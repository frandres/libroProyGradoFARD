% file thesis.tex
% Archivo thesis.tex
% Documento maestro que incluye todos los paquetes necesarios para el documento
% principal.

% Documento obtenido por un sinfin de iteraciones de administradores del LDC
% Estructura actual hecha por:
% Jairo Lopez <jairo@ldc.usb.ve>
% Actualizado ligeramente por:
% Alexander Tough 

\documentclass[oneside,11pt,letterpaper]{report}
\tolerance=1000  
\hbadness=10000  
\raggedbottom

\usepackage{fancybox}
\usepackage[utf8]{inputenc} % Caracteres especiales
\usepackage[T1]{fontenc}
\usepackage{ae,aecompl}
\usepackage{appendix} 
% Paquetes para manejar graficos
\usepackage{epsf}
\usepackage[pdftex]{graphicx}
\graphicspath{{./diagramas/}}
\DeclareGraphicsExtensions{.jpg,.jpeg,.png}
\usepackage{epsfig}
\usepackage{pgf}            % Gráficos con LaTeX (PGF/Tikz)
\usepackage{pgfsys}         % Gráficos con LaTeX (PGF/Tikz)
\usepackage{pgfcore}        % Gráficos con LaTeX (PGF/Tikz)
%\usepackage{pgfmath}        % Gráficos con LaTeX (PGF/Tikz)
\usepackage{pgfpages}       % Gráficos con LaTeX (PGF/Tikz)
\usepackage{tikz}           % Gráficos con LaTeX (PGF/Tikz)
\usepackage{pgfplots}       % Ploteo de Gráficos con LaTeX (PGF/Tikz)
\usepackage{subfig}         % Dibujos (operadores vecindad) (PGF/Tikz)
\usepackage{pdflscape}      % Poner tablas grandes horizontalmente
\usepgflibrary{arrows,shapes}      % PGF/Tikz: Flechas
\usepgflibrary{plotmarks}
% Simbolos matematicos
\usepackage{latexsym,amssymb}
% Paquetes para presentar una tesis decente.
\usepackage{setspace,cite} % Doble espacio para texto, espacio singular para
                           % los caption y pie de pagina

% Paquetes no utilizados para citas
%\usepackage{mcite} 
%\usepackage{draft} 

\usepackage{multirow}
\usepackage{verbatim} 
\usepackage{wrapfig}
\usepackage{alltt}
\usepackage{threeparttable}

% Acentos 
\usepackage[spanish,activeacute]{babel}
\usepackage[spanish]{translator}
\usepackage{color, xcolor, colortbl}
\usepackage{multirow}
\usepackage{subfig}
\usepackage[OT1]{fontenc}
\usepackage{tocbibind}
\usepackage{anysize}
\usepackage{listings}


% Para poder tener texto asiatico
%\usepackage{CJK}

% Opciones para los glosarios
\usepackage[style=altlist,toc,numberline,acronym]{glossaries}
\usepackage{url}
\usepackage{amsthm}
\usepackage{amsmath}
\usepackage{amssymb}        % Símbolos matemáticos
\usepackage{fancyhdr} % Necesario para los encabezados
\usepackage{fancyvrb}
\usepackage{makeidx} % En caso de necesitar indices.
\usepackage{float}
\usepackage{caption}
\DeclareCaptionFont{white}{\color{white}}
\DeclareCaptionFormat{listing}{\colorbox{white}{\parbox{\textwidth}{#1#2#3}}}
\lstset{
 		 language=Pascal,
         numbers=left,               % Ort der Zeilennummern
         numberstyle=\tiny,          % Stil der Zeilennummern
         numbersep=5pt,              % Abstand der Nummern zum Text
         tabsize=5,                  % Groesse von Tabs
         extendedchars=true,         %
         breaklines=true,            % Zeilen werden Umgebrochen
         keywordstyle=\bfseries,
         frame=b,
         showspaces=false,           % Leerzeichen anzeigen ?
         showtabs=false,             % Tabs anzeigen ?
         xleftmargin=17pt,
         framexleftmargin=17pt,
         framexrightmargin=5pt,
         framexbottommargin=4pt,
         backgroundcolor=\color{white},
         showstringspaces=false      % Leerzeichen in Strings anzeigen ?        
}
\floatstyle{boxed} 
\restylefloat{figure}

\hyphenation{ins-tan-cias}
\hyphenation{ope-ra-do-res}
\hyphenation{indi-vi-duos}
\hyphenation{re-le-van-tes}
\hyphenation{li-ge-ra-mente}
\hyphenation{ne-ce-saria-men-te}
\hyphenation{cam-bia-das}
\hyphenation{na-tu-ral}
\hyphenation{com-bi-na-to-ria}
\hyphenation{fo-ca-li-za-da}
\hyphenation{mo-di-fi-ca-cio-nes}
\hyphenation{re-gu-la-res}
\hyphenation{con-ti-nua-cion}
\hyphenation{an-te-rio-res}
\hyphenation{fo-ca-li-za-do}
\hyphenation{ge-ne-rar}
\hyphenation{des-co-no-ci-mien-to}
\hyphenation{he-rra-mien-tas}

\makeindex  % Necesario para los indices

% Definiciones para definicions, teoremas y lemas
\theoremstyle{definition} \newtheorem{definicion}{Definici\'{o}n}
\theoremstyle{plain} \newtheorem{teorema}{Teorema}
\theoremstyle{plain} \newtheorem{lema}{Lema}

\usepackage[subfigure]{tocloft}
\setlength{\cfttabnumwidth}{3em}

% Para la creacion de los pdfs
\usepackage{hyperref}

% Para resolver el lio del Unicode para la informacion de los PDFs
% En pdftitle coloca el nombre de su proyecto de grado/pasantia.
% En pdfauthor coloca su nombre.
\hypersetup{
    pdftitle = {EXTRACCIÓN DE INFORMACIÓN FOCALIZADA BASADA EN RESPUESTAS INCOMPLETAS (FIE)},
    pdfauthor={FRANCISCO RODRÍGUEZ DRUMOND},
    colorlinks,
    citecolor=black,
    filecolor=black,
    linkcolor=black,
    urlcolor=black,
    backref,
    pdftex
}

\renewcommand{\appendixname}{Apéndice}
\renewcommand{\appendixtocname}{Apéndices}
\renewcommand{\appendixpagename}{Apéndices}

% Crea el glosario
\makeglossaries

% Incluye el glosario
\newacronym{as-h}{as-H}{proceso autosimilar con par\'{a}metro autosimilar $H$}

\newacronym{asie-h}{asie-H}{proceso autosimilar con par\'{a}metro autosimilar
$H$ e incrementos estacionarios}

\newacronym{aas-h}{aas-H}{proceso asint\'{o}ticamente autosimilar con
par\'{a}metro autosimilar $H$}


% Para crear la hoja escaneada de las firmas
\usepackage[absolute]{textpos}

% Pone los nombres y las opciones para mostrar los codigos fuentes
\renewcommand\lstlistingname{Código}
\renewcommand\lstlistlistingname{Índice de algoritmos}
\captionsetup{tablename=Tabla}
\captionsetup{listtablename=Índice de tablas}

% Dimensiones de la pagina
\setlength{\headheight}{14pt}
\marginsize{2cm}{2cm}{2cm}{2cm}

% Se pueden omitir para que no compile ciertos capitulos.
%\includeonly{header, resultados, conclusiones}

\lstset{
    literate={´} {$\'$}{1} % set tilde as a literal (no process)
}

\lstnewenvironment{code}[1][]%
{
   \noindent
   \minipage{\linewidth} 
   \vspace{0.5\baselineskip}
   \lstset{basicstyle=\footnotesize,frame=single,#1}}
{\endminipage}

%%%%%%%%%%%%%%%%%%%%%%%%%%%%%%%%%%%%%%%%%%%%%%%%%%%%%%%%%%%%%%%%%%%%%%%%%%%
%%%%%%%%%%%%%%%%      end of preamble and start of document     %%%%%%%%%%%
%%%%%%%%%%%%%%%%%%%%%%%%%%%%%%%%%%%%%%%%%%%%%%%%%%%%%%%%%%%%%%%%%%%%%%%%%%%
\begin{document}

% Pagina de titulo
% Pagina de titulo
\begin{titlepage}
\begin{center}

% Upper part (aqui ya esta incluido el logo de la USB).
\includegraphics[scale=0.5,type=png,ext=.png,read=.png]{figures/cebolla} \\

% Encabezado
\textsc {\large UNIVERSIDAD SIMÓN BOLÍVAR} \\
\textsc{\bfseries DECANATO DE ESTUDIOS PROFESIONALES\\
COORDINACIÓN DE INGENIERÍA DE LA COMPUTACIÓN}

\bigskip
\bigskip
\bigskip
\bigskip
\bigskip
\bigskip
\bigskip
\bigskip
\bigskip

% Title/Titulo
% Aqui ponga el nombre de su proyecto de grado/pasantia larga
\textsc{\bfseries EXTRACCIÓN DE INFORMACIÓN FOCALIZADA BASADA EN RESPUESTAS INCOMPLETAS (FIE)}
\bigskip
\bigskip
\bigskip
\bigskip
\bigskip

% Author and supervisor/Autor y tutor
\begin{minipage}{\textwidth}
\centering
Por: \\
FRANCISCO RODRÍGUEZ DRUMOND \\

\bigskip
\bigskip
\bigskip

Realizado con la asesoría de: \\
PROF. SORAYA ABAD
\end{minipage}

\bigskip
\bigskip
\bigskip
\bigskip
\bigskip
\bigskip
\bigskip
\bigskip
\bigskip

% Bottom half
{PROYECTO DE GRADO \\ Presentado ante la Ilustre Universidad Simón Bolívar \\
como requisito parcial para optar al título de \\ Ingeniero de Computación} \\

\bigskip
\bigskip
\vfill

% Date/Fecha 
{\large \bfseries Sartenejas, Septiembre de 2012}

\end{center}
\end{titlepage}


% Pagina de acta final (vacio)
% Pagina del acta final
\begin{titlepage}
\begin{center}

% Upper part
\includegraphics[scale=0.5,type=png,ext=.png,read=.png]{figures/cebolla} \\

\textsc {\large UNIVERSIDAD SIM'ON BOL'IVAR} \\
\textsc{DECANATO DE ESTUDIOS PROFESIONALES\\
COORDINACI'ON DE INGENIER'IA DE LA COMPUTACI'ON}

\bigskip
\bigskip
\bigskip
\bigskip
\bigskip
\bigskip

% Title
\textsc{ACTA FINAL PROYECTO DE GRADO}

\bigskip
\bigskip

% Aqui coloca el nombre de su proyecto de grado/pasantia larga.
\textsc{\bfseries  EXTRACCIÓN DE INFORMACIÓN FOCALIZADA BASADA EN RESPUESTAS INCOMPLETAS (FIE)}

\bigskip
\bigskip
\bigskip
\bigskip

\begin{minipage}{\textwidth}
\centering
Presentado por: \\
% Aqui coloca su nombre.
\textsc{\bfseries FRANCISCO RODRÍGUEZ DRUMOND} \\

\bigskip
\bigskip
\bigskip
\bigskip

Este Proyecto de Grado ha sido aprobado por el siguiente jurado examinador: \\

\bigskip
\bigskip

% Despues de cada line coloca el (los) nombre(s) de
% cada uno de los integrantes del jurado.
\line(1,0){200} \\
PROF. 1\\

\bigskip
\bigskip

\line(1,0){200} \\
PROF. 2\\

\bigskip
\bigskip

\line(1,0){200} \\
PROF. 3\\
\end{minipage}

\bigskip
\bigskip
\vfill

% Date/Fecha
{\large \bfseries Sartenejas, X/X/12}

\end{center}
\end{titlepage}

\setcounter{secnumdepth}{3}
\setcounter{tocdepth}{4}

% Define encabezado numeros romanos y como se separan los captiulos y las
% secciones
\addtolength{\headheight}{3pt}
\pagenumbering{roman}
\pagestyle{fancyplain}


\renewcommand{\chaptermark}[1]{\markboth{\chaptername\ \thechapter:\,\ #1}{}}
\renewcommand{\sectionmark}[1]{\markright{\thesection\,\ #1}}

\onehalfspacing

\lhead{}
\chead{}
\rhead{}
\renewcommand{\headrulewidth}{0.0pt}
\lfoot{}
\cfoot{\fancyplain{}{\thepage}}
\rfoot{}


% Pagina de resumen
\setcounter{page}{4}
\chapter*{Resumen}
%
%El Problema de Enrutamiento de Vehículos (VRP por las siglas de Vehicle Routing Problem) es un 
%problema de optimización combinatoria del área de logística y distribución de bienes. Este problema busca 
%determinar las rutas que deben transitar un grupo de vehículos con el objetivo de transportar la cantidad de
%productos demandada por un conjunto de consumidores de forma tal que se minimice la distancia total recorrida. 
%El Problema de Enrutamiento de Vehículos con Entrega y Recepción Simúltanea (VRPSPD por las siglas de Vehicle Routing 
%Problem with Simultaneus Pick-up and Delivery) es una variante de VRP que permite a los clientes tanto recibir como entregar bienes de manera simultánea. Este problema está clasificado como un problema complejo, pues resolverlo de manera exacta por un algoritmo implica tiempos que se incrementan de manera exponencial a medida que el número de clientes aumenta de tamaño. Las metaheurísticas constituyen un conjunto de técnicas que  permiten conseguir buena calidad de soluciones y tiempos computacionales aceptables para este timpo de problemas. En este trabajo se presenta la implementación de seis metaheurísticas para la resolución de VRPSPD. Experimentos son realizados, así como entonación de los parámetros más influyentes para cada metaheurística. Finalmente, se comparan los resultados obtenidos entre las mejores metaheurísticas y se recomienda la utilización de la mejor de ellas, la cual tiene resultados competitivos en comparación con los mejores algoritmos en la literatura.

El problema de enrutamiento de vehículos (VRP por las siglas de Vehicle Routing Problem) es un problema de optimización combinatoria del área de logística y distribución de bienes. Una de sus variantes es VRPSPD, el cual tiene la particularidad que los clientes pueden recibir y entregar bienes de manera simultánea.\\

En este trabajo se busca resolver VRPSPD por medio de diferentes metaheurísticas híbridas, con el propósito de comparar sus resultados y poder recomendar la que genere la solución de mejor calidad y tiempo.\\

Para ello se realizó una investigación sobre las metaheurísticas más utilizadas para resolver VRPSPD y se seleccionaron seis. Esas seis se implementaron y para cada una de ellas se estudiaron los parámetros que más influyen en su comportamiento, para luego entonar cada uno de ellos. Una vez entonada cada metaheurística, se comparó su desempeño con los resultados del trabajo utilizado como base para la implementación de la metaheurística. Las metaheurísticas son ejecutadas con un conjunto de instancias del problema recomendadas en la literatura, mayormente utilizadas en los trabajos referenciados. Finalmente, se seleccionaron las metaheurísticas que proporcionaron la solución de mejor calidad y menor tiempo según cada clase de instancia, y se recomendaron las mejores.

\newpage



% Pagina de dedicatoria (opcional)
%\setcounter{page}{5}

\vspace*{8cm} 
\pdfbookmark[0]{Dedication}{dedication} % Sets a PDF bookmark for the dedication
\begin{center} 
\large DEDICATORIA
\end{center}
\newpage


% Pagina de agradecimientos (opcional)
\chapter*{Acrónimos}

\newcommand{\TERM}[3]{
	\textbf{#1} & #3 & #2 \\[5pt]
}

\begin{longtable}{p{1in}p{2.2in}p{3in}}

		\TERM{FIE}{Extracción Focalizada de Información}{Focalized Information Extraction}
		\TERM{DIA}{Arquitectura de Interrogación de Documentos}{Document Interrogation Architecture}
		\TERM{MBFI}{Módulo de Búsqueda Focalizada de Información}{Módulo de Búsqueda Focalizada de Información}
		\TERM{MDFI}{Módulo de Determinación de la Fuente de Incompletitud}{Módulo de Determinación de la Fuente de Incompletitud}		
		\TERM{EFI}{Extractor Focalizado de Información}{Extractor Focalizado de Información}								
%		\TERM{}{}{}		
		DUDA: Que hago con los acrónimos que son sólo en español? Deberia poner los acronimos siempre en internet?
\end{longtable}

\newpage


% More space for figure numbers
% Space between elements of the list

% Crea la tabla de contenidos
\tableofcontents
\clearpage

% Crea la lista de cuadros
\listoftables
\clearpage

% Crea la lista de figuras
\listoffigures
\clearpage

% Crea la lista de algoritmos
%\lstlistoflistings
%\clearpage


% Crea la lista de codigos fuentes
%\lstlistoflistings

\clearpage

% Define encabezado en numeros arabicos  
\pagenumbering{arabic}

\fancyhf{} % Redefine el encabezado 
\lhead{}
\chead{}
\rhead{\fancyplain{}{\thepage}}
\renewcommand{\headrulewidth}{0.0pt}
\lfoot{}
\cfoot{}
\rfoot{}

%\doublespacing

% Incluye los archivos deseados - El contenido de
% su proyecto de grado/pasantia larga.
\chapter*{Introducción} \label{chap:intro}
\addcontentsline{toc}{chapter}{Introducción}

Uno de los retos aún presentes en el mundo de la computación hoy por hoy es el de construir herramientas que permitan realizar operaciones de procesamiento de un lenguaje natural. Se entiende por lenguaje natural un idioma como el inglés, español, italiano, que no es reconocible por una computadora sin herramientas pertientes. Cada día surgen más y más tecnologías en este ámbito: existen procesadores de comandos de voz que son capaces de entender instrucciones dadas por un usuario, programas que permiten tomar dictados de un humano, traductores de textos y programas que colocan signos de puntuación automáticamente, entre otros. A la par de dichos avances, sin embargo, hay muchísimos campos los cuales se hacen más difíciles para la computación; en la medida en la que los objetivos trazados requieren una mayor ``comprensión'' semántica del lenguaje natural se hace más complicado construir herramientas que resuelvan los problemas planteados. \\ 

Uno de las áreas dentro del mundo de la computación que se está desarrollando actualmente y cuya utilidad es altísima es la Extracción de Información (IE, que significa Information Extraction). La IE tiene como objetivo ubicar dentro de un texto datos que puedan ser útiles para responder preguntas sobre el dominio del texto. (o en general datos en cualquier medio que se encuentre en algún lenguaje humano). Por ejemplo, puede ser deseable construir un extractor que analice noticias de un periódico para poblar una base de datos con los principales eventos que han ocurrido. Dicha base de datos puede luego ser consultada de forma que un usuario pueda buscar algún dato que sea necesario sin tener que leer todas las noticias. Generalizando, los avances en esta  área pueden ser aplicados en cualquier dominio en el cual se tenga una amplia cantidad de información en lenguaje natural no entendible por una máquina: ámbitos corporativos, educativos, diplomáticos, etcétera. Las posibles aplicaciones de ésto son numerosas.\\

Abad Mota (2009) propuso en \cite{documentInterrogationArchitecture} una arquitectura para la Interrogación de Documentos (DIA, por sus siglas Document Interrogation Architecture) que busca resolver el problema de extracción de información en documentos escritos. La interrogación de documentos se entiende como un proceso en el cual tienen lugar varias actividades: una primera extracción de información de un conjunto de documentos, la realización de consultas sobre una ontología poblada con información previa, y la resolución de consultas con información que no se encuentra presente en la ontología o en la base de datos, accediendo a los documentos. \\

Ruiz en \cite{ruiz-HMM} y \cite{SemistructuredTextExtraction} propuso mecanismos para realizar la primera extracción definida en DIA para poblar inicialmente la ontología. Apaza y Mirisola en \cite{ODILImplementation} e Ituarte en \cite{ODILImplementationDefinition} implementaron el lenguaje ODIL (Lenguaje de Interrogación de Documentos basado en una ontología) diseñado a la par de DIA para la interrogación de Documentos. Todos estos avances están relacionados con las primeras fases propuestas en DIA.  \\

El propósito del presente proyecto de grado es diseñar e implementar un mecanismo de \emph{extracción focalizada}, que permita resolver consultas cuando la base de datos poblada no se encuentre completa. Dicho en otras palabras, el objetivo general del mismo es implementar un mecanismo que permita mediante búsquedas focalizadas encontrar datos faltantes en la base de datos a partir de un conjunto de documentos e información presente en la ontología y la base de datos. Esta fase es la última propuesta en DIA y opera bajo la premisa de que es posible obtener mejores resultados en IE cuando se tiene información de contexto que ayude a encontrar la información faltante en documentos mediante una nueva extracción. \\

Para poder realizar la extracción focalizada es necesario una serie de pasos que serán descritos en profundidad en este informe. Por ahora conviene mencionar como objetivos específicos o pasos intermedios: el estudio del lenguaje de consultas ODIL basado en ontologías y clasifiación de los tipos de respuestas incompletas que se pueden obtener de las consultas escritas en este lenguaje, el estudio del concepto y técnicas de extracción de información, el análisis de los metadatos útiles para facilitar la búsqueda de datos faltantes, la definición e implementación de un mecanismo de extracción focalizada y el estudio de las métricas para la descripción de las respuestas aproximadas y síntesis de una métrica adecuada para evaluar el mecanismo de extracción focalizada implementado. \\

El presente informe busca sintetizar los principales resultados obtenidos en este proyecto. Para ello está estructurado en 7 capítulos. Primeramente se hace un planteamiento del problema introduciendo más en detalle la arquitectura DIA y presentando en profundidad el concepto de extracción focalizada. Seguidamente se presenta una breve revisión de algunos trabajos existentes en el ámbito de extracción de información. Posteriormente se presentan en los capítulos 3 y 4 el diseño propuesto para la extracción focalizada de la información y el mecanismo concebido para determinar los origenes de incompletitud propios de una ontología incompleta. Luego se realizan algunos comentarios relevantes sobre la implementación del sistema, se presentan las pruebas realizadas para probar el sistema en conjunto con sus resultados y el análisis pertinente. \\

\chapter{Planteamiento del Problema} \label{chap:planteamientoProblema}

El problema a tratar en el presente trabajo de investigación consiste en diseñar, implementar y probar el proceso de Extracción Focalizada de Información \emph{(FIE)} definida la Arquitectura para la Interrogación de Documentos \emph{(DIA)} por Abad Mota \cite{documentInterrogationArchitecture}. 

\chapter{Marco Teórico} \label{chap:marcoTeorico}

Marco Teórico goes here.\cite{TCL}
\chapter{Análisis de incompletitudes en consulta y propuesta para determinar las fuentes de incompletitud} \label{chap:analisisFuenteIncompletitud}


\chapter{Diseño} \label{chap:diseno}

El sistema para realizar la Extracción Focalizada de Información está conformado por dos módulos: el Determinador del Origen de Incompletitud y el Buscador Focalizado de Información. 

\section{El Determinador del Origen de Incompletitud (DOI)}\label{sect:diseno-DOI}

Extractor plano: no se toman en cuenta ors. Quizas se debería.

\section{El Buscador Focalizado de Información (BFI)}\label{sect:diseno-BFI}

El Buscador Focalizado de Información es el encargado de encontrar los valores de los campos cuyos valores son desconocidos o incompletos para una consulta dada. Dichos valores son buscados en un conjunto de documentos presentes en el sistema de archivos donde se ejecute el sistema utilizando información de contexto proporcionada por el Determinador del Origen de Incompletitud (DOI).\\

Para llevar a cabo esta tarea, el BFI tiene dos componentes: un preprocesador de texto y un extractor focalizado de información. \\


\begin{figure}[h]
  \centering
  \includegraphics[width=1\textwidth]%
    {BFIdiagram}
  \caption[Diagrama del Buscador Focalizado de Información]
   {Diagrama del Buscador Focalizado de Información}
\end{figure}

\begin{comment}
\begin{figure}[hb]
  \centering
  \includegraphics[width=1\textwidth]%
    {DOIdiagram}
  \caption[Diagrama del Determinador del Origen de Incompletitud]
   {Diagrama del Determinador del Origen de Incompletitud}
\end{figure}
\end{comment}

El Preprocesador de Texto tiene como objetivo preparar el conjunto de documentos sobre el cual se hará la Búsqueda Focalizada de Información y dejarlo listo para la realización de la extracción. Las tareas para ello incluyen leer el documento en su formato original y extraer el fragmento o el conjunto de fragmentos del documento que están relacionados con el dominio sobre el cual se hará la Búsqueda Focalizada de Información.\\

El preprocesamiento de texto es una tarea que se realiza \emph{offline}. Es decir, cada conjunto de documentos puede ser preprocesado previamente de forma para que las consultas sean respondidas lo más rápidamente posible. \\

Por su parte, el Extractor Focalizado de Información (EFI) tiene como objetivo extraer el valor del campo faltante utilizando como materia prima el conjunto de documentos preprocesados y un contexto de extracción. El contexto de extracción consiste en toda la información que el DOI puede recopilar sobre la consulta que se está realizando que puede ser de útil para hacer la extracción focalizada de información. Puede incluir por ejemplo valores de campos que están relacionados con el campo faltante dentro del dominio sobre el cual se hace la búsqueda.  \\

El DOI trabaja directamente con el Extractor Focalizado de Información para encontrar los valores faltantes. Esto es, una vez que se determinan las fuentes de incompletitud, el EFI recibe solicitudes de búsqueda para cada uno de los campos con valores faltantes. Esas solicitudes permiten descubrir y completar la ontología y dar respuesta a las consultas que tiene el usuario. El preprocesador, como se ha dicho, se ejecuta una única vez por conjunto de documentos. \\


\chapter{Detalles de implementación} \label{chap:implementacion}

\section{Consideraciones generales} \label{sect:implementacion-consideraciones}

\subsection{Lenguaje de programación} \label{sect:implementacion-lenguaje}

El lenguaje de programación utilizado para la implementación de las seis metaheurísticas fue Java, compilado usando g++ en su versión 
1.6.0\_20.
\chapter{Pruebas} \label{chap:pruebas}

Descripción del conjunto de pruebas < Hablar de los dominios: se trabajó sobre designaciones, etc>\\

Duda: esto es una especie de marco metodológico?\\

\section{Selección de Documentos de Prueba}

La selección de los documentos se realizó sobre el conjunto de las actas de los Consejos Directivos y Académicos de la Universidad Simón Bolívar. Con el objetivo de probar apropiadamente el Extractor, se decidió trabajar con las actas desde el año 1998 hasta el año 2012. Esto permite que las pruebas se hagan sobre un conjunto de documentos semiestructurado, pero que puede mostrar variabilidad en su redacción y estructura en los años.\\

Del conjunto de actas en el intervalo de tiempo especificado, se hizo una preselección de los documentos para utilizar las actas de los Consejos Extraordinarios. Adicionalmente, según cada dominio de aplicación se extrajeron las actas que sí contienen información sobre los dominios. Es decir, se utilizaron los documentos que contienen información sobre los dominios elegidos.\\

Para probar el preprocesador, se tomó un conjunto de archivos seleccionados aleatoriamente con uniformidad sobre los años. La prueba consistió en ver si el fragmento de texto extraído por el preprocesador coincidía exactamente, estaba contenido, contenia o era completamente disjunto con el segmento de texto que debería ser extraído. Esta prueba se hizo manualmente. \\

\section{Pruebas de Extractor Focalizado}

Para probar el extractor focalizado, se tomó un conjunto de prueba seleccionado aleatoriamente con uniformidad sobre los años. De cada archivo seleccionado, se tomaban hasta 3 designaciones (falta explicalro para los otros dominios), que se utilizaban para realizar pruebas. Una prueba consite en hacer una búsqueda focalizada sobre cada uno de los campos de una designación: se asume que se tienen algunos valores de esa designación (un contexto de extracción) y se buscan los campos faltantes. \\

Estas pruebas se hicieron variando 2 parámetros. En primera instancia, se varió sobre el minimum hit measure (varió entre 0.25;0.5;0.75;1) que mide la proporción de los campos con valores conocidos que aparecen en una unidad de extracción (designación). Luego para cada hit measure se varió sobre la probabilidad de que cada uno de los campos fuese desconocido. Esto es, para cada hit measure al hacer una prueba se determinaba aleatoriamente cuantos campos tenían valores conocidos. Esto para poder simular los casos de la vida real en los que no se tienen todos los campos de una designación menos el conocido. Las probabilidades de no tener un valor puntual variaron entre 0.1;0.25;0.75. \\


% Resultados
\chapter{Resultados} \label{chap:resultados}

A continuación se muestran los resultados de las pruebas realizadas.

\section{Resultados de las pruebas unitarias sobre el Módulo de Determinación de Origen de la Incompletitud.} 

\section{Resultados de las pruebas unitarias sobre el Módulo de Búsqueda Focalizada de Información}


\begin{table}[h]
\caption{Resultados detallados la evaluación del Preprocesador de Textos}
\centering
\scriptsize
\begin{tabular*}{1\textwidth}{@{\extracolsep{\fill}} !{\vrule width 1pt} c !{\vrule width 1pt} c | c | c!{\vrule width 1pt} c | c | c!{\vrule width 1pt}}
\hline
Dominio & \multicolumn{3}{c!{\vrule width 1pt}}{\bf{Aprobados}} & \multicolumn{3}{c |}{\bf{No aprobados}}\\
\hline
 & Correctos & Con texto de más & \bf{Total aprobados} & Incompletos & Incorrectos & \bf{Total no aprobados}\\
\hline
Designaciones & 87.69\% & 7.69\% & \bf{95.39\%} & 3.07\% & 1.54\% & \bf{4.61\%}\\
\hline
Escalafón & 80.51\% & 12.98\% & \bf{93.6\%}  & 6.4\% & 0\% & \bf{6.4\%} \\
\hline
Jurados de Ascenso & 77.55\% & 16.32\% & \bf{93.88\%} & 0\% & 6.12\% & \bf{6.12\%} \\
\hline
\end{tabular*}
\label{tabla-resultados-preprocesamientoDatosDesignacion}

\end{table}

%------------------------

\begin{comment}
\begin{table}[h]
\caption{Resultados detallados la evaluación del Preprocesador de Textos: Escalafón}
\centering
\scriptsize
\begin{tabular*}{.68\textwidth}{@{\extracolsep{\fill}} !{\vrule width 1pt} c !{\vrule width 1pt} c | c !{\vrule width 1pt} c | c !{\vrule width 1pt}}
\hline
Dominio & \multicolumn{2}{c!{\vrule width 1pt}}{\bf{Aprobados}} & \multicolumn{2}{c |}{\bf{No aprobados}}\\
\hline
 & Correctos & Con texto de más & Incompletos & Incorrectos \\
\hline
\multirow{2}{*}{Designaciones} & 80.51\% & 12.98\% & 6.4\% & 0\% \\\cline{2-5}
& \multicolumn{2}{c!{\vrule width 1pt}}{Aprobados: \bf{93.6}} & \multicolumn{2}{c |}{No aprobados: \bf{6.4\%}}\\
\hline
\end{tabular*}
\label{tabla-resultados-preprocesamientoDatosEscalafon}
\\El conjunto de prueba es de tama~no 77, un tercio de la población.
\end{table}
\end{comment}

%------------------------

\begin{table}[h]
\caption{ Resultados de la evaluación del Extractor Focalizado - Dominio: Designaciones. UnitHit Measure mínimo:.75}
\centering
\scriptsize
\begin{tabular*}{1\textwidth}{@{\extracolsep{\fill}} | c | c | c | c | c | c |}
\hline
Campo & Prob. Campo Faltante & \multicolumn{2}{c|}{\bf{P. Aprobados}} & \multicolumn{2}{c |}{\bf{P. No aprobados}}\\
\hline

\multicolumn{2}{|c|}{ } & Correctos & Con texto de más & Incompletos & Incorrectos \\
\hline
\multirow{8}{*}{EsAsignado.calificacion} 

	& \multirow{2}{*}{0} 
	& 87.69\% & 7.69\% & 3.07\% & 1.54\% \\
	\cline{3-6}
	& & \multicolumn{2}{c|}{Aprobados: \bf{95.38\%}} & \multicolumn{2}{c |}{No aprobados: \bf{4.61\%}}\\
	\cline{2-6}
	
	& \multirow{2}{*}{0.10} 
	& 87.69\% & 7.69\% & 3.07\% & 1.54\% \\
	\cline{3-6}
	& & \multicolumn{2}{c|}{Aprobados: \bf{95.38\%}} & \multicolumn{2}{c |}{No aprobados: \bf{4.61\%}}\\
	\cline{2-6}

	& \multirow{2}{*}{0.25} 
	& 87.69\% & 7.69\% & 3.07\% & 1.54\% \\
	\cline{3-6}
	& & \multicolumn{2}{c|}{Aprobados: \bf{95.38\%}} & \multicolumn{2}{c |}{No aprobados: \bf{4.61\%}}\\
	\cline{2-6}
	
	& \multirow{2}{*}{0.50} 
	& 87.69\% & 7.69\% & 3.07\% & 1.54\% \\
	\cline{3-6}
	& & \multicolumn{2}{c|}{Aprobados: \bf{95.38\%}} & \multicolumn{2}{c |}{No aprobados: \bf{4.61\%}}\\

\hline
	
\multirow{8}{*}{EsAsignado.fechaAsignacion} 

	& \multirow{2}{*}{0} 
	& 87.69\% & 7.69\% & 3.07\% & 1.54\% \\
	\cline{3-6}
	& & \multicolumn{2}{c|}{Aprobados: \bf{95.38\%}} & \multicolumn{2}{c |}{No aprobados: \bf{4.61\%}}\\
	\cline{2-6}
	
	& \multirow{2}{*}{0.10} 
	& 87.69\% & 7.69\% & 3.07\% & 1.54\% \\
	\cline{3-6}
	& & \multicolumn{2}{c|}{Aprobados: \bf{95.38\%}} & \multicolumn{2}{c |}{No aprobados: \bf{4.61\%}}\\
	\cline{2-6}

	& \multirow{2}{*}{0.25} 
	& 87.69\% & 7.69\% & 3.07\% & 1.54\% \\
	\cline{3-6}
	& & \multicolumn{2}{c|}{Aprobados: \bf{95.38\%}} & \multicolumn{2}{c |}{No aprobados: \bf{4.61\%}}\\
	\cline{2-6}
	
	& \multirow{2}{*}{0.50} 
	& 87.69\% & 7.69\% & 3.07\% & 1.54\% \\
	\cline{3-6}
	& & \multicolumn{2}{c|}{Aprobados: \bf{95.38\%}} & \multicolumn{2}{c |}{No aprobados: \bf{4.61\%}}\\
	\cline{2-6}
\hline

\multirow{8}{*}{EsAsignado.fechaFinal} 

	& \multirow{2}{*}{0} 
	& 87.69\% & 7.69\% & 3.07\% & 1.54\% \\
	\cline{3-6}
	& & \multicolumn{2}{c|}{Aprobados: \bf{95.38\%}} & \multicolumn{2}{c |}{No aprobados: \bf{4.61\%}}\\
	\cline{2-6}
	
	& \multirow{2}{*}{0.10} 
	& 87.69\% & 7.69\% & 3.07\% & 1.54\% \\
	\cline{3-6}
	& & \multicolumn{2}{c|}{Aprobados: \bf{95.38\%}} & \multicolumn{2}{c |}{No aprobados: \bf{4.61\%}}\\
	\cline{2-6}

	& \multirow{2}{*}{0.25} 
	& 87.69\% & 7.69\% & 3.07\% & 1.54\% \\
	\cline{3-6}
	& & \multicolumn{2}{c|}{Aprobados: \bf{95.38\%}} & \multicolumn{2}{c |}{No aprobados: \bf{4.61\%}}\\
	\cline{2-6}
	
	& \multirow{2}{*}{0.50} 
	& 87.69\% & 7.69\% & 3.07\% & 1.54\% \\
	\cline{3-6}
	& & \multicolumn{2}{c|}{Aprobados: \bf{95.38\%}} & \multicolumn{2}{c |}{No aprobados: \bf{4.61\%}}\\
	\cline{2-6}
\hline

\multirow{8}{*}{EsAsignado.motivo} 

	& \multirow{2}{*}{0} 
	& 87.69\% & 7.69\% & 3.07\% & 1.54\% \\
	\cline{3-6}
	& & \multicolumn{2}{c|}{Aprobados: \bf{95.38\%}} & \multicolumn{2}{c |}{No aprobados: \bf{4.61\%}}\\
	\cline{2-6}
	
	& \multirow{2}{*}{0.10} 
	& 87.69\% & 7.69\% & 3.07\% & 1.54\% \\
	\cline{3-6}
	& & \multicolumn{2}{c|}{Aprobados: \bf{95.38\%}} & \multicolumn{2}{c |}{No aprobados: \bf{4.61\%}}\\
	\cline{2-6}

	& \multirow{2}{*}{0.25} 
	& 87.69\% & 7.69\% & 3.07\% & 1.54\% \\
	\cline{3-6}
	& & \multicolumn{2}{c|}{Aprobados: \bf{95.38\%}} & \multicolumn{2}{c |}{No aprobados: \bf{4.61\%}}\\
	\cline{2-6}
	
	& \multirow{2}{*}{0.50} 
	& 87.69\% & 7.69\% & 3.07\% & 1.54\% \\
	\cline{3-6}
	& & \multicolumn{2}{c|}{Aprobados: \bf{95.38\%}} & \multicolumn{2}{c |}{No aprobados: \bf{4.61\%}}\\
	\cline{2-6}
\hline

\multirow{8}{*}{Profesor.Nombre} 

	& \multirow{2}{*}{0} 
	& 87.69\% & 7.69\% & 3.07\% & 1.54\% \\
	\cline{3-6}
	& & \multicolumn{2}{c|}{Aprobados: \bf{95.38\%}} & \multicolumn{2}{c |}{No aprobados: \bf{4.61\%}}\\
	\cline{2-6}
	
	& \multirow{2}{*}{0.10} 
	& 87.69\% & 7.69\% & 3.07\% & 1.54\% \\
	\cline{3-6}
	& & \multicolumn{2}{c|}{Aprobados: \bf{95.38\%}} & \multicolumn{2}{c |}{No aprobados: \bf{4.61\%}}\\
	\cline{2-6}

	& \multirow{2}{*}{0.25} 
	& 87.69\% & 7.69\% & 3.07\% & 1.54\% \\
	\cline{3-6}
	& & \multicolumn{2}{c|}{Aprobados: \bf{95.38\%}} & \multicolumn{2}{c |}{No aprobados: \bf{4.61\%}}\\
	\cline{2-6}
	
	& \multirow{2}{*}{0.50} 
	& 87.69\% & 7.69\% & 3.07\% & 1.54\% \\
	\cline{3-6}
	& & \multicolumn{2}{c|}{Aprobados: \bf{95.38\%}} & \multicolumn{2}{c |}{No aprobados: \bf{4.61\%}}\\
	\cline{2-6}
\hline
\end{tabular*}
\label{tabla-resultados-EFDesignaciones.75}
\\
Prob. Campo Faltante es la probabilidad de que no se tenga el valor uno de los campos que se utilizan para hacer extracción focalizada.
\end{table}

Por que el extractor falla con designaciones? (Observaciones hechas al hacer las pruebas que pueden ser útiles en el análisis de datos)

1. Las unidades de informacion no son precisas. A veces hay multiples designaciones en una misma linea. \\
2. Hay multiples designaciones que coinciden en un mismo día para una misma persona. \\
3. Hay ratificaciones, correcciones, posteriores a la designacoines que modifican el resultado que uno pensaria correcto.\\
4. Hay errores de tipeo por parte de la secretaria.\\
5. Hay casos "patologicos" que es imposible generalizar con expresiones regulares que no sean hechas a la medida. \\
6. Hay casos en los que los campos tienden a tener muchos valores null y al no encontrar la respuesta en el mejor hit, se procede al segundo. Arreglar esto en el extractor? Hay muchos casos en los que un segundo match ayuda. \\

% Conclusiones
\chapter*{Conclusiones y recomendaciones} \label{chap:conclusiones}
\addcontentsline{toc}{chapter}{Conclusiones y recomendaciones}

En este capítulo se presentan los hallazgos y contribuciones de este trabajo y se dan algunas recomendaciones.\\


%\begin{table}[ht]
\caption{Resultados de la ejecución de la metaheurística GTS, utilizando instancias de Dethloff con la configuración XXX}
\centering
\begin{tabular}{c c c c c c}
\hline\hline
Instancia & Costo mínimo & Tiempo(seg.) & Costo promedio & Tiempo promedio(seg.) & Costo GTS \\ [0.5ex]
\hline
c0530.txt & \bf{636.06} & 0.88 & 638.305 & 1.07 & 636.06\\
c0531.txt & 700.50 & 2.53 & 703.7 & 1.6 & \bf{697.84}\\
c0532.txt & \bf{659.34} & 1.68 & 659.34 & 1.215 & 659.34\\
c0533.txt & 682.41 & 1.08 & 686.655 & 1.07 & \bf{680.04}\\
c0534.txt & \bf{690.50} & 1.09 & 690.5 & 1.42 & 690.50\\
c0535.txt & 672.94 & 1.42 & 672.94 & 1.96 & \bf{659.90}\\
c0536.txt & \bf{651.09} & 1.64 & 652.015 & 1.36 & 651.09\\
c0537.txt & 666.15 & 0.70 & 672.28 & 0.805 & \bf{659.17}\\
c0538.txt & \bf{719.47} & 2.31 & 719.47 & 2.745 & 719.47\\
c0539.txt & \bf{681.00} & 0.92 & 681 & 1.28 & 681.00\\
c0580.txt & 981.32 & 1.16 & 983.43 & 1.665 & \bf{961.50}\\
c0581.txt & 1074.02 & 1.14 & 1086.92 & 0.97 & \bf{1050.20}\\
c0582.txt & 1042.17 & 1.34 & 1046.27 & 0.985 & \bf{1039.64}\\
c0583.txt & 1018.04 & 0.61 & 1018.68 & 1.2 & \bf{983.34}\\
c0584.txt & 1067.55 & 2.08 & 1067.55 & 1.8 & \bf{1065.49}\\
c0585.txt & 1062.43 & 0.74 & 1063.66 & 1.03 & \bf{1027.08}\\
c0586.txt & 989.51 & 1.68 & 990.08 & 1.58 & \bf{971.82}\\
c0587.txt & 1066.65 & 1.31 & 1076.1 & 1.325 & \bf{1052.17}\\
c0588.txt & 1084.41 & 1.45 & 1084.41 & 1.105 & \bf{1071.18}\\
c0589.txt & 1070.34 & 1.75 & 1078.05 & 1.615 & \bf{1060.50}\\
cc0530.txt & 617.59 & 1.38 & 630.385 & 1.49 & \bf{616.52}\\
cc0531.txt & 556.04 & 1.20 & 557 & 1.14 & \bf{554.47}\\
cc0532.txt & 523.23 & 1.74 & 523.27 & 1.335 & \bf{519.26}\\
cc0533.txt & 591.20 & 2.01 & 606.565 & 1.47 & \bf{591.19}\\
cc0534.txt & 596.29 & 1.90 & 601.925 & 2.455 & \bf{589.32}\\
cc0535.txt & 572.58 & 0.58 & 572.58 & 0.955 & \bf{563.70}\\
cc0536.txt & \bf{\textit{499.05}} & 0.76 & 499.925 & 1.265 & 500.80\\
cc0537.txt & 578.41 & 1.07 & 580.265 & 0.98 & \bf{576.48}\\
cc0538.txt & \bf{523.05} & 0.66 & 523.095 & 1.3 & 523.05\\
cc0539.txt & \bf{\textit{578.25}} & 2.05 & 580.52 & 1.46 & 580.05\\
cc0580.txt & 891.02 & 1.34 & 899.045 & 2.3 & \bf{857.17}\\
cc0581.txt & \bf{740.85} & 2.72 & 744.905 & 1.8 & 740.85\\
cc0582.txt & 716.03 & 2.40 & 718.685 & 2.81 & \bf{713.44}\\
cc0583.txt & 838.58 & 1.50 & 840.695 & 1.64 & \bf{811.07}\\
cc0584.txt & \bf{772.25} & 1.13 & 782.965 & 1.725 & 772.25\\
cc0585.txt & 758.12 & 2.12 & 758.855 & 2.045 & \bf{756.91}\\
cc0586.txt & 702.42 & 1.14 & 710.475 & 1.2 & \bf{678.92}\\
cc0587.txt & 815.71 & 1.54 & 834.37 & 1.99 & \bf{811.96}\\
cc0588.txt & 782.34 & 1.33 & 784.41 & 1.41 & \bf{767.53}\\
cc0589.txt & 810.91 & 2.97 & 814.875 & 2.095 & \bf{809.00}\\
[1ex]\hline
\end{tabular}
\label{table:nonlin}
\end{table} \clearpage
\begin{table}[ht]
\caption{Resultados de la ejecución de la metaheurística GTS, utilizando instancias de Dethloff con la configuración XXX}
\centering
\begin{tabular}{c c c c c c}
\hline\hline
Instancia & Costo mínimo & Tiempo(seg.) & Costo promedio & Tiempo promedio(seg.) & Costo GTS \\ [0.5ex]
\hline
c0530.txt & \bf{636.06} & 3.11 & 638.13 & 2.748 & 636.06\\
c0531.txt & \bf{697.84} & 3.42 & 698.372 & 2.698 & 697.84\\
c0532.txt & \bf{659.34} & 5.01 & 659.34 & 3.832 & 659.34\\
c0533.txt & \bf{680.04} & 3.25 & 684.834 & 2.79 & 680.04\\
c0534.txt & \bf{690.50} & 1.56 & 690.5 & 2.226 & 690.50\\
c0535.txt & \bf{659.90} & 3.35 & 665.116 & 3.386 & 659.90\\
c0536.txt & \bf{651.09} & 1.36 & 653.284 & 2.206 & 651.09\\
c0537.txt & 666.15 & 3.09 & 667.646 & 2.736 & \bf{659.17}\\
c0538.txt & \bf{719.47} & 2.02 & 729.93 & 3.148 & 719.47\\
c0539.txt & \bf{681.00} & 4.09 & 681 & 2.938 & 681.00\\
c0580.txt & \bf{961.50} & 6.86 & 989.944 & 4.284 & 961.50\\
c0581.txt & \bf{1050.20} & 2.22 & 1057.63 & 3.072 & 1050.20\\
c0582.txt & \bf{1039.64} & 3.88 & 1054.9 & 3.79 & 1039.64\\
c0583.txt & 1012.77 & 3.41 & 1014.48 & 2.746 & \bf{983.34}\\
c0584.txt & 1067.28 & 4.04 & 1068.73 & 4.182 & \bf{1065.49}\\
c0585.txt & \bf{1027.08} & 2.39 & 1032.84 & 3.3 & 1027.08\\
c0586.txt & 972.48 & 1.53 & 977.756 & 2.682 & \bf{971.82}\\
c0587.txt & 1066.65 & 3.59 & 1075.31 & 2.912 & \bf{1052.17}\\
c0588.txt & 1080.58 & 1.88 & 1082.73 & 2.042 & \bf{1071.18}\\
c0589.txt & \bf{1060.50} & 3.12 & 1066.21 & 3.274 & 1060.50\\
cc0530.txt & \bf{616.52} & 3.70 & 626.294 & 2.808 & 616.52\\
cc0531.txt & \bf{554.47} & 3.06 & 558.104 & 2.854 & 554.47\\
cc0532.txt & 522.86 & 3.83 & 524.258 & 2.684 & \bf{519.26}\\
cc0533.txt & \bf{591.19} & 2.04 & 591.19 & 3.196 & 591.19\\
cc0534.txt & 596.29 & 2.91 & 596.858 & 2.782 & \bf{589.32}\\
cc0535.txt & \bf{563.70} & 1.43 & 570.226 & 2.806 & 563.70\\
cc0536.txt & \bf{\underline{499.05}} & 2.98 & 499.506 & 2.986 & 500.80\\
cc0537.txt & \bf{576.48} & 2.69 & 585.672 & 2.636 & 576.48\\
cc0538.txt & \bf{523.05} & 4.06 & 523.176 & 2.722 & 523.05\\
cc0539.txt & \bf{\underline{578.25}} & 1.76 & 582.954 & 4.476 & 580.05\\
cc0580.txt & 858.03 & 1.91 & 881.19 & 3.15 & \bf{857.17}\\
cc0581.txt & \bf{740.85} & 8.65 & 752.58 & 4.388 & 740.85\\
cc0582.txt & 715.59 & 3.22 & 736.96 & 3.212 & \bf{713.44}\\
cc0583.txt & \bf{811.07} & 2.00 & 819.996 & 3.124 & 811.07\\
cc0584.txt & \bf{772.25} & 2.01 & 781.31 & 2.066 & 772.25\\
cc0585.txt & \bf{756.91} & 1.94 & 762.676 & 3.276 & 756.91\\
cc0586.txt & 683.83 & 5.82 & 695.032 & 3.91 & \bf{678.92}\\
cc0587.txt & 813.00 & 3.03 & 832.39 & 2.626 & \bf{811.96}\\
cc0588.txt & \bf{767.53} & 3.56 & 777.356 & 3.334 & 767.53\\
cc0589.txt & \bf{809.00} & 2.00 & 834.95 & 3.238 & 809.00\\
[1ex]\hline
\end{tabular}
\label{table:nonlin}
\end{table} \clearpage
\begin{table}[ht]
\caption{Resultados de la ejecución de la metaheurística GTS, utilizando instancias de Dethloff con la configuración -h GTS -mni 6000 -lambda1 0.01 -lambda2 0.05 -tabu 20}
\centering
\begin{tabular}{c c c c c c}
\hline\hline
Instancia & Costo mínimo & Tiempo(seg.) & Costo promedio & Tiempo promedio(seg.) & Costo GTS \\ [0.5ex]
\hline
c0530.txt & \bf{636.06} & 4.53 & 638.305 & 4.7675 & 636.06\\
c0531.txt & \bf{697.84} & 6.91 & 697.84 & 6.1825 & 697.84\\
c0532.txt & \bf{659.34} & 6.88 & 659.34 & 4.765 & 659.34\\
c0533.txt & \bf{680.04} & 4.04 & 680.32 & 4.055 & 680.04\\
c0534.txt & \bf{690.50} & 4.55 & 690.5 & 4.7975 & 690.50\\
c0535.txt & \bf{659.90} & 4.10 & 663.16 & 4.9825 & 659.90\\
c0536.txt & \bf{651.09} & 4.35 & 651.553 & 3.275 & 651.09\\
c0537.txt & 666.15 & 8.20 & 666.15 & 7.4025 & \bf{659.17}\\
c0538.txt & \bf{719.47} & 6.71 & 719.47 & 9.41 & 719.47\\
c0539.txt & \bf{681.00} & 6.32 & 681 & 5.495 & 681.00\\
c0580.txt & \bf{961.50} & 4.57 & 970.377 & 4.49 & 961.50\\
c0581.txt & 1068.31 & 5.60 & 1070.3 & 5.015 & \bf{1050.20}\\
c0582.txt & 1050.37 & 3.88 & 1061.01 & 6.255 & \bf{1039.64}\\
c0583.txt & \bf{983.34} & 7.90 & 1002.05 & 5.39 & 983.34\\
c0584.txt & \bf{1065.49} & 4.22 & 1068.66 & 5.3775 & 1065.49\\
c0585.txt & 1040.66 & 6.56 & 1052.07 & 4.225 & \bf{1027.08}\\
c0586.txt & 972.48 & 2.92 & 972.48 & 3.925 & \bf{971.82}\\
c0587.txt & \bf{\underline{1051.28}} & 7.91 & 1059.26 & 6.3425 & 1052.17\\
c0588.txt & \bf{1071.18} & 2.40 & 1076.65 & 3.555 & 1071.18\\
c0589.txt & \bf{1060.50} & 7.06 & 1062.55 & 6.28 & 1060.50\\
cc0530.txt & \bf{616.52} & 5.20 & 623.558 & 3.3925 & 616.52\\
cc0531.txt & \bf{554.47} & 5.30 & 556.433 & 5.0975 & 554.47\\
cc0532.txt & 522.24 & 4.01 & 522.983 & 4.4775 & \bf{519.26}\\
cc0533.txt & \bf{591.19} & 7.16 & 601.778 & 5.11 & 591.19\\
cc0534.txt & \bf{\underline{588.79}} & 3.67 & 594.02 & 4.1075 & 589.32\\
cc0535.txt & \bf{563.70} & 3.84 & 572.263 & 3.5575 & 563.70\\
cc0536.txt & \bf{\underline{499.05}} & 8.29 & 500.265 & 6.1175 & 500.80\\
cc0537.txt & \bf{576.48} & 2.94 & 578.102 & 2.9775 & 576.48\\
cc0538.txt & \bf{523.05} & 2.91 & 523.207 & 3.6025 & 523.05\\
cc0539.txt & \bf{\underline{578.25}} & 3.26 & 585.308 & 5.54 & 580.05\\
cc0580.txt & 860.28 & 4.88 & 875.62 & 5.3475 & \bf{857.17}\\
cc0581.txt & 751.76 & 4.28 & 763.463 & 4.925 & \bf{740.85}\\
cc0582.txt & \bf{713.44} & 10.83 & 720.56 & 6.35 & 713.44\\
cc0583.txt & \bf{811.07} & 3.22 & 860.582 & 3.42 & 811.07\\
cc0584.txt & \bf{772.25} & 3.00 & 778.045 & 2.6825 & 772.25\\
cc0585.txt & \bf{\underline{754.88}} & 7.51 & 758.667 & 5.0975 & 756.91\\
cc0586.txt & 683.83 & 16.41 & 690.237 & 7.72 & \bf{678.92}\\
cc0587.txt & 812.89 & 5.63 & 816.472 & 4.905 & \bf{811.96}\\
cc0588.txt & 773.60 & 6.54 & 778.707 & 4.5075 & \bf{767.53}\\
cc0589.txt & \bf{809.00} & 6.23 & 811.59 & 4.855 & 809.00\\
[1ex]\hline
\end{tabular}
\label{table:nonlin}
\end{table} \clearpage
\begin{table}[ht]
\caption{Resultados de la ejecución de la metaheurística GTS, utilizando instancias de Dethloff con la configuración -mni 6000 -lambda1 0.10 -lambda2 0.05 -tabu 20}
\centering
\begin{tabular}{c c c c c c}
\hline\hline
Instancia & Costo mínimo & Tiempo(seg.) & Costo promedio & Tiempo promedio(seg.) & Costo GTS \\ [0.5ex]
\hline
c0530.txt & \bf{636.06} & 5.39 & 637.182 & 5.8875 & 636.06\\
c0531.txt & \bf{697.84} & 10.69 & 698.505 & 5.82 & 697.84\\
c0532.txt & \bf{659.34} & 8.55 & 659.34 & 5.1475 & 659.34\\
c0533.txt & \bf{680.04} & 8.24 & 682.755 & 4.8825 & 680.04\\
c0534.txt & \bf{690.50} & 6.67 & 690.5 & 6.57 & 690.50\\
c0535.txt & \bf{659.90} & 4.64 & 663.16 & 3.5275 & 659.90\\
c0536.txt & \bf{651.09} & 5.78 & 651.09 & 4.37 & 651.09\\
c0537.txt & 666.15 & 2.67 & 668.465 & 4.085 & \bf{659.17}\\
c0538.txt & \bf{719.47} & 3.25 & 719.47 & 5.3625 & 719.47\\
c0539.txt & \bf{681.00} & 2.56 & 681 & 3.4325 & 681.00\\
c0580.txt & \bf{961.50} & 9.52 & 972.877 & 10.6325 & 961.50\\
c0581.txt & \bf{\underline{1049.65}} & 7.14 & 1054.41 & 6.1425 & 1050.20\\
c0582.txt & 1042.10 & 3.36 & 1049.72 & 3.895 & \bf{1039.64}\\
c0583.txt & \bf{983.34} & 8.99 & 1003.83 & 7.46 & 983.34\\
c0584.txt & 1067.28 & 2.44 & 1071.14 & 3.83 & \bf{1065.49}\\
c0585.txt & \bf{1027.08} & 4.76 & 1049.93 & 4.1075 & 1027.08\\
c0586.txt & 972.48 & 4.28 & 977.023 & 5.0175 & \bf{971.82}\\
c0587.txt & 1060.98 & 4.65 & 1065.7 & 6.6725 & \bf{1052.17}\\
c0588.txt & \bf{1071.18} & 3.75 & 1076.65 & 4.2925 & 1071.18\\
c0589.txt & \bf{1060.50} & 5.04 & 1060.5 & 5.2725 & 1060.50\\
cc0530.txt & 616.81 & 3.73 & 625.3 & 3.465 & \bf{616.52}\\
cc0531.txt & \bf{554.47} & 5.03 & 557.61 & 6.165 & 554.47\\
cc0532.txt & 519.61 & 3.22 & 522.57 & 4.18 & \bf{519.26}\\
cc0533.txt & \bf{591.19} & 9.65 & 603.398 & 5.805 & 591.19\\
cc0534.txt & \bf{\underline{588.79}} & 3.09 & 588.79 & 4.035 & 589.32\\
cc0535.txt & \bf{563.70} & 8.27 & 570.97 & 5.045 & 563.70\\
cc0536.txt & \bf{\underline{499.05}} & 5.92 & 500.605 & 5.5575 & 500.80\\
cc0537.txt & \bf{576.48} & 2.90 & 586.502 & 3.9275 & 576.48\\
cc0538.txt & \bf{523.05} & 2.43 & 523.05 & 4.5075 & 523.05\\
cc0539.txt & \bf{\underline{578.25}} & 3.62 & 580.648 & 3.8575 & 580.05\\
cc0580.txt & 858.03 & 4.04 & 867.37 & 3.0425 & \bf{857.17}\\
cc0581.txt & 752.95 & 2.42 & 759.635 & 2.8875 & \bf{740.85}\\
cc0582.txt & \bf{\underline{712.89}} & 2.54 & 720.26 & 3.9275 & 713.44\\
cc0583.txt & 821.26 & 4.17 & 825.65 & 5.7975 & \bf{811.07}\\
cc0584.txt & \bf{772.25} & 6.48 & 786.112 & 5.2025 & 772.25\\
cc0585.txt & 757.23 & 2.54 & 760.092 & 3.5225 & \bf{756.91}\\
cc0586.txt & 688.47 & 6.26 & 700.25 & 4.2475 & \bf{678.92}\\
cc0587.txt & 812.89 & 7.52 & 813.872 & 7.04 & \bf{811.96}\\
cc0588.txt & \bf{767.53} & 5.53 & 774.497 & 4.9525 & 767.53\\
cc0589.txt & \bf{809.00} & 2.92 & 814.635 & 4.76 & 809.00\\
[1ex]\hline
\end{tabular}
\label{table:nonlin}
\end{table} \clearpage
\begin{table}[ht]
\caption{Resultados de la ejecución de la metaheurística GTS, utilizando instancias de Dethloff con la configuración -mni 6000 -lambda1 0.40 -lambda2 0.05 -tabu 20}
\centering
\begin{tabular}{c c c c c c}
\hline\hline
Instancia & Costo mínimo & Tiempo(seg.) & Costo promedio & Tiempo promedio(seg.) & Costo GTS \\ [0.5ex]
\hline
c0530.txt & \bf{636.06} & 12.86 & 638.305 & 6.56 & 636.06\\
c0531.txt & \bf{697.84} & 3.78 & 702.058 & 4.585 & 697.84\\
c0532.txt & \bf{659.34} & 3.73 & 659.34 & 4.5925 & 659.34\\
c0533.txt & \bf{680.04} & 5.53 & 680.18 & 5.8075 & 680.04\\
c0534.txt & \bf{690.50} & 3.96 & 698.28 & 4.735 & 690.50\\
c0535.txt & \bf{659.90} & 6.19 & 659.9 & 6.4625 & 659.90\\
c0536.txt & \bf{651.09} & 4.02 & 651.09 & 4.0625 & 651.09\\
c0537.txt & 666.15 & 3.94 & 668.02 & 4.435 & \bf{659.17}\\
c0538.txt & \bf{719.47} & 2.49 & 719.47 & 3.5075 & 719.47\\
c0539.txt & \bf{681.00} & 4.55 & 681 & 4.475 & 681.00\\
c0580.txt & \bf{961.50} & 6.63 & 981.893 & 4.15 & 961.50\\
c0581.txt & \bf{1050.20} & 7.86 & 1060.38 & 5.25 & 1050.20\\
c0582.txt & \bf{1039.64} & 3.36 & 1045.62 & 4.735 & 1039.64\\
c0583.txt & \bf{983.34} & 10.71 & 1002.08 & 6.675 & 983.34\\
c0584.txt & 1067.28 & 6.49 & 1069.7 & 6.6725 & \bf{1065.49}\\
c0585.txt & \bf{1027.08} & 4.53 & 1042.31 & 3.7225 & 1027.08\\
c0586.txt & 972.48 & 5.36 & 977.085 & 4.72 & \bf{971.82}\\
c0587.txt & 1061.78 & 6.10 & 1062.86 & 3.94 & \bf{1052.17}\\
c0588.txt & 1082.12 & 4.21 & 1083.86 & 3.975 & \bf{1071.18}\\
c0589.txt & \bf{1060.50} & 8.06 & 1061.3 & 5.35 & 1060.50\\
cc0530.txt & \bf{616.52} & 9.79 & 624.192 & 6.6725 & 616.52\\
cc0531.txt & \bf{554.47} & 8.04 & 556.725 & 5.93 & 554.47\\
cc0532.txt & 521.09 & 3.16 & 522.775 & 4.8775 & \bf{519.26}\\
cc0533.txt & \bf{591.19} & 8.38 & 591.193 & 6.5925 & 591.19\\
cc0534.txt & \bf{\underline{588.79}} & 4.62 & 590.11 & 5.335 & 589.32\\
cc0535.txt & \bf{563.70} & 2.45 & 571.852 & 3.335 & 563.70\\
cc0536.txt & \bf{\underline{499.05}} & 5.98 & 501.382 & 6.09 & 500.80\\
cc0537.txt & \bf{576.48} & 5.95 & 584.697 & 6.055 & 576.48\\
cc0538.txt & \bf{523.05} & 4.48 & 523.05 & 3.02 & 523.05\\
cc0539.txt & \bf{\underline{578.25}} & 3.20 & 579.567 & 4.2575 & 580.05\\
cc0580.txt & \bf{857.17} & 9.63 & 866.857 & 6.875 & 857.17\\
cc0581.txt & \bf{740.85} & 6.16 & 761.533 & 5.5425 & 740.85\\
cc0582.txt & \bf{\underline{712.89}} & 2.75 & 723.087 & 3.1725 & 713.44\\
cc0583.txt & \bf{811.07} & 7.05 & 818.638 & 4.605 & 811.07\\
cc0584.txt & \bf{772.25} & 2.44 & 775.278 & 4.67 & 772.25\\
cc0585.txt & \bf{756.91} & 7.08 & 758.955 & 6.1375 & 756.91\\
cc0586.txt & \bf{678.92} & 3.46 & 688.99 & 4.215 & 678.92\\
cc0587.txt & 813.91 & 3.49 & 814.215 & 4.9075 & \bf{811.96}\\
cc0588.txt & \bf{767.53} & 3.58 & 775.743 & 3.4975 & 767.53\\
cc0589.txt & \bf{809.00} & 4.34 & 821.612 & 4.7675 & 809.00\\
[1ex]\hline
\end{tabular}
\label{table:nonlin}
\end{table} \clearpage
\begin{table}[ht]
\caption{Resultados de la ejecución de la metaheurística GTS, utilizando instancias de Dethloff con la configuración -mni 6000 -lambda1 0.80 -lambda2 0.05 -tabu 20}
\centering
\begin{tabular}{c c c c c c}
\hline\hline
Instancia & Costo mínimo & Tiempo(seg.) & Costo promedio & Tiempo promedio(seg.) & Costo GTS \\ [0.5ex]
\hline
c0530.txt & \bf{\underline{635.62}} & 6.53 & 637.072 & 4.4 & 636.06\\
c0531.txt & \bf{697.84} & 7.50 & 697.84 & 6.495 & 697.84\\
c0532.txt & \bf{659.34} & 3.22 & 659.34 & 3.9875 & 659.34\\
c0533.txt & \bf{680.04} & 3.08 & 680.32 & 4.775 & 680.04\\
c0534.txt & \bf{690.50} & 8.65 & 690.5 & 5.98 & 690.50\\
c0535.txt & \bf{659.90} & 9.66 & 659.9 & 7.1425 & 659.90\\
c0536.txt & \bf{651.09} & 5.77 & 651.553 & 4.49 & 651.09\\
c0537.txt & 666.15 & 4.60 & 668.02 & 6.415 & \bf{659.17}\\
c0538.txt & \bf{719.47} & 6.32 & 719.47 & 7.845 & 719.47\\
c0539.txt & \bf{681.00} & 2.42 & 681 & 6.285 & 681.00\\
c0580.txt & \bf{961.50} & 4.38 & 972.477 & 6.065 & 961.50\\
c0581.txt & \bf{\underline{1049.65}} & 9.17 & 1062.63 & 4.515 & 1050.20\\
c0582.txt & \bf{1039.64} & 7.14 & 1051.15 & 4.9075 & 1039.64\\
c0583.txt & \bf{983.34} & 4.16 & 998.505 & 5.8275 & 983.34\\
c0584.txt & 1067.28 & 6.66 & 1070.01 & 6.495 & \bf{1065.49}\\
c0585.txt & \bf{1027.08} & 5.68 & 1036.79 & 5.5175 & 1027.08\\
c0586.txt & \bf{971.82} & 3.12 & 976.858 & 3.925 & 971.82\\
c0587.txt & \bf{\underline{1052.04}} & 3.74 & 1069.34 & 3.985 & 1052.17\\
c0588.txt & \bf{1071.18} & 6.80 & 1079.38 & 6.4525 & 1071.18\\
c0589.txt & \bf{1060.50} & 4.07 & 1064.68 & 5 & 1060.50\\
cc0530.txt & \bf{616.52} & 6.86 & 626.685 & 5.925 & 616.52\\
cc0531.txt & \bf{554.47} & 6.90 & 556.98 & 4.0975 & 554.47\\
cc0532.txt & 519.61 & 5.76 & 524.075 & 4.7425 & \bf{519.26}\\
cc0533.txt & \bf{591.19} & 3.01 & 591.19 & 5.8125 & 591.19\\
cc0534.txt & 589.88 & 4.20 & 593.085 & 5.1675 & \bf{589.32}\\
cc0535.txt & \bf{563.70} & 5.74 & 570.548 & 5.4325 & 563.70\\
cc0536.txt & \bf{\underline{499.05}} & 6.42 & 500.655 & 6.195 & 500.80\\
cc0537.txt & \bf{576.48} & 4.64 & 584.697 & 5.105 & 576.48\\
cc0538.txt & \bf{523.05} & 4.92 & 532.28 & 3.8525 & 523.05\\
cc0539.txt & \bf{\underline{578.25}} & 4.50 & 581.655 & 4.365 & 580.05\\
cc0580.txt & \bf{857.17} & 6.42 & 866.01 & 5.43 & 857.17\\
cc0581.txt & 751.76 & 2.97 & 760.09 & 3.665 & \bf{740.85}\\
cc0582.txt & \bf{\underline{712.89}} & 5.91 & 714.793 & 5.565 & 713.44\\
cc0583.txt & \bf{811.07} & 3.60 & 818.963 & 3.2675 & 811.07\\
cc0584.txt & \bf{772.25} & 8.07 & 775.395 & 4.42 & 772.25\\
cc0585.txt & \bf{\underline{755.67}} & 15.19 & 757.232 & 7.6375 & 756.91\\
cc0586.txt & \bf{678.92} & 7.62 & 687.957 & 5.8275 & 678.92\\
cc0587.txt & 812.89 & 3.82 & 815.957 & 4.9575 & \bf{811.96}\\
cc0588.txt & \bf{767.53} & 2.89 & 773.557 & 4.3725 & 767.53\\
cc0589.txt & \bf{809.00} & 4.04 & 819.092 & 4.8125 & 809.00\\
[1ex]\hline
\end{tabular}
\label{table:nonlin}
\end{table} \clearpage
\begin{table}[ht]
\caption{Resultados de la ejecución de la metaheurística GTS, utilizando instancias de Dethloff con la configuración -mni 6000 -lambda1 0.05 -lambda2 0.01 -tabu 20}
\centering
\begin{tabular}{c c c c c c}
\hline\hline
Instancia & Costo mínimo & Tiempo(seg.) & Costo promedio & Tiempo promedio(seg.) & Costo GTS \\ [0.5ex]
\hline
c0530.txt & \bf{636.06} & 1.99 & 639.428 & 2.0425 & 636.06\\
c0531.txt & \bf{697.84} & 2.10 & 699.17 & 3.9125 & 697.84\\
c0532.txt & \bf{659.34} & 2.98 & 659.34 & 4.3525 & 659.34\\
c0533.txt & \bf{680.04} & 2.08 & 686.062 & 2.865 & 680.04\\
c0534.txt & \bf{690.50} & 3.23 & 690.5 & 3.5775 & 690.50\\
c0535.txt & \bf{659.90} & 3.46 & 663.273 & 2.7425 & 659.90\\
c0536.txt & \bf{651.09} & 2.74 & 654.16 & 2.7625 & 651.09\\
c0537.txt & 666.15 & 7.10 & 667.085 & 4.965 & \bf{659.17}\\
c0538.txt & \bf{719.47} & 2.16 & 719.47 & 2.2025 & 719.47\\
c0539.txt & \bf{681.00} & 2.87 & 685.115 & 2.7175 & 681.00\\
c0580.txt & 970.64 & 4.07 & 977.065 & 3.445 & \bf{961.50}\\
c0581.txt & \bf{1050.20} & 5.74 & 1075.84 & 3.6225 & 1050.20\\
c0582.txt & \bf{1039.64} & 5.50 & 1049.72 & 4.1 & 1039.64\\
c0583.txt & \bf{983.34} & 4.56 & 989.468 & 7.32 & 983.34\\
c0584.txt & \bf{1065.49} & 7.18 & 1067.25 & 6.3075 & 1065.49\\
c0585.txt & \bf{1027.08} & 5.34 & 1045.03 & 3.3975 & 1027.08\\
c0586.txt & 972.48 & 3.98 & 980.75 & 4.595 & \bf{971.82}\\
c0587.txt & 1061.65 & 12.27 & 1075.66 & 6.2975 & \bf{1052.17}\\
c0588.txt & \bf{1071.18} & 6.12 & 1076.65 & 6.17 & 1071.18\\
c0589.txt & 1067.26 & 4.13 & 1073.68 & 3.03 & \bf{1060.50}\\
cc0530.txt & \bf{616.52} & 3.49 & 627.258 & 3.525 & 616.52\\
cc0531.txt & 556.04 & 1.81 & 557.57 & 3.405 & \bf{554.47}\\
cc0532.txt & 523.23 & 3.99 & 526.237 & 3.5575 & \bf{519.26}\\
cc0533.txt & \bf{591.19} & 4.81 & 594.58 & 4.1175 & 591.19\\
cc0534.txt & 591.43 & 3.14 & 593.355 & 2.4425 & \bf{589.32}\\
cc0535.txt & \bf{563.70} & 3.88 & 565.92 & 2.5625 & 563.70\\
cc0536.txt & \bf{\underline{499.05}} & 7.33 & 505.333 & 4.5125 & 500.80\\
cc0537.txt & \bf{576.48} & 10.14 & 583.01 & 5.0725 & 576.48\\
cc0538.txt & \bf{523.05} & 3.50 & 523.05 & 3.3425 & 523.05\\
cc0539.txt & 582.79 & 2.15 & 585.635 & 3.3825 & \bf{580.05}\\
cc0580.txt & \bf{857.17} & 6.62 & 877.515 & 7.67 & 857.17\\
cc0581.txt & 751.76 & 4.68 & 759.942 & 5.65 & \bf{740.85}\\
cc0582.txt & \bf{713.44} & 3.80 & 736.342 & 3.625 & 713.44\\
cc0583.txt & \bf{811.07} & 4.32 & 836.835 & 3.98 & 811.07\\
cc0584.txt & \bf{772.25} & 3.71 & 781.37 & 3.59 & 772.25\\
cc0585.txt & \bf{756.91} & 3.35 & 760.32 & 3.43 & 756.91\\
cc0586.txt & 698.64 & 5.56 & 710.202 & 4.28 & \bf{678.92}\\
cc0587.txt & 812.89 & 10.09 & 813.852 & 6.1625 & \bf{811.96}\\
cc0588.txt & \bf{767.53} & 2.00 & 774.295 & 2.4275 & 767.53\\
cc0589.txt & \bf{809.00} & 9.46 & 810.815 & 6.49 & 809.00\\
[1ex]\hline
\end{tabular}
\label{table:nonlin}
\end{table} \clearpage
\begin{table}[ht]
\caption{Resultados de la ejecución de la metaheurística GTS, utilizando instancias de Dethloff con la configuración -mni 6000 -lambda1 0.05 -lambda2 0.10 -tabu 20}
\centering
\begin{tabular}{c c c c c c}
\hline\hline
Instancia & Costo mínimo & Tiempo(seg.) & Costo promedio & Tiempo promedio(seg.) & Costo GTS \\ [0.5ex]
\hline
c0530.txt & \bf{636.06} & 2.99 & 639.428 & 5.0375 & 636.06\\
c0531.txt & \bf{697.84} & 3.80 & 697.84 & 5.395 & 697.84\\
c0532.txt & \bf{659.34} & 4.10 & 659.34 & 4.7575 & 659.34\\
c0533.txt & \bf{680.04} & 13.57 & 680.04 & 8.7975 & 680.04\\
c0534.txt & \bf{690.50} & 6.68 & 690.5 & 5.5625 & 690.50\\
c0535.txt & \bf{659.90} & 5.30 & 663.16 & 6.42 & 659.90\\
c0536.txt & \bf{651.09} & 3.54 & 651.09 & 4.7025 & 651.09\\
c0537.txt & \bf{659.17} & 3.35 & 664.405 & 7.0175 & 659.17\\
c0538.txt & \bf{719.47} & 7.26 & 721.898 & 5.875 & 719.47\\
c0539.txt & \bf{681.00} & 4.39 & 681 & 5.08 & 681.00\\
c0580.txt & \bf{961.50} & 10.22 & 968.092 & 5.875 & 961.50\\
c0581.txt & \bf{\underline{1049.65}} & 3.32 & 1059.87 & 4.8925 & 1050.20\\
c0582.txt & \bf{1039.64} & 4.50 & 1049.1 & 3.39 & 1039.64\\
c0583.txt & \bf{983.34} & 6.97 & 996.825 & 9.32 & 983.34\\
c0584.txt & \bf{1065.49} & 10.74 & 1067.77 & 6.515 & 1065.49\\
c0585.txt & \bf{1027.08} & 5.05 & 1044.86 & 5.4 & 1027.08\\
c0586.txt & 972.48 & 5.99 & 972.48 & 5.93 & \bf{971.82}\\
c0587.txt & \bf{1052.17} & 3.34 & 1063.69 & 3.74 & 1052.17\\
c0588.txt & \bf{1071.18} & 2.73 & 1079.96 & 3.2375 & 1071.18\\
c0589.txt & \bf{1060.50} & 3.24 & 1065.26 & 5.0125 & 1060.50\\
cc0530.txt & \bf{616.52} & 3.58 & 619.507 & 4.9475 & 616.52\\
cc0531.txt & \bf{554.47} & 4.90 & 556.178 & 5.15 & 554.47\\
cc0532.txt & 523.23 & 6.68 & 524.098 & 7.1 & \bf{519.26}\\
cc0533.txt & \bf{591.19} & 3.59 & 599.713 & 5.29 & 591.19\\
cc0534.txt & \bf{\underline{588.79}} & 3.16 & 591.597 & 6.75 & 589.32\\
cc0535.txt & \bf{563.70} & 11.25 & 567.423 & 5.53 & 563.70\\
cc0536.txt & \bf{\underline{499.05}} & 4.84 & 499.827 & 6.3575 & 500.80\\
cc0537.txt & \bf{576.48} & 4.06 & 580.773 & 4.805 & 576.48\\
cc0538.txt & \bf{523.05} & 4.52 & 523.05 & 5.51 & 523.05\\
cc0539.txt & \bf{\underline{578.25}} & 5.87 & 584.485 & 5.16 & 580.05\\
cc0580.txt & \bf{857.17} & 4.27 & 869.525 & 3.6275 & 857.17\\
cc0581.txt & 751.76 & 9.40 & 757.08 & 6.405 & \bf{740.85}\\
cc0582.txt & 718.64 & 9.90 & 728.878 & 5.7475 & \bf{713.44}\\
cc0583.txt & 812.54 & 3.94 & 829.337 & 4.6425 & \bf{811.07}\\
cc0584.txt & \bf{772.25} & 3.53 & 780.752 & 3.88 & 772.25\\
cc0585.txt & \bf{\underline{754.88}} & 4.68 & 757.762 & 6.2475 & 756.91\\
cc0586.txt & 690.63 & 6.73 & 694 & 4.5825 & \bf{678.92}\\
cc0587.txt & \bf{811.96} & 2.94 & 812.89 & 3.44 & 811.96\\
cc0588.txt & \bf{767.53} & 5.00 & 775.7 & 4.6825 & 767.53\\
cc0589.txt & 811.16 & 4.97 & 824.812 & 4.36 & \bf{809.00}\\
[1ex]\hline
\end{tabular}
\label{table:nonlin}
\end{table} \clearpage
\begin{table}[ht]
\caption{Resultados de la ejecución de la metaheurística GTS, utilizando instancias de Dethloff con la configuración -mni 6000 -lambda1 0.05 -lambda2 0.40 -tabu 20}
\centering
\begin{tabular}{c c c c c c}
\hline\hline
Instancia & Costo mínimo & Tiempo(seg.) & Costo promedio & Tiempo promedio(seg.) & Costo GTS \\ [0.5ex]
\hline
c0530.txt & \bf{\underline{635.62}} & 8.75 & 639.317 & 5.785 & 636.06\\
c0531.txt & \bf{697.84} & 5.69 & 698.505 & 5.3975 & 697.84\\
c0532.txt & \bf{659.34} & 6.01 & 659.34 & 8.6975 & 659.34\\
c0533.txt & \bf{680.04} & 4.30 & 682.895 & 4.2875 & 680.04\\
c0534.txt & \bf{690.50} & 6.45 & 695.88 & 6.0325 & 690.50\\
c0535.txt & \bf{659.90} & 6.90 & 663.16 & 5.16 & 659.90\\
c0536.txt & \bf{651.09} & 3.35 & 651.553 & 6.375 & 651.09\\
c0537.txt & 666.15 & 8.10 & 666.15 & 7.58 & \bf{659.17}\\
c0538.txt & \bf{719.47} & 6.70 & 719.47 & 5.41 & 719.47\\
c0539.txt & \bf{681.00} & 6.53 & 681 & 6.035 & 681.00\\
c0580.txt & \bf{961.50} & 3.11 & 973.752 & 3.7775 & 961.50\\
c0581.txt & \bf{\underline{1049.65}} & 4.85 & 1058.9 & 4.5525 & 1050.20\\
c0582.txt & 1042.10 & 4.86 & 1062.17 & 4.8075 & \bf{1039.64}\\
c0583.txt & \bf{983.34} & 13.18 & 996.825 & 6.9775 & 983.34\\
c0584.txt & \bf{1065.49} & 4.90 & 1070.1 & 5.255 & 1065.49\\
c0585.txt & 1034.32 & 3.20 & 1047.36 & 4.3375 & \bf{1027.08}\\
c0586.txt & 972.48 & 6.55 & 976.737 & 4.8125 & \bf{971.82}\\
c0587.txt & \bf{\underline{1051.28}} & 6.26 & 1059.89 & 5.5775 & 1052.17\\
c0588.txt & \bf{1071.18} & 2.61 & 1081.51 & 4.595 & 1071.18\\
c0589.txt & \bf{1060.50} & 8.16 & 1064.4 & 6.0575 & 1060.50\\
cc0530.txt & \bf{616.52} & 5.84 & 624.562 & 7.235 & 616.52\\
cc0531.txt & \bf{554.47} & 5.20 & 557.118 & 5.2275 & 554.47\\
cc0532.txt & 521.38 & 3.20 & 522.768 & 6.01 & \bf{519.26}\\
cc0533.txt & \bf{591.19} & 6.03 & 599.86 & 5.365 & 591.19\\
cc0534.txt & \bf{\underline{588.79}} & 6.70 & 593.91 & 7.065 & 589.32\\
cc0535.txt & \bf{563.70} & 4.19 & 567.207 & 6.3925 & 563.70\\
cc0536.txt & \bf{\underline{499.05}} & 13.00 & 501.895 & 7.69 & 500.80\\
cc0537.txt & \bf{576.48} & 8.53 & 581.995 & 5.365 & 576.48\\
cc0538.txt & \bf{523.05} & 6.94 & 523.05 & 6.2775 & 523.05\\
cc0539.txt & \bf{\underline{578.25}} & 7.73 & 583.115 & 5.8225 & 580.05\\
cc0580.txt & \bf{857.17} & 13.41 & 862.27 & 8.6025 & 857.17\\
cc0581.txt & \bf{740.85} & 3.00 & 744.762 & 3.115 & 740.85\\
cc0582.txt & \bf{713.44} & 2.78 & 719.942 & 4.7875 & 713.44\\
cc0583.txt & 821.26 & 5.45 & 829.153 & 3.795 & \bf{811.07}\\
cc0584.txt & \bf{772.25} & 3.36 & 779.55 & 4.3075 & 772.25\\
cc0585.txt & 758.12 & 6.17 & 759.31 & 4.9525 & \bf{756.91}\\
cc0586.txt & 685.49 & 3.77 & 690.11 & 5.61 & \bf{678.92}\\
cc0587.txt & 814.50 & 5.19 & 828.45 & 5.2625 & \bf{811.96}\\
cc0588.txt & \bf{767.53} & 7.72 & 774.825 & 6.3875 & 767.53\\
cc0589.txt & 811.16 & 7.13 & 821.815 & 5.545 & \bf{809.00}\\
[1ex]\hline
\end{tabular}
\label{table:nonlin}
\end{table} \clearpage
\begin{table}[ht]
\caption{Resultados de la ejecución de la metaheurística GTS, utilizando instancias de Dethloff con la configuración -mni 6000 -lambda1 0.05 -lambda2 0.80 -tabu 20}
\centering
\begin{tabular}{c c c c c c}
\hline\hline
Instancia & Costo mínimo & Tiempo(seg.) & Costo promedio & Tiempo promedio(seg.) & Costo GTS \\ [0.5ex]
\hline
c0530.txt & \bf{636.06} & 3.84 & 637.182 & 5.685 & 636.06\\
c0531.txt & \bf{697.84} & 6.59 & 697.84 & 8.425 & 697.84\\
c0532.txt & \bf{659.34} & 8.51 & 659.34 & 5.7425 & 659.34\\
c0533.txt & \bf{680.04} & 4.20 & 680.18 & 7.3625 & 680.04\\
c0534.txt & \bf{690.50} & 13.44 & 690.5 & 7.005 & 690.50\\
c0535.txt & \bf{659.90} & 3.69 & 663.273 & 3.4675 & 659.90\\
c0536.txt & \bf{651.09} & 5.29 & 651.738 & 6.31 & 651.09\\
c0537.txt & 666.15 & 6.94 & 667.085 & 5.315 & \bf{659.17}\\
c0538.txt & \bf{719.47} & 7.73 & 721.213 & 6.5575 & 719.47\\
c0539.txt & \bf{681.00} & 7.22 & 683.237 & 5.84 & 681.00\\
c0580.txt & \bf{961.50} & 4.48 & 971.143 & 5.8575 & 961.50\\
c0581.txt & \bf{\underline{1049.65}} & 6.97 & 1055.68 & 5.0075 & 1050.20\\
c0582.txt & 1064.23 & 3.19 & 1072.31 & 3.2775 & \bf{1039.64}\\
c0583.txt & \bf{983.34} & 13.12 & 1005.3 & 6.425 & 983.34\\
c0584.txt & 1067.55 & 2.90 & 1071.17 & 4.465 & \bf{1065.49}\\
c0585.txt & \bf{1027.08} & 9.20 & 1041.99 & 5.7325 & 1027.08\\
c0586.txt & 972.48 & 4.72 & 976.615 & 3.3275 & \bf{971.82}\\
c0587.txt & 1060.98 & 4.59 & 1069.16 & 3.8525 & \bf{1052.17}\\
c0588.txt & \bf{1071.18} & 5.73 & 1079.99 & 4.605 & 1071.18\\
c0589.txt & \bf{1060.50} & 5.63 & 1064.32 & 4.5125 & 1060.50\\
cc0530.txt & 619.09 & 4.10 & 625.795 & 3.91 & \bf{616.52}\\
cc0531.txt & 556.04 & 7.01 & 556.1 & 5.8175 & \bf{554.47}\\
cc0532.txt & 519.61 & 6.36 & 522.233 & 7.1325 & \bf{519.26}\\
cc0533.txt & \bf{591.19} & 7.59 & 591.19 & 5.34 & 591.19\\
cc0534.txt & 589.88 & 4.36 & 607.335 & 4.4975 & \bf{589.32}\\
cc0535.txt & \bf{563.70} & 9.51 & 565.368 & 6.6125 & 563.70\\
cc0536.txt & 500.82 & 5.98 & 501.825 & 5.495 & \bf{500.80}\\
cc0537.txt & \bf{576.48} & 6.19 & 580.84 & 6.0175 & 576.48\\
cc0538.txt & \bf{523.05} & 3.22 & 523.05 & 5.435 & 523.05\\
cc0539.txt & \bf{\underline{578.98}} & 8.64 & 582.83 & 8.88 & 580.05\\
cc0580.txt & 875.98 & 10.65 & 899.12 & 6.97 & \bf{857.17}\\
cc0581.txt & \bf{740.85} & 5.86 & 749.412 & 5.3375 & 740.85\\
cc0582.txt & \bf{713.44} & 3.26 & 720.32 & 3.98 & 713.44\\
cc0583.txt & \bf{811.07} & 3.96 & 825.248 & 3.8325 & 811.07\\
cc0584.txt & \bf{772.25} & 4.41 & 780.752 & 5.095 & 772.25\\
cc0585.txt & \bf{\underline{754.88}} & 8.49 & 758.735 & 6.7575 & 756.91\\
cc0586.txt & 683.83 & 6.17 & 688.838 & 4.8475 & \bf{678.92}\\
cc0587.txt & 813.00 & 5.18 & 813.648 & 6.75 & \bf{811.96}\\
cc0588.txt & \bf{767.53} & 12.32 & 772.935 & 6.9225 & 767.53\\
cc0589.txt & 811.16 & 2.49 & 819.945 & 4.38 & \bf{809.00}\\
[1ex]\hline
\end{tabular}
\label{table:nonlin}
\end{table} \clearpage
\begin{table}[ht]
\caption{Resultados de la ejecución de la metaheurística GTS, utilizando instancias de Dethloff con la configuración -mni 3000 -lambda1 0.05 -lambda2 0.05 -tabu 15}
\centering
\begin{tabular}{c c c c c c}
\hline\hline
Instancia & Costo mínimo & Tiempo(seg.) & Costo promedio & Tiempo promedio(seg.) & Costo GTS \\ [0.5ex]
\hline
c0530.txt & \bf{636.06} & 2.50 & 639.428 & 2.355 & 636.06\\
c0531.txt & \bf{697.84} & 1.34 & 698.505 & 1.7225 & 697.84\\
c0532.txt & \bf{659.34} & 4.05 & 659.34 & 2.835 & 659.34\\
c0533.txt & \bf{680.04} & 1.58 & 685.622 & 1.9325 & 680.04\\
c0534.txt & \bf{690.50} & 1.42 & 690.5 & 1.935 & 690.50\\
c0535.txt & \bf{659.90} & 1.79 & 666.42 & 1.97 & 659.90\\
c0536.txt & \bf{651.09} & 2.54 & 654.285 & 1.9 & 651.09\\
c0537.txt & 666.15 & 3.19 & 667.53 & 3.045 & \bf{659.17}\\
c0538.txt & \bf{719.47} & 3.31 & 722.305 & 2.2275 & 719.47\\
c0539.txt & \bf{681.00} & 3.62 & 681 & 1.7025 & 681.00\\
c0580.txt & 979.79 & 1.64 & 995.642 & 3.0175 & \bf{961.50}\\
c0581.txt & \bf{1050.20} & 2.05 & 1059.02 & 1.8875 & 1050.20\\
c0582.txt & 1042.10 & 2.31 & 1047.7 & 1.8325 & \bf{1039.64}\\
c0583.txt & 1010.50 & 2.58 & 1012.6 & 3.51 & \bf{983.34}\\
c0584.txt & 1070.75 & 2.35 & 1078.48 & 2.8275 & \bf{1065.49}\\
c0585.txt & 1042.30 & 1.84 & 1062.77 & 1.4025 & \bf{1027.08}\\
c0586.txt & 972.48 & 3.91 & 987.067 & 2.3775 & \bf{971.82}\\
c0587.txt & \bf{\underline{1052.04}} & 2.46 & 1072.11 & 2.685 & 1052.17\\
c0588.txt & \bf{1071.18} & 1.88 & 1081.12 & 1.51 & 1071.18\\
c0589.txt & \bf{1060.50} & 1.97 & 1067.4 & 2.2675 & 1060.50\\
cc0530.txt & \bf{616.52} & 5.28 & 625.707 & 2.895 & 616.52\\
cc0531.txt & 556.38 & 1.22 & 559.723 & 1.9825 & \bf{554.47}\\
cc0532.txt & 523.23 & 2.31 & 523.837 & 1.7775 & \bf{519.26}\\
cc0533.txt & \bf{591.19} & 5.30 & 594.582 & 3.515 & 591.19\\
cc0534.txt & 591.43 & 2.03 & 592.645 & 2.195 & \bf{589.32}\\
cc0535.txt & \bf{563.70} & 1.09 & 567.868 & 1.615 & 563.70\\
cc0536.txt & \bf{\underline{499.05}} & 2.14 & 501.712 & 2.165 & 500.80\\
cc0537.txt & \bf{576.48} & 1.85 & 582.72 & 2.36 & 576.48\\
cc0538.txt & \bf{523.05} & 1.63 & 523.05 & 1.8525 & 523.05\\
cc0539.txt & 588.11 & 2.28 & 592.207 & 1.9625 & \bf{580.05}\\
cc0580.txt & 857.40 & 5.76 & 874.567 & 3.38 & \bf{857.17}\\
cc0581.txt & 752.18 & 3.29 & 770.51 & 2.36 & \bf{740.85}\\
cc0582.txt & \bf{713.44} & 3.76 & 732.395 & 2.1375 & 713.44\\
cc0583.txt & \bf{811.07} & 2.27 & 826.798 & 1.74 & 811.07\\
cc0584.txt & \bf{772.25} & 1.91 & 776.013 & 2.1425 & 772.25\\
cc0585.txt & \bf{756.91} & 1.80 & 762.783 & 1.6375 & 756.91\\
cc0586.txt & 699.33 & 2.10 & 711.02 & 2.1275 & \bf{678.92}\\
cc0587.txt & 814.50 & 1.60 & 829.57 & 2.285 & \bf{811.96}\\
cc0588.txt & \bf{767.53} & 2.99 & 771.79 & 2.68 & 767.53\\
cc0589.txt & 811.16 & 1.85 & 815.417 & 2.145 & \bf{809.00}\\
[1ex]\hline
\end{tabular}
\label{table:nonlin}
\end{table} \clearpage
\begin{table}[ht]
\caption{Resultados de la ejecución de la metaheurística GTS, utilizando instancias de Dethloff con la configuración -mni 3500 -lambda1 0.05 -lambda2 0.05 -tabu 15}
\centering
\begin{tabular}{c c c c c c}
\hline\hline
Instancia & Costo mínimo & Tiempo(seg.) & Costo promedio & Tiempo promedio(seg.) & Costo GTS \\ [0.5ex]
\hline
c0530.txt & 640.55 & 2.04 & 641.023 & 2.4625 & \bf{636.06}\\
c0531.txt & \bf{697.84} & 2.61 & 699.17 & 1.9875 & 697.84\\
c0532.txt & \bf{659.34} & 3.38 & 659.34 & 3.1175 & 659.34\\
c0533.txt & \bf{680.04} & 1.53 & 680.46 & 2.34 & 680.04\\
c0534.txt & \bf{690.50} & 4.71 & 690.5 & 2.8025 & 690.50\\
c0535.txt & \bf{659.90} & 3.34 & 666.42 & 3.58 & 659.90\\
c0536.txt & \bf{651.09} & 1.89 & 653.638 & 2.1925 & 651.09\\
c0537.txt & 666.15 & 1.94 & 666.263 & 1.95 & \bf{659.17}\\
c0538.txt & \bf{719.47} & 1.73 & 724.228 & 1.5125 & 719.47\\
c0539.txt & \bf{681.00} & 4.40 & 681 & 2.5975 & 681.00\\
c0580.txt & 979.68 & 3.32 & 993.32 & 2.0625 & \bf{961.50}\\
c0581.txt & 1050.38 & 5.04 & 1060.06 & 3.05 & \bf{1050.20}\\
c0582.txt & 1050.37 & 1.98 & 1065.18 & 3.1375 & \bf{1039.64}\\
c0583.txt & \bf{983.34} & 4.90 & 1004.66 & 2.8375 & 983.34\\
c0584.txt & 1067.28 & 1.93 & 1071.73 & 2.8275 & \bf{1065.49}\\
c0585.txt & \bf{1027.08} & 3.69 & 1054.38 & 2.25 & 1027.08\\
c0586.txt & 972.48 & 2.18 & 972.48 & 2.615 & \bf{971.82}\\
c0587.txt & 1060.98 & 4.09 & 1070.74 & 2.8475 & \bf{1052.17}\\
c0588.txt & \bf{1071.18} & 2.28 & 1080.26 & 2.5425 & 1071.18\\
c0589.txt & \bf{1060.50} & 2.62 & 1065.28 & 3.4975 & 1060.50\\
cc0530.txt & \bf{616.52} & 2.72 & 628.942 & 2.2725 & 616.52\\
cc0531.txt & \bf{554.47} & 3.06 & 557.118 & 2.805 & 554.47\\
cc0532.txt & \bf{\underline{519.11}} & 5.38 & 521.738 & 4.3275 & 519.26\\
cc0533.txt & \bf{591.19} & 2.44 & 600.65 & 3.91 & 591.19\\
cc0534.txt & 589.88 & 4.93 & 593.472 & 2.6225 & \bf{589.32}\\
cc0535.txt & \bf{563.70} & 1.89 & 569.428 & 3.245 & 563.70\\
cc0536.txt & \bf{\underline{499.05}} & 4.64 & 501.433 & 2.865 & 500.80\\
cc0537.txt & \bf{576.48} & 3.78 & 587.33 & 2.5475 & 576.48\\
cc0538.txt & \bf{523.05} & 2.50 & 523.05 & 2.5625 & 523.05\\
cc0539.txt & \bf{\underline{578.25}} & 4.48 & 585.038 & 3.7075 & 580.05\\
cc0580.txt & 871.29 & 3.27 & 909.185 & 2.79 & \bf{857.17}\\
cc0581.txt & 751.76 & 1.48 & 753.975 & 2.6675 & \bf{740.85}\\
cc0582.txt & \bf{713.44} & 2.06 & 719.008 & 2.3775 & 713.44\\
cc0583.txt & 814.73 & 1.51 & 829.528 & 2.5975 & \bf{811.07}\\
cc0584.txt & \bf{772.25} & 2.92 & 808.158 & 2.41 & 772.25\\
cc0585.txt & \bf{756.91} & 1.76 & 759.225 & 2.4475 & 756.91\\
cc0586.txt & 683.83 & 2.64 & 688.688 & 2.1925 & \bf{678.92}\\
cc0587.txt & 812.89 & 3.61 & 826.527 & 2.4125 & \bf{811.96}\\
cc0588.txt & \bf{767.53} & 1.96 & 778.303 & 2.03 & 767.53\\
cc0589.txt & \bf{809.00} & 4.17 & 814.67 & 3.71 & 809.00\\
[1ex]\hline
\end{tabular}
\label{table:nonlin}
\end{table} \clearpage
\begin{table}[ht]
\caption{Resultados de la ejecución de la metaheurística GTS, utilizando instancias de Dethloff con la configuración -mni 4000 -lambda1 0.05 -lambda2 0.05 -tabu 15}
\centering
\begin{tabular}{c c c c c c}
\hline\hline
Instancia & Costo mínimo & Tiempo(seg.) & Costo promedio & Tiempo promedio(seg.) & Costo GTS \\ [0.5ex]
\hline
c0530.txt & 640.55 & 1.42 & 649.585 & 3.265 & \bf{636.06}\\
c0531.txt & \bf{697.84} & 2.31 & 698.505 & 2.19 & 697.84\\
c0532.txt & \bf{659.34} & 3.52 & 659.34 & 2.685 & 659.34\\
c0533.txt & 680.60 & 1.65 & 683.175 & 3.515 & \bf{680.04}\\
c0534.txt & \bf{690.50} & 2.94 & 690.5 & 2.4 & 690.50\\
c0535.txt & \bf{659.90} & 3.62 & 659.9 & 3.2 & 659.90\\
c0536.txt & \bf{651.09} & 2.74 & 653.942 & 3.1125 & 651.09\\
c0537.txt & 666.15 & 3.60 & 668.02 & 3.29 & \bf{659.17}\\
c0538.txt & \bf{719.47} & 1.94 & 719.47 & 3.8625 & 719.47\\
c0539.txt & \bf{681.00} & 1.42 & 681 & 2.4 & 681.00\\
c0580.txt & \bf{961.50} & 1.84 & 975.87 & 2.62 & 961.50\\
c0581.txt & \bf{1050.20} & 3.42 & 1065.17 & 3.34 & 1050.20\\
c0582.txt & \bf{1039.64} & 2.01 & 1051.39 & 2.6825 & 1039.64\\
c0583.txt & \bf{983.34} & 4.80 & 997.488 & 5.2875 & 983.34\\
c0584.txt & 1070.75 & 2.42 & 1074.17 & 2.125 & \bf{1065.49}\\
c0585.txt & \bf{1027.08} & 2.81 & 1047.74 & 3.3575 & 1027.08\\
c0586.txt & 972.48 & 2.74 & 976.615 & 2.35 & \bf{971.82}\\
c0587.txt & 1069.46 & 3.90 & 1078.87 & 3.12 & \bf{1052.17}\\
c0588.txt & 1082.12 & 3.28 & 1083.37 & 2.5125 & \bf{1071.18}\\
c0589.txt & \bf{1060.50} & 3.35 & 1067.27 & 3.5775 & 1060.50\\
cc0530.txt & 617.59 & 2.52 & 625.42 & 2.37 & \bf{616.52}\\
cc0531.txt & \bf{554.47} & 4.52 & 558.003 & 2.4175 & 554.47\\
cc0532.txt & 522.86 & 2.69 & 523.587 & 2.995 & \bf{519.26}\\
cc0533.txt & \bf{591.19} & 2.86 & 591.19 & 5.3225 & 591.19\\
cc0534.txt & \bf{\underline{588.79}} & 4.94 & 595.615 & 3.5625 & 589.32\\
cc0535.txt & \bf{563.70} & 1.70 & 572.93 & 3.2125 & 563.70\\
cc0536.txt & \bf{\underline{499.05}} & 3.00 & 499.827 & 2.425 & 500.80\\
cc0537.txt & \bf{576.48} & 3.22 & 577.44 & 3.3025 & 576.48\\
cc0538.txt & \bf{523.05} & 1.49 & 523.365 & 1.94 & 523.05\\
cc0539.txt & 582.79 & 3.68 & 586.013 & 3.0525 & \bf{580.05}\\
cc0580.txt & 860.28 & 2.98 & 901.43 & 2.5575 & \bf{857.17}\\
cc0581.txt & 756.50 & 2.66 & 758.43 & 4.1875 & \bf{740.85}\\
cc0582.txt & \bf{713.44} & 5.10 & 719.86 & 3.2875 & 713.44\\
cc0583.txt & 821.26 & 3.68 & 832.947 & 2.7675 & \bf{811.07}\\
cc0584.txt & \bf{772.25} & 4.77 & 781.445 & 4.0925 & 772.25\\
cc0585.txt & \bf{\underline{754.88}} & 4.95 & 762.24 & 3.295 & 756.91\\
cc0586.txt & 690.63 & 3.24 & 694.45 & 3.45 & \bf{678.92}\\
cc0587.txt & 812.89 & 6.20 & 833.365 & 4.825 & \bf{811.96}\\
cc0588.txt & 773.60 & 2.76 & 778.255 & 3.23 & \bf{767.53}\\
cc0589.txt & \bf{809.00} & 5.44 & 811.812 & 3.3975 & 809.00\\
[1ex]\hline
\end{tabular}
\label{table:nonlin}
\end{table} \clearpage
\begin{table}[ht]
\caption{Resultados de la ejecución de la metaheurística GTS, utilizando instancias de Dethloff con la configuración -mni 6000 -lambda1 0.05 -lambda2 0.05 -tabu 15}
\centering
\begin{tabular}{c c c c c c}
\hline\hline
Instancia & Costo mínimo & Tiempo(seg.) & Costo promedio & Tiempo promedio(seg.) & Costo GTS \\ [0.5ex]
\hline
c0530.txt & \bf{636.06} & 7.23 & 647.428 & 4.8875 & 636.06\\
c0531.txt & \bf{697.84} & 2.71 & 698.505 & 3.3775 & 697.84\\
c0532.txt & \bf{659.34} & 7.90 & 659.34 & 5.54 & 659.34\\
c0533.txt & \bf{680.04} & 6.80 & 682.622 & 5.3975 & 680.04\\
c0534.txt & \bf{690.50} & 6.30 & 690.5 & 5.1975 & 690.50\\
c0535.txt & \bf{659.90} & 5.60 & 659.9 & 5.075 & 659.90\\
c0536.txt & \bf{651.09} & 5.01 & 651.09 & 6.1225 & 651.09\\
c0537.txt & 666.15 & 4.18 & 666.15 & 4.7425 & \bf{659.17}\\
c0538.txt & \bf{719.47} & 8.97 & 719.47 & 5.2025 & 719.47\\
c0539.txt & \bf{681.00} & 6.45 & 681 & 6.1725 & 681.00\\
c0580.txt & \bf{961.50} & 6.40 & 970.227 & 4.8225 & 961.50\\
c0581.txt & \bf{\underline{1049.65}} & 8.63 & 1063.14 & 4.7925 & 1050.20\\
c0582.txt & 1050.37 & 3.79 & 1053.35 & 3.3325 & \bf{1039.64}\\
c0583.txt & \bf{983.34} & 8.14 & 1001.02 & 4.6375 & 983.34\\
c0584.txt & 1067.28 & 8.11 & 1068.57 & 4.7275 & \bf{1065.49}\\
c0585.txt & \bf{1027.08} & 2.63 & 1034.88 & 6.5625 & 1027.08\\
c0586.txt & \bf{971.82} & 3.01 & 976.857 & 4.51 & 971.82\\
c0587.txt & 1066.65 & 4.87 & 1070.76 & 4.0925 & \bf{1052.17}\\
c0588.txt & \bf{1071.18} & 1.96 & 1088.64 & 2.4125 & 1071.18\\
c0589.txt & \bf{1060.50} & 3.87 & 1064.76 & 5.3025 & 1060.50\\
cc0530.txt & \bf{616.52} & 11.24 & 623.06 & 6.605 & 616.52\\
cc0531.txt & \bf{554.47} & 7.24 & 557.61 & 4.8375 & 554.47\\
cc0532.txt & 522.90 & 6.63 & 523.205 & 5.29 & \bf{519.26}\\
cc0533.txt & \bf{591.19} & 4.24 & 594.58 & 4.7625 & 591.19\\
cc0534.txt & 591.43 & 2.12 & 596.677 & 3.1625 & \bf{589.32}\\
cc0535.txt & \bf{563.70} & 2.09 & 571.62 & 6.5275 & 563.70\\
cc0536.txt & \bf{\underline{499.05}} & 7.24 & 501.787 & 5.9125 & 500.80\\
cc0537.txt & \bf{576.48} & 5.96 & 576.48 & 5.7975 & 576.48\\
cc0538.txt & \bf{523.05} & 2.28 & 523.207 & 3.0125 & 523.05\\
cc0539.txt & \bf{\underline{578.25}} & 6.92 & 590.447 & 5.7725 & 580.05\\
cc0580.txt & 875.52 & 2.56 & 891.03 & 4.4 & \bf{857.17}\\
cc0581.txt & \bf{740.85} & 4.14 & 747.482 & 4.8125 & 740.85\\
cc0582.txt & \bf{713.44} & 2.14 & 720.175 & 5.455 & 713.44\\
cc0583.txt & \bf{811.07} & 7.08 & 841.9 & 5.13 & 811.07\\
cc0584.txt & \bf{772.25} & 4.04 & 786.795 & 3.1775 & 772.25\\
cc0585.txt & \bf{756.91} & 5.70 & 800.847 & 4.14 & 756.91\\
cc0586.txt & 688.68 & 7.06 & 694.582 & 5.3875 & \bf{678.92}\\
cc0587.txt & 812.70 & 6.96 & 814.263 & 4.0325 & \bf{811.96}\\
cc0588.txt & \bf{767.53} & 7.95 & 780.055 & 5.02 & 767.53\\
cc0589.txt & 811.14 & 3.21 & 814.822 & 4.555 & \bf{809.00}\\
[1ex]\hline
\end{tabular}
\label{table:nonlin}
\end{table} \clearpage
\begin{table}[ht]
\caption{Resultados de la ejecución de la metaheurística GTS, utilizando instancias de Dethloff con la configuración -mni 6000 -lambda1 0.05 -lambda2 0.05 -tabu 20}
\centering
\begin{tabular}{c c c c c c}
\hline\hline
Instancia & Costo mínimo & Tiempo(seg.) & Costo promedio & Tiempo promedio(seg.) & Costo GTS \\ [0.5ex]
\hline
c0530.txt & \bf{636.06} & 5.89 & 637.182 & 5.85 & 636.06\\
c0531.txt & \bf{697.84} & 2.86 & 697.84 & 6.0575 & 697.84\\
c0532.txt & \bf{659.34} & 6.69 & 659.34 & 4.26 & 659.34\\
c0533.txt & 680.60 & 8.21 & 682.903 & 4.6775 & \bf{680.04}\\
c0534.txt & \bf{690.50} & 8.57 & 690.5 & 5.545 & 690.50\\
c0535.txt & \bf{659.90} & 3.25 & 666.42 & 3.935 & 659.90\\
c0536.txt & \bf{651.09} & 4.33 & 656.488 & 4.0775 & 651.09\\
c0537.txt & 666.15 & 4.98 & 667.085 & 3.6725 & \bf{659.17}\\
c0538.txt & \bf{719.47} & 2.80 & 719.47 & 4.26 & 719.47\\
c0539.txt & \bf{681.00} & 5.46 & 681 & 5.045 & 681.00\\
c0580.txt & \bf{961.50} & 6.96 & 967.82 & 5.26 & 961.50\\
c0581.txt & 1068.31 & 2.66 & 1070.07 & 3.9725 & \bf{1050.20}\\
c0582.txt & 1042.17 & 8.55 & 1051.78 & 4.6725 & \bf{1039.64}\\
c0583.txt & \bf{983.34} & 5.20 & 1005.33 & 3.5525 & 983.34\\
c0584.txt & \bf{1065.49} & 7.00 & 1069.8 & 5.225 & 1065.49\\
c0585.txt & \bf{1027.08} & 2.28 & 1033.78 & 3.2725 & 1027.08\\
c0586.txt & 972.48 & 3.24 & 977.023 & 3.3975 & \bf{971.82}\\
c0587.txt & \bf{\underline{1051.28}} & 12.42 & 1060.49 & 6.555 & 1052.17\\
c0588.txt & 1082.12 & 3.99 & 1085.07 & 3.76 & \bf{1071.18}\\
c0589.txt & \bf{1060.50} & 6.55 & 1064.4 & 6.485 & 1060.50\\
cc0530.txt & \bf{616.52} & 9.95 & 624.415 & 7.975 & 616.52\\
cc0531.txt & \bf{554.47} & 5.13 & 556.315 & 4.22 & 554.47\\
cc0532.txt & 519.61 & 3.25 & 522.233 & 6.6475 & \bf{519.26}\\
cc0533.txt & \bf{591.19} & 4.20 & 591.19 & 4.7 & 591.19\\
cc0534.txt & \bf{\underline{588.79}} & 4.31 & 593.2 & 4.3175 & 589.32\\
cc0535.txt & \bf{563.70} & 6.14 & 565.915 & 6.39 & 563.70\\
cc0536.txt & \bf{\underline{499.05}} & 3.56 & 499.38 & 3.475 & 500.80\\
cc0537.txt & \bf{576.48} & 5.57 & 576.962 & 5.375 & 576.48\\
cc0538.txt & \bf{523.05} & 2.69 & 523.05 & 3.6025 & 523.05\\
cc0539.txt & \bf{\underline{578.25}} & 9.85 & 580.495 & 5.15 & 580.05\\
cc0580.txt & \bf{857.17} & 5.08 & 866.695 & 3.98 & 857.17\\
cc0581.txt & \bf{740.85} & 2.74 & 753.865 & 4.52 & 740.85\\
cc0582.txt & \bf{713.44} & 5.07 & 719.04 & 5.265 & 713.44\\
cc0583.txt & 825.94 & 5.59 & 830.66 & 4.0525 & \bf{811.07}\\
cc0584.txt & 783.99 & 3.54 & 786.885 & 4.175 & \bf{772.25}\\
cc0585.txt & \bf{\underline{754.88}} & 4.47 & 757.778 & 5.0725 & 756.91\\
cc0586.txt & 690.58 & 6.01 & 692.905 & 5.63 & \bf{678.92}\\
cc0587.txt & 812.89 & 4.54 & 816.41 & 5.0825 & \bf{811.96}\\
cc0588.txt & \bf{767.53} & 3.60 & 776.125 & 3.155 & 767.53\\
cc0589.txt & 811.16 & 5.37 & 818.985 & 5.98 & \bf{809.00}\\
[1ex]\hline
\end{tabular}
\label{table:nonlin}
\end{table} \clearpage
\begin{table}[ht]
\caption{Resultados de la ejecución de la metaheurística GTS, utilizando instancias de Dethloff con la configuración -mni 100 -lambda1 0.05 -lambda2 0.05 -tabu 15}
\centering
\begin{tabular}{c c c c c c}
\hline\hline
Instancia & Costo mínimo & Tiempo(seg.) & Costo promedio & Tiempo promedio(seg.) & Costo GTS \\ [0.5ex]
\hline
c0530.txt & 642.44 & 0.15 & 642.44 & 0.1525 & \bf{636.06}\\
c0531.txt & 728.57 & 0.12 & 728.57 & 0.1325 & \bf{697.84}\\
c0532.txt & 664.18 & 0.19 & 680.742 & 0.17 & \bf{659.34}\\
c0533.txt & 681.16 & 0.15 & 688.45 & 0.1475 & \bf{680.04}\\
c0534.txt & \bf{690.50} & 0.14 & 702.56 & 0.135 & 690.50\\
c0535.txt & 687.15 & 0.13 & 695.062 & 0.115 & \bf{659.90}\\
c0536.txt & 661.28 & 0.15 & 667.055 & 0.1375 & \bf{651.09}\\
c0537.txt & 671.77 & 0.12 & 672.115 & 0.1375 & \bf{659.17}\\
c0538.txt & 724.28 & 0.12 & 732.732 & 0.15 & \bf{719.47}\\
c0539.txt & 689.95 & 0.18 & 699.807 & 0.1675 & \bf{681.00}\\
c0580.txt & 1013.42 & 0.15 & 1033.75 & 0.1675 & \bf{961.50}\\
c0581.txt & 1099.93 & 0.12 & 1100.69 & 0.1375 & \bf{1050.20}\\
c0582.txt & 1072.93 & 0.28 & 1129.72 & 0.25 & \bf{1039.64}\\
c0583.txt & 1013.56 & 0.19 & 1051.27 & 0.165 & \bf{983.34}\\
c0584.txt & 1111.67 & 0.18 & 1147.32 & 0.1775 & \bf{1065.49}\\
c0585.txt & 1042.97 & 0.25 & 1056.7 & 0.2375 & \bf{1027.08}\\
c0586.txt & 976.37 & 0.20 & 1016.98 & 0.165 & \bf{971.82}\\
c0587.txt & 1087.13 & 0.17 & 1120.05 & 0.1875 & \bf{1052.17}\\
c0588.txt & 1084.41 & 0.23 & 1085.54 & 0.165 & \bf{1071.18}\\
c0589.txt & 1071.79 & 0.36 & 1097.19 & 0.3025 & \bf{1060.50}\\
cc0530.txt & 633.24 & 0.16 & 635.045 & 0.1625 & \bf{616.52}\\
cc0531.txt & 561.16 & 0.18 & 569.8 & 0.16 & \bf{554.47}\\
cc0532.txt & 528.10 & 0.14 & 528.332 & 0.225 & \bf{519.26}\\
cc0533.txt & 591.20 & 0.17 & 613.743 & 0.165 & \bf{591.19}\\
cc0534.txt & 595.25 & 0.18 & 599.383 & 0.1625 & \bf{589.32}\\
cc0535.txt & \bf{563.70} & 0.13 & 578.835 & 0.1675 & 563.70\\
cc0536.txt & 505.82 & 0.19 & 515.098 & 0.1875 & \bf{500.80}\\
cc0537.txt & 586.01 & 0.17 & 587.97 & 0.135 & \bf{576.48}\\
cc0538.txt & \bf{523.05} & 0.22 & 528.665 & 0.1775 & 523.05\\
cc0539.txt & 591.24 & 0.14 & 601.15 & 0.175 & \bf{580.05}\\
cc0580.txt & 904.17 & 0.41 & 914.633 & 0.295 & \bf{857.17}\\
cc0581.txt & 765.40 & 0.16 & 807.928 & 0.245 & \bf{740.85}\\
cc0582.txt & 737.13 & 0.26 & 741.625 & 0.235 & \bf{713.44}\\
cc0583.txt & 830.07 & 0.16 & 839.493 & 0.165 & \bf{811.07}\\
cc0584.txt & 804.11 & 0.18 & 842.038 & 0.18 & \bf{772.25}\\
cc0585.txt & 760.03 & 0.26 & 784.467 & 0.2425 & \bf{756.91}\\
cc0586.txt & 705.37 & 0.34 & 711.228 & 0.2325 & \bf{678.92}\\
cc0587.txt & 815.54 & 0.30 & 853.095 & 0.2175 & \bf{811.96}\\
cc0588.txt & 815.10 & 0.17 & 828.337 & 0.1775 & \bf{767.53}\\
cc0589.txt & 812.03 & 0.22 & 879.28 & 0.19 & \bf{809.00}\\
[1ex]\hline
\end{tabular}
\label{table:nonlin}
\end{table} \clearpage
\begin{table}[ht]
\caption{Resultados de la ejecución de la metaheurística ILS, utilizando instancias de Dethloff con la configuración -n 15 -LS 80 -y 0.5}
\centering
\begin{tabular}{c c c c c c}
\hline\hline
Instancia & Costo mínimo & Tiempo(seg.) & Costo promedio & Tiempo promedio(seg.) & Costo ILS \\ [0.5ex]
\hline
c0530.txt & 636.06 & 4.18 & 640.77 & 4.1725 & \bf{635.62}\\
c0531.txt & \bf{697.84} & 4.19 & 700.092 & 4.84 & 697.84\\
c0532.txt & 664.21 & 4.48 & 669.303 & 4.4625 & \bf{659.34}\\
c0533.txt & 681.16 & 5.10 & 687.605 & 4.75 & \bf{680.04}\\
c0534.txt & \bf{690.50} & 4.12 & 691.183 & 4.365 & 690.50\\
c0535.txt & 665.04 & 2.94 & 674.27 & 4.135 & \bf{659.90}\\
c0536.txt & \bf{651.09} & 4.48 & 657.942 & 3.9975 & 651.09\\
c0537.txt & 671.67 & 4.04 & 671.695 & 4.395 & \bf{659.17}\\
c0538.txt & \bf{719.47} & 5.19 & 723.807 & 4.6275 & 719.47\\
c0539.txt & 689.95 & 3.93 & 694.773 & 4.65 & \bf{681.00}\\
c0580.txt & 1010.18 & 3.53 & 1022.11 & 3.575 & \bf{961.50}\\
c0581.txt & 1076.39 & 4.32 & 1092.36 & 3.765 & \bf{1049.65}\\
c0582.txt & 1052.94 & 4.14 & 1065.32 & 4.215 & \bf{1039.64}\\
c0583.txt & 1005.90 & 4.07 & 1034.68 & 3.7125 & \bf{983.34}\\
c0584.txt & 1109.65 & 4.10 & 1120.7 & 3.8775 & \bf{1065.49}\\
c0585.txt & 1065.10 & 4.44 & 1074.49 & 4.52 & \bf{1027.08}\\
c0586.txt & 992.65 & 3.51 & 997.602 & 3.84 & \bf{971.82}\\
c0587.txt & 1081.65 & 3.96 & 1085.74 & 3.8575 & \bf{1051.28}\\
c0588.txt & \bf{1071.18} & 3.96 & 1093.71 & 3.475 & 1071.18\\
c0589.txt & 1101.45 & 3.68 & 1109 & 3.535 & \bf{1060.50}\\
cc0530.txt & 632.57 & 5.10 & 634.288 & 4.6775 & \bf{616.52}\\
cc0531.txt & 557.21 & 4.89 & 560.423 & 4.7175 & \bf{554.47}\\
cc0532.txt & 521.38 & 4.98 & 524.013 & 4.58 & \bf{518.00}\\
cc0533.txt & 591.48 & 4.16 & 595.788 & 4.29 & \bf{591.19}\\
cc0534.txt & 591.43 & 4.91 & 596.36 & 5.2425 & \bf{588.79}\\
cc0535.txt & 569.88 & 4.57 & 580.4 & 4.17 & \bf{563.70}\\
cc0536.txt & 508.05 & 4.58 & 511.487 & 4.2725 & \bf{499.05}\\
cc0537.txt & 578.41 & 5.06 & 588.315 & 4.535 & \bf{576.48}\\
cc0538.txt & 524.59 & 4.36 & 534.25 & 4.4825 & \bf{523.05}\\
cc0539.txt & 589.57 & 4.10 & 590.59 & 4.42 & \bf{578.24}\\
cc0580.txt & 894.75 & 4.42 & 906.475 & 3.8125 & \bf{857.17}\\
cc0581.txt & 758.51 & 3.68 & 772.592 & 3.8525 & \bf{740.85}\\
cc0582.txt & 732.47 & 3.50 & 740.54 & 3.83 & \bf{712.89}\\
cc0583.txt & 833.81 & 4.46 & 853.533 & 4.23 & \bf{811.07}\\
cc0584.txt & 781.78 & 3.37 & 797.53 & 3.6375 & \bf{772.25}\\
cc0585.txt & 766.55 & 4.56 & 780.183 & 3.9425 & \bf{754.88}\\
cc0586.txt & 708.84 & 3.56 & 711.372 & 3.825 & \bf{678.92}\\
cc0587.txt & 815.79 & 3.42 & 832.93 & 3.8925 & \bf{811.96}\\
cc0588.txt & 787.59 & 2.97 & 794.138 & 3.48 & \bf{767.53}\\
cc0589.txt & 839.37 & 2.87 & 845.09 & 3.18 & \bf{809.00}\\
[1ex]\hline
\end{tabular}
\label{table:nonlin}
\end{table} \clearpage
\begin{table}[ht]
\caption{Resultados de la ejecución de la metaheurística ACO, utilizando instancias de Dethloff con la configuración -n 20 -alpha 1.0 -beta 3.0 -q 0.8 -ro 0.015}
\centering
\begin{tabular}{c c c c c c}
\hline\hline
Instancia & Costo mínimo & Tiempo(seg.) & Costo promedio & Tiempo promedio(seg.) & Costo ACO \\ [0.5ex]
\hline
c0530.txt & \bf{\underline{636.06}} & 14.73 & 636.06 & 14.25 & 636.10\\
c0531.txt & \bf{\underline{697.84}} & 15.31 & 697.84 & 15.06 & 700.10\\
c0532.txt & 659.34 & 14.30 & 659.788 & 13.25 & \bf{659.30}\\
c0533.txt & 680.04 & 15.10 & 680.04 & 14.4 & \bf{680.00}\\
c0534.txt & \bf{690.50} & 14.67 & 690.5 & 14.33 & 690.50\\
c0535.txt & \bf{\underline{659.90}} & 14.79 & 663.482 & 14.67 & 671.10\\
c0536.txt & 652.94 & 12.99 & 652.94 & 13.36 & \bf{651.10}\\
c0537.txt & 666.15 & 12.53 & 666.15 & 11.915 & \bf{666.10}\\
c0538.txt & \bf{\underline{719.47}} & 11.81 & 719.545 & 12.315 & 719.50\\
c0539.txt & \bf{681.00} & 11.24 & 681 & 10.5825 & 681.00\\
c0580.txt & \bf{\underline{961.50}} & 15.00 & 975.753 & 15.3125 & 961.60\\
c0581.txt & \bf{\underline{1061.71}} & 12.13 & 1065.27 & 12.2925 & 1063.00\\
c0582.txt & 1046.29 & 11.58 & 1048.54 & 10.74 & \bf{1040.60}\\
c0583.txt & 1013.00 & 16.03 & 1015.98 & 15.68 & \bf{985.90}\\
c0584.txt & \bf{\underline{1067.66}} & 15.25 & 1070.72 & 15.395 & 1071.00\\
c0585.txt & \bf{\underline{1047.55}} & 16.83 & 1052.78 & 16.465 & 1054.30\\
c0586.txt & \bf{\underline{972.48}} & 16.28 & 978.168 & 16.035 & 972.50\\
c0587.txt & 1067.03 & 16.70 & 1068.86 & 17.4725 & \bf{1059.70}\\
c0588.txt & \bf{\underline{1071.18}} & 15.36 & 1073.91 & 15.2975 & 1082.70\\
c0589.txt & \bf{\underline{1067.42}} & 11.86 & 1067.42 & 12.2025 & 1081.40\\
cc0530.txt & 616.52 & 17.32 & 620.758 & 16.3675 & \bf{616.50}\\
cc0531.txt & \bf{\underline{554.47}} & 13.88 & 556.233 & 14.3725 & 555.60\\
cc0532.txt & \bf{\underline{521.38}} & 12.99 & 521.442 & 12.77 & 521.40\\
cc0533.txt & \bf{\underline{591.19}} & 14.78 & 591.195 & 15.1975 & 591.20\\
cc0534.txt & \bf{\underline{588.79}} & 12.92 & 588.79 & 14.6675 & 589.30\\
cc0535.txt & \bf{563.70} & 13.89 & 565.653 & 14.4775 & 563.70\\
cc0536.txt & 500.80 & 17.76 & 501.14 & 17.305 & \bf{499.20}\\
cc0537.txt & \bf{\underline{576.48}} & 13.54 & 578.415 & 12.9575 & 577.50\\
cc0538.txt & 523.68 & 12.46 & 524.145 & 12.605 & \bf{523.10}\\
cc0539.txt & 580.78 & 13.05 & 584.158 & 13.15 & \bf{578.20}\\
cc0580.txt & 866.22 & 15.21 & 871.605 & 14.7025 & \bf{858.90}\\
cc0581.txt & \bf{\underline{740.85}} & 15.59 & 742.53 & 14.92 & 740.90\\
cc0582.txt & \bf{\underline{713.44}} & 19.24 & 713.785 & 20.995 & 714.30\\
cc0583.txt & \bf{\underline{811.23}} & 14.04 & 815.543 & 15.02 & 812.30\\
cc0584.txt & 778.28 & 14.41 & 787.852 & 14.4275 & \bf{770.10}\\
cc0585.txt & \bf{\underline{759.93}} & 14.63 & 760.967 & 14.1275 & 766.60\\
cc0586.txt & \bf{\underline{688.29}} & 17.02 & 693.885 & 16.7675 & 697.20\\
cc0587.txt & \bf{\underline{814.79}} & 12.89 & 814.95 & 12.6475 & 814.80\\
cc0588.txt & 781.54 & 17.40 & 786.413 & 16.995 & \bf{771.30}\\
cc0589.txt & \bf{\underline{811.75}} & 15.30 & 813.832 & 15.905 & 815.10\\
[1ex]\hline
\end{tabular}
\label{table:nonlin}
\end{table} \clearpage
\begin{table}[ht]
\caption{Resultados de la ejecución de la metaheurística ILS, utilizando instancias de Dethloff con la configuración -n 15 -LS 80}
\centering
\begin{tabular}{c c c c c c}
\hline\hline
Instancia & Costo mínimo & Tiempo(seg.) & Costo promedio & Tiempo promedio(seg.) & Costo ILS \\ [0.5ex]
\hline
635.62 & 640.55 & 3.81 & 642.755 & 4.2775 & \bf{0.40}\\
697.84 & 701.53 & 3.40 & 706.707 & 3.9075 & \bf{0.00}\\
659.34 & 664.92 & 5.08 & 670.315 & 4.3975 & \bf{0.00}\\
680.04 & 680.04 & 4.35 & 683.895 & 4.79 & \bf{0.10}\\
690.50 & 690.50 & 4.48 & 706.8 & 4.365 & \bf{0.00}\\
659.90 & 677.30 & 6.33 & 685.798 & 4.875 & \bf{0.10}\\
651.09 & 652.94 & 4.01 & 654.168 & 4.38 & \bf{0.00}\\
659.17 & 671.67 & 4.26 & 671.72 & 4.2725 & \bf{1.00}\\
719.47 & 719.77 & 4.55 & 726.412 & 3.85 & \bf{0.00}\\
681.00 & 685.88 & 3.68 & 693.168 & 4.19 & \bf{0.00}\\
961.50 & 1007.51 & 3.51 & 1018.95 & 3.385 & \bf{0.10}\\
1049.65 & 1082.54 & 3.98 & 1089.95 & 3.8525 & \bf{0.60}\\
1039.64 & 1069.86 & 4.69 & 1071.06 & 4.1325 & \bf{0.40}\\
983.34 & 1022.22 & 4.04 & 1029.63 & 3.8 & \bf{0.80}\\
1065.49 & 1077.80 & 5.34 & 1104.68 & 3.9925 & \bf{0.90}\\
1027.08 & 1070.80 & 3.24 & 1076.26 & 3.6675 & \bf{0.70}\\
971.82 & 999.07 & 3.98 & 1015.29 & 3.4975 & \bf{0.00}\\
1051.28 & 1083.70 & 3.84 & 1095.4 & 3.6625 & \bf{0.40}\\
1071.18 & 1080.75 & 4.23 & 1097.61 & 3.275 & \bf{0.00}\\
1060.50 & 1101.58 & 4.26 & 1111.53 & 3.9 & \bf{0.60}\\
616.52 & 620.49 & 4.22 & 633.175 & 4.43 & \bf{1.00}\\
554.47 & 559.72 & 4.18 & 562.152 & 5.0825 & \bf{1.00}\\
518.00 & 521.38 & 4.32 & 525.45 & 4.395 & \bf{0.70}\\
591.19 & 595.43 & 4.59 & 601.81 & 4.185 & \bf{0.00}\\
588.79 & 591.43 & 4.22 & 600.452 & 4.61 & \bf{0.90}\\
563.70 & 569.04 & 4.17 & 583.755 & 3.975 & \bf{0.50}\\
499.05 & 502.16 & 4.57 & 507.353 & 4.28 & \bf{1.00}\\
576.48 & 586.01 & 4.86 & 593.215 & 4.755 & \bf{0.60}\\
523.05 & 524.38 & 4.02 & 533.66 & 4.7075 & \bf{0.10}\\
578.24 & 590.39 & 4.24 & 592.94 & 4.5325 & \bf{0.70}\\
857.17 & 897.83 & 3.90 & 915.822 & 3.3175 & \bf{0.90}\\
740.85 & 755.14 & 5.54 & 765.105 & 4.3975 & \bf{0.90}\\
712.89 & 727.27 & 2.94 & 737.797 & 3.775 & \bf{1.00}\\
811.07 & 836.24 & 3.94 & 844.12 & 4.055 & \bf{0.80}\\
772.25 & 817.96 & 4.02 & 829.635 & 3.485 & \bf{0.80}\\
754.88 & 761.01 & 4.46 & 782.56 & 3.655 & \bf{0.40}\\
678.92 & 699.68 & 3.00 & 710.005 & 3.47 & \bf{0.80}\\
811.96 & 815.72 & 3.71 & 822.74 & 3.8525 & \bf{0.30}\\
767.53 & 782.61 & 3.84 & 800.133 & 3.425 & \bf{0.80}\\
809.00 & 812.45 & 3.90 & 835.973 & 3.53 & \bf{0.70}\\
[1ex]\hline
\end{tabular}
\label{table:nonlin}
\end{table} \clearpage
\begin{table}[ht]
\caption{Resultados de la ejecución de la metaheurística ILS, utilizando instancias de Dethloff con la configuración -n 15 -LS 80}
\centering
\begin{tabular}{c c c c c c}
\hline\hline
Instancia & Costo mínimo & Tiempo(seg.) & Costo promedio & Tiempo promedio(seg.) & Costo ILS \\ [0.5ex]
\hline
c0530.txt & 640.55 & 3.81 & 642.755 & 4.2775 & \bf{635.62}\\
c0531.txt & 701.53 & 3.40 & 706.707 & 3.9075 & \bf{697.84}\\
c0532.txt & 664.92 & 5.08 & 670.315 & 4.3975 & \bf{659.34}\\
c0533.txt & \bf{680.04} & 4.35 & 683.895 & 4.79 & 680.04\\
c0534.txt & \bf{690.50} & 4.48 & 706.8 & 4.365 & 690.50\\
c0535.txt & 677.30 & 6.33 & 685.798 & 4.875 & \bf{659.90}\\
c0536.txt & 652.94 & 4.01 & 654.168 & 4.38 & \bf{651.09}\\
c0537.txt & 671.67 & 4.26 & 671.72 & 4.2725 & \bf{659.17}\\
c0538.txt & 719.77 & 4.55 & 726.412 & 3.85 & \bf{719.47}\\
c0539.txt & 685.88 & 3.68 & 693.168 & 4.19 & \bf{681.00}\\
c0580.txt & 1007.51 & 3.51 & 1018.95 & 3.385 & \bf{961.50}\\
c0581.txt & 1082.54 & 3.98 & 1089.95 & 3.8525 & \bf{1049.65}\\
c0582.txt & 1069.86 & 4.69 & 1071.06 & 4.1325 & \bf{1039.64}\\
c0583.txt & 1022.22 & 4.04 & 1029.63 & 3.8 & \bf{983.34}\\
c0584.txt & 1077.80 & 5.34 & 1104.68 & 3.9925 & \bf{1065.49}\\
c0585.txt & 1070.80 & 3.24 & 1076.26 & 3.6675 & \bf{1027.08}\\
c0586.txt & 999.07 & 3.98 & 1015.29 & 3.4975 & \bf{971.82}\\
c0587.txt & 1083.70 & 3.84 & 1095.4 & 3.6625 & \bf{1051.28}\\
c0588.txt & 1080.75 & 4.23 & 1097.61 & 3.275 & \bf{1071.18}\\
c0589.txt & 1101.58 & 4.26 & 1111.53 & 3.9 & \bf{1060.50}\\
cc0530.txt & 620.49 & 4.22 & 633.175 & 4.43 & \bf{616.52}\\
cc0531.txt & 559.72 & 4.18 & 562.152 & 5.0825 & \bf{554.47}\\
cc0532.txt & 521.38 & 4.32 & 525.45 & 4.395 & \bf{518.00}\\
cc0533.txt & 595.43 & 4.59 & 601.81 & 4.185 & \bf{591.19}\\
cc0534.txt & 591.43 & 4.22 & 600.452 & 4.61 & \bf{588.79}\\
cc0535.txt & 569.04 & 4.17 & 583.755 & 3.975 & \bf{563.70}\\
cc0536.txt & 502.16 & 4.57 & 507.353 & 4.28 & \bf{499.05}\\
cc0537.txt & 586.01 & 4.86 & 593.215 & 4.755 & \bf{576.48}\\
cc0538.txt & 524.38 & 4.02 & 533.66 & 4.7075 & \bf{523.05}\\
cc0539.txt & 590.39 & 4.24 & 592.94 & 4.5325 & \bf{578.24}\\
cc0580.txt & 897.83 & 3.90 & 915.822 & 3.3175 & \bf{857.17}\\
cc0581.txt & 755.14 & 5.54 & 765.105 & 4.3975 & \bf{740.85}\\
cc0582.txt & 727.27 & 2.94 & 737.797 & 3.775 & \bf{712.89}\\
cc0583.txt & 836.24 & 3.94 & 844.12 & 4.055 & \bf{811.07}\\
cc0584.txt & 817.96 & 4.02 & 829.635 & 3.485 & \bf{772.25}\\
cc0585.txt & 761.01 & 4.46 & 782.56 & 3.655 & \bf{754.88}\\
cc0586.txt & 699.68 & 3.00 & 710.005 & 3.47 & \bf{678.92}\\
cc0587.txt & 815.72 & 3.71 & 822.74 & 3.8525 & \bf{811.96}\\
cc0588.txt & 782.61 & 3.84 & 800.133 & 3.425 & \bf{767.53}\\
cc0589.txt & 812.45 & 3.90 & 835.973 & 3.53 & \bf{809.00}\\
[1ex]\hline
\end{tabular}
\label{table:nonlin}
\end{table}
\begin{table}[ht]
\caption{Resultados de la ejecución de la metaheurística ILS, utilizando instancias de Dethloff con la configuración -n 30 -LS 80}
\centering
\begin{tabular}{c c c c c c}
\hline\hline
Instancia & Costo mínimo & Tiempo(seg.) & Costo promedio & Tiempo promedio(seg.) & Costo ILS \\ [0.5ex]
\hline
c0530.txt & 636.34 & 8.16 & 641.258 & 8.48 & \bf{635.62}\\
c0531.txt & \bf{697.84} & 8.15 & 700.148 & 8.0425 & 697.84\\
c0532.txt & 661.13 & 9.66 & 669.408 & 8.7875 & \bf{659.34}\\
c0533.txt & \bf{680.04} & 8.72 & 684.07 & 8.4575 & 680.04\\
c0534.txt & \bf{690.50} & 8.49 & 692.383 & 7.955 & 690.50\\
c0535.txt & 666.67 & 7.68 & 675.36 & 7.2875 & \bf{659.90}\\
c0536.txt & 653.68 & 7.60 & 656.605 & 8.4875 & \bf{651.09}\\
c0537.txt & 669.89 & 7.88 & 671.388 & 7.9075 & \bf{659.17}\\
c0538.txt & \bf{719.47} & 6.57 & 724.348 & 7.8975 & 719.47\\
c0539.txt & 684.25 & 8.05 & 688.267 & 7.49 & \bf{681.00}\\
c0580.txt & 1011.07 & 5.86 & 1029.35 & 6.0375 & \bf{961.50}\\
c0581.txt & 1082.86 & 6.80 & 1093.6 & 6.2125 & \bf{1049.65}\\
c0582.txt & 1066.10 & 7.58 & 1078.2 & 6.9275 & \bf{1039.64}\\
c0583.txt & 1024.92 & 5.95 & 1037.18 & 6.545 & \bf{983.34}\\
c0584.txt & 1105.22 & 7.95 & 1118.04 & 6.82 & \bf{1065.49}\\
c0585.txt & 1072.62 & 5.00 & 1088.57 & 5.39 & \bf{1027.08}\\
c0586.txt & 993.90 & 7.16 & 996.275 & 6.86 & \bf{971.82}\\
c0587.txt & 1075.62 & 7.02 & 1088.14 & 6.6425 & \bf{1051.28}\\
c0588.txt & \bf{1071.18} & 7.44 & 1089.83 & 6.4325 & 1071.18\\
c0589.txt & 1068.10 & 6.78 & 1079.61 & 6.56 & \bf{1060.50}\\
cc0530.txt & 624.41 & 6.84 & 632.125 & 8.3475 & \bf{616.52}\\
cc0531.txt & 560.75 & 8.49 & 563.942 & 8.62 & \bf{554.47}\\
cc0532.txt & 521.38 & 7.84 & 523.503 & 8.58 & \bf{518.00}\\
cc0533.txt & 592.41 & 7.37 & 594.547 & 7.8725 & \bf{591.19}\\
cc0534.txt & 591.43 & 8.58 & 602.482 & 7.75 & \bf{588.79}\\
cc0535.txt & 568.69 & 9.01 & 575.142 & 8.2925 & \bf{563.70}\\
cc0536.txt & 502.85 & 9.12 & 510.418 & 8.5175 & \bf{499.05}\\
cc0537.txt & 578.41 & 7.28 & 584.322 & 7.885 & \bf{576.48}\\
cc0538.txt & 523.68 & 8.22 & 525.597 & 8.095 & \bf{523.05}\\
cc0539.txt & 588.99 & 8.13 & 589.965 & 8.1825 & \bf{578.24}\\
cc0580.txt & 903.17 & 6.96 & 907.077 & 6.8325 & \bf{857.17}\\
cc0581.txt & 764.79 & 7.11 & 772.42 & 6.77 & \bf{740.85}\\
cc0582.txt & 728.75 & 5.72 & 732.535 & 6.185 & \bf{712.89}\\
cc0583.txt & 825.74 & 8.92 & 849.8 & 7.5525 & \bf{811.07}\\
cc0584.txt & 789.79 & 7.80 & 808.46 & 7.91 & \bf{772.25}\\
cc0585.txt & 762.61 & 7.23 & 783.803 & 6.9575 & \bf{754.88}\\
cc0586.txt & 699.00 & 6.98 & 711.987 & 6.575 & \bf{678.92}\\
cc0587.txt & 814.79 & 7.28 & 825.963 & 6.955 & \bf{811.96}\\
cc0588.txt & 790.07 & 5.80 & 799.532 & 6.0575 & \bf{767.53}\\
cc0589.txt & 815.50 & 6.60 & 827.077 & 6.7725 & \bf{809.00}\\
[1ex]\hline
\end{tabular}
\label{table:nonlin}
\end{table}
\begin{table}[ht]
\caption{Resultados de la ejecución de la metaheurística ILS, utilizando instancias de Dethloff con la configuración -n 60 -LS 80}
\centering
\begin{tabular}{c c c c c c}
\hline\hline
Instancia & Costo mínimo & Tiempo(seg.) & Costo promedio & Tiempo promedio(seg.) & Costo ILS \\ [0.5ex]
\hline
c0530.txt & 636.06 & 17.05 & 637.977 & 16.435 & \bf{635.62}\\
c0531.txt & 700.50 & 17.16 & 704.165 & 16.2575 & \bf{697.84}\\
c0532.txt & \bf{659.34} & 16.54 & 664.665 & 16.495 & 659.34\\
c0533.txt & \bf{680.04} & 15.21 & 681.75 & 15.94 & 680.04\\
c0534.txt & \bf{690.50} & 16.22 & 696.73 & 15.66 & 690.50\\
c0535.txt & \bf{659.90} & 18.52 & 672.247 & 15.6925 & 659.90\\
c0536.txt & 652.94 & 15.82 & 653.345 & 15.555 & \bf{651.09}\\
c0537.txt & 666.60 & 15.10 & 670.428 & 15.53 & \bf{659.17}\\
c0538.txt & \bf{719.47} & 15.53 & 720.672 & 16.1825 & 719.47\\
c0539.txt & 686.11 & 17.62 & 688.45 & 16.0825 & \bf{681.00}\\
c0580.txt & 995.48 & 14.98 & 1010 & 14.64 & \bf{961.50}\\
c0581.txt & 1081.68 & 10.96 & 1084.17 & 13.1525 & \bf{1049.65}\\
c0582.txt & 1057.56 & 13.59 & 1064.2 & 13.46 & \bf{1039.64}\\
c0583.txt & 1014.18 & 13.14 & 1026.6 & 12.2125 & \bf{983.34}\\
c0584.txt & 1069.33 & 14.00 & 1089.22 & 13.7125 & \bf{1065.49}\\
c0585.txt & 1067.06 & 12.56 & 1071.68 & 14.1775 & \bf{1027.08}\\
c0586.txt & 981.37 & 12.69 & 992.893 & 12.6775 & \bf{971.82}\\
c0587.txt & 1070.92 & 14.62 & 1089.51 & 14.3475 & \bf{1051.28}\\
c0588.txt & 1085.34 & 12.82 & 1088.96 & 13.83 & \bf{1071.18}\\
c0589.txt & 1068.10 & 12.58 & 1084.52 & 13.0875 & \bf{1060.50}\\
cc0530.txt & 617.59 & 16.11 & 624.995 & 16.4125 & \bf{616.52}\\
cc0531.txt & \bf{554.47} & 17.40 & 559.46 & 17.69 & 554.47\\
cc0532.txt & 521.38 & 15.07 & 523.08 & 16.4725 & \bf{518.00}\\
cc0533.txt & \bf{591.19} & 17.00 & 594.6 & 18.21 & 591.19\\
cc0534.txt & \bf{588.79} & 15.76 & 597.133 & 16.53 & 588.79\\
cc0535.txt & 564.88 & 14.28 & 567.963 & 14.775 & \bf{563.70}\\
cc0536.txt & 502.16 & 16.38 & 504.663 & 16.15 & \bf{499.05}\\
cc0537.txt & 578.41 & 14.87 & 589.435 & 15.4575 & \bf{576.48}\\
cc0538.txt & 523.14 & 16.28 & 523.637 & 16.34 & \bf{523.05}\\
cc0539.txt & 588.38 & 16.31 & 588.985 & 16.9025 & \bf{578.24}\\
cc0580.txt & 870.80 & 13.23 & 898.47 & 13.2675 & \bf{857.17}\\
cc0581.txt & 740.93 & 16.78 & 758.73 & 15.3775 & \bf{740.85}\\
cc0582.txt & 717.31 & 13.09 & 726.403 & 13.5225 & \bf{712.89}\\
cc0583.txt & 826.29 & 16.27 & 832.017 & 14.8925 & \bf{811.07}\\
cc0584.txt & 801.90 & 12.77 & 817.805 & 13.1675 & \bf{772.25}\\
cc0585.txt & 762.36 & 12.24 & 766.98 & 12.525 & \bf{754.88}\\
cc0586.txt & 694.04 & 15.08 & 699.598 & 14.525 & \bf{678.92}\\
cc0587.txt & 816.74 & 12.79 & 826.235 & 12.7025 & \bf{811.96}\\
cc0588.txt & 788.92 & 16.12 & 792.148 & 15.005 & \bf{767.53}\\
cc0589.txt & 814.67 & 15.08 & 827.873 & 13.15 & \bf{809.00}\\
[1ex]\hline
\end{tabular}
\label{table:nonlin}
\end{table}
\begin{table}[ht]
\caption{Resultados de la ejecución de la metaheurística ILS, utilizando instancias de Dethloff con la configuración -n 15 -LS 160}
\centering
\begin{tabular}{c c c c c c}
\hline\hline
Instancia & Costo mínimo & Tiempo(seg.) & Costo promedio & Tiempo promedio(seg.) & Costo ILS \\ [0.5ex]
\hline
c0530.txt & 640.55 & 8.67 & 640.55 & 8.565 & \bf{635.62}\\
c0531.txt & 707.07 & 7.82 & 709.85 & 8.6225 & \bf{697.84}\\
c0532.txt & \bf{659.34} & 10.56 & 665.555 & 8.7625 & 659.34\\
c0533.txt & 681.16 & 7.49 & 683.48 & 8.965 & \bf{680.04}\\
c0534.txt & \bf{690.50} & 6.63 & 702.91 & 8.1525 & 690.50\\
c0535.txt & 681.81 & 7.44 & 684.49 & 8.02 & \bf{659.90}\\
c0536.txt & 653.81 & 9.08 & 661.805 & 8.2675 & \bf{651.09}\\
c0537.txt & 667.24 & 7.14 & 670.638 & 7.77 & \bf{659.17}\\
c0538.txt & \bf{719.47} & 8.36 & 724.997 & 8.66 & 719.47\\
c0539.txt & 684.25 & 8.58 & 691.66 & 7.8625 & \bf{681.00}\\
c0580.txt & 979.68 & 6.99 & 1010.12 & 5.755 & \bf{961.50}\\
c0581.txt & 1072.88 & 5.15 & 1081.31 & 6.2275 & \bf{1049.65}\\
c0582.txt & 1061.46 & 7.58 & 1072.27 & 7.3925 & \bf{1039.64}\\
c0583.txt & 1025.59 & 6.38 & 1038.65 & 6.1 & \bf{983.34}\\
c0584.txt & 1108.42 & 5.74 & 1115.36 & 6.36 & \bf{1065.49}\\
c0585.txt & 1052.77 & 7.57 & 1067.34 & 6.765 & \bf{1027.08}\\
c0586.txt & 1002.62 & 5.76 & 1005.88 & 5.7325 & \bf{971.82}\\
c0587.txt & 1089.45 & 7.32 & 1099.42 & 6.505 & \bf{1051.28}\\
c0588.txt & \bf{1071.18} & 7.09 & 1075.35 & 6.925 & 1071.18\\
c0589.txt & 1084.07 & 5.32 & 1101.29 & 6.0375 & \bf{1060.50}\\
cc0530.txt & 623.97 & 7.91 & 630.433 & 8.33 & \bf{616.52}\\
cc0531.txt & 560.61 & 8.80 & 562.743 & 8.395 & \bf{554.47}\\
cc0532.txt & 521.63 & 9.39 & 525.357 & 8.2075 & \bf{518.00}\\
cc0533.txt & 591.48 & 9.06 & 591.985 & 9.3025 & \bf{591.19}\\
cc0534.txt & 591.43 & 9.64 & 595.168 & 8.595 & \bf{588.79}\\
cc0535.txt & 569.88 & 7.97 & 580.015 & 9.1425 & \bf{563.70}\\
cc0536.txt & 502.26 & 8.60 & 507.09 & 7.855 & \bf{499.05}\\
cc0537.txt & 586.01 & 8.59 & 589.37 & 8.705 & \bf{576.48}\\
cc0538.txt & 526.59 & 8.46 & 533.848 & 8.3375 & \bf{523.05}\\
cc0539.txt & 590.64 & 8.61 & 591.962 & 7.9625 & \bf{578.24}\\
cc0580.txt & 904.04 & 5.34 & 912.995 & 6.31 & \bf{857.17}\\
cc0581.txt & 761.24 & 6.25 & 769.825 & 5.645 & \bf{740.85}\\
cc0582.txt & 719.66 & 6.78 & 731.597 & 7.1425 & \bf{712.89}\\
cc0583.txt & 832.62 & 6.12 & 842.548 & 7.26 & \bf{811.07}\\
cc0584.txt & 796.04 & 7.19 & 810.005 & 6.5225 & \bf{772.25}\\
cc0585.txt & 764.15 & 6.58 & 779.74 & 6.79 & \bf{754.88}\\
cc0586.txt & 702.04 & 6.65 & 710.477 & 6.8975 & \bf{678.92}\\
cc0587.txt & 814.86 & 7.14 & 828.497 & 7.0475 & \bf{811.96}\\
cc0588.txt & 788.30 & 5.71 & 803.447 & 6.1125 & \bf{767.53}\\
cc0589.txt & 819.00 & 7.39 & 844.65 & 6.4025 & \bf{809.00}\\
[1ex]\hline
\end{tabular}
\label{table:nonlin}
\end{table}
\begin{table}[ht]
\caption{Resultados de la ejecución de la metaheurística ILS, utilizando instancias de Dethloff con la configuración -n 30 -LS 80}
\centering
\begin{tabular}{c c c c c c}
\hline\hline
Instancia & Costo mínimo & Tiempo(seg.) & Costo promedio & Tiempo promedio(seg.) & Costo ILS \\ [0.5ex]
\hline
c0530.txt & 636.06 & 7.97 & 642.142 & 8.508 & \bf{635.62}\\
c0531.txt & \bf{697.84} & 8.15 & 703.167 & 8.11 & 697.84\\
c0532.txt & \bf{659.34} & 8.38 & 668.972 & 8.6005 & 659.34\\
c0533.txt & \bf{680.04} & 7.42 & 682.934 & 8.0165 & 680.04\\
c0534.txt & \bf{690.50} & 8.36 & 702.067 & 8.362 & 690.50\\
c0535.txt & \bf{659.90} & 7.58 & 677.616 & 8.1705 & 659.90\\
c0536.txt & \bf{651.09} & 9.04 & 654.054 & 8.2435 & 651.09\\
c0537.txt & 666.15 & 8.21 & 671.615 & 8.309 & \bf{659.17}\\
c0538.txt & \bf{719.47} & 8.44 & 722.996 & 8.4235 & 719.47\\
c0539.txt & \bf{681.00} & 7.84 & 688.252 & 7.815 & 681.00\\
c0580.txt & 995.48 & 7.59 & 1011.37 & 6.9885 & \bf{961.50}\\
c0581.txt & 1070.29 & 6.03 & 1084.87 & 6.513 & \bf{1049.65}\\
c0582.txt & 1054.30 & 6.81 & 1068.96 & 7.1205 & \bf{1039.64}\\
c0583.txt & 1018.15 & 6.62 & 1031.47 & 6.722 & \bf{983.34}\\
c0584.txt & 1067.28 & 5.64 & 1093.37 & 6.628 & \bf{1065.49}\\
c0585.txt & 1043.36 & 7.00 & 1075.7 & 6.6925 & \bf{1027.08}\\
c0586.txt & 982.49 & 7.02 & 1003.68 & 6.9035 & \bf{971.82}\\
c0587.txt & 1070.92 & 8.57 & 1087.96 & 6.9385 & \bf{1051.28}\\
c0588.txt & \bf{1071.18} & 7.13 & 1094.66 & 6.4865 & 1071.18\\
c0589.txt & 1074.71 & 6.44 & 1097.93 & 6.4785 & \bf{1060.50}\\
cc0530.txt & 627.51 & 7.94 & 637.745 & 8.052 & \bf{616.52}\\
cc0531.txt & 556.92 & 8.46 & 562.634 & 8.639 & \bf{554.47}\\
cc0532.txt & 521.38 & 8.56 & 524.807 & 8.6245 & \bf{518.00}\\
cc0533.txt & 591.36 & 7.66 & 599.297 & 8.0945 & \bf{591.19}\\
cc0534.txt & 589.32 & 8.56 & 596.935 & 8.2745 & \bf{588.79}\\
cc0535.txt & 564.88 & 7.94 & 577.104 & 7.985 & \bf{563.70}\\
cc0536.txt & 502.16 & 15.63 & 507.699 & 8.512 & \bf{499.05}\\
cc0537.txt & 578.41 & 7.66 & 589.108 & 7.339 & \bf{576.48}\\
cc0538.txt & 523.14 & 9.32 & 531.072 & 8.2635 & \bf{523.05}\\
cc0539.txt & 578.25 & 8.49 & 589.758 & 8.637 & \bf{578.24}\\
cc0580.txt & 858.63 & 5.59 & 901.964 & 6.9365 & \bf{857.17}\\
cc0581.txt & 754.22 & 7.33 & 770.252 & 7.099 & \bf{740.85}\\
cc0582.txt & 716.30 & 7.16 & 732.804 & 7.3955 & \bf{712.89}\\
cc0583.txt & 812.22 & 6.79 & 840.165 & 7.3835 & \bf{811.07}\\
cc0584.txt & 787.15 & 5.56 & 813.745 & 6.5235 & \bf{772.25}\\
cc0585.txt & 762.01 & 7.62 & 778.342 & 6.9485 & \bf{754.88}\\
cc0586.txt & 686.95 & 6.35 & 709.284 & 7.4165 & \bf{678.92}\\
cc0587.txt & 814.50 & 6.24 & 834.82 & 6.8245 & \bf{811.96}\\
cc0588.txt & 784.40 & 7.14 & 794.633 & 6.8375 & \bf{767.53}\\
cc0589.txt & 817.60 & 7.44 & 840.478 & 7.5605 & \bf{809.00}\\
[1ex]\hline
\end{tabular}
\label{table:nonlin}
\end{table}
\begin{table}[ht]
\caption{Resultados de la ejecución de la metaheurística GTS, utilizando instancias de Dethloff con la configuración -mni 100 -lambda1 0.05 -lambda2 0.05 -tabu 15}
\centering
\small
\begin{tabular}{c c c c c c c}
\hline\hline
Instancia & Costo mínimo & Tiempo(seg.) & Costo promedio & Tiempo promedio(seg.) & Costo GTS & \%Gap \\ [0.5ex]
\hline
c0530.txt & 644.06 & 0.12 & 649.69 & 0.12 & \bf{636.06} & 0.00\\
c0531.txt & 714.92 & 0.17 & 719.607 & 0.14 & \bf{697.84} & 0.00\\
c0532.txt & 696.54 & 0.12 & 696.54 & 0.135 & \bf{659.34} & 0.00\\
c0533.txt & \bf{680.04} & 0.16 & 680.04 & 0.16 & 680.04 & 0.00\\
c0534.txt & \bf{690.50} & 0.12 & 692.548 & 0.1175 & 690.50 & 0.00\\
c0535.txt & 697.30 & 0.14 & 700.293 & 0.1575 & \bf{659.90} & 0.00\\
c0536.txt & 653.68 & 0.16 & 659.6 & 0.1225 & \bf{651.09} & 0.00\\
c0537.txt & 673.15 & 0.11 & 674.465 & 0.12 & \bf{659.17} & 0.00\\
c0538.txt & 721.81 & 0.14 & 735.647 & 0.145 & \bf{719.47} & 0.00\\
c0539.txt & 689.95 & 0.14 & 696.652 & 0.1675 & \bf{681.00} & 0.00\\
c0580.txt & 987.08 & 0.24 & 1014.1 & 0.1925 & \bf{961.50} & 0.00\\
c0581.txt & 1075.73 & 0.20 & 1118.64 & 0.16 & \bf{1050.20} & 0.00\\
c0582.txt & 1064.27 & 0.14 & 1064.98 & 0.165 & \bf{1039.64} & 0.00\\
c0583.txt & 1016.08 & 0.16 & 1020.29 & 0.19 & \bf{983.34} & 0.00\\
c0584.txt & 1071.16 & 0.15 & 1101.42 & 0.1575 & \bf{1065.49} & 0.00\\
c0585.txt & 1066.92 & 0.12 & 1069.97 & 0.15 & \bf{1027.08} & 0.00\\
c0586.txt & 1035.00 & 0.13 & 1039.47 & 0.1475 & \bf{971.82} & 0.00\\
c0587.txt & 1085.30 & 0.19 & 1110.19 & 0.19 & \bf{1052.17} & 0.00\\
c0588.txt & 1100.57 & 0.28 & 1100.92 & 0.245 & \bf{1071.18} & 0.00\\
c0589.txt & 1082.02 & 0.23 & 1089.42 & 0.1875 & \bf{1060.50} & 0.00\\
cc0530.txt & 631.02 & 0.21 & 632.53 & 0.1725 & \bf{616.52} & 0.00\\
cc0531.txt & 571.60 & 0.12 & 573.107 & 0.12 & \bf{554.47} & 0.00\\
cc0532.txt & 533.55 & 0.11 & 533.55 & 0.11 & \bf{519.26} & 0.00\\
cc0533.txt & 605.05 & 0.23 & 608.913 & 0.18 & \bf{591.19} & 0.00\\
cc0534.txt & 613.27 & 0.14 & 621.237 & 0.1375 & \bf{589.32} & 0.00\\
cc0535.txt & \bf{563.70} & 0.24 & 583.945 & 0.215 & 563.70 & 0.00\\
cc0536.txt & 512.91 & 0.13 & 518.805 & 0.1725 & \bf{500.80} & 0.00\\
cc0537.txt & 603.94 & 0.11 & 603.94 & 0.14 & \bf{576.48} & 0.00\\
cc0538.txt & 523.68 & 0.18 & 525.173 & 0.1725 & \bf{523.05} & 0.00\\
cc0539.txt & 592.78 & 0.20 & 597.63 & 0.22 & \bf{580.05} & 0.00\\
cc0580.txt & 896.19 & 0.20 & 907.02 & 0.235 & \bf{857.17} & 0.00\\
cc0581.txt & 759.51 & 0.23 & 775.225 & 0.27 & \bf{740.85} & 0.00\\
cc0582.txt & 724.34 & 0.24 & 735.31 & 0.2275 & \bf{713.44} & 0.00\\
cc0583.txt & 841.64 & 0.19 & 855.115 & 0.1825 & \bf{811.07} & 0.00\\
cc0584.txt & 833.07 & 0.14 & 836.425 & 0.1675 & \bf{772.25} & 0.00\\
cc0585.txt & 761.40 & 0.31 & 765.78 & 0.2275 & \bf{756.91} & 0.00\\
cc0586.txt & 716.61 & 0.18 & 731.145 & 0.185 & \bf{678.92} & 0.00\\
cc0587.txt & 920.01 & 0.10 & 920.01 & 0.1 & \bf{811.96} & 0.00\\
cc0588.txt & 790.66 & 0.21 & 808.712 & 0.18 & \bf{767.53} & 0.00\\
cc0589.txt & 829.35 & 0.36 & 890.602 & 0.21 & \bf{809.00} & 0.00\\
[1ex]\hline
\end{tabular}
\label{table:nonlin}
\end{table}
\begin{table}[ht]
\caption{Resultados de la ejecución de la metaheurística GTS, utilizando instancias de Dethloff con la configuración -mni 100 -lambda1 0.05 -lambda2 0.05 -tabu 15}
\centering
\small
\begin{tabular}{c c c c c c c}
\hline\hline
Instancia & Costo mínimo & Tiempo(seg.) & Costo promedio & Tiempo promedio(seg.) & Costo GTS & \%Gap \\ [0.5ex]
\hline
c0530.txt & 643.15 & 0.14 & 654.205 & 0.1125 & \bf{636.06} & 0.00\\
c0531.txt & 719.33 & 0.11 & 726.545 & 0.125 & \bf{697.84} & 0.00\\
c0532.txt & \bf{659.34} & 0.14 & 664.965 & 0.155 & 659.34 & 0.00\\
c0533.txt & 684.44 & 0.20 & 688.12 & 0.17 & \bf{680.04} & 0.00\\
c0534.txt & 693.23 & 0.11 & 728.585 & 0.1125 & \bf{690.50} & 0.00\\
c0535.txt & 686.81 & 0.12 & 693.545 & 0.14 & \bf{659.90} & 0.00\\
c0536.txt & 661.28 & 0.14 & 662.945 & 0.1175 & \bf{651.09} & 0.00\\
c0537.txt & 676.85 & 0.16 & 678.343 & 0.15 & \bf{659.17} & 0.00\\
c0538.txt & 719.77 & 0.16 & 728.55 & 0.1425 & \bf{719.47} & 0.00\\
c0539.txt & 684.25 & 0.14 & 686.17 & 0.155 & \bf{681.00} & 0.00\\
c0580.txt & 1001.48 & 0.22 & 1027.1 & 0.25 & \bf{961.50} & 0.00\\
c0581.txt & 1079.84 & 0.22 & 1088.76 & 0.1725 & \bf{1050.20} & 0.00\\
c0582.txt & 1056.02 & 0.15 & 1067.36 & 0.155 & \bf{1039.64} & 0.00\\
c0583.txt & 1014.12 & 0.17 & 1017.46 & 0.1975 & \bf{983.34} & 0.00\\
c0584.txt & 1082.69 & 0.15 & 1097.89 & 0.1625 & \bf{1065.49} & 0.00\\
c0585.txt & 1064.12 & 0.24 & 1070.94 & 0.245 & \bf{1027.08} & 0.00\\
c0586.txt & 987.01 & 0.20 & 995.66 & 0.195 & \bf{971.82} & 0.00\\
c0587.txt & 1084.27 & 0.14 & 1141.55 & 0.2225 & \bf{1052.17} & 0.00\\
c0588.txt & 1094.99 & 0.11 & 1097.22 & 0.1275 & \bf{1071.18} & 0.00\\
c0589.txt & 1108.28 & 0.19 & 1131.59 & 0.185 & \bf{1060.50} & 0.00\\
cc0530.txt & 633.74 & 0.18 & 639.02 & 0.1475 & \bf{616.52} & 0.00\\
cc0531.txt & 568.24 & 0.12 & 580.895 & 0.1425 & \bf{554.47} & 0.00\\
cc0532.txt & 524.58 & 0.17 & 526.033 & 0.16 & \bf{519.26} & 0.00\\
cc0533.txt & 606.16 & 0.24 & 615.862 & 0.1875 & \bf{591.19} & 0.00\\
cc0534.txt & 591.43 & 0.14 & 591.43 & 0.1475 & \bf{589.32} & 0.00\\
cc0535.txt & 573.25 & 0.11 & 575.183 & 0.1425 & \bf{563.70} & 0.00\\
cc0536.txt & 509.00 & 0.14 & 510.75 & 0.15 & \bf{500.80} & 0.00\\
cc0537.txt & 578.41 & 0.13 & 586.832 & 0.1425 & \bf{576.48} & 0.00\\
cc0538.txt & 552.32 & 0.16 & 568.065 & 0.215 & \bf{523.05} & 0.00\\
cc0539.txt & 587.23 & 0.20 & 597.385 & 0.1675 & \bf{580.05} & 0.00\\
cc0580.txt & 910.68 & 0.21 & 918.53 & 0.195 & \bf{857.17} & 0.00\\
cc0581.txt & 765.87 & 0.14 & 768.308 & 0.135 & \bf{740.85} & 0.00\\
cc0582.txt & 745.03 & 0.23 & 773.257 & 0.2125 & \bf{713.44} & 0.00\\
cc0583.txt & 848.81 & 0.12 & 857.635 & 0.215 & \bf{811.07} & 0.00\\
cc0584.txt & 787.47 & 0.14 & 844.42 & 0.145 & \bf{772.25} & 0.00\\
cc0585.txt & 784.77 & 0.18 & 805.055 & 0.15 & \bf{756.91} & 0.00\\
cc0586.txt & 690.63 & 0.22 & 709.235 & 0.19 & \bf{678.92} & 0.00\\
cc0587.txt & 863.88 & 0.20 & 865.807 & 0.18 & \bf{811.96} & 0.00\\
cc0588.txt & 793.93 & 0.19 & 800.087 & 0.19 & \bf{767.53} & 0.00\\
cc0589.txt & 819.73 & 0.29 & 833.553 & 0.2475 & \bf{809.00} & 0.00\\
[1ex]\hline
\end{tabular}
\label{table:nonlin}
\end{table}
\begin{table}[ht]
\caption{Resultados de la ejecución de la metaheurística GTS, utilizando instancias de Dethloff con la configuración -mni 100 -lambda1 0.05 -lambda2 0.05 -tabu 15}
\centering
\small
\begin{tabular}{c c c c c c c}
\hline\hline
Instancia & Costo mínimo & Tiempo(seg.) & Costo promedio & Tiempo promedio(seg.) & Costo GTS & \%Gap \\ [0.5ex]
\hline
SCA3-0 & 814.87 & 0.33 & 847.45 & 0.27 & \bf{636.06} & 0.00\\
SCA3-1 & 814.87 & 0.33 & 847.45 & 0.27 & \bf{697.84} & 0.00\\
SCA3-2 & 814.87 & 0.33 & 847.45 & 0.27 & \bf{659.34} & 0.00\\
SCA3-3 & 814.87 & 0.33 & 847.45 & 0.27 & \bf{680.04} & 0.00\\
SCA3-4 & 814.87 & 0.33 & 847.45 & 0.27 & \bf{690.50} & 0.00\\
SCA3-5 & 814.87 & 0.33 & 847.45 & 0.27 & \bf{659.90} & 0.00\\
SCA3-6 & 814.87 & 0.33 & 847.45 & 0.27 & \bf{651.09} & 0.00\\
SCA3-7 & 814.87 & 0.33 & 847.45 & 0.27 & \bf{659.17} & 0.00\\
SCA3-8 & 814.87 & 0.33 & 847.45 & 0.27 & \bf{719.47} & 0.00\\
SCA3-9 & 814.87 & 0.33 & 847.45 & 0.27 & \bf{681.00} & 0.00\\
SCA8-0 & \bf{\underline{814.87}} & 0.33 & 847.45 & 0.27 & 961.50 & 15.2501\\
SCA8-1 & \bf{\underline{814.87}} & 0.33 & 847.45 & 0.27 & 1050.20 & 22.4081\\
SCA8-2 & \bf{\underline{814.87}} & 0.33 & 847.45 & 0.27 & 1039.64 & 21.62\\
SCA8-3 & \bf{\underline{814.87}} & 0.33 & 847.45 & 0.27 & 983.34 & 17.1324\\
SCA8-4 & \bf{\underline{814.87}} & 0.33 & 847.45 & 0.27 & 1065.49 & 23.5216\\
SCA8-5 & \bf{\underline{814.87}} & 0.33 & 847.45 & 0.27 & 1027.08 & 20.6615\\
SCA8-6 & \bf{\underline{814.87}} & 0.33 & 847.45 & 0.27 & 971.82 & 16.1501\\
SCA8-7 & \bf{\underline{814.87}} & 0.33 & 847.45 & 0.27 & 1052.17 & 22.5534\\
SCA8-8 & \bf{\underline{814.87}} & 0.33 & 847.45 & 0.27 & 1071.18 & 23.9278\\
SCA8-9 & \bf{\underline{814.87}} & 0.33 & 847.45 & 0.27 & 1060.50 & 23.1617\\
CON3-0 & 814.87 & 0.33 & 847.45 & 0.27 & \bf{616.52} & 0.00\\
CON3-1 & 814.87 & 0.33 & 847.45 & 0.27 & \bf{554.47} & 0.00\\
CON3-2 & 814.87 & 0.33 & 847.45 & 0.27 & \bf{519.26} & 0.00\\
CON3-3 & 814.87 & 0.33 & 847.45 & 0.27 & \bf{591.19} & 0.00\\
CON3-4 & 814.87 & 0.33 & 847.45 & 0.27 & \bf{589.32} & 0.00\\
CON3-5 & 814.87 & 0.33 & 847.45 & 0.27 & \bf{563.70} & 0.00\\
CON3-6 & 814.87 & 0.33 & 847.45 & 0.27 & \bf{500.80} & 0.00\\
CON3-7 & 814.87 & 0.33 & 847.45 & 0.27 & \bf{576.48} & 0.00\\
CON3-8 & 814.87 & 0.33 & 847.45 & 0.27 & \bf{523.05} & 0.00\\
CON3-9 & 814.87 & 0.33 & 847.45 & 0.27 & \bf{580.05} & 0.00\\
CON8-0 & \bf{\underline{814.87}} & 0.33 & 847.45 & 0.27 & 857.17 & 4.93484\\
CON8-1 & 814.87 & 0.33 & 847.45 & 0.27 & \bf{740.85} & 0.00\\
CON8-2 & 814.87 & 0.33 & 847.45 & 0.27 & \bf{713.44} & 0.00\\
CON8-3 & 814.87 & 0.33 & 847.45 & 0.27 & \bf{811.07} & 0.00\\
CON8-4 & 814.87 & 0.33 & 847.45 & 0.27 & \bf{772.25} & 0.00\\
CON8-5 & 814.87 & 0.33 & 847.45 & 0.27 & \bf{756.91} & 0.00\\
CON8-6 & 814.87 & 0.33 & 847.45 & 0.27 & \bf{678.92} & 0.00\\
CON8-7 & 814.87 & 0.33 & 847.45 & 0.27 & \bf{811.96} & 0.00\\
CON8-8 & 814.87 & 0.33 & 847.45 & 0.27 & \bf{767.53} & 0.00\\
CON8-9 & 814.87 & 0.33 & 847.45 & 0.27 & \bf{809.00} & 0.00\\
[1ex]\hline
\end{tabular}
\label{table:nonlin}
\end{table}
\begin{table}[ht]
\caption{Resultados de la ejecución de la metaheurística GTS, utilizando instancias de Dethloff con la configuración -mni 100 -lambda1 0.05 -lambda2 0.05 -tabu 15}
\centering
\small
\begin{tabular}{c c c c c c c}
\hline\hline
Instancia & Costo mínimo & Tiempo(seg.) & Costo promedio & Tiempo promedio(seg.) & Costo GTS & \%Gap \\ [0.5ex]
\hline
SCA3-0 & 643.15 & 0.21 & 643.605 & 0.1925 & \bf{636.06} & 0.00\\
SCA3-1 & 708.65 & 0.16 & 713.613 & 0.1275 & \bf{697.84} & 0.00\\
SCA3-2 & \bf{659.34} & 0.22 & 661.76 & 0.2 & 659.34 & 0.00\\
SCA3-3 & 683.16 & 0.12 & 690.092 & 0.1575 & \bf{680.04} & 0.00\\
SCA3-4 & 713.35 & 0.12 & 719.92 & 0.1325 & \bf{690.50} & 0.00\\
SCA3-5 & 670.48 & 0.16 & 704.192 & 0.135 & \bf{659.90} & 0.00\\
SCA3-6 & 665.57 & 0.12 & 680.173 & 0.1075 & \bf{651.09} & 0.00\\
SCA3-7 & 671.77 & 0.15 & 677.185 & 0.135 & \bf{659.17} & 0.00\\
SCA3-8 & 724.28 & 0.14 & 747.892 & 0.13 & \bf{719.47} & 0.00\\
SCA3-9 & \bf{681.00} & 0.18 & 687.965 & 0.185 & 681.00 & 0.00\\
SCA8-0 & 1033.41 & 0.12 & 1035.47 & 0.145 & \bf{961.50} & 0.00\\
SCA8-1 & 1105.42 & 0.14 & 1110.37 & 0.1375 & \bf{1050.20} & 0.00\\
SCA8-2 & 1054.47 & 0.21 & 1084.39 & 0.175 & \bf{1039.64} & 0.00\\
SCA8-3 & 1031.31 & 0.14 & 1031.31 & 0.1375 & \bf{983.34} & 0.00\\
SCA8-4 & 1081.79 & 0.19 & 1092.38 & 0.1675 & \bf{1065.49} & 0.00\\
SCA8-5 & 1104.32 & 0.18 & 1110.39 & 0.175 & \bf{1027.08} & 0.00\\
SCA8-6 & 972.48 & 0.32 & 1004.25 & 0.23 & \bf{971.82} & 0.00\\
SCA8-7 & 1109.63 & 0.12 & 1109.63 & 0.115 & \bf{1052.17} & 0.00\\
SCA8-8 & 1098.39 & 0.15 & 1098.39 & 0.14 & \bf{1071.18} & 0.00\\
SCA8-9 & 1101.50 & 0.21 & 1136.08 & 0.1725 & \bf{1060.50} & 0.00\\
CON3-0 & 633.97 & 0.18 & 640.815 & 0.1975 & \bf{616.52} & 0.00\\
CON3-1 & 562.43 & 0.13 & 576.207 & 0.11 & \bf{554.47} & 0.00\\
CON3-2 & 524.13 & 0.18 & 535.402 & 0.125 & \bf{519.26} & 0.00\\
CON3-3 & 591.48 & 0.14 & 607.05 & 0.1375 & \bf{591.19} & 0.00\\
CON3-4 & 607.57 & 0.11 & 610.653 & 0.1275 & \bf{589.32} & 0.00\\
CON3-5 & 572.56 & 0.16 & 572.56 & 0.165 & \bf{563.70} & 0.00\\
CON3-6 & 510.15 & 0.12 & 510.635 & 0.14 & \bf{500.80} & 0.00\\
CON3-7 & 594.23 & 0.14 & 608.195 & 0.1425 & \bf{576.48} & 0.00\\
CON3-8 & 545.45 & 0.13 & 550.885 & 0.1375 & \bf{523.05} & 0.00\\
CON3-9 & 588.38 & 0.21 & 590.383 & 0.1675 & \bf{580.05} & 0.00\\
CON8-0 & 944.61 & 0.19 & 946.142 & 0.2125 & \bf{857.17} & 0.00\\
CON8-1 & 764.76 & 0.11 & 765.747 & 0.1825 & \bf{740.85} & 0.00\\
CON8-2 & 768.72 & 0.15 & 800.518 & 0.1825 & \bf{713.44} & 0.00\\
CON8-3 & 828.56 & 0.23 & 850.84 & 0.1725 & \bf{811.07} & 0.00\\
CON8-4 & 792.53 & 0.32 & 815.992 & 0.2125 & \bf{772.25} & 0.00\\
CON8-5 & 772.05 & 0.17 & 801.495 & 0.2325 & \bf{756.91} & 0.00\\
CON8-6 & 687.18 & 0.21 & 731.97 & 0.165 & \bf{678.92} & 0.00\\
CON8-7 & 867.71 & 0.19 & 879.843 & 0.1875 & \bf{811.96} & 0.00\\
CON8-8 & 822.62 & 0.22 & 828.82 & 0.1975 & \bf{767.53} & 0.00\\
CON8-9 & 812.22 & 0.22 & 826.56 & 0.2675 & \bf{809.00} & 0.00\\
[1ex]\hline
\end{tabular}
\label{table:nonlin}
\end{table}
\begin{table}[ht]
\caption{Resultados de la ejecución de la metaheurística GTS, utilizando instancias de Dethloff con la configuración -mni 3000 -lambda1 0.05 -lambda2 0.05 -tabu 15}
\centering
\small
\begin{tabular}{c c c c c c c}
\hline\hline
Instancia & Costo mínimo & Tiempo(seg.) & Costo promedio & Tiempo promedio(seg.) & Costo GTS & \%Gap \\ [0.5ex]
\hline
SCA3-0 & \bf{636.06} & 1.64 & 639.428 & 1.9975 & 636.06 & 0.00\\
SCA3-1 & \bf{697.84} & 1.14 & 703.063 & 1.3625 & 697.84 & 0.00\\
SCA3-2 & \bf{659.34} & 2.91 & 659.34 & 2.6125 & 659.34 & 0.00\\
SCA3-3 & 680.60 & 2.94 & 685.52 & 1.8125 & \bf{680.04} & 0.00\\
SCA3-4 & \bf{690.50} & 2.37 & 690.5 & 2.6675 & 690.50 & 0.00\\
SCA3-5 & \bf{659.90} & 1.94 & 666.533 & 1.8325 & 659.90 & 0.00\\
SCA3-6 & \bf{651.09} & 2.00 & 653.638 & 1.6825 & 651.09 & 0.00\\
SCA3-7 & 666.15 & 1.74 & 670.29 & 1.63 & \bf{659.17} & 0.00\\
SCA3-8 & \bf{719.47} & 4.46 & 719.47 & 2.2025 & 719.47 & 0.00\\
SCA3-9 & \bf{681.00} & 2.99 & 685.115 & 2.2375 & 681.00 & 0.00\\
SCA8-0 & 970.64 & 1.48 & 977.857 & 1.755 & \bf{961.50} & 0.00\\
SCA8-1 & \bf{\underline{1049.65}} & 3.02 & 1071.56 & 2.9925 & 1050.20 & 0.052371\\
SCA8-2 & 1042.10 & 2.04 & 1076.3 & 2.0675 & \bf{1039.64} & 0.00\\
SCA8-3 & 1012.77 & 2.24 & 1014.22 & 2.81 & \bf{983.34} & 0.00\\
SCA8-4 & 1067.28 & 4.88 & 1079.22 & 2.7775 & \bf{1065.49} & 0.00\\
SCA8-5 & 1042.30 & 2.58 & 1056.66 & 2.2675 & \bf{1027.08} & 0.00\\
SCA8-6 & \bf{971.82} & 1.96 & 977.817 & 2.1025 & 971.82 & 0.00\\
SCA8-7 & 1074.08 & 1.98 & 1081.41 & 2.435 & \bf{1052.17} & 0.00\\
SCA8-8 & 1080.58 & 1.00 & 1082.01 & 1.035 & \bf{1071.18} & 0.00\\
SCA8-9 & \bf{1060.50} & 2.49 & 1070.07 & 1.93 & 1060.50 & 0.00\\
CON3-0 & \bf{616.52} & 3.10 & 624.787 & 2.455 & 616.52 & 0.00\\
CON3-1 & \bf{554.47} & 2.83 & 557.41 & 2.1425 & 554.47 & 0.00\\
CON3-2 & 521.38 & 1.59 & 524.817 & 1.495 & \bf{519.26} & 0.00\\
CON3-3 & \bf{591.19} & 1.24 & 594.58 & 1.65 & 591.19 & 0.00\\
CON3-4 & \bf{\underline{588.79}} & 1.05 & 594.407 & 1.4075 & 589.32 & 0.0899342\\
CON3-5 & \bf{563.70} & 0.97 & 563.7 & 2.3125 & 563.70 & 0.00\\
CON3-6 & 502.16 & 3.13 & 508.027 & 1.9625 & \bf{500.80} & 0.00\\
CON3-7 & 578.41 & 1.61 & 590.26 & 2.35 & \bf{576.48} & 0.00\\
CON3-8 & \bf{523.05} & 3.25 & 523.05 & 2.92 & 523.05 & 0.00\\
CON3-9 & \bf{\underline{578.25}} & 1.62 & 584.925 & 2.105 & 580.05 & 0.310318\\
CON8-0 & \bf{857.17} & 4.18 & 876.89 & 3.195 & 857.17 & 0.00\\
CON8-1 & \bf{740.85} & 2.38 & 766.168 & 2.015 & 740.85 & 0.00\\
CON8-2 & 718.52 & 1.68 & 721.865 & 2.3075 & \bf{713.44} & 0.00\\
CON8-3 & \bf{811.07} & 1.75 & 841.717 & 1.6675 & 811.07 & 0.00\\
CON8-4 & \bf{772.25} & 2.59 & 796.727 & 2.3925 & 772.25 & 0.00\\
CON8-5 & 758.99 & 2.55 & 759.983 & 2.555 & \bf{756.91} & 0.00\\
CON8-6 & 692.52 & 3.99 & 697.485 & 2.5725 & \bf{678.92} & 0.00\\
CON8-7 & 814.50 & 4.40 & 823.168 & 2.385 & \bf{811.96} & 0.00\\
CON8-8 & 773.60 & 1.30 & 781.795 & 2.025 & \bf{767.53} & 0.00\\
CON8-9 & \bf{809.00} & 4.76 & 812.545 & 3.3975 & 809.00 & 0.00\\
[1ex]\hline
\end{tabular}
\label{table:nonlin}
\end{table}
\begin{table}[ht]
\caption{Resultados de la ejecución de la metaheurística GTS, utilizando instancias de Dethloff con la configuración -mni 3000 -lambda1 0.05 -lambda2 0.05 -tabu 15}
\centering
\small
\begin{tabular}{c c c c c c c}
\hline\hline
Instancia & Costo mínimo & Tiempo(seg.) & Costo promedio & Tiempo promedio(seg.) & Costo GTS & \%Gap \\ [0.5ex]
\hline
SCA3-0 & \bf{636.06} & 1.35 & 640.86 & 2 & 636.06 & 0.00\\
SCA3-1 & \bf{697.84} & 1.92 & 699.17 & 1.755 & 697.84 & 0.00\\
SCA3-2 & \bf{659.34} & 2.13 & 659.34 & 1.9375 & 659.34 & 0.00\\
SCA3-3 & \bf{680.04} & 1.47 & 682.878 & 1.4525 & 680.04 & 0.00\\
SCA3-4 & \bf{690.50} & 1.78 & 690.5 & 1.8875 & 690.50 & 0.00\\
SCA3-5 & \bf{659.90} & 2.76 & 669.793 & 2.1675 & 659.90 & 0.00\\
SCA3-6 & \bf{651.09} & 2.32 & 654.345 & 1.655 & 651.09 & 0.00\\
SCA3-7 & 666.15 & 4.86 & 667.967 & 2.365 & \bf{659.17} & 0.00\\
SCA3-8 & \bf{719.47} & 2.76 & 719.47 & 2.33 & 719.47 & 0.00\\
SCA3-9 & \bf{681.00} & 1.32 & 681 & 2.005 & 681.00 & 0.00\\
SCA8-0 & 970.64 & 3.46 & 986.718 & 2.7575 & \bf{961.50} & 0.00\\
SCA8-1 & \bf{\underline{1049.65}} & 1.74 & 1060.96 & 2.47 & 1050.20 & 
0.05\\SCA8-2 & \bf{1039.64} & 1.33 & 1062.34 & 1.855 & 1039.64 & 0.00\\
SCA8-3 & \bf{983.34} & 2.41 & 1004.16 & 1.5475 & 983.34 & 0.00\\
SCA8-4 & \bf{1065.49} & 2.95 & 1075.19 & 2.325 & 1065.49 & 0.00\\
SCA8-5 & \bf{1027.08} & 2.48 & 1048.16 & 1.965 & 1027.08 & 0.00\\
SCA8-6 & 976.69 & 1.23 & 988.065 & 1.51 & \bf{971.82} & 0.00\\
SCA8-7 & \bf{\underline{1051.28}} & 3.59 & 1068.58 & 2.4675 & 1052.17 & 
0.08\\SCA8-8 & 1082.12 & 2.03 & 1085.5 & 2.3625 & \bf{1071.18} & 0.00\\
SCA8-9 & 1061.40 & 1.40 & 1074.82 & 1.76 & \bf{1060.50} & 0.00\\
CON3-0 & \bf{616.52} & 4.76 & 623.4 & 2.2925 & 616.52 & 0.00\\
CON3-1 & 556.04 & 1.12 & 560.587 & 1.885 & \bf{554.47} & 0.00\\
CON3-2 & 521.38 & 2.37 & 523.163 & 2.5175 & \bf{519.26} & 0.00\\
CON3-3 & \bf{591.19} & 1.84 & 591.19 & 1.9325 & 591.19 & 0.00\\
CON3-4 & \bf{\underline{588.79}} & 3.77 & 591.985 & 2.0375 & 589.32 & 
0.09\\CON3-5 & \bf{563.70} & 1.10 & 563.7 & 1.195 & 563.70 & 0.00\\
CON3-6 & \bf{\underline{499.05}} & 2.66 & 501.043 & 2.965 & 500.80 & 
0.35\\CON3-7 & \bf{576.48} & 1.89 & 586.538 & 3.1475 & 576.48 & 0.00\\
CON3-8 & \bf{523.05} & 2.45 & 523.05 & 1.52 & 523.05 & 0.00\\
CON3-9 & 588.11 & 2.51 & 589.612 & 1.8125 & \bf{580.05} & 0.00\\
CON8-0 & 868.12 & 2.27 & 895.562 & 2.5275 & \bf{857.17} & 0.00\\
CON8-1 & 751.76 & 2.31 & 772.122 & 1.9575 & \bf{740.85} & 0.00\\
CON8-2 & 718.64 & 1.01 & 724.34 & 1.7575 & \bf{713.44} & 0.00\\
CON8-3 & 821.26 & 1.65 & 829.822 & 1.2875 & \bf{811.07} & 0.00\\
CON8-4 & \bf{772.25} & 1.09 & 782.51 & 1.81 & 772.25 & 0.00\\
CON8-5 & \bf{\underline{754.95}} & 3.22 & 761.535 & 2.095 & 756.91 & 
0.26\\CON8-6 & 683.77 & 1.75 & 692.865 & 2.795 & \bf{678.92} & 0.00\\
CON8-7 & 812.89 & 4.11 & 820.125 & 3.0375 & \bf{811.96} & 0.00\\
CON8-8 & 776.55 & 2.82 & 783.825 & 2.395 & \bf{767.53} & 0.00\\
CON8-9 & 811.16 & 1.90 & 813.26 & 2.2725 & \bf{809.00} & 0.00\\
[1ex]\hline
\end{tabular}
\label{table:nonlin}
\end{table}
\begin{table}[ht]
\caption{Resultados de la ejecución de la metaheurística GTS, utilizando instancias de Dethloff con la configuración -mni 6000 -lambda1 0.05 -lambda2 0.05 -tabu 15}
\centering
\small
\begin{tabular}{c c c c c c c}
\hline\hline
Instancia & Costo mínimo & Tiempo(seg.) & Costo promedio & Tiempo promedio(seg.) & Costo GTS & \%Gap \\ [0.5ex]
\hline
SCA3-0 & \bf{\underline{635.62}} & 3.37 & 637.072 & 4.815 & 636.06 & 
0.07\\SCA3-1 & \bf{697.84} & 3.55 & 697.84 & 5.87 & 697.84 & 0.00\\
SCA3-2 & \bf{659.34} & 5.16 & 659.34 & 4.27 & 659.34 & 0.00\\
SCA3-3 & \bf{680.04} & 5.80 & 682.622 & 5.8725 & 680.04 & 0.00\\
SCA3-4 & \bf{690.50} & 8.55 & 690.5 & 7.0425 & 690.50 & 0.00\\
SCA3-5 & \bf{659.90} & 4.89 & 659.9 & 4.9025 & 659.90 & 0.00\\
SCA3-6 & \bf{651.09} & 4.12 & 651.09 & 4.2725 & 651.09 & 0.00\\
SCA3-7 & 666.15 & 9.16 & 668.465 & 4.82 & \bf{659.17} & 0.00\\
SCA3-8 & \bf{719.47} & 3.28 & 719.47 & 4.16 & 719.47 & 0.00\\
SCA3-9 & \bf{681.00} & 7.60 & 681 & 5.4275 & 681.00 & 0.00\\
SCA8-0 & 977.93 & 15.77 & 984.662 & 9.0875 & \bf{961.50} & 0.00\\
SCA8-1 & \bf{\underline{1049.65}} & 5.62 & 1054.45 & 4.0275 & 1050.20 & 
0.05\\SCA8-2 & 1042.10 & 10.83 & 1059.05 & 5.7225 & \bf{1039.64} & 0.00\\
SCA8-3 & \bf{983.34} & 5.86 & 1004.38 & 4.53 & 983.34 & 0.00\\
SCA8-4 & 1067.55 & 3.31 & 1068.62 & 3.3875 & \bf{1065.49} & 0.00\\
SCA8-5 & \bf{1027.08} & 4.40 & 1041.05 & 3.95 & 1027.08 & 0.00\\
SCA8-6 & 972.48 & 8.63 & 977.023 & 4.985 & \bf{971.82} & 0.00\\
SCA8-7 & 1060.98 & 3.36 & 1075.68 & 4.37 & \bf{1052.17} & 0.00\\
SCA8-8 & \bf{1071.18} & 2.09 & 1077.53 & 3.1475 & 1071.18 & 0.00\\
SCA8-9 & \bf{1060.50} & 5.89 & 1064.16 & 4.805 & 1060.50 & 0.00\\
CON3-0 & \bf{616.52} & 3.54 & 620.01 & 4.765 & 616.52 & 0.00\\
CON3-1 & \bf{554.47} & 7.86 & 556.128 & 4.635 & 554.47 & 0.00\\
CON3-2 & \bf{\underline{519.11}} & 5.90 & 521.19 & 4.5325 & 519.26 & 
0.03\\CON3-3 & \bf{591.19} & 6.01 & 591.19 & 6.5425 & 591.19 & 0.00\\
CON3-4 & \bf{\underline{588.79}} & 7.30 & 594.457 & 4.895 & 589.32 & 
0.09\\CON3-5 & \bf{563.70} & 5.35 & 563.7 & 4.235 & 563.70 & 0.00\\
CON3-6 & \bf{\underline{499.05}} & 3.90 & 502.673 & 3.3475 & 500.80 & 
0.35\\CON3-7 & \bf{576.48} & 5.15 & 584.697 & 6.025 & 576.48 & 0.00\\
CON3-8 & \bf{523.05} & 3.67 & 523.05 & 5.23 & 523.05 & 0.00\\
CON3-9 & \bf{\underline{578.25}} & 5.27 & 583.42 & 3.52 & 580.05 & 
0.31\\CON8-0 & 857.40 & 6.08 & 879.648 & 4 & \bf{857.17} & 0.00\\
CON8-1 & 758.92 & 4.12 & 760.202 & 3.6825 & \bf{740.85} & 0.00\\
CON8-2 & \bf{\underline{712.89}} & 6.06 & 721.923 & 3.8625 & 713.44 & 
0.08\\CON8-3 & 821.26 & 6.33 & 834.71 & 5.985 & \bf{811.07} & 0.00\\
CON8-4 & \bf{772.25} & 2.39 & 784.855 & 3.75 & 772.25 & 0.00\\
CON8-5 & \bf{\underline{754.88}} & 4.03 & 756.668 & 4.1125 & 756.91 & 
0.27\\CON8-6 & 691.08 & 8.24 & 695.543 & 5.2775 & \bf{678.92} & 0.00\\
CON8-7 & 812.89 & 3.32 & 813.397 & 4.73 & \bf{811.96} & 0.00\\
CON8-8 & 782.34 & 6.53 & 793.18 & 6 & \bf{767.53} & 0.00\\
CON8-9 & 809.42 & 4.64 & 812.312 & 4.585 & \bf{809.00} & 0.00\\
[1ex]\hline
\end{tabular}
\label{table:nonlin}
\end{table}
\begin{table}[ht]
\caption{Resultados de la ejecución de la metaheurística GTS, utilizando instancias de Dethloff con la configuración -mni 8000 -lambda1 0.05 -lambda2 0.05 -tabu 15}
\centering
\small
\begin{tabular}{c c c c c c c}
\hline\hline
Instancia & Costo mínimo & Tiempo(seg.) & Costo promedio & Tiempo promedio(seg.) & Costo GTS & \%Gap \\ [0.5ex]
\hline
SCA3-0 & \bf{636.06} & 4.63 & 638.305 & 6.7075 & 636.06 & 0.00\\
SCA3-1 & \bf{697.84} & 9.12 & 698.505 & 6.565 & 697.84 & 0.00\\
SCA3-2 & \bf{659.34} & 5.16 & 659.34 & 4.5025 & 659.34 & 0.00\\
SCA3-3 & \bf{680.04} & 22.02 & 680.32 & 9.58 & 680.04 & 0.00\\
SCA3-4 & \bf{690.50} & 5.38 & 690.5 & 8.0275 & 690.50 & 0.00\\
SCA3-5 & \bf{659.90} & 9.50 & 663.16 & 8.24 & 659.90 & 0.00\\
SCA3-6 & \bf{651.09} & 4.60 & 651.09 & 5.3075 & 651.09 & 0.00\\
SCA3-7 & 664.88 & 4.91 & 666.767 & 5.9925 & \bf{659.17} & 0.00\\
SCA3-8 & \bf{719.47} & 4.31 & 719.47 & 5.14 & 719.47 & 0.00\\
SCA3-9 & \bf{681.00} & 3.90 & 681 & 5.095 & 681.00 & 0.00\\
SCA8-0 & \bf{961.50} & 7.13 & 975.62 & 6.67 & 961.50 & 0.00\\
SCA8-1 & \bf{1050.20} & 7.03 & 1054.91 & 5.2975 & 1050.20 & 0.00\\
SCA8-2 & \bf{1039.64} & 4.72 & 1058.26 & 5.935 & 1039.64 & 0.00\\
SCA8-3 & \bf{983.34} & 20.41 & 989.468 & 9.8375 & 983.34 & 0.00\\
SCA8-4 & \bf{1065.49} & 4.52 & 1071.5 & 5.59 & 1065.49 & 0.00\\
SCA8-5 & 1040.18 & 5.64 & 1048.42 & 5.2425 & \bf{1027.08} & 0.00\\
SCA8-6 & \bf{971.82} & 11.13 & 980.993 & 7.8325 & 971.82 & 0.00\\
SCA8-7 & 1060.98 & 4.36 & 1073.47 & 5.55 & \bf{1052.17} & 0.00\\
SCA8-8 & \bf{1071.18} & 5.56 & 1071.18 & 4.345 & 1071.18 & 0.00\\
SCA8-9 & \bf{1060.50} & 3.78 & 1063.54 & 5.3775 & 1060.50 & 0.00\\
CON3-0 & \bf{616.52} & 6.11 & 619.48 & 4.9825 & 616.52 & 0.00\\
CON3-1 & \bf{554.47} & 7.70 & 556.233 & 6.3275 & 554.47 & 0.00\\
CON3-2 & \bf{\underline{519.11}} & 4.11 & 521.48 & 6.4175 & 519.26 & 
0.03\\CON3-3 & \bf{591.19} & 5.64 & 591.19 & 6.5025 & 591.19 & 0.00\\
CON3-4 & \bf{\underline{588.79}} & 2.96 & 590.665 & 5.3825 & 589.32 & 
0.09\\CON3-5 & \bf{563.70} & 6.15 & 567.582 & 8.6275 & 563.70 & 0.00\\
CON3-6 & \bf{\underline{499.05}} & 9.91 & 499.62 & 6.8675 & 500.80 & 
0.35\\CON3-7 & \bf{576.48} & 7.40 & 576.962 & 7.3925 & 576.48 & 0.00\\
CON3-8 & \bf{523.05} & 8.42 & 523.072 & 5.815 & 523.05 & 0.00\\
CON3-9 & \bf{\underline{578.25}} & 4.61 & 581.85 & 5.78 & 580.05 & 
0.31\\CON8-0 & \bf{857.17} & 4.36 & 862.495 & 4.855 & 857.17 & 0.00\\
CON8-1 & 742.47 & 10.88 & 768.493 & 7.6925 & \bf{740.85} & 0.00\\
CON8-2 & \bf{713.44} & 8.30 & 720.885 & 5.18 & 713.44 & 0.00\\
CON8-3 & \bf{811.07} & 5.87 & 825.58 & 4.3375 & 811.07 & 0.00\\
CON8-4 & \bf{772.25} & 11.14 & 780.752 & 5.985 & 772.25 & 0.00\\
CON8-5 & \bf{\underline{754.88}} & 4.22 & 756.705 & 5.205 & 756.91 & 
0.27\\CON8-6 & \bf{678.92} & 5.27 & 686.455 & 6.4025 & 678.92 & 0.00\\
CON8-7 & \bf{811.96} & 5.60 & 812.452 & 6.395 & 811.96 & 0.00\\
CON8-8 & \bf{767.53} & 11.06 & 777.955 & 7.5025 & 767.53 & 0.00\\
CON8-9 & \bf{809.00} & 4.56 & 813.41 & 5.1325 & 809.00 & 0.00\\
[1ex]\hline
\end{tabular}
\label{table:nonlin}
\end{table}
\begin{table}[ht]
\caption{Resultados de la ejecución de la metaheurística GTS, utilizando instancias de Dethloff con la configuración -mni 7000 -lambda1 0.05 -lambda2 0.05 -tabu 15}
\centering
\small
\begin{tabular}{c c c c c c c}
\hline\hline
Instancia & Costo mínimo & Tiempo(seg.) & Costo promedio & Tiempo promedio(seg.) & Costo GTS & \%Gap \\ [0.5ex]
\hline
SCA3-0 & 640.55 & 4.03 & 641.982 & 4.2275 & \bf{636.06} & 0.00\\
SCA3-1 & \bf{697.84} & 3.07 & 697.84 & 4.3825 & 697.84 & 0.00\\
SCA3-2 & \bf{659.34} & 4.94 & 659.34 & 6.7675 & 659.34 & 0.00\\
SCA3-3 & \bf{680.04} & 3.40 & 680.04 & 6.045 & 680.04 & 0.00\\
SCA3-4 & \bf{690.50} & 3.91 & 690.5 & 5.395 & 690.50 & 0.00\\
SCA3-5 & \bf{659.90} & 3.26 & 663.16 & 4.495 & 659.90 & 0.00\\
SCA3-6 & \bf{651.09} & 3.74 & 651.738 & 4.06 & 651.09 & 0.00\\
SCA3-7 & 666.15 & 4.03 & 667.085 & 5.4525 & \bf{659.17} & 0.00\\
SCA3-8 & \bf{719.47} & 4.71 & 719.47 & 6.0475 & 719.47 & 0.00\\
SCA3-9 & \bf{681.00} & 6.24 & 681 & 5.025 & 681.00 & 0.00\\
SCA8-0 & \bf{961.50} & 8.49 & 979.095 & 5.56 & 961.50 & 0.00\\
SCA8-1 & 1067.45 & 5.40 & 1069.8 & 4.985 & \bf{1050.20} & 0.00\\
SCA8-2 & \bf{1039.64} & 7.03 & 1051.95 & 6.9075 & 1039.64 & 0.00\\
SCA8-3 & \bf{983.34} & 12.42 & 1003.83 & 8.2175 & 983.34 & 0.00\\
SCA8-4 & 1068.97 & 7.18 & 1077.25 & 5.0675 & \bf{1065.49} & 0.00\\
SCA8-5 & \bf{1027.08} & 4.81 & 1047.14 & 4.505 & 1027.08 & 0.00\\
SCA8-6 & \bf{971.82} & 3.66 & 972.315 & 4.7625 & 971.82 & 0.00\\
SCA8-7 & 1060.98 & 3.82 & 1069.63 & 5.94 & \bf{1052.17} & 0.00\\
SCA8-8 & 1082.12 & 3.64 & 1082.69 & 4 & \bf{1071.18} & 0.00\\
SCA8-9 & 1063.68 & 3.95 & 1067.5 & 4.6175 & \bf{1060.50} & 0.00\\
CON3-0 & \bf{616.52} & 10.37 & 622.88 & 7.9725 & 616.52 & 0.00\\
CON3-1 & \bf{554.47} & 5.28 & 558.372 & 4.0475 & 554.47 & 0.00\\
CON3-2 & 519.85 & 3.28 & 522.293 & 5.835 & \bf{519.26} & 0.00\\
CON3-3 & \bf{591.19} & 4.89 & 591.19 & 5.8225 & 591.19 & 0.00\\
CON3-4 & \bf{\underline{588.79}} & 8.90 & 589.45 & 5.7375 & 589.32 & 
0.09\\CON3-5 & \bf{563.70} & 2.48 & 565.915 & 3.38 & 563.70 & 0.00\\
CON3-6 & \bf{\underline{499.05}} & 5.15 & 501.383 & 6.385 & 500.80 & 
0.35\\CON3-7 & \bf{576.48} & 3.24 & 581.587 & 4.83 & 576.48 & 0.00\\
CON3-8 & \bf{523.05} & 2.68 & 523.05 & 5.35 & 523.05 & 0.00\\
CON3-9 & \bf{\underline{578.25}} & 5.23 & 581.892 & 4.55 & 580.05 & 
0.31\\CON8-0 & 857.40 & 6.03 & 887.923 & 6.46 & \bf{857.17} & 0.00\\
CON8-1 & \bf{740.85} & 5.11 & 749.125 & 5.5925 & 740.85 & 0.00\\
CON8-2 & 718.22 & 4.62 & 723.263 & 5.5875 & \bf{713.44} & 0.00\\
CON8-3 & \bf{811.07} & 3.48 & 831.508 & 4.4175 & 811.07 & 0.00\\
CON8-4 & \bf{772.25} & 4.90 & 772.25 & 3.385 & 772.25 & 0.00\\
CON8-5 & \bf{\underline{754.88}} & 4.80 & 758.433 & 5.5725 & 756.91 & 
0.27\\CON8-6 & 688.47 & 5.54 & 693.013 & 4.8675 & \bf{678.92} & 0.00\\
CON8-7 & 812.89 & 6.05 & 813.62 & 4 & \bf{811.96} & 0.00\\
CON8-8 & \bf{767.53} & 6.89 & 777.645 & 5.0375 & 767.53 & 0.00\\
CON8-9 & \bf{809.00} & 5.22 & 818.882 & 3.815 & 809.00 & 0.00\\
[1ex]\hline
\end{tabular}
\label{table:nonlin}
\end{table}
\begin{table}[ht]
\caption{Resultados de la ejecución de la metaheurística GTS, utilizando instancias de Dethloff con la configuración -mni 6000 -lambda1 0.05 -lambda2 0.05 -tabu 2}
\centering
\small
\begin{tabular}{c c c c c c c}
\hline\hline
Instancia & Costo mínimo & Tiempo(seg.) & Costo promedio & Tiempo promedio(seg.) & Costo GTS & \%Gap \\ [0.5ex]
\hline
SCA3-0 & \bf{636.06} & 3.99 & 640.86 & 3.08 & 636.06 & 0.00\\
SCA3-1 & \bf{697.84} & 2.58 & 697.84 & 4.2975 & 697.84 & 0.00\\
SCA3-2 & \bf{659.34} & 3.06 & 659.34 & 3.265 & 659.34 & 0.00\\
SCA3-3 & \bf{680.04} & 5.04 & 685.712 & 3.0375 & 680.04 & 0.00\\
SCA3-4 & \bf{690.50} & 3.00 & 690.5 & 3.035 & 690.50 & 0.00\\
SCA3-5 & \bf{659.90} & 3.11 & 666.533 & 3.05 & 659.90 & 0.00\\
SCA3-6 & \bf{651.09} & 2.52 & 652.385 & 3.33 & 651.09 & 0.00\\
SCA3-7 & 666.15 & 4.23 & 666.263 & 3.115 & \bf{659.17} & 0.00\\
SCA3-8 & \bf{719.47} & 1.85 & 719.47 & 4.9575 & 719.47 & 0.00\\
SCA3-9 & \bf{681.00} & 2.86 & 681 & 3.195 & 681.00 & 0.00\\
SCA8-0 & 979.79 & 2.25 & 990.643 & 4.2475 & \bf{961.50} & 0.00\\
SCA8-1 & 1050.38 & 2.53 & 1064.19 & 4.9525 & \bf{1050.20} & 0.00\\
SCA8-2 & 1042.10 & 4.80 & 1059.4 & 3.775 & \bf{1039.64} & 0.00\\
SCA8-3 & \bf{983.34} & 2.48 & 1003.14 & 4 & 983.34 & 0.00\\
SCA8-4 & 1066.37 & 2.55 & 1072.22 & 3.29 & \bf{1065.49} & 0.00\\
SCA8-5 & \bf{1027.08} & 5.64 & 1045.96 & 3.9475 & 1027.08 & 0.00\\
SCA8-6 & \bf{971.82} & 3.88 & 976.808 & 4.385 & 971.82 & 0.00\\
SCA8-7 & 1061.95 & 3.92 & 1069.78 & 3.565 & \bf{1052.17} & 0.00\\
SCA8-8 & \bf{1071.18} & 1.66 & 1080.26 & 2.4825 & 1071.18 & 0.00\\
SCA8-9 & \bf{1060.50} & 5.20 & 1069.92 & 3.1525 & 1060.50 & 0.00\\
CON3-0 & \bf{616.52} & 6.80 & 628.135 & 4.8275 & 616.52 & 0.00\\
CON3-1 & 556.04 & 4.48 & 557.405 & 3.455 & \bf{554.47} & 0.00\\
CON3-2 & 522.86 & 1.80 & 524.648 & 2.77 & \bf{519.26} & 0.00\\
CON3-3 & \bf{591.19} & 3.28 & 597.893 & 2.7675 & 591.19 & 0.00\\
CON3-4 & \bf{\underline{588.79}} & 1.86 & 594.142 & 2.0925 & 589.32 & 
0.09\\CON3-5 & \bf{563.70} & 3.80 & 563.7 & 2.615 & 563.70 & 0.00\\
CON3-6 & \bf{\underline{499.05}} & 2.62 & 506.412 & 2.5075 & 500.80 & 
0.35\\CON3-7 & 578.22 & 5.70 & 582.665 & 3.165 & \bf{576.48} & 0.00\\
CON3-8 & \bf{523.05} & 3.20 & 523.05 & 3.9125 & 523.05 & 0.00\\
CON3-9 & 582.79 & 2.44 & 586.742 & 4.275 & \bf{580.05} & 0.00\\
CON8-0 & 857.40 & 4.56 & 873.603 & 3.0775 & \bf{857.17} & 0.00\\
CON8-1 & 751.76 & 3.06 & 755.5 & 4.445 & \bf{740.85} & 0.00\\
CON8-2 & \bf{\underline{712.89}} & 4.35 & 719.002 & 3.6575 & 713.44 & 
0.08\\CON8-3 & \bf{811.07} & 3.62 & 825.28 & 4.14 & 811.07 & 0.00\\
CON8-4 & \bf{772.25} & 6.38 & 787.165 & 3.8 & 772.25 & 0.00\\
CON8-5 & 761.01 & 2.47 & 766.485 & 3.0375 & \bf{756.91} & 0.00\\
CON8-6 & 691.20 & 1.85 & 699.972 & 3.0725 & \bf{678.92} & 0.00\\
CON8-7 & 812.26 & 3.92 & 818.335 & 4.0875 & \bf{811.96} & 0.00\\
CON8-8 & \bf{767.53} & 5.55 & 782.658 & 4.5 & 767.53 & 0.00\\
CON8-9 & \bf{809.00} & 3.03 & 817.227 & 4.3675 & 809.00 & 0.00\\
[1ex]\hline
\end{tabular}
\label{table:nonlin}
\end{table}
\begin{table}[ht]
\caption{Resultados de la ejecución de la metaheurística GTS, utilizando instancias de Dethloff con la configuración -mni 6000 -lambda1 0.05 -lambda2 0.05 -tabu 5}
\centering
\small
\begin{tabular}{c c c c c c c}
\hline\hline
Instancia & Costo mínimo & Tiempo(seg.) & Costo promedio & Tiempo promedio(seg.) & Costo GTS & \%Gap \\ [0.5ex]
\hline
SCA3-0 & \bf{\underline{635.62}} & 4.42 & 638.195 & 4.26 & 636.06 & 
0.07\\SCA3-1 & \bf{697.84} & 4.00 & 697.84 & 3.47 & 697.84 & 0.00\\
SCA3-2 & \bf{659.34} & 2.53 & 659.34 & 4.535 & 659.34 & 0.00\\
SCA3-3 & \bf{680.04} & 5.12 & 680.46 & 3.4025 & 680.04 & 0.00\\
SCA3-4 & \bf{690.50} & 4.62 & 690.5 & 4.0675 & 690.50 & 0.00\\
SCA3-5 & \bf{659.90} & 2.27 & 670.22 & 3.185 & 659.90 & 0.00\\
SCA3-6 & \bf{651.09} & 7.80 & 656.49 & 3.4225 & 651.09 & 0.00\\
SCA3-7 & 666.15 & 2.74 & 667.085 & 3.3075 & \bf{659.17} & 0.00\\
SCA3-8 & \bf{719.47} & 3.26 & 719.47 & 3.95 & 719.47 & 0.00\\
SCA3-9 & \bf{681.00} & 2.27 & 681 & 4.445 & 681.00 & 0.00\\
SCA8-0 & 970.64 & 2.50 & 997.825 & 3.315 & \bf{961.50} & 0.00\\
SCA8-1 & 1067.45 & 5.10 & 1069.09 & 3.8325 & \bf{1050.20} & 0.00\\
SCA8-2 & \bf{1039.64} & 6.80 & 1056.9 & 4.8825 & 1039.64 & 0.00\\
SCA8-3 & 1005.65 & 4.40 & 1010.55 & 4.3575 & \bf{983.34} & 0.00\\
SCA8-4 & 1068.97 & 5.02 & 1078.67 & 4.3575 & \bf{1065.49} & 0.00\\
SCA8-5 & \bf{1027.08} & 3.82 & 1046.15 & 3.2725 & 1027.08 & 0.00\\
SCA8-6 & \bf{971.82} & 2.68 & 978.275 & 3.0775 & 971.82 & 0.00\\
SCA8-7 & 1066.65 & 4.66 & 1074.81 & 3.465 & \bf{1052.17} & 0.00\\
SCA8-8 & \bf{1071.18} & 1.67 & 1078.51 & 2.585 & 1071.18 & 0.00\\
SCA8-9 & \bf{1060.50} & 8.03 & 1061.3 & 5.44 & 1060.50 & 0.00\\
CON3-0 & 617.59 & 5.85 & 628.25 & 3.505 & \bf{616.52} & 0.00\\
CON3-1 & \bf{554.47} & 11.84 & 555.255 & 5.07 & 554.47 & 0.00\\
CON3-2 & 523.08 & 4.04 & 523.53 & 2.6575 & \bf{519.26} & 0.00\\
CON3-3 & \bf{591.19} & 5.64 & 591.19 & 5.1925 & 591.19 & 0.00\\
CON3-4 & \bf{\underline{588.79}} & 1.90 & 599.647 & 2.0875 & 589.32 & 
0.09\\CON3-5 & \bf{563.70} & 3.14 & 571.145 & 3.045 & 563.70 & 0.00\\
CON3-6 & \bf{\underline{499.05}} & 3.65 & 501.88 & 3.83 & 500.80 & 
0.35\\CON3-7 & \bf{576.48} & 3.92 & 577.06 & 3.1625 & 576.48 & 0.00\\
CON3-8 & \bf{523.05} & 3.33 & 523.05 & 3.12 & 523.05 & 0.00\\
CON3-9 & 582.79 & 2.92 & 585.882 & 3.0525 & \bf{580.05} & 0.00\\
CON8-0 & 891.39 & 3.69 & 904.775 & 3.77 & \bf{857.17} & 0.00\\
CON8-1 & 751.76 & 3.31 & 754.297 & 3.745 & \bf{740.85} & 0.00\\
CON8-2 & 718.64 & 2.43 & 732.485 & 4.55 & \bf{713.44} & 0.00\\
CON8-3 & \bf{811.07} & 2.98 & 846.28 & 3.8925 & 811.07 & 0.00\\
CON8-4 & \bf{772.25} & 2.16 & 791.23 & 3.5525 & 772.25 & 0.00\\
CON8-5 & \bf{\underline{755.67}} & 4.76 & 757.355 & 4.7225 & 756.91 & 
0.16\\CON8-6 & \bf{678.92} & 2.56 & 693.992 & 2.73 & 678.92 & 0.00\\
CON8-7 & \bf{811.96} & 4.19 & 815.94 & 3.85 & 811.96 & 0.00\\
CON8-8 & \bf{767.53} & 2.64 & 773.807 & 4.145 & 767.53 & 0.00\\
CON8-9 & \bf{809.00} & 5.21 & 826.105 & 4.285 & 809.00 & 0.00\\
[1ex]\hline
\end{tabular}
\label{table:nonlin}
\end{table}
\begin{table}[ht]
\caption{Resultados de la ejecución de la metaheurística GTS, utilizando instancias de Dethloff con la configuración -mni 6000 -lambda1 0.05 -lambda2 0.05 -tabu 10}
\centering
\small
\begin{tabular}{c c c c c c c}
\hline\hline
Instancia & Costo mínimo & Tiempo(seg.) & Costo promedio & Tiempo promedio(seg.) & Costo GTS & \%Gap \\ [0.5ex]
\hline
SCA3-0 & \bf{636.06} & 3.21 & 639.428 & 4.0425 & 636.06 & 0.00\\
SCA3-1 & \bf{697.84} & 2.79 & 698.505 & 3.6725 & 697.84 & 0.00\\
SCA3-2 & \bf{659.34} & 2.67 & 659.34 & 3.4425 & 659.34 & 0.00\\
SCA3-3 & 680.60 & 3.10 & 683.175 & 4.255 & \bf{680.04} & 0.00\\
SCA3-4 & \bf{690.50} & 3.42 & 690.5 & 5.46 & 690.50 & 0.00\\
SCA3-5 & \bf{659.90} & 4.66 & 663.16 & 5.15 & 659.90 & 0.00\\
SCA3-6 & \bf{651.09} & 4.66 & 652.2 & 3.925 & 651.09 & 0.00\\
SCA3-7 & 666.15 & 3.47 & 667.085 & 4.1225 & \bf{659.17} & 0.00\\
SCA3-8 & \bf{719.47} & 2.42 & 719.47 & 3.1725 & 719.47 & 0.00\\
SCA3-9 & \bf{681.00} & 5.49 & 681 & 4.9925 & 681.00 & 0.00\\
SCA8-0 & \bf{961.50} & 7.22 & 970.585 & 5.9125 & 961.50 & 0.00\\
SCA8-1 & \bf{\underline{1049.65}} & 11.24 & 1065.81 & 5.31 & 1050.20 & 
0.05\\SCA8-2 & \bf{1039.64} & 5.53 & 1071.4 & 3.6625 & 1039.64 & 0.00\\
SCA8-3 & \bf{983.34} & 3.90 & 1004.58 & 6.64 & 983.34 & 0.00\\
SCA8-4 & 1065.83 & 3.65 & 1068.73 & 3.5275 & \bf{1065.49} & 0.00\\
SCA8-5 & \bf{1027.08} & 3.86 & 1046.55 & 4.615 & 1027.08 & 0.00\\
SCA8-6 & \bf{971.82} & 2.69 & 976.857 & 4.345 & 971.82 & 0.00\\
SCA8-7 & \bf{\underline{1051.28}} & 6.58 & 1061.09 & 5.535 & 1052.17 & 
0.08\\SCA8-8 & \bf{1071.18} & 2.77 & 1080.26 & 3.63 & 1071.18 & 0.00\\
SCA8-9 & \bf{1060.50} & 4.44 & 1067.68 & 2.92 & 1060.50 & 0.00\\
CON3-0 & \bf{616.52} & 2.38 & 625.622 & 3.465 & 616.52 & 0.00\\
CON3-1 & 556.04 & 3.71 & 559.572 & 3.2075 & \bf{554.47} & 0.00\\
CON3-2 & \bf{\underline{518.00}} & 5.40 & 521.46 & 4.1975 & 519.26 & 
0.24\\CON3-3 & \bf{591.19} & 3.74 & 599.548 & 4.5575 & 591.19 & 0.00\\
CON3-4 & \bf{\underline{588.79}} & 2.15 & 594.415 & 3.1425 & 589.32 & 
0.09\\CON3-5 & \bf{563.70} & 3.60 & 570.345 & 3.88 & 563.70 & 0.00\\
CON3-6 & \bf{\underline{499.05}} & 6.22 & 501.835 & 3.915 & 500.80 & 
0.35\\CON3-7 & \bf{576.48} & 3.84 & 581.747 & 5.345 & 576.48 & 0.00\\
CON3-8 & \bf{523.05} & 7.09 & 523.05 & 4.35 & 523.05 & 0.00\\
CON3-9 & \bf{\underline{578.25}} & 3.30 & 583.097 & 3.2875 & 580.05 & 
0.31\\CON8-0 & 857.38 & 5.74 & 873.817 & 5.32 & \bf{857.17} & 0.00\\
CON8-1 & \bf{740.85} & 4.49 & 754.44 & 5.3625 & 740.85 & 0.00\\
CON8-2 & 718.22 & 2.32 & 730.38 & 2.9875 & \bf{713.44} & 0.00\\
CON8-3 & \bf{811.07} & 5.49 & 827.097 & 4.435 & 811.07 & 0.00\\
CON8-4 & \bf{772.25} & 3.48 & 776.013 & 2.885 & 772.25 & 0.00\\
CON8-5 & 758.12 & 5.00 & 761.355 & 3.7025 & \bf{756.91} & 0.00\\
CON8-6 & 690.01 & 6.38 & 690.902 & 4.805 & \bf{678.92} & 0.00\\
CON8-7 & 812.26 & 3.22 & 823.945 & 4.6725 & \bf{811.96} & 0.00\\
CON8-8 & \bf{767.53} & 4.66 & 773.947 & 4.61 & 767.53 & 0.00\\
CON8-9 & 812.35 & 4.36 & 819.242 & 3.7175 & \bf{809.00} & 0.00\\
[1ex]\hline
\end{tabular}
\label{table:nonlin}
\end{table} \clearpage
\begin{table}[ht]
\caption{Resultados de la ejecución de la metaheurística GTS, utilizando instancias de Dethloff con la configuración -mni 6000 -lambda1 0.05 -lambda2 0.05 -tabu 50}
\centering
\small
\begin{tabular}{c c c c c c c}
\hline\hline
Instancia & Costo mínimo & Tiempo(seg.) & Costo promedio & Tiempo promedio(seg.) & Costo GTS & \%Gap \\ [0.5ex]
\hline
SCA3-0 & \bf{636.06} & 4.26 & 637.207 & 4.105 & 636.06 & 0.00\\
SCA3-1 & \bf{697.84} & 10.21 & 698.505 & 11.72 & 697.84 & 0.00\\
SCA3-2 & \bf{659.34} & 10.38 & 659.34 & 5.87 & 659.34 & 0.00\\
SCA3-3 & \bf{680.04} & 10.45 & 680.32 & 7.08 & 680.04 & 0.00\\
SCA3-4 & \bf{690.50} & 6.91 & 690.5 & 10.745 & 690.50 & 0.00\\
SCA3-5 & \bf{659.90} & 3.83 & 659.9 & 5.18 & 659.90 & 0.00\\
SCA3-6 & \bf{651.09} & 7.06 & 651.553 & 6.125 & 651.09 & 0.00\\
SCA3-7 & 666.15 & 5.42 & 667.085 & 6.8025 & \bf{659.17} & 0.00\\
SCA3-8 & \bf{719.47} & 11.52 & 719.47 & 9.07 & 719.47 & 0.00\\
SCA3-9 & \bf{681.00} & 10.02 & 681 & 9.9675 & 681.00 & 0.00\\
SCA8-0 & 970.65 & 10.60 & 991.638 & 7.4 & \bf{961.50} & 0.00\\
SCA8-1 & \bf{\underline{1049.66}} & 9.17 & 1054.64 & 8.615 & 1050.20 & 
0.05\\SCA8-2 & 1042.10 & 7.22 & 1055.23 & 7.9275 & \bf{1039.64} & 0.00\\
SCA8-3 & \bf{983.34} & 10.58 & 1001.78 & 7.0525 & 983.34 & 0.00\\
SCA8-4 & \bf{1065.49} & 4.80 & 1067.75 & 7.7575 & 1065.49 & 0.00\\
SCA8-5 & 1029.95 & 15.73 & 1045.55 & 7.885 & \bf{1027.08} & 0.00\\
SCA8-6 & 972.48 & 5.71 & 976.678 & 5.3275 & \bf{971.82} & 0.00\\
SCA8-7 & \bf{\underline{1052.04}} & 15.35 & 1062.14 & 8.1375 & 1052.17 & 
0.01\\SCA8-8 & 1082.12 & 3.64 & 1082.12 & 4.5225 & \bf{1071.18} & 0.00\\
SCA8-9 & \bf{1060.50} & 3.55 & 1061.55 & 4.92 & 1060.50 & 0.00\\
CON3-0 & \bf{616.52} & 4.80 & 629.695 & 7.6325 & 616.52 & 0.00\\
CON3-1 & \bf{554.47} & 19.32 & 555.255 & 11.635 & 554.47 & 0.00\\
CON3-2 & 519.61 & 3.86 & 521.895 & 6.9425 & \bf{519.26} & 0.00\\
CON3-3 & \bf{591.19} & 5.20 & 591.19 & 8.15 & 591.19 & 0.00\\
CON3-4 & \bf{\underline{588.79}} & 8.07 & 594.658 & 6.2825 & 589.32 & 
0.09\\CON3-5 & \bf{563.70} & 13.27 & 563.7 & 10.275 & 563.70 & 0.00\\
CON3-6 & \bf{\underline{499.05}} & 7.04 & 501.383 & 7.155 & 500.80 & 
0.35\\CON3-7 & \bf{576.48} & 12.20 & 576.48 & 6.5825 & 576.48 & 0.00\\
CON3-8 & \bf{523.05} & 11.63 & 523.05 & 8.8375 & 523.05 & 0.00\\
CON3-9 & \bf{\underline{578.25}} & 11.44 & 584.233 & 8.455 & 580.05 & 
0.31\\CON8-0 & 858.03 & 3.55 & 881.57 & 4.62 & \bf{857.17} & 0.00\\
CON8-1 & \bf{740.85} & 4.43 & 754.017 & 6.21 & 740.85 & 0.00\\
CON8-2 & \bf{713.44} & 7.59 & 718.377 & 7.1275 & 713.44 & 0.00\\
CON8-3 & \bf{811.07} & 5.43 & 822.418 & 6.4 & 811.07 & 0.00\\
CON8-4 & \bf{772.25} & 18.82 & 775.278 & 7.57 & 772.25 & 0.00\\
CON8-5 & \bf{\underline{754.95}} & 5.01 & 757.725 & 5.525 & 756.91 & 
0.26\\CON8-6 & 693.62 & 6.29 & 695.78 & 6.105 & \bf{678.92} & 0.00\\
CON8-7 & 812.89 & 4.45 & 814.005 & 5.7875 & \bf{811.96} & 0.00\\
CON8-8 & \bf{767.53} & 9.60 & 771.232 & 6.89 & 767.53 & 0.00\\
CON8-9 & \bf{809.00} & 7.99 & 819.713 & 6.305 & 809.00 & 0.00\\
[1ex]\hline
\end{tabular}
\label{table:nonlin}
\end{table} \clearpage
\begin{table}[ht]
\caption{Resultados de la ejecución de la metaheurística GTS, utilizando instancias de SalhiNagy con la configuración -mni 6000 -lambda1 0.05 -lambda2 0.05 -tabu 20}
\centering
\small
\begin{tabular}{c c c c c c c}
\hline\hline
Instancia & Costo mínimo & Tiempo(seg.) & Costo promedio & Tiempo promedio(seg.) & Costo GTS & \%Gap \\ [0.5ex]
\hline
CMT1X & 470.48 & 3.35 & 470.952 & 4.625 & \bf{469.80} & 0.00\\
CMT1Y & 470.48 & 3.74 & 470.952 & 4.67 & \bf{469.80} & 0.00\\
CMT2X & 684.29 & 15.47 & 686.898 & 9.215 & \bf{684.21} & 0.00\\
CMT2Y & 684.24 & 16.16 & 687.715 & 10.8625 & \bf{684.21} & 0.00\\
CMT3X & 724.57 & 13.48 & 727.32 & 14.1375 & \bf{721.27} & 0.00\\
CMT3Y & 728.50 & 13.91 & 731.145 & 12.165 & \bf{721.27} & 0.00\\
CMT4X & 855.58 & 60.68 & 875.528 & 57.565 & \bf{852.46} & 0.00\\
CMT4Y & 868.59 & 49.93 & 874.66 & 59.785 & \bf{852.46} & 0.00\\
CMT5X & 1055.91 & 77.50 & 1065.74 & 71.3675 & \bf{1030.55} & 0.00\\
CMT5Y & 1043.25 & 107.67 & 1047.44 & 103.35 & \bf{1030.55} & 0.00\\
CMT11X & 878.19 & 54.51 & 880.127 & 34.8375 & \bf{838.66} & 0.00\\
CMT11Y & 889.50 & 30.80 & 930.818 & 43.4625 & \bf{837.08} & 0.00\\
CMT12X & 672.97 & 35.24 & 681.83 & 20.6325 & \bf{662.22} & 0.00\\
CMT12Y & 664.80 & 59.07 & 675.555 & 30.555 & \bf{662.22} & 0.00\\
[1ex]\hline
\end{tabular}
\label{table:nonlin}
\end{table}
\begin{table}[ht]
\caption{Resultados de la ejecución de la metaheurística GTS, utilizando instancias de SalhiNagy con la configuración -mni 12000 -lambda1 0.05 -lambda2 0.05 -tabu 20}
\centering
\small
\begin{tabular}{c c c c c c c}
\hline\hline
Instancia & Costo mínimo & Tiempo(seg.) & Costo promedio & Tiempo promedio(seg.) & Costo GTS & \%Gap \\ [0.5ex]
\hline
CMT1X & 470.48 & 6.05 & 470.48 & 6.3975 & \bf{469.80} & 0.00\\
CMT1Y & 470.48 & 10.15 & 470.48 & 10.7825 & \bf{469.80} & 0.00\\
CMT2X & \bf{\underline{682.47}} & 14.95 & 686.265 & 18.1875 & 684.21 & 
0.25\\CMT2Y & \bf{\underline{682.39}} & 15.63 & 685.205 & 14.82 & 684.21 & 
0.27\\CMT3X & 726.09 & 24.90 & 729.055 & 26.345 & \bf{721.27} & 0.00\\
CMT3Y & 726.23 & 32.71 & 726.365 & 26.5425 & \bf{721.27} & 0.00\\
CMT4X & \bf{852.46} & 161.99 & 858.36 & 114.293 & 852.46 & 0.00\\
CMT4Y & 863.34 & 54.39 & 876.498 & 67.4875 & \bf{852.46} & 0.00\\
CMT5X & 1045.23 & 260.47 & 1066.07 & 194.628 & \bf{1030.55} & 0.00\\
CMT5Y & 1053.62 & 87.33 & 1069.32 & 133.678 & \bf{1030.55} & 0.00\\
CMT11X & 895.96 & 59.01 & 918.228 & 62.9125 & \bf{838.66} & 0.00\\
CMT11Y & 875.66 & 113.79 & 911.647 & 70.7175 & \bf{837.08} & 0.00\\
CMT12X & 671.32 & 38.80 & 674.965 & 27.6 & \bf{662.22} & 0.00\\
CMT12Y & 673.59 & 42.44 & 679.905 & 32.065 & \bf{662.22} & 0.00\\
[1ex]\hline
\end{tabular}
\label{table:nonlin}
\end{table}
\begin{table}[ht]
\caption{Resultados de la ejecución de la metaheurística ILS, utilizando instancias de Dethloff con la configuración -n 1 -LS 5}
\centering
\small
\begin{tabular}{c c c c c c c}
\hline\hline
Instancia & Costo mínimo & Tiempo(seg.) & Costo promedio & Tiempo promedio(seg.) & Costo ILS & \%Gap \\ [0.5ex]
\hline
SCA3-0 & 643.32 & 0.04 & 681.308 & 0.065 & \bf{635.62} & 1.21142\\
SCA3-1 & 710.84 & 0.06 & 715.398 & 0.0675 & \bf{697.84} & 1.86289\\
SCA3-2 & 693.07 & 0.06 & 693.07 & 0.0575 & \bf{659.34} & 5.11572\\
SCA3-3 & 700.34 & 0.04 & 700.34 & 0.04 & \bf{680.04} & 2.98512\\
SCA3-4 & \bf{690.50} & 0.04 & 690.5 & 0.04 & 690.50 & 0.00\\
SCA3-5 & 681.03 & 0.07 & 741.998 & 0.0475 & \bf{659.90} & 3.202\\
SCA3-6 & 695.32 & 0.05 & 698.8 & 0.045 & \bf{651.09} & 6.79322\\
SCA3-7 & 691.07 & 0.04 & 691.07 & 0.055 & \bf{659.17} & 4.83942\\
SCA3-8 & 807.81 & 0.06 & 807.81 & 0.0625 & \bf{719.47} & 12.2785\\
SCA3-9 & 767.75 & 0.04 & 767.75 & 0.0425 & \bf{681.00} & 12.7386\\
SCA8-0 & 1035.33 & 0.06 & 1035.33 & 0.0475 & \bf{961.50} & 7.67863\\
SCA8-1 & 1097.65 & 0.09 & 1155.58 & 0.0525 & \bf{1049.65} & 4.57295\\
SCA8-2 & 1093.93 & 0.06 & 1144.83 & 0.0525 & \bf{1039.64} & 5.222\\
SCA8-3 & 1048.51 & 0.04 & 1112.94 & 0.0425 & \bf{983.34} & 6.62741\\
SCA8-4 & 1091.78 & 0.05 & 1154.9 & 0.0425 & \bf{1065.49} & 2.46741\\
SCA8-5 & 1079.38 & 0.06 & 1108.05 & 0.05 & \bf{1027.08} & 5.09211\\
SCA8-6 & 1030.07 & 0.03 & 1030.07 & 0.0375 & \bf{971.82} & 5.99391\\
SCA8-7 & 1173.42 & 0.04 & 1173.42 & 0.0475 & \bf{1051.28} & 11.6182\\
SCA8-8 & 1123.78 & 0.03 & 1150.97 & 0.0475 & \bf{1071.18} & 4.91047\\
SCA8-9 & 1121.43 & 0.05 & 1124.17 & 0.0575 & \bf{1060.50} & 5.7454\\
CON3-0 & 659.87 & 0.07 & 659.87 & 0.0625 & \bf{616.52} & 7.0314\\
CON3-1 & 631.55 & 0.04 & 631.55 & 0.0325 & \bf{554.47} & 13.9016\\
CON3-2 & 521.38 & 0.06 & 521.38 & 0.06 & \bf{518.00} & 0.65251\\
CON3-3 & 617.91 & 0.05 & 630.555 & 0.0475 & \bf{591.19} & 4.5197\\
CON3-4 & 610.84 & 0.06 & 614.767 & 0.0525 & \bf{588.79} & 3.74497\\
CON3-5 & 603.37 & 0.05 & 607.607 & 0.0525 & \bf{563.70} & 7.03743\\
CON3-6 & 505.82 & 0.05 & 505.82 & 0.0575 & \bf{499.05} & 1.35658\\
CON3-7 & 612.41 & 0.05 & 612.41 & 0.055 & \bf{576.48} & 6.23265\\
CON3-8 & 534.92 & 0.06 & 543.03 & 0.0725 & \bf{523.05} & 2.26938\\
CON3-9 & 590.48 & 0.05 & 590.48 & 0.05 & \bf{578.24} & 2.11677\\
CON8-0 & 936.94 & 0.04 & 958.275 & 0.04 & \bf{857.17} & 9.30621\\
CON8-1 & 777.35 & 0.06 & 778.985 & 0.0525 & \bf{740.85} & 4.92677\\
CON8-2 & 771.83 & 0.03 & 783.258 & 0.0375 & \bf{712.89} & 8.26776\\
CON8-3 & 867.01 & 0.04 & 880.32 & 0.05 & \bf{811.07} & 6.89706\\
CON8-4 & 907.31 & 0.04 & 907.31 & 0.0425 & \bf{772.25} & 17.4892\\
CON8-5 & 879.96 & 0.04 & 879.96 & 0.04 & \bf{754.88} & 16.5695\\
CON8-6 & 741.13 & 0.04 & 746.845 & 0.045 & \bf{678.92} & 9.16308\\
CON8-7 & 839.28 & 0.05 & 881.97 & 0.045 & \bf{811.96} & 3.3647\\
CON8-8 & 804.18 & 0.05 & 837.145 & 0.0525 & \bf{767.53} & 4.77506\\
CON8-9 & 909.90 & 0.10 & 909.9 & 0.0875 & \bf{809.00} & 12.4722\\
[1ex]\hline
\end{tabular}
\label{table:nonlin}
\end{table} \clearpage
\begin{table}[ht]
\caption{Resultados de la ejecución de la metaheurística ILS, utilizando instancias de Dethloff con la configuración -n 1 -LS 1}
\centering
\small
\begin{tabular}{c c c c c c c}
\hline\hline
Instancia & Costo mínimo & Tiempo(seg.) & Costo promedio & Tiempo promedio(seg.) & Costo ILS & \%Gap \\ [0.5ex]
\hline
SCA3-0 & 656.60 & 0.04 & 669.275 & 0.0375 & \bf{635.62} & 
3.30\\SCA3-1 & 708.65 & 0.05 & 749.817 & 0.0325 & \bf{697.84} & 
1.55\\SCA3-2 & 701.80 & 0.02 & 701.8 & 0.0225 & \bf{659.34} & 
6.44\\SCA3-3 & 697.23 & 0.04 & 697.23 & 0.0375 & \bf{680.04} & 
2.53\\SCA3-4 & 698.98 & 0.04 & 711.34 & 0.0375 & \bf{690.50} & 
1.23\\SCA3-5 & 684.57 & 0.04 & 694.255 & 0.0475 & \bf{659.90} & 
3.74\\SCA3-6 & 674.67 & 0.04 & 674.67 & 0.03 & \bf{651.09} & 
3.62\\SCA3-7 & 710.01 & 0.03 & 710.01 & 0.03 & \bf{659.17} & 
7.71\\SCA3-8 & 727.73 & 0.04 & 727.73 & 0.035 & \bf{719.47} & 
1.15\\SCA3-9 & 697.46 & 0.04 & 700.875 & 0.04 & \bf{681.00} & 
2.42\\SCA8-0 & 1093.64 & 0.04 & 1120.07 & 0.0525 & \bf{961.50} & 
13.74\\SCA8-1 & 1119.55 & 0.03 & 1157.66 & 0.035 & \bf{1049.65} & 
6.66\\SCA8-2 & 1065.89 & 0.04 & 1065.89 & 0.0325 & \bf{1039.64} & 
2.52\\SCA8-3 & 1029.69 & 0.03 & 1029.69 & 0.035 & \bf{983.34} & 
4.71\\SCA8-4 & 1166.50 & 0.03 & 1207.19 & 0.035 & \bf{1065.49} & 
9.48\\SCA8-5 & 1117.60 & 0.03 & 1126.51 & 0.0375 & \bf{1027.08} & 
8.81\\SCA8-6 & 1052.90 & 0.04 & 1052.9 & 0.035 & \bf{971.82} & 
8.34\\SCA8-7 & 1092.65 & 0.04 & 1092.65 & 0.0325 & \bf{1051.28} & 
3.94\\SCA8-8 & 1093.54 & 0.04 & 1115.41 & 0.035 & \bf{1071.18} & 
2.09\\SCA8-9 & 1154.24 & 0.02 & 1154.24 & 0.0275 & \bf{1060.50} & 
8.84\\CON3-0 & 657.96 & 0.03 & 668.57 & 0.0325 & \bf{616.52} & 
6.72\\CON3-1 & 581.04 & 0.03 & 581.04 & 0.0325 & \bf{554.47} & 
4.79\\CON3-2 & 528.76 & 0.06 & 528.76 & 0.05 & \bf{518.00} & 
2.08\\CON3-3 & 614.77 & 0.04 & 628.01 & 0.04 & \bf{591.19} & 
3.99\\CON3-4 & 620.59 & 0.04 & 620.59 & 0.0375 & \bf{588.79} & 
5.40\\CON3-5 & 597.86 & 0.03 & 597.86 & 0.025 & \bf{563.70} & 
6.06\\CON3-6 & 517.62 & 0.04 & 517.62 & 0.0375 & \bf{499.05} & 
3.72\\CON3-7 & 609.19 & 0.03 & 613.92 & 0.035 & \bf{576.48} & 
5.67\\CON3-8 & 627.52 & 0.04 & 627.52 & 0.0375 & \bf{523.05} & 
19.97\\CON3-9 & 609.38 & 0.04 & 609.38 & 0.035 & \bf{578.24} & 
5.39\\CON8-0 & 941.89 & 0.03 & 959.315 & 0.025 & \bf{857.17} & 
9.88\\CON8-1 & 796.95 & 0.05 & 800.913 & 0.0425 & \bf{740.85} & 
7.57\\CON8-2 & 791.79 & 0.06 & 791.79 & 0.0625 & \bf{712.89} & 
11.07\\CON8-3 & 965.25 & 0.03 & 965.25 & 0.035 & \bf{811.07} & 
19.01\\CON8-4 & 819.73 & 0.04 & 823.562 & 0.035 & \bf{772.25} & 
6.15\\CON8-5 & 834.95 & 0.03 & 834.95 & 0.0325 & \bf{754.88} & 
10.61\\CON8-6 & 705.52 & 0.04 & 708.955 & 0.0375 & \bf{678.92} & 
3.92\\CON8-7 & 852.49 & 0.04 & 852.49 & 0.04 & \bf{811.96} & 
4.99\\CON8-8 & 841.37 & 0.03 & 841.37 & 0.03 & \bf{767.53} & 
9.62\\CON8-9 & 864.97 & 0.04 & 872.065 & 0.04 & \bf{809.00} & 
6.92\\[1ex]\hline
\end{tabular}
\label{table:nonlin}
\end{table} \clearpage
\begin{table}[ht]
\caption{Resultados de la ejecución de la metaheurística ILS, utilizando instancias de Dethloff con la configuración -n 15 -LS 80}
\centering
\small
\begin{tabular}{c c c c c c c}
\hline\hline
Instancia & Costo mínimo & Tiempo(seg.) & Costo promedio & Tiempo promedio(seg.) & Costo ILS & \%Gap \\ [0.5ex]
\hline
SCA3-0 & 636.06 & 5.66 & 639.712 & 5.3275 & \bf{635.62} & 
0.07\\SCA3-1 & \bf{697.84} & 5.03 & 707.825 & 4.4575 & 697.84 & 0.00\\
SCA3-2 & \bf{659.34} & 5.41 & 664.37 & 4.9275 & 659.34 & 0.00\\
SCA3-3 & 681.31 & 5.01 & 685.033 & 4.4025 & \bf{680.04} & 
0.19\\SCA3-4 & \bf{690.50} & 4.76 & 696.442 & 4.84 & 690.50 & 0.00\\
SCA3-5 & 661.07 & 4.60 & 676.11 & 4.5075 & \bf{659.90} & 
0.18\\SCA3-6 & \bf{651.09} & 4.97 & 657.087 & 4.7725 & 651.09 & 0.00\\
SCA3-7 & 669.89 & 4.42 & 671.387 & 4.48 & \bf{659.17} & 
1.63\\SCA3-8 & 719.77 & 3.97 & 737.743 & 3.6325 & \bf{719.47} & 
0.04\\SCA3-9 & \bf{681.00} & 4.63 & 683.325 & 4.4275 & 681.00 & 0.00\\
SCA8-0 & 987.52 & 3.32 & 1033.55 & 3.4225 & \bf{961.50} & 
2.71\\SCA8-1 & 1080.05 & 3.69 & 1088.1 & 3.4025 & \bf{1049.65} & 
2.90\\SCA8-2 & 1070.14 & 3.69 & 1075.76 & 3.5925 & \bf{1039.64} & 
2.93\\SCA8-3 & 1021.53 & 2.99 & 1040.07 & 3.015 & \bf{983.34} & 
3.88\\SCA8-4 & 1088.73 & 3.60 & 1113.32 & 3.25 & \bf{1065.49} & 
2.18\\SCA8-5 & 1054.85 & 4.23 & 1086.62 & 3.755 & \bf{1027.08} & 
2.70\\SCA8-6 & 994.26 & 4.21 & 1007.8 & 3.6175 & \bf{971.82} & 
2.31\\SCA8-7 & 1089.22 & 3.16 & 1106.49 & 3.47 & \bf{1051.28} & 
3.61\\SCA8-8 & \bf{1071.18} & 3.39 & 1093.75 & 3.39 & 1071.18 & 0.00\\
SCA8-9 & 1084.38 & 4.19 & 1114.04 & 3.38 & \bf{1060.50} & 
2.25\\CON3-0 & 632.16 & 5.43 & 637.005 & 4.5275 & \bf{616.52} & 
2.54\\CON3-1 & 561.63 & 3.90 & 568.41 & 4.095 & \bf{554.47} & 
1.29\\CON3-2 & 521.38 & 4.95 & 531.38 & 3.6425 & \bf{518.00} & 
0.65\\CON3-3 & 592.43 & 4.18 & 606.615 & 4.2925 & \bf{591.19} & 
0.21\\CON3-4 & 594.59 & 3.48 & 602.183 & 4.3525 & \bf{588.79} & 
0.99\\CON3-5 & 569.88 & 4.95 & 575.487 & 4.4075 & \bf{563.70} & 
1.10\\CON3-6 & 502.16 & 3.22 & 506.715 & 3.6575 & \bf{499.05} & 
0.62\\CON3-7 & 594.23 & 4.10 & 600.188 & 4.48 & \bf{576.48} & 
3.08\\CON3-8 & 524.59 & 4.77 & 534.19 & 4.245 & \bf{523.05} & 
0.29\\CON3-9 & 589.72 & 4.82 & 590.707 & 4.435 & \bf{578.24} & 
1.99\\CON8-0 & 875.52 & 3.11 & 899.438 & 3.365 & \bf{857.17} & 
2.14\\CON8-1 & 766.24 & 3.50 & 772.355 & 3.4075 & \bf{740.85} & 
3.43\\CON8-2 & 730.66 & 3.26 & 742.16 & 3.3475 & \bf{712.89} & 
2.49\\CON8-3 & 846.72 & 4.37 & 852.94 & 3.7875 & \bf{811.07} & 
4.40\\CON8-4 & 781.78 & 3.62 & 815.925 & 3.445 & \bf{772.25} & 
1.23\\CON8-5 & 774.37 & 4.44 & 791.785 & 3.85 & \bf{754.88} & 
2.58\\CON8-6 & 700.28 & 3.41 & 709.862 & 3.7875 & \bf{678.92} & 
3.15\\CON8-7 & 815.80 & 3.32 & 837.035 & 3.385 & \bf{811.96} & 
0.47\\CON8-8 & 790.17 & 4.49 & 799.27 & 3.685 & \bf{767.53} & 
2.95\\CON8-9 & 814.10 & 3.64 & 831.503 & 3.8725 & \bf{809.00} & 
0.63\\[1ex]\hline
\end{tabular}
\label{table:nonlin}
\end{table} \clearpage
\begin{table}[ht]
\caption{Resultados de la ejecución de la metaheurística GTS, utilizando instancias de Dethloff con la configuración -mni 10 -lambda1 0.05 -lambda2 0.05 -tabu 20}
\centering
\small
\begin{tabular}{c c c c c c c}
\hline\hline
Instancia & Costo mínimo & Tiempo(seg.) & Costo promedio & Tiempo promedio(seg.) & Costo GTS & \%Gap \\ [0.5ex]
\hline
SCA3-0 & 642.44 & 0.10 & 
643.60 & 0.12 & \bf{636.06} & 1.00305\\
SCA3-1 & 707.07 & 0.10 & 
720.05 & 0.10 & \bf{697.84} & 1.32265\\
SCA3-2 & 669.06 & 0.10 & 
683.12 & 0.10 & \bf{659.34} & 1.4742\\
SCA3-3 & 688.95 & 0.13 & 
699.04 & 0.11 & \bf{680.04} & 1.31022\\
SCA3-4 & 704.54 & 0.09 & 
721.36 & 0.09 & \bf{690.50} & 2.03331\\
SCA3-5 & 672.37 & 0.15 & 
677.82 & 0.11 & \bf{659.90} & 1.88968\\
SCA3-6 & 664.86 & 0.09 & 
670.63 & 0.10 & \bf{651.09} & 2.11491\\
SCA3-7 & 671.67 & 0.10 & 
695.48 & 0.12 & \bf{659.17} & 1.89632\\
SCA3-8 & 724.67 & 0.16 & 
736.35 & 0.11 & \bf{719.47} & 0.722754\\
SCA3-9 & 685.88 & 0.10 & 
706.45 & 0.10 & \bf{681.00} & 0.716593\\
SCA8-0 & 1032.21 & 0.14 & 
1095.30 & 0.10 & \bf{961.50} & 7.35413\\
SCA8-1 & 1105.80 & 0.08 & 
1137.48 & 0.11 & \bf{1050.20} & 5.29423\\
SCA8-2 & 1125.14 & 0.11 & 
1143.33 & 0.10 & \bf{1039.64} & 8.224\\
SCA8-3 & 1021.27 & 0.11 & 
1058.08 & 0.12 & \bf{983.34} & 3.85726\\
SCA8-4 & 1105.95 & 0.11 & 
1171.82 & 0.10 & \bf{1065.49} & 3.79731\\
SCA8-5 & 1101.06 & 0.10 & 
1106.39 & 0.11 & \bf{1027.08} & 7.20294\\
SCA8-6 & 1028.50 & 0.14 & 
1037.05 & 0.10 & \bf{971.82} & 5.83236\\
SCA8-7 & 1162.16 & 0.11 & 
1162.16 & 0.12 & \bf{1052.17} & 10.4536\\
SCA8-8 & 1104.29 & 0.10 & 
1104.29 & 0.10 & \bf{1071.18} & 3.09098\\
SCA8-9 & 1082.42 & 0.12 & 
1105.25 & 0.12 & \bf{1060.50} & 2.06695\\
CON3-0 & 633.24 & 0.15 & 
638.53 & 0.12 & \bf{616.52} & 2.712\\
CON3-1 & 599.37 & 0.09 & 
599.37 & 0.10 & \bf{554.47} & 8.09782\\
CON3-2 & 538.92 & 0.10 & 
538.92 & 0.10 & \bf{519.26} & 3.78616\\
CON3-3 & 634.36 & 0.09 & 
634.36 & 0.09 & \bf{591.19} & 7.30222\\
CON3-4 & 595.00 & 0.13 & 
601.95 & 0.12 & \bf{589.32} & 0.963823\\
CON3-5 & 590.65 & 0.10 & 
591.89 & 0.11 & \bf{563.70} & 4.78091\\
CON3-6 & 517.62 & 0.16 & 
526.31 & 0.12 & \bf{500.80} & 3.35863\\
CON3-7 & 582.75 & 0.15 & 
593.61 & 0.12 & \bf{576.48} & 1.08764\\
CON3-8 & 526.59 & 0.10 & 
567.20 & 0.11 & \bf{523.05} & 0.6768\\
CON3-9 & 600.43 & 0.12 & 
602.46 & 0.12 & \bf{580.05} & 3.51349\\
CON8-0 & 934.85 & 0.13 & 
945.63 & 0.11 & \bf{857.17} & 9.06238\\
CON8-1 & 771.74 & 0.14 & 
793.25 & 0.10 & \bf{740.85} & 4.16953\\
CON8-2 & 729.94 & 0.11 & 
743.80 & 0.12 & \bf{713.44} & 2.31274\\
CON8-3 & 852.60 & 0.12 & 
860.42 & 0.10 & \bf{811.07} & 5.1204\\
CON8-4 & 837.56 & 0.09 & 
846.10 & 0.11 & \bf{772.25} & 8.45711\\
CON8-5 & 777.37 & 0.10 & 
777.92 & 0.10 & \bf{756.91} & 2.7031\\
CON8-6 & 719.80 & 0.10 & 
764.01 & 0.11 & \bf{678.92} & 6.02133\\
CON8-7 & 823.81 & 0.10 & 
823.81 & 0.10 & \bf{811.96} & 1.45943\\
CON8-8 & 835.76 & 0.10 & 
835.76 & 0.10 & \bf{767.53} & 8.88955\\
CON8-9 & 851.30 & 0.16 & 
868.52 & 0.15 & \bf{809.00} & 5.22868\\
[1ex]\hline
\end{tabular}
\label{table:nonlin}
\end{table} \clearpage
\begin{table}[ht]
\caption{Resultados de la ejecución de la metaheurística GTS, utilizando instancias de Dethloff con la configuración -mni 10 -lambda1 0.05 -lambda2 0.05 -tabu 20}
\centering
\small
\begin{tabular}{c c c c c c c}
\hline\hline
Instancia & Costo mínimo & Tiempo(seg.) & Costo promedio & Tiempo promedio(seg.) & Costo GTS & \%Gap \\ [0.5ex]
\hline
SCA3-0 & 643.83 & 0.09 & 
682.82 & 0.10 & \bf{636.06} & 
1.22\\SCA3-1 & 701.53 & 0.13 & 
704.73 & 0.13 & \bf{697.84} & 
0.53\\SCA3-2 & 664.21 & 0.12 & 
685.86 & 0.10 & \bf{659.34} & 
0.74\\SCA3-3 & 693.74 & 0.10 & 
693.74 & 0.11 & \bf{680.04} & 
2.01\\SCA3-4 & 746.50 & 0.09 & 
746.50 & 0.09 & \bf{690.50} & 
8.11\\SCA3-5 & 697.30 & 0.09 & 
697.30 & 0.11 & \bf{659.90} & 
5.67\\SCA3-6 & 663.50 & 0.08 & 
713.83 & 0.09 & \bf{651.09} & 
1.91\\SCA3-7 & 706.11 & 0.10 & 
706.11 & 0.09 & \bf{659.17} & 
7.12\\SCA3-8 & 754.80 & 0.14 & 
756.17 & 0.12 & \bf{719.47} & 
4.91\\SCA3-9 & 698.97 & 0.10 & 
698.97 & 0.12 & \bf{681.00} & 
2.64\\SCA8-0 & 1047.27 & 0.09 & 
1047.27 & 0.09 & \bf{961.50} & 
8.92\\SCA8-1 & 1261.11 & 0.12 & 
1261.11 & 0.10 & \bf{1050.20} & 
20.08\\SCA8-2 & 1142.01 & 0.15 & 
1239.40 & 0.12 & \bf{1039.64} & 
9.85\\SCA8-3 & 1064.13 & 0.15 & 
1083.66 & 0.14 & \bf{983.34} & 
8.22\\SCA8-4 & 1115.73 & 0.12 & 
1131.38 & 0.15 & \bf{1065.49} & 
4.72\\SCA8-5 & 1113.51 & 0.14 & 
1115.25 & 0.13 & \bf{1027.08} & 
8.42\\SCA8-6 & 1040.96 & 0.12 & 
1040.96 & 0.10 & \bf{971.82} & 
7.11\\SCA8-7 & 1150.40 & 0.09 & 
1150.40 & 0.09 & \bf{1052.17} & 
9.34\\SCA8-8 & 1115.47 & 0.10 & 
1115.47 & 0.10 & \bf{1071.18} & 
4.13\\SCA8-9 & 1106.43 & 0.11 & 
1110.58 & 0.12 & \bf{1060.50} & 
4.33\\CON3-0 & 679.42 & 0.09 & 
684.21 & 0.12 & \bf{616.52} & 
10.20\\CON3-1 & 589.70 & 0.09 & 
589.70 & 0.09 & \bf{554.47} & 
6.35\\CON3-2 & 530.73 & 0.10 & 
531.26 & 0.10 & \bf{519.26} & 
2.21\\CON3-3 & 633.10 & 0.11 & 
633.10 & 0.12 & \bf{591.19} & 
7.09\\CON3-4 & 603.94 & 0.13 & 
603.94 & 0.12 & \bf{589.32} & 
2.48\\CON3-5 & 577.64 & 0.10 & 
577.64 & 0.12 & \bf{563.70} & 
2.47\\CON3-6 & 508.52 & 0.13 & 
508.52 & 0.12 & \bf{500.80} & 
1.54\\CON3-7 & 604.49 & 0.15 & 
627.10 & 0.13 & \bf{576.48} & 
4.86\\CON3-8 & 534.96 & 0.15 & 
546.17 & 0.11 & \bf{523.05} & 
2.28\\CON3-9 & 600.67 & 0.09 & 
621.83 & 0.11 & \bf{580.05} & 
3.55\\CON8-0 & 885.81 & 0.10 & 
921.30 & 0.11 & \bf{857.17} & 
3.34\\CON8-1 & 758.14 & 0.14 & 
771.29 & 0.12 & \bf{740.85} & 
2.33\\CON8-2 & 744.66 & 0.07 & 
744.66 & 0.09 & \bf{713.44} & 
4.38\\CON8-3 & 833.76 & 0.10 & 
833.76 & 0.10 & \bf{811.07} & 
2.80\\CON8-4 & 798.25 & 0.13 & 
798.25 & 0.10 & \bf{772.25} & 
3.37\\CON8-5 & 761.01 & 0.10 & 
761.01 & 0.12 & \bf{756.91} & 
0.54\\CON8-6 & 734.46 & 0.11 & 
734.46 & 0.10 & \bf{678.92} & 
8.18\\CON8-7 & 848.50 & 0.09 & 
848.50 & 0.09 & \bf{811.96} & 
4.50\\CON8-8 & 799.17 & 0.10 & 
835.15 & 0.10 & \bf{767.53} & 
4.12\\CON8-9 & 874.49 & 0.09 & 
880.23 & 0.09 & \bf{809.00} & 
8.10\\[1ex]\hline
\end{tabular}
\label{table:nonlin}
\end{table} \clearpage
\begin{table}[ht]
\caption{Resultados de la ejecución de la metaheurística GTS, utilizando instancias de Dethloff con la configuración -mni 6000 -lambda1 0.05 -lambda2 0.05 -tabu 20}
\centering
\small
\begin{tabular}{c c c c c c c}
\hline\hline
Instancia & Costo mínimo & Tiempo(seg.) & Costo promedio & Tiempo promedio(seg.) & Costo GTS & \%Gap \\ [0.5ex]
\hline
SCA3-0 & \bf{636.06} & 4.52 & 
639.43 & 4.21 & 636.06 & 0.00\\
SCA3-1 & \bf{697.84} & 3.07 & 
697.84 & 3.17 & 697.84 & 0.00\\
SCA3-2 & \bf{659.34} & 6.97 & 
659.34 & 6.27 & 659.34 & 0.00\\
SCA3-3 & \bf{680.04} & 5.10 & 
680.45 & 3.61 & 680.04 & 0.00\\
SCA3-4 & \bf{690.50} & 3.25 & 
690.50 & 6.90 & 690.50 & 0.00\\
SCA3-5 & \bf{659.90} & 2.74 & 
666.42 & 3.81 & 659.90 & 0.00\\
SCA3-6 & \bf{651.09} & 5.56 & 
653.70 & 5.75 & 651.09 & 0.00\\
SCA3-7 & 666.15 & 6.72 & 
666.15 & 5.46 & \bf{659.17} & 
1.06\\SCA3-8 & \bf{719.47} & 4.08 & 
719.47 & 5.07 & 719.47 & 0.00\\
SCA3-9 & \bf{681.00} & 3.92 & 
681.00 & 5.67 & 681.00 & 0.00\\
SCA8-0 & \bf{961.50} & 2.26 & 
981.05 & 3.75 & 961.50 & 0.00\\
SCA8-1 & \bf{1050.20} & 7.87 & 
1065.27 & 6.73 & 1050.20 & 0.00\\
SCA8-2 & \bf{1039.64} & 4.13 & 
1054.05 & 3.94 & 1039.64 & 0.00\\
SCA8-3 & \bf{983.34} & 15.47 & 
999.38 & 7.09 & 983.34 & 0.00\\
SCA8-4 & 1067.55 & 5.58 & 
1068.70 & 5.46 & \bf{1065.49} & 
0.19\\SCA8-5 & \bf{1027.08} & 6.64 & 
1048.72 & 4.93 & 1027.08 & 0.00\\
SCA8-6 & 972.48 & 5.22 & 
977.02 & 3.88 & \bf{971.82} & 
0.07\\SCA8-7 & 1054.59 & 4.54 & 
1075.56 & 4.05 & \bf{1052.17} & 
0.23\\SCA8-8 & \bf{1071.18} & 1.88 & 
1077.22 & 3.45 & 1071.18 & 0.00\\
SCA8-9 & 1063.68 & 6.57 & 
1067.29 & 6.01 & \bf{1060.50} & 
0.30\\CON3-0 & 617.59 & 3.94 & 
627.49 & 5.26 & \bf{616.52} & 
0.17\\CON3-1 & \bf{554.47} & 3.93 & 
556.88 & 3.27 & 554.47 & 0.00\\
-CON3-2 & \bf{\underline{519.11}} & 11.60 & 
520.83 & 8.42 & 519.26 & 
0.03\\CON3-3 & \bf{591.19} & 2.77 & 
591.19 & 4.66 & 591.19 & 0.00\\
-CON3-4 & \bf{\underline{588.79}} & 4.60 & 
595.12 & 3.56 & 589.32 & 
0.09\\CON3-5 & \bf{563.70} & 2.84 & 
563.70 & 3.01 & 563.70 & 0.00\\
-CON3-6 & \bf{\underline{499.05}} & 4.09 & 
501.04 & 5.56 & 500.80 & 
0.35\\CON3-7 & \bf{576.48} & 7.00 & 
585.20 & 5.85 & 576.48 & 0.00\\
CON3-8 & \bf{523.05} & 5.32 & 
523.07 & 3.27 & 523.05 & 0.00\\
-CON3-9 & \bf{\underline{578.25}} & 3.72 & 
579.38 & 6.28 & 580.05 & 
0.31\\CON8-0 & \bf{857.17} & 5.49 & 
881.11 & 3.67 & 857.17 & 0.00\\
CON8-1 & \bf{740.85} & 7.62 & 
766.52 & 5.62 & 740.85 & 0.00\\
CON8-2 & 716.03 & 9.56 & 
728.10 & 5.85 & \bf{713.44} & 
0.36\\CON8-3 & \bf{811.07} & 7.18 & 
816.33 & 6.48 & 811.07 & 0.00\\
CON8-4 & \bf{772.25} & 7.38 & 
783.80 & 3.67 & 772.25 & 0.00\\
-CON8-5 & \bf{\underline{755.67}} & 2.78 & 
758.25 & 4.50 & 756.91 & 
0.16\\CON8-6 & 688.47 & 6.86 & 
693.18 & 5.78 & \bf{678.92} & 
1.41\\CON8-7 & 812.26 & 5.32 & 
827.57 & 5.18 & \bf{811.96} & 
0.04\\CON8-8 & \bf{767.53} & 4.31 & 
767.71 & 4.52 & 767.53 & 0.00\\
CON8-9 & \bf{809.00} & 4.37 & 
813.58 & 5.41 & 809.00 & 0.00\\
[1ex]\hline
\end{tabular}
\label{table:nonlin}
\end{table} \clearpage
\begin{table}[ht]
\caption{Resultados de la ejecución de la metaheurística GTS, utilizando instancias de Dethloff con la configuración -mni 6000 -lambda1 0.05 -lambda2 0.05 -tabu 20}
\centering
\small
\begin{tabular}{c c c c c c c}
\hline\hline
Instancia & Costo mínimo & Tiempo(seg.) & Costo promedio & Tiempo promedio(seg.) & Costo GTS & \%Gap \\ [0.5ex]
\hline
SCA3-0 & \bf{636.06} & 5.46 & 
638.30 & 3.92 & 636.06 & 0.00\\
SCA3-1 & \bf{697.84} & 5.74 & 
702.06 & 4.55 & 697.84 & 0.00\\
SCA3-2 & \bf{659.34} & 2.84 & 
659.34 & 3.75 & 659.34 & 0.00\\
SCA3-3 & \bf{680.04} & 3.17 & 
682.76 & 3.48 & 680.04 & 0.00\\
SCA3-4 & \bf{690.50} & 5.03 & 
690.50 & 3.23 & 690.50 & 0.00\\
SCA3-5 & \bf{659.90} & 3.70 & 
659.90 & 4.26 & 659.90 & 0.00\\
SCA3-6 & \bf{651.09} & 5.19 & 
654.35 & 3.33 & 651.09 & 0.00\\
SCA3-7 & 666.15 & 5.89 & 
666.15 & 4.44 & \bf{659.17} & 
1.06\\SCA3-8 & \bf{719.47} & 8.62 & 
719.47 & 6.55 & 719.47 & 0.00\\
SCA3-9 & \bf{681.00} & 3.26 & 
681.00 & 3.73 & 681.00 & 0.00\\
SCA8-0 & 970.64 & 5.16 & 
976.48 & 5.41 & \bf{961.50} & 
0.95\\SCA8-1 & \bf{\underline{1049.65}} & 2.86 & 
1059.56 & 4.59 & 1050.20 & 
-0.05\\SCA8-2 & \bf{1039.64} & 4.22 & 
1045.64 & 3.88 & 1039.64 & 0.00\\
SCA8-3 & \bf{983.34} & 6.28 & 
1001.83 & 5.02 & 983.34 & 0.00\\
SCA8-4 & \bf{1065.49} & 6.57 & 
1068.10 & 4.61 & 1065.49 & 0.00\\
SCA8-5 & 1042.30 & 5.09 & 
1048.88 & 5.42 & \bf{1027.08} & 
1.48\\SCA8-6 & 972.48 & 4.08 & 
977.02 & 3.40 & \bf{971.82} & 
0.07\\SCA8-7 & 1063.22 & 5.22 & 
1073.26 & 3.88 & \bf{1052.17} & 
1.05\\SCA8-8 & \bf{1071.18} & 2.88 & 
1079.57 & 3.05 & 1071.18 & 0.00\\
SCA8-9 & \bf{1060.50} & 3.75 & 
1062.34 & 4.56 & 1060.50 & 0.00\\
CON3-0 & \bf{616.52} & 3.96 & 
622.18 & 3.81 & 616.52 & 0.00\\
CON3-1 & \bf{554.47} & 4.54 & 
556.16 & 3.17 & 554.47 & 0.00\\
CON3-2 & \bf{\underline{519.11}} & 7.47 & 
522.20 & 6.73 & 519.26 & 
-0.03\\CON3-3 & \bf{591.19} & 5.50 & 
591.92 & 4.58 & 591.19 & 0.00\\
CON3-4 & \bf{\underline{588.79}} & 7.07 & 
594.41 & 4.73 & 589.32 & 
-0.09\\CON3-5 & \bf{563.70} & 6.89 & 
565.91 & 4.03 & 563.70 & 0.00\\
CON3-6 & \bf{\underline{499.05}} & 3.66 & 
502.24 & 3.30 & 500.80 & 
-0.35\\CON3-7 & \bf{576.48} & 10.39 & 
577.65 & 6.42 & 576.48 & 0.00\\
CON3-8 & \bf{523.05} & 2.42 & 
523.05 & 3.04 & 523.05 & 0.00\\
CON3-9 & 581.35 & 5.03 & 
584.30 & 4.81 & \bf{580.05} & 
0.22\\CON8-0 & 857.42 & 2.70 & 
880.81 & 4.25 & \bf{857.17} & 
0.03\\CON8-1 & \bf{740.85} & 7.98 & 
746.30 & 5.72 & 740.85 & 0.00\\
CON8-2 & 713.60 & 6.27 & 
718.11 & 5.65 & \bf{713.44} & 
0.02\\CON8-3 & \bf{811.07} & 3.49 & 
823.28 & 4.33 & 811.07 & 0.00\\
CON8-4 & \bf{772.25} & 6.74 & 
776.01 & 4.48 & 772.25 & 0.00\\
CON8-5 & \bf{756.91} & 3.61 & 
758.18 & 4.07 & 756.91 & 0.00\\
CON8-6 & \bf{678.92} & 4.21 & 
692.35 & 5.39 & 678.92 & 0.00\\
CON8-7 & 812.89 & 4.12 & 
813.21 & 4.86 & \bf{811.96} & 
0.11\\CON8-8 & \bf{767.53} & 9.00 & 
772.04 & 5.48 & 767.53 & 0.00\\
CON8-9 & 826.64 & 3.50 & 
845.12 & 4.99 & \bf{809.00} & 
2.18\\[1ex]\hline
\end{tabular}
\label{table:nonlin}
\end{table} \clearpage
\begin{table}[ht]
\caption{Resultados de la ejecución de la metaheurística GTS, utilizando instancias de SalhiNagy con la configuración -mni 6000 -lambda1 0.05 -lambda2 0.05 -tabu 20}
\centering
\small
\begin{tabular}{c c c c c c c}
\hline\hline
Instancia & Costo mínimo & Tiempo(seg.) & Costo promedio & Tiempo promedio(seg.) & Costo GTS & \%Gap \\ [0.5ex]
\hline
CMT1X & 470.48 & 3.03 & 
470.48 & 5.04 & \bf{469.80} & 
0.14\\CMT1Y & 470.48 & 9.46 & 
471.19 & 5.97 & \bf{469.80} & 
0.14\\CMT2X & \bf{\underline{683.64}} & 14.89 & 
685.60 & 15.80 & 684.21 & 
-0.08\\CMT2Y & \bf{\underline{683.73}} & 8.53 & 
687.15 & 6.92 & 684.21 & 
-0.07\\CMT3X & \bf{\underline{720.18}} & 25.96 & 
726.57 & 18.73 & 721.27 & 
-0.15\\CMT3Y & \bf{\underline{718.40}} & 32.16 & 
724.38 & 20.30 & 721.27 & 
-0.40\\CMT4X & 862.53 & 70.40 & 
867.82 & 54.43 & \bf{852.46} & 
1.18\\CMT4Y & 857.71 & 94.36 & 
868.20 & 62.59 & \bf{852.46} & 
0.62\\CMT5X & 1058.94 & 70.81 & 
1073.58 & 73.11 & \bf{1030.55} & 
2.75\\CMT5Y & 1044.08 & 124.11 & 
1064.46 & 89.51 & \bf{1030.55} & 
1.31\\CMT11X & 876.35 & 59.41 & 
921.99 & 42.27 & \bf{838.66} & 
4.49\\CMT11Y & 876.71 & 29.13 & 
902.49 & 55.58 & \bf{837.08} & 
4.73\\CMT12X & 673.45 & 15.28 & 
673.77 & 16.30 & \bf{662.22} & 
1.70\\CMT12Y & 673.64 & 15.40 & 
679.67 & 18.35 & \bf{662.22} & 
1.72\\[1ex]\hline
\end{tabular}
\label{table:nonlin}
\end{table} \clearpage
\begin{table}[ht]
\caption{Resultados de la ejecución de la metaheurística ILS, utilizando instancias de Dethloff con la configuración -n 1 -LS 5}
\centering
\small
\begin{tabular}{c c c c c c c}
\hline\hline
Instancia & Costo mínimo & Tiempo(seg.) & Costo promedio & Tiempo promedio(seg.) & Costo ILS & \%Gap \\ [0.5ex]
\hline
SCA3-0 & 640.55 & 0.07 & 
659.09 & 0.06 & \bf{635.62} & 
0.78\\SCA3-1 & 736.71 & 0.04 & 
738.55 & 0.05 & \bf{697.84} & 
5.57\\SCA3-2 & 699.25 & 0.06 & 
699.25 & 0.05 & \bf{659.34} & 
6.05\\SCA3-3 & 682.46 & 0.06 & 
682.46 & 0.06 & \bf{680.04} & 
0.36\\SCA3-4 & 748.61 & 0.06 & 
748.61 & 0.07 & \bf{690.50} & 
8.42\\SCA3-5 & 673.39 & 0.07 & 
677.06 & 0.04 & \bf{659.90} & 
2.04\\SCA3-6 & 663.05 & 0.06 & 
666.38 & 0.05 & \bf{651.09} & 
1.84\\SCA3-7 & 688.49 & 0.05 & 
688.49 & 0.04 & \bf{659.17} & 
4.45\\SCA3-8 & 732.24 & 0.05 & 
755.56 & 0.05 & \bf{719.47} & 
1.77\\SCA3-9 & 784.22 & 0.04 & 
784.22 & 0.04 & \bf{681.00} & 
15.16\\SCA8-0 & 1048.65 & 0.04 & 
1048.65 & 0.06 & \bf{961.50} & 
9.06\\SCA8-1 & 1111.96 & 0.04 & 
1191.92 & 0.05 & \bf{1049.65} & 
5.94\\SCA8-2 & 1069.59 & 0.04 & 
1209.38 & 0.04 & \bf{1039.64} & 
2.88\\SCA8-3 & 1063.56 & 0.04 & 
1072.89 & 0.04 & \bf{983.34} & 
8.16\\SCA8-4 & 1127.99 & 0.04 & 
1127.99 & 0.04 & \bf{1065.49} & 
5.87\\SCA8-5 & 1136.22 & 0.06 & 
1136.22 & 0.06 & \bf{1027.08} & 
10.63\\SCA8-6 & 1040.73 & 0.04 & 
1040.73 & 0.04 & \bf{971.82} & 
7.09\\SCA8-7 & 1134.90 & 0.05 & 
1171.04 & 0.05 & \bf{1051.28} & 
7.95\\SCA8-8 & 1125.75 & 0.04 & 
1125.75 & 0.04 & \bf{1071.18} & 
5.09\\SCA8-9 & 1175.59 & 0.05 & 
1175.59 & 0.04 & \bf{1060.50} & 
10.85\\CON3-0 & 657.63 & 0.05 & 
657.63 & 0.04 & \bf{616.52} & 
6.67\\CON3-1 & 587.53 & 0.06 & 
590.43 & 0.06 & \bf{554.47} & 
5.96\\CON3-2 & 534.95 & 0.02 & 
534.95 & 0.03 & \bf{518.00} & 
3.27\\CON3-3 & 603.19 & 0.06 & 
603.19 & 0.05 & \bf{591.19} & 
2.03\\CON3-4 & 593.78 & 0.05 & 
593.78 & 0.05 & \bf{588.79} & 
0.85\\CON3-5 & 575.00 & 0.06 & 
605.04 & 0.06 & \bf{563.70} & 
2.00\\CON3-6 & 545.10 & 0.08 & 
546.56 & 0.07 & \bf{499.05} & 
9.23\\CON3-7 & 591.17 & 0.05 & 
598.57 & 0.05 & \bf{576.48} & 
2.55\\CON3-8 & 540.17 & 0.04 & 
540.17 & 0.04 & \bf{523.05} & 
3.27\\CON3-9 & 591.10 & 0.06 & 
591.10 & 0.06 & \bf{578.24} & 
2.22\\CON8-0 & 957.85 & 0.04 & 
964.28 & 0.04 & \bf{857.17} & 
11.75\\CON8-1 & 779.49 & 0.06 & 
797.48 & 0.05 & \bf{740.85} & 
5.22\\CON8-2 & 754.35 & 0.05 & 
758.05 & 0.04 & \bf{712.89} & 
5.82\\CON8-3 & 865.57 & 0.04 & 
865.57 & 0.04 & \bf{811.07} & 
6.72\\CON8-4 & 835.61 & 0.06 & 
851.96 & 0.05 & \bf{772.25} & 
8.20\\CON8-5 & 795.51 & 0.06 & 
797.46 & 0.06 & \bf{754.88} & 
5.38\\CON8-6 & 706.23 & 0.07 & 
712.22 & 0.06 & \bf{678.92} & 
4.02\\CON8-7 & 922.35 & 0.05 & 
922.35 & 0.05 & \bf{811.96} & 
13.60\\CON8-8 & 821.47 & 0.05 & 
821.94 & 0.05 & \bf{767.53} & 
7.03\\CON8-9 & 856.37 & 0.04 & 
875.52 & 0.05 & \bf{809.00} & 
5.86\\[1ex]\hline
\end{tabular}
\label{table:nonlin}
\end{table} \clearpage
\begin{table}[ht]
\caption{Resultados de la ejecución de la metaheurística ILS, utilizando instancias de SalhiNagy con la configuración -n 30 -LS 80}
\centering
\small
\begin{tabular}{c c c c c c c}
\hline\hline
Instancia & Costo mínimo & Tiempo(seg.) & Costo promedio & Tiempo promedio(seg.) & Costo ILS & \%Gap \\ [0.5ex]
\hline
CMT1X & 470.48 & 7.15 & 
478.58 & 7.63 & \bf{466.77} & 
0.79\\CMT1Y & 472.58 & 6.54 & 
481.51 & 7.03 & \bf{466.77} & 
1.24\\CMT2X & 705.53 & 14.09 & 
711.34 & 14.46 & \bf{684.21} & 
3.12\\CMT2Y & 707.99 & 13.07 & 
711.35 & 14.37 & \bf{684.21} & 
3.48\\CMT3X & 729.38 & 35.45 & 
734.79 & 35.05 & \bf{721.40} & 
1.11\\CMT3Y & 723.85 & 35.77 & 
734.15 & 35.56 & \bf{721.40} & 
0.34\\CMT4X & 897.46 & 88.12 & 
902.82 & 85.71 & \bf{852.83} & 
5.23\\CMT4Y & 903.48 & 93.70 & 
908.91 & 89.79 & \bf{852.46} & 
5.99\\CMT5X & 1105.15 & 233.37 & 
1106.81 & 255.15 & \bf{1030.55} & 
7.24\\CMT5Y & 1096.06 & 261.62 & 
1106.46 & 214.43 & \bf{1031.17} & 
6.29\\CMT11X & 848.73 & 84.97 & 
890.79 & 72.98 & \bf{839.39} & 
1.11\\CMT11Y & 891.13 & 63.31 & 
905.89 & 68.50 & \bf{841.88} & 
5.85\\CMT12X & 676.97 & 31.07 & 
679.78 & 35.05 & \bf{662.22} & 
2.23\\CMT12Y & 676.34 & 30.35 & 
683.40 & 28.71 & \bf{662.22} & 
2.13\\[1ex]\hline
\end{tabular}
\label{table:nonlin}
\end{table} \clearpage
\begin{table}[ht]
\caption{Resultados de la ejecución de la metaheurística ACO, utilizando instancias de Dethloff con la configuración -n 20 -alpha 1.0 -beta 3.0 -q 0.8 -ro 0.015}
\centering
\small
\begin{tabular}{c c c c c c c}
\hline\hline
Instancia & Costo mínimo & Tiempo(seg.) & Costo promedio & Tiempo promedio(seg.) & Costo ACO & \%Gap \\ [0.5ex]
\hline
SCA3-0 & \bf{\underline{636.06}} & 13.96 & 
636.06 & 13.35 & 636.10 & 
-0.01\\SCA3-1 & \bf{\underline{697.84}} & 14.36 & 
697.84 & 14.34 & 700.10 & 
-0.32\\SCA3-2 & 659.34 & 12.54 & 
659.79 & 12.60 & \bf{659.30} & 
0.01\\SCA3-3 & 680.04 & 14.06 & 
680.18 & 13.17 & \bf{680.00} & 
0.01\\SCA3-4 & \bf{690.50} & 13.78 & 
690.50 & 14.23 & 690.50 & 0.00\\
SCA3-5 & \bf{\underline{661.07}} & 14.92 & 
664.35 & 14.52 & 671.10 & 
-1.49\\SCA3-6 & 652.94 & 13.04 & 
652.94 & 13.40 & \bf{651.10} & 
0.28\\SCA3-7 & 666.15 & 10.51 & 
666.15 & 10.51 & \bf{666.10} & 
0.01\\SCA3-8 & \bf{\underline{719.47}} & 11.26 & 
721.21 & 12.70 & 719.50 & 
-0.00\\SCA3-9 & \bf{681.00} & 12.26 & 
681.00 & 11.26 & 681.00 & 0.00\\
SCA8-0 & 968.79 & 16.14 & 
971.89 & 15.02 & \bf{961.60} & 
0.75\\SCA8-1 & \bf{\underline{1059.21}} & 11.76 & 
1064.75 & 11.82 & 1063.00 & 
-0.36\\SCA8-2 & 1046.29 & 10.39 & 
1050.02 & 10.56 & \bf{1040.60} & 
0.55\\SCA8-3 & 1007.63 & 14.48 & 
1014.41 & 14.37 & \bf{985.90} & 
2.20\\SCA8-4 & \bf{\underline{1067.66}} & 14.58 & 
1074.68 & 14.19 & 1071.00 & 
-0.31\\SCA8-5 & \bf{\underline{1051.94}} & 16.04 & 
1055.20 & 15.45 & 1054.30 & 
-0.22\\SCA8-6 & 977.03 & 15.09 & 
978.34 & 15.42 & \bf{972.50} & 
0.47\\SCA8-7 & 1067.20 & 16.46 & 
1067.20 & 15.98 & \bf{1059.70} & 
0.71\\SCA8-8 & \bf{\underline{1071.18}} & 14.66 & 
1071.18 & 14.60 & 1082.70 & 
-1.06\\SCA8-9 & \bf{\underline{1067.42}} & 11.22 & 
1067.42 & 11.42 & 1081.40 & 
-1.29\\CON3-0 & 616.52 & 14.39 & 
621.69 & 14.66 & \bf{616.50} & 
0.00\\CON3-1 & \bf{\underline{554.47}} & 13.79 & 
555.15 & 13.63 & 555.60 & 
-0.20\\CON3-2 & \bf{\underline{519.11}} & 11.92 & 
520.81 & 12.46 & 521.40 & 
-0.44\\CON3-3 & \bf{\underline{591.19}} & 16.34 & 
591.20 & 15.05 & 591.20 & 
-0.00\\CON3-4 & \bf{\underline{588.79}} & 12.96 & 
588.79 & 12.78 & 589.30 & 
-0.09\\CON3-5 & \bf{563.70} & 13.22 & 
564.59 & 13.31 & 563.70 & 0.00\\
CON3-6 & 500.80 & 17.36 & 
500.80 & 16.44 & \bf{499.20} & 
0.32\\CON3-7 & 578.22 & 11.09 & 
578.36 & 11.91 & \bf{577.50} & 
0.12\\CON3-8 & 523.68 & 11.72 & 
525.35 & 11.56 & \bf{523.10} & 
0.11\\CON3-9 & 586.31 & 12.50 & 
587.39 & 12.48 & \bf{578.20} & 
1.40\\CON8-0 & 865.86 & 14.49 & 
876.91 & 13.93 & \bf{858.90} & 
0.81\\CON8-1 & \bf{\underline{740.85}} & 13.56 & 
745.00 & 13.46 & 740.90 & 
-0.01\\CON8-2 & \bf{\underline{713.44}} & 18.66 & 
713.73 & 18.24 & 714.30 & 
-0.12\\CON8-3 & \bf{\underline{811.07}} & 13.24 & 
815.50 & 13.35 & 812.30 & 
-0.15\\CON8-4 & 776.37 & 13.74 & 
783.87 & 14.23 & \bf{770.10} & 
0.81\\CON8-5 & \bf{\underline{760.03}} & 13.82 & 
762.80 & 13.14 & 766.60 & 
-0.86\\CON8-6 & \bf{\underline{685.06}} & 15.37 & 
693.07 & 15.40 & 697.20 & 
-1.74\\CON8-7 & \bf{\underline{814.79}} & 11.03 & 
816.72 & 11.89 & 814.80 & 
-0.00\\CON8-8 & 782.86 & 15.56 & 
787.86 & 15.34 & \bf{771.30} & 
1.50\\CON8-9 & \bf{\underline{810.18}} & 14.80 & 
812.58 & 14.74 & 815.10 & 
-0.60\\[1ex]\hline
\end{tabular}
\label{table:nonlin}
\end{table} \clearpage
\begin{table}[ht]
\caption{Resultados de la ejecución de la metaheurística ACO, utilizando instancias de SalhiNagy con la configuración -n 20 -alpha 1.0 -beta 3.0 -q 0.8 -ro 0.015}
\centering
\small
\begin{tabular}{c c c c c c c}
\hline\hline
Instancia & Costo mínimo & Tiempo(seg.) & Costo promedio & Tiempo promedio(seg.) & Costo ACO & \%Gap \\ [0.5ex]
\hline
CMT1X & \bf{\underline{470.48}} & 13.16 & 
474.88 & 12.47 & 470.67 & 
-0.04\\CMT1Y & \bf{472.37} & 12.43 & 
475.17 & 12.62 & 472.37 & 0.00\\
CMT2X & \bf{\underline{691.23}} & 59.61 & 
697.12 & 58.31 & 705.24 & 
-1.99\\CMT2Y & \bf{\underline{696.83}} & 54.87 & 
697.66 & 55.74 & 704.16 & 
-1.04\\CMT3X & \bf{\underline{726.36}} & 181.67 & 
727.75 & 178.53 & 726.55 & 
-0.03\\CMT3Y & \bf{\underline{727.90}} & 178.97 & 
729.25 & 185.24 & 729.02 & 
-0.15\\CMT4X & \bf{\underline{877.42}} & 780.46 & 
882.70 & 794.30 & 893.90 & 
-1.84\\CMT4Y & \bf{\underline{866.19}} & 811.52 & 
878.62 & 787.42 & 895.25 & 
-3.25\\CMT5X & \bf{\underline{1081.05}} & 2301.47 & 
1083.93 & 2428.96 & 1115.75 & 
-3.11\\CMT5Y & \bf{\underline{1076.89}} & 2893.52 & 
1077.91 & 2338.42 & 1112.61 & 
-3.21\\CMT11X & \bf{\underline{851.98}} & 366.81 & 
858.39 & 361.19 & 887.36 & 
-3.99\\CMT11Y & \bf{\underline{852.18}} & 262.41 & 
861.63 & 287.29 & 874.13 & 
-2.51\\CMT12X & \bf{\underline{672.79}} & 162.30 & 
673.74 & 165.12 & 681.02 & 
-1.21\\CMT12Y & 672.74 & 131.84 & 
676.71 & 136.62 & \bf{671.32} & 
0.21\\[1ex]\hline
\end{tabular}
\label{table:nonlin}
\end{table} \clearpage
\begin{table}[ht]
\caption{Resultados de la ejecución de la metaheurística ACO, utilizando instancias de Dethloff con la configuración -n 1 -alpha 1.0 -beta 3.0 -q 0.8 -ro 0.015}
\centering
\small
\begin{tabular}{c c c c c c c}
\hline\hline
Instancia & Costo mínimo & Tiempo(seg.) & Costo promedio & Tiempo promedio(seg.) & Costo ACO & \%Gap \\ [0.5ex]
\hline
SCA3-0 & \bf{\underline{636.06}} & 0.72 & 
640.79 & 0.67 & 636.10 & 
-0.01\\SCA3-1 & \bf{\underline{697.84}} & 0.70 & 
698.76 & 0.72 & 700.10 & 
-0.32\\SCA3-2 & 667.37 & 0.62 & 
669.07 & 0.66 & \bf{659.30} & 
1.22\\SCA3-3 & 680.60 & 0.58 & 
680.92 & 0.71 & \bf{680.00} & 
0.09\\SCA3-4 & \bf{690.50} & 0.62 & 
690.50 & 0.65 & 690.50 & 0.00\\
SCA3-5 & \bf{\underline{662.75}} & 0.66 & 
673.02 & 0.68 & 671.10 & 
-1.24\\SCA3-6 & 652.94 & 0.68 & 
656.34 & 0.64 & \bf{651.10} & 
0.28\\SCA3-7 & 666.15 & 0.46 & 
666.42 & 0.47 & \bf{666.10} & 
0.01\\SCA3-8 & 726.22 & 0.60 & 
729.05 & 0.64 & \bf{719.50} & 
0.93\\SCA3-9 & \bf{681.00} & 0.52 & 
681.86 & 0.55 & 681.00 & 0.00\\
SCA8-0 & \bf{\underline{961.50}} & 0.76 & 
971.38 & 0.81 & 961.60 & 
-0.01\\SCA8-1 & 1070.43 & 0.62 & 
1073.62 & 0.60 & \bf{1063.00} & 
0.70\\SCA8-2 & 1050.37 & 0.60 & 
1050.84 & 0.58 & \bf{1040.60} & 
0.94\\SCA8-3 & 1018.45 & 0.72 & 
1024.84 & 0.70 & \bf{985.90} & 
3.30\\SCA8-4 & \bf{\underline{1065.49}} & 0.69 & 
1070.74 & 0.70 & 1071.00 & 
-0.51\\SCA8-5 & 1058.14 & 0.82 & 
1060.02 & 0.79 & \bf{1054.30} & 
0.36\\SCA8-6 & \bf{\underline{972.48}} & 0.78 & 
981.36 & 0.78 & 972.50 & 
-0.00\\SCA8-7 & 1075.87 & 0.82 & 
1083.14 & 0.80 & \bf{1059.70} & 
1.53\\SCA8-8 & \bf{\underline{1071.18}} & 0.73 & 
1079.95 & 0.71 & 1082.70 & 
-1.06\\SCA8-9 & \bf{\underline{1067.42}} & 0.58 & 
1067.42 & 0.58 & 1081.40 & 
-1.29\\CON3-0 & 617.59 & 0.77 & 
624.09 & 0.77 & \bf{616.50} & 
0.18\\CON3-1 & 557.38 & 0.73 & 
559.51 & 0.72 & \bf{555.60} & 
0.32\\CON3-2 & \bf{\underline{519.11}} & 0.63 & 
526.15 & 0.65 & 521.40 & 
-0.44\\CON3-3 & \bf{591.20} & 0.74 & 
593.94 & 0.74 & 591.20 & 0.00\\
CON3-4 & 589.32 & 0.63 & 
592.29 & 0.60 & \bf{589.30} & 
0.00\\CON3-5 & 568.69 & 0.73 & 
568.78 & 0.72 & \bf{563.70} & 
0.89\\CON3-6 & 504.15 & 0.76 & 
508.66 & 0.81 & \bf{499.20} & 
0.99\\CON3-7 & 582.33 & 0.57 & 
586.78 & 0.62 & \bf{577.50} & 
0.84\\CON3-8 & 523.14 & 0.70 & 
526.41 & 0.66 & \bf{523.10} & 
0.01\\CON3-9 & 587.78 & 0.66 & 
589.33 & 0.63 & \bf{578.20} & 
1.66\\CON8-0 & 871.92 & 0.75 & 
881.45 & 0.76 & \bf{858.90} & 
1.52\\CON8-1 & 742.44 & 0.71 & 
744.89 & 0.71 & \bf{740.90} & 
0.21\\CON8-2 & 716.72 & 0.94 & 
718.02 & 0.96 & \bf{714.30} & 
0.34\\CON8-3 & 817.57 & 0.80 & 
829.32 & 0.77 & \bf{812.30} & 
0.65\\CON8-4 & 792.11 & 0.74 & 
796.19 & 0.73 & \bf{770.10} & 
2.86\\CON8-5 & \bf{\underline{762.61}} & 0.69 & 
763.85 & 0.69 & 766.60 & 
-0.52\\CON8-6 & \bf{\underline{689.11}} & 0.88 & 
693.70 & 0.81 & 697.20 & 
-1.16\\CON8-7 & 815.79 & 0.60 & 
824.03 & 0.61 & \bf{814.80} & 
0.12\\CON8-8 & 790.88 & 0.79 & 
794.49 & 0.77 & \bf{771.30} & 
2.54\\CON8-9 & 815.49 & 0.82 & 
817.21 & 0.79 & \bf{815.10} & 
0.05\\[1ex]\hline
\end{tabular}
\label{table:nonlin}
\end{table} \clearpage
\begin{table}[ht]
\caption{Resultados de la ejecución de la metaheurística ACO, utilizando instancias de Dethloff con la configuración -n 5 -alpha 1.0 -beta 3.0 -q 0.8 -ro 0.015}
\centering
\small
\begin{tabular}{c c c c c c c}
\hline\hline
Instancia & Costo mínimo & Tiempo(seg.) & Costo promedio & Tiempo promedio(seg.) & Costo ACO & \%Gap \\ [0.5ex]
\hline
SCA3-0 & \bf{\underline{636.06}} & 3.25 & 
636.06 & 3.31 & 636.10 & 
-0.01\\SCA3-1 & \bf{\underline{697.84}} & 3.79 & 
697.84 & 3.67 & 700.10 & 
-0.32\\SCA3-2 & 659.34 & 3.16 & 
662.21 & 3.23 & \bf{659.30} & 
0.01\\SCA3-3 & 680.04 & 3.52 & 
680.32 & 3.42 & \bf{680.00} & 
0.01\\SCA3-4 & \bf{690.50} & 3.66 & 
690.50 & 3.54 & 690.50 & 0.00\\
SCA3-5 & \bf{\underline{662.75}} & 3.59 & 
664.20 & 3.64 & 671.10 & 
-1.24\\SCA3-6 & 652.94 & 3.24 & 
653.72 & 3.31 & \bf{651.10} & 
0.28\\SCA3-7 & 666.15 & 2.80 & 
666.15 & 2.85 & \bf{666.10} & 
0.01\\SCA3-8 & \bf{\underline{719.47}} & 3.00 & 
722.49 & 3.08 & 719.50 & 
-0.00\\SCA3-9 & \bf{681.00} & 2.66 & 
681.00 & 2.72 & 681.00 & 0.00\\
SCA8-0 & 973.03 & 3.45 & 
980.88 & 3.60 & \bf{961.60} & 
1.19\\SCA8-1 & 1065.63 & 2.90 & 
1067.88 & 2.95 & \bf{1063.00} & 
0.25\\SCA8-2 & 1050.17 & 2.62 & 
1050.99 & 2.61 & \bf{1040.60} & 
0.92\\SCA8-3 & 1010.76 & 3.59 & 
1016.14 & 3.57 & \bf{985.90} & 
2.52\\SCA8-4 & \bf{\underline{1065.49}} & 3.90 & 
1070.50 & 3.70 & 1071.00 & 
-0.51\\SCA8-5 & \bf{\underline{1034.74}} & 3.67 & 
1046.45 & 4.16 & 1054.30 & 
-1.86\\SCA8-6 & 977.03 & 4.00 & 
980.19 & 3.92 & \bf{972.50} & 
0.47\\SCA8-7 & 1063.22 & 3.74 & 
1068.67 & 3.92 & \bf{1059.70} & 
0.33\\SCA8-8 & \bf{\underline{1082.11}} & 3.56 & 
1084.12 & 3.75 & 1082.70 & 
-0.05\\SCA8-9 & \bf{\underline{1067.42}} & 2.83 & 
1067.42 & 2.83 & 1081.40 & 
-1.29\\CON3-0 & 621.82 & 4.07 & 
624.06 & 3.88 & \bf{616.50} & 
0.86\\CON3-1 & \bf{\underline{554.47}} & 3.39 & 
556.61 & 3.57 & 555.60 & 
-0.20\\CON3-2 & \bf{\underline{521.38}} & 3.37 & 
522.75 & 3.08 & 521.40 & 
-0.00\\CON3-3 & \bf{\underline{591.19}} & 3.88 & 
591.20 & 3.64 & 591.20 & 
-0.00\\CON3-4 & \bf{\underline{588.79}} & 3.05 & 
588.79 & 3.12 & 589.30 & 
-0.09\\CON3-5 & 564.88 & 3.82 & 
566.06 & 3.44 & \bf{563.70} & 
0.21\\CON3-6 & 500.80 & 3.98 & 
502.77 & 4.27 & \bf{499.20} & 
0.32\\CON3-7 & \bf{\underline{576.84}} & 3.00 & 
578.02 & 3.07 & 577.50 & 
-0.11\\CON3-8 & 523.14 & 3.16 & 
524.01 & 3.02 & \bf{523.10} & 
0.01\\CON3-9 & 582.86 & 3.17 & 
586.81 & 2.95 & \bf{578.20} & 
0.81\\CON8-0 & 869.43 & 3.44 & 
875.34 & 3.59 & \bf{858.90} & 
1.23\\CON8-1 & \bf{\underline{740.85}} & 3.48 & 
746.22 & 3.56 & 740.90 & 
-0.01\\CON8-2 & \bf{\underline{713.44}} & 4.63 & 
715.21 & 4.61 & 714.30 & 
-0.12\\CON8-3 & 812.54 & 3.26 & 
816.31 & 3.43 & \bf{812.30} & 
0.03\\CON8-4 & 776.84 & 3.38 & 
787.27 & 3.43 & \bf{770.10} & 
0.88\\CON8-5 & \bf{\underline{759.93}} & 3.25 & 
763.13 & 3.41 & 766.60 & 
-0.87\\CON8-6 & \bf{\underline{691.59}} & 3.81 & 
695.34 & 4.01 & 697.20 & 
-0.80\\CON8-7 & \bf{\underline{814.77}} & 2.82 & 
816.52 & 2.92 & 814.80 & 
-0.00\\CON8-8 & 788.09 & 3.93 & 
793.58 & 3.89 & \bf{771.30} & 
2.18\\CON8-9 & \bf{\underline{812.60}} & 3.76 & 
814.48 & 3.77 & 815.10 & 
-0.31\\[1ex]\hline
\end{tabular}
\label{table:nonlin}
\end{table} \clearpage
\begin{table}[ht]
\caption{Resultados de la ejecución de la metaheurística ACO, utilizando instancias de Dethloff con la configuración -n 50 -alpha 1.0 -beta 3.0 -q 0.8 -ro 0.015}
\centering
\small
\begin{tabular}{c c c c c c c}
\hline\hline
Instancia & Costo mínimo & Tiempo(seg.) & Costo promedio & Tiempo promedio(seg.) & Costo ACO & \%Gap \\ [0.5ex]
\hline
SCA3-0 & \bf{\underline{636.06}} & 32.69 & 
636.06 & 32.48 & 636.10 & 
-0.01\\SCA3-1 & \bf{\underline{697.84}} & 33.91 & 
697.84 & 35.08 & 700.10 & 
-0.32\\SCA3-2 & 659.34 & 29.80 & 
659.34 & 32.05 & \bf{659.30} & 
0.01\\SCA3-3 & 680.04 & 33.06 & 
680.04 & 32.41 & \bf{680.00} & 
0.01\\SCA3-4 & \bf{690.50} & 33.65 & 
690.50 & 34.28 & 690.50 & 0.00\\
SCA3-5 & \bf{\underline{659.90}} & 34.65 & 
663.33 & 36.15 & 671.10 & 
-1.67\\SCA3-6 & \bf{\underline{651.09}} & 32.62 & 
652.01 & 33.25 & 651.10 & 
-0.00\\SCA3-7 & 666.15 & 27.84 & 
666.15 & 28.84 & \bf{666.10} & 
0.01\\SCA3-8 & \bf{\underline{719.47}} & 28.78 & 
719.47 & 29.68 & 719.50 & 
-0.00\\SCA3-9 & \bf{681.00} & 29.64 & 
681.00 & 27.04 & 681.00 & 0.00\\
SCA8-0 & \bf{\underline{961.50}} & 35.31 & 
968.64 & 36.23 & 961.60 & 
-0.01\\SCA8-1 & \bf{\underline{1054.87}} & 29.58 & 
1060.17 & 29.04 & 1063.00 & 
-0.76\\SCA8-2 & 1042.61 & 26.56 & 
1049.36 & 26.47 & \bf{1040.60} & 
0.19\\SCA8-3 & 1008.22 & 38.11 & 
1012.51 & 37.80 & \bf{985.90} & 
2.26\\SCA8-4 & \bf{\underline{1065.49}} & 34.59 & 
1067.42 & 35.40 & 1071.00 & 
-0.51\\SCA8-5 & \bf{\underline{1034.74}} & 37.86 & 
1046.47 & 37.05 & 1054.30 & 
-1.86\\SCA8-6 & \bf{\underline{972.48}} & 40.91 & 
977.64 & 38.89 & 972.50 & 
-0.00\\SCA8-7 & 1067.20 & 39.89 & 
1067.20 & 41.73 & \bf{1059.70} & 
0.71\\SCA8-8 & \bf{\underline{1071.18}} & 36.02 & 
1071.18 & 35.94 & 1082.70 & 
-1.06\\SCA8-9 & \bf{\underline{1067.42}} & 29.94 & 
1067.42 & 28.75 & 1081.40 & 
-1.29\\CON3-0 & 616.52 & 40.21 & 
618.63 & 37.67 & \bf{616.50} & 
0.00\\CON3-1 & \bf{\underline{554.47}} & 35.39 & 
555.15 & 34.16 & 555.60 & 
-0.20\\CON3-2 & \bf{\underline{521.38}} & 33.02 & 
522.51 & 31.72 & 521.40 & 
-0.00\\CON3-3 & \bf{\underline{591.19}} & 36.33 & 
591.19 & 36.62 & 591.20 & 
-0.00\\CON3-4 & \bf{\underline{588.79}} & 31.30 & 
588.79 & 31.67 & 589.30 & 
-0.09\\CON3-5 & \bf{563.70} & 31.60 & 
564.81 & 33.18 & 563.70 & 0.00\\
CON3-6 & 500.37 & 39.09 & 
501.25 & 39.34 & \bf{499.20} & 
0.23\\CON3-7 & 577.54 & 32.13 & 
578.14 & 30.15 & \bf{577.50} & 
0.01\\CON3-8 & \bf{\underline{523.05}} & 28.46 & 
523.54 & 29.59 & 523.10 & 
-0.01\\CON3-9 & 580.78 & 29.41 & 
583.42 & 30.80 & \bf{578.20} & 
0.45\\CON8-0 & 870.22 & 35.22 & 
873.63 & 34.87 & \bf{858.90} & 
1.32\\CON8-1 & \bf{\underline{740.85}} & 33.35 & 
741.21 & 33.65 & 740.90 & 
-0.01\\CON8-2 & \bf{\underline{712.89}} & 46.06 & 
713.40 & 45.92 & 714.30 & 
-0.20\\CON8-3 & \bf{\underline{811.23}} & 35.71 & 
813.00 & 34.73 & 812.30 & 
-0.13\\CON8-4 & 776.37 & 35.67 & 
780.64 & 34.63 & \bf{770.10} & 
0.81\\CON8-5 & \bf{\underline{759.93}} & 33.97 & 
761.88 & 34.52 & 766.60 & 
-0.87\\CON8-6 & \bf{\underline{689.23}} & 37.69 & 
692.13 & 37.46 & 697.20 & 
-1.14\\CON8-7 & \bf{\underline{814.50}} & 28.43 & 
814.74 & 28.96 & 814.80 & 
-0.04\\CON8-8 & 777.98 & 40.32 & 
785.56 & 39.12 & \bf{771.30} & 
0.87\\CON8-9 & \bf{\underline{814.57}} & 34.67 & 
815.42 & 36.67 & 815.10 & 
-0.07\\[1ex]\hline
\end{tabular}
\label{table:nonlin}
\end{table} \clearpage
\begin{table}[ht]
\caption{Resultados de la ejecución de la metaheurística ACO, utilizando instancias de Dethloff con la configuración -n 100 -alpha 1.0 -beta 3.0 -q 0.8 -ro 0.015}
\centering
\small
\begin{tabular}{c c c c c c c}
\hline\hline
Instancia & Costo mínimo & Tiempo(seg.) & Costo promedio & Tiempo promedio(seg.) & Costo ACO & \%Gap \\ [0.5ex]
\hline
SCA3-0 & \bf{\underline{636.06}} & 67.33 & 
636.06 & 67.13 & 636.10 & 
-0.01\\SCA3-1 & \bf{\underline{697.84}} & 74.43 & 
697.84 & 71.70 & 700.10 & 
-0.32\\SCA3-2 & 659.34 & 64.34 & 
661.00 & 63.94 & \bf{659.30} & 
0.01\\SCA3-3 & 680.04 & 66.84 & 
680.04 & 66.86 & \bf{680.00} & 
0.01\\SCA3-4 & \bf{690.50} & 68.36 & 
690.50 & 68.07 & 690.50 & 0.00\\
SCA3-5 & \bf{\underline{659.90}} & 70.78 & 
660.61 & 72.51 & 671.10 & 
-1.67\\SCA3-6 & \bf{\underline{651.09}} & 66.98 & 
651.09 & 69.02 & 651.10 & 
-0.00\\SCA3-7 & 666.15 & 57.52 & 
666.15 & 59.44 & \bf{666.10} & 
0.01\\SCA3-8 & \bf{\underline{719.47}} & 61.35 & 
719.47 & 60.59 & 719.50 & 
-0.00\\SCA3-9 & \bf{681.00} & 55.24 & 
681.00 & 53.83 & 681.00 & 0.00\\
SCA8-0 & \bf{\underline{961.50}} & 68.04 & 
971.38 & 75.64 & 961.60 & 
-0.01\\SCA8-1 & \bf{\underline{1057.04}} & 60.15 & 
1059.19 & 57.45 & 1063.00 & 
-0.56\\SCA8-2 & 1049.22 & 49.63 & 
1050.29 & 54.27 & \bf{1040.60} & 
0.83\\SCA8-3 & 1007.97 & 78.98 & 
1010.89 & 73.64 & \bf{985.90} & 
2.24\\SCA8-4 & \bf{\underline{1067.66}} & 66.56 & 
1073.98 & 67.28 & 1071.00 & 
-0.31\\SCA8-5 & \bf{\underline{1050.09}} & 79.89 & 
1052.47 & 76.86 & 1054.30 & 
-0.40\\SCA8-6 & \bf{\underline{972.48}} & 69.16 & 
973.82 & 76.97 & 972.50 & 
-0.00\\SCA8-7 & \bf{\underline{1054.73}} & 78.26 & 
1065.05 & 81.27 & 1059.70 & 
-0.47\\SCA8-8 & \bf{\underline{1071.18}} & 78.91 & 
1071.18 & 74.48 & 1082.70 & 
-1.06\\SCA8-9 & \bf{\underline{1067.42}} & 58.09 & 
1067.42 & 57.64 & 1081.40 & 
-1.29\\CON3-0 & 617.59 & 76.08 & 
619.17 & 75.50 & \bf{616.50} & 
0.18\\CON3-1 & \bf{\underline{554.47}} & 66.86 & 
554.86 & 68.38 & 555.60 & 
-0.20\\CON3-2 & \bf{\underline{519.11}} & 65.63 & 
520.25 & 60.52 & 521.40 & 
-0.44\\CON3-3 & \bf{\underline{591.19}} & 70.87 & 
591.19 & 73.21 & 591.20 & 
-0.00\\CON3-4 & \bf{\underline{588.79}} & 62.58 & 
588.79 & 60.59 & 589.30 & 
-0.09\\CON3-5 & \bf{563.70} & 61.18 & 
564.81 & 68.29 & 563.70 & 0.00\\
CON3-6 & \bf{\underline{499.05}} & 89.11 & 
500.25 & 84.09 & 499.20 & 
-0.03\\CON3-7 & 577.54 & 60.58 & 
578.01 & 60.40 & \bf{577.50} & 
0.01\\CON3-8 & \bf{\underline{523.05}} & 57.58 & 
523.37 & 59.45 & 523.10 & 
-0.01\\CON3-9 & 578.98 & 59.08 & 
585.54 & 61.79 & \bf{578.20} & 
0.13\\CON8-0 & 861.87 & 71.59 & 
868.07 & 70.00 & \bf{858.90} & 
0.35\\CON8-1 & \bf{\underline{740.85}} & 69.61 & 
740.85 & 67.88 & 740.90 & 
-0.01\\CON8-2 & \bf{\underline{713.44}} & 90.98 & 
713.63 & 95.44 & 714.30 & 
-0.12\\CON8-3 & \bf{\underline{811.23}} & 69.91 & 
813.81 & 66.69 & 812.30 & 
-0.13\\CON8-4 & 776.34 & 72.90 & 
776.51 & 70.59 & \bf{770.10} & 
0.81\\CON8-5 & \bf{\underline{758.12}} & 73.30 & 
760.07 & 69.31 & 766.60 & 
-1.11\\CON8-6 & \bf{\underline{684.05}} & 85.14 & 
689.75 & 79.43 & 697.20 & 
-1.89\\CON8-7 & \bf{\underline{814.79}} & 59.31 & 
814.89 & 58.97 & 814.80 & 
-0.00\\CON8-8 & 778.62 & 76.46 & 
782.47 & 75.53 & \bf{771.30} & 
0.95\\CON8-9 & \bf{\underline{813.16}} & 72.85 & 
814.87 & 72.87 & 815.10 & 
-0.24\\[1ex]\hline
\end{tabular}
\label{table:nonlin}
\end{table} \clearpage
\begin{table}[ht]
\caption{Resultados de la ejecución de la metaheurística ACO, utilizando instancias de Dethloff con la configuración -n 200 -alpha 1.0 -beta 3.0 -q 0.8 -ro 0.015}
\centering
\small
\begin{tabular}{c c c c c c c}
\hline\hline
Instancia & Costo mínimo & Tiempo(seg.) & Costo promedio & Tiempo promedio(seg.) & Costo ACO & \%Gap \\ [0.5ex]
\hline
SCA3-0 & \bf{\underline{636.06}} & 126.70 & 
636.06 & 126.71 & 636.10 & 
-0.01\\SCA3-1 & \bf{\underline{697.84}} & 143.18 & 
697.84 & 142.09 & 700.10 & 
-0.32\\SCA3-2 & 659.34 & 120.27 & 
659.34 & 131.02 & \bf{659.30} & 
0.01\\SCA3-3 & 680.04 & 128.01 & 
680.04 & 134.50 & \bf{680.00} & 
0.01\\SCA3-4 & \bf{690.50} & 146.95 & 
690.50 & 139.34 & 690.50 & 0.00\\
SCA3-5 & \bf{\underline{659.90}} & 138.51 & 
660.61 & 140.60 & 671.10 & 
-1.67\\SCA3-6 & \bf{\underline{651.09}} & 119.52 & 
651.90 & 131.68 & 651.10 & 
-0.00\\SCA3-7 & \bf{\underline{664.88}} & 96.94 & 
665.83 & 106.41 & 666.10 & 
-0.18\\SCA3-8 & \bf{\underline{719.47}} & 113.11 & 
719.47 & 120.25 & 719.50 & 
-0.00\\SCA3-9 & \bf{681.00} & 98.17 & 
681.00 & 102.90 & 681.00 & 0.00\\
SCA8-0 & 965.26 & 176.99 & 
969.00 & 155.99 & \bf{961.60} & 
0.38\\SCA8-1 & \bf{\underline{1050.38}} & 123.99 & 
1054.38 & 115.23 & 1063.00 & 
-1.19\\SCA8-2 & 1049.54 & 93.09 & 
1050.16 & 100.99 & \bf{1040.60} & 
0.86\\SCA8-3 & 1008.29 & 141.16 & 
1012.12 & 145.25 & \bf{985.90} & 
2.27\\SCA8-4 & \bf{\underline{1065.49}} & 142.96 & 
1067.06 & 148.38 & 1071.00 & 
-0.51\\SCA8-5 & \bf{\underline{1034.74}} & 160.67 & 
1038.52 & 152.46 & 1054.30 & 
-1.86\\SCA8-6 & \bf{\underline{972.48}} & 165.22 & 
972.89 & 160.94 & 972.50 & 
-0.00\\SCA8-7 & 1067.20 & 167.69 & 
1067.31 & 160.88 & \bf{1059.70} & 
0.71\\SCA8-8 & \bf{\underline{1071.18}} & 132.19 & 
1071.18 & 146.71 & 1082.70 & 
-1.06\\SCA8-9 & \bf{\underline{1066.61}} & 118.17 & 
1067.01 & 118.24 & 1081.40 & 
-1.37\\CON3-0 & 616.52 & 168.56 & 
619.70 & 154.38 & \bf{616.50} & 
0.00\\CON3-1 & \bf{\underline{554.47}} & 134.81 & 
554.47 & 137.37 & 555.60 & 
-0.20\\CON3-2 & \bf{\underline{519.61}} & 135.40 & 
520.50 & 121.06 & 521.40 & 
-0.34\\CON3-3 & \bf{\underline{591.19}} & 147.70 & 
591.19 & 144.14 & 591.20 & 
-0.00\\CON3-4 & \bf{\underline{588.79}} & 131.34 & 
588.79 & 126.78 & 589.30 & 
-0.09\\CON3-5 & \bf{563.70} & 140.08 & 
564.29 & 127.39 & 563.70 & 0.00\\
CON3-6 & \bf{\underline{499.05}} & 160.47 & 
500.25 & 160.34 & 499.20 & 
-0.03\\CON3-7 & \bf{\underline{576.48}} & 123.53 & 
577.51 & 118.64 & 577.50 & 
-0.18\\CON3-8 & \bf{\underline{523.05}} & 141.00 & 
523.12 & 127.26 & 523.10 & 
-0.01\\CON3-9 & 578.25 & 128.88 & 
582.96 & 125.32 & \bf{578.20} & 
0.01\\CON8-0 & 866.22 & 143.41 & 
867.87 & 138.98 & \bf{858.90} & 
0.85\\CON8-1 & \bf{\underline{740.85}} & 130.26 & 
740.85 & 132.78 & 740.90 & 
-0.01\\CON8-2 & \bf{\underline{712.89}} & 180.16 & 
713.16 & 180.84 & 714.30 & 
-0.20\\CON8-3 & \bf{\underline{811.23}} & 135.87 & 
814.62 & 154.38 & 812.30 & 
-0.13\\CON8-4 & 776.34 & 132.08 & 
779.76 & 140.50 & \bf{770.10} & 
0.81\\CON8-5 & \bf{\underline{758.12}} & 129.26 & 
760.37 & 130.79 & 766.60 & 
-1.11\\CON8-6 & \bf{\underline{684.69}} & 177.38 & 
688.52 & 161.67 & 697.20 & 
-1.79\\CON8-7 & \bf{\underline{814.79}} & 117.64 & 
814.81 & 117.83 & 814.80 & 
-0.00\\CON8-8 & 782.86 & 148.83 & 
785.59 & 156.35 & \bf{771.30} & 
1.50\\CON8-9 & \bf{\underline{812.60}} & 148.80 & 
813.77 & 147.13 & 815.10 & 
-0.31\\[1ex]\hline
\end{tabular}
\label{table:nonlin}
\end{table} \clearpage
\begin{table}[ht]
\caption{Resultados de la ejecución de la metaheurística ACO, utilizando instancias de Dethloff con la configuración -n 20 -alpha 1.0 -beta 3.0 -q 1.0 -ro 0.015}
\centering
\small
\begin{tabular}{c c c c c c c}
\hline\hline
Instancia & Costo mínimo & Tiempo(seg.) & Costo promedio & Tiempo promedio(seg.) & Costo ACO & \%Gap \\ [0.5ex]
\hline
SCA3-0 & 640.55 & 13.00 & 
640.55 & 12.97 & \bf{636.10} & 
0.70\\SCA3-1 & \bf{\underline{697.84}} & 14.79 & 
697.84 & 13.99 & 700.10 & 
-0.32\\SCA3-2 & 664.18 & 12.97 & 
664.18 & 12.90 & \bf{659.30} & 
0.74\\SCA3-3 & 680.60 & 13.64 & 
680.74 & 13.55 & \bf{680.00} & 
0.09\\SCA3-4 & \bf{690.50} & 13.24 & 
690.50 & 13.72 & 690.50 & 0.00\\
SCA3-5 & \bf{\underline{665.04}} & 15.03 & 
665.04 & 14.74 & 671.10 & 
-0.90\\SCA3-6 & 653.69 & 12.46 & 
654.82 & 12.44 & \bf{651.10} & 
0.40\\SCA3-7 & 666.15 & 10.16 & 
666.15 & 10.33 & \bf{666.10} & 
0.01\\SCA3-8 & 721.45 & 10.40 & 
724.65 & 11.47 & \bf{719.50} & 
0.27\\SCA3-9 & \bf{681.00} & 9.47 & 
681.00 & 9.43 & 681.00 & 0.00\\
SCA8-0 & 991.07 & 13.46 & 
991.65 & 13.89 & \bf{961.60} & 
3.06\\SCA8-1 & 1066.49 & 11.54 & 
1066.49 & 11.27 & \bf{1063.00} & 
0.33\\SCA8-2 & 1056.87 & 10.61 & 
1056.87 & 9.95 & \bf{1040.60} & 
1.56\\SCA8-3 & 1031.08 & 13.92 & 
1031.08 & 14.47 & \bf{985.90} & 
4.58\\SCA8-4 & 1098.34 & 14.81 & 
1098.88 & 14.48 & \bf{1071.00} & 
2.55\\SCA8-5 & \bf{\underline{1053.09}} & 15.97 & 
1054.78 & 16.13 & 1054.30 & 
-0.11\\SCA8-6 & \bf{\underline{972.48}} & 16.95 & 
972.48 & 16.29 & 972.50 & 
-0.00\\SCA8-7 & 1082.59 & 16.31 & 
1082.59 & 16.30 & \bf{1059.70} & 
2.16\\SCA8-8 & \bf{\underline{1071.18}} & 14.50 & 
1081.34 & 14.12 & 1082.70 & 
-1.06\\SCA8-9 & \bf{\underline{1067.42}} & 10.55 & 
1067.42 & 10.89 & 1081.40 & 
-1.29\\CON3-0 & 624.96 & 15.02 & 
624.96 & 15.63 & \bf{616.50} & 
1.37\\CON3-1 & 557.38 & 13.30 & 
557.38 & 13.47 & \bf{555.60} & 
0.32\\CON3-2 & \bf{\underline{521.38}} & 12.90 & 
522.94 & 12.65 & 521.40 & 
-0.00\\CON3-3 & \bf{591.20} & 15.06 & 
591.60 & 14.79 & 591.20 & 0.00\\
CON3-4 & \bf{\underline{588.79}} & 12.37 & 
589.19 & 12.53 & 589.30 & 
-0.09\\CON3-5 & 570.70 & 12.50 & 
573.00 & 12.76 & \bf{563.70} & 
1.24\\CON3-6 & 503.58 & 17.33 & 
504.58 & 17.18 & \bf{499.20} & 
0.88\\CON3-7 & 578.41 & 12.87 & 
578.41 & 13.14 & \bf{577.50} & 
0.16\\CON3-8 & 524.30 & 11.80 & 
524.45 & 11.51 & \bf{523.10} & 
0.23\\CON3-9 & 578.25 & 12.26 & 
580.81 & 12.10 & \bf{578.20} & 
0.01\\CON8-0 & 879.00 & 13.12 & 
879.00 & 13.40 & \bf{858.90} & 
2.34\\CON8-1 & 758.26 & 14.03 & 
758.26 & 13.79 & \bf{740.90} & 
2.34\\CON8-2 & 716.53 & 19.66 & 
716.53 & 18.71 & \bf{714.30} & 
0.31\\CON8-3 & 817.57 & 14.06 & 
817.57 & 13.39 & \bf{812.30} & 
0.65\\CON8-4 & 781.64 & 14.88 & 
787.89 & 14.80 & \bf{770.10} & 
1.50\\CON8-5 & \bf{\underline{764.36}} & 13.81 & 
764.36 & 13.86 & 766.60 & 
-0.29\\CON8-6 & \bf{\underline{696.08}} & 16.89 & 
699.40 & 16.57 & 697.20 & 
-0.16\\CON8-7 & 822.42 & 12.70 & 
822.92 & 12.30 & \bf{814.80} & 
0.94\\CON8-8 & 795.77 & 14.85 & 
798.31 & 14.97 & \bf{771.30} & 
3.17\\CON8-9 & 816.12 & 15.90 & 
816.12 & 15.38 & \bf{815.10} & 
0.13\\[1ex]\hline
\end{tabular}
\label{table:nonlin}
\end{table} \clearpage
\begin{table}[ht]
\caption{Resultados de la ejecución de la metaheurística ACO, utilizando instancias de Dethloff con la configuración -n 20 -alpha 1.0 -beta 3.0 -q 0.1 -ro 0.015}
\centering
\small
\begin{tabular}{c c c c c c c}
\hline\hline
Instancia & Costo mínimo & Tiempo(seg.) & Costo promedio & Tiempo promedio(seg.) & Costo ACO & \%Gap \\ [0.5ex]
\hline
SCA3-0 & \bf{\underline{636.06}} & 13.83 & 
636.06 & 13.85 & 636.10 & 
-0.01\\SCA3-1 & \bf{\underline{697.84}} & 14.60 & 
697.84 & 15.33 & 700.10 & 
-0.32\\SCA3-2 & 659.34 & 13.11 & 
659.34 & 13.59 & \bf{659.30} & 
0.01\\SCA3-3 & 680.04 & 12.75 & 
680.04 & 13.29 & \bf{680.00} & 
0.01\\SCA3-4 & \bf{690.50} & 14.78 & 
690.50 & 14.71 & 690.50 & 0.00\\
SCA3-5 & \bf{\underline{662.75}} & 15.38 & 
663.47 & 14.35 & 671.10 & 
-1.24\\SCA3-6 & \bf{\underline{651.09}} & 14.47 & 
651.09 & 14.29 & 651.10 & 
-0.00\\SCA3-7 & 666.15 & 12.09 & 
666.15 & 11.96 & \bf{666.10} & 
0.01\\SCA3-8 & \bf{\underline{719.47}} & 13.78 & 
719.47 & 14.03 & 719.50 & 
-0.00\\SCA3-9 & \bf{681.00} & 12.28 & 
681.00 & 12.12 & 681.00 & 0.00\\
SCA8-0 & \bf{\underline{961.50}} & 14.86 & 
970.36 & 14.74 & 961.60 & 
-0.01\\SCA8-1 & \bf{\underline{1052.71}} & 12.58 & 
1055.12 & 12.82 & 1063.00 & 
-0.97\\SCA8-2 & 1047.63 & 11.67 & 
1049.35 & 11.46 & \bf{1040.60} & 
0.68\\SCA8-3 & \bf{\underline{985.60}} & 14.09 & 
995.52 & 15.86 & 985.90 & 
-0.03\\SCA8-4 & \bf{\underline{1065.49}} & 14.78 & 
1065.49 & 14.50 & 1071.00 & 
-0.51\\SCA8-5 & \bf{\underline{1034.74}} & 16.93 & 
1044.51 & 16.32 & 1054.30 & 
-1.86\\SCA8-6 & \bf{\underline{972.48}} & 15.67 & 
978.38 & 15.41 & 972.50 & 
-0.00\\SCA8-7 & 1065.47 & 15.12 & 
1066.74 & 15.35 & \bf{1059.70} & 
0.54\\SCA8-8 & \bf{\underline{1071.18}} & 15.24 & 
1072.13 & 15.31 & 1082.70 & 
-1.06\\SCA8-9 & \bf{\underline{1063.68}} & 12.99 & 
1066.03 & 12.60 & 1081.40 & 
-1.64\\CON3-0 & 617.59 & 14.91 & 
617.59 & 14.98 & \bf{616.50} & 
0.18\\CON3-1 & \bf{\underline{554.47}} & 14.38 & 
554.86 & 14.29 & 555.60 & 
-0.20\\CON3-2 & \bf{\underline{519.11}} & 14.26 & 
520.37 & 14.16 & 521.40 & 
-0.44\\CON3-3 & \bf{\underline{591.19}} & 14.76 & 
591.19 & 15.26 & 591.20 & 
-0.00\\CON3-4 & \bf{\underline{588.79}} & 13.32 & 
588.92 & 13.36 & 589.30 & 
-0.09\\CON3-5 & \bf{563.70} & 14.85 & 
564.59 & 14.04 & 563.70 & 0.00\\
CON3-6 & \bf{\underline{499.07}} & 15.89 & 
500.80 & 17.65 & 499.20 & 
-0.03\\CON3-7 & \bf{\underline{576.84}} & 11.92 & 
577.79 & 13.41 & 577.50 & 
-0.11\\CON3-8 & \bf{\underline{523.05}} & 12.80 & 
523.10 & 13.67 & 523.10 & 
-0.01\\CON3-9 & 578.98 & 13.70 & 
581.62 & 13.22 & \bf{578.20} & 
0.13\\CON8-0 & 859.73 & 20.13 & 
867.41 & 15.71 & \bf{858.90} & 
0.10\\CON8-1 & \bf{\underline{740.85}} & 14.48 & 
741.21 & 14.55 & 740.90 & 
-0.01\\CON8-2 & \bf{\underline{712.89}} & 17.60 & 
713.34 & 17.98 & 714.30 & 
-0.20\\CON8-3 & \bf{\underline{812.11}} & 15.25 & 
814.01 & 14.94 & 812.30 & 
-0.02\\CON8-4 & 776.60 & 14.18 & 
780.33 & 13.34 & \bf{770.10} & 
0.84\\CON8-5 & \bf{\underline{758.12}} & 15.77 & 
759.47 & 14.89 & 766.60 & 
-1.11\\CON8-6 & \bf{\underline{684.05}} & 16.97 & 
687.13 & 17.28 & 697.20 & 
-1.89\\CON8-7 & \bf{\underline{814.79}} & 13.73 & 
814.79 & 13.31 & 814.80 & 
-0.00\\CON8-8 & \bf{\underline{771.26}} & 15.57 & 
776.17 & 16.60 & 771.30 & 
-0.01\\CON8-9 & \bf{\underline{810.18}} & 15.82 & 
812.10 & 16.20 & 815.10 & 
-0.60\\[1ex]\hline
\end{tabular}
\label{table:nonlin}
\end{table} \clearpage
\begin{table}[ht]
\caption{Resultados de la ejecución de la metaheurística ACO, utilizando instancias de Dethloff con la configuración -n 20 -alpha 1.0 -beta 3.0 -q 0.5 -ro 0.015}
\centering
\small
\begin{tabular}{c c c c c c c}
\hline\hline
Instancia & Costo mínimo & Tiempo(seg.) & Costo promedio & Tiempo promedio(seg.) & Costo ACO & \%Gap \\ [0.5ex]
\hline
SCA3-0 & \bf{\underline{636.06}} & 13.59 & 
636.06 & 13.24 & 636.10 & 
-0.01\\SCA3-1 & \bf{\underline{697.84}} & 14.43 & 
697.84 & 14.31 & 700.10 & 
-0.32\\SCA3-2 & 659.34 & 13.35 & 
659.79 & 13.12 & \bf{659.30} & 
0.01\\SCA3-3 & 680.04 & 13.47 & 
680.18 & 13.24 & \bf{680.00} & 
0.01\\SCA3-4 & \bf{690.50} & 14.54 & 
690.50 & 14.45 & 690.50 & 0.00\\
SCA3-5 & \bf{\underline{659.90}} & 14.03 & 
661.34 & 14.31 & 671.10 & 
-1.67\\SCA3-6 & 652.47 & 12.38 & 
652.82 & 12.74 & \bf{651.10} & 
0.21\\SCA3-7 & \bf{\underline{664.88}} & 11.53 & 
665.83 & 11.14 & 666.10 & 
-0.18\\SCA3-8 & \bf{\underline{719.47}} & 12.98 & 
719.47 & 13.37 & 719.50 & 
-0.00\\SCA3-9 & \bf{681.00} & 10.47 & 
681.00 & 11.25 & 681.00 & 0.00\\
SCA8-0 & \bf{\underline{961.50}} & 13.96 & 
972.59 & 14.82 & 961.60 & 
-0.01\\SCA8-1 & \bf{\underline{1052.71}} & 11.71 & 
1057.56 & 11.78 & 1063.00 & 
-0.97\\SCA8-2 & 1046.29 & 9.91 & 
1047.99 & 10.68 & \bf{1040.60} & 
0.55\\SCA8-3 & 1008.29 & 14.70 & 
1010.53 & 14.19 & \bf{985.90} & 
2.27\\SCA8-4 & \bf{\underline{1065.49}} & 14.18 & 
1070.48 & 14.39 & 1071.00 & 
-0.51\\SCA8-5 & \bf{\underline{1034.74}} & 16.20 & 
1050.19 & 15.78 & 1054.30 & 
-1.86\\SCA8-6 & \bf{\underline{972.48}} & 16.02 & 
977.28 & 15.69 & 972.50 & 
-0.00\\SCA8-7 & 1066.65 & 15.17 & 
1067.06 & 15.71 & \bf{1059.70} & 
0.66\\SCA8-8 & \bf{\underline{1071.18}} & 15.64 & 
1071.18 & 15.16 & 1082.70 & 
-1.06\\SCA8-9 & \bf{\underline{1067.42}} & 12.26 & 
1067.42 & 12.06 & 1081.40 & 
-1.29\\CON3-0 & 617.59 & 14.70 & 
620.04 & 15.06 & \bf{616.50} & 
0.18\\CON3-1 & \bf{\underline{554.47}} & 14.04 & 
554.86 & 14.48 & 555.60 & 
-0.20\\CON3-2 & \bf{\underline{519.11}} & 13.63 & 
520.81 & 13.47 & 521.40 & 
-0.44\\CON3-3 & \bf{\underline{591.19}} & 14.13 & 
591.19 & 14.85 & 591.20 & 
-0.00\\CON3-4 & \bf{\underline{588.79}} & 12.99 & 
588.79 & 12.60 & 589.30 & 
-0.09\\CON3-5 & \bf{563.70} & 13.54 & 
564.59 & 13.62 & 563.70 & 0.00\\
CON3-6 & 500.80 & 15.74 & 
501.83 & 15.71 & \bf{499.20} & 
0.32\\CON3-7 & \bf{\underline{576.48}} & 11.98 & 
577.83 & 12.38 & 577.50 & 
-0.18\\CON3-8 & \bf{\underline{523.05}} & 14.14 & 
523.43 & 13.99 & 523.10 & 
-0.01\\CON3-9 & 578.98 & 13.01 & 
585.71 & 13.05 & \bf{578.20} & 
0.13\\CON8-0 & 866.22 & 14.51 & 
869.23 & 14.25 & \bf{858.90} & 
0.85\\CON8-1 & \bf{\underline{740.85}} & 14.22 & 
741.25 & 13.77 & 740.90 & 
-0.01\\CON8-2 & \bf{\underline{712.89}} & 17.94 & 
714.08 & 18.12 & 714.30 & 
-0.20\\CON8-3 & \bf{\underline{812.11}} & 13.40 & 
814.99 & 13.71 & 812.30 & 
-0.02\\CON8-4 & 776.72 & 13.75 & 
779.25 & 14.69 & \bf{770.10} & 
0.86\\CON8-5 & \bf{\underline{758.84}} & 13.85 & 
761.16 & 13.66 & 766.60 & 
-1.01\\CON8-6 & \bf{\underline{683.83}} & 17.94 & 
692.36 & 16.68 & 697.20 & 
-1.92\\CON8-7 & \bf{\underline{811.96}} & 12.36 & 
814.08 & 12.63 & 814.80 & 
-0.35\\CON8-8 & 773.63 & 16.22 & 
782.97 & 15.38 & \bf{771.30} & 
0.30\\CON8-9 & \bf{\underline{810.18}} & 15.59 & 
813.44 & 15.11 & 815.10 & 
-0.60\\[1ex]\hline
\end{tabular}
\label{table:nonlin}
\end{table} \clearpage
\begin{table}[ht]
\caption{Resultados de la ejecución de la metaheurística ACO, utilizando instancias de Dethloff con la configuración -n 20 -alpha 0.1 -beta 3.0 -q 0.8 -ro 0.015}
\centering
\small
\begin{tabular}{c c c c c c c}
\hline\hline
Instancia & Costo mínimo & Tiempo(seg.) & Costo promedio & Tiempo promedio(seg.) & Costo ACO & \%Gap \\ [0.5ex]
\hline
SCA3-0 & \bf{\underline{636.06}} & 13.21 & 
636.06 & 12.92 & 636.10 & 
-0.01\\SCA3-1 & \bf{\underline{697.84}} & 13.95 & 
697.84 & 14.48 & 700.10 & 
-0.32\\SCA3-2 & 659.34 & 13.47 & 
659.79 & 12.80 & \bf{659.30} & 
0.01\\SCA3-3 & 680.04 & 13.40 & 
680.04 & 13.53 & \bf{680.00} & 
0.01\\SCA3-4 & \bf{690.50} & 14.04 & 
690.50 & 14.26 & 690.50 & 0.00\\
SCA3-5 & \bf{\underline{662.75}} & 13.34 & 
662.75 & 13.43 & 671.10 & 
-1.24\\SCA3-6 & 652.94 & 12.47 & 
652.94 & 12.90 & \bf{651.10} & 
0.28\\SCA3-7 & 666.15 & 10.00 & 
666.15 & 9.94 & \bf{666.10} & 
0.01\\SCA3-8 & \bf{\underline{719.47}} & 12.29 & 
719.47 & 12.35 & 719.50 & 
-0.00\\SCA3-9 & \bf{681.00} & 10.24 & 
681.00 & 10.08 & 681.00 & 0.00\\
SCA8-0 & \bf{\underline{961.50}} & 15.12 & 
971.06 & 16.09 & 961.60 & 
-0.01\\SCA8-1 & \bf{\underline{1049.65}} & 12.56 & 
1052.32 & 12.14 & 1063.00 & 
-1.26\\SCA8-2 & 1046.29 & 10.62 & 
1049.94 & 10.53 & \bf{1040.60} & 
0.55\\SCA8-3 & 1004.25 & 13.76 & 
1010.07 & 13.72 & \bf{985.90} & 
1.86\\SCA8-4 & \bf{\underline{1065.49}} & 13.24 & 
1065.49 & 14.07 & 1071.00 & 
-0.51\\SCA8-5 & \bf{\underline{1034.74}} & 15.90 & 
1044.91 & 15.88 & 1054.30 & 
-1.86\\SCA8-6 & \bf{\underline{972.48}} & 14.71 & 
975.73 & 15.13 & 972.50 & 
-0.00\\SCA8-7 & 1067.11 & 15.28 & 
1067.18 & 15.34 & \bf{1059.70} & 
0.70\\SCA8-8 & \bf{\underline{1071.18}} & 14.44 & 
1071.18 & 14.45 & 1082.70 & 
-1.06\\SCA8-9 & \bf{\underline{1067.26}} & 11.24 & 
1067.38 & 11.62 & 1081.40 & 
-1.31\\CON3-0 & 620.58 & 15.57 & 
622.64 & 15.59 & \bf{616.50} & 
0.66\\CON3-1 & \bf{\underline{554.47}} & 13.98 & 
554.86 & 14.45 & 555.60 & 
-0.20\\CON3-2 & \bf{\underline{521.38}} & 12.05 & 
521.44 & 12.05 & 521.40 & 
-0.00\\CON3-3 & \bf{\underline{591.19}} & 14.82 & 
591.20 & 14.66 & 591.20 & 
-0.00\\CON3-4 & \bf{\underline{588.79}} & 13.01 & 
588.79 & 12.60 & 589.30 & 
-0.09\\CON3-5 & 564.88 & 13.98 & 
566.80 & 14.53 & \bf{563.70} & 
0.21\\CON3-6 & 500.80 & 16.39 & 
501.14 & 16.69 & \bf{499.20} & 
0.32\\CON3-7 & 577.54 & 11.48 & 
578.10 & 11.52 & \bf{577.50} & 
0.01\\CON3-8 & \bf{\underline{523.05}} & 12.46 & 
523.52 & 12.04 & 523.10 & 
-0.01\\CON3-9 & 578.25 & 12.66 & 
582.03 & 12.58 & \bf{578.20} & 
0.01\\CON8-0 & 868.49 & 13.36 & 
869.59 & 14.01 & \bf{858.90} & 
1.12\\CON8-1 & \bf{\underline{740.85}} & 12.71 & 
740.89 & 13.76 & 740.90 & 
-0.01\\CON8-2 & \bf{\underline{713.44}} & 17.27 & 
713.44 & 17.70 & 714.30 & 
-0.12\\CON8-3 & \bf{\underline{811.07}} & 14.02 & 
811.11 & 13.91 & 812.30 & 
-0.15\\CON8-4 & 778.63 & 13.62 & 
782.05 & 14.05 & \bf{770.10} & 
1.11\\CON8-5 & \bf{\underline{754.95}} & 14.75 & 
759.22 & 15.87 & 766.60 & 
-1.52\\CON8-6 & \bf{\underline{688.68}} & 15.46 & 
693.39 & 16.09 & 697.20 & 
-1.22\\CON8-7 & \bf{\underline{814.79}} & 12.22 & 
814.79 & 11.91 & 814.80 & 
-0.00\\CON8-8 & 771.32 & 15.33 & 
782.82 & 15.74 & \bf{771.30} & 
0.00\\CON8-9 & \bf{\underline{810.18}} & 15.83 & 
810.68 & 15.29 & 815.10 & 
-0.60\\[1ex]\hline
\end{tabular}
\label{table:nonlin}
\end{table} \clearpage
\begin{table}[ht]
\caption{Resultados de la ejecución de la metaheurística ACO, utilizando instancias de Dethloff con la configuración -n 20 -alpha 0.5 -beta 3.0 -q 0.8 -ro 0.015}
\centering
\small
\begin{tabular}{c c c c c c c}
\hline\hline
Instancia & Costo mínimo & Tiempo(seg.) & Costo promedio & Tiempo promedio(seg.) & Costo ACO & \%Gap \\ [0.5ex]
\hline
SCA3-0 & \bf{\underline{636.06}} & 13.34 & 
636.06 & 12.80 & 636.10 & 
-0.01\\SCA3-1 & \bf{\underline{697.84}} & 14.40 & 
697.84 & 14.46 & 700.10 & 
-0.32\\SCA3-2 & 659.34 & 13.22 & 
659.34 & 13.00 & \bf{659.30} & 
0.01\\SCA3-3 & 680.04 & 13.00 & 
680.04 & 13.12 & \bf{680.00} & 
0.01\\SCA3-4 & \bf{690.50} & 14.54 & 
690.50 & 13.90 & 690.50 & 0.00\\
SCA3-5 & \bf{\underline{659.90}} & 13.84 & 
660.19 & 13.85 & 671.10 & 
-1.67\\SCA3-6 & \bf{\underline{651.09}} & 13.08 & 
652.01 & 13.20 & 651.10 & 
-0.00\\SCA3-7 & 666.15 & 10.01 & 
666.15 & 10.10 & \bf{666.10} & 
0.01\\SCA3-8 & \bf{\underline{719.47}} & 11.50 & 
719.47 & 12.21 & 719.50 & 
-0.00\\SCA3-9 & \bf{681.00} & 11.04 & 
681.00 & 10.68 & 681.00 & 0.00\\
SCA8-0 & 968.79 & 15.30 & 
972.39 & 14.92 & \bf{961.60} & 
0.75\\SCA8-1 & \bf{\underline{1054.87}} & 10.97 & 
1061.11 & 11.32 & 1063.00 & 
-0.76\\SCA8-2 & 1049.22 & 10.25 & 
1050.29 & 10.39 & \bf{1040.60} & 
0.83\\SCA8-3 & 995.50 & 14.25 & 
1008.64 & 14.41 & \bf{985.90} & 
0.97\\SCA8-4 & \bf{\underline{1065.49}} & 14.98 & 
1067.18 & 14.63 & 1071.00 & 
-0.51\\SCA8-5 & \bf{\underline{1034.74}} & 16.04 & 
1045.57 & 15.75 & 1054.30 & 
-1.86\\SCA8-6 & \bf{\underline{972.48}} & 15.65 & 
976.29 & 15.52 & 972.50 & 
-0.00\\SCA8-7 & 1067.20 & 15.38 & 
1068.13 & 15.21 & \bf{1059.70} & 
0.71\\SCA8-8 & \bf{\underline{1071.18}} & 13.93 & 
1073.91 & 13.80 & 1082.70 & 
-1.06\\SCA8-9 & \bf{\underline{1067.42}} & 11.68 & 
1067.42 & 11.44 & 1081.40 & 
-1.29\\CON3-0 & 617.59 & 14.84 & 
620.79 & 15.27 & \bf{616.50} & 
0.18\\CON3-1 & \bf{\underline{554.47}} & 13.18 & 
555.92 & 13.73 & 555.60 & 
-0.20\\CON3-2 & \bf{\underline{521.38}} & 11.94 & 
522.59 & 12.08 & 521.40 & 
-0.00\\CON3-3 & \bf{\underline{591.19}} & 14.76 & 
591.20 & 14.68 & 591.20 & 
-0.00\\CON3-4 & \bf{\underline{588.79}} & 12.09 & 
589.05 & 12.79 & 589.30 & 
-0.09\\CON3-5 & 564.88 & 13.00 & 
565.83 & 13.75 & \bf{563.70} & 
0.21\\CON3-6 & 500.80 & 16.42 & 
502.02 & 16.30 & \bf{499.20} & 
0.32\\CON3-7 & 578.22 & 12.62 & 
578.98 & 12.84 & \bf{577.50} & 
0.12\\CON3-8 & \bf{\underline{523.05}} & 11.27 & 
523.84 & 11.59 & 523.10 & 
-0.01\\CON3-9 & 583.32 & 12.42 & 
586.61 & 12.47 & \bf{578.20} & 
0.89\\CON8-0 & \bf{\underline{858.88}} & 13.41 & 
867.00 & 14.03 & 858.90 & 
-0.00\\CON8-1 & \bf{\underline{740.85}} & 14.04 & 
740.85 & 13.28 & 740.90 & 
-0.01\\CON8-2 & \bf{\underline{712.89}} & 18.38 & 
713.74 & 18.31 & 714.30 & 
-0.20\\CON8-3 & 812.75 & 13.84 & 
814.98 & 13.10 & \bf{812.30} & 
0.06\\CON8-4 & 776.37 & 13.98 & 
780.13 & 14.15 & \bf{770.10} & 
0.81\\CON8-5 & \bf{\underline{758.99}} & 13.38 & 
762.08 & 13.20 & 766.60 & 
-0.99\\CON8-6 & \bf{\underline{690.27}} & 15.94 & 
693.76 & 15.56 & 697.20 & 
-0.99\\CON8-7 & \bf{\underline{814.50}} & 11.68 & 
814.74 & 11.71 & 814.80 & 
-0.04\\CON8-8 & 781.96 & 16.58 & 
784.94 & 15.88 & \bf{771.30} & 
1.38\\CON8-9 & \bf{\underline{810.61}} & 14.78 & 
813.88 & 15.13 & 815.10 & 
-0.55\\[1ex]\hline
\end{tabular}
\label{table:nonlin}
\end{table} \clearpage
\begin{table}[ht]
\caption{Resultados de la ejecución de la metaheurística ACO, utilizando instancias de Dethloff con la configuración -n 20 -alpha 2.0 -beta 3.0 -q 0.8 -ro 0.015}
\centering
\small
\begin{tabular}{c c c c c c c}
\hline\hline
Instancia & Costo mínimo & Tiempo(seg.) & Costo promedio & Tiempo promedio(seg.) & Costo ACO & \%Gap \\ [0.5ex]
\hline
SCA3-0 & \bf{\underline{636.06}} & 16.38 & 
636.06 & 14.29 & 636.10 & 
-0.01\\SCA3-1 & \bf{\underline{697.84}} & 14.06 & 
697.84 & 14.37 & 700.10 & 
-0.32\\SCA3-2 & 659.34 & 14.64 & 
661.00 & 13.69 & \bf{659.30} & 
0.01\\SCA3-3 & 680.04 & 13.52 & 
680.18 & 13.51 & \bf{680.00} & 
0.01\\SCA3-4 & \bf{690.50} & 13.10 & 
690.50 & 13.63 & 690.50 & 0.00\\
SCA3-5 & \bf{\underline{662.75}} & 14.03 & 
663.89 & 14.97 & 671.10 & 
-1.24\\SCA3-6 & 652.94 & 13.86 & 
654.03 & 15.26 & \bf{651.10} & 
0.28\\SCA3-7 & \bf{\underline{664.88}} & 12.85 & 
665.83 & 13.30 & 666.10 & 
-0.18\\SCA3-8 & \bf{\underline{719.47}} & 11.96 & 
720.26 & 12.16 & 719.50 & 
-0.00\\SCA3-9 & \bf{681.00} & 11.40 & 
681.00 & 11.36 & 681.00 & 0.00\\
SCA8-0 & 973.03 & 14.75 & 
978.15 & 14.93 & \bf{961.60} & 
1.19\\SCA8-1 & \bf{\underline{1050.38}} & 11.41 & 
1055.74 & 11.91 & 1063.00 & 
-1.19\\SCA8-2 & 1046.29 & 10.96 & 
1051.01 & 10.46 & \bf{1040.60} & 
0.55\\SCA8-3 & 1007.97 & 15.50 & 
1013.94 & 15.28 & \bf{985.90} & 
2.24\\SCA8-4 & \bf{\underline{1065.49}} & 15.65 & 
1077.85 & 15.03 & 1071.00 & 
-0.51\\SCA8-5 & \bf{\underline{1034.74}} & 15.82 & 
1045.11 & 15.47 & 1054.30 & 
-1.86\\SCA8-6 & 977.87 & 15.93 & 
978.83 & 15.95 & \bf{972.50} & 
0.55\\SCA8-7 & 1067.20 & 16.66 & 
1068.66 & 16.42 & \bf{1059.70} & 
0.71\\SCA8-8 & \bf{\underline{1071.18}} & 15.23 & 
1072.14 & 14.79 & 1082.70 & 
-1.06\\SCA8-9 & \bf{\underline{1066.61}} & 10.88 & 
1067.22 & 11.41 & 1081.40 & 
-1.37\\CON3-0 & 620.76 & 14.94 & 
624.31 & 15.95 & \bf{616.50} & 
0.69\\CON3-1 & \bf{\underline{554.47}} & 13.48 & 
555.32 & 13.75 & 555.60 & 
-0.20\\CON3-2 & \bf{\underline{521.38}} & 12.06 & 
521.50 & 12.57 & 521.40 & 
-0.00\\CON3-3 & \bf{\underline{591.19}} & 14.09 & 
591.20 & 14.83 & 591.20 & 
-0.00\\CON3-4 & \bf{\underline{588.79}} & 12.80 & 
589.45 & 12.72 & 589.30 & 
-0.09\\CON3-5 & 564.88 & 14.07 & 
565.82 & 14.13 & \bf{563.70} & 
0.21\\CON3-6 & 501.33 & 15.64 & 
501.95 & 16.66 & \bf{499.20} & 
0.43\\CON3-7 & 577.68 & 12.41 & 
578.23 & 12.38 & \bf{577.50} & 
0.03\\CON3-8 & \bf{\underline{523.05}} & 11.78 & 
523.90 & 12.91 & 523.10 & 
-0.01\\CON3-9 & 578.98 & 14.24 & 
585.30 & 13.80 & \bf{578.20} & 
0.13\\CON8-0 & 873.62 & 12.64 & 
878.04 & 13.40 & \bf{858.90} & 
1.71\\CON8-1 & \bf{\underline{740.85}} & 15.45 & 
741.25 & 14.46 & 740.90 & 
-0.01\\CON8-2 & \bf{\underline{713.44}} & 19.32 & 
713.78 & 19.45 & 714.30 & 
-0.12\\CON8-3 & \bf{\underline{812.11}} & 14.11 & 
815.91 & 14.12 & 812.30 & 
-0.02\\CON8-4 & 784.92 & 15.66 & 
788.80 & 15.50 & \bf{770.10} & 
1.92\\CON8-5 & \bf{\underline{761.40}} & 13.76 & 
764.25 & 13.73 & 766.60 & 
-0.68\\CON8-6 & \bf{\underline{690.27}} & 14.98 & 
694.76 & 15.31 & 697.20 & 
-0.99\\CON8-7 & \bf{\underline{814.79}} & 13.37 & 
817.42 & 13.29 & 814.80 & 
-0.00\\CON8-8 & 778.62 & 15.98 & 
787.32 & 16.25 & \bf{771.30} & 
0.95\\CON8-9 & \bf{\underline{813.16}} & 14.54 & 
815.39 & 14.51 & 815.10 & 
-0.24\\[1ex]\hline
\end{tabular}
\label{table:nonlin}
\end{table} \clearpage
\begin{table}[ht]
\caption{Resultados de la ejecución de la metaheurística ACO, utilizando instancias de Dethloff con la configuración -n 20 -alpha 3.0 -beta 3.0 -q 0.8 -ro 0.015}
\centering
\small
\begin{tabular}{c c c c c c c}
\hline\hline
Instancia & Costo mínimo & Tiempo(seg.) & Costo promedio & Tiempo promedio(seg.) & Costo ACO & \%Gap \\ [0.5ex]
\hline
SCA3-0 & \bf{\underline{636.06}} & 14.69 & 
636.06 & 14.59 & 636.10 & 
-0.01\\SCA3-1 & \bf{\underline{697.84}} & 13.52 & 
697.84 & 14.29 & 700.10 & 
-0.32\\SCA3-2 & 659.34 & 13.60 & 
661.00 & 12.79 & \bf{659.30} & 
0.01\\SCA3-3 & 680.04 & 14.82 & 
680.18 & 13.51 & \bf{680.00} & 
0.01\\SCA3-4 & \bf{690.50} & 15.03 & 
690.50 & 14.71 & 690.50 & 0.00\\
SCA3-5 & \bf{\underline{659.90}} & 13.56 & 
663.48 & 14.38 & 671.10 & 
-1.67\\SCA3-6 & \bf{\underline{651.09}} & 13.64 & 
652.94 & 13.21 & 651.10 & 
-0.00\\SCA3-7 & \bf{\underline{664.88}} & 13.90 & 
665.83 & 14.58 & 666.10 & 
-0.18\\SCA3-8 & \bf{\underline{719.47}} & 12.36 & 
719.54 & 12.81 & 719.50 & 
-0.00\\SCA3-9 & \bf{681.00} & 11.90 & 
681.00 & 11.81 & 681.00 & 0.00\\
SCA8-0 & 987.51 & 15.10 & 
993.82 & 15.79 & \bf{961.60} & 
2.69\\SCA8-1 & \bf{\underline{1055.60}} & 11.69 & 
1065.25 & 12.02 & 1063.00 & 
-0.70\\SCA8-2 & 1050.98 & 11.84 & 
1052.14 & 11.04 & \bf{1040.60} & 
1.00\\SCA8-3 & 1005.59 & 15.54 & 
1017.12 & 16.15 & \bf{985.90} & 
2.00\\SCA8-4 & \bf{\underline{1067.82}} & 15.24 & 
1073.62 & 15.28 & 1071.00 & 
-0.30\\SCA8-5 & \bf{\underline{1034.74}} & 16.24 & 
1040.58 & 16.16 & 1054.30 & 
-1.86\\SCA8-6 & \bf{\underline{972.48}} & 16.04 & 
977.11 & 15.55 & 972.50 & 
-0.00\\SCA8-7 & 1066.65 & 16.06 & 
1067.18 & 16.66 & \bf{1059.70} & 
0.66\\SCA8-8 & \bf{\underline{1071.18}} & 14.74 & 
1071.18 & 15.19 & 1082.70 & 
-1.06\\SCA8-9 & \bf{\underline{1067.42}} & 11.57 & 
1067.42 & 11.63 & 1081.40 & 
-1.29\\CON3-0 & 620.76 & 14.76 & 
622.06 & 15.68 & \bf{616.50} & 
0.69\\CON3-1 & \bf{\underline{554.47}} & 15.10 & 
556.11 & 14.68 & 555.60 & 
-0.20\\CON3-2 & \bf{\underline{521.38}} & 11.57 & 
522.10 & 12.18 & 521.40 & 
-0.00\\CON3-3 & \bf{\underline{591.19}} & 14.74 & 
591.20 & 14.70 & 591.20 & 
-0.00\\CON3-4 & \bf{\underline{588.79}} & 13.21 & 
588.79 & 13.24 & 589.30 & 
-0.09\\CON3-5 & 564.88 & 14.54 & 
565.84 & 14.61 & \bf{563.70} & 
0.21\\CON3-6 & 502.16 & 18.26 & 
503.11 & 17.27 & \bf{499.20} & 
0.59\\CON3-7 & 577.54 & 12.74 & 
578.10 & 12.29 & \bf{577.50} & 
0.01\\CON3-8 & 523.14 & 13.09 & 
523.87 & 13.15 & \bf{523.10} & 
0.01\\CON3-9 & 588.40 & 13.86 & 
588.46 & 13.27 & \bf{578.20} & 
1.76\\CON8-0 & 871.80 & 15.01 & 
875.99 & 14.38 & \bf{858.90} & 
1.50\\CON8-1 & \bf{\underline{740.85}} & 15.72 & 
744.79 & 15.01 & 740.90 & 
-0.01\\CON8-2 & \bf{\underline{713.44}} & 19.83 & 
714.61 & 20.15 & 714.30 & 
-0.12\\CON8-3 & 817.57 & 15.01 & 
821.05 & 15.37 & \bf{812.30} & 
0.65\\CON8-4 & 776.37 & 15.18 & 
780.78 & 15.76 & \bf{770.10} & 
0.81\\CON8-5 & \bf{\underline{759.93}} & 15.01 & 
762.56 & 14.40 & 766.60 & 
-0.87\\CON8-6 & \bf{\underline{696.30}} & 16.31 & 
697.15 & 16.62 & 697.20 & 
-0.13\\CON8-7 & \bf{\underline{814.50}} & 13.97 & 
816.41 & 13.43 & 814.80 & 
-0.04\\CON8-8 & 782.86 & 14.71 & 
788.50 & 15.78 & \bf{771.30} & 
1.50\\CON8-9 & \bf{\underline{813.16}} & 14.97 & 
816.26 & 15.14 & 815.10 & 
-0.24\\[1ex]\hline
\end{tabular}
\label{table:nonlin}
\end{table} \clearpage
\begin{table}[ht]
\caption{Resultados de la ejecución de la metaheurística ACO, utilizando instancias de Dethloff con la configuración -n 20 -alpha 4.0 -beta 3.0 -q 0.8 -ro 0.015}
\centering
\small
\begin{tabular}{c c c c c c c}
\hline\hline
Instancia & Costo mínimo & Tiempo(seg.) & Costo promedio & Tiempo promedio(seg.) & Costo ACO & \%Gap \\ [0.5ex]
\hline
SCA3-0 & \bf{\underline{636.06}} & 15.52 & 
636.06 & 14.84 & 636.10 & 
-0.01\\SCA3-1 & \bf{\underline{697.84}} & 14.96 & 
697.84 & 14.58 & 700.10 & 
-0.32\\SCA3-2 & 659.34 & 12.72 & 
661.00 & 13.04 & \bf{659.30} & 
0.01\\SCA3-3 & 680.60 & 13.32 & 
680.78 & 13.61 & \bf{680.00} & 
0.09\\SCA3-4 & \bf{690.50} & 16.11 & 
690.50 & 14.87 & 690.50 & 0.00\\
SCA3-5 & \bf{\underline{665.04}} & 15.24 & 
666.61 & 15.59 & 671.10 & 
-0.90\\SCA3-6 & 652.94 & 13.53 & 
653.49 & 13.92 & \bf{651.10} & 
0.28\\SCA3-7 & 666.15 & 12.89 & 
666.15 & 14.16 & \bf{666.10} & 
0.01\\SCA3-8 & \bf{\underline{719.47}} & 12.62 & 
719.54 & 14.22 & 719.50 & 
-0.00\\SCA3-9 & \bf{681.00} & 12.13 & 
681.00 & 12.22 & 681.00 & 0.00\\
SCA8-0 & \bf{\underline{961.50}} & 16.08 & 
977.51 & 16.05 & 961.60 & 
-0.01\\SCA8-1 & 1063.66 & 11.50 & 
1067.87 & 12.62 & \bf{1063.00} & 
0.06\\SCA8-2 & 1048.75 & 10.90 & 
1050.99 & 11.70 & \bf{1040.60} & 
0.78\\SCA8-3 & 1007.63 & 16.34 & 
1012.92 & 16.17 & \bf{985.90} & 
2.20\\SCA8-4 & \bf{\underline{1067.66}} & 15.36 & 
1073.46 & 15.04 & 1071.00 & 
-0.31\\SCA8-5 & \bf{\underline{1034.74}} & 15.01 & 
1045.75 & 15.86 & 1054.30 & 
-1.86\\SCA8-6 & \bf{\underline{972.48}} & 16.92 & 
974.97 & 15.47 & 972.50 & 
-0.00\\SCA8-7 & 1067.20 & 16.68 & 
1070.22 & 16.59 & \bf{1059.70} & 
0.71\\SCA8-8 & \bf{\underline{1071.18}} & 15.09 & 
1071.18 & 16.13 & 1082.70 & 
-1.06\\SCA8-9 & \bf{\underline{1067.42}} & 12.03 & 
1067.42 & 11.84 & 1081.40 & 
-1.29\\CON3-0 & 617.59 & 15.26 & 
620.95 & 15.59 & \bf{616.50} & 
0.18\\CON3-1 & \bf{\underline{554.47}} & 14.01 & 
555.15 & 14.70 & 555.60 & 
-0.20\\CON3-2 & \bf{\underline{521.38}} & 13.00 & 
521.88 & 13.91 & 521.40 & 
-0.00\\CON3-3 & \bf{\underline{591.19}} & 14.95 & 
591.19 & 14.91 & 591.20 & 
-0.00\\CON3-4 & \bf{\underline{588.79}} & 14.45 & 
589.45 & 14.29 & 589.30 & 
-0.09\\CON3-5 & \bf{563.70} & 16.13 & 
566.82 & 15.21 & 563.70 & 0.00\\
CON3-6 & 500.80 & 16.64 & 
502.77 & 17.45 & \bf{499.20} & 
0.32\\CON3-7 & 578.41 & 13.49 & 
580.52 & 12.70 & \bf{577.50} & 
0.16\\CON3-8 & 523.14 & 13.74 & 
525.82 & 13.68 & \bf{523.10} & 
0.01\\CON3-9 & 580.05 & 12.56 & 
584.93 & 13.71 & \bf{578.20} & 
0.32\\CON8-0 & 875.63 & 14.33 & 
878.28 & 14.24 & \bf{858.90} & 
1.95\\CON8-1 & 740.93 & 14.69 & 
741.61 & 13.83 & \bf{740.90} & 
0.00\\CON8-2 & 714.94 & 19.21 & 
715.91 & 19.34 & \bf{714.30} & 
0.09\\CON8-3 & 817.57 & 14.46 & 
817.57 & 14.99 & \bf{812.30} & 
0.65\\CON8-4 & 776.37 & 17.62 & 
785.54 & 16.54 & \bf{770.10} & 
0.81\\CON8-5 & \bf{\underline{761.40}} & 13.84 & 
762.83 & 13.83 & 766.60 & 
-0.68\\CON8-6 & \bf{\underline{685.49}} & 17.57 & 
693.29 & 16.65 & 697.20 & 
-1.68\\CON8-7 & \bf{\underline{814.50}} & 13.09 & 
815.05 & 13.67 & 814.80 & 
-0.04\\CON8-8 & 788.70 & 16.51 & 
790.37 & 16.28 & \bf{771.30} & 
2.26\\CON8-9 & \bf{\underline{812.60}} & 14.88 & 
813.83 & 14.69 & 815.10 & 
-0.31\\[1ex]\hline
\end{tabular}
\label{table:nonlin}
\end{table} \clearpage
\begin{table}[ht]
\caption{Resultados de la ejecución de la metaheurística ACO, utilizando instancias de Dethloff con la configuración -n 20 -alpha 1.0 -beta 0.1 -q 0.8 -ro 0.015}
\centering
\small
\begin{tabular}{c c c c c c c}
\hline\hline
Instancia & Costo mínimo & Tiempo(seg.) & Costo promedio & Tiempo promedio(seg.) & Costo ACO & \%Gap \\ [0.5ex]
\hline
SCA3-0 & \bf{\underline{636.06}} & 20.29 & 
636.06 & 18.96 & 636.10 & 
-0.01\\SCA3-1 & \bf{\underline{697.84}} & 18.71 & 
697.84 & 18.37 & 700.10 & 
-0.32\\SCA3-2 & 659.34 & 16.53 & 
660.55 & 17.05 & \bf{659.30} & 
0.01\\SCA3-3 & 680.04 & 18.44 & 
680.18 & 17.67 & \bf{680.00} & 
0.01\\SCA3-4 & \bf{690.50} & 19.60 & 
690.50 & 18.67 & 690.50 & 0.00\\
SCA3-5 & \bf{\underline{662.75}} & 20.60 & 
663.32 & 18.87 & 671.10 & 
-1.24\\SCA3-6 & 652.94 & 20.18 & 
652.94 & 18.73 & \bf{651.10} & 
0.28\\SCA3-7 & \bf{\underline{659.17}} & 18.64 & 
664.40 & 18.48 & 666.10 & 
-1.04\\SCA3-8 & \bf{\underline{719.47}} & 20.20 & 
719.47 & 19.07 & 719.50 & 
-0.00\\SCA3-9 & \bf{681.00} & 17.46 & 
681.00 & 17.66 & 681.00 & 0.00\\
SCA8-0 & 973.22 & 19.55 & 
980.97 & 18.93 & \bf{961.60} & 
1.21\\SCA8-1 & \bf{\underline{1050.38}} & 16.29 & 
1057.57 & 16.61 & 1063.00 & 
-1.19\\SCA8-2 & 1045.05 & 14.86 & 
1046.82 & 15.20 & \bf{1040.60} & 
0.43\\SCA8-3 & 995.60 & 19.51 & 
1000.49 & 18.37 & \bf{985.90} & 
0.98\\SCA8-4 & \bf{\underline{1065.49}} & 17.33 & 
1069.40 & 17.80 & 1071.00 & 
-0.51\\SCA8-5 & \bf{\underline{1034.74}} & 19.86 & 
1048.47 & 20.01 & 1054.30 & 
-1.86\\SCA8-6 & \bf{\underline{972.48}} & 19.10 & 
974.18 & 18.96 & 972.50 & 
-0.00\\SCA8-7 & 1066.65 & 18.74 & 
1069.66 & 19.17 & \bf{1059.70} & 
0.66\\SCA8-8 & \bf{\underline{1071.18}} & 19.71 & 
1071.18 & 18.74 & 1082.70 & 
-1.06\\SCA8-9 & \bf{\underline{1060.50}} & 15.80 & 
1063.69 & 15.46 & 1081.40 & 
-1.93\\CON3-0 & 619.09 & 17.21 & 
620.92 & 18.19 & \bf{616.50} & 
0.42\\CON3-1 & \bf{\underline{554.47}} & 17.12 & 
554.86 & 17.68 & 555.60 & 
-0.20\\CON3-2 & \bf{\underline{519.11}} & 18.75 & 
519.68 & 19.03 & 521.40 & 
-0.44\\CON3-3 & \bf{\underline{591.19}} & 18.18 & 
591.19 & 18.64 & 591.20 & 
-0.00\\CON3-4 & \bf{\underline{588.79}} & 17.94 & 
588.92 & 18.11 & 589.30 & 
-0.09\\CON3-5 & \bf{563.70} & 18.02 & 
565.11 & 18.16 & 563.70 & 0.00\\
CON3-6 & \bf{\underline{499.07}} & 20.19 & 
501.07 & 20.24 & 499.20 & 
-0.03\\CON3-7 & \bf{\underline{576.84}} & 17.36 & 
578.30 & 16.78 & 577.50 & 
-0.11\\CON3-8 & \bf{\underline{523.05}} & 19.80 & 
523.05 & 20.02 & 523.10 & 
-0.01\\CON3-9 & 578.25 & 17.07 & 
581.66 & 18.34 & \bf{578.20} & 
0.01\\CON8-0 & 867.71 & 17.63 & 
868.06 & 17.52 & \bf{858.90} & 
1.03\\CON8-1 & \bf{\underline{740.85}} & 18.51 & 
741.61 & 19.07 & 740.90 & 
-0.01\\CON8-2 & \bf{\underline{713.53}} & 22.47 & 
715.32 & 22.20 & 714.30 & 
-0.11\\CON8-3 & \bf{\underline{812.11}} & 18.50 & 
814.30 & 18.64 & 812.30 & 
-0.02\\CON8-4 & 776.72 & 17.00 & 
783.23 & 17.85 & \bf{770.10} & 
0.86\\CON8-5 & \bf{\underline{758.84}} & 19.82 & 
760.03 & 17.80 & 766.60 & 
-1.01\\CON8-6 & \bf{\underline{687.66}} & 21.92 & 
691.04 & 20.41 & 697.20 & 
-1.37\\CON8-7 & \bf{\underline{814.50}} & 22.39 & 
814.74 & 18.83 & 814.80 & 
-0.04\\CON8-8 & 778.77 & 19.85 & 
782.88 & 19.12 & \bf{771.30} & 
0.97\\CON8-9 & \bf{\underline{811.43}} & 18.87 & 
812.11 & 18.81 & 815.10 & 
-0.45\\[1ex]\hline
\end{tabular}
\label{table:nonlin}
\end{table} \clearpage
\begin{table}[ht]
\caption{Resultados de la ejecución de la metaheurística ACO, utilizando instancias de Dethloff con la configuración -n 20 -alpha 1.0 -beta 0.5 -q 0.8 -ro 0.015}
\centering
\small
\begin{tabular}{c c c c c c c}
\hline\hline
Instancia & Costo mínimo & Tiempo(seg.) & Costo promedio & Tiempo promedio(seg.) & Costo ACO & \%Gap \\ [0.5ex]
\hline
SCA3-0 & \bf{\underline{636.06}} & 14.98 & 
636.06 & 15.43 & 636.10 & 
-0.01\\SCA3-1 & \bf{\underline{697.84}} & 15.66 & 
697.84 & 16.01 & 700.10 & 
-0.32\\SCA3-2 & 659.34 & 13.91 & 
660.55 & 14.35 & \bf{659.30} & 
0.01\\SCA3-3 & 680.04 & 15.38 & 
680.18 & 14.79 & \bf{680.00} & 
0.01\\SCA3-4 & \bf{690.50} & 15.83 & 
690.50 & 16.81 & 690.50 & 0.00\\
SCA3-5 & \bf{\underline{661.07}} & 16.44 & 
662.90 & 15.96 & 671.10 & 
-1.49\\SCA3-6 & 652.47 & 16.11 & 
652.82 & 15.02 & \bf{651.10} & 
0.21\\SCA3-7 & 666.15 & 14.40 & 
666.15 & 14.74 & \bf{666.10} & 
0.01\\SCA3-8 & \bf{\underline{719.47}} & 15.98 & 
719.47 & 15.89 & 719.50 & 
-0.00\\SCA3-9 & \bf{681.00} & 14.35 & 
681.00 & 13.92 & 681.00 & 0.00\\
SCA8-0 & 975.84 & 17.17 & 
981.12 & 16.66 & \bf{961.60} & 
1.48\\SCA8-1 & \bf{\underline{1050.38}} & 14.31 & 
1057.23 & 13.93 & 1063.00 & 
-1.19\\SCA8-2 & 1049.22 & 14.35 & 
1049.70 & 14.13 & \bf{1040.60} & 
0.83\\SCA8-3 & 995.12 & 16.04 & 
1006.04 & 15.72 & \bf{985.90} & 
0.94\\SCA8-4 & \bf{\underline{1065.49}} & 18.98 & 
1068.76 & 17.30 & 1071.00 & 
-0.51\\SCA8-5 & \bf{\underline{1034.74}} & 16.67 & 
1042.22 & 17.25 & 1054.30 & 
-1.86\\SCA8-6 & \bf{\underline{972.48}} & 18.32 & 
973.62 & 17.52 & 972.50 & 
-0.00\\SCA8-7 & 1067.20 & 17.47 & 
1070.49 & 16.99 & \bf{1059.70} & 
0.71\\SCA8-8 & \bf{\underline{1071.18}} & 16.72 & 
1071.18 & 16.63 & 1082.70 & 
-1.06\\SCA8-9 & \bf{\underline{1061.23}} & 14.24 & 
1065.67 & 13.56 & 1081.40 & 
-1.87\\CON3-0 & 619.09 & 17.16 & 
621.19 & 16.62 & \bf{616.50} & 
0.42\\CON3-1 & \bf{\underline{554.47}} & 16.64 & 
554.47 & 16.37 & 555.60 & 
-0.20\\CON3-2 & \bf{\underline{521.38}} & 18.17 & 
521.38 & 16.93 & 521.40 & 
-0.00\\CON3-3 & \bf{\underline{591.19}} & 16.89 & 
591.19 & 16.87 & 591.20 & 
-0.00\\CON3-4 & \bf{\underline{588.79}} & 15.27 & 
588.92 & 14.99 & 589.30 & 
-0.09\\CON3-5 & \bf{563.70} & 16.88 & 
564.59 & 16.37 & 563.70 & 0.00\\
CON3-6 & 501.33 & 18.35 & 
501.77 & 18.40 & \bf{499.20} & 
0.43\\CON3-7 & \bf{\underline{576.48}} & 14.06 & 
577.62 & 14.20 & 577.50 & 
-0.18\\CON3-8 & \bf{\underline{523.05}} & 15.26 & 
523.21 & 15.22 & 523.10 & 
-0.01\\CON3-9 & 578.25 & 16.24 & 
585.29 & 16.25 & \bf{578.20} & 
0.01\\CON8-0 & 866.22 & 17.66 & 
868.76 & 17.20 & \bf{858.90} & 
0.85\\CON8-1 & \bf{\underline{740.85}} & 16.82 & 
742.63 & 16.12 & 740.90 & 
-0.01\\CON8-2 & \bf{\underline{713.44}} & 19.71 & 
714.74 & 20.07 & 714.30 & 
-0.12\\CON8-3 & \bf{\underline{811.07}} & 16.72 & 
812.19 & 16.82 & 812.30 & 
-0.15\\CON8-4 & 776.37 & 16.00 & 
784.79 & 16.85 & \bf{770.10} & 
0.81\\CON8-5 & \bf{\underline{759.93}} & 14.86 & 
761.58 & 14.57 & 766.60 & 
-0.87\\CON8-6 & \bf{\underline{680.40}} & 17.47 & 
690.19 & 17.72 & 697.20 & 
-2.41\\CON8-7 & \bf{\underline{814.50}} & 15.03 & 
817.34 & 16.82 & 814.80 & 
-0.04\\CON8-8 & 782.86 & 17.81 & 
784.45 & 16.95 & \bf{771.30} & 
1.50\\CON8-9 & \bf{\underline{810.18}} & 16.56 & 
811.74 & 16.99 & 815.10 & 
-0.60\\[1ex]\hline
\end{tabular}
\label{table:nonlin}
\end{table} \clearpage
\begin{table}[ht]
\caption{Resultados de la ejecución de la metaheurística ACO, utilizando instancias de Dethloff con la configuración -n 20 -alpha 1.0 -beta 1.0 -q 0.8 -ro 0.015}
\centering
\small
\begin{tabular}{c c c c c c c}
\hline\hline
Instancia & Costo mínimo & Tiempo(seg.) & Costo promedio & Tiempo promedio(seg.) & Costo ACO & \%Gap \\ [0.5ex]
\hline
SCA3-0 & \bf{\underline{636.06}} & 14.31 & 
636.06 & 14.50 & 636.10 & 
-0.01\\SCA3-1 & \bf{\underline{697.84}} & 15.34 & 
697.84 & 14.55 & 700.10 & 
-0.32\\SCA3-2 & 659.34 & 12.62 & 
662.21 & 13.04 & \bf{659.30} & 
0.01\\SCA3-3 & 680.04 & 14.14 & 
680.18 & 13.65 & \bf{680.00} & 
0.01\\SCA3-4 & \bf{690.50} & 15.17 & 
690.50 & 14.81 & 690.50 & 0.00\\
SCA3-5 & \bf{\underline{659.90}} & 14.68 & 
661.90 & 15.29 & 671.10 & 
-1.67\\SCA3-6 & 652.94 & 13.80 & 
652.94 & 13.89 & \bf{651.10} & 
0.28\\SCA3-7 & 666.15 & 13.76 & 
666.15 & 14.19 & \bf{666.10} & 
0.01\\SCA3-8 & \bf{\underline{719.47}} & 14.45 & 
719.47 & 13.72 & 719.50 & 
-0.00\\SCA3-9 & \bf{681.00} & 13.69 & 
681.00 & 12.44 & 681.00 & 0.00\\
SCA8-0 & 973.03 & 15.96 & 
978.12 & 15.97 & \bf{961.60} & 
1.19\\SCA8-1 & \bf{\underline{1054.45}} & 12.35 & 
1058.48 & 12.28 & 1063.00 & 
-0.80\\SCA8-2 & 1046.29 & 11.76 & 
1047.32 & 11.14 & \bf{1040.60} & 
0.55\\SCA8-3 & 1000.96 & 16.08 & 
1007.58 & 15.25 & \bf{985.90} & 
1.53\\SCA8-4 & \bf{\underline{1065.49}} & 14.80 & 
1066.88 & 14.87 & 1071.00 & 
-0.51\\SCA8-5 & \bf{\underline{1034.74}} & 16.78 & 
1043.73 & 16.79 & 1054.30 & 
-1.86\\SCA8-6 & \bf{\underline{972.48}} & 16.39 & 
973.83 & 16.82 & 972.50 & 
-0.00\\SCA8-7 & 1067.03 & 16.56 & 
1067.16 & 16.64 & \bf{1059.70} & 
0.69\\SCA8-8 & \bf{\underline{1071.18}} & 16.11 & 
1071.18 & 15.63 & 1082.70 & 
-1.06\\SCA8-9 & \bf{\underline{1060.50}} & 12.07 & 
1065.24 & 13.20 & 1081.40 & 
-1.93\\CON3-0 & 619.09 & 15.02 & 
621.72 & 15.24 & \bf{616.50} & 
0.42\\CON3-1 & \bf{\underline{554.47}} & 14.15 & 
554.47 & 14.37 & 555.60 & 
-0.20\\CON3-2 & \bf{\underline{521.38}} & 12.74 & 
521.38 & 13.40 & 521.40 & 
-0.00\\CON3-3 & \bf{591.20} & 16.70 & 
591.20 & 16.65 & 591.20 & 0.00\\
CON3-4 & \bf{\underline{588.79}} & 12.92 & 
588.92 & 13.21 & 589.30 & 
-0.09\\CON3-5 & \bf{563.70} & 15.82 & 
564.29 & 15.47 & 563.70 & 0.00\\
CON3-6 & 500.37 & 16.42 & 
502.25 & 17.14 & \bf{499.20} & 
0.23\\CON3-7 & \bf{\underline{576.48}} & 12.95 & 
577.49 & 13.11 & 577.50 & 
-0.18\\CON3-8 & \bf{\underline{523.05}} & 14.47 & 
523.23 & 13.51 & 523.10 & 
-0.01\\CON3-9 & 580.05 & 12.68 & 
582.52 & 13.70 & \bf{578.20} & 
0.32\\CON8-0 & 870.24 & 14.43 & 
873.80 & 14.44 & \bf{858.90} & 
1.32\\CON8-1 & \bf{\underline{740.85}} & 14.22 & 
740.85 & 14.06 & 740.90 & 
-0.01\\CON8-2 & \bf{\underline{713.44}} & 21.02 & 
713.93 & 20.64 & 714.30 & 
-0.12\\CON8-3 & \bf{\underline{812.11}} & 15.71 & 
814.66 & 15.53 & 812.30 & 
-0.02\\CON8-4 & 780.15 & 14.73 & 
785.25 & 14.67 & \bf{770.10} & 
1.31\\CON8-5 & \bf{\underline{755.16}} & 14.08 & 
760.54 & 14.32 & 766.60 & 
-1.49\\CON8-6 & \bf{\underline{686.39}} & 16.57 & 
691.33 & 16.05 & 697.20 & 
-1.55\\CON8-7 & \bf{\underline{814.50}} & 13.24 & 
814.80 & 13.00 & 814.80 & 
-0.04\\CON8-8 & 777.70 & 16.18 & 
780.78 & 16.67 & \bf{771.30} & 
0.83\\CON8-9 & \bf{\underline{811.75}} & 16.17 & 
812.90 & 15.59 & 815.10 & 
-0.41\\[1ex]\hline
\end{tabular}
\label{table:nonlin}
\end{table} \clearpage
\begin{table}[ht]
\caption{Resultados de la ejecución de la metaheurística ACO, utilizando instancias de Dethloff con la configuración -n 20 -alpha 1.0 -beta 2.0 -q 0.8 -ro 0.015}
\centering
\small
\begin{tabular}{c c c c c c c}
\hline\hline
Instancia & Costo mínimo & Tiempo(seg.) & Costo promedio & Tiempo promedio(seg.) & Costo ACO & \%Gap \\ [0.5ex]
\hline
SCA3-0 & \bf{\underline{636.06}} & 13.15 & 
637.18 & 13.61 & 636.10 & 
-0.01\\SCA3-1 & \bf{\underline{697.84}} & 14.30 & 
697.84 & 13.99 & 700.10 & 
-0.32\\SCA3-2 & 659.34 & 12.48 & 
661.77 & 13.09 & \bf{659.30} & 
0.01\\SCA3-3 & 680.04 & 13.52 & 
680.04 & 14.22 & \bf{680.00} & 
0.01\\SCA3-4 & \bf{690.50} & 14.30 & 
690.50 & 14.04 & 690.50 & 0.00\\
SCA3-5 & \bf{\underline{659.90}} & 14.38 & 
662.24 & 14.31 & 671.10 & 
-1.67\\SCA3-6 & 652.94 & 12.39 & 
653.16 & 13.15 & \bf{651.10} & 
0.28\\SCA3-7 & 666.15 & 12.76 & 
666.15 & 11.66 & \bf{666.10} & 
0.01\\SCA3-8 & \bf{\underline{719.47}} & 12.82 & 
719.97 & 13.73 & 719.50 & 
-0.00\\SCA3-9 & \bf{681.00} & 11.16 & 
681.00 & 11.47 & 681.00 & 0.00\\
SCA8-0 & \bf{\underline{961.50}} & 12.87 & 
969.79 & 14.64 & 961.60 & 
-0.01\\SCA8-1 & \bf{\underline{1053.40}} & 11.40 & 
1056.69 & 11.17 & 1063.00 & 
-0.90\\SCA8-2 & 1042.17 & 9.95 & 
1046.12 & 10.36 & \bf{1040.60} & 
0.15\\SCA8-3 & 1002.63 & 15.52 & 
1011.12 & 14.74 & \bf{985.90} & 
1.70\\SCA8-4 & \bf{\underline{1067.82}} & 14.37 & 
1071.25 & 14.64 & 1071.00 & 
-0.30\\SCA8-5 & \bf{\underline{1034.74}} & 15.50 & 
1042.62 & 15.44 & 1054.30 & 
-1.86\\SCA8-6 & \bf{\underline{972.48}} & 15.84 & 
973.62 & 15.82 & 972.50 & 
-0.00\\SCA8-7 & 1067.20 & 15.52 & 
1068.66 & 15.72 & \bf{1059.70} & 
0.71\\SCA8-8 & \bf{\underline{1071.18}} & 15.40 & 
1072.14 & 14.57 & 1082.70 & 
-1.06\\SCA8-9 & \bf{\underline{1067.42}} & 11.29 & 
1067.42 & 11.65 & 1081.40 & 
-1.29\\CON3-0 & 617.59 & 15.25 & 
621.16 & 15.27 & \bf{616.50} & 
0.18\\CON3-1 & \bf{\underline{554.47}} & 14.46 & 
555.65 & 14.61 & 555.60 & 
-0.20\\CON3-2 & \bf{\underline{521.38}} & 11.60 & 
521.44 & 12.44 & 521.40 & 
-0.00\\CON3-3 & \bf{\underline{591.19}} & 16.18 & 
591.20 & 15.13 & 591.20 & 
-0.00\\CON3-4 & \bf{\underline{588.79}} & 12.37 & 
589.45 & 12.66 & 589.30 & 
-0.09\\CON3-5 & 564.88 & 13.85 & 
564.88 & 13.64 & \bf{563.70} & 
0.21\\CON3-6 & 500.80 & 16.19 & 
501.29 & 16.33 & \bf{499.20} & 
0.32\\CON3-7 & \bf{\underline{576.84}} & 12.78 & 
577.97 & 12.46 & 577.50 & 
-0.11\\CON3-8 & \bf{\underline{523.05}} & 12.12 & 
523.23 & 12.40 & 523.10 & 
-0.01\\CON3-9 & 578.98 & 13.86 & 
586.11 & 12.75 & \bf{578.20} & 
0.13\\CON8-0 & 869.43 & 13.72 & 
870.69 & 14.16 & \bf{858.90} & 
1.23\\CON8-1 & \bf{\underline{740.85}} & 13.89 & 
741.23 & 13.98 & 740.90 & 
-0.01\\CON8-2 & \bf{\underline{712.89}} & 18.66 & 
713.53 & 19.15 & 714.30 & 
-0.20\\CON8-3 & \bf{\underline{811.07}} & 14.02 & 
815.95 & 14.47 & 812.30 & 
-0.15\\CON8-4 & 776.37 & 14.84 & 
781.36 & 14.49 & \bf{770.10} & 
0.81\\CON8-5 & \bf{\underline{760.91}} & 14.28 & 
762.50 & 13.95 & 766.60 & 
-0.74\\CON8-6 & \bf{\underline{689.23}} & 15.90 & 
690.82 & 15.88 & 697.20 & 
-1.14\\CON8-7 & \bf{\underline{814.79}} & 12.33 & 
814.88 & 11.93 & 814.80 & 
-0.00\\CON8-8 & 782.09 & 16.39 & 
786.01 & 15.88 & \bf{771.30} & 
1.40\\CON8-9 & \bf{\underline{810.18}} & 15.47 & 
812.61 & 15.10 & 815.10 & 
-0.60\\[1ex]\hline
\end{tabular}
\label{table:nonlin}
\end{table} \clearpage
\begin{table}[ht]
\caption{Resultados de la ejecución de la metaheurística ACO, utilizando instancias de Dethloff con la configuración -n 20 -alpha 1.0 -beta 4.0 -q 0.8 -ro 0.015}
\centering
\small
\begin{tabular}{c c c c c c c}
\hline\hline
Instancia & Costo mínimo & Tiempo(seg.) & Costo promedio & Tiempo promedio(seg.) & Costo ACO & \%Gap \\ [0.5ex]
\hline
SCA3-0 & \bf{\underline{636.06}} & 12.17 & 
636.20 & 12.84 & 636.10 & 
-0.01\\SCA3-1 & \bf{\underline{697.84}} & 14.03 & 
697.84 & 14.09 & 700.10 & 
-0.32\\SCA3-2 & 659.34 & 12.73 & 
660.55 & 12.86 & \bf{659.30} & 
0.01\\SCA3-3 & 680.04 & 14.28 & 
680.32 & 13.51 & \bf{680.00} & 
0.01\\SCA3-4 & \bf{690.50} & 14.37 & 
690.50 & 13.96 & 690.50 & 0.00\\
SCA3-5 & \bf{\underline{662.75}} & 14.58 & 
664.77 & 14.07 & 671.10 & 
-1.24\\SCA3-6 & 652.94 & 12.65 & 
653.99 & 12.32 & \bf{651.10} & 
0.28\\SCA3-7 & 666.15 & 10.38 & 
666.15 & 10.35 & \bf{666.10} & 
0.01\\SCA3-8 & \bf{\underline{719.47}} & 11.17 & 
719.97 & 11.38 & 719.50 & 
-0.00\\SCA3-9 & \bf{681.00} & 10.35 & 
681.00 & 10.22 & 681.00 & 0.00\\
SCA8-0 & \bf{\underline{961.50}} & 15.48 & 
968.32 & 14.17 & 961.60 & 
-0.01\\SCA8-1 & \bf{\underline{1062.88}} & 10.90 & 
1065.74 & 11.39 & 1063.00 & 
-0.01\\SCA8-2 & 1044.24 & 10.00 & 
1049.05 & 9.94 & \bf{1040.60} & 
0.35\\SCA8-3 & 1011.22 & 13.72 & 
1017.06 & 14.01 & \bf{985.90} & 
2.57\\SCA8-4 & \bf{\underline{1065.49}} & 13.73 & 
1067.14 & 13.95 & 1071.00 & 
-0.51\\SCA8-5 & \bf{\underline{1041.29}} & 14.97 & 
1050.54 & 15.54 & 1054.30 & 
-1.23\\SCA8-6 & 977.03 & 15.42 & 
978.00 & 15.17 & \bf{972.50} & 
0.47\\SCA8-7 & 1067.20 & 14.52 & 
1067.20 & 15.22 & \bf{1059.70} & 
0.71\\SCA8-8 & \bf{\underline{1071.18}} & 14.64 & 
1073.91 & 14.16 & 1082.70 & 
-1.06\\SCA8-9 & \bf{\underline{1067.42}} & 10.95 & 
1067.42 & 11.62 & 1081.40 & 
-1.29\\CON3-0 & 617.59 & 14.63 & 
622.07 & 15.13 & \bf{616.50} & 
0.18\\CON3-1 & \bf{\underline{554.47}} & 13.88 & 
556.73 & 13.87 & 555.60 & 
-0.20\\CON3-2 & \bf{\underline{521.38}} & 12.37 & 
521.87 & 12.44 & 521.40 & 
-0.00\\CON3-3 & \bf{\underline{591.19}} & 14.87 & 
591.31 & 14.96 & 591.20 & 
-0.00\\CON3-4 & \bf{\underline{588.79}} & 12.55 & 
588.92 & 12.24 & 589.30 & 
-0.09\\CON3-5 & \bf{563.70} & 14.44 & 
565.53 & 13.30 & 563.70 & 0.00\\
CON3-6 & 500.80 & 16.83 & 
501.82 & 16.30 & \bf{499.20} & 
0.32\\CON3-7 & 578.22 & 17.99 & 
578.36 & 13.16 & \bf{577.50} & 
0.12\\CON3-8 & \bf{\underline{523.05}} & 10.75 & 
523.56 & 10.64 & 523.10 & 
-0.01\\CON3-9 & 588.48 & 12.08 & 
588.48 & 12.47 & \bf{578.20} & 
1.78\\CON8-0 & 869.43 & 13.70 & 
876.09 & 13.35 & \bf{858.90} & 
1.23\\CON8-1 & \bf{\underline{740.85}} & 12.55 & 
740.85 & 12.93 & 740.90 & 
-0.01\\CON8-2 & \bf{\underline{713.44}} & 19.34 & 
714.30 & 18.97 & 714.30 & 
-0.12\\CON8-3 & \bf{\underline{811.07}} & 12.63 & 
814.69 & 13.20 & 812.30 & 
-0.15\\CON8-4 & 776.37 & 14.09 & 
786.58 & 13.56 & \bf{770.10} & 
0.81\\CON8-5 & \bf{\underline{758.12}} & 12.78 & 
762.39 & 12.78 & 766.60 & 
-1.11\\CON8-6 & \bf{\underline{694.29}} & 16.27 & 
695.57 & 15.88 & 697.20 & 
-0.42\\CON8-7 & 814.86 & 10.96 & 
816.50 & 11.12 & \bf{814.80} & 
0.01\\CON8-8 & 788.15 & 15.89 & 
790.13 & 14.91 & \bf{771.30} & 
2.18\\CON8-9 & \bf{\underline{810.18}} & 15.42 & 
814.00 & 15.31 & 815.10 & 
-0.60\\[1ex]\hline
\end{tabular}
\label{table:nonlin}
\end{table} \clearpage
\begin{table}[ht]
\caption{Resultados de la ejecución de la metaheurística ILS, utilizando instancias de Dethloff con la configuración -n 300 -LS 80}
\centering
\small
\begin{tabular}{c c c c c c c}
\hline\hline
Instancia & Costo mínimo & Tiempo(seg.) & Costo promedio & Tiempo promedio(seg.) & Costo ILS & \%Gap \\ [0.5ex]
\hline
SCA3-0 & 636.06 & 84.90 & 
638.30 & 84.72 & \bf{635.62} & 
0.07\\SCA3-1 & \bf{697.84} & 85.29 & 
698.50 & 84.14 & 697.84 & 0.00\\
SCA3-2 & \bf{659.34} & 78.74 & 
659.34 & 83.28 & 659.34 & 0.00\\
SCA3-3 & \bf{680.04} & 89.41 & 
680.18 & 86.87 & 680.04 & 0.00\\
SCA3-4 & \bf{690.50} & 113.24 & 
690.50 & 91.97 & 690.50 & 0.00\\
SCA3-5 & \bf{659.90} & 82.44 & 
664.71 & 85.72 & 659.90 & 0.00\\
SCA3-6 & \bf{651.09} & 80.78 & 
651.43 & 81.41 & 651.09 & 0.00\\
SCA3-7 & 666.15 & 84.16 & 
668.58 & 83.67 & \bf{659.17} & 
1.06\\SCA3-8 & \bf{719.47} & 88.59 & 
719.54 & 84.81 & 719.47 & 0.00\\
SCA3-9 & \bf{681.00} & 83.97 & 
681.00 & 92.04 & 681.00 & 0.00\\
SCA8-0 & 970.64 & 65.88 & 
977.27 & 67.97 & \bf{961.50} & 
0.95\\SCA8-1 & 1066.73 & 62.66 & 
1073.36 & 65.75 & \bf{1049.65} & 
1.63\\SCA8-2 & 1052.94 & 63.39 & 
1054.00 & 68.25 & \bf{1039.64} & 
1.28\\SCA8-3 & 1007.85 & 72.10 & 
1012.49 & 66.28 & \bf{983.34} & 
2.49\\SCA8-4 & 1067.55 & 81.60 & 
1068.67 & 81.92 & \bf{1065.49} & 
0.19\\SCA8-5 & 1048.28 & 78.40 & 
1057.08 & 71.17 & \bf{1027.08} & 
2.06\\SCA8-6 & 975.81 & 79.08 & 
983.61 & 73.36 & \bf{971.82} & 
0.41\\SCA8-7 & 1064.45 & 81.07 & 
1069.67 & 77.67 & \bf{1051.28} & 
1.25\\SCA8-8 & \bf{1071.18} & 72.26 & 
1071.18 & 76.15 & 1071.18 & 0.00\\
SCA8-9 & 1065.60 & 65.77 & 
1070.68 & 68.38 & \bf{1060.50} & 
0.48\\CON3-0 & \bf{616.52} & 86.00 & 
620.96 & 86.86 & 616.52 & 0.00\\
CON3-1 & 558.09 & 88.21 & 
559.31 & 92.33 & \bf{554.47} & 
0.65\\CON3-2 & 519.11 & 87.91 & 
520.81 & 87.69 & \bf{518.00} & 
0.21\\CON3-3 & \bf{591.19} & 87.73 & 
591.19 & 90.19 & 591.19 & 0.00\\
CON3-4 & \bf{588.79} & 119.80 & 
590.70 & 98.44 & 588.79 & 0.00\\
CON3-5 & \bf{563.70} & 84.76 & 
565.35 & 86.74 & 563.70 & 0.00\\
CON3-6 & 502.16 & 90.88 & 
502.28 & 89.72 & \bf{499.05} & 
0.62\\CON3-7 & \bf{576.48} & 86.36 & 
577.54 & 87.19 & 576.48 & 0.00\\
CON3-8 & 100000 & 0 & 
523.05 & 0.00 & \bf{523.05} & 
19018.63\\CON3-9 & 100000 & 0 & 
nan & nan & \bf{578.24} & 
17193.86\\CON8-0 & 100000 & 0 & 
nan & nan & \bf{857.17} & 
11566.30\\CON8-1 & 100000 & 0 & 
nan & nan & \bf{740.85} & 
13398.01\\CON8-2 & 100000 & 0 & 
nan & nan & \bf{712.89} & 
13927.41\\CON8-3 & 100000 & 0 & 
nan & nan & \bf{811.07} & 
12229.39\\CON8-4 & 100000 & 0 & 
nan & nan & \bf{772.25} & 
12849.17\\CON8-5 & 100000 & 0 & 
nan & nan & \bf{754.88} & 
13147.14\\CON8-6 & 100000 & 0 & 
nan & nan & \bf{678.92} & 
14629.28\\CON8-7 & 100000 & 0 & 
nan & nan & \bf{811.96} & 
12215.88\\CON8-8 & \bf{\underline{85.00}} & Command & 
85.00 & 0.00 & 767.53 & 
-88.93\\CON8-9 & 100000 & 0 & 
nan & nan & \bf{809.00} & 
12260.94\\[1ex]\hline
\end{tabular}
\label{table:nonlin}
\end{table} \clearpage
\begin{table}[ht]
\caption{Resultados de la ejecución de la metaheurística ACO, utilizando instancias de Dethloff con la configuración -n 50 -alpha 1.0 -beta 3.0 -q 0.8 -ro 0.015}
\centering
\small
\begin{tabular}{c c c c c c c}
\hline\hline
Instancia & Costo mínimo & Tiempo(seg.) & Costo promedio & Tiempo promedio(seg.) & Costo ACO & \%Gap \\ [0.5ex]
\hline
SCA3-0 & 100000 & 0 & 
nan & nan & \bf{636.10} & 
15620.80\\SCA3-1 & 100000 & 0 & 
nan & nan & \bf{700.10} & 
14183.67\\SCA3-2 & 100000 & 0 & 
nan & nan & \bf{659.30} & 
15067.60\\SCA3-3 & 100000 & 0 & 
nan & nan & \bf{680.00} & 
14605.88\\SCA3-4 & 100000 & 0 & 
nan & nan & \bf{690.50} & 
14382.26\\SCA3-5 & 100000 & 0 & 
nan & nan & \bf{671.10} & 
14800.91\\SCA3-6 & 100000 & 0 & 
nan & nan & \bf{651.10} & 
15258.62\\SCA3-7 & 100000 & 0 & 
nan & nan & \bf{666.10} & 
14912.76\\SCA3-8 & 100000 & 0 & 
nan & nan & \bf{719.50} & 
13798.54\\SCA3-9 & 100000 & 0 & 
nan & nan & \bf{681.00} & 
14584.29\\SCA8-0 & 100000 & 0 & 
nan & nan & \bf{961.60} & 
10299.33\\SCA8-1 & 100000 & 0 & 
nan & nan & \bf{1063.00} & 
9307.34\\SCA8-2 & 100000 & 0 & 
nan & nan & \bf{1040.60} & 
9509.84\\SCA8-3 & 100000 & 0 & 
nan & nan & \bf{985.90} & 
10043.02\\SCA8-4 & 100000 & 0 & 
nan & nan & \bf{1071.00} & 
9237.07\\SCA8-5 & 100000 & 0 & 
nan & nan & \bf{1054.30} & 
9384.97\\SCA8-6 & 100000 & 0 & 
nan & nan & \bf{972.50} & 
10182.78\\SCA8-7 & 100000 & 0 & 
nan & nan & \bf{1059.70} & 
9336.63\\SCA8-8 & 100000 & 0 & 
nan & nan & \bf{1082.70} & 
9136.17\\SCA8-9 & 100000 & 0 & 
nan & nan & \bf{1081.40} & 
9147.27\\CON3-0 & 100000 & 0 & 
nan & nan & \bf{616.50} & 
16120.60\\CON3-1 & 100000 & 0 & 
nan & nan & \bf{555.60} & 
17898.56\\CON3-2 & 100000 & 0 & 
nan & nan & \bf{521.40} & 
19079.13\\CON3-3 & 100000 & 0 & 
nan & nan & \bf{591.20} & 
16814.75\\CON3-4 & 100000 & 0 & 
nan & nan & \bf{589.30} & 
16869.29\\CON3-5 & 100000 & 0 & 
nan & nan & \bf{563.70} & 
17639.93\\CON3-6 & 100000 & 0 & 
nan & nan & \bf{499.20} & 
19932.05\\CON3-7 & 100000 & 0 & 
nan & nan & \bf{577.50} & 
17216.02\\CON3-8 & 100000 & 0 & 
nan & nan & \bf{523.10} & 
19016.80\\CON3-9 & 100000 & 0 & 
nan & nan & \bf{578.20} & 
17195.05\\CON8-0 & 100000 & 0 & 
nan & nan & \bf{858.90} & 
11542.80\\CON8-1 & 100000 & 0 & 
nan & nan & \bf{740.90} & 
13397.10\\CON8-2 & 100000 & 0 & 
nan & nan & \bf{714.30} & 
13899.72\\CON8-3 & 100000 & 0 & 
nan & nan & \bf{812.30} & 
12210.72\\CON8-4 & 100000 & 0 & 
nan & nan & \bf{770.10} & 
12885.33\\CON8-5 & 100000 & 0 & 
nan & nan & \bf{766.60} & 
12944.61\\CON8-6 & 100000 & 0 & 
nan & nan & \bf{697.20} & 
14243.09\\CON8-7 & 100000 & 0 & 
nan & nan & \bf{814.80} & 
12172.95\\CON8-8 & 100000 & 0 & 
nan & nan & \bf{771.30} & 
12865.12\\CON8-9 & 100000 & 0 & 
nan & nan & \bf{815.10} & 
12168.43\\[1ex]\hline
\end{tabular}
\label{table:nonlin}
\end{table} \clearpage
\begin{table}[ht]
\caption{Resultados de la ejecución de la metaheurística ILS, utilizando instancias de Dethloff con la configuración -n 300 -LS 80}
\centering
\small
\begin{tabular}{c c c c c c c}
\hline\hline
Instancia & Costo mínimo & Tiempo(seg.) & Costo promedio & Tiempo promedio(seg.) & Costo ILS & \%Gap \\ [0.5ex]
\hline
SCA3-0 & 636.06 & 90.54 & 
636.06 & 89.58 & \bf{635.62} & 
0.07\\SCA3-1 & \bf{697.84} & 92.95 & 
697.84 & 88.80 & 697.84 & 0.00\\
SCA3-2 & \bf{659.34} & 87.27 & 
660.55 & 87.50 & 659.34 & 0.00\\
SCA3-3 & \bf{680.04} & 89.54 & 
680.04 & 90.85 & 680.04 & 0.00\\
SCA3-4 & \bf{690.50} & 95.42 & 
690.50 & 88.47 & 690.50 & 0.00\\
SCA3-5 & \bf{659.90} & 90.68 & 
660.61 & 87.99 & 659.90 & 0.00\\
SCA3-6 & \bf{651.09} & 88.70 & 
651.09 & 85.86 & 651.09 & 0.00\\
SCA3-7 & 666.15 & 90.66 & 
667.20 & 87.54 & \bf{659.17} & 
1.06\\SCA3-8 & \bf{719.47} & 81.66 & 
719.47 & 86.14 & 719.47 & 0.00\\
SCA3-9 & \bf{681.00} & 86.96 & 
681.00 & 87.09 & 681.00 & 0.00\\
SCA8-0 & 973.22 & 73.08 & 
978.48 & 71.99 & \bf{961.50} & 
1.22\\SCA8-1 & 1059.16 & 66.98 & 
1064.28 & 68.67 & \bf{1049.65} & 
0.91\\SCA8-2 & 1047.46 & 73.68 & 
1052.34 & 73.30 & \bf{1039.64} & 
0.75\\SCA8-3 & 1012.37 & 68.77 & 
1013.20 & 73.17 & \bf{983.34} & 
2.95\\SCA8-4 & \bf{1065.49} & 72.17 & 
1067.46 & 70.67 & 1065.49 & 0.00\\
SCA8-5 & 1043.65 & 73.58 & 
1055.75 & 74.28 & \bf{1027.08} & 
1.61\\SCA8-6 & 981.28 & 69.54 & 
984.89 & 72.83 & \bf{971.82} & 
0.97\\SCA8-7 & 1067.11 & 74.43 & 
1067.34 & 73.77 & \bf{1051.28} & 
1.51\\SCA8-8 & \bf{1071.18} & 71.52 & 
1072.14 & 72.56 & 1071.18 & 0.00\\
SCA8-9 & 1067.42 & 67.45 & 
1074.69 & 73.57 & \bf{1060.50} & 
0.65\\CON3-0 & 617.59 & 92.46 & 
620.43 & 91.11 & \bf{616.52} & 
0.17\\CON3-1 & \bf{554.47} & 95.74 & 
557.26 & 95.99 & 554.47 & 0.00\\
CON3-2 & 519.11 & 91.26 & 
520.81 & 92.23 & \bf{518.00} & 
0.21\\CON3-3 & 591.20 & 89.34 & 
591.27 & 91.70 & \bf{591.19} & 
0.00\\CON3-4 & \bf{588.79} & 89.30 & 
589.58 & 92.40 & 588.79 & 0.00\\
CON3-5 & 564.88 & 89.15 & 
566.47 & 85.39 & \bf{563.70} & 
0.21\\CON3-6 & \bf{499.05} & 89.02 & 
501.04 & 88.14 & 499.05 & 0.00\\
CON3-7 & 578.41 & 88.99 & 
578.41 & 86.65 & \bf{576.48} & 
0.33\\CON3-8 & \bf{523.05} & 89.71 & 
523.38 & 94.44 & 523.05 & 0.00\\
CON3-9 & 582.98 & 90.06 & 
585.94 & 89.19 & \bf{578.24} & 
0.82\\CON8-0 & 865.04 & 76.39 & 
873.55 & 74.89 & \bf{857.17} & 
0.92\\CON8-1 & 743.13 & 69.62 & 
752.15 & 68.48 & \bf{740.85} & 
0.31\\CON8-2 & 716.03 & 79.87 & 
716.88 & 77.52 & \bf{712.89} & 
0.44\\CON8-3 & 811.23 & 79.32 & 
817.64 & 78.67 & \bf{811.07} & 
0.02\\CON8-4 & 776.72 & 67.32 & 
784.91 & 69.75 & \bf{772.25} & 
0.58\\CON8-5 & 758.12 & 73.20 & 
760.31 & 72.00 & \bf{754.88} & 
0.43\\CON8-6 & 683.83 & 72.65 & 
690.18 & 73.04 & \bf{678.92} & 
0.72\\CON8-7 & \bf{811.96} & 72.89 & 
814.27 & 73.92 & 811.96 & 0.00\\
CON8-8 & 780.71 & 70.39 & 
783.33 & 70.28 & \bf{767.53} & 
1.72\\CON8-9 & 812.03 & 77.68 & 
815.36 & 75.84 & \bf{809.00} & 
0.37\\[1ex]\hline
\end{tabular}
\label{table:nonlin}
\end{table} \clearpage
\begin{table}[ht]
\caption{Resultados de la ejecución de la metaheurística SCA, utilizando instancias de Dethloff con la configuración -n 50 -b 10 -y 0.1}
\centering
\small
\begin{tabular}{c c c c c c c}
\hline\hline
Instancia & Costo mínimo & Tiempo(seg.) & Costo promedio & Tiempo promedio(seg.) & Costo SCA & \%Gap \\ [0.5ex]
\hline
SCA3-0 & 640.55 & 13.57 & 
640.55 & 13.57 & \bf{636.06} & 
0.71\\SCA3-1 & 701.86 & 9.05 & 
701.86 & 9.05 & \bf{697.84} & 
0.58\\SCA3-2 & 661.13 & 10.53 & 
661.13 & 10.53 & \bf{659.34} & 
0.27\\SCA3-3 & 681.74 & 13.14 & 
681.74 & 13.14 & \bf{680.04} & 
0.25\\SCA3-4 & \bf{690.50} & 19.12 & 
690.50 & 19.12 & 690.50 & 0.00\\
SCA3-5 & 681.81 & 9.40 & 
681.81 & 9.40 & \bf{659.90} & 
3.32\\SCA3-6 & 653.81 & 13.10 & 
653.81 & 13.10 & \bf{651.09} & 
0.42\\SCA3-7 & 666.15 & 5.01 & 
666.15 & 5.01 & \bf{659.17} & 
1.06\\SCA3-8 & \bf{719.47} & 14.20 & 
719.47 & 14.20 & 719.47 & 0.00\\
SCA3-9 & \bf{681.00} & 14.50 & 
681.00 & 14.50 & 681.00 & 0.00\\
SCA8-0 & 987.90 & 38.75 & 
987.90 & 38.75 & \bf{961.50} & 
2.75\\SCA8-1 & 1089.61 & 21.64 & 
1089.61 & 21.64 & \bf{1050.20} & 
3.75\\SCA8-2 & 1050.98 & 50.95 & 
1050.98 & 50.95 & \bf{1039.64} & 
1.09\\SCA8-3 & 998.53 & 41.11 & 
998.53 & 41.11 & \bf{983.34} & 
1.54\\SCA8-4 & 1090.83 & 23.64 & 
1090.83 & 23.64 & \bf{1065.49} & 
2.38\\SCA8-5 & 1048.31 & 49.42 & 
1048.31 & 49.42 & \bf{1027.08} & 
2.07\\SCA8-6 & 972.48 & 52.08 & 
972.48 & 52.08 & \bf{971.82} & 
0.07\\SCA8-7 & 1070.53 & 23.37 & 
1070.53 & 23.37 & \bf{1052.17} & 
1.74\\SCA8-8 & 1086.88 & 23.68 & 
1086.88 & 23.68 & \bf{1071.18} & 
1.47\\SCA8-9 & 1091.19 & 42.43 & 
1091.19 & 42.43 & \bf{1060.50} & 
2.89\\CON3-0 & 619.09 & 5.64 & 
619.09 & 5.64 & \bf{616.52} & 
0.42\\CON3-1 & 560.61 & 14.53 & 
560.61 & 14.53 & \bf{554.47} & 
1.11\\CON3-2 & 521.38 & 7.94 & 
521.38 & 7.94 & \bf{519.26} & 
0.41\\CON3-3 & 591.20 & 11.30 & 
591.20 & 11.30 & \bf{591.19} & 
0.00\\CON3-4 & 598.47 & 6.59 & 
598.47 & 6.59 & \bf{589.32} & 
1.55\\CON3-5 & 566.96 & 20.87 & 
566.96 & 20.87 & \bf{563.70} & 
0.58\\CON3-6 & 502.16 & 11.73 & 
502.16 & 11.73 & \bf{500.80} & 
0.27\\CON3-7 & 586.01 & 10.41 & 
586.01 & 10.41 & \bf{576.48} & 
1.65\\CON3-8 & 523.68 & 7.84 & 
523.68 & 7.84 & \bf{523.05} & 
0.12\\CON3-9 & 582.79 & 16.74 & 
582.79 & 16.74 & \bf{580.05} & 
0.47\\CON8-0 & 872.74 & 25.82 & 
872.74 & 25.82 & \bf{857.17} & 
1.82\\CON8-1 & 748.39 & 65.10 & 
748.39 & 65.10 & \bf{740.85} & 
1.02\\CON8-2 & 716.07 & 59.75 & 
716.07 & 59.75 & \bf{713.44} & 
0.37\\CON8-3 & 832.15 & 44.36 & 
832.15 & 44.36 & \bf{811.07} & 
2.60\\CON8-4 & 788.77 & 33.45 & 
788.77 & 33.45 & \bf{772.25} & 
2.14\\CON8-5 & \bf{\underline{754.95}} & 42.06 & 
754.95 & 42.06 & 756.91 & 
-0.26\\CON8-6 & 695.86 & 53.89 & 
695.86 & 53.89 & \bf{678.92} & 
2.50\\CON8-7 & 815.71 & 47.24 & 
815.71 & 47.24 & \bf{811.96} & 
0.46\\CON8-8 & 793.84 & 27.50 & 
793.84 & 27.50 & \bf{767.53} & 
3.43\\CON8-9 & 827.05 & 41.39 & 
827.05 & 41.39 & \bf{809.00} & 
2.23\\[1ex]\hline
\end{tabular}
\label{table:nonlin}
\end{table} \clearpage
\begin{table}[ht]
\caption{Resultados de la ejecución de la metaheurística IGA, utilizando instancias de Dethloff con la configuración -n 50 -p 10}
\centering
\small
\begin{tabular}{c c c c c c c}
\hline\hline
Instancia & Costo mínimo & Tiempo(seg.) & Costo promedio & Tiempo promedio(seg.) & Costo IGA & \%Gap \\ [0.5ex]
\hline
SCA3-0 & 641.69 & 0.44 & 
641.69 & 0.44 & \bf{636.06} & 
0.89\\SCA3-1 & 701.53 & 0.45 & 
701.53 & 0.45 & \bf{697.84} & 
0.53\\SCA3-2 & 666.33 & 0.41 & 
666.33 & 0.41 & \bf{659.34} & 
1.06\\SCA3-3 & 682.46 & 0.42 & 
682.46 & 0.42 & \bf{680.04} & 
0.36\\SCA3-4 & \bf{690.50} & 0.44 & 
690.50 & 0.44 & 690.50 & 0.00\\
SCA3-5 & 665.04 & 0.39 & 
665.04 & 0.39 & \bf{659.90} & 
0.78\\SCA3-6 & 663.50 & 0.50 & 
663.50 & 0.50 & \bf{651.09} & 
1.91\\SCA3-7 & 671.77 & 0.40 & 
671.77 & 0.40 & \bf{659.17} & 
1.91\\SCA3-8 & 727.65 & 0.45 & 
727.65 & 0.45 & \bf{719.47} & 
1.14\\SCA3-9 & 685.00 & 0.45 & 
685.00 & 0.45 & \bf{681.00} & 
0.59\\SCA8-0 & 1027.51 & 0.44 & 
1027.51 & 0.44 & \bf{961.50} & 
6.87\\SCA8-1 & 1086.83 & 0.45 & 
1086.83 & 0.45 & \bf{1050.20} & 
3.49\\SCA8-2 & 1075.20 & 0.49 & 
1075.20 & 0.49 & \bf{1039.64} & 
3.42\\SCA8-3 & 1035.53 & 0.48 & 
1035.53 & 0.48 & \bf{983.34} & 
5.31\\SCA8-4 & 1106.32 & 0.48 & 
1106.32 & 0.48 & \bf{1065.49} & 
3.83\\SCA8-5 & 1065.97 & 0.44 & 
1065.97 & 0.44 & \bf{1027.08} & 
3.79\\SCA8-6 & 1006.74 & 0.44 & 
1006.74 & 0.44 & \bf{971.82} & 
3.59\\SCA8-7 & 1107.00 & 0.43 & 
1107.00 & 0.43 & \bf{1052.17} & 
5.21\\SCA8-8 & 1089.07 & 0.38 & 
1089.07 & 0.38 & \bf{1071.18} & 
1.67\\SCA8-9 & 1088.36 & 0.36 & 
1088.36 & 0.36 & \bf{1060.50} & 
2.63\\CON3-0 & 639.93 & 0.42 & 
639.93 & 0.42 & \bf{616.52} & 
3.80\\CON3-1 & 573.55 & 0.48 & 
573.55 & 0.48 & \bf{554.47} & 
3.44\\CON3-2 & 530.73 & 0.50 & 
530.73 & 0.50 & \bf{519.26} & 
2.21\\CON3-3 & 610.22 & 0.40 & 
610.22 & 0.40 & \bf{591.19} & 
3.22\\CON3-4 & 604.64 & 0.40 & 
604.64 & 0.40 & \bf{589.32} & 
2.60\\CON3-5 & 584.38 & 0.46 & 
584.38 & 0.46 & \bf{563.70} & 
3.67\\CON3-6 & 508.98 & 0.48 & 
508.98 & 0.48 & \bf{500.80} & 
1.63\\CON3-7 & 599.76 & 0.48 & 
599.76 & 0.48 & \bf{576.48} & 
4.04\\CON3-8 & 536.56 & 0.44 & 
536.56 & 0.44 & \bf{523.05} & 
2.58\\CON3-9 & 590.64 & 0.49 & 
590.64 & 0.49 & \bf{580.05} & 
1.83\\CON8-0 & 885.37 & 0.46 & 
885.37 & 0.46 & \bf{857.17} & 
3.29\\CON8-1 & 800.86 & 0.47 & 
800.86 & 0.47 & \bf{740.85} & 
8.10\\CON8-2 & 743.05 & 0.42 & 
743.05 & 0.42 & \bf{713.44} & 
4.15\\CON8-3 & 844.03 & 0.44 & 
844.03 & 0.44 & \bf{811.07} & 
4.06\\CON8-4 & 818.03 & 0.43 & 
818.03 & 0.43 & \bf{772.25} & 
5.93\\CON8-5 & 775.25 & 0.43 & 
775.25 & 0.43 & \bf{756.91} & 
2.42\\CON8-6 & 703.11 & 0.38 & 
703.11 & 0.38 & \bf{678.92} & 
3.56\\CON8-7 & 821.28 & 0.47 & 
821.28 & 0.47 & \bf{811.96} & 
1.15\\CON8-8 & 807.36 & 0.40 & 
807.36 & 0.40 & \bf{767.53} & 
5.19\\CON8-9 & 833.43 & 0.40 & 
833.43 & 0.40 & \bf{809.00} & 
3.02\\[1ex]\hline
\end{tabular}
\label{table:nonlin}
\end{table} \clearpage
\begin{table}[ht]
\caption{Resultados de la ejecución de la metaheurística IGA, utilizando instancias de Dethloff con la configuración -n 200 -p 40}
\centering
\small
\begin{tabular}{c c c c c c c}
\hline\hline
Instancia & Costo mínimo & Tiempo(seg.) & Costo promedio & Tiempo promedio(seg.) & Costo IGA & \%Gap \\ [0.5ex]
\hline
SCA3-0 & 640.55 & 2.38 & 
640.55 & 2.38 & \bf{636.06} & 
0.71\\SCA3-1 & 700.50 & 2.37 & 
700.50 & 2.37 & \bf{697.84} & 
0.38\\SCA3-2 & 666.85 & 2.22 & 
666.85 & 2.22 & \bf{659.34} & 
1.14\\SCA3-3 & \bf{680.04} & 2.28 & 
680.04 & 2.28 & 680.04 & 0.00\\
SCA3-4 & \bf{690.50} & 2.67 & 
690.50 & 2.67 & 690.50 & 0.00\\
SCA3-5 & 679.84 & 2.36 & 
679.84 & 2.36 & \bf{659.90} & 
3.02\\SCA3-6 & 652.94 & 2.40 & 
652.94 & 2.40 & \bf{651.09} & 
0.28\\SCA3-7 & 671.67 & 2.25 & 
671.67 & 2.25 & \bf{659.17} & 
1.90\\SCA3-8 & 719.77 & 2.36 & 
719.77 & 2.36 & \bf{719.47} & 
0.04\\SCA3-9 & \bf{681.00} & 2.35 & 
681.00 & 2.35 & 681.00 & 0.00\\
SCA8-0 & 100000 & 0 & 
nan & nan & \bf{961.50} & 
10300.42\\SCA8-1 & 1073.02 & 2.53 & 
1073.02 & 2.53 & \bf{1050.20} & 
2.17\\SCA8-2 & 1058.01 & 2.45 & 
1058.01 & 2.45 & \bf{1039.64} & 
1.77\\SCA8-3 & 1006.40 & 2.38 & 
1006.40 & 2.38 & \bf{983.34} & 
2.35\\SCA8-4 & 1082.58 & 2.64 & 
1082.58 & 2.64 & \bf{1065.49} & 
1.60\\SCA8-5 & 1058.41 & 2.40 & 
1058.41 & 2.40 & \bf{1027.08} & 
3.05\\SCA8-6 & 992.74 & 2.51 & 
992.74 & 2.51 & \bf{971.82} & 
2.15\\SCA8-7 & 1083.73 & 1.86 & 
1083.73 & 1.86 & \bf{1052.17} & 
3.00\\SCA8-8 & 1082.11 & 2.03 & 
1082.11 & 2.03 & \bf{1071.18} & 
1.02\\SCA8-9 & 1079.93 & 2.43 & 
1079.93 & 2.43 & \bf{1060.50} & 
1.83\\CON3-0 & 633.24 & 2.48 & 
633.24 & 2.48 & \bf{616.52} & 
2.71\\CON3-1 & 560.75 & 2.76 & 
560.75 & 2.76 & \bf{554.47} & 
1.13\\CON3-2 & 521.38 & 2.72 & 
521.38 & 2.72 & \bf{519.26} & 
0.41\\CON3-3 & 594.31 & 2.40 & 
594.31 & 2.40 & \bf{591.19} & 
0.53\\CON3-4 & 592.58 & 2.32 & 
592.58 & 2.32 & \bf{589.32} & 
0.55\\CON3-5 & 569.74 & 2.40 & 
569.74 & 2.40 & \bf{563.70} & 
1.07\\CON3-6 & 513.44 & 2.62 & 
513.44 & 2.62 & \bf{500.80} & 
2.52\\CON3-7 & 582.12 & 1.94 & 
582.12 & 1.94 & \bf{576.48} & 
0.98\\CON3-8 & \bf{523.05} & 2.69 & 
523.05 & 2.69 & 523.05 & 0.00\\
CON3-9 & 582.79 & 2.52 & 
582.79 & 2.52 & \bf{580.05} & 
0.47\\CON8-0 & 891.03 & 2.65 & 
891.03 & 2.65 & \bf{857.17} & 
3.95\\CON8-1 & 741.70 & 3.25 & 
741.70 & 3.25 & \bf{740.85} & 
0.11\\CON8-2 & 721.98 & 3.26 & 
721.98 & 3.26 & \bf{713.44} & 
1.20\\CON8-3 & 828.78 & 2.45 & 
828.78 & 2.45 & \bf{811.07} & 
2.18\\CON8-4 & 815.46 & 2.84 & 
815.46 & 2.84 & \bf{772.25} & 
5.60\\CON8-5 & 766.70 & 2.40 & 
766.70 & 2.40 & \bf{756.91} & 
1.29\\CON8-6 & 694.84 & 2.53 & 
694.84 & 2.53 & \bf{678.92} & 
2.34\\CON8-7 & 827.34 & 2.43 & 
827.34 & 2.43 & \bf{811.96} & 
1.89\\CON8-8 & 794.02 & 2.31 & 
794.02 & 2.31 & \bf{767.53} & 
3.45\\CON8-9 & 822.78 & 2.60 & 
822.78 & 2.60 & \bf{809.00} & 
1.70\\[1ex]\hline
\end{tabular}
\label{table:nonlin}
\end{table} \clearpage
\begin{table}[ht]
\caption{Resultados de la ejecución de la metaheurística ACO, utilizando instancias de Dethloff con la configuración -n 2.0 -alpha 1.0 -beta 3.0 -q 0.1 -ro 0.015}
\centering
\small
\begin{tabular}{c c c c c c c}
\hline\hline
Instancia & Costo mínimo & Tiempo(seg.) & Costo promedio & Tiempo promedio(seg.) & Costo ACO & \%Gap \\ [0.5ex]
\hline
SCA3-0 & \bf{\underline{636.06}} & 1.56 & 
638.38 & 1.46 & 636.10 & 
-0.01\\SCA3-1 & \bf{\underline{697.84}} & 1.49 & 
697.84 & 1.54 & 700.10 & 
-0.32\\SCA3-2 & 659.34 & 1.48 & 
663.43 & 1.43 & \bf{659.30} & 
0.01\\SCA3-3 & 680.04 & 1.36 & 
680.04 & 1.40 & \bf{680.00} & 
0.01\\SCA3-4 & \bf{690.50} & 1.40 & 
690.50 & 1.42 & 690.50 & 0.00\\
SCA3-5 & \bf{\underline{659.90}} & 1.48 & 
663.33 & 1.47 & 671.10 & 
-1.67\\SCA3-6 & 652.94 & 1.49 & 
654.59 & 1.43 & \bf{651.10} & 
0.28\\SCA3-7 & 666.15 & 1.52 & 
667.20 & 1.40 & \bf{666.10} & 
0.01\\SCA3-8 & \bf{\underline{719.47}} & 1.55 & 
719.62 & 1.50 & 719.50 & 
-0.00\\SCA3-9 & \bf{681.00} & 1.32 & 
681.59 & 1.34 & 681.00 & 0.00\\
SCA8-0 & 973.50 & 1.59 & 
979.59 & 1.58 & \bf{961.60} & 
1.24\\SCA8-1 & \bf{\underline{1056.89}} & 1.41 & 
1063.63 & 1.45 & 1063.00 & 
-0.57\\SCA8-2 & 1049.22 & 1.27 & 
1050.08 & 1.28 & \bf{1040.60} & 
0.83\\SCA8-3 & 995.50 & 1.59 & 
1009.30 & 1.51 & \bf{985.90} & 
0.97\\SCA8-4 & \bf{\underline{1065.49}} & 1.62 & 
1070.30 & 1.58 & 1071.00 & 
-0.51\\SCA8-5 & \bf{\underline{1051.46}} & 1.77 & 
1058.64 & 1.72 & 1054.30 & 
-0.27\\SCA8-6 & \bf{\underline{972.48}} & 1.72 & 
983.22 & 1.72 & 972.50 & 
-0.00\\SCA8-7 & 1066.65 & 1.60 & 
1069.21 & 1.58 & \bf{1059.70} & 
0.66\\SCA8-8 & \bf{\underline{1082.11}} & 1.72 & 
1087.96 & 1.71 & 1082.70 & 
-0.05\\SCA8-9 & \bf{\underline{1067.42}} & 1.39 & 
1068.15 & 1.42 & 1081.40 & 
-1.29\\CON3-0 & 620.49 & 1.74 & 
623.88 & 1.67 & \bf{616.50} & 
0.65\\CON3-1 & 556.04 & 1.70 & 
558.42 & 1.63 & \bf{555.60} & 
0.08\\CON3-2 & \bf{\underline{521.38}} & 1.63 & 
521.38 & 1.50 & 521.40 & 
-0.00\\CON3-3 & \bf{\underline{591.19}} & 1.72 & 
591.31 & 1.71 & 591.20 & 
-0.00\\CON3-4 & \bf{\underline{588.79}} & 1.30 & 
592.84 & 1.32 & 589.30 & 
-0.09\\CON3-5 & \bf{563.70} & 1.48 & 
567.93 & 1.42 & 563.70 & 0.00\\
CON3-6 & 500.80 & 1.68 & 
502.63 & 1.69 & \bf{499.20} & 
0.32\\CON3-7 & 581.27 & 1.36 & 
585.24 & 1.33 & \bf{577.50} & 
0.65\\CON3-8 & \bf{\underline{523.05}} & 1.38 & 
525.43 & 1.44 & 523.10 & 
-0.01\\CON3-9 & 588.40 & 1.64 & 
588.97 & 1.56 & \bf{578.20} & 
1.76\\CON8-0 & 870.49 & 1.56 & 
876.35 & 1.58 & \bf{858.90} & 
1.35\\CON8-1 & 740.93 & 1.54 & 
741.80 & 1.61 & \bf{740.90} & 
0.00\\CON8-2 & 716.53 & 1.88 & 
718.75 & 1.88 & \bf{714.30} & 
0.31\\CON8-3 & 817.57 & 1.53 & 
823.42 & 1.53 & \bf{812.30} & 
0.65\\CON8-4 & 777.24 & 1.55 & 
787.07 & 1.47 & \bf{770.10} & 
0.93\\CON8-5 & \bf{\underline{760.03}} & 1.54 & 
761.99 & 1.56 & 766.60 & 
-0.86\\CON8-6 & \bf{\underline{688.68}} & 1.68 & 
693.14 & 1.75 & 697.20 & 
-1.22\\CON8-7 & 814.86 & 1.49 & 
818.51 & 1.41 & \bf{814.80} & 
0.01\\CON8-8 & 775.92 & 1.87 & 
783.73 & 1.79 & \bf{771.30} & 
0.60\\CON8-9 & 817.21 & 1.81 & 
818.15 & 1.73 & \bf{815.10} & 
0.26\\[1ex]\hline
\end{tabular}
\label{table:nonlin}
\end{table} \clearpage
\begin{table}[ht]
\caption{Resultados de la ejecución de la metaheurística ACO, utilizando instancias de Dethloff con la configuración -n 2.0 -alpha 1.0 -beta 3.0 -q 0.1 -ro 0.015}
\centering
\small
\begin{tabular}{c c c c c c c}
\hline\hline
Instancia & Costo mínimo & Tiempo(seg.) & Costo promedio & Tiempo promedio(seg.) & Costo ACO & \%Gap \\ [0.5ex]
\hline
SCA3-0 & \bf{\underline{636.06}} & 1.42 & 
636.06 & 1.47 & 636.10 & 
-0.01\\SCA3-1 & \bf{\underline{697.84}} & 1.52 & 
697.84 & 1.60 & 700.10 & 
-0.32\\SCA3-2 & 661.13 & 1.58 & 
664.06 & 1.47 & \bf{659.30} & 
0.28\\SCA3-3 & 680.04 & 1.41 & 
680.51 & 1.46 & \bf{680.00} & 
0.01\\SCA3-4 & \bf{690.50} & 1.60 & 
690.50 & 1.58 & 690.50 & 0.00\\
SCA3-5 & \bf{\underline{665.04}} & 1.60 & 
666.71 & 1.57 & 671.10 & 
-0.90\\SCA3-6 & 652.47 & 1.99 & 
653.37 & 1.64 & \bf{651.10} & 
0.21\\SCA3-7 & 666.15 & 1.26 & 
666.15 & 1.41 & \bf{666.10} & 
0.01\\SCA3-8 & \bf{\underline{719.47}} & 1.47 & 
721.32 & 1.56 & 719.50 & 
-0.00\\SCA3-9 & \bf{681.00} & 1.30 & 
681.00 & 1.33 & 681.00 & 0.00\\
SCA8-0 & 968.79 & 1.55 & 
987.39 & 1.55 & \bf{961.60} & 
0.75\\SCA8-1 & \bf{\underline{1059.27}} & 1.50 & 
1065.80 & 1.41 & 1063.00 & 
-0.35\\SCA8-2 & 1047.63 & 1.30 & 
1049.89 & 1.28 & \bf{1040.60} & 
0.68\\SCA8-3 & 1006.20 & 1.50 & 
1011.24 & 1.53 & \bf{985.90} & 
2.06\\SCA8-4 & \bf{\underline{1067.66}} & 1.98 & 
1079.43 & 1.71 & 1071.00 & 
-0.31\\SCA8-5 & \bf{\underline{1038.59}} & 1.74 & 
1052.47 & 1.78 & 1054.30 & 
-1.49\\SCA8-6 & \bf{\underline{971.82}} & 1.66 & 
975.60 & 1.68 & 972.50 & 
-0.07\\SCA8-7 & 1070.67 & 1.61 & 
1077.35 & 1.63 & \bf{1059.70} & 
1.04\\SCA8-8 & \bf{\underline{1071.18}} & 1.65 & 
1073.91 & 1.62 & 1082.70 & 
-1.06\\SCA8-9 & \bf{\underline{1063.68}} & 1.39 & 
1068.77 & 1.42 & 1081.40 & 
-1.64\\CON3-0 & 617.59 & 1.52 & 
624.79 & 1.60 & \bf{616.50} & 
0.18\\CON3-1 & \bf{\underline{554.47}} & 1.62 & 
557.67 & 1.61 & 555.60 & 
-0.20\\CON3-2 & \bf{\underline{519.11}} & 1.48 & 
520.25 & 1.46 & 521.40 & 
-0.44\\CON3-3 & \bf{\underline{591.19}} & 1.72 & 
591.36 & 1.65 & 591.20 & 
-0.00\\CON3-4 & \bf{\underline{588.79}} & 1.41 & 
590.82 & 1.46 & 589.30 & 
-0.09\\CON3-5 & 564.88 & 1.48 & 
568.32 & 1.47 & \bf{563.70} & 
0.21\\CON3-6 & 502.16 & 1.86 & 
503.94 & 1.76 & \bf{499.20} & 
0.59\\CON3-7 & 578.41 & 1.40 & 
581.88 & 1.38 & \bf{577.50} & 
0.16\\CON3-8 & 523.14 & 1.58 & 
524.18 & 1.59 & \bf{523.10} & 
0.01\\CON3-9 & 580.05 & 1.38 & 
586.98 & 1.48 & \bf{578.20} & 
0.32\\CON8-0 & 861.11 & 1.56 & 
874.88 & 1.53 & \bf{858.90} & 
0.26\\CON8-1 & 742.61 & 1.66 & 
747.52 & 1.67 & \bf{740.90} & 
0.23\\CON8-2 & 716.31 & 1.84 & 
721.65 & 1.91 & \bf{714.30} & 
0.28\\CON8-3 & \bf{\underline{811.07}} & 1.58 & 
814.58 & 1.64 & 812.30 & 
-0.15\\CON8-4 & 773.64 & 1.44 & 
785.29 & 1.51 & \bf{770.10} & 
0.46\\CON8-5 & \bf{\underline{760.62}} & 1.55 & 
763.16 & 1.56 & 766.60 & 
-0.78\\CON8-6 & \bf{\underline{691.42}} & 1.83 & 
695.38 & 1.76 & 697.20 & 
-0.83\\CON8-7 & 814.86 & 1.58 & 
815.39 & 1.50 & \bf{814.80} & 
0.01\\CON8-8 & 784.71 & 1.82 & 
789.18 & 1.80 & \bf{771.30} & 
1.74\\CON8-9 & \bf{\underline{815.02}} & 1.67 & 
816.21 & 1.69 & 815.10 & 
-0.01\\[1ex]\hline
\end{tabular}
\label{table:nonlin}
\end{table} \clearpage
\begin{table}[ht]
\caption{Resultados de la ejecución de la metaheurística ACO, utilizando instancias de Dethloff con la configuración -n 2.0 -alpha 1.0 -beta 3.0 -q .2 -ro 0.015}
\centering
\small
\begin{tabular}{c c c c c c c}
\hline\hline
Instancia & Costo mínimo & Tiempo(seg.) & Costo promedio & Tiempo promedio(seg.) & Costo ACO & \%Gap \\ [0.5ex]
\hline
SCA3-0 & \bf{\underline{636.06}} & 1.37 & 
637.18 & 1.39 & 636.10 & 
-0.01\\SCA3-1 & \bf{\underline{697.84}} & 1.47 & 
697.84 & 1.56 & 700.10 & 
-0.32\\SCA3-2 & 661.13 & 1.48 & 
664.91 & 1.45 & \bf{659.30} & 
0.28\\SCA3-3 & 680.04 & 1.38 & 
680.18 & 1.41 & \bf{680.00} & 
0.01\\SCA3-4 & \bf{690.50} & 1.48 & 
690.50 & 1.53 & 690.50 & 0.00\\
SCA3-5 & \bf{\underline{665.04}} & 1.52 & 
666.90 & 1.44 & 671.10 & 
-0.90\\SCA3-6 & 652.94 & 1.48 & 
653.40 & 1.46 & \bf{651.10} & 
0.28\\SCA3-7 & 666.15 & 1.26 & 
666.53 & 1.32 & \bf{666.10} & 
0.01\\SCA3-8 & \bf{\underline{719.47}} & 1.48 & 
720.19 & 1.46 & 719.50 & 
-0.00\\SCA3-9 & \bf{681.00} & 1.14 & 
681.86 & 1.25 & 681.00 & 0.00\\
SCA8-0 & 970.64 & 1.52 & 
984.42 & 1.59 & \bf{961.60} & 
0.94\\SCA8-1 & 1063.82 & 1.36 & 
1070.11 & 1.38 & \bf{1063.00} & 
0.08\\SCA8-2 & 1051.21 & 1.14 & 
1052.95 & 1.26 & \bf{1040.60} & 
1.02\\SCA8-3 & 1007.85 & 1.57 & 
1014.74 & 1.48 & \bf{985.90} & 
2.23\\SCA8-4 & \bf{\underline{1067.55}} & 1.56 & 
1068.22 & 1.59 & 1071.00 & 
-0.32\\SCA8-5 & \bf{\underline{1050.49}} & 1.63 & 
1056.08 & 1.69 & 1054.30 & 
-0.36\\SCA8-6 & \bf{\underline{972.48}} & 1.60 & 
982.43 & 1.64 & 972.50 & 
-0.00\\SCA8-7 & 1067.03 & 1.63 & 
1071.60 & 1.58 & \bf{1059.70} & 
0.69\\SCA8-8 & \bf{\underline{1071.18}} & 1.85 & 
1081.64 & 1.71 & 1082.70 & 
-1.06\\SCA8-9 & \bf{\underline{1065.60}} & 1.44 & 
1067.70 & 1.41 & 1081.40 & 
-1.46\\CON3-0 & 616.52 & 1.46 & 
620.33 & 1.51 & \bf{616.50} & 
0.00\\CON3-1 & 556.28 & 1.62 & 
559.22 & 1.62 & \bf{555.60} & 
0.12\\CON3-2 & \bf{\underline{521.36}} & 1.79 & 
522.03 & 1.56 & 521.40 & 
-0.01\\CON3-3 & \bf{\underline{591.19}} & 1.62 & 
591.20 & 1.63 & 591.20 & 
-0.00\\CON3-4 & 592.58 & 1.35 & 
593.38 & 1.41 & \bf{589.30} & 
0.56\\CON3-5 & \bf{563.70} & 1.35 & 
565.95 & 1.48 & 563.70 & 0.00\\
CON3-6 & 500.37 & 1.81 & 
502.52 & 1.71 & \bf{499.20} & 
0.23\\CON3-7 & 578.22 & 1.41 & 
580.79 & 1.42 & \bf{577.50} & 
0.12\\CON3-8 & 523.68 & 1.46 & 
525.01 & 1.45 & \bf{523.10} & 
0.11\\CON3-9 & 588.40 & 1.40 & 
589.20 & 1.44 & \bf{578.20} & 
1.76\\CON8-0 & 865.86 & 1.43 & 
875.14 & 1.55 & \bf{858.90} & 
0.81\\CON8-1 & 742.29 & 1.58 & 
746.98 & 1.58 & \bf{740.90} & 
0.19\\CON8-2 & \bf{\underline{713.68}} & 1.88 & 
715.25 & 1.83 & 714.30 & 
-0.09\\CON8-3 & 817.22 & 1.52 & 
817.48 & 1.58 & \bf{812.30} & 
0.61\\CON8-4 & 785.27 & 1.39 & 
789.33 & 1.50 & \bf{770.10} & 
1.97\\CON8-5 & \bf{\underline{762.01}} & 1.60 & 
763.46 & 1.58 & 766.60 & 
-0.60\\CON8-6 & \bf{\underline{691.28}} & 1.73 & 
696.14 & 1.73 & 697.20 & 
-0.85\\CON8-7 & 814.86 & 1.47 & 
816.82 & 1.44 & \bf{814.80} & 
0.01\\CON8-8 & 782.34 & 1.70 & 
787.33 & 1.77 & \bf{771.30} & 
1.43\\CON8-9 & \bf{\underline{811.18}} & 1.70 & 
814.66 & 1.73 & 815.10 & 
-0.48\\[1ex]\hline
\end{tabular}
\label{table:nonlin}
\end{table} \clearpage
\begin{table}[ht]
\caption{Resultados de la ejecución de la metaheurística ACO, utilizando instancias de Dethloff con la configuración -n 2.0 -alpha 1.0 -beta 3.0 -q .3 -ro 0.015}
\centering
\small
\begin{tabular}{c c c c c c c}
\hline\hline
Instancia & Costo mínimo & Tiempo(seg.) & Costo promedio & Tiempo promedio(seg.) & Costo ACO & \%Gap \\ [0.5ex]
\hline
SCA3-0 & \bf{\underline{636.06}} & 1.36 & 
636.13 & 1.41 & 636.10 & 
-0.01\\SCA3-1 & \bf{\underline{697.84}} & 1.54 & 
697.84 & 1.54 & 700.10 & 
-0.32\\SCA3-2 & 659.34 & 1.45 & 
662.22 & 1.40 & \bf{659.30} & 
0.01\\SCA3-3 & 680.04 & 1.43 & 
680.67 & 1.42 & \bf{680.00} & 
0.01\\SCA3-4 & \bf{690.50} & 1.40 & 
690.50 & 1.50 & 690.50 & 0.00\\
SCA3-5 & \bf{\underline{665.64}} & 1.43 & 
667.78 & 1.47 & 671.10 & 
-0.81\\SCA3-6 & 652.94 & 1.39 & 
653.81 & 1.46 & \bf{651.10} & 
0.28\\SCA3-7 & 666.15 & 1.17 & 
667.09 & 1.27 & \bf{666.10} & 
0.01\\SCA3-8 & \bf{\underline{719.47}} & 1.38 & 
724.74 & 1.41 & 719.50 & 
-0.00\\SCA3-9 & \bf{681.00} & 1.26 & 
681.98 & 1.26 & 681.00 & 0.00\\
SCA8-0 & 968.79 & 1.60 & 
989.32 & 1.59 & \bf{961.60} & 
0.75\\SCA8-1 & 1065.28 & 1.48 & 
1068.58 & 1.42 & \bf{1063.00} & 
0.21\\SCA8-2 & 1046.29 & 1.32 & 
1049.27 & 1.25 & \bf{1040.60} & 
0.55\\SCA8-3 & 1002.56 & 1.41 & 
1008.46 & 1.44 & \bf{985.90} & 
1.69\\SCA8-4 & \bf{\underline{1065.49}} & 1.66 & 
1070.95 & 1.54 & 1071.00 & 
-0.51\\SCA8-5 & \bf{\underline{1050.49}} & 1.64 & 
1055.81 & 1.67 & 1054.30 & 
-0.36\\SCA8-6 & \bf{\underline{972.48}} & 1.74 & 
980.71 & 1.75 & 972.50 & 
-0.00\\SCA8-7 & 1075.87 & 1.56 & 
1079.36 & 1.57 & \bf{1059.70} & 
1.53\\SCA8-8 & \bf{\underline{1071.18}} & 1.63 & 
1079.14 & 1.63 & 1082.70 & 
-1.06\\SCA8-9 & \bf{\underline{1067.42}} & 1.29 & 
1067.42 & 1.37 & 1081.40 & 
-1.29\\CON3-0 & 620.58 & 1.53 & 
624.59 & 1.66 & \bf{616.50} & 
0.66\\CON3-1 & \bf{\underline{554.47}} & 1.50 & 
558.83 & 1.53 & 555.60 & 
-0.20\\CON3-2 & \bf{\underline{521.38}} & 1.47 & 
523.22 & 1.50 & 521.40 & 
-0.00\\CON3-3 & \bf{591.20} & 1.53 & 
591.20 & 1.57 & 591.20 & 0.00\\
CON3-4 & 591.43 & 1.49 & 
593.48 & 1.42 & \bf{589.30} & 
0.36\\CON3-5 & 564.88 & 1.48 & 
567.48 & 1.41 & \bf{563.70} & 
0.21\\CON3-6 & 502.16 & 1.72 & 
503.15 & 1.74 & \bf{499.20} & 
0.59\\CON3-7 & \bf{\underline{576.87}} & 1.48 & 
581.04 & 1.42 & 577.50 & 
-0.11\\CON3-8 & 524.59 & 1.47 & 
525.40 & 1.41 & \bf{523.10} & 
0.28\\CON3-9 & 589.00 & 1.50 & 
589.29 & 1.44 & \bf{578.20} & 
1.87\\CON8-0 & 870.22 & 1.55 & 
880.97 & 1.57 & \bf{858.90} & 
1.32\\CON8-1 & \bf{\underline{740.85}} & 1.55 & 
748.58 & 1.56 & 740.90 & 
-0.01\\CON8-2 & 716.19 & 1.90 & 
717.19 & 1.87 & \bf{714.30} & 
0.26\\CON8-3 & 812.54 & 1.46 & 
813.82 & 1.53 & \bf{812.30} & 
0.03\\CON8-4 & 776.72 & 1.41 & 
784.68 & 1.52 & \bf{770.10} & 
0.86\\CON8-5 & \bf{\underline{760.03}} & 1.57 & 
763.74 & 1.54 & 766.60 & 
-0.86\\CON8-6 & \bf{\underline{688.51}} & 1.71 & 
695.88 & 1.70 & 697.20 & 
-1.25\\CON8-7 & \bf{\underline{814.79}} & 1.45 & 
822.32 & 1.35 & 814.80 & 
-0.00\\CON8-8 & 782.34 & 1.78 & 
786.53 & 1.79 & \bf{771.30} & 
1.43\\CON8-9 & \bf{\underline{814.37}} & 1.66 & 
816.90 & 1.70 & 815.10 & 
-0.09\\[1ex]\hline
\end{tabular}
\label{table:nonlin}
\end{table} \clearpage
\begin{table}[ht]
\caption{Resultados de la ejecución de la metaheurística ACO, utilizando instancias de Dethloff con la configuración -n 2.0 -alpha 1.0 -beta 3.0 -q .4 -ro 0.015}
\centering
\small
\begin{tabular}{c c c c c c c}
\hline\hline
Instancia & Costo mínimo & Tiempo(seg.) & Costo promedio & Tiempo promedio(seg.) & Costo ACO & \%Gap \\ [0.5ex]
\hline
SCA3-0 & \bf{\underline{636.06}} & 1.42 & 
637.18 & 1.40 & 636.10 & 
-0.01\\SCA3-1 & \bf{\underline{697.84}} & 1.52 & 
697.84 & 1.52 & 700.10 & 
-0.32\\SCA3-2 & 659.34 & 1.34 & 
661.00 & 1.38 & \bf{659.30} & 
0.01\\SCA3-3 & 680.04 & 1.37 & 
680.32 & 1.40 & \bf{680.00} & 
0.01\\SCA3-4 & \bf{690.50} & 1.42 & 
690.50 & 1.44 & 690.50 & 0.00\\
SCA3-5 & \bf{\underline{662.75}} & 1.45 & 
666.06 & 1.43 & 671.10 & 
-1.24\\SCA3-6 & 652.47 & 1.40 & 
653.66 & 1.41 & \bf{651.10} & 
0.21\\SCA3-7 & 666.15 & 1.22 & 
666.15 & 1.22 & \bf{666.10} & 
0.01\\SCA3-8 & \bf{\underline{719.47}} & 1.33 & 
723.95 & 1.38 & 719.50 & 
-0.00\\SCA3-9 & \bf{681.00} & 1.07 & 
681.81 & 1.15 & 681.00 & 0.00\\
SCA8-0 & 971.49 & 1.59 & 
982.37 & 1.54 & \bf{961.60} & 
1.03\\SCA8-1 & \bf{\underline{1058.59}} & 1.32 & 
1065.98 & 1.36 & 1063.00 & 
-0.41\\SCA8-2 & 1050.17 & 1.22 & 
1051.06 & 1.27 & \bf{1040.60} & 
0.92\\SCA8-3 & 1011.61 & 1.50 & 
1013.34 & 1.52 & \bf{985.90} & 
2.61\\SCA8-4 & \bf{\underline{1065.49}} & 1.55 & 
1066.55 & 1.64 & 1071.00 & 
-0.51\\SCA8-5 & \bf{\underline{1050.09}} & 1.78 & 
1052.63 & 1.71 & 1054.30 & 
-0.40\\SCA8-6 & 980.91 & 1.52 & 
982.27 & 1.61 & \bf{972.50} & 
0.86\\SCA8-7 & 1067.20 & 1.62 & 
1075.32 & 1.66 & \bf{1059.70} & 
0.71\\SCA8-8 & \bf{\underline{1082.11}} & 1.56 & 
1083.03 & 1.60 & 1082.70 & 
-0.05\\SCA8-9 & \bf{\underline{1067.27}} & 1.31 & 
1067.38 & 1.36 & 1081.40 & 
-1.31\\CON3-0 & 617.59 & 1.66 & 
621.82 & 1.63 & \bf{616.50} & 
0.18\\CON3-1 & \bf{\underline{554.47}} & 1.50 & 
557.88 & 1.55 & 555.60 & 
-0.20\\CON3-2 & \bf{\underline{521.38}} & 1.57 & 
522.19 & 1.50 & 521.40 & 
-0.00\\CON3-3 & \bf{\underline{591.19}} & 1.51 & 
591.98 & 1.57 & 591.20 & 
-0.00\\CON3-4 & \bf{\underline{588.79}} & 1.49 & 
590.11 & 1.46 & 589.30 & 
-0.09\\CON3-5 & 564.88 & 1.42 & 
567.42 & 1.47 & \bf{563.70} & 
0.21\\CON3-6 & 502.09 & 1.62 & 
502.74 & 1.63 & \bf{499.20} & 
0.58\\CON3-7 & \bf{\underline{576.84}} & 1.34 & 
579.49 & 1.35 & 577.50 & 
-0.11\\CON3-8 & 524.30 & 1.36 & 
525.25 & 1.34 & \bf{523.10} & 
0.23\\CON3-9 & 588.40 & 1.48 & 
588.67 & 1.40 & \bf{578.20} & 
1.76\\CON8-0 & 866.22 & 1.50 & 
872.15 & 1.51 & \bf{858.90} & 
0.85\\CON8-1 & 742.29 & 1.46 & 
746.51 & 1.58 & \bf{740.90} & 
0.19\\CON8-2 & 714.94 & 1.94 & 
715.38 & 1.86 & \bf{714.30} & 
0.09\\CON8-3 & 815.14 & 1.56 & 
819.18 & 1.53 & \bf{812.30} & 
0.35\\CON8-4 & 789.56 & 1.57 & 
792.74 & 1.50 & \bf{770.10} & 
2.53\\CON8-5 & \bf{\underline{764.09}} & 1.40 & 
766.00 & 1.49 & 766.60 & 
-0.33\\CON8-6 & \bf{\underline{689.69}} & 1.70 & 
693.41 & 1.72 & 697.20 & 
-1.08\\CON8-7 & 814.86 & 1.50 & 
821.12 & 1.40 & \bf{814.80} & 
0.01\\CON8-8 & 784.56 & 1.65 & 
792.03 & 1.60 & \bf{771.30} & 
1.72\\CON8-9 & \bf{\underline{813.16}} & 1.72 & 
816.34 & 1.64 & 815.10 & 
-0.24\\[1ex]\hline
\end{tabular}
\label{table:nonlin}
\end{table} \clearpage
\begin{table}[ht]
\caption{Resultados de la ejecución de la metaheurística ACO, utilizando instancias de Dethloff con la configuración -n 2.0 -alpha 1.0 -beta 3.0 -q .5 -ro 0.015}
\centering
\small
\begin{tabular}{c c c c c c c}
\hline\hline
Instancia & Costo mínimo & Tiempo(seg.) & Costo promedio & Tiempo promedio(seg.) & Costo ACO & \%Gap \\ [0.5ex]
\hline
SCA3-0 & \bf{\underline{636.06}} & 1.28 & 
637.25 & 1.36 & 636.10 & 
-0.01\\SCA3-1 & \bf{\underline{697.84}} & 1.58 & 
697.84 & 1.52 & 700.10 & 
-0.32\\SCA3-2 & 659.34 & 1.32 & 
661.13 & 1.39 & \bf{659.30} & 
0.01\\SCA3-3 & 680.04 & 1.36 & 
680.82 & 1.39 & \bf{680.00} & 
0.01\\SCA3-4 & \bf{690.50} & 1.36 & 
690.50 & 1.42 & 690.50 & 0.00\\
SCA3-5 & \bf{\underline{665.04}} & 1.58 & 
665.90 & 1.42 & 671.10 & 
-0.90\\SCA3-6 & 652.94 & 1.34 & 
654.33 & 1.39 & \bf{651.10} & 
0.28\\SCA3-7 & 666.15 & 1.07 & 
666.42 & 1.14 & \bf{666.10} & 
0.01\\SCA3-8 & \bf{\underline{719.47}} & 1.42 & 
724.88 & 1.35 & 719.50 & 
-0.00\\SCA3-9 & \bf{681.00} & 1.26 & 
681.00 & 1.16 & 681.00 & 0.00\\
SCA8-0 & \bf{\underline{961.50}} & 1.46 & 
983.36 & 1.51 & 961.60 & 
-0.01\\SCA8-1 & \bf{\underline{1054.87}} & 1.44 & 
1062.84 & 1.30 & 1063.00 & 
-0.76\\SCA8-2 & 1043.79 & 1.20 & 
1050.01 & 1.18 & \bf{1040.60} & 
0.31\\SCA8-3 & 996.02 & 1.43 & 
1012.52 & 1.49 & \bf{985.90} & 
1.03\\SCA8-4 & \bf{\underline{1065.49}} & 1.46 & 
1068.36 & 1.54 & 1071.00 & 
-0.51\\SCA8-5 & \bf{\underline{1045.30}} & 1.70 & 
1055.22 & 1.74 & 1054.30 & 
-0.85\\SCA8-6 & \bf{\underline{972.48}} & 1.58 & 
983.38 & 1.58 & 972.50 & 
-0.00\\SCA8-7 & 1067.20 & 1.64 & 
1070.92 & 1.59 & \bf{1059.70} & 
0.71\\SCA8-8 & \bf{\underline{1071.18}} & 1.70 & 
1077.42 & 1.66 & 1082.70 & 
-1.06\\SCA8-9 & \bf{\underline{1067.42}} & 1.35 & 
1067.42 & 1.34 & 1081.40 & 
-1.29\\CON3-0 & 624.96 & 1.60 & 
625.84 & 1.57 & \bf{616.50} & 
1.37\\CON3-1 & \bf{\underline{554.47}} & 1.53 & 
558.61 & 1.50 & 555.60 & 
-0.20\\CON3-2 & \bf{\underline{521.38}} & 1.31 & 
523.91 & 1.35 & 521.40 & 
-0.00\\CON3-3 & \bf{\underline{591.19}} & 1.48 & 
591.20 & 1.55 & 591.20 & 
-0.00\\CON3-4 & \bf{\underline{588.79}} & 1.38 & 
589.05 & 1.33 & 589.30 & 
-0.09\\CON3-5 & 564.88 & 1.50 & 
567.26 & 1.50 & \bf{563.70} & 
0.21\\CON3-6 & 502.34 & 1.55 & 
504.20 & 1.68 & \bf{499.20} & 
0.63\\CON3-7 & 578.22 & 1.26 & 
583.01 & 1.30 & \bf{577.50} & 
0.12\\CON3-8 & \bf{\underline{523.05}} & 1.42 & 
524.53 & 1.37 & 523.10 & 
-0.01\\CON3-9 & 578.25 & 1.35 & 
584.59 & 1.37 & \bf{578.20} & 
0.01\\CON8-0 & 877.31 & 1.58 & 
885.88 & 1.56 & \bf{858.90} & 
2.14\\CON8-1 & \bf{\underline{740.85}} & 1.52 & 
746.97 & 1.53 & 740.90 & 
-0.01\\CON8-2 & \bf{\underline{713.68}} & 1.73 & 
714.89 & 1.74 & 714.30 & 
-0.09\\CON8-3 & \bf{\underline{811.07}} & 1.42 & 
816.59 & 1.51 & 812.30 & 
-0.15\\CON8-4 & 787.67 & 1.42 & 
789.83 & 1.42 & \bf{770.10} & 
2.28\\CON8-5 & \bf{\underline{759.93}} & 1.52 & 
762.43 & 1.51 & 766.60 & 
-0.87\\CON8-6 & \bf{\underline{690.00}} & 1.74 & 
697.15 & 1.77 & 697.20 & 
-1.03\\CON8-7 & 815.06 & 1.41 & 
819.71 & 1.31 & \bf{814.80} & 
0.03\\CON8-8 & 788.25 & 1.77 & 
793.19 & 1.70 & \bf{771.30} & 
2.20\\CON8-9 & \bf{\underline{812.89}} & 1.62 & 
815.86 & 1.62 & 815.10 & 
-0.27\\[1ex]\hline
\end{tabular}
\label{table:nonlin}
\end{table} \clearpage
\begin{table}[ht]
\caption{Resultados de la ejecución de la metaheurística ACO, utilizando instancias de Dethloff con la configuración -n 2.0 -alpha 1.0 -beta 3.0 -q .6 -ro 0.015}
\centering
\small
\begin{tabular}{c c c c c c c}
\hline\hline
Instancia & Costo mínimo & Tiempo(seg.) & Costo promedio & Tiempo promedio(seg.) & Costo ACO & \%Gap \\ [0.5ex]
\hline
SCA3-0 & \bf{\underline{636.06}} & 1.50 & 
636.13 & 1.37 & 636.10 & 
-0.01\\SCA3-1 & \bf{\underline{697.84}} & 1.56 & 
697.84 & 1.48 & 700.10 & 
-0.32\\SCA3-2 & 659.34 & 1.38 & 
661.01 & 1.33 & \bf{659.30} & 
0.01\\SCA3-3 & 680.04 & 1.39 & 
680.32 & 1.33 & \bf{680.00} & 
0.01\\SCA3-4 & \bf{690.50} & 1.51 & 
690.50 & 1.46 & 690.50 & 0.00\\
SCA3-5 & \bf{\underline{662.75}} & 1.37 & 
666.63 & 1.44 & 671.10 & 
-1.24\\SCA3-6 & 652.94 & 1.36 & 
653.95 & 1.36 & \bf{651.10} & 
0.28\\SCA3-7 & \bf{\underline{659.17}} & 1.14 & 
665.34 & 1.17 & 666.10 & 
-1.04\\SCA3-8 & \bf{\underline{719.47}} & 1.31 & 
722.87 & 1.42 & 719.50 & 
-0.00\\SCA3-9 & \bf{681.00} & 1.10 & 
682.00 & 1.15 & 681.00 & 0.00\\
SCA8-0 & 977.94 & 1.86 & 
986.85 & 1.65 & \bf{961.60} & 
1.70\\SCA8-1 & 1063.98 & 1.16 & 
1073.28 & 1.42 & \bf{1063.00} & 
0.09\\SCA8-2 & 1050.37 & 1.20 & 
1053.40 & 1.15 & \bf{1040.60} & 
0.94\\SCA8-3 & 1014.62 & 1.58 & 
1017.91 & 1.50 & \bf{985.90} & 
2.91\\SCA8-4 & 1072.75 & 1.63 & 
1081.36 & 1.50 & \bf{1071.00} & 
0.16\\SCA8-5 & \bf{\underline{1039.12}} & 1.72 & 
1048.97 & 1.73 & 1054.30 & 
-1.44\\SCA8-6 & 977.03 & 1.64 & 
980.68 & 1.60 & \bf{972.50} & 
0.47\\SCA8-7 & 1075.87 & 1.68 & 
1082.32 & 1.65 & \bf{1059.70} & 
1.53\\SCA8-8 & \bf{\underline{1071.18}} & 1.60 & 
1077.60 & 1.60 & 1082.70 & 
-1.06\\SCA8-9 & \bf{\underline{1067.42}} & 1.31 & 
1067.42 & 1.29 & 1081.40 & 
-1.29\\CON3-0 & 617.59 & 1.56 & 
625.38 & 1.66 & \bf{616.50} & 
0.18\\CON3-1 & 557.38 & 1.56 & 
559.28 & 1.53 & \bf{555.60} & 
0.32\\CON3-2 & 521.63 & 1.31 & 
523.12 & 1.43 & \bf{521.40} & 
0.04\\CON3-3 & \bf{\underline{591.19}} & 1.58 & 
591.92 & 1.59 & 591.20 & 
-0.00\\CON3-4 & \bf{\underline{588.79}} & 1.39 & 
589.58 & 1.37 & 589.30 & 
-0.09\\CON3-5 & 564.88 & 1.28 & 
567.77 & 1.39 & \bf{563.70} & 
0.21\\CON3-6 & 500.80 & 1.69 & 
503.09 & 1.63 & \bf{499.20} & 
0.32\\CON3-7 & 578.22 & 1.28 & 
580.76 & 1.32 & \bf{577.50} & 
0.12\\CON3-8 & \bf{\underline{523.05}} & 1.39 & 
524.94 & 1.35 & 523.10 & 
-0.01\\CON3-9 & 588.99 & 1.34 & 
590.01 & 1.32 & \bf{578.20} & 
1.87\\CON8-0 & 872.05 & 1.46 & 
883.45 & 1.47 & \bf{858.90} & 
1.53\\CON8-1 & \bf{\underline{740.85}} & 1.54 & 
745.25 & 1.46 & 740.90 & 
-0.01\\CON8-2 & \bf{\underline{713.44}} & 1.92 & 
715.45 & 1.92 & 714.30 & 
-0.12\\CON8-3 & 812.75 & 1.52 & 
817.71 & 1.53 & \bf{812.30} & 
0.06\\CON8-4 & 785.76 & 1.44 & 
789.59 & 1.48 & \bf{770.10} & 
2.03\\CON8-5 & \bf{\underline{760.03}} & 1.44 & 
766.50 & 1.44 & 766.60 & 
-0.86\\CON8-6 & \bf{\underline{686.39}} & 1.62 & 
693.70 & 1.69 & 697.20 & 
-1.55\\CON8-7 & 821.11 & 1.26 & 
823.33 & 1.29 & \bf{814.80} & 
0.77\\CON8-8 & 788.09 & 1.73 & 
791.64 & 1.65 & \bf{771.30} & 
2.18\\CON8-9 & 815.49 & 1.71 & 
818.30 & 1.65 & \bf{815.10} & 
0.05\\[1ex]\hline
\end{tabular}
\label{table:nonlin}
\end{table} \clearpage
\begin{table}[ht]
\caption{Resultados de la ejecución de la metaheurística ACO, utilizando instancias de Dethloff con la configuración -n 2.0 -alpha 1.0 -beta 3.0 -q .7 -ro 0.015}
\centering
\small
\begin{tabular}{c c c c c c c}
\hline\hline
Instancia & Costo mínimo & Tiempo(seg.) & Costo promedio & Tiempo promedio(seg.) & Costo ACO & \%Gap \\ [0.5ex]
\hline
SCA3-0 & \bf{\underline{636.06}} & 1.35 & 
636.06 & 1.36 & 636.10 & 
-0.01\\SCA3-1 & \bf{\underline{697.84}} & 1.60 & 
697.84 & 1.51 & 700.10 & 
-0.32\\SCA3-2 & 659.34 & 1.31 & 
661.77 & 1.35 & \bf{659.30} & 
0.01\\SCA3-3 & 680.04 & 1.29 & 
680.46 & 1.34 & \bf{680.00} & 
0.01\\SCA3-4 & \bf{690.50} & 1.43 & 
690.50 & 1.45 & 690.50 & 0.00\\
SCA3-5 & \bf{\underline{662.75}} & 1.54 & 
668.11 & 1.44 & 671.10 & 
-1.24\\SCA3-6 & \bf{\underline{651.09}} & 1.34 & 
652.48 & 1.38 & 651.10 & 
-0.00\\SCA3-7 & 666.15 & 1.14 & 
666.15 & 1.10 & \bf{666.10} & 
0.01\\SCA3-8 & \bf{\underline{719.47}} & 1.35 & 
722.94 & 1.34 & 719.50 & 
-0.00\\SCA3-9 & \bf{681.00} & 1.10 & 
683.08 & 1.10 & 681.00 & 0.00\\
SCA8-0 & 970.64 & 1.48 & 
984.03 & 1.51 & \bf{961.60} & 
0.94\\SCA8-1 & \bf{\underline{1061.67}} & 1.20 & 
1071.19 & 1.24 & 1063.00 & 
-0.13\\SCA8-2 & 1044.24 & 1.18 & 
1049.77 & 1.15 & \bf{1040.60} & 
0.35\\SCA8-3 & 1012.31 & 1.36 & 
1017.73 & 1.45 & \bf{985.90} & 
2.68\\SCA8-4 & \bf{\underline{1065.49}} & 1.50 & 
1073.82 & 1.49 & 1071.00 & 
-0.51\\SCA8-5 & \bf{\underline{1034.74}} & 1.56 & 
1050.40 & 1.58 & 1054.30 & 
-1.86\\SCA8-6 & 976.74 & 1.66 & 
981.67 & 1.64 & \bf{972.50} & 
0.44\\SCA8-7 & 1067.20 & 1.70 & 
1074.39 & 1.64 & \bf{1059.70} & 
0.71\\SCA8-8 & \bf{\underline{1071.18}} & 1.59 & 
1073.91 & 1.54 & 1082.70 & 
-1.06\\SCA8-9 & \bf{\underline{1067.42}} & 1.26 & 
1068.72 & 1.21 & 1081.40 & 
-1.29\\CON3-0 & 617.59 & 1.58 & 
622.94 & 1.59 & \bf{616.50} & 
0.18\\CON3-1 & \bf{\underline{554.47}} & 1.46 & 
556.49 & 1.47 & 555.60 & 
-0.20\\CON3-2 & \bf{\underline{521.38}} & 1.37 & 
523.53 & 1.26 & 521.40 & 
-0.00\\CON3-3 & \bf{\underline{591.19}} & 1.62 & 
593.00 & 1.57 & 591.20 & 
-0.00\\CON3-4 & \bf{\underline{588.79}} & 1.41 & 
591.73 & 1.36 & 589.30 & 
-0.09\\CON3-5 & 564.88 & 1.56 & 
568.09 & 1.46 & \bf{563.70} & 
0.21\\CON3-6 & 502.95 & 1.76 & 
504.11 & 1.67 & \bf{499.20} & 
0.75\\CON3-7 & 578.22 & 1.26 & 
581.21 & 1.22 & \bf{577.50} & 
0.12\\CON3-8 & 523.14 & 1.35 & 
525.49 & 1.30 & \bf{523.10} & 
0.01\\CON3-9 & 585.05 & 1.36 & 
589.29 & 1.40 & \bf{578.20} & 
1.18\\CON8-0 & 876.30 & 1.42 & 
880.56 & 1.42 & \bf{858.90} & 
2.03\\CON8-1 & \bf{\underline{740.85}} & 1.47 & 
743.43 & 1.43 & 740.90 & 
-0.01\\CON8-2 & 716.16 & 1.85 & 
718.28 & 1.86 & \bf{714.30} & 
0.26\\CON8-3 & 812.54 & 1.51 & 
816.31 & 1.49 & \bf{812.30} & 
0.03\\CON8-4 & 791.18 & 1.46 & 
795.39 & 1.48 & \bf{770.10} & 
2.74\\CON8-5 & \bf{\underline{761.62}} & 1.38 & 
763.83 & 1.39 & 766.60 & 
-0.65\\CON8-6 & \bf{\underline{693.10}} & 1.58 & 
699.43 & 1.60 & 697.20 & 
-0.59\\CON8-7 & 815.54 & 1.32 & 
819.62 & 1.26 & \bf{814.80} & 
0.09\\CON8-8 & 782.86 & 1.50 & 
789.71 & 1.66 & \bf{771.30} & 
1.50\\CON8-9 & 815.44 & 1.64 & 
816.21 & 1.65 & \bf{815.10} & 
0.04\\[1ex]\hline
\end{tabular}
\label{table:nonlin}
\end{table} \clearpage
\begin{table}[ht]
\caption{Resultados de la ejecución de la metaheurística ACO, utilizando instancias de Dethloff con la configuración -n 2.0 -alpha 1.0 -beta 3.0 -q .8 -ro 0.015}
\centering
\small
\begin{tabular}{c c c c c c c}
\hline\hline
Instancia & Costo mínimo & Tiempo(seg.) & Costo promedio & Tiempo promedio(seg.) & Costo ACO & \%Gap \\ [0.5ex]
\hline
SCA3-0 & \bf{\underline{636.06}} & 1.38 & 
637.25 & 1.34 & 636.10 & 
-0.01\\SCA3-1 & \bf{\underline{697.84}} & 1.58 & 
698.76 & 1.54 & 700.10 & 
-0.32\\SCA3-2 & 659.34 & 1.38 & 
662.22 & 1.31 & \bf{659.30} & 
0.01\\SCA3-3 & 680.04 & 1.42 & 
680.46 & 1.44 & \bf{680.00} & 
0.01\\SCA3-4 & \bf{690.50} & 1.42 & 
690.50 & 1.44 & 690.50 & 0.00\\
SCA3-5 & \bf{\underline{659.90}} & 1.39 & 
667.61 & 1.45 & 671.10 & 
-1.67\\SCA3-6 & 652.94 & 1.41 & 
654.73 & 1.39 & \bf{651.10} & 
0.28\\SCA3-7 & 666.15 & 1.12 & 
667.09 & 1.08 & \bf{666.10} & 
0.01\\SCA3-8 & \bf{\underline{719.47}} & 1.23 & 
723.67 & 1.26 & 719.50 & 
-0.00\\SCA3-9 & \bf{681.00} & 1.04 & 
682.00 & 1.09 & 681.00 & 0.00\\
SCA8-0 & 977.26 & 1.59 & 
985.14 & 1.51 & \bf{961.60} & 
1.63\\SCA8-1 & \bf{\underline{1059.74}} & 1.28 & 
1066.78 & 1.25 & 1063.00 & 
-0.31\\SCA8-2 & 1048.75 & 0.98 & 
1051.01 & 1.04 & \bf{1040.60} & 
0.78\\SCA8-3 & 1017.58 & 1.45 & 
1020.74 & 1.47 & \bf{985.90} & 
3.21\\SCA8-4 & \bf{\underline{1067.66}} & 1.38 & 
1074.32 & 1.35 & 1071.00 & 
-0.31\\SCA8-5 & \bf{\underline{1036.88}} & 1.57 & 
1049.48 & 1.63 & 1054.30 & 
-1.65\\SCA8-6 & 981.41 & 1.59 & 
982.19 & 1.63 & \bf{972.50} & 
0.92\\SCA8-7 & 1067.20 & 1.90 & 
1070.68 & 1.69 & \bf{1059.70} & 
0.71\\SCA8-8 & \bf{\underline{1071.18}} & 1.54 & 
1073.97 & 1.49 & 1082.70 & 
-1.06\\SCA8-9 & \bf{\underline{1067.42}} & 1.32 & 
1067.42 & 1.29 & 1081.40 & 
-1.29\\CON3-0 & 624.96 & 1.56 & 
625.84 & 1.51 & \bf{616.50} & 
1.37\\CON3-1 & 556.04 & 1.49 & 
558.97 & 1.45 & \bf{555.60} & 
0.08\\CON3-2 & 521.63 & 1.18 & 
525.08 & 1.26 & \bf{521.40} & 
0.04\\CON3-3 & \bf{\underline{591.19}} & 1.49 & 
591.43 & 1.52 & 591.20 & 
-0.00\\CON3-4 & \bf{\underline{588.79}} & 1.23 & 
589.05 & 1.41 & 589.30 & 
-0.09\\CON3-5 & 564.89 & 1.44 & 
567.48 & 1.50 & \bf{563.70} & 
0.21\\CON3-6 & 502.16 & 1.67 & 
503.12 & 1.71 & \bf{499.20} & 
0.59\\CON3-7 & 578.41 & 1.28 & 
581.01 & 1.25 & \bf{577.50} & 
0.16\\CON3-8 & 523.14 & 1.29 & 
526.15 & 1.23 & \bf{523.10} & 
0.01\\CON3-9 & 586.17 & 1.24 & 
589.13 & 1.30 & \bf{578.20} & 
1.38\\CON8-0 & 870.22 & 1.38 & 
873.66 & 1.45 & \bf{858.90} & 
1.32\\CON8-1 & 740.93 & 1.31 & 
745.43 & 1.41 & \bf{740.90} & 
0.00\\CON8-2 & \bf{\underline{713.44}} & 1.90 & 
715.23 & 1.86 & 714.30 & 
-0.12\\CON8-3 & 817.57 & 1.44 & 
817.57 & 1.46 & \bf{812.30} & 
0.65\\CON8-4 & 776.72 & 1.48 & 
787.74 & 1.55 & \bf{770.10} & 
0.86\\CON8-5 & \bf{\underline{762.36}} & 1.35 & 
764.13 & 1.32 & 766.60 & 
-0.55\\CON8-6 & \bf{\underline{695.01}} & 1.72 & 
699.89 & 1.66 & 697.20 & 
-0.31\\CON8-7 & 815.09 & 1.20 & 
821.35 & 1.21 & \bf{814.80} & 
0.04\\CON8-8 & 788.09 & 1.56 & 
793.01 & 1.60 & \bf{771.30} & 
2.18\\CON8-9 & \bf{\underline{810.18}} & 1.47 & 
814.50 & 1.59 & 815.10 & 
-0.60\\[1ex]\hline
\end{tabular}
\label{table:nonlin}
\end{table} \clearpage
\begin{table}[ht]
\caption{Resultados de la ejecución de la metaheurística ACO, utilizando instancias de Dethloff con la configuración -n 2.0 -alpha 1.0 -beta 3.0 -q .9 -ro 0.015}
\centering
\small
\begin{tabular}{c c c c c c c}
\hline\hline
Instancia & Costo mínimo & Tiempo(seg.) & Costo promedio & Tiempo promedio(seg.) & Costo ACO & \%Gap \\ [0.5ex]
\hline
SCA3-0 & 640.55 & 1.33 & 
640.84 & 1.31 & \bf{636.10} & 
0.70\\SCA3-1 & \bf{\underline{697.84}} & 1.59 & 
698.76 & 1.50 & 700.10 & 
-0.32\\SCA3-2 & 664.18 & 1.33 & 
666.64 & 1.29 & \bf{659.30} & 
0.74\\SCA3-3 & 680.60 & 1.29 & 
680.60 & 1.39 & \bf{680.00} & 
0.09\\SCA3-4 & \bf{690.50} & 1.65 & 
690.50 & 1.48 & 690.50 & 0.00\\
SCA3-5 & \bf{\underline{662.75}} & 1.46 & 
666.05 & 1.37 & 671.10 & 
-1.24\\SCA3-6 & 652.94 & 1.31 & 
655.67 & 1.35 & \bf{651.10} & 
0.28\\SCA3-7 & 666.15 & 1.08 & 
666.15 & 1.09 & \bf{666.10} & 
0.01\\SCA3-8 & \bf{\underline{719.47}} & 1.27 & 
725.05 & 1.20 & 719.50 & 
-0.00\\SCA3-9 & \bf{681.00} & 1.12 & 
681.00 & 1.05 & 681.00 & 0.00\\
SCA8-0 & 973.03 & 1.56 & 
987.32 & 1.48 & \bf{961.60} & 
1.19\\SCA8-1 & 1069.40 & 1.18 & 
1072.90 & 1.19 & \bf{1063.00} & 
0.60\\SCA8-2 & 1050.37 & 0.98 & 
1052.93 & 1.06 & \bf{1040.60} & 
0.94\\SCA8-3 & 1010.50 & 1.50 & 
1017.62 & 1.48 & \bf{985.90} & 
2.50\\SCA8-4 & \bf{\underline{1065.49}} & 1.40 & 
1088.12 & 1.46 & 1071.00 & 
-0.51\\SCA8-5 & 1055.35 & 1.60 & 
1056.36 & 1.59 & \bf{1054.30} & 
0.10\\SCA8-6 & 981.37 & 1.58 & 
981.69 & 1.61 & \bf{972.50} & 
0.91\\SCA8-7 & 1067.20 & 1.66 & 
1073.74 & 1.60 & \bf{1059.70} & 
0.71\\SCA8-8 & \bf{\underline{1071.18}} & 1.45 & 
1082.11 & 1.48 & 1082.70 & 
-1.06\\SCA8-9 & \bf{\underline{1067.42}} & 1.15 & 
1067.42 & 1.17 & 1081.40 & 
-1.29\\CON3-0 & 624.90 & 1.57 & 
624.95 & 1.59 & \bf{616.50} & 
1.36\\CON3-1 & 557.38 & 1.49 & 
560.35 & 1.46 & \bf{555.60} & 
0.32\\CON3-2 & 523.99 & 1.16 & 
527.81 & 1.13 & \bf{521.40} & 
0.50\\CON3-3 & \bf{\underline{591.19}} & 1.46 & 
591.20 & 1.46 & 591.20 & 
-0.00\\CON3-4 & 589.32 & 1.32 & 
589.85 & 1.31 & \bf{589.30} & 
0.00\\CON3-5 & 564.88 & 1.53 & 
569.22 & 1.44 & \bf{563.70} & 
0.21\\CON3-6 & 500.80 & 1.76 & 
504.38 & 1.84 & \bf{499.20} & 
0.32\\CON3-7 & 578.41 & 1.13 & 
582.26 & 1.16 & \bf{577.50} & 
0.16\\CON3-8 & 524.59 & 1.50 & 
528.55 & 1.28 & \bf{523.10} & 
0.28\\CON3-9 & 578.98 & 1.33 & 
586.07 & 1.25 & \bf{578.20} & 
0.13\\CON8-0 & 868.49 & 1.47 & 
875.81 & 1.47 & \bf{858.90} & 
1.12\\CON8-1 & 742.29 & 1.38 & 
747.02 & 1.43 & \bf{740.90} & 
0.19\\CON8-2 & \bf{\underline{713.90}} & 1.95 & 
717.60 & 1.90 & 714.30 & 
-0.06\\CON8-3 & \bf{\underline{811.07}} & 1.37 & 
815.95 & 1.45 & 812.30 & 
-0.15\\CON8-4 & 781.64 & 1.43 & 
787.67 & 1.44 & \bf{770.10} & 
1.50\\CON8-5 & \bf{\underline{759.93}} & 1.42 & 
762.84 & 1.37 & 766.60 & 
-0.87\\CON8-6 & \bf{\underline{690.19}} & 1.60 & 
695.56 & 1.61 & 697.20 & 
-1.01\\CON8-7 & \bf{\underline{814.79}} & 1.20 & 
825.09 & 1.19 & 814.80 & 
-0.00\\CON8-8 & 790.80 & 1.53 & 
794.87 & 1.60 & \bf{771.30} & 
2.53\\CON8-9 & \bf{\underline{813.95}} & 1.57 & 
816.50 & 1.56 & 815.10 & 
-0.14\\[1ex]\hline
\end{tabular}
\label{table:nonlin}
\end{table} \clearpage
\begin{table}[ht]
\caption{Resultados de la ejecución de la metaheurística ACO, utilizando instancias de SalhiNagy con la configuración -n 2.0 -alpha 1.0 -beta 3.0 -q 0.1 -ro 0.015}
\centering
\small
\begin{tabular}{c c c c c c c}
\hline\hline
Instancia & Costo mínimo & Tiempo(seg.) & Costo promedio & Tiempo promedio(seg.) & Costo ACO & \%Gap \\ [0.5ex]
\hline
CMT1X & 475.20 & 1.44 & 
479.20 & 1.36 & \bf{470.67} & 
0.96\\CMT1Y & 472.87 & 1.32 & 
477.83 & 1.34 & \bf{472.37} & 
0.11\\CMT2X & 706.30 & 6.40 & 
708.22 & 6.36 & \bf{705.24} & 
0.15\\CMT2Y & 708.19 & 6.34 & 
711.41 & 6.46 & \bf{704.16} & 
0.57\\CMT3X & 729.61 & 19.50 & 
732.90 & 20.13 & \bf{726.55} & 
0.42\\CMT3Y & \bf{\underline{728.15}} & 20.49 & 
734.90 & 20.68 & 729.02 & 
-0.12\\CMT4X & \bf{\underline{893.62}} & 90.78 & 
896.88 & 88.11 & 893.90 & 
-0.03\\CMT4Y & \bf{\underline{885.27}} & 93.29 & 
890.79 & 91.76 & 895.25 & 
-1.11\\CMT5X & \bf{\underline{1077.10}} & 273.96 & 
1078.69 & 278.39 & 1115.75 & 
-3.46\\CMT5Y & 100000 & 0 & 
nan & nan & \bf{1112.61} & 
8887.88\\CMT11X & \bf{\underline{856.51855.69}} & 39.92 & 
869.69 & 38.73 & 887.36 & 
-3.48\\CMT11Y & \bf{\underline{855.60}} & 33.45 & 
886.99 & 34.41 & 874.13 & 
-2.12\\CMT12X & \bf{\underline{675.10}} & 15.51 & 
676.80 & 16.16 & 681.02 & 
-0.87\\CMT12Y & \bf{\underline{666.83}} & 16.57 & 
675.95 & 16.59 & 671.32 & 
-0.67\\[1ex]\hline
\end{tabular}
\label{table:nonlin}
\end{table} \clearpage
\begin{table}[ht]
\caption{Resultados de la ejecución de la metaheurística ACO, utilizando instancias de Dethloff con la configuración -n 2.0 -alpha 1.0 -beta 3.0 -q 1.0 -ro 0.015}
\centering
\small
\begin{tabular}{c c c c c c c}
\hline\hline
Instancia & Costo mínimo & Tiempo(seg.) & Costo promedio & Tiempo promedio(seg.) & Costo ACO & \%Gap \\ [0.5ex]
\hline
SCA3-0 & 640.55 & 1.29 & 
640.55 & 1.33 & \bf{636.10} & 
0.70\\SCA3-1 & \bf{\underline{697.84}} & 1.50 & 
697.84 & 1.51 & 700.10 & 
-0.32\\SCA3-2 & 659.34 & 1.39 & 
661.76 & 1.34 & \bf{659.30} & 
0.01\\SCA3-3 & 681.31 & 1.57 & 
681.31 & 1.46 & \bf{680.00} & 
0.19\\SCA3-4 & \bf{690.50} & 1.41 & 
690.50 & 1.36 & 690.50 & 0.00\\
SCA3-5 & \bf{\underline{665.04}} & 1.34 & 
665.60 & 1.40 & 671.10 & 
-0.90\\SCA3-6 & 655.19 & 1.42 & 
655.19 & 1.35 & \bf{651.10} & 
0.63\\SCA3-7 & 666.15 & 0.94 & 
666.15 & 0.97 & \bf{666.10} & 
0.01\\SCA3-8 & 721.45 & 1.06 & 
723.34 & 1.08 & \bf{719.50} & 
0.27\\SCA3-9 & \bf{681.00} & 1.05 & 
681.00 & 1.06 & 681.00 & 0.00\\
SCA8-0 & 991.07 & 1.53 & 
991.07 & 1.47 & \bf{961.60} & 
3.06\\SCA8-1 & 1069.40 & 1.23 & 
1073.34 & 1.19 & \bf{1063.00} & 
0.60\\SCA8-2 & 1056.87 & 1.01 & 
1056.87 & 1.00 & \bf{1040.60} & 
1.56\\SCA8-3 & 1031.08 & 1.48 & 
1031.08 & 1.41 & \bf{985.90} & 
4.58\\SCA8-4 & 1099.06 & 1.48 & 
1099.06 & 1.48 & \bf{1071.00} & 
2.62\\SCA8-5 & 1055.35 & 1.64 & 
1055.35 & 1.24 & \bf{1054.30} & 
0.10\\SCA8-6 & \bf{\underline{972.48}} & 1.67 & 
972.48 & 1.64 & 972.50 & 
-0.00\\SCA8-7 & 1092.57 & 1.62 & 
1092.57 & 1.63 & \bf{1059.70} & 
3.10\\SCA8-8 & 1091.49 & 1.45 & 
1091.75 & 1.45 & \bf{1082.70} & 
0.81\\SCA8-9 & \bf{\underline{1067.42}} & 1.11 & 
1067.42 & 1.12 & 1081.40 & 
-1.29\\CON3-0 & 624.96 & 1.64 & 
624.96 & 1.60 & \bf{616.50} & 
1.37\\CON3-1 & 557.38 & 1.56 & 
558.01 & 1.49 & \bf{555.60} & 
0.32\\CON3-2 & 524.07 & 1.13 & 
524.62 & 1.16 & \bf{521.40} & 
0.51\\CON3-3 & 592.95 & 1.56 & 
593.53 & 1.54 & \bf{591.20} & 
0.30\\CON3-4 & \bf{\underline{588.79}} & 1.38 & 
589.19 & 1.38 & 589.30 & 
-0.09\\CON3-5 & 569.88 & 1.44 & 
574.79 & 1.39 & \bf{563.70} & 
1.10\\CON3-6 & 504.15 & 1.89 & 
504.98 & 1.79 & \bf{499.20} & 
0.99\\CON3-7 & 578.41 & 1.28 & 
579.12 & 1.24 & \bf{577.50} & 
0.16\\CON3-8 & 524.30 & 1.14 & 
524.52 & 1.17 & \bf{523.10} & 
0.23\\CON3-9 & 588.48 & 1.29 & 
588.48 & 1.25 & \bf{578.20} & 
1.78\\CON8-0 & 879.00 & 1.38 & 
879.00 & 1.44 & \bf{858.90} & 
2.34\\CON8-1 & 758.26 & 1.34 & 
758.26 & 1.38 & \bf{740.90} & 
2.34\\CON8-2 & 716.53 & 1.97 & 
716.54 & 2.00 & \bf{714.30} & 
0.31\\CON8-3 & 817.57 & 1.47 & 
817.57 & 1.47 & \bf{812.30} & 
0.65\\CON8-4 & 781.64 & 1.64 & 
783.73 & 1.58 & \bf{770.10} & 
1.50\\CON8-5 & \bf{\underline{764.36}} & 1.64 & 
764.36 & 1.44 & 766.60 & 
-0.29\\CON8-6 & 705.61 & 1.82 & 
706.63 & 1.72 & \bf{697.20} & 
1.21\\CON8-7 & 822.42 & 1.21 & 
822.67 & 1.21 & \bf{814.80} & 
0.94\\CON8-8 & 799.32 & 1.54 & 
799.46 & 1.69 & \bf{771.30} & 
3.63\\CON8-9 & 816.12 & 1.64 & 
816.12 & 1.64 & \bf{815.10} & 
0.13\\[1ex]\hline
\end{tabular}
\label{table:nonlin}
\end{table} \clearpage
\begin{table}[ht]
\caption{Resultados de la ejecución de la metaheurística ACO, utilizando instancias de Dethloff con la configuración -n 2.0 -alpha 1.0 -beta 3.0 -q 1.1 -ro 0.015}
\centering
\small
\begin{tabular}{c c c c c c c}
\hline\hline
Instancia & Costo mínimo & Tiempo(seg.) & Costo promedio & Tiempo promedio(seg.) & Costo ACO & \%Gap \\ [0.5ex]
\hline
SCA3-0 & 640.55 & 1.38 & 
640.55 & 1.37 & \bf{636.10} & 
0.70\\SCA3-1 & \bf{\underline{697.84}} & 1.54 & 
697.84 & 1.55 & 700.10 & 
-0.32\\SCA3-2 & 659.34 & 1.36 & 
662.97 & 1.31 & \bf{659.30} & 
0.01\\SCA3-3 & 680.60 & 1.49 & 
680.92 & 1.50 & \bf{680.00} & 
0.09\\SCA3-4 & \bf{690.50} & 1.30 & 
690.50 & 1.37 & 690.50 & 0.00\\
SCA3-5 & \bf{\underline{665.04}} & 1.42 & 
665.39 & 1.47 & 671.10 & 
-0.90\\SCA3-6 & 655.19 & 1.32 & 
655.19 & 1.30 & \bf{651.10} & 
0.63\\SCA3-7 & 666.15 & 0.92 & 
666.15 & 0.96 & \bf{666.10} & 
0.01\\SCA3-8 & 721.45 & 1.13 & 
722.70 & 1.17 & \bf{719.50} & 
0.27\\SCA3-9 & \bf{681.00} & 0.94 & 
681.00 & 0.94 & 681.00 & 0.00\\
SCA8-0 & 991.07 & 1.44 & 
991.65 & 1.48 & \bf{961.60} & 
3.06\\SCA8-1 & 1074.65 & 1.16 & 
1074.65 & 1.16 & \bf{1063.00} & 
1.10\\SCA8-2 & 1056.87 & 1.06 & 
1056.87 & 1.03 & \bf{1040.60} & 
1.56\\SCA8-3 & 1031.08 & 1.49 & 
1031.08 & 1.47 & \bf{985.90} & 
4.58\\SCA8-4 & 1099.06 & 1.38 & 
1099.17 & 1.44 & \bf{1071.00} & 
2.62\\SCA8-5 & 1055.35 & 1.57 & 
1055.35 & 1.66 & \bf{1054.30} & 
0.10\\SCA8-6 & \bf{\underline{972.48}} & 1.74 & 
975.00 & 1.65 & 972.50 & 
-0.00\\SCA8-7 & 1092.57 & 1.64 & 
1092.57 & 1.60 & \bf{1059.70} & 
3.10\\SCA8-8 & 1092.02 & 1.48 & 
1092.02 & 1.43 & \bf{1082.70} & 
0.86\\SCA8-9 & \bf{\underline{1067.42}} & 1.14 & 
1067.42 & 1.14 & 1081.40 & 
-1.29\\CON3-0 & 624.96 & 1.56 & 
624.96 & 1.71 & \bf{616.50} & 
1.37\\CON3-1 & 557.38 & 1.45 & 
558.22 & 1.43 & \bf{555.60} & 
0.32\\CON3-2 & 524.07 & 1.07 & 
524.07 & 1.09 & \bf{521.40} & 
0.51\\CON3-3 & \bf{591.20} & 1.58 & 
593.38 & 1.59 & 591.20 & 0.00\\
CON3-4 & 589.32 & 1.31 & 
589.32 & 1.30 & \bf{589.30} & 
0.00\\CON3-5 & 569.88 & 1.37 & 
574.79 & 1.42 & \bf{563.70} & 
1.10\\CON3-6 & 504.15 & 1.94 & 
506.15 & 1.83 & \bf{499.20} & 
0.99\\CON3-7 & 578.41 & 1.22 & 
579.12 & 1.24 & \bf{577.50} & 
0.16\\CON3-8 & 524.59 & 1.15 & 
524.59 & 1.21 & \bf{523.10} & 
0.28\\CON3-9 & 578.25 & 1.29 & 
585.35 & 1.25 & \bf{578.20} & 
0.01\\CON8-0 & 879.00 & 1.46 & 
879.00 & 1.45 & \bf{858.90} & 
2.34\\CON8-1 & 758.26 & 1.33 & 
758.26 & 1.33 & \bf{740.90} & 
2.34\\CON8-2 & 716.53 & 1.97 & 
716.55 & 2.00 & \bf{714.30} & 
0.31\\CON8-3 & 817.57 & 1.49 & 
817.57 & 1.46 & \bf{812.30} & 
0.65\\CON8-4 & 781.64 & 1.45 & 
785.81 & 1.52 & \bf{770.10} & 
1.50\\CON8-5 & \bf{\underline{764.36}} & 1.34 & 
764.36 & 1.40 & 766.60 & 
-0.29\\CON8-6 & 705.61 & 1.64 & 
706.75 & 1.72 & \bf{697.20} & 
1.21\\CON8-7 & 822.42 & 1.19 & 
822.67 & 1.18 & \bf{814.80} & 
0.94\\CON8-8 & 799.32 & 1.61 & 
799.37 & 1.53 & \bf{771.30} & 
3.63\\CON8-9 & 816.12 & 1.53 & 
816.12 & 1.56 & \bf{815.10} & 
0.13\\[1ex]\hline
\end{tabular}
\label{table:nonlin}
\end{table} \clearpage
\begin{table}[ht]
\caption{Resultados de la ejecución de la metaheurística ACO, utilizando instancias de Dethloff con la configuración -n 2.0 -alpha 1.0 -beta 3.0 -q 1.2 -ro 0.015}
\centering
\small
\begin{tabular}{c c c c c c c}
\hline\hline
Instancia & Costo mínimo & Tiempo(seg.) & Costo promedio & Tiempo promedio(seg.) & Costo ACO & \%Gap \\ [0.5ex]
\hline
SCA3-0 & 640.55 & 1.35 & 
640.55 & 1.33 & \bf{636.10} & 
0.70\\SCA3-1 & \bf{\underline{697.84}} & 1.45 & 
697.84 & 1.49 & 700.10 & 
-0.32\\SCA3-2 & 664.18 & 1.26 & 
664.18 & 1.33 & \bf{659.30} & 
0.74\\SCA3-3 & 680.60 & 1.42 & 
680.60 & 1.44 & \bf{680.00} & 
0.09\\SCA3-4 & \bf{690.50} & 1.40 & 
690.50 & 1.43 & 690.50 & 0.00\\
SCA3-5 & \bf{\underline{665.04}} & 1.42 & 
668.74 & 1.43 & 671.10 & 
-0.90\\SCA3-6 & 655.19 & 1.31 & 
655.19 & 1.38 & \bf{651.10} & 
0.63\\SCA3-7 & 666.15 & 0.95 & 
666.15 & 0.95 & \bf{666.10} & 
0.01\\SCA3-8 & 721.45 & 1.12 & 
722.95 & 1.16 & \bf{719.50} & 
0.27\\SCA3-9 & \bf{681.00} & 0.90 & 
681.00 & 0.95 & 681.00 & 0.00\\
SCA8-0 & 991.07 & 1.58 & 
991.07 & 1.54 & \bf{961.60} & 
3.06\\SCA8-1 & 1069.40 & 1.22 & 
1072.03 & 1.19 & \bf{1063.00} & 
0.60\\SCA8-2 & 1056.87 & 1.00 & 
1056.87 & 1.03 & \bf{1040.60} & 
1.56\\SCA8-3 & 1031.08 & 1.43 & 
1031.08 & 1.40 & \bf{985.90} & 
4.58\\SCA8-4 & 1099.06 & 1.48 & 
1099.06 & 1.50 & \bf{1071.00} & 
2.62\\SCA8-5 & 1055.35 & 1.54 & 
1055.35 & 1.60 & \bf{1054.30} & 
0.10\\SCA8-6 & \bf{\underline{972.48}} & 1.62 & 
980.01 & 1.65 & 972.50 & 
-0.00\\SCA8-7 & 1092.57 & 1.62 & 
1092.57 & 1.61 & \bf{1059.70} & 
3.10\\SCA8-8 & 1092.02 & 1.34 & 
1092.02 & 1.41 & \bf{1082.70} & 
0.86\\SCA8-9 & \bf{\underline{1067.42}} & 1.25 & 
1067.42 & 1.21 & 1081.40 & 
-1.29\\CON3-0 & 624.96 & 1.67 & 
624.96 & 1.65 & \bf{616.50} & 
1.37\\CON3-1 & 557.38 & 1.60 & 
558.22 & 1.51 & \bf{555.60} & 
0.32\\CON3-2 & 525.17 & 1.05 & 
525.50 & 1.13 & \bf{521.40} & 
0.72\\CON3-3 & 594.11 & 1.45 & 
594.11 & 1.51 & \bf{591.20} & 
0.49\\CON3-4 & 589.32 & 1.35 & 
589.32 & 1.33 & \bf{589.30} & 
0.00\\CON3-5 & 574.57 & 1.34 & 
575.97 & 1.40 & \bf{563.70} & 
1.93\\CON3-6 & 505.26 & 1.83 & 
506.43 & 1.78 & \bf{499.20} & 
1.21\\CON3-7 & 578.41 & 1.22 & 
579.12 & 1.23 & \bf{577.50} & 
0.16\\CON3-8 & 524.30 & 1.15 & 
524.52 & 1.22 & \bf{523.10} & 
0.23\\CON3-9 & 588.48 & 1.31 & 
588.48 & 1.28 & \bf{578.20} & 
1.78\\CON8-0 & 879.00 & 1.44 & 
879.00 & 1.47 & \bf{858.90} & 
2.34\\CON8-1 & 758.26 & 1.37 & 
758.26 & 1.37 & \bf{740.90} & 
2.34\\CON8-2 & 716.53 & 2.00 & 
717.20 & 2.02 & \bf{714.30} & 
0.31\\CON8-3 & 817.57 & 1.52 & 
817.57 & 1.47 & \bf{812.30} & 
0.65\\CON8-4 & 781.64 & 1.64 & 
787.89 & 1.56 & \bf{770.10} & 
1.50\\CON8-5 & \bf{\underline{764.36}} & 1.37 & 
764.36 & 1.39 & 766.60 & 
-0.29\\CON8-6 & \bf{\underline{693.83}} & 1.67 & 
700.17 & 1.70 & 697.20 & 
-0.48\\CON8-7 & 822.42 & 1.20 & 
822.67 & 1.21 & \bf{814.80} & 
0.94\\CON8-8 & 799.51 & 1.62 & 
799.51 & 1.53 & \bf{771.30} & 
3.66\\CON8-9 & 816.12 & 1.49 & 
816.12 & 1.50 & \bf{815.10} & 
0.13\\[1ex]\hline
\end{tabular}
\label{table:nonlin}
\end{table} \clearpage
\begin{table}[ht]
\caption{Resultados de la ejecución de la metaheurística ACO, utilizando instancias de Dethloff con la configuración -n 2.0 -alpha 1.0 -beta 3.0 -q 1.3 -ro 0.015}
\centering
\small
\begin{tabular}{c c c c c c c}
\hline\hline
Instancia & Costo mínimo & Tiempo(seg.) & Costo promedio & Tiempo promedio(seg.) & Costo ACO & \%Gap \\ [0.5ex]
\hline
SCA3-0 & 640.55 & 1.35 & 
640.55 & 1.39 & \bf{636.10} & 
0.70\\SCA3-1 & \bf{\underline{697.84}} & 1.59 & 
698.76 & 1.52 & 700.10 & 
-0.32\\SCA3-2 & 664.18 & 1.39 & 
664.18 & 1.38 & \bf{659.30} & 
0.74\\SCA3-3 & 680.60 & 1.42 & 
680.78 & 1.43 & \bf{680.00} & 
0.09\\SCA3-4 & \bf{690.50} & 1.38 & 
690.50 & 1.35 & 690.50 & 0.00\\
SCA3-5 & \bf{\underline{665.04}} & 1.36 & 
665.19 & 1.42 & 671.10 & 
-0.90\\SCA3-6 & 655.19 & 1.47 & 
655.30 & 1.34 & \bf{651.10} & 
0.63\\SCA3-7 & 666.15 & 0.98 & 
666.15 & 0.99 & \bf{666.10} & 
0.01\\SCA3-8 & 721.45 & 1.16 & 
725.97 & 1.10 & \bf{719.50} & 
0.27\\SCA3-9 & \bf{681.00} & 1.02 & 
681.00 & 1.01 & 681.00 & 0.00\\
SCA8-0 & 991.07 & 1.51 & 
991.07 & 1.56 & \bf{961.60} & 
3.06\\SCA8-1 & 1074.65 & 1.20 & 
1074.68 & 1.21 & \bf{1063.00} & 
1.10\\SCA8-2 & 1056.87 & 1.03 & 
1056.87 & 1.03 & \bf{1040.60} & 
1.56\\SCA8-3 & 1031.08 & 1.45 & 
1031.08 & 1.44 & \bf{985.90} & 
4.58\\SCA8-4 & 1099.06 & 1.37 & 
1099.06 & 1.42 & \bf{1071.00} & 
2.62\\SCA8-5 & 1055.35 & 1.63 & 
1055.35 & 1.60 & \bf{1054.30} & 
0.10\\SCA8-6 & \bf{\underline{972.48}} & 1.67 & 
972.48 & 1.68 & 972.50 & 
-0.00\\SCA8-7 & 1092.57 & 1.58 & 
1092.57 & 1.61 & \bf{1059.70} & 
3.10\\SCA8-8 & 1092.02 & 1.47 & 
1092.02 & 1.46 & \bf{1082.70} & 
0.86\\SCA8-9 & \bf{\underline{1067.42}} & 1.12 & 
1067.42 & 1.11 & 1081.40 & 
-1.29\\CON3-0 & 624.96 & 1.60 & 
624.96 & 1.59 & \bf{616.50} & 
1.37\\CON3-1 & 557.38 & 1.47 & 
558.22 & 1.47 & \bf{555.60} & 
0.32\\CON3-2 & 524.07 & 1.12 & 
525.72 & 1.11 & \bf{521.40} & 
0.51\\CON3-3 & \bf{591.20} & 1.48 & 
592.65 & 1.51 & 591.20 & 0.00\\
CON3-4 & 589.32 & 1.28 & 
589.32 & 1.34 & \bf{589.30} & 
0.00\\CON3-5 & 568.76 & 1.43 & 
572.69 & 1.42 & \bf{563.70} & 
0.90\\CON3-6 & 505.26 & 1.72 & 
506.43 & 1.79 & \bf{499.20} & 
1.21\\CON3-7 & 578.41 & 1.14 & 
579.84 & 1.21 & \bf{577.50} & 
0.16\\CON3-8 & 524.59 & 1.29 & 
524.59 & 1.28 & \bf{523.10} & 
0.28\\CON3-9 & 588.48 & 1.24 & 
588.48 & 1.24 & \bf{578.20} & 
1.78\\CON8-0 & 879.00 & 1.83 & 
879.00 & 1.54 & \bf{858.90} & 
2.34\\CON8-1 & 754.98 & 1.28 & 
757.44 & 1.33 & \bf{740.90} & 
1.90\\CON8-2 & 716.53 & 1.98 & 
716.53 & 1.99 & \bf{714.30} & 
0.31\\CON8-3 & 817.57 & 1.45 & 
817.57 & 1.45 & \bf{812.30} & 
0.65\\CON8-4 & 789.98 & 1.68 & 
790.76 & 1.56 & \bf{770.10} & 
2.58\\CON8-5 & \bf{\underline{764.36}} & 1.31 & 
764.36 & 1.36 & 766.60 & 
-0.29\\CON8-6 & 705.61 & 1.75 & 
706.89 & 1.72 & \bf{697.20} & 
1.21\\CON8-7 & 822.42 & 1.20 & 
823.18 & 1.29 & \bf{814.80} & 
0.94\\CON8-8 & 799.32 & 1.55 & 
799.46 & 1.57 & \bf{771.30} & 
3.63\\CON8-9 & 816.12 & 1.63 & 
817.40 & 1.56 & \bf{815.10} & 
0.13\\[1ex]\hline
\end{tabular}
\label{table:nonlin}
\end{table} \clearpage
\begin{table}[ht]
\caption{Resultados de la ejecución de la metaheurística ACO, utilizando instancias de Dethloff con la configuración -n 2.0 -alpha 1.0 -beta 3.0 -q 1.4 -ro 0.015}
\centering
\small
\begin{tabular}{c c c c c c c}
\hline\hline
Instancia & Costo mínimo & Tiempo(seg.) & Costo promedio & Tiempo promedio(seg.) & Costo ACO & \%Gap \\ [0.5ex]
\hline
SCA3-0 & 640.55 & 1.46 & 
640.55 & 1.39 & \bf{636.10} & 
0.70\\SCA3-1 & \bf{\underline{697.84}} & 1.46 & 
699.68 & 1.51 & 700.10 & 
-0.32\\SCA3-2 & 659.34 & 1.42 & 
662.97 & 1.33 & \bf{659.30} & 
0.01\\SCA3-3 & 680.60 & 1.50 & 
680.96 & 1.46 & \bf{680.00} & 
0.09\\SCA3-4 & \bf{690.50} & 1.38 & 
690.50 & 1.41 & 690.50 & 0.00\\
SCA3-5 & \bf{\underline{665.04}} & 1.36 & 
665.04 & 1.38 & 671.10 & 
-0.90\\SCA3-6 & 655.19 & 1.30 & 
655.19 & 1.32 & \bf{651.10} & 
0.63\\SCA3-7 & 666.15 & 0.96 & 
666.15 & 0.99 & \bf{666.10} & 
0.01\\SCA3-8 & 721.45 & 1.10 & 
723.34 & 1.08 & \bf{719.50} & 
0.27\\SCA3-9 & \bf{681.00} & 0.98 & 
681.00 & 0.96 & 681.00 & 0.00\\
SCA8-0 & 991.07 & 1.40 & 
993.89 & 1.46 & \bf{961.60} & 
3.06\\SCA8-1 & 1069.40 & 1.16 & 
1072.03 & 1.15 & \bf{1063.00} & 
0.60\\SCA8-2 & 1056.87 & 1.05 & 
1056.87 & 1.05 & \bf{1040.60} & 
1.56\\SCA8-3 & 1031.08 & 1.45 & 
1031.08 & 1.43 & \bf{985.90} & 
4.58\\SCA8-4 & 1099.06 & 1.58 & 
1099.06 & 1.50 & \bf{1071.00} & 
2.62\\SCA8-5 & 1055.35 & 1.67 & 
1055.35 & 1.60 & \bf{1054.30} & 
0.10\\SCA8-6 & \bf{\underline{972.48}} & 1.64 & 
972.48 & 1.70 & 972.50 & 
-0.00\\SCA8-7 & 1075.42 & 1.62 & 
1088.28 & 1.58 & \bf{1059.70} & 
1.48\\SCA8-8 & 1092.02 & 1.47 & 
1092.22 & 1.40 & \bf{1082.70} & 
0.86\\SCA8-9 & \bf{\underline{1067.42}} & 1.04 & 
1067.42 & 1.09 & 1081.40 & 
-1.29\\CON3-0 & 624.96 & 1.55 & 
624.96 & 1.55 & \bf{616.50} & 
1.37\\CON3-1 & 557.38 & 1.51 & 
557.82 & 1.47 & \bf{555.60} & 
0.32\\CON3-2 & 524.07 & 1.14 & 
525.08 & 1.13 & \bf{521.40} & 
0.51\\CON3-3 & \bf{591.20} & 1.39 & 
592.65 & 1.47 & 591.20 & 0.00\\
CON3-4 & \bf{\underline{588.79}} & 1.28 & 
589.19 & 1.28 & 589.30 & 
-0.09\\CON3-5 & 569.15 & 1.38 & 
574.61 & 1.39 & \bf{563.70} & 
0.97\\CON3-6 & 504.15 & 1.64 & 
507.33 & 1.68 & \bf{499.20} & 
0.99\\CON3-7 & 578.41 & 1.14 & 
579.12 & 1.18 & \bf{577.50} & 
0.16\\CON3-8 & 524.59 & 1.21 & 
524.59 & 1.18 & \bf{523.10} & 
0.28\\CON3-9 & 588.48 & 1.27 & 
588.48 & 1.24 & \bf{578.20} & 
1.78\\CON8-0 & 879.00 & 1.38 & 
879.00 & 1.37 & \bf{858.90} & 
2.34\\CON8-1 & 758.26 & 1.26 & 
758.26 & 1.33 & \bf{740.90} & 
2.34\\CON8-2 & 716.53 & 2.08 & 
716.54 & 2.00 & \bf{714.30} & 
0.31\\CON8-3 & 817.57 & 1.44 & 
817.57 & 1.43 & \bf{812.30} & 
0.65\\CON8-4 & 781.64 & 1.62 & 
786.59 & 1.59 & \bf{770.10} & 
1.50\\CON8-5 & \bf{\underline{764.36}} & 1.33 & 
764.36 & 1.38 & 766.60 & 
-0.29\\CON8-6 & 707.41 & 1.68 & 
707.53 & 1.69 & \bf{697.20} & 
1.46\\CON8-7 & 822.42 & 1.13 & 
822.92 & 1.16 & \bf{814.80} & 
0.94\\CON8-8 & 799.16 & 1.45 & 
799.33 & 1.56 & \bf{771.30} & 
3.61\\CON8-9 & 816.12 & 1.54 & 
816.12 & 1.62 & \bf{815.10} & 
0.13\\[1ex]\hline
\end{tabular}
\label{table:nonlin}
\end{table} \clearpage
\begin{table}[ht]
\caption{Resultados de la ejecución de la metaheurística ACO, utilizando instancias de Dethloff con la configuración -n 2.0 -alpha 1.0 -beta 3.0 -q 1.5 -ro 0.015}
\centering
\small
\begin{tabular}{c c c c c c c}
\hline\hline
Instancia & Costo mínimo & Tiempo(seg.) & Costo promedio & Tiempo promedio(seg.) & Costo ACO & \%Gap \\ [0.5ex]
\hline
SCA3-0 & 640.55 & 1.44 & 
640.55 & 1.40 & \bf{636.10} & 
0.70\\SCA3-1 & \bf{\underline{697.84}} & 1.51 & 
698.76 & 1.53 & 700.10 & 
-0.32\\SCA3-2 & 664.18 & 1.28 & 
664.18 & 1.33 & \bf{659.30} & 
0.74\\SCA3-3 & 680.60 & 1.41 & 
680.78 & 1.44 & \bf{680.00} & 
0.09\\SCA3-4 & \bf{690.50} & 1.31 & 
690.50 & 1.39 & 690.50 & 0.00\\
SCA3-5 & \bf{\underline{665.04}} & 1.50 & 
665.04 & 1.40 & 671.10 & 
-0.90\\SCA3-6 & 655.19 & 1.41 & 
655.19 & 1.35 & \bf{651.10} & 
0.63\\SCA3-7 & 666.15 & 0.94 & 
666.15 & 0.94 & \bf{666.10} & 
0.01\\SCA3-8 & 726.44 & 1.12 & 
727.73 & 1.11 & \bf{719.50} & 
0.96\\SCA3-9 & \bf{681.00} & 0.93 & 
681.00 & 0.94 & 681.00 & 0.00\\
SCA8-0 & 991.07 & 1.47 & 
991.65 & 1.46 & \bf{961.60} & 
3.06\\SCA8-1 & 1074.65 & 1.17 & 
1074.68 & 1.18 & \bf{1063.00} & 
1.10\\SCA8-2 & 1056.87 & 1.03 & 
1056.87 & 1.01 & \bf{1040.60} & 
1.56\\SCA8-3 & 1031.08 & 1.44 & 
1031.08 & 1.46 & \bf{985.90} & 
4.58\\SCA8-4 & 1098.34 & 1.48 & 
1098.88 & 1.51 & \bf{1071.00} & 
2.55\\SCA8-5 & 1055.35 & 1.76 & 
1055.35 & 1.68 & \bf{1054.30} & 
0.10\\SCA8-6 & \bf{\underline{972.48}} & 1.58 & 
972.48 & 1.58 & 972.50 & 
-0.00\\SCA8-7 & 1092.57 & 1.56 & 
1092.57 & 1.61 & \bf{1059.70} & 
3.10\\SCA8-8 & 1091.49 & 1.36 & 
1091.76 & 1.42 & \bf{1082.70} & 
0.81\\SCA8-9 & \bf{\underline{1067.42}} & 1.17 & 
1067.42 & 1.16 & 1081.40 & 
-1.29\\CON3-0 & 624.96 & 1.64 & 
624.96 & 1.62 & \bf{616.50} & 
1.37\\CON3-1 & 557.38 & 1.45 & 
558.22 & 1.45 & \bf{555.60} & 
0.32\\CON3-2 & 524.07 & 1.13 & 
524.51 & 1.10 & \bf{521.40} & 
0.51\\CON3-3 & \bf{591.20} & 1.52 & 
592.65 & 1.51 & 591.20 & 0.00\\
CON3-4 & 589.32 & 1.44 & 
589.32 & 1.33 & \bf{589.30} & 
0.00\\CON3-5 & 576.43 & 1.40 & 
576.43 & 1.40 & \bf{563.70} & 
2.26\\CON3-6 & 505.26 & 1.75 & 
508.52 & 1.77 & \bf{499.20} & 
1.21\\CON3-7 & 578.41 & 1.30 & 
579.12 & 1.26 & \bf{577.50} & 
0.16\\CON3-8 & 524.30 & 1.86 & 
524.52 & 1.37 & \bf{523.10} & 
0.23\\CON3-9 & 588.48 & 1.26 & 
588.48 & 1.28 & \bf{578.20} & 
1.78\\CON8-0 & 879.00 & 1.39 & 
879.00 & 1.44 & \bf{858.90} & 
2.34\\CON8-1 & 758.26 & 1.30 & 
758.26 & 1.31 & \bf{740.90} & 
2.34\\CON8-2 & 716.53 & 2.03 & 
716.54 & 2.03 & \bf{714.30} & 
0.31\\CON8-3 & 817.57 & 1.46 & 
817.57 & 1.45 & \bf{812.30} & 
0.65\\CON8-4 & 781.64 & 1.51 & 
787.89 & 1.56 & \bf{770.10} & 
1.50\\CON8-5 & \bf{\underline{764.36}} & 1.36 & 
764.36 & 1.38 & 766.60 & 
-0.29\\CON8-6 & 705.61 & 1.68 & 
706.51 & 1.72 & \bf{697.20} & 
1.21\\CON8-7 & 822.42 & 1.22 & 
822.75 & 1.19 & \bf{814.80} & 
0.94\\CON8-8 & 799.32 & 1.52 & 
799.46 & 1.54 & \bf{771.30} & 
3.63\\CON8-9 & 816.12 & 1.55 & 
816.12 & 1.52 & \bf{815.10} & 
0.13\\[1ex]\hline
\end{tabular}
\label{table:nonlin}
\end{table} \clearpage
\begin{table}[ht]
\caption{Resultados de la ejecución de la metaheurística ACO, utilizando instancias de Dethloff con la configuración -n 2.0 -alpha 1.0 -beta 3.0 -q 1.6 -ro 0.015}
\centering
\small
\begin{tabular}{c c c c c c c}
\hline\hline
Instancia & Costo mínimo & Tiempo(seg.) & Costo promedio & Tiempo promedio(seg.) & Costo ACO & \%Gap \\ [0.5ex]
\hline
SCA3-0 & 636.34 & 1.32 & 
639.50 & 1.35 & \bf{636.10} & 
0.04\\SCA3-1 & \bf{\underline{697.84}} & 1.52 & 
697.84 & 1.53 & 700.10 & 
-0.32\\SCA3-2 & 659.34 & 1.28 & 
661.76 & 1.30 & \bf{659.30} & 
0.01\\SCA3-3 & 680.60 & 1.43 & 
680.78 & 1.41 & \bf{680.00} & 
0.09\\SCA3-4 & \bf{690.50} & 1.33 & 
690.50 & 1.36 & 690.50 & 0.00\\
SCA3-5 & \bf{\underline{665.04}} & 1.48 & 
668.60 & 1.46 & 671.10 & 
-0.90\\SCA3-6 & 655.19 & 1.44 & 
655.19 & 1.33 & \bf{651.10} & 
0.63\\SCA3-7 & 666.15 & 1.06 & 
666.15 & 1.05 & \bf{666.10} & 
0.01\\SCA3-8 & 721.45 & 1.13 & 
727.22 & 1.22 & \bf{719.50} & 
0.27\\SCA3-9 & \bf{681.00} & 0.98 & 
681.00 & 0.95 & 681.00 & 0.00\\
SCA8-0 & 991.07 & 1.50 & 
998.34 & 1.53 & \bf{961.60} & 
3.06\\SCA8-1 & 1074.65 & 1.26 & 
1074.68 & 1.20 & \bf{1063.00} & 
1.10\\SCA8-2 & 1056.87 & 0.98 & 
1056.87 & 0.97 & \bf{1040.60} & 
1.56\\SCA8-3 & 1031.08 & 1.42 & 
1031.08 & 1.44 & \bf{985.90} & 
4.58\\SCA8-4 & 1099.06 & 1.42 & 
1099.17 & 1.48 & \bf{1071.00} & 
2.62\\SCA8-5 & 1055.35 & 1.70 & 
1055.35 & 1.66 & \bf{1054.30} & 
0.10\\SCA8-6 & \bf{\underline{972.48}} & 1.70 & 
982.30 & 1.65 & 972.50 & 
-0.00\\SCA8-7 & 1092.57 & 1.60 & 
1092.57 & 1.58 & \bf{1059.70} & 
3.10\\SCA8-8 & 1091.49 & 1.39 & 
1091.89 & 1.39 & \bf{1082.70} & 
0.81\\SCA8-9 & \bf{\underline{1067.42}} & 1.15 & 
1067.42 & 1.14 & 1081.40 & 
-1.29\\CON3-0 & 624.96 & 1.66 & 
624.96 & 1.64 & \bf{616.50} & 
1.37\\CON3-1 & 557.38 & 1.43 & 
557.58 & 1.45 & \bf{555.60} & 
0.32\\CON3-2 & 524.07 & 1.08 & 
525.23 & 1.06 & \bf{521.40} & 
0.51\\CON3-3 & 594.11 & 1.45 & 
594.11 & 1.53 & \bf{591.20} & 
0.49\\CON3-4 & 589.32 & 1.35 & 
589.32 & 1.32 & \bf{589.30} & 
0.00\\CON3-5 & 569.15 & 1.38 & 
574.14 & 1.42 & \bf{563.70} & 
0.97\\CON3-6 & 505.26 & 1.69 & 
507.34 & 1.79 & \bf{499.20} & 
1.21\\CON3-7 & 578.41 & 1.20 & 
578.41 & 1.27 & \bf{577.50} & 
0.16\\CON3-8 & 524.30 & 1.18 & 
524.52 & 1.19 & \bf{523.10} & 
0.23\\CON3-9 & 588.48 & 1.30 & 
588.48 & 1.26 & \bf{578.20} & 
1.78\\CON8-0 & 879.00 & 1.38 & 
879.00 & 1.43 & \bf{858.90} & 
2.34\\CON8-1 & 758.26 & 1.35 & 
758.26 & 1.34 & \bf{740.90} & 
2.34\\CON8-2 & 716.53 & 1.90 & 
716.54 & 1.98 & \bf{714.30} & 
0.31\\CON8-3 & 817.57 & 1.34 & 
817.57 & 1.38 & \bf{812.30} & 
0.65\\CON8-4 & 781.64 & 1.50 & 
788.67 & 1.54 & \bf{770.10} & 
1.50\\CON8-5 & \bf{\underline{764.36}} & 1.34 & 
764.36 & 1.33 & 766.60 & 
-0.29\\CON8-6 & 707.41 & 1.76 & 
707.76 & 1.72 & \bf{697.20} & 
1.46\\CON8-7 & 822.42 & 1.23 & 
822.67 & 1.21 & \bf{814.80} & 
0.94\\CON8-8 & 799.16 & 1.50 & 
799.38 & 1.52 & \bf{771.30} & 
3.61\\CON8-9 & 816.12 & 1.48 & 
817.40 & 1.53 & \bf{815.10} & 
0.13\\[1ex]\hline
\end{tabular}
\label{table:nonlin}
\end{table} \clearpage
\begin{table}[ht]
\caption{Resultados de la ejecución de la metaheurística ACO, utilizando instancias de Dethloff con la configuración -n 2.0 -alpha 1.0 -beta 3.0 -q 1.7 -ro 0.015}
\centering
\small
\begin{tabular}{c c c c c c c}
\hline\hline
Instancia & Costo mínimo & Tiempo(seg.) & Costo promedio & Tiempo promedio(seg.) & Costo ACO & \%Gap \\ [0.5ex]
\hline
SCA3-0 & 636.34 & 1.38 & 
639.50 & 1.36 & \bf{636.10} & 
0.04\\SCA3-1 & \bf{\underline{697.84}} & 1.43 & 
697.84 & 1.46 & 700.10 & 
-0.32\\SCA3-2 & 659.34 & 1.30 & 
662.97 & 1.34 & \bf{659.30} & 
0.01\\SCA3-3 & 680.60 & 1.58 & 
680.96 & 1.50 & \bf{680.00} & 
0.09\\SCA3-4 & \bf{690.50} & 1.36 & 
690.50 & 1.65 & 690.50 & 0.00\\
SCA3-5 & \bf{\underline{665.04}} & 1.42 & 
665.04 & 1.40 & 671.10 & 
-0.90\\SCA3-6 & 654.26 & 1.28 & 
654.96 & 1.36 & \bf{651.10} & 
0.49\\SCA3-7 & 666.15 & 1.05 & 
666.15 & 0.99 & \bf{666.10} & 
0.01\\SCA3-8 & 721.45 & 1.13 & 
727.86 & 1.11 & \bf{719.50} & 
0.27\\SCA3-9 & \bf{681.00} & 0.95 & 
681.00 & 0.97 & 681.00 & 0.00\\
SCA8-0 & 991.07 & 1.48 & 
997.76 & 1.47 & \bf{961.60} & 
3.06\\SCA8-1 & 1074.65 & 1.17 & 
1074.65 & 1.20 & \bf{1063.00} & 
1.10\\SCA8-2 & 1056.87 & 0.99 & 
1056.87 & 1.03 & \bf{1040.60} & 
1.56\\SCA8-3 & 1031.08 & 1.49 & 
1031.08 & 1.41 & \bf{985.90} & 
4.58\\SCA8-4 & 1099.06 & 1.37 & 
1099.06 & 1.46 & \bf{1071.00} & 
2.62\\SCA8-5 & 1055.35 & 1.59 & 
1055.35 & 1.66 & \bf{1054.30} & 
0.10\\SCA8-6 & \bf{\underline{972.48}} & 1.68 & 
976.70 & 1.68 & 972.50 & 
-0.00\\SCA8-7 & 1092.57 & 1.63 & 
1092.57 & 1.65 & \bf{1059.70} & 
3.10\\SCA8-8 & 1091.49 & 1.39 & 
1091.76 & 1.40 & \bf{1082.70} & 
0.81\\SCA8-9 & \bf{\underline{1067.42}} & 1.12 & 
1067.42 & 1.12 & 1081.40 & 
-1.29\\CON3-0 & 624.96 & 1.57 & 
624.96 & 1.60 & \bf{616.50} & 
1.37\\CON3-1 & 557.38 & 1.43 & 
558.22 & 1.45 & \bf{555.60} & 
0.32\\CON3-2 & 524.07 & 1.10 & 
524.07 & 1.08 & \bf{521.40} & 
0.51\\CON3-3 & 594.11 & 1.58 & 
594.11 & 1.54 & \bf{591.20} & 
0.49\\CON3-4 & 589.32 & 1.33 & 
589.32 & 1.36 & \bf{589.30} & 
0.00\\CON3-5 & 569.15 & 1.38 & 
574.61 & 1.40 & \bf{563.70} & 
0.97\\CON3-6 & 505.26 & 1.78 & 
505.26 & 1.76 & \bf{499.20} & 
1.21\\CON3-7 & 578.41 & 1.30 & 
578.41 & 1.22 & \bf{577.50} & 
0.16\\CON3-8 & 524.59 & 1.24 & 
524.59 & 1.17 & \bf{523.10} & 
0.28\\CON3-9 & 588.48 & 1.21 & 
588.48 & 1.21 & \bf{578.20} & 
1.78\\CON8-0 & 879.00 & 1.34 & 
879.00 & 1.37 & \bf{858.90} & 
2.34\\CON8-1 & 758.26 & 1.35 & 
758.26 & 1.31 & \bf{740.90} & 
2.34\\CON8-2 & 716.53 & 1.98 & 
716.55 & 1.98 & \bf{714.30} & 
0.31\\CON8-3 & 817.57 & 1.41 & 
817.57 & 1.41 & \bf{812.30} & 
0.65\\CON8-4 & 781.64 & 1.46 & 
790.23 & 1.54 & \bf{770.10} & 
1.50\\CON8-5 & \bf{\underline{764.36}} & 1.39 & 
764.36 & 1.35 & 766.60 & 
-0.29\\CON8-6 & 706.20 & 1.60 & 
707.23 & 1.66 & \bf{697.20} & 
1.29\\CON8-7 & 822.42 & 1.20 & 
822.92 & 1.18 & \bf{814.80} & 
0.94\\CON8-8 & 799.32 & 1.64 & 
799.46 & 1.59 & \bf{771.30} & 
3.63\\CON8-9 & 816.12 & 1.68 & 
816.12 & 1.59 & \bf{815.10} & 
0.13\\[1ex]\hline
\end{tabular}
\label{table:nonlin}
\end{table} \clearpage
\begin{table}[ht]
\caption{Resultados de la ejecución de la metaheurística ACO, utilizando instancias de Dethloff con la configuración -n 2.0 -alpha 1.0 -beta 3.0 -q 1.8 -ro 0.015}
\centering
\small
\begin{tabular}{c c c c c c c}
\hline\hline
Instancia & Costo mínimo & Tiempo(seg.) & Costo promedio & Tiempo promedio(seg.) & Costo ACO & \%Gap \\ [0.5ex]
\hline
SCA3-0 & 640.55 & 1.35 & 
640.55 & 1.37 & \bf{636.10} & 
0.70\\SCA3-1 & \bf{\underline{697.84}} & 1.57 & 
697.84 & 1.51 & 700.10 & 
-0.32\\SCA3-2 & 659.34 & 1.29 & 
662.97 & 1.32 & \bf{659.30} & 
0.01\\SCA3-3 & 680.60 & 1.50 & 
680.96 & 1.47 & \bf{680.00} & 
0.09\\SCA3-4 & \bf{690.50} & 1.38 & 
690.50 & 1.39 & 690.50 & 0.00\\
SCA3-5 & \bf{\underline{665.04}} & 1.33 & 
668.60 & 1.37 & 671.10 & 
-0.90\\SCA3-6 & 655.19 & 1.30 & 
655.19 & 1.31 & \bf{651.10} & 
0.63\\SCA3-7 & 666.15 & 1.00 & 
666.15 & 0.99 & \bf{666.10} & 
0.01\\SCA3-8 & 721.45 & 1.19 & 
726.48 & 1.12 & \bf{719.50} & 
0.27\\SCA3-9 & \bf{681.00} & 0.96 & 
681.00 & 0.95 & 681.00 & 0.00\\
SCA8-0 & 991.07 & 1.42 & 
991.07 & 1.48 & \bf{961.60} & 
3.06\\SCA8-1 & 1074.65 & 1.16 & 
1074.65 & 1.20 & \bf{1063.00} & 
1.10\\SCA8-2 & 1056.87 & 1.03 & 
1056.87 & 1.01 & \bf{1040.60} & 
1.56\\SCA8-3 & 1031.08 & 1.51 & 
1031.08 & 1.48 & \bf{985.90} & 
4.58\\SCA8-4 & 1099.06 & 1.43 & 
1099.06 & 1.43 & \bf{1071.00} & 
2.62\\SCA8-5 & 1055.35 & 1.72 & 
1055.35 & 1.65 & \bf{1054.30} & 
0.10\\SCA8-6 & \bf{\underline{972.48}} & 1.66 & 
972.48 & 1.65 & 972.50 & 
-0.00\\SCA8-7 & 1092.57 & 1.72 & 
1092.57 & 1.67 & \bf{1059.70} & 
3.10\\SCA8-8 & 1091.49 & 1.46 & 
1091.89 & 1.43 & \bf{1082.70} & 
0.81\\SCA8-9 & \bf{\underline{1067.42}} & 1.07 & 
1067.42 & 1.19 & 1081.40 & 
-1.29\\CON3-0 & 624.96 & 1.62 & 
624.96 & 1.62 & \bf{616.50} & 
1.37\\CON3-1 & 557.38 & 1.44 & 
557.58 & 1.46 & \bf{555.60} & 
0.32\\CON3-2 & 525.24 & 1.08 & 
525.93 & 1.16 & \bf{521.40} & 
0.74\\CON3-3 & 594.11 & 1.48 & 
594.11 & 1.52 & \bf{591.20} & 
0.49\\CON3-4 & 589.32 & 1.46 & 
589.32 & 1.35 & \bf{589.30} & 
0.00\\CON3-5 & 576.43 & 1.39 & 
576.43 & 1.41 & \bf{563.70} & 
2.26\\CON3-6 & 505.26 & 1.75 & 
507.61 & 1.77 & \bf{499.20} & 
1.21\\CON3-7 & 578.41 & 1.17 & 
579.12 & 1.21 & \bf{577.50} & 
0.16\\CON3-8 & 524.30 & 1.26 & 
524.52 & 1.19 & \bf{523.10} & 
0.23\\CON3-9 & 588.48 & 1.35 & 
588.48 & 1.29 & \bf{578.20} & 
1.78\\CON8-0 & 879.00 & 1.44 & 
879.00 & 1.42 & \bf{858.90} & 
2.34\\CON8-1 & 758.26 & 1.20 & 
758.26 & 1.28 & \bf{740.90} & 
2.34\\CON8-2 & 716.53 & 1.99 & 
716.54 & 2.02 & \bf{714.30} & 
0.31\\CON8-3 & 817.57 & 1.45 & 
817.57 & 1.45 & \bf{812.30} & 
0.65\\CON8-4 & 781.64 & 1.54 & 
789.45 & 1.53 & \bf{770.10} & 
1.50\\CON8-5 & \bf{\underline{764.36}} & 1.30 & 
764.36 & 1.32 & 766.60 & 
-0.29\\CON8-6 & 705.61 & 1.74 & 
706.75 & 1.68 & \bf{697.20} & 
1.21\\CON8-7 & 822.42 & 1.20 & 
823.18 & 1.17 & \bf{814.80} & 
0.94\\CON8-8 & 799.16 & 1.57 & 
799.42 & 1.74 & \bf{771.30} & 
3.61\\CON8-9 & 816.12 & 1.68 & 
816.12 & 1.64 & \bf{815.10} & 
0.13\\[1ex]\hline
\end{tabular}
\label{table:nonlin}
\end{table} \clearpage
\begin{table}[ht]
\caption{Resultados de la ejecución de la metaheurística ACO, utilizando instancias de Dethloff con la configuración -n 2.0 -alpha 1.0 -beta 3.0 -q 1.9 -ro 0.015}
\centering
\small
\begin{tabular}{c c c c c c c}
\hline\hline
Instancia & Costo mínimo & Tiempo(seg.) & Costo promedio & Tiempo promedio(seg.) & Costo ACO & \%Gap \\ [0.5ex]
\hline
SCA3-0 & 640.55 & 1.34 & 
640.55 & 1.36 & \bf{636.10} & 
0.70\\SCA3-1 & \bf{\underline{697.84}} & 1.54 & 
697.84 & 1.50 & 700.10 & 
-0.32\\SCA3-2 & 664.18 & 1.42 & 
664.18 & 1.34 & \bf{659.30} & 
0.74\\SCA3-3 & 680.60 & 1.42 & 
680.60 & 1.48 & \bf{680.00} & 
0.09\\SCA3-4 & \bf{690.50} & 1.50 & 
690.50 & 1.43 & 690.50 & 0.00\\
SCA3-5 & \bf{\underline{665.04}} & 1.37 & 
665.19 & 1.48 & 671.10 & 
-0.90\\SCA3-6 & 655.19 & 1.30 & 
655.19 & 1.32 & \bf{651.10} & 
0.63\\SCA3-7 & 666.15 & 1.10 & 
666.15 & 1.04 & \bf{666.10} & 
0.01\\SCA3-8 & 721.45 & 1.12 & 
724.59 & 1.12 & \bf{719.50} & 
0.27\\SCA3-9 & \bf{681.00} & 0.94 & 
681.00 & 0.96 & 681.00 & 0.00\\
SCA8-0 & 991.07 & 1.51 & 
991.07 & 1.49 & \bf{961.60} & 
3.06\\SCA8-1 & 1074.65 & 1.22 & 
1074.65 & 1.19 & \bf{1063.00} & 
1.10\\SCA8-2 & 1056.87 & 1.02 & 
1056.87 & 1.00 & \bf{1040.60} & 
1.56\\SCA8-3 & 1031.08 & 1.42 & 
1031.08 & 1.43 & \bf{985.90} & 
4.58\\SCA8-4 & 1099.06 & 1.46 & 
1099.06 & 1.50 & \bf{1071.00} & 
2.62\\SCA8-5 & 1055.35 & 1.64 & 
1055.35 & 1.70 & \bf{1054.30} & 
0.10\\SCA8-6 & \bf{\underline{972.48}} & 1.61 & 
972.48 & 1.63 & 972.50 & 
-0.00\\SCA8-7 & 1092.57 & 1.64 & 
1092.57 & 1.63 & \bf{1059.70} & 
3.10\\SCA8-8 & 1091.49 & 1.33 & 
1091.89 & 1.42 & \bf{1082.70} & 
0.81\\SCA8-9 & \bf{\underline{1067.42}} & 1.14 & 
1067.42 & 1.14 & 1081.40 & 
-1.29\\CON3-0 & 624.96 & 1.54 & 
624.96 & 1.56 & \bf{616.50} & 
1.37\\CON3-1 & 557.38 & 1.38 & 
557.82 & 1.48 & \bf{555.60} & 
0.32\\CON3-2 & 524.07 & 1.09 & 
526.99 & 1.10 & \bf{521.40} & 
0.51\\CON3-3 & \bf{591.20} & 1.48 & 
592.65 & 1.54 & 591.20 & 0.00\\
CON3-4 & 589.32 & 1.43 & 
589.32 & 1.36 & \bf{589.30} & 
0.00\\CON3-5 & 576.43 & 1.51 & 
577.42 & 1.42 & \bf{563.70} & 
2.26\\CON3-6 & 505.26 & 1.81 & 
506.43 & 1.80 & \bf{499.20} & 
1.21\\CON3-7 & 578.41 & 1.28 & 
579.12 & 1.21 & \bf{577.50} & 
0.16\\CON3-8 & 524.59 & 1.30 & 
524.59 & 1.20 & \bf{523.10} & 
0.28\\CON3-9 & 588.48 & 1.22 & 
588.48 & 1.24 & \bf{578.20} & 
1.78\\CON8-0 & 879.00 & 1.42 & 
879.00 & 1.47 & \bf{858.90} & 
2.34\\CON8-1 & 758.26 & 1.31 & 
758.26 & 1.33 & \bf{740.90} & 
2.34\\CON8-2 & 716.53 & 1.97 & 
717.20 & 1.95 & \bf{714.30} & 
0.31\\CON8-3 & 817.57 & 1.41 & 
817.57 & 1.41 & \bf{812.30} & 
0.65\\CON8-4 & 781.64 & 1.48 & 
787.37 & 1.52 & \bf{770.10} & 
1.50\\CON8-5 & \bf{\underline{764.36}} & 1.36 & 
764.36 & 1.37 & 766.60 & 
-0.29\\CON8-6 & 705.61 & 1.66 & 
706.96 & 1.73 & \bf{697.20} & 
1.21\\CON8-7 & 822.42 & 1.21 & 
823.18 & 1.19 & \bf{814.80} & 
0.94\\CON8-8 & 799.16 & 1.67 & 
799.28 & 1.57 & \bf{771.30} & 
3.61\\CON8-9 & 816.12 & 1.60 & 
817.40 & 1.57 & \bf{815.10} & 
0.13\\[1ex]\hline
\end{tabular}
\label{table:nonlin}
\end{table} \clearpage
\begin{table}[ht]
\caption{Resultados de la ejecución de la metaheurística ACO, utilizando instancias de Dethloff con la configuración -n 2.0 -alpha 1.0 -beta 3.0 -q 2.0 -ro 0.015}
\centering
\small
\begin{tabular}{c c c c c c c}
\hline\hline
Instancia & Costo mínimo & Tiempo(seg.) & Costo promedio & Tiempo promedio(seg.) & Costo ACO & \%Gap \\ [0.5ex]
\hline
SCA3-0 & 640.55 & 1.36 & 
640.55 & 1.34 & \bf{636.10} & 
0.70\\SCA3-1 & \bf{\underline{697.84}} & 1.50 & 
697.84 & 1.44 & 700.10 & 
-0.32\\SCA3-2 & 659.34 & 1.38 & 
662.97 & 1.34 & \bf{659.30} & 
0.01\\SCA3-3 & 680.60 & 1.36 & 
680.96 & 1.41 & \bf{680.00} & 
0.09\\SCA3-4 & \bf{690.50} & 1.32 & 
690.50 & 1.42 & 690.50 & 0.00\\
SCA3-5 & \bf{\underline{665.04}} & 1.44 & 
665.04 & 1.46 & 671.10 & 
-0.90\\SCA3-6 & 653.69 & 1.40 & 
654.82 & 1.35 & \bf{651.10} & 
0.40\\SCA3-7 & 666.15 & 0.94 & 
666.15 & 0.97 & \bf{666.10} & 
0.01\\SCA3-8 & 726.44 & 1.04 & 
727.09 & 1.11 & \bf{719.50} & 
0.96\\SCA3-9 & \bf{681.00} & 0.97 & 
681.00 & 0.94 & 681.00 & 0.00\\
SCA8-0 & 991.07 & 1.49 & 
991.65 & 1.57 & \bf{961.60} & 
3.06\\SCA8-1 & 1074.65 & 1.20 & 
1074.68 & 1.16 & \bf{1063.00} & 
1.10\\SCA8-2 & 1056.87 & 1.02 & 
1056.87 & 1.01 & \bf{1040.60} & 
1.56\\SCA8-3 & 1031.08 & 1.40 & 
1031.08 & 1.42 & \bf{985.90} & 
4.58\\SCA8-4 & 1098.34 & 1.55 & 
1098.88 & 1.49 & \bf{1071.00} & 
2.55\\SCA8-5 & 1055.35 & 1.62 & 
1055.35 & 1.67 & \bf{1054.30} & 
0.10\\SCA8-6 & \bf{\underline{972.48}} & 1.64 & 
972.48 & 1.65 & 972.50 & 
-0.00\\SCA8-7 & 1092.57 & 1.63 & 
1092.57 & 1.58 & \bf{1059.70} & 
3.10\\SCA8-8 & 1092.02 & 1.39 & 
1092.02 & 1.43 & \bf{1082.70} & 
0.86\\SCA8-9 & \bf{\underline{1067.42}} & 1.07 & 
1067.42 & 1.13 & 1081.40 & 
-1.29\\CON3-0 & 624.96 & 1.72 & 
624.96 & 1.64 & \bf{616.50} & 
1.37\\CON3-1 & 557.38 & 1.46 & 
557.82 & 1.45 & \bf{555.60} & 
0.32\\CON3-2 & 524.07 & 1.13 & 
524.84 & 1.07 & \bf{521.40} & 
0.51\\CON3-3 & \bf{591.20} & 1.49 & 
593.09 & 1.54 & 591.20 & 0.00\\
CON3-4 & \bf{\underline{588.79}} & 1.35 & 
589.05 & 1.32 & 589.30 & 
-0.09\\CON3-5 & 569.88 & 1.39 & 
574.79 & 1.40 & \bf{563.70} & 
1.10\\CON3-6 & 504.15 & 1.79 & 
504.98 & 1.78 & \bf{499.20} & 
0.99\\CON3-7 & 578.41 & 1.18 & 
579.84 & 1.19 & \bf{577.50} & 
0.16\\CON3-8 & 524.30 & 1.12 & 
524.52 & 1.18 & \bf{523.10} & 
0.23\\CON3-9 & 588.48 & 1.25 & 
588.48 & 1.26 & \bf{578.20} & 
1.78\\CON8-0 & 879.00 & 1.37 & 
879.00 & 1.41 & \bf{858.90} & 
2.34\\CON8-1 & 758.26 & 1.39 & 
758.26 & 1.35 & \bf{740.90} & 
2.34\\CON8-2 & 716.53 & 1.93 & 
716.54 & 1.99 & \bf{714.30} & 
0.31\\CON8-3 & 817.57 & 1.42 & 
817.57 & 1.43 & \bf{812.30} & 
0.65\\CON8-4 & 781.64 & 1.61 & 
788.67 & 1.62 & \bf{770.10} & 
1.50\\CON8-5 & \bf{\underline{764.36}} & 1.31 & 
764.36 & 1.37 & 766.60 & 
-0.29\\CON8-6 & 705.61 & 1.68 & 
706.63 & 1.68 & \bf{697.20} & 
1.21\\CON8-7 & 822.42 & 1.12 & 
822.67 & 1.15 & \bf{814.80} & 
0.94\\CON8-8 & 799.32 & 1.58 & 
799.41 & 1.50 & \bf{771.30} & 
3.63\\CON8-9 & 816.12 & 1.61 & 
816.12 & 1.56 & \bf{815.10} & 
0.13\\[1ex]\hline
\end{tabular}
\label{table:nonlin}
\end{table} \clearpage
\begin{table}[ht]
\caption{Resultados de la ejecución de la metaheurística ACO, utilizando instancias de Dethloff con la configuración -n 2.0 -alpha 1.0 -beta 3.0 -q 2.1 -ro 0.015}
\centering
\small
\begin{tabular}{c c c c c c c}
\hline\hline
Instancia & Costo mínimo & Tiempo(seg.) & Costo promedio & Tiempo promedio(seg.) & Costo ACO & \%Gap \\ [0.5ex]
\hline
SCA3-0 & 640.55 & 1.32 & 
640.55 & 1.39 & \bf{636.10} & 
0.70\\SCA3-1 & \bf{\underline{697.84}} & 1.44 & 
697.84 & 1.47 & 700.10 & 
-0.32\\SCA3-2 & 659.34 & 1.28 & 
662.97 & 1.30 & \bf{659.30} & 
0.01\\SCA3-3 & 680.60 & 1.39 & 
680.78 & 1.47 & \bf{680.00} & 
0.09\\SCA3-4 & \bf{690.50} & 1.34 & 
690.50 & 1.39 & 690.50 & 0.00\\
SCA3-5 & \bf{\underline{665.04}} & 1.34 & 
665.19 & 1.36 & 671.10 & 
-0.90\\SCA3-6 & 655.19 & 1.30 & 
655.19 & 1.29 & \bf{651.10} & 
0.63\\SCA3-7 & 666.15 & 0.99 & 
666.15 & 0.99 & \bf{666.10} & 
0.01\\SCA3-8 & 721.45 & 1.08 & 
724.59 & 1.11 & \bf{719.50} & 
0.27\\SCA3-9 & \bf{681.00} & 1.02 & 
681.00 & 0.99 & 681.00 & 0.00\\
SCA8-0 & 991.07 & 1.63 & 
991.07 & 1.52 & \bf{961.60} & 
3.06\\SCA8-1 & 1074.65 & 1.20 & 
1074.65 & 1.15 & \bf{1063.00} & 
1.10\\SCA8-2 & 1056.87 & 0.97 & 
1056.87 & 1.02 & \bf{1040.60} & 
1.56\\SCA8-3 & 1031.08 & 1.52 & 
1031.08 & 1.50 & \bf{985.90} & 
4.58\\SCA8-4 & 1099.06 & 1.50 & 
1099.06 & 1.48 & \bf{1071.00} & 
2.62\\SCA8-5 & 1055.35 & 1.64 & 
1055.35 & 1.64 & \bf{1054.30} & 
0.10\\SCA8-6 & \bf{\underline{972.48}} & 1.75 & 
973.62 & 1.66 & 972.50 & 
-0.00\\SCA8-7 & 1092.57 & 1.65 & 
1092.57 & 1.57 & \bf{1059.70} & 
3.10\\SCA8-8 & 1092.02 & 1.40 & 
1092.02 & 1.45 & \bf{1082.70} & 
0.86\\SCA8-9 & \bf{\underline{1067.42}} & 1.14 & 
1067.42 & 1.14 & 1081.40 & 
-1.29\\CON3-0 & 624.96 & 1.91 & 
624.96 & 1.69 & \bf{616.50} & 
1.37\\CON3-1 & 557.38 & 1.43 & 
557.38 & 1.41 & \bf{555.60} & 
0.32\\CON3-2 & 524.07 & 1.19 & 
525.48 & 1.11 & \bf{521.40} & 
0.51\\CON3-3 & \bf{591.20} & 1.53 & 
593.38 & 1.47 & 591.20 & 0.00\\
CON3-4 & 589.32 & 1.40 & 
589.32 & 1.41 & \bf{589.30} & 
0.00\\CON3-5 & 569.88 & 1.49 & 
574.79 & 1.42 & \bf{563.70} & 
1.10\\CON3-6 & 505.26 & 1.80 & 
505.86 & 1.92 & \bf{499.20} & 
1.21\\CON3-7 & 578.41 & 1.20 & 
578.41 & 1.23 & \bf{577.50} & 
0.16\\CON3-8 & 524.30 & 1.18 & 
524.52 & 1.16 & \bf{523.10} & 
0.23\\CON3-9 & 588.48 & 1.24 & 
588.48 & 1.31 & \bf{578.20} & 
1.78\\CON8-0 & 879.00 & 1.43 & 
879.00 & 1.55 & \bf{858.90} & 
2.34\\CON8-1 & 754.98 & 1.36 & 
757.44 & 1.33 & \bf{740.90} & 
1.90\\CON8-2 & 716.53 & 1.93 & 
716.55 & 1.96 & \bf{714.30} & 
0.31\\CON8-3 & 817.57 & 1.34 & 
817.57 & 1.37 & \bf{812.30} & 
0.65\\CON8-4 & 789.98 & 1.44 & 
790.76 & 1.53 & \bf{770.10} & 
2.58\\CON8-5 & \bf{\underline{764.36}} & 1.35 & 
764.36 & 1.35 & 766.60 & 
-0.29\\CON8-6 & \bf{\underline{693.83}} & 1.70 & 
702.66 & 1.66 & 697.20 & 
-0.48\\CON8-7 & 822.42 & 1.12 & 
822.92 & 1.14 & \bf{814.80} & 
0.94\\CON8-8 & 799.16 & 1.45 & 
799.16 & 1.52 & \bf{771.30} & 
3.61\\CON8-9 & 816.12 & 1.45 & 
816.12 & 1.53 & \bf{815.10} & 
0.13\\[1ex]\hline
\end{tabular}
\label{table:nonlin}
\end{table} \clearpage
\begin{table}[ht]
\caption{Resultados de la ejecución de la metaheurística ACO, utilizando instancias de Dethloff con la configuración -n 2.0 -alpha 1.0 -beta 3.0 -q 2.2 -ro 0.015}
\centering
\small
\begin{tabular}{c c c c c c c}
\hline\hline
Instancia & Costo mínimo & Tiempo(seg.) & Costo promedio & Tiempo promedio(seg.) & Costo ACO & \%Gap \\ [0.5ex]
\hline
SCA3-0 & 640.55 & 1.35 & 
640.55 & 1.38 & \bf{636.10} & 
0.70\\SCA3-1 & \bf{\underline{697.84}} & 1.64 & 
698.76 & 1.55 & 700.10 & 
-0.32\\SCA3-2 & 659.34 & 1.25 & 
662.97 & 1.31 & \bf{659.30} & 
0.01\\SCA3-3 & 680.60 & 1.49 & 
680.96 & 1.50 & \bf{680.00} & 
0.09\\SCA3-4 & \bf{690.50} & 1.37 & 
690.50 & 1.40 & 690.50 & 0.00\\
SCA3-5 & \bf{\underline{665.04}} & 1.40 & 
665.39 & 1.54 & 671.10 & 
-0.90\\SCA3-6 & 655.19 & 1.34 & 
655.19 & 1.33 & \bf{651.10} & 
0.63\\SCA3-7 & 666.15 & 1.04 & 
666.15 & 1.00 & \bf{666.10} & 
0.01\\SCA3-8 & 721.45 & 1.18 & 
723.34 & 1.12 & \bf{719.50} & 
0.27\\SCA3-9 & \bf{681.00} & 1.04 & 
681.00 & 0.99 & 681.00 & 0.00\\
SCA8-0 & 991.07 & 1.51 & 
991.07 & 1.49 & \bf{961.60} & 
3.06\\SCA8-1 & 1074.65 & 1.19 & 
1074.65 & 1.18 & \bf{1063.00} & 
1.10\\SCA8-2 & 1056.87 & 0.98 & 
1056.87 & 1.01 & \bf{1040.60} & 
1.56\\SCA8-3 & 1031.08 & 1.48 & 
1031.08 & 1.47 & \bf{985.90} & 
4.58\\SCA8-4 & 1098.34 & 1.52 & 
1098.70 & 1.47 & \bf{1071.00} & 
2.55\\SCA8-5 & 1055.35 & 1.66 & 
1055.35 & 1.66 & \bf{1054.30} & 
0.10\\SCA8-6 & \bf{\underline{972.48}} & 1.70 & 
976.70 & 1.68 & 972.50 & 
-0.00\\SCA8-7 & 1092.57 & 1.60 & 
1092.57 & 1.61 & \bf{1059.70} & 
3.10\\SCA8-8 & 1085.93 & 1.43 & 
1090.50 & 1.43 & \bf{1082.70} & 
0.30\\SCA8-9 & \bf{\underline{1067.42}} & 1.16 & 
1067.42 & 1.16 & 1081.40 & 
-1.29\\CON3-0 & 624.96 & 1.52 & 
624.96 & 1.62 & \bf{616.50} & 
1.37\\CON3-1 & 557.38 & 1.47 & 
557.38 & 1.49 & \bf{555.60} & 
0.32\\CON3-2 & 524.07 & 1.06 & 
524.62 & 1.09 & \bf{521.40} & 
0.51\\CON3-3 & \bf{591.20} & 1.49 & 
592.65 & 1.49 & 591.20 & 0.00\\
CON3-4 & 589.32 & 1.37 & 
589.32 & 1.33 & \bf{589.30} & 
0.00\\CON3-5 & 576.43 & 1.39 & 
576.43 & 1.41 & \bf{563.70} & 
2.26\\CON3-6 & 504.15 & 1.74 & 
507.25 & 1.86 & \bf{499.20} & 
0.99\\CON3-7 & 578.41 & 1.14 & 
579.12 & 1.17 & \bf{577.50} & 
0.16\\CON3-8 & 524.30 & 1.18 & 
524.52 & 1.22 & \bf{523.10} & 
0.23\\CON3-9 & 588.48 & 1.33 & 
588.77 & 1.28 & \bf{578.20} & 
1.78\\CON8-0 & 879.00 & 1.56 & 
879.00 & 1.46 & \bf{858.90} & 
2.34\\CON8-1 & 758.26 & 1.34 & 
758.26 & 1.35 & \bf{740.90} & 
2.34\\CON8-2 & 716.53 & 2.25 & 
717.20 & 2.08 & \bf{714.30} & 
0.31\\CON8-3 & 817.57 & 1.41 & 
817.57 & 1.45 & \bf{812.30} & 
0.65\\CON8-4 & 781.64 & 1.47 & 
783.73 & 1.50 & \bf{770.10} & 
1.50\\CON8-5 & \bf{\underline{764.36}} & 1.32 & 
764.36 & 1.36 & 766.60 & 
-0.29\\CON8-6 & 705.61 & 1.63 & 
706.75 & 1.65 & \bf{697.20} & 
1.21\\CON8-7 & 822.42 & 1.10 & 
822.92 & 1.15 & \bf{814.80} & 
0.94\\CON8-8 & 799.16 & 1.51 & 
799.34 & 1.49 & \bf{771.30} & 
3.61\\CON8-9 & 816.12 & 1.56 & 
816.12 & 1.58 & \bf{815.10} & 
0.13\\[1ex]\hline
\end{tabular}
\label{table:nonlin}
\end{table} \clearpage
\begin{table}[ht]
\caption{Resultados de la ejecución de la metaheurística ACO, utilizando instancias de Dethloff con la configuración -n 2.0 -alpha 1.0 -beta 3.0 -q 2.3 -ro 0.015}
\centering
\small
\begin{tabular}{c c c c c c c}
\hline\hline
Instancia & Costo mínimo & Tiempo(seg.) & Costo promedio & Tiempo promedio(seg.) & Costo ACO & \%Gap \\ [0.5ex]
\hline
SCA3-0 & 640.55 & 1.30 & 
640.55 & 1.37 & \bf{636.10} & 
0.70\\SCA3-1 & \bf{\underline{697.84}} & 1.53 & 
698.76 & 1.45 & 700.10 & 
-0.32\\SCA3-2 & 659.34 & 1.32 & 
662.97 & 1.34 & \bf{659.30} & 
0.01\\SCA3-3 & 680.60 & 1.47 & 
680.96 & 1.46 & \bf{680.00} & 
0.09\\SCA3-4 & \bf{690.50} & 1.37 & 
690.50 & 1.45 & 690.50 & 0.00\\
SCA3-5 & \bf{\underline{665.04}} & 1.54 & 
665.19 & 1.45 & 671.10 & 
-0.90\\SCA3-6 & 655.19 & 1.30 & 
655.19 & 1.34 & \bf{651.10} & 
0.63\\SCA3-7 & 666.15 & 0.98 & 
666.15 & 1.01 & \bf{666.10} & 
0.01\\SCA3-8 & 721.45 & 1.10 & 
722.95 & 1.11 & \bf{719.50} & 
0.27\\SCA3-9 & \bf{681.00} & 0.97 & 
681.00 & 0.96 & 681.00 & 0.00\\
SCA8-0 & 991.07 & 1.56 & 
991.65 & 1.52 & \bf{961.60} & 
3.06\\SCA8-1 & 1074.65 & 1.24 & 
1074.65 & 1.18 & \bf{1063.00} & 
1.10\\SCA8-2 & 1056.87 & 1.04 & 
1056.87 & 1.02 & \bf{1040.60} & 
1.56\\SCA8-3 & 1031.08 & 1.50 & 
1031.08 & 1.42 & \bf{985.90} & 
4.58\\SCA8-4 & 1099.06 & 1.47 & 
1099.17 & 1.46 & \bf{1071.00} & 
2.62\\SCA8-5 & 1055.35 & 1.72 & 
1055.35 & 1.67 & \bf{1054.30} & 
0.10\\SCA8-6 & \bf{\underline{972.48}} & 1.69 & 
975.00 & 1.68 & 972.50 & 
-0.00\\SCA8-7 & 1092.57 & 1.51 & 
1092.57 & 1.56 & \bf{1059.70} & 
3.10\\SCA8-8 & 1091.49 & 1.39 & 
1091.89 & 1.44 & \bf{1082.70} & 
0.81\\SCA8-9 & \bf{\underline{1067.42}} & 1.16 & 
1067.42 & 1.16 & 1081.40 & 
-1.29\\CON3-0 & 624.96 & 1.60 & 
624.96 & 1.61 & \bf{616.50} & 
1.37\\CON3-1 & 557.38 & 1.43 & 
557.38 & 1.45 & \bf{555.60} & 
0.32\\CON3-2 & 525.02 & 1.17 & 
525.54 & 1.12 & \bf{521.40} & 
0.69\\CON3-3 & \bf{591.20} & 1.44 & 
593.09 & 1.48 & 591.20 & 0.00\\
CON3-4 & 589.32 & 1.38 & 
589.32 & 1.37 & \bf{589.30} & 
0.00\\CON3-5 & 569.15 & 1.48 & 
575.06 & 1.45 & \bf{563.70} & 
0.97\\CON3-6 & 505.26 & 1.75 & 
508.01 & 1.79 & \bf{499.20} & 
1.21\\CON3-7 & 578.41 & 1.18 & 
579.12 & 1.18 & \bf{577.50} & 
0.16\\CON3-8 & 524.30 & 1.14 & 
524.52 & 1.18 & \bf{523.10} & 
0.23\\CON3-9 & 588.48 & 1.28 & 
588.48 & 1.31 & \bf{578.20} & 
1.78\\CON8-0 & 879.00 & 1.43 & 
879.00 & 1.45 & \bf{858.90} & 
2.34\\CON8-1 & 754.98 & 1.34 & 
757.44 & 1.33 & \bf{740.90} & 
1.90\\CON8-2 & 716.53 & 2.08 & 
717.19 & 2.06 & \bf{714.30} & 
0.31\\CON8-3 & 817.57 & 1.36 & 
817.57 & 1.43 & \bf{812.30} & 
0.65\\CON8-4 & 778.60 & 1.44 & 
787.91 & 1.52 & \bf{770.10} & 
1.10\\CON8-5 & \bf{\underline{764.36}} & 1.38 & 
764.36 & 1.34 & 766.60 & 
-0.29\\CON8-6 & 705.61 & 1.68 & 
706.51 & 1.67 & \bf{697.20} & 
1.21\\CON8-7 & 822.42 & 1.20 & 
822.92 & 1.20 & \bf{814.80} & 
0.94\\CON8-8 & 799.32 & 1.51 & 
799.41 & 1.57 & \bf{771.30} & 
3.63\\CON8-9 & 816.12 & 1.60 & 
816.12 & 1.56 & \bf{815.10} & 
0.13\\[1ex]\hline
\end{tabular}
\label{table:nonlin}
\end{table} \clearpage
\begin{table}[ht]
\caption{Resultados de la ejecución de la metaheurística ACO, utilizando instancias de Dethloff con la configuración -n 2.0 -alpha 1.0 -beta 3.0 -q 2.4 -ro 0.015}
\centering
\small
\begin{tabular}{c c c c c c c}
\hline\hline
Instancia & Costo mínimo & Tiempo(seg.) & Costo promedio & Tiempo promedio(seg.) & Costo ACO & \%Gap \\ [0.5ex]
\hline
SCA3-0 & 640.55 & 1.35 & 
640.55 & 1.36 & \bf{636.10} & 
0.70\\SCA3-1 & \bf{\underline{697.84}} & 1.51 & 
698.76 & 1.48 & 700.10 & 
-0.32\\SCA3-2 & 659.34 & 1.30 & 
662.97 & 1.36 & \bf{659.30} & 
0.01\\SCA3-3 & 680.60 & 1.53 & 
680.78 & 1.47 & \bf{680.00} & 
0.09\\SCA3-4 & \bf{690.50} & 1.36 & 
690.50 & 1.37 & 690.50 & 0.00\\
SCA3-5 & \bf{\underline{665.04}} & 1.40 & 
669.08 & 1.40 & 671.10 & 
-0.90\\SCA3-6 & 655.19 & 1.30 & 
655.19 & 1.34 & \bf{651.10} & 
0.63\\SCA3-7 & 666.15 & 0.96 & 
666.15 & 0.99 & \bf{666.10} & 
0.01\\SCA3-8 & 721.45 & 1.07 & 
721.45 & 1.09 & \bf{719.50} & 
0.27\\SCA3-9 & \bf{681.00} & 0.94 & 
681.00 & 0.94 & 681.00 & 0.00\\
SCA8-0 & 991.07 & 1.77 & 
991.65 & 1.53 & \bf{961.60} & 
3.06\\SCA8-1 & 1074.65 & 1.10 & 
1074.68 & 1.18 & \bf{1063.00} & 
1.10\\SCA8-2 & 1056.87 & 1.03 & 
1056.87 & 0.99 & \bf{1040.60} & 
1.56\\SCA8-3 & 1031.08 & 1.50 & 
1031.08 & 1.47 & \bf{985.90} & 
4.58\\SCA8-4 & 1098.34 & 1.52 & 
1098.88 & 1.43 & \bf{1071.00} & 
2.55\\SCA8-5 & 1055.35 & 1.66 & 
1055.35 & 1.64 & \bf{1054.30} & 
0.10\\SCA8-6 & \bf{\underline{972.48}} & 1.67 & 
972.48 & 1.66 & 972.50 & 
-0.00\\SCA8-7 & 1092.57 & 1.50 & 
1092.57 & 1.57 & \bf{1059.70} & 
3.10\\SCA8-8 & 1091.49 & 1.48 & 
1091.89 & 1.46 & \bf{1082.70} & 
0.81\\SCA8-9 & \bf{\underline{1067.42}} & 1.13 & 
1067.42 & 1.14 & 1081.40 & 
-1.29\\CON3-0 & 624.96 & 1.55 & 
624.96 & 1.60 & \bf{616.50} & 
1.37\\CON3-1 & 557.38 & 1.43 & 
557.38 & 1.44 & \bf{555.60} & 
0.32\\CON3-2 & 524.07 & 1.05 & 
525.33 & 1.05 & \bf{521.40} & 
0.51\\CON3-3 & 594.11 & 1.47 & 
594.11 & 1.50 & \bf{591.20} & 
0.49\\CON3-4 & 589.32 & 1.36 & 
589.32 & 1.36 & \bf{589.30} & 
0.00\\CON3-5 & 576.43 & 1.30 & 
576.43 & 1.38 & \bf{563.70} & 
2.26\\CON3-6 & 504.15 & 1.74 & 
505.89 & 1.77 & \bf{499.20} & 
0.99\\CON3-7 & 578.41 & 1.18 & 
578.41 & 1.19 & \bf{577.50} & 
0.16\\CON3-8 & 524.30 & 1.19 & 
524.52 & 1.17 & \bf{523.10} & 
0.23\\CON3-9 & 588.48 & 1.22 & 
588.48 & 1.27 & \bf{578.20} & 
1.78\\CON8-0 & 879.00 & 1.56 & 
879.00 & 1.43 & \bf{858.90} & 
2.34\\CON8-1 & 758.26 & 1.30 & 
758.26 & 1.34 & \bf{740.90} & 
2.34\\CON8-2 & 716.53 & 1.84 & 
716.54 & 1.97 & \bf{714.30} & 
0.31\\CON8-3 & 817.57 & 1.39 & 
817.57 & 1.44 & \bf{812.30} & 
0.65\\CON8-4 & 778.60 & 1.61 & 
787.91 & 1.57 & \bf{770.10} & 
1.10\\CON8-5 & \bf{\underline{764.36}} & 1.31 & 
764.36 & 1.35 & 766.60 & 
-0.29\\CON8-6 & \bf{\underline{693.83}} & 1.67 & 
700.29 & 1.69 & 697.20 & 
-0.48\\CON8-7 & 822.42 & 1.26 & 
823.18 & 1.18 & \bf{814.80} & 
0.94\\CON8-8 & 799.16 & 1.54 & 
799.34 & 1.49 & \bf{771.30} & 
3.61\\CON8-9 & 816.12 & 1.45 & 
816.12 & 1.54 & \bf{815.10} & 
0.13\\[1ex]\hline
\end{tabular}
\label{table:nonlin}
\end{table} \clearpage
\begin{table}[ht]
\caption{Resultados de la ejecución de la metaheurística ACO, utilizando instancias de Dethloff con la configuración -n 2.0 -alpha 1.0 -beta 3.0 -q 2.5 -ro 0.015}
\centering
\small
\begin{tabular}{c c c c c c c}
\hline\hline
Instancia & Costo mínimo & Tiempo(seg.) & Costo promedio & Tiempo promedio(seg.) & Costo ACO & \%Gap \\ [0.5ex]
\hline
SCA3-0 & 640.55 & 1.43 & 
640.55 & 1.39 & \bf{636.10} & 
0.70\\SCA3-1 & \bf{\underline{697.84}} & 1.49 & 
697.84 & 1.47 & 700.10 & 
-0.32\\SCA3-2 & 664.18 & 1.23 & 
664.18 & 1.31 & \bf{659.30} & 
0.74\\SCA3-3 & 680.60 & 1.56 & 
681.13 & 1.48 & \bf{680.00} & 
0.09\\SCA3-4 & \bf{690.50} & 1.46 & 
690.50 & 1.43 & 690.50 & 0.00\\
SCA3-5 & \bf{\underline{665.04}} & 1.44 & 
665.39 & 1.38 & 671.10 & 
-0.90\\SCA3-6 & 655.19 & 1.40 & 
655.19 & 1.34 & \bf{651.10} & 
0.63\\SCA3-7 & 666.15 & 0.92 & 
666.15 & 0.99 & \bf{666.10} & 
0.01\\SCA3-8 & 721.45 & 1.13 & 
725.24 & 1.12 & \bf{719.50} & 
0.27\\SCA3-9 & \bf{681.00} & 0.96 & 
681.00 & 0.97 & 681.00 & 0.00\\
SCA8-0 & 991.07 & 1.55 & 
992.23 & 1.54 & \bf{961.60} & 
3.06\\SCA8-1 & 1074.65 & 1.23 & 
1074.65 & 1.20 & \bf{1063.00} & 
1.10\\SCA8-2 & 1056.87 & 0.95 & 
1056.87 & 1.00 & \bf{1040.60} & 
1.56\\SCA8-3 & 1031.08 & 1.47 & 
1031.08 & 1.40 & \bf{985.90} & 
4.58\\SCA8-4 & 1099.06 & 1.43 & 
1099.06 & 1.50 & \bf{1071.00} & 
2.62\\SCA8-5 & 1055.35 & 1.71 & 
1055.35 & 1.63 & \bf{1054.30} & 
0.10\\SCA8-6 & \bf{\underline{972.48}} & 1.68 & 
972.48 & 1.65 & 972.50 & 
-0.00\\SCA8-7 & 1092.57 & 1.68 & 
1092.57 & 1.67 & \bf{1059.70} & 
3.10\\SCA8-8 & 1091.49 & 1.48 & 
1091.89 & 1.43 & \bf{1082.70} & 
0.81\\SCA8-9 & \bf{\underline{1067.42}} & 1.16 & 
1067.42 & 1.15 & 1081.40 & 
-1.29\\CON3-0 & 624.96 & 1.62 & 
624.96 & 1.65 & \bf{616.50} & 
1.37\\CON3-1 & 557.38 & 1.44 & 
558.22 & 1.46 & \bf{555.60} & 
0.32\\CON3-2 & 524.07 & 1.10 & 
526.41 & 1.12 & \bf{521.40} & 
0.51\\CON3-3 & 594.11 & 1.53 & 
594.11 & 1.65 & \bf{591.20} & 
0.49\\CON3-4 & 589.32 & 1.32 & 
589.32 & 1.32 & \bf{589.30} & 
0.00\\CON3-5 & 574.57 & 1.47 & 
575.97 & 1.45 & \bf{563.70} & 
1.93\\CON3-6 & 505.26 & 1.78 & 
506.12 & 1.76 & \bf{499.20} & 
1.21\\CON3-7 & 578.41 & 1.18 & 
578.41 & 1.21 & \bf{577.50} & 
0.16\\CON3-8 & 524.59 & 1.12 & 
524.59 & 1.18 & \bf{523.10} & 
0.28\\CON3-9 & 588.48 & 1.49 & 
588.48 & 1.30 & \bf{578.20} & 
1.78\\CON8-0 & 879.00 & 1.48 & 
879.00 & 1.46 & \bf{858.90} & 
2.34\\CON8-1 & 758.26 & 1.32 & 
758.26 & 1.37 & \bf{740.90} & 
2.34\\CON8-2 & 716.53 & 2.10 & 
716.55 & 1.99 & \bf{714.30} & 
0.31\\CON8-3 & 817.57 & 1.36 & 
817.57 & 1.42 & \bf{812.30} & 
0.65\\CON8-4 & 778.60 & 1.44 & 
783.74 & 1.51 & \bf{770.10} & 
1.10\\CON8-5 & \bf{\underline{764.36}} & 1.36 & 
764.36 & 1.36 & 766.60 & 
-0.29\\CON8-6 & 705.61 & 1.66 & 
706.26 & 1.69 & \bf{697.20} & 
1.21\\CON8-7 & 822.42 & 1.20 & 
823.18 & 1.17 & \bf{814.80} & 
0.94\\CON8-8 & 799.16 & 1.56 & 
799.33 & 1.58 & \bf{771.30} & 
3.61\\CON8-9 & 816.12 & 1.62 & 
816.12 & 1.58 & \bf{815.10} & 
0.13\\[1ex]\hline
\end{tabular}
\label{table:nonlin}
\end{table} \clearpage
\begin{table}[ht]
\caption{Resultados de la ejecución de la metaheurística ACO, utilizando instancias de Dethloff con la configuración -n 2.0 -alpha 1.0 -beta 3.0 -q 2.6 -ro 0.015}
\centering
\small
\begin{tabular}{c c c c c c c}
\hline\hline
Instancia & Costo mínimo & Tiempo(seg.) & Costo promedio & Tiempo promedio(seg.) & Costo ACO & \%Gap \\ [0.5ex]
\hline
SCA3-0 & 640.55 & 1.37 & 
640.55 & 1.35 & \bf{636.10} & 
0.70\\SCA3-1 & \bf{\underline{697.84}} & 1.45 & 
697.84 & 1.44 & 700.10 & 
-0.32\\SCA3-2 & 659.34 & 1.41 & 
662.97 & 1.41 & \bf{659.30} & 
0.01\\SCA3-3 & 680.60 & 1.53 & 
680.78 & 1.47 & \bf{680.00} & 
0.09\\SCA3-4 & \bf{690.50} & 1.45 & 
690.50 & 1.41 & 690.50 & 0.00\\
SCA3-5 & \bf{\underline{665.04}} & 1.41 & 
668.74 & 1.40 & 671.10 & 
-0.90\\SCA3-6 & 655.19 & 1.28 & 
655.25 & 1.35 & \bf{651.10} & 
0.63\\SCA3-7 & 666.15 & 1.16 & 
666.15 & 1.00 & \bf{666.10} & 
0.01\\SCA3-8 & 721.45 & 1.14 & 
723.95 & 1.16 & \bf{719.50} & 
0.27\\SCA3-9 & \bf{681.00} & 1.02 & 
681.00 & 0.96 & 681.00 & 0.00\\
SCA8-0 & 991.07 & 1.58 & 
991.65 & 1.52 & \bf{961.60} & 
3.06\\SCA8-1 & 1074.65 & 1.20 & 
1074.65 & 1.23 & \bf{1063.00} & 
1.10\\SCA8-2 & 1056.87 & 1.01 & 
1056.87 & 1.03 & \bf{1040.60} & 
1.56\\SCA8-3 & 1031.08 & 1.49 & 
1031.08 & 1.49 & \bf{985.90} & 
4.58\\SCA8-4 & 1098.34 & 1.44 & 
1098.88 & 1.47 & \bf{1071.00} & 
2.55\\SCA8-5 & 1055.35 & 1.58 & 
1055.35 & 1.67 & \bf{1054.30} & 
0.10\\SCA8-6 & \bf{\underline{972.48}} & 1.66 & 
972.48 & 1.70 & 972.50 & 
-0.00\\SCA8-7 & 1092.57 & 1.63 & 
1092.57 & 1.61 & \bf{1059.70} & 
3.10\\SCA8-8 & 1091.49 & 1.47 & 
1091.89 & 1.45 & \bf{1082.70} & 
0.81\\SCA8-9 & \bf{\underline{1067.42}} & 1.18 & 
1067.42 & 1.15 & 1081.40 & 
-1.29\\CON3-0 & 624.96 & 1.57 & 
624.96 & 1.62 & \bf{616.50} & 
1.37\\CON3-1 & 557.38 & 1.42 & 
557.82 & 1.43 & \bf{555.60} & 
0.32\\CON3-2 & 524.07 & 1.05 & 
524.92 & 1.03 & \bf{521.40} & 
0.51\\CON3-3 & 592.95 & 1.55 & 
593.82 & 1.70 & \bf{591.20} & 
0.30\\CON3-4 & 589.32 & 1.42 & 
589.32 & 1.36 & \bf{589.30} & 
0.00\\CON3-5 & 576.43 & 1.46 & 
576.43 & 1.46 & \bf{563.70} & 
2.26\\CON3-6 & 505.26 & 1.82 & 
506.43 & 1.78 & \bf{499.20} & 
1.21\\CON3-7 & 578.41 & 1.25 & 
579.12 & 1.22 & \bf{577.50} & 
0.16\\CON3-8 & 524.59 & 1.28 & 
524.59 & 1.22 & \bf{523.10} & 
0.28\\CON3-9 & 588.48 & 1.32 & 
588.48 & 1.31 & \bf{578.20} & 
1.78\\CON8-0 & 879.00 & 1.41 & 
879.00 & 1.46 & \bf{858.90} & 
2.34\\CON8-1 & 758.26 & 1.35 & 
758.26 & 1.35 & \bf{740.90} & 
2.34\\CON8-2 & 716.53 & 2.00 & 
716.54 & 2.01 & \bf{714.30} & 
0.31\\CON8-3 & 817.57 & 1.50 & 
817.57 & 1.47 & \bf{812.30} & 
0.65\\CON8-4 & 781.64 & 1.49 & 
789.45 & 1.57 & \bf{770.10} & 
1.50\\CON8-5 & \bf{\underline{764.36}} & 1.40 & 
764.36 & 1.40 & 766.60 & 
-0.29\\CON8-6 & 705.61 & 1.79 & 
707.20 & 1.71 & \bf{697.20} & 
1.21\\CON8-7 & 822.42 & 1.20 & 
822.67 & 1.21 & \bf{814.80} & 
0.94\\CON8-8 & 799.16 & 1.63 & 
799.29 & 1.58 & \bf{771.30} & 
3.61\\CON8-9 & 816.12 & 1.65 & 
816.12 & 1.63 & \bf{815.10} & 
0.13\\[1ex]\hline
\end{tabular}
\label{table:nonlin}
\end{table} \clearpage
\begin{table}[ht]
\caption{Resultados de la ejecución de la metaheurística ACO, utilizando instancias de Dethloff con la configuración -n 2.0 -alpha 1.0 -beta 3.0 -q 2.7 -ro 0.015}
\centering
\small
\begin{tabular}{c c c c c c c}
\hline\hline
Instancia & Costo mínimo & Tiempo(seg.) & Costo promedio & Tiempo promedio(seg.) & Costo ACO & \%Gap \\ [0.5ex]
\hline
SCA3-0 & 640.55 & 1.34 & 
640.55 & 1.42 & \bf{636.10} & 
0.70\\SCA3-1 & \bf{\underline{697.84}} & 1.45 & 
697.84 & 1.49 & 700.10 & 
-0.32\\SCA3-2 & 659.34 & 1.36 & 
662.97 & 1.34 & \bf{659.30} & 
0.01\\SCA3-3 & 680.60 & 1.44 & 
680.78 & 1.45 & \bf{680.00} & 
0.09\\SCA3-4 & \bf{690.50} & 1.42 & 
690.50 & 1.35 & 690.50 & 0.00\\
SCA3-5 & \bf{\underline{665.04}} & 1.48 & 
665.04 & 1.42 & 671.10 & 
-0.90\\SCA3-6 & 655.19 & 1.71 & 
655.19 & 1.41 & \bf{651.10} & 
0.63\\SCA3-7 & 666.15 & 0.97 & 
666.15 & 1.00 & \bf{666.10} & 
0.01\\SCA3-8 & 721.45 & 1.16 & 
724.59 & 1.19 & \bf{719.50} & 
0.27\\SCA3-9 & \bf{681.00} & 0.94 & 
681.00 & 0.95 & 681.00 & 0.00\\
SCA8-0 & 991.07 & 1.43 & 
991.65 & 1.47 & \bf{961.60} & 
3.06\\SCA8-1 & 1074.65 & 1.24 & 
1074.65 & 1.20 & \bf{1063.00} & 
1.10\\SCA8-2 & 1056.87 & 0.94 & 
1056.87 & 1.00 & \bf{1040.60} & 
1.56\\SCA8-3 & 1031.08 & 1.43 & 
1031.08 & 1.46 & \bf{985.90} & 
4.58\\SCA8-4 & 1098.34 & 1.54 & 
1098.88 & 1.49 & \bf{1071.00} & 
2.55\\SCA8-5 & 1055.35 & 1.58 & 
1055.35 & 1.65 & \bf{1054.30} & 
0.10\\SCA8-6 & \bf{\underline{972.48}} & 1.61 & 
972.48 & 1.66 & 972.50 & 
-0.00\\SCA8-7 & 1092.57 & 1.59 & 
1092.57 & 1.56 & \bf{1059.70} & 
3.10\\SCA8-8 & 1092.02 & 1.42 & 
1092.02 & 1.50 & \bf{1082.70} & 
0.86\\SCA8-9 & \bf{\underline{1067.42}} & 1.10 & 
1067.42 & 1.10 & 1081.40 & 
-1.29\\CON3-0 & 624.96 & 1.56 & 
624.96 & 1.56 & \bf{616.50} & 
1.37\\CON3-1 & 557.38 & 1.45 & 
557.58 & 1.45 & \bf{555.60} & 
0.32\\CON3-2 & 524.07 & 1.12 & 
524.79 & 1.16 & \bf{521.40} & 
0.51\\CON3-3 & 594.11 & 1.62 & 
594.11 & 1.56 & \bf{591.20} & 
0.49\\CON3-4 & 589.32 & 1.32 & 
590.74 & 1.36 & \bf{589.30} & 
0.00\\CON3-5 & 576.43 & 1.48 & 
576.43 & 1.39 & \bf{563.70} & 
2.26\\CON3-6 & 505.26 & 1.84 & 
508.13 & 1.78 & \bf{499.20} & 
1.21\\CON3-7 & 578.41 & 1.26 & 
579.84 & 1.26 & \bf{577.50} & 
0.16\\CON3-8 & 524.30 & 1.22 & 
524.52 & 1.19 & \bf{523.10} & 
0.23\\CON3-9 & 588.48 & 1.30 & 
588.48 & 1.29 & \bf{578.20} & 
1.78\\CON8-0 & 879.00 & 1.40 & 
879.00 & 1.44 & \bf{858.90} & 
2.34\\CON8-1 & 758.26 & 1.37 & 
758.26 & 1.32 & \bf{740.90} & 
2.34\\CON8-2 & 716.53 & 1.90 & 
716.53 & 1.98 & \bf{714.30} & 
0.31\\CON8-3 & 817.57 & 1.38 & 
817.57 & 1.40 & \bf{812.30} & 
0.65\\CON8-4 & 781.64 & 1.52 & 
787.37 & 1.52 & \bf{770.10} & 
1.50\\CON8-5 & \bf{\underline{764.36}} & 1.36 & 
764.36 & 1.35 & 766.60 & 
-0.29\\CON8-6 & \bf{\underline{693.83}} & 1.73 & 
703.12 & 1.71 & 697.20 & 
-0.48\\CON8-7 & 822.42 & 1.16 & 
822.67 & 1.20 & \bf{814.80} & 
0.94\\CON8-8 & 799.16 & 1.55 & 
799.28 & 1.53 & \bf{771.30} & 
3.61\\CON8-9 & 816.12 & 1.58 & 
816.12 & 1.56 & \bf{815.10} & 
0.13\\[1ex]\hline
\end{tabular}
\label{table:nonlin}
\end{table} \clearpage
\begin{table}[ht]
\caption{Resultados de la ejecución de la metaheurística ACO, utilizando instancias de Dethloff con la configuración -n 2.0 -alpha 1.0 -beta 3.0 -q 2.8 -ro 0.015}
\centering
\small
\begin{tabular}{c c c c c c c}
\hline\hline
Instancia & Costo mínimo & Tiempo(seg.) & Costo promedio & Tiempo promedio(seg.) & Costo ACO & \%Gap \\ [0.5ex]
\hline
SCA3-0 & 640.55 & 1.32 & 
640.55 & 1.34 & \bf{636.10} & 
0.70\\SCA3-1 & \bf{\underline{697.84}} & 1.51 & 
699.68 & 1.46 & 700.10 & 
-0.32\\SCA3-2 & 659.34 & 1.30 & 
661.76 & 1.37 & \bf{659.30} & 
0.01\\SCA3-3 & 680.60 & 1.42 & 
680.78 & 1.41 & \bf{680.00} & 
0.09\\SCA3-4 & \bf{690.50} & 1.40 & 
690.50 & 1.48 & 690.50 & 0.00\\
SCA3-5 & \bf{\underline{665.04}} & 1.46 & 
665.04 & 1.47 & 671.10 & 
-0.90\\SCA3-6 & 655.19 & 1.32 & 
655.19 & 1.33 & \bf{651.10} & 
0.63\\SCA3-7 & 666.15 & 0.96 & 
666.15 & 0.94 & \bf{666.10} & 
0.01\\SCA3-8 & 721.45 & 1.05 & 
726.48 & 1.13 & \bf{719.50} & 
0.27\\SCA3-9 & \bf{681.00} & 0.97 & 
681.00 & 0.93 & 681.00 & 0.00\\
SCA8-0 & 991.07 & 1.46 & 
991.07 & 1.46 & \bf{961.60} & 
3.06\\SCA8-1 & 1069.40 & 1.18 & 
1073.34 & 1.18 & \bf{1063.00} & 
0.60\\SCA8-2 & 1056.87 & 1.04 & 
1056.87 & 1.49 & \bf{1040.60} & 
1.56\\SCA8-3 & 1031.08 & 1.48 & 
1031.08 & 1.46 & \bf{985.90} & 
4.58\\SCA8-4 & 1098.34 & 1.44 & 
1098.88 & 1.68 & \bf{1071.00} & 
2.55\\SCA8-5 & 1055.35 & 1.62 & 
1055.35 & 1.64 & \bf{1054.30} & 
0.10\\SCA8-6 & \bf{\underline{972.48}} & 1.71 & 
977.39 & 1.63 & 972.50 & 
-0.00\\SCA8-7 & 1092.57 & 1.59 & 
1092.57 & 1.61 & \bf{1059.70} & 
3.10\\SCA8-8 & 1091.49 & 1.51 & 
1091.89 & 1.47 & \bf{1082.70} & 
0.81\\SCA8-9 & \bf{\underline{1067.42}} & 1.12 & 
1067.42 & 1.15 & 1081.40 & 
-1.29\\CON3-0 & 624.96 & 1.64 & 
624.96 & 1.63 & \bf{616.50} & 
1.37\\CON3-1 & 557.38 & 1.44 & 
559.50 & 1.47 & \bf{555.60} & 
0.32\\CON3-2 & 524.07 & 1.22 & 
524.35 & 1.11 & \bf{521.40} & 
0.51\\CON3-3 & \bf{591.20} & 1.59 & 
593.38 & 1.53 & 591.20 & 0.00\\
CON3-4 & 589.32 & 1.41 & 
589.32 & 1.36 & \bf{589.30} & 
0.00\\CON3-5 & 569.15 & 1.40 & 
575.15 & 1.42 & \bf{563.70} & 
0.97\\CON3-6 & 505.26 & 1.79 & 
505.26 & 1.80 & \bf{499.20} & 
1.21\\CON3-7 & 578.41 & 1.26 & 
579.12 & 1.23 & \bf{577.50} & 
0.16\\CON3-8 & 524.59 & 1.23 & 
524.59 & 1.20 & \bf{523.10} & 
0.28\\CON3-9 & 588.48 & 1.37 & 
588.48 & 1.26 & \bf{578.20} & 
1.78\\CON8-0 & 879.00 & 1.43 & 
879.00 & 1.43 & \bf{858.90} & 
2.34\\CON8-1 & 758.26 & 1.40 & 
758.26 & 1.35 & \bf{740.90} & 
2.34\\CON8-2 & 716.53 & 2.00 & 
716.55 & 2.00 & \bf{714.30} & 
0.31\\CON8-3 & 817.57 & 1.45 & 
817.57 & 1.45 & \bf{812.30} & 
0.65\\CON8-4 & 789.98 & 1.52 & 
790.76 & 1.53 & \bf{770.10} & 
2.58\\CON8-5 & \bf{\underline{764.36}} & 1.28 & 
764.36 & 1.30 & 766.60 & 
-0.29\\CON8-6 & 705.61 & 1.66 & 
707.08 & 1.70 & \bf{697.20} & 
1.21\\CON8-7 & 822.42 & 1.16 & 
822.67 & 1.15 & \bf{814.80} & 
0.94\\CON8-8 & 799.32 & 1.56 & 
799.46 & 1.53 & \bf{771.30} & 
3.63\\CON8-9 & 816.12 & 1.57 & 
816.12 & 1.53 & \bf{815.10} & 
0.13\\[1ex]\hline
\end{tabular}
\label{table:nonlin}
\end{table} \clearpage
\begin{table}[ht]
\caption{Resultados de la ejecución de la metaheurística ACO, utilizando instancias de Dethloff con la configuración -n 2.0 -alpha 1.0 -beta 3.0 -q 2.9 -ro 0.015}
\centering
\small
\begin{tabular}{c c c c c c c}
\hline\hline
Instancia & Costo mínimo & Tiempo(seg.) & Costo promedio & Tiempo promedio(seg.) & Costo ACO & \%Gap \\ [0.5ex]
\hline
SCA3-0 & 640.55 & 1.34 & 
640.55 & 1.35 & \bf{636.10} & 
0.70\\SCA3-1 & \bf{\underline{697.84}} & 1.49 & 
697.84 & 1.49 & 700.10 & 
-0.32\\SCA3-2 & 659.34 & 1.22 & 
661.76 & 1.29 & \bf{659.30} & 
0.01\\SCA3-3 & 680.60 & 1.40 & 
680.96 & 1.42 & \bf{680.00} & 
0.09\\SCA3-4 & \bf{690.50} & 1.38 & 
690.50 & 1.39 & 690.50 & 0.00\\
SCA3-5 & \bf{\underline{665.04}} & 1.42 & 
668.60 & 1.44 & 671.10 & 
-0.90\\SCA3-6 & 653.69 & 1.35 & 
654.82 & 1.34 & \bf{651.10} & 
0.40\\SCA3-7 & 666.15 & 1.07 & 
666.15 & 1.03 & \bf{666.10} & 
0.01\\SCA3-8 & 721.45 & 1.10 & 
723.34 & 1.10 & \bf{719.50} & 
0.27\\SCA3-9 & \bf{681.00} & 0.91 & 
681.00 & 0.92 & 681.00 & 0.00\\
SCA8-0 & 991.07 & 1.53 & 
991.07 & 1.53 & \bf{961.60} & 
3.06\\SCA8-1 & 1069.40 & 1.13 & 
1073.40 & 1.14 & \bf{1063.00} & 
0.60\\SCA8-2 & 1056.87 & 1.03 & 
1056.87 & 1.02 & \bf{1040.60} & 
1.56\\SCA8-3 & 1031.08 & 1.53 & 
1031.08 & 1.47 & \bf{985.90} & 
4.58\\SCA8-4 & 1098.34 & 1.52 & 
1098.88 & 1.49 & \bf{1071.00} & 
2.55\\SCA8-5 & 1055.35 & 1.76 & 
1055.35 & 1.66 & \bf{1054.30} & 
0.10\\SCA8-6 & \bf{\underline{972.48}} & 1.69 & 
972.48 & 1.64 & 972.50 & 
-0.00\\SCA8-7 & 1092.57 & 1.65 & 
1092.57 & 1.63 & \bf{1059.70} & 
3.10\\SCA8-8 & 1087.30 & 1.40 & 
1090.84 & 1.42 & \bf{1082.70} & 
0.42\\SCA8-9 & \bf{\underline{1067.42}} & 1.11 & 
1067.42 & 1.09 & 1081.40 & 
-1.29\\CON3-0 & 624.96 & 1.61 & 
624.96 & 1.62 & \bf{616.50} & 
1.37\\CON3-1 & 557.38 & 1.52 & 
558.42 & 1.48 & \bf{555.60} & 
0.32\\CON3-2 & 524.07 & 1.11 & 
525.02 & 1.09 & \bf{521.40} & 
0.51\\CON3-3 & \bf{591.20} & 1.52 & 
593.09 & 1.52 & 591.20 & 0.00\\
CON3-4 & 589.32 & 1.24 & 
589.32 & 1.36 & \bf{589.30} & 
0.00\\CON3-5 & 576.43 & 1.34 & 
576.43 & 1.39 & \bf{563.70} & 
2.26\\CON3-6 & 505.26 & 1.83 & 
505.26 & 1.86 & \bf{499.20} & 
1.21\\CON3-7 & 578.41 & 1.30 & 
579.12 & 1.24 & \bf{577.50} & 
0.16\\CON3-8 & 524.30 & 1.15 & 
524.52 & 1.18 & \bf{523.10} & 
0.23\\CON3-9 & 588.48 & 1.32 & 
588.48 & 1.31 & \bf{578.20} & 
1.78\\CON8-0 & 871.80 & 1.33 & 
877.20 & 1.38 & \bf{858.90} & 
1.50\\CON8-1 & 758.26 & 1.30 & 
758.26 & 1.29 & \bf{740.90} & 
2.34\\CON8-2 & 716.53 & 1.99 & 
716.54 & 1.98 & \bf{714.30} & 
0.31\\CON8-3 & 817.57 & 1.45 & 
817.57 & 1.44 & \bf{812.30} & 
0.65\\CON8-4 & 778.60 & 1.63 & 
787.13 & 1.58 & \bf{770.10} & 
1.10\\CON8-5 & \bf{\underline{764.36}} & 1.40 & 
764.36 & 1.39 & 766.60 & 
-0.29\\CON8-6 & \bf{\underline{693.83}} & 1.76 & 
704.13 & 1.71 & 697.20 & 
-0.48\\CON8-7 & 822.42 & 1.19 & 
822.92 & 1.19 & \bf{814.80} & 
0.94\\CON8-8 & 799.32 & 2.67 & 
799.46 & 1.81 & \bf{771.30} & 
3.63\\CON8-9 & 816.12 & 1.59 & 
816.12 & 1.59 & \bf{815.10} & 
0.13\\[1ex]\hline
\end{tabular}
\label{table:nonlin}
\end{table} \clearpage
\begin{table}[ht]
\caption{Resultados de la ejecución de la metaheurística ACO, utilizando instancias de Dethloff con la configuración -n 2.0 -alpha 1.0 -beta 3.0 -q 3.0 -ro 0.015}
\centering
\small
\begin{tabular}{c c c c c c c}
\hline\hline
Instancia & Costo mínimo & Tiempo(seg.) & Costo promedio & Tiempo promedio(seg.) & Costo ACO & \%Gap \\ [0.5ex]
\hline
SCA3-0 & 640.55 & 1.32 & 
640.55 & 1.38 & \bf{636.10} & 
0.70\\SCA3-1 & \bf{\underline{697.84}} & 1.59 & 
697.84 & 1.52 & 700.10 & 
-0.32\\SCA3-2 & 664.18 & 1.33 & 
664.18 & 1.34 & \bf{659.30} & 
0.74\\SCA3-3 & 680.60 & 1.50 & 
680.78 & 1.50 & \bf{680.00} & 
0.09\\SCA3-4 & \bf{690.50} & 1.40 & 
690.50 & 1.43 & 690.50 & 0.00\\
SCA3-5 & \bf{\underline{665.04}} & 1.38 & 
665.19 & 1.38 & 671.10 & 
-0.90\\SCA3-6 & 655.19 & 1.39 & 
655.19 & 1.41 & \bf{651.10} & 
0.63\\SCA3-7 & 666.15 & 1.01 & 
666.15 & 1.02 & \bf{666.10} & 
0.01\\SCA3-8 & 721.45 & 1.10 & 
723.34 & 1.12 & \bf{719.50} & 
0.27\\SCA3-9 & \bf{681.00} & 1.11 & 
681.00 & 1.05 & 681.00 & 0.00\\
SCA8-0 & 991.07 & 1.52 & 
991.07 & 1.53 & \bf{961.60} & 
3.06\\SCA8-1 & 1074.65 & 1.18 & 
1074.68 & 1.16 & \bf{1063.00} & 
1.10\\SCA8-2 & 1056.87 & 1.04 & 
1056.87 & 1.02 & \bf{1040.60} & 
1.56\\SCA8-3 & 1031.08 & 1.45 & 
1031.08 & 1.44 & \bf{985.90} & 
4.58\\SCA8-4 & 1098.34 & 1.43 & 
1098.88 & 1.46 & \bf{1071.00} & 
2.55\\SCA8-5 & 1055.35 & 1.66 & 
1055.35 & 1.61 & \bf{1054.30} & 
0.10\\SCA8-6 & \bf{\underline{972.48}} & 1.72 & 
973.62 & 1.73 & 972.50 & 
-0.00\\SCA8-7 & 1092.57 & 1.61 & 
1092.57 & 1.65 & \bf{1059.70} & 
3.10\\SCA8-8 & 1092.02 & 1.48 & 
1092.02 & 1.46 & \bf{1082.70} & 
0.86\\SCA8-9 & \bf{\underline{1067.42}} & 1.10 & 
1067.42 & 1.12 & 1081.40 & 
-1.29\\CON3-0 & 624.96 & 1.65 & 
624.96 & 1.68 & \bf{616.50} & 
1.37\\CON3-1 & 557.38 & 1.38 & 
558.30 & 1.43 & \bf{555.60} & 
0.32\\CON3-2 & 524.07 & 1.03 & 
524.35 & 1.07 & \bf{521.40} & 
0.51\\CON3-3 & 594.11 & 1.62 & 
594.11 & 1.58 & \bf{591.20} & 
0.49\\CON3-4 & 589.32 & 1.29 & 
589.32 & 1.35 & \bf{589.30} & 
0.00\\CON3-5 & 576.43 & 1.38 & 
576.43 & 1.41 & \bf{563.70} & 
2.26\\CON3-6 & 505.26 & 1.74 & 
505.26 & 1.76 & \bf{499.20} & 
1.21\\CON3-7 & 578.41 & 1.16 & 
578.41 & 1.18 & \bf{577.50} & 
0.16\\CON3-8 & 524.30 & 1.12 & 
524.52 & 1.18 & \bf{523.10} & 
0.23\\CON3-9 & 588.48 & 1.19 & 
588.48 & 1.27 & \bf{578.20} & 
1.78\\CON8-0 & 879.00 & 1.33 & 
879.00 & 1.42 & \bf{858.90} & 
2.34\\CON8-1 & 758.26 & 1.35 & 
758.26 & 1.34 & \bf{740.90} & 
2.34\\CON8-2 & 716.53 & 1.86 & 
717.20 & 2.02 & \bf{714.30} & 
0.31\\CON8-3 & 817.57 & 1.46 & 
817.57 & 1.45 & \bf{812.30} & 
0.65\\CON8-4 & 793.09 & 1.57 & 
793.09 & 1.50 & \bf{770.10} & 
2.99\\CON8-5 & \bf{\underline{764.36}} & 1.34 & 
764.36 & 1.45 & 766.60 & 
-0.29\\CON8-6 & 705.61 & 1.67 & 
706.51 & 1.69 & \bf{697.20} & 
1.21\\CON8-7 & 822.42 & 1.16 & 
822.67 & 1.18 & \bf{814.80} & 
0.94\\CON8-8 & 799.16 & 1.43 & 
799.38 & 1.48 & \bf{771.30} & 
3.61\\CON8-9 & 816.12 & 1.52 & 
816.12 & 1.57 & \bf{815.10} & 
0.13\\[1ex]\hline
\end{tabular}
\label{table:nonlin}
\end{table} \clearpage
\begin{table}[ht]
\caption{Resultados de la ejecución de la metaheurística ACO, utilizando instancias de Dethloff con la configuración -n 2.0 -alpha 1.0 -beta 3.0 -q 3.1 -ro 0.015}
\centering
\small
\begin{tabular}{c c c c c c c}
\hline\hline
Instancia & Costo mínimo & Tiempo(seg.) & Costo promedio & Tiempo promedio(seg.) & Costo ACO & \%Gap \\ [0.5ex]
\hline
SCA3-0 & 640.55 & 1.34 & 
640.55 & 1.40 & \bf{636.10} & 
0.70\\SCA3-1 & \bf{\underline{697.84}} & 1.50 & 
697.84 & 1.51 & 700.10 & 
-0.32\\SCA3-2 & 659.34 & 1.38 & 
662.97 & 1.33 & \bf{659.30} & 
0.01\\SCA3-3 & 680.60 & 1.47 & 
680.78 & 1.54 & \bf{680.00} & 
0.09\\SCA3-4 & \bf{690.50} & 1.39 & 
690.50 & 1.50 & 690.50 & 0.00\\
SCA3-5 & \bf{\underline{665.04}} & 1.48 & 
665.34 & 1.42 & 671.10 & 
-0.90\\SCA3-6 & 655.19 & 1.30 & 
655.19 & 1.34 & \bf{651.10} & 
0.63\\SCA3-7 & 666.15 & 1.04 & 
666.15 & 1.00 & \bf{666.10} & 
0.01\\SCA3-8 & 721.45 & 1.12 & 
726.73 & 1.12 & \bf{719.50} & 
0.27\\SCA3-9 & \bf{681.00} & 0.97 & 
681.00 & 0.98 & 681.00 & 0.00\\
SCA8-0 & 991.07 & 1.59 & 
992.23 & 1.51 & \bf{961.60} & 
3.06\\SCA8-1 & 1069.40 & 1.19 & 
1073.34 & 1.20 & \bf{1063.00} & 
0.60\\SCA8-2 & 1056.87 & 1.03 & 
1056.87 & 1.07 & \bf{1040.60} & 
1.56\\SCA8-3 & 1031.08 & 1.50 & 
1031.08 & 1.44 & \bf{985.90} & 
4.58\\SCA8-4 & 1098.34 & 1.60 & 
1098.88 & 1.53 & \bf{1071.00} & 
2.55\\SCA8-5 & 1055.35 & 1.62 & 
1055.35 & 1.56 & \bf{1054.30} & 
0.10\\SCA8-6 & \bf{\underline{972.48}} & 1.96 & 
979.23 & 1.71 & 972.50 & 
-0.00\\SCA8-7 & 1092.57 & 1.50 & 
1092.57 & 2.53 & \bf{1059.70} & 
3.10\\SCA8-8 & 1091.49 & 1.48 & 
1091.89 & 1.45 & \bf{1082.70} & 
0.81\\SCA8-9 & \bf{\underline{1067.42}} & 1.15 & 
1067.42 & 1.15 & 1081.40 & 
-1.29\\CON3-0 & 624.96 & 1.57 & 
624.96 & 1.60 & \bf{616.50} & 
1.37\\CON3-1 & 557.38 & 1.55 & 
557.38 & 1.52 & \bf{555.60} & 
0.32\\CON3-2 & 524.07 & 1.07 & 
524.35 & 1.10 & \bf{521.40} & 
0.51\\CON3-3 & 594.11 & 1.43 & 
594.11 & 1.45 & \bf{591.20} & 
0.49\\CON3-4 & 589.32 & 1.27 & 
589.32 & 1.32 & \bf{589.30} & 
0.00\\CON3-5 & 576.43 & 1.38 & 
576.43 & 1.39 & \bf{563.70} & 
2.26\\CON3-6 & 504.15 & 1.75 & 
506.15 & 1.76 & \bf{499.20} & 
0.99\\CON3-7 & 578.41 & 1.28 & 
579.12 & 1.26 & \bf{577.50} & 
0.16\\CON3-8 & 524.30 & 1.14 & 
524.52 & 1.18 & \bf{523.10} & 
0.23\\CON3-9 & 588.48 & 1.39 & 
588.48 & 1.32 & \bf{578.20} & 
1.78\\CON8-0 & 879.00 & 1.44 & 
879.00 & 1.42 & \bf{858.90} & 
2.34\\CON8-1 & 758.26 & 1.36 & 
758.26 & 1.34 & \bf{740.90} & 
2.34\\CON8-2 & 716.53 & 1.96 & 
716.54 & 2.06 & \bf{714.30} & 
0.31\\CON8-3 & 817.57 & 1.44 & 
817.57 & 1.45 & \bf{812.30} & 
0.65\\CON8-4 & 778.60 & 1.50 & 
785.05 & 1.52 & \bf{770.10} & 
1.10\\CON8-5 & \bf{\underline{764.36}} & 1.37 & 
764.36 & 1.40 & 766.60 & 
-0.29\\CON8-6 & 705.61 & 1.69 & 
706.63 & 1.73 & \bf{697.20} & 
1.21\\CON8-7 & 823.43 & 1.18 & 
823.43 & 1.18 & \bf{814.80} & 
1.06\\CON8-8 & 799.32 & 1.48 & 
799.46 & 1.67 & \bf{771.30} & 
3.63\\CON8-9 & 816.12 & 1.53 & 
816.12 & 1.55 & \bf{815.10} & 
0.13\\[1ex]\hline
\end{tabular}
\label{table:nonlin}
\end{table} \clearpage
\begin{table}[ht]
\caption{Resultados de la ejecución de la metaheurística ACO, utilizando instancias de Dethloff con la configuración -n 2.0 -alpha 1.0 -beta 3.0 -q 3.2 -ro 0.015}
\centering
\small
\begin{tabular}{c c c c c c c}
\hline\hline
Instancia & Costo mínimo & Tiempo(seg.) & Costo promedio & Tiempo promedio(seg.) & Costo ACO & \%Gap \\ [0.5ex]
\hline
SCA3-0 & 640.55 & 1.32 & 
640.55 & 1.34 & \bf{636.10} & 
0.70\\SCA3-1 & \bf{\underline{697.84}} & 1.49 & 
698.76 & 1.50 & 700.10 & 
-0.32\\SCA3-2 & 664.18 & 1.37 & 
664.18 & 1.35 & \bf{659.30} & 
0.74\\SCA3-3 & 680.60 & 1.50 & 
680.60 & 1.48 & \bf{680.00} & 
0.09\\SCA3-4 & \bf{690.50} & 1.38 & 
690.50 & 1.63 & 690.50 & 0.00\\
SCA3-5 & \bf{\underline{665.04}} & 1.34 & 
665.04 & 1.40 & 671.10 & 
-0.90\\SCA3-6 & 655.19 & 1.32 & 
655.19 & 1.30 & \bf{651.10} & 
0.63\\SCA3-7 & 666.15 & 0.94 & 
666.15 & 0.97 & \bf{666.10} & 
0.01\\SCA3-8 & 721.45 & 1.10 & 
721.45 & 1.11 & \bf{719.50} & 
0.27\\SCA3-9 & \bf{681.00} & 1.02 & 
681.00 & 1.02 & 681.00 & 0.00\\
SCA8-0 & 991.07 & 1.46 & 
992.23 & 1.52 & \bf{961.60} & 
3.06\\SCA8-1 & 1074.65 & 1.20 & 
1074.68 & 1.19 & \bf{1063.00} & 
1.10\\SCA8-2 & 1056.87 & 1.02 & 
1056.87 & 1.02 & \bf{1040.60} & 
1.56\\SCA8-3 & 1031.08 & 1.31 & 
1031.08 & 1.38 & \bf{985.90} & 
4.58\\SCA8-4 & 1099.06 & 1.57 & 
1099.06 & 1.52 & \bf{1071.00} & 
2.62\\SCA8-5 & 1055.35 & 1.74 & 
1055.35 & 1.69 & \bf{1054.30} & 
0.10\\SCA8-6 & \bf{\underline{972.48}} & 1.62 & 
975.00 & 1.65 & 972.50 & 
-0.00\\SCA8-7 & 1092.57 & 1.68 & 
1092.57 & 1.62 & \bf{1059.70} & 
3.10\\SCA8-8 & 1092.02 & 1.35 & 
1092.02 & 1.38 & \bf{1082.70} & 
0.86\\SCA8-9 & \bf{\underline{1067.42}} & 1.10 & 
1067.42 & 1.14 & 1081.40 & 
-1.29\\CON3-0 & 624.96 & 1.60 & 
624.96 & 1.61 & \bf{616.50} & 
1.37\\CON3-1 & 557.38 & 1.48 & 
558.22 & 1.47 & \bf{555.60} & 
0.32\\CON3-2 & 524.07 & 1.19 & 
524.54 & 1.19 & \bf{521.40} & 
0.51\\CON3-3 & 594.11 & 1.54 & 
594.11 & 1.49 & \bf{591.20} & 
0.49\\CON3-4 & \bf{\underline{588.79}} & 1.29 & 
589.05 & 1.32 & 589.30 & 
-0.09\\CON3-5 & 576.43 & 1.63 & 
576.43 & 1.44 & \bf{563.70} & 
2.26\\CON3-6 & 505.26 & 1.82 & 
507.61 & 1.83 & \bf{499.20} & 
1.21\\CON3-7 & 578.41 & 1.21 & 
579.12 & 1.19 & \bf{577.50} & 
0.16\\CON3-8 & 524.30 & 1.20 & 
524.52 & 1.19 & \bf{523.10} & 
0.23\\CON3-9 & 588.48 & 1.27 & 
588.48 & 1.27 & \bf{578.20} & 
1.78\\CON8-0 & 879.00 & 1.42 & 
879.00 & 1.44 & \bf{858.90} & 
2.34\\CON8-1 & 758.26 & 1.33 & 
758.26 & 1.32 & \bf{740.90} & 
2.34\\CON8-2 & 716.53 & 2.10 & 
716.54 & 2.06 & \bf{714.30} & 
0.31\\CON8-3 & 817.57 & 1.44 & 
817.57 & 1.47 & \bf{812.30} & 
0.65\\CON8-4 & 781.64 & 1.47 & 
787.89 & 1.51 & \bf{770.10} & 
1.50\\CON8-5 & \bf{\underline{764.36}} & 1.36 & 
764.36 & 1.36 & 766.60 & 
-0.29\\CON8-6 & 705.61 & 1.71 & 
706.26 & 1.72 & \bf{697.20} & 
1.21\\CON8-7 & 822.42 & 1.13 & 
823.18 & 1.16 & \bf{814.80} & 
0.94\\CON8-8 & 799.16 & 1.65 & 
799.29 & 1.53 & \bf{771.30} & 
3.61\\CON8-9 & 816.12 & 1.63 & 
816.12 & 1.57 & \bf{815.10} & 
0.13\\[1ex]\hline
\end{tabular}
\label{table:nonlin}
\end{table} \clearpage
\begin{table}[ht]
\caption{Resultados de la ejecución de la metaheurística ACO, utilizando instancias de Dethloff con la configuración -n 2.0 -alpha 1.0 -beta 3.0 -q 3.3 -ro 0.015}
\centering
\small
\begin{tabular}{c c c c c c c}
\hline\hline
Instancia & Costo mínimo & Tiempo(seg.) & Costo promedio & Tiempo promedio(seg.) & Costo ACO & \%Gap \\ [0.5ex]
\hline
SCA3-0 & 640.55 & 1.31 & 
640.55 & 1.38 & \bf{636.10} & 
0.70\\SCA3-1 & \bf{\underline{697.84}} & 1.63 & 
697.84 & 1.51 & 700.10 & 
-0.32\\SCA3-2 & 664.18 & 1.25 & 
664.18 & 1.35 & \bf{659.30} & 
0.74\\SCA3-3 & 680.60 & 1.50 & 
681.13 & 1.55 & \bf{680.00} & 
0.09\\SCA3-4 & \bf{690.50} & 1.49 & 
690.50 & 1.42 & 690.50 & 0.00\\
SCA3-5 & \bf{\underline{665.04}} & 1.50 & 
665.04 & 1.41 & 671.10 & 
-0.90\\SCA3-6 & 655.19 & 1.33 & 
655.19 & 1.34 & \bf{651.10} & 
0.63\\SCA3-7 & 666.15 & 1.00 & 
666.15 & 1.02 & \bf{666.10} & 
0.01\\SCA3-8 & 721.45 & 1.06 & 
723.34 & 1.09 & \bf{719.50} & 
0.27\\SCA3-9 & \bf{681.00} & 0.93 & 
681.00 & 0.97 & 681.00 & 0.00\\
SCA8-0 & 991.07 & 1.53 & 
998.34 & 1.49 & \bf{961.60} & 
3.06\\SCA8-1 & 1074.65 & 1.19 & 
1074.65 & 1.20 & \bf{1063.00} & 
1.10\\SCA8-2 & 1056.87 & 0.98 & 
1056.87 & 1.00 & \bf{1040.60} & 
1.56\\SCA8-3 & 1031.08 & 1.43 & 
1031.08 & 1.45 & \bf{985.90} & 
4.58\\SCA8-4 & 1099.06 & 1.50 & 
1099.06 & 1.50 & \bf{1071.00} & 
2.62\\SCA8-5 & 1055.35 & 1.62 & 
1055.35 & 1.62 & \bf{1054.30} & 
0.10\\SCA8-6 & \bf{\underline{972.48}} & 1.64 & 
972.48 & 1.68 & 972.50 & 
-0.00\\SCA8-7 & 1092.57 & 1.62 & 
1092.57 & 1.61 & \bf{1059.70} & 
3.10\\SCA8-8 & 1091.49 & 1.46 & 
1091.89 & 1.43 & \bf{1082.70} & 
0.81\\SCA8-9 & \bf{\underline{1067.42}} & 1.19 & 
1067.42 & 1.15 & 1081.40 & 
-1.29\\CON3-0 & 624.96 & 1.70 & 
624.96 & 1.64 & \bf{616.50} & 
1.37\\CON3-1 & 557.38 & 1.42 & 
558.42 & 1.43 & \bf{555.60} & 
0.32\\CON3-2 & 524.07 & 1.04 & 
524.35 & 1.09 & \bf{521.40} & 
0.51\\CON3-3 & 594.11 & 1.45 & 
594.11 & 1.46 & \bf{591.20} & 
0.49\\CON3-4 & 589.32 & 1.33 & 
589.32 & 1.33 & \bf{589.30} & 
0.00\\CON3-5 & 576.43 & 1.47 & 
576.43 & 1.75 & \bf{563.70} & 
2.26\\CON3-6 & 505.26 & 1.81 & 
505.86 & 1.82 & \bf{499.20} & 
1.21\\CON3-7 & 578.41 & 1.21 & 
578.41 & 1.23 & \bf{577.50} & 
0.16\\CON3-8 & 524.30 & 1.16 & 
524.52 & 1.15 & \bf{523.10} & 
0.23\\CON3-9 & 588.48 & 1.26 & 
588.48 & 1.31 & \bf{578.20} & 
1.78\\CON8-0 & 879.00 & 1.70 & 
879.00 & 1.48 & \bf{858.90} & 
2.34\\CON8-1 & 758.26 & 1.27 & 
758.26 & 1.33 & \bf{740.90} & 
2.34\\CON8-2 & 716.53 & 1.95 & 
716.55 & 1.99 & \bf{714.30} & 
0.31\\CON8-3 & 817.57 & 1.44 & 
817.57 & 1.43 & \bf{812.30} & 
0.65\\CON8-4 & 781.64 & 1.56 & 
789.45 & 1.54 & \bf{770.10} & 
1.50\\CON8-5 & \bf{\underline{764.36}} & 1.34 & 
764.36 & 1.39 & 766.60 & 
-0.29\\CON8-6 & 705.61 & 1.70 & 
707.20 & 1.68 & \bf{697.20} & 
1.21\\CON8-7 & 822.72 & 1.21 & 
823.25 & 1.18 & \bf{814.80} & 
0.97\\CON8-8 & 799.32 & 1.54 & 
799.46 & 1.55 & \bf{771.30} & 
3.63\\CON8-9 & 816.12 & 1.50 & 
816.12 & 1.55 & \bf{815.10} & 
0.13\\[1ex]\hline
\end{tabular}
\label{table:nonlin}
\end{table} \clearpage
\begin{table}[ht]
\caption{Resultados de la ejecución de la metaheurística ACO, utilizando instancias de Dethloff con la configuración -n 2.0 -alpha 1.0 -beta 3.0 -q 3.4 -ro 0.015}
\centering
\small
\begin{tabular}{c c c c c c c}
\hline\hline
Instancia & Costo mínimo & Tiempo(seg.) & Costo promedio & Tiempo promedio(seg.) & Costo ACO & \%Gap \\ [0.5ex]
\hline
SCA3-0 & 640.55 & 1.45 & 
640.84 & 1.42 & \bf{636.10} & 
0.70\\SCA3-1 & \bf{\underline{697.84}} & 1.54 & 
697.84 & 1.46 & 700.10 & 
-0.32\\SCA3-2 & 664.18 & 1.36 & 
665.45 & 1.31 & \bf{659.30} & 
0.74\\SCA3-3 & 680.60 & 1.51 & 
681.13 & 1.49 & \bf{680.00} & 
0.09\\SCA3-4 & \bf{690.50} & 1.40 & 
690.50 & 1.35 & 690.50 & 0.00\\
SCA3-5 & \bf{\underline{665.04}} & 1.47 & 
665.04 & 1.42 & 671.10 & 
-0.90\\SCA3-6 & 655.19 & 1.41 & 
655.30 & 1.45 & \bf{651.10} & 
0.63\\SCA3-7 & 666.15 & 0.94 & 
666.15 & 0.96 & \bf{666.10} & 
0.01\\SCA3-8 & 721.45 & 1.12 & 
725.24 & 1.16 & \bf{719.50} & 
0.27\\SCA3-9 & \bf{681.00} & 0.96 & 
681.00 & 0.99 & 681.00 & 0.00\\
SCA8-0 & 991.07 & 1.56 & 
991.65 & 1.59 & \bf{961.60} & 
3.06\\SCA8-1 & 1074.65 & 1.18 & 
1074.65 & 1.19 & \bf{1063.00} & 
1.10\\SCA8-2 & 1056.87 & 0.95 & 
1056.87 & 0.99 & \bf{1040.60} & 
1.56\\SCA8-3 & 1031.08 & 1.47 & 
1031.08 & 1.42 & \bf{985.90} & 
4.58\\SCA8-4 & 1099.06 & 1.41 & 
1099.06 & 1.48 & \bf{1071.00} & 
2.62\\SCA8-5 & 1055.35 & 1.72 & 
1055.35 & 1.70 & \bf{1054.30} & 
0.10\\SCA8-6 & \bf{\underline{972.48}} & 1.72 & 
972.48 & 1.69 & 972.50 & 
-0.00\\SCA8-7 & 1092.57 & 1.59 & 
1092.57 & 1.60 & \bf{1059.70} & 
3.10\\SCA8-8 & \bf{\underline{1082.11}} & 1.37 & 
1089.54 & 1.41 & 1082.70 & 
-0.05\\SCA8-9 & \bf{\underline{1067.42}} & 1.15 & 
1067.42 & 1.20 & 1081.40 & 
-1.29\\CON3-0 & 624.96 & 1.62 & 
624.96 & 1.56 & \bf{616.50} & 
1.37\\CON3-1 & 557.38 & 1.42 & 
557.82 & 1.44 & \bf{555.60} & 
0.32\\CON3-2 & 524.07 & 1.03 & 
524.51 & 1.08 & \bf{521.40} & 
0.51\\CON3-3 & 594.11 & 1.52 & 
594.11 & 1.53 & \bf{591.20} & 
0.49\\CON3-4 & \bf{\underline{588.79}} & 1.37 & 
589.19 & 1.38 & 589.30 & 
-0.09\\CON3-5 & 569.88 & 1.34 & 
573.15 & 1.37 & \bf{563.70} & 
1.10\\CON3-6 & 505.26 & 1.90 & 
506.84 & 1.76 & \bf{499.20} & 
1.21\\CON3-7 & 578.41 & 1.13 & 
579.12 & 1.18 & \bf{577.50} & 
0.16\\CON3-8 & 524.59 & 1.14 & 
524.59 & 1.19 & \bf{523.10} & 
0.28\\CON3-9 & 588.48 & 1.30 & 
588.48 & 1.26 & \bf{578.20} & 
1.78\\CON8-0 & 879.00 & 1.44 & 
879.00 & 1.42 & \bf{858.90} & 
2.34\\CON8-1 & 758.26 & 1.40 & 
758.26 & 1.37 & \bf{740.90} & 
2.34\\CON8-2 & 716.53 & 2.02 & 
716.55 & 2.00 & \bf{714.30} & 
0.31\\CON8-3 & 817.57 & 1.44 & 
817.57 & 1.41 & \bf{812.30} & 
0.65\\CON8-4 & 793.09 & 1.59 & 
793.09 & 1.59 & \bf{770.10} & 
2.99\\CON8-5 & \bf{\underline{764.36}} & 1.31 & 
764.36 & 1.36 & 766.60 & 
-0.29\\CON8-6 & 705.61 & 1.66 & 
706.75 & 1.67 & \bf{697.20} & 
1.21\\CON8-7 & 822.42 & 1.20 & 
822.92 & 1.18 & \bf{814.80} & 
0.94\\CON8-8 & 799.32 & 1.50 & 
799.37 & 1.50 & \bf{771.30} & 
3.63\\CON8-9 & 816.12 & 1.56 & 
817.40 & 1.56 & \bf{815.10} & 
0.13\\[1ex]\hline
\end{tabular}
\label{table:nonlin}
\end{table} \clearpage
\begin{table}[ht]
\caption{Resultados de la ejecución de la metaheurística ACO, utilizando instancias de Dethloff con la configuración -n 2.0 -alpha 1.0 -beta 3.0 -q 3.5 -ro 0.015}
\centering
\small
\begin{tabular}{c c c c c c c}
\hline\hline
Instancia & Costo mínimo & Tiempo(seg.) & Costo promedio & Tiempo promedio(seg.) & Costo ACO & \%Gap \\ [0.5ex]
\hline
SCA3-0 & 640.55 & 1.42 & 
640.55 & 1.35 & \bf{636.10} & 
0.70\\SCA3-1 & \bf{\underline{697.84}} & 1.50 & 
697.84 & 1.54 & 700.10 & 
-0.32\\SCA3-2 & 659.34 & 1.42 & 
662.97 & 1.40 & \bf{659.30} & 
0.01\\SCA3-3 & 680.60 & 1.47 & 
680.78 & 1.49 & \bf{680.00} & 
0.09\\SCA3-4 & \bf{690.50} & 1.38 & 
690.50 & 1.45 & 690.50 & 0.00\\
SCA3-5 & \bf{\underline{665.04}} & 1.42 & 
672.15 & 1.43 & 671.10 & 
-0.90\\SCA3-6 & 655.19 & 1.44 & 
655.19 & 1.36 & \bf{651.10} & 
0.63\\SCA3-7 & 666.15 & 0.93 & 
666.15 & 0.97 & \bf{666.10} & 
0.01\\SCA3-8 & 721.45 & 1.25 & 
724.19 & 1.14 & \bf{719.50} & 
0.27\\SCA3-9 & \bf{681.00} & 1.03 & 
681.00 & 0.97 & 681.00 & 0.00\\
SCA8-0 & 991.07 & 1.49 & 
991.65 & 1.50 & \bf{961.60} & 
3.06\\SCA8-1 & 1074.65 & 1.18 & 
1074.65 & 1.21 & \bf{1063.00} & 
1.10\\SCA8-2 & 1056.87 & 1.05 & 
1056.87 & 1.03 & \bf{1040.60} & 
1.56\\SCA8-3 & 1031.08 & 1.56 & 
1031.08 & 1.50 & \bf{985.90} & 
4.58\\SCA8-4 & 1099.06 & 1.39 & 
1099.06 & 1.49 & \bf{1071.00} & 
2.62\\SCA8-5 & 1055.35 & 1.64 & 
1055.35 & 1.67 & \bf{1054.30} & 
0.10\\SCA8-6 & \bf{\underline{972.48}} & 1.70 & 
972.48 & 1.76 & 972.50 & 
-0.00\\SCA8-7 & 1092.57 & 1.60 & 
1092.57 & 1.64 & \bf{1059.70} & 
3.10\\SCA8-8 & 1092.02 & 1.47 & 
1092.02 & 1.44 & \bf{1082.70} & 
0.86\\SCA8-9 & \bf{\underline{1067.42}} & 1.18 & 
1067.42 & 1.17 & 1081.40 & 
-1.29\\CON3-0 & 624.96 & 1.59 & 
624.96 & 1.62 & \bf{616.50} & 
1.37\\CON3-1 & 557.38 & 1.42 & 
559.07 & 1.47 & \bf{555.60} & 
0.32\\CON3-2 & 524.07 & 1.24 & 
525.40 & 1.14 & \bf{521.40} & 
0.51\\CON3-3 & 594.11 & 1.46 & 
594.11 & 1.46 & \bf{591.20} & 
0.49\\CON3-4 & 589.32 & 1.27 & 
589.32 & 1.36 & \bf{589.30} & 
0.00\\CON3-5 & 576.43 & 1.42 & 
576.43 & 1.48 & \bf{563.70} & 
2.26\\CON3-6 & 505.26 & 1.76 & 
505.26 & 1.82 & \bf{499.20} & 
1.21\\CON3-7 & 578.41 & 1.25 & 
578.41 & 1.21 & \bf{577.50} & 
0.16\\CON3-8 & 524.30 & 1.22 & 
524.52 & 1.22 & \bf{523.10} & 
0.23\\CON3-9 & 588.48 & 1.17 & 
588.48 & 1.24 & \bf{578.20} & 
1.78\\CON8-0 & 879.00 & 1.45 & 
879.00 & 1.46 & \bf{858.90} & 
2.34\\CON8-1 & 758.26 & 1.36 & 
758.28 & 1.33 & \bf{740.90} & 
2.34\\CON8-2 & 716.53 & 2.00 & 
716.55 & 1.94 & \bf{714.30} & 
0.31\\CON8-3 & 817.57 & 1.42 & 
817.57 & 1.45 & \bf{812.30} & 
0.65\\CON8-4 & 781.64 & 1.48 & 
787.89 & 1.54 & \bf{770.10} & 
1.50\\CON8-5 & \bf{\underline{764.36}} & 1.42 & 
764.36 & 1.39 & 766.60 & 
-0.29\\CON8-6 & 705.61 & 1.71 & 
707.31 & 1.71 & \bf{697.20} & 
1.21\\CON8-7 & 822.42 & 1.21 & 
822.78 & 1.18 & \bf{814.80} & 
0.94\\CON8-8 & 799.32 & 1.62 & 
799.37 & 1.55 & \bf{771.30} & 
3.63\\CON8-9 & 816.12 & 1.57 & 
816.12 & 1.55 & \bf{815.10} & 
0.13\\[1ex]\hline
\end{tabular}
\label{table:nonlin}
\end{table} \clearpage
\begin{table}[ht]
\caption{Resultados de la ejecución de la metaheurística ACO, utilizando instancias de Dethloff con la configuración -n 2.0 -alpha 1.0 -beta 3.0 -q 3.6 -ro 0.015}
\centering
\small
\begin{tabular}{c c c c c c c}
\hline\hline
Instancia & Costo mínimo & Tiempo(seg.) & Costo promedio & Tiempo promedio(seg.) & Costo ACO & \%Gap \\ [0.5ex]
\hline
SCA3-0 & 640.55 & 1.36 & 
640.55 & 1.36 & \bf{636.10} & 
0.70\\SCA3-1 & \bf{\underline{697.84}} & 1.47 & 
697.84 & 1.50 & 700.10 & 
-0.32\\SCA3-2 & 659.34 & 1.30 & 
662.97 & 1.35 & \bf{659.30} & 
0.01\\SCA3-3 & 680.60 & 1.41 & 
680.96 & 1.48 & \bf{680.00} & 
0.09\\SCA3-4 & \bf{690.50} & 1.35 & 
690.50 & 1.44 & 690.50 & 0.00\\
SCA3-5 & \bf{\underline{665.04}} & 1.32 & 
668.95 & 1.37 & 671.10 & 
-0.90\\SCA3-6 & 655.19 & 1.31 & 
655.19 & 1.36 & \bf{651.10} & 
0.63\\SCA3-7 & 666.15 & 0.95 & 
666.15 & 1.04 & \bf{666.10} & 
0.01\\SCA3-8 & 721.45 & 1.12 & 
725.24 & 1.11 & \bf{719.50} & 
0.27\\SCA3-9 & \bf{681.00} & 1.03 & 
681.00 & 0.99 & 681.00 & 0.00\\
SCA8-0 & 991.07 & 1.59 & 
991.65 & 1.52 & \bf{961.60} & 
3.06\\SCA8-1 & 1074.65 & 1.19 & 
1074.68 & 1.21 & \bf{1063.00} & 
1.10\\SCA8-2 & 1056.87 & 1.67 & 
1056.87 & 1.17 & \bf{1040.60} & 
1.56\\SCA8-3 & 1031.08 & 1.48 & 
1031.08 & 1.46 & \bf{985.90} & 
4.58\\SCA8-4 & 1099.06 & 1.62 & 
1099.06 & 1.58 & \bf{1071.00} & 
2.62\\SCA8-5 & 1055.35 & 1.62 & 
1055.35 & 1.68 & \bf{1054.30} & 
0.10\\SCA8-6 & \bf{\underline{972.48}} & 1.68 & 
972.48 & 1.65 & 972.50 & 
-0.00\\SCA8-7 & 1092.57 & 1.70 & 
1092.57 & 1.61 & \bf{1059.70} & 
3.10\\SCA8-8 & 1091.49 & 1.36 & 
1091.76 & 1.42 & \bf{1082.70} & 
0.81\\SCA8-9 & \bf{\underline{1067.42}} & 1.18 & 
1067.42 & 1.17 & 1081.40 & 
-1.29\\CON3-0 & 624.96 & 1.61 & 
624.96 & 1.64 & \bf{616.50} & 
1.37\\CON3-1 & 557.38 & 1.45 & 
558.30 & 1.48 & \bf{555.60} & 
0.32\\CON3-2 & 524.07 & 1.28 & 
525.20 & 1.19 & \bf{521.40} & 
0.51\\CON3-3 & \bf{591.20} & 1.45 & 
591.93 & 1.48 & 591.20 & 0.00\\
CON3-4 & 589.32 & 1.34 & 
589.32 & 1.33 & \bf{589.30} & 
0.00\\CON3-5 & 576.43 & 1.46 & 
576.43 & 1.42 & \bf{563.70} & 
2.26\\CON3-6 & 504.15 & 1.85 & 
506.15 & 1.77 & \bf{499.20} & 
0.99\\CON3-7 & 578.41 & 1.21 & 
579.12 & 1.19 & \bf{577.50} & 
0.16\\CON3-8 & 524.59 & 1.25 & 
524.59 & 1.21 & \bf{523.10} & 
0.28\\CON3-9 & 588.48 & 1.22 & 
588.48 & 1.27 & \bf{578.20} & 
1.78\\CON8-0 & 879.00 & 1.46 & 
879.00 & 1.44 & \bf{858.90} & 
2.34\\CON8-1 & 758.26 & 1.57 & 
758.30 & 1.39 & \bf{740.90} & 
2.34\\CON8-2 & 716.53 & 1.91 & 
716.54 & 1.98 & \bf{714.30} & 
0.31\\CON8-3 & 817.57 & 1.41 & 
817.57 & 1.42 & \bf{812.30} & 
0.65\\CON8-4 & 778.60 & 1.57 & 
785.83 & 1.53 & \bf{770.10} & 
1.10\\CON8-5 & \bf{\underline{764.36}} & 1.36 & 
764.36 & 1.43 & 766.60 & 
-0.29\\CON8-6 & \bf{\underline{693.83}} & 1.66 & 
704.25 & 1.67 & 697.20 & 
-0.48\\CON8-7 & 822.42 & 1.17 & 
822.75 & 1.16 & \bf{814.80} & 
0.94\\CON8-8 & 799.16 & 1.63 & 
799.29 & 1.56 & \bf{771.30} & 
3.61\\CON8-9 & 816.12 & 1.56 & 
816.12 & 1.53 & \bf{815.10} & 
0.13\\[1ex]\hline
\end{tabular}
\label{table:nonlin}
\end{table} \clearpage
\begin{table}[ht]
\caption{Resultados de la ejecución de la metaheurística ACO, utilizando instancias de Dethloff con la configuración -n 2.0 -alpha 1.0 -beta 3.0 -q 3.7 -ro 0.015}
\centering
\small
\begin{tabular}{c c c c c c c}
\hline\hline
Instancia & Costo mínimo & Tiempo(seg.) & Costo promedio & Tiempo promedio(seg.) & Costo ACO & \%Gap \\ [0.5ex]
\hline
SCA3-0 & 640.55 & 1.33 & 
640.55 & 1.39 & \bf{636.10} & 
0.70\\SCA3-1 & \bf{\underline{697.84}} & 1.54 & 
697.84 & 1.56 & 700.10 & 
-0.32\\SCA3-2 & 659.34 & 1.38 & 
661.76 & 1.32 & \bf{659.30} & 
0.01\\SCA3-3 & 680.60 & 1.46 & 
680.78 & 1.46 & \bf{680.00} & 
0.09\\SCA3-4 & \bf{690.50} & 1.37 & 
690.50 & 1.43 & 690.50 & 0.00\\
SCA3-5 & \bf{\underline{659.90}} & 1.40 & 
663.75 & 1.40 & 671.10 & 
-1.67\\SCA3-6 & 655.19 & 1.42 & 
655.19 & 1.36 & \bf{651.10} & 
0.63\\SCA3-7 & 666.15 & 1.04 & 
666.15 & 1.02 & \bf{666.10} & 
0.01\\SCA3-8 & 721.45 & 1.07 & 
726.09 & 1.12 & \bf{719.50} & 
0.27\\SCA3-9 & \bf{681.00} & 0.94 & 
681.00 & 0.94 & 681.00 & 0.00\\
SCA8-0 & 991.07 & 1.54 & 
991.07 & 1.50 & \bf{961.60} & 
3.06\\SCA8-1 & 1069.40 & 1.19 & 
1073.37 & 1.22 & \bf{1063.00} & 
0.60\\SCA8-2 & 1056.87 & 1.04 & 
1056.87 & 1.03 & \bf{1040.60} & 
1.56\\SCA8-3 & 1031.08 & 1.41 & 
1031.08 & 1.45 & \bf{985.90} & 
4.58\\SCA8-4 & 1099.06 & 1.59 & 
1099.06 & 1.48 & \bf{1071.00} & 
2.62\\SCA8-5 & 1055.35 & 1.60 & 
1055.35 & 1.65 & \bf{1054.30} & 
0.10\\SCA8-6 & \bf{\underline{972.48}} & 1.67 & 
975.00 & 1.67 & 972.50 & 
-0.00\\SCA8-7 & 1092.57 & 1.54 & 
1092.57 & 1.60 & \bf{1059.70} & 
3.10\\SCA8-8 & 1092.02 & 1.45 & 
1092.02 & 1.44 & \bf{1082.70} & 
0.86\\SCA8-9 & \bf{\underline{1067.42}} & 1.18 & 
1067.42 & 1.16 & 1081.40 & 
-1.29\\CON3-0 & 624.96 & 1.60 & 
624.96 & 1.64 & \bf{616.50} & 
1.37\\CON3-1 & 557.38 & 1.58 & 
557.38 & 1.45 & \bf{555.60} & 
0.32\\CON3-2 & 524.07 & 1.06 & 
524.07 & 1.08 & \bf{521.40} & 
0.51\\CON3-3 & 594.11 & 1.47 & 
594.11 & 1.53 & \bf{591.20} & 
0.49\\CON3-4 & \bf{\underline{588.79}} & 1.26 & 
589.19 & 1.33 & 589.30 & 
-0.09\\CON3-5 & 576.43 & 1.34 & 
576.43 & 1.39 & \bf{563.70} & 
2.26\\CON3-6 & 504.15 & 1.77 & 
506.18 & 1.74 & \bf{499.20} & 
0.99\\CON3-7 & 578.41 & 1.20 & 
579.84 & 1.21 & \bf{577.50} & 
0.16\\CON3-8 & 524.30 & 1.22 & 
524.52 & 1.19 & \bf{523.10} & 
0.23\\CON3-9 & 588.48 & 1.31 & 
588.48 & 1.28 & \bf{578.20} & 
1.78\\CON8-0 & 879.00 & 1.42 & 
879.00 & 1.50 & \bf{858.90} & 
2.34\\CON8-1 & 758.26 & 1.39 & 
758.26 & 1.34 & \bf{740.90} & 
2.34\\CON8-2 & 716.53 & 1.95 & 
716.54 & 1.97 & \bf{714.30} & 
0.31\\CON8-3 & 817.57 & 1.34 & 
817.57 & 1.38 & \bf{812.30} & 
0.65\\CON8-4 & 781.64 & 1.56 & 
788.67 & 1.54 & \bf{770.10} & 
1.50\\CON8-5 & \bf{\underline{764.36}} & 1.40 & 
764.36 & 1.34 & 766.60 & 
-0.29\\CON8-6 & 705.61 & 1.68 & 
707.08 & 1.70 & \bf{697.20} & 
1.21\\CON8-7 & 822.42 & 1.23 & 
822.92 & 1.21 & \bf{814.80} & 
0.94\\CON8-8 & 799.32 & 1.55 & 
799.46 & 1.58 & \bf{771.30} & 
3.63\\CON8-9 & 816.12 & 1.58 & 
816.12 & 1.59 & \bf{815.10} & 
0.13\\[1ex]\hline
\end{tabular}
\label{table:nonlin}
\end{table} \clearpage
\begin{table}[ht]
\caption{Resultados de la ejecución de la metaheurística ACO, utilizando instancias de Dethloff con la configuración -n 2.0 -alpha 1.0 -beta 3.0 -q 3.8 -ro 0.015}
\centering
\small
\begin{tabular}{c c c c c c c}
\hline\hline
Instancia & Costo mínimo & Tiempo(seg.) & Costo promedio & Tiempo promedio(seg.) & Costo ACO & \%Gap \\ [0.5ex]
\hline
SCA3-0 & 640.55 & 1.35 & 
640.55 & 1.40 & \bf{636.10} & 
0.70\\SCA3-1 & \bf{\underline{697.84}} & 1.47 & 
697.84 & 1.45 & 700.10 & 
-0.32\\SCA3-2 & 664.18 & 1.26 & 
664.18 & 1.33 & \bf{659.30} & 
0.74\\SCA3-3 & 680.60 & 1.49 & 
680.60 & 1.53 & \bf{680.00} & 
0.09\\SCA3-4 & \bf{690.50} & 1.39 & 
690.50 & 1.36 & 690.50 & 0.00\\
SCA3-5 & \bf{\underline{665.04}} & 1.47 & 
668.95 & 1.38 & 671.10 & 
-0.90\\SCA3-6 & 653.69 & 1.40 & 
654.93 & 1.38 & \bf{651.10} & 
0.40\\SCA3-7 & 666.15 & 1.09 & 
666.15 & 1.09 & \bf{666.10} & 
0.01\\SCA3-8 & 721.45 & 1.28 & 
724.59 & 1.17 & \bf{719.50} & 
0.27\\SCA3-9 & \bf{681.00} & 1.02 & 
681.00 & 1.01 & 681.00 & 0.00\\
SCA8-0 & 991.07 & 1.50 & 
991.07 & 1.54 & \bf{961.60} & 
3.06\\SCA8-1 & 1074.65 & 1.20 & 
1074.68 & 1.19 & \bf{1063.00} & 
1.10\\SCA8-2 & 1056.87 & 1.03 & 
1056.87 & 1.02 & \bf{1040.60} & 
1.56\\SCA8-3 & 1031.08 & 1.44 & 
1031.08 & 1.45 & \bf{985.90} & 
4.58\\SCA8-4 & 1099.06 & 1.46 & 
1099.06 & 1.48 & \bf{1071.00} & 
2.62\\SCA8-5 & 1055.35 & 1.73 & 
1055.35 & 1.64 & \bf{1054.30} & 
0.10\\SCA8-6 & \bf{\underline{972.48}} & 2.32 & 
972.48 & 1.85 & 972.50 & 
-0.00\\SCA8-7 & 1092.57 & 1.62 & 
1092.57 & 1.59 & \bf{1059.70} & 
3.10\\SCA8-8 & 1091.49 & 1.41 & 
1091.89 & 1.45 & \bf{1082.70} & 
0.81\\SCA8-9 & \bf{\underline{1067.42}} & 1.15 & 
1067.42 & 1.16 & 1081.40 & 
-1.29\\CON3-0 & 624.96 & 1.63 & 
624.96 & 1.62 & \bf{616.50} & 
1.37\\CON3-1 & 557.38 & 1.35 & 
559.22 & 1.45 & \bf{555.60} & 
0.32\\CON3-2 & 524.07 & 1.10 & 
525.21 & 1.10 & \bf{521.40} & 
0.51\\CON3-3 & 594.11 & 1.54 & 
594.11 & 1.57 & \bf{591.20} & 
0.49\\CON3-4 & 589.32 & 1.32 & 
589.32 & 1.33 & \bf{589.30} & 
0.00\\CON3-5 & 574.57 & 1.44 & 
575.97 & 1.46 & \bf{563.70} & 
1.93\\CON3-6 & 504.15 & 1.78 & 
506.18 & 1.82 & \bf{499.20} & 
0.99\\CON3-7 & 578.41 & 1.14 & 
578.41 & 1.20 & \bf{577.50} & 
0.16\\CON3-8 & 524.30 & 1.27 & 
524.52 & 1.24 & \bf{523.10} & 
0.23\\CON3-9 & 588.48 & 1.21 & 
588.48 & 1.27 & \bf{578.20} & 
1.78\\CON8-0 & 879.00 & 1.38 & 
879.00 & 1.43 & \bf{858.90} & 
2.34\\CON8-1 & 758.26 & 1.38 & 
758.30 & 1.38 & \bf{740.90} & 
2.34\\CON8-2 & 716.53 & 2.02 & 
716.54 & 2.00 & \bf{714.30} & 
0.31\\CON8-3 & 817.57 & 1.40 & 
817.57 & 1.44 & \bf{812.30} & 
0.65\\CON8-4 & 781.64 & 1.48 & 
783.73 & 1.55 & \bf{770.10} & 
1.50\\CON8-5 & \bf{\underline{764.36}} & 1.40 & 
764.36 & 1.41 & 766.60 & 
-0.29\\CON8-6 & 705.61 & 1.74 & 
707.08 & 1.77 & \bf{697.20} & 
1.21\\CON8-7 & 822.42 & 1.20 & 
822.67 & 1.21 & \bf{814.80} & 
0.94\\CON8-8 & 799.32 & 1.57 & 
799.41 & 1.52 & \bf{771.30} & 
3.63\\CON8-9 & 816.12 & 1.51 & 
816.12 & 1.53 & \bf{815.10} & 
0.13\\[1ex]\hline
\end{tabular}
\label{table:nonlin}
\end{table} \clearpage
\begin{table}[ht]
\caption{Resultados de la ejecución de la metaheurística ACO, utilizando instancias de Dethloff con la configuración -n 2.0 -alpha 1.0 -beta 3.0 -q 3.9 -ro 0.015}
\centering
\small
\begin{tabular}{c c c c c c c}
\hline\hline
Instancia & Costo mínimo & Tiempo(seg.) & Costo promedio & Tiempo promedio(seg.) & Costo ACO & \%Gap \\ [0.5ex]
\hline
SCA3-0 & 640.55 & 1.42 & 
640.55 & 1.40 & \bf{636.10} & 
0.70\\SCA3-1 & \bf{\underline{697.84}} & 1.46 & 
697.84 & 1.52 & 700.10 & 
-0.32\\SCA3-2 & 659.34 & 1.30 & 
662.97 & 1.32 & \bf{659.30} & 
0.01\\SCA3-3 & 680.60 & 1.49 & 
680.78 & 1.50 & \bf{680.00} & 
0.09\\SCA3-4 & \bf{690.50} & 1.35 & 
690.50 & 1.38 & 690.50 & 0.00\\
SCA3-5 & \bf{\underline{665.04}} & 1.49 & 
665.19 & 1.46 & 671.10 & 
-0.90\\SCA3-6 & 655.19 & 1.44 & 
655.19 & 1.33 & \bf{651.10} & 
0.63\\SCA3-7 & 666.15 & 0.98 & 
666.15 & 0.96 & \bf{666.10} & 
0.01\\SCA3-8 & 721.45 & 1.18 & 
724.08 & 1.17 & \bf{719.50} & 
0.27\\SCA3-9 & \bf{681.00} & 1.05 & 
681.00 & 1.03 & 681.00 & 0.00\\
SCA8-0 & 991.07 & 1.48 & 
991.07 & 1.53 & \bf{961.60} & 
3.06\\SCA8-1 & 1069.40 & 1.18 & 
1073.27 & 1.18 & \bf{1063.00} & 
0.60\\SCA8-2 & 1056.87 & 0.96 & 
1056.87 & 1.01 & \bf{1040.60} & 
1.56\\SCA8-3 & 1031.08 & 1.39 & 
1031.08 & 1.48 & \bf{985.90} & 
4.58\\SCA8-4 & 1099.06 & 1.62 & 
1099.06 & 1.52 & \bf{1071.00} & 
2.62\\SCA8-5 & 1055.35 & 1.72 & 
1055.35 & 1.64 & \bf{1054.30} & 
0.10\\SCA8-6 & \bf{\underline{972.48}} & 1.70 & 
972.48 & 1.64 & 972.50 & 
-0.00\\SCA8-7 & 1075.42 & 1.79 & 
1088.28 & 1.67 & \bf{1059.70} & 
1.48\\SCA8-8 & 1092.02 & 1.44 & 
1092.02 & 1.43 & \bf{1082.70} & 
0.86\\SCA8-9 & \bf{\underline{1067.42}} & 1.09 & 
1067.42 & 1.15 & 1081.40 & 
-1.29\\CON3-0 & 624.96 & 1.58 & 
624.96 & 1.63 & \bf{616.50} & 
1.37\\CON3-1 & 557.38 & 1.40 & 
559.07 & 1.45 & \bf{555.60} & 
0.32\\CON3-2 & 524.07 & 1.05 & 
524.75 & 1.08 & \bf{521.40} & 
0.51\\CON3-3 & 594.11 & 1.52 & 
594.11 & 1.52 & \bf{591.20} & 
0.49\\CON3-4 & 589.32 & 1.36 & 
589.32 & 1.31 & \bf{589.30} & 
0.00\\CON3-5 & 570.70 & 1.47 & 
575.00 & 1.43 & \bf{563.70} & 
1.24\\CON3-6 & 504.15 & 1.78 & 
505.88 & 1.78 & \bf{499.20} & 
0.99\\CON3-7 & 578.41 & 1.26 & 
579.12 & 1.25 & \bf{577.50} & 
0.16\\CON3-8 & 524.59 & 1.22 & 
524.59 & 1.22 & \bf{523.10} & 
0.28\\CON3-9 & 588.48 & 1.27 & 
588.48 & 1.25 & \bf{578.20} & 
1.78\\CON8-0 & 879.00 & 1.48 & 
879.00 & 1.45 & \bf{858.90} & 
2.34\\CON8-1 & 758.26 & 1.31 & 
758.26 & 1.35 & \bf{740.90} & 
2.34\\CON8-2 & 716.53 & 2.69 & 
716.54 & 2.22 & \bf{714.30} & 
0.31\\CON8-3 & 817.57 & 1.42 & 
817.57 & 1.46 & \bf{812.30} & 
0.65\\CON8-4 & 781.64 & 1.69 & 
784.50 & 1.57 & \bf{770.10} & 
1.50\\CON8-5 & \bf{\underline{764.36}} & 1.44 & 
764.36 & 1.37 & 766.60 & 
-0.29\\CON8-6 & 705.61 & 1.79 & 
706.06 & 1.75 & \bf{697.20} & 
1.21\\CON8-7 & 822.42 & 1.20 & 
822.42 & 1.20 & \bf{814.80} & 
0.94\\CON8-8 & 799.32 & 1.57 & 
799.41 & 1.53 & \bf{771.30} & 
3.63\\CON8-9 & 816.12 & 1.57 & 
816.12 & 1.56 & \bf{815.10} & 
0.13\\[1ex]\hline
\end{tabular}
\label{table:nonlin}
\end{table} \clearpage
\begin{table}[ht]
\caption{Resultados de la ejecución de la metaheurística ACO, utilizando instancias de Dethloff con la configuración -n 2.0 -alpha 1.0 -beta 3.0 -q 4.0 -ro 0.015}
\centering
\small
\begin{tabular}{c c c c c c c}
\hline\hline
Instancia & Costo mínimo & Tiempo(seg.) & Costo promedio & Tiempo promedio(seg.) & Costo ACO & \%Gap \\ [0.5ex]
\hline
SCA3-0 & 640.55 & 1.38 & 
640.55 & 1.38 & \bf{636.10} & 
0.70\\SCA3-1 & \bf{\underline{697.84}} & 1.46 & 
697.84 & 1.45 & 700.10 & 
-0.32\\SCA3-2 & 664.18 & 1.37 & 
664.18 & 1.35 & \bf{659.30} & 
0.74\\SCA3-3 & 680.60 & 1.49 & 
680.96 & 1.71 & \bf{680.00} & 
0.09\\SCA3-4 & \bf{690.50} & 1.41 & 
690.50 & 1.36 & 690.50 & 0.00\\
SCA3-5 & \bf{\underline{665.04}} & 1.50 & 
665.19 & 1.42 & 671.10 & 
-0.90\\SCA3-6 & 655.19 & 1.44 & 
655.30 & 1.33 & \bf{651.10} & 
0.63\\SCA3-7 & 666.15 & 0.99 & 
666.15 & 1.00 & \bf{666.10} & 
0.01\\SCA3-8 & 721.45 & 1.09 & 
725.84 & 1.19 & \bf{719.50} & 
0.27\\SCA3-9 & \bf{681.00} & 0.96 & 
681.00 & 0.99 & 681.00 & 0.00\\
SCA8-0 & 991.07 & 1.55 & 
991.07 & 1.53 & \bf{961.60} & 
3.06\\SCA8-1 & 1074.65 & 1.19 & 
1074.71 & 1.16 & \bf{1063.00} & 
1.10\\SCA8-2 & 1056.87 & 0.95 & 
1056.87 & 1.00 & \bf{1040.60} & 
1.56\\SCA8-3 & 1031.08 & 1.46 & 
1031.08 & 1.48 & \bf{985.90} & 
4.58\\SCA8-4 & 1099.06 & 1.45 & 
1099.06 & 1.50 & \bf{1071.00} & 
2.62\\SCA8-5 & 1055.35 & 1.68 & 
1055.35 & 1.65 & \bf{1054.30} & 
0.10\\SCA8-6 & \bf{\underline{972.48}} & 1.61 & 
973.62 & 1.61 & 972.50 & 
-0.00\\SCA8-7 & 1092.57 & 1.70 & 
1092.57 & 1.63 & \bf{1059.70} & 
3.10\\SCA8-8 & 1084.41 & 1.42 & 
1090.12 & 1.47 & \bf{1082.70} & 
0.16\\SCA8-9 & \bf{\underline{1067.42}} & 1.06 & 
1067.42 & 1.11 & 1081.40 & 
-1.29\\CON3-0 & 624.96 & 1.62 & 
624.96 & 1.62 & \bf{616.50} & 
1.37\\CON3-1 & 557.38 & 1.47 & 
557.58 & 1.51 & \bf{555.60} & 
0.32\\CON3-2 & \bf{\underline{521.38}} & 1.05 & 
524.48 & 1.12 & 521.40 & 
-0.00\\CON3-3 & \bf{591.20} & 1.58 & 
593.09 & 1.51 & 591.20 & 0.00\\
CON3-4 & 589.32 & 1.28 & 
589.32 & 1.28 & \bf{589.30} & 
0.00\\CON3-5 & 569.15 & 1.38 & 
574.61 & 1.40 & \bf{563.70} & 
0.97\\CON3-6 & 505.26 & 1.73 & 
506.43 & 1.74 & \bf{499.20} & 
1.21\\CON3-7 & 578.41 & 1.14 & 
579.12 & 1.24 & \bf{577.50} & 
0.16\\CON3-8 & 524.30 & 1.18 & 
524.52 & 1.17 & \bf{523.10} & 
0.23\\CON3-9 & 586.17 & 1.40 & 
587.90 & 1.33 & \bf{578.20} & 
1.38\\CON8-0 & 879.00 & 1.42 & 
879.00 & 1.46 & \bf{858.90} & 
2.34\\CON8-1 & 758.26 & 1.38 & 
758.26 & 1.36 & \bf{740.90} & 
2.34\\CON8-2 & 716.53 & 1.96 & 
716.54 & 1.98 & \bf{714.30} & 
0.31\\CON8-3 & 817.57 & 1.45 & 
817.57 & 1.43 & \bf{812.30} & 
0.65\\CON8-4 & 781.64 & 1.63 & 
785.81 & 1.57 & \bf{770.10} & 
1.50\\CON8-5 & \bf{\underline{764.36}} & 1.38 & 
764.36 & 1.35 & 766.60 & 
-0.29\\CON8-6 & 705.61 & 1.60 & 
707.20 & 1.68 & \bf{697.20} & 
1.21\\CON8-7 & 822.42 & 1.19 & 
822.75 & 1.20 & \bf{814.80} & 
0.94\\CON8-8 & 796.81 & 1.49 & 
798.70 & 1.59 & \bf{771.30} & 
3.31\\CON8-9 & 816.12 & 1.56 & 
816.12 & 1.59 & \bf{815.10} & 
0.13\\[1ex]\hline
\end{tabular}
\label{table:nonlin}
\end{table} \clearpage
\begin{table}[ht]
\caption{Resultados de la ejecución de la metaheurística ACO, utilizando instancias de Dethloff con la configuración -n 2.0 -alpha 1.0 -beta 3.0 -q 4.1 -ro 0.015}
\centering
\small
\begin{tabular}{c c c c c c c}
\hline\hline
Instancia & Costo mínimo & Tiempo(seg.) & Costo promedio & Tiempo promedio(seg.) & Costo ACO & \%Gap \\ [0.5ex]
\hline
SCA3-0 & 640.55 & 1.44 & 
640.55 & 1.35 & \bf{636.10} & 
0.70\\SCA3-1 & \bf{\underline{697.84}} & 1.44 & 
698.76 & 1.45 & 700.10 & 
-0.32\\SCA3-2 & 659.34 & 1.25 & 
662.97 & 1.29 & \bf{659.30} & 
0.01\\SCA3-3 & 680.60 & 1.46 & 
680.60 & 1.48 & \bf{680.00} & 
0.09\\SCA3-4 & \bf{690.50} & 1.35 & 
690.50 & 1.46 & 690.50 & 0.00\\
SCA3-5 & \bf{\underline{665.04}} & 1.38 & 
665.04 & 1.45 & 671.10 & 
-0.90\\SCA3-6 & 653.69 & 1.40 & 
654.82 & 1.33 & \bf{651.10} & 
0.40\\SCA3-7 & 666.15 & 0.96 & 
666.26 & 0.98 & \bf{666.10} & 
0.01\\SCA3-8 & 721.45 & 1.12 & 
724.84 & 1.17 & \bf{719.50} & 
0.27\\SCA3-9 & \bf{681.00} & 0.96 & 
681.00 & 0.96 & 681.00 & 0.00\\
SCA8-0 & 991.07 & 1.49 & 
991.65 & 1.50 & \bf{961.60} & 
3.06\\SCA8-1 & 1069.40 & 1.20 & 
1073.34 & 1.19 & \bf{1063.00} & 
0.60\\SCA8-2 & 1056.87 & 0.99 & 
1056.87 & 1.02 & \bf{1040.60} & 
1.56\\SCA8-3 & 1031.08 & 1.53 & 
1031.08 & 1.48 & \bf{985.90} & 
4.58\\SCA8-4 & 1099.06 & 1.48 & 
1099.06 & 1.52 & \bf{1071.00} & 
2.62\\SCA8-5 & 1055.35 & 1.63 & 
1055.35 & 1.71 & \bf{1054.30} & 
0.10\\SCA8-6 & \bf{\underline{972.48}} & 1.64 & 
972.48 & 1.66 & 972.50 & 
-0.00\\SCA8-7 & 1092.57 & 1.54 & 
1092.57 & 1.57 & \bf{1059.70} & 
3.10\\SCA8-8 & 1085.93 & 1.51 & 
1090.50 & 1.50 & \bf{1082.70} & 
0.30\\SCA8-9 & \bf{\underline{1067.42}} & 1.18 & 
1067.42 & 1.16 & 1081.40 & 
-1.29\\CON3-0 & 624.96 & 1.62 & 
624.96 & 1.65 & \bf{616.50} & 
1.37\\CON3-1 & 557.38 & 1.42 & 
559.58 & 1.46 & \bf{555.60} & 
0.32\\CON3-2 & 524.07 & 1.15 & 
524.35 & 1.13 & \bf{521.40} & 
0.51\\CON3-3 & \bf{591.20} & 1.52 & 
593.38 & 1.50 & 591.20 & 0.00\\
CON3-4 & \bf{\underline{588.79}} & 1.28 & 
589.19 & 1.36 & 589.30 & 
-0.09\\CON3-5 & 574.57 & 1.46 & 
575.97 & 1.45 & \bf{563.70} & 
1.93\\CON3-6 & 505.26 & 1.72 & 
507.61 & 1.80 & \bf{499.20} & 
1.21\\CON3-7 & 578.41 & 1.31 & 
580.55 & 1.28 & \bf{577.50} & 
0.16\\CON3-8 & 524.30 & 1.30 & 
524.45 & 1.23 & \bf{523.10} & 
0.23\\CON3-9 & 588.48 & 1.32 & 
588.48 & 1.29 & \bf{578.20} & 
1.78\\CON8-0 & 879.00 & 1.58 & 
879.00 & 1.44 & \bf{858.90} & 
2.34\\CON8-1 & 758.26 & 1.34 & 
758.26 & 1.31 & \bf{740.90} & 
2.34\\CON8-2 & 716.53 & 2.03 & 
716.54 & 2.04 & \bf{714.30} & 
0.31\\CON8-3 & 817.57 & 1.45 & 
817.57 & 1.44 & \bf{812.30} & 
0.65\\CON8-4 & 781.64 & 1.58 & 
789.45 & 1.57 & \bf{770.10} & 
1.50\\CON8-5 & \bf{\underline{764.36}} & 1.40 & 
764.36 & 1.39 & 766.60 & 
-0.29\\CON8-6 & 705.61 & 1.79 & 
706.18 & 1.70 & \bf{697.20} & 
1.21\\CON8-7 & 822.42 & 1.13 & 
822.92 & 1.16 & \bf{814.80} & 
0.94\\CON8-8 & 799.32 & 1.55 & 
799.32 & 1.57 & \bf{771.30} & 
3.63\\CON8-9 & 816.12 & 1.52 & 
816.12 & 1.55 & \bf{815.10} & 
0.13\\[1ex]\hline
\end{tabular}
\label{table:nonlin}
\end{table} \clearpage
\begin{table}[ht]
\caption{Resultados de la ejecución de la metaheurística ACO, utilizando instancias de Dethloff con la configuración -n 2.0 -alpha 1.0 -beta 3.0 -q 4.2 -ro 0.015}
\centering
\small
\begin{tabular}{c c c c c c c}
\hline\hline
Instancia & Costo mínimo & Tiempo(seg.) & Costo promedio & Tiempo promedio(seg.) & Costo ACO & \%Gap \\ [0.5ex]
\hline
SCA3-0 & 640.55 & 1.39 & 
640.55 & 1.35 & \bf{636.10} & 
0.70\\SCA3-1 & \bf{\underline{697.84}} & 1.50 & 
697.84 & 1.53 & 700.10 & 
-0.32\\SCA3-2 & 659.34 & 1.29 & 
662.97 & 1.35 & \bf{659.30} & 
0.01\\SCA3-3 & 680.60 & 1.47 & 
680.78 & 1.48 & \bf{680.00} & 
0.09\\SCA3-4 & \bf{690.50} & 1.43 & 
690.50 & 1.38 & 690.50 & 0.00\\
SCA3-5 & \bf{\underline{665.04}} & 1.44 & 
668.74 & 1.40 & 671.10 & 
-0.90\\SCA3-6 & 655.19 & 1.28 & 
655.19 & 1.35 & \bf{651.10} & 
0.63\\SCA3-7 & 666.15 & 0.97 & 
666.15 & 0.97 & \bf{666.10} & 
0.01\\SCA3-8 & 721.45 & 1.12 & 
727.86 & 1.15 & \bf{719.50} & 
0.27\\SCA3-9 & \bf{681.00} & 0.95 & 
681.00 & 0.99 & 681.00 & 0.00\\
SCA8-0 & 991.07 & 1.44 & 
992.80 & 1.45 & \bf{961.60} & 
3.06\\SCA8-1 & 1074.39 & 1.17 & 
1074.59 & 1.20 & \bf{1063.00} & 
1.07\\SCA8-2 & 1056.87 & 1.01 & 
1056.87 & 1.01 & \bf{1040.60} & 
1.56\\SCA8-3 & 1031.08 & 1.44 & 
1031.08 & 1.66 & \bf{985.90} & 
4.58\\SCA8-4 & 1099.06 & 1.40 & 
1099.17 & 1.46 & \bf{1071.00} & 
2.62\\SCA8-5 & 1055.35 & 1.68 & 
1055.35 & 1.71 & \bf{1054.30} & 
0.10\\SCA8-6 & \bf{\underline{972.48}} & 1.58 & 
972.48 & 1.69 & 972.50 & 
-0.00\\SCA8-7 & 1075.42 & 1.63 & 
1088.28 & 1.63 & \bf{1059.70} & 
1.48\\SCA8-8 & 1091.49 & 1.41 & 
1091.89 & 1.44 & \bf{1082.70} & 
0.81\\SCA8-9 & \bf{\underline{1067.42}} & 1.14 & 
1067.42 & 1.16 & 1081.40 & 
-1.29\\CON3-0 & 624.96 & 1.55 & 
624.96 & 1.54 & \bf{616.50} & 
1.37\\CON3-1 & 557.38 & 1.42 & 
559.14 & 1.47 & \bf{555.60} & 
0.32\\CON3-2 & 524.07 & 1.18 & 
524.89 & 1.12 & \bf{521.40} & 
0.51\\CON3-3 & \bf{591.20} & 1.56 & 
593.38 & 1.49 & 591.20 & 0.00\\
CON3-4 & 589.32 & 1.38 & 
589.32 & 1.30 & \bf{589.30} & 
0.00\\CON3-5 & 576.43 & 1.39 & 
576.43 & 1.40 & \bf{563.70} & 
2.26\\CON3-6 & 505.26 & 1.73 & 
506.43 & 1.78 & \bf{499.20} & 
1.21\\CON3-7 & 578.41 & 1.31 & 
579.12 & 1.24 & \bf{577.50} & 
0.16\\CON3-8 & 524.30 & 1.14 & 
524.52 & 1.20 & \bf{523.10} & 
0.23\\CON3-9 & 588.48 & 1.21 & 
588.48 & 1.25 & \bf{578.20} & 
1.78\\CON8-0 & 879.00 & 1.36 & 
879.00 & 1.40 & \bf{858.90} & 
2.34\\CON8-1 & 758.26 & 1.30 & 
758.26 & 1.30 & \bf{740.90} & 
2.34\\CON8-2 & 716.53 & 2.10 & 
716.54 & 2.02 & \bf{714.30} & 
0.31\\CON8-3 & 817.57 & 1.43 & 
817.57 & 1.41 & \bf{812.30} & 
0.65\\CON8-4 & 781.64 & 1.47 & 
786.59 & 1.56 & \bf{770.10} & 
1.50\\CON8-5 & \bf{\underline{764.36}} & 1.39 & 
764.36 & 1.47 & 766.60 & 
-0.29\\CON8-6 & 705.61 & 1.75 & 
706.63 & 1.98 & \bf{697.20} & 
1.21\\CON8-7 & 822.42 & 1.23 & 
822.92 & 1.20 & \bf{814.80} & 
0.94\\CON8-8 & 799.16 & 1.56 & 
799.24 & 1.55 & \bf{771.30} & 
3.61\\CON8-9 & 816.12 & 1.54 & 
816.12 & 1.58 & \bf{815.10} & 
0.13\\[1ex]\hline
\end{tabular}
\label{table:nonlin}
\end{table} \clearpage
\begin{table}[ht]
\caption{Resultados de la ejecución de la metaheurística ACO, utilizando instancias de Dethloff con la configuración -n 2.0 -alpha 1.0 -beta 3.0 -q 4.3 -ro 0.015}
\centering
\small
\begin{tabular}{c c c c c c c}
\hline\hline
Instancia & Costo mínimo & Tiempo(seg.) & Costo promedio & Tiempo promedio(seg.) & Costo ACO & \%Gap \\ [0.5ex]
\hline
SCA3-0 & 640.55 & 1.36 & 
640.55 & 1.38 & \bf{636.10} & 
0.70\\SCA3-1 & \bf{\underline{697.84}} & 1.50 & 
698.76 & 1.49 & 700.10 & 
-0.32\\SCA3-2 & 659.34 & 1.30 & 
660.55 & 1.34 & \bf{659.30} & 
0.01\\SCA3-3 & 680.60 & 1.52 & 
680.96 & 1.48 & \bf{680.00} & 
0.09\\SCA3-4 & \bf{690.50} & 1.43 & 
690.50 & 1.37 & 690.50 & 0.00\\
SCA3-5 & \bf{\underline{659.90}} & 1.39 & 
662.47 & 1.41 & 671.10 & 
-1.67\\SCA3-6 & 655.19 & 1.34 & 
655.19 & 1.33 & \bf{651.10} & 
0.63\\SCA3-7 & 666.15 & 1.03 & 
666.15 & 0.97 & \bf{666.10} & 
0.01\\SCA3-8 & 721.45 & 1.14 & 
725.24 & 1.11 & \bf{719.50} & 
0.27\\SCA3-9 & \bf{681.00} & 0.89 & 
681.00 & 0.94 & 681.00 & 0.00\\
SCA8-0 & 991.07 & 1.55 & 
992.23 & 1.49 & \bf{961.60} & 
3.06\\SCA8-1 & 1074.65 & 1.23 & 
1074.65 & 1.20 & \bf{1063.00} & 
1.10\\SCA8-2 & 1056.87 & 1.03 & 
1056.87 & 1.16 & \bf{1040.60} & 
1.56\\SCA8-3 & 1031.08 & 1.44 & 
1031.08 & 1.44 & \bf{985.90} & 
4.58\\SCA8-4 & 1098.34 & 1.55 & 
1098.88 & 1.48 & \bf{1071.00} & 
2.55\\SCA8-5 & 1055.35 & 1.56 & 
1055.35 & 1.57 & \bf{1054.30} & 
0.10\\SCA8-6 & \bf{\underline{972.48}} & 1.69 & 
972.48 & 1.67 & 972.50 & 
-0.00\\SCA8-7 & 1092.57 & 1.64 & 
1092.57 & 1.60 & \bf{1059.70} & 
3.10\\SCA8-8 & 1087.30 & 1.44 & 
1090.84 & 1.44 & \bf{1082.70} & 
0.42\\SCA8-9 & \bf{\underline{1067.42}} & 1.16 & 
1067.42 & 1.16 & 1081.40 & 
-1.29\\CON3-0 & 624.96 & 1.65 & 
624.96 & 1.65 & \bf{616.50} & 
1.37\\CON3-1 & 557.38 & 1.44 & 
558.42 & 1.46 & \bf{555.60} & 
0.32\\CON3-2 & 524.07 & 1.04 & 
524.64 & 1.09 & \bf{521.40} & 
0.51\\CON3-3 & 594.11 & 1.50 & 
594.11 & 1.50 & \bf{591.20} & 
0.49\\CON3-4 & \bf{\underline{588.79}} & 1.26 & 
589.19 & 1.26 & 589.30 & 
-0.09\\CON3-5 & 576.43 & 1.43 & 
576.97 & 1.40 & \bf{563.70} & 
2.26\\CON3-6 & 505.26 & 1.74 & 
506.43 & 1.93 & \bf{499.20} & 
1.21\\CON3-7 & 578.41 & 1.22 & 
578.41 & 1.22 & \bf{577.50} & 
0.16\\CON3-8 & 524.30 & 1.24 & 
524.37 & 1.18 & \bf{523.10} & 
0.23\\CON3-9 & 588.48 & 1.28 & 
588.48 & 1.29 & \bf{578.20} & 
1.78\\CON8-0 & 879.00 & 1.48 & 
879.00 & 1.47 & \bf{858.90} & 
2.34\\CON8-1 & 758.26 & 1.23 & 
758.26 & 1.28 & \bf{740.90} & 
2.34\\CON8-2 & 716.53 & 1.95 & 
716.54 & 1.99 & \bf{714.30} & 
0.31\\CON8-3 & 817.57 & 1.38 & 
817.57 & 1.51 & \bf{812.30} & 
0.65\\CON8-4 & 781.64 & 1.65 & 
785.81 & 1.60 & \bf{770.10} & 
1.50\\CON8-5 & \bf{\underline{764.36}} & 1.38 & 
764.36 & 1.39 & 766.60 & 
-0.29\\CON8-6 & 705.61 & 1.66 & 
706.63 & 1.72 & \bf{697.20} & 
1.21\\CON8-7 & 822.42 & 1.17 & 
823.18 & 1.42 & \bf{814.80} & 
0.94\\CON8-8 & 799.51 & 1.50 & 
799.51 & 1.50 & \bf{771.30} & 
3.66\\CON8-9 & 816.12 & 1.53 & 
816.12 & 1.48 & \bf{815.10} & 
0.13\\[1ex]\hline
\end{tabular}
\label{table:nonlin}
\end{table} \clearpage
\begin{table}[ht]
\caption{Resultados de la ejecución de la metaheurística ACO, utilizando instancias de Dethloff con la configuración -n 2.0 -alpha 1.0 -beta 3.0 -q 4.4 -ro 0.015}
\centering
\small
\begin{tabular}{c c c c c c c}
\hline\hline
Instancia & Costo mínimo & Tiempo(seg.) & Costo promedio & Tiempo promedio(seg.) & Costo ACO & \%Gap \\ [0.5ex]
\hline
SCA3-0 & 640.55 & 1.34 & 
640.55 & 1.35 & \bf{636.10} & 
0.70\\SCA3-1 & \bf{\underline{697.84}} & 1.52 & 
697.84 & 1.48 & 700.10 & 
-0.32\\SCA3-2 & 659.34 & 1.33 & 
660.55 & 1.39 & \bf{659.30} & 
0.01\\SCA3-3 & 680.60 & 1.42 & 
680.78 & 1.43 & \bf{680.00} & 
0.09\\SCA3-4 & \bf{690.50} & 1.43 & 
690.50 & 1.40 & 690.50 & 0.00\\
SCA3-5 & \bf{\underline{665.04}} & 1.44 & 
665.04 & 1.42 & 671.10 & 
-0.90\\SCA3-6 & 655.19 & 1.42 & 
655.30 & 1.36 & \bf{651.10} & 
0.63\\SCA3-7 & 666.15 & 1.05 & 
666.15 & 1.00 & \bf{666.10} & 
0.01\\SCA3-8 & 721.45 & 1.09 & 
721.45 & 1.29 & \bf{719.50} & 
0.27\\SCA3-9 & \bf{681.00} & 1.03 & 
681.00 & 0.98 & 681.00 & 0.00\\
SCA8-0 & 991.07 & 1.52 & 
991.65 & 1.53 & \bf{961.60} & 
3.06\\SCA8-1 & 1074.39 & 1.17 & 
1074.62 & 1.29 & \bf{1063.00} & 
1.07\\SCA8-2 & 1056.87 & 1.10 & 
1056.87 & 1.05 & \bf{1040.60} & 
1.56\\SCA8-3 & 1031.08 & 1.43 & 
1031.08 & 1.44 & \bf{985.90} & 
4.58\\SCA8-4 & 1098.34 & 1.52 & 
1098.88 & 1.50 & \bf{1071.00} & 
2.55\\SCA8-5 & 1055.35 & 1.68 & 
1055.35 & 1.66 & \bf{1054.30} & 
0.10\\SCA8-6 & \bf{\underline{972.48}} & 1.65 & 
981.71 & 1.71 & 972.50 & 
-0.00\\SCA8-7 & 1092.57 & 1.67 & 
1092.57 & 1.66 & \bf{1059.70} & 
3.10\\SCA8-8 & 1092.02 & 1.45 & 
1092.02 & 1.44 & \bf{1082.70} & 
0.86\\SCA8-9 & \bf{\underline{1067.42}} & 1.14 & 
1067.42 & 1.15 & 1081.40 & 
-1.29\\CON3-0 & 624.96 & 1.70 & 
624.96 & 1.66 & \bf{616.50} & 
1.37\\CON3-1 & 557.38 & 1.44 & 
557.38 & 1.47 & \bf{555.60} & 
0.32\\CON3-2 & 524.07 & 1.12 & 
525.47 & 1.07 & \bf{521.40} & 
0.51\\CON3-3 & \bf{591.20} & 1.57 & 
593.09 & 1.52 & 591.20 & 0.00\\
CON3-4 & 589.32 & 1.35 & 
589.32 & 1.37 & \bf{589.30} & 
0.00\\CON3-5 & 576.43 & 1.44 & 
576.43 & 1.46 & \bf{563.70} & 
2.26\\CON3-6 & 505.26 & 1.76 & 
506.17 & 1.75 & \bf{499.20} & 
1.21\\CON3-7 & 578.41 & 1.19 & 
578.41 & 1.23 & \bf{577.50} & 
0.16\\CON3-8 & 524.59 & 1.24 & 
524.59 & 1.20 & \bf{523.10} & 
0.28\\CON3-9 & 588.48 & 1.40 & 
588.48 & 1.32 & \bf{578.20} & 
1.78\\CON8-0 & 879.00 & 1.50 & 
879.00 & 1.47 & \bf{858.90} & 
2.34\\CON8-1 & 758.26 & 1.37 & 
758.26 & 1.38 & \bf{740.90} & 
2.34\\CON8-2 & 716.53 & 2.10 & 
716.54 & 2.10 & \bf{714.30} & 
0.31\\CON8-3 & 817.57 & 1.34 & 
817.57 & 1.42 & \bf{812.30} & 
0.65\\CON8-4 & 781.64 & 1.46 & 
785.81 & 1.53 & \bf{770.10} & 
1.50\\CON8-5 & \bf{\underline{764.36}} & 1.38 & 
764.36 & 1.46 & 766.60 & 
-0.29\\CON8-6 & 705.61 & 1.76 & 
707.20 & 1.72 & \bf{697.20} & 
1.21\\CON8-7 & 822.42 & 1.14 & 
823.18 & 1.19 & \bf{814.80} & 
0.94\\CON8-8 & 799.16 & 1.57 & 
799.38 & 1.56 & \bf{771.30} & 
3.61\\CON8-9 & 816.12 & 1.57 & 
817.40 & 1.56 & \bf{815.10} & 
0.13\\[1ex]\hline
\end{tabular}
\label{table:nonlin}
\end{table} \clearpage
\begin{table}[ht]
\caption{Resultados de la ejecución de la metaheurística ACO, utilizando instancias de Dethloff con la configuración -n 2.0 -alpha 1.0 -beta 3.0 -q 4.5 -ro 0.015}
\centering
\small
\begin{tabular}{c c c c c c c}
\hline\hline
Instancia & Costo mínimo & Tiempo(seg.) & Costo promedio & Tiempo promedio(seg.) & Costo ACO & \%Gap \\ [0.5ex]
\hline
SCA3-0 & 640.55 & 1.41 & 
640.55 & 1.37 & \bf{636.10} & 
0.70\\SCA3-1 & \bf{\underline{697.84}} & 1.48 & 
697.84 & 1.47 & 700.10 & 
-0.32\\SCA3-2 & 664.18 & 1.34 & 
664.18 & 1.34 & \bf{659.30} & 
0.74\\SCA3-3 & 680.60 & 1.48 & 
680.78 & 1.57 & \bf{680.00} & 
0.09\\SCA3-4 & \bf{690.50} & 1.37 & 
690.50 & 1.41 & 690.50 & 0.00\\
SCA3-5 & \bf{\underline{665.04}} & 1.46 & 
665.39 & 1.43 & 671.10 & 
-0.90\\SCA3-6 & 655.19 & 1.26 & 
655.19 & 1.36 & \bf{651.10} & 
0.63\\SCA3-7 & 666.15 & 0.96 & 
666.26 & 0.96 & \bf{666.10} & 
0.01\\SCA3-8 & 721.45 & 1.10 & 
726.48 & 1.10 & \bf{719.50} & 
0.27\\SCA3-9 & \bf{681.00} & 0.94 & 
681.00 & 0.95 & 681.00 & 0.00\\
SCA8-0 & 991.07 & 1.45 & 
998.16 & 1.52 & \bf{961.60} & 
3.06\\SCA8-1 & 1074.65 & 1.23 & 
1074.68 & 1.17 & \bf{1063.00} & 
1.10\\SCA8-2 & 1056.87 & 1.03 & 
1056.87 & 1.00 & \bf{1040.60} & 
1.56\\SCA8-3 & 1031.08 & 1.48 & 
1031.08 & 1.48 & \bf{985.90} & 
4.58\\SCA8-4 & 1099.06 & 1.56 & 
1099.06 & 1.51 & \bf{1071.00} & 
2.62\\SCA8-5 & 1055.35 & 1.68 & 
1055.35 & 1.69 & \bf{1054.30} & 
0.10\\SCA8-6 & \bf{\underline{972.48}} & 1.71 & 
976.70 & 1.68 & 972.50 & 
-0.00\\SCA8-7 & 1092.57 & 1.68 & 
1092.57 & 1.63 & \bf{1059.70} & 
3.10\\SCA8-8 & 1091.49 & 1.66 & 
1091.89 & 1.50 & \bf{1082.70} & 
0.81\\SCA8-9 & \bf{\underline{1067.42}} & 1.10 & 
1067.42 & 1.12 & 1081.40 & 
-1.29\\CON3-0 & 624.96 & 1.54 & 
624.96 & 1.65 & \bf{616.50} & 
1.37\\CON3-1 & 557.38 & 1.38 & 
558.42 & 1.41 & \bf{555.60} & 
0.32\\CON3-2 & 524.07 & 1.13 & 
524.96 & 1.12 & \bf{521.40} & 
0.51\\CON3-3 & \bf{591.20} & 1.42 & 
593.38 & 1.49 & 591.20 & 0.00\\
CON3-4 & 589.32 & 1.35 & 
589.32 & 1.35 & \bf{589.30} & 
0.00\\CON3-5 & 574.57 & 1.48 & 
575.97 & 1.47 & \bf{563.70} & 
1.93\\CON3-6 & 504.15 & 1.80 & 
504.98 & 1.76 & \bf{499.20} & 
0.99\\CON3-7 & 578.41 & 1.30 & 
580.55 & 1.23 & \bf{577.50} & 
0.16\\CON3-8 & 524.59 & 1.13 & 
526.97 & 1.19 & \bf{523.10} & 
0.28\\CON3-9 & 588.48 & 1.27 & 
588.48 & 1.28 & \bf{578.20} & 
1.78\\CON8-0 & 879.00 & 1.36 & 
879.00 & 1.41 & \bf{858.90} & 
2.34\\CON8-1 & 758.26 & 1.34 & 
758.26 & 1.39 & \bf{740.90} & 
2.34\\CON8-2 & 716.53 & 1.98 & 
716.54 & 1.97 & \bf{714.30} & 
0.31\\CON8-3 & 817.57 & 1.46 & 
817.57 & 1.48 & \bf{812.30} & 
0.65\\CON8-4 & 778.60 & 1.54 & 
786.61 & 1.54 & \bf{770.10} & 
1.10\\CON8-5 & \bf{\underline{764.36}} & 1.34 & 
764.36 & 1.36 & 766.60 & 
-0.29\\CON8-6 & 703.66 & 1.72 & 
706.02 & 1.68 & \bf{697.20} & 
0.93\\CON8-7 & 822.42 & 1.19 & 
822.92 & 1.18 & \bf{814.80} & 
0.94\\CON8-8 & 799.51 & 1.57 & 
799.51 & 1.55 & \bf{771.30} & 
3.66\\CON8-9 & 816.12 & 1.63 & 
816.12 & 1.60 & \bf{815.10} & 
0.13\\[1ex]\hline
\end{tabular}
\label{table:nonlin}
\end{table} \clearpage
\begin{table}[ht]
\caption{Resultados de la ejecución de la metaheurística ACO, utilizando instancias de Dethloff con la configuración -n 2.0 -alpha 1.0 -beta 3.0 -q 4.6 -ro 0.015}
\centering
\small
\begin{tabular}{c c c c c c c}
\hline\hline
Instancia & Costo mínimo & Tiempo(seg.) & Costo promedio & Tiempo promedio(seg.) & Costo ACO & \%Gap \\ [0.5ex]
\hline
SCA3-0 & 640.55 & 1.44 & 
640.55 & 1.39 & \bf{636.10} & 
0.70\\SCA3-1 & \bf{\underline{697.84}} & 1.58 & 
697.84 & 1.55 & 700.10 & 
-0.32\\SCA3-2 & 664.18 & 1.37 & 
664.18 & 1.34 & \bf{659.30} & 
0.74\\SCA3-3 & 680.60 & 1.50 & 
680.78 & 1.50 & \bf{680.00} & 
0.09\\SCA3-4 & \bf{690.50} & 1.43 & 
690.50 & 1.43 & 690.50 & 0.00\\
SCA3-5 & \bf{\underline{665.04}} & 1.38 & 
665.19 & 1.44 & 671.10 & 
-0.90\\SCA3-6 & 655.19 & 1.30 & 
655.19 & 1.40 & \bf{651.10} & 
0.63\\SCA3-7 & 666.15 & 0.97 & 
666.15 & 1.03 & \bf{666.10} & 
0.01\\SCA3-8 & 721.45 & 1.08 & 
721.45 & 1.13 & \bf{719.50} & 
0.27\\SCA3-9 & \bf{681.00} & 0.94 & 
681.00 & 0.97 & 681.00 & 0.00\\
SCA8-0 & 991.07 & 1.48 & 
991.65 & 1.48 & \bf{961.60} & 
3.06\\SCA8-1 & 1074.65 & 1.15 & 
1074.65 & 1.20 & \bf{1063.00} & 
1.10\\SCA8-2 & 1056.87 & 1.08 & 
1056.87 & 1.06 & \bf{1040.60} & 
1.56\\SCA8-3 & 1031.08 & 1.46 & 
1031.08 & 1.44 & \bf{985.90} & 
4.58\\SCA8-4 & 1099.06 & 1.50 & 
1099.06 & 1.51 & \bf{1071.00} & 
2.62\\SCA8-5 & 1055.35 & 1.66 & 
1055.35 & 1.66 & \bf{1054.30} & 
0.10\\SCA8-6 & \bf{\underline{972.48}} & 1.61 & 
972.48 & 1.64 & 972.50 & 
-0.00\\SCA8-7 & 1092.57 & 1.60 & 
1092.57 & 1.64 & \bf{1059.70} & 
3.10\\SCA8-8 & 1091.49 & 1.35 & 
1091.89 & 1.39 & \bf{1082.70} & 
0.81\\SCA8-9 & \bf{\underline{1067.42}} & 1.14 & 
1067.42 & 1.16 & 1081.40 & 
-1.29\\CON3-0 & 624.96 & 1.67 & 
624.96 & 1.65 & \bf{616.50} & 
1.37\\CON3-1 & 557.38 & 1.41 & 
559.22 & 1.43 & \bf{555.60} & 
0.32\\CON3-2 & 524.07 & 1.08 & 
524.79 & 1.10 & \bf{521.40} & 
0.51\\CON3-3 & \bf{591.20} & 1.84 & 
593.38 & 1.56 & 591.20 & 0.00\\
CON3-4 & 589.32 & 1.33 & 
589.32 & 1.30 & \bf{589.30} & 
0.00\\CON3-5 & 575.30 & 1.39 & 
576.15 & 1.40 & \bf{563.70} & 
2.06\\CON3-6 & 505.26 & 1.66 & 
505.26 & 1.92 & \bf{499.20} & 
1.21\\CON3-7 & 578.41 & 1.27 & 
578.41 & 1.20 & \bf{577.50} & 
0.16\\CON3-8 & 524.59 & 1.14 & 
524.59 & 1.19 & \bf{523.10} & 
0.28\\CON3-9 & 588.48 & 1.26 & 
588.48 & 1.28 & \bf{578.20} & 
1.78\\CON8-0 & 879.00 & 1.45 & 
879.00 & 1.43 & \bf{858.90} & 
2.34\\CON8-1 & 758.26 & 1.30 & 
758.26 & 1.34 & \bf{740.90} & 
2.34\\CON8-2 & 716.53 & 2.08 & 
716.54 & 2.06 & \bf{714.30} & 
0.31\\CON8-3 & 817.57 & 1.35 & 
817.57 & 1.43 & \bf{812.30} & 
0.65\\CON8-4 & 781.64 & 1.58 & 
787.89 & 1.58 & \bf{770.10} & 
1.50\\CON8-5 & \bf{\underline{764.36}} & 1.37 & 
764.36 & 1.39 & 766.60 & 
-0.29\\CON8-6 & \bf{\underline{693.83}} & 1.65 & 
703.23 & 1.69 & 697.20 & 
-0.48\\CON8-7 & 822.42 & 1.20 & 
822.92 & 1.20 & \bf{814.80} & 
0.94\\CON8-8 & 799.32 & 1.57 & 
799.37 & 1.52 & \bf{771.30} & 
3.63\\CON8-9 & 816.12 & 1.63 & 
816.12 & 1.54 & \bf{815.10} & 
0.13\\[1ex]\hline
\end{tabular}
\label{table:nonlin}
\end{table} \clearpage
\begin{table}[ht]
\caption{Resultados de la ejecución de la metaheurística ACO, utilizando instancias de Dethloff con la configuración -n 2.0 -alpha 1.0 -beta 3.0 -q 4.7 -ro 0.015}
\centering
\small
\begin{tabular}{c c c c c c c}
\hline\hline
Instancia & Costo mínimo & Tiempo(seg.) & Costo promedio & Tiempo promedio(seg.) & Costo ACO & \%Gap \\ [0.5ex]
\hline
SCA3-0 & 640.55 & 1.37 & 
640.55 & 1.36 & \bf{636.10} & 
0.70\\SCA3-1 & \bf{\underline{697.84}} & 1.56 & 
697.84 & 1.52 & 700.10 & 
-0.32\\SCA3-2 & 664.18 & 1.33 & 
664.18 & 1.34 & \bf{659.30} & 
0.74\\SCA3-3 & 680.60 & 1.51 & 
680.96 & 1.51 & \bf{680.00} & 
0.09\\SCA3-4 & \bf{690.50} & 1.30 & 
690.50 & 1.39 & 690.50 & 0.00\\
SCA3-5 & \bf{\underline{665.04}} & 1.38 & 
665.79 & 1.43 & 671.10 & 
-0.90\\SCA3-6 & 655.19 & 1.26 & 
655.19 & 1.34 & \bf{651.10} & 
0.63\\SCA3-7 & 666.15 & 1.00 & 
666.15 & 1.01 & \bf{666.10} & 
0.01\\SCA3-8 & 721.45 & 1.21 & 
727.22 & 1.14 & \bf{719.50} & 
0.27\\SCA3-9 & \bf{681.00} & 0.93 & 
681.00 & 0.98 & 681.00 & 0.00\\
SCA8-0 & 991.07 & 1.50 & 
991.65 & 1.52 & \bf{961.60} & 
3.06\\SCA8-1 & 1074.65 & 1.26 & 
1074.68 & 1.22 & \bf{1063.00} & 
1.10\\SCA8-2 & 1056.87 & 0.99 & 
1056.87 & 1.00 & \bf{1040.60} & 
1.56\\SCA8-3 & 1031.08 & 1.48 & 
1031.08 & 1.45 & \bf{985.90} & 
4.58\\SCA8-4 & 1099.06 & 1.54 & 
1099.06 & 1.52 & \bf{1071.00} & 
2.62\\SCA8-5 & 1055.35 & 1.58 & 
1055.35 & 1.61 & \bf{1054.30} & 
0.10\\SCA8-6 & \bf{\underline{972.48}} & 1.74 & 
975.00 & 1.74 & 972.50 & 
-0.00\\SCA8-7 & 1092.57 & 1.62 & 
1092.57 & 1.67 & \bf{1059.70} & 
3.10\\SCA8-8 & 1091.49 & 1.44 & 
1091.89 & 1.42 & \bf{1082.70} & 
0.81\\SCA8-9 & \bf{\underline{1067.42}} & 1.18 & 
1067.42 & 1.15 & 1081.40 & 
-1.29\\CON3-0 & 624.96 & 1.55 & 
624.96 & 1.58 & \bf{616.50} & 
1.37\\CON3-1 & 557.38 & 1.38 & 
559.50 & 1.43 & \bf{555.60} & 
0.32\\CON3-2 & 524.07 & 1.08 & 
524.86 & 1.07 & \bf{521.40} & 
0.51\\CON3-3 & \bf{591.20} & 1.44 & 
593.38 & 1.48 & 591.20 & 0.00\\
CON3-4 & 589.32 & 1.30 & 
589.32 & 1.39 & \bf{589.30} & 
0.00\\CON3-5 & 569.15 & 1.51 & 
572.79 & 1.47 & \bf{563.70} & 
0.97\\CON3-6 & 505.26 & 1.80 & 
508.78 & 1.79 & \bf{499.20} & 
1.21\\CON3-7 & 578.41 & 1.27 & 
579.84 & 1.25 & \bf{577.50} & 
0.16\\CON3-8 & 524.59 & 1.28 & 
524.59 & 1.24 & \bf{523.10} & 
0.28\\CON3-9 & 578.25 & 1.20 & 
585.92 & 1.26 & \bf{578.20} & 
0.01\\CON8-0 & 879.00 & 1.46 & 
879.00 & 1.43 & \bf{858.90} & 
2.34\\CON8-1 & 758.26 & 1.36 & 
758.26 & 1.35 & \bf{740.90} & 
2.34\\CON8-2 & 716.53 & 1.93 & 
716.54 & 1.95 & \bf{714.30} & 
0.31\\CON8-3 & 817.57 & 1.52 & 
817.57 & 1.47 & \bf{812.30} & 
0.65\\CON8-4 & 781.64 & 1.54 & 
789.45 & 1.59 & \bf{770.10} & 
1.50\\CON8-5 & \bf{\underline{764.36}} & 1.39 & 
764.36 & 1.39 & 766.60 & 
-0.29\\CON8-6 & 705.61 & 1.63 & 
706.71 & 1.68 & \bf{697.20} & 
1.21\\CON8-7 & 823.43 & 1.10 & 
823.43 & 1.17 & \bf{814.80} & 
1.06\\CON8-8 & 799.32 & 1.60 & 
799.41 & 1.54 & \bf{771.30} & 
3.63\\CON8-9 & 816.12 & 1.58 & 
817.40 & 1.58 & \bf{815.10} & 
0.13\\[1ex]\hline
\end{tabular}
\label{table:nonlin}
\end{table} \clearpage
\begin{table}[ht]
\caption{Resultados de la ejecución de la metaheurística ACO, utilizando instancias de Dethloff con la configuración -n 2.0 -alpha 1.0 -beta 3.0 -q 4.8 -ro 0.015}
\centering
\small
\begin{tabular}{c c c c c c c}
\hline\hline
Instancia & Costo mínimo & Tiempo(seg.) & Costo promedio & Tiempo promedio(seg.) & Costo ACO & \%Gap \\ [0.5ex]
\hline
SCA3-0 & 640.55 & 1.47 & 
640.55 & 1.43 & \bf{636.10} & 
0.70\\SCA3-1 & \bf{\underline{697.84}} & 1.54 & 
698.76 & 1.56 & 700.10 & 
-0.32\\SCA3-2 & 659.34 & 1.37 & 
661.76 & 1.35 & \bf{659.30} & 
0.01\\SCA3-3 & 680.60 & 1.42 & 
680.78 & 1.47 & \bf{680.00} & 
0.09\\SCA3-4 & \bf{690.50} & 1.48 & 
690.50 & 1.43 & 690.50 & 0.00\\
SCA3-5 & \bf{\underline{665.04}} & 1.53 & 
668.60 & 1.47 & 671.10 & 
-0.90\\SCA3-6 & 655.19 & 1.44 & 
655.19 & 1.37 & \bf{651.10} & 
0.63\\SCA3-7 & 666.15 & 0.97 & 
666.15 & 0.99 & \bf{666.10} & 
0.01\\SCA3-8 & 721.45 & 1.08 & 
724.59 & 1.14 & \bf{719.50} & 
0.27\\SCA3-9 & \bf{681.00} & 1.02 & 
681.00 & 0.94 & 681.00 & 0.00\\
SCA8-0 & 991.07 & 1.50 & 
991.07 & 1.49 & \bf{961.60} & 
3.06\\SCA8-1 & 1074.65 & 1.18 & 
1074.65 & 1.19 & \bf{1063.00} & 
1.10\\SCA8-2 & 1056.87 & 1.01 & 
1056.87 & 1.03 & \bf{1040.60} & 
1.56\\SCA8-3 & 1031.08 & 1.55 & 
1031.08 & 1.49 & \bf{985.90} & 
4.58\\SCA8-4 & 1099.06 & 1.49 & 
1099.06 & 1.48 & \bf{1071.00} & 
2.62\\SCA8-5 & 1055.35 & 1.53 & 
1055.35 & 1.59 & \bf{1054.30} & 
0.10\\SCA8-6 & \bf{\underline{972.48}} & 1.59 & 
977.52 & 1.62 & 972.50 & 
-0.00\\SCA8-7 & 1092.57 & 1.52 & 
1092.57 & 1.59 & \bf{1059.70} & 
3.10\\SCA8-8 & 1092.02 & 1.42 & 
1092.02 & 1.44 & \bf{1082.70} & 
0.86\\SCA8-9 & \bf{\underline{1067.42}} & 1.19 & 
1067.42 & 1.16 & 1081.40 & 
-1.29\\CON3-0 & 624.96 & 1.65 & 
624.96 & 1.67 & \bf{616.50} & 
1.37\\CON3-1 & 557.38 & 1.60 & 
559.34 & 1.49 & \bf{555.60} & 
0.32\\CON3-2 & 524.07 & 1.01 & 
526.65 & 1.08 & \bf{521.40} & 
0.51\\CON3-3 & 594.11 & 1.50 & 
594.11 & 1.58 & \bf{591.20} & 
0.49\\CON3-4 & 589.32 & 1.35 & 
589.32 & 1.28 & \bf{589.30} & 
0.00\\CON3-5 & 576.43 & 1.34 & 
576.43 & 1.41 & \bf{563.70} & 
2.26\\CON3-6 & 504.15 & 1.77 & 
507.33 & 1.78 & \bf{499.20} & 
0.99\\CON3-7 & 578.41 & 1.13 & 
579.84 & 1.18 & \bf{577.50} & 
0.16\\CON3-8 & 524.59 & 1.22 & 
524.59 & 1.20 & \bf{523.10} & 
0.28\\CON3-9 & 588.48 & 1.28 & 
588.48 & 1.26 & \bf{578.20} & 
1.78\\CON8-0 & 879.00 & 1.43 & 
879.00 & 1.45 & \bf{858.90} & 
2.34\\CON8-1 & 758.26 & 1.40 & 
758.26 & 1.34 & \bf{740.90} & 
2.34\\CON8-2 & 716.53 & 1.82 & 
716.53 & 1.92 & \bf{714.30} & 
0.31\\CON8-3 & 817.57 & 1.40 & 
817.57 & 1.44 & \bf{812.30} & 
0.65\\CON8-4 & 789.98 & 1.56 & 
789.98 & 1.52 & \bf{770.10} & 
2.58\\CON8-5 & \bf{\underline{764.36}} & 1.42 & 
764.36 & 1.40 & 766.60 & 
-0.29\\CON8-6 & 705.61 & 1.78 & 
706.51 & 1.67 & \bf{697.20} & 
1.21\\CON8-7 & 822.42 & 1.18 & 
822.92 & 1.19 & \bf{814.80} & 
0.94\\CON8-8 & 799.51 & 1.57 & 
799.51 & 1.59 & \bf{771.30} & 
3.66\\CON8-9 & 816.12 & 1.50 & 
816.12 & 1.56 & \bf{815.10} & 
0.13\\[1ex]\hline
\end{tabular}
\label{table:nonlin}
\end{table} \clearpage
\begin{table}[ht]
\caption{Resultados de la ejecución de la metaheurística ACO, utilizando instancias de Dethloff con la configuración -n 2.0 -alpha 1.0 -beta 3.0 -q 4.9 -ro 0.015}
\centering
\small
\begin{tabular}{c c c c c c c}
\hline\hline
Instancia & Costo mínimo & Tiempo(seg.) & Costo promedio & Tiempo promedio(seg.) & Costo ACO & \%Gap \\ [0.5ex]
\hline
SCA3-0 & 640.55 & 1.31 & 
640.55 & 1.34 & \bf{636.10} & 
0.70\\SCA3-1 & \bf{\underline{697.84}} & 1.45 & 
697.84 & 1.50 & 700.10 & 
-0.32\\SCA3-2 & 664.18 & 1.30 & 
664.18 & 1.29 & \bf{659.30} & 
0.74\\SCA3-3 & 680.60 & 1.45 & 
680.60 & 1.44 & \bf{680.00} & 
0.09\\SCA3-4 & \bf{690.50} & 1.39 & 
690.50 & 1.39 & 690.50 & 0.00\\
SCA3-5 & \bf{\underline{665.04}} & 1.45 & 
665.04 & 1.46 & 671.10 & 
-0.90\\SCA3-6 & 655.19 & 1.34 & 
655.19 & 1.35 & \bf{651.10} & 
0.63\\SCA3-7 & 666.15 & 0.95 & 
666.15 & 1.01 & \bf{666.10} & 
0.01\\SCA3-8 & 721.45 & 1.29 & 
725.24 & 1.19 & \bf{719.50} & 
0.27\\SCA3-9 & \bf{681.00} & 1.07 & 
681.00 & 0.98 & 681.00 & 0.00\\
SCA8-0 & 991.07 & 1.60 & 
991.65 & 1.54 & \bf{961.60} & 
3.06\\SCA8-1 & 1074.65 & 1.24 & 
1074.65 & 1.21 & \bf{1063.00} & 
1.10\\SCA8-2 & 1056.87 & 1.07 & 
1056.87 & 1.04 & \bf{1040.60} & 
1.56\\SCA8-3 & 1031.08 & 1.49 & 
1031.08 & 1.46 & \bf{985.90} & 
4.58\\SCA8-4 & 1098.34 & 1.44 & 
1098.88 & 1.45 & \bf{1071.00} & 
2.55\\SCA8-5 & 1055.35 & 1.64 & 
1055.35 & 1.61 & \bf{1054.30} & 
0.10\\SCA8-6 & \bf{\underline{972.48}} & 1.68 & 
972.48 & 1.72 & 972.50 & 
-0.00\\SCA8-7 & 1092.57 & 1.56 & 
1092.57 & 1.59 & \bf{1059.70} & 
3.10\\SCA8-8 & 1092.02 & 1.48 & 
1092.02 & 1.43 & \bf{1082.70} & 
0.86\\SCA8-9 & \bf{\underline{1067.42}} & 1.16 & 
1067.42 & 1.16 & 1081.40 & 
-1.29\\CON3-0 & 624.96 & 1.61 & 
624.96 & 1.63 & \bf{616.50} & 
1.37\\CON3-1 & 557.38 & 1.45 & 
560.06 & 1.44 & \bf{555.60} & 
0.32\\CON3-2 & 524.07 & 1.21 & 
524.27 & 1.16 & \bf{521.40} & 
0.51\\CON3-3 & 594.11 & 1.60 & 
594.11 & 1.56 & \bf{591.20} & 
0.49\\CON3-4 & 589.32 & 1.24 & 
589.32 & 1.29 & \bf{589.30} & 
0.00\\CON3-5 & 569.88 & 1.34 & 
573.36 & 1.41 & \bf{563.70} & 
1.10\\CON3-6 & 505.26 & 1.72 & 
505.26 & 1.75 & \bf{499.20} & 
1.21\\CON3-7 & 578.41 & 1.33 & 
579.84 & 1.20 & \bf{577.50} & 
0.16\\CON3-8 & 524.30 & 1.15 & 
524.52 & 1.20 & \bf{523.10} & 
0.23\\CON3-9 & 588.48 & 1.30 & 
588.48 & 1.28 & \bf{578.20} & 
1.78\\CON8-0 & 879.00 & 1.40 & 
879.00 & 1.43 & \bf{858.90} & 
2.34\\CON8-1 & 758.26 & 1.27 & 
758.26 & 1.33 & \bf{740.90} & 
2.34\\CON8-2 & 716.53 & 2.06 & 
717.85 & 2.06 & \bf{714.30} & 
0.31\\CON8-3 & 817.57 & 1.47 & 
817.57 & 1.41 & \bf{812.30} & 
0.65\\CON8-4 & 778.60 & 1.60 & 
782.97 & 1.56 & \bf{770.10} & 
1.10\\CON8-5 & \bf{\underline{764.36}} & 1.30 & 
764.36 & 1.32 & 766.60 & 
-0.29\\CON8-6 & 706.20 & 1.71 & 
707.34 & 1.73 & \bf{697.20} & 
1.29\\CON8-7 & 822.42 & 1.20 & 
823.18 & 1.20 & \bf{814.80} & 
0.94\\CON8-8 & 799.16 & 1.46 & 
799.38 & 1.48 & \bf{771.30} & 
3.61\\CON8-9 & 816.12 & 1.64 & 
816.12 & 1.56 & \bf{815.10} & 
0.13\\[1ex]\hline
\end{tabular}
\label{table:nonlin}
\end{table} \clearpage
\begin{table}[ht]
\caption{Resultados de la ejecución de la metaheurística ACO, utilizando instancias de Dethloff con la configuración -n 2.0 -alpha 1.0 -beta 3.0 -q 5.0 -ro 0.015}
\centering
\small
\begin{tabular}{c c c c c c c}
\hline\hline
Instancia & Costo mínimo & Tiempo(seg.) & Costo promedio & Tiempo promedio(seg.) & Costo ACO & \%Gap \\ [0.5ex]
\hline
SCA3-0 & 640.55 & 1.41 & 
640.55 & 1.35 & \bf{636.10} & 
0.70\\SCA3-1 & \bf{\underline{697.84}} & 1.46 & 
698.76 & 1.51 & 700.10 & 
-0.32\\SCA3-2 & 659.34 & 1.29 & 
662.97 & 1.28 & \bf{659.30} & 
0.01\\SCA3-3 & 680.60 & 1.43 & 
680.60 & 1.44 & \bf{680.00} & 
0.09\\SCA3-4 & \bf{690.50} & 1.34 & 
690.50 & 1.39 & 690.50 & 0.00\\
SCA3-5 & \bf{\underline{665.04}} & 1.42 & 
665.75 & 1.43 & 671.10 & 
-0.90\\SCA3-6 & 655.19 & 1.36 & 
655.19 & 1.34 & \bf{651.10} & 
0.63\\SCA3-7 & 666.15 & 1.00 & 
666.15 & 1.16 & \bf{666.10} & 
0.01\\SCA3-8 & 721.45 & 1.06 & 
724.59 & 1.10 & \bf{719.50} & 
0.27\\SCA3-9 & \bf{681.00} & 0.92 & 
681.00 & 0.92 & 681.00 & 0.00\\
SCA8-0 & 991.07 & 1.54 & 
991.65 & 1.51 & \bf{961.60} & 
3.06\\SCA8-1 & 1074.65 & 1.17 & 
1074.68 & 1.17 & \bf{1063.00} & 
1.10\\SCA8-2 & 1056.87 & 1.02 & 
1056.87 & 1.03 & \bf{1040.60} & 
1.56\\SCA8-3 & 1031.08 & 1.41 & 
1031.08 & 1.49 & \bf{985.90} & 
4.58\\SCA8-4 & 1099.06 & 1.49 & 
1099.17 & 1.51 & \bf{1071.00} & 
2.62\\SCA8-5 & 1055.35 & 1.67 & 
1055.35 & 1.71 & \bf{1054.30} & 
0.10\\SCA8-6 & \bf{\underline{972.48}} & 1.56 & 
972.48 & 1.58 & 972.50 & 
-0.00\\SCA8-7 & 1092.57 & 1.59 & 
1092.57 & 1.71 & \bf{1059.70} & 
3.10\\SCA8-8 & 1091.49 & 1.40 & 
1091.75 & 1.41 & \bf{1082.70} & 
0.81\\SCA8-9 & \bf{\underline{1067.42}} & 1.14 & 
1067.42 & 1.16 & 1081.40 & 
-1.29\\CON3-0 & 624.96 & 1.69 & 
624.96 & 1.63 & \bf{616.50} & 
1.37\\CON3-1 & 557.38 & 1.49 & 
558.66 & 1.49 & \bf{555.60} & 
0.32\\CON3-2 & 524.07 & 1.10 & 
526.21 & 1.12 & \bf{521.40} & 
0.51\\CON3-3 & \bf{591.20} & 1.53 & 
593.09 & 1.47 & 591.20 & 0.00\\
CON3-4 & 589.32 & 1.39 & 
589.32 & 1.40 & \bf{589.30} & 
0.00\\CON3-5 & 569.88 & 1.38 & 
574.33 & 1.38 & \bf{563.70} & 
1.10\\CON3-6 & 504.15 & 1.69 & 
504.98 & 1.77 & \bf{499.20} & 
0.99\\CON3-7 & 578.41 & 1.19 & 
579.12 & 1.19 & \bf{577.50} & 
0.16\\CON3-8 & 524.59 & 1.17 & 
524.59 & 1.20 & \bf{523.10} & 
0.28\\CON3-9 & 588.48 & 1.34 & 
589.00 & 1.31 & \bf{578.20} & 
1.78\\CON8-0 & 879.00 & 1.41 & 
879.00 & 1.56 & \bf{858.90} & 
2.34\\CON8-1 & 754.98 & 1.29 & 
757.44 & 1.31 & \bf{740.90} & 
1.90\\CON8-2 & 716.53 & 1.96 & 
716.54 & 1.98 & \bf{714.30} & 
0.31\\CON8-3 & 817.57 & 1.39 & 
817.57 & 1.42 & \bf{812.30} & 
0.65\\CON8-4 & 781.64 & 1.63 & 
783.73 & 1.61 & \bf{770.10} & 
1.50\\CON8-5 & \bf{\underline{764.36}} & 1.31 & 
764.36 & 1.40 & 766.60 & 
-0.29\\CON8-6 & 705.61 & 1.76 & 
706.77 & 1.72 & \bf{697.20} & 
1.21\\CON8-7 & 822.42 & 1.22 & 
823.18 & 1.20 & \bf{814.80} & 
0.94\\CON8-8 & 799.32 & 1.40 & 
799.41 & 1.49 & \bf{771.30} & 
3.63\\CON8-9 & 816.12 & 1.56 & 
816.12 & 1.53 & \bf{815.10} & 
0.13\\[1ex]\hline
\end{tabular}
\label{table:nonlin}
\end{table} \clearpage
\begin{table}[ht]
\caption{Resultados de la ejecución de la metaheurística ACO, utilizando instancias de Dethloff con la configuración -n 2.0 -alpha 1.0 -beta 3.0 -q 5.1 -ro 0.015}
\centering
\small
\begin{tabular}{c c c c c c c}
\hline\hline
Instancia & Costo mínimo & Tiempo(seg.) & Costo promedio & Tiempo promedio(seg.) & Costo ACO & \%Gap \\ [0.5ex]
\hline
SCA3-0 & 640.55 & 1.34 & 
640.55 & 1.36 & \bf{636.10} & 
0.70\\SCA3-1 & \bf{\underline{697.84}} & 1.52 & 
697.84 & 1.54 & 700.10 & 
-0.32\\SCA3-2 & 664.18 & 1.28 & 
664.18 & 1.40 & \bf{659.30} & 
0.74\\SCA3-3 & 680.60 & 1.45 & 
681.13 & 1.49 & \bf{680.00} & 
0.09\\SCA3-4 & \bf{690.50} & 1.47 & 
690.50 & 1.40 & 690.50 & 0.00\\
SCA3-5 & \bf{\underline{665.04}} & 1.46 & 
665.04 & 1.41 & 671.10 & 
-0.90\\SCA3-6 & 655.19 & 1.37 & 
655.19 & 1.38 & \bf{651.10} & 
0.63\\SCA3-7 & 666.15 & 0.99 & 
666.15 & 1.02 & \bf{666.10} & 
0.01\\SCA3-8 & 721.45 & 1.12 & 
722.70 & 1.15 & \bf{719.50} & 
0.27\\SCA3-9 & \bf{681.00} & 0.93 & 
681.00 & 0.95 & 681.00 & 0.00\\
SCA8-0 & 991.07 & 1.65 & 
992.23 & 1.56 & \bf{961.60} & 
3.06\\SCA8-1 & 1074.65 & 1.19 & 
1074.68 & 1.18 & \bf{1063.00} & 
1.10\\SCA8-2 & 1056.87 & 1.02 & 
1056.87 & 1.01 & \bf{1040.60} & 
1.56\\SCA8-3 & 1031.08 & 1.58 & 
1031.08 & 1.70 & \bf{985.90} & 
4.58\\SCA8-4 & 1099.06 & 1.44 & 
1099.06 & 1.45 & \bf{1071.00} & 
2.62\\SCA8-5 & 1055.35 & 1.68 & 
1055.35 & 1.63 & \bf{1054.30} & 
0.10\\SCA8-6 & \bf{\underline{972.48}} & 1.68 & 
975.00 & 1.69 & 972.50 & 
-0.00\\SCA8-7 & 1092.57 & 1.57 & 
1092.57 & 1.60 & \bf{1059.70} & 
3.10\\SCA8-8 & 1092.02 & 1.47 & 
1092.02 & 1.47 & \bf{1082.70} & 
0.86\\SCA8-9 & \bf{\underline{1067.42}} & 1.19 & 
1067.42 & 1.18 & 1081.40 & 
-1.29\\CON3-0 & 624.96 & 1.73 & 
624.96 & 1.67 & \bf{616.50} & 
1.37\\CON3-1 & 557.38 & 1.46 & 
559.14 & 1.49 & \bf{555.60} & 
0.32\\CON3-2 & 524.07 & 1.09 & 
525.61 & 1.13 & \bf{521.40} & 
0.51\\CON3-3 & 594.11 & 1.54 & 
594.11 & 1.50 & \bf{591.20} & 
0.49\\CON3-4 & 589.32 & 1.38 & 
589.32 & 1.36 & \bf{589.30} & 
0.00\\CON3-5 & 576.43 & 1.42 & 
576.97 & 1.39 & \bf{563.70} & 
2.26\\CON3-6 & 505.26 & 1.75 & 
506.43 & 1.76 & \bf{499.20} & 
1.21\\CON3-7 & 578.41 & 1.25 & 
578.41 & 1.45 & \bf{577.50} & 
0.16\\CON3-8 & 524.59 & 1.14 & 
524.59 & 1.21 & \bf{523.10} & 
0.28\\CON3-9 & 588.48 & 1.22 & 
588.48 & 1.30 & \bf{578.20} & 
1.78\\CON8-0 & 879.00 & 1.46 & 
879.00 & 1.43 & \bf{858.90} & 
2.34\\CON8-1 & 758.26 & 1.42 & 
758.26 & 1.38 & \bf{740.90} & 
2.34\\CON8-2 & 716.53 & 2.00 & 
716.55 & 1.99 & \bf{714.30} & 
0.31\\CON8-3 & 817.57 & 1.43 & 
817.57 & 1.43 & \bf{812.30} & 
0.65\\CON8-4 & 781.64 & 1.52 & 
788.67 & 1.55 & \bf{770.10} & 
1.50\\CON8-5 & \bf{\underline{764.36}} & 1.34 & 
764.36 & 1.34 & 766.60 & 
-0.29\\CON8-6 & 705.61 & 1.74 & 
706.21 & 1.68 & \bf{697.20} & 
1.21\\CON8-7 & 822.42 & 1.13 & 
822.75 & 1.14 & \bf{814.80} & 
0.94\\CON8-8 & 799.32 & 1.49 & 
799.41 & 1.53 & \bf{771.30} & 
3.63\\CON8-9 & 816.12 & 1.55 & 
818.68 & 1.56 & \bf{815.10} & 
0.13\\[1ex]\hline
\end{tabular}
\label{table:nonlin}
\end{table} \clearpage
\begin{table}[ht]
\caption{Resultados de la ejecución de la metaheurística ACO, utilizando instancias de Dethloff con la configuración -n 2.0 -alpha 1.0 -beta 3.0 -q 5.2 -ro 0.015}
\centering
\small
\begin{tabular}{c c c c c c c}
\hline\hline
Instancia & Costo mínimo & Tiempo(seg.) & Costo promedio & Tiempo promedio(seg.) & Costo ACO & \%Gap \\ [0.5ex]
\hline
SCA3-0 & 640.55 & 1.37 & 
640.55 & 1.38 & \bf{636.10} & 
0.70\\SCA3-1 & \bf{\underline{697.84}} & 1.52 & 
697.84 & 1.45 & 700.10 & 
-0.32\\SCA3-2 & 659.34 & 1.32 & 
662.97 & 1.32 & \bf{659.30} & 
0.01\\SCA3-3 & 680.60 & 1.55 & 
680.96 & 1.49 & \bf{680.00} & 
0.09\\SCA3-4 & \bf{690.50} & 1.33 & 
690.50 & 1.38 & 690.50 & 0.00\\
SCA3-5 & \bf{\underline{665.04}} & 1.41 & 
665.04 & 1.43 & 671.10 & 
-0.90\\SCA3-6 & 655.19 & 1.33 & 
655.19 & 1.32 & \bf{651.10} & 
0.63\\SCA3-7 & 666.15 & 1.04 & 
666.15 & 0.97 & \bf{666.10} & 
0.01\\SCA3-8 & 721.45 & 1.11 & 
725.84 & 1.25 & \bf{719.50} & 
0.27\\SCA3-9 & \bf{681.00} & 0.95 & 
681.00 & 0.99 & 681.00 & 0.00\\
SCA8-0 & 991.07 & 1.44 & 
991.65 & 1.58 & \bf{961.60} & 
3.06\\SCA8-1 & 1074.65 & 1.20 & 
1074.68 & 1.21 & \bf{1063.00} & 
1.10\\SCA8-2 & 1056.87 & 1.06 & 
1056.87 & 1.03 & \bf{1040.60} & 
1.56\\SCA8-3 & 1031.08 & 1.52 & 
1031.08 & 1.43 & \bf{985.90} & 
4.58\\SCA8-4 & 1098.34 & 1.42 & 
1098.88 & 1.46 & \bf{1071.00} & 
2.55\\SCA8-5 & 1055.35 & 1.64 & 
1055.35 & 1.69 & \bf{1054.30} & 
0.10\\SCA8-6 & \bf{\underline{972.48}} & 1.77 & 
972.48 & 1.71 & 972.50 & 
-0.00\\SCA8-7 & 1092.57 & 1.57 & 
1092.57 & 1.64 & \bf{1059.70} & 
3.10\\SCA8-8 & 1092.02 & 1.43 & 
1092.02 & 1.45 & \bf{1082.70} & 
0.86\\SCA8-9 & \bf{\underline{1067.42}} & 1.14 & 
1067.42 & 1.15 & 1081.40 & 
-1.29\\CON3-0 & 624.96 & 1.58 & 
624.96 & 1.60 & \bf{616.50} & 
1.37\\CON3-1 & 557.38 & 1.42 & 
557.58 & 1.41 & \bf{555.60} & 
0.32\\CON3-2 & 524.07 & 1.09 & 
524.36 & 1.09 & \bf{521.40} & 
0.51\\CON3-3 & 592.95 & 1.46 & 
593.53 & 1.49 & \bf{591.20} & 
0.30\\CON3-4 & \bf{\underline{588.79}} & 1.23 & 
589.19 & 1.34 & 589.30 & 
-0.09\\CON3-5 & 576.43 & 1.39 & 
576.43 & 1.41 & \bf{563.70} & 
2.26\\CON3-6 & 504.15 & 1.74 & 
504.98 & 1.75 & \bf{499.20} & 
0.99\\CON3-7 & 578.41 & 1.18 & 
579.12 & 1.21 & \bf{577.50} & 
0.16\\CON3-8 & 524.30 & 1.17 & 
524.52 & 1.23 & \bf{523.10} & 
0.23\\CON3-9 & 588.48 & 1.35 & 
588.48 & 1.25 & \bf{578.20} & 
1.78\\CON8-0 & 879.00 & 1.41 & 
879.00 & 1.41 & \bf{858.90} & 
2.34\\CON8-1 & 758.26 & 1.40 & 
758.26 & 1.35 & \bf{740.90} & 
2.34\\CON8-2 & 716.53 & 1.89 & 
717.20 & 1.97 & \bf{714.30} & 
0.31\\CON8-3 & 817.57 & 1.44 & 
817.57 & 1.46 & \bf{812.30} & 
0.65\\CON8-4 & 781.64 & 1.69 & 
783.73 & 1.59 & \bf{770.10} & 
1.50\\CON8-5 & \bf{\underline{764.36}} & 1.32 & 
764.36 & 1.33 & 766.60 & 
-0.29\\CON8-6 & 701.31 & 1.80 & 
705.13 & 1.70 & \bf{697.20} & 
0.59\\CON8-7 & 822.42 & 1.19 & 
822.92 & 1.15 & \bf{814.80} & 
0.94\\CON8-8 & 799.32 & 1.49 & 
799.46 & 1.53 & \bf{771.30} & 
3.63\\CON8-9 & 816.12 & 1.56 & 
816.12 & 1.58 & \bf{815.10} & 
0.13\\[1ex]\hline
\end{tabular}
\label{table:nonlin}
\end{table} \clearpage
\begin{table}[ht]
\caption{Resultados de la ejecución de la metaheurística ACO, utilizando instancias de Dethloff con la configuración -n 2.0 -alpha 1.0 -beta 3.0 -q 5.3 -ro 0.015}
\centering
\small
\begin{tabular}{c c c c c c c}
\hline\hline
Instancia & Costo mínimo & Tiempo(seg.) & Costo promedio & Tiempo promedio(seg.) & Costo ACO & \%Gap \\ [0.5ex]
\hline
SCA3-0 & 640.55 & 1.36 & 
640.55 & 1.38 & \bf{636.10} & 
0.70\\SCA3-1 & \bf{\underline{697.84}} & 1.50 & 
697.84 & 1.51 & 700.10 & 
-0.32\\SCA3-2 & 659.34 & 1.33 & 
662.97 & 1.37 & \bf{659.30} & 
0.01\\SCA3-3 & 680.60 & 1.46 & 
680.78 & 1.50 & \bf{680.00} & 
0.09\\SCA3-4 & \bf{690.50} & 1.39 & 
690.50 & 1.38 & 690.50 & 0.00\\
SCA3-5 & \bf{\underline{665.04}} & 1.41 & 
665.04 & 1.46 & 671.10 & 
-0.90\\SCA3-6 & 655.19 & 1.29 & 
655.30 & 1.37 & \bf{651.10} & 
0.63\\SCA3-7 & 666.15 & 0.95 & 
666.15 & 0.95 & \bf{666.10} & 
0.01\\SCA3-8 & 721.45 & 1.11 & 
727.13 & 1.13 & \bf{719.50} & 
0.27\\SCA3-9 & \bf{681.00} & 0.93 & 
681.00 & 0.99 & 681.00 & 0.00\\
SCA8-0 & 991.07 & 1.50 & 
991.07 & 1.47 & \bf{961.60} & 
3.06\\SCA8-1 & 1074.65 & 1.17 & 
1074.68 & 1.21 & \bf{1063.00} & 
1.10\\SCA8-2 & 1056.87 & 1.07 & 
1056.87 & 1.01 & \bf{1040.60} & 
1.56\\SCA8-3 & 1031.08 & 1.40 & 
1031.08 & 1.43 & \bf{985.90} & 
4.58\\SCA8-4 & 1098.34 & 1.62 & 
1098.88 & 1.52 & \bf{1071.00} & 
2.55\\SCA8-5 & 1055.35 & 1.76 & 
1055.35 & 1.76 & \bf{1054.30} & 
0.10\\SCA8-6 & \bf{\underline{972.48}} & 1.55 & 
976.27 & 1.67 & 972.50 & 
-0.00\\SCA8-7 & 1092.57 & 1.64 & 
1092.57 & 1.60 & \bf{1059.70} & 
3.10\\SCA8-8 & 1091.49 & 1.36 & 
1091.89 & 1.40 & \bf{1082.70} & 
0.81\\SCA8-9 & \bf{\underline{1067.42}} & 1.10 & 
1067.42 & 1.12 & 1081.40 & 
-1.29\\CON3-0 & 624.96 & 1.58 & 
624.96 & 1.64 & \bf{616.50} & 
1.37\\CON3-1 & 557.38 & 1.41 & 
559.34 & 1.43 & \bf{555.60} & 
0.32\\CON3-2 & 524.07 & 1.09 & 
526.41 & 1.10 & \bf{521.40} & 
0.51\\CON3-3 & \bf{591.20} & 1.48 & 
593.38 & 1.49 & 591.20 & 0.00\\
CON3-4 & \bf{\underline{588.79}} & 1.26 & 
589.19 & 1.34 & 589.30 & 
-0.09\\CON3-5 & 576.43 & 1.44 & 
576.43 & 1.43 & \bf{563.70} & 
2.26\\CON3-6 & 505.26 & 1.84 & 
507.61 & 1.76 & \bf{499.20} & 
1.21\\CON3-7 & 578.41 & 1.27 & 
579.84 & 1.20 & \bf{577.50} & 
0.16\\CON3-8 & 524.59 & 1.28 & 
524.59 & 1.18 & \bf{523.10} & 
0.28\\CON3-9 & 588.48 & 1.26 & 
588.48 & 1.25 & \bf{578.20} & 
1.78\\CON8-0 & 879.00 & 1.51 & 
879.00 & 1.45 & \bf{858.90} & 
2.34\\CON8-1 & 758.26 & 1.29 & 
758.26 & 1.29 & \bf{740.90} & 
2.34\\CON8-2 & 716.56 & 1.86 & 
716.56 & 2.01 & \bf{714.30} & 
0.32\\CON8-3 & 817.57 & 1.43 & 
817.57 & 1.42 & \bf{812.30} & 
0.65\\CON8-4 & 781.64 & 1.52 & 
783.73 & 1.53 & \bf{770.10} & 
1.50\\CON8-5 & \bf{\underline{764.36}} & 1.32 & 
764.36 & 1.43 & 766.60 & 
-0.29\\CON8-6 & \bf{\underline{693.83}} & 1.74 & 
703.12 & 1.78 & 697.20 & 
-0.48\\CON8-7 & 822.42 & 1.17 & 
822.67 & 1.20 & \bf{814.80} & 
0.94\\CON8-8 & 799.32 & 1.45 & 
799.37 & 1.52 & \bf{771.30} & 
3.63\\CON8-9 & 816.12 & 1.54 & 
816.12 & 1.54 & \bf{815.10} & 
0.13\\[1ex]\hline
\end{tabular}
\label{table:nonlin}
\end{table} \clearpage
\begin{table}[ht]
\caption{Resultados de la ejecución de la metaheurística ACO, utilizando instancias de Dethloff con la configuración -n 2.0 -alpha 1.0 -beta 3.0 -q 5.4 -ro 0.015}
\centering
\small
\begin{tabular}{c c c c c c c}
\hline\hline
Instancia & Costo mínimo & Tiempo(seg.) & Costo promedio & Tiempo promedio(seg.) & Costo ACO & \%Gap \\ [0.5ex]
\hline
SCA3-0 & 640.55 & 1.39 & 
640.55 & 1.42 & \bf{636.10} & 
0.70\\SCA3-1 & \bf{\underline{697.84}} & 1.46 & 
697.84 & 1.51 & 700.10 & 
-0.32\\SCA3-2 & 659.34 & 1.31 & 
662.97 & 1.35 & \bf{659.30} & 
0.01\\SCA3-3 & 680.60 & 1.53 & 
680.78 & 1.50 & \bf{680.00} & 
0.09\\SCA3-4 & \bf{690.50} & 1.34 & 
690.50 & 1.35 & 690.50 & 0.00\\
SCA3-5 & \bf{\underline{665.04}} & 1.50 & 
665.19 & 1.49 & 671.10 & 
-0.90\\SCA3-6 & 655.19 & 1.35 & 
655.19 & 1.34 & \bf{651.10} & 
0.63\\SCA3-7 & 666.15 & 1.00 & 
666.15 & 1.02 & \bf{666.10} & 
0.01\\SCA3-8 & 721.45 & 1.19 & 
726.48 & 1.19 & \bf{719.50} & 
0.27\\SCA3-9 & \bf{681.00} & 0.97 & 
681.00 & 0.99 & 681.00 & 0.00\\
SCA8-0 & 991.07 & 1.56 & 
991.07 & 1.54 & \bf{961.60} & 
3.06\\SCA8-1 & \bf{\underline{1059.89}} & 1.25 & 
1070.99 & 1.21 & 1063.00 & 
-0.29\\SCA8-2 & 1056.87 & 1.00 & 
1056.87 & 1.00 & \bf{1040.60} & 
1.56\\SCA8-3 & 1031.08 & 1.48 & 
1031.08 & 1.46 & \bf{985.90} & 
4.58\\SCA8-4 & 1098.34 & 1.50 & 
1098.88 & 1.45 & \bf{1071.00} & 
2.55\\SCA8-5 & 1055.35 & 1.68 & 
1055.35 & 1.67 & \bf{1054.30} & 
0.10\\SCA8-6 & \bf{\underline{972.48}} & 1.67 & 
972.48 & 1.72 & 972.50 & 
-0.00\\SCA8-7 & 1092.57 & 1.72 & 
1092.57 & 1.68 & \bf{1059.70} & 
3.10\\SCA8-8 & 1092.02 & 1.49 & 
1092.02 & 1.44 & \bf{1082.70} & 
0.86\\SCA8-9 & \bf{\underline{1067.42}} & 1.19 & 
1067.42 & 1.16 & 1081.40 & 
-1.29\\CON3-0 & 624.96 & 1.67 & 
624.96 & 1.64 & \bf{616.50} & 
1.37\\CON3-1 & 557.38 & 1.50 & 
557.38 & 1.48 & \bf{555.60} & 
0.32\\CON3-2 & 524.07 & 1.14 & 
524.53 & 1.09 & \bf{521.40} & 
0.51\\CON3-3 & 592.95 & 1.56 & 
593.82 & 1.52 & \bf{591.20} & 
0.30\\CON3-4 & 589.32 & 1.30 & 
589.32 & 1.28 & \bf{589.30} & 
0.00\\CON3-5 & 570.70 & 1.41 & 
575.00 & 1.41 & \bf{563.70} & 
1.24\\CON3-6 & 504.15 & 2.30 & 
506.15 & 1.90 & \bf{499.20} & 
0.99\\CON3-7 & 578.41 & 1.20 & 
578.41 & 1.21 & \bf{577.50} & 
0.16\\CON3-8 & 524.30 & 1.17 & 
524.52 & 1.21 & \bf{523.10} & 
0.23\\CON3-9 & 588.48 & 1.20 & 
588.48 & 1.21 & \bf{578.20} & 
1.78\\CON8-0 & 879.00 & 1.42 & 
879.00 & 1.39 & \bf{858.90} & 
2.34\\CON8-1 & 758.26 & 1.40 & 
758.26 & 1.40 & \bf{740.90} & 
2.34\\CON8-2 & 716.53 & 2.07 & 
716.54 & 2.01 & \bf{714.30} & 
0.31\\CON8-3 & 817.57 & 1.48 & 
817.57 & 1.42 & \bf{812.30} & 
0.65\\CON8-4 & 781.64 & 1.54 & 
789.45 & 1.56 & \bf{770.10} & 
1.50\\CON8-5 & \bf{\underline{764.36}} & 1.35 & 
764.36 & 1.37 & 766.60 & 
-0.29\\CON8-6 & 705.61 & 1.75 & 
705.61 & 1.69 & \bf{697.20} & 
1.21\\CON8-7 & 822.42 & 1.20 & 
822.92 & 1.18 & \bf{814.80} & 
0.94\\CON8-8 & 795.08 & 1.60 & 
798.36 & 1.62 & \bf{771.30} & 
3.08\\CON8-9 & 816.12 & 1.52 & 
817.40 & 1.56 & \bf{815.10} & 
0.13\\[1ex]\hline
\end{tabular}
\label{table:nonlin}
\end{table} \clearpage
\begin{table}[ht]
\caption{Resultados de la ejecución de la metaheurística ACO, utilizando instancias de Dethloff con la configuración -n 2.0 -alpha 1.0 -beta 3.0 -q 5.5 -ro 0.015}
\centering
\small
\begin{tabular}{c c c c c c c}
\hline\hline
Instancia & Costo mínimo & Tiempo(seg.) & Costo promedio & Tiempo promedio(seg.) & Costo ACO & \%Gap \\ [0.5ex]
\hline
SCA3-0 & 640.55 & 1.34 & 
640.55 & 1.34 & \bf{636.10} & 
0.70\\SCA3-1 & \bf{\underline{697.84}} & 1.56 & 
698.76 & 1.48 & 700.10 & 
-0.32\\SCA3-2 & 664.18 & 1.26 & 
664.18 & 1.34 & \bf{659.30} & 
0.74\\SCA3-3 & 680.60 & 1.52 & 
680.60 & 1.49 & \bf{680.00} & 
0.09\\SCA3-4 & \bf{690.50} & 1.38 & 
690.50 & 1.39 & 690.50 & 0.00\\
SCA3-5 & \bf{\underline{665.04}} & 1.48 & 
668.75 & 1.45 & 671.10 & 
-0.90\\SCA3-6 & 655.19 & 1.36 & 
655.19 & 1.34 & \bf{651.10} & 
0.63\\SCA3-7 & 666.15 & 0.99 & 
666.15 & 0.99 & \bf{666.10} & 
0.01\\SCA3-8 & 721.45 & 1.12 & 
726.48 & 1.10 & \bf{719.50} & 
0.27\\SCA3-9 & \bf{681.00} & 0.96 & 
681.00 & 0.95 & 681.00 & 0.00\\
SCA8-0 & 991.07 & 1.52 & 
991.65 & 1.53 & \bf{961.60} & 
3.06\\SCA8-1 & 1074.39 & 1.20 & 
1074.59 & 1.18 & \bf{1063.00} & 
1.07\\SCA8-2 & 1056.87 & 0.95 & 
1056.87 & 1.01 & \bf{1040.60} & 
1.56\\SCA8-3 & 1031.08 & 1.54 & 
1031.08 & 1.48 & \bf{985.90} & 
4.58\\SCA8-4 & 1098.34 & 1.52 & 
1098.99 & 1.50 & \bf{1071.00} & 
2.55\\SCA8-5 & 1055.35 & 1.58 & 
1055.35 & 1.62 & \bf{1054.30} & 
0.10\\SCA8-6 & \bf{\underline{972.48}} & 1.75 & 
972.48 & 1.71 & 972.50 & 
-0.00\\SCA8-7 & 1092.57 & 1.66 & 
1092.57 & 1.71 & \bf{1059.70} & 
3.10\\SCA8-8 & 1092.02 & 1.49 & 
1092.02 & 1.43 & \bf{1082.70} & 
0.86\\SCA8-9 & \bf{\underline{1067.42}} & 1.10 & 
1067.42 & 1.15 & 1081.40 & 
-1.29\\CON3-0 & 624.96 & 1.62 & 
624.96 & 1.64 & \bf{616.50} & 
1.37\\CON3-1 & 557.38 & 1.52 & 
558.22 & 1.51 & \bf{555.60} & 
0.32\\CON3-2 & 524.07 & 1.00 & 
524.51 & 1.14 & \bf{521.40} & 
0.51\\CON3-3 & \bf{591.20} & 1.47 & 
592.65 & 1.50 & 591.20 & 0.00\\
CON3-4 & 589.32 & 1.68 & 
589.32 & 1.41 & \bf{589.30} & 
0.00\\CON3-5 & 574.44 & 1.38 & 
575.93 & 1.43 & \bf{563.70} & 
1.91\\CON3-6 & 505.26 & 1.78 & 
505.26 & 1.77 & \bf{499.20} & 
1.21\\CON3-7 & 578.41 & 1.24 & 
579.12 & 1.23 & \bf{577.50} & 
0.16\\CON3-8 & 524.59 & 1.17 & 
524.59 & 1.21 & \bf{523.10} & 
0.28\\CON3-9 & 588.48 & 1.22 & 
588.48 & 1.28 & \bf{578.20} & 
1.78\\CON8-0 & 879.00 & 1.50 & 
879.00 & 1.46 & \bf{858.90} & 
2.34\\CON8-1 & 758.26 & 1.35 & 
758.30 & 1.31 & \bf{740.90} & 
2.34\\CON8-2 & 716.53 & 2.06 & 
716.54 & 2.00 & \bf{714.30} & 
0.31\\CON8-3 & 817.57 & 1.50 & 
817.57 & 1.44 & \bf{812.30} & 
0.65\\CON8-4 & 781.64 & 1.56 & 
787.89 & 1.57 & \bf{770.10} & 
1.50\\CON8-5 & \bf{\underline{764.36}} & 1.35 & 
764.36 & 1.33 & 766.60 & 
-0.29\\CON8-6 & 705.61 & 1.77 & 
706.06 & 1.73 & \bf{697.20} & 
1.21\\CON8-7 & 822.42 & 1.14 & 
822.92 & 1.15 & \bf{814.80} & 
0.94\\CON8-8 & 799.32 & 1.52 & 
799.41 & 1.53 & \bf{771.30} & 
3.63\\CON8-9 & 816.12 & 1.56 & 
816.12 & 1.59 & \bf{815.10} & 
0.13\\[1ex]\hline
\end{tabular}
\label{table:nonlin}
\end{table} \clearpage
\begin{table}[ht]
\caption{Resultados de la ejecución de la metaheurística ACO, utilizando instancias de Dethloff con la configuración -n 2.0 -alpha 1.0 -beta 3.0 -q 5.6 -ro 0.015}
\centering
\small
\begin{tabular}{c c c c c c c}
\hline\hline
Instancia & Costo mínimo & Tiempo(seg.) & Costo promedio & Tiempo promedio(seg.) & Costo ACO & \%Gap \\ [0.5ex]
\hline
SCA3-0 & 640.55 & 1.42 & 
640.55 & 1.52 & \bf{636.10} & 
0.70\\SCA3-1 & \bf{\underline{697.84}} & 1.50 & 
697.84 & 1.51 & 700.10 & 
-0.32\\SCA3-2 & 659.34 & 1.30 & 
662.97 & 1.29 & \bf{659.30} & 
0.01\\SCA3-3 & 680.60 & 1.38 & 
680.60 & 1.41 & \bf{680.00} & 
0.09\\SCA3-4 & \bf{690.50} & 1.48 & 
690.50 & 1.40 & 690.50 & 0.00\\
SCA3-5 & \bf{\underline{665.04}} & 1.36 & 
668.75 & 1.41 & 671.10 & 
-0.90\\SCA3-6 & 655.19 & 1.26 & 
655.19 & 1.32 & \bf{651.10} & 
0.63\\SCA3-7 & 666.15 & 0.92 & 
666.15 & 1.02 & \bf{666.10} & 
0.01\\SCA3-8 & 721.45 & 1.09 & 
725.97 & 1.12 & \bf{719.50} & 
0.27\\SCA3-9 & \bf{681.00} & 0.96 & 
681.00 & 1.00 & 681.00 & 0.00\\
SCA8-0 & 991.07 & 1.46 & 
991.07 & 1.49 & \bf{961.60} & 
3.06\\SCA8-1 & 1074.65 & 1.22 & 
1074.65 & 1.19 & \bf{1063.00} & 
1.10\\SCA8-2 & 1056.87 & 1.06 & 
1056.87 & 1.06 & \bf{1040.60} & 
1.56\\SCA8-3 & 1031.08 & 1.40 & 
1031.08 & 1.44 & \bf{985.90} & 
4.58\\SCA8-4 & 1098.34 & 1.45 & 
1098.99 & 1.44 & \bf{1071.00} & 
2.55\\SCA8-5 & 1055.35 & 1.69 & 
1055.35 & 1.64 & \bf{1054.30} & 
0.10\\SCA8-6 & \bf{\underline{972.48}} & 1.68 & 
976.70 & 1.71 & 972.50 & 
-0.00\\SCA8-7 & 1092.57 & 1.69 & 
1092.57 & 1.59 & \bf{1059.70} & 
3.10\\SCA8-8 & 1091.49 & 1.32 & 
1091.76 & 1.42 & \bf{1082.70} & 
0.81\\SCA8-9 & \bf{\underline{1067.42}} & 1.09 & 
1067.42 & 1.12 & 1081.40 & 
-1.29\\CON3-0 & 624.96 & 1.66 & 
624.96 & 1.64 & \bf{616.50} & 
1.37\\CON3-1 & 557.38 & 1.56 & 
558.22 & 1.50 & \bf{555.60} & 
0.32\\CON3-2 & 524.07 & 1.07 & 
524.35 & 1.14 & \bf{521.40} & 
0.51\\CON3-3 & \bf{591.20} & 1.45 & 
591.93 & 1.50 & 591.20 & 0.00\\
CON3-4 & 589.32 & 1.33 & 
589.32 & 1.28 & \bf{589.30} & 
0.00\\CON3-5 & 570.70 & 1.38 & 
575.00 & 1.49 & \bf{563.70} & 
1.24\\CON3-6 & 505.26 & 1.73 & 
505.26 & 1.75 & \bf{499.20} & 
1.21\\CON3-7 & 578.41 & 1.20 & 
578.41 & 1.21 & \bf{577.50} & 
0.16\\CON3-8 & 524.30 & 1.26 & 
524.45 & 1.22 & \bf{523.10} & 
0.23\\CON3-9 & 588.48 & 1.24 & 
588.48 & 1.25 & \bf{578.20} & 
1.78\\CON8-0 & 879.00 & 1.41 & 
879.00 & 1.42 & \bf{858.90} & 
2.34\\CON8-1 & 758.26 & 1.34 & 
758.26 & 1.31 & \bf{740.90} & 
2.34\\CON8-2 & 716.53 & 1.88 & 
716.54 & 2.00 & \bf{714.30} & 
0.31\\CON8-3 & 817.57 & 1.46 & 
817.57 & 1.42 & \bf{812.30} & 
0.65\\CON8-4 & 781.64 & 1.48 & 
787.89 & 1.51 & \bf{770.10} & 
1.50\\CON8-5 & \bf{\underline{764.36}} & 1.48 & 
764.36 & 1.38 & 766.60 & 
-0.29\\CON8-6 & 705.61 & 1.60 & 
705.76 & 1.65 & \bf{697.20} & 
1.21\\CON8-7 & 822.42 & 1.20 & 
823.18 & 1.18 & \bf{814.80} & 
0.94\\CON8-8 & 799.32 & 1.59 & 
799.46 & 1.58 & \bf{771.30} & 
3.63\\CON8-9 & 816.12 & 1.60 & 
817.40 & 1.59 & \bf{815.10} & 
0.13\\[1ex]\hline
\end{tabular}
\label{table:nonlin}
\end{table} \clearpage
\begin{table}[ht]
\caption{Resultados de la ejecución de la metaheurística ACO, utilizando instancias de Dethloff con la configuración -n 2.0 -alpha 1.0 -beta 3.0 -q 5.7 -ro 0.015}
\centering
\small
\begin{tabular}{c c c c c c c}
\hline\hline
Instancia & Costo mínimo & Tiempo(seg.) & Costo promedio & Tiempo promedio(seg.) & Costo ACO & \%Gap \\ [0.5ex]
\hline
SCA3-0 & 640.55 & 1.35 & 
640.55 & 1.38 & \bf{636.10} & 
0.70\\SCA3-1 & \bf{\underline{697.84}} & 1.47 & 
697.84 & 1.49 & 700.10 & 
-0.32\\SCA3-2 & 664.18 & 1.24 & 
664.18 & 1.28 & \bf{659.30} & 
0.74\\SCA3-3 & 680.60 & 2.11 & 
680.78 & 1.69 & \bf{680.00} & 
0.09\\SCA3-4 & \bf{690.50} & 1.37 & 
690.50 & 1.41 & 690.50 & 0.00\\
SCA3-5 & \bf{\underline{665.04}} & 1.41 & 
665.75 & 1.43 & 671.10 & 
-0.90\\SCA3-6 & 653.69 & 1.37 & 
654.72 & 1.37 & \bf{651.10} & 
0.40\\SCA3-7 & 666.15 & 1.05 & 
666.15 & 1.00 & \bf{666.10} & 
0.01\\SCA3-8 & 721.45 & 1.20 & 
724.59 & 1.13 & \bf{719.50} & 
0.27\\SCA3-9 & \bf{681.00} & 1.09 & 
681.00 & 0.98 & 681.00 & 0.00\\
SCA8-0 & 991.07 & 1.47 & 
991.65 & 1.47 & \bf{961.60} & 
3.06\\SCA8-1 & 1074.39 & 1.19 & 
1074.59 & 1.20 & \bf{1063.00} & 
1.07\\SCA8-2 & 1056.87 & 1.02 & 
1056.87 & 1.06 & \bf{1040.60} & 
1.56\\SCA8-3 & 1031.08 & 1.56 & 
1031.08 & 1.51 & \bf{985.90} & 
4.58\\SCA8-4 & 1098.34 & 1.54 & 
1098.88 & 1.57 & \bf{1071.00} & 
2.55\\SCA8-5 & 1055.35 & 1.56 & 
1055.35 & 1.59 & \bf{1054.30} & 
0.10\\SCA8-6 & \bf{\underline{972.48}} & 1.57 & 
973.62 & 1.68 & 972.50 & 
-0.00\\SCA8-7 & 1092.57 & 1.55 & 
1092.57 & 1.60 & \bf{1059.70} & 
3.10\\SCA8-8 & 1085.93 & 1.35 & 
1090.50 & 1.38 & \bf{1082.70} & 
0.30\\SCA8-9 & \bf{\underline{1067.42}} & 1.10 & 
1067.42 & 1.16 & 1081.40 & 
-1.29\\CON3-0 & 624.96 & 1.59 & 
624.96 & 1.59 & \bf{616.50} & 
1.37\\CON3-1 & 557.38 & 1.46 & 
558.42 & 1.45 & \bf{555.60} & 
0.32\\CON3-2 & 524.07 & 1.17 & 
526.58 & 1.14 & \bf{521.40} & 
0.51\\CON3-3 & 594.11 & 1.51 & 
594.11 & 1.51 & \bf{591.20} & 
0.49\\CON3-4 & 589.32 & 1.35 & 
589.32 & 1.32 & \bf{589.30} & 
0.00\\CON3-5 & 576.43 & 1.40 & 
576.43 & 1.44 & \bf{563.70} & 
2.26\\CON3-6 & 505.26 & 1.78 & 
505.26 & 1.80 & \bf{499.20} & 
1.21\\CON3-7 & 578.41 & 1.22 & 
579.12 & 1.23 & \bf{577.50} & 
0.16\\CON3-8 & 524.30 & 1.41 & 
524.52 & 1.25 & \bf{523.10} & 
0.23\\CON3-9 & 588.48 & 1.28 & 
588.48 & 1.25 & \bf{578.20} & 
1.78\\CON8-0 & 879.00 & 1.47 & 
879.00 & 1.39 & \bf{858.90} & 
2.34\\CON8-1 & 758.26 & 1.34 & 
758.26 & 1.38 & \bf{740.90} & 
2.34\\CON8-2 & 716.53 & 2.06 & 
717.20 & 2.06 & \bf{714.30} & 
0.31\\CON8-3 & 817.57 & 1.47 & 
817.57 & 1.45 & \bf{812.30} & 
0.65\\CON8-4 & 781.64 & 1.62 & 
789.45 & 1.55 & \bf{770.10} & 
1.50\\CON8-5 & \bf{\underline{764.36}} & 1.32 & 
764.36 & 1.36 & 766.60 & 
-0.29\\CON8-6 & 705.61 & 1.75 & 
707.08 & 1.77 & \bf{697.20} & 
1.21\\CON8-7 & 822.42 & 1.11 & 
822.42 & 1.15 & \bf{814.80} & 
0.94\\CON8-8 & 799.32 & 1.52 & 
799.37 & 1.54 & \bf{771.30} & 
3.63\\CON8-9 & 816.12 & 1.66 & 
816.12 & 1.57 & \bf{815.10} & 
0.13\\[1ex]\hline
\end{tabular}
\label{table:nonlin}
\end{table} \clearpage
\begin{table}[ht]
\caption{Resultados de la ejecución de la metaheurística ACO, utilizando instancias de Dethloff con la configuración -n 2.0 -alpha 1.0 -beta 3.0 -q 5.8 -ro 0.015}
\centering
\small
\begin{tabular}{c c c c c c c}
\hline\hline
Instancia & Costo mínimo & Tiempo(seg.) & Costo promedio & Tiempo promedio(seg.) & Costo ACO & \%Gap \\ [0.5ex]
\hline
SCA3-0 & 640.55 & 1.66 & 
640.55 & 1.41 & \bf{636.10} & 
0.70\\SCA3-1 & \bf{\underline{697.84}} & 1.54 & 
697.84 & 1.52 & 700.10 & 
-0.32\\SCA3-2 & 664.18 & 1.32 & 
664.18 & 1.33 & \bf{659.30} & 
0.74\\SCA3-3 & 680.60 & 1.44 & 
680.96 & 1.49 & \bf{680.00} & 
0.09\\SCA3-4 & \bf{690.50} & 1.38 & 
690.50 & 1.39 & 690.50 & 0.00\\
SCA3-5 & \bf{\underline{665.04}} & 1.44 & 
665.79 & 1.38 & 671.10 & 
-0.90\\SCA3-6 & 655.19 & 1.28 & 
655.19 & 1.32 & \bf{651.10} & 
0.63\\SCA3-7 & 666.15 & 1.02 & 
666.15 & 1.00 & \bf{666.10} & 
0.01\\SCA3-8 & 721.45 & 1.12 & 
727.13 & 1.15 & \bf{719.50} & 
0.27\\SCA3-9 & \bf{681.00} & 0.96 & 
681.00 & 0.98 & 681.00 & 0.00\\
SCA8-0 & 991.07 & 1.50 & 
991.07 & 1.56 & \bf{961.60} & 
3.06\\SCA8-1 & 1074.65 & 1.12 & 
1074.68 & 1.18 & \bf{1063.00} & 
1.10\\SCA8-2 & 1056.87 & 0.98 & 
1056.87 & 0.99 & \bf{1040.60} & 
1.56\\SCA8-3 & 1031.08 & 1.49 & 
1031.08 & 1.42 & \bf{985.90} & 
4.58\\SCA8-4 & 1098.34 & 1.55 & 
1098.70 & 1.47 & \bf{1071.00} & 
2.55\\SCA8-5 & 1055.35 & 1.60 & 
1055.35 & 1.63 & \bf{1054.30} & 
0.10\\SCA8-6 & \bf{\underline{972.48}} & 1.64 & 
973.62 & 1.68 & 972.50 & 
-0.00\\SCA8-7 & 1092.57 & 1.58 & 
1092.57 & 1.62 & \bf{1059.70} & 
3.10\\SCA8-8 & 1091.49 & 1.70 & 
1091.76 & 1.51 & \bf{1082.70} & 
0.81\\SCA8-9 & \bf{\underline{1067.42}} & 1.14 & 
1067.42 & 1.15 & 1081.40 & 
-1.29\\CON3-0 & 624.96 & 1.52 & 
624.96 & 1.60 & \bf{616.50} & 
1.37\\CON3-1 & 557.38 & 1.44 & 
557.38 & 1.44 & \bf{555.60} & 
0.32\\CON3-2 & 524.07 & 1.18 & 
524.55 & 1.11 & \bf{521.40} & 
0.51\\CON3-3 & \bf{591.20} & 1.56 & 
593.38 & 1.55 & 591.20 & 0.00\\
CON3-4 & 589.32 & 1.32 & 
589.32 & 1.30 & \bf{589.30} & 
0.00\\CON3-5 & 574.57 & 1.52 & 
575.97 & 1.42 & \bf{563.70} & 
1.93\\CON3-6 & 505.26 & 1.74 & 
506.35 & 1.75 & \bf{499.20} & 
1.21\\CON3-7 & 578.41 & 1.28 & 
579.84 & 1.22 & \bf{577.50} & 
0.16\\CON3-8 & 524.59 & 1.31 & 
524.59 & 1.22 & \bf{523.10} & 
0.28\\CON3-9 & 588.48 & 1.36 & 
588.77 & 1.27 & \bf{578.20} & 
1.78\\CON8-0 & 879.00 & 1.36 & 
879.00 & 1.44 & \bf{858.90} & 
2.34\\CON8-1 & 758.26 & 1.51 & 
758.26 & 1.41 & \bf{740.90} & 
2.34\\CON8-2 & 716.53 & 1.94 & 
716.54 & 1.95 & \bf{714.30} & 
0.31\\CON8-3 & 817.57 & 1.71 & 
817.57 & 1.45 & \bf{812.30} & 
0.65\\CON8-4 & 781.64 & 1.50 & 
788.67 & 1.47 & \bf{770.10} & 
1.50\\CON8-5 & \bf{\underline{764.36}} & 1.39 & 
764.36 & 1.38 & 766.60 & 
-0.29\\CON8-6 & 705.61 & 1.74 & 
705.76 & 1.87 & \bf{697.20} & 
1.21\\CON8-7 & 822.42 & 1.22 & 
822.92 & 1.18 & \bf{814.80} & 
0.94\\CON8-8 & 799.32 & 1.43 & 
799.46 & 1.50 & \bf{771.30} & 
3.63\\CON8-9 & 816.12 & 1.60 & 
817.40 & 1.58 & \bf{815.10} & 
0.13\\[1ex]\hline
\end{tabular}
\label{table:nonlin}
\end{table} \clearpage
\begin{table}[ht]
\caption{Resultados de la ejecución de la metaheurística ACO, utilizando instancias de Dethloff con la configuración -n 2.0 -alpha 1.0 -beta 3.0 -q 5.9 -ro 0.015}
\centering
\small
\begin{tabular}{c c c c c c c}
\hline\hline
Instancia & Costo mínimo & Tiempo(seg.) & Costo promedio & Tiempo promedio(seg.) & Costo ACO & \%Gap \\ [0.5ex]
\hline
SCA3-0 & 636.34 & 1.40 & 
639.50 & 1.37 & \bf{636.10} & 
0.04\\SCA3-1 & \bf{\underline{697.84}} & 1.43 & 
697.84 & 1.45 & 700.10 & 
-0.32\\SCA3-2 & 664.18 & 1.31 & 
664.18 & 1.32 & \bf{659.30} & 
0.74\\SCA3-3 & 680.60 & 1.44 & 
680.60 & 1.55 & \bf{680.00} & 
0.09\\SCA3-4 & \bf{690.50} & 1.34 & 
690.50 & 1.43 & 690.50 & 0.00\\
SCA3-5 & \bf{\underline{665.04}} & 1.34 & 
665.39 & 1.40 & 671.10 & 
-0.90\\SCA3-6 & 655.19 & 1.30 & 
655.19 & 1.32 & \bf{651.10} & 
0.63\\SCA3-7 & 666.15 & 1.05 & 
666.26 & 1.00 & \bf{666.10} & 
0.01\\SCA3-8 & 721.45 & 1.11 & 
727.86 & 1.12 & \bf{719.50} & 
0.27\\SCA3-9 & \bf{681.00} & 1.03 & 
681.00 & 0.96 & 681.00 & 0.00\\
SCA8-0 & 991.07 & 1.50 & 
991.07 & 1.50 & \bf{961.60} & 
3.06\\SCA8-1 & \bf{\underline{1059.89}} & 1.15 & 
1070.96 & 1.18 & 1063.00 & 
-0.29\\SCA8-2 & 1056.87 & 1.03 & 
1056.87 & 1.01 & \bf{1040.60} & 
1.56\\SCA8-3 & 1031.08 & 1.43 & 
1031.08 & 1.45 & \bf{985.90} & 
4.58\\SCA8-4 & 1099.06 & 1.45 & 
1099.06 & 1.47 & \bf{1071.00} & 
2.62\\SCA8-5 & 1055.35 & 1.60 & 
1055.35 & 1.57 & \bf{1054.30} & 
0.10\\SCA8-6 & \bf{\underline{972.48}} & 1.65 & 
976.70 & 1.66 & 972.50 & 
-0.00\\SCA8-7 & 1092.57 & 1.62 & 
1092.57 & 1.64 & \bf{1059.70} & 
3.10\\SCA8-8 & 1091.49 & 1.47 & 
1091.89 & 1.44 & \bf{1082.70} & 
0.81\\SCA8-9 & \bf{\underline{1067.42}} & 1.19 & 
1067.42 & 1.12 & 1081.40 & 
-1.29\\CON3-0 & 624.96 & 1.53 & 
624.96 & 1.56 & \bf{616.50} & 
1.37\\CON3-1 & 557.38 & 1.38 & 
557.58 & 1.42 & \bf{555.60} & 
0.32\\CON3-2 & 524.07 & 1.13 & 
524.35 & 1.10 & \bf{521.40} & 
0.51\\CON3-3 & \bf{591.20} & 1.58 & 
593.38 & 1.52 & 591.20 & 0.00\\
CON3-4 & 589.32 & 1.35 & 
589.32 & 1.35 & \bf{589.30} & 
0.00\\CON3-5 & 576.43 & 1.52 & 
576.43 & 1.52 & \bf{563.70} & 
2.26\\CON3-6 & 505.26 & 1.78 & 
506.43 & 1.76 & \bf{499.20} & 
1.21\\CON3-7 & 578.41 & 1.25 & 
578.41 & 1.23 & \bf{577.50} & 
0.16\\CON3-8 & 524.30 & 1.11 & 
524.52 & 1.17 & \bf{523.10} & 
0.23\\CON3-9 & 588.48 & 1.24 & 
588.48 & 1.30 & \bf{578.20} & 
1.78\\CON8-0 & 879.00 & 1.44 & 
879.00 & 1.43 & \bf{858.90} & 
2.34\\CON8-1 & 758.26 & 1.41 & 
758.28 & 1.36 & \bf{740.90} & 
2.34\\CON8-2 & 716.53 & 1.86 & 
716.54 & 1.97 & \bf{714.30} & 
0.31\\CON8-3 & 817.57 & 1.48 & 
817.57 & 1.49 & \bf{812.30} & 
0.65\\CON8-4 & 781.64 & 1.52 & 
784.50 & 1.53 & \bf{770.10} & 
1.50\\CON8-5 & \bf{\underline{764.36}} & 1.32 & 
764.36 & 1.37 & 766.60 & 
-0.29\\CON8-6 & 707.41 & 1.70 & 
707.76 & 1.68 & \bf{697.20} & 
1.46\\CON8-7 & 822.42 & 1.19 & 
823.18 & 1.16 & \bf{814.80} & 
0.94\\CON8-8 & 799.32 & 1.55 & 
799.46 & 1.53 & \bf{771.30} & 
3.63\\CON8-9 & 816.12 & 1.47 & 
816.12 & 1.55 & \bf{815.10} & 
0.13\\[1ex]\hline
\end{tabular}
\label{table:nonlin}
\end{table} \clearpage
\begin{table}[ht]
\caption{Resultados de la ejecución de la metaheurística ACO, utilizando instancias de Dethloff con la configuración -n 2.0 -alpha 1.0 -beta 3.0 -q 6.0 -ro 0.015}
\centering
\small
\begin{tabular}{c c c c c c c}
\hline\hline
Instancia & Costo mínimo & Tiempo(seg.) & Costo promedio & Tiempo promedio(seg.) & Costo ACO & \%Gap \\ [0.5ex]
\hline
SCA3-0 & 640.55 & 1.33 & 
640.55 & 1.39 & \bf{636.10} & 
0.70\\SCA3-1 & \bf{\underline{697.84}} & 1.43 & 
697.84 & 1.44 & 700.10 & 
-0.32\\SCA3-2 & 664.18 & 1.40 & 
664.18 & 1.35 & \bf{659.30} & 
0.74\\SCA3-3 & 680.60 & 1.54 & 
680.78 & 1.48 & \bf{680.00} & 
0.09\\SCA3-4 & \bf{690.50} & 1.38 & 
690.50 & 1.43 & 690.50 & 0.00\\
SCA3-5 & \bf{\underline{665.04}} & 1.45 & 
665.04 & 1.42 & 671.10 & 
-0.90\\SCA3-6 & 655.19 & 1.38 & 
655.19 & 1.39 & \bf{651.10} & 
0.63\\SCA3-7 & 666.15 & 1.07 & 
666.15 & 1.01 & \bf{666.10} & 
0.01\\SCA3-8 & 721.45 & 1.22 & 
725.24 & 1.19 & \bf{719.50} & 
0.27\\SCA3-9 & \bf{681.00} & 1.01 & 
681.00 & 0.99 & 681.00 & 0.00\\
SCA8-0 & 991.07 & 1.56 & 
992.80 & 1.53 & \bf{961.60} & 
3.06\\SCA8-1 & 1074.39 & 1.14 & 
1074.59 & 1.20 & \bf{1063.00} & 
1.07\\SCA8-2 & 1056.87 & 1.04 & 
1056.87 & 1.03 & \bf{1040.60} & 
1.56\\SCA8-3 & 1031.08 & 1.44 & 
1031.08 & 1.44 & \bf{985.90} & 
4.58\\SCA8-4 & 1098.34 & 1.54 & 
1098.88 & 1.48 & \bf{1071.00} & 
2.55\\SCA8-5 & 1055.35 & 1.58 & 
1055.35 & 1.67 & \bf{1054.30} & 
0.10\\SCA8-6 & \bf{\underline{972.48}} & 1.68 & 
975.00 & 1.69 & 972.50 & 
-0.00\\SCA8-7 & 1092.57 & 1.62 & 
1092.57 & 1.76 & \bf{1059.70} & 
3.10\\SCA8-8 & 1092.02 & 1.53 & 
1092.02 & 1.48 & \bf{1082.70} & 
0.86\\SCA8-9 & \bf{\underline{1067.42}} & 1.20 & 
1067.42 & 1.16 & 1081.40 & 
-1.29\\CON3-0 & 624.96 & 1.72 & 
624.96 & 1.70 & \bf{616.50} & 
1.37\\CON3-1 & 557.38 & 1.40 & 
558.30 & 1.68 & \bf{555.60} & 
0.32\\CON3-2 & 524.07 & 1.09 & 
524.79 & 1.16 & \bf{521.40} & 
0.51\\CON3-3 & \bf{591.20} & 1.51 & 
593.38 & 1.54 & 591.20 & 0.00\\
CON3-4 & 589.32 & 1.36 & 
589.32 & 1.36 & \bf{589.30} & 
0.00\\CON3-5 & 576.43 & 1.34 & 
576.43 & 1.35 & \bf{563.70} & 
2.26\\CON3-6 & 504.15 & 1.85 & 
505.88 & 1.83 & \bf{499.20} & 
0.99\\CON3-7 & 578.41 & 1.18 & 
578.41 & 1.22 & \bf{577.50} & 
0.16\\CON3-8 & 524.30 & 1.17 & 
524.52 & 1.18 & \bf{523.10} & 
0.23\\CON3-9 & 588.48 & 1.41 & 
588.48 & 1.34 & \bf{578.20} & 
1.78\\CON8-0 & 879.00 & 1.44 & 
879.00 & 1.48 & \bf{858.90} & 
2.34\\CON8-1 & 758.26 & 1.38 & 
758.26 & 1.35 & \bf{740.90} & 
2.34\\CON8-2 & 716.53 & 1.85 & 
717.20 & 1.92 & \bf{714.30} & 
0.31\\CON8-3 & 817.57 & 1.43 & 
817.57 & 1.40 & \bf{812.30} & 
0.65\\CON8-4 & 789.98 & 1.60 & 
791.53 & 1.58 & \bf{770.10} & 
2.58\\CON8-5 & \bf{\underline{764.36}} & 1.42 & 
764.36 & 1.42 & 766.60 & 
-0.29\\CON8-6 & \bf{\underline{693.83}} & 1.69 & 
703.12 & 1.69 & 697.20 & 
-0.48\\CON8-7 & 822.42 & 1.18 & 
823.18 & 1.16 & \bf{814.80} & 
0.94\\CON8-8 & 799.32 & 1.52 & 
799.46 & 1.50 & \bf{771.30} & 
3.63\\CON8-9 & 816.12 & 1.64 & 
816.12 & 1.57 & \bf{815.10} & 
0.13\\[1ex]\hline
\end{tabular}
\label{table:nonlin}
\end{table} \clearpage
\begin{table}[ht]
\caption{Resultados de la ejecución de la metaheurística ACO, utilizando instancias de Dethloff con la configuración -n 2.0 -alpha 1.0 -beta 3.0 -q 6.1 -ro 0.015}
\centering
\small
\begin{tabular}{c c c c c c c}
\hline\hline
Instancia & Costo mínimo & Tiempo(seg.) & Costo promedio & Tiempo promedio(seg.) & Costo ACO & \%Gap \\ [0.5ex]
\hline
SCA3-0 & 640.55 & 1.40 & 
640.55 & 1.40 & \bf{636.10} & 
0.70\\SCA3-1 & \bf{\underline{697.84}} & 1.52 & 
697.84 & 1.52 & 700.10 & 
-0.32\\SCA3-2 & 659.34 & 1.36 & 
662.97 & 1.31 & \bf{659.30} & 
0.01\\SCA3-3 & 680.60 & 1.60 & 
680.96 & 1.49 & \bf{680.00} & 
0.09\\SCA3-4 & \bf{690.50} & 1.49 & 
690.50 & 1.43 & 690.50 & 0.00\\
SCA3-5 & \bf{\underline{665.04}} & 1.39 & 
665.04 & 1.44 & 671.10 & 
-0.90\\SCA3-6 & 655.19 & 1.27 & 
655.30 & 1.29 & \bf{651.10} & 
0.63\\SCA3-7 & 666.15 & 1.02 & 
666.15 & 1.00 & \bf{666.10} & 
0.01\\SCA3-8 & 721.45 & 1.20 & 
726.48 & 1.19 & \bf{719.50} & 
0.27\\SCA3-9 & \bf{681.00} & 1.08 & 
681.00 & 1.04 & 681.00 & 0.00\\
SCA8-0 & 991.07 & 1.51 & 
991.65 & 1.53 & \bf{961.60} & 
3.06\\SCA8-1 & 1074.65 & 1.11 & 
1074.68 & 1.20 & \bf{1063.00} & 
1.10\\SCA8-2 & 1056.87 & 1.01 & 
1056.87 & 1.03 & \bf{1040.60} & 
1.56\\SCA8-3 & 1031.08 & 1.52 & 
1031.08 & 1.44 & \bf{985.90} & 
4.58\\SCA8-4 & 1098.34 & 1.45 & 
1098.52 & 1.44 & \bf{1071.00} & 
2.55\\SCA8-5 & 1055.35 & 1.57 & 
1055.35 & 1.68 & \bf{1054.30} & 
0.10\\SCA8-6 & \bf{\underline{972.48}} & 1.65 & 
972.48 & 1.70 & 972.50 & 
-0.00\\SCA8-7 & 1092.57 & 1.74 & 
1092.57 & 1.69 & \bf{1059.70} & 
3.10\\SCA8-8 & 1092.02 & 1.45 & 
1092.02 & 1.44 & \bf{1082.70} & 
0.86\\SCA8-9 & \bf{\underline{1067.42}} & 1.16 & 
1067.42 & 1.17 & 1081.40 & 
-1.29\\CON3-0 & 624.96 & 1.56 & 
624.96 & 1.56 & \bf{616.50} & 
1.37\\CON3-1 & 557.38 & 1.45 & 
559.07 & 1.43 & \bf{555.60} & 
0.32\\CON3-2 & 524.07 & 1.29 & 
524.83 & 1.17 & \bf{521.40} & 
0.51\\CON3-3 & 592.95 & 3.02 & 
593.82 & 1.86 & \bf{591.20} & 
0.30\\CON3-4 & 589.32 & 1.39 & 
589.32 & 1.33 & \bf{589.30} & 
0.00\\CON3-5 & 576.43 & 1.34 & 
576.43 & 1.41 & \bf{563.70} & 
2.26\\CON3-6 & 504.15 & 1.86 & 
504.70 & 1.82 & \bf{499.20} & 
0.99\\CON3-7 & 578.41 & 1.20 & 
579.84 & 1.23 & \bf{577.50} & 
0.16\\CON3-8 & 524.30 & 1.23 & 
525.33 & 1.18 & \bf{523.10} & 
0.23\\CON3-9 & 588.48 & 1.34 & 
588.48 & 1.27 & \bf{578.20} & 
1.78\\CON8-0 & 879.00 & 1.46 & 
879.00 & 1.49 & \bf{858.90} & 
2.34\\CON8-1 & 754.98 & 1.37 & 
757.44 & 1.35 & \bf{740.90} & 
1.90\\CON8-2 & 716.53 & 1.96 & 
716.54 & 1.92 & \bf{714.30} & 
0.31\\CON8-3 & 817.57 & 1.37 & 
817.57 & 1.44 & \bf{812.30} & 
0.65\\CON8-4 & 789.98 & 1.55 & 
791.53 & 1.56 & \bf{770.10} & 
2.58\\CON8-5 & \bf{\underline{764.36}} & 1.39 & 
764.36 & 1.31 & 766.60 & 
-0.29\\CON8-6 & 705.61 & 1.62 & 
705.76 & 1.70 & \bf{697.20} & 
1.21\\CON8-7 & 822.42 & 1.19 & 
823.00 & 1.21 & \bf{814.80} & 
0.94\\CON8-8 & 799.16 & 1.71 & 
799.38 & 1.60 & \bf{771.30} & 
3.61\\CON8-9 & 816.12 & 1.46 & 
816.12 & 1.51 & \bf{815.10} & 
0.13\\[1ex]\hline
\end{tabular}
\label{table:nonlin}
\end{table} \clearpage
\begin{table}[ht]
\caption{Resultados de la ejecución de la metaheurística ACO, utilizando instancias de Dethloff con la configuración -n 2.0 -alpha 1.0 -beta 3.0 -q 6.2 -ro 0.015}
\centering
\small
\begin{tabular}{c c c c c c c}
\hline\hline
Instancia & Costo mínimo & Tiempo(seg.) & Costo promedio & Tiempo promedio(seg.) & Costo ACO & \%Gap \\ [0.5ex]
\hline
SCA3-0 & 640.55 & 1.33 & 
640.55 & 1.40 & \bf{636.10} & 
0.70\\SCA3-1 & \bf{\underline{697.84}} & 1.42 & 
697.84 & 1.53 & 700.10 & 
-0.32\\SCA3-2 & 664.18 & 1.28 & 
664.18 & 1.33 & \bf{659.30} & 
0.74\\SCA3-3 & 680.60 & 1.56 & 
680.96 & 1.51 & \bf{680.00} & 
0.09\\SCA3-4 & \bf{690.50} & 1.43 & 
690.50 & 1.40 & 690.50 & 0.00\\
SCA3-5 & \bf{\underline{665.04}} & 1.53 & 
665.04 & 1.45 & 671.10 & 
-0.90\\SCA3-6 & 655.19 & 1.20 & 
655.19 & 1.34 & \bf{651.10} & 
0.63\\SCA3-7 & 666.15 & 0.97 & 
666.15 & 0.97 & \bf{666.10} & 
0.01\\SCA3-8 & 721.45 & 1.11 & 
723.95 & 1.12 & \bf{719.50} & 
0.27\\SCA3-9 & \bf{681.00} & 1.02 & 
681.00 & 0.96 & 681.00 & 0.00\\
SCA8-0 & 991.07 & 1.52 & 
991.07 & 1.64 & \bf{961.60} & 
3.06\\SCA8-1 & 1074.65 & 1.21 & 
1074.65 & 1.20 & \bf{1063.00} & 
1.10\\SCA8-2 & 1056.87 & 1.01 & 
1056.87 & 1.03 & \bf{1040.60} & 
1.56\\SCA8-3 & 1031.08 & 1.49 & 
1031.08 & 1.46 & \bf{985.90} & 
4.58\\SCA8-4 & 1098.34 & 1.44 & 
1098.88 & 1.54 & \bf{1071.00} & 
2.55\\SCA8-5 & 1055.35 & 1.67 & 
1055.35 & 1.66 & \bf{1054.30} & 
0.10\\SCA8-6 & \bf{\underline{972.48}} & 1.60 & 
975.00 & 1.63 & 972.50 & 
-0.00\\SCA8-7 & 1092.57 & 1.62 & 
1092.57 & 1.62 & \bf{1059.70} & 
3.10\\SCA8-8 & 1084.41 & 1.44 & 
1090.12 & 1.39 & \bf{1082.70} & 
0.16\\SCA8-9 & \bf{\underline{1067.42}} & 1.14 & 
1067.42 & 1.12 & 1081.40 & 
-1.29\\CON3-0 & 624.96 & 1.62 & 
624.96 & 1.62 & \bf{616.50} & 
1.37\\CON3-1 & 557.38 & 1.54 & 
558.30 & 1.47 & \bf{555.60} & 
0.32\\CON3-2 & 524.07 & 1.15 & 
524.31 & 1.17 & \bf{521.40} & 
0.51\\CON3-3 & 594.11 & 1.47 & 
594.11 & 1.46 & \bf{591.20} & 
0.49\\CON3-4 & \bf{\underline{588.79}} & 1.20 & 
589.19 & 1.28 & 589.30 & 
-0.09\\CON3-5 & 569.88 & 1.34 & 
574.79 & 1.43 & \bf{563.70} & 
1.10\\CON3-6 & 505.26 & 1.65 & 
505.26 & 1.77 & \bf{499.20} & 
1.21\\CON3-7 & 578.41 & 1.32 & 
578.41 & 1.26 & \bf{577.50} & 
0.16\\CON3-8 & 524.59 & 1.13 & 
524.59 & 1.17 & \bf{523.10} & 
0.28\\CON3-9 & 588.48 & 1.24 & 
588.48 & 1.26 & \bf{578.20} & 
1.78\\CON8-0 & 879.00 & 1.43 & 
879.00 & 1.42 & \bf{858.90} & 
2.34\\CON8-1 & 758.26 & 1.39 & 
758.26 & 1.38 & \bf{740.90} & 
2.34\\CON8-2 & 716.53 & 2.06 & 
716.54 & 2.00 & \bf{714.30} & 
0.31\\CON8-3 & 817.57 & 1.49 & 
817.57 & 1.46 & \bf{812.30} & 
0.65\\CON8-4 & 781.64 & 1.51 & 
788.67 & 1.61 & \bf{770.10} & 
1.50\\CON8-5 & \bf{\underline{764.36}} & 1.40 & 
764.36 & 1.41 & 766.60 & 
-0.29\\CON8-6 & \bf{\underline{693.83}} & 1.65 & 
699.72 & 1.68 & 697.20 & 
-0.48\\CON8-7 & 822.42 & 1.20 & 
823.18 & 1.16 & \bf{814.80} & 
0.94\\CON8-8 & 799.32 & 1.54 & 
799.41 & 1.51 & \bf{771.30} & 
3.63\\CON8-9 & 816.12 & 1.65 & 
816.12 & 1.59 & \bf{815.10} & 
0.13\\[1ex]\hline
\end{tabular}
\label{table:nonlin}
\end{table} \clearpage
\begin{table}[ht]
\caption{Resultados de la ejecución de la metaheurística ACO, utilizando instancias de Dethloff con la configuración -n 2.0 -alpha 1.0 -beta 3.0 -q 6.3 -ro 0.015}
\centering
\small
\begin{tabular}{c c c c c c c}
\hline\hline
Instancia & Costo mínimo & Tiempo(seg.) & Costo promedio & Tiempo promedio(seg.) & Costo ACO & \%Gap \\ [0.5ex]
\hline
SCA3-0 & 640.55 & 1.48 & 
640.55 & 1.40 & \bf{636.10} & 
0.70\\SCA3-1 & \bf{\underline{697.84}} & 1.53 & 
697.84 & 1.51 & 700.10 & 
-0.32\\SCA3-2 & 659.34 & 1.33 & 
662.97 & 1.35 & \bf{659.30} & 
0.01\\SCA3-3 & 680.60 & 1.43 & 
680.78 & 1.47 & \bf{680.00} & 
0.09\\SCA3-4 & \bf{690.50} & 1.39 & 
690.50 & 1.44 & 690.50 & 0.00\\
SCA3-5 & \bf{\underline{665.04}} & 1.54 & 
665.04 & 1.45 & 671.10 & 
-0.90\\SCA3-6 & 655.19 & 1.39 & 
655.19 & 1.34 & \bf{651.10} & 
0.63\\SCA3-7 & 666.15 & 1.05 & 
666.15 & 1.01 & \bf{666.10} & 
0.01\\SCA3-8 & 721.45 & 1.10 & 
725.24 & 1.11 & \bf{719.50} & 
0.27\\SCA3-9 & \bf{681.00} & 0.96 & 
681.00 & 0.96 & 681.00 & 0.00\\
SCA8-0 & 991.07 & 1.46 & 
991.65 & 1.52 & \bf{961.60} & 
3.06\\SCA8-1 & 1074.39 & 1.22 & 
1074.59 & 1.21 & \bf{1063.00} & 
1.07\\SCA8-2 & 1056.87 & 0.99 & 
1056.87 & 1.02 & \bf{1040.60} & 
1.56\\SCA8-3 & 1031.08 & 1.48 & 
1031.08 & 1.48 & \bf{985.90} & 
4.58\\SCA8-4 & 1099.06 & 1.47 & 
1099.06 & 1.49 & \bf{1071.00} & 
2.62\\SCA8-5 & 1055.35 & 1.62 & 
1055.35 & 1.60 & \bf{1054.30} & 
0.10\\SCA8-6 & \bf{\underline{972.48}} & 1.70 & 
975.00 & 1.70 & 972.50 & 
-0.00\\SCA8-7 & 1092.57 & 1.66 & 
1092.57 & 1.62 & \bf{1059.70} & 
3.10\\SCA8-8 & 1092.02 & 1.48 & 
1092.02 & 1.44 & \bf{1082.70} & 
0.86\\SCA8-9 & \bf{\underline{1067.42}} & 1.11 & 
1067.42 & 1.15 & 1081.40 & 
-1.29\\CON3-0 & 624.96 & 1.67 & 
624.96 & 1.65 & \bf{616.50} & 
1.37\\CON3-1 & 557.38 & 1.40 & 
557.38 & 1.46 & \bf{555.60} & 
0.32\\CON3-2 & 524.07 & 1.10 & 
524.79 & 1.09 & \bf{521.40} & 
0.51\\CON3-3 & 594.11 & 1.52 & 
594.11 & 1.52 & \bf{591.20} & 
0.49\\CON3-4 & 589.32 & 1.38 & 
589.32 & 1.35 & \bf{589.30} & 
0.00\\CON3-5 & 576.43 & 1.39 & 
576.43 & 1.39 & \bf{563.70} & 
2.26\\CON3-6 & 504.70 & 1.79 & 
506.05 & 1.80 & \bf{499.20} & 
1.10\\CON3-7 & 578.41 & 1.22 & 
579.12 & 1.23 & \bf{577.50} & 
0.16\\CON3-8 & 524.30 & 1.16 & 
524.52 & 1.18 & \bf{523.10} & 
0.23\\CON3-9 & 588.48 & 1.24 & 
588.48 & 1.27 & \bf{578.20} & 
1.78\\CON8-0 & 879.00 & 1.42 & 
879.00 & 1.39 & \bf{858.90} & 
2.34\\CON8-1 & 758.26 & 1.28 & 
758.26 & 1.32 & \bf{740.90} & 
2.34\\CON8-2 & 716.53 & 2.00 & 
716.54 & 2.06 & \bf{714.30} & 
0.31\\CON8-3 & 817.57 & 1.35 & 
817.57 & 1.42 & \bf{812.30} & 
0.65\\CON8-4 & 789.98 & 1.56 & 
791.53 & 1.54 & \bf{770.10} & 
2.58\\CON8-5 & \bf{\underline{764.36}} & 1.40 & 
764.36 & 1.37 & 766.60 & 
-0.29\\CON8-6 & 707.41 & 1.73 & 
707.64 & 1.72 & \bf{697.20} & 
1.46\\CON8-7 & 822.42 & 1.25 & 
822.67 & 1.17 & \bf{814.80} & 
0.94\\CON8-8 & 799.32 & 1.48 & 
799.46 & 1.47 & \bf{771.30} & 
3.63\\CON8-9 & 816.12 & 1.58 & 
817.40 & 1.65 & \bf{815.10} & 
0.13\\[1ex]\hline
\end{tabular}
\label{table:nonlin}
\end{table} \clearpage
\begin{table}[ht]
\caption{Resultados de la ejecución de la metaheurística ACO, utilizando instancias de Dethloff con la configuración -n 2.0 -alpha 1.0 -beta 3.0 -q 6.4 -ro 0.015}
\centering
\small
\begin{tabular}{c c c c c c c}
\hline\hline
Instancia & Costo mínimo & Tiempo(seg.) & Costo promedio & Tiempo promedio(seg.) & Costo ACO & \%Gap \\ [0.5ex]
\hline
SCA3-0 & 640.55 & 1.38 & 
640.55 & 1.36 & \bf{636.10} & 
0.70\\SCA3-1 & \bf{\underline{697.84}} & 1.48 & 
697.84 & 1.50 & 700.10 & 
-0.32\\SCA3-2 & 664.18 & 1.29 & 
664.18 & 1.32 & \bf{659.30} & 
0.74\\SCA3-3 & 680.60 & 1.46 & 
680.78 & 1.47 & \bf{680.00} & 
0.09\\SCA3-4 & \bf{690.50} & 1.46 & 
690.50 & 1.40 & 690.50 & 0.00\\
SCA3-5 & \bf{\underline{665.04}} & 1.48 & 
668.60 & 1.45 & 671.10 & 
-0.90\\SCA3-6 & 655.19 & 1.43 & 
655.19 & 1.38 & \bf{651.10} & 
0.63\\SCA3-7 & 666.15 & 0.98 & 
666.15 & 1.01 & \bf{666.10} & 
0.01\\SCA3-8 & 721.45 & 1.20 & 
726.09 & 1.17 & \bf{719.50} & 
0.27\\SCA3-9 & \bf{681.00} & 0.99 & 
681.00 & 1.01 & 681.00 & 0.00\\
SCA8-0 & 991.07 & 1.60 & 
991.65 & 1.53 & \bf{961.60} & 
3.06\\SCA8-1 & 1074.65 & 1.14 & 
1074.65 & 1.16 & \bf{1063.00} & 
1.10\\SCA8-2 & 1056.87 & 1.08 & 
1056.87 & 1.04 & \bf{1040.60} & 
1.56\\SCA8-3 & 1031.08 & 1.41 & 
1031.08 & 1.46 & \bf{985.90} & 
4.58\\SCA8-4 & 1099.06 & 1.44 & 
1099.06 & 1.60 & \bf{1071.00} & 
2.62\\SCA8-5 & 1055.35 & 1.62 & 
1055.35 & 1.64 & \bf{1054.30} & 
0.10\\SCA8-6 & \bf{\underline{972.48}} & 1.59 & 
972.48 & 1.62 & 972.50 & 
-0.00\\SCA8-7 & 1092.57 & 1.67 & 
1092.57 & 1.62 & \bf{1059.70} & 
3.10\\SCA8-8 & 1091.49 & 1.40 & 
1091.76 & 1.37 & \bf{1082.70} & 
0.81\\SCA8-9 & \bf{\underline{1067.42}} & 1.16 & 
1067.42 & 1.15 & 1081.40 & 
-1.29\\CON3-0 & 624.96 & 1.56 & 
624.96 & 1.60 & \bf{616.50} & 
1.37\\CON3-1 & 557.38 & 1.52 & 
557.58 & 1.48 & \bf{555.60} & 
0.32\\CON3-2 & 524.07 & 1.15 & 
524.07 & 1.24 & \bf{521.40} & 
0.51\\CON3-3 & \bf{591.20} & 1.48 & 
593.09 & 1.52 & 591.20 & 0.00\\
CON3-4 & 589.32 & 1.34 & 
589.32 & 1.34 & \bf{589.30} & 
0.00\\CON3-5 & 576.43 & 1.43 & 
576.43 & 1.38 & \bf{563.70} & 
2.26\\CON3-6 & 505.26 & 1.76 & 
506.43 & 1.80 & \bf{499.20} & 
1.21\\CON3-7 & 578.41 & 1.31 & 
579.12 & 1.27 & \bf{577.50} & 
0.16\\CON3-8 & 524.59 & 1.21 & 
524.59 & 1.19 & \bf{523.10} & 
0.28\\CON3-9 & 588.48 & 1.35 & 
588.48 & 1.33 & \bf{578.20} & 
1.78\\CON8-0 & 879.00 & 1.37 & 
879.00 & 1.47 & \bf{858.90} & 
2.34\\CON8-1 & 758.26 & 1.36 & 
758.26 & 1.36 & \bf{740.90} & 
2.34\\CON8-2 & 716.53 & 2.30 & 
716.55 & 2.06 & \bf{714.30} & 
0.31\\CON8-3 & 817.57 & 1.44 & 
817.57 & 1.41 & \bf{812.30} & 
0.65\\CON8-4 & 789.98 & 1.57 & 
791.53 & 1.57 & \bf{770.10} & 
2.58\\CON8-5 & \bf{\underline{764.36}} & 1.40 & 
764.36 & 1.47 & 766.60 & 
-0.29\\CON8-6 & 707.41 & 1.76 & 
707.64 & 1.70 & \bf{697.20} & 
1.46\\CON8-7 & 823.43 & 1.12 & 
823.43 & 1.17 & \bf{814.80} & 
1.06\\CON8-8 & 799.16 & 1.51 & 
799.42 & 1.53 & \bf{771.30} & 
3.61\\CON8-9 & 816.12 & 1.56 & 
817.40 & 1.54 & \bf{815.10} & 
0.13\\[1ex]\hline
\end{tabular}
\label{table:nonlin}
\end{table} \clearpage
\begin{table}[ht]
\caption{Resultados de la ejecución de la metaheurística ACO, utilizando instancias de Dethloff con la configuración -n 2.0 -alpha 1.0 -beta 3.0 -q 6.5 -ro 0.015}
\centering
\small
\begin{tabular}{c c c c c c c}
\hline\hline
Instancia & Costo mínimo & Tiempo(seg.) & Costo promedio & Tiempo promedio(seg.) & Costo ACO & \%Gap \\ [0.5ex]
\hline
SCA3-0 & 640.55 & 1.30 & 
640.55 & 1.35 & \bf{636.10} & 
0.70\\SCA3-1 & \bf{\underline{697.84}} & 1.44 & 
697.84 & 1.48 & 700.10 & 
-0.32\\SCA3-2 & 659.34 & 1.27 & 
662.97 & 1.33 & \bf{659.30} & 
0.01\\SCA3-3 & 680.60 & 1.55 & 
680.78 & 1.51 & \bf{680.00} & 
0.09\\SCA3-4 & \bf{690.50} & 1.42 & 
690.50 & 1.48 & 690.50 & 0.00\\
SCA3-5 & \bf{\underline{665.04}} & 1.36 & 
665.04 & 1.43 & 671.10 & 
-0.90\\SCA3-6 & 655.19 & 1.36 & 
655.19 & 1.35 & \bf{651.10} & 
0.63\\SCA3-7 & 666.15 & 0.93 & 
666.15 & 0.95 & \bf{666.10} & 
0.01\\SCA3-8 & 721.45 & 1.10 & 
726.48 & 1.14 & \bf{719.50} & 
0.27\\SCA3-9 & \bf{681.00} & 0.95 & 
681.00 & 0.99 & 681.00 & 0.00\\
SCA8-0 & 991.07 & 1.59 & 
991.07 & 1.51 & \bf{961.60} & 
3.06\\SCA8-1 & 1069.40 & 1.19 & 
1073.34 & 1.20 & \bf{1063.00} & 
0.60\\SCA8-2 & 1056.87 & 1.03 & 
1056.87 & 1.27 & \bf{1040.60} & 
1.56\\SCA8-3 & 1031.08 & 1.44 & 
1031.08 & 1.49 & \bf{985.90} & 
4.58\\SCA8-4 & 1099.06 & 1.58 & 
1099.06 & 1.57 & \bf{1071.00} & 
2.62\\SCA8-5 & 1055.35 & 1.77 & 
1055.35 & 1.69 & \bf{1054.30} & 
0.10\\SCA8-6 & \bf{\underline{972.48}} & 1.66 & 
972.48 & 1.70 & 972.50 & 
-0.00\\SCA8-7 & 1092.57 & 1.52 & 
1092.57 & 1.61 & \bf{1059.70} & 
3.10\\SCA8-8 & 1091.49 & 1.48 & 
1091.89 & 1.45 & \bf{1082.70} & 
0.81\\SCA8-9 & \bf{\underline{1067.42}} & 1.21 & 
1067.42 & 1.12 & 1081.40 & 
-1.29\\CON3-0 & 624.96 & 1.73 & 
624.96 & 1.66 & \bf{616.50} & 
1.37\\CON3-1 & 557.38 & 1.42 & 
558.30 & 1.45 & \bf{555.60} & 
0.32\\CON3-2 & 525.84 & 1.07 & 
526.22 & 1.06 & \bf{521.40} & 
0.85\\CON3-3 & \bf{591.20} & 1.51 & 
593.38 & 1.56 & 591.20 & 0.00\\
CON3-4 & 589.32 & 1.39 & 
589.32 & 1.36 & \bf{589.30} & 
0.00\\CON3-5 & 576.43 & 1.41 & 
576.43 & 1.44 & \bf{563.70} & 
2.26\\CON3-6 & 505.26 & 1.70 & 
508.78 & 1.80 & \bf{499.20} & 
1.21\\CON3-7 & 578.41 & 1.16 & 
579.84 & 1.28 & \bf{577.50} & 
0.16\\CON3-8 & 524.30 & 1.24 & 
524.37 & 1.17 & \bf{523.10} & 
0.23\\CON3-9 & 588.48 & 1.28 & 
588.48 & 1.26 & \bf{578.20} & 
1.78\\CON8-0 & 879.00 & 1.50 & 
879.00 & 1.44 & \bf{858.90} & 
2.34\\CON8-1 & 758.26 & 1.43 & 
758.26 & 1.37 & \bf{740.90} & 
2.34\\CON8-2 & 716.53 & 2.04 & 
716.54 & 2.03 & \bf{714.30} & 
0.31\\CON8-3 & 817.57 & 1.47 & 
817.57 & 1.41 & \bf{812.30} & 
0.65\\CON8-4 & 781.64 & 1.56 & 
787.37 & 1.55 & \bf{770.10} & 
1.50\\CON8-5 & \bf{\underline{764.36}} & 1.40 & 
764.36 & 1.35 & 766.60 & 
-0.29\\CON8-6 & \bf{\underline{693.83}} & 1.70 & 
703.12 & 1.71 & 697.20 & 
-0.48\\CON8-7 & 822.42 & 1.18 & 
823.00 & 1.18 & \bf{814.80} & 
0.94\\CON8-8 & 799.16 & 1.52 & 
799.33 & 1.51 & \bf{771.30} & 
3.61\\CON8-9 & 816.12 & 1.60 & 
818.68 & 1.54 & \bf{815.10} & 
0.13\\[1ex]\hline
\end{tabular}
\label{table:nonlin}
\end{table} \clearpage
\begin{table}[ht]
\caption{Resultados de la ejecución de la metaheurística ACO, utilizando instancias de Dethloff con la configuración -n 2.0 -alpha 1.0 -beta 3.0 -q 6.6 -ro 0.015}
\centering
\small
\begin{tabular}{c c c c c c c}
\hline\hline
Instancia & Costo mínimo & Tiempo(seg.) & Costo promedio & Tiempo promedio(seg.) & Costo ACO & \%Gap \\ [0.5ex]
\hline
SCA3-0 & 640.55 & 1.35 & 
640.55 & 1.43 & \bf{636.10} & 
0.70\\SCA3-1 & \bf{\underline{697.84}} & 1.44 & 
697.84 & 1.54 & 700.10 & 
-0.32\\SCA3-2 & 664.18 & 1.33 & 
664.18 & 1.31 & \bf{659.30} & 
0.74\\SCA3-3 & 680.60 & 1.50 & 
680.96 & 1.45 & \bf{680.00} & 
0.09\\SCA3-4 & \bf{690.50} & 1.42 & 
690.50 & 1.45 & 690.50 & 0.00\\
SCA3-5 & \bf{\underline{665.04}} & 1.52 & 
665.04 & 1.46 & 671.10 & 
-0.90\\SCA3-6 & 655.19 & 1.37 & 
655.41 & 1.38 & \bf{651.10} & 
0.63\\SCA3-7 & 666.15 & 1.09 & 
666.15 & 1.11 & \bf{666.10} & 
0.01\\SCA3-8 & 726.44 & 1.18 & 
727.73 & 1.15 & \bf{719.50} & 
0.96\\SCA3-9 & \bf{681.00} & 0.96 & 
681.00 & 0.93 & 681.00 & 0.00\\
SCA8-0 & 991.07 & 1.52 & 
991.07 & 1.48 & \bf{961.60} & 
3.06\\SCA8-1 & 1074.65 & 1.22 & 
1074.65 & 1.21 & \bf{1063.00} & 
1.10\\SCA8-2 & 1056.87 & 1.07 & 
1056.87 & 1.02 & \bf{1040.60} & 
1.56\\SCA8-3 & 1031.08 & 1.45 & 
1031.08 & 1.44 & \bf{985.90} & 
4.58\\SCA8-4 & 1099.06 & 1.45 & 
1099.06 & 1.64 & \bf{1071.00} & 
2.62\\SCA8-5 & 1055.35 & 1.60 & 
1055.35 & 1.61 & \bf{1054.30} & 
0.10\\SCA8-6 & \bf{\underline{972.48}} & 1.64 & 
972.48 & 1.63 & 972.50 & 
-0.00\\SCA8-7 & 1092.57 & 1.66 & 
1092.57 & 1.62 & \bf{1059.70} & 
3.10\\SCA8-8 & 1092.02 & 1.46 & 
1092.02 & 1.45 & \bf{1082.70} & 
0.86\\SCA8-9 & \bf{\underline{1067.42}} & 1.18 & 
1067.42 & 1.14 & 1081.40 & 
-1.29\\CON3-0 & 624.96 & 1.66 & 
624.96 & 1.64 & \bf{616.50} & 
1.37\\CON3-1 & 557.38 & 1.50 & 
557.58 & 1.52 & \bf{555.60} & 
0.32\\CON3-2 & 524.07 & 1.04 & 
524.35 & 1.09 & \bf{521.40} & 
0.51\\CON3-3 & 592.95 & 1.46 & 
593.82 & 1.47 & \bf{591.20} & 
0.30\\CON3-4 & 589.32 & 1.22 & 
589.32 & 1.24 & \bf{589.30} & 
0.00\\CON3-5 & 576.43 & 1.47 & 
576.43 & 1.45 & \bf{563.70} & 
2.26\\CON3-6 & 505.26 & 1.76 & 
506.84 & 1.82 & \bf{499.20} & 
1.21\\CON3-7 & 578.41 & 1.32 & 
579.12 & 1.28 & \bf{577.50} & 
0.16\\CON3-8 & 524.59 & 1.19 & 
524.59 & 1.17 & \bf{523.10} & 
0.28\\CON3-9 & 588.48 & 1.37 & 
588.48 & 1.29 & \bf{578.20} & 
1.78\\CON8-0 & 879.00 & 1.44 & 
879.00 & 1.46 & \bf{858.90} & 
2.34\\CON8-1 & 758.26 & 1.25 & 
758.26 & 1.31 & \bf{740.90} & 
2.34\\CON8-2 & 716.53 & 1.92 & 
716.55 & 1.97 & \bf{714.30} & 
0.31\\CON8-3 & 817.57 & 1.43 & 
817.57 & 1.38 & \bf{812.30} & 
0.65\\CON8-4 & 781.64 & 1.53 & 
783.73 & 1.65 & \bf{770.10} & 
1.50\\CON8-5 & \bf{\underline{764.36}} & 1.34 & 
764.36 & 1.36 & 766.60 & 
-0.29\\CON8-6 & 705.61 & 1.69 & 
706.77 & 1.69 & \bf{697.20} & 
1.21\\CON8-7 & 822.42 & 1.21 & 
822.92 & 1.15 & \bf{814.80} & 
0.94\\CON8-8 & 799.32 & 1.53 & 
799.46 & 1.55 & \bf{771.30} & 
3.63\\CON8-9 & 816.12 & 1.52 & 
816.12 & 1.61 & \bf{815.10} & 
0.13\\[1ex]\hline
\end{tabular}
\label{table:nonlin}
\end{table} \clearpage
\begin{table}[ht]
\caption{Resultados de la ejecución de la metaheurística ACO, utilizando instancias de Dethloff con la configuración -n 2.0 -alpha 1.0 -beta 3.0 -q 6.7 -ro 0.015}
\centering
\small
\begin{tabular}{c c c c c c c}
\hline\hline
Instancia & Costo mínimo & Tiempo(seg.) & Costo promedio & Tiempo promedio(seg.) & Costo ACO & \%Gap \\ [0.5ex]
\hline
SCA3-0 & 640.55 & 1.43 & 
640.55 & 1.39 & \bf{636.10} & 
0.70\\SCA3-1 & \bf{\underline{697.84}} & 1.61 & 
698.76 & 1.51 & 700.10 & 
-0.32\\SCA3-2 & 664.18 & 1.32 & 
664.18 & 1.39 & \bf{659.30} & 
0.74\\SCA3-3 & 680.60 & 1.53 & 
680.60 & 1.54 & \bf{680.00} & 
0.09\\SCA3-4 & \bf{690.50} & 1.32 & 
690.50 & 1.45 & 690.50 & 0.00\\
SCA3-5 & \bf{\underline{665.04}} & 1.38 & 
665.04 & 1.45 & 671.10 & 
-0.90\\SCA3-6 & 653.81 & 1.41 & 
654.85 & 1.33 & \bf{651.10} & 
0.42\\SCA3-7 & 666.15 & 1.00 & 
666.15 & 1.01 & \bf{666.10} & 
0.01\\SCA3-8 & 721.45 & 1.12 & 
727.86 & 1.13 & \bf{719.50} & 
0.27\\SCA3-9 & \bf{681.00} & 1.02 & 
681.00 & 1.00 & 681.00 & 0.00\\
SCA8-0 & 991.07 & 1.52 & 
991.07 & 1.51 & \bf{961.60} & 
3.06\\SCA8-1 & 1074.65 & 1.13 & 
1074.65 & 1.18 & \bf{1063.00} & 
1.10\\SCA8-2 & 1056.87 & 1.08 & 
1056.87 & 1.05 & \bf{1040.60} & 
1.56\\SCA8-3 & 1031.08 & 1.41 & 
1031.08 & 1.39 & \bf{985.90} & 
4.58\\SCA8-4 & 1099.06 & 1.42 & 
1099.06 & 1.48 & \bf{1071.00} & 
2.62\\SCA8-5 & 1055.35 & 1.63 & 
1055.35 & 1.67 & \bf{1054.30} & 
0.10\\SCA8-6 & \bf{\underline{972.48}} & 1.68 & 
975.00 & 1.65 & 972.50 & 
-0.00\\SCA8-7 & 1092.57 & 1.68 & 
1092.57 & 1.65 & \bf{1059.70} & 
3.10\\SCA8-8 & 1085.93 & 1.40 & 
1090.37 & 1.43 & \bf{1082.70} & 
0.30\\SCA8-9 & \bf{\underline{1067.42}} & 1.08 & 
1067.42 & 1.11 & 1081.40 & 
-1.29\\CON3-0 & 624.96 & 1.64 & 
624.96 & 1.58 & \bf{616.50} & 
1.37\\CON3-1 & 557.38 & 1.50 & 
557.38 & 1.48 & \bf{555.60} & 
0.32\\CON3-2 & 524.07 & 1.12 & 
524.27 & 1.11 & \bf{521.40} & 
0.51\\CON3-3 & 594.11 & 1.50 & 
594.11 & 1.53 & \bf{591.20} & 
0.49\\CON3-4 & 589.32 & 1.29 & 
589.32 & 1.59 & \bf{589.30} & 
0.00\\CON3-5 & 570.70 & 1.36 & 
575.00 & 1.43 & \bf{563.70} & 
1.24\\CON3-6 & 505.26 & 1.77 & 
506.76 & 1.80 & \bf{499.20} & 
1.21\\CON3-7 & 578.41 & 1.17 & 
579.84 & 1.20 & \bf{577.50} & 
0.16\\CON3-8 & 524.59 & 1.16 & 
524.59 & 1.19 & \bf{523.10} & 
0.28\\CON3-9 & 588.48 & 1.32 & 
588.48 & 1.26 & \bf{578.20} & 
1.78\\CON8-0 & 879.00 & 1.44 & 
879.00 & 1.47 & \bf{858.90} & 
2.34\\CON8-1 & 758.26 & 1.38 & 
758.28 & 1.33 & \bf{740.90} & 
2.34\\CON8-2 & 716.53 & 1.97 & 
716.54 & 1.92 & \bf{714.30} & 
0.31\\CON8-3 & 817.57 & 1.41 & 
817.57 & 1.41 & \bf{812.30} & 
0.65\\CON8-4 & 781.64 & 1.67 & 
786.59 & 1.70 & \bf{770.10} & 
1.50\\CON8-5 & \bf{\underline{764.36}} & 1.36 & 
764.36 & 1.38 & 766.60 & 
-0.29\\CON8-6 & \bf{\underline{693.83}} & 1.78 & 
703.57 & 1.78 & 697.20 & 
-0.48\\CON8-7 & 822.42 & 1.21 & 
822.92 & 1.21 & \bf{814.80} & 
0.94\\CON8-8 & 799.16 & 1.59 & 
799.38 & 1.58 & \bf{771.30} & 
3.61\\CON8-9 & 816.12 & 1.46 & 
816.12 & 1.53 & \bf{815.10} & 
0.13\\[1ex]\hline
\end{tabular}
\label{table:nonlin}
\end{table} \clearpage
\begin{table}[ht]
\caption{Resultados de la ejecución de la metaheurística ACO, utilizando instancias de Dethloff con la configuración -n 2.0 -alpha 1.0 -beta 3.0 -q 6.8 -ro 0.015}
\centering
\small
\begin{tabular}{c c c c c c c}
\hline\hline
Instancia & Costo mínimo & Tiempo(seg.) & Costo promedio & Tiempo promedio(seg.) & Costo ACO & \%Gap \\ [0.5ex]
\hline
SCA3-0 & 640.55 & 1.35 & 
640.55 & 1.35 & \bf{636.10} & 
0.70\\SCA3-1 & \bf{\underline{697.84}} & 1.56 & 
697.84 & 1.53 & 700.10 & 
-0.32\\SCA3-2 & 659.34 & 1.35 & 
663.10 & 1.31 & \bf{659.30} & 
0.01\\SCA3-3 & 680.60 & 1.38 & 
681.13 & 1.40 & \bf{680.00} & 
0.09\\SCA3-4 & \bf{690.50} & 1.33 & 
690.50 & 1.41 & 690.50 & 0.00\\
SCA3-5 & \bf{\underline{665.04}} & 1.34 & 
668.60 & 1.41 & 671.10 & 
-0.90\\SCA3-6 & 655.19 & 1.27 & 
655.19 & 1.30 & \bf{651.10} & 
0.63\\SCA3-7 & 666.15 & 1.06 & 
666.15 & 1.00 & \bf{666.10} & 
0.01\\SCA3-8 & 721.45 & 1.18 & 
723.34 & 1.19 & \bf{719.50} & 
0.27\\SCA3-9 & \bf{681.00} & 0.96 & 
681.00 & 0.95 & 681.00 & 0.00\\
SCA8-0 & 991.07 & 1.58 & 
992.23 & 1.50 & \bf{961.60} & 
3.06\\SCA8-1 & 1074.65 & 1.19 & 
1074.65 & 1.20 & \bf{1063.00} & 
1.10\\SCA8-2 & 1056.87 & 1.00 & 
1056.87 & 0.99 & \bf{1040.60} & 
1.56\\SCA8-3 & 1031.08 & 1.54 & 
1031.08 & 1.51 & \bf{985.90} & 
4.58\\SCA8-4 & 1099.06 & 1.51 & 
1099.06 & 1.53 & \bf{1071.00} & 
2.62\\SCA8-5 & 1055.35 & 1.62 & 
1055.35 & 1.66 & \bf{1054.30} & 
0.10\\SCA8-6 & \bf{\underline{972.48}} & 1.66 & 
979.23 & 1.64 & 972.50 & 
-0.00\\SCA8-7 & 1092.57 & 1.58 & 
1092.57 & 1.61 & \bf{1059.70} & 
3.10\\SCA8-8 & 1091.49 & 1.37 & 
1091.76 & 1.43 & \bf{1082.70} & 
0.81\\SCA8-9 & \bf{\underline{1067.42}} & 1.20 & 
1067.42 & 1.17 & 1081.40 & 
-1.29\\CON3-0 & 624.96 & 1.62 & 
624.96 & 1.68 & \bf{616.50} & 
1.37\\CON3-1 & 557.38 & 1.57 & 
558.22 & 1.59 & \bf{555.60} & 
0.32\\CON3-2 & 524.07 & 1.08 & 
524.86 & 1.10 & \bf{521.40} & 
0.51\\CON3-3 & \bf{591.20} & 1.55 & 
593.38 & 1.53 & 591.20 & 0.00\\
CON3-4 & \bf{\underline{588.79}} & 1.28 & 
589.19 & 1.30 & 589.30 & 
-0.09\\CON3-5 & 576.43 & 1.41 & 
576.43 & 1.46 & \bf{563.70} & 
2.26\\CON3-6 & 504.15 & 1.74 & 
506.24 & 1.80 & \bf{499.20} & 
0.99\\CON3-7 & 578.41 & 1.28 & 
578.41 & 1.26 & \bf{577.50} & 
0.16\\CON3-8 & 524.59 & 1.28 & 
525.40 & 1.23 & \bf{523.10} & 
0.28\\CON3-9 & 588.48 & 1.32 & 
588.48 & 1.30 & \bf{578.20} & 
1.78\\CON8-0 & 879.00 & 1.44 & 
879.00 & 1.45 & \bf{858.90} & 
2.34\\CON8-1 & 758.26 & 1.38 & 
758.26 & 1.33 & \bf{740.90} & 
2.34\\CON8-2 & 716.53 & 2.10 & 
716.54 & 1.93 & \bf{714.30} & 
0.31\\CON8-3 & 817.57 & 1.44 & 
817.57 & 1.42 & \bf{812.30} & 
0.65\\CON8-4 & 781.64 & 1.46 & 
788.67 & 1.52 & \bf{770.10} & 
1.50\\CON8-5 & \bf{\underline{764.36}} & 1.38 & 
764.36 & 1.37 & 766.60 & 
-0.29\\CON8-6 & 705.61 & 1.70 & 
706.96 & 1.71 & \bf{697.20} & 
1.21\\CON8-7 & 822.42 & 1.20 & 
823.00 & 1.20 & \bf{814.80} & 
0.94\\CON8-8 & 799.16 & 1.58 & 
799.29 & 1.52 & \bf{771.30} & 
3.61\\CON8-9 & 816.12 & 1.58 & 
816.12 & 1.56 & \bf{815.10} & 
0.13\\[1ex]\hline
\end{tabular}
\label{table:nonlin}
\end{table} \clearpage
\begin{table}[ht]
\caption{Resultados de la ejecución de la metaheurística ACO, utilizando instancias de Dethloff con la configuración -n 2.0 -alpha 1.0 -beta 3.0 -q 6.9 -ro 0.015}
\centering
\small
\begin{tabular}{c c c c c c c}
\hline\hline
Instancia & Costo mínimo & Tiempo(seg.) & Costo promedio & Tiempo promedio(seg.) & Costo ACO & \%Gap \\ [0.5ex]
\hline
SCA3-0 & 640.55 & 1.37 & 
640.55 & 1.41 & \bf{636.10} & 
0.70\\SCA3-1 & \bf{\underline{697.84}} & 1.58 & 
697.84 & 1.48 & 700.10 & 
-0.32\\SCA3-2 & 659.34 & 1.34 & 
662.97 & 1.33 & \bf{659.30} & 
0.01\\SCA3-3 & 680.60 & 1.46 & 
681.13 & 1.50 & \bf{680.00} & 
0.09\\SCA3-4 & \bf{690.50} & 1.39 & 
690.50 & 1.36 & 690.50 & 0.00\\
SCA3-5 & \bf{\underline{665.04}} & 1.57 & 
665.04 & 1.50 & 671.10 & 
-0.90\\SCA3-6 & 655.19 & 1.24 & 
655.19 & 1.30 & \bf{651.10} & 
0.63\\SCA3-7 & 666.15 & 0.98 & 
666.15 & 0.97 & \bf{666.10} & 
0.01\\SCA3-8 & 721.45 & 1.13 & 
725.84 & 1.13 & \bf{719.50} & 
0.27\\SCA3-9 & \bf{681.00} & 0.93 & 
681.00 & 0.98 & 681.00 & 0.00\\
SCA8-0 & 991.07 & 1.52 & 
992.23 & 1.56 & \bf{961.60} & 
3.06\\SCA8-1 & 1074.65 & 1.21 & 
1074.65 & 1.24 & \bf{1063.00} & 
1.10\\SCA8-2 & 1056.87 & 1.02 & 
1056.87 & 1.03 & \bf{1040.60} & 
1.56\\SCA8-3 & 1031.08 & 1.42 & 
1031.08 & 1.46 & \bf{985.90} & 
4.58\\SCA8-4 & 1099.06 & 1.42 & 
1099.06 & 1.43 & \bf{1071.00} & 
2.62\\SCA8-5 & 1055.35 & 1.65 & 
1055.35 & 1.65 & \bf{1054.30} & 
0.10\\SCA8-6 & \bf{\underline{972.48}} & 1.81 & 
976.70 & 1.75 & 972.50 & 
-0.00\\SCA8-7 & 1092.57 & 1.64 & 
1092.57 & 1.64 & \bf{1059.70} & 
3.10\\SCA8-8 & 1091.49 & 1.48 & 
1091.76 & 1.44 & \bf{1082.70} & 
0.81\\SCA8-9 & \bf{\underline{1067.42}} & 1.16 & 
1067.42 & 1.21 & 1081.40 & 
-1.29\\CON3-0 & 624.96 & 1.69 & 
624.96 & 1.65 & \bf{616.50} & 
1.37\\CON3-1 & 557.38 & 1.56 & 
559.07 & 1.47 & \bf{555.60} & 
0.32\\CON3-2 & 524.07 & 1.24 & 
524.79 & 1.17 & \bf{521.40} & 
0.51\\CON3-3 & 594.11 & 1.56 & 
594.11 & 1.51 & \bf{591.20} & 
0.49\\CON3-4 & \bf{\underline{588.79}} & 1.38 & 
589.19 & 1.38 & 589.30 & 
-0.09\\CON3-5 & 574.57 & 1.40 & 
575.97 & 1.43 & \bf{563.70} & 
1.93\\CON3-6 & 505.26 & 1.89 & 
506.43 & 1.83 & \bf{499.20} & 
1.21\\CON3-7 & 578.41 & 1.21 & 
578.41 & 1.25 & \bf{577.50} & 
0.16\\CON3-8 & 524.59 & 1.28 & 
524.59 & 1.28 & \bf{523.10} & 
0.28\\CON3-9 & 588.48 & 1.23 & 
588.48 & 1.29 & \bf{578.20} & 
1.78\\CON8-0 & 879.00 & 1.40 & 
879.00 & 1.42 & \bf{858.90} & 
2.34\\CON8-1 & 758.26 & 1.37 & 
758.26 & 1.33 & \bf{740.90} & 
2.34\\CON8-2 & 716.53 & 1.90 & 
716.54 & 1.96 & \bf{714.30} & 
0.31\\CON8-3 & 817.57 & 1.34 & 
817.57 & 1.40 & \bf{812.30} & 
0.65\\CON8-4 & 789.98 & 1.50 & 
790.76 & 1.58 & \bf{770.10} & 
2.58\\CON8-5 & \bf{\underline{764.36}} & 1.42 & 
764.36 & 1.40 & 766.60 & 
-0.29\\CON8-6 & 705.61 & 1.66 & 
706.51 & 1.71 & \bf{697.20} & 
1.21\\CON8-7 & 822.42 & 1.23 & 
823.03 & 1.19 & \bf{814.80} & 
0.94\\CON8-8 & 799.16 & 1.46 & 
799.42 & 1.53 & \bf{771.30} & 
3.61\\CON8-9 & 816.12 & 1.61 & 
816.12 & 1.55 & \bf{815.10} & 
0.13\\[1ex]\hline
\end{tabular}
\label{table:nonlin}
\end{table} \clearpage
\begin{table}[ht]
\caption{Resultados de la ejecución de la metaheurística ACO, utilizando instancias de Dethloff con la configuración -n 2.0 -alpha 1.0 -beta 3.0 -q 7.0 -ro 0.015}
\centering
\small
\begin{tabular}{c c c c c c c}
\hline\hline
Instancia & Costo mínimo & Tiempo(seg.) & Costo promedio & Tiempo promedio(seg.) & Costo ACO & \%Gap \\ [0.5ex]
\hline
SCA3-0 & 640.55 & 1.36 & 
640.55 & 1.35 & \bf{636.10} & 
0.70\\SCA3-1 & \bf{\underline{697.84}} & 1.61 & 
697.84 & 1.74 & 700.10 & 
-0.32\\SCA3-2 & 664.18 & 1.39 & 
664.18 & 1.36 & \bf{659.30} & 
0.74\\SCA3-3 & 680.60 & 1.46 & 
680.78 & 1.48 & \bf{680.00} & 
0.09\\SCA3-4 & \bf{690.50} & 1.42 & 
690.50 & 1.42 & 690.50 & 0.00\\
SCA3-5 & \bf{\underline{665.04}} & 1.43 & 
665.19 & 1.47 & 671.10 & 
-0.90\\SCA3-6 & 655.19 & 1.34 & 
655.19 & 1.38 & \bf{651.10} & 
0.63\\SCA3-7 & 666.15 & 1.01 & 
666.15 & 0.99 & \bf{666.10} & 
0.01\\SCA3-8 & 721.45 & 1.11 & 
721.45 & 1.15 & \bf{719.50} & 
0.27\\SCA3-9 & \bf{681.00} & 1.03 & 
681.00 & 0.98 & 681.00 & 0.00\\
SCA8-0 & 993.38 & 1.66 & 
993.38 & 1.58 & \bf{961.60} & 
3.30\\SCA8-1 & 1069.40 & 1.23 & 
1073.34 & 1.19 & \bf{1063.00} & 
0.60\\SCA8-2 & 1056.87 & 1.01 & 
1056.87 & 1.04 & \bf{1040.60} & 
1.56\\SCA8-3 & 1031.08 & 1.49 & 
1031.08 & 1.42 & \bf{985.90} & 
4.58\\SCA8-4 & 1099.06 & 1.57 & 
1099.06 & 1.51 & \bf{1071.00} & 
2.62\\SCA8-5 & 1055.35 & 1.66 & 
1055.35 & 1.67 & \bf{1054.30} & 
0.10\\SCA8-6 & \bf{\underline{972.48}} & 1.71 & 
972.48 & 1.69 & 972.50 & 
-0.00\\SCA8-7 & 1092.57 & 1.65 & 
1092.57 & 1.68 & \bf{1059.70} & 
3.10\\SCA8-8 & 1091.49 & 1.45 & 
1091.89 & 1.45 & \bf{1082.70} & 
0.81\\SCA8-9 & \bf{\underline{1067.42}} & 1.19 & 
1067.42 & 1.15 & 1081.40 & 
-1.29\\CON3-0 & 624.96 & 1.59 & 
624.96 & 1.60 & \bf{616.50} & 
1.37\\CON3-1 & 557.38 & 1.46 & 
558.30 & 1.43 & \bf{555.60} & 
0.32\\CON3-2 & 524.07 & 1.04 & 
524.86 & 1.09 & \bf{521.40} & 
0.51\\CON3-3 & 592.95 & 1.48 & 
593.82 & 1.51 & \bf{591.20} & 
0.30\\CON3-4 & 589.32 & 1.38 & 
589.32 & 1.35 & \bf{589.30} & 
0.00\\CON3-5 & 576.43 & 1.44 & 
576.43 & 1.44 & \bf{563.70} & 
2.26\\CON3-6 & 504.15 & 1.86 & 
504.70 & 1.85 & \bf{499.20} & 
0.99\\CON3-7 & 578.41 & 1.18 & 
579.84 & 1.21 & \bf{577.50} & 
0.16\\CON3-8 & 524.59 & 1.20 & 
524.59 & 1.18 & \bf{523.10} & 
0.28\\CON3-9 & 588.48 & 1.28 & 
588.48 & 1.26 & \bf{578.20} & 
1.78\\CON8-0 & 879.00 & 1.41 & 
879.00 & 1.42 & \bf{858.90} & 
2.34\\CON8-1 & 758.26 & 1.28 & 
758.26 & 1.35 & \bf{740.90} & 
2.34\\CON8-2 & 716.53 & 1.88 & 
716.54 & 2.00 & \bf{714.30} & 
0.31\\CON8-3 & 817.57 & 1.46 & 
817.57 & 1.46 & \bf{812.30} & 
0.65\\CON8-4 & 781.64 & 1.49 & 
783.73 & 1.54 & \bf{770.10} & 
1.50\\CON8-5 & \bf{\underline{764.36}} & 1.36 & 
764.36 & 1.33 & 766.60 & 
-0.29\\CON8-6 & 707.41 & 1.72 & 
707.76 & 1.69 & \bf{697.20} & 
1.46\\CON8-7 & 822.42 & 1.18 & 
822.92 & 1.18 & \bf{814.80} & 
0.94\\CON8-8 & 799.32 & 1.52 & 
799.46 & 1.52 & \bf{771.30} & 
3.63\\CON8-9 & 816.12 & 1.50 & 
816.12 & 1.57 & \bf{815.10} & 
0.13\\[1ex]\hline
\end{tabular}
\label{table:nonlin}
\end{table} \clearpage
\begin{table}[ht]
\caption{Resultados de la ejecución de la metaheurística ACO, utilizando instancias de Dethloff con la configuración -n 2.0 -alpha 1.0 -beta 3.0 -q 7.1 -ro 0.015}
\centering
\small
\begin{tabular}{c c c c c c c}
\hline\hline
Instancia & Costo mínimo & Tiempo(seg.) & Costo promedio & Tiempo promedio(seg.) & Costo ACO & \%Gap \\ [0.5ex]
\hline
SCA3-0 & 640.55 & 1.42 & 
640.55 & 1.39 & \bf{636.10} & 
0.70\\SCA3-1 & \bf{\underline{697.84}} & 1.40 & 
697.84 & 1.43 & 700.10 & 
-0.32\\SCA3-2 & 664.18 & 1.30 & 
664.18 & 1.37 & \bf{659.30} & 
0.74\\SCA3-3 & 680.60 & 1.43 & 
680.78 & 1.44 & \bf{680.00} & 
0.09\\SCA3-4 & \bf{690.50} & 1.29 & 
690.50 & 1.37 & 690.50 & 0.00\\
SCA3-5 & \bf{\underline{665.04}} & 1.42 & 
668.75 & 1.39 & 671.10 & 
-0.90\\SCA3-6 & 655.19 & 1.50 & 
655.30 & 1.41 & \bf{651.10} & 
0.63\\SCA3-7 & 666.15 & 1.00 & 
666.15 & 1.01 & \bf{666.10} & 
0.01\\SCA3-8 & 721.45 & 1.18 & 
722.70 & 1.15 & \bf{719.50} & 
0.27\\SCA3-9 & \bf{681.00} & 0.96 & 
681.00 & 0.94 & 681.00 & 0.00\\
SCA8-0 & 991.07 & 1.56 & 
992.23 & 1.52 & \bf{961.60} & 
3.06\\SCA8-1 & 1074.65 & 1.19 & 
1074.68 & 1.15 & \bf{1063.00} & 
1.10\\SCA8-2 & 1056.87 & 1.04 & 
1056.87 & 1.02 & \bf{1040.60} & 
1.56\\SCA8-3 & 1031.08 & 1.39 & 
1031.08 & 1.43 & \bf{985.90} & 
4.58\\SCA8-4 & 1098.34 & 1.52 & 
1098.88 & 1.76 & \bf{1071.00} & 
2.55\\SCA8-5 & 1055.35 & 1.63 & 
1055.35 & 1.66 & \bf{1054.30} & 
0.10\\SCA8-6 & \bf{\underline{972.48}} & 1.70 & 
972.48 & 1.72 & 972.50 & 
-0.00\\SCA8-7 & 1092.57 & 1.64 & 
1092.57 & 1.63 & \bf{1059.70} & 
3.10\\SCA8-8 & 1091.49 & 1.44 & 
1091.76 & 1.44 & \bf{1082.70} & 
0.81\\SCA8-9 & \bf{\underline{1067.42}} & 1.16 & 
1067.42 & 1.15 & 1081.40 & 
-1.29\\CON3-0 & 624.96 & 1.60 & 
624.96 & 1.69 & \bf{616.50} & 
1.37\\CON3-1 & 557.38 & 1.49 & 
558.01 & 1.59 & \bf{555.60} & 
0.32\\CON3-2 & 524.07 & 1.07 & 
524.51 & 1.10 & \bf{521.40} & 
0.51\\CON3-3 & 594.11 & 1.48 & 
594.11 & 1.52 & \bf{591.20} & 
0.49\\CON3-4 & 589.32 & 1.33 & 
589.32 & 1.32 & \bf{589.30} & 
0.00\\CON3-5 & 576.43 & 1.39 & 
576.43 & 1.38 & \bf{563.70} & 
2.26\\CON3-6 & 504.15 & 1.82 & 
506.15 & 1.84 & \bf{499.20} & 
0.99\\CON3-7 & 578.41 & 1.15 & 
579.12 & 1.19 & \bf{577.50} & 
0.16\\CON3-8 & 524.30 & 1.09 & 
524.52 & 1.19 & \bf{523.10} & 
0.23\\CON3-9 & 578.25 & 1.24 & 
585.92 & 1.25 & \bf{578.20} & 
0.01\\CON8-0 & 879.00 & 1.59 & 
879.00 & 1.47 & \bf{858.90} & 
2.34\\CON8-1 & 758.26 & 1.37 & 
758.26 & 1.32 & \bf{740.90} & 
2.34\\CON8-2 & 716.56 & 1.98 & 
717.21 & 1.99 & \bf{714.30} & 
0.32\\CON8-3 & 817.57 & 1.40 & 
817.57 & 1.41 & \bf{812.30} & 
0.65\\CON8-4 & 777.81 & 1.61 & 
786.94 & 1.58 & \bf{770.10} & 
1.00\\CON8-5 & \bf{\underline{764.36}} & 1.38 & 
764.36 & 1.36 & 766.60 & 
-0.29\\CON8-6 & \bf{\underline{693.83}} & 1.61 & 
703.12 & 1.66 & 697.20 & 
-0.48\\CON8-7 & 822.42 & 1.17 & 
823.18 & 1.13 & \bf{814.80} & 
0.94\\CON8-8 & 799.32 & 1.62 & 
799.46 & 1.54 & \bf{771.30} & 
3.63\\CON8-9 & 816.12 & 1.56 & 
816.12 & 1.59 & \bf{815.10} & 
0.13\\[1ex]\hline
\end{tabular}
\label{table:nonlin}
\end{table} \clearpage
\begin{table}[ht]
\caption{Resultados de la ejecución de la metaheurística ACO, utilizando instancias de Dethloff con la configuración -n 2.0 -alpha 1.0 -beta 3.0 -q 7.2 -ro 0.015}
\centering
\small
\begin{tabular}{c c c c c c c}
\hline\hline
Instancia & Costo mínimo & Tiempo(seg.) & Costo promedio & Tiempo promedio(seg.) & Costo ACO & \%Gap \\ [0.5ex]
\hline
SCA3-0 & 640.55 & 1.33 & 
640.55 & 1.50 & \bf{636.10} & 
0.70\\SCA3-1 & \bf{\underline{697.84}} & 1.50 & 
697.84 & 1.47 & 700.10 & 
-0.32\\SCA3-2 & 664.18 & 1.39 & 
664.18 & 1.34 & \bf{659.30} & 
0.74\\SCA3-3 & 680.60 & 1.54 & 
680.96 & 1.48 & \bf{680.00} & 
0.09\\SCA3-4 & \bf{690.50} & 1.41 & 
690.50 & 1.40 & 690.50 & 0.00\\
SCA3-5 & \bf{\underline{665.04}} & 1.39 & 
668.60 & 1.43 & 671.10 & 
-0.90\\SCA3-6 & 655.19 & 1.30 & 
655.30 & 1.30 & \bf{651.10} & 
0.63\\SCA3-7 & 666.15 & 1.05 & 
666.15 & 1.03 & \bf{666.10} & 
0.01\\SCA3-8 & 721.45 & 1.17 & 
726.70 & 1.15 & \bf{719.50} & 
0.27\\SCA3-9 & \bf{681.00} & 0.96 & 
681.00 & 0.94 & 681.00 & 0.00\\
SCA8-0 & 991.07 & 1.49 & 
991.65 & 1.49 & \bf{961.60} & 
3.06\\SCA8-1 & 1069.40 & 1.18 & 
1073.37 & 1.19 & \bf{1063.00} & 
0.60\\SCA8-2 & 1056.87 & 1.07 & 
1056.87 & 1.02 & \bf{1040.60} & 
1.56\\SCA8-3 & 1031.08 & 1.45 & 
1031.08 & 1.44 & \bf{985.90} & 
4.58\\SCA8-4 & 1099.06 & 1.40 & 
1099.06 & 1.46 & \bf{1071.00} & 
2.62\\SCA8-5 & 1055.35 & 1.56 & 
1055.35 & 1.61 & \bf{1054.30} & 
0.10\\SCA8-6 & \bf{\underline{972.48}} & 1.67 & 
972.48 & 1.64 & 972.50 & 
-0.00\\SCA8-7 & 1092.57 & 1.62 & 
1092.57 & 1.60 & \bf{1059.70} & 
3.10\\SCA8-8 & 1091.49 & 1.40 & 
1091.89 & 1.45 & \bf{1082.70} & 
0.81\\SCA8-9 & \bf{\underline{1067.42}} & 1.13 & 
1067.42 & 1.14 & 1081.40 & 
-1.29\\CON3-0 & 624.96 & 1.66 & 
624.96 & 1.59 & \bf{616.50} & 
1.37\\CON3-1 & 557.38 & 1.52 & 
557.38 & 1.49 & \bf{555.60} & 
0.32\\CON3-2 & 524.07 & 1.14 & 
524.79 & 1.09 & \bf{521.40} & 
0.51\\CON3-3 & 594.11 & 1.57 & 
594.11 & 1.55 & \bf{591.20} & 
0.49\\CON3-4 & 589.32 & 1.59 & 
589.32 & 1.44 & \bf{589.30} & 
0.00\\CON3-5 & 570.70 & 1.44 & 
575.00 & 1.44 & \bf{563.70} & 
1.24\\CON3-6 & 505.26 & 1.75 & 
506.43 & 1.79 & \bf{499.20} & 
1.21\\CON3-7 & 578.41 & 1.19 & 
579.84 & 1.17 & \bf{577.50} & 
0.16\\CON3-8 & 524.59 & 1.14 & 
524.59 & 1.20 & \bf{523.10} & 
0.28\\CON3-9 & 588.48 & 1.26 & 
588.48 & 1.31 & \bf{578.20} & 
1.78\\CON8-0 & 879.00 & 1.39 & 
879.00 & 1.41 & \bf{858.90} & 
2.34\\CON8-1 & 758.26 & 1.26 & 
758.26 & 1.35 & \bf{740.90} & 
2.34\\CON8-2 & 716.53 & 2.00 & 
716.54 & 1.99 & \bf{714.30} & 
0.31\\CON8-3 & 817.57 & 1.46 & 
817.57 & 1.48 & \bf{812.30} & 
0.65\\CON8-4 & 781.64 & 1.46 & 
786.59 & 1.52 & \bf{770.10} & 
1.50\\CON8-5 & \bf{\underline{764.36}} & 1.37 & 
764.36 & 1.37 & 766.60 & 
-0.29\\CON8-6 & \bf{\underline{693.83}} & 1.68 & 
703.12 & 1.72 & 697.20 & 
-0.48\\CON8-7 & 822.42 & 1.22 & 
822.92 & 1.19 & \bf{814.80} & 
0.94\\CON8-8 & 799.32 & 1.59 & 
799.46 & 1.55 & \bf{771.30} & 
3.63\\CON8-9 & 816.12 & 1.59 & 
816.12 & 1.58 & \bf{815.10} & 
0.13\\[1ex]\hline
\end{tabular}
\label{table:nonlin}
\end{table} \clearpage
\begin{table}[ht]
\caption{Resultados de la ejecución de la metaheurística ACO, utilizando instancias de Dethloff con la configuración -n 2.0 -alpha 1.0 -beta 3.0 -q 7.3 -ro 0.015}
\centering
\small
\begin{tabular}{c c c c c c c}
\hline\hline
Instancia & Costo mínimo & Tiempo(seg.) & Costo promedio & Tiempo promedio(seg.) & Costo ACO & \%Gap \\ [0.5ex]
\hline
SCA3-0 & 640.55 & 1.31 & 
640.55 & 1.34 & \bf{636.10} & 
0.70\\SCA3-1 & \bf{\underline{697.84}} & 1.41 & 
697.84 & 1.51 & 700.10 & 
-0.32\\SCA3-2 & 664.18 & 1.23 & 
664.18 & 1.27 & \bf{659.30} & 
0.74\\SCA3-3 & 680.60 & 1.50 & 
680.96 & 1.55 & \bf{680.00} & 
0.09\\SCA3-4 & \bf{690.50} & 1.48 & 
690.50 & 1.43 & 690.50 & 0.00\\
SCA3-5 & \bf{\underline{665.04}} & 1.47 & 
665.04 & 1.46 & 671.10 & 
-0.90\\SCA3-6 & 655.19 & 1.38 & 
655.19 & 1.35 & \bf{651.10} & 
0.63\\SCA3-7 & 666.15 & 0.96 & 
666.15 & 0.99 & \bf{666.10} & 
0.01\\SCA3-8 & 721.45 & 1.07 & 
724.59 & 1.10 & \bf{719.50} & 
0.27\\SCA3-9 & \bf{681.00} & 0.88 & 
681.00 & 0.94 & 681.00 & 0.00\\
SCA8-0 & 991.07 & 1.48 & 
991.07 & 1.51 & \bf{961.60} & 
3.06\\SCA8-1 & 1074.65 & 1.22 & 
1074.68 & 1.19 & \bf{1063.00} & 
1.10\\SCA8-2 & 1056.87 & 1.07 & 
1056.87 & 1.04 & \bf{1040.60} & 
1.56\\SCA8-3 & 1031.08 & 1.36 & 
1031.08 & 1.44 & \bf{985.90} & 
4.58\\SCA8-4 & 1099.06 & 1.52 & 
1099.06 & 1.52 & \bf{1071.00} & 
2.62\\SCA8-5 & 1055.35 & 1.53 & 
1055.35 & 1.59 & \bf{1054.30} & 
0.10\\SCA8-6 & \bf{\underline{972.48}} & 1.59 & 
972.48 & 1.65 & 972.50 & 
-0.00\\SCA8-7 & 1092.57 & 1.56 & 
1092.57 & 1.62 & \bf{1059.70} & 
3.10\\SCA8-8 & 1092.02 & 1.46 & 
1092.02 & 1.45 & \bf{1082.70} & 
0.86\\SCA8-9 & \bf{\underline{1067.42}} & 1.10 & 
1067.42 & 1.14 & 1081.40 & 
-1.29\\CON3-0 & 624.96 & 1.60 & 
624.96 & 1.61 & \bf{616.50} & 
1.37\\CON3-1 & 557.38 & 1.53 & 
557.58 & 1.49 & \bf{555.60} & 
0.32\\CON3-2 & 524.07 & 1.06 & 
525.20 & 1.15 & \bf{521.40} & 
0.51\\CON3-3 & 594.11 & 1.47 & 
594.11 & 1.49 & \bf{591.20} & 
0.49\\CON3-4 & \bf{\underline{588.79}} & 1.67 & 
589.19 & 1.40 & 589.30 & 
-0.09\\CON3-5 & 576.43 & 1.41 & 
576.43 & 1.42 & \bf{563.70} & 
2.26\\CON3-6 & 505.26 & 1.78 & 
507.61 & 1.82 & \bf{499.20} & 
1.21\\CON3-7 & 578.41 & 1.21 & 
579.12 & 1.24 & \bf{577.50} & 
0.16\\CON3-8 & 524.59 & 1.12 & 
524.59 & 1.15 & \bf{523.10} & 
0.28\\CON3-9 & 588.48 & 1.22 & 
588.48 & 1.24 & \bf{578.20} & 
1.78\\CON8-0 & 879.00 & 1.37 & 
879.00 & 1.43 & \bf{858.90} & 
2.34\\CON8-1 & 758.26 & 1.34 & 
758.26 & 1.33 & \bf{740.90} & 
2.34\\CON8-2 & 716.53 & 2.10 & 
716.54 & 2.01 & \bf{714.30} & 
0.31\\CON8-3 & 817.57 & 1.46 & 
817.57 & 1.47 & \bf{812.30} & 
0.65\\CON8-4 & 781.64 & 1.53 & 
789.45 & 1.54 & \bf{770.10} & 
1.50\\CON8-5 & \bf{\underline{764.36}} & 1.30 & 
764.36 & 1.37 & 766.60 & 
-0.29\\CON8-6 & 701.31 & 1.74 & 
705.10 & 1.69 & \bf{697.20} & 
0.59\\CON8-7 & 822.42 & 1.14 & 
822.92 & 1.17 & \bf{814.80} & 
0.94\\CON8-8 & 799.32 & 1.56 & 
799.37 & 1.54 & \bf{771.30} & 
3.63\\CON8-9 & 816.12 & 1.52 & 
817.40 & 1.53 & \bf{815.10} & 
0.13\\[1ex]\hline
\end{tabular}
\label{table:nonlin}
\end{table} \clearpage
\begin{table}[ht]
\caption{Resultados de la ejecución de la metaheurística ACO, utilizando instancias de Dethloff con la configuración -n 2.0 -alpha 1.0 -beta 3.0 -q 7.4 -ro 0.015}
\centering
\small
\begin{tabular}{c c c c c c c}
\hline\hline
Instancia & Costo mínimo & Tiempo(seg.) & Costo promedio & Tiempo promedio(seg.) & Costo ACO & \%Gap \\ [0.5ex]
\hline
SCA3-0 & 640.55 & 1.48 & 
640.55 & 1.38 & \bf{636.10} & 
0.70\\SCA3-1 & \bf{\underline{697.84}} & 1.49 & 
698.76 & 1.64 & 700.10 & 
-0.32\\SCA3-2 & 664.18 & 1.42 & 
664.18 & 1.38 & \bf{659.30} & 
0.74\\SCA3-3 & 680.60 & 1.52 & 
680.60 & 1.70 & \bf{680.00} & 
0.09\\SCA3-4 & \bf{690.50} & 1.42 & 
690.50 & 1.44 & 690.50 & 0.00\\
SCA3-5 & \bf{\underline{665.04}} & 1.39 & 
665.04 & 1.42 & 671.10 & 
-0.90\\SCA3-6 & 653.69 & 1.40 & 
654.82 & 1.43 & \bf{651.10} & 
0.40\\SCA3-7 & 666.15 & 1.05 & 
666.15 & 1.01 & \bf{666.10} & 
0.01\\SCA3-8 & 721.45 & 1.14 & 
726.48 & 1.15 & \bf{719.50} & 
0.27\\SCA3-9 & \bf{681.00} & 1.01 & 
681.00 & 0.99 & 681.00 & 0.00\\
SCA8-0 & 991.07 & 1.45 & 
991.65 & 1.49 & \bf{961.60} & 
3.06\\SCA8-1 & 1074.65 & 1.12 & 
1074.71 & 1.17 & \bf{1063.00} & 
1.10\\SCA8-2 & 1056.87 & 1.08 & 
1056.87 & 1.01 & \bf{1040.60} & 
1.56\\SCA8-3 & 1031.08 & 1.39 & 
1031.08 & 1.45 & \bf{985.90} & 
4.58\\SCA8-4 & 1098.34 & 1.53 & 
1098.88 & 1.49 & \bf{1071.00} & 
2.55\\SCA8-5 & 1055.35 & 1.69 & 
1055.35 & 1.67 & \bf{1054.30} & 
0.10\\SCA8-6 & \bf{\underline{972.48}} & 1.65 & 
972.48 & 1.70 & 972.50 & 
-0.00\\SCA8-7 & 1092.57 & 1.64 & 
1092.57 & 1.62 & \bf{1059.70} & 
3.10\\SCA8-8 & 1092.02 & 1.41 & 
1092.02 & 1.43 & \bf{1082.70} & 
0.86\\SCA8-9 & \bf{\underline{1067.42}} & 1.16 & 
1067.42 & 1.16 & 1081.40 & 
-1.29\\CON3-0 & 624.96 & 1.52 & 
624.96 & 1.58 & \bf{616.50} & 
1.37\\CON3-1 & 557.38 & 1.46 & 
559.99 & 1.64 & \bf{555.60} & 
0.32\\CON3-2 & 524.07 & 1.05 & 
524.55 & 1.07 & \bf{521.40} & 
0.51\\CON3-3 & 594.11 & 1.50 & 
594.11 & 1.49 & \bf{591.20} & 
0.49\\CON3-4 & 589.32 & 1.29 & 
589.32 & 1.34 & \bf{589.30} & 
0.00\\CON3-5 & 576.43 & 1.46 & 
576.43 & 1.44 & \bf{563.70} & 
2.26\\CON3-6 & 504.15 & 1.91 & 
506.15 & 1.83 & \bf{499.20} & 
0.99\\CON3-7 & 578.41 & 1.26 & 
579.12 & 1.25 & \bf{577.50} & 
0.16\\CON3-8 & 524.30 & 1.16 & 
524.37 & 1.18 & \bf{523.10} & 
0.23\\CON3-9 & 588.48 & 1.27 & 
588.48 & 1.29 & \bf{578.20} & 
1.78\\CON8-0 & 879.00 & 1.38 & 
879.00 & 1.43 & \bf{858.90} & 
2.34\\CON8-1 & 758.26 & 1.31 & 
758.26 & 1.32 & \bf{740.90} & 
2.34\\CON8-2 & 716.53 & 1.97 & 
716.55 & 1.97 & \bf{714.30} & 
0.31\\CON8-3 & 817.57 & 1.47 & 
817.57 & 1.44 & \bf{812.30} & 
0.65\\CON8-4 & 789.98 & 1.52 & 
790.76 & 1.49 & \bf{770.10} & 
2.58\\CON8-5 & \bf{\underline{764.36}} & 1.40 & 
764.36 & 1.36 & 766.60 & 
-0.29\\CON8-6 & \bf{\underline{693.83}} & 1.72 & 
703.12 & 1.71 & 697.20 & 
-0.48\\CON8-7 & 822.42 & 1.23 & 
822.92 & 1.21 & \bf{814.80} & 
0.94\\CON8-8 & 799.32 & 1.60 & 
799.46 & 1.55 & \bf{771.30} & 
3.63\\CON8-9 & 816.12 & 1.53 & 
817.40 & 1.61 & \bf{815.10} & 
0.13\\[1ex]\hline
\end{tabular}
\label{table:nonlin}
\end{table} \clearpage
\begin{table}[ht]
\caption{Resultados de la ejecución de la metaheurística ACO, utilizando instancias de Dethloff con la configuración -n 2.0 -alpha 1.0 -beta 3.0 -q 7.5 -ro 0.015}
\centering
\small
\begin{tabular}{c c c c c c c}
\hline\hline
Instancia & Costo mínimo & Tiempo(seg.) & Costo promedio & Tiempo promedio(seg.) & Costo ACO & \%Gap \\ [0.5ex]
\hline
SCA3-0 & 640.55 & 1.40 & 
640.55 & 1.36 & \bf{636.10} & 
0.70\\SCA3-1 & \bf{\underline{697.84}} & 1.41 & 
697.84 & 1.44 & 700.10 & 
-0.32\\SCA3-2 & 659.34 & 1.37 & 
662.97 & 1.34 & \bf{659.30} & 
0.01\\SCA3-3 & 680.60 & 1.47 & 
680.96 & 1.51 & \bf{680.00} & 
0.09\\SCA3-4 & \bf{690.50} & 1.38 & 
690.50 & 1.40 & 690.50 & 0.00\\
SCA3-5 & \bf{\underline{665.04}} & 1.38 & 
665.04 & 1.45 & 671.10 & 
-0.90\\SCA3-6 & 653.69 & 1.26 & 
654.82 & 1.44 & \bf{651.10} & 
0.40\\SCA3-7 & 666.15 & 1.07 & 
666.15 & 1.03 & \bf{666.10} & 
0.01\\SCA3-8 & 721.45 & 1.14 & 
723.34 & 1.14 & \bf{719.50} & 
0.27\\SCA3-9 & \bf{681.00} & 0.96 & 
681.00 & 0.97 & 681.00 & 0.00\\
SCA8-0 & 991.07 & 1.47 & 
991.65 & 1.50 & \bf{961.60} & 
3.06\\SCA8-1 & 1074.65 & 1.27 & 
1074.68 & 1.22 & \bf{1063.00} & 
1.10\\SCA8-2 & 1056.87 & 0.97 & 
1056.87 & 1.02 & \bf{1040.60} & 
1.56\\SCA8-3 & 1031.08 & 1.46 & 
1031.08 & 1.48 & \bf{985.90} & 
4.58\\SCA8-4 & 1099.06 & 1.53 & 
1099.06 & 1.50 & \bf{1071.00} & 
2.62\\SCA8-5 & 1055.35 & 1.54 & 
1055.35 & 1.58 & \bf{1054.30} & 
0.10\\SCA8-6 & \bf{\underline{972.48}} & 1.68 & 
972.48 & 1.65 & 972.50 & 
-0.00\\SCA8-7 & 1092.57 & 1.63 & 
1092.57 & 1.64 & \bf{1059.70} & 
3.10\\SCA8-8 & 1091.49 & 1.37 & 
1091.89 & 1.44 & \bf{1082.70} & 
0.81\\SCA8-9 & \bf{\underline{1067.42}} & 1.14 & 
1067.42 & 1.15 & 1081.40 & 
-1.29\\CON3-0 & 624.96 & 1.69 & 
624.96 & 1.66 & \bf{616.50} & 
1.37\\CON3-1 & 557.38 & 1.44 & 
558.22 & 1.44 & \bf{555.60} & 
0.32\\CON3-2 & 524.07 & 1.03 & 
525.97 & 1.09 & \bf{521.40} & 
0.51\\CON3-3 & 594.11 & 1.48 & 
594.11 & 1.52 & \bf{591.20} & 
0.49\\CON3-4 & 589.32 & 1.46 & 
589.32 & 1.45 & \bf{589.30} & 
0.00\\CON3-5 & 569.15 & 1.45 & 
574.61 & 1.44 & \bf{563.70} & 
0.97\\CON3-6 & 505.26 & 1.79 & 
506.43 & 1.79 & \bf{499.20} & 
1.21\\CON3-7 & 578.41 & 1.20 & 
579.12 & 1.23 & \bf{577.50} & 
0.16\\CON3-8 & 524.30 & 1.20 & 
524.52 & 1.34 & \bf{523.10} & 
0.23\\CON3-9 & 588.48 & 1.34 & 
588.48 & 1.26 & \bf{578.20} & 
1.78\\CON8-0 & 879.00 & 1.48 & 
879.00 & 1.43 & \bf{858.90} & 
2.34\\CON8-1 & 758.26 & 1.28 & 
758.26 & 1.31 & \bf{740.90} & 
2.34\\CON8-2 & 716.56 & 2.01 & 
716.56 & 2.01 & \bf{714.30} & 
0.32\\CON8-3 & 817.57 & 1.45 & 
817.57 & 1.45 & \bf{812.30} & 
0.65\\CON8-4 & 778.60 & 1.46 & 
782.97 & 1.64 & \bf{770.10} & 
1.10\\CON8-5 & \bf{\underline{764.36}} & 1.41 & 
764.36 & 1.38 & 766.60 & 
-0.29\\CON8-6 & \bf{\underline{693.83}} & 1.75 & 
703.68 & 1.73 & 697.20 & 
-0.48\\CON8-7 & 822.42 & 1.18 & 
822.67 & 1.17 & \bf{814.80} & 
0.94\\CON8-8 & 799.32 & 1.44 & 
799.41 & 1.51 & \bf{771.30} & 
3.63\\CON8-9 & 816.12 & 1.62 & 
817.40 & 1.59 & \bf{815.10} & 
0.13\\[1ex]\hline
\end{tabular}
\label{table:nonlin}
\end{table} \clearpage
\begin{table}[ht]
\caption{Resultados de la ejecución de la metaheurística ACO, utilizando instancias de Dethloff con la configuración -n 2.0 -alpha 1.0 -beta 3.0 -q 7.6 -ro 0.015}
\centering
\small
\begin{tabular}{c c c c c c c}
\hline\hline
Instancia & Costo mínimo & Tiempo(seg.) & Costo promedio & Tiempo promedio(seg.) & Costo ACO & \%Gap \\ [0.5ex]
\hline
SCA3-0 & 640.55 & 1.38 & 
640.55 & 1.34 & \bf{636.10} & 
0.70\\SCA3-1 & \bf{\underline{697.84}} & 1.54 & 
698.76 & 1.48 & 700.10 & 
-0.32\\SCA3-2 & 664.18 & 1.29 & 
664.18 & 1.32 & \bf{659.30} & 
0.74\\SCA3-3 & 680.60 & 1.48 & 
681.13 & 1.51 & \bf{680.00} & 
0.09\\SCA3-4 & \bf{690.50} & 1.49 & 
690.50 & 1.42 & 690.50 & 0.00\\
SCA3-5 & \bf{\underline{665.04}} & 1.34 & 
665.04 & 1.39 & 671.10 & 
-0.90\\SCA3-6 & 655.19 & 1.32 & 
655.19 & 1.35 & \bf{651.10} & 
0.63\\SCA3-7 & 666.15 & 1.08 & 
666.15 & 1.00 & \bf{666.10} & 
0.01\\SCA3-8 & 721.45 & 1.20 & 
725.97 & 1.18 & \bf{719.50} & 
0.27\\SCA3-9 & \bf{681.00} & 0.98 & 
681.00 & 0.96 & 681.00 & 0.00\\
SCA8-0 & 991.07 & 1.53 & 
992.23 & 1.50 & \bf{961.60} & 
3.06\\SCA8-1 & 1074.65 & 1.20 & 
1074.68 & 1.18 & \bf{1063.00} & 
1.10\\SCA8-2 & 1056.87 & 1.05 & 
1056.87 & 1.01 & \bf{1040.60} & 
1.56\\SCA8-3 & 1031.08 & 1.50 & 
1031.08 & 1.52 & \bf{985.90} & 
4.58\\SCA8-4 & 1099.06 & 1.59 & 
1099.06 & 1.54 & \bf{1071.00} & 
2.62\\SCA8-5 & 1055.35 & 1.75 & 
1055.35 & 1.61 & \bf{1054.30} & 
0.10\\SCA8-6 & \bf{\underline{972.48}} & 1.60 & 
976.70 & 1.63 & 972.50 & 
-0.00\\SCA8-7 & 1092.57 & 1.54 & 
1092.57 & 1.62 & \bf{1059.70} & 
3.10\\SCA8-8 & 1091.49 & 1.40 & 
1091.89 & 1.44 & \bf{1082.70} & 
0.81\\SCA8-9 & \bf{\underline{1067.42}} & 1.19 & 
1067.42 & 1.18 & 1081.40 & 
-1.29\\CON3-0 & 624.96 & 1.64 & 
624.96 & 1.69 & \bf{616.50} & 
1.37\\CON3-1 & 557.38 & 1.48 & 
559.50 & 1.42 & \bf{555.60} & 
0.32\\CON3-2 & 524.07 & 1.12 & 
525.23 & 1.13 & \bf{521.40} & 
0.51\\CON3-3 & 594.11 & 1.50 & 
594.11 & 1.48 & \bf{591.20} & 
0.49\\CON3-4 & 589.32 & 1.25 & 
589.32 & 1.34 & \bf{589.30} & 
0.00\\CON3-5 & 570.70 & 1.43 & 
575.00 & 1.42 & \bf{563.70} & 
1.24\\CON3-6 & 505.26 & 1.85 & 
507.61 & 1.78 & \bf{499.20} & 
1.21\\CON3-7 & 578.41 & 1.17 & 
579.12 & 1.20 & \bf{577.50} & 
0.16\\CON3-8 & 524.30 & 1.28 & 
524.52 & 1.21 & \bf{523.10} & 
0.23\\CON3-9 & 588.48 & 1.26 & 
588.48 & 1.26 & \bf{578.20} & 
1.78\\CON8-0 & 879.00 & 1.36 & 
879.00 & 1.43 & \bf{858.90} & 
2.34\\CON8-1 & 758.26 & 1.44 & 
758.26 & 1.38 & \bf{740.90} & 
2.34\\CON8-2 & 716.53 & 1.94 & 
716.54 & 1.97 & \bf{714.30} & 
0.31\\CON8-3 & 817.57 & 1.40 & 
817.57 & 1.43 & \bf{812.30} & 
0.65\\CON8-4 & 781.64 & 1.56 & 
788.67 & 1.61 & \bf{770.10} & 
1.50\\CON8-5 & \bf{\underline{764.36}} & 1.41 & 
764.36 & 1.37 & 766.60 & 
-0.29\\CON8-6 & 705.61 & 1.76 & 
707.31 & 1.72 & \bf{697.20} & 
1.21\\CON8-7 & 822.42 & 1.16 & 
822.92 & 1.15 & \bf{814.80} & 
0.94\\CON8-8 & 796.81 & 1.54 & 
798.79 & 1.51 & \bf{771.30} & 
3.31\\CON8-9 & 816.12 & 1.54 & 
817.40 & 1.54 & \bf{815.10} & 
0.13\\[1ex]\hline
\end{tabular}
\label{table:nonlin}
\end{table} \clearpage
\begin{table}[ht]
\caption{Resultados de la ejecución de la metaheurística ACO, utilizando instancias de Dethloff con la configuración -n 2.0 -alpha 1.0 -beta 3.0 -q 7.7 -ro 0.015}
\centering
\small
\begin{tabular}{c c c c c c c}
\hline\hline
Instancia & Costo mínimo & Tiempo(seg.) & Costo promedio & Tiempo promedio(seg.) & Costo ACO & \%Gap \\ [0.5ex]
\hline
SCA3-0 & 640.55 & 1.42 & 
640.55 & 1.39 & \bf{636.10} & 
0.70\\SCA3-1 & \bf{\underline{697.84}} & 1.47 & 
697.84 & 1.47 & 700.10 & 
-0.32\\SCA3-2 & 659.34 & 1.32 & 
661.76 & 1.29 & \bf{659.30} & 
0.01\\SCA3-3 & 680.60 & 1.60 & 
680.96 & 1.53 & \bf{680.00} & 
0.09\\SCA3-4 & \bf{690.50} & 1.38 & 
690.50 & 1.38 & 690.50 & 0.00\\
SCA3-5 & \bf{\underline{665.04}} & 1.29 & 
665.19 & 1.42 & 671.10 & 
-0.90\\SCA3-6 & 655.19 & 1.24 & 
655.19 & 1.33 & \bf{651.10} & 
0.63\\SCA3-7 & 666.15 & 0.99 & 
666.15 & 1.01 & \bf{666.10} & 
0.01\\SCA3-8 & 721.45 & 1.05 & 
723.34 & 1.08 & \bf{719.50} & 
0.27\\SCA3-9 & \bf{681.00} & 0.96 & 
681.00 & 0.97 & 681.00 & 0.00\\
SCA8-0 & 991.07 & 1.51 & 
991.07 & 1.51 & \bf{961.60} & 
3.06\\SCA8-1 & 1074.65 & 1.12 & 
1074.68 & 1.15 & \bf{1063.00} & 
1.10\\SCA8-2 & 1056.87 & 1.14 & 
1056.87 & 1.03 & \bf{1040.60} & 
1.56\\SCA8-3 & 1031.08 & 1.41 & 
1031.08 & 1.42 & \bf{985.90} & 
4.58\\SCA8-4 & 1099.06 & 1.41 & 
1099.06 & 1.53 & \bf{1071.00} & 
2.62\\SCA8-5 & 1055.35 & 1.54 & 
1055.35 & 1.59 & \bf{1054.30} & 
0.10\\SCA8-6 & \bf{\underline{972.48}} & 1.64 & 
974.39 & 1.67 & 972.50 & 
-0.00\\SCA8-7 & 1092.57 & 1.58 & 
1092.57 & 1.63 & \bf{1059.70} & 
3.10\\SCA8-8 & 1085.93 & 1.40 & 
1090.50 & 1.39 & \bf{1082.70} & 
0.30\\SCA8-9 & \bf{\underline{1067.42}} & 1.17 & 
1067.42 & 1.15 & 1081.40 & 
-1.29\\CON3-0 & 624.96 & 1.67 & 
624.96 & 1.64 & \bf{616.50} & 
1.37\\CON3-1 & 557.38 & 1.42 & 
558.22 & 1.43 & \bf{555.60} & 
0.32\\CON3-2 & 524.07 & 1.07 & 
524.07 & 1.11 & \bf{521.40} & 
0.51\\CON3-3 & \bf{591.20} & 1.55 & 
593.09 & 1.48 & 591.20 & 0.00\\
CON3-4 & 589.32 & 1.31 & 
589.32 & 1.27 & \bf{589.30} & 
0.00\\CON3-5 & 576.43 & 1.38 & 
576.43 & 1.50 & \bf{563.70} & 
2.26\\CON3-6 & 505.26 & 1.82 & 
507.03 & 1.77 & \bf{499.20} & 
1.21\\CON3-7 & 578.41 & 1.18 & 
579.12 & 1.23 & \bf{577.50} & 
0.16\\CON3-8 & 524.30 & 1.16 & 
524.52 & 1.23 & \bf{523.10} & 
0.23\\CON3-9 & 588.48 & 1.31 & 
588.48 & 1.28 & \bf{578.20} & 
1.78\\CON8-0 & 879.00 & 1.50 & 
879.00 & 1.47 & \bf{858.90} & 
2.34\\CON8-1 & 758.26 & 1.25 & 
758.28 & 1.29 & \bf{740.90} & 
2.34\\CON8-2 & 716.53 & 2.00 & 
716.54 & 1.97 & \bf{714.30} & 
0.31\\CON8-3 & 817.57 & 1.44 & 
817.57 & 1.47 & \bf{812.30} & 
0.65\\CON8-4 & 789.98 & 1.60 & 
790.76 & 1.54 & \bf{770.10} & 
2.58\\CON8-5 & \bf{\underline{764.36}} & 1.44 & 
764.36 & 1.35 & 766.60 & 
-0.29\\CON8-6 & 707.88 & 1.57 & 
707.88 & 1.66 & \bf{697.20} & 
1.53\\CON8-7 & 822.42 & 1.20 & 
823.03 & 1.17 & \bf{814.80} & 
0.94\\CON8-8 & 799.32 & 1.50 & 
799.41 & 1.55 & \bf{771.30} & 
3.63\\CON8-9 & 816.12 & 1.51 & 
817.40 & 1.54 & \bf{815.10} & 
0.13\\[1ex]\hline
\end{tabular}
\label{table:nonlin}
\end{table} \clearpage
\begin{table}[ht]
\caption{Resultados de la ejecución de la metaheurística ACO, utilizando instancias de Dethloff con la configuración -n 2.0 -alpha 1.0 -beta 3.0 -q 7.8 -ro 0.015}
\centering
\small
\begin{tabular}{c c c c c c c}
\hline\hline
Instancia & Costo mínimo & Tiempo(seg.) & Costo promedio & Tiempo promedio(seg.) & Costo ACO & \%Gap \\ [0.5ex]
\hline
SCA3-0 & 640.55 & 1.40 & 
640.55 & 1.38 & \bf{636.10} & 
0.70\\SCA3-1 & \bf{\underline{697.84}} & 1.56 & 
697.84 & 1.50 & 700.10 & 
-0.32\\SCA3-2 & 659.34 & 1.36 & 
662.97 & 1.38 & \bf{659.30} & 
0.01\\SCA3-3 & 680.60 & 1.47 & 
681.06 & 1.47 & \bf{680.00} & 
0.09\\SCA3-4 & \bf{690.50} & 1.46 & 
690.50 & 1.42 & 690.50 & 0.00\\
SCA3-5 & \bf{\underline{665.04}} & 1.50 & 
665.04 & 1.42 & 671.10 & 
-0.90\\SCA3-6 & 655.19 & 1.42 & 
655.41 & 1.36 & \bf{651.10} & 
0.63\\SCA3-7 & 666.15 & 0.93 & 
666.15 & 0.98 & \bf{666.10} & 
0.01\\SCA3-8 & 721.45 & 1.10 & 
724.19 & 1.14 & \bf{719.50} & 
0.27\\SCA3-9 & \bf{681.00} & 0.94 & 
681.00 & 0.94 & 681.00 & 0.00\\
SCA8-0 & 991.07 & 1.57 & 
991.07 & 1.52 & \bf{961.60} & 
3.06\\SCA8-1 & 1074.65 & 1.18 & 
1074.65 & 1.20 & \bf{1063.00} & 
1.10\\SCA8-2 & 1056.87 & 1.06 & 
1056.87 & 1.02 & \bf{1040.60} & 
1.56\\SCA8-3 & 1031.08 & 1.39 & 
1031.08 & 1.43 & \bf{985.90} & 
4.58\\SCA8-4 & 1099.06 & 1.42 & 
1099.06 & 1.48 & \bf{1071.00} & 
2.62\\SCA8-5 & 1055.35 & 1.80 & 
1055.35 & 1.69 & \bf{1054.30} & 
0.10\\SCA8-6 & \bf{\underline{972.48}} & 1.56 & 
974.39 & 1.65 & 972.50 & 
-0.00\\SCA8-7 & 1092.57 & 1.61 & 
1092.57 & 1.62 & \bf{1059.70} & 
3.10\\SCA8-8 & 1091.49 & 1.44 & 
1091.89 & 1.45 & \bf{1082.70} & 
0.81\\SCA8-9 & \bf{\underline{1067.42}} & 1.18 & 
1067.42 & 1.17 & 1081.40 & 
-1.29\\CON3-0 & 624.96 & 1.70 & 
624.96 & 1.69 & \bf{616.50} & 
1.37\\CON3-1 & 557.38 & 1.45 & 
557.38 & 1.51 & \bf{555.60} & 
0.32\\CON3-2 & 524.07 & 1.18 & 
525.06 & 1.09 & \bf{521.40} & 
0.51\\CON3-3 & 594.11 & 1.50 & 
594.11 & 1.54 & \bf{591.20} & 
0.49\\CON3-4 & 589.32 & 1.28 & 
589.32 & 1.31 & \bf{589.30} & 
0.00\\CON3-5 & 574.57 & 1.39 & 
575.97 & 1.41 & \bf{563.70} & 
1.93\\CON3-6 & 505.26 & 1.74 & 
505.26 & 1.77 & \bf{499.20} & 
1.21\\CON3-7 & 578.41 & 1.22 & 
579.84 & 1.23 & \bf{577.50} & 
0.16\\CON3-8 & 524.59 & 1.18 & 
524.59 & 1.22 & \bf{523.10} & 
0.28\\CON3-9 & 578.25 & 1.18 & 
585.92 & 1.21 & \bf{578.20} & 
0.01\\CON8-0 & 879.00 & 1.42 & 
879.00 & 1.43 & \bf{858.90} & 
2.34\\CON8-1 & 758.26 & 1.34 & 
758.26 & 1.35 & \bf{740.90} & 
2.34\\CON8-2 & 716.56 & 1.98 & 
716.56 & 2.01 & \bf{714.30} & 
0.32\\CON8-3 & 817.57 & 1.47 & 
817.57 & 1.48 & \bf{812.30} & 
0.65\\CON8-4 & 778.60 & 1.56 & 
782.97 & 1.56 & \bf{770.10} & 
1.10\\CON8-5 & \bf{\underline{764.36}} & 1.32 & 
764.36 & 1.38 & 766.60 & 
-0.29\\CON8-6 & 705.61 & 1.67 & 
706.51 & 1.81 & \bf{697.20} & 
1.21\\CON8-7 & 822.42 & 1.16 & 
822.78 & 1.15 & \bf{814.80} & 
0.94\\CON8-8 & 799.16 & 1.63 & 
799.42 & 1.57 & \bf{771.30} & 
3.61\\CON8-9 & 816.12 & 1.62 & 
817.40 & 1.57 & \bf{815.10} & 
0.13\\[1ex]\hline
\end{tabular}
\label{table:nonlin}
\end{table} \clearpage
\begin{table}[ht]
\caption{Resultados de la ejecución de la metaheurística ACO, utilizando instancias de Dethloff con la configuración -n 2.0 -alpha 1.0 -beta 3.0 -q 7.9 -ro 0.015}
\centering
\small
\begin{tabular}{c c c c c c c}
\hline\hline
Instancia & Costo mínimo & Tiempo(seg.) & Costo promedio & Tiempo promedio(seg.) & Costo ACO & \%Gap \\ [0.5ex]
\hline
SCA3-0 & 640.55 & 1.42 & 
640.55 & 1.36 & \bf{636.10} & 
0.70\\SCA3-1 & \bf{\underline{697.84}} & 1.52 & 
697.84 & 1.51 & 700.10 & 
-0.32\\SCA3-2 & 664.18 & 1.36 & 
664.18 & 1.34 & \bf{659.30} & 
0.74\\SCA3-3 & 680.60 & 1.58 & 
680.96 & 1.49 & \bf{680.00} & 
0.09\\SCA3-4 & \bf{690.50} & 1.38 & 
690.50 & 1.47 & 690.50 & 0.00\\
SCA3-5 & \bf{\underline{665.04}} & 1.44 & 
665.04 & 1.44 & 671.10 & 
-0.90\\SCA3-6 & 655.19 & 1.43 & 
655.19 & 1.46 & \bf{651.10} & 
0.63\\SCA3-7 & 666.15 & 1.04 & 
666.15 & 1.01 & \bf{666.10} & 
0.01\\SCA3-8 & 721.45 & 1.12 & 
721.45 & 1.12 & \bf{719.50} & 
0.27\\SCA3-9 & \bf{681.00} & 0.96 & 
681.00 & 0.95 & 681.00 & 0.00\\
SCA8-0 & 991.07 & 1.53 & 
992.23 & 1.54 & \bf{961.60} & 
3.06\\SCA8-1 & 1074.65 & 1.14 & 
1074.65 & 1.22 & \bf{1063.00} & 
1.10\\SCA8-2 & 1056.87 & 1.05 & 
1056.87 & 1.06 & \bf{1040.60} & 
1.56\\SCA8-3 & 1031.08 & 1.51 & 
1031.08 & 1.47 & \bf{985.90} & 
4.58\\SCA8-4 & 1099.06 & 1.44 & 
1099.06 & 1.50 & \bf{1071.00} & 
2.62\\SCA8-5 & 1055.35 & 1.65 & 
1055.35 & 1.70 & \bf{1054.30} & 
0.10\\SCA8-6 & \bf{\underline{972.48}} & 1.62 & 
976.70 & 1.74 & 972.50 & 
-0.00\\SCA8-7 & 1092.57 & 1.67 & 
1092.57 & 1.67 & \bf{1059.70} & 
3.10\\SCA8-8 & \bf{\underline{1075.00}} & 1.48 & 
1087.63 & 1.48 & 1082.70 & 
-0.71\\SCA8-9 & \bf{\underline{1067.42}} & 1.12 & 
1067.42 & 1.15 & 1081.40 & 
-1.29\\CON3-0 & 624.96 & 1.77 & 
624.96 & 1.67 & \bf{616.50} & 
1.37\\CON3-1 & 557.38 & 1.44 & 
557.38 & 1.49 & \bf{555.60} & 
0.32\\CON3-2 & 524.07 & 1.12 & 
526.92 & 1.10 & \bf{521.40} & 
0.51\\CON3-3 & \bf{591.20} & 1.54 & 
591.93 & 1.54 & 591.20 & 0.00\\
CON3-4 & 589.32 & 1.28 & 
589.32 & 1.31 & \bf{589.30} & 
0.00\\CON3-5 & 576.43 & 1.40 & 
577.34 & 1.44 & \bf{563.70} & 
2.26\\CON3-6 & 505.26 & 1.66 & 
505.78 & 1.82 & \bf{499.20} & 
1.21\\CON3-7 & 578.41 & 1.18 & 
578.41 & 1.21 & \bf{577.50} & 
0.16\\CON3-8 & 524.59 & 1.23 & 
524.59 & 1.23 & \bf{523.10} & 
0.28\\CON3-9 & 588.48 & 1.36 & 
588.48 & 1.32 & \bf{578.20} & 
1.78\\CON8-0 & 879.00 & 1.46 & 
879.00 & 1.44 & \bf{858.90} & 
2.34\\CON8-1 & 758.26 & 1.40 & 
758.26 & 1.41 & \bf{740.90} & 
2.34\\CON8-2 & 716.53 & 1.91 & 
716.54 & 2.02 & \bf{714.30} & 
0.31\\CON8-3 & 817.57 & 1.50 & 
817.57 & 1.46 & \bf{812.30} & 
0.65\\CON8-4 & 778.60 & 1.64 & 
785.05 & 1.61 & \bf{770.10} & 
1.10\\CON8-5 & \bf{\underline{764.36}} & 1.44 & 
764.36 & 1.43 & 766.60 & 
-0.29\\CON8-6 & \bf{\underline{693.83}} & 1.76 & 
703.68 & 1.75 & 697.20 & 
-0.48\\CON8-7 & 822.42 & 1.22 & 
822.67 & 1.23 & \bf{814.80} & 
0.94\\CON8-8 & 799.32 & 1.68 & 
799.46 & 1.65 & \bf{771.30} & 
3.63\\CON8-9 & 816.12 & 1.64 & 
816.12 & 1.61 & \bf{815.10} & 
0.13\\[1ex]\hline
\end{tabular}
\label{table:nonlin}
\end{table} \clearpage
\begin{table}[ht]
\caption{Resultados de la ejecución de la metaheurística ACO, utilizando instancias de Dethloff con la configuración -n 2.0 -alpha 1.0 -beta 3.0 -q 8.0 -ro 0.015}
\centering
\small
\begin{tabular}{c c c c c c c}
\hline\hline
Instancia & Costo mínimo & Tiempo(seg.) & Costo promedio & Tiempo promedio(seg.) & Costo ACO & \%Gap \\ [0.5ex]
\hline
SCA3-0 & 640.55 & 1.38 & 
640.55 & 1.42 & \bf{636.10} & 
0.70\\SCA3-1 & \bf{\underline{697.84}} & 1.50 & 
697.84 & 1.56 & 700.10 & 
-0.32\\SCA3-2 & 664.18 & 1.34 & 
664.18 & 1.38 & \bf{659.30} & 
0.74\\SCA3-3 & 680.60 & 1.52 & 
680.60 & 1.55 & \bf{680.00} & 
0.09\\SCA3-4 & \bf{690.50} & 1.45 & 
690.50 & 1.45 & 690.50 & 0.00\\
SCA3-5 & \bf{\underline{665.04}} & 1.35 & 
665.19 & 1.46 & 671.10 & 
-0.90\\SCA3-6 & 655.19 & 1.38 & 
655.19 & 1.34 & \bf{651.10} & 
0.63\\SCA3-7 & 666.15 & 1.01 & 
666.15 & 1.05 & \bf{666.10} & 
0.01\\SCA3-8 & 721.45 & 1.16 & 
726.48 & 1.17 & \bf{719.50} & 
0.27\\SCA3-9 & \bf{681.00} & 0.96 & 
681.00 & 1.00 & 681.00 & 0.00\\
SCA8-0 & 991.07 & 1.54 & 
993.89 & 1.54 & \bf{961.60} & 
3.06\\SCA8-1 & 1074.65 & 1.24 & 
1074.65 & 1.22 & \bf{1063.00} & 
1.10\\SCA8-2 & 1056.87 & 0.98 & 
1056.87 & 1.05 & \bf{1040.60} & 
1.56\\SCA8-3 & 1031.08 & 1.55 & 
1031.08 & 1.57 & \bf{985.90} & 
4.58\\SCA8-4 & 1099.06 & 1.45 & 
1099.06 & 1.52 & \bf{1071.00} & 
2.62\\SCA8-5 & 1055.35 & 1.60 & 
1055.35 & 1.64 & \bf{1054.30} & 
0.10\\SCA8-6 & \bf{\underline{972.48}} & 1.74 & 
976.70 & 1.73 & 972.50 & 
-0.00\\SCA8-7 & 1092.57 & 1.65 & 
1092.57 & 1.65 & \bf{1059.70} & 
3.10\\SCA8-8 & 1091.49 & 1.49 & 
1091.89 & 1.47 & \bf{1082.70} & 
0.81\\SCA8-9 & \bf{\underline{1067.42}} & 1.20 & 
1067.42 & 1.18 & 1081.40 & 
-1.29\\CON3-0 & 624.96 & 1.71 & 
624.96 & 1.69 & \bf{616.50} & 
1.37\\CON3-1 & 557.38 & 1.46 & 
557.38 & 1.50 & \bf{555.60} & 
0.32\\CON3-2 & 524.07 & 1.18 & 
524.64 & 1.11 & \bf{521.40} & 
0.51\\CON3-3 & 592.95 & 1.53 & 
593.82 & 1.53 & \bf{591.20} & 
0.30\\CON3-4 & \bf{\underline{588.79}} & 1.37 & 
589.19 & 1.41 & 589.30 & 
-0.09\\CON3-5 & 569.15 & 1.52 & 
573.14 & 1.49 & \bf{563.70} & 
0.97\\CON3-6 & 504.15 & 1.78 & 
505.78 & 1.78 & \bf{499.20} & 
0.99\\CON3-7 & 578.41 & 1.30 & 
579.12 & 1.26 & \bf{577.50} & 
0.16\\CON3-8 & 524.59 & 1.36 & 
524.59 & 1.26 & \bf{523.10} & 
0.28\\CON3-9 & 588.48 & 1.35 & 
588.48 & 1.31 & \bf{578.20} & 
1.78\\CON8-0 & 879.00 & 1.54 & 
879.00 & 1.47 & \bf{858.90} & 
2.34\\CON8-1 & 758.26 & 1.40 & 
758.26 & 1.37 & \bf{740.90} & 
2.34\\CON8-2 & 716.53 & 1.95 & 
716.54 & 2.06 & \bf{714.30} & 
0.31\\CON8-3 & 817.57 & 1.50 & 
817.57 & 1.46 & \bf{812.30} & 
0.65\\CON8-4 & 781.64 & 1.56 & 
787.89 & 1.59 & \bf{770.10} & 
1.50\\CON8-5 & \bf{\underline{764.36}} & 1.40 & 
764.36 & 1.41 & 766.60 & 
-0.29\\CON8-6 & \bf{\underline{693.83}} & 1.67 & 
703.57 & 1.67 & 697.20 & 
-0.48\\CON8-7 & 823.43 & 1.12 & 
823.43 & 1.17 & \bf{814.80} & 
1.06\\CON8-8 & 799.32 & 1.54 & 
799.41 & 1.57 & \bf{771.30} & 
3.63\\CON8-9 & 816.12 & 1.55 & 
816.12 & 1.61 & \bf{815.10} & 
0.13\\[1ex]\hline
\end{tabular}
\label{table:nonlin}
\end{table} \clearpage
\begin{table}[ht]
\caption{Resultados de la ejecución de la metaheurística ACO, utilizando instancias de Dethloff con la configuración -n 2.0 -alpha 1.0 -beta 3.0 -q 8.1 -ro 0.015}
\centering
\small
\begin{tabular}{c c c c c c c}
\hline\hline
Instancia & Costo mínimo & Tiempo(seg.) & Costo promedio & Tiempo promedio(seg.) & Costo ACO & \%Gap \\ [0.5ex]
\hline
SCA3-0 & 640.55 & 1.48 & 
640.55 & 1.43 & \bf{636.10} & 
0.70\\SCA3-1 & \bf{\underline{697.84}} & 1.52 & 
697.84 & 1.55 & 700.10 & 
-0.32\\SCA3-2 & 664.18 & 1.30 & 
664.18 & 1.37 & \bf{659.30} & 
0.74\\SCA3-3 & 680.60 & 1.38 & 
680.78 & 1.48 & \bf{680.00} & 
0.09\\SCA3-4 & \bf{690.50} & 1.42 & 
690.50 & 1.44 & 690.50 & 0.00\\
SCA3-5 & \bf{\underline{665.04}} & 1.44 & 
665.19 & 1.43 & 671.10 & 
-0.90\\SCA3-6 & 653.69 & 1.30 & 
654.93 & 1.34 & \bf{651.10} & 
0.40\\SCA3-7 & 666.15 & 0.92 & 
666.15 & 1.02 & \bf{666.10} & 
0.01\\SCA3-8 & 721.45 & 1.12 & 
722.70 & 1.19 & \bf{719.50} & 
0.27\\SCA3-9 & \bf{681.00} & 0.97 & 
681.00 & 0.96 & 681.00 & 0.00\\
SCA8-0 & 991.07 & 1.48 & 
991.65 & 1.55 & \bf{961.60} & 
3.06\\SCA8-1 & 1074.65 & 1.22 & 
1074.68 & 1.18 & \bf{1063.00} & 
1.10\\SCA8-2 & 1056.87 & 1.05 & 
1056.87 & 1.06 & \bf{1040.60} & 
1.56\\SCA8-3 & 1031.08 & 1.46 & 
1031.08 & 1.43 & \bf{985.90} & 
4.58\\SCA8-4 & 1099.06 & 1.43 & 
1099.06 & 1.50 & \bf{1071.00} & 
2.62\\SCA8-5 & 1055.35 & 2.14 & 
1055.35 & 1.80 & \bf{1054.30} & 
0.10\\SCA8-6 & \bf{\underline{972.48}} & 1.66 & 
972.48 & 1.67 & 972.50 & 
-0.00\\SCA8-7 & 1092.57 & 1.60 & 
1092.57 & 1.59 & \bf{1059.70} & 
3.10\\SCA8-8 & 1091.49 & 1.35 & 
1091.89 & 1.45 & \bf{1082.70} & 
0.81\\SCA8-9 & \bf{\underline{1067.42}} & 1.08 & 
1067.42 & 1.14 & 1081.40 & 
-1.29\\CON3-0 & 624.96 & 1.60 & 
624.96 & 1.61 & \bf{616.50} & 
1.37\\CON3-1 & 557.38 & 1.57 & 
558.30 & 1.51 & \bf{555.60} & 
0.32\\CON3-2 & 524.07 & 1.09 & 
525.23 & 1.12 & \bf{521.40} & 
0.51\\CON3-3 & \bf{591.20} & 1.52 & 
593.38 & 1.51 & 591.20 & 0.00\\
CON3-4 & 589.32 & 2.11 & 
589.32 & 1.51 & \bf{589.30} & 
0.00\\CON3-5 & 569.88 & 1.35 & 
574.79 & 1.43 & \bf{563.70} & 
1.10\\CON3-6 & 505.26 & 1.79 & 
505.86 & 1.81 & \bf{499.20} & 
1.21\\CON3-7 & 578.41 & 1.35 & 
579.12 & 1.24 & \bf{577.50} & 
0.16\\CON3-8 & 524.59 & 1.12 & 
525.40 & 1.14 & \bf{523.10} & 
0.28\\CON3-9 & 588.48 & 1.26 & 
588.48 & 1.25 & \bf{578.20} & 
1.78\\CON8-0 & 879.00 & 1.44 & 
879.00 & 1.44 & \bf{858.90} & 
2.34\\CON8-1 & 758.26 & 1.35 & 
758.26 & 1.39 & \bf{740.90} & 
2.34\\CON8-2 & 716.53 & 2.03 & 
716.54 & 2.00 & \bf{714.30} & 
0.31\\CON8-3 & 817.57 & 1.51 & 
817.57 & 1.46 & \bf{812.30} & 
0.65\\CON8-4 & 778.60 & 1.64 & 
782.21 & 1.54 & \bf{770.10} & 
1.10\\CON8-5 & \bf{\underline{764.36}} & 1.33 & 
764.36 & 1.37 & 766.60 & 
-0.29\\CON8-6 & 705.61 & 1.72 & 
706.82 & 1.71 & \bf{697.20} & 
1.21\\CON8-7 & 822.42 & 1.24 & 
822.67 & 1.20 & \bf{814.80} & 
0.94\\CON8-8 & 799.32 & 1.50 & 
799.41 & 1.53 & \bf{771.30} & 
3.63\\CON8-9 & 816.12 & 1.53 & 
817.40 & 1.55 & \bf{815.10} & 
0.13\\[1ex]\hline
\end{tabular}
\label{table:nonlin}
\end{table} \clearpage
\begin{table}[ht]
\caption{Resultados de la ejecución de la metaheurística ACO, utilizando instancias de Dethloff con la configuración -n 2.0 -alpha 1.0 -beta 3.0 -q 8.2 -ro 0.015}
\centering
\small
\begin{tabular}{c c c c c c c}
\hline\hline
Instancia & Costo mínimo & Tiempo(seg.) & Costo promedio & Tiempo promedio(seg.) & Costo ACO & \%Gap \\ [0.5ex]
\hline
SCA3-0 & 640.55 & 1.30 & 
640.55 & 1.37 & \bf{636.10} & 
0.70\\SCA3-1 & \bf{\underline{697.84}} & 1.50 & 
697.84 & 1.48 & 700.10 & 
-0.32\\SCA3-2 & 659.34 & 1.28 & 
662.97 & 1.35 & \bf{659.30} & 
0.01\\SCA3-3 & 680.60 & 1.55 & 
680.78 & 1.58 & \bf{680.00} & 
0.09\\SCA3-4 & \bf{690.50} & 1.38 & 
690.50 & 1.40 & 690.50 & 0.00\\
SCA3-5 & \bf{\underline{665.04}} & 1.41 & 
665.19 & 1.40 & 671.10 & 
-0.90\\SCA3-6 & 655.19 & 1.28 & 
655.19 & 1.41 & \bf{651.10} & 
0.63\\SCA3-7 & 666.15 & 0.96 & 
666.15 & 0.99 & \bf{666.10} & 
0.01\\SCA3-8 & 721.45 & 1.11 & 
725.24 & 1.18 & \bf{719.50} & 
0.27\\SCA3-9 & \bf{681.00} & 0.96 & 
681.00 & 0.95 & 681.00 & 0.00\\
SCA8-0 & 991.07 & 1.49 & 
991.07 & 1.53 & \bf{961.60} & 
3.06\\SCA8-1 & 1074.65 & 1.19 & 
1074.65 & 1.20 & \bf{1063.00} & 
1.10\\SCA8-2 & 1056.87 & 1.04 & 
1056.87 & 1.03 & \bf{1040.60} & 
1.56\\SCA8-3 & 1031.08 & 1.46 & 
1031.08 & 1.48 & \bf{985.90} & 
4.58\\SCA8-4 & 1098.34 & 1.43 & 
1098.88 & 1.47 & \bf{1071.00} & 
2.55\\SCA8-5 & 1055.35 & 1.62 & 
1055.35 & 1.63 & \bf{1054.30} & 
0.10\\SCA8-6 & \bf{\underline{972.48}} & 1.68 & 
975.30 & 1.77 & 972.50 & 
-0.00\\SCA8-7 & 1092.57 & 1.71 & 
1092.57 & 1.67 & \bf{1059.70} & 
3.10\\SCA8-8 & 1091.49 & 1.46 & 
1091.89 & 1.50 & \bf{1082.70} & 
0.81\\SCA8-9 & \bf{\underline{1067.42}} & 1.20 & 
1067.42 & 1.20 & 1081.40 & 
-1.29\\CON3-0 & 624.96 & 1.61 & 
624.96 & 1.61 & \bf{616.50} & 
1.37\\CON3-1 & 557.38 & 1.44 & 
558.22 & 1.49 & \bf{555.60} & 
0.32\\CON3-2 & 524.07 & 1.06 & 
524.75 & 1.12 & \bf{521.40} & 
0.51\\CON3-3 & 594.11 & 1.46 & 
594.11 & 1.48 & \bf{591.20} & 
0.49\\CON3-4 & 589.32 & 1.24 & 
589.32 & 1.35 & \bf{589.30} & 
0.00\\CON3-5 & 570.70 & 1.40 & 
575.00 & 1.45 & \bf{563.70} & 
1.24\\CON3-6 & 505.26 & 1.84 & 
506.43 & 1.84 & \bf{499.20} & 
1.21\\CON3-7 & 578.41 & 1.24 & 
579.12 & 1.22 & \bf{577.50} & 
0.16\\CON3-8 & 524.59 & 1.24 & 
524.59 & 1.24 & \bf{523.10} & 
0.28\\CON3-9 & 588.48 & 1.30 & 
588.48 & 1.28 & \bf{578.20} & 
1.78\\CON8-0 & 879.00 & 1.39 & 
879.00 & 1.45 & \bf{858.90} & 
2.34\\CON8-1 & 758.26 & 1.37 & 
758.26 & 1.35 & \bf{740.90} & 
2.34\\CON8-2 & 716.53 & 2.02 & 
716.54 & 2.08 & \bf{714.30} & 
0.31\\CON8-3 & 817.57 & 1.41 & 
817.57 & 1.45 & \bf{812.30} & 
0.65\\CON8-4 & 778.60 & 1.50 & 
787.91 & 1.57 & \bf{770.10} & 
1.10\\CON8-5 & \bf{\underline{764.36}} & 1.44 & 
764.36 & 1.39 & 766.60 & 
-0.29\\CON8-6 & 707.41 & 1.64 & 
707.64 & 1.69 & \bf{697.20} & 
1.46\\CON8-7 & 822.42 & 1.10 & 
822.92 & 1.18 & \bf{814.80} & 
0.94\\CON8-8 & 799.32 & 1.60 & 
799.46 & 1.59 & \bf{771.30} & 
3.63\\CON8-9 & 816.12 & 1.61 & 
816.12 & 1.54 & \bf{815.10} & 
0.13\\[1ex]\hline
\end{tabular}
\label{table:nonlin}
\end{table} \clearpage
\begin{table}[ht]
\caption{Resultados de la ejecución de la metaheurística ACO, utilizando instancias de Dethloff con la configuración -n 2.0 -alpha 1.0 -beta 3.0 -q 8.3 -ro 0.015}
\centering
\small
\begin{tabular}{c c c c c c c}
\hline\hline
Instancia & Costo mínimo & Tiempo(seg.) & Costo promedio & Tiempo promedio(seg.) & Costo ACO & \%Gap \\ [0.5ex]
\hline
SCA3-0 & 640.55 & 1.48 & 
640.55 & 1.39 & \bf{636.10} & 
0.70\\SCA3-1 & \bf{\underline{697.84}} & 1.62 & 
697.84 & 1.52 & 700.10 & 
-0.32\\SCA3-2 & 659.34 & 1.33 & 
661.76 & 1.32 & \bf{659.30} & 
0.01\\SCA3-3 & 680.60 & 1.50 & 
680.78 & 1.47 & \bf{680.00} & 
0.09\\SCA3-4 & \bf{690.50} & 1.33 & 
690.50 & 1.46 & 690.50 & 0.00\\
SCA3-5 & \bf{\underline{665.04}} & 1.36 & 
665.79 & 1.45 & 671.10 & 
-0.90\\SCA3-6 & 655.19 & 1.29 & 
655.30 & 1.39 & \bf{651.10} & 
0.63\\SCA3-7 & 666.15 & 1.02 & 
666.15 & 1.00 & \bf{666.10} & 
0.01\\SCA3-8 & 721.45 & 1.12 & 
725.24 & 1.14 & \bf{719.50} & 
0.27\\SCA3-9 & \bf{681.00} & 1.10 & 
681.00 & 1.04 & 681.00 & 0.00\\
SCA8-0 & 991.07 & 1.49 & 
991.65 & 1.56 & \bf{961.60} & 
3.06\\SCA8-1 & 1074.65 & 1.16 & 
1074.68 & 1.19 & \bf{1063.00} & 
1.10\\SCA8-2 & 1056.87 & 1.01 & 
1056.87 & 1.04 & \bf{1040.60} & 
1.56\\SCA8-3 & 1031.08 & 1.54 & 
1031.08 & 1.54 & \bf{985.90} & 
4.58\\SCA8-4 & 1099.06 & 1.48 & 
1099.06 & 1.46 & \bf{1071.00} & 
2.62\\SCA8-5 & 1055.35 & 1.64 & 
1055.35 & 1.64 & \bf{1054.30} & 
0.10\\SCA8-6 & \bf{\underline{972.48}} & 1.68 & 
975.00 & 1.70 & 972.50 & 
-0.00\\SCA8-7 & 1092.57 & 1.68 & 
1092.57 & 1.65 & \bf{1059.70} & 
3.10\\SCA8-8 & 1092.02 & 1.46 & 
1092.02 & 1.50 & \bf{1082.70} & 
0.86\\SCA8-9 & \bf{\underline{1067.42}} & 1.22 & 
1067.42 & 1.15 & 1081.40 & 
-1.29\\CON3-0 & 624.96 & 1.64 & 
624.96 & 1.62 & \bf{616.50} & 
1.37\\CON3-1 & 557.38 & 1.49 & 
558.22 & 1.47 & \bf{555.60} & 
0.32\\CON3-2 & 524.07 & 1.14 & 
524.79 & 1.10 & \bf{521.40} & 
0.51\\CON3-3 & 592.78 & 1.48 & 
593.49 & 1.56 & \bf{591.20} & 
0.27\\CON3-4 & 589.32 & 1.48 & 
589.32 & 1.40 & \bf{589.30} & 
0.00\\CON3-5 & 575.30 & 1.41 & 
576.15 & 1.43 & \bf{563.70} & 
2.06\\CON3-6 & 504.15 & 1.75 & 
504.98 & 1.78 & \bf{499.20} & 
0.99\\CON3-7 & 578.41 & 1.15 & 
579.84 & 1.22 & \bf{577.50} & 
0.16\\CON3-8 & 524.59 & 1.21 & 
524.59 & 1.21 & \bf{523.10} & 
0.28\\CON3-9 & 588.48 & 1.32 & 
588.48 & 1.33 & \bf{578.20} & 
1.78\\CON8-0 & 879.00 & 1.65 & 
879.00 & 1.50 & \bf{858.90} & 
2.34\\CON8-1 & 758.26 & 1.32 & 
758.26 & 1.33 & \bf{740.90} & 
2.34\\CON8-2 & 716.56 & 1.98 & 
717.21 & 2.02 & \bf{714.30} & 
0.32\\CON8-3 & 817.57 & 1.48 & 
817.57 & 1.47 & \bf{812.30} & 
0.65\\CON8-4 & 781.64 & 1.55 & 
785.81 & 1.58 & \bf{770.10} & 
1.50\\CON8-5 & \bf{\underline{764.36}} & 1.43 & 
764.36 & 1.39 & 766.60 & 
-0.29\\CON8-6 & 705.61 & 1.79 & 
707.08 & 1.74 & \bf{697.20} & 
1.21\\CON8-7 & 822.42 & 1.21 & 
822.67 & 1.20 & \bf{814.80} & 
0.94\\CON8-8 & 799.32 & 1.45 & 
799.46 & 1.52 & \bf{771.30} & 
3.63\\CON8-9 & 816.12 & 1.56 & 
817.40 & 1.58 & \bf{815.10} & 
0.13\\[1ex]\hline
\end{tabular}
\label{table:nonlin}
\end{table} \clearpage
\begin{table}[ht]
\caption{Resultados de la ejecución de la metaheurística ACO, utilizando instancias de Dethloff con la configuración -n 2.0 -alpha 1.0 -beta 3.0 -q 8.4 -ro 0.015}
\centering
\small
\begin{tabular}{c c c c c c c}
\hline\hline
Instancia & Costo mínimo & Tiempo(seg.) & Costo promedio & Tiempo promedio(seg.) & Costo ACO & \%Gap \\ [0.5ex]
\hline
SCA3-0 & 640.55 & 1.30 & 
640.55 & 1.39 & \bf{636.10} & 
0.70\\SCA3-1 & \bf{\underline{697.84}} & 1.55 & 
698.76 & 1.50 & 700.10 & 
-0.32\\SCA3-2 & 664.18 & 1.35 & 
664.18 & 1.35 & \bf{659.30} & 
0.74\\SCA3-3 & 680.60 & 1.55 & 
681.13 & 1.51 & \bf{680.00} & 
0.09\\SCA3-4 & \bf{690.50} & 1.46 & 
690.50 & 1.43 & 690.50 & 0.00\\
SCA3-5 & \bf{\underline{665.04}} & 1.43 & 
665.19 & 1.45 & 671.10 & 
-0.90\\SCA3-6 & 655.19 & 1.30 & 
655.19 & 1.30 & \bf{651.10} & 
0.63\\SCA3-7 & 666.15 & 1.00 & 
666.15 & 1.01 & \bf{666.10} & 
0.01\\SCA3-8 & 721.45 & 1.14 & 
724.59 & 1.17 & \bf{719.50} & 
0.27\\SCA3-9 & \bf{681.00} & 0.96 & 
681.00 & 0.95 & 681.00 & 0.00\\
SCA8-0 & 991.07 & 1.56 & 
991.65 & 1.54 & \bf{961.60} & 
3.06\\SCA8-1 & 1074.65 & 1.13 & 
1074.65 & 1.18 & \bf{1063.00} & 
1.10\\SCA8-2 & 1056.87 & 1.04 & 
1056.87 & 1.02 & \bf{1040.60} & 
1.56\\SCA8-3 & 1031.08 & 1.47 & 
1031.08 & 1.46 & \bf{985.90} & 
4.58\\SCA8-4 & 1099.06 & 1.42 & 
1099.06 & 1.50 & \bf{1071.00} & 
2.62\\SCA8-5 & 1055.35 & 1.67 & 
1055.35 & 1.91 & \bf{1054.30} & 
0.10\\SCA8-6 & \bf{\underline{972.48}} & 1.71 & 
975.00 & 1.69 & 972.50 & 
-0.00\\SCA8-7 & 1092.57 & 1.61 & 
1092.57 & 1.61 & \bf{1059.70} & 
3.10\\SCA8-8 & 1091.49 & 1.46 & 
1091.89 & 1.44 & \bf{1082.70} & 
0.81\\SCA8-9 & \bf{\underline{1067.42}} & 1.07 & 
1067.42 & 1.15 & 1081.40 & 
-1.29\\CON3-0 & 624.96 & 1.62 & 
624.96 & 1.61 & \bf{616.50} & 
1.37\\CON3-1 & 557.38 & 1.41 & 
557.58 & 1.46 & \bf{555.60} & 
0.32\\CON3-2 & 524.07 & 1.13 & 
524.35 & 1.08 & \bf{521.40} & 
0.51\\CON3-3 & 594.11 & 1.46 & 
594.11 & 1.48 & \bf{591.20} & 
0.49\\CON3-4 & \bf{\underline{588.79}} & 1.44 & 
589.19 & 1.31 & 589.30 & 
-0.09\\CON3-5 & 576.43 & 1.38 & 
576.43 & 1.53 & \bf{563.70} & 
2.26\\CON3-6 & 505.26 & 1.71 & 
505.26 & 1.79 & \bf{499.20} & 
1.21\\CON3-7 & 578.41 & 1.30 & 
578.41 & 1.22 & \bf{577.50} & 
0.16\\CON3-8 & 524.59 & 1.24 & 
524.59 & 1.24 & \bf{523.10} & 
0.28\\CON3-9 & 588.48 & 1.24 & 
588.48 & 1.28 & \bf{578.20} & 
1.78\\CON8-0 & 879.00 & 1.46 & 
879.00 & 1.46 & \bf{858.90} & 
2.34\\CON8-1 & 758.26 & 1.43 & 
758.26 & 1.35 & \bf{740.90} & 
2.34\\CON8-2 & 716.56 & 2.06 & 
716.56 & 2.02 & \bf{714.30} & 
0.32\\CON8-3 & 817.57 & 1.53 & 
817.57 & 1.48 & \bf{812.30} & 
0.65\\CON8-4 & 778.60 & 1.54 & 
785.05 & 1.56 & \bf{770.10} & 
1.10\\CON8-5 & \bf{\underline{764.36}} & 1.39 & 
764.36 & 1.40 & 766.60 & 
-0.29\\CON8-6 & 705.61 & 1.90 & 
706.75 & 1.73 & \bf{697.20} & 
1.21\\CON8-7 & 822.42 & 1.26 & 
823.18 & 1.22 & \bf{814.80} & 
0.94\\CON8-8 & 799.32 & 1.51 & 
799.37 & 1.55 & \bf{771.30} & 
3.63\\CON8-9 & 816.12 & 1.48 & 
816.12 & 1.55 & \bf{815.10} & 
0.13\\[1ex]\hline
\end{tabular}
\label{table:nonlin}
\end{table} \clearpage
\begin{table}[ht]
\caption{Resultados de la ejecución de la metaheurística ACO, utilizando instancias de Dethloff con la configuración -n 2.0 -alpha 1.0 -beta 3.0 -q 8.5 -ro 0.015}
\centering
\small
\begin{tabular}{c c c c c c c}
\hline\hline
Instancia & Costo mínimo & Tiempo(seg.) & Costo promedio & Tiempo promedio(seg.) & Costo ACO & \%Gap \\ [0.5ex]
\hline
SCA3-0 & 640.55 & 1.46 & 
640.55 & 1.48 & \bf{636.10} & 
0.70\\SCA3-1 & \bf{\underline{697.84}} & 1.58 & 
697.84 & 1.51 & 700.10 & 
-0.32\\SCA3-2 & 659.34 & 1.47 & 
662.97 & 1.37 & \bf{659.30} & 
0.01\\SCA3-3 & 680.60 & 1.45 & 
680.92 & 1.47 & \bf{680.00} & 
0.09\\SCA3-4 & \bf{690.50} & 1.36 & 
690.50 & 1.39 & 690.50 & 0.00\\
SCA3-5 & \bf{\underline{665.04}} & 1.36 & 
665.04 & 1.47 & 671.10 & 
-0.90\\SCA3-6 & 655.19 & 1.34 & 
655.19 & 1.34 & \bf{651.10} & 
0.63\\SCA3-7 & 666.15 & 0.97 & 
666.15 & 0.99 & \bf{666.10} & 
0.01\\SCA3-8 & 726.44 & 1.13 & 
726.69 & 1.21 & \bf{719.50} & 
0.96\\SCA3-9 & \bf{681.00} & 0.98 & 
681.00 & 0.98 & 681.00 & 0.00\\
SCA8-0 & 991.07 & 1.60 & 
992.23 & 1.59 & \bf{961.60} & 
3.06\\SCA8-1 & 1074.65 & 1.30 & 
1074.65 & 1.20 & \bf{1063.00} & 
1.10\\SCA8-2 & 1056.87 & 1.03 & 
1056.87 & 1.03 & \bf{1040.60} & 
1.56\\SCA8-3 & 1031.08 & 1.45 & 
1031.08 & 1.48 & \bf{985.90} & 
4.58\\SCA8-4 & 1099.06 & 1.42 & 
1099.06 & 1.50 & \bf{1071.00} & 
2.62\\SCA8-5 & 1055.35 & 1.70 & 
1055.35 & 1.70 & \bf{1054.30} & 
0.10\\SCA8-6 & \bf{\underline{972.48}} & 1.70 & 
972.48 & 1.68 & 972.50 & 
-0.00\\SCA8-7 & 1092.57 & 1.72 & 
1092.57 & 1.66 & \bf{1059.70} & 
3.10\\SCA8-8 & 1092.02 & 1.47 & 
1092.02 & 1.45 & \bf{1082.70} & 
0.86\\SCA8-9 & \bf{\underline{1067.42}} & 1.14 & 
1067.42 & 1.13 & 1081.40 & 
-1.29\\CON3-0 & 624.96 & 1.64 & 
624.96 & 1.66 & \bf{616.50} & 
1.37\\CON3-1 & 557.21 & 1.53 & 
557.53 & 1.49 & \bf{555.60} & 
0.29\\CON3-2 & 524.07 & 1.10 & 
524.79 & 1.16 & \bf{521.40} & 
0.51\\CON3-3 & \bf{591.20} & 1.49 & 
593.09 & 1.63 & 591.20 & 0.00\\
CON3-4 & \bf{\underline{588.79}} & 1.28 & 
589.19 & 1.34 & 589.30 & 
-0.09\\CON3-5 & 574.57 & 1.40 & 
575.97 & 1.41 & \bf{563.70} & 
1.93\\CON3-6 & 504.15 & 1.80 & 
507.33 & 1.85 & \bf{499.20} & 
0.99\\CON3-7 & 578.41 & 1.27 & 
579.12 & 1.28 & \bf{577.50} & 
0.16\\CON3-8 & 524.59 & 1.11 & 
524.59 & 1.21 & \bf{523.10} & 
0.28\\CON3-9 & 588.48 & 1.38 & 
588.48 & 1.37 & \bf{578.20} & 
1.78\\CON8-0 & 879.00 & 1.53 & 
879.00 & 1.48 & \bf{858.90} & 
2.34\\CON8-1 & 758.26 & 1.47 & 
758.26 & 1.40 & \bf{740.90} & 
2.34\\CON8-2 & 716.56 & 2.06 & 
717.21 & 2.07 & \bf{714.30} & 
0.32\\CON8-3 & 817.57 & 1.50 & 
817.57 & 1.45 & \bf{812.30} & 
0.65\\CON8-4 & 789.98 & 1.62 & 
792.31 & 1.60 & \bf{770.10} & 
2.58\\CON8-5 & \bf{\underline{764.36}} & 1.41 & 
764.36 & 1.38 & 766.60 & 
-0.29\\CON8-6 & 701.31 & 1.76 & 
705.43 & 1.72 & \bf{697.20} & 
0.59\\CON8-7 & 822.42 & 1.32 & 
822.92 & 1.23 & \bf{814.80} & 
0.94\\CON8-8 & 796.81 & 1.54 & 
798.79 & 1.58 & \bf{771.30} & 
3.31\\CON8-9 & 816.12 & 1.64 & 
818.68 & 1.62 & \bf{815.10} & 
0.13\\[1ex]\hline
\end{tabular}
\label{table:nonlin}
\end{table} \clearpage
\begin{table}[ht]
\caption{Resultados de la ejecución de la metaheurística ACO, utilizando instancias de Dethloff con la configuración -n 2.0 -alpha 1.0 -beta 3.0 -q 8.6 -ro 0.015}
\centering
\small
\begin{tabular}{c c c c c c c}
\hline\hline
Instancia & Costo mínimo & Tiempo(seg.) & Costo promedio & Tiempo promedio(seg.) & Costo ACO & \%Gap \\ [0.5ex]
\hline
SCA3-0 & 640.55 & 1.42 & 
640.55 & 1.44 & \bf{636.10} & 
0.70\\SCA3-1 & \bf{\underline{697.84}} & 1.54 & 
697.84 & 1.55 & 700.10 & 
-0.32\\SCA3-2 & 659.34 & 1.32 & 
662.97 & 1.33 & \bf{659.30} & 
0.01\\SCA3-3 & 680.60 & 1.48 & 
680.78 & 1.49 & \bf{680.00} & 
0.09\\SCA3-4 & \bf{690.50} & 1.44 & 
690.50 & 1.44 & 690.50 & 0.00\\
SCA3-5 & \bf{\underline{665.04}} & 1.52 & 
668.75 & 1.43 & 671.10 & 
-0.90\\SCA3-6 & 655.19 & 1.50 & 
655.19 & 1.43 & \bf{651.10} & 
0.63\\SCA3-7 & 666.15 & 1.01 & 
666.15 & 1.03 & \bf{666.10} & 
0.01\\SCA3-8 & 721.45 & 1.13 & 
725.24 & 1.12 & \bf{719.50} & 
0.27\\SCA3-9 & \bf{681.00} & 0.94 & 
681.00 & 0.99 & 681.00 & 0.00\\
SCA8-0 & 991.07 & 1.57 & 
991.65 & 1.59 & \bf{961.60} & 
3.06\\SCA8-1 & 1074.39 & 1.15 & 
1074.59 & 1.20 & \bf{1063.00} & 
1.07\\SCA8-2 & 1056.87 & 1.10 & 
1056.87 & 1.08 & \bf{1040.60} & 
1.56\\SCA8-3 & 1031.08 & 1.50 & 
1031.08 & 1.49 & \bf{985.90} & 
4.58\\SCA8-4 & 1099.06 & 1.48 & 
1099.06 & 1.53 & \bf{1071.00} & 
2.62\\SCA8-5 & 1055.35 & 1.68 & 
1055.35 & 1.76 & \bf{1054.30} & 
0.10\\SCA8-6 & \bf{\underline{972.48}} & 1.70 & 
972.48 & 1.74 & 972.50 & 
-0.00\\SCA8-7 & 1092.57 & 1.67 & 
1092.57 & 1.66 & \bf{1059.70} & 
3.10\\SCA8-8 & 1091.49 & 1.42 & 
1091.89 & 1.47 & \bf{1082.70} & 
0.81\\SCA8-9 & \bf{\underline{1067.42}} & 1.10 & 
1067.42 & 1.20 & 1081.40 & 
-1.29\\CON3-0 & 624.96 & 1.70 & 
624.96 & 1.73 & \bf{616.50} & 
1.37\\CON3-1 & 557.38 & 1.58 & 
559.14 & 1.52 & \bf{555.60} & 
0.32\\CON3-2 & 524.07 & 1.12 & 
524.75 & 1.16 & \bf{521.40} & 
0.51\\CON3-3 & 594.11 & 1.55 & 
594.11 & 1.60 & \bf{591.20} & 
0.49\\CON3-4 & \bf{\underline{588.79}} & 1.32 & 
589.19 & 1.40 & 589.30 & 
-0.09\\CON3-5 & 576.43 & 1.45 & 
576.43 & 1.44 & \bf{563.70} & 
2.26\\CON3-6 & 504.15 & 1.84 & 
505.50 & 1.87 & \bf{499.20} & 
0.99\\CON3-7 & 578.41 & 1.24 & 
579.84 & 1.26 & \bf{577.50} & 
0.16\\CON3-8 & 524.30 & 1.18 & 
524.45 & 1.22 & \bf{523.10} & 
0.23\\CON3-9 & 588.48 & 1.26 & 
588.48 & 1.29 & \bf{578.20} & 
1.78\\CON8-0 & 879.00 & 1.52 & 
879.00 & 1.49 & \bf{858.90} & 
2.34\\CON8-1 & 758.26 & 1.40 & 
758.26 & 1.38 & \bf{740.90} & 
2.34\\CON8-2 & 716.53 & 2.01 & 
716.53 & 2.03 & \bf{714.30} & 
0.31\\CON8-3 & 817.57 & 1.41 & 
817.57 & 1.41 & \bf{812.30} & 
0.65\\CON8-4 & 789.98 & 1.48 & 
790.76 & 1.53 & \bf{770.10} & 
2.58\\CON8-5 & \bf{\underline{764.36}} & 1.41 & 
764.36 & 1.40 & 766.60 & 
-0.29\\CON8-6 & \bf{\underline{693.83}} & 1.74 & 
703.26 & 1.76 & 697.20 & 
-0.48\\CON8-7 & 822.42 & 1.22 & 
822.78 & 1.22 & \bf{814.80} & 
0.94\\CON8-8 & 799.32 & 1.55 & 
799.41 & 1.56 & \bf{771.30} & 
3.63\\CON8-9 & 816.12 & 1.61 & 
816.12 & 1.60 & \bf{815.10} & 
0.13\\[1ex]\hline
\end{tabular}
\label{table:nonlin}
\end{table} \clearpage
\begin{table}[ht]
\caption{Resultados de la ejecución de la metaheurística ACO, utilizando instancias de Dethloff con la configuración -n 2.0 -alpha 1.0 -beta 3.0 -q 8.7 -ro 0.015}
\centering
\small
\begin{tabular}{c c c c c c c}
\hline\hline
Instancia & Costo mínimo & Tiempo(seg.) & Costo promedio & Tiempo promedio(seg.) & Costo ACO & \%Gap \\ [0.5ex]
\hline
SCA3-0 & 640.55 & 1.40 & 
640.55 & 1.43 & \bf{636.10} & 
0.70\\SCA3-1 & \bf{\underline{697.84}} & 1.60 & 
697.84 & 1.56 & 700.10 & 
-0.32\\SCA3-2 & 664.18 & 1.34 & 
664.18 & 1.38 & \bf{659.30} & 
0.74\\SCA3-3 & 680.60 & 1.48 & 
680.60 & 1.51 & \bf{680.00} & 
0.09\\SCA3-4 & \bf{690.50} & 1.44 & 
690.50 & 1.52 & 690.50 & 0.00\\
SCA3-5 & \bf{\underline{665.04}} & 1.50 & 
665.04 & 1.45 & 671.10 & 
-0.90\\SCA3-6 & 655.19 & 1.37 & 
655.19 & 1.42 & \bf{651.10} & 
0.63\\SCA3-7 & 666.15 & 1.01 & 
666.15 & 1.02 & \bf{666.10} & 
0.01\\SCA3-8 & 721.45 & 1.14 & 
727.46 & 1.20 & \bf{719.50} & 
0.27\\SCA3-9 & \bf{681.00} & 0.95 & 
681.00 & 0.99 & 681.00 & 0.00\\
SCA8-0 & 991.07 & 1.58 & 
991.07 & 1.58 & \bf{961.60} & 
3.06\\SCA8-1 & \bf{\underline{1059.89}} & 1.22 & 
1070.96 & 1.24 & 1063.00 & 
-0.29\\SCA8-2 & 1056.87 & 1.08 & 
1056.87 & 1.08 & \bf{1040.60} & 
1.56\\SCA8-3 & 1031.08 & 1.46 & 
1031.08 & 1.48 & \bf{985.90} & 
4.58\\SCA8-4 & 1099.06 & 1.52 & 
1099.06 & 1.51 & \bf{1071.00} & 
2.62\\SCA8-5 & 1055.35 & 1.65 & 
1055.35 & 1.68 & \bf{1054.30} & 
0.10\\SCA8-6 & \bf{\underline{972.48}} & 1.77 & 
972.48 & 1.74 & 972.50 & 
-0.00\\SCA8-7 & 1092.57 & 1.65 & 
1092.57 & 1.69 & \bf{1059.70} & 
3.10\\SCA8-8 & \bf{\underline{1071.18}} & 1.41 & 
1086.81 & 1.48 & 1082.70 & 
-1.06\\SCA8-9 & \bf{\underline{1067.42}} & 1.19 & 
1067.42 & 1.19 & 1081.40 & 
-1.29\\CON3-0 & 624.96 & 1.62 & 
624.96 & 1.64 & \bf{616.50} & 
1.37\\CON3-1 & 557.38 & 1.48 & 
558.42 & 1.51 & \bf{555.60} & 
0.32\\CON3-2 & 524.07 & 1.21 & 
524.07 & 1.15 & \bf{521.40} & 
0.51\\CON3-3 & \bf{591.20} & 1.55 & 
593.38 & 1.55 & 591.20 & 0.00\\
CON3-4 & 589.32 & 1.26 & 
589.32 & 1.38 & \bf{589.30} & 
0.00\\CON3-5 & 576.43 & 1.46 & 
576.43 & 1.48 & \bf{563.70} & 
2.26\\CON3-6 & 505.26 & 1.90 & 
505.26 & 1.83 & \bf{499.20} & 
1.21\\CON3-7 & 578.41 & 1.23 & 
578.41 & 1.31 & \bf{577.50} & 
0.16\\CON3-8 & 524.30 & 1.22 & 
524.45 & 1.22 & \bf{523.10} & 
0.23\\CON3-9 & 588.48 & 1.24 & 
588.48 & 1.31 & \bf{578.20} & 
1.78\\CON8-0 & 879.00 & 1.51 & 
879.00 & 1.50 & \bf{858.90} & 
2.34\\CON8-1 & 758.26 & 1.37 & 
758.26 & 1.35 & \bf{740.90} & 
2.34\\CON8-2 & 716.53 & 2.04 & 
716.54 & 2.08 & \bf{714.30} & 
0.31\\CON8-3 & 817.57 & 1.52 & 
817.57 & 1.48 & \bf{812.30} & 
0.65\\CON8-4 & 778.60 & 2.17 & 
782.97 & 1.76 & \bf{770.10} & 
1.10\\CON8-5 & \bf{\underline{764.36}} & 1.42 & 
764.36 & 1.39 & 766.60 & 
-0.29\\CON8-6 & \bf{\underline{693.83}} & 1.83 & 
699.72 & 1.75 & 697.20 & 
-0.48\\CON8-7 & 822.42 & 1.28 & 
822.67 & 1.25 & \bf{814.80} & 
0.94\\CON8-8 & 799.32 & 1.58 & 
799.41 & 1.57 & \bf{771.30} & 
3.63\\CON8-9 & 816.12 & 1.69 & 
816.12 & 1.61 & \bf{815.10} & 
0.13\\[1ex]\hline
\end{tabular}
\label{table:nonlin}
\end{table} \clearpage
\begin{table}[ht]
\caption{Resultados de la ejecución de la metaheurística ACO, utilizando instancias de Dethloff con la configuración -n 2.0 -alpha 1.0 -beta 3.0 -q 8.8 -ro 0.015}
\centering
\small
\begin{tabular}{c c c c c c c}
\hline\hline
Instancia & Costo mínimo & Tiempo(seg.) & Costo promedio & Tiempo promedio(seg.) & Costo ACO & \%Gap \\ [0.5ex]
\hline
SCA3-0 & 640.55 & 1.34 & 
640.55 & 1.41 & \bf{636.10} & 
0.70\\SCA3-1 & \bf{\underline{697.84}} & 1.50 & 
697.84 & 1.54 & 700.10 & 
-0.32\\SCA3-2 & 664.18 & 1.32 & 
664.18 & 1.35 & \bf{659.30} & 
0.74\\SCA3-3 & 680.60 & 1.48 & 
680.78 & 1.50 & \bf{680.00} & 
0.09\\SCA3-4 & \bf{690.50} & 1.48 & 
690.50 & 1.49 & 690.50 & 0.00\\
SCA3-5 & \bf{\underline{665.04}} & 1.49 & 
669.23 & 1.46 & 671.10 & 
-0.90\\SCA3-6 & 655.19 & 1.33 & 
655.19 & 1.39 & \bf{651.10} & 
0.63\\SCA3-7 & 666.15 & 1.03 & 
666.15 & 1.04 & \bf{666.10} & 
0.01\\SCA3-8 & 721.45 & 1.11 & 
727.22 & 1.15 & \bf{719.50} & 
0.27\\SCA3-9 & \bf{681.00} & 1.00 & 
681.00 & 0.99 & 681.00 & 0.00\\
SCA8-0 & 991.07 & 1.52 & 
991.07 & 1.57 & \bf{961.60} & 
3.06\\SCA8-1 & 1069.40 & 1.20 & 
1073.34 & 1.23 & \bf{1063.00} & 
0.60\\SCA8-2 & 1056.87 & 1.05 & 
1056.87 & 1.06 & \bf{1040.60} & 
1.56\\SCA8-3 & 1031.08 & 1.51 & 
1031.08 & 1.51 & \bf{985.90} & 
4.58\\SCA8-4 & 1099.06 & 1.61 & 
1099.06 & 1.51 & \bf{1071.00} & 
2.62\\SCA8-5 & 1055.35 & 1.78 & 
1055.35 & 1.72 & \bf{1054.30} & 
0.10\\SCA8-6 & \bf{\underline{972.48}} & 1.64 & 
974.39 & 1.64 & 972.50 & 
-0.00\\SCA8-7 & 1092.57 & 1.76 & 
1092.57 & 1.66 & \bf{1059.70} & 
3.10\\SCA8-8 & 1091.49 & 1.45 & 
1091.62 & 1.47 & \bf{1082.70} & 
0.81\\SCA8-9 & \bf{\underline{1067.42}} & 1.11 & 
1067.42 & 1.17 & 1081.40 & 
-1.29\\CON3-0 & 624.96 & 1.60 & 
624.96 & 1.66 & \bf{616.50} & 
1.37\\CON3-1 & 557.38 & 1.51 & 
558.22 & 1.51 & \bf{555.60} & 
0.32\\CON3-2 & 524.07 & 1.10 & 
525.77 & 1.10 & \bf{521.40} & 
0.51\\CON3-3 & 592.95 & 1.40 & 
593.82 & 1.52 & \bf{591.20} & 
0.30\\CON3-4 & 589.32 & 1.25 & 
589.32 & 1.33 & \bf{589.30} & 
0.00\\CON3-5 & 574.57 & 1.38 & 
575.97 & 1.43 & \bf{563.70} & 
1.93\\CON3-6 & 505.26 & 1.78 & 
505.26 & 1.81 & \bf{499.20} & 
1.21\\CON3-7 & 578.41 & 1.20 & 
579.12 & 1.22 & \bf{577.50} & 
0.16\\CON3-8 & 524.30 & 1.18 & 
524.52 & 1.18 & \bf{523.10} & 
0.23\\CON3-9 & 588.48 & 1.22 & 
588.48 & 1.25 & \bf{578.20} & 
1.78\\CON8-0 & 879.00 & 1.38 & 
879.00 & 1.44 & \bf{858.90} & 
2.34\\CON8-1 & 758.26 & 1.40 & 
758.30 & 1.30 & \bf{740.90} & 
2.34\\CON8-2 & 716.53 & 1.98 & 
716.54 & 2.02 & \bf{714.30} & 
0.31\\CON8-3 & 817.57 & 1.46 & 
817.57 & 1.50 & \bf{812.30} & 
0.65\\CON8-4 & 778.60 & 1.53 & 
783.74 & 1.58 & \bf{770.10} & 
1.10\\CON8-5 & \bf{\underline{764.36}} & 1.68 & 
764.36 & 1.46 & 766.60 & 
-0.29\\CON8-6 & \bf{\underline{693.83}} & 1.68 & 
700.17 & 1.70 & 697.20 & 
-0.48\\CON8-7 & 822.42 & 1.22 & 
822.42 & 1.21 & \bf{814.80} & 
0.94\\CON8-8 & 799.32 & 1.64 & 
799.41 & 1.58 & \bf{771.30} & 
3.63\\CON8-9 & 816.12 & 1.61 & 
816.12 & 1.61 & \bf{815.10} & 
0.13\\[1ex]\hline
\end{tabular}
\label{table:nonlin}
\end{table} \clearpage
\begin{table}[ht]
\caption{Resultados de la ejecución de la metaheurística ACO, utilizando instancias de Dethloff con la configuración -n 2.0 -alpha 1.0 -beta 3.0 -q 8.9 -ro 0.015}
\centering
\small
\begin{tabular}{c c c c c c c}
\hline\hline
Instancia & Costo mínimo & Tiempo(seg.) & Costo promedio & Tiempo promedio(seg.) & Costo ACO & \%Gap \\ [0.5ex]
\hline
SCA3-0 & 640.55 & 1.38 & 
640.55 & 1.38 & \bf{636.10} & 
0.70\\SCA3-1 & \bf{\underline{697.84}} & 1.54 & 
697.84 & 1.52 & 700.10 & 
-0.32\\SCA3-2 & 659.34 & 1.36 & 
662.97 & 1.36 & \bf{659.30} & 
0.01\\SCA3-3 & 680.60 & 1.42 & 
680.96 & 1.47 & \bf{680.00} & 
0.09\\SCA3-4 & \bf{690.50} & 1.36 & 
690.50 & 1.39 & 690.50 & 0.00\\
SCA3-5 & \bf{\underline{665.04}} & 1.56 & 
668.60 & 1.43 & 671.10 & 
-0.90\\SCA3-6 & 655.19 & 1.36 & 
655.30 & 1.36 & \bf{651.10} & 
0.63\\SCA3-7 & 666.15 & 0.96 & 
666.15 & 1.01 & \bf{666.10} & 
0.01\\SCA3-8 & 721.45 & 1.22 & 
722.70 & 1.16 & \bf{719.50} & 
0.27\\SCA3-9 & \bf{681.00} & 0.99 & 
681.00 & 0.96 & 681.00 & 0.00\\
SCA8-0 & 991.07 & 1.53 & 
991.65 & 1.55 & \bf{961.60} & 
3.06\\SCA8-1 & 1074.65 & 1.22 & 
1074.65 & 1.20 & \bf{1063.00} & 
1.10\\SCA8-2 & 1056.87 & 1.02 & 
1056.87 & 1.04 & \bf{1040.60} & 
1.56\\SCA8-3 & 1031.08 & 1.46 & 
1031.08 & 1.43 & \bf{985.90} & 
4.58\\SCA8-4 & 1099.06 & 1.55 & 
1099.06 & 1.46 & \bf{1071.00} & 
2.62\\SCA8-5 & 1055.35 & 1.67 & 
1055.35 & 1.62 & \bf{1054.30} & 
0.10\\SCA8-6 & \bf{\underline{972.48}} & 1.74 & 
975.00 & 1.80 & 972.50 & 
-0.00\\SCA8-7 & 1092.57 & 1.58 & 
1092.57 & 1.59 & \bf{1059.70} & 
3.10\\SCA8-8 & 1091.49 & 1.45 & 
1091.76 & 1.44 & \bf{1082.70} & 
0.81\\SCA8-9 & \bf{\underline{1067.42}} & 1.17 & 
1067.42 & 1.13 & 1081.40 & 
-1.29\\CON3-0 & 624.96 & 1.67 & 
624.96 & 1.65 & \bf{616.50} & 
1.37\\CON3-1 & 557.38 & 1.42 & 
559.14 & 1.43 & \bf{555.60} & 
0.32\\CON3-2 & 524.07 & 1.74 & 
525.23 & 1.24 & \bf{521.40} & 
0.51\\CON3-3 & 594.11 & 1.46 & 
594.11 & 1.50 & \bf{591.20} & 
0.49\\CON3-4 & 589.32 & 1.39 & 
589.32 & 1.38 & \bf{589.30} & 
0.00\\CON3-5 & 576.43 & 1.34 & 
576.43 & 1.37 & \bf{563.70} & 
2.26\\CON3-6 & 502.16 & 1.80 & 
504.21 & 1.79 & \bf{499.20} & 
0.59\\CON3-7 & 578.41 & 1.27 & 
578.41 & 1.24 & \bf{577.50} & 
0.16\\CON3-8 & 524.30 & 1.22 & 
524.45 & 1.21 & \bf{523.10} & 
0.23\\CON3-9 & 588.48 & 1.23 & 
589.00 & 1.28 & \bf{578.20} & 
1.78\\CON8-0 & 879.00 & 1.48 & 
879.00 & 1.44 & \bf{858.90} & 
2.34\\CON8-1 & 758.26 & 1.30 & 
758.26 & 1.35 & \bf{740.90} & 
2.34\\CON8-2 & 716.53 & 1.99 & 
716.55 & 2.00 & \bf{714.30} & 
0.31\\CON8-3 & 817.57 & 1.42 & 
817.57 & 1.43 & \bf{812.30} & 
0.65\\CON8-4 & 778.60 & 1.61 & 
785.83 & 1.63 & \bf{770.10} & 
1.10\\CON8-5 & \bf{\underline{764.36}} & 1.41 & 
764.36 & 1.41 & 766.60 & 
-0.29\\CON8-6 & 705.61 & 1.70 & 
707.08 & 1.71 & \bf{697.20} & 
1.21\\CON8-7 & 822.42 & 1.23 & 
822.42 & 1.21 & \bf{814.80} & 
0.94\\CON8-8 & 792.72 & 1.54 & 
797.73 & 1.58 & \bf{771.30} & 
2.78\\CON8-9 & 816.12 & 1.51 & 
816.12 & 1.54 & \bf{815.10} & 
0.13\\[1ex]\hline
\end{tabular}
\label{table:nonlin}
\end{table} \clearpage
\begin{table}[ht]
\caption{Resultados de la ejecución de la metaheurística ACO, utilizando instancias de Dethloff con la configuración -n 2.0 -alpha 1.0 -beta 3.0 -q 9.0 -ro 0.015}
\centering
\small
\begin{tabular}{c c c c c c c}
\hline\hline
Instancia & Costo mínimo & Tiempo(seg.) & Costo promedio & Tiempo promedio(seg.) & Costo ACO & \%Gap \\ [0.5ex]
\hline
SCA3-0 & 640.55 & 1.37 & 
640.55 & 1.38 & \bf{636.10} & 
0.70\\SCA3-1 & \bf{\underline{697.84}} & 1.48 & 
697.84 & 1.53 & 700.10 & 
-0.32\\SCA3-2 & 659.34 & 1.35 & 
662.97 & 1.34 & \bf{659.30} & 
0.01\\SCA3-3 & 680.60 & 1.42 & 
681.13 & 1.52 & \bf{680.00} & 
0.09\\SCA3-4 & \bf{690.50} & 1.35 & 
690.50 & 1.40 & 690.50 & 0.00\\
SCA3-5 & \bf{\underline{665.04}} & 1.43 & 
665.19 & 1.38 & 671.10 & 
-0.90\\SCA3-6 & 655.19 & 1.30 & 
655.30 & 1.32 & \bf{651.10} & 
0.63\\SCA3-7 & 666.15 & 0.97 & 
666.15 & 0.96 & \bf{666.10} & 
0.01\\SCA3-8 & 721.45 & 1.22 & 
723.34 & 1.17 & \bf{719.50} & 
0.27\\SCA3-9 & \bf{681.00} & 0.96 & 
681.00 & 1.01 & 681.00 & 0.00\\
SCA8-0 & 991.07 & 1.64 & 
991.65 & 1.55 & \bf{961.60} & 
3.06\\SCA8-1 & 1074.65 & 1.18 & 
1074.68 & 1.19 & \bf{1063.00} & 
1.10\\SCA8-2 & 1056.87 & 0.98 & 
1056.87 & 1.01 & \bf{1040.60} & 
1.56\\SCA8-3 & 1031.08 & 1.45 & 
1031.08 & 1.43 & \bf{985.90} & 
4.58\\SCA8-4 & 1099.06 & 1.50 & 
1099.06 & 1.48 & \bf{1071.00} & 
2.62\\SCA8-5 & 1055.35 & 1.56 & 
1055.35 & 1.64 & \bf{1054.30} & 
0.10\\SCA8-6 & \bf{\underline{972.48}} & 1.70 & 
972.48 & 1.71 & 972.50 & 
-0.00\\SCA8-7 & 1092.57 & 1.65 & 
1092.57 & 1.62 & \bf{1059.70} & 
3.10\\SCA8-8 & 1092.02 & 1.46 & 
1092.02 & 1.44 & \bf{1082.70} & 
0.86\\SCA8-9 & \bf{\underline{1067.42}} & 1.17 & 
1067.42 & 1.19 & 1081.40 & 
-1.29\\CON3-0 & 624.96 & 1.62 & 
624.96 & 1.61 & \bf{616.50} & 
1.37\\CON3-1 & 557.38 & 1.46 & 
557.38 & 1.48 & \bf{555.60} & 
0.32\\CON3-2 & 524.07 & 1.13 & 
524.80 & 1.12 & \bf{521.40} & 
0.51\\CON3-3 & \bf{591.20} & 1.44 & 
592.37 & 1.48 & 591.20 & 0.00\\
CON3-4 & 589.32 & 1.21 & 
589.32 & 1.27 & \bf{589.30} & 
0.00\\CON3-5 & 569.15 & 1.32 & 
574.61 & 1.49 & \bf{563.70} & 
0.97\\CON3-6 & 505.26 & 1.80 & 
505.26 & 1.83 & \bf{499.20} & 
1.21\\CON3-7 & 578.41 & 1.20 & 
579.12 & 1.24 & \bf{577.50} & 
0.16\\CON3-8 & 524.59 & 1.18 & 
524.59 & 1.22 & \bf{523.10} & 
0.28\\CON3-9 & 588.48 & 1.21 & 
588.48 & 1.28 & \bf{578.20} & 
1.78\\CON8-0 & 879.00 & 1.48 & 
879.00 & 1.46 & \bf{858.90} & 
2.34\\CON8-1 & 758.26 & 1.28 & 
758.26 & 1.33 & \bf{740.90} & 
2.34\\CON8-2 & 716.53 & 1.91 & 
716.53 & 1.99 & \bf{714.30} & 
0.31\\CON8-3 & 817.57 & 1.39 & 
817.57 & 1.41 & \bf{812.30} & 
0.65\\CON8-4 & 781.64 & 1.45 & 
790.23 & 1.49 & \bf{770.10} & 
1.50\\CON8-5 & \bf{\underline{764.36}} & 1.40 & 
764.36 & 1.40 & 766.60 & 
-0.29\\CON8-6 & \bf{\underline{693.83}} & 1.76 & 
699.72 & 1.80 & 697.20 & 
-0.48\\CON8-7 & 822.42 & 1.14 & 
822.92 & 1.18 & \bf{814.80} & 
0.94\\CON8-8 & 799.16 & 1.59 & 
799.42 & 1.56 & \bf{771.30} & 
3.61\\CON8-9 & 816.12 & 1.62 & 
818.68 & 1.55 & \bf{815.10} & 
0.13\\[1ex]\hline
\end{tabular}
\label{table:nonlin}
\end{table} \clearpage
\begin{table}[ht]
\caption{Resultados de la ejecución de la metaheurística ACO, utilizando instancias de Dethloff con la configuración -n 2.0 -alpha 1.0 -beta 3.0 -q 9.1 -ro 0.015}
\centering
\small
\begin{tabular}{c c c c c c c}
\hline\hline
Instancia & Costo mínimo & Tiempo(seg.) & Costo promedio & Tiempo promedio(seg.) & Costo ACO & \%Gap \\ [0.5ex]
\hline
SCA3-0 & 640.55 & 1.41 & 
640.55 & 1.37 & \bf{636.10} & 
0.70\\SCA3-1 & \bf{\underline{697.84}} & 1.42 & 
697.84 & 1.49 & 700.10 & 
-0.32\\SCA3-2 & 664.18 & 1.37 & 
664.18 & 1.31 & \bf{659.30} & 
0.74\\SCA3-3 & 680.60 & 1.40 & 
680.60 & 1.47 & \bf{680.00} & 
0.09\\SCA3-4 & \bf{690.50} & 1.40 & 
690.50 & 1.41 & 690.50 & 0.00\\
SCA3-5 & \bf{\underline{665.04}} & 1.47 & 
665.04 & 1.48 & 671.10 & 
-0.90\\SCA3-6 & 655.19 & 1.29 & 
655.19 & 1.30 & \bf{651.10} & 
0.63\\SCA3-7 & 666.15 & 1.09 & 
666.15 & 1.00 & \bf{666.10} & 
0.01\\SCA3-8 & 721.45 & 1.18 & 
725.24 & 1.12 & \bf{719.50} & 
0.27\\SCA3-9 & \bf{681.00} & 0.96 & 
681.00 & 0.97 & 681.00 & 0.00\\
SCA8-0 & 991.07 & 1.49 & 
991.07 & 1.55 & \bf{961.60} & 
3.06\\SCA8-1 & 1074.65 & 1.22 & 
1074.68 & 1.21 & \bf{1063.00} & 
1.10\\SCA8-2 & 1056.87 & 1.02 & 
1056.87 & 1.03 & \bf{1040.60} & 
1.56\\SCA8-3 & 1031.08 & 1.47 & 
1031.08 & 1.46 & \bf{985.90} & 
4.58\\SCA8-4 & 1099.06 & 1.79 & 
1099.06 & 1.53 & \bf{1071.00} & 
2.62\\SCA8-5 & 1055.35 & 1.68 & 
1055.35 & 1.65 & \bf{1054.30} & 
0.10\\SCA8-6 & \bf{\underline{972.48}} & 1.65 & 
977.39 & 1.64 & 972.50 & 
-0.00\\SCA8-7 & 1092.57 & 1.68 & 
1092.57 & 1.65 & \bf{1059.70} & 
3.10\\SCA8-8 & 1091.49 & 1.45 & 
1091.89 & 1.45 & \bf{1082.70} & 
0.81\\SCA8-9 & \bf{\underline{1067.42}} & 1.12 & 
1067.42 & 1.15 & 1081.40 & 
-1.29\\CON3-0 & 624.96 & 1.59 & 
624.96 & 1.81 & \bf{616.50} & 
1.37\\CON3-1 & 557.38 & 1.52 & 
558.86 & 1.49 & \bf{555.60} & 
0.32\\CON3-2 & 524.07 & 1.08 & 
524.96 & 1.11 & \bf{521.40} & 
0.51\\CON3-3 & 594.11 & 1.52 & 
594.11 & 1.49 & \bf{591.20} & 
0.49\\CON3-4 & 589.32 & 1.36 & 
589.32 & 1.32 & \bf{589.30} & 
0.00\\CON3-5 & 569.88 & 1.52 & 
573.36 & 1.46 & \bf{563.70} & 
1.10\\CON3-6 & 505.26 & 1.82 & 
507.61 & 1.82 & \bf{499.20} & 
1.21\\CON3-7 & 578.41 & 1.20 & 
579.12 & 1.26 & \bf{577.50} & 
0.16\\CON3-8 & 524.59 & 1.24 & 
524.59 & 1.24 & \bf{523.10} & 
0.28\\CON3-9 & 588.48 & 1.25 & 
588.48 & 1.26 & \bf{578.20} & 
1.78\\CON8-0 & 879.00 & 1.50 & 
879.00 & 1.43 & \bf{858.90} & 
2.34\\CON8-1 & 758.26 & 1.42 & 
758.26 & 1.39 & \bf{740.90} & 
2.34\\CON8-2 & 716.53 & 2.07 & 
716.55 & 2.09 & \bf{714.30} & 
0.31\\CON8-3 & 817.57 & 1.40 & 
817.57 & 1.44 & \bf{812.30} & 
0.65\\CON8-4 & 781.64 & 2.32 & 
788.67 & 1.74 & \bf{770.10} & 
1.50\\CON8-5 & \bf{\underline{764.36}} & 1.36 & 
764.36 & 1.37 & 766.60 & 
-0.29\\CON8-6 & \bf{\underline{693.83}} & 1.64 & 
700.62 & 1.70 & 697.20 & 
-0.48\\CON8-7 & 822.84 & 1.26 & 
823.28 & 1.22 & \bf{814.80} & 
0.99\\CON8-8 & 799.32 & 1.76 & 
799.37 & 1.61 & \bf{771.30} & 
3.63\\CON8-9 & 816.12 & 1.62 & 
816.12 & 1.60 & \bf{815.10} & 
0.13\\[1ex]\hline
\end{tabular}
\label{table:nonlin}
\end{table} \clearpage
\begin{table}[ht]
\caption{Resultados de la ejecución de la metaheurística ACO, utilizando instancias de Dethloff con la configuración -n 2.0 -alpha 1.0 -beta 3.0 -q 9.2 -ro 0.015}
\centering
\small
\begin{tabular}{c c c c c c c}
\hline\hline
Instancia & Costo mínimo & Tiempo(seg.) & Costo promedio & Tiempo promedio(seg.) & Costo ACO & \%Gap \\ [0.5ex]
\hline
SCA3-0 & 640.55 & 1.40 & 
640.55 & 1.40 & \bf{636.10} & 
0.70\\SCA3-1 & \bf{\underline{697.84}} & 1.52 & 
698.76 & 1.50 & 700.10 & 
-0.32\\SCA3-2 & 659.34 & 1.31 & 
662.97 & 1.32 & \bf{659.30} & 
0.01\\SCA3-3 & 680.60 & 1.56 & 
680.96 & 1.47 & \bf{680.00} & 
0.09\\SCA3-4 & \bf{690.50} & 1.32 & 
690.50 & 1.47 & 690.50 & 0.00\\
SCA3-5 & \bf{\underline{665.04}} & 1.42 & 
665.04 & 1.45 & 671.10 & 
-0.90\\SCA3-6 & 655.19 & 1.39 & 
655.19 & 1.39 & \bf{651.10} & 
0.63\\SCA3-7 & 666.15 & 1.02 & 
666.15 & 1.05 & \bf{666.10} & 
0.01\\SCA3-8 & 721.45 & 1.12 & 
723.34 & 1.11 & \bf{719.50} & 
0.27\\SCA3-9 & \bf{681.00} & 1.00 & 
681.00 & 1.00 & 681.00 & 0.00\\
SCA8-0 & 991.07 & 1.52 & 
992.80 & 1.61 & \bf{961.60} & 
3.06\\SCA8-1 & 1074.65 & 1.18 & 
1074.65 & 1.16 & \bf{1063.00} & 
1.10\\SCA8-2 & 1056.87 & 1.02 & 
1056.87 & 1.03 & \bf{1040.60} & 
1.56\\SCA8-3 & 1031.08 & 1.46 & 
1031.08 & 1.45 & \bf{985.90} & 
4.58\\SCA8-4 & 1098.34 & 1.51 & 
1098.52 & 1.55 & \bf{1071.00} & 
2.55\\SCA8-5 & 1055.35 & 1.76 & 
1055.35 & 1.71 & \bf{1054.30} & 
0.10\\SCA8-6 & \bf{\underline{972.48}} & 1.68 & 
972.48 & 1.65 & 972.50 & 
-0.00\\SCA8-7 & 1092.57 & 1.65 & 
1092.57 & 1.64 & \bf{1059.70} & 
3.10\\SCA8-8 & 1092.02 & 1.46 & 
1092.02 & 1.43 & \bf{1082.70} & 
0.86\\SCA8-9 & \bf{\underline{1067.42}} & 1.14 & 
1067.42 & 1.13 & 1081.40 & 
-1.29\\CON3-0 & 624.96 & 1.62 & 
624.96 & 1.60 & \bf{616.50} & 
1.37\\CON3-1 & 557.38 & 1.54 & 
557.38 & 1.51 & \bf{555.60} & 
0.32\\CON3-2 & 525.02 & 1.06 & 
525.71 & 1.08 & \bf{521.40} & 
0.69\\CON3-3 & 594.11 & 1.52 & 
594.11 & 1.51 & \bf{591.20} & 
0.49\\CON3-4 & 589.32 & 1.40 & 
589.32 & 1.38 & \bf{589.30} & 
0.00\\CON3-5 & 576.43 & 1.39 & 
576.43 & 1.47 & \bf{563.70} & 
2.26\\CON3-6 & 505.26 & 1.80 & 
505.26 & 1.79 & \bf{499.20} & 
1.21\\CON3-7 & 578.41 & 1.24 & 
578.41 & 1.22 & \bf{577.50} & 
0.16\\CON3-8 & 524.30 & 1.16 & 
524.52 & 1.17 & \bf{523.10} & 
0.23\\CON3-9 & 588.48 & 1.30 & 
590.09 & 1.31 & \bf{578.20} & 
1.78\\CON8-0 & 879.00 & 1.48 & 
879.00 & 1.46 & \bf{858.90} & 
2.34\\CON8-1 & 758.26 & 1.31 & 
758.26 & 1.34 & \bf{740.90} & 
2.34\\CON8-2 & 716.53 & 1.85 & 
717.20 & 1.95 & \bf{714.30} & 
0.31\\CON8-3 & 817.57 & 1.40 & 
817.57 & 1.42 & \bf{812.30} & 
0.65\\CON8-4 & 789.98 & 1.64 & 
790.76 & 1.60 & \bf{770.10} & 
2.58\\CON8-5 & \bf{\underline{764.36}} & 1.44 & 
764.36 & 1.41 & 766.60 & 
-0.29\\CON8-6 & 705.61 & 1.79 & 
706.33 & 1.73 & \bf{697.20} & 
1.21\\CON8-7 & 822.42 & 1.22 & 
822.92 & 1.20 & \bf{814.80} & 
0.94\\CON8-8 & 799.32 & 1.70 & 
799.41 & 1.61 & \bf{771.30} & 
3.63\\CON8-9 & 816.12 & 1.59 & 
816.12 & 1.52 & \bf{815.10} & 
0.13\\[1ex]\hline
\end{tabular}
\label{table:nonlin}
\end{table} \clearpage
\begin{table}[ht]
\caption{Resultados de la ejecución de la metaheurística ACO, utilizando instancias de Dethloff con la configuración -n 2.0 -alpha 1.0 -beta 3.0 -q 9.3 -ro 0.015}
\centering
\small
\begin{tabular}{c c c c c c c}
\hline\hline
Instancia & Costo mínimo & Tiempo(seg.) & Costo promedio & Tiempo promedio(seg.) & Costo ACO & \%Gap \\ [0.5ex]
\hline
SCA3-0 & 640.55 & 1.36 & 
640.55 & 1.39 & \bf{636.10} & 
0.70\\SCA3-1 & \bf{\underline{697.84}} & 1.44 & 
698.76 & 1.48 & 700.10 & 
-0.32\\SCA3-2 & 664.18 & 1.37 & 
664.18 & 1.35 & \bf{659.30} & 
0.74\\SCA3-3 & 680.60 & 1.54 & 
680.78 & 1.48 & \bf{680.00} & 
0.09\\SCA3-4 & \bf{690.50} & 1.36 & 
690.50 & 1.40 & 690.50 & 0.00\\
SCA3-5 & \bf{\underline{665.04}} & 1.49 & 
665.04 & 1.44 & 671.10 & 
-0.90\\SCA3-6 & 655.19 & 1.27 & 
655.19 & 1.33 & \bf{651.10} & 
0.63\\SCA3-7 & 666.15 & 0.98 & 
666.15 & 0.96 & \bf{666.10} & 
0.01\\SCA3-8 & 721.45 & 1.13 & 
725.24 & 1.16 & \bf{719.50} & 
0.27\\SCA3-9 & \bf{681.00} & 0.96 & 
681.00 & 0.97 & 681.00 & 0.00\\
SCA8-0 & 991.07 & 1.52 & 
997.59 & 1.52 & \bf{961.60} & 
3.06\\SCA8-1 & 1069.40 & 1.21 & 
1073.37 & 1.20 & \bf{1063.00} & 
0.60\\SCA8-2 & 1056.87 & 0.94 & 
1056.87 & 0.97 & \bf{1040.60} & 
1.56\\SCA8-3 & 1031.08 & 1.43 & 
1031.08 & 1.46 & \bf{985.90} & 
4.58\\SCA8-4 & 1099.06 & 1.42 & 
1099.06 & 1.50 & \bf{1071.00} & 
2.62\\SCA8-5 & 1055.35 & 1.67 & 
1055.35 & 1.63 & \bf{1054.30} & 
0.10\\SCA8-6 & \bf{\underline{972.48}} & 1.70 & 
972.48 & 1.71 & 972.50 & 
-0.00\\SCA8-7 & 1092.57 & 1.64 & 
1092.57 & 1.58 & \bf{1059.70} & 
3.10\\SCA8-8 & 1091.49 & 1.38 & 
1091.89 & 1.42 & \bf{1082.70} & 
0.81\\SCA8-9 & \bf{\underline{1067.42}} & 1.14 & 
1067.42 & 1.15 & 1081.40 & 
-1.29\\CON3-0 & 624.96 & 1.62 & 
624.96 & 1.64 & \bf{616.50} & 
1.37\\CON3-1 & 557.38 & 1.48 & 
557.77 & 1.47 & \bf{555.60} & 
0.32\\CON3-2 & 524.07 & 1.15 & 
524.51 & 1.12 & \bf{521.40} & 
0.51\\CON3-3 & 594.11 & 1.49 & 
594.11 & 1.49 & \bf{591.20} & 
0.49\\CON3-4 & 589.32 & 1.35 & 
589.32 & 1.31 & \bf{589.30} & 
0.00\\CON3-5 & 569.88 & 1.37 & 
573.36 & 1.45 & \bf{563.70} & 
1.10\\CON3-6 & 505.26 & 1.87 & 
506.43 & 1.82 & \bf{499.20} & 
1.21\\CON3-7 & 578.41 & 1.18 & 
578.41 & 1.20 & \bf{577.50} & 
0.16\\CON3-8 & 524.30 & 1.18 & 
524.52 & 1.24 & \bf{523.10} & 
0.23\\CON3-9 & 588.48 & 1.30 & 
588.48 & 1.27 & \bf{578.20} & 
1.78\\CON8-0 & 879.00 & 1.48 & 
879.00 & 1.49 & \bf{858.90} & 
2.34\\CON8-1 & 758.26 & 1.32 & 
758.26 & 1.33 & \bf{740.90} & 
2.34\\CON8-2 & 716.53 & 2.02 & 
716.54 & 2.02 & \bf{714.30} & 
0.31\\CON8-3 & 817.57 & 1.41 & 
817.57 & 1.42 & \bf{812.30} & 
0.65\\CON8-4 & 781.64 & 1.57 & 
783.73 & 1.60 & \bf{770.10} & 
1.50\\CON8-5 & \bf{\underline{764.36}} & 1.43 & 
764.36 & 1.42 & 766.60 & 
-0.29\\CON8-6 & 705.61 & 1.72 & 
706.63 & 1.70 & \bf{697.20} & 
1.21\\CON8-7 & 822.42 & 1.25 & 
822.85 & 1.20 & \bf{814.80} & 
0.94\\CON8-8 & 799.32 & 1.55 & 
799.46 & 1.55 & \bf{771.30} & 
3.63\\CON8-9 & 816.12 & 1.64 & 
816.12 & 1.67 & \bf{815.10} & 
0.13\\[1ex]\hline
\end{tabular}
\label{table:nonlin}
\end{table} \clearpage
\begin{table}[ht]
\caption{Resultados de la ejecución de la metaheurística ACO, utilizando instancias de Dethloff con la configuración -n 2.0 -alpha 1.0 -beta 3.0 -q 9.4 -ro 0.015}
\centering
\small
\begin{tabular}{c c c c c c c}
\hline\hline
Instancia & Costo mínimo & Tiempo(seg.) & Costo promedio & Tiempo promedio(seg.) & Costo ACO & \%Gap \\ [0.5ex]
\hline
SCA3-0 & 640.55 & 1.44 & 
640.55 & 1.36 & \bf{636.10} & 
0.70\\SCA3-1 & \bf{\underline{697.84}} & 1.48 & 
697.84 & 1.49 & 700.10 & 
-0.32\\SCA3-2 & 659.34 & 1.41 & 
662.97 & 1.35 & \bf{659.30} & 
0.01\\SCA3-3 & 680.60 & 1.42 & 
680.96 & 1.50 & \bf{680.00} & 
0.09\\SCA3-4 & \bf{690.50} & 1.39 & 
690.50 & 1.41 & 690.50 & 0.00\\
SCA3-5 & \bf{\underline{665.04}} & 1.45 & 
665.04 & 1.45 & 671.10 & 
-0.90\\SCA3-6 & 655.19 & 1.34 & 
655.19 & 1.33 & \bf{651.10} & 
0.63\\SCA3-7 & 666.15 & 0.96 & 
666.15 & 1.01 & \bf{666.10} & 
0.01\\SCA3-8 & 721.45 & 1.19 & 
723.95 & 1.13 & \bf{719.50} & 
0.27\\SCA3-9 & \bf{681.00} & 1.01 & 
681.00 & 0.99 & 681.00 & 0.00\\
SCA8-0 & 991.07 & 1.61 & 
991.65 & 1.56 & \bf{961.60} & 
3.06\\SCA8-1 & 1074.65 & 1.22 & 
1074.65 & 1.21 & \bf{1063.00} & 
1.10\\SCA8-2 & 1056.87 & 1.00 & 
1056.87 & 0.99 & \bf{1040.60} & 
1.56\\SCA8-3 & 1031.08 & 1.50 & 
1031.08 & 1.43 & \bf{985.90} & 
4.58\\SCA8-4 & 1099.06 & 1.51 & 
1099.06 & 1.65 & \bf{1071.00} & 
2.62\\SCA8-5 & 1055.35 & 1.64 & 
1055.35 & 1.65 & \bf{1054.30} & 
0.10\\SCA8-6 & \bf{\underline{972.48}} & 1.72 & 
979.23 & 1.69 & 972.50 & 
-0.00\\SCA8-7 & 1092.57 & 1.64 & 
1092.57 & 1.63 & \bf{1059.70} & 
3.10\\SCA8-8 & 1091.49 & 1.47 & 
1091.89 & 1.44 & \bf{1082.70} & 
0.81\\SCA8-9 & \bf{\underline{1067.42}} & 1.25 & 
1067.42 & 1.16 & 1081.40 & 
-1.29\\CON3-0 & 624.96 & 1.74 & 
624.96 & 1.70 & \bf{616.50} & 
1.37\\CON3-1 & 557.38 & 1.38 & 
558.22 & 1.47 & \bf{555.60} & 
0.32\\CON3-2 & 524.07 & 1.01 & 
525.70 & 1.06 & \bf{521.40} & 
0.51\\CON3-3 & 594.11 & 1.54 & 
594.11 & 1.53 & \bf{591.20} & 
0.49\\CON3-4 & 589.32 & 1.28 & 
589.32 & 1.40 & \bf{589.30} & 
0.00\\CON3-5 & 570.70 & 1.38 & 
575.00 & 1.48 & \bf{563.70} & 
1.24\\CON3-6 & 505.26 & 1.74 & 
506.43 & 1.77 & \bf{499.20} & 
1.21\\CON3-7 & 578.41 & 1.16 & 
579.12 & 1.24 & \bf{577.50} & 
0.16\\CON3-8 & 524.59 & 1.24 & 
524.59 & 1.22 & \bf{523.10} & 
0.28\\CON3-9 & 588.48 & 1.29 & 
588.48 & 1.28 & \bf{578.20} & 
1.78\\CON8-0 & 879.00 & 1.48 & 
879.00 & 1.46 & \bf{858.90} & 
2.34\\CON8-1 & 758.26 & 1.33 & 
758.26 & 1.36 & \bf{740.90} & 
2.34\\CON8-2 & 716.53 & 1.92 & 
716.54 & 2.02 & \bf{714.30} & 
0.31\\CON8-3 & 817.57 & 1.42 & 
817.57 & 1.47 & \bf{812.30} & 
0.65\\CON8-4 & 778.60 & 1.61 & 
786.61 & 1.59 & \bf{770.10} & 
1.10\\CON8-5 & \bf{\underline{764.36}} & 1.41 & 
764.36 & 1.40 & 766.60 & 
-0.29\\CON8-6 & 705.61 & 1.65 & 
706.63 & 1.71 & \bf{697.20} & 
1.21\\CON8-7 & 822.42 & 1.17 & 
822.67 & 1.19 & \bf{814.80} & 
0.94\\CON8-8 & 799.32 & 1.55 & 
799.41 & 1.53 & \bf{771.30} & 
3.63\\CON8-9 & 816.12 & 1.49 & 
818.68 & 1.58 & \bf{815.10} & 
0.13\\[1ex]\hline
\end{tabular}
\label{table:nonlin}
\end{table} \clearpage
\begin{table}[ht]
\caption{Resultados de la ejecución de la metaheurística ACO, utilizando instancias de Dethloff con la configuración -n 2.0 -alpha 1.0 -beta 3.0 -q 9.5 -ro 0.015}
\centering
\small
\begin{tabular}{c c c c c c c}
\hline\hline
Instancia & Costo mínimo & Tiempo(seg.) & Costo promedio & Tiempo promedio(seg.) & Costo ACO & \%Gap \\ [0.5ex]
\hline
SCA3-0 & 640.55 & 1.42 & 
640.55 & 1.41 & \bf{636.10} & 
0.70\\SCA3-1 & \bf{\underline{697.84}} & 1.50 & 
697.84 & 1.52 & 700.10 & 
-0.32\\SCA3-2 & 659.34 & 1.45 & 
662.97 & 1.33 & \bf{659.30} & 
0.01\\SCA3-3 & 680.60 & 1.41 & 
680.78 & 1.49 & \bf{680.00} & 
0.09\\SCA3-4 & \bf{690.50} & 1.33 & 
690.50 & 1.40 & 690.50 & 0.00\\
SCA3-5 & \bf{\underline{665.04}} & 1.42 & 
665.04 & 1.45 & 671.10 & 
-0.90\\SCA3-6 & 655.19 & 1.37 & 
655.19 & 1.36 & \bf{651.10} & 
0.63\\SCA3-7 & 666.15 & 1.05 & 
666.15 & 1.08 & \bf{666.10} & 
0.01\\SCA3-8 & 721.45 & 1.11 & 
721.45 & 1.21 & \bf{719.50} & 
0.27\\SCA3-9 & \bf{681.00} & 1.06 & 
681.00 & 1.01 & 681.00 & 0.00\\
SCA8-0 & 991.07 & 1.60 & 
991.07 & 1.55 & \bf{961.60} & 
3.06\\SCA8-1 & 1074.65 & 1.15 & 
1074.65 & 1.17 & \bf{1063.00} & 
1.10\\SCA8-2 & 1056.87 & 1.04 & 
1056.87 & 1.02 & \bf{1040.60} & 
1.56\\SCA8-3 & 1031.08 & 1.46 & 
1031.08 & 1.46 & \bf{985.90} & 
4.58\\SCA8-4 & 1099.06 & 1.49 & 
1099.06 & 1.48 & \bf{1071.00} & 
2.62\\SCA8-5 & 1055.35 & 1.74 & 
1055.35 & 1.66 & \bf{1054.30} & 
0.10\\SCA8-6 & \bf{\underline{972.48}} & 1.69 & 
972.48 & 1.70 & 972.50 & 
-0.00\\SCA8-7 & 1092.57 & 1.58 & 
1092.57 & 1.58 & \bf{1059.70} & 
3.10\\SCA8-8 & 1091.49 & 1.43 & 
1091.75 & 1.47 & \bf{1082.70} & 
0.81\\SCA8-9 & \bf{\underline{1067.42}} & 1.13 & 
1067.42 & 1.11 & 1081.40 & 
-1.29\\CON3-0 & 624.96 & 1.64 & 
624.96 & 1.65 & \bf{616.50} & 
1.37\\CON3-1 & 557.38 & 1.50 & 
557.38 & 1.47 & \bf{555.60} & 
0.32\\CON3-2 & 524.07 & 1.15 & 
525.33 & 1.08 & \bf{521.40} & 
0.51\\CON3-3 & \bf{591.20} & 1.54 & 
591.93 & 1.61 & 591.20 & 0.00\\
CON3-4 & 589.32 & 1.27 & 
589.32 & 1.34 & \bf{589.30} & 
0.00\\CON3-5 & 570.70 & 1.38 & 
574.53 & 1.40 & \bf{563.70} & 
1.24\\CON3-6 & 505.26 & 1.80 & 
506.43 & 1.74 & \bf{499.20} & 
1.21\\CON3-7 & 578.41 & 1.28 & 
579.84 & 1.24 & \bf{577.50} & 
0.16\\CON3-8 & 524.30 & 1.14 & 
526.89 & 1.20 & \bf{523.10} & 
0.23\\CON3-9 & 588.48 & 1.26 & 
588.48 & 1.29 & \bf{578.20} & 
1.78\\CON8-0 & 879.00 & 1.38 & 
879.00 & 1.44 & \bf{858.90} & 
2.34\\CON8-1 & 758.26 & 1.28 & 
758.26 & 1.35 & \bf{740.90} & 
2.34\\CON8-2 & 716.53 & 2.12 & 
716.55 & 2.06 & \bf{714.30} & 
0.31\\CON8-3 & 817.57 & 1.40 & 
817.57 & 1.46 & \bf{812.30} & 
0.65\\CON8-4 & 789.98 & 1.60 & 
790.76 & 1.54 & \bf{770.10} & 
2.58\\CON8-5 & \bf{\underline{764.36}} & 1.28 & 
764.36 & 1.40 & 766.60 & 
-0.29\\CON8-6 & 705.61 & 1.68 & 
707.08 & 1.72 & \bf{697.20} & 
1.21\\CON8-7 & 822.42 & 1.22 & 
822.92 & 1.20 & \bf{814.80} & 
0.94\\CON8-8 & 799.32 & 1.54 & 
799.46 & 1.55 & \bf{771.30} & 
3.63\\CON8-9 & 816.12 & 1.62 & 
817.40 & 1.58 & \bf{815.10} & 
0.13\\[1ex]\hline
\end{tabular}
\label{table:nonlin}
\end{table} \clearpage
\begin{table}[ht]
\caption{Resultados de la ejecución de la metaheurística ACO, utilizando instancias de Dethloff con la configuración -n 2.0 -alpha 1.0 -beta 3.0 -q 9.6 -ro 0.015}
\centering
\small
\begin{tabular}{c c c c c c c}
\hline\hline
Instancia & Costo mínimo & Tiempo(seg.) & Costo promedio & Tiempo promedio(seg.) & Costo ACO & \%Gap \\ [0.5ex]
\hline
SCA3-0 & 640.55 & 1.42 & 
640.55 & 1.43 & \bf{636.10} & 
0.70\\SCA3-1 & \bf{\underline{697.84}} & 1.54 & 
698.76 & 1.50 & 700.10 & 
-0.32\\SCA3-2 & 659.34 & 1.37 & 
662.97 & 1.33 & \bf{659.30} & 
0.01\\SCA3-3 & 681.31 & 1.41 & 
681.31 & 1.45 & \bf{680.00} & 
0.19\\SCA3-4 & \bf{690.50} & 1.45 & 
690.50 & 1.47 & 690.50 & 0.00\\
SCA3-5 & \bf{\underline{665.04}} & 1.42 & 
665.04 & 1.44 & 671.10 & 
-0.90\\SCA3-6 & 655.19 & 1.29 & 
655.19 & 1.32 & \bf{651.10} & 
0.63\\SCA3-7 & 666.15 & 0.98 & 
666.15 & 1.00 & \bf{666.10} & 
0.01\\SCA3-8 & 721.45 & 1.13 & 
723.34 & 1.16 & \bf{719.50} & 
0.27\\SCA3-9 & \bf{681.00} & 0.98 & 
681.00 & 0.96 & 681.00 & 0.00\\
SCA8-0 & 991.07 & 1.46 & 
991.07 & 1.86 & \bf{961.60} & 
3.06\\SCA8-1 & 1074.65 & 1.15 & 
1074.68 & 1.18 & \bf{1063.00} & 
1.10\\SCA8-2 & 1056.87 & 1.01 & 
1056.87 & 1.02 & \bf{1040.60} & 
1.56\\SCA8-3 & 1031.08 & 1.46 & 
1031.08 & 1.46 & \bf{985.90} & 
4.58\\SCA8-4 & 1099.06 & 1.46 & 
1099.17 & 1.50 & \bf{1071.00} & 
2.62\\SCA8-5 & 1055.35 & 1.68 & 
1055.35 & 1.69 & \bf{1054.30} & 
0.10\\SCA8-6 & \bf{\underline{972.48}} & 1.53 & 
972.48 & 1.64 & 972.50 & 
-0.00\\SCA8-7 & 1075.42 & 1.70 & 
1088.28 & 1.64 & \bf{1059.70} & 
1.48\\SCA8-8 & 1091.49 & 1.43 & 
1091.76 & 1.45 & \bf{1082.70} & 
0.81\\SCA8-9 & \bf{\underline{1067.42}} & 1.34 & 
1067.42 & 1.21 & 1081.40 & 
-1.29\\CON3-0 & 624.96 & 1.54 & 
624.96 & 1.62 & \bf{616.50} & 
1.37\\CON3-1 & 557.38 & 1.38 & 
558.22 & 1.45 & \bf{555.60} & 
0.32\\CON3-2 & 524.07 & 1.16 & 
524.36 & 1.11 & \bf{521.40} & 
0.51\\CON3-3 & \bf{591.20} & 1.49 & 
593.38 & 1.50 & 591.20 & 0.00\\
CON3-4 & 589.32 & 1.38 & 
589.32 & 1.29 & \bf{589.30} & 
0.00\\CON3-5 & 569.88 & 1.38 & 
574.79 & 1.40 & \bf{563.70} & 
1.10\\CON3-6 & 505.26 & 1.78 & 
506.43 & 1.77 & \bf{499.20} & 
1.21\\CON3-7 & 578.41 & 1.22 & 
578.41 & 1.24 & \bf{577.50} & 
0.16\\CON3-8 & 524.59 & 1.18 & 
524.59 & 1.17 & \bf{523.10} & 
0.28\\CON3-9 & 588.48 & 1.39 & 
588.48 & 1.29 & \bf{578.20} & 
1.78\\CON8-0 & 879.00 & 1.42 & 
879.00 & 1.42 & \bf{858.90} & 
2.34\\CON8-1 & 758.26 & 1.41 & 
758.26 & 1.36 & \bf{740.90} & 
2.34\\CON8-2 & 716.53 & 1.94 & 
716.54 & 2.06 & \bf{714.30} & 
0.31\\CON8-3 & 817.57 & 1.43 & 
817.57 & 1.46 & \bf{812.30} & 
0.65\\CON8-4 & 781.64 & 1.60 & 
788.67 & 1.57 & \bf{770.10} & 
1.50\\CON8-5 & \bf{\underline{764.36}} & 1.39 & 
764.36 & 1.39 & 766.60 & 
-0.29\\CON8-6 & 705.61 & 1.82 & 
706.51 & 1.75 & \bf{697.20} & 
1.21\\CON8-7 & 822.42 & 1.13 & 
822.75 & 1.17 & \bf{814.80} & 
0.94\\CON8-8 & 799.32 & 1.52 & 
799.41 & 1.52 & \bf{771.30} & 
3.63\\CON8-9 & 816.12 & 1.62 & 
816.12 & 1.62 & \bf{815.10} & 
0.13\\[1ex]\hline
\end{tabular}
\label{table:nonlin}
\end{table} \clearpage
\begin{table}[ht]
\caption{Resultados de la ejecución de la metaheurística ACO, utilizando instancias de Dethloff con la configuración -n 2.0 -alpha 1.0 -beta 3.0 -q 9.7 -ro 0.015}
\centering
\small
\begin{tabular}{c c c c c c c}
\hline\hline
Instancia & Costo mínimo & Tiempo(seg.) & Costo promedio & Tiempo promedio(seg.) & Costo ACO & \%Gap \\ [0.5ex]
\hline
SCA3-0 & 640.55 & 1.45 & 
640.55 & 1.37 & \bf{636.10} & 
0.70\\SCA3-1 & \bf{\underline{697.84}} & 1.50 & 
697.84 & 1.51 & 700.10 & 
-0.32\\SCA3-2 & 664.18 & 1.38 & 
664.18 & 1.33 & \bf{659.30} & 
0.74\\SCA3-3 & 680.60 & 1.45 & 
680.78 & 1.48 & \bf{680.00} & 
0.09\\SCA3-4 & \bf{690.50} & 1.43 & 
690.50 & 1.43 & 690.50 & 0.00\\
SCA3-5 & \bf{\underline{665.04}} & 1.43 & 
668.60 & 1.43 & 671.10 & 
-0.90\\SCA3-6 & 655.19 & 1.34 & 
655.19 & 1.33 & \bf{651.10} & 
0.63\\SCA3-7 & 666.15 & 1.05 & 
666.15 & 1.00 & \bf{666.10} & 
0.01\\SCA3-8 & 726.44 & 1.08 & 
727.09 & 1.14 & \bf{719.50} & 
0.96\\SCA3-9 & \bf{681.00} & 0.99 & 
681.00 & 0.97 & 681.00 & 0.00\\
SCA8-0 & 991.07 & 1.52 & 
992.80 & 1.51 & \bf{961.60} & 
3.06\\SCA8-1 & 1074.65 & 1.22 & 
1074.65 & 1.21 & \bf{1063.00} & 
1.10\\SCA8-2 & 1056.87 & 1.03 & 
1056.87 & 1.06 & \bf{1040.60} & 
1.56\\SCA8-3 & 1031.08 & 1.50 & 
1031.08 & 1.50 & \bf{985.90} & 
4.58\\SCA8-4 & 1099.06 & 1.51 & 
1099.06 & 1.53 & \bf{1071.00} & 
2.62\\SCA8-5 & 1055.35 & 1.64 & 
1055.35 & 1.70 & \bf{1054.30} & 
0.10\\SCA8-6 & \bf{\underline{972.48}} & 1.73 & 
972.48 & 1.67 & 972.50 & 
-0.00\\SCA8-7 & 1092.57 & 1.60 & 
1092.57 & 1.64 & \bf{1059.70} & 
3.10\\SCA8-8 & 1092.02 & 1.39 & 
1092.02 & 1.44 & \bf{1082.70} & 
0.86\\SCA8-9 & \bf{\underline{1067.42}} & 1.19 & 
1067.42 & 1.16 & 1081.40 & 
-1.29\\CON3-0 & 624.96 & 1.59 & 
624.96 & 1.66 & \bf{616.50} & 
1.37\\CON3-1 & 557.38 & 1.43 & 
558.22 & 1.43 & \bf{555.60} & 
0.32\\CON3-2 & 524.07 & 1.20 & 
525.23 & 1.13 & \bf{521.40} & 
0.51\\CON3-3 & 594.11 & 1.53 & 
594.11 & 1.48 & \bf{591.20} & 
0.49\\CON3-4 & 589.32 & 1.37 & 
589.32 & 1.32 & \bf{589.30} & 
0.00\\CON3-5 & 576.43 & 1.34 & 
576.43 & 1.43 & \bf{563.70} & 
2.26\\CON3-6 & 505.26 & 1.77 & 
507.61 & 1.79 & \bf{499.20} & 
1.21\\CON3-7 & 578.41 & 1.23 & 
578.41 & 1.25 & \bf{577.50} & 
0.16\\CON3-8 & 524.59 & 1.12 & 
524.59 & 1.18 & \bf{523.10} & 
0.28\\CON3-9 & 588.48 & 1.32 & 
588.48 & 1.30 & \bf{578.20} & 
1.78\\CON8-0 & 879.00 & 1.42 & 
879.00 & 1.44 & \bf{858.90} & 
2.34\\CON8-1 & 758.26 & 1.34 & 
758.26 & 1.37 & \bf{740.90} & 
2.34\\CON8-2 & 716.53 & 1.96 & 
716.55 & 1.98 & \bf{714.30} & 
0.31\\CON8-3 & 817.57 & 1.48 & 
817.57 & 1.42 & \bf{812.30} & 
0.65\\CON8-4 & 778.60 & 1.53 & 
782.98 & 1.50 & \bf{770.10} & 
1.10\\CON8-5 & \bf{\underline{764.36}} & 1.34 & 
764.36 & 1.34 & 766.60 & 
-0.29\\CON8-6 & 705.61 & 1.64 & 
706.18 & 1.71 & \bf{697.20} & 
1.21\\CON8-7 & 822.84 & 1.13 & 
823.28 & 1.19 & \bf{814.80} & 
0.99\\CON8-8 & 799.32 & 1.48 & 
799.46 & 1.51 & \bf{771.30} & 
3.63\\CON8-9 & 816.12 & 1.60 & 
817.40 & 1.54 & \bf{815.10} & 
0.13\\[1ex]\hline
\end{tabular}
\label{table:nonlin}
\end{table} \clearpage
\begin{table}[ht]
\caption{Resultados de la ejecución de la metaheurística ACO, utilizando instancias de Dethloff con la configuración -n 2.0 -alpha 1.0 -beta 3.0 -q 9.8 -ro 0.015}
\centering
\small
\begin{tabular}{c c c c c c c}
\hline\hline
Instancia & Costo mínimo & Tiempo(seg.) & Costo promedio & Tiempo promedio(seg.) & Costo ACO & \%Gap \\ [0.5ex]
\hline
SCA3-0 & 640.55 & 1.39 & 
640.55 & 1.40 & \bf{636.10} & 
0.70\\SCA3-1 & \bf{\underline{697.84}} & 1.48 & 
698.76 & 1.49 & 700.10 & 
-0.32\\SCA3-2 & 659.34 & 1.65 & 
662.97 & 1.42 & \bf{659.30} & 
0.01\\SCA3-3 & 680.60 & 1.44 & 
680.78 & 1.48 & \bf{680.00} & 
0.09\\SCA3-4 & \bf{690.50} & 1.47 & 
690.50 & 1.38 & 690.50 & 0.00\\
SCA3-5 & \bf{\underline{665.04}} & 1.32 & 
665.04 & 1.35 & 671.10 & 
-0.90\\SCA3-6 & 655.19 & 1.24 & 
655.30 & 1.25 & \bf{651.10} & 
0.63\\SCA3-7 & 666.15 & 0.99 & 
666.15 & 0.98 & \bf{666.10} & 
0.01\\SCA3-8 & 721.45 & 1.12 & 
725.97 & 1.10 & \bf{719.50} & 
0.27\\SCA3-9 & \bf{681.00} & 0.90 & 
681.00 & 0.92 & 681.00 & 0.00\\
SCA8-0 & 991.07 & 1.52 & 
991.65 & 1.55 & \bf{961.60} & 
3.06\\SCA8-1 & 1074.65 & 1.20 & 
1074.65 & 1.18 & \bf{1063.00} & 
1.10\\SCA8-2 & 1056.87 & 1.06 & 
1056.87 & 1.03 & \bf{1040.60} & 
1.56\\SCA8-3 & 1031.08 & 1.50 & 
1031.08 & 1.50 & \bf{985.90} & 
4.58\\SCA8-4 & 1098.34 & 1.56 & 
1098.88 & 1.53 & \bf{1071.00} & 
2.55\\SCA8-5 & 1055.35 & 1.65 & 
1055.35 & 1.61 & \bf{1054.30} & 
0.10\\SCA8-6 & \bf{\underline{972.48}} & 1.74 & 
972.48 & 1.67 & 972.50 & 
-0.00\\SCA8-7 & 1092.57 & 1.61 & 
1092.57 & 1.64 & \bf{1059.70} & 
3.10\\SCA8-8 & 1091.49 & 1.49 & 
1091.89 & 1.46 & \bf{1082.70} & 
0.81\\SCA8-9 & \bf{\underline{1067.42}} & 1.19 & 
1067.42 & 1.17 & 1081.40 & 
-1.29\\CON3-0 & 624.96 & 1.71 & 
624.96 & 1.67 & \bf{616.50} & 
1.37\\CON3-1 & 557.38 & 1.37 & 
559.26 & 1.44 & \bf{555.60} & 
0.32\\CON3-2 & 524.07 & 1.10 & 
524.79 & 1.17 & \bf{521.40} & 
0.51\\CON3-3 & \bf{591.20} & 1.44 & 
593.38 & 1.49 & 591.20 & 0.00\\
CON3-4 & 589.32 & 1.46 & 
589.32 & 1.35 & \bf{589.30} & 
0.00\\CON3-5 & 576.43 & 1.46 & 
577.34 & 1.41 & \bf{563.70} & 
2.26\\CON3-6 & 504.15 & 1.84 & 
504.98 & 1.78 & \bf{499.20} & 
0.99\\CON3-7 & 578.41 & 1.19 & 
578.41 & 1.20 & \bf{577.50} & 
0.16\\CON3-8 & 524.30 & 1.22 & 
524.45 & 1.20 & \bf{523.10} & 
0.23\\CON3-9 & 588.48 & 1.48 & 
588.48 & 1.31 & \bf{578.20} & 
1.78\\CON8-0 & 879.00 & 1.34 & 
879.00 & 1.40 & \bf{858.90} & 
2.34\\CON8-1 & 758.26 & 1.30 & 
758.26 & 1.32 & \bf{740.90} & 
2.34\\CON8-2 & 716.53 & 1.94 & 
716.55 & 2.05 & \bf{714.30} & 
0.31\\CON8-3 & 817.57 & 1.44 & 
817.57 & 1.43 & \bf{812.30} & 
0.65\\CON8-4 & 778.60 & 1.64 & 
782.97 & 1.61 & \bf{770.10} & 
1.10\\CON8-5 & \bf{\underline{764.36}} & 1.36 & 
764.36 & 1.35 & 766.60 & 
-0.29\\CON8-6 & \bf{\underline{693.83}} & 1.70 & 
703.23 & 1.70 & 697.20 & 
-0.48\\CON8-7 & 822.42 & 1.16 & 
822.50 & 1.20 & \bf{814.80} & 
0.94\\CON8-8 & 799.32 & 1.61 & 
799.46 & 1.61 & \bf{771.30} & 
3.63\\CON8-9 & 816.12 & 1.58 & 
816.12 & 1.54 & \bf{815.10} & 
0.13\\[1ex]\hline
\end{tabular}
\label{table:nonlin}
\end{table} \clearpage
\begin{table}[ht]
\caption{Resultados de la ejecución de la metaheurística ACO, utilizando instancias de Dethloff con la configuración -n 2.0 -alpha 1.0 -beta 3.0 -q 9.9 -ro 0.015}
\centering
\small
\begin{tabular}{c c c c c c c}
\hline\hline
Instancia & Costo mínimo & Tiempo(seg.) & Costo promedio & Tiempo promedio(seg.) & Costo ACO & \%Gap \\ [0.5ex]
\hline
SCA3-0 & 640.55 & 1.34 & 
640.55 & 1.40 & \bf{636.10} & 
0.70\\SCA3-1 & \bf{\underline{697.84}} & 1.52 & 
697.84 & 1.47 & 700.10 & 
-0.32\\SCA3-2 & 664.18 & 1.34 & 
664.18 & 1.34 & \bf{659.30} & 
0.74\\SCA3-3 & 680.60 & 1.54 & 
680.78 & 1.46 & \bf{680.00} & 
0.09\\SCA3-4 & \bf{690.50} & 1.37 & 
690.50 & 1.40 & 690.50 & 0.00\\
SCA3-5 & \bf{\underline{665.04}} & 1.50 & 
665.04 & 1.52 & 671.10 & 
-0.90\\SCA3-6 & 655.19 & 1.38 & 
655.19 & 1.40 & \bf{651.10} & 
0.63\\SCA3-7 & 666.15 & 1.00 & 
666.15 & 1.01 & \bf{666.10} & 
0.01\\SCA3-8 & 721.45 & 1.12 & 
724.59 & 1.15 & \bf{719.50} & 
0.27\\SCA3-9 & \bf{681.00} & 1.07 & 
681.00 & 0.99 & 681.00 & 0.00\\
SCA8-0 & 991.07 & 1.48 & 
991.07 & 1.50 & \bf{961.60} & 
3.06\\SCA8-1 & 1074.65 & 1.22 & 
1074.68 & 1.22 & \bf{1063.00} & 
1.10\\SCA8-2 & 1056.87 & 1.05 & 
1056.87 & 1.01 & \bf{1040.60} & 
1.56\\SCA8-3 & 1031.08 & 1.67 & 
1031.08 & 1.52 & \bf{985.90} & 
4.58\\SCA8-4 & 1098.34 & 1.60 & 
1098.88 & 1.53 & \bf{1071.00} & 
2.55\\SCA8-5 & 1055.35 & 1.65 & 
1055.35 & 1.64 & \bf{1054.30} & 
0.10\\SCA8-6 & \bf{\underline{972.48}} & 1.71 & 
977.49 & 1.70 & 972.50 & 
-0.00\\SCA8-7 & 1092.57 & 1.61 & 
1092.57 & 1.60 & \bf{1059.70} & 
3.10\\SCA8-8 & 1091.49 & 1.47 & 
1091.89 & 1.46 & \bf{1082.70} & 
0.81\\SCA8-9 & \bf{\underline{1067.42}} & 1.09 & 
1067.42 & 1.14 & 1081.40 & 
-1.29\\CON3-0 & 624.96 & 1.69 & 
624.96 & 1.63 & \bf{616.50} & 
1.37\\CON3-1 & 557.38 & 1.43 & 
558.86 & 1.83 & \bf{555.60} & 
0.32\\CON3-2 & 524.07 & 1.10 & 
524.51 & 1.13 & \bf{521.40} & 
0.51\\CON3-3 & 594.11 & 1.54 & 
594.11 & 1.53 & \bf{591.20} & 
0.49\\CON3-4 & 589.32 & 1.32 & 
589.32 & 1.34 & \bf{589.30} & 
0.00\\CON3-5 & 576.43 & 1.52 & 
576.43 & 1.51 & \bf{563.70} & 
2.26\\CON3-6 & 505.26 & 1.88 & 
507.56 & 1.80 & \bf{499.20} & 
1.21\\CON3-7 & 578.41 & 1.32 & 
578.41 & 1.23 & \bf{577.50} & 
0.16\\CON3-8 & 524.30 & 1.16 & 
524.52 & 1.17 & \bf{523.10} & 
0.23\\CON3-9 & 588.48 & 1.37 & 
588.48 & 1.28 & \bf{578.20} & 
1.78\\CON8-0 & 879.00 & 1.41 & 
879.00 & 1.45 & \bf{858.90} & 
2.34\\CON8-1 & 758.26 & 1.36 & 
758.26 & 1.34 & \bf{740.90} & 
2.34\\CON8-2 & 716.56 & 1.95 & 
717.21 & 2.02 & \bf{714.30} & 
0.32\\CON8-3 & 817.57 & 1.45 & 
817.57 & 1.45 & \bf{812.30} & 
0.65\\CON8-4 & 778.60 & 1.62 & 
785.07 & 1.58 & \bf{770.10} & 
1.10\\CON8-5 & \bf{\underline{764.36}} & 1.42 & 
764.36 & 1.37 & 766.60 & 
-0.29\\CON8-6 & \bf{\underline{693.83}} & 1.74 & 
703.57 & 1.69 & 697.20 & 
-0.48\\CON8-7 & 822.42 & 1.23 & 
823.18 & 1.19 & \bf{814.80} & 
0.94\\CON8-8 & 799.16 & 1.45 & 
799.38 & 1.50 & \bf{771.30} & 
3.61\\CON8-9 & 816.12 & 1.56 & 
816.12 & 1.53 & \bf{815.10} & 
0.13\\[1ex]\hline
\end{tabular}
\label{table:nonlin}
\end{table} \clearpage
\begin{table}[ht]
\caption{Resultados de la ejecución de la metaheurística ACO, utilizando instancias de Dethloff con la configuración -n 2.0 -alpha 1.0 -beta 3.0 -q 10.0 -ro 0.015}
\centering
\small
\begin{tabular}{c c c c c c c}
\hline\hline
Instancia & Costo mínimo & Tiempo(seg.) & Costo promedio & Tiempo promedio(seg.) & Costo ACO & \%Gap \\ [0.5ex]
\hline
SCA3-0 & 640.55 & 1.36 & 
640.55 & 1.38 & \bf{636.10} & 
0.70\\SCA3-1 & \bf{\underline{697.84}} & 1.47 & 
698.76 & 1.46 & 700.10 & 
-0.32\\SCA3-2 & 664.18 & 1.35 & 
664.18 & 1.35 & \bf{659.30} & 
0.74\\SCA3-3 & 680.60 & 1.50 & 
680.96 & 1.48 & \bf{680.00} & 
0.09\\SCA3-4 & \bf{690.50} & 1.45 & 
690.50 & 1.40 & 690.50 & 0.00\\
SCA3-5 & \bf{\underline{665.04}} & 1.80 & 
665.19 & 1.51 & 671.10 & 
-0.90\\SCA3-6 & 655.19 & 1.30 & 
655.19 & 1.35 & \bf{651.10} & 
0.63\\SCA3-7 & 666.15 & 0.96 & 
666.15 & 1.01 & \bf{666.10} & 
0.01\\SCA3-8 & 721.45 & 1.14 & 
724.59 & 1.15 & \bf{719.50} & 
0.27\\SCA3-9 & \bf{681.00} & 0.96 & 
681.00 & 1.00 & 681.00 & 0.00\\
SCA8-0 & 991.07 & 1.49 & 
993.89 & 1.50 & \bf{961.60} & 
3.06\\SCA8-1 & 1074.65 & 1.28 & 
1074.68 & 1.20 & \bf{1063.00} & 
1.10\\SCA8-2 & 1056.87 & 0.96 & 
1056.87 & 0.98 & \bf{1040.60} & 
1.56\\SCA8-3 & 1031.08 & 1.51 & 
1031.08 & 1.48 & \bf{985.90} & 
4.58\\SCA8-4 & 1099.06 & 1.71 & 
1099.06 & 1.56 & \bf{1071.00} & 
2.62\\SCA8-5 & 1055.35 & 1.72 & 
1055.35 & 1.70 & \bf{1054.30} & 
0.10\\SCA8-6 & \bf{\underline{972.48}} & 1.74 & 
975.00 & 1.70 & 972.50 & 
-0.00\\SCA8-7 & 1092.57 & 1.53 & 
1092.57 & 1.62 & \bf{1059.70} & 
3.10\\SCA8-8 & 1092.02 & 1.50 & 
1092.02 & 1.48 & \bf{1082.70} & 
0.86\\SCA8-9 & \bf{\underline{1067.42}} & 1.17 & 
1067.42 & 1.17 & 1081.40 & 
-1.29\\CON3-0 & 624.96 & 1.62 & 
624.96 & 1.61 & \bf{616.50} & 
1.37\\CON3-1 & 557.38 & 1.52 & 
557.82 & 1.52 & \bf{555.60} & 
0.32\\CON3-2 & 524.07 & 1.05 & 
524.82 & 1.10 & \bf{521.40} & 
0.51\\CON3-3 & 592.95 & 1.56 & 
593.82 & 1.55 & \bf{591.20} & 
0.30\\CON3-4 & 589.32 & 1.25 & 
589.32 & 1.37 & \bf{589.30} & 
0.00\\CON3-5 & 576.43 & 1.39 & 
576.43 & 1.41 & \bf{563.70} & 
2.26\\CON3-6 & 505.26 & 1.87 & 
507.82 & 1.80 & \bf{499.20} & 
1.21\\CON3-7 & 578.41 & 2.31 & 
578.41 & 1.51 & \bf{577.50} & 
0.16\\CON3-8 & 524.30 & 1.27 & 
525.25 & 1.32 & \bf{523.10} & 
0.23\\CON3-9 & 588.48 & 1.30 & 
588.48 & 1.31 & \bf{578.20} & 
1.78\\CON8-0 & 879.00 & 1.50 & 
879.00 & 1.47 & \bf{858.90} & 
2.34\\CON8-1 & 758.26 & 1.28 & 
758.30 & 1.30 & \bf{740.90} & 
2.34\\CON8-2 & 716.53 & 1.95 & 
717.20 & 1.99 & \bf{714.30} & 
0.31\\CON8-3 & 817.57 & 1.39 & 
817.57 & 1.42 & \bf{812.30} & 
0.65\\CON8-4 & 789.98 & 1.54 & 
791.53 & 1.57 & \bf{770.10} & 
2.58\\CON8-5 & \bf{\underline{764.36}} & 1.38 & 
764.36 & 1.35 & 766.60 & 
-0.29\\CON8-6 & 705.61 & 1.68 & 
706.18 & 1.71 & \bf{697.20} & 
1.21\\CON8-7 & 823.43 & 1.13 & 
823.43 & 1.19 & \bf{814.80} & 
1.06\\CON8-8 & 796.81 & 1.64 & 
798.69 & 1.57 & \bf{771.30} & 
3.31\\CON8-9 & 816.12 & 1.56 & 
816.12 & 1.57 & \bf{815.10} & 
0.13\\[1ex]\hline
\end{tabular}
\label{table:nonlin}
\end{table} \clearpage
\begin{table}[ht]
\caption{Resultados de la ejecución de la metaheurística ACO, utilizando instancias de Dethloff con la configuración -n 2.0 -alpha 1.0 -beta 3.0 -q 10.1 -ro 0.015}
\centering
\small
\begin{tabular}{c c c c c c c}
\hline\hline
Instancia & Costo mínimo & Tiempo(seg.) & Costo promedio & Tiempo promedio(seg.) & Costo ACO & \%Gap \\ [0.5ex]
\hline
SCA3-0 & 640.55 & 1.42 & 
640.55 & 1.34 & \bf{636.10} & 
0.70\\SCA3-1 & \bf{\underline{697.84}} & 1.53 & 
698.76 & 1.46 & 700.10 & 
-0.32\\SCA3-2 & 664.18 & 1.42 & 
664.18 & 1.36 & \bf{659.30} & 
0.74\\SCA3-3 & 680.60 & 1.44 & 
680.60 & 1.47 & \bf{680.00} & 
0.09\\SCA3-4 & \bf{690.50} & 1.38 & 
690.50 & 1.42 & 690.50 & 0.00\\
SCA3-5 & \bf{\underline{665.04}} & 1.40 & 
665.04 & 1.43 & 671.10 & 
-0.90\\SCA3-6 & 655.19 & 1.27 & 
655.19 & 1.38 & \bf{651.10} & 
0.63\\SCA3-7 & 666.15 & 1.06 & 
666.15 & 1.09 & \bf{666.10} & 
0.01\\SCA3-8 & 721.45 & 1.15 & 
727.86 & 1.14 & \bf{719.50} & 
0.27\\SCA3-9 & \bf{681.00} & 0.97 & 
681.00 & 1.07 & 681.00 & 0.00\\
SCA8-0 & 991.07 & 1.52 & 
991.07 & 1.52 & \bf{961.60} & 
3.06\\SCA8-1 & 1074.39 & 1.15 & 
1074.59 & 1.19 & \bf{1063.00} & 
1.07\\SCA8-2 & 1056.87 & 1.06 & 
1056.87 & 1.05 & \bf{1040.60} & 
1.56\\SCA8-3 & 1031.08 & 1.41 & 
1031.08 & 1.39 & \bf{985.90} & 
4.58\\SCA8-4 & 1098.34 & 1.42 & 
1098.88 & 1.48 & \bf{1071.00} & 
2.55\\SCA8-5 & 1055.35 & 1.70 & 
1055.35 & 1.65 & \bf{1054.30} & 
0.10\\SCA8-6 & \bf{\underline{972.48}} & 1.66 & 
972.48 & 1.69 & 972.50 & 
-0.00\\SCA8-7 & 1092.57 & 1.73 & 
1092.57 & 1.69 & \bf{1059.70} & 
3.10\\SCA8-8 & 1091.49 & 1.45 & 
1091.89 & 1.47 & \bf{1082.70} & 
0.81\\SCA8-9 & \bf{\underline{1067.42}} & 1.16 & 
1067.42 & 1.16 & 1081.40 & 
-1.29\\CON3-0 & 624.96 & 1.60 & 
624.96 & 1.67 & \bf{616.50} & 
1.37\\CON3-1 & 557.38 & 1.49 & 
558.50 & 1.49 & \bf{555.60} & 
0.32\\CON3-2 & 524.07 & 1.08 & 
525.08 & 1.09 & \bf{521.40} & 
0.51\\CON3-3 & \bf{591.20} & 1.58 & 
593.38 & 1.54 & 591.20 & 0.00\\
CON3-4 & 589.32 & 1.38 & 
589.32 & 1.33 & \bf{589.30} & 
0.00\\CON3-5 & 570.70 & 1.41 & 
575.00 & 1.45 & \bf{563.70} & 
1.24\\CON3-6 & 505.26 & 1.83 & 
505.26 & 1.84 & \bf{499.20} & 
1.21\\CON3-7 & 578.41 & 1.24 & 
580.55 & 1.21 & \bf{577.50} & 
0.16\\CON3-8 & 524.30 & 1.26 & 
524.52 & 1.29 & \bf{523.10} & 
0.23\\CON3-9 & 588.48 & 1.28 & 
588.48 & 1.29 & \bf{578.20} & 
1.78\\CON8-0 & 879.00 & 1.44 & 
879.00 & 1.46 & \bf{858.90} & 
2.34\\CON8-1 & 758.26 & 1.33 & 
758.26 & 1.34 & \bf{740.90} & 
2.34\\CON8-2 & 716.53 & 1.93 & 
716.55 & 1.96 & \bf{714.30} & 
0.31\\CON8-3 & 817.57 & 1.46 & 
817.57 & 1.50 & \bf{812.30} & 
0.65\\CON8-4 & 778.60 & 1.50 & 
785.05 & 1.52 & \bf{770.10} & 
1.10\\CON8-5 & \bf{\underline{764.36}} & 1.28 & 
764.36 & 1.33 & 766.60 & 
-0.29\\CON8-6 & \bf{\underline{693.83}} & 1.76 & 
704.13 & 1.72 & 697.20 & 
-0.48\\CON8-7 & 822.42 & 1.14 & 
822.42 & 1.21 & \bf{814.80} & 
0.94\\CON8-8 & 799.32 & 1.48 & 
799.41 & 1.49 & \bf{771.30} & 
3.63\\CON8-9 & 816.12 & 1.52 & 
816.12 & 1.56 & \bf{815.10} & 
0.13\\[1ex]\hline
\end{tabular}
\label{table:nonlin}
\end{table} \clearpage
\begin{table}[ht]
\caption{Resultados de la ejecución de la metaheurística ACO, utilizando instancias de Dethloff con la configuración -n 2.0 -alpha 1.0 -beta 3.0 -q 10.2 -ro 0.015}
\centering
\small
\begin{tabular}{c c c c c c c}
\hline\hline
Instancia & Costo mínimo & Tiempo(seg.) & Costo promedio & Tiempo promedio(seg.) & Costo ACO & \%Gap \\ [0.5ex]
\hline
SCA3-0 & 640.55 & 1.44 & 
640.55 & 1.39 & \bf{636.10} & 
0.70\\SCA3-1 & \bf{\underline{697.84}} & 1.48 & 
697.84 & 1.52 & 700.10 & 
-0.32\\SCA3-2 & 659.34 & 1.28 & 
662.97 & 1.30 & \bf{659.30} & 
0.01\\SCA3-3 & 680.60 & 1.54 & 
680.60 & 1.65 & \bf{680.00} & 
0.09\\SCA3-4 & \bf{690.50} & 1.37 & 
690.50 & 1.41 & 690.50 & 0.00\\
SCA3-5 & \bf{\underline{665.04}} & 1.38 & 
665.19 & 1.37 & 671.10 & 
-0.90\\SCA3-6 & 655.19 & 1.33 & 
655.30 & 1.32 & \bf{651.10} & 
0.63\\SCA3-7 & 666.15 & 0.99 & 
666.15 & 1.02 & \bf{666.10} & 
0.01\\SCA3-8 & 726.44 & 1.10 & 
729.11 & 1.14 & \bf{719.50} & 
0.96\\SCA3-9 & \bf{681.00} & 1.01 & 
681.00 & 0.98 & 681.00 & 0.00\\
SCA8-0 & 991.07 & 1.54 & 
992.80 & 1.52 & \bf{961.60} & 
3.06\\SCA8-1 & 1074.65 & 1.25 & 
1074.65 & 1.21 & \bf{1063.00} & 
1.10\\SCA8-2 & 1056.87 & 1.65 & 
1056.87 & 1.20 & \bf{1040.60} & 
1.56\\SCA8-3 & 1031.08 & 1.50 & 
1031.08 & 1.46 & \bf{985.90} & 
4.58\\SCA8-4 & 1099.06 & 1.60 & 
1099.06 & 1.59 & \bf{1071.00} & 
2.62\\SCA8-5 & 1055.35 & 1.74 & 
1055.35 & 1.65 & \bf{1054.30} & 
0.10\\SCA8-6 & \bf{\underline{972.48}} & 1.69 & 
976.70 & 1.67 & 972.50 & 
-0.00\\SCA8-7 & 1092.57 & 1.68 & 
1092.57 & 1.69 & \bf{1059.70} & 
3.10\\SCA8-8 & 1091.49 & 1.45 & 
1091.89 & 1.46 & \bf{1082.70} & 
0.81\\SCA8-9 & \bf{\underline{1067.42}} & 1.18 & 
1067.42 & 1.18 & 1081.40 & 
-1.29\\CON3-0 & 624.96 & 1.68 & 
624.96 & 1.64 & \bf{616.50} & 
1.37\\CON3-1 & 557.38 & 1.56 & 
558.30 & 1.57 & \bf{555.60} & 
0.32\\CON3-2 & 524.07 & 1.17 & 
524.79 & 1.10 & \bf{521.40} & 
0.51\\CON3-3 & 592.95 & 1.48 & 
593.82 & 1.49 & \bf{591.20} & 
0.30\\CON3-4 & \bf{\underline{588.79}} & 1.33 & 
589.19 & 1.34 & 589.30 & 
-0.09\\CON3-5 & 576.43 & 1.46 & 
576.43 & 1.52 & \bf{563.70} & 
2.26\\CON3-6 & 504.15 & 1.85 & 
504.98 & 1.82 & \bf{499.20} & 
0.99\\CON3-7 & 578.41 & 1.19 & 
579.12 & 1.20 & \bf{577.50} & 
0.16\\CON3-8 & 524.59 & 1.10 & 
527.01 & 1.23 & \bf{523.10} & 
0.28\\CON3-9 & 588.48 & 1.26 & 
588.48 & 1.26 & \bf{578.20} & 
1.78\\CON8-0 & 879.00 & 1.41 & 
879.00 & 1.40 & \bf{858.90} & 
2.34\\CON8-1 & 758.26 & 1.36 & 
758.26 & 1.31 & \bf{740.90} & 
2.34\\CON8-2 & 716.56 & 2.11 & 
717.21 & 2.06 & \bf{714.30} & 
0.32\\CON8-3 & 817.57 & 1.42 & 
817.57 & 1.45 & \bf{812.30} & 
0.65\\CON8-4 & 778.60 & 1.58 & 
785.05 & 1.55 & \bf{770.10} & 
1.10\\CON8-5 & \bf{\underline{764.36}} & 1.37 & 
764.36 & 1.36 & 766.60 & 
-0.29\\CON8-6 & 705.61 & 1.80 & 
706.18 & 1.70 & \bf{697.20} & 
1.21\\CON8-7 & 822.42 & 1.22 & 
823.18 & 1.20 & \bf{814.80} & 
0.94\\CON8-8 & 799.51 & 1.52 & 
799.51 & 1.55 & \bf{771.30} & 
3.66\\CON8-9 & 816.12 & 1.56 & 
816.12 & 1.56 & \bf{815.10} & 
0.13\\[1ex]\hline
\end{tabular}
\label{table:nonlin}
\end{table} \clearpage
\begin{table}[ht]
\caption{Resultados de la ejecución de la metaheurística ACO, utilizando instancias de Dethloff con la configuración -n 2.0 -alpha 1.0 -beta 3.0 -q 10.3 -ro 0.015}
\centering
\small
\begin{tabular}{c c c c c c c}
\hline\hline
Instancia & Costo mínimo & Tiempo(seg.) & Costo promedio & Tiempo promedio(seg.) & Costo ACO & \%Gap \\ [0.5ex]
\hline
SCA3-0 & 640.55 & 1.44 & 
640.55 & 1.37 & \bf{636.10} & 
0.70\\SCA3-1 & \bf{\underline{697.84}} & 1.44 & 
697.84 & 1.45 & 700.10 & 
-0.32\\SCA3-2 & 659.34 & 1.36 & 
662.97 & 1.36 & \bf{659.30} & 
0.01\\SCA3-3 & 680.60 & 1.52 & 
680.96 & 1.49 & \bf{680.00} & 
0.09\\SCA3-4 & \bf{690.50} & 1.28 & 
690.50 & 1.38 & 690.50 & 0.00\\
SCA3-5 & \bf{\underline{665.04}} & 1.40 & 
665.34 & 1.46 & 671.10 & 
-0.90\\SCA3-6 & 655.19 & 1.29 & 
655.19 & 1.38 & \bf{651.10} & 
0.63\\SCA3-7 & 666.15 & 0.94 & 
666.15 & 0.99 & \bf{666.10} & 
0.01\\SCA3-8 & 721.45 & 1.15 & 
723.34 & 1.11 & \bf{719.50} & 
0.27\\SCA3-9 & \bf{681.00} & 0.96 & 
681.00 & 0.97 & 681.00 & 0.00\\
SCA8-0 & 991.07 & 1.55 & 
993.32 & 1.51 & \bf{961.60} & 
3.06\\SCA8-1 & 1074.65 & 1.20 & 
1074.65 & 1.19 & \bf{1063.00} & 
1.10\\SCA8-2 & 1056.87 & 1.03 & 
1056.87 & 1.03 & \bf{1040.60} & 
1.56\\SCA8-3 & 1031.08 & 1.52 & 
1031.55 & 1.47 & \bf{985.90} & 
4.58\\SCA8-4 & 1099.06 & 1.50 & 
1099.17 & 1.52 & \bf{1071.00} & 
2.62\\SCA8-5 & 1055.35 & 1.58 & 
1055.35 & 1.69 & \bf{1054.30} & 
0.10\\SCA8-6 & \bf{\underline{972.48}} & 1.82 & 
972.48 & 1.72 & 972.50 & 
-0.00\\SCA8-7 & 1092.57 & 1.61 & 
1092.57 & 1.63 & \bf{1059.70} & 
3.10\\SCA8-8 & 1092.02 & 1.50 & 
1092.02 & 1.47 & \bf{1082.70} & 
0.86\\SCA8-9 & \bf{\underline{1067.42}} & 1.13 & 
1067.42 & 1.14 & 1081.40 & 
-1.29\\CON3-0 & 624.96 & 1.63 & 
624.96 & 1.65 & \bf{616.50} & 
1.37\\CON3-1 & 557.38 & 1.46 & 
559.99 & 1.49 & \bf{555.60} & 
0.32\\CON3-2 & 524.07 & 1.18 & 
524.35 & 1.14 & \bf{521.40} & 
0.51\\CON3-3 & \bf{591.20} & 1.49 & 
592.65 & 1.50 & 591.20 & 0.00\\
CON3-4 & 589.32 & 1.40 & 
589.32 & 1.37 & \bf{589.30} & 
0.00\\CON3-5 & 570.70 & 1.48 & 
575.00 & 1.45 & \bf{563.70} & 
1.24\\CON3-6 & 505.26 & 1.77 & 
505.79 & 1.85 & \bf{499.20} & 
1.21\\CON3-7 & 578.41 & 1.16 & 
578.41 & 1.22 & \bf{577.50} & 
0.16\\CON3-8 & 524.30 & 1.30 & 
524.52 & 1.24 & \bf{523.10} & 
0.23\\CON3-9 & 588.48 & 1.27 & 
588.48 & 1.39 & \bf{578.20} & 
1.78\\CON8-0 & 879.00 & 1.38 & 
879.00 & 1.49 & \bf{858.90} & 
2.34\\CON8-1 & 758.26 & 1.32 & 
758.26 & 1.34 & \bf{740.90} & 
2.34\\CON8-2 & 716.53 & 2.01 & 
716.54 & 1.98 & \bf{714.30} & 
0.31\\CON8-3 & 817.57 & 1.46 & 
817.57 & 1.44 & \bf{812.30} & 
0.65\\CON8-4 & 781.64 & 1.57 & 
788.67 & 1.54 & \bf{770.10} & 
1.50\\CON8-5 & \bf{\underline{764.36}} & 1.42 & 
764.36 & 1.39 & 766.60 & 
-0.29\\CON8-6 & \bf{\underline{693.83}} & 1.76 & 
703.80 & 1.77 & 697.20 & 
-0.48\\CON8-7 & 822.42 & 1.22 & 
822.92 & 1.20 & \bf{814.80} & 
0.94\\CON8-8 & 799.32 & 1.52 & 
799.46 & 1.53 & \bf{771.30} & 
3.63\\CON8-9 & 816.12 & 1.62 & 
816.12 & 1.56 & \bf{815.10} & 
0.13\\[1ex]\hline
\end{tabular}
\label{table:nonlin}
\end{table} \clearpage
\begin{table}[ht]
\caption{Resultados de la ejecución de la metaheurística ACO, utilizando instancias de Dethloff con la configuración -n 2.0 -alpha 1.0 -beta 3.0 -q 10.4 -ro 0.015}
\centering
\small
\begin{tabular}{c c c c c c c}
\hline\hline
Instancia & Costo mínimo & Tiempo(seg.) & Costo promedio & Tiempo promedio(seg.) & Costo ACO & \%Gap \\ [0.5ex]
\hline
SCA3-0 & 640.55 & 1.49 & 
640.55 & 1.45 & \bf{636.10} & 
0.70\\SCA3-1 & \bf{\underline{697.84}} & 1.52 & 
697.84 & 1.53 & 700.10 & 
-0.32\\SCA3-2 & 659.34 & 1.31 & 
662.97 & 1.30 & \bf{659.30} & 
0.01\\SCA3-3 & 680.60 & 1.48 & 
680.74 & 1.48 & \bf{680.00} & 
0.09\\SCA3-4 & \bf{690.50} & 1.38 & 
690.50 & 1.41 & 690.50 & 0.00\\
SCA3-5 & \bf{\underline{665.04}} & 1.44 & 
665.04 & 1.44 & 671.10 & 
-0.90\\SCA3-6 & 655.19 & 1.56 & 
655.19 & 1.38 & \bf{651.10} & 
0.63\\SCA3-7 & 666.15 & 1.00 & 
666.15 & 1.01 & \bf{666.10} & 
0.01\\SCA3-8 & 721.45 & 1.07 & 
725.24 & 1.10 & \bf{719.50} & 
0.27\\SCA3-9 & \bf{681.00} & 0.94 & 
681.00 & 0.98 & 681.00 & 0.00\\
SCA8-0 & 991.07 & 1.53 & 
991.07 & 1.52 & \bf{961.60} & 
3.06\\SCA8-1 & 1074.39 & 1.21 & 
1074.59 & 1.18 & \bf{1063.00} & 
1.07\\SCA8-2 & 1056.87 & 0.98 & 
1056.87 & 1.04 & \bf{1040.60} & 
1.56\\SCA8-3 & 1031.08 & 1.42 & 
1031.08 & 1.47 & \bf{985.90} & 
4.58\\SCA8-4 & 1098.34 & 1.44 & 
1098.88 & 1.47 & \bf{1071.00} & 
2.55\\SCA8-5 & 1055.35 & 1.75 & 
1055.35 & 1.66 & \bf{1054.30} & 
0.10\\SCA8-6 & \bf{\underline{972.48}} & 1.65 & 
972.48 & 1.66 & 972.50 & 
-0.00\\SCA8-7 & 1092.57 & 1.76 & 
1092.57 & 1.70 & \bf{1059.70} & 
3.10\\SCA8-8 & \bf{\underline{1071.18}} & 1.54 & 
1086.81 & 1.47 & 1082.70 & 
-1.06\\SCA8-9 & \bf{\underline{1067.42}} & 1.34 & 
1067.42 & 1.20 & 1081.40 & 
-1.29\\CON3-0 & 624.96 & 1.60 & 
624.96 & 1.63 & \bf{616.50} & 
1.37\\CON3-1 & 557.38 & 1.50 & 
557.38 & 1.42 & \bf{555.60} & 
0.32\\CON3-2 & 524.07 & 1.08 & 
526.14 & 1.12 & \bf{521.40} & 
0.51\\CON3-3 & 594.11 & 1.43 & 
594.11 & 1.50 & \bf{591.20} & 
0.49\\CON3-4 & 589.32 & 1.35 & 
589.32 & 1.35 & \bf{589.30} & 
0.00\\CON3-5 & 569.88 & 1.44 & 
574.79 & 1.44 & \bf{563.70} & 
1.10\\CON3-6 & 505.26 & 1.85 & 
505.26 & 1.77 & \bf{499.20} & 
1.21\\CON3-7 & 578.41 & 1.32 & 
578.41 & 1.26 & \bf{577.50} & 
0.16\\CON3-8 & 524.30 & 1.71 & 
524.52 & 1.35 & \bf{523.10} & 
0.23\\CON3-9 & 588.48 & 1.23 & 
588.48 & 1.28 & \bf{578.20} & 
1.78\\CON8-0 & 879.00 & 1.48 & 
879.00 & 1.48 & \bf{858.90} & 
2.34\\CON8-1 & 758.26 & 1.38 & 
758.26 & 1.31 & \bf{740.90} & 
2.34\\CON8-2 & 716.53 & 1.96 & 
716.53 & 2.00 & \bf{714.30} & 
0.31\\CON8-3 & 817.57 & 1.48 & 
817.57 & 1.43 & \bf{812.30} & 
0.65\\CON8-4 & 781.64 & 1.64 & 
787.89 & 1.57 & \bf{770.10} & 
1.50\\CON8-5 & \bf{\underline{764.36}} & 1.39 & 
764.36 & 1.40 & 766.60 & 
-0.29\\CON8-6 & \bf{\underline{693.83}} & 1.78 & 
703.72 & 1.76 & 697.20 & 
-0.48\\CON8-7 & 822.84 & 1.12 & 
823.28 & 1.17 & \bf{814.80} & 
0.99\\CON8-8 & 799.16 & 1.54 & 
799.38 & 1.54 & \bf{771.30} & 
3.61\\CON8-9 & 816.12 & 1.50 & 
817.40 & 1.54 & \bf{815.10} & 
0.13\\[1ex]\hline
\end{tabular}
\label{table:nonlin}
\end{table} \clearpage
\begin{table}[ht]
\caption{Resultados de la ejecución de la metaheurística ACO, utilizando instancias de Dethloff con la configuración -n 2.0 -alpha 1.0 -beta 3.0 -q 10.5 -ro 0.015}
\centering
\small
\begin{tabular}{c c c c c c c}
\hline\hline
Instancia & Costo mínimo & Tiempo(seg.) & Costo promedio & Tiempo promedio(seg.) & Costo ACO & \%Gap \\ [0.5ex]
\hline
SCA3-0 & 640.55 & 1.30 & 
640.55 & 1.40 & \bf{636.10} & 
0.70\\SCA3-1 & \bf{\underline{697.84}} & 1.46 & 
698.76 & 1.48 & 700.10 & 
-0.32\\SCA3-2 & 659.34 & 1.30 & 
662.97 & 1.34 & \bf{659.30} & 
0.01\\SCA3-3 & 680.60 & 1.45 & 
680.78 & 1.43 & \bf{680.00} & 
0.09\\SCA3-4 & \bf{690.50} & 1.48 & 
690.50 & 1.43 & 690.50 & 0.00\\
SCA3-5 & \bf{\underline{665.04}} & 1.41 & 
665.04 & 1.52 & 671.10 & 
-0.90\\SCA3-6 & 653.81 & 1.39 & 
654.85 & 1.36 & \bf{651.10} & 
0.42\\SCA3-7 & 666.15 & 0.96 & 
666.15 & 1.00 & \bf{666.10} & 
0.01\\SCA3-8 & 721.45 & 1.13 & 
725.97 & 1.10 & \bf{719.50} & 
0.27\\SCA3-9 & \bf{681.00} & 0.96 & 
681.00 & 0.99 & 681.00 & 0.00\\
SCA8-0 & 991.07 & 1.51 & 
991.65 & 1.52 & \bf{961.60} & 
3.06\\SCA8-1 & 1074.65 & 1.21 & 
1074.65 & 1.19 & \bf{1063.00} & 
1.10\\SCA8-2 & 1056.87 & 0.99 & 
1056.87 & 1.03 & \bf{1040.60} & 
1.56\\SCA8-3 & 1031.08 & 1.40 & 
1031.08 & 1.39 & \bf{985.90} & 
4.58\\SCA8-4 & 1099.06 & 1.46 & 
1099.06 & 1.47 & \bf{1071.00} & 
2.62\\SCA8-5 & 1055.35 & 1.58 & 
1055.35 & 1.63 & \bf{1054.30} & 
0.10\\SCA8-6 & \bf{\underline{972.48}} & 1.60 & 
972.48 & 1.68 & 972.50 & 
-0.00\\SCA8-7 & 1075.42 & 1.70 & 
1088.28 & 1.65 & \bf{1059.70} & 
1.48\\SCA8-8 & 1092.02 & 1.50 & 
1092.02 & 1.46 & \bf{1082.70} & 
0.86\\SCA8-9 & \bf{\underline{1067.42}} & 1.17 & 
1067.42 & 1.15 & 1081.40 & 
-1.29\\CON3-0 & 624.96 & 1.58 & 
624.96 & 1.63 & \bf{616.50} & 
1.37\\CON3-1 & 557.38 & 1.37 & 
558.42 & 1.47 & \bf{555.60} & 
0.32\\CON3-2 & 524.07 & 1.18 & 
524.62 & 1.16 & \bf{521.40} & 
0.51\\CON3-3 & \bf{591.20} & 1.50 & 
593.38 & 1.54 & 591.20 & 0.00\\
CON3-4 & 589.32 & 1.37 & 
589.32 & 1.35 & \bf{589.30} & 
0.00\\CON3-5 & 576.43 & 1.42 & 
576.43 & 1.43 & \bf{563.70} & 
2.26\\CON3-6 & 505.26 & 1.74 & 
505.26 & 1.77 & \bf{499.20} & 
1.21\\CON3-7 & 578.41 & 1.21 & 
579.12 & 1.25 & \bf{577.50} & 
0.16\\CON3-8 & 524.30 & 1.29 & 
524.52 & 1.20 & \bf{523.10} & 
0.23\\CON3-9 & 588.48 & 1.22 & 
588.48 & 1.27 & \bf{578.20} & 
1.78\\CON8-0 & 879.00 & 1.45 & 
879.00 & 1.46 & \bf{858.90} & 
2.34\\CON8-1 & 758.26 & 1.32 & 
758.26 & 1.35 & \bf{740.90} & 
2.34\\CON8-2 & 716.53 & 2.00 & 
716.54 & 2.02 & \bf{714.30} & 
0.31\\CON8-3 & 817.57 & 1.42 & 
817.57 & 1.43 & \bf{812.30} & 
0.65\\CON8-4 & 781.64 & 1.55 & 
786.59 & 1.54 & \bf{770.10} & 
1.50\\CON8-5 & \bf{\underline{764.36}} & 1.37 & 
764.36 & 1.34 & 766.60 & 
-0.29\\CON8-6 & \bf{\underline{693.83}} & 1.73 & 
703.57 & 1.68 & 697.20 & 
-0.48\\CON8-7 & 823.43 & 1.18 & 
823.43 & 1.17 & \bf{814.80} & 
1.06\\CON8-8 & 799.32 & 1.59 & 
799.41 & 1.58 & \bf{771.30} & 
3.63\\CON8-9 & 816.12 & 1.52 & 
816.12 & 1.56 & \bf{815.10} & 
0.13\\[1ex]\hline
\end{tabular}
\label{table:nonlin}
\end{table} \clearpage
\begin{table}[ht]
\caption{Resultados de la ejecución de la metaheurística ACO, utilizando instancias de Dethloff con la configuración -n 2.0 -alpha 1.0 -beta 3.0 -q 10.6 -ro 0.015}
\centering
\small
\begin{tabular}{c c c c c c c}
\hline\hline
Instancia & Costo mínimo & Tiempo(seg.) & Costo promedio & Tiempo promedio(seg.) & Costo ACO & \%Gap \\ [0.5ex]
\hline
SCA3-0 & 640.55 & 1.34 & 
640.55 & 1.36 & \bf{636.10} & 
0.70\\SCA3-1 & \bf{\underline{697.84}} & 1.48 & 
697.84 & 1.53 & 700.10 & 
-0.32\\SCA3-2 & 664.18 & 1.42 & 
664.18 & 1.41 & \bf{659.30} & 
0.74\\SCA3-3 & 680.60 & 1.49 & 
680.60 & 1.47 & \bf{680.00} & 
0.09\\SCA3-4 & \bf{690.50} & 1.38 & 
690.50 & 1.45 & 690.50 & 0.00\\
SCA3-5 & \bf{\underline{665.04}} & 1.40 & 
665.39 & 1.41 & 671.10 & 
-0.90\\SCA3-6 & 655.19 & 1.37 & 
655.19 & 1.32 & \bf{651.10} & 
0.63\\SCA3-7 & 666.15 & 1.01 & 
666.26 & 0.99 & \bf{666.10} & 
0.01\\SCA3-8 & 721.45 & 1.21 & 
726.48 & 1.13 & \bf{719.50} & 
0.27\\SCA3-9 & \bf{681.00} & 1.06 & 
681.00 & 0.98 & 681.00 & 0.00\\
SCA8-0 & 991.07 & 1.48 & 
991.65 & 1.53 & \bf{961.60} & 
3.06\\SCA8-1 & 1074.65 & 1.12 & 
1074.68 & 1.17 & \bf{1063.00} & 
1.10\\SCA8-2 & 1056.87 & 1.04 & 
1056.87 & 1.01 & \bf{1040.60} & 
1.56\\SCA8-3 & 1031.08 & 1.53 & 
1031.08 & 1.54 & \bf{985.90} & 
4.58\\SCA8-4 & 1099.06 & 1.48 & 
1099.06 & 1.53 & \bf{1071.00} & 
2.62\\SCA8-5 & 1055.35 & 1.56 & 
1055.35 & 1.66 & \bf{1054.30} & 
0.10\\SCA8-6 & \bf{\underline{972.48}} & 1.66 & 
972.48 & 1.71 & 972.50 & 
-0.00\\SCA8-7 & 1075.42 & 1.68 & 
1088.28 & 1.62 & \bf{1059.70} & 
1.48\\SCA8-8 & 1092.02 & 1.36 & 
1092.02 & 1.37 & \bf{1082.70} & 
0.86\\SCA8-9 & \bf{\underline{1067.42}} & 1.08 & 
1067.42 & 1.14 & 1081.40 & 
-1.29\\CON3-0 & 624.96 & 1.63 & 
624.96 & 1.59 & \bf{616.50} & 
1.37\\CON3-1 & 560.75 & 1.53 & 
560.90 & 1.46 & \bf{555.60} & 
0.93\\CON3-2 & 524.07 & 1.10 & 
525.33 & 1.10 & \bf{521.40} & 
0.51\\CON3-3 & \bf{591.20} & 1.50 & 
593.38 & 1.47 & 591.20 & 0.00\\
CON3-4 & 589.32 & 1.38 & 
589.32 & 1.37 & \bf{589.30} & 
0.00\\CON3-5 & 569.88 & 1.42 & 
574.79 & 1.50 & \bf{563.70} & 
1.10\\CON3-6 & 505.26 & 1.82 & 
506.96 & 1.82 & \bf{499.20} & 
1.21\\CON3-7 & 578.41 & 1.22 & 
579.84 & 1.25 & \bf{577.50} & 
0.16\\CON3-8 & 524.30 & 1.17 & 
524.52 & 1.20 & \bf{523.10} & 
0.23\\CON3-9 & 588.48 & 1.31 & 
588.48 & 1.32 & \bf{578.20} & 
1.78\\CON8-0 & 879.00 & 1.46 & 
879.00 & 1.43 & \bf{858.90} & 
2.34\\CON8-1 & 758.26 & 1.33 & 
758.26 & 1.35 & \bf{740.90} & 
2.34\\CON8-2 & 716.53 & 2.10 & 
716.54 & 2.07 & \bf{714.30} & 
0.31\\CON8-3 & 817.57 & 1.53 & 
817.57 & 1.46 & \bf{812.30} & 
0.65\\CON8-4 & 781.64 & 1.61 & 
787.89 & 1.64 & \bf{770.10} & 
1.50\\CON8-5 & \bf{\underline{764.36}} & 1.40 & 
764.36 & 1.37 & 766.60 & 
-0.29\\CON8-6 & 705.61 & 1.68 & 
707.20 & 1.66 & \bf{697.20} & 
1.21\\CON8-7 & 822.42 & 1.18 & 
822.92 & 1.17 & \bf{814.80} & 
0.94\\CON8-8 & 799.32 & 1.48 & 
799.46 & 1.59 & \bf{771.30} & 
3.63\\CON8-9 & 816.12 & 1.63 & 
816.12 & 1.60 & \bf{815.10} & 
0.13\\[1ex]\hline
\end{tabular}
\label{table:nonlin}
\end{table} \clearpage
\begin{table}[ht]
\caption{Resultados de la ejecución de la metaheurística ACO, utilizando instancias de Dethloff con la configuración -n 2.0 -alpha 1.0 -beta 3.0 -q 10.7 -ro 0.015}
\centering
\small
\begin{tabular}{c c c c c c c}
\hline\hline
Instancia & Costo mínimo & Tiempo(seg.) & Costo promedio & Tiempo promedio(seg.) & Costo ACO & \%Gap \\ [0.5ex]
\hline
SCA3-0 & 640.55 & 1.32 & 
640.55 & 1.39 & \bf{636.10} & 
0.70\\SCA3-1 & \bf{\underline{697.84}} & 1.52 & 
697.84 & 1.53 & 700.10 & 
-0.32\\SCA3-2 & 659.34 & 1.26 & 
662.97 & 1.35 & \bf{659.30} & 
0.01\\SCA3-3 & 680.60 & 1.44 & 
680.96 & 1.48 & \bf{680.00} & 
0.09\\SCA3-4 & \bf{690.50} & 1.36 & 
690.50 & 1.42 & 690.50 & 0.00\\
SCA3-5 & \bf{\underline{665.04}} & 1.48 & 
666.14 & 1.44 & 671.10 & 
-0.90\\SCA3-6 & 655.19 & 1.34 & 
655.19 & 1.33 & \bf{651.10} & 
0.63\\SCA3-7 & 666.15 & 0.97 & 
666.15 & 0.99 & \bf{666.10} & 
0.01\\SCA3-8 & 721.45 & 1.07 & 
725.24 & 1.16 & \bf{719.50} & 
0.27\\SCA3-9 & \bf{681.00} & 0.98 & 
681.00 & 0.97 & 681.00 & 0.00\\
SCA8-0 & 991.07 & 1.55 & 
992.23 & 1.49 & \bf{961.60} & 
3.06\\SCA8-1 & 1074.65 & 1.25 & 
1074.65 & 1.22 & \bf{1063.00} & 
1.10\\SCA8-2 & 1056.87 & 1.03 & 
1056.87 & 1.02 & \bf{1040.60} & 
1.56\\SCA8-3 & 1031.08 & 1.42 & 
1031.08 & 1.47 & \bf{985.90} & 
4.58\\SCA8-4 & 1098.34 & 1.51 & 
1098.88 & 1.52 & \bf{1071.00} & 
2.55\\SCA8-5 & 1055.35 & 1.77 & 
1055.35 & 1.73 & \bf{1054.30} & 
0.10\\SCA8-6 & \bf{\underline{972.48}} & 1.72 & 
972.48 & 1.73 & 972.50 & 
-0.00\\SCA8-7 & 1092.57 & 1.69 & 
1092.57 & 1.66 & \bf{1059.70} & 
3.10\\SCA8-8 & 1092.02 & 1.36 & 
1092.02 & 1.42 & \bf{1082.70} & 
0.86\\SCA8-9 & \bf{\underline{1067.42}} & 1.16 & 
1067.42 & 1.18 & 1081.40 & 
-1.29\\CON3-0 & 624.96 & 1.65 & 
624.96 & 1.69 & \bf{616.50} & 
1.37\\CON3-1 & 557.38 & 1.43 & 
557.77 & 1.51 & \bf{555.60} & 
0.32\\CON3-2 & 524.07 & 1.10 & 
524.79 & 1.12 & \bf{521.40} & 
0.51\\CON3-3 & 594.11 & 1.53 & 
594.11 & 1.54 & \bf{591.20} & 
0.49\\CON3-4 & 589.32 & 1.24 & 
589.32 & 1.32 & \bf{589.30} & 
0.00\\CON3-5 & 569.88 & 1.32 & 
574.79 & 1.40 & \bf{563.70} & 
1.10\\CON3-6 & 504.15 & 1.74 & 
505.58 & 1.78 & \bf{499.20} & 
0.99\\CON3-7 & 578.41 & 1.44 & 
579.84 & 1.31 & \bf{577.50} & 
0.16\\CON3-8 & 524.30 & 1.28 & 
524.45 & 1.18 & \bf{523.10} & 
0.23\\CON3-9 & 588.48 & 1.30 & 
588.48 & 1.35 & \bf{578.20} & 
1.78\\CON8-0 & 879.00 & 1.41 & 
879.00 & 1.43 & \bf{858.90} & 
2.34\\CON8-1 & 758.26 & 1.37 & 
758.26 & 1.34 & \bf{740.90} & 
2.34\\CON8-2 & 716.53 & 2.00 & 
717.20 & 1.97 & \bf{714.30} & 
0.31\\CON8-3 & 817.57 & 1.40 & 
817.57 & 1.40 & \bf{812.30} & 
0.65\\CON8-4 & 781.64 & 1.58 & 
787.37 & 1.61 & \bf{770.10} & 
1.50\\CON8-5 & \bf{\underline{764.36}} & 1.42 & 
764.36 & 1.38 & 766.60 & 
-0.29\\CON8-6 & 705.61 & 1.96 & 
707.31 & 1.79 & \bf{697.20} & 
1.21\\CON8-7 & 822.42 & 1.22 & 
823.18 & 1.19 & \bf{814.80} & 
0.94\\CON8-8 & 799.16 & 1.54 & 
799.24 & 1.58 & \bf{771.30} & 
3.61\\CON8-9 & 816.12 & 1.49 & 
817.40 & 1.61 & \bf{815.10} & 
0.13\\[1ex]\hline
\end{tabular}
\label{table:nonlin}
\end{table} \clearpage
\begin{table}[ht]
\caption{Resultados de la ejecución de la metaheurística ACO, utilizando instancias de Dethloff con la configuración -n 2.0 -alpha 1.0 -beta 3.0 -q 10.8 -ro 0.015}
\centering
\small
\begin{tabular}{c c c c c c c}
\hline\hline
Instancia & Costo mínimo & Tiempo(seg.) & Costo promedio & Tiempo promedio(seg.) & Costo ACO & \%Gap \\ [0.5ex]
\hline
SCA3-0 & 640.55 & 1.29 & 
640.55 & 1.31 & \bf{636.10} & 
0.70\\SCA3-1 & \bf{\underline{697.84}} & 1.46 & 
697.84 & 1.49 & 700.10 & 
-0.32\\SCA3-2 & 664.18 & 1.32 & 
664.18 & 1.36 & \bf{659.30} & 
0.74\\SCA3-3 & 680.60 & 1.46 & 
680.78 & 1.50 & \bf{680.00} & 
0.09\\SCA3-4 & \bf{690.50} & 1.37 & 
690.50 & 1.41 & 690.50 & 0.00\\
SCA3-5 & \bf{\underline{665.04}} & 1.42 & 
665.39 & 1.42 & 671.10 & 
-0.90\\SCA3-6 & 655.19 & 1.32 & 
655.19 & 1.31 & \bf{651.10} & 
0.63\\SCA3-7 & 666.15 & 0.97 & 
666.15 & 1.02 & \bf{666.10} & 
0.01\\SCA3-8 & 721.45 & 1.13 & 
722.95 & 1.13 & \bf{719.50} & 
0.27\\SCA3-9 & \bf{681.00} & 0.92 & 
681.00 & 0.97 & 681.00 & 0.00\\
SCA8-0 & 991.07 & 1.48 & 
991.07 & 1.50 & \bf{961.60} & 
3.06\\SCA8-1 & 1074.65 & 1.22 & 
1074.65 & 1.19 & \bf{1063.00} & 
1.10\\SCA8-2 & 1056.87 & 1.04 & 
1056.87 & 1.03 & \bf{1040.60} & 
1.56\\SCA8-3 & 1031.08 & 1.44 & 
1031.08 & 1.43 & \bf{985.90} & 
4.58\\SCA8-4 & 1099.06 & 1.45 & 
1099.06 & 1.46 & \bf{1071.00} & 
2.62\\SCA8-5 & 1055.35 & 1.63 & 
1055.35 & 1.63 & \bf{1054.30} & 
0.10\\SCA8-6 & \bf{\underline{972.48}} & 1.76 & 
972.48 & 1.66 & 972.50 & 
-0.00\\SCA8-7 & 1092.57 & 1.57 & 
1092.57 & 1.65 & \bf{1059.70} & 
3.10\\SCA8-8 & \bf{\underline{1082.11}} & 1.46 & 
1089.41 & 1.46 & 1082.70 & 
-0.05\\SCA8-9 & \bf{\underline{1067.42}} & 1.17 & 
1067.42 & 1.14 & 1081.40 & 
-1.29\\CON3-0 & 624.96 & 1.62 & 
624.96 & 1.64 & \bf{616.50} & 
1.37\\CON3-1 & 557.38 & 1.38 & 
558.22 & 1.43 & \bf{555.60} & 
0.32\\CON3-2 & \bf{\underline{521.38}} & 1.09 & 
524.12 & 1.11 & 521.40 & 
-0.00\\CON3-3 & 594.11 & 1.57 & 
594.11 & 1.52 & \bf{591.20} & 
0.49\\CON3-4 & 589.32 & 1.30 & 
589.32 & 1.34 & \bf{589.30} & 
0.00\\CON3-5 & 569.88 & 1.52 & 
574.79 & 1.59 & \bf{563.70} & 
1.10\\CON3-6 & 505.26 & 1.76 & 
506.43 & 1.79 & \bf{499.20} & 
1.21\\CON3-7 & 578.41 & 1.23 & 
579.12 & 1.23 & \bf{577.50} & 
0.16\\CON3-8 & 524.59 & 1.23 & 
524.59 & 1.21 & \bf{523.10} & 
0.28\\CON3-9 & 588.48 & 1.24 & 
588.48 & 1.28 & \bf{578.20} & 
1.78\\CON8-0 & 879.00 & 1.48 & 
879.00 & 1.50 & \bf{858.90} & 
2.34\\CON8-1 & 758.26 & 1.42 & 
758.26 & 1.40 & \bf{740.90} & 
2.34\\CON8-2 & 716.53 & 2.00 & 
716.54 & 2.02 & \bf{714.30} & 
0.31\\CON8-3 & 817.57 & 1.47 & 
817.57 & 1.46 & \bf{812.30} & 
0.65\\CON8-4 & 781.64 & 1.55 & 
787.89 & 1.52 & \bf{770.10} & 
1.50\\CON8-5 & \bf{\underline{764.36}} & 1.44 & 
764.36 & 1.39 & 766.60 & 
-0.29\\CON8-6 & \bf{\underline{693.83}} & 1.74 & 
703.12 & 1.70 & 697.20 & 
-0.48\\CON8-7 & 822.42 & 1.22 & 
822.92 & 1.21 & \bf{814.80} & 
0.94\\CON8-8 & 799.51 & 1.65 & 
799.51 & 1.59 & \bf{771.30} & 
3.66\\CON8-9 & 816.12 & 1.47 & 
817.40 & 1.53 & \bf{815.10} & 
0.13\\[1ex]\hline
\end{tabular}
\label{table:nonlin}
\end{table} \clearpage
\begin{table}[ht]
\caption{Resultados de la ejecución de la metaheurística ACO, utilizando instancias de Dethloff con la configuración -n 2.0 -alpha 1.0 -beta 3.0 -q 10.9 -ro 0.015}
\centering
\small
\begin{tabular}{c c c c c c c}
\hline\hline
Instancia & Costo mínimo & Tiempo(seg.) & Costo promedio & Tiempo promedio(seg.) & Costo ACO & \%Gap \\ [0.5ex]
\hline
SCA3-0 & 640.55 & 1.36 & 
640.55 & 1.36 & \bf{636.10} & 
0.70\\SCA3-1 & \bf{\underline{697.84}} & 1.46 & 
697.84 & 1.44 & 700.10 & 
-0.32\\SCA3-2 & 664.18 & 1.34 & 
664.18 & 1.31 & \bf{659.30} & 
0.74\\SCA3-3 & 680.60 & 1.44 & 
680.78 & 1.51 & \bf{680.00} & 
0.09\\SCA3-4 & \bf{690.50} & 1.36 & 
690.50 & 1.43 & 690.50 & 0.00\\
SCA3-5 & \bf{\underline{665.04}} & 1.46 & 
665.04 & 1.46 & 671.10 & 
-0.90\\SCA3-6 & 655.19 & 1.25 & 
655.30 & 1.30 & \bf{651.10} & 
0.63\\SCA3-7 & 666.15 & 1.06 & 
666.15 & 0.99 & \bf{666.10} & 
0.01\\SCA3-8 & 721.45 & 1.11 & 
726.48 & 1.09 & \bf{719.50} & 
0.27\\SCA3-9 & \bf{681.00} & 1.05 & 
681.00 & 0.94 & 681.00 & 0.00\\
SCA8-0 & 991.07 & 1.43 & 
998.16 & 1.49 & \bf{961.60} & 
3.06\\SCA8-1 & 1074.65 & 1.18 & 
1074.71 & 1.20 & \bf{1063.00} & 
1.10\\SCA8-2 & 1056.87 & 1.05 & 
1056.87 & 1.05 & \bf{1040.60} & 
1.56\\SCA8-3 & 1031.08 & 1.50 & 
1031.08 & 1.44 & \bf{985.90} & 
4.58\\SCA8-4 & 1098.34 & 1.47 & 
1098.88 & 1.52 & \bf{1071.00} & 
2.55\\SCA8-5 & 1055.35 & 1.67 & 
1055.35 & 1.64 & \bf{1054.30} & 
0.10\\SCA8-6 & \bf{\underline{972.48}} & 1.62 & 
972.48 & 1.67 & 972.50 & 
-0.00\\SCA8-7 & 1092.57 & 1.67 & 
1092.57 & 1.67 & \bf{1059.70} & 
3.10\\SCA8-8 & 1091.49 & 1.44 & 
1091.89 & 1.43 & \bf{1082.70} & 
0.81\\SCA8-9 & \bf{\underline{1067.42}} & 1.20 & 
1067.42 & 1.18 & 1081.40 & 
-1.29\\CON3-0 & 624.96 & 1.72 & 
624.96 & 1.68 & \bf{616.50} & 
1.37\\CON3-1 & 557.38 & 1.49 & 
558.01 & 1.49 & \bf{555.60} & 
0.32\\CON3-2 & 524.07 & 1.09 & 
526.43 & 1.16 & \bf{521.40} & 
0.51\\CON3-3 & \bf{591.20} & 1.55 & 
593.38 & 1.57 & 591.20 & 0.00\\
CON3-4 & 589.32 & 1.47 & 
589.32 & 1.38 & \bf{589.30} & 
0.00\\CON3-5 & 576.43 & 1.41 & 
576.43 & 1.35 & \bf{563.70} & 
2.26\\CON3-6 & 505.26 & 1.86 & 
507.61 & 1.80 & \bf{499.20} & 
1.21\\CON3-7 & 578.41 & 1.21 & 
579.12 & 1.25 & \bf{577.50} & 
0.16\\CON3-8 & 524.30 & 1.21 & 
524.45 & 1.17 & \bf{523.10} & 
0.23\\CON3-9 & 588.48 & 1.39 & 
588.48 & 1.32 & \bf{578.20} & 
1.78\\CON8-0 & 879.00 & 1.42 & 
879.00 & 1.45 & \bf{858.90} & 
2.34\\CON8-1 & 758.26 & 1.35 & 
758.26 & 1.40 & \bf{740.90} & 
2.34\\CON8-2 & 716.53 & 2.04 & 
716.55 & 1.99 & \bf{714.30} & 
0.31\\CON8-3 & 817.57 & 1.45 & 
817.57 & 1.50 & \bf{812.30} & 
0.65\\CON8-4 & 778.60 & 1.62 & 
785.83 & 1.60 & \bf{770.10} & 
1.10\\CON8-5 & \bf{\underline{764.36}} & 1.40 & 
764.36 & 1.38 & 766.60 & 
-0.29\\CON8-6 & \bf{\underline{693.83}} & 1.63 & 
702.66 & 1.69 & 697.20 & 
-0.48\\CON8-7 & 822.42 & 1.26 & 
823.18 & 1.21 & \bf{814.80} & 
0.94\\CON8-8 & 796.81 & 1.68 & 
798.69 & 1.58 & \bf{771.30} & 
3.31\\CON8-9 & 816.12 & 1.56 & 
816.12 & 1.59 & \bf{815.10} & 
0.13\\[1ex]\hline
\end{tabular}
\label{table:nonlin}
\end{table} \clearpage
\begin{table}[ht]
\caption{Resultados de la ejecución de la metaheurística ACO, utilizando instancias de Dethloff con la configuración -n 2.0 -alpha 1.0 -beta 3.0 -q 11.0 -ro 0.015}
\centering
\small
\begin{tabular}{c c c c c c c}
\hline\hline
Instancia & Costo mínimo & Tiempo(seg.) & Costo promedio & Tiempo promedio(seg.) & Costo ACO & \%Gap \\ [0.5ex]
\hline
SCA3-0 & 640.55 & 1.46 & 
640.55 & 1.41 & \bf{636.10} & 
0.70\\SCA3-1 & \bf{\underline{697.84}} & 1.43 & 
697.84 & 1.55 & 700.10 & 
-0.32\\SCA3-2 & 659.34 & 1.37 & 
662.97 & 1.33 & \bf{659.30} & 
0.01\\SCA3-3 & 680.60 & 1.60 & 
680.78 & 1.54 & \bf{680.00} & 
0.09\\SCA3-4 & \bf{690.50} & 1.32 & 
690.50 & 1.41 & 690.50 & 0.00\\
SCA3-5 & \bf{\underline{665.04}} & 1.45 & 
665.04 & 1.54 & 671.10 & 
-0.90\\SCA3-6 & 655.19 & 1.31 & 
655.30 & 1.33 & \bf{651.10} & 
0.63\\SCA3-7 & 666.15 & 0.99 & 
666.15 & 1.02 & \bf{666.10} & 
0.01\\SCA3-8 & 721.45 & 1.20 & 
725.24 & 1.16 & \bf{719.50} & 
0.27\\SCA3-9 & \bf{681.00} & 0.93 & 
681.00 & 0.96 & 681.00 & 0.00\\
SCA8-0 & 991.07 & 1.58 & 
991.65 & 1.56 & \bf{961.60} & 
3.06\\SCA8-1 & 1074.65 & 1.42 & 
1074.65 & 1.25 & \bf{1063.00} & 
1.10\\SCA8-2 & 1056.87 & 1.02 & 
1056.87 & 1.02 & \bf{1040.60} & 
1.56\\SCA8-3 & 1031.08 & 1.42 & 
1031.08 & 1.42 & \bf{985.90} & 
4.58\\SCA8-4 & 1099.06 & 1.60 & 
1099.17 & 1.52 & \bf{1071.00} & 
2.62\\SCA8-5 & 1055.35 & 1.69 & 
1055.35 & 1.64 & \bf{1054.30} & 
0.10\\SCA8-6 & \bf{\underline{972.48}} & 1.60 & 
972.48 & 1.67 & 972.50 & 
-0.00\\SCA8-7 & 1092.57 & 1.66 & 
1092.57 & 1.63 & \bf{1059.70} & 
3.10\\SCA8-8 & 1091.49 & 1.48 & 
1091.89 & 1.47 & \bf{1082.70} & 
0.81\\SCA8-9 & \bf{\underline{1067.42}} & 1.14 & 
1067.42 & 1.17 & 1081.40 & 
-1.29\\CON3-0 & 624.96 & 1.60 & 
624.96 & 1.59 & \bf{616.50} & 
1.37\\CON3-1 & 557.38 & 1.44 & 
558.22 & 1.46 & \bf{555.60} & 
0.32\\CON3-2 & 524.07 & 1.09 & 
526.99 & 1.09 & \bf{521.40} & 
0.51\\CON3-3 & \bf{591.20} & 1.49 & 
593.09 & 1.49 & 591.20 & 0.00\\
CON3-4 & \bf{\underline{588.79}} & 1.25 & 
589.19 & 1.33 & 589.30 & 
-0.09\\CON3-5 & 569.88 & 1.38 & 
574.79 & 1.43 & \bf{563.70} & 
1.10\\CON3-6 & 505.26 & 1.82 & 
505.26 & 1.77 & \bf{499.20} & 
1.21\\CON3-7 & 578.41 & 1.18 & 
578.41 & 1.23 & \bf{577.50} & 
0.16\\CON3-8 & 524.59 & 1.21 & 
524.59 & 1.20 & \bf{523.10} & 
0.28\\CON3-9 & 588.48 & 1.20 & 
588.48 & 1.29 & \bf{578.20} & 
1.78\\CON8-0 & 879.00 & 1.51 & 
879.00 & 1.49 & \bf{858.90} & 
2.34\\CON8-1 & 758.26 & 1.44 & 
758.26 & 1.37 & \bf{740.90} & 
2.34\\CON8-2 & 716.53 & 2.02 & 
716.54 & 1.98 & \bf{714.30} & 
0.31\\CON8-3 & 817.57 & 1.46 & 
817.57 & 1.45 & \bf{812.30} & 
0.65\\CON8-4 & 781.64 & 1.52 & 
786.59 & 1.55 & \bf{770.10} & 
1.50\\CON8-5 & \bf{\underline{764.36}} & 1.41 & 
764.36 & 1.40 & 766.60 & 
-0.29\\CON8-6 & 705.61 & 1.76 & 
705.61 & 1.72 & \bf{697.20} & 
1.21\\CON8-7 & 822.42 & 1.21 & 
823.18 & 1.19 & \bf{814.80} & 
0.94\\CON8-8 & 796.81 & 1.46 & 
798.65 & 1.48 & \bf{771.30} & 
3.31\\CON8-9 & 816.12 & 1.46 & 
816.12 & 1.53 & \bf{815.10} & 
0.13\\[1ex]\hline
\end{tabular}
\label{table:nonlin}
\end{table} \clearpage
\begin{table}[ht]
\caption{Resultados de la ejecución de la metaheurística ACO, utilizando instancias de Dethloff con la configuración -n 2.0 -alpha 1.0 -beta 3.0 -q 11.1 -ro 0.015}
\centering
\small
\begin{tabular}{c c c c c c c}
\hline\hline
Instancia & Costo mínimo & Tiempo(seg.) & Costo promedio & Tiempo promedio(seg.) & Costo ACO & \%Gap \\ [0.5ex]
\hline
SCA3-0 & 640.55 & 1.42 & 
640.55 & 1.38 & \bf{636.10} & 
0.70\\SCA3-1 & \bf{\underline{697.84}} & 1.47 & 
697.84 & 1.52 & 700.10 & 
-0.32\\SCA3-2 & 659.34 & 1.30 & 
662.97 & 1.30 & \bf{659.30} & 
0.01\\SCA3-3 & 680.60 & 1.49 & 
680.74 & 1.48 & \bf{680.00} & 
0.09\\SCA3-4 & \bf{690.50} & 1.82 & 
690.50 & 1.51 & 690.50 & 0.00\\
SCA3-5 & \bf{\underline{665.04}} & 1.42 & 
665.04 & 1.40 & 671.10 & 
-0.90\\SCA3-6 & 655.19 & 1.37 & 
655.19 & 1.37 & \bf{651.10} & 
0.63\\SCA3-7 & 666.15 & 0.95 & 
666.15 & 0.96 & \bf{666.10} & 
0.01\\SCA3-8 & 721.45 & 1.22 & 
724.59 & 1.17 & \bf{719.50} & 
0.27\\SCA3-9 & \bf{681.00} & 1.00 & 
681.00 & 0.97 & 681.00 & 0.00\\
SCA8-0 & 991.07 & 1.53 & 
992.23 & 1.49 & \bf{961.60} & 
3.06\\SCA8-1 & 1074.65 & 1.10 & 
1074.65 & 1.19 & \bf{1063.00} & 
1.10\\SCA8-2 & 1056.87 & 1.10 & 
1056.87 & 1.05 & \bf{1040.60} & 
1.56\\SCA8-3 & 1031.08 & 1.50 & 
1031.08 & 1.48 & \bf{985.90} & 
4.58\\SCA8-4 & 1098.34 & 1.49 & 
1098.88 & 1.49 & \bf{1071.00} & 
2.55\\SCA8-5 & 1055.35 & 1.67 & 
1055.35 & 1.66 & \bf{1054.30} & 
0.10\\SCA8-6 & \bf{\underline{972.48}} & 1.74 & 
985.03 & 1.70 & 972.50 & 
-0.00\\SCA8-7 & 1092.57 & 1.56 & 
1092.57 & 1.60 & \bf{1059.70} & 
3.10\\SCA8-8 & 1092.02 & 1.40 & 
1092.02 & 1.65 & \bf{1082.70} & 
0.86\\SCA8-9 & \bf{\underline{1067.42}} & 1.24 & 
1067.42 & 1.18 & 1081.40 & 
-1.29\\CON3-0 & 624.96 & 1.56 & 
624.96 & 1.62 & \bf{616.50} & 
1.37\\CON3-1 & 557.38 & 1.40 & 
557.38 & 1.44 & \bf{555.60} & 
0.32\\CON3-2 & 524.07 & 1.02 & 
525.60 & 1.10 & \bf{521.40} & 
0.51\\CON3-3 & \bf{591.20} & 1.56 & 
592.65 & 1.53 & 591.20 & 0.00\\
CON3-4 & 589.32 & 1.33 & 
589.32 & 1.30 & \bf{589.30} & 
0.00\\CON3-5 & 576.43 & 1.53 & 
576.43 & 1.50 & \bf{563.70} & 
2.26\\CON3-6 & 505.26 & 1.74 & 
505.26 & 1.73 & \bf{499.20} & 
1.21\\CON3-7 & 578.41 & 1.25 & 
579.12 & 1.26 & \bf{577.50} & 
0.16\\CON3-8 & 524.30 & 1.14 & 
524.52 & 1.20 & \bf{523.10} & 
0.23\\CON3-9 & 588.48 & 1.29 & 
588.48 & 1.27 & \bf{578.20} & 
1.78\\CON8-0 & 879.00 & 1.46 & 
879.00 & 1.54 & \bf{858.90} & 
2.34\\CON8-1 & 758.26 & 1.32 & 
758.26 & 1.35 & \bf{740.90} & 
2.34\\CON8-2 & 716.56 & 1.99 & 
716.56 & 2.19 & \bf{714.30} & 
0.32\\CON8-3 & 817.57 & 1.44 & 
817.57 & 1.41 & \bf{812.30} & 
0.65\\CON8-4 & 781.64 & 1.53 & 
789.45 & 1.54 & \bf{770.10} & 
1.50\\CON8-5 & \bf{\underline{764.36}} & 1.38 & 
764.36 & 1.38 & 766.60 & 
-0.29\\CON8-6 & 705.61 & 1.68 & 
706.18 & 1.67 & \bf{697.20} & 
1.21\\CON8-7 & 822.42 & 1.22 & 
823.18 & 1.18 & \bf{814.80} & 
0.94\\CON8-8 & 799.16 & 1.55 & 
799.34 & 1.57 & \bf{771.30} & 
3.61\\CON8-9 & 816.12 & 1.72 & 
816.12 & 1.64 & \bf{815.10} & 
0.13\\[1ex]\hline
\end{tabular}
\label{table:nonlin}
\end{table} \clearpage
\begin{table}[ht]
\caption{Resultados de la ejecución de la metaheurística ACO, utilizando instancias de Dethloff con la configuración -n 2.0 -alpha 1.0 -beta 3.0 -q 11.2 -ro 0.015}
\centering
\small
\begin{tabular}{c c c c c c c}
\hline\hline
Instancia & Costo mínimo & Tiempo(seg.) & Costo promedio & Tiempo promedio(seg.) & Costo ACO & \%Gap \\ [0.5ex]
\hline
SCA3-0 & 640.55 & 1.38 & 
640.55 & 1.37 & \bf{636.10} & 
0.70\\SCA3-1 & \bf{\underline{697.84}} & 1.58 & 
697.84 & 1.55 & 700.10 & 
-0.32\\SCA3-2 & 659.34 & 1.28 & 
661.76 & 1.33 & \bf{659.30} & 
0.01\\SCA3-3 & 680.60 & 1.56 & 
680.78 & 1.50 & \bf{680.00} & 
0.09\\SCA3-4 & \bf{690.50} & 1.48 & 
690.50 & 1.43 & 690.50 & 0.00\\
SCA3-5 & \bf{\underline{665.04}} & 1.46 & 
665.04 & 1.46 & 671.10 & 
-0.90\\SCA3-6 & 655.19 & 1.34 & 
655.19 & 1.37 & \bf{651.10} & 
0.63\\SCA3-7 & 666.15 & 0.94 & 
666.38 & 1.00 & \bf{666.10} & 
0.01\\SCA3-8 & 721.45 & 1.22 & 
725.24 & 1.17 & \bf{719.50} & 
0.27\\SCA3-9 & \bf{681.00} & 0.92 & 
681.00 & 0.94 & 681.00 & 0.00\\
SCA8-0 & 991.07 & 1.51 & 
991.65 & 1.47 & \bf{961.60} & 
3.06\\SCA8-1 & 1074.65 & 1.22 & 
1074.68 & 1.20 & \bf{1063.00} & 
1.10\\SCA8-2 & 1056.87 & 1.04 & 
1056.87 & 1.05 & \bf{1040.60} & 
1.56\\SCA8-3 & 1031.08 & 1.50 & 
1031.08 & 1.44 & \bf{985.90} & 
4.58\\SCA8-4 & 1099.06 & 1.46 & 
1099.06 & 1.45 & \bf{1071.00} & 
2.62\\SCA8-5 & 1055.35 & 1.68 & 
1055.35 & 1.71 & \bf{1054.30} & 
0.10\\SCA8-6 & \bf{\underline{972.48}} & 1.69 & 
976.70 & 1.74 & 972.50 & 
-0.00\\SCA8-7 & 1092.57 & 1.67 & 
1092.57 & 1.61 & \bf{1059.70} & 
3.10\\SCA8-8 & 1091.49 & 1.44 & 
1091.89 & 1.43 & \bf{1082.70} & 
0.81\\SCA8-9 & \bf{\underline{1067.42}} & 1.09 & 
1067.42 & 1.15 & 1081.40 & 
-1.29\\CON3-0 & 624.96 & 1.61 & 
624.96 & 1.65 & \bf{616.50} & 
1.37\\CON3-1 & 557.38 & 1.55 & 
558.66 & 1.54 & \bf{555.60} & 
0.32\\CON3-2 & 524.07 & 1.00 & 
525.40 & 1.08 & \bf{521.40} & 
0.51\\CON3-3 & \bf{591.20} & 1.47 & 
593.38 & 1.50 & 591.20 & 0.00\\
CON3-4 & 589.32 & 1.28 & 
589.32 & 1.31 & \bf{589.30} & 
0.00\\CON3-5 & 576.43 & 1.37 & 
576.88 & 1.43 & \bf{563.70} & 
2.26\\CON3-6 & 505.26 & 1.79 & 
506.43 & 1.83 & \bf{499.20} & 
1.21\\CON3-7 & 578.41 & 1.30 & 
578.41 & 1.26 & \bf{577.50} & 
0.16\\CON3-8 & 524.30 & 1.28 & 
524.45 & 1.24 & \bf{523.10} & 
0.23\\CON3-9 & 588.48 & 1.33 & 
588.48 & 1.33 & \bf{578.20} & 
1.78\\CON8-0 & 879.00 & 1.48 & 
879.00 & 1.46 & \bf{858.90} & 
2.34\\CON8-1 & 758.26 & 1.36 & 
758.26 & 1.32 & \bf{740.90} & 
2.34\\CON8-2 & 716.53 & 2.31 & 
716.54 & 2.06 & \bf{714.30} & 
0.31\\CON8-3 & 817.57 & 1.42 & 
817.57 & 1.42 & \bf{812.30} & 
0.65\\CON8-4 & 781.64 & 1.56 & 
789.45 & 1.58 & \bf{770.10} & 
1.50\\CON8-5 & \bf{\underline{764.36}} & 1.31 & 
764.36 & 1.32 & 766.60 & 
-0.29\\CON8-6 & 705.61 & 1.67 & 
706.77 & 1.73 & \bf{697.20} & 
1.21\\CON8-7 & 822.42 & 1.26 & 
823.18 & 1.21 & \bf{814.80} & 
0.94\\CON8-8 & 796.81 & 1.56 & 
798.75 & 1.63 & \bf{771.30} & 
3.31\\CON8-9 & 816.12 & 1.52 & 
817.40 & 1.58 & \bf{815.10} & 
0.13\\[1ex]\hline
\end{tabular}
\label{table:nonlin}
\end{table} \clearpage
\begin{table}[ht]
\caption{Resultados de la ejecución de la metaheurística ACO, utilizando instancias de Dethloff con la configuración -n 2.0 -alpha 1.0 -beta 3.0 -q 11.3 -ro 0.015}
\centering
\small
\begin{tabular}{c c c c c c c}
\hline\hline
Instancia & Costo mínimo & Tiempo(seg.) & Costo promedio & Tiempo promedio(seg.) & Costo ACO & \%Gap \\ [0.5ex]
\hline
SCA3-0 & 640.55 & 1.36 & 
640.55 & 1.38 & \bf{636.10} & 
0.70\\SCA3-1 & \bf{\underline{697.84}} & 1.45 & 
697.84 & 1.47 & 700.10 & 
-0.32\\SCA3-2 & 664.18 & 1.32 & 
664.18 & 1.33 & \bf{659.30} & 
0.74\\SCA3-3 & 680.60 & 1.44 & 
681.13 & 1.50 & \bf{680.00} & 
0.09\\SCA3-4 & \bf{690.50} & 1.40 & 
690.50 & 1.43 & 690.50 & 0.00\\
SCA3-5 & \bf{\underline{665.04}} & 1.38 & 
665.39 & 1.41 & 671.10 & 
-0.90\\SCA3-6 & 655.19 & 1.30 & 
655.19 & 1.35 & \bf{651.10} & 
0.63\\SCA3-7 & 666.15 & 1.06 & 
666.15 & 1.08 & \bf{666.10} & 
0.01\\SCA3-8 & 721.45 & 1.08 & 
724.59 & 1.12 & \bf{719.50} & 
0.27\\SCA3-9 & \bf{681.00} & 0.94 & 
681.00 & 0.95 & 681.00 & 0.00\\
SCA8-0 & 991.07 & 1.46 & 
991.65 & 1.51 & \bf{961.60} & 
3.06\\SCA8-1 & 1074.65 & 1.21 & 
1074.71 & 1.21 & \bf{1063.00} & 
1.10\\SCA8-2 & 1056.87 & 1.04 & 
1056.87 & 1.01 & \bf{1040.60} & 
1.56\\SCA8-3 & 1031.08 & 1.47 & 
1031.08 & 1.46 & \bf{985.90} & 
4.58\\SCA8-4 & 1098.34 & 1.74 & 
1098.70 & 1.57 & \bf{1071.00} & 
2.55\\SCA8-5 & 1055.35 & 1.66 & 
1055.35 & 1.70 & \bf{1054.30} & 
0.10\\SCA8-6 & \bf{\underline{972.48}} & 1.76 & 
972.48 & 1.68 & 972.50 & 
-0.00\\SCA8-7 & 1092.57 & 1.64 & 
1092.57 & 1.62 & \bf{1059.70} & 
3.10\\SCA8-8 & 1085.93 & 1.44 & 
1090.37 & 1.45 & \bf{1082.70} & 
0.30\\SCA8-9 & \bf{\underline{1067.42}} & 1.18 & 
1067.42 & 1.16 & 1081.40 & 
-1.29\\CON3-0 & 624.96 & 1.57 & 
624.96 & 1.60 & \bf{616.50} & 
1.37\\CON3-1 & 557.38 & 1.46 & 
557.38 & 1.45 & \bf{555.60} & 
0.32\\CON3-2 & 525.84 & 1.11 & 
526.35 & 1.11 & \bf{521.40} & 
0.85\\CON3-3 & \bf{591.20} & 1.50 & 
591.93 & 1.51 & 591.20 & 0.00\\
CON3-4 & 589.32 & 1.39 & 
589.32 & 1.39 & \bf{589.30} & 
0.00\\CON3-5 & 569.88 & 1.46 & 
574.79 & 1.46 & \bf{563.70} & 
1.10\\CON3-6 & 505.26 & 1.84 & 
505.26 & 1.82 & \bf{499.20} & 
1.21\\CON3-7 & 578.41 & 1.33 & 
579.12 & 1.28 & \bf{577.50} & 
0.16\\CON3-8 & 524.30 & 1.17 & 
524.52 & 1.22 & \bf{523.10} & 
0.23\\CON3-9 & 588.48 & 1.30 & 
588.48 & 1.31 & \bf{578.20} & 
1.78\\CON8-0 & 879.00 & 1.51 & 
879.00 & 1.47 & \bf{858.90} & 
2.34\\CON8-1 & 758.26 & 1.31 & 
758.26 & 1.35 & \bf{740.90} & 
2.34\\CON8-2 & 716.53 & 2.12 & 
716.55 & 2.02 & \bf{714.30} & 
0.31\\CON8-3 & 817.57 & 1.41 & 
817.57 & 1.45 & \bf{812.30} & 
0.65\\CON8-4 & 789.98 & 1.54 & 
790.76 & 1.56 & \bf{770.10} & 
2.58\\CON8-5 & \bf{\underline{764.36}} & 1.29 & 
764.36 & 1.39 & 766.60 & 
-0.29\\CON8-6 & \bf{\underline{693.83}} & 1.74 & 
703.68 & 1.76 & 697.20 & 
-0.48\\CON8-7 & 822.42 & 1.15 & 
823.18 & 1.19 & \bf{814.80} & 
0.94\\CON8-8 & 799.51 & 1.58 & 
799.51 & 1.59 & \bf{771.30} & 
3.66\\CON8-9 & 816.12 & 1.67 & 
816.12 & 1.61 & \bf{815.10} & 
0.13\\[1ex]\hline
\end{tabular}
\label{table:nonlin}
\end{table} \clearpage
\begin{table}[ht]
\caption{Resultados de la ejecución de la metaheurística ACO, utilizando instancias de Dethloff con la configuración -n 2.0 -alpha 1.0 -beta 3.0 -q 11.4 -ro 0.015}
\centering
\small
\begin{tabular}{c c c c c c c}
\hline\hline
Instancia & Costo mínimo & Tiempo(seg.) & Costo promedio & Tiempo promedio(seg.) & Costo ACO & \%Gap \\ [0.5ex]
\hline
SCA3-0 & 640.55 & 1.40 & 
640.55 & 1.42 & \bf{636.10} & 
0.70\\SCA3-1 & \bf{\underline{697.84}} & 1.47 & 
697.84 & 1.48 & 700.10 & 
-0.32\\SCA3-2 & 659.34 & 1.53 & 
662.97 & 1.41 & \bf{659.30} & 
0.01\\SCA3-3 & 680.60 & 1.63 & 
680.96 & 1.55 & \bf{680.00} & 
0.09\\SCA3-4 & \bf{690.50} & 1.42 & 
690.50 & 1.41 & 690.50 & 0.00\\
SCA3-5 & \bf{\underline{665.04}} & 1.41 & 
668.60 & 1.39 & 671.10 & 
-0.90\\SCA3-6 & 655.19 & 1.32 & 
655.19 & 1.39 & \bf{651.10} & 
0.63\\SCA3-7 & 666.15 & 0.99 & 
666.15 & 1.01 & \bf{666.10} & 
0.01\\SCA3-8 & 721.45 & 1.14 & 
725.24 & 1.14 & \bf{719.50} & 
0.27\\SCA3-9 & \bf{681.00} & 1.08 & 
681.00 & 1.00 & 681.00 & 0.00\\
SCA8-0 & 991.07 & 1.55 & 
992.23 & 1.53 & \bf{961.60} & 
3.06\\SCA8-1 & 1074.65 & 1.26 & 
1074.65 & 1.23 & \bf{1063.00} & 
1.10\\SCA8-2 & 1056.87 & 0.96 & 
1056.87 & 0.99 & \bf{1040.60} & 
1.56\\SCA8-3 & 1031.08 & 1.47 & 
1031.08 & 1.45 & \bf{985.90} & 
4.58\\SCA8-4 & 1099.06 & 1.60 & 
1099.06 & 1.52 & \bf{1071.00} & 
2.62\\SCA8-5 & 1055.35 & 1.71 & 
1055.35 & 1.70 & \bf{1054.30} & 
0.10\\SCA8-6 & \bf{\underline{972.48}} & 1.62 & 
972.48 & 1.70 & 972.50 & 
-0.00\\SCA8-7 & 1092.57 & 1.59 & 
1092.57 & 1.61 & \bf{1059.70} & 
3.10\\SCA8-8 & 1091.49 & 1.43 & 
1091.76 & 1.43 & \bf{1082.70} & 
0.81\\SCA8-9 & \bf{\underline{1067.42}} & 1.21 & 
1067.42 & 1.16 & 1081.40 & 
-1.29\\CON3-0 & 624.96 & 1.75 & 
624.96 & 1.62 & \bf{616.50} & 
1.37\\CON3-1 & 557.38 & 1.37 & 
558.66 & 1.44 & \bf{555.60} & 
0.32\\CON3-2 & 524.07 & 1.10 & 
524.07 & 1.17 & \bf{521.40} & 
0.51\\CON3-3 & 594.11 & 1.66 & 
594.11 & 1.52 & \bf{591.20} & 
0.49\\CON3-4 & 589.32 & 1.28 & 
589.32 & 1.31 & \bf{589.30} & 
0.00\\CON3-5 & 576.43 & 1.49 & 
576.43 & 1.49 & \bf{563.70} & 
2.26\\CON3-6 & 505.26 & 1.73 & 
507.03 & 1.79 & \bf{499.20} & 
1.21\\CON3-7 & 578.41 & 1.18 & 
579.84 & 1.24 & \bf{577.50} & 
0.16\\CON3-8 & 524.59 & 1.30 & 
524.59 & 1.26 & \bf{523.10} & 
0.28\\CON3-9 & 588.48 & 1.32 & 
588.96 & 1.32 & \bf{578.20} & 
1.78\\CON8-0 & 879.00 & 1.50 & 
879.00 & 1.47 & \bf{858.90} & 
2.34\\CON8-1 & 758.26 & 1.29 & 
758.26 & 1.34 & \bf{740.90} & 
2.34\\CON8-2 & 716.56 & 1.98 & 
716.56 & 2.00 & \bf{714.30} & 
0.32\\CON8-3 & 817.57 & 1.50 & 
817.57 & 1.48 & \bf{812.30} & 
0.65\\CON8-4 & 789.98 & 1.58 & 
789.98 & 1.61 & \bf{770.10} & 
2.58\\CON8-5 & \bf{\underline{764.36}} & 1.39 & 
764.36 & 1.41 & 766.60 & 
-0.29\\CON8-6 & \bf{\underline{693.83}} & 1.74 & 
703.12 & 1.77 & 697.20 & 
-0.48\\CON8-7 & 822.42 & 1.22 & 
823.00 & 1.19 & \bf{814.80} & 
0.94\\CON8-8 & 799.32 & 1.56 & 
799.46 & 1.57 & \bf{771.30} & 
3.63\\CON8-9 & 816.12 & 1.66 & 
816.12 & 1.58 & \bf{815.10} & 
0.13\\[1ex]\hline
\end{tabular}
\label{table:nonlin}
\end{table} \clearpage
\begin{table}[ht]
\caption{Resultados de la ejecución de la metaheurística ACO, utilizando instancias de Dethloff con la configuración -n 2.0 -alpha 1.0 -beta 3.0 -q 11.5 -ro 0.015}
\centering
\small
\begin{tabular}{c c c c c c c}
\hline\hline
Instancia & Costo mínimo & Tiempo(seg.) & Costo promedio & Tiempo promedio(seg.) & Costo ACO & \%Gap \\ [0.5ex]
\hline
SCA3-0 & 640.55 & 1.40 & 
640.55 & 1.44 & \bf{636.10} & 
0.70\\SCA3-1 & \bf{\underline{697.84}} & 1.52 & 
697.84 & 1.57 & 700.10 & 
-0.32\\SCA3-2 & 664.18 & 1.34 & 
664.18 & 1.35 & \bf{659.30} & 
0.74\\SCA3-3 & 680.60 & 1.56 & 
680.78 & 1.48 & \bf{680.00} & 
0.09\\SCA3-4 & \bf{690.50} & 1.42 & 
690.50 & 1.41 & 690.50 & 0.00\\
SCA3-5 & \bf{\underline{665.04}} & 1.43 & 
665.04 & 1.41 & 671.10 & 
-0.90\\SCA3-6 & 655.19 & 1.30 & 
655.19 & 1.41 & \bf{651.10} & 
0.63\\SCA3-7 & 666.15 & 1.00 & 
666.15 & 0.98 & \bf{666.10} & 
0.01\\SCA3-8 & 721.45 & 1.09 & 
724.59 & 1.11 & \bf{719.50} & 
0.27\\SCA3-9 & \bf{681.00} & 0.97 & 
681.00 & 1.07 & 681.00 & 0.00\\
SCA8-0 & 991.07 & 1.51 & 
991.07 & 1.50 & \bf{961.60} & 
3.06\\SCA8-1 & 1074.65 & 1.22 & 
1074.65 & 1.22 & \bf{1063.00} & 
1.10\\SCA8-2 & 1056.87 & 0.97 & 
1056.87 & 1.03 & \bf{1040.60} & 
1.56\\SCA8-3 & 1031.08 & 1.50 & 
1031.08 & 1.46 & \bf{985.90} & 
4.58\\SCA8-4 & 1098.34 & 1.56 & 
1098.99 & 1.53 & \bf{1071.00} & 
2.55\\SCA8-5 & 1055.35 & 1.67 & 
1055.35 & 1.67 & \bf{1054.30} & 
0.10\\SCA8-6 & \bf{\underline{972.48}} & 1.67 & 
975.00 & 1.72 & 972.50 & 
-0.00\\SCA8-7 & 1092.57 & 1.67 & 
1092.57 & 1.67 & \bf{1059.70} & 
3.10\\SCA8-8 & 1092.02 & 1.49 & 
1092.02 & 1.45 & \bf{1082.70} & 
0.86\\SCA8-9 & \bf{\underline{1067.42}} & 1.16 & 
1067.42 & 1.18 & 1081.40 & 
-1.29\\CON3-0 & 624.96 & 1.69 & 
624.96 & 1.65 & \bf{616.50} & 
1.37\\CON3-1 & 557.38 & 1.43 & 
558.42 & 1.49 & \bf{555.60} & 
0.32\\CON3-2 & 524.07 & 1.12 & 
524.51 & 1.13 & \bf{521.40} & 
0.51\\CON3-3 & 594.11 & 1.59 & 
594.11 & 1.59 & \bf{591.20} & 
0.49\\CON3-4 & 589.32 & 1.29 & 
589.32 & 1.27 & \bf{589.30} & 
0.00\\CON3-5 & 576.43 & 1.51 & 
576.88 & 1.46 & \bf{563.70} & 
2.26\\CON3-6 & 504.15 & 1.89 & 
504.98 & 1.83 & \bf{499.20} & 
0.99\\CON3-7 & 578.41 & 1.28 & 
578.41 & 1.26 & \bf{577.50} & 
0.16\\CON3-8 & 524.59 & 1.23 & 
524.59 & 1.24 & \bf{523.10} & 
0.28\\CON3-9 & 588.48 & 1.29 & 
588.48 & 1.30 & \bf{578.20} & 
1.78\\CON8-0 & 879.00 & 1.38 & 
879.00 & 1.43 & \bf{858.90} & 
2.34\\CON8-1 & 758.26 & 1.38 & 
758.26 & 1.41 & \bf{740.90} & 
2.34\\CON8-2 & 716.56 & 2.05 & 
717.21 & 2.02 & \bf{714.30} & 
0.32\\CON8-3 & 817.57 & 1.48 & 
817.57 & 1.45 & \bf{812.30} & 
0.65\\CON8-4 & 781.64 & 1.53 & 
785.81 & 1.58 & \bf{770.10} & 
1.50\\CON8-5 & \bf{\underline{764.36}} & 1.33 & 
764.36 & 1.37 & 766.60 & 
-0.29\\CON8-6 & 706.20 & 1.70 & 
707.23 & 1.71 & \bf{697.20} & 
1.29\\CON8-7 & 822.42 & 1.20 & 
822.92 & 1.25 & \bf{814.80} & 
0.94\\CON8-8 & 799.16 & 1.59 & 
799.33 & 1.57 & \bf{771.30} & 
3.61\\CON8-9 & 816.12 & 1.66 & 
816.12 & 1.59 & \bf{815.10} & 
0.13\\[1ex]\hline
\end{tabular}
\label{table:nonlin}
\end{table} \clearpage
\begin{table}[ht]
\caption{Resultados de la ejecución de la metaheurística ACO, utilizando instancias de Dethloff con la configuración -n 2.0 -alpha 1.0 -beta 3.0 -q 11.6 -ro 0.015}
\centering
\small
\begin{tabular}{c c c c c c c}
\hline\hline
Instancia & Costo mínimo & Tiempo(seg.) & Costo promedio & Tiempo promedio(seg.) & Costo ACO & \%Gap \\ [0.5ex]
\hline
SCA3-0 & 640.55 & 1.34 & 
640.55 & 1.38 & \bf{636.10} & 
0.70\\SCA3-1 & \bf{\underline{697.84}} & 1.59 & 
697.84 & 1.55 & 700.10 & 
-0.32\\SCA3-2 & 664.18 & 1.39 & 
664.18 & 1.39 & \bf{659.30} & 
0.74\\SCA3-3 & 680.60 & 1.52 & 
680.96 & 1.53 & \bf{680.00} & 
0.09\\SCA3-4 & \bf{690.50} & 1.43 & 
690.50 & 1.39 & 690.50 & 0.00\\
SCA3-5 & \bf{\underline{665.04}} & 1.44 & 
668.60 & 1.42 & 671.10 & 
-0.90\\SCA3-6 & 655.19 & 1.29 & 
655.19 & 1.35 & \bf{651.10} & 
0.63\\SCA3-7 & 666.15 & 0.99 & 
666.15 & 0.97 & \bf{666.10} & 
0.01\\SCA3-8 & 721.45 & 1.12 & 
723.34 & 1.13 & \bf{719.50} & 
0.27\\SCA3-9 & \bf{681.00} & 0.93 & 
681.00 & 0.95 & 681.00 & 0.00\\
SCA8-0 & 991.07 & 1.45 & 
991.07 & 1.53 & \bf{961.60} & 
3.06\\SCA8-1 & 1074.65 & 1.17 & 
1074.68 & 1.24 & \bf{1063.00} & 
1.10\\SCA8-2 & 1056.87 & 1.07 & 
1056.87 & 1.06 & \bf{1040.60} & 
1.56\\SCA8-3 & 1031.08 & 1.48 & 
1031.55 & 1.47 & \bf{985.90} & 
4.58\\SCA8-4 & 1099.06 & 1.44 & 
1099.06 & 1.53 & \bf{1071.00} & 
2.62\\SCA8-5 & 1055.35 & 1.72 & 
1055.35 & 1.70 & \bf{1054.30} & 
0.10\\SCA8-6 & \bf{\underline{972.48}} & 1.67 & 
972.48 & 1.66 & 972.50 & 
-0.00\\SCA8-7 & 1075.42 & 1.54 & 
1088.28 & 1.60 & \bf{1059.70} & 
1.48\\SCA8-8 & 1091.49 & 1.45 & 
1091.76 & 1.46 & \bf{1082.70} & 
0.81\\SCA8-9 & \bf{\underline{1067.42}} & 1.11 & 
1067.42 & 1.17 & 1081.40 & 
-1.29\\CON3-0 & 624.96 & 1.63 & 
624.96 & 1.65 & \bf{616.50} & 
1.37\\CON3-1 & 557.38 & 1.48 & 
558.30 & 1.45 & \bf{555.60} & 
0.32\\CON3-2 & 524.07 & 1.22 & 
525.19 & 1.14 & \bf{521.40} & 
0.51\\CON3-3 & 594.11 & 1.44 & 
594.11 & 1.50 & \bf{591.20} & 
0.49\\CON3-4 & 589.32 & 1.35 & 
589.32 & 1.31 & \bf{589.30} & 
0.00\\CON3-5 & 569.15 & 1.38 & 
574.61 & 1.40 & \bf{563.70} & 
0.97\\CON3-6 & 504.15 & 1.82 & 
506.68 & 1.83 & \bf{499.20} & 
0.99\\CON3-7 & 578.41 & 1.27 & 
579.12 & 1.26 & \bf{577.50} & 
0.16\\CON3-8 & 524.59 & 1.17 & 
524.59 & 1.22 & \bf{523.10} & 
0.28\\CON3-9 & 588.48 & 1.26 & 
588.48 & 1.28 & \bf{578.20} & 
1.78\\CON8-0 & 879.00 & 1.44 & 
879.00 & 1.41 & \bf{858.90} & 
2.34\\CON8-1 & 758.26 & 1.28 & 
758.30 & 1.34 & \bf{740.90} & 
2.34\\CON8-2 & 716.53 & 1.90 & 
716.54 & 2.01 & \bf{714.30} & 
0.31\\CON8-3 & 817.57 & 1.43 & 
817.57 & 1.46 & \bf{812.30} & 
0.65\\CON8-4 & 781.64 & 1.51 & 
786.59 & 1.55 & \bf{770.10} & 
1.50\\CON8-5 & \bf{\underline{764.36}} & 1.33 & 
764.36 & 1.40 & 766.60 & 
-0.29\\CON8-6 & \bf{\underline{693.83}} & 1.82 & 
702.66 & 1.76 & 697.20 & 
-0.48\\CON8-7 & 822.42 & 1.24 & 
823.18 & 1.18 & \bf{814.80} & 
0.94\\CON8-8 & 799.16 & 1.59 & 
799.38 & 1.56 & \bf{771.30} & 
3.61\\CON8-9 & 816.12 & 1.67 & 
816.12 & 1.64 & \bf{815.10} & 
0.13\\[1ex]\hline
\end{tabular}
\label{table:nonlin}
\end{table} \clearpage
\begin{table}[ht]
\caption{Resultados de la ejecución de la metaheurística ACO, utilizando instancias de Dethloff con la configuración -n 2.0 -alpha 1.0 -beta 3.0 -q 11.7 -ro 0.015}
\centering
\small
\begin{tabular}{c c c c c c c}
\hline\hline
Instancia & Costo mínimo & Tiempo(seg.) & Costo promedio & Tiempo promedio(seg.) & Costo ACO & \%Gap \\ [0.5ex]
\hline
SCA3-0 & 640.55 & 1.48 & 
640.55 & 1.42 & \bf{636.10} & 
0.70\\SCA3-1 & \bf{\underline{697.84}} & 1.59 & 
698.76 & 1.54 & 700.10 & 
-0.32\\SCA3-2 & 659.34 & 1.31 & 
662.97 & 1.33 & \bf{659.30} & 
0.01\\SCA3-3 & 680.60 & 1.47 & 
680.96 & 1.53 & \bf{680.00} & 
0.09\\SCA3-4 & \bf{690.50} & 1.44 & 
690.50 & 1.40 & 690.50 & 0.00\\
SCA3-5 & \bf{\underline{665.04}} & 1.44 & 
665.04 & 1.47 & 671.10 & 
-0.90\\SCA3-6 & 653.69 & 1.28 & 
654.82 & 1.32 & \bf{651.10} & 
0.40\\SCA3-7 & 666.15 & 1.00 & 
666.15 & 1.01 & \bf{666.10} & 
0.01\\SCA3-8 & 721.45 & 1.11 & 
722.70 & 1.16 & \bf{719.50} & 
0.27\\SCA3-9 & \bf{681.00} & 0.99 & 
681.00 & 0.95 & 681.00 & 0.00\\
SCA8-0 & 991.07 & 1.50 & 
991.65 & 1.51 & \bf{961.60} & 
3.06\\SCA8-1 & 1074.65 & 1.23 & 
1074.65 & 1.19 & \bf{1063.00} & 
1.10\\SCA8-2 & 1056.87 & 1.03 & 
1056.87 & 1.01 & \bf{1040.60} & 
1.56\\SCA8-3 & 1031.08 & 1.42 & 
1031.08 & 1.55 & \bf{985.90} & 
4.58\\SCA8-4 & 1099.06 & 1.52 & 
1099.06 & 1.47 & \bf{1071.00} & 
2.62\\SCA8-5 & 1055.35 & 1.69 & 
1055.35 & 1.68 & \bf{1054.30} & 
0.10\\SCA8-6 & \bf{\underline{972.48}} & 1.70 & 
974.70 & 1.69 & 972.50 & 
-0.00\\SCA8-7 & 1075.42 & 1.61 & 
1088.28 & 1.61 & \bf{1059.70} & 
1.48\\SCA8-8 & 1092.02 & 1.43 & 
1092.02 & 1.42 & \bf{1082.70} & 
0.86\\SCA8-9 & \bf{\underline{1067.42}} & 1.16 & 
1067.42 & 1.16 & 1081.40 & 
-1.29\\CON3-0 & 624.96 & 1.59 & 
624.96 & 1.67 & \bf{616.50} & 
1.37\\CON3-1 & 557.38 & 1.44 & 
557.38 & 1.54 & \bf{555.60} & 
0.32\\CON3-2 & 524.07 & 1.10 & 
524.51 & 1.12 & \bf{521.40} & 
0.51\\CON3-3 & 594.11 & 1.41 & 
594.11 & 1.53 & \bf{591.20} & 
0.49\\CON3-4 & \bf{\underline{588.79}} & 1.28 & 
589.19 & 1.32 & 589.30 & 
-0.09\\CON3-5 & 576.43 & 1.46 & 
576.43 & 1.43 & \bf{563.70} & 
2.26\\CON3-6 & 505.26 & 1.80 & 
505.26 & 1.77 & \bf{499.20} & 
1.21\\CON3-7 & 578.41 & 1.30 & 
579.12 & 1.23 & \bf{577.50} & 
0.16\\CON3-8 & 524.30 & 1.21 & 
524.52 & 1.18 & \bf{523.10} & 
0.23\\CON3-9 & 588.48 & 1.30 & 
588.48 & 1.31 & \bf{578.20} & 
1.78\\CON8-0 & 879.00 & 1.48 & 
879.00 & 1.45 & \bf{858.90} & 
2.34\\CON8-1 & 758.26 & 1.24 & 
758.26 & 1.35 & \bf{740.90} & 
2.34\\CON8-2 & 716.53 & 1.97 & 
716.54 & 1.98 & \bf{714.30} & 
0.31\\CON8-3 & 817.57 & 1.48 & 
817.57 & 1.47 & \bf{812.30} & 
0.65\\CON8-4 & 778.60 & 1.60 & 
782.97 & 1.61 & \bf{770.10} & 
1.10\\CON8-5 & \bf{\underline{764.36}} & 1.41 & 
764.36 & 1.39 & 766.60 & 
-0.29\\CON8-6 & \bf{\underline{693.83}} & 1.69 & 
703.57 & 1.71 & 697.20 & 
-0.48\\CON8-7 & 822.42 & 1.21 & 
823.18 & 1.21 & \bf{814.80} & 
0.94\\CON8-8 & 799.16 & 1.65 & 
799.34 & 1.62 & \bf{771.30} & 
3.61\\CON8-9 & 816.12 & 1.56 & 
816.12 & 1.55 & \bf{815.10} & 
0.13\\[1ex]\hline
\end{tabular}
\label{table:nonlin}
\end{table} \clearpage
\begin{table}[ht]
\caption{Resultados de la ejecución de la metaheurística ACO, utilizando instancias de Dethloff con la configuración -n 2.0 -alpha 1.0 -beta 3.0 -q 11.8 -ro 0.015}
\centering
\small
\begin{tabular}{c c c c c c c}
\hline\hline
Instancia & Costo mínimo & Tiempo(seg.) & Costo promedio & Tiempo promedio(seg.) & Costo ACO & \%Gap \\ [0.5ex]
\hline
SCA3-0 & 640.55 & 1.38 & 
640.55 & 1.41 & \bf{636.10} & 
0.70\\SCA3-1 & \bf{\underline{697.84}} & 1.48 & 
698.76 & 1.54 & 700.10 & 
-0.32\\SCA3-2 & 664.18 & 1.30 & 
664.18 & 1.35 & \bf{659.30} & 
0.74\\SCA3-3 & 680.60 & 1.46 & 
680.78 & 1.48 & \bf{680.00} & 
0.09\\SCA3-4 & \bf{690.50} & 1.37 & 
690.50 & 1.42 & 690.50 & 0.00\\
SCA3-5 & \bf{\underline{665.04}} & 1.46 & 
665.04 & 1.44 & 671.10 & 
-0.90\\SCA3-6 & 655.19 & 1.40 & 
655.19 & 1.39 & \bf{651.10} & 
0.63\\SCA3-7 & 666.15 & 1.03 & 
666.15 & 1.00 & \bf{666.10} & 
0.01\\SCA3-8 & 721.45 & 1.11 & 
721.45 & 1.12 & \bf{719.50} & 
0.27\\SCA3-9 & \bf{681.00} & 1.02 & 
681.00 & 0.99 & 681.00 & 0.00\\
SCA8-0 & 991.07 & 1.52 & 
991.65 & 1.51 & \bf{961.60} & 
3.06\\SCA8-1 & 1069.40 & 1.21 & 
1073.37 & 1.21 & \bf{1063.00} & 
0.60\\SCA8-2 & 1056.87 & 0.98 & 
1056.87 & 1.00 & \bf{1040.60} & 
1.56\\SCA8-3 & 1031.08 & 1.48 & 
1031.08 & 1.53 & \bf{985.90} & 
4.58\\SCA8-4 & 1099.06 & 1.44 & 
1099.06 & 1.45 & \bf{1071.00} & 
2.62\\SCA8-5 & 1055.35 & 1.62 & 
1055.35 & 1.73 & \bf{1054.30} & 
0.10\\SCA8-6 & \bf{\underline{972.48}} & 1.75 & 
973.62 & 1.74 & 972.50 & 
-0.00\\SCA8-7 & 1092.57 & 1.75 & 
1092.57 & 1.68 & \bf{1059.70} & 
3.10\\SCA8-8 & 1091.49 & 1.47 & 
1091.49 & 1.44 & \bf{1082.70} & 
0.81\\SCA8-9 & \bf{\underline{1067.42}} & 1.17 & 
1067.42 & 1.16 & 1081.40 & 
-1.29\\CON3-0 & 624.96 & 1.64 & 
624.96 & 1.66 & \bf{616.50} & 
1.37\\CON3-1 & 557.38 & 1.40 & 
559.14 & 1.50 & \bf{555.60} & 
0.32\\CON3-2 & 524.07 & 1.14 & 
524.35 & 1.15 & \bf{521.40} & 
0.51\\CON3-3 & 594.11 & 1.49 & 
594.11 & 1.52 & \bf{591.20} & 
0.49\\CON3-4 & 589.32 & 1.28 & 
589.32 & 1.29 & \bf{589.30} & 
0.00\\CON3-5 & 570.70 & 1.40 & 
575.00 & 1.44 & \bf{563.70} & 
1.24\\CON3-6 & 505.26 & 1.82 & 
505.78 & 1.80 & \bf{499.20} & 
1.21\\CON3-7 & 578.41 & 1.34 & 
579.84 & 1.29 & \bf{577.50} & 
0.16\\CON3-8 & 524.59 & 1.23 & 
524.59 & 1.20 & \bf{523.10} & 
0.28\\CON3-9 & 588.48 & 1.34 & 
588.48 & 1.31 & \bf{578.20} & 
1.78\\CON8-0 & 879.00 & 1.47 & 
879.00 & 1.46 & \bf{858.90} & 
2.34\\CON8-1 & 758.26 & 1.44 & 
758.26 & 1.38 & \bf{740.90} & 
2.34\\CON8-2 & 716.53 & 1.90 & 
716.54 & 1.99 & \bf{714.30} & 
0.31\\CON8-3 & 817.57 & 1.41 & 
817.57 & 1.42 & \bf{812.30} & 
0.65\\CON8-4 & 781.64 & 1.59 & 
785.81 & 1.58 & \bf{770.10} & 
1.50\\CON8-5 & \bf{\underline{764.36}} & 1.42 & 
764.36 & 1.36 & 766.60 & 
-0.29\\CON8-6 & 705.61 & 1.76 & 
706.77 & 1.70 & \bf{697.20} & 
1.21\\CON8-7 & 822.42 & 1.23 & 
822.92 & 1.21 & \bf{814.80} & 
0.94\\CON8-8 & 799.16 & 1.62 & 
799.34 & 1.56 & \bf{771.30} & 
3.61\\CON8-9 & 816.12 & 1.65 & 
817.40 & 1.58 & \bf{815.10} & 
0.13\\[1ex]\hline
\end{tabular}
\label{table:nonlin}
\end{table} \clearpage
\begin{table}[ht]
\caption{Resultados de la ejecución de la metaheurística ACO, utilizando instancias de Dethloff con la configuración -n 2.0 -alpha 1.0 -beta 3.0 -q 11.9 -ro 0.015}
\centering
\small
\begin{tabular}{c c c c c c c}
\hline\hline
Instancia & Costo mínimo & Tiempo(seg.) & Costo promedio & Tiempo promedio(seg.) & Costo ACO & \%Gap \\ [0.5ex]
\hline
SCA3-0 & 640.55 & 1.36 & 
640.55 & 1.39 & \bf{636.10} & 
0.70\\SCA3-1 & \bf{\underline{697.84}} & 1.52 & 
697.84 & 1.50 & 700.10 & 
-0.32\\SCA3-2 & 659.34 & 1.34 & 
661.76 & 1.33 & \bf{659.30} & 
0.01\\SCA3-3 & 680.60 & 1.45 & 
680.96 & 1.46 & \bf{680.00} & 
0.09\\SCA3-4 & \bf{690.50} & 1.46 & 
690.50 & 1.42 & 690.50 & 0.00\\
SCA3-5 & \bf{\underline{665.04}} & 1.45 & 
668.79 & 1.44 & 671.10 & 
-0.90\\SCA3-6 & 655.19 & 1.36 & 
655.19 & 1.33 & \bf{651.10} & 
0.63\\SCA3-7 & 666.15 & 0.96 & 
666.15 & 0.96 & \bf{666.10} & 
0.01\\SCA3-8 & 721.45 & 1.22 & 
725.97 & 1.15 & \bf{719.50} & 
0.27\\SCA3-9 & \bf{681.00} & 0.95 & 
681.00 & 1.00 & 681.00 & 0.00\\
SCA8-0 & 991.07 & 1.54 & 
991.07 & 1.56 & \bf{961.60} & 
3.06\\SCA8-1 & 1074.65 & 1.23 & 
1074.65 & 1.21 & \bf{1063.00} & 
1.10\\SCA8-2 & 1056.87 & 1.09 & 
1056.87 & 1.03 & \bf{1040.60} & 
1.56\\SCA8-3 & 1031.08 & 1.39 & 
1031.08 & 1.43 & \bf{985.90} & 
4.58\\SCA8-4 & 1098.34 & 1.54 & 
1098.99 & 1.49 & \bf{1071.00} & 
2.55\\SCA8-5 & 1055.35 & 1.64 & 
1055.35 & 1.68 & \bf{1054.30} & 
0.10\\SCA8-6 & \bf{\underline{972.48}} & 1.61 & 
972.48 & 1.70 & 972.50 & 
-0.00\\SCA8-7 & 1092.57 & 1.64 & 
1092.57 & 1.67 & \bf{1059.70} & 
3.10\\SCA8-8 & 1091.49 & 1.39 & 
1091.89 & 1.42 & \bf{1082.70} & 
0.81\\SCA8-9 & \bf{\underline{1067.42}} & 1.13 & 
1067.42 & 1.13 & 1081.40 & 
-1.29\\CON3-0 & 624.96 & 1.59 & 
624.96 & 1.66 & \bf{616.50} & 
1.37\\CON3-1 & 557.38 & 1.43 & 
558.22 & 1.48 & \bf{555.60} & 
0.32\\CON3-2 & 525.17 & 1.10 & 
525.36 & 1.09 & \bf{521.40} & 
0.72\\CON3-3 & 592.95 & 1.67 & 
593.82 & 1.58 & \bf{591.20} & 
0.30\\CON3-4 & 589.32 & 1.44 & 
589.32 & 1.35 & \bf{589.30} & 
0.00\\CON3-5 & 576.43 & 1.39 & 
576.43 & 1.41 & \bf{563.70} & 
2.26\\CON3-6 & 504.15 & 1.80 & 
505.07 & 1.78 & \bf{499.20} & 
0.99\\CON3-7 & 578.41 & 1.22 & 
579.84 & 1.20 & \bf{577.50} & 
0.16\\CON3-8 & 524.30 & 1.30 & 
524.45 & 1.18 & \bf{523.10} & 
0.23\\CON3-9 & 588.48 & 1.29 & 
588.48 & 1.28 & \bf{578.20} & 
1.78\\CON8-0 & 879.00 & 1.46 & 
879.00 & 1.45 & \bf{858.90} & 
2.34\\CON8-1 & 758.26 & 1.33 & 
758.26 & 1.38 & \bf{740.90} & 
2.34\\CON8-2 & 716.53 & 2.02 & 
716.55 & 2.08 & \bf{714.30} & 
0.31\\CON8-3 & 817.57 & 1.45 & 
817.57 & 1.46 & \bf{812.30} & 
0.65\\CON8-4 & 789.98 & 1.60 & 
792.31 & 1.55 & \bf{770.10} & 
2.58\\CON8-5 & \bf{\underline{764.36}} & 1.42 & 
764.36 & 1.45 & 766.60 & 
-0.29\\CON8-6 & \bf{\underline{693.83}} & 1.79 & 
699.72 & 1.77 & 697.20 & 
-0.48\\CON8-7 & 822.42 & 1.15 & 
822.67 & 1.22 & \bf{814.80} & 
0.94\\CON8-8 & 799.16 & 1.57 & 
799.33 & 1.58 & \bf{771.30} & 
3.61\\CON8-9 & 816.12 & 1.66 & 
817.40 & 1.59 & \bf{815.10} & 
0.13\\[1ex]\hline
\end{tabular}
\label{table:nonlin}
\end{table} \clearpage
\begin{table}[ht]
\caption{Resultados de la ejecución de la metaheurística ACO, utilizando instancias de Dethloff con la configuración -n 2.0 -alpha 1.0 -beta 3.0 -q 12.0 -ro 0.015}
\centering
\small
\begin{tabular}{c c c c c c c}
\hline\hline
Instancia & Costo mínimo & Tiempo(seg.) & Costo promedio & Tiempo promedio(seg.) & Costo ACO & \%Gap \\ [0.5ex]
\hline
SCA3-0 & 640.55 & 1.35 & 
640.55 & 1.39 & \bf{636.10} & 
0.70\\SCA3-1 & \bf{\underline{697.84}} & 1.60 & 
697.84 & 1.55 & 700.10 & 
-0.32\\SCA3-2 & 664.18 & 1.31 & 
664.18 & 1.32 & \bf{659.30} & 
0.74\\SCA3-3 & 680.60 & 1.55 & 
680.96 & 1.46 & \bf{680.00} & 
0.09\\SCA3-4 & \bf{690.50} & 1.50 & 
690.50 & 1.41 & 690.50 & 0.00\\
SCA3-5 & \bf{\underline{665.04}} & 1.49 & 
665.19 & 1.42 & 671.10 & 
-0.90\\SCA3-6 & 655.19 & 1.37 & 
655.19 & 1.38 & \bf{651.10} & 
0.63\\SCA3-7 & 666.15 & 1.11 & 
666.15 & 1.04 & \bf{666.10} & 
0.01\\SCA3-8 & 721.45 & 1.10 & 
725.24 & 1.13 & \bf{719.50} & 
0.27\\SCA3-9 & \bf{681.00} & 0.94 & 
681.00 & 0.96 & 681.00 & 0.00\\
SCA8-0 & 991.07 & 1.56 & 
992.23 & 1.53 & \bf{961.60} & 
3.06\\SCA8-1 & 1069.40 & 1.21 & 
1073.34 & 1.17 & \bf{1063.00} & 
0.60\\SCA8-2 & 1056.87 & 0.98 & 
1056.87 & 1.01 & \bf{1040.60} & 
1.56\\SCA8-3 & 1031.08 & 1.46 & 
1031.08 & 1.46 & \bf{985.90} & 
4.58\\SCA8-4 & 1098.34 & 1.47 & 
1098.99 & 1.50 & \bf{1071.00} & 
2.55\\SCA8-5 & 1055.35 & 1.80 & 
1055.35 & 1.73 & \bf{1054.30} & 
0.10\\SCA8-6 & \bf{\underline{972.48}} & 1.73 & 
972.48 & 1.74 & 972.50 & 
-0.00\\SCA8-7 & 1092.57 & 1.56 & 
1092.57 & 1.61 & \bf{1059.70} & 
3.10\\SCA8-8 & 1091.49 & 1.42 & 
1091.76 & 1.54 & \bf{1082.70} & 
0.81\\SCA8-9 & \bf{\underline{1067.42}} & 1.13 & 
1067.42 & 1.15 & 1081.40 & 
-1.29\\CON3-0 & 624.96 & 1.57 & 
624.96 & 1.58 & \bf{616.50} & 
1.37\\CON3-1 & 557.38 & 1.38 & 
557.38 & 1.58 & \bf{555.60} & 
0.32\\CON3-2 & 524.07 & 1.22 & 
525.88 & 1.14 & \bf{521.40} & 
0.51\\CON3-3 & \bf{591.20} & 1.50 & 
593.38 & 1.51 & 591.20 & 0.00\\
CON3-4 & 589.32 & 1.28 & 
589.32 & 1.37 & \bf{589.30} & 
0.00\\CON3-5 & 576.43 & 1.40 & 
576.43 & 1.45 & \bf{563.70} & 
2.26\\CON3-6 & 508.89 & 1.77 & 
510.09 & 1.78 & \bf{499.20} & 
1.94\\CON3-7 & 578.41 & 1.16 & 
579.84 & 1.17 & \bf{577.50} & 
0.16\\CON3-8 & 524.30 & 1.22 & 
524.52 & 1.19 & \bf{523.10} & 
0.23\\CON3-9 & 588.48 & 1.28 & 
588.48 & 1.27 & \bf{578.20} & 
1.78\\CON8-0 & 879.00 & 1.50 & 
879.00 & 1.47 & \bf{858.90} & 
2.34\\CON8-1 & 758.26 & 1.38 & 
758.26 & 1.36 & \bf{740.90} & 
2.34\\CON8-2 & 716.53 & 2.00 & 
716.54 & 2.00 & \bf{714.30} & 
0.31\\CON8-3 & 817.57 & 1.40 & 
817.57 & 1.43 & \bf{812.30} & 
0.65\\CON8-4 & 777.81 & 1.56 & 
786.94 & 1.54 & \bf{770.10} & 
1.00\\CON8-5 & \bf{\underline{764.36}} & 1.40 & 
764.36 & 1.40 & 766.60 & 
-0.29\\CON8-6 & 705.61 & 1.68 & 
706.18 & 1.73 & \bf{697.20} & 
1.21\\CON8-7 & 822.42 & 1.21 & 
822.92 & 1.20 & \bf{814.80} & 
0.94\\CON8-8 & 799.32 & 1.65 & 
799.41 & 1.58 & \bf{771.30} & 
3.63\\CON8-9 & 816.12 & 1.70 & 
817.40 & 1.60 & \bf{815.10} & 
0.13\\[1ex]\hline
\end{tabular}
\label{table:nonlin}
\end{table} \clearpage
\begin{table}[ht]
\caption{Resultados de la ejecución de la metaheurística ACO, utilizando instancias de Dethloff con la configuración -n 2.0 -alpha 1.0 -beta 3.0 -q 12.1 -ro 0.015}
\centering
\small
\begin{tabular}{c c c c c c c}
\hline\hline
Instancia & Costo mínimo & Tiempo(seg.) & Costo promedio & Tiempo promedio(seg.) & Costo ACO & \%Gap \\ [0.5ex]
\hline
SCA3-0 & 640.55 & 1.46 & 
640.55 & 1.39 & \bf{636.10} & 
0.70\\SCA3-1 & \bf{\underline{697.84}} & 1.60 & 
697.84 & 1.50 & 700.10 & 
-0.32\\SCA3-2 & 659.34 & 1.34 & 
662.97 & 1.36 & \bf{659.30} & 
0.01\\SCA3-3 & 680.60 & 1.47 & 
680.78 & 1.50 & \bf{680.00} & 
0.09\\SCA3-4 & \bf{690.50} & 1.46 & 
690.50 & 1.47 & 690.50 & 0.00\\
SCA3-5 & \bf{\underline{665.04}} & 1.42 & 
665.19 & 1.47 & 671.10 & 
-0.90\\SCA3-6 & 655.19 & 1.32 & 
655.19 & 1.31 & \bf{651.10} & 
0.63\\SCA3-7 & 666.15 & 0.95 & 
666.15 & 1.00 & \bf{666.10} & 
0.01\\SCA3-8 & 721.45 & 1.12 & 
723.34 & 1.13 & \bf{719.50} & 
0.27\\SCA3-9 & \bf{681.00} & 0.93 & 
681.00 & 0.94 & 681.00 & 0.00\\
SCA8-0 & 991.07 & 1.58 & 
991.65 & 1.53 & \bf{961.60} & 
3.06\\SCA8-1 & 1074.39 & 1.19 & 
1074.59 & 1.18 & \bf{1063.00} & 
1.07\\SCA8-2 & 1056.87 & 1.03 & 
1056.87 & 1.02 & \bf{1040.60} & 
1.56\\SCA8-3 & 1031.08 & 1.46 & 
1031.08 & 1.50 & \bf{985.90} & 
4.58\\SCA8-4 & 1099.06 & 1.62 & 
1099.06 & 1.57 & \bf{1071.00} & 
2.62\\SCA8-5 & 1055.35 & 1.64 & 
1055.35 & 1.66 & \bf{1054.30} & 
0.10\\SCA8-6 & \bf{\underline{972.48}} & 1.69 & 
972.48 & 1.65 & 972.50 & 
-0.00\\SCA8-7 & 1075.42 & 1.67 & 
1088.28 & 1.65 & \bf{1059.70} & 
1.48\\SCA8-8 & 1091.49 & 1.35 & 
1091.76 & 1.42 & \bf{1082.70} & 
0.81\\SCA8-9 & \bf{\underline{1067.42}} & 1.09 & 
1067.42 & 1.14 & 1081.40 & 
-1.29\\CON3-0 & 624.96 & 1.68 & 
624.96 & 1.64 & \bf{616.50} & 
1.37\\CON3-1 & 557.38 & 1.38 & 
559.22 & 1.49 & \bf{555.60} & 
0.32\\CON3-2 & 524.07 & 1.07 & 
524.51 & 1.08 & \bf{521.40} & 
0.51\\CON3-3 & \bf{591.20} & 1.46 & 
593.38 & 1.48 & 591.20 & 0.00\\
CON3-4 & \bf{\underline{588.79}} & 1.22 & 
589.19 & 1.31 & 589.30 & 
-0.09\\CON3-5 & 576.43 & 1.40 & 
576.43 & 1.52 & \bf{563.70} & 
2.26\\CON3-6 & 504.15 & 1.92 & 
506.15 & 1.85 & \bf{499.20} & 
0.99\\CON3-7 & 578.41 & 1.15 & 
579.12 & 1.21 & \bf{577.50} & 
0.16\\CON3-8 & 524.30 & 1.18 & 
524.52 & 1.21 & \bf{523.10} & 
0.23\\CON3-9 & 588.48 & 1.31 & 
588.48 & 1.39 & \bf{578.20} & 
1.78\\CON8-0 & 879.00 & 1.44 & 
879.00 & 1.46 & \bf{858.90} & 
2.34\\CON8-1 & 758.26 & 1.37 & 
758.26 & 1.33 & \bf{740.90} & 
2.34\\CON8-2 & 716.53 & 1.95 & 
717.20 & 2.02 & \bf{714.30} & 
0.31\\CON8-3 & 817.57 & 1.39 & 
817.57 & 1.42 & \bf{812.30} & 
0.65\\CON8-4 & 781.64 & 1.55 & 
790.23 & 1.53 & \bf{770.10} & 
1.50\\CON8-5 & \bf{\underline{764.36}} & 1.42 & 
764.36 & 1.41 & 766.60 & 
-0.29\\CON8-6 & 705.61 & 1.75 & 
706.52 & 1.73 & \bf{697.20} & 
1.21\\CON8-7 & 822.42 & 1.12 & 
822.92 & 1.16 & \bf{814.80} & 
0.94\\CON8-8 & 799.16 & 1.62 & 
799.29 & 1.54 & \bf{771.30} & 
3.61\\CON8-9 & 816.12 & 1.50 & 
816.12 & 1.53 & \bf{815.10} & 
0.13\\[1ex]\hline
\end{tabular}
\label{table:nonlin}
\end{table} \clearpage
\begin{table}[ht]
\caption{Resultados de la ejecución de la metaheurística ACO, utilizando instancias de Dethloff con la configuración -n 2.0 -alpha 1.0 -beta 3.0 -q 12.2 -ro 0.015}
\centering
\small
\begin{tabular}{c c c c c c c}
\hline\hline
Instancia & Costo mínimo & Tiempo(seg.) & Costo promedio & Tiempo promedio(seg.) & Costo ACO & \%Gap \\ [0.5ex]
\hline
SCA3-0 & 640.55 & 1.44 & 
640.55 & 1.40 & \bf{636.10} & 
0.70\\SCA3-1 & \bf{\underline{697.84}} & 1.52 & 
697.84 & 1.52 & 700.10 & 
-0.32\\SCA3-2 & 659.34 & 1.40 & 
662.97 & 1.38 & \bf{659.30} & 
0.01\\SCA3-3 & 680.60 & 1.48 & 
681.13 & 1.49 & \bf{680.00} & 
0.09\\SCA3-4 & \bf{690.50} & 1.42 & 
690.50 & 1.40 & 690.50 & 0.00\\
SCA3-5 & \bf{\underline{665.04}} & 1.43 & 
665.04 & 1.42 & 671.10 & 
-0.90\\SCA3-6 & 655.19 & 1.33 & 
655.19 & 1.36 & \bf{651.10} & 
0.63\\SCA3-7 & 666.15 & 0.97 & 
666.15 & 1.00 & \bf{666.10} & 
0.01\\SCA3-8 & 721.45 & 1.14 & 
723.34 & 1.13 & \bf{719.50} & 
0.27\\SCA3-9 & \bf{681.00} & 0.96 & 
681.00 & 1.06 & 681.00 & 0.00\\
SCA8-0 & 991.07 & 1.42 & 
991.65 & 1.52 & \bf{961.60} & 
3.06\\SCA8-1 & 1069.40 & 1.23 & 
1073.37 & 1.20 & \bf{1063.00} & 
0.60\\SCA8-2 & 1056.87 & 1.04 & 
1056.87 & 1.02 & \bf{1040.60} & 
1.56\\SCA8-3 & 1031.08 & 1.53 & 
1031.08 & 1.45 & \bf{985.90} & 
4.58\\SCA8-4 & 1099.06 & 1.44 & 
1099.06 & 1.49 & \bf{1071.00} & 
2.62\\SCA8-5 & 1055.35 & 1.59 & 
1055.35 & 1.67 & \bf{1054.30} & 
0.10\\SCA8-6 & \bf{\underline{972.48}} & 1.70 & 
976.70 & 1.67 & 972.50 & 
-0.00\\SCA8-7 & 1092.57 & 1.58 & 
1092.57 & 1.63 & \bf{1059.70} & 
3.10\\SCA8-8 & 1085.93 & 1.30 & 
1090.50 & 1.41 & \bf{1082.70} & 
0.30\\SCA8-9 & \bf{\underline{1067.42}} & 1.47 & 
1067.42 & 1.22 & 1081.40 & 
-1.29\\CON3-0 & 624.96 & 1.59 & 
624.96 & 1.64 & \bf{616.50} & 
1.37\\CON3-1 & 557.38 & 1.39 & 
557.38 & 1.47 & \bf{555.60} & 
0.32\\CON3-2 & 524.07 & 1.07 & 
524.53 & 1.11 & \bf{521.40} & 
0.51\\CON3-3 & \bf{591.20} & 1.43 & 
593.38 & 1.47 & 591.20 & 0.00\\
CON3-4 & 589.32 & 1.47 & 
589.32 & 1.39 & \bf{589.30} & 
0.00\\CON3-5 & 576.43 & 1.38 & 
576.43 & 1.41 & \bf{563.70} & 
2.26\\CON3-6 & 505.26 & 1.77 & 
505.26 & 1.77 & \bf{499.20} & 
1.21\\CON3-7 & 578.41 & 1.15 & 
578.41 & 1.19 & \bf{577.50} & 
0.16\\CON3-8 & 524.30 & 1.28 & 
524.52 & 1.19 & \bf{523.10} & 
0.23\\CON3-9 & 588.48 & 1.31 & 
588.48 & 1.27 & \bf{578.20} & 
1.78\\CON8-0 & 879.00 & 1.46 & 
879.00 & 1.50 & \bf{858.90} & 
2.34\\CON8-1 & 758.26 & 1.40 & 
758.26 & 1.33 & \bf{740.90} & 
2.34\\CON8-2 & 716.53 & 1.97 & 
716.54 & 2.03 & \bf{714.30} & 
0.31\\CON8-3 & 817.57 & 1.44 & 
817.57 & 1.46 & \bf{812.30} & 
0.65\\CON8-4 & 778.60 & 1.51 & 
785.05 & 1.57 & \bf{770.10} & 
1.10\\CON8-5 & \bf{\underline{764.36}} & 1.43 & 
764.36 & 1.41 & 766.60 & 
-0.29\\CON8-6 & \bf{\underline{693.83}} & 1.79 & 
703.80 & 1.72 & 697.20 & 
-0.48\\CON8-7 & 822.42 & 1.14 & 
822.67 & 1.18 & \bf{814.80} & 
0.94\\CON8-8 & 799.32 & 1.56 & 
799.41 & 1.55 & \bf{771.30} & 
3.63\\CON8-9 & 816.12 & 1.48 & 
816.12 & 1.56 & \bf{815.10} & 
0.13\\[1ex]\hline
\end{tabular}
\label{table:nonlin}
\end{table} \clearpage
\begin{table}[ht]
\caption{Resultados de la ejecución de la metaheurística ACO, utilizando instancias de Dethloff con la configuración -n 2.0 -alpha 1.0 -beta 3.0 -q 12.3 -ro 0.015}
\centering
\small
\begin{tabular}{c c c c c c c}
\hline\hline
Instancia & Costo mínimo & Tiempo(seg.) & Costo promedio & Tiempo promedio(seg.) & Costo ACO & \%Gap \\ [0.5ex]
\hline
SCA3-0 & 640.55 & 1.38 & 
640.55 & 1.39 & \bf{636.10} & 
0.70\\SCA3-1 & \bf{\underline{697.84}} & 1.53 & 
697.84 & 1.57 & 700.10 & 
-0.32\\SCA3-2 & 659.34 & 1.26 & 
662.97 & 1.33 & \bf{659.30} & 
0.01\\SCA3-3 & 680.60 & 1.48 & 
680.78 & 1.53 & \bf{680.00} & 
0.09\\SCA3-4 & \bf{690.50} & 1.42 & 
690.50 & 1.44 & 690.50 & 0.00\\
SCA3-5 & \bf{\underline{665.04}} & 1.43 & 
665.19 & 1.42 & 671.10 & 
-0.90\\SCA3-6 & 655.19 & 1.38 & 
655.30 & 1.34 & \bf{651.10} & 
0.63\\SCA3-7 & 666.15 & 0.94 & 
666.26 & 0.97 & \bf{666.10} & 
0.01\\SCA3-8 & 721.45 & 1.15 & 
725.84 & 1.17 & \bf{719.50} & 
0.27\\SCA3-9 & \bf{681.00} & 0.94 & 
681.00 & 0.95 & 681.00 & 0.00\\
SCA8-0 & 991.07 & 1.58 & 
991.07 & 1.52 & \bf{961.60} & 
3.06\\SCA8-1 & 1074.65 & 1.24 & 
1074.65 & 1.22 & \bf{1063.00} & 
1.10\\SCA8-2 & 1056.87 & 0.96 & 
1056.87 & 1.02 & \bf{1040.60} & 
1.56\\SCA8-3 & 1031.08 & 1.44 & 
1031.08 & 1.47 & \bf{985.90} & 
4.58\\SCA8-4 & 1099.06 & 1.44 & 
1099.06 & 1.46 & \bf{1071.00} & 
2.62\\SCA8-5 & 1055.35 & 1.63 & 
1055.35 & 1.67 & \bf{1054.30} & 
0.10\\SCA8-6 & \bf{\underline{972.48}} & 1.72 & 
976.25 & 1.68 & 972.50 & 
-0.00\\SCA8-7 & 1092.57 & 1.61 & 
1092.57 & 1.64 & \bf{1059.70} & 
3.10\\SCA8-8 & 1085.22 & 1.45 & 
1090.19 & 1.44 & \bf{1082.70} & 
0.23\\SCA8-9 & \bf{\underline{1067.42}} & 1.18 & 
1067.42 & 1.19 & 1081.40 & 
-1.29\\CON3-0 & 624.96 & 1.72 & 
624.96 & 1.64 & \bf{616.50} & 
1.37\\CON3-1 & 557.38 & 1.47 & 
558.22 & 1.49 & \bf{555.60} & 
0.32\\CON3-2 & 524.07 & 1.09 & 
524.51 & 1.08 & \bf{521.40} & 
0.51\\CON3-3 & 594.11 & 1.52 & 
594.11 & 1.49 & \bf{591.20} & 
0.49\\CON3-4 & 589.32 & 1.45 & 
589.32 & 1.35 & \bf{589.30} & 
0.00\\CON3-5 & 569.15 & 1.49 & 
572.51 & 1.45 & \bf{563.70} & 
0.97\\CON3-6 & 505.26 & 1.75 & 
505.86 & 1.80 & \bf{499.20} & 
1.21\\CON3-7 & 578.41 & 1.20 & 
579.84 & 1.21 & \bf{577.50} & 
0.16\\CON3-8 & 524.30 & 1.28 & 
524.52 & 1.21 & \bf{523.10} & 
0.23\\CON3-9 & 588.48 & 1.22 & 
588.48 & 1.20 & \bf{578.20} & 
1.78\\CON8-0 & 879.00 & 1.43 & 
879.00 & 1.42 & \bf{858.90} & 
2.34\\CON8-1 & 758.26 & 1.36 & 
758.26 & 1.39 & \bf{740.90} & 
2.34\\CON8-2 & 716.53 & 1.96 & 
716.54 & 2.10 & \bf{714.30} & 
0.31\\CON8-3 & 817.57 & 1.44 & 
817.57 & 1.44 & \bf{812.30} & 
0.65\\CON8-4 & 781.64 & 1.59 & 
787.37 & 1.57 & \bf{770.10} & 
1.50\\CON8-5 & \bf{\underline{764.36}} & 1.40 & 
764.36 & 1.39 & 766.60 & 
-0.29\\CON8-6 & 705.61 & 1.64 & 
706.51 & 1.71 & \bf{697.20} & 
1.21\\CON8-7 & 822.42 & 1.12 & 
822.67 & 1.15 & \bf{814.80} & 
0.94\\CON8-8 & 799.16 & 1.57 & 
799.33 & 1.54 & \bf{771.30} & 
3.61\\CON8-9 & 816.12 & 1.56 & 
816.12 & 1.55 & \bf{815.10} & 
0.13\\[1ex]\hline
\end{tabular}
\label{table:nonlin}
\end{table} \clearpage
\begin{table}[ht]
\caption{Resultados de la ejecución de la metaheurística ACO, utilizando instancias de Dethloff con la configuración -n 2.0 -alpha 1.0 -beta 3.0 -q 12.4 -ro 0.015}
\centering
\small
\begin{tabular}{c c c c c c c}
\hline\hline
Instancia & Costo mínimo & Tiempo(seg.) & Costo promedio & Tiempo promedio(seg.) & Costo ACO & \%Gap \\ [0.5ex]
\hline
SCA3-0 & 640.55 & 1.36 & 
640.55 & 1.37 & \bf{636.10} & 
0.70\\SCA3-1 & \bf{\underline{697.84}} & 1.50 & 
697.84 & 1.50 & 700.10 & 
-0.32\\SCA3-2 & 664.18 & 1.31 & 
665.49 & 1.38 & \bf{659.30} & 
0.74\\SCA3-3 & 680.60 & 1.46 & 
681.13 & 1.44 & \bf{680.00} & 
0.09\\SCA3-4 & \bf{690.50} & 1.38 & 
690.50 & 1.41 & 690.50 & 0.00\\
SCA3-5 & \bf{\underline{665.04}} & 1.41 & 
665.04 & 1.41 & 671.10 & 
-0.90\\SCA3-6 & 655.19 & 1.34 & 
655.19 & 1.33 & \bf{651.10} & 
0.63\\SCA3-7 & 666.15 & 1.06 & 
666.15 & 1.05 & \bf{666.10} & 
0.01\\SCA3-8 & 721.45 & 1.22 & 
722.95 & 1.20 & \bf{719.50} & 
0.27\\SCA3-9 & \bf{681.00} & 0.95 & 
681.00 & 0.97 & 681.00 & 0.00\\
SCA8-0 & 991.07 & 1.86 & 
991.07 & 1.56 & \bf{961.60} & 
3.06\\SCA8-1 & 1074.65 & 1.24 & 
1074.68 & 1.19 & \bf{1063.00} & 
1.10\\SCA8-2 & 1056.87 & 0.97 & 
1056.87 & 1.02 & \bf{1040.60} & 
1.56\\SCA8-3 & 1031.08 & 1.45 & 
1031.08 & 1.43 & \bf{985.90} & 
4.58\\SCA8-4 & 1099.06 & 1.49 & 
1099.06 & 1.48 & \bf{1071.00} & 
2.62\\SCA8-5 & 1055.35 & 1.63 & 
1055.35 & 1.65 & \bf{1054.30} & 
0.10\\SCA8-6 & \bf{\underline{972.48}} & 1.69 & 
972.48 & 1.69 & 972.50 & 
-0.00\\SCA8-7 & 1092.57 & 1.62 & 
1092.57 & 1.65 & \bf{1059.70} & 
3.10\\SCA8-8 & 1091.49 & 1.45 & 
1091.89 & 1.44 & \bf{1082.70} & 
0.81\\SCA8-9 & \bf{\underline{1067.42}} & 1.18 & 
1067.42 & 1.18 & 1081.40 & 
-1.29\\CON3-0 & 624.96 & 2.04 & 
624.96 & 1.69 & \bf{616.50} & 
1.37\\CON3-1 & 557.38 & 1.53 & 
558.66 & 1.47 & \bf{555.60} & 
0.32\\CON3-2 & 524.07 & 1.15 & 
524.51 & 1.14 & \bf{521.40} & 
0.51\\CON3-3 & 592.95 & 1.52 & 
593.82 & 1.49 & \bf{591.20} & 
0.30\\CON3-4 & 589.32 & 1.30 & 
589.32 & 1.31 & \bf{589.30} & 
0.00\\CON3-5 & 572.75 & 1.38 & 
575.51 & 1.40 & \bf{563.70} & 
1.61\\CON3-6 & 504.06 & 1.75 & 
504.82 & 1.73 & \bf{499.20} & 
0.97\\CON3-7 & 578.41 & 1.21 & 
579.84 & 1.21 & \bf{577.50} & 
0.16\\CON3-8 & 524.59 & 1.11 & 
524.59 & 1.13 & \bf{523.10} & 
0.28\\CON3-9 & 588.48 & 1.24 & 
588.48 & 1.24 & \bf{578.20} & 
1.78\\CON8-0 & 879.00 & 1.44 & 
879.00 & 1.43 & \bf{858.90} & 
2.34\\CON8-1 & 758.26 & 1.35 & 
758.26 & 1.36 & \bf{740.90} & 
2.34\\CON8-2 & 716.53 & 1.99 & 
716.54 & 1.94 & \bf{714.30} & 
0.31\\CON8-3 & 817.57 & 1.38 & 
817.57 & 1.44 & \bf{812.30} & 
0.65\\CON8-4 & 778.60 & 1.46 & 
785.05 & 1.53 & \bf{770.10} & 
1.10\\CON8-5 & \bf{\underline{764.36}} & 1.32 & 
764.36 & 1.38 & 766.60 & 
-0.29\\CON8-6 & 705.61 & 1.74 & 
707.20 & 1.72 & \bf{697.20} & 
1.21\\CON8-7 & 822.42 & 1.20 & 
822.92 & 1.18 & \bf{814.80} & 
0.94\\CON8-8 & 799.16 & 1.65 & 
799.42 & 1.55 & \bf{771.30} & 
3.61\\CON8-9 & 816.12 & 1.62 & 
816.12 & 1.58 & \bf{815.10} & 
0.13\\[1ex]\hline
\end{tabular}
\label{table:nonlin}
\end{table} \clearpage
\begin{table}[ht]
\caption{Resultados de la ejecución de la metaheurística ACO, utilizando instancias de Dethloff con la configuración -n 2.0 -alpha 1.0 -beta 3.0 -q 12.5 -ro 0.015}
\centering
\small
\begin{tabular}{c c c c c c c}
\hline\hline
Instancia & Costo mínimo & Tiempo(seg.) & Costo promedio & Tiempo promedio(seg.) & Costo ACO & \%Gap \\ [0.5ex]
\hline
SCA3-0 & 640.55 & 1.33 & 
640.55 & 1.38 & \bf{636.10} & 
0.70\\SCA3-1 & \bf{\underline{697.84}} & 1.57 & 
697.84 & 1.50 & 700.10 & 
-0.32\\SCA3-2 & 664.18 & 1.40 & 
664.18 & 1.37 & \bf{659.30} & 
0.74\\SCA3-3 & 680.60 & 1.49 & 
680.96 & 1.45 & \bf{680.00} & 
0.09\\SCA3-4 & \bf{690.50} & 1.38 & 
690.50 & 1.41 & 690.50 & 0.00\\
SCA3-5 & \bf{\underline{665.04}} & 1.41 & 
668.90 & 1.39 & 671.10 & 
-0.90\\SCA3-6 & 655.19 & 1.34 & 
655.19 & 1.36 & \bf{651.10} & 
0.63\\SCA3-7 & 666.15 & 0.97 & 
666.15 & 0.98 & \bf{666.10} & 
0.01\\SCA3-8 & 721.45 & 1.09 & 
722.70 & 1.12 & \bf{719.50} & 
0.27\\SCA3-9 & \bf{681.00} & 0.97 & 
681.00 & 0.99 & 681.00 & 0.00\\
SCA8-0 & 991.07 & 1.56 & 
991.07 & 1.55 & \bf{961.60} & 
3.06\\SCA8-1 & 1074.65 & 1.13 & 
1074.65 & 1.18 & \bf{1063.00} & 
1.10\\SCA8-2 & 1056.87 & 1.01 & 
1056.87 & 1.05 & \bf{1040.60} & 
1.56\\SCA8-3 & 1031.08 & 1.47 & 
1031.08 & 1.49 & \bf{985.90} & 
4.58\\SCA8-4 & 1099.06 & 1.48 & 
1099.06 & 1.49 & \bf{1071.00} & 
2.62\\SCA8-5 & 1055.35 & 1.70 & 
1055.35 & 1.66 & \bf{1054.30} & 
0.10\\SCA8-6 & \bf{\underline{972.48}} & 1.63 & 
972.48 & 1.64 & 972.50 & 
-0.00\\SCA8-7 & 1092.57 & 1.60 & 
1092.57 & 1.63 & \bf{1059.70} & 
3.10\\SCA8-8 & 1092.02 & 1.43 & 
1092.02 & 1.41 & \bf{1082.70} & 
0.86\\SCA8-9 & \bf{\underline{1067.42}} & 1.22 & 
1067.42 & 1.33 & 1081.40 & 
-1.29\\CON3-0 & 624.96 & 1.66 & 
624.96 & 1.63 & \bf{616.50} & 
1.37\\CON3-1 & 557.38 & 1.44 & 
557.82 & 1.45 & \bf{555.60} & 
0.32\\CON3-2 & 524.07 & 1.12 & 
524.07 & 1.14 & \bf{521.40} & 
0.51\\CON3-3 & \bf{591.20} & 1.48 & 
593.38 & 1.48 & 591.20 & 0.00\\
CON3-4 & 589.32 & 1.40 & 
590.74 & 1.39 & \bf{589.30} & 
0.00\\CON3-5 & 569.88 & 1.40 & 
574.33 & 1.40 & \bf{563.70} & 
1.10\\CON3-6 & 505.26 & 1.76 & 
506.43 & 1.75 & \bf{499.20} & 
1.21\\CON3-7 & 578.41 & 1.16 & 
578.41 & 1.27 & \bf{577.50} & 
0.16\\CON3-8 & 524.59 & 1.22 & 
524.59 & 1.20 & \bf{523.10} & 
0.28\\CON3-9 & 588.48 & 1.26 & 
588.48 & 1.26 & \bf{578.20} & 
1.78\\CON8-0 & 879.00 & 1.44 & 
879.00 & 1.43 & \bf{858.90} & 
2.34\\CON8-1 & 758.26 & 1.29 & 
758.26 & 1.32 & \bf{740.90} & 
2.34\\CON8-2 & 716.53 & 1.95 & 
716.54 & 1.96 & \bf{714.30} & 
0.31\\CON8-3 & 817.57 & 1.44 & 
817.57 & 1.44 & \bf{812.30} & 
0.65\\CON8-4 & 789.98 & 1.62 & 
789.98 & 1.54 & \bf{770.10} & 
2.58\\CON8-5 & \bf{\underline{764.36}} & 1.40 & 
764.36 & 1.37 & 766.60 & 
-0.29\\CON8-6 & 705.61 & 1.80 & 
706.96 & 1.69 & \bf{697.20} & 
1.21\\CON8-7 & 822.42 & 1.16 & 
823.18 & 1.16 & \bf{814.80} & 
0.94\\CON8-8 & 799.16 & 1.56 & 
799.42 & 1.58 & \bf{771.30} & 
3.61\\CON8-9 & 816.12 & 1.62 & 
816.12 & 1.58 & \bf{815.10} & 
0.13\\[1ex]\hline
\end{tabular}
\label{table:nonlin}
\end{table} \clearpage
\begin{table}[ht]
\caption{Resultados de la ejecución de la metaheurística ACO, utilizando instancias de Dethloff con la configuración -n 2.0 -alpha 1.0 -beta 3.0 -q 12.6 -ro 0.015}
\centering
\small
\begin{tabular}{c c c c c c c}
\hline\hline
Instancia & Costo mínimo & Tiempo(seg.) & Costo promedio & Tiempo promedio(seg.) & Costo ACO & \%Gap \\ [0.5ex]
\hline
SCA3-0 & 640.55 & 1.42 & 
640.55 & 1.39 & \bf{636.10} & 
0.70\\SCA3-1 & \bf{\underline{697.84}} & 1.59 & 
697.84 & 1.51 & 700.10 & 
-0.32\\SCA3-2 & 659.34 & 1.42 & 
662.97 & 1.39 & \bf{659.30} & 
0.01\\SCA3-3 & 680.60 & 1.47 & 
680.78 & 1.48 & \bf{680.00} & 
0.09\\SCA3-4 & \bf{690.50} & 1.31 & 
690.50 & 1.40 & 690.50 & 0.00\\
SCA3-5 & \bf{\underline{665.04}} & 1.45 & 
665.54 & 1.44 & 671.10 & 
-0.90\\SCA3-6 & 655.19 & 1.38 & 
655.19 & 1.34 & \bf{651.10} & 
0.63\\SCA3-7 & 666.15 & 0.98 & 
666.15 & 1.01 & \bf{666.10} & 
0.01\\SCA3-8 & 721.45 & 1.15 & 
727.13 & 1.12 & \bf{719.50} & 
0.27\\SCA3-9 & \bf{681.00} & 1.06 & 
681.00 & 1.02 & 681.00 & 0.00\\
SCA8-0 & 991.07 & 1.53 & 
992.80 & 1.50 & \bf{961.60} & 
3.06\\SCA8-1 & 1074.65 & 1.14 & 
1074.68 & 1.19 & \bf{1063.00} & 
1.10\\SCA8-2 & 1056.87 & 0.97 & 
1056.87 & 1.02 & \bf{1040.60} & 
1.56\\SCA8-3 & 1031.08 & 1.45 & 
1031.08 & 1.45 & \bf{985.90} & 
4.58\\SCA8-4 & 1098.34 & 1.41 & 
1098.88 & 1.74 & \bf{1071.00} & 
2.55\\SCA8-5 & 1055.35 & 1.58 & 
1055.35 & 1.65 & \bf{1054.30} & 
0.10\\SCA8-6 & \bf{\underline{972.48}} & 1.60 & 
976.70 & 1.69 & 972.50 & 
-0.00\\SCA8-7 & 1092.57 & 1.68 & 
1092.57 & 1.64 & \bf{1059.70} & 
3.10\\SCA8-8 & 1091.49 & 1.45 & 
1091.76 & 1.44 & \bf{1082.70} & 
0.81\\SCA8-9 & \bf{\underline{1067.42}} & 1.10 & 
1067.42 & 1.14 & 1081.40 & 
-1.29\\CON3-0 & 624.96 & 1.69 & 
624.96 & 1.66 & \bf{616.50} & 
1.37\\CON3-1 & 557.38 & 1.49 & 
557.58 & 1.47 & \bf{555.60} & 
0.32\\CON3-2 & 524.07 & 1.08 & 
526.14 & 1.08 & \bf{521.40} & 
0.51\\CON3-3 & 594.11 & 1.56 & 
594.11 & 1.59 & \bf{591.20} & 
0.49\\CON3-4 & \bf{\underline{588.79}} & 1.31 & 
589.19 & 1.67 & 589.30 & 
-0.09\\CON3-5 & 576.43 & 1.53 & 
576.43 & 1.43 & \bf{563.70} & 
2.26\\CON3-6 & 505.26 & 1.82 & 
506.43 & 1.81 & \bf{499.20} & 
1.21\\CON3-7 & 578.41 & 1.20 & 
579.84 & 1.22 & \bf{577.50} & 
0.16\\CON3-8 & 524.59 & 1.24 & 
524.59 & 1.22 & \bf{523.10} & 
0.28\\CON3-9 & 588.48 & 1.22 & 
588.48 & 1.24 & \bf{578.20} & 
1.78\\CON8-0 & 879.00 & 1.46 & 
879.00 & 1.44 & \bf{858.90} & 
2.34\\CON8-1 & 758.26 & 1.29 & 
758.26 & 1.32 & \bf{740.90} & 
2.34\\CON8-2 & 716.53 & 1.94 & 
716.54 & 1.96 & \bf{714.30} & 
0.31\\CON8-3 & 817.57 & 1.42 & 
817.57 & 1.42 & \bf{812.30} & 
0.65\\CON8-4 & 781.64 & 1.56 & 
790.23 & 1.52 & \bf{770.10} & 
1.50\\CON8-5 & \bf{\underline{764.36}} & 1.40 & 
764.36 & 1.38 & 766.60 & 
-0.29\\CON8-6 & 705.61 & 1.70 & 
707.08 & 1.71 & \bf{697.20} & 
1.21\\CON8-7 & 822.42 & 1.23 & 
823.18 & 1.17 & \bf{814.80} & 
0.94\\CON8-8 & 799.16 & 1.56 & 
799.38 & 1.60 & \bf{771.30} & 
3.61\\CON8-9 & 816.12 & 1.52 & 
816.12 & 1.53 & \bf{815.10} & 
0.13\\[1ex]\hline
\end{tabular}
\label{table:nonlin}
\end{table} \clearpage
\begin{table}[ht]
\caption{Resultados de la ejecución de la metaheurística ACO, utilizando instancias de Dethloff con la configuración -n 2.0 -alpha 1.0 -beta 3.0 -q 12.7 -ro 0.015}
\centering
\small
\begin{tabular}{c c c c c c c}
\hline\hline
Instancia & Costo mínimo & Tiempo(seg.) & Costo promedio & Tiempo promedio(seg.) & Costo ACO & \%Gap \\ [0.5ex]
\hline
SCA3-0 & 640.55 & 1.40 & 
640.55 & 1.34 & \bf{636.10} & 
0.70\\SCA3-1 & \bf{\underline{697.84}} & 1.43 & 
697.84 & 1.46 & 700.10 & 
-0.32\\SCA3-2 & 659.34 & 1.34 & 
662.97 & 1.35 & \bf{659.30} & 
0.01\\SCA3-3 & 680.60 & 1.48 & 
680.78 & 1.50 & \bf{680.00} & 
0.09\\SCA3-4 & \bf{690.50} & 1.46 & 
690.50 & 1.43 & 690.50 & 0.00\\
SCA3-5 & \bf{\underline{665.04}} & 1.45 & 
665.04 & 1.48 & 671.10 & 
-0.90\\SCA3-6 & 655.19 & 1.29 & 
655.19 & 1.32 & \bf{651.10} & 
0.63\\SCA3-7 & 666.15 & 1.30 & 
666.15 & 1.06 & \bf{666.10} & 
0.01\\SCA3-8 & 721.45 & 1.08 & 
724.59 & 1.36 & \bf{719.50} & 
0.27\\SCA3-9 & \bf{681.00} & 0.98 & 
681.00 & 1.01 & 681.00 & 0.00\\
SCA8-0 & 991.07 & 1.49 & 
991.65 & 1.53 & \bf{961.60} & 
3.06\\SCA8-1 & 1074.65 & 1.21 & 
1074.68 & 1.17 & \bf{1063.00} & 
1.10\\SCA8-2 & 1056.87 & 1.02 & 
1056.87 & 1.03 & \bf{1040.60} & 
1.56\\SCA8-3 & 1031.08 & 1.44 & 
1031.08 & 1.47 & \bf{985.90} & 
4.58\\SCA8-4 & 1098.34 & 1.45 & 
1098.88 & 1.49 & \bf{1071.00} & 
2.55\\SCA8-5 & 1055.35 & 1.70 & 
1055.35 & 1.69 & \bf{1054.30} & 
0.10\\SCA8-6 & \bf{\underline{972.48}} & 1.78 & 
972.48 & 1.67 & 972.50 & 
-0.00\\SCA8-7 & 1092.57 & 1.64 & 
1092.57 & 1.67 & \bf{1059.70} & 
3.10\\SCA8-8 & 1091.49 & 1.46 & 
1091.89 & 1.46 & \bf{1082.70} & 
0.81\\SCA8-9 & \bf{\underline{1067.42}} & 1.10 & 
1067.42 & 1.23 & 1081.40 & 
-1.29\\CON3-0 & 624.96 & 1.73 & 
624.96 & 1.70 & \bf{616.50} & 
1.37\\CON3-1 & 557.38 & 1.52 & 
558.30 & 1.47 & \bf{555.60} & 
0.32\\CON3-2 & 525.17 & 1.08 & 
525.52 & 1.14 & \bf{521.40} & 
0.72\\CON3-3 & \bf{591.20} & 1.57 & 
593.38 & 1.52 & 591.20 & 0.00\\
CON3-4 & 589.32 & 1.29 & 
589.32 & 1.30 & \bf{589.30} & 
0.00\\CON3-5 & 576.43 & 1.36 & 
576.43 & 1.41 & \bf{563.70} & 
2.26\\CON3-6 & 504.15 & 1.85 & 
507.33 & 1.80 & \bf{499.20} & 
0.99\\CON3-7 & 578.41 & 1.16 & 
578.41 & 1.19 & \bf{577.50} & 
0.16\\CON3-8 & 524.59 & 1.14 & 
524.59 & 1.18 & \bf{523.10} & 
0.28\\CON3-9 & 588.48 & 1.25 & 
588.48 & 1.27 & \bf{578.20} & 
1.78\\CON8-0 & 879.00 & 1.46 & 
879.00 & 1.46 & \bf{858.90} & 
2.34\\CON8-1 & 758.26 & 1.33 & 
758.26 & 1.34 & \bf{740.90} & 
2.34\\CON8-2 & 716.53 & 2.00 & 
716.54 & 2.08 & \bf{714.30} & 
0.31\\CON8-3 & 817.57 & 1.52 & 
817.57 & 1.47 & \bf{812.30} & 
0.65\\CON8-4 & 781.64 & 1.59 & 
788.67 & 1.58 & \bf{770.10} & 
1.50\\CON8-5 & \bf{\underline{764.36}} & 1.41 & 
764.36 & 1.38 & 766.60 & 
-0.29\\CON8-6 & 705.61 & 1.62 & 
706.06 & 1.75 & \bf{697.20} & 
1.21\\CON8-7 & 822.42 & 1.20 & 
822.92 & 1.20 & \bf{814.80} & 
0.94\\CON8-8 & 799.32 & 1.48 & 
799.37 & 1.55 & \bf{771.30} & 
3.63\\CON8-9 & 816.12 & 1.64 & 
816.12 & 1.57 & \bf{815.10} & 
0.13\\[1ex]\hline
\end{tabular}
\label{table:nonlin}
\end{table} \clearpage
\begin{table}[ht]
\caption{Resultados de la ejecución de la metaheurística ACO, utilizando instancias de Dethloff con la configuración -n 2.0 -alpha 1.0 -beta 3.0 -q 12.8 -ro 0.015}
\centering
\small
\begin{tabular}{c c c c c c c}
\hline\hline
Instancia & Costo mínimo & Tiempo(seg.) & Costo promedio & Tiempo promedio(seg.) & Costo ACO & \%Gap \\ [0.5ex]
\hline
SCA3-0 & 640.55 & 1.36 & 
640.55 & 1.37 & \bf{636.10} & 
0.70\\SCA3-1 & \bf{\underline{697.84}} & 1.44 & 
697.84 & 1.50 & 700.10 & 
-0.32\\SCA3-2 & 664.18 & 1.36 & 
664.18 & 1.34 & \bf{659.30} & 
0.74\\SCA3-3 & 680.60 & 1.44 & 
680.78 & 1.49 & \bf{680.00} & 
0.09\\SCA3-4 & \bf{690.50} & 1.36 & 
690.50 & 1.40 & 690.50 & 0.00\\
SCA3-5 & \bf{\underline{665.04}} & 1.38 & 
665.04 & 1.42 & 671.10 & 
-0.90\\SCA3-6 & 655.19 & 1.32 & 
655.19 & 1.36 & \bf{651.10} & 
0.63\\SCA3-7 & 666.15 & 0.96 & 
666.26 & 1.01 & \bf{666.10} & 
0.01\\SCA3-8 & 721.45 & 1.24 & 
727.13 & 1.22 & \bf{719.50} & 
0.27\\SCA3-9 & \bf{681.00} & 0.95 & 
681.00 & 0.99 & 681.00 & 0.00\\
SCA8-0 & 991.07 & 1.56 & 
991.07 & 1.52 & \bf{961.60} & 
3.06\\SCA8-1 & 1074.65 & 1.21 & 
1074.65 & 1.18 & \bf{1063.00} & 
1.10\\SCA8-2 & 1056.87 & 1.05 & 
1056.87 & 1.04 & \bf{1040.60} & 
1.56\\SCA8-3 & 1031.08 & 1.44 & 
1031.08 & 1.42 & \bf{985.90} & 
4.58\\SCA8-4 & 1099.06 & 1.49 & 
1099.06 & 1.48 & \bf{1071.00} & 
2.62\\SCA8-5 & 1055.35 & 1.75 & 
1055.35 & 1.70 & \bf{1054.30} & 
0.10\\SCA8-6 & \bf{\underline{972.48}} & 1.77 & 
977.77 & 1.70 & 972.50 & 
-0.00\\SCA8-7 & 1092.57 & 1.57 & 
1092.57 & 1.63 & \bf{1059.70} & 
3.10\\SCA8-8 & 1091.49 & 1.50 & 
1091.89 & 1.46 & \bf{1082.70} & 
0.81\\SCA8-9 & \bf{\underline{1067.42}} & 1.23 & 
1067.42 & 1.18 & 1081.40 & 
-1.29\\CON3-0 & 624.96 & 1.71 & 
624.96 & 1.63 & \bf{616.50} & 
1.37\\CON3-1 & 559.13 & 1.43 & 
560.02 & 1.45 & \bf{555.60} & 
0.64\\CON3-2 & 524.07 & 1.08 & 
525.19 & 1.11 & \bf{521.40} & 
0.51\\CON3-3 & 594.11 & 1.51 & 
594.11 & 1.55 & \bf{591.20} & 
0.49\\CON3-4 & 589.32 & 1.53 & 
589.32 & 1.45 & \bf{589.30} & 
0.00\\CON3-5 & 576.43 & 1.50 & 
576.43 & 1.46 & \bf{563.70} & 
2.26\\CON3-6 & 505.26 & 1.70 & 
505.26 & 1.78 & \bf{499.20} & 
1.21\\CON3-7 & 578.41 & 1.18 & 
579.84 & 1.23 & \bf{577.50} & 
0.16\\CON3-8 & 524.59 & 1.19 & 
524.59 & 1.19 & \bf{523.10} & 
0.28\\CON3-9 & 588.48 & 1.27 & 
588.48 & 1.29 & \bf{578.20} & 
1.78\\CON8-0 & 879.00 & 1.49 & 
879.00 & 1.49 & \bf{858.90} & 
2.34\\CON8-1 & 758.26 & 1.39 & 
758.26 & 1.50 & \bf{740.90} & 
2.34\\CON8-2 & 716.53 & 2.58 & 
716.54 & 2.15 & \bf{714.30} & 
0.31\\CON8-3 & 817.57 & 1.42 & 
817.57 & 1.43 & \bf{812.30} & 
0.65\\CON8-4 & 781.64 & 1.89 & 
788.67 & 1.81 & \bf{770.10} & 
1.50\\CON8-5 & \bf{\underline{764.36}} & 1.36 & 
764.36 & 1.35 & 766.60 & 
-0.29\\CON8-6 & \bf{\underline{693.83}} & 1.81 & 
700.17 & 1.74 & 697.20 & 
-0.48\\CON8-7 & 822.42 & 1.23 & 
822.67 & 1.19 & \bf{814.80} & 
0.94\\CON8-8 & 799.16 & 1.60 & 
799.28 & 1.58 & \bf{771.30} & 
3.61\\CON8-9 & 816.12 & 1.60 & 
816.12 & 1.55 & \bf{815.10} & 
0.13\\[1ex]\hline
\end{tabular}
\label{table:nonlin}
\end{table} \clearpage
\begin{table}[ht]
\caption{Resultados de la ejecución de la metaheurística ACO, utilizando instancias de Dethloff con la configuración -n 2.0 -alpha 1.0 -beta 3.0 -q 12.9 -ro 0.015}
\centering
\small
\begin{tabular}{c c c c c c c}
\hline\hline
Instancia & Costo mínimo & Tiempo(seg.) & Costo promedio & Tiempo promedio(seg.) & Costo ACO & \%Gap \\ [0.5ex]
\hline
SCA3-0 & 640.55 & 1.44 & 
640.55 & 1.45 & \bf{636.10} & 
0.70\\SCA3-1 & \bf{\underline{697.84}} & 1.48 & 
697.84 & 1.49 & 700.10 & 
-0.32\\SCA3-2 & 664.18 & 1.31 & 
664.18 & 1.36 & \bf{659.30} & 
0.74\\SCA3-3 & 680.60 & 1.43 & 
680.60 & 1.46 & \bf{680.00} & 
0.09\\SCA3-4 & \bf{690.50} & 1.43 & 
690.50 & 1.43 & 690.50 & 0.00\\
SCA3-5 & \bf{\underline{665.04}} & 1.38 & 
665.79 & 1.40 & 671.10 & 
-0.90\\SCA3-6 & 655.19 & 1.32 & 
655.19 & 1.40 & \bf{651.10} & 
0.63\\SCA3-7 & 666.15 & 0.98 & 
666.15 & 1.02 & \bf{666.10} & 
0.01\\SCA3-8 & 721.45 & 1.14 & 
721.45 & 1.13 & \bf{719.50} & 
0.27\\SCA3-9 & \bf{681.00} & 0.96 & 
681.00 & 0.94 & 681.00 & 0.00\\
SCA8-0 & 991.07 & 1.54 & 
991.07 & 1.51 & \bf{961.60} & 
3.06\\SCA8-1 & 1074.65 & 1.28 & 
1074.65 & 1.25 & \bf{1063.00} & 
1.10\\SCA8-2 & 1056.87 & 1.04 & 
1056.87 & 1.03 & \bf{1040.60} & 
1.56\\SCA8-3 & 1031.08 & 1.42 & 
1031.08 & 1.44 & \bf{985.90} & 
4.58\\SCA8-4 & 1099.06 & 1.60 & 
1099.06 & 1.48 & \bf{1071.00} & 
2.62\\SCA8-5 & 1055.35 & 1.68 & 
1055.35 & 1.68 & \bf{1054.30} & 
0.10\\SCA8-6 & \bf{\underline{972.48}} & 1.60 & 
972.48 & 1.67 & 972.50 & 
-0.00\\SCA8-7 & 1075.42 & 1.59 & 
1088.28 & 1.62 & \bf{1059.70} & 
1.48\\SCA8-8 & 1092.02 & 1.50 & 
1092.02 & 1.54 & \bf{1082.70} & 
0.86\\SCA8-9 & \bf{\underline{1067.42}} & 1.19 & 
1067.42 & 1.15 & 1081.40 & 
-1.29\\CON3-0 & 624.96 & 1.69 & 
624.96 & 1.68 & \bf{616.50} & 
1.37\\CON3-1 & 557.38 & 1.44 & 
558.66 & 1.47 & \bf{555.60} & 
0.32\\CON3-2 & 524.07 & 1.06 & 
525.05 & 1.09 & \bf{521.40} & 
0.51\\CON3-3 & 592.95 & 1.48 & 
593.82 & 1.50 & \bf{591.20} & 
0.30\\CON3-4 & 589.32 & 1.48 & 
589.32 & 1.35 & \bf{589.30} & 
0.00\\CON3-5 & 569.15 & 1.43 & 
574.61 & 1.45 & \bf{563.70} & 
0.97\\CON3-6 & 505.26 & 1.86 & 
505.86 & 1.81 & \bf{499.20} & 
1.21\\CON3-7 & 578.41 & 1.26 & 
579.84 & 1.22 & \bf{577.50} & 
0.16\\CON3-8 & 524.59 & 1.22 & 
524.59 & 1.18 & \bf{523.10} & 
0.28\\CON3-9 & 588.48 & 1.22 & 
588.48 & 1.26 & \bf{578.20} & 
1.78\\CON8-0 & 879.00 & 1.45 & 
879.00 & 1.44 & \bf{858.90} & 
2.34\\CON8-1 & 758.26 & 1.35 & 
758.26 & 1.34 & \bf{740.90} & 
2.34\\CON8-2 & 716.53 & 2.04 & 
716.54 & 2.01 & \bf{714.30} & 
0.31\\CON8-3 & 817.57 & 1.37 & 
817.57 & 1.43 & \bf{812.30} & 
0.65\\CON8-4 & 781.64 & 1.49 & 
789.45 & 1.55 & \bf{770.10} & 
1.50\\CON8-5 & \bf{\underline{764.36}} & 1.27 & 
764.36 & 1.31 & 766.60 & 
-0.29\\CON8-6 & 705.61 & 1.67 & 
706.06 & 1.72 & \bf{697.20} & 
1.21\\CON8-7 & 822.42 & 1.20 & 
822.67 & 1.24 & \bf{814.80} & 
0.94\\CON8-8 & 799.16 & 1.54 & 
799.28 & 1.56 & \bf{771.30} & 
3.61\\CON8-9 & 816.12 & 1.65 & 
816.12 & 1.59 & \bf{815.10} & 
0.13\\[1ex]\hline
\end{tabular}
\label{table:nonlin}
\end{table} \clearpage
\begin{table}[ht]
\caption{Resultados de la ejecución de la metaheurística ACO, utilizando instancias de Dethloff con la configuración -n 2.0 -alpha 1.0 -beta 3.0 -q 13.0 -ro 0.015}
\centering
\small
\begin{tabular}{c c c c c c c}
\hline\hline
Instancia & Costo mínimo & Tiempo(seg.) & Costo promedio & Tiempo promedio(seg.) & Costo ACO & \%Gap \\ [0.5ex]
\hline
SCA3-0 & 640.55 & 1.31 & 
640.55 & 1.38 & \bf{636.10} & 
0.70\\SCA3-1 & \bf{\underline{697.84}} & 1.56 & 
697.84 & 1.58 & 700.10 & 
-0.32\\SCA3-2 & 659.34 & 1.34 & 
662.97 & 1.35 & \bf{659.30} & 
0.01\\SCA3-3 & 680.60 & 1.46 & 
680.78 & 1.51 & \bf{680.00} & 
0.09\\SCA3-4 & \bf{690.50} & 1.38 & 
690.50 & 1.45 & 690.50 & 0.00\\
SCA3-5 & \bf{\underline{665.04}} & 1.42 & 
665.04 & 1.44 & 671.10 & 
-0.90\\SCA3-6 & 655.19 & 1.28 & 
655.19 & 1.31 & \bf{651.10} & 
0.63\\SCA3-7 & 666.15 & 0.94 & 
666.15 & 0.97 & \bf{666.10} & 
0.01\\SCA3-8 & 721.45 & 1.14 & 
722.70 & 1.15 & \bf{719.50} & 
0.27\\SCA3-9 & \bf{681.00} & 1.04 & 
681.00 & 0.96 & 681.00 & 0.00\\
SCA8-0 & 991.07 & 1.52 & 
991.07 & 1.48 & \bf{961.60} & 
3.06\\SCA8-1 & 1074.65 & 1.23 & 
1074.65 & 1.19 & \bf{1063.00} & 
1.10\\SCA8-2 & 1056.87 & 1.06 & 
1056.87 & 1.04 & \bf{1040.60} & 
1.56\\SCA8-3 & 1031.08 & 1.37 & 
1031.08 & 1.48 & \bf{985.90} & 
4.58\\SCA8-4 & 1099.06 & 1.64 & 
1099.06 & 1.55 & \bf{1071.00} & 
2.62\\SCA8-5 & 1055.35 & 1.75 & 
1055.35 & 1.69 & \bf{1054.30} & 
0.10\\SCA8-6 & \bf{\underline{972.48}} & 1.76 & 
976.70 & 1.68 & 972.50 & 
-0.00\\SCA8-7 & 1092.57 & 1.62 & 
1092.57 & 1.60 & \bf{1059.70} & 
3.10\\SCA8-8 & 1091.49 & 1.39 & 
1091.76 & 1.42 & \bf{1082.70} & 
0.81\\SCA8-9 & \bf{\underline{1067.42}} & 1.17 & 
1067.42 & 1.13 & 1081.40 & 
-1.29\\CON3-0 & 624.96 & 1.70 & 
624.96 & 1.66 & \bf{616.50} & 
1.37\\CON3-1 & 557.38 & 1.50 & 
558.30 & 1.71 & \bf{555.60} & 
0.32\\CON3-2 & 524.07 & 1.18 & 
524.79 & 1.14 & \bf{521.40} & 
0.51\\CON3-3 & 594.11 & 1.48 & 
594.11 & 1.55 & \bf{591.20} & 
0.49\\CON3-4 & 589.32 & 1.36 & 
589.32 & 1.34 & \bf{589.30} & 
0.00\\CON3-5 & 576.43 & 1.47 & 
576.43 & 1.48 & \bf{563.70} & 
2.26\\CON3-6 & 505.26 & 1.89 & 
506.43 & 1.78 & \bf{499.20} & 
1.21\\CON3-7 & 578.41 & 1.26 & 
579.12 & 1.25 & \bf{577.50} & 
0.16\\CON3-8 & 524.59 & 1.16 & 
524.59 & 1.21 & \bf{523.10} & 
0.28\\CON3-9 & 588.48 & 1.28 & 
588.48 & 1.28 & \bf{578.20} & 
1.78\\CON8-0 & 879.00 & 1.54 & 
879.00 & 1.46 & \bf{858.90} & 
2.34\\CON8-1 & 758.26 & 1.30 & 
758.26 & 1.32 & \bf{740.90} & 
2.34\\CON8-2 & 716.53 & 2.00 & 
716.55 & 1.98 & \bf{714.30} & 
0.31\\CON8-3 & 817.57 & 1.40 & 
817.57 & 1.41 & \bf{812.30} & 
0.65\\CON8-4 & 781.64 & 1.57 & 
785.81 & 1.59 & \bf{770.10} & 
1.50\\CON8-5 & \bf{\underline{764.36}} & 1.41 & 
764.36 & 1.43 & 766.60 & 
-0.29\\CON8-6 & 705.61 & 1.78 & 
706.06 & 1.71 & \bf{697.20} & 
1.21\\CON8-7 & 822.42 & 1.20 & 
823.18 & 1.18 & \bf{814.80} & 
0.94\\CON8-8 & 799.32 & 1.53 & 
799.41 & 1.54 & \bf{771.30} & 
3.63\\CON8-9 & 816.12 & 1.58 & 
817.40 & 1.57 & \bf{815.10} & 
0.13\\[1ex]\hline
\end{tabular}
\label{table:nonlin}
\end{table} \clearpage
\begin{table}[ht]
\caption{Resultados de la ejecución de la metaheurística ACO, utilizando instancias de Dethloff con la configuración -n 2.0 -alpha 1.0 -beta 3.0 -q 13.1 -ro 0.015}
\centering
\small
\begin{tabular}{c c c c c c c}
\hline\hline
Instancia & Costo mínimo & Tiempo(seg.) & Costo promedio & Tiempo promedio(seg.) & Costo ACO & \%Gap \\ [0.5ex]
\hline
SCA3-0 & 640.55 & 1.32 & 
640.55 & 1.41 & \bf{636.10} & 
0.70\\SCA3-1 & \bf{\underline{697.84}} & 1.50 & 
697.84 & 1.50 & 700.10 & 
-0.32\\SCA3-2 & 664.18 & 1.35 & 
664.18 & 1.35 & \bf{659.30} & 
0.74\\SCA3-3 & 680.60 & 1.51 & 
680.78 & 1.50 & \bf{680.00} & 
0.09\\SCA3-4 & \bf{690.50} & 1.34 & 
690.50 & 1.43 & 690.50 & 0.00\\
SCA3-5 & \bf{\underline{665.04}} & 1.47 & 
665.04 & 1.47 & 671.10 & 
-0.90\\SCA3-6 & 655.19 & 1.42 & 
655.19 & 1.36 & \bf{651.10} & 
0.63\\SCA3-7 & 666.15 & 0.98 & 
666.15 & 0.97 & \bf{666.10} & 
0.01\\SCA3-8 & 721.45 & 1.16 & 
722.95 & 1.21 & \bf{719.50} & 
0.27\\SCA3-9 & \bf{681.00} & 0.92 & 
681.00 & 0.96 & 681.00 & 0.00\\
SCA8-0 & 991.07 & 1.49 & 
991.65 & 1.52 & \bf{961.60} & 
3.06\\SCA8-1 & 1074.65 & 1.20 & 
1074.65 & 1.20 & \bf{1063.00} & 
1.10\\SCA8-2 & 1056.87 & 1.00 & 
1056.87 & 1.03 & \bf{1040.60} & 
1.56\\SCA8-3 & 1031.08 & 1.38 & 
1031.08 & 1.46 & \bf{985.90} & 
4.58\\SCA8-4 & 1099.06 & 1.53 & 
1099.17 & 1.49 & \bf{1071.00} & 
2.62\\SCA8-5 & 1055.35 & 1.64 & 
1055.35 & 1.65 & \bf{1054.30} & 
0.10\\SCA8-6 & \bf{\underline{972.48}} & 1.69 & 
972.48 & 1.70 & 972.50 & 
-0.00\\SCA8-7 & 1092.57 & 1.68 & 
1092.57 & 1.63 & \bf{1059.70} & 
3.10\\SCA8-8 & 1091.49 & 1.40 & 
1091.76 & 1.42 & \bf{1082.70} & 
0.81\\SCA8-9 & \bf{\underline{1067.42}} & 1.19 & 
1067.42 & 1.13 & 1081.40 & 
-1.29\\CON3-0 & 624.96 & 1.68 & 
624.96 & 1.67 & \bf{616.50} & 
1.37\\CON3-1 & 557.38 & 1.46 & 
558.22 & 1.46 & \bf{555.60} & 
0.32\\CON3-2 & 524.07 & 1.12 & 
525.97 & 1.11 & \bf{521.40} & 
0.51\\CON3-3 & \bf{591.20} & 1.49 & 
592.65 & 1.49 & 591.20 & 0.00\\
CON3-4 & 589.32 & 1.25 & 
589.32 & 1.33 & \bf{589.30} & 
0.00\\CON3-5 & 576.43 & 1.44 & 
576.43 & 1.47 & \bf{563.70} & 
2.26\\CON3-6 & 504.15 & 1.81 & 
506.01 & 1.75 & \bf{499.20} & 
0.99\\CON3-7 & 578.41 & 1.28 & 
579.84 & 1.20 & \bf{577.50} & 
0.16\\CON3-8 & 524.59 & 1.16 & 
524.59 & 1.16 & \bf{523.10} & 
0.28\\CON3-9 & 588.48 & 1.33 & 
588.48 & 1.29 & \bf{578.20} & 
1.78\\CON8-0 & 879.00 & 1.49 & 
879.00 & 1.46 & \bf{858.90} & 
2.34\\CON8-1 & 758.26 & 1.37 & 
758.26 & 1.36 & \bf{740.90} & 
2.34\\CON8-2 & 716.53 & 2.04 & 
716.53 & 1.99 & \bf{714.30} & 
0.31\\CON8-3 & 817.57 & 1.40 & 
817.57 & 1.43 & \bf{812.30} & 
0.65\\CON8-4 & 781.64 & 1.56 & 
788.67 & 1.53 & \bf{770.10} & 
1.50\\CON8-5 & \bf{\underline{764.36}} & 1.36 & 
764.36 & 1.47 & 766.60 & 
-0.29\\CON8-6 & \bf{\underline{693.83}} & 1.70 & 
703.23 & 1.69 & 697.20 & 
-0.48\\CON8-7 & 822.42 & 1.24 & 
822.92 & 1.18 & \bf{814.80} & 
0.94\\CON8-8 & 792.72 & 1.63 & 
797.63 & 1.59 & \bf{771.30} & 
2.78\\CON8-9 & 816.12 & 1.57 & 
816.12 & 1.59 & \bf{815.10} & 
0.13\\[1ex]\hline
\end{tabular}
\label{table:nonlin}
\end{table} \clearpage
\begin{table}[ht]
\caption{Resultados de la ejecución de la metaheurística ACO, utilizando instancias de Dethloff con la configuración -n 2.0 -alpha 1.0 -beta 3.0 -q 13.2 -ro 0.015}
\centering
\small
\begin{tabular}{c c c c c c c}
\hline\hline
Instancia & Costo mínimo & Tiempo(seg.) & Costo promedio & Tiempo promedio(seg.) & Costo ACO & \%Gap \\ [0.5ex]
\hline
SCA3-0 & 640.55 & 1.49 & 
640.55 & 1.41 & \bf{636.10} & 
0.70\\SCA3-1 & \bf{\underline{697.84}} & 1.48 & 
698.76 & 1.48 & 700.10 & 
-0.32\\SCA3-2 & 659.34 & 1.32 & 
662.97 & 1.34 & \bf{659.30} & 
0.01\\SCA3-3 & 680.60 & 1.45 & 
680.60 & 1.49 & \bf{680.00} & 
0.09\\SCA3-4 & \bf{690.50} & 1.49 & 
690.50 & 1.47 & 690.50 & 0.00\\
SCA3-5 & \bf{\underline{665.04}} & 1.44 & 
665.19 & 1.42 & 671.10 & 
-0.90\\SCA3-6 & 655.19 & 1.42 & 
655.30 & 1.37 & \bf{651.10} & 
0.63\\SCA3-7 & 666.15 & 1.07 & 
666.26 & 1.05 & \bf{666.10} & 
0.01\\SCA3-8 & 721.45 & 1.14 & 
725.24 & 1.16 & \bf{719.50} & 
0.27\\SCA3-9 & \bf{681.00} & 0.97 & 
681.00 & 1.00 & 681.00 & 0.00\\
SCA8-0 & 991.07 & 1.56 & 
992.80 & 1.52 & \bf{961.60} & 
3.06\\SCA8-1 & 1074.65 & 1.23 & 
1074.71 & 1.21 & \bf{1063.00} & 
1.10\\SCA8-2 & 1056.87 & 0.98 & 
1056.87 & 1.00 & \bf{1040.60} & 
1.56\\SCA8-3 & 1031.08 & 1.53 & 
1031.08 & 1.45 & \bf{985.90} & 
4.58\\SCA8-4 & 1099.06 & 1.46 & 
1099.06 & 1.50 & \bf{1071.00} & 
2.62\\SCA8-5 & 1055.35 & 1.74 & 
1055.35 & 1.68 & \bf{1054.30} & 
0.10\\SCA8-6 & \bf{\underline{972.48}} & 1.67 & 
972.48 & 1.69 & 972.50 & 
-0.00\\SCA8-7 & 1092.57 & 1.68 & 
1092.57 & 1.67 & \bf{1059.70} & 
3.10\\SCA8-8 & 1091.49 & 1.78 & 
1091.76 & 1.54 & \bf{1082.70} & 
0.81\\SCA8-9 & \bf{\underline{1067.42}} & 1.18 & 
1067.42 & 1.17 & 1081.40 & 
-1.29\\CON3-0 & 624.96 & 1.60 & 
624.96 & 1.64 & \bf{616.50} & 
1.37\\CON3-1 & 557.38 & 1.50 & 
558.50 & 1.48 & \bf{555.60} & 
0.32\\CON3-2 & 524.07 & 1.16 & 
524.31 & 1.11 & \bf{521.40} & 
0.51\\CON3-3 & 594.11 & 1.48 & 
594.11 & 1.53 & \bf{591.20} & 
0.49\\CON3-4 & 589.32 & 1.38 & 
589.32 & 1.36 & \bf{589.30} & 
0.00\\CON3-5 & 576.43 & 1.52 & 
576.43 & 1.48 & \bf{563.70} & 
2.26\\CON3-6 & 505.26 & 1.79 & 
505.26 & 1.77 & \bf{499.20} & 
1.21\\CON3-7 & 578.41 & 1.18 & 
579.12 & 1.20 & \bf{577.50} & 
0.16\\CON3-8 & 524.59 & 1.26 & 
524.59 & 1.24 & \bf{523.10} & 
0.28\\CON3-9 & 588.48 & 1.26 & 
588.48 & 1.25 & \bf{578.20} & 
1.78\\CON8-0 & 879.00 & 1.46 & 
879.00 & 1.50 & \bf{858.90} & 
2.34\\CON8-1 & 758.26 & 1.26 & 
758.26 & 1.29 & \bf{740.90} & 
2.34\\CON8-2 & 716.53 & 2.00 & 
716.54 & 1.99 & \bf{714.30} & 
0.31\\CON8-3 & 817.57 & 1.43 & 
817.57 & 1.44 & \bf{812.30} & 
0.65\\CON8-4 & 789.98 & 1.57 & 
791.53 & 1.52 & \bf{770.10} & 
2.58\\CON8-5 & \bf{\underline{764.36}} & 1.38 & 
764.36 & 1.35 & 766.60 & 
-0.29\\CON8-6 & 705.61 & 1.70 & 
706.18 & 1.70 & \bf{697.20} & 
1.21\\CON8-7 & 822.42 & 1.20 & 
822.67 & 1.21 & \bf{814.80} & 
0.94\\CON8-8 & 799.16 & 1.62 & 
799.33 & 1.58 & \bf{771.30} & 
3.61\\CON8-9 & 816.12 & 1.46 & 
816.12 & 1.54 & \bf{815.10} & 
0.13\\[1ex]\hline
\end{tabular}
\label{table:nonlin}
\end{table} \clearpage
\begin{table}[ht]
\caption{Resultados de la ejecución de la metaheurística ACO, utilizando instancias de Dethloff con la configuración -n 2.0 -alpha 1.0 -beta 3.0 -q 13.3 -ro 0.015}
\centering
\small
\begin{tabular}{c c c c c c c}
\hline\hline
Instancia & Costo mínimo & Tiempo(seg.) & Costo promedio & Tiempo promedio(seg.) & Costo ACO & \%Gap \\ [0.5ex]
\hline
SCA3-0 & 640.55 & 1.46 & 
640.55 & 1.33 & \bf{636.10} & 
0.70\\SCA3-1 & \bf{\underline{697.84}} & 1.46 & 
698.76 & 1.47 & 700.10 & 
-0.32\\SCA3-2 & 659.34 & 1.30 & 
662.97 & 1.32 & \bf{659.30} & 
0.01\\SCA3-3 & 680.60 & 1.42 & 
681.24 & 1.47 & \bf{680.00} & 
0.09\\SCA3-4 & \bf{690.50} & 1.49 & 
690.50 & 1.40 & 690.50 & 0.00\\
SCA3-5 & \bf{\underline{665.04}} & 1.44 & 
668.75 & 1.50 & 671.10 & 
-0.90\\SCA3-6 & 655.19 & 1.34 & 
655.19 & 1.41 & \bf{651.10} & 
0.63\\SCA3-7 & 666.15 & 0.98 & 
666.15 & 1.00 & \bf{666.10} & 
0.01\\SCA3-8 & 726.44 & 1.20 & 
727.98 & 1.18 & \bf{719.50} & 
0.96\\SCA3-9 & \bf{681.00} & 0.94 & 
681.00 & 0.93 & 681.00 & 0.00\\
SCA8-0 & 991.07 & 1.49 & 
991.07 & 1.54 & \bf{961.60} & 
3.06\\SCA8-1 & 1074.65 & 1.26 & 
1074.65 & 1.20 & \bf{1063.00} & 
1.10\\SCA8-2 & 1056.87 & 0.98 & 
1056.87 & 1.02 & \bf{1040.60} & 
1.56\\SCA8-3 & 1031.08 & 1.45 & 
1031.08 & 1.47 & \bf{985.90} & 
4.58\\SCA8-4 & 1099.06 & 1.62 & 
1099.06 & 1.49 & \bf{1071.00} & 
2.62\\SCA8-5 & 1055.35 & 1.61 & 
1055.35 & 1.67 & \bf{1054.30} & 
0.10\\SCA8-6 & \bf{\underline{972.48}} & 1.76 & 
975.00 & 1.75 & 972.50 & 
-0.00\\SCA8-7 & 1092.57 & 1.65 & 
1092.57 & 1.62 & \bf{1059.70} & 
3.10\\SCA8-8 & 1092.02 & 1.40 & 
1092.02 & 1.45 & \bf{1082.70} & 
0.86\\SCA8-9 & \bf{\underline{1067.42}} & 1.20 & 
1067.42 & 1.16 & 1081.40 & 
-1.29\\CON3-0 & 624.96 & 1.68 & 
624.96 & 1.74 & \bf{616.50} & 
1.37\\CON3-1 & 558.16 & 1.46 & 
559.70 & 1.46 & \bf{555.60} & 
0.46\\CON3-2 & 524.07 & 1.06 & 
524.79 & 1.09 & \bf{521.40} & 
0.51\\CON3-3 & 594.11 & 1.68 & 
594.11 & 1.62 & \bf{591.20} & 
0.49\\CON3-4 & \bf{\underline{588.79}} & 1.31 & 
589.19 & 1.33 & 589.30 & 
-0.09\\CON3-5 & 574.57 & 1.49 & 
575.97 & 1.44 & \bf{563.70} & 
1.93\\CON3-6 & 504.15 & 1.77 & 
506.15 & 1.82 & \bf{499.20} & 
0.99\\CON3-7 & 578.41 & 1.24 & 
578.41 & 1.26 & \bf{577.50} & 
0.16\\CON3-8 & 524.30 & 1.16 & 
524.52 & 1.24 & \bf{523.10} & 
0.23\\CON3-9 & 588.48 & 1.25 & 
588.48 & 1.28 & \bf{578.20} & 
1.78\\CON8-0 & 879.00 & 1.48 & 
879.00 & 1.47 & \bf{858.90} & 
2.34\\CON8-1 & 758.26 & 1.32 & 
758.26 & 1.36 & \bf{740.90} & 
2.34\\CON8-2 & 716.53 & 1.90 & 
716.53 & 2.07 & \bf{714.30} & 
0.31\\CON8-3 & 817.57 & 1.46 & 
817.57 & 1.47 & \bf{812.30} & 
0.65\\CON8-4 & 789.98 & 1.64 & 
791.53 & 1.59 & \bf{770.10} & 
2.58\\CON8-5 & \bf{\underline{764.36}} & 1.38 & 
764.36 & 1.35 & 766.60 & 
-0.29\\CON8-6 & \bf{\underline{693.83}} & 1.62 & 
703.12 & 1.70 & 697.20 & 
-0.48\\CON8-7 & 822.42 & 1.22 & 
823.18 & 1.21 & \bf{814.80} & 
0.94\\CON8-8 & 799.32 & 1.61 & 
799.41 & 1.52 & \bf{771.30} & 
3.63\\CON8-9 & 816.12 & 1.57 & 
816.12 & 1.54 & \bf{815.10} & 
0.13\\[1ex]\hline
\end{tabular}
\label{table:nonlin}
\end{table} \clearpage
\begin{table}[ht]
\caption{Resultados de la ejecución de la metaheurística ACO, utilizando instancias de Dethloff con la configuración -n 2.0 -alpha 1.0 -beta 3.0 -q 13.4 -ro 0.015}
\centering
\small
\begin{tabular}{c c c c c c c}
\hline\hline
Instancia & Costo mínimo & Tiempo(seg.) & Costo promedio & Tiempo promedio(seg.) & Costo ACO & \%Gap \\ [0.5ex]
\hline
SCA3-0 & 640.55 & 1.48 & 
640.55 & 1.42 & \bf{636.10} & 
0.70\\SCA3-1 & \bf{\underline{697.84}} & 1.46 & 
697.84 & 1.50 & 700.10 & 
-0.32\\SCA3-2 & 659.34 & 1.28 & 
661.76 & 1.31 & \bf{659.30} & 
0.01\\SCA3-3 & 680.60 & 1.44 & 
680.96 & 1.48 & \bf{680.00} & 
0.09\\SCA3-4 & \bf{690.50} & 1.48 & 
690.50 & 1.45 & 690.50 & 0.00\\
SCA3-5 & \bf{\underline{665.04}} & 1.36 & 
665.04 & 1.44 & 671.10 & 
-0.90\\SCA3-6 & 655.19 & 1.34 & 
655.19 & 1.32 & \bf{651.10} & 
0.63\\SCA3-7 & 666.15 & 0.94 & 
666.15 & 0.95 & \bf{666.10} & 
0.01\\SCA3-8 & 721.45 & 1.22 & 
726.48 & 1.19 & \bf{719.50} & 
0.27\\SCA3-9 & \bf{681.00} & 0.94 & 
681.00 & 0.94 & 681.00 & 0.00\\
SCA8-0 & 991.07 & 1.56 & 
991.07 & 1.50 & \bf{961.60} & 
3.06\\SCA8-1 & 1074.65 & 1.14 & 
1074.68 & 1.16 & \bf{1063.00} & 
1.10\\SCA8-2 & 1056.87 & 1.04 & 
1056.87 & 1.01 & \bf{1040.60} & 
1.56\\SCA8-3 & 1031.08 & 1.50 & 
1031.08 & 1.46 & \bf{985.90} & 
4.58\\SCA8-4 & 1099.06 & 1.47 & 
1099.06 & 1.48 & \bf{1071.00} & 
2.62\\SCA8-5 & 1055.35 & 1.59 & 
1055.35 & 1.60 & \bf{1054.30} & 
0.10\\SCA8-6 & \bf{\underline{972.48}} & 1.60 & 
972.48 & 1.67 & 972.50 & 
-0.00\\SCA8-7 & 1075.42 & 1.55 & 
1088.28 & 1.62 & \bf{1059.70} & 
1.48\\SCA8-8 & 1091.49 & 1.46 & 
1091.76 & 1.50 & \bf{1082.70} & 
0.81\\SCA8-9 & \bf{\underline{1067.42}} & 1.09 & 
1067.42 & 1.14 & 1081.40 & 
-1.29\\CON3-0 & 624.96 & 1.68 & 
624.96 & 1.62 & \bf{616.50} & 
1.37\\CON3-1 & 557.21 & 1.42 & 
559.10 & 1.47 & \bf{555.60} & 
0.29\\CON3-2 & 524.07 & 1.25 & 
525.47 & 1.12 & \bf{521.40} & 
0.51\\CON3-3 & 594.11 & 1.50 & 
594.11 & 1.52 & \bf{591.20} & 
0.49\\CON3-4 & \bf{\underline{588.79}} & 1.31 & 
589.19 & 1.38 & 589.30 & 
-0.09\\CON3-5 & 576.43 & 1.44 & 
576.43 & 1.48 & \bf{563.70} & 
2.26\\CON3-6 & 504.15 & 1.79 & 
505.47 & 1.79 & \bf{499.20} & 
0.99\\CON3-7 & 578.41 & 1.22 & 
579.12 & 1.24 & \bf{577.50} & 
0.16\\CON3-8 & 524.30 & 1.17 & 
524.45 & 1.20 & \bf{523.10} & 
0.23\\CON3-9 & 588.48 & 1.38 & 
588.48 & 1.33 & \bf{578.20} & 
1.78\\CON8-0 & 879.00 & 1.38 & 
879.00 & 1.45 & \bf{858.90} & 
2.34\\CON8-1 & 758.26 & 1.44 & 
758.26 & 1.38 & \bf{740.90} & 
2.34\\CON8-2 & 716.53 & 2.02 & 
716.54 & 1.98 & \bf{714.30} & 
0.31\\CON8-3 & 817.57 & 1.40 & 
817.57 & 1.41 & \bf{812.30} & 
0.65\\CON8-4 & 781.64 & 1.60 & 
785.81 & 1.58 & \bf{770.10} & 
1.50\\CON8-5 & \bf{\underline{764.36}} & 1.34 & 
764.36 & 1.36 & 766.60 & 
-0.29\\CON8-6 & 705.61 & 1.70 & 
706.21 & 1.75 & \bf{697.20} & 
1.21\\CON8-7 & 822.42 & 1.22 & 
822.67 & 1.18 & \bf{814.80} & 
0.94\\CON8-8 & 799.32 & 1.90 & 
799.37 & 1.69 & \bf{771.30} & 
3.63\\CON8-9 & 816.12 & 1.56 & 
816.12 & 1.59 & \bf{815.10} & 
0.13\\[1ex]\hline
\end{tabular}
\label{table:nonlin}
\end{table} \clearpage
\begin{table}[ht]
\caption{Resultados de la ejecución de la metaheurística ACO, utilizando instancias de Dethloff con la configuración -n 2.0 -alpha 1.0 -beta 3.0 -q 13.5 -ro 0.015}
\centering
\small
\begin{tabular}{c c c c c c c}
\hline\hline
Instancia & Costo mínimo & Tiempo(seg.) & Costo promedio & Tiempo promedio(seg.) & Costo ACO & \%Gap \\ [0.5ex]
\hline
SCA3-0 & 640.55 & 1.46 & 
640.55 & 1.37 & \bf{636.10} & 
0.70\\SCA3-1 & \bf{\underline{697.84}} & 1.54 & 
697.84 & 1.51 & 700.10 & 
-0.32\\SCA3-2 & 664.18 & 1.54 & 
664.18 & 1.41 & \bf{659.30} & 
0.74\\SCA3-3 & 680.60 & 1.61 & 
680.96 & 1.50 & \bf{680.00} & 
0.09\\SCA3-4 & \bf{690.50} & 1.44 & 
690.50 & 1.41 & 690.50 & 0.00\\
SCA3-5 & \bf{\underline{665.04}} & 1.40 & 
665.04 & 1.44 & 671.10 & 
-0.90\\SCA3-6 & 655.19 & 1.27 & 
655.19 & 1.35 & \bf{651.10} & 
0.63\\SCA3-7 & 666.15 & 0.97 & 
666.15 & 0.98 & \bf{666.10} & 
0.01\\SCA3-8 & 721.45 & 1.18 & 
723.34 & 1.15 & \bf{719.50} & 
0.27\\SCA3-9 & \bf{681.00} & 0.98 & 
681.00 & 1.02 & 681.00 & 0.00\\
SCA8-0 & 991.07 & 1.49 & 
991.65 & 1.52 & \bf{961.60} & 
3.06\\SCA8-1 & 1069.40 & 1.18 & 
1073.37 & 1.19 & \bf{1063.00} & 
0.60\\SCA8-2 & 1056.87 & 1.00 & 
1056.87 & 1.00 & \bf{1040.60} & 
1.56\\SCA8-3 & 1031.08 & 1.48 & 
1031.08 & 1.44 & \bf{985.90} & 
4.58\\SCA8-4 & 1099.06 & 1.46 & 
1099.06 & 1.47 & \bf{1071.00} & 
2.62\\SCA8-5 & 1055.35 & 1.76 & 
1055.35 & 1.69 & \bf{1054.30} & 
0.10\\SCA8-6 & \bf{\underline{972.48}} & 1.77 & 
975.00 & 1.67 & 972.50 & 
-0.00\\SCA8-7 & 1092.57 & 1.69 & 
1092.57 & 1.66 & \bf{1059.70} & 
3.10\\SCA8-8 & 1091.49 & 1.37 & 
1091.89 & 1.44 & \bf{1082.70} & 
0.81\\SCA8-9 & \bf{\underline{1067.42}} & 1.12 & 
1067.42 & 1.11 & 1081.40 & 
-1.29\\CON3-0 & 624.96 & 1.59 & 
624.96 & 1.61 & \bf{616.50} & 
1.37\\CON3-1 & 557.38 & 1.44 & 
559.07 & 1.42 & \bf{555.60} & 
0.32\\CON3-2 & 524.07 & 1.10 & 
525.71 & 1.09 & \bf{521.40} & 
0.51\\CON3-3 & 594.11 & 1.54 & 
594.11 & 1.56 & \bf{591.20} & 
0.49\\CON3-4 & \bf{\underline{588.79}} & 1.25 & 
589.19 & 1.31 & 589.30 & 
-0.09\\CON3-5 & 569.88 & 1.36 & 
573.36 & 1.43 & \bf{563.70} & 
1.10\\CON3-6 & 504.15 & 1.80 & 
504.98 & 1.75 & \bf{499.20} & 
0.99\\CON3-7 & 578.41 & 1.32 & 
579.12 & 1.25 & \bf{577.50} & 
0.16\\CON3-8 & 524.30 & 1.28 & 
524.52 & 1.18 & \bf{523.10} & 
0.23\\CON3-9 & 588.48 & 1.30 & 
588.48 & 1.29 & \bf{578.20} & 
1.78\\CON8-0 & 879.00 & 1.46 & 
879.00 & 1.45 & \bf{858.90} & 
2.34\\CON8-1 & 758.26 & 1.28 & 
758.26 & 1.38 & \bf{740.90} & 
2.34\\CON8-2 & 716.53 & 1.92 & 
716.55 & 2.01 & \bf{714.30} & 
0.31\\CON8-3 & 817.57 & 1.47 & 
817.57 & 1.40 & \bf{812.30} & 
0.65\\CON8-4 & 778.60 & 1.68 & 
783.74 & 1.57 & \bf{770.10} & 
1.10\\CON8-5 & \bf{\underline{764.36}} & 1.32 & 
764.36 & 1.37 & 766.60 & 
-0.29\\CON8-6 & \bf{\underline{693.83}} & 1.76 & 
703.68 & 1.75 & 697.20 & 
-0.48\\CON8-7 & 822.42 & 1.20 & 
823.03 & 1.22 & \bf{814.80} & 
0.94\\CON8-8 & 799.16 & 1.72 & 
799.38 & 1.58 & \bf{771.30} & 
3.61\\CON8-9 & 816.12 & 1.60 & 
816.12 & 1.61 & \bf{815.10} & 
0.13\\[1ex]\hline
\end{tabular}
\label{table:nonlin}
\end{table} \clearpage
\begin{table}[ht]
\caption{Resultados de la ejecución de la metaheurística ACO, utilizando instancias de Dethloff con la configuración -n 2.0 -alpha 1.0 -beta 3.0 -q 13.6 -ro 0.015}
\centering
\small
\begin{tabular}{c c c c c c c}
\hline\hline
Instancia & Costo mínimo & Tiempo(seg.) & Costo promedio & Tiempo promedio(seg.) & Costo ACO & \%Gap \\ [0.5ex]
\hline
SCA3-0 & 640.55 & 1.33 & 
640.55 & 1.39 & \bf{636.10} & 
0.70\\SCA3-1 & \bf{\underline{697.84}} & 1.45 & 
697.84 & 1.48 & 700.10 & 
-0.32\\SCA3-2 & 659.34 & 1.33 & 
661.76 & 1.34 & \bf{659.30} & 
0.01\\SCA3-3 & 680.60 & 1.43 & 
681.13 & 1.46 & \bf{680.00} & 
0.09\\SCA3-4 & \bf{690.50} & 1.43 & 
690.50 & 1.40 & 690.50 & 0.00\\
SCA3-5 & \bf{\underline{665.04}} & 1.35 & 
665.39 & 1.37 & 671.10 & 
-0.90\\SCA3-6 & 655.19 & 1.32 & 
655.19 & 1.39 & \bf{651.10} & 
0.63\\SCA3-7 & 666.15 & 1.08 & 
666.15 & 1.03 & \bf{666.10} & 
0.01\\SCA3-8 & 721.45 & 1.22 & 
724.19 & 1.16 & \bf{719.50} & 
0.27\\SCA3-9 & \bf{681.00} & 1.04 & 
681.00 & 0.99 & 681.00 & 0.00\\
SCA8-0 & 991.07 & 2.70 & 
991.65 & 1.81 & \bf{961.60} & 
3.06\\SCA8-1 & 1074.39 & 1.22 & 
1074.59 & 1.19 & \bf{1063.00} & 
1.07\\SCA8-2 & 1056.87 & 1.04 & 
1056.87 & 1.02 & \bf{1040.60} & 
1.56\\SCA8-3 & 1031.08 & 1.44 & 
1031.08 & 1.46 & \bf{985.90} & 
4.58\\SCA8-4 & 1099.06 & 1.61 & 
1099.06 & 1.54 & \bf{1071.00} & 
2.62\\SCA8-5 & 1055.35 & 1.67 & 
1055.35 & 1.72 & \bf{1054.30} & 
0.10\\SCA8-6 & \bf{\underline{972.48}} & 1.72 & 
972.48 & 1.70 & 972.50 & 
-0.00\\SCA8-7 & 1092.57 & 1.73 & 
1092.57 & 1.67 & \bf{1059.70} & 
3.10\\SCA8-8 & 1092.02 & 1.44 & 
1092.02 & 1.46 & \bf{1082.70} & 
0.86\\SCA8-9 & \bf{\underline{1067.42}} & 1.13 & 
1067.42 & 1.13 & 1081.40 & 
-1.29\\CON3-0 & 624.96 & 1.64 & 
624.96 & 1.76 & \bf{616.50} & 
1.37\\CON3-1 & 557.38 & 1.42 & 
559.50 & 1.45 & \bf{555.60} & 
0.32\\CON3-2 & 524.07 & 1.15 & 
524.51 & 1.16 & \bf{521.40} & 
0.51\\CON3-3 & \bf{591.20} & 1.52 & 
593.38 & 1.48 & 591.20 & 0.00\\
CON3-4 & 589.32 & 1.38 & 
589.32 & 1.36 & \bf{589.30} & 
0.00\\CON3-5 & 576.43 & 1.47 & 
576.97 & 1.54 & \bf{563.70} & 
2.26\\CON3-6 & 502.29 & 1.80 & 
504.52 & 1.82 & \bf{499.20} & 
0.62\\CON3-7 & 578.41 & 1.34 & 
579.12 & 1.21 & \bf{577.50} & 
0.16\\CON3-8 & 524.30 & 1.30 & 
524.45 & 1.21 & \bf{523.10} & 
0.23\\CON3-9 & 588.48 & 1.20 & 
588.48 & 1.25 & \bf{578.20} & 
1.78\\CON8-0 & 879.00 & 1.52 & 
879.00 & 1.47 & \bf{858.90} & 
2.34\\CON8-1 & 758.26 & 1.35 & 
758.26 & 1.35 & \bf{740.90} & 
2.34\\CON8-2 & 716.53 & 2.16 & 
716.55 & 2.09 & \bf{714.30} & 
0.31\\CON8-3 & 817.57 & 1.48 & 
817.57 & 1.46 & \bf{812.30} & 
0.65\\CON8-4 & 789.98 & 1.50 & 
789.98 & 1.57 & \bf{770.10} & 
2.58\\CON8-5 & \bf{\underline{764.36}} & 1.41 & 
764.36 & 1.38 & 766.60 & 
-0.29\\CON8-6 & \bf{\underline{693.83}} & 1.70 & 
703.23 & 1.74 & 697.20 & 
-0.48\\CON8-7 & 822.42 & 1.20 & 
822.67 & 1.18 & \bf{814.80} & 
0.94\\CON8-8 & 799.16 & 1.52 & 
799.38 & 1.53 & \bf{771.30} & 
3.61\\CON8-9 & 816.12 & 1.64 & 
816.12 & 1.61 & \bf{815.10} & 
0.13\\[1ex]\hline
\end{tabular}
\label{table:nonlin}
\end{table} \clearpage
\begin{table}[ht]
\caption{Resultados de la ejecución de la metaheurística ACO, utilizando instancias de Dethloff con la configuración -n 2.0 -alpha 1.0 -beta 3.0 -q 13.7 -ro 0.015}
\centering
\small
\begin{tabular}{c c c c c c c}
\hline\hline
Instancia & Costo mínimo & Tiempo(seg.) & Costo promedio & Tiempo promedio(seg.) & Costo ACO & \%Gap \\ [0.5ex]
\hline
SCA3-0 & 640.55 & 1.42 & 
640.55 & 1.42 & \bf{636.10} & 
0.70\\SCA3-1 & \bf{\underline{697.84}} & 1.58 & 
697.84 & 1.54 & 700.10 & 
-0.32\\SCA3-2 & 659.34 & 1.37 & 
662.97 & 1.39 & \bf{659.30} & 
0.01\\SCA3-3 & 680.60 & 1.48 & 
680.96 & 1.50 & \bf{680.00} & 
0.09\\SCA3-4 & \bf{690.50} & 1.42 & 
690.50 & 1.39 & 690.50 & 0.00\\
SCA3-5 & \bf{\underline{665.04}} & 1.50 & 
665.04 & 1.48 & 671.10 & 
-0.90\\SCA3-6 & 655.19 & 1.47 & 
655.19 & 1.43 & \bf{651.10} & 
0.63\\SCA3-7 & 666.15 & 0.96 & 
666.15 & 1.00 & \bf{666.10} & 
0.01\\SCA3-8 & 721.45 & 1.11 & 
724.59 & 1.17 & \bf{719.50} & 
0.27\\SCA3-9 & \bf{681.00} & 1.57 & 
681.00 & 1.33 & 681.00 & 0.00\\
SCA8-0 & 991.07 & 1.55 & 
991.07 & 1.62 & \bf{961.60} & 
3.06\\SCA8-1 & 1074.65 & 1.15 & 
1074.65 & 1.20 & \bf{1063.00} & 
1.10\\SCA8-2 & 1056.87 & 1.04 & 
1056.87 & 1.03 & \bf{1040.60} & 
1.56\\SCA8-3 & 1031.08 & 1.55 & 
1031.08 & 1.48 & \bf{985.90} & 
4.58\\SCA8-4 & 1098.34 & 1.50 & 
1098.70 & 1.49 & \bf{1071.00} & 
2.55\\SCA8-5 & 1055.35 & 1.64 & 
1055.35 & 1.75 & \bf{1054.30} & 
0.10\\SCA8-6 & 982.57 & 1.56 & 
985.97 & 1.68 & \bf{972.50} & 
1.04\\SCA8-7 & 1092.57 & 1.66 & 
1092.57 & 1.61 & \bf{1059.70} & 
3.10\\SCA8-8 & 1092.02 & 1.47 & 
1092.02 & 1.43 & \bf{1082.70} & 
0.86\\SCA8-9 & \bf{\underline{1067.42}} & 1.08 & 
1067.42 & 1.11 & 1081.40 & 
-1.29\\CON3-0 & 624.96 & 1.56 & 
624.96 & 1.74 & \bf{616.50} & 
1.37\\CON3-1 & 557.38 & 1.48 & 
559.14 & 1.46 & \bf{555.60} & 
0.32\\CON3-2 & 525.17 & 1.14 & 
525.50 & 1.12 & \bf{521.40} & 
0.72\\CON3-3 & \bf{591.20} & 1.46 & 
593.38 & 1.51 & 591.20 & 0.00\\
CON3-4 & 589.32 & 1.23 & 
590.74 & 1.35 & \bf{589.30} & 
0.00\\CON3-5 & 569.15 & 2.21 & 
574.61 & 1.63 & \bf{563.70} & 
0.97\\CON3-6 & 505.26 & 1.87 & 
507.04 & 1.83 & \bf{499.20} & 
1.21\\CON3-7 & 578.41 & 1.19 & 
578.41 & 1.21 & \bf{577.50} & 
0.16\\CON3-8 & 524.59 & 1.27 & 
524.59 & 1.25 & \bf{523.10} & 
0.28\\CON3-9 & 588.48 & 1.40 & 
588.48 & 1.28 & \bf{578.20} & 
1.78\\CON8-0 & 879.00 & 1.40 & 
879.00 & 1.42 & \bf{858.90} & 
2.34\\CON8-1 & 758.26 & 1.29 & 
758.26 & 1.35 & \bf{740.90} & 
2.34\\CON8-2 & 716.53 & 2.05 & 
716.53 & 2.02 & \bf{714.30} & 
0.31\\CON8-3 & 817.57 & 1.48 & 
817.57 & 1.46 & \bf{812.30} & 
0.65\\CON8-4 & 781.64 & 1.56 & 
781.64 & 1.57 & \bf{770.10} & 
1.50\\CON8-5 & \bf{\underline{764.36}} & 1.30 & 
764.36 & 1.33 & 766.60 & 
-0.29\\CON8-6 & 705.61 & 1.68 & 
706.63 & 1.70 & \bf{697.20} & 
1.21\\CON8-7 & 822.42 & 1.23 & 
823.18 & 1.20 & \bf{814.80} & 
0.94\\CON8-8 & 799.16 & 1.63 & 
799.33 & 1.60 & \bf{771.30} & 
3.61\\CON8-9 & 816.12 & 1.48 & 
818.68 & 1.51 & \bf{815.10} & 
0.13\\[1ex]\hline
\end{tabular}
\label{table:nonlin}
\end{table} \clearpage
\begin{table}[ht]
\caption{Resultados de la ejecución de la metaheurística ACO, utilizando instancias de Dethloff con la configuración -n 2.0 -alpha 1.0 -beta 3.0 -q 13.8 -ro 0.015}
\centering
\small
\begin{tabular}{c c c c c c c}
\hline\hline
Instancia & Costo mínimo & Tiempo(seg.) & Costo promedio & Tiempo promedio(seg.) & Costo ACO & \%Gap \\ [0.5ex]
\hline
SCA3-0 & 640.55 & 1.34 & 
640.55 & 1.36 & \bf{636.10} & 
0.70\\SCA3-1 & \bf{\underline{697.84}} & 1.49 & 
697.84 & 1.51 & 700.10 & 
-0.32\\SCA3-2 & 664.18 & 1.28 & 
664.31 & 1.34 & \bf{659.30} & 
0.74\\SCA3-3 & 680.60 & 1.57 & 
680.78 & 1.52 & \bf{680.00} & 
0.09\\SCA3-4 & \bf{690.50} & 1.40 & 
690.50 & 1.43 & 690.50 & 0.00\\
SCA3-5 & \bf{\underline{665.04}} & 1.53 & 
665.04 & 1.50 & 671.10 & 
-0.90\\SCA3-6 & 653.69 & 1.31 & 
654.93 & 1.33 & \bf{651.10} & 
0.40\\SCA3-7 & 666.15 & 0.97 & 
666.15 & 1.00 & \bf{666.10} & 
0.01\\SCA3-8 & 721.45 & 1.14 & 
724.59 & 1.13 & \bf{719.50} & 
0.27\\SCA3-9 & \bf{681.00} & 1.03 & 
681.00 & 1.01 & 681.00 & 0.00\\
SCA8-0 & 991.07 & 1.60 & 
992.23 & 1.54 & \bf{961.60} & 
3.06\\SCA8-1 & 1074.39 & 1.23 & 
1074.59 & 1.24 & \bf{1063.00} & 
1.07\\SCA8-2 & 1056.87 & 1.02 & 
1056.87 & 1.03 & \bf{1040.60} & 
1.56\\SCA8-3 & 1031.08 & 1.48 & 
1031.08 & 1.49 & \bf{985.90} & 
4.58\\SCA8-4 & 1099.06 & 1.53 & 
1099.06 & 1.53 & \bf{1071.00} & 
2.62\\SCA8-5 & 1055.35 & 1.70 & 
1055.35 & 1.68 & \bf{1054.30} & 
0.10\\SCA8-6 & \bf{\underline{972.48}} & 1.65 & 
976.70 & 1.65 & 972.50 & 
-0.00\\SCA8-7 & 1075.42 & 1.66 & 
1088.28 & 1.62 & \bf{1059.70} & 
1.48\\SCA8-8 & 1091.49 & 1.45 & 
1091.75 & 1.46 & \bf{1082.70} & 
0.81\\SCA8-9 & \bf{\underline{1067.42}} & 1.12 & 
1067.42 & 1.16 & 1081.40 & 
-1.29\\CON3-0 & 624.96 & 1.72 & 
624.96 & 1.67 & \bf{616.50} & 
1.37\\CON3-1 & 557.38 & 1.42 & 
557.77 & 1.42 & \bf{555.60} & 
0.32\\CON3-2 & 524.07 & 1.12 & 
524.83 & 1.12 & \bf{521.40} & 
0.51\\CON3-3 & \bf{591.20} & 1.41 & 
593.38 & 1.46 & 591.20 & 0.00\\
CON3-4 & 589.32 & 1.42 & 
589.32 & 1.38 & \bf{589.30} & 
0.00\\CON3-5 & 570.70 & 1.36 & 
575.00 & 1.42 & \bf{563.70} & 
1.24\\CON3-6 & 504.15 & 1.90 & 
505.88 & 1.80 & \bf{499.20} & 
0.99\\CON3-7 & 578.41 & 1.17 & 
578.41 & 1.19 & \bf{577.50} & 
0.16\\CON3-8 & 524.59 & 1.32 & 
524.59 & 1.23 & \bf{523.10} & 
0.28\\CON3-9 & 586.17 & 1.23 & 
587.90 & 1.27 & \bf{578.20} & 
1.38\\CON8-0 & 879.00 & 1.38 & 
879.00 & 1.44 & \bf{858.90} & 
2.34\\CON8-1 & 758.26 & 1.32 & 
758.26 & 1.33 & \bf{740.90} & 
2.34\\CON8-2 & 716.53 & 2.13 & 
716.54 & 2.03 & \bf{714.30} & 
0.31\\CON8-3 & 817.57 & 1.39 & 
817.57 & 1.43 & \bf{812.30} & 
0.65\\CON8-4 & 781.64 & 1.53 & 
784.50 & 1.57 & \bf{770.10} & 
1.50\\CON8-5 & \bf{\underline{764.36}} & 1.36 & 
764.36 & 1.33 & 766.60 & 
-0.29\\CON8-6 & 705.61 & 1.60 & 
707.20 & 1.68 & \bf{697.20} & 
1.21\\CON8-7 & 822.42 & 1.23 & 
822.92 & 1.33 & \bf{814.80} & 
0.94\\CON8-8 & 799.32 & 1.67 & 
799.41 & 1.58 & \bf{771.30} & 
3.63\\CON8-9 & 816.12 & 1.65 & 
816.12 & 1.61 & \bf{815.10} & 
0.13\\[1ex]\hline
\end{tabular}
\label{table:nonlin}
\end{table} \clearpage
\begin{table}[ht]
\caption{Resultados de la ejecución de la metaheurística ACO, utilizando instancias de Dethloff con la configuración -n 2.0 -alpha 1.0 -beta 3.0 -q 0.1 -ro 0.015}
\centering
\small
\begin{tabular}{c c c c c c c}
\hline\hline
Instancia & Costo mínimo & Tiempo(seg.) & Costo promedio & Tiempo promedio(seg.) & Costo ACO & \%Gap \\ [0.5ex]
\hline
SCA3-0 & \bf{\underline{636.06}} & 1.80 & 
636.20 & 1.56 & 636.10 & 
-0.01\\SCA3-1 & \bf{\underline{697.84}} & 1.53 & 
698.76 & 1.56 & 700.10 & 
-0.32\\SCA3-2 & 664.18 & 1.48 & 
666.29 & 1.43 & \bf{659.30} & 
0.74\\SCA3-3 & 680.04 & 1.38 & 
680.18 & 1.44 & \bf{680.00} & 
0.01\\SCA3-4 & \bf{690.50} & 1.65 & 
690.50 & 1.55 & 690.50 & 0.00\\
SCA3-5 & \bf{\underline{669.68}} & 1.47 & 
671.18 & 1.49 & 671.10 & 
-0.21\\SCA3-6 & 652.94 & 1.56 & 
653.47 & 1.50 & \bf{651.10} & 
0.28\\SCA3-7 & \bf{\underline{664.88}} & 1.09 & 
665.83 & 1.21 & 666.10 & 
-0.18\\SCA3-8 & \bf{\underline{719.47}} & 1.66 & 
724.10 & 1.55 & 719.50 & 
-0.00\\SCA3-9 & \bf{681.00} & 1.20 & 
681.86 & 1.26 & 681.00 & 0.00\\
SCA8-0 & 982.79 & 1.54 & 
989.70 & 1.59 & \bf{961.60} & 
2.20\\SCA8-1 & \bf{\underline{1060.94}} & 1.37 & 
1069.19 & 1.42 & 1063.00 & 
-0.19\\SCA8-2 & 1047.63 & 1.26 & 
1051.43 & 1.42 & \bf{1040.60} & 
0.68\\SCA8-3 & 991.84 & 1.54 & 
1007.21 & 1.52 & \bf{985.90} & 
0.60\\SCA8-4 & \bf{\underline{1067.28}} & 1.54 & 
1072.08 & 1.60 & 1071.00 & 
-0.35\\SCA8-5 & \bf{\underline{1042.30}} & 1.75 & 
1053.96 & 1.78 & 1054.30 & 
-1.14\\SCA8-6 & 982.49 & 1.64 & 
984.99 & 1.63 & \bf{972.50} & 
1.03\\SCA8-7 & 1063.22 & 1.65 & 
1069.92 & 1.67 & \bf{1059.70} & 
0.33\\SCA8-8 & \bf{\underline{1071.18}} & 1.66 & 
1078.86 & 1.66 & 1082.70 & 
-1.06\\SCA8-9 & \bf{\underline{1067.42}} & 1.48 & 
1069.29 & 1.47 & 1081.40 & 
-1.29\\CON3-0 & 619.09 & 1.59 & 
624.72 & 1.59 & \bf{616.50} & 
0.42\\CON3-1 & \bf{\underline{554.47}} & 1.46 & 
557.58 & 1.48 & 555.60 & 
-0.20\\CON3-2 & \bf{\underline{521.38}} & 1.52 & 
523.37 & 1.46 & 521.40 & 
-0.00\\CON3-3 & \bf{591.20} & 1.59 & 
591.20 & 1.68 & 591.20 & 0.00\\
CON3-4 & \bf{\underline{588.79}} & 1.40 & 
589.06 & 1.39 & 589.30 & 
-0.09\\CON3-5 & 564.88 & 1.55 & 
569.77 & 1.45 & \bf{563.70} & 
0.21\\CON3-6 & 502.16 & 1.88 & 
503.42 & 1.76 & \bf{499.20} & 
0.59\\CON3-7 & 578.41 & 1.41 & 
580.56 & 1.50 & \bf{577.50} & 
0.16\\CON3-8 & 523.68 & 1.75 & 
524.29 & 1.56 & \bf{523.10} & 
0.11\\CON3-9 & 578.98 & 1.42 & 
586.36 & 1.44 & \bf{578.20} & 
0.13\\CON8-0 & 866.32 & 1.52 & 
873.22 & 1.57 & \bf{858.90} & 
0.86\\CON8-1 & 749.41 & 1.54 & 
752.61 & 1.58 & \bf{740.90} & 
1.15\\CON8-2 & \bf{\underline{713.60}} & 1.76 & 
716.01 & 1.81 & 714.30 & 
-0.10\\CON8-3 & 822.15 & 1.73 & 
824.70 & 1.66 & \bf{812.30} & 
1.21\\CON8-4 & 787.80 & 1.51 & 
790.20 & 1.54 & \bf{770.10} & 
2.30\\CON8-5 & \bf{\underline{760.86}} & 1.56 & 
764.55 & 1.66 & 766.60 & 
-0.75\\CON8-6 & \bf{\underline{689.11}} & 1.76 & 
694.38 & 1.74 & 697.20 & 
-1.16\\CON8-7 & \bf{\underline{814.79}} & 1.51 & 
818.00 & 1.53 & 814.80 & 
-0.00\\CON8-8 & 782.93 & 1.99 & 
785.94 & 1.82 & \bf{771.30} & 
1.51\\CON8-9 & \bf{\underline{811.43}} & 1.63 & 
814.62 & 1.67 & 815.10 & 
-0.45\\[1ex]\hline
\end{tabular}
\label{table:nonlin}
\end{table} \clearpage
\begin{table}[ht]
\caption{Resultados de la ejecución de la metaheurística ACO, utilizando instancias de Dethloff con la configuración -n 2.0 -alpha 1.0 -beta 3.0 -q .2 -ro 0.015}
\centering
\small
\begin{tabular}{c c c c c c c}
\hline\hline
Instancia & Costo mínimo & Tiempo(seg.) & Costo promedio & Tiempo promedio(seg.) & Costo ACO & \%Gap \\ [0.5ex]
\hline
SCA3-0 & \bf{\underline{636.06}} & 1.38 & 
637.47 & 1.51 & 636.10 & 
-0.01\\SCA3-1 & \bf{\underline{697.84}} & 1.52 & 
698.76 & 1.56 & 700.10 & 
-0.32\\SCA3-2 & 659.34 & 1.30 & 
663.59 & 1.35 & \bf{659.30} & 
0.01\\SCA3-3 & 680.04 & 1.33 & 
680.04 & 1.31 & \bf{680.00} & 
0.01\\SCA3-4 & \bf{690.50} & 1.47 & 
691.02 & 1.44 & 690.50 & 0.00\\
SCA3-5 & \bf{\underline{659.90}} & 1.40 & 
663.75 & 1.42 & 671.10 & 
-1.67\\SCA3-6 & 652.94 & 1.43 & 
653.47 & 1.42 & \bf{651.10} & 
0.28\\SCA3-7 & 666.15 & 1.25 & 
666.95 & 1.23 & \bf{666.10} & 
0.01\\SCA3-8 & \bf{\underline{719.47}} & 1.50 & 
721.66 & 1.49 & 719.50 & 
-0.00\\SCA3-9 & \bf{681.00} & 1.29 & 
681.86 & 1.23 & 681.00 & 0.00\\
SCA8-0 & 965.26 & 1.53 & 
976.03 & 1.55 & \bf{961.60} & 
0.38\\SCA8-1 & \bf{\underline{1060.23}} & 1.34 & 
1065.87 & 1.38 & 1063.00 & 
-0.26\\SCA8-2 & 1050.37 & 1.34 & 
1051.46 & 1.32 & \bf{1040.60} & 
0.94\\SCA8-3 & 1006.94 & 1.47 & 
1013.42 & 1.52 & \bf{985.90} & 
2.13\\SCA8-4 & \bf{\underline{1067.55}} & 1.51 & 
1073.91 & 1.58 & 1071.00 & 
-0.32\\SCA8-5 & \bf{\underline{1034.74}} & 2.10 & 
1048.32 & 1.77 & 1054.30 & 
-1.86\\SCA8-6 & \bf{\underline{972.48}} & 1.75 & 
980.80 & 1.67 & 972.50 & 
-0.00\\SCA8-7 & 1066.65 & 1.55 & 
1073.44 & 1.65 & \bf{1059.70} & 
0.66\\SCA8-8 & \bf{\underline{1080.75}} & 1.72 & 
1083.82 & 1.78 & 1082.70 & 
-0.18\\SCA8-9 & \bf{\underline{1063.68}} & 1.49 & 
1066.49 & 1.43 & 1081.40 & 
-1.64\\CON3-0 & 617.59 & 1.68 & 
619.97 & 1.62 & \bf{616.50} & 
0.18\\CON3-1 & 557.21 & 1.48 & 
558.75 & 1.54 & \bf{555.60} & 
0.29\\CON3-2 & \bf{\underline{521.38}} & 1.51 & 
522.90 & 1.49 & 521.40 & 
-0.00\\CON3-3 & \bf{591.20} & 1.75 & 
591.20 & 1.66 & 591.20 & 0.00\\
CON3-4 & 589.32 & 1.39 & 
593.48 & 1.42 & \bf{589.30} & 
0.00\\CON3-5 & \bf{563.70} & 1.32 & 
569.06 & 1.36 & 563.70 & 0.00\\
CON3-6 & 501.05 & 1.78 & 
503.34 & 1.73 & \bf{499.20} & 
0.37\\CON3-7 & 578.22 & 1.32 & 
580.26 & 1.38 & \bf{577.50} & 
0.12\\CON3-8 & 523.14 & 1.35 & 
525.77 & 1.43 & \bf{523.10} & 
0.01\\CON3-9 & 588.11 & 1.53 & 
588.60 & 1.49 & \bf{578.20} & 
1.71\\CON8-0 & 870.22 & 1.63 & 
875.78 & 1.55 & \bf{858.90} & 
1.32\\CON8-1 & 742.44 & 1.68 & 
746.81 & 1.64 & \bf{740.90} & 
0.21\\CON8-2 & 715.10 & 1.98 & 
716.76 & 1.85 & \bf{714.30} & 
0.11\\CON8-3 & \bf{\underline{811.07}} & 1.50 & 
819.18 & 1.61 & 812.30 & 
-0.15\\CON8-4 & 788.11 & 1.54 & 
792.54 & 1.54 & \bf{770.10} & 
2.34\\CON8-5 & \bf{\underline{758.84}} & 1.47 & 
764.94 & 1.57 & 766.60 & 
-1.01\\CON8-6 & \bf{\underline{690.36}} & 1.74 & 
694.16 & 1.73 & 697.20 & 
-0.98\\CON8-7 & \bf{\underline{814.50}} & 1.45 & 
814.77 & 1.43 & 814.80 & 
-0.04\\CON8-8 & 782.40 & 1.75 & 
789.85 & 1.75 & \bf{771.30} & 
1.44\\CON8-9 & \bf{\underline{814.87}} & 1.72 & 
816.31 & 1.72 & 815.10 & 
-0.03\\[1ex]\hline
\end{tabular}
\label{table:nonlin}
\end{table} \clearpage
\begin{table}[ht]
\caption{Resultados de la ejecución de la metaheurística ACO, utilizando instancias de Dethloff con la configuración -n 2.0 -alpha 1.0 -beta 3.0 -q .3 -ro 0.015}
\centering
\small
\begin{tabular}{c c c c c c c}
\hline\hline
Instancia & Costo mínimo & Tiempo(seg.) & Costo promedio & Tiempo promedio(seg.) & Costo ACO & \%Gap \\ [0.5ex]
\hline
SCA3-0 & \bf{\underline{636.06}} & 1.30 & 
637.18 & 1.35 & 636.10 & 
-0.01\\SCA3-1 & \bf{\underline{697.84}} & 1.59 & 
699.43 & 1.60 & 700.10 & 
-0.32\\SCA3-2 & 661.13 & 1.38 & 
666.30 & 1.44 & \bf{659.30} & 
0.28\\SCA3-3 & 680.04 & 1.33 & 
680.64 & 1.41 & \bf{680.00} & 
0.01\\SCA3-4 & \bf{690.50} & 1.66 & 
690.50 & 1.58 & 690.50 & 0.00\\
SCA3-5 & \bf{\underline{659.90}} & 1.54 & 
667.74 & 1.56 & 671.10 & 
-1.67\\SCA3-6 & \bf{\underline{651.09}} & 1.52 & 
653.00 & 1.53 & 651.10 & 
-0.00\\SCA3-7 & 666.15 & 1.34 & 
666.15 & 1.31 & \bf{666.10} & 
0.01\\SCA3-8 & \bf{\underline{719.47}} & 1.61 & 
720.75 & 1.58 & 719.50 & 
-0.00\\SCA3-9 & \bf{681.00} & 1.38 & 
681.86 & 1.30 & 681.00 & 0.00\\
SCA8-0 & 982.79 & 1.54 & 
987.26 & 1.58 & \bf{961.60} & 
2.20\\SCA8-1 & \bf{\underline{1056.27}} & 1.31 & 
1065.14 & 1.32 & 1063.00 & 
-0.63\\SCA8-2 & 1050.37 & 1.30 & 
1050.94 & 1.27 & \bf{1040.60} & 
0.94\\SCA8-3 & 1002.89 & 1.62 & 
1009.31 & 1.50 & \bf{985.90} & 
1.72\\SCA8-4 & \bf{\underline{1067.55}} & 1.82 & 
1069.87 & 1.66 & 1071.00 & 
-0.32\\SCA8-5 & \bf{\underline{1051.73}} & 1.56 & 
1057.75 & 1.63 & 1054.30 & 
-0.24\\SCA8-6 & \bf{\underline{972.48}} & 1.58 & 
980.44 & 1.65 & 972.50 & 
-0.00\\SCA8-7 & 1067.20 & 1.65 & 
1070.09 & 1.65 & \bf{1059.70} & 
0.71\\SCA8-8 & \bf{\underline{1071.18}} & 1.72 & 
1080.18 & 1.71 & 1082.70 & 
-1.06\\SCA8-9 & \bf{\underline{1067.42}} & 1.24 & 
1067.42 & 1.32 & 1081.40 & 
-1.29\\CON3-0 & 620.76 & 1.63 & 
627.20 & 1.62 & \bf{616.50} & 
0.69\\CON3-1 & 558.16 & 1.58 & 
559.55 & 1.51 & \bf{555.60} & 
0.46\\CON3-2 & \bf{\underline{519.11}} & 1.46 & 
521.34 & 1.44 & 521.40 & 
-0.44\\CON3-3 & \bf{591.20} & 1.65 & 
593.38 & 1.69 & 591.20 & 0.00\\
CON3-4 & 589.32 & 1.40 & 
592.09 & 1.44 & \bf{589.30} & 
0.00\\CON3-5 & 564.89 & 1.51 & 
568.30 & 1.44 & \bf{563.70} & 
0.21\\CON3-6 & 501.33 & 1.70 & 
502.90 & 1.67 & \bf{499.20} & 
0.43\\CON3-7 & 578.22 & 1.32 & 
582.19 & 1.38 & \bf{577.50} & 
0.12\\CON3-8 & \bf{\underline{523.05}} & 1.68 & 
523.84 & 1.53 & 523.10 & 
-0.01\\CON3-9 & 578.25 & 1.50 & 
585.36 & 1.53 & \bf{578.20} & 
0.01\\CON8-0 & 871.03 & 1.58 & 
878.62 & 1.57 & \bf{858.90} & 
1.41\\CON8-1 & 740.93 & 1.57 & 
743.94 & 1.66 & \bf{740.90} & 
0.00\\CON8-2 & \bf{\underline{713.44}} & 1.97 & 
714.71 & 1.90 & 714.30 & 
-0.12\\CON8-3 & 812.70 & 1.54 & 
820.52 & 1.55 & \bf{812.30} & 
0.05\\CON8-4 & 787.25 & 1.54 & 
790.71 & 1.52 & \bf{770.10} & 
2.23\\CON8-5 & \bf{\underline{760.03}} & 1.58 & 
762.72 & 1.61 & 766.60 & 
-0.86\\CON8-6 & \bf{\underline{686.39}} & 1.79 & 
690.28 & 1.77 & 697.20 & 
-1.55\\CON8-7 & \bf{\underline{814.79}} & 1.51 & 
816.90 & 1.46 & 814.80 & 
-0.00\\CON8-8 & 784.58 & 1.73 & 
787.75 & 1.75 & \bf{771.30} & 
1.72\\CON8-9 & 815.44 & 1.73 & 
816.70 & 1.71 & \bf{815.10} & 
0.04\\[1ex]\hline
\end{tabular}
\label{table:nonlin}
\end{table} \clearpage
\begin{table}[ht]
\caption{Resultados de la ejecución de la metaheurística ACO, utilizando instancias de Dethloff con la configuración -n 2.0 -alpha 1.0 -beta 3.0 -q .4 -ro 0.015}
\centering
\small
\begin{tabular}{c c c c c c c}
\hline\hline
Instancia & Costo mínimo & Tiempo(seg.) & Costo promedio & Tiempo promedio(seg.) & Costo ACO & \%Gap \\ [0.5ex]
\hline
SCA3-0 & \bf{\underline{636.06}} & 1.33 & 
637.18 & 1.43 & 636.10 & 
-0.01\\SCA3-1 & \bf{\underline{697.84}} & 1.66 & 
697.84 & 1.53 & 700.10 & 
-0.32\\SCA3-2 & 659.34 & 1.43 & 
661.00 & 1.42 & \bf{659.30} & 
0.01\\SCA3-3 & 680.04 & 1.36 & 
680.04 & 1.37 & \bf{680.00} & 
0.01\\SCA3-4 & \bf{690.50} & 1.30 & 
690.50 & 1.39 & 690.50 & 0.00\\
SCA3-5 & \bf{\underline{662.75}} & 1.45 & 
666.30 & 1.45 & 671.10 & 
-1.24\\SCA3-6 & 652.94 & 1.41 & 
653.16 & 1.39 & \bf{651.10} & 
0.28\\SCA3-7 & 666.15 & 1.28 & 
667.09 & 1.24 & \bf{666.10} & 
0.01\\SCA3-8 & \bf{\underline{719.47}} & 1.55 & 
719.62 & 1.50 & 719.50 & 
-0.00\\SCA3-9 & \bf{681.00} & 1.18 & 
682.85 & 1.17 & 681.00 & 0.00\\
SCA8-0 & 968.79 & 1.48 & 
983.14 & 1.53 & \bf{961.60} & 
0.75\\SCA8-1 & \bf{\underline{1050.20}} & 1.38 & 
1058.46 & 1.39 & 1063.00 & 
-1.20\\SCA8-2 & 1044.23 & 1.25 & 
1049.19 & 1.19 & \bf{1040.60} & 
0.35\\SCA8-3 & 1015.17 & 1.46 & 
1017.88 & 1.48 & \bf{985.90} & 
2.97\\SCA8-4 & \bf{\underline{1065.49}} & 1.42 & 
1073.40 & 1.49 & 1071.00 & 
-0.51\\SCA8-5 & \bf{\underline{1045.30}} & 1.65 & 
1057.35 & 1.70 & 1054.30 & 
-0.85\\SCA8-6 & \bf{\underline{972.48}} & 1.58 & 
980.50 & 1.60 & 972.50 & 
-0.00\\SCA8-7 & 1067.20 & 1.60 & 
1074.85 & 1.63 & \bf{1059.70} & 
0.71\\SCA8-8 & \bf{\underline{1071.18}} & 1.58 & 
1071.18 & 1.57 & 1082.70 & 
-1.06\\SCA8-9 & \bf{\underline{1067.42}} & 1.35 & 
1067.59 & 1.32 & 1081.40 & 
-1.29\\CON3-0 & 622.79 & 1.58 & 
626.40 & 1.64 & \bf{616.50} & 
1.02\\CON3-1 & \bf{\underline{554.47}} & 1.59 & 
557.22 & 1.47 & 555.60 & 
-0.20\\CON3-2 & \bf{\underline{521.38}} & 2.33 & 
522.13 & 1.69 & 521.40 & 
-0.00\\CON3-3 & \bf{\underline{591.19}} & 1.52 & 
591.90 & 1.50 & 591.20 & 
-0.00\\CON3-4 & \bf{\underline{588.79}} & 1.41 & 
592.35 & 1.36 & 589.30 & 
-0.09\\CON3-5 & \bf{563.70} & 1.40 & 
566.58 & 1.36 & 563.70 & 0.00\\
CON3-6 & 500.37 & 1.80 & 
503.38 & 1.70 & \bf{499.20} & 
0.23\\CON3-7 & 578.22 & 1.41 & 
580.26 & 1.37 & \bf{577.50} & 
0.12\\CON3-8 & \bf{\underline{523.05}} & 1.44 & 
523.41 & 1.43 & 523.10 & 
-0.01\\CON3-9 & 588.48 & 1.36 & 
588.99 & 1.41 & \bf{578.20} & 
1.78\\CON8-0 & 865.86 & 1.63 & 
879.22 & 1.55 & \bf{858.90} & 
0.81\\CON8-1 & \bf{\underline{740.85}} & 1.41 & 
745.22 & 1.47 & 740.90 & 
-0.01\\CON8-2 & \bf{\underline{713.44}} & 1.97 & 
717.90 & 1.81 & 714.30 & 
-0.12\\CON8-3 & \bf{\underline{811.07}} & 1.51 & 
817.00 & 1.51 & 812.30 & 
-0.15\\CON8-4 & 784.79 & 1.40 & 
786.43 & 1.41 & \bf{770.10} & 
1.91\\CON8-5 & \bf{\underline{759.93}} & 1.45 & 
763.55 & 1.46 & 766.60 & 
-0.87\\CON8-6 & \bf{\underline{692.75}} & 1.71 & 
695.88 & 1.72 & 697.20 & 
-0.64\\CON8-7 & \bf{\underline{814.79}} & 1.29 & 
819.05 & 1.39 & 814.80 & 
-0.00\\CON8-8 & 789.62 & 1.59 & 
791.98 & 1.65 & \bf{771.30} & 
2.38\\CON8-9 & \bf{\underline{811.18}} & 1.59 & 
814.71 & 1.62 & 815.10 & 
-0.48\\[1ex]\hline
\end{tabular}
\label{table:nonlin}
\end{table} \clearpage
\begin{table}[ht]
\caption{Resultados de la ejecución de la metaheurística ACO, utilizando instancias de Dethloff con la configuración -n 2.0 -alpha 1.0 -beta 3.0 -q .5 -ro 0.015}
\centering
\small
\begin{tabular}{c c c c c c c}
\hline\hline
Instancia & Costo mínimo & Tiempo(seg.) & Costo promedio & Tiempo promedio(seg.) & Costo ACO & \%Gap \\ [0.5ex]
\hline
SCA3-0 & \bf{\underline{636.06}} & 1.30 & 
636.13 & 1.37 & 636.10 & 
-0.01\\SCA3-1 & \bf{\underline{697.84}} & 1.48 & 
698.76 & 1.54 & 700.10 & 
-0.32\\SCA3-2 & 659.34 & 1.38 & 
662.14 & 1.35 & \bf{659.30} & 
0.01\\SCA3-3 & 680.04 & 1.40 & 
680.50 & 1.41 & \bf{680.00} & 
0.01\\SCA3-4 & \bf{690.50} & 1.54 & 
690.50 & 1.42 & 690.50 & 0.00\\
SCA3-5 & \bf{\underline{662.75}} & 1.32 & 
667.73 & 1.44 & 671.10 & 
-1.24\\SCA3-6 & 652.94 & 1.36 & 
654.18 & 1.42 & \bf{651.10} & 
0.28\\SCA3-7 & 666.15 & 1.14 & 
667.67 & 1.17 & \bf{666.10} & 
0.01\\SCA3-8 & \bf{\underline{719.47}} & 1.39 & 
723.60 & 1.37 & 719.50 & 
-0.00\\SCA3-9 & \bf{681.00} & 1.23 & 
681.00 & 1.27 & 681.00 & 0.00\\
SCA8-0 & 977.94 & 1.61 & 
991.61 & 1.50 & \bf{961.60} & 
1.70\\SCA8-1 & \bf{\underline{1062.88}} & 1.30 & 
1068.01 & 1.31 & 1063.00 & 
-0.01\\SCA8-2 & 1047.63 & 1.12 & 
1051.39 & 1.17 & \bf{1040.60} & 
0.68\\SCA8-3 & 1004.76 & 1.39 & 
1012.83 & 1.49 & \bf{985.90} & 
1.91\\SCA8-4 & \bf{\underline{1065.49}} & 1.62 & 
1071.27 & 1.50 & 1071.00 & 
-0.51\\SCA8-5 & \bf{\underline{1045.30}} & 1.76 & 
1056.99 & 1.69 & 1054.30 & 
-0.85\\SCA8-6 & 980.91 & 1.55 & 
982.85 & 1.63 & \bf{972.50} & 
0.86\\SCA8-7 & 1066.65 & 1.57 & 
1072.05 & 1.56 & \bf{1059.70} & 
0.66\\SCA8-8 & \bf{\underline{1071.18}} & 1.51 & 
1071.18 & 1.58 & 1082.70 & 
-1.06\\SCA8-9 & \bf{\underline{1067.42}} & 1.30 & 
1067.42 & 1.31 & 1081.40 & 
-1.29\\CON3-0 & 617.59 & 1.60 & 
620.35 & 1.58 & \bf{616.50} & 
0.18\\CON3-1 & \bf{\underline{554.47}} & 1.44 & 
558.91 & 1.49 & 555.60 & 
-0.20\\CON3-2 & \bf{\underline{521.38}} & 1.45 & 
522.03 & 1.41 & 521.40 & 
-0.00\\CON3-3 & \bf{\underline{591.19}} & 1.58 & 
591.20 & 1.57 & 591.20 & 
-0.00\\CON3-4 & \bf{\underline{588.79}} & 1.34 & 
593.17 & 1.33 & 589.30 & 
-0.09\\CON3-5 & 567.63 & 1.49 & 
569.11 & 1.49 & \bf{563.70} & 
0.70\\CON3-6 & \bf{\underline{499.05}} & 1.51 & 
502.51 & 1.65 & 499.20 & 
-0.03\\CON3-7 & 578.41 & 1.42 & 
581.76 & 1.37 & \bf{577.50} & 
0.16\\CON3-8 & 524.59 & 1.42 & 
527.82 & 1.32 & \bf{523.10} & 
0.28\\CON3-9 & 588.48 & 1.39 & 
588.88 & 1.45 & \bf{578.20} & 
1.78\\CON8-0 & 865.86 & 1.56 & 
872.77 & 1.52 & \bf{858.90} & 
0.81\\CON8-1 & 742.29 & 1.60 & 
746.32 & 1.53 & \bf{740.90} & 
0.19\\CON8-2 & \bf{\underline{713.68}} & 1.89 & 
715.89 & 1.87 & 714.30 & 
-0.09\\CON8-3 & \bf{\underline{812.11}} & 1.47 & 
816.12 & 1.55 & 812.30 & 
-0.02\\CON8-4 & 788.57 & 1.36 & 
790.03 & 1.43 & \bf{770.10} & 
2.40\\CON8-5 & \bf{\underline{759.93}} & 1.50 & 
765.04 & 1.48 & 766.60 & 
-0.87\\CON8-6 & \bf{\underline{690.01}} & 1.76 & 
696.26 & 1.73 & 697.20 & 
-1.03\\CON8-7 & \bf{\underline{814.79}} & 1.40 & 
817.01 & 1.37 & 814.80 & 
-0.00\\CON8-8 & 784.89 & 1.67 & 
789.55 & 1.77 & \bf{771.30} & 
1.76\\CON8-9 & \bf{\underline{814.37}} & 1.77 & 
816.03 & 1.67 & 815.10 & 
-0.09\\[1ex]\hline
\end{tabular}
\label{table:nonlin}
\end{table} \clearpage
\begin{table}[ht]
\caption{Resultados de la ejecución de la metaheurística ACO, utilizando instancias de Dethloff con la configuración -n 2.0 -alpha 1.0 -beta 3.0 -q .6 -ro 0.015}
\centering
\small
\begin{tabular}{c c c c c c c}
\hline\hline
Instancia & Costo mínimo & Tiempo(seg.) & Costo promedio & Tiempo promedio(seg.) & Costo ACO & \%Gap \\ [0.5ex]
\hline
SCA3-0 & \bf{\underline{636.06}} & 1.45 & 
637.32 & 1.45 & 636.10 & 
-0.01\\SCA3-1 & \bf{\underline{697.84}} & 1.51 & 
697.84 & 1.50 & 700.10 & 
-0.32\\SCA3-2 & 661.13 & 1.37 & 
663.88 & 1.37 & \bf{659.30} & 
0.28\\SCA3-3 & 680.04 & 1.40 & 
680.50 & 1.38 & \bf{680.00} & 
0.01\\SCA3-4 & \bf{690.50} & 1.48 & 
690.50 & 1.41 & 690.50 & 0.00\\
SCA3-5 & \bf{\underline{665.04}} & 1.53 & 
667.99 & 1.47 & 671.10 & 
-0.90\\SCA3-6 & \bf{\underline{651.09}} & 1.43 & 
653.00 & 1.42 & 651.10 & 
-0.00\\SCA3-7 & 666.15 & 1.18 & 
667.78 & 1.19 & \bf{666.10} & 
0.01\\SCA3-8 & \bf{\underline{719.47}} & 1.66 & 
720.68 & 1.47 & 719.50 & 
-0.00\\SCA3-9 & \bf{681.00} & 1.18 & 
682.45 & 1.18 & 681.00 & 0.00\\
SCA8-0 & 977.94 & 1.55 & 
989.19 & 1.58 & \bf{961.60} & 
1.70\\SCA8-1 & \bf{\underline{1049.65}} & 1.26 & 
1064.98 & 1.28 & 1063.00 & 
-1.26\\SCA8-2 & 1050.37 & 1.24 & 
1051.04 & 1.14 & \bf{1040.60} & 
0.94\\SCA8-3 & 1010.01 & 1.39 & 
1015.51 & 1.45 & \bf{985.90} & 
2.45\\SCA8-4 & \bf{\underline{1065.49}} & 1.28 & 
1068.71 & 1.51 & 1071.00 & 
-0.51\\SCA8-5 & \bf{\underline{1047.55}} & 1.65 & 
1051.03 & 1.61 & 1054.30 & 
-0.64\\SCA8-6 & \bf{\underline{972.48}} & 1.54 & 
980.31 & 1.69 & 972.50 & 
-0.00\\SCA8-7 & 1067.20 & 1.67 & 
1069.75 & 1.61 & \bf{1059.70} & 
0.71\\SCA8-8 & \bf{\underline{1071.18}} & 1.69 & 
1071.18 & 1.61 & 1082.70 & 
-1.06\\SCA8-9 & \bf{\underline{1067.42}} & 1.18 & 
1067.42 & 1.24 & 1081.40 & 
-1.29\\CON3-0 & 619.09 & 1.52 & 
624.47 & 1.59 & \bf{616.50} & 
0.42\\CON3-1 & \bf{\underline{554.47}} & 1.59 & 
558.11 & 1.47 & 555.60 & 
-0.20\\CON3-2 & \bf{\underline{521.38}} & 1.32 & 
523.68 & 1.34 & 521.40 & 
-0.00\\CON3-3 & \bf{591.20} & 1.56 & 
591.71 & 1.60 & 591.20 & 0.00\\
CON3-4 & 589.32 & 1.40 & 
591.48 & 1.32 & \bf{589.30} & 
0.00\\CON3-5 & \bf{563.70} & 1.48 & 
567.84 & 1.45 & 563.70 & 0.00\\
CON3-6 & 503.97 & 1.86 & 
504.70 & 1.76 & \bf{499.20} & 
0.96\\CON3-7 & 578.22 & 1.25 & 
580.00 & 1.27 & \bf{577.50} & 
0.12\\CON3-8 & 524.30 & 1.32 & 
530.58 & 1.36 & \bf{523.10} & 
0.23\\CON3-9 & 586.31 & 1.24 & 
587.46 & 1.30 & \bf{578.20} & 
1.40\\CON8-0 & 869.98 & 1.56 & 
877.08 & 1.49 & \bf{858.90} & 
1.29\\CON8-1 & \bf{\underline{740.85}} & 1.54 & 
745.56 & 1.50 & 740.90 & 
-0.01\\CON8-2 & 716.16 & 1.86 & 
717.52 & 1.83 & \bf{714.30} & 
0.26\\CON8-3 & 812.75 & 1.54 & 
816.37 & 1.48 & \bf{812.30} & 
0.06\\CON8-4 & 778.63 & 1.46 & 
785.54 & 1.46 & \bf{770.10} & 
1.11\\CON8-5 & \bf{\underline{760.03}} & 1.34 & 
762.77 & 1.33 & 766.60 & 
-0.86\\CON8-6 & \bf{\underline{689.56}} & 1.47 & 
694.20 & 1.62 & 697.20 & 
-1.10\\CON8-7 & 814.86 & 1.23 & 
818.05 & 1.26 & \bf{814.80} & 
0.01\\CON8-8 & 789.95 & 1.72 & 
794.12 & 1.64 & \bf{771.30} & 
2.42\\CON8-9 & \bf{\underline{815.07}} & 1.69 & 
817.77 & 1.66 & 815.10 & 
-0.00\\[1ex]\hline
\end{tabular}
\label{table:nonlin}
\end{table} \clearpage
\begin{table}[ht]
\caption{Resultados de la ejecución de la metaheurística ACO, utilizando instancias de Dethloff con la configuración -n 2.0 -alpha 1.0 -beta 3.0 -q .7 -ro 0.015}
\centering
\small
\begin{tabular}{c c c c c c c}
\hline\hline
Instancia & Costo mínimo & Tiempo(seg.) & Costo promedio & Tiempo promedio(seg.) & Costo ACO & \%Gap \\ [0.5ex]
\hline
SCA3-0 & \bf{\underline{636.06}} & 1.37 & 
637.25 & 1.37 & 636.10 & 
-0.01\\SCA3-1 & \bf{\underline{697.84}} & 1.48 & 
697.84 & 1.44 & 700.10 & 
-0.32\\SCA3-2 & 659.34 & 1.42 & 
661.45 & 1.36 & \bf{659.30} & 
0.01\\SCA3-3 & 680.04 & 1.23 & 
680.32 & 1.31 & \bf{680.00} & 
0.01\\SCA3-4 & \bf{690.50} & 2.35 & 
690.50 & 1.69 & 690.50 & 0.00\\
SCA3-5 & \bf{\underline{662.75}} & 1.42 & 
670.82 & 1.53 & 671.10 & 
-1.24\\SCA3-6 & 652.94 & 1.38 & 
653.27 & 1.34 & \bf{651.10} & 
0.28\\SCA3-7 & 666.15 & 1.10 & 
666.15 & 1.17 & \bf{666.10} & 
0.01\\SCA3-8 & 724.28 & 1.28 & 
726.39 & 1.34 & \bf{719.50} & 
0.66\\SCA3-9 & \bf{681.00} & 1.26 & 
681.17 & 1.15 & 681.00 & 0.00\\
SCA8-0 & 968.79 & 1.54 & 
984.90 & 1.56 & \bf{961.60} & 
0.75\\SCA8-1 & \bf{\underline{1059.55}} & 1.28 & 
1069.87 & 1.26 & 1063.00 & 
-0.32\\SCA8-2 & 1046.29 & 1.18 & 
1049.01 & 1.13 & \bf{1040.60} & 
0.55\\SCA8-3 & 1005.13 & 1.51 & 
1014.88 & 1.53 & \bf{985.90} & 
1.95\\SCA8-4 & \bf{\underline{1067.66}} & 1.62 & 
1073.49 & 1.48 & 1071.00 & 
-0.31\\SCA8-5 & 1054.99 & 1.56 & 
1058.41 & 1.59 & \bf{1054.30} & 
0.07\\SCA8-6 & 980.13 & 1.72 & 
983.68 & 1.58 & \bf{972.50} & 
0.78\\SCA8-7 & 1067.20 & 1.50 & 
1069.48 & 1.57 & \bf{1059.70} & 
0.71\\SCA8-8 & \bf{\underline{1071.18}} & 1.43 & 
1073.91 & 1.53 & 1082.70 & 
-1.06\\SCA8-9 & \bf{\underline{1067.42}} & 1.32 & 
1067.42 & 1.24 & 1081.40 & 
-1.29\\CON3-0 & 623.60 & 1.56 & 
627.20 & 1.58 & \bf{616.50} & 
1.15\\CON3-1 & 556.28 & 1.55 & 
557.24 & 1.51 & \bf{555.60} & 
0.12\\CON3-2 & \bf{\underline{521.38}} & 1.24 & 
523.45 & 1.27 & 521.40 & 
-0.00\\CON3-3 & \bf{591.20} & 1.56 & 
593.28 & 1.62 & 591.20 & 0.00\\
CON3-4 & \bf{\underline{588.79}} & 1.36 & 
593.52 & 1.30 & 589.30 & 
-0.09\\CON3-5 & 564.88 & 1.52 & 
566.89 & 1.50 & \bf{563.70} & 
0.21\\CON3-6 & 502.16 & 1.68 & 
504.02 & 1.70 & \bf{499.20} & 
0.59\\CON3-7 & 578.41 & 1.21 & 
580.66 & 1.24 & \bf{577.50} & 
0.16\\CON3-8 & 523.68 & 1.35 & 
524.37 & 1.33 & \bf{523.10} & 
0.11\\CON3-9 & 588.48 & 1.40 & 
589.46 & 1.26 & \bf{578.20} & 
1.78\\CON8-0 & 871.96 & 1.39 & 
875.58 & 1.50 & \bf{858.90} & 
1.52\\CON8-1 & \bf{\underline{740.85}} & 1.54 & 
746.97 & 1.43 & 740.90 & 
-0.01\\CON8-2 & 714.94 & 2.02 & 
717.20 & 1.89 & \bf{714.30} & 
0.09\\CON8-3 & 815.80 & 1.42 & 
816.92 & 1.45 & \bf{812.30} & 
0.43\\CON8-4 & 780.19 & 1.56 & 
788.20 & 1.51 & \bf{770.10} & 
1.31\\CON8-5 & \bf{\underline{764.41}} & 1.52 & 
766.57 & 1.43 & 766.60 & 
-0.29\\CON8-6 & \bf{\underline{695.66}} & 1.58 & 
697.64 & 1.60 & 697.20 & 
-0.22\\CON8-7 & \bf{\underline{814.79}} & 1.28 & 
816.73 & 1.30 & 814.80 & 
-0.00\\CON8-8 & 780.12 & 1.58 & 
789.25 & 1.65 & \bf{771.30} & 
1.14\\CON8-9 & \bf{\underline{812.89}} & 1.56 & 
815.19 & 1.59 & 815.10 & 
-0.27\\[1ex]\hline
\end{tabular}
\label{table:nonlin}
\end{table} \clearpage
\begin{table}[ht]
\caption{Resultados de la ejecución de la metaheurística ACO, utilizando instancias de Dethloff con la configuración -n 2.0 -alpha 1.0 -beta 3.0 -q .8 -ro 0.015}
\centering
\small
\begin{tabular}{c c c c c c c}
\hline\hline
Instancia & Costo mínimo & Tiempo(seg.) & Costo promedio & Tiempo promedio(seg.) & Costo ACO & \%Gap \\ [0.5ex]
\hline
SCA3-0 & \bf{\underline{636.06}} & 1.29 & 
636.27 & 1.32 & 636.10 & 
-0.01\\SCA3-1 & \bf{\underline{697.84}} & 1.59 & 
698.50 & 1.52 & 700.10 & 
-0.32\\SCA3-2 & 659.34 & 1.28 & 
662.97 & 1.38 & \bf{659.30} & 
0.01\\SCA3-3 & 680.04 & 1.33 & 
680.46 & 1.40 & \bf{680.00} & 
0.01\\SCA3-4 & \bf{690.50} & 1.32 & 
690.50 & 1.40 & 690.50 & 0.00\\
SCA3-5 & \bf{\underline{665.64}} & 1.40 & 
667.58 & 1.41 & 671.10 & 
-0.81\\SCA3-6 & \bf{\underline{651.09}} & 1.40 & 
654.59 & 1.37 & 651.10 & 
-0.00\\SCA3-7 & 666.15 & 1.15 & 
666.15 & 1.14 & \bf{666.10} & 
0.01\\SCA3-8 & \bf{\underline{719.47}} & 1.33 & 
721.29 & 1.32 & 719.50 & 
-0.00\\SCA3-9 & \bf{681.00} & 1.25 & 
681.00 & 1.12 & 681.00 & 0.00\\
SCA8-0 & \bf{\underline{961.50}} & 1.53 & 
976.28 & 1.52 & 961.60 & 
-0.01\\SCA8-1 & 1064.33 & 1.28 & 
1070.00 & 1.24 & \bf{1063.00} & 
0.13\\SCA8-2 & 1046.29 & 1.07 & 
1051.56 & 1.27 & \bf{1040.60} & 
0.55\\SCA8-3 & 1016.51 & 1.41 & 
1021.71 & 1.40 & \bf{985.90} & 
3.10\\SCA8-4 & \bf{\underline{1067.82}} & 1.34 & 
1083.99 & 1.43 & 1071.00 & 
-0.30\\SCA8-5 & \bf{\underline{1039.12}} & 1.72 & 
1050.16 & 1.62 & 1054.30 & 
-1.44\\SCA8-6 & 980.91 & 1.58 & 
985.05 & 1.57 & \bf{972.50} & 
0.86\\SCA8-7 & 1067.20 & 1.64 & 
1073.93 & 1.55 & \bf{1059.70} & 
0.71\\SCA8-8 & \bf{\underline{1071.18}} & 1.52 & 
1074.49 & 1.54 & 1082.70 & 
-1.06\\SCA8-9 & \bf{\underline{1067.42}} & 1.17 & 
1067.42 & 1.22 & 1081.40 & 
-1.29\\CON3-0 & 623.60 & 1.56 & 
625.50 & 1.64 & \bf{616.50} & 
1.15\\CON3-1 & 556.28 & 1.54 & 
559.51 & 1.48 & \bf{555.60} & 
0.12\\CON3-2 & \bf{\underline{519.11}} & 1.22 & 
523.32 & 1.24 & 521.40 & 
-0.44\\CON3-3 & \bf{591.20} & 1.46 & 
593.39 & 1.46 & 591.20 & 0.00\\
CON3-4 & \bf{\underline{588.79}} & 1.18 & 
591.35 & 1.25 & 589.30 & 
-0.09\\CON3-5 & 568.69 & 1.40 & 
571.55 & 1.45 & \bf{563.70} & 
0.89\\CON3-6 & 500.80 & 1.67 & 
505.12 & 1.76 & \bf{499.20} & 
0.32\\CON3-7 & 578.41 & 1.23 & 
579.15 & 1.22 & \bf{577.50} & 
0.16\\CON3-8 & 524.59 & 1.30 & 
525.09 & 1.22 & \bf{523.10} & 
0.28\\CON3-9 & 581.79 & 1.24 & 
587.99 & 1.31 & \bf{578.20} & 
0.62\\CON8-0 & 879.00 & 1.53 & 
886.00 & 1.51 & \bf{858.90} & 
2.34\\CON8-1 & 742.44 & 1.27 & 
747.46 & 1.39 & \bf{740.90} & 
0.21\\CON8-2 & \bf{\underline{713.44}} & 2.05 & 
715.70 & 1.90 & 714.30 & 
-0.12\\CON8-3 & \bf{\underline{812.11}} & 1.58 & 
816.21 & 1.48 & 812.30 & 
-0.02\\CON8-4 & 789.98 & 1.48 & 
792.23 & 1.41 & \bf{770.10} & 
2.58\\CON8-5 & \bf{\underline{760.91}} & 1.42 & 
763.67 & 1.35 & 766.60 & 
-0.74\\CON8-6 & \bf{\underline{688.77}} & 1.66 & 
693.94 & 1.59 & 697.20 & 
-1.21\\CON8-7 & \bf{\underline{814.77}} & 1.22 & 
821.60 & 1.22 & 814.80 & 
-0.00\\CON8-8 & 790.88 & 1.54 & 
795.79 & 1.60 & \bf{771.30} & 
2.54\\CON8-9 & \bf{\underline{812.25}} & 1.70 & 
813.74 & 1.66 & 815.10 & 
-0.35\\[1ex]\hline
\end{tabular}
\label{table:nonlin}
\end{table} \clearpage
\begin{table}[ht]
\caption{Resultados de la ejecución de la metaheurística ACO, utilizando instancias de Dethloff con la configuración -n 2.0 -alpha 1.0 -beta 3.0 -q .9 -ro 0.015}
\centering
\small
\begin{tabular}{c c c c c c c}
\hline\hline
Instancia & Costo mínimo & Tiempo(seg.) & Costo promedio & Tiempo promedio(seg.) & Costo ACO & \%Gap \\ [0.5ex]
\hline
SCA3-0 & \bf{\underline{636.06}} & 1.28 & 
638.66 & 1.35 & 636.10 & 
-0.01\\SCA3-1 & \bf{\underline{697.84}} & 2.12 & 
698.76 & 1.58 & 700.10 & 
-0.32\\SCA3-2 & 659.34 & 1.47 & 
662.97 & 1.33 & \bf{659.30} & 
0.01\\SCA3-3 & 680.60 & 1.39 & 
680.60 & 1.54 & \bf{680.00} & 
0.09\\SCA3-4 & \bf{690.50} & 1.35 & 
690.50 & 1.39 & 690.50 & 0.00\\
SCA3-5 & \bf{\underline{665.04}} & 1.52 & 
665.34 & 1.50 & 671.10 & 
-0.90\\SCA3-6 & 652.94 & 1.31 & 
655.08 & 1.33 & \bf{651.10} & 
0.28\\SCA3-7 & 666.15 & 0.98 & 
666.15 & 1.07 & \bf{666.10} & 
0.01\\SCA3-8 & \bf{\underline{719.47}} & 1.15 & 
723.60 & 1.22 & 719.50 & 
-0.00\\SCA3-9 & \bf{681.00} & 1.08 & 
681.00 & 1.09 & 681.00 & 0.00\\
SCA8-0 & \bf{\underline{961.50}} & 1.55 & 
979.13 & 1.54 & 961.60 & 
-0.01\\SCA8-1 & 1068.31 & 1.20 & 
1073.42 & 1.17 & \bf{1063.00} & 
0.50\\SCA8-2 & 1051.21 & 0.99 & 
1053.81 & 1.05 & \bf{1040.60} & 
1.02\\SCA8-3 & 1008.29 & 1.38 & 
1020.00 & 1.42 & \bf{985.90} & 
2.27\\SCA8-4 & 1071.50 & 1.44 & 
1087.15 & 1.44 & \bf{1071.00} & 
0.05\\SCA8-5 & \bf{\underline{1053.09}} & 1.62 & 
1055.48 & 1.65 & 1054.30 & 
-0.11\\SCA8-6 & 981.72 & 1.70 & 
985.53 & 1.62 & \bf{972.50} & 
0.95\\SCA8-7 & 1067.20 & 1.63 & 
1072.72 & 1.62 & \bf{1059.70} & 
0.71\\SCA8-8 & \bf{\underline{1071.18}} & 1.43 & 
1076.64 & 1.46 & 1082.70 & 
-1.06\\SCA8-9 & \bf{\underline{1066.61}} & 1.22 & 
1067.22 & 1.19 & 1081.40 & 
-1.37\\CON3-0 & 624.51 & 1.49 & 
624.84 & 1.54 & \bf{616.50} & 
1.30\\CON3-1 & 557.38 & 1.39 & 
558.67 & 1.44 & \bf{555.60} & 
0.32\\CON3-2 & \bf{\underline{521.38}} & 1.25 & 
523.37 & 1.17 & 521.40 & 
-0.00\\CON3-3 & \bf{591.20} & 1.74 & 
592.24 & 1.59 & 591.20 & 0.00\\
CON3-4 & \bf{\underline{588.79}} & 1.45 & 
589.74 & 1.32 & 589.30 & 
-0.09\\CON3-5 & 564.88 & 1.49 & 
568.84 & 1.46 & \bf{563.70} & 
0.21\\CON3-6 & 502.16 & 1.62 & 
504.45 & 1.74 & \bf{499.20} & 
0.59\\CON3-7 & 578.22 & 1.20 & 
578.36 & 1.21 & \bf{577.50} & 
0.12\\CON3-8 & 527.52 & 1.23 & 
530.19 & 1.26 & \bf{523.10} & 
0.84\\CON3-9 & 578.98 & 1.18 & 
587.31 & 1.21 & \bf{578.20} & 
0.13\\CON8-0 & 873.25 & 1.41 & 
882.83 & 1.47 & \bf{858.90} & 
1.67\\CON8-1 & 742.29 & 1.52 & 
747.92 & 1.39 & \bf{740.90} & 
0.19\\CON8-2 & \bf{\underline{713.44}} & 1.81 & 
715.69 & 1.84 & 714.30 & 
-0.12\\CON8-3 & 817.57 & 1.49 & 
818.49 & 1.46 & \bf{812.30} & 
0.65\\CON8-4 & 789.98 & 1.49 & 
790.76 & 1.49 & \bf{770.10} & 
2.58\\CON8-5 & \bf{\underline{760.03}} & 1.38 & 
763.62 & 1.41 & 766.60 & 
-0.86\\CON8-6 & \bf{\underline{693.83}} & 1.69 & 
696.38 & 1.63 & 697.20 & 
-0.48\\CON8-7 & 814.86 & 1.20 & 
820.78 & 1.20 & \bf{814.80} & 
0.01\\CON8-8 & 789.35 & 1.40 & 
792.90 & 1.53 & \bf{771.30} & 
2.34\\CON8-9 & \bf{\underline{814.67}} & 1.59 & 
817.35 & 1.57 & 815.10 & 
-0.05\\[1ex]\hline
\end{tabular}
\label{table:nonlin}
\end{table} \clearpage
\begin{table}[ht]
\caption{Resultados de la ejecución de la metaheurística ACO, utilizando instancias de Dethloff con la configuración -n 12.0 -alpha 1.0 -beta 3.0 -q 0.1 -ro 0.015}
\centering
\small
\begin{tabular}{c c c c c c c}
\hline\hline
Instancia & Costo mínimo & Tiempo(seg.) & Costo promedio & Tiempo promedio(seg.) & Costo ACO & \%Gap \\ [0.5ex]
\hline
SCA3-0 & \bf{\underline{636.06}} & 8.14 & 
636.06 & 8.49 & 636.10 & 
-0.01\\SCA3-1 & \bf{\underline{697.84}} & 9.85 & 
697.84 & 9.52 & 700.10 & 
-0.32\\SCA3-2 & 659.34 & 8.90 & 
659.34 & 8.40 & \bf{659.30} & 
0.01\\SCA3-3 & 680.04 & 8.14 & 
680.04 & 8.15 & \bf{680.00} & 
0.01\\SCA3-4 & \bf{690.50} & 9.24 & 
690.50 & 9.53 & 690.50 & 0.00\\
SCA3-5 & \bf{\underline{659.90}} & 8.32 & 
661.90 & 8.71 & 671.10 & 
-1.67\\SCA3-6 & \bf{\underline{651.09}} & 8.66 & 
652.01 & 9.10 & 651.10 & 
-0.00\\SCA3-7 & 666.15 & 7.86 & 
666.15 & 9.06 & \bf{666.10} & 
0.01\\SCA3-8 & \bf{\underline{719.47}} & 8.68 & 
720.04 & 8.85 & 719.50 & 
-0.00\\SCA3-9 & \bf{681.00} & 7.64 & 
681.00 & 7.96 & 681.00 & 0.00\\
SCA8-0 & \bf{\underline{961.50}} & 9.64 & 
968.97 & 9.79 & 961.60 & 
-0.01\\SCA8-1 & \bf{\underline{1053.44}} & 8.02 & 
1057.72 & 8.39 & 1063.00 & 
-0.90\\SCA8-2 & 1043.79 & 7.40 & 
1048.03 & 8.20 & \bf{1040.60} & 
0.31\\SCA8-3 & \bf{\underline{983.34}} & 9.32 & 
992.33 & 9.36 & 985.90 & 
-0.26\\SCA8-4 & \bf{\underline{1065.49}} & 9.23 & 
1066.38 & 9.47 & 1071.00 & 
-0.51\\SCA8-5 & \bf{\underline{1038.59}} & 10.52 & 
1047.51 & 10.27 & 1054.30 & 
-1.49\\SCA8-6 & \bf{\underline{972.48}} & 10.23 & 
972.48 & 10.09 & 972.50 & 
-0.00\\SCA8-7 & 1066.65 & 9.87 & 
1066.88 & 10.11 & \bf{1059.70} & 
0.66\\SCA8-8 & \bf{\underline{1071.18}} & 9.92 & 
1071.18 & 10.08 & 1082.70 & 
-1.06\\SCA8-9 & \bf{\underline{1063.68}} & 8.52 & 
1066.49 & 8.12 & 1081.40 & 
-1.64\\CON3-0 & 617.59 & 9.64 & 
619.27 & 9.59 & \bf{616.50} & 
0.18\\CON3-1 & \bf{\underline{554.47}} & 9.04 & 
555.05 & 8.91 & 555.60 & 
-0.20\\CON3-2 & \bf{\underline{519.11}} & 9.56 & 
520.81 & 8.87 & 521.40 & 
-0.44\\CON3-3 & \bf{\underline{591.19}} & 9.54 & 
591.19 & 9.49 & 591.20 & 
-0.00\\CON3-4 & \bf{\underline{588.79}} & 7.68 & 
589.58 & 8.32 & 589.30 & 
-0.09\\CON3-5 & \bf{563.70} & 8.94 & 
564.59 & 8.85 & 563.70 & 0.00\\
CON3-6 & \bf{\underline{499.07}} & 10.00 & 
500.71 & 9.93 & 499.20 & 
-0.03\\CON3-7 & 578.22 & 7.99 & 
578.32 & 7.95 & \bf{577.50} & 
0.12\\CON3-8 & 523.14 & 8.21 & 
523.54 & 8.20 & \bf{523.10} & 
0.01\\CON3-9 & 578.25 & 8.28 & 
583.41 & 8.39 & \bf{578.20} & 
0.01\\CON8-0 & 869.32 & 8.82 & 
869.66 & 9.09 & \bf{858.90} & 
1.21\\CON8-1 & \bf{\underline{740.85}} & 9.17 & 
741.59 & 9.16 & 740.90 & 
-0.01\\CON8-2 & \bf{\underline{713.05}} & 10.78 & 
713.44 & 11.09 & 714.30 & 
-0.17\\CON8-3 & \bf{\underline{811.07}} & 9.01 & 
812.95 & 9.50 & 812.30 & 
-0.15\\CON8-4 & 776.37 & 8.50 & 
777.71 & 8.59 & \bf{770.10} & 
0.81\\CON8-5 & \bf{\underline{754.95}} & 9.27 & 
757.06 & 9.20 & 766.60 & 
-1.52\\CON8-6 & \bf{\underline{690.53}} & 10.64 & 
693.29 & 10.16 & 697.20 & 
-0.96\\CON8-7 & \bf{\underline{814.79}} & 8.32 & 
814.83 & 8.12 & 814.80 & 
-0.00\\CON8-8 & 777.98 & 10.20 & 
782.01 & 10.36 & \bf{771.30} & 
0.87\\CON8-9 & \bf{\underline{812.03}} & 10.30 & 
812.93 & 10.53 & 815.10 & 
-0.38\\[1ex]\hline
\end{tabular}
\label{table:nonlin}
\end{table} \clearpage
\begin{table}[ht]
\caption{Resultados de la ejecución de la metaheurística ACO, utilizando instancias de Dethloff con la configuración -n 12.0 -alpha 1.0 -beta 3.0 -q .2 -ro 0.015}
\centering
\small
\begin{tabular}{c c c c c c c}
\hline\hline
Instancia & Costo mínimo & Tiempo(seg.) & Costo promedio & Tiempo promedio(seg.) & Costo ACO & \%Gap \\ [0.5ex]
\hline
SCA3-0 & \bf{\underline{636.06}} & 8.45 & 
636.13 & 8.43 & 636.10 & 
-0.01\\SCA3-1 & \bf{\underline{697.84}} & 9.20 & 
697.84 & 9.15 & 700.10 & 
-0.32\\SCA3-2 & 659.34 & 8.32 & 
661.00 & 7.93 & \bf{659.30} & 
0.01\\SCA3-3 & 680.04 & 8.58 & 
680.04 & 8.49 & \bf{680.00} & 
0.01\\SCA3-4 & \bf{690.50} & 9.04 & 
690.50 & 9.12 & 690.50 & 0.00\\
SCA3-5 & \bf{\underline{661.07}} & 9.01 & 
663.05 & 8.93 & 671.10 & 
-1.49\\SCA3-6 & \bf{\underline{651.09}} & 8.58 & 
651.55 & 8.75 & 651.10 & 
-0.00\\SCA3-7 & 666.15 & 7.93 & 
666.15 & 7.73 & \bf{666.10} & 
0.01\\SCA3-8 & \bf{\underline{719.47}} & 9.09 & 
720.12 & 8.66 & 719.50 & 
-0.00\\SCA3-9 & \bf{681.00} & 7.09 & 
681.00 & 7.23 & 681.00 & 0.00\\
SCA8-0 & 968.79 & 9.45 & 
976.21 & 9.40 & \bf{961.60} & 
0.75\\SCA8-1 & \bf{\underline{1054.87}} & 7.16 & 
1056.33 & 7.80 & 1063.00 & 
-0.76\\SCA8-2 & 1047.63 & 7.92 & 
1049.63 & 7.38 & \bf{1040.60} & 
0.68\\SCA8-3 & 991.84 & 8.26 & 
1003.26 & 8.80 & \bf{985.90} & 
0.60\\SCA8-4 & \bf{\underline{1065.49}} & 8.41 & 
1069.48 & 9.26 & 1071.00 & 
-0.51\\SCA8-5 & \bf{\underline{1037.06}} & 10.13 & 
1046.46 & 9.85 & 1054.30 & 
-1.64\\SCA8-6 & \bf{\underline{972.48}} & 9.27 & 
975.73 & 9.48 & 972.50 & 
-0.00\\SCA8-7 & 1066.65 & 9.65 & 
1066.97 & 9.56 & \bf{1059.70} & 
0.66\\SCA8-8 & \bf{\underline{1071.18}} & 9.87 & 
1073.53 & 9.92 & 1082.70 & 
-1.06\\SCA8-9 & \bf{\underline{1061.23}} & 7.67 & 
1065.87 & 7.59 & 1081.40 & 
-1.87\\CON3-0 & 616.52 & 9.18 & 
618.42 & 9.44 & \bf{616.50} & 
0.00\\CON3-1 & \bf{\underline{554.47}} & 9.46 & 
554.92 & 8.98 & 555.60 & 
-0.20\\CON3-2 & \bf{\underline{519.11}} & 8.34 & 
520.81 & 8.62 & 521.40 & 
-0.44\\CON3-3 & \bf{\underline{591.19}} & 9.02 & 
591.19 & 9.67 & 591.20 & 
-0.00\\CON3-4 & \bf{\underline{588.79}} & 7.88 & 
589.45 & 8.09 & 589.30 & 
-0.09\\CON3-5 & \bf{563.70} & 9.79 & 
564.29 & 9.05 & 563.70 & 0.00\\
CON3-6 & 500.80 & 10.16 & 
502.27 & 10.40 & \bf{499.20} & 
0.32\\CON3-7 & \bf{\underline{576.84}} & 7.89 & 
577.18 & 7.87 & 577.50 & 
-0.11\\CON3-8 & \bf{\underline{523.05}} & 9.00 & 
523.75 & 8.34 & 523.10 & 
-0.01\\CON3-9 & 578.25 & 8.03 & 
580.27 & 8.23 & \bf{578.20} & 
0.01\\CON8-0 & 865.19 & 9.67 & 
867.97 & 9.31 & \bf{858.90} & 
0.73\\CON8-1 & \bf{\underline{740.85}} & 9.39 & 
742.14 & 9.04 & 740.90 & 
-0.01\\CON8-2 & \bf{\underline{713.44}} & 11.37 & 
714.23 & 11.33 & 714.30 & 
-0.12\\CON8-3 & \bf{\underline{811.07}} & 9.82 & 
812.28 & 9.32 & 812.30 & 
-0.15\\CON8-4 & 776.37 & 9.23 & 
780.03 & 9.04 & \bf{770.10} & 
0.81\\CON8-5 & \bf{\underline{755.14}} & 8.92 & 
757.85 & 8.96 & 766.60 & 
-1.49\\CON8-6 & \bf{\underline{688.00}} & 10.37 & 
691.15 & 10.21 & 697.20 & 
-1.32\\CON8-7 & \bf{\underline{814.50}} & 8.20 & 
814.80 & 7.72 & 814.80 & 
-0.04\\CON8-8 & \bf{\underline{771.26}} & 9.90 & 
782.12 & 10.36 & 771.30 & 
-0.01\\CON8-9 & \bf{\underline{812.45}} & 10.10 & 
813.14 & 9.95 & 815.10 & 
-0.33\\[1ex]\hline
\end{tabular}
\label{table:nonlin}
\end{table} \clearpage
\begin{table}[ht]
\caption{Resultados de la ejecución de la metaheurística ACO, utilizando instancias de Dethloff con la configuración -n 12.0 -alpha 1.0 -beta 3.0 -q .3 -ro 0.015}
\centering
\small
\begin{tabular}{c c c c c c c}
\hline\hline
Instancia & Costo mínimo & Tiempo(seg.) & Costo promedio & Tiempo promedio(seg.) & Costo ACO & \%Gap \\ [0.5ex]
\hline
SCA3-0 & \bf{\underline{636.06}} & 9.22 & 
636.06 & 8.88 & 636.10 & 
-0.01\\SCA3-1 & \bf{\underline{697.84}} & 9.06 & 
697.84 & 9.10 & 700.10 & 
-0.32\\SCA3-2 & 659.34 & 8.00 & 
659.79 & 8.18 & \bf{659.30} & 
0.01\\SCA3-3 & 680.04 & 8.08 & 
680.04 & 8.12 & \bf{680.00} & 
0.01\\SCA3-4 & \bf{690.50} & 8.68 & 
690.50 & 8.88 & 690.50 & 0.00\\
SCA3-5 & \bf{\underline{659.90}} & 8.49 & 
662.04 & 9.24 & 671.10 & 
-1.67\\SCA3-6 & \bf{\underline{651.09}} & 8.71 & 
652.70 & 8.87 & 651.10 & 
-0.00\\SCA3-7 & 666.15 & 7.19 & 
666.15 & 7.43 & \bf{666.10} & 
0.01\\SCA3-8 & \bf{\underline{719.47}} & 8.72 & 
719.47 & 8.44 & 719.50 & 
-0.00\\SCA3-9 & \bf{681.00} & 7.12 & 
681.00 & 7.04 & 681.00 & 0.00\\
SCA8-0 & 968.79 & 8.89 & 
974.26 & 9.14 & \bf{961.60} & 
0.75\\SCA8-1 & \bf{\underline{1053.09}} & 7.47 & 
1056.85 & 7.87 & 1063.00 & 
-0.93\\SCA8-2 & 1046.29 & 7.05 & 
1048.28 & 7.16 & \bf{1040.60} & 
0.55\\SCA8-3 & 991.84 & 9.03 & 
997.15 & 8.57 & \bf{985.90} & 
0.60\\SCA8-4 & \bf{\underline{1065.49}} & 9.53 & 
1069.54 & 9.22 & 1071.00 & 
-0.51\\SCA8-5 & \bf{\underline{1034.74}} & 9.70 & 
1043.91 & 10.09 & 1054.30 & 
-1.86\\SCA8-6 & \bf{\underline{972.48}} & 9.51 & 
975.75 & 9.71 & 972.50 & 
-0.00\\SCA8-7 & 1066.65 & 9.75 & 
1067.02 & 9.71 & \bf{1059.70} & 
0.66\\SCA8-8 & \bf{\underline{1071.18}} & 9.74 & 
1071.18 & 9.77 & 1082.70 & 
-1.06\\SCA8-9 & \bf{\underline{1065.60}} & 7.26 & 
1066.97 & 7.79 & 1081.40 & 
-1.46\\CON3-0 & 616.52 & 9.46 & 
617.43 & 9.51 & \bf{616.50} & 
0.00\\CON3-1 & 556.04 & 9.36 & 
556.62 & 8.93 & \bf{555.60} & 
0.08\\CON3-2 & \bf{\underline{521.38}} & 8.11 & 
521.38 & 8.54 & 521.40 & 
-0.00\\CON3-3 & \bf{\underline{591.19}} & 9.44 & 
591.20 & 9.31 & 591.20 & 
-0.00\\CON3-4 & \bf{\underline{588.79}} & 7.67 & 
588.79 & 7.96 & 589.30 & 
-0.09\\CON3-5 & \bf{563.70} & 8.35 & 
565.11 & 8.68 & 563.70 & 0.00\\
CON3-6 & 500.37 & 10.77 & 
501.32 & 10.53 & \bf{499.20} & 
0.23\\CON3-7 & \bf{\underline{576.84}} & 8.04 & 
577.97 & 7.69 & 577.50 & 
-0.11\\CON3-8 & \bf{\underline{523.05}} & 8.10 & 
523.54 & 8.42 & 523.10 & 
-0.01\\CON3-9 & 578.98 & 8.30 & 
585.53 & 8.67 & \bf{578.20} & 
0.13\\CON8-0 & 866.22 & 9.18 & 
868.39 & 9.11 & \bf{858.90} & 
0.85\\CON8-1 & \bf{\underline{740.85}} & 8.39 & 
744.15 & 8.95 & 740.90 & 
-0.01\\CON8-2 & \bf{\underline{713.05}} & 10.61 & 
713.25 & 13.04 & 714.30 & 
-0.17\\CON8-3 & \bf{\underline{811.07}} & 9.19 & 
811.75 & 9.04 & 812.30 & 
-0.15\\CON8-4 & 776.72 & 7.84 & 
781.49 & 8.76 & \bf{770.10} & 
0.86\\CON8-5 & \bf{\underline{758.12}} & 8.22 & 
760.21 & 8.63 & 766.60 & 
-1.11\\CON8-6 & \bf{\underline{683.83}} & 9.99 & 
687.07 & 10.21 & 697.20 & 
-1.92\\CON8-7 & \bf{\underline{814.50}} & 8.30 & 
815.01 & 8.12 & 814.80 & 
-0.04\\CON8-8 & 779.68 & 10.64 & 
785.24 & 10.09 & \bf{771.30} & 
1.09\\CON8-9 & \bf{\underline{810.18}} & 10.73 & 
812.67 & 9.81 & 815.10 & 
-0.60\\[1ex]\hline
\end{tabular}
\label{table:nonlin}
\end{table} \clearpage
\begin{table}[ht]
\caption{Resultados de la ejecución de la metaheurística ACO, utilizando instancias de Dethloff con la configuración -n 12.0 -alpha 1.0 -beta 3.0 -q .4 -ro 0.015}
\centering
\small
\begin{tabular}{c c c c c c c}
\hline\hline
Instancia & Costo mínimo & Tiempo(seg.) & Costo promedio & Tiempo promedio(seg.) & Costo ACO & \%Gap \\ [0.5ex]
\hline
SCA3-0 & \bf{\underline{636.06}} & 8.89 & 
637.18 & 8.32 & 636.10 & 
-0.01\\SCA3-1 & \bf{\underline{697.84}} & 9.52 & 
697.84 & 9.29 & 700.10 & 
-0.32\\SCA3-2 & 659.34 & 7.90 & 
661.00 & 8.22 & \bf{659.30} & 
0.01\\SCA3-3 & 680.04 & 7.87 & 
680.04 & 8.24 & \bf{680.00} & 
0.01\\SCA3-4 & \bf{690.50} & 8.47 & 
690.50 & 8.63 & 690.50 & 0.00\\
SCA3-5 & \bf{\underline{662.75}} & 9.19 & 
663.32 & 8.98 & 671.10 & 
-1.24\\SCA3-6 & 652.94 & 9.00 & 
652.94 & 8.67 & \bf{651.10} & 
0.28\\SCA3-7 & \bf{\underline{664.88}} & 7.32 & 
665.83 & 7.25 & 666.10 & 
-0.18\\SCA3-8 & \bf{\underline{719.47}} & 7.98 & 
719.54 & 8.16 & 719.50 & 
-0.00\\SCA3-9 & \bf{681.00} & 6.85 & 
681.00 & 7.46 & 681.00 & 0.00\\
SCA8-0 & \bf{\underline{961.50}} & 8.79 & 
965.14 & 9.10 & 961.60 & 
-0.01\\SCA8-1 & \bf{\underline{1052.71}} & 7.15 & 
1054.88 & 7.62 & 1063.00 & 
-0.97\\SCA8-2 & 1043.79 & 6.74 & 
1047.42 & 6.75 & \bf{1040.60} & 
0.31\\SCA8-3 & 995.50 & 9.27 & 
1004.58 & 8.95 & \bf{985.90} & 
0.97\\SCA8-4 & \bf{\underline{1067.66}} & 8.96 & 
1068.41 & 8.93 & 1071.00 & 
-0.31\\SCA8-5 & \bf{\underline{1029.95}} & 9.52 & 
1040.54 & 9.89 & 1054.30 & 
-2.31\\SCA8-6 & \bf{\underline{972.48}} & 10.20 & 
976.86 & 9.71 & 972.50 & 
-0.00\\SCA8-7 & 1066.65 & 9.36 & 
1067.86 & 9.32 & \bf{1059.70} & 
0.66\\SCA8-8 & \bf{\underline{1071.18}} & 9.54 & 
1071.18 & 10.64 & 1082.70 & 
-1.06\\SCA8-9 & \bf{\underline{1067.27}} & 7.65 & 
1067.38 & 7.59 & 1081.40 & 
-1.31\\CON3-0 & 617.59 & 9.17 & 
619.83 & 9.34 & \bf{616.50} & 
0.18\\CON3-1 & \bf{\underline{554.47}} & 8.60 & 
555.32 & 9.12 & 555.60 & 
-0.20\\CON3-2 & \bf{\underline{519.11}} & 8.48 & 
520.81 & 8.48 & 521.40 & 
-0.44\\CON3-3 & \bf{\underline{591.19}} & 12.92 & 
591.20 & 10.32 & 591.20 & 
-0.00\\CON3-4 & \bf{\underline{588.79}} & 8.70 & 
588.92 & 8.11 & 589.30 & 
-0.09\\CON3-5 & \bf{563.70} & 8.74 & 
564.29 & 8.55 & 563.70 & 0.00\\
CON3-6 & 500.80 & 10.06 & 
501.59 & 10.26 & \bf{499.20} & 
0.32\\CON3-7 & \bf{\underline{576.84}} & 7.76 & 
577.97 & 7.67 & 577.50 & 
-0.11\\CON3-8 & \bf{\underline{523.05}} & 7.52 & 
523.62 & 8.20 & 523.10 & 
-0.01\\CON3-9 & 580.05 & 8.42 & 
586.47 & 8.22 & \bf{578.20} & 
0.32\\CON8-0 & 869.43 & 8.38 & 
871.15 & 8.84 & \bf{858.90} & 
1.23\\CON8-1 & \bf{\underline{740.85}} & 9.05 & 
741.23 & 9.19 & 740.90 & 
-0.01\\CON8-2 & \bf{\underline{712.89}} & 12.35 & 
714.22 & 12.05 & 714.30 & 
-0.20\\CON8-3 & \bf{\underline{811.07}} & 8.34 & 
813.11 & 8.86 & 812.30 & 
-0.15\\CON8-4 & 776.37 & 8.53 & 
783.64 & 8.54 & \bf{770.10} & 
0.81\\CON8-5 & \bf{\underline{758.99}} & 8.82 & 
762.00 & 8.57 & 766.60 & 
-0.99\\CON8-6 & \bf{\underline{686.39}} & 10.27 & 
687.73 & 10.05 & 697.20 & 
-1.55\\CON8-7 & \bf{\underline{814.77}} & 7.41 & 
814.87 & 7.70 & 814.80 & 
-0.00\\CON8-8 & 781.44 & 10.02 & 
785.62 & 10.07 & \bf{771.30} & 
1.31\\CON8-9 & \bf{\underline{811.66}} & 10.39 & 
812.50 & 10.28 & 815.10 & 
-0.42\\[1ex]\hline
\end{tabular}
\label{table:nonlin}
\end{table} \clearpage
\begin{table}[ht]
\caption{Resultados de la ejecución de la metaheurística ACO, utilizando instancias de Dethloff con la configuración -n 12.0 -alpha 1.0 -beta 3.0 -q .5 -ro 0.015}
\centering
\small
\begin{tabular}{c c c c c c c}
\hline\hline
Instancia & Costo mínimo & Tiempo(seg.) & Costo promedio & Tiempo promedio(seg.) & Costo ACO & \%Gap \\ [0.5ex]
\hline
SCA3-0 & \bf{\underline{636.06}} & 8.76 & 
636.06 & 8.41 & 636.10 & 
-0.01\\SCA3-1 & \bf{\underline{697.84}} & 9.18 & 
697.84 & 8.97 & 700.10 & 
-0.32\\SCA3-2 & 659.34 & 7.74 & 
659.34 & 8.24 & \bf{659.30} & 
0.01\\SCA3-3 & 680.04 & 7.82 & 
680.04 & 9.99 & \bf{680.00} & 
0.01\\SCA3-4 & \bf{690.50} & 8.94 & 
690.50 & 8.98 & 690.50 & 0.00\\
SCA3-5 & \bf{\underline{659.90}} & 9.33 & 
662.04 & 8.97 & 671.10 & 
-1.67\\SCA3-6 & \bf{\underline{651.09}} & 7.90 & 
652.48 & 8.24 & 651.10 & 
-0.00\\SCA3-7 & 666.15 & 6.45 & 
666.15 & 7.04 & \bf{666.10} & 
0.01\\SCA3-8 & \bf{\underline{719.47}} & 7.82 & 
719.54 & 8.13 & 719.50 & 
-0.00\\SCA3-9 & \bf{681.00} & 7.60 & 
681.00 & 7.00 & 681.00 & 0.00\\
SCA8-0 & \bf{\underline{961.50}} & 9.30 & 
976.45 & 9.10 & 961.60 & 
-0.01\\SCA8-1 & \bf{\underline{1053.44}} & 7.22 & 
1056.80 & 7.20 & 1063.00 & 
-0.90\\SCA8-2 & 1046.29 & 6.74 & 
1049.80 & 6.86 & \bf{1040.60} & 
0.55\\SCA8-3 & 1001.69 & 9.56 & 
1011.19 & 9.30 & \bf{985.90} & 
1.60\\SCA8-4 & \bf{\underline{1065.49}} & 8.36 & 
1069.65 & 8.89 & 1071.00 & 
-0.51\\SCA8-5 & \bf{\underline{1044.03}} & 9.32 & 
1051.38 & 9.38 & 1054.30 & 
-0.97\\SCA8-6 & 977.03 & 9.81 & 
978.97 & 9.88 & \bf{972.50} & 
0.47\\SCA8-7 & 1067.11 & 10.03 & 
1067.18 & 9.90 & \bf{1059.70} & 
0.70\\SCA8-8 & \bf{\underline{1071.18}} & 9.19 & 
1071.18 & 9.28 & 1082.70 & 
-1.06\\SCA8-9 & \bf{\underline{1067.42}} & 7.73 & 
1067.42 & 7.76 & 1081.40 & 
-1.29\\CON3-0 & 616.52 & 9.17 & 
618.12 & 9.28 & \bf{616.50} & 
0.00\\CON3-1 & \bf{\underline{554.47}} & 9.35 & 
555.53 & 8.90 & 555.60 & 
-0.20\\CON3-2 & \bf{\underline{519.61}} & 8.59 & 
521.00 & 8.44 & 521.40 & 
-0.34\\CON3-3 & \bf{\underline{591.19}} & 9.53 & 
591.24 & 9.59 & 591.20 & 
-0.00\\CON3-4 & \bf{\underline{588.79}} & 8.01 & 
588.92 & 7.98 & 589.30 & 
-0.09\\CON3-5 & \bf{563.70} & 11.63 & 
564.29 & 9.62 & 563.70 & 0.00\\
CON3-6 & 500.37 & 10.06 & 
501.52 & 9.93 & \bf{499.20} & 
0.23\\CON3-7 & \bf{\underline{576.48}} & 7.58 & 
577.53 & 7.60 & 577.50 & 
-0.18\\CON3-8 & 523.14 & 7.74 & 
523.57 & 7.75 & \bf{523.10} & 
0.01\\CON3-9 & 586.31 & 8.04 & 
587.92 & 8.38 & \bf{578.20} & 
1.40\\CON8-0 & 867.76 & 8.90 & 
870.17 & 9.06 & \bf{858.90} & 
1.03\\CON8-1 & \bf{\underline{740.85}} & 9.04 & 
742.67 & 9.04 & 740.90 & 
-0.01\\CON8-2 & \bf{\underline{712.89}} & 12.00 & 
714.75 & 11.46 & 714.30 & 
-0.20\\CON8-3 & \bf{\underline{811.07}} & 8.54 & 
813.54 & 9.20 & 812.30 & 
-0.15\\CON8-4 & 781.64 & 8.71 & 
786.29 & 8.91 & \bf{770.10} & 
1.50\\CON8-5 & \bf{\underline{758.12}} & 8.60 & 
759.55 & 8.65 & 766.60 & 
-1.11\\CON8-6 & \bf{\underline{686.39}} & 10.32 & 
692.62 & 10.12 & 697.20 & 
-1.55\\CON8-7 & \bf{\underline{814.50}} & 8.07 & 
815.02 & 7.54 & 814.80 & 
-0.04\\CON8-8 & 782.86 & 10.85 & 
783.41 & 10.73 & \bf{771.30} & 
1.50\\CON8-9 & \bf{\underline{810.18}} & 10.71 & 
813.02 & 10.29 & 815.10 & 
-0.60\\[1ex]\hline
\end{tabular}
\label{table:nonlin}
\end{table} \clearpage
\begin{table}[ht]
\caption{Resultados de la ejecución de la metaheurística ACO, utilizando instancias de Dethloff con la configuración -n 12.0 -alpha 1.0 -beta 3.0 -q .6 -ro 0.015}
\centering
\small
\begin{tabular}{c c c c c c c}
\hline\hline
Instancia & Costo mínimo & Tiempo(seg.) & Costo promedio & Tiempo promedio(seg.) & Costo ACO & \%Gap \\ [0.5ex]
\hline
SCA3-0 & \bf{\underline{636.06}} & 8.41 & 
636.06 & 8.30 & 636.10 & 
-0.01\\SCA3-1 & \bf{\underline{697.84}} & 9.11 & 
697.84 & 9.09 & 700.10 & 
-0.32\\SCA3-2 & 659.34 & 8.01 & 
659.79 & 7.91 & \bf{659.30} & 
0.01\\SCA3-3 & 680.04 & 7.95 & 
680.04 & 9.93 & \bf{680.00} & 
0.01\\SCA3-4 & \bf{690.50} & 8.36 & 
690.50 & 8.54 & 690.50 & 0.00\\
SCA3-5 & \bf{\underline{662.75}} & 9.14 & 
663.89 & 8.77 & 671.10 & 
-1.24\\SCA3-6 & \bf{\underline{651.09}} & 8.56 & 
652.48 & 8.20 & 651.10 & 
-0.00\\SCA3-7 & 666.15 & 7.39 & 
666.15 & 8.24 & \bf{666.10} & 
0.01\\SCA3-8 & \bf{\underline{719.47}} & 8.18 & 
719.54 & 7.83 & 719.50 & 
-0.00\\SCA3-9 & \bf{681.00} & 7.20 & 
681.00 & 7.03 & 681.00 & 0.00\\
SCA8-0 & \bf{\underline{961.50}} & 9.34 & 
969.71 & 9.30 & 961.60 & 
-0.01\\SCA8-1 & \bf{\underline{1056.61}} & 7.26 & 
1058.73 & 7.48 & 1063.00 & 
-0.60\\SCA8-2 & 1046.29 & 6.46 & 
1050.15 & 6.63 & \bf{1040.60} & 
0.55\\SCA8-3 & 1011.36 & 9.16 & 
1014.46 & 9.05 & \bf{985.90} & 
2.58\\SCA8-4 & \bf{\underline{1065.49}} & 8.48 & 
1070.48 & 9.16 & 1071.00 & 
-0.51\\SCA8-5 & \bf{\underline{1034.74}} & 9.94 & 
1045.74 & 10.00 & 1054.30 & 
-1.86\\SCA8-6 & \bf{\underline{972.48}} & 9.54 & 
978.04 & 9.57 & 972.50 & 
-0.00\\SCA8-7 & 1066.65 & 10.09 & 
1066.93 & 10.03 & \bf{1059.70} & 
0.66\\SCA8-8 & \bf{\underline{1071.18}} & 9.87 & 
1073.91 & 9.28 & 1082.70 & 
-1.06\\SCA8-9 & \bf{\underline{1067.42}} & 7.22 & 
1067.42 & 7.33 & 1081.40 & 
-1.29\\CON3-0 & 616.52 & 9.45 & 
621.45 & 9.52 & \bf{616.50} & 
0.00\\CON3-1 & \bf{\underline{554.47}} & 9.01 & 
555.88 & 9.12 & 555.60 & 
-0.20\\CON3-2 & \bf{\underline{519.11}} & 9.05 & 
520.37 & 8.69 & 521.40 & 
-0.44\\CON3-3 & \bf{\underline{591.19}} & 9.80 & 
591.19 & 9.58 & 591.20 & 
-0.00\\CON3-4 & \bf{\underline{588.79}} & 7.76 & 
588.92 & 7.78 & 589.30 & 
-0.09\\CON3-5 & \bf{563.70} & 9.23 & 
565.53 & 8.74 & 563.70 & 0.00\\
CON3-6 & \bf{\underline{499.05}} & 10.40 & 
500.70 & 10.10 & 499.20 & 
-0.03\\CON3-7 & 577.68 & 8.12 & 
578.18 & 7.77 & \bf{577.50} & 
0.03\\CON3-8 & \bf{\underline{523.05}} & 7.67 & 
523.84 & 7.92 & 523.10 & 
-0.01\\CON3-9 & 578.25 & 7.60 & 
585.73 & 8.08 & \bf{578.20} & 
0.01\\CON8-0 & 868.49 & 8.71 & 
869.63 & 8.96 & \bf{858.90} & 
1.12\\CON8-1 & \bf{\underline{740.85}} & 9.00 & 
740.85 & 8.93 & 740.90 & 
-0.01\\CON8-2 & \bf{\underline{713.44}} & 11.57 & 
713.52 & 11.37 & 714.30 & 
-0.12\\CON8-3 & \bf{\underline{811.07}} & 8.72 & 
813.46 & 8.77 & 812.30 & 
-0.15\\CON8-4 & 776.37 & 8.21 & 
782.38 & 8.74 & \bf{770.10} & 
0.81\\CON8-5 & \bf{\underline{758.99}} & 8.80 & 
761.13 & 8.61 & 766.60 & 
-0.99\\CON8-6 & \bf{\underline{689.11}} & 9.90 & 
693.39 & 9.80 & 697.20 & 
-1.16\\CON8-7 & \bf{\underline{814.77}} & 7.73 & 
814.94 & 9.62 & 814.80 & 
-0.00\\CON8-8 & 784.02 & 10.72 & 
789.64 & 9.75 & \bf{771.30} & 
1.65\\CON8-9 & \bf{\underline{811.59}} & 9.94 & 
813.82 & 9.62 & 815.10 & 
-0.43\\[1ex]\hline
\end{tabular}
\label{table:nonlin}
\end{table} \clearpage
\begin{table}[ht]
\caption{Resultados de la ejecución de la metaheurística ACO, utilizando instancias de Dethloff con la configuración -n 12.0 -alpha 1.0 -beta 3.0 -q .7 -ro 0.015}
\centering
\small
\begin{tabular}{c c c c c c c}
\hline\hline
Instancia & Costo mínimo & Tiempo(seg.) & Costo promedio & Tiempo promedio(seg.) & Costo ACO & \%Gap \\ [0.5ex]
\hline
SCA3-0 & \bf{\underline{636.06}} & 7.86 & 
636.06 & 8.32 & 636.10 & 
-0.01\\SCA3-1 & \bf{\underline{697.84}} & 9.54 & 
697.84 & 9.27 & 700.10 & 
-0.32\\SCA3-2 & 659.34 & 7.88 & 
660.55 & 7.91 & \bf{659.30} & 
0.01\\SCA3-3 & 680.04 & 8.73 & 
680.04 & 8.21 & \bf{680.00} & 
0.01\\SCA3-4 & \bf{690.50} & 8.58 & 
690.50 & 8.88 & 690.50 & 0.00\\
SCA3-5 & \bf{\underline{659.90}} & 9.19 & 
663.06 & 8.73 & 671.10 & 
-1.67\\SCA3-6 & \bf{\underline{651.09}} & 7.60 & 
652.01 & 8.24 & 651.10 & 
-0.00\\SCA3-7 & 666.15 & 7.00 & 
666.15 & 6.71 & \bf{666.10} & 
0.01\\SCA3-8 & \bf{\underline{719.47}} & 7.24 & 
720.04 & 7.51 & 719.50 & 
-0.00\\SCA3-9 & \bf{681.00} & 7.44 & 
681.00 & 6.83 & 681.00 & 0.00\\
SCA8-0 & \bf{\underline{961.50}} & 9.15 & 
968.64 & 9.35 & 961.60 & 
-0.01\\SCA8-1 & \bf{\underline{1053.40}} & 7.47 & 
1063.83 & 7.27 & 1063.00 & 
-0.90\\SCA8-2 & 1045.85 & 6.70 & 
1048.89 & 6.34 & \bf{1040.60} & 
0.50\\SCA8-3 & 1013.17 & 9.40 & 
1016.31 & 9.59 & \bf{985.90} & 
2.77\\SCA8-4 & \bf{\underline{1067.66}} & 9.16 & 
1073.70 & 9.25 & 1071.00 & 
-0.31\\SCA8-5 & \bf{\underline{1034.74}} & 9.87 & 
1042.53 & 10.08 & 1054.30 & 
-1.86\\SCA8-6 & \bf{\underline{972.48}} & 10.22 & 
978.04 & 10.09 & 972.50 & 
-0.00\\SCA8-7 & 1067.20 & 9.63 & 
1069.25 & 9.54 & \bf{1059.70} & 
0.71\\SCA8-8 & \bf{\underline{1071.18}} & 9.36 & 
1072.14 & 9.24 & 1082.70 & 
-1.06\\SCA8-9 & \bf{\underline{1063.68}} & 7.09 & 
1066.03 & 7.33 & 1081.40 & 
-1.64\\CON3-0 & 619.09 & 9.43 & 
620.86 & 9.49 & \bf{616.50} & 
0.42\\CON3-1 & \bf{\underline{554.47}} & 9.25 & 
555.08 & 8.97 & 555.60 & 
-0.20\\CON3-2 & \bf{\underline{519.11}} & 8.54 & 
522.12 & 8.08 & 521.40 & 
-0.44\\CON3-3 & \bf{\underline{591.19}} & 9.32 & 
591.20 & 9.18 & 591.20 & 
-0.00\\CON3-4 & \bf{\underline{588.79}} & 7.53 & 
588.92 & 8.03 & 589.30 & 
-0.09\\CON3-5 & \bf{563.70} & 8.58 & 
566.51 & 8.44 & 563.70 & 0.00\\
CON3-6 & 500.37 & 10.07 & 
501.03 & 9.89 & \bf{499.20} & 
0.23\\CON3-7 & 578.22 & 8.23 & 
578.36 & 7.65 & \bf{577.50} & 
0.12\\CON3-8 & 523.68 & 7.41 & 
524.01 & 7.71 & \bf{523.10} & 
0.11\\CON3-9 & 578.98 & 7.60 & 
584.16 & 7.76 & \bf{578.20} & 
0.13\\CON8-0 & 870.22 & 8.88 & 
872.04 & 8.79 & \bf{858.90} & 
1.32\\CON8-1 & \bf{\underline{740.85}} & 8.49 & 
741.61 & 8.30 & 740.90 & 
-0.01\\CON8-2 & \bf{\underline{712.89}} & 11.75 & 
714.24 & 11.55 & 714.30 & 
-0.20\\CON8-3 & \bf{\underline{812.11}} & 8.39 & 
816.17 & 8.38 & 812.30 & 
-0.02\\CON8-4 & 777.59 & 9.25 & 
785.00 & 9.03 & \bf{770.10} & 
0.97\\CON8-5 & \bf{\underline{758.12}} & 8.91 & 
760.12 & 8.48 & 766.60 & 
-1.11\\CON8-6 & \bf{\underline{692.75}} & 9.89 & 
695.27 & 9.92 & 697.20 & 
-0.64\\CON8-7 & \bf{\underline{814.50}} & 7.34 & 
814.97 & 7.52 & 814.80 & 
-0.04\\CON8-8 & 776.55 & 10.18 & 
783.29 & 9.93 & \bf{771.30} & 
0.68\\CON8-9 & \bf{\underline{811.16}} & 9.32 & 
814.32 & 9.38 & 815.10 & 
-0.48\\[1ex]\hline
\end{tabular}
\label{table:nonlin}
\end{table} \clearpage
\begin{table}[ht]
\caption{Resultados de la ejecución de la metaheurística ACO, utilizando instancias de Dethloff con la configuración -n 12.0 -alpha 1.0 -beta 3.0 -q .8 -ro 0.015}
\centering
\small
\begin{tabular}{c c c c c c c}
\hline\hline
Instancia & Costo mínimo & Tiempo(seg.) & Costo promedio & Tiempo promedio(seg.) & Costo ACO & \%Gap \\ [0.5ex]
\hline
SCA3-0 & \bf{\underline{636.06}} & 8.43 & 
636.06 & 8.24 & 636.10 & 
-0.01\\SCA3-1 & \bf{\underline{697.84}} & 8.67 & 
697.84 & 9.01 & 700.10 & 
-0.32\\SCA3-2 & 659.34 & 8.38 & 
660.55 & 7.98 & \bf{659.30} & 
0.01\\SCA3-3 & 680.04 & 8.71 & 
680.04 & 8.33 & \bf{680.00} & 
0.01\\SCA3-4 & \bf{690.50} & 8.33 & 
690.50 & 8.65 & 690.50 & 0.00\\
SCA3-5 & \bf{\underline{659.90}} & 9.14 & 
663.48 & 9.20 & 671.10 & 
-1.67\\SCA3-6 & 652.94 & 8.06 & 
653.16 & 8.14 & \bf{651.10} & 
0.28\\SCA3-7 & \bf{\underline{659.17}} & 7.93 & 
664.40 & 7.20 & 666.10 & 
-1.04\\SCA3-8 & \bf{\underline{719.47}} & 7.99 & 
720.67 & 7.45 & 719.50 & 
-0.00\\SCA3-9 & \bf{681.00} & 6.46 & 
681.00 & 6.43 & 681.00 & 0.00\\
SCA8-0 & 968.79 & 8.52 & 
975.65 & 8.75 & \bf{961.60} & 
0.75\\SCA8-1 & 1064.33 & 6.64 & 
1067.99 & 7.04 & \bf{1063.00} & 
0.13\\SCA8-2 & 1046.29 & 6.79 & 
1048.80 & 6.45 & \bf{1040.60} & 
0.55\\SCA8-3 & 1010.76 & 9.55 & 
1014.24 & 9.33 & \bf{985.90} & 
2.52\\SCA8-4 & \bf{\underline{1065.49}} & 9.00 & 
1074.38 & 9.07 & 1071.00 & 
-0.51\\SCA8-5 & \bf{\underline{1034.74}} & 9.42 & 
1044.50 & 9.73 & 1054.30 & 
-1.86\\SCA8-6 & 977.87 & 9.76 & 
980.08 & 9.87 & \bf{972.50} & 
0.55\\SCA8-7 & 1067.20 & 9.75 & 
1070.13 & 9.59 & \bf{1059.70} & 
0.71\\SCA8-8 & \bf{\underline{1071.18}} & 9.23 & 
1072.14 & 9.07 & 1082.70 & 
-1.06\\SCA8-9 & \bf{\underline{1067.42}} & 7.13 & 
1067.42 & 7.49 & 1081.40 & 
-1.29\\CON3-0 & 616.52 & 9.15 & 
621.26 & 9.20 & \bf{616.50} & 
0.00\\CON3-1 & \bf{\underline{554.47}} & 8.73 & 
555.38 & 9.10 & 555.60 & 
-0.20\\CON3-2 & \bf{\underline{521.38}} & 7.48 & 
522.10 & 7.71 & 521.40 & 
-0.00\\CON3-3 & \bf{\underline{591.19}} & 9.11 & 
591.20 & 9.91 & 591.20 & 
-0.00\\CON3-4 & \bf{\underline{588.79}} & 8.16 & 
588.92 & 8.15 & 589.30 & 
-0.09\\CON3-5 & \bf{563.70} & 7.93 & 
565.56 & 8.16 & 563.70 & 0.00\\
CON3-6 & 500.80 & 10.19 & 
502.76 & 10.25 & \bf{499.20} & 
0.32\\CON3-7 & 577.68 & 7.94 & 
578.97 & 7.22 & \bf{577.50} & 
0.03\\CON3-8 & 523.14 & 7.28 & 
523.57 & 7.56 & \bf{523.10} & 
0.01\\CON3-9 & 578.98 & 7.22 & 
585.47 & 7.69 & \bf{578.20} & 
0.13\\CON8-0 & 869.43 & 8.57 & 
876.75 & 8.54 & \bf{858.90} & 
1.23\\CON8-1 & \bf{\underline{740.85}} & 8.66 & 
741.61 & 8.59 & 740.90 & 
-0.01\\CON8-2 & \bf{\underline{713.44}} & 11.64 & 
714.37 & 11.71 & 714.30 & 
-0.12\\CON8-3 & \bf{\underline{812.11}} & 8.29 & 
813.02 & 8.34 & 812.30 & 
-0.02\\CON8-4 & 776.72 & 8.52 & 
782.50 & 8.82 & \bf{770.10} & 
0.86\\CON8-5 & \bf{\underline{760.03}} & 8.06 & 
761.93 & 8.25 & 766.60 & 
-0.86\\CON8-6 & \bf{\underline{683.83}} & 9.36 & 
692.39 & 9.83 & 697.20 & 
-1.92\\CON8-7 & 814.86 & 7.39 & 
818.36 & 7.40 & \bf{814.80} & 
0.01\\CON8-8 & 780.90 & 9.82 & 
788.10 & 9.61 & \bf{771.30} & 
1.24\\CON8-9 & \bf{\underline{815.07}} & 9.82 & 
815.38 & 9.79 & 815.10 & 
-0.00\\[1ex]\hline
\end{tabular}
\label{table:nonlin}
\end{table} \clearpage
\begin{table}[ht]
\caption{Resultados de la ejecución de la metaheurística ACO, utilizando instancias de Dethloff con la configuración -n 12.0 -alpha 1.0 -beta 3.0 -q .9 -ro 0.015}
\centering
\small
\begin{tabular}{c c c c c c c}
\hline\hline
Instancia & Costo mínimo & Tiempo(seg.) & Costo promedio & Tiempo promedio(seg.) & Costo ACO & \%Gap \\ [0.5ex]
\hline
SCA3-0 & \bf{\underline{636.06}} & 9.08 & 
636.20 & 8.44 & 636.10 & 
-0.01\\SCA3-1 & \bf{\underline{697.84}} & 8.71 & 
697.84 & 8.71 & 700.10 & 
-0.32\\SCA3-2 & 664.18 & 8.03 & 
664.56 & 7.89 & \bf{659.30} & 
0.74\\SCA3-3 & 680.60 & 8.67 & 
680.60 & 8.55 & \bf{680.00} & 
0.09\\SCA3-4 & \bf{690.50} & 8.42 & 
690.50 & 8.29 & 690.50 & 0.00\\
SCA3-5 & \bf{\underline{662.75}} & 8.91 & 
664.45 & 9.01 & 671.10 & 
-1.24\\SCA3-6 & \bf{\underline{651.09}} & 8.24 & 
653.80 & 7.89 & 651.10 & 
-0.00\\SCA3-7 & 666.15 & 6.50 & 
666.15 & 6.54 & \bf{666.10} & 
0.01\\SCA3-8 & \bf{\underline{719.47}} & 6.78 & 
722.35 & 7.22 & 719.50 & 
-0.00\\SCA3-9 & \bf{681.00} & 7.12 & 
681.00 & 6.93 & 681.00 & 0.00\\
SCA8-0 & 980.45 & 9.37 & 
989.69 & 9.07 & \bf{961.60} & 
1.96\\SCA8-1 & \bf{\underline{1052.71}} & 7.17 & 
1063.33 & 7.10 & 1063.00 & 
-0.97\\SCA8-2 & 1049.22 & 6.19 & 
1051.79 & 6.21 & \bf{1040.60} & 
0.83\\SCA8-3 & 1013.30 & 9.01 & 
1014.30 & 9.00 & \bf{985.90} & 
2.78\\SCA8-4 & \bf{\underline{1067.82}} & 9.44 & 
1079.00 & 9.05 & 1071.00 & 
-0.30\\SCA8-5 & \bf{\underline{1047.55}} & 9.77 & 
1053.61 & 9.47 & 1054.30 & 
-0.64\\SCA8-6 & 978.03 & 10.00 & 
980.32 & 9.67 & \bf{972.50} & 
0.57\\SCA8-7 & 1067.20 & 10.35 & 
1072.82 & 10.06 & \bf{1059.70} & 
0.71\\SCA8-8 & \bf{\underline{1071.18}} & 9.04 & 
1071.18 & 8.90 & 1082.70 & 
-1.06\\SCA8-9 & \bf{\underline{1067.42}} & 7.09 & 
1067.42 & 7.09 & 1081.40 & 
-1.29\\CON3-0 & 620.76 & 9.64 & 
623.01 & 9.46 & \bf{616.50} & 
0.69\\CON3-1 & \bf{\underline{554.47}} & 7.75 & 
555.59 & 8.49 & 555.60 & 
-0.20\\CON3-2 & \bf{\underline{521.38}} & 7.54 & 
523.44 & 7.55 & 521.40 & 
-0.00\\CON3-3 & \bf{591.20} & 9.63 & 
591.64 & 9.39 & 591.20 & 0.00\\
CON3-4 & \bf{\underline{588.79}} & 7.70 & 
588.92 & 7.73 & 589.30 & 
-0.09\\CON3-5 & 564.88 & 8.04 & 
568.73 & 8.14 & \bf{563.70} & 
0.21\\CON3-6 & 500.80 & 10.69 & 
502.52 & 10.80 & \bf{499.20} & 
0.32\\CON3-7 & 578.41 & 7.31 & 
581.05 & 7.41 & \bf{577.50} & 
0.16\\CON3-8 & \bf{\underline{523.05}} & 7.39 & 
524.06 & 7.36 & 523.10 & 
-0.01\\CON3-9 & 578.98 & 7.46 & 
586.64 & 7.97 & \bf{578.20} & 
0.13\\CON8-0 & 879.00 & 8.24 & 
880.39 & 8.63 & \bf{858.90} & 
2.34\\CON8-1 & \bf{\underline{740.85}} & 8.93 & 
742.85 & 8.38 & 740.90 & 
-0.01\\CON8-2 & \bf{\underline{712.89}} & 12.09 & 
714.71 & 12.11 & 714.30 & 
-0.20\\CON8-3 & 817.57 & 8.29 & 
817.57 & 8.58 & \bf{812.30} & 
0.65\\CON8-4 & 781.64 & 9.28 & 
783.73 & 8.83 & \bf{770.10} & 
1.50\\CON8-5 & \bf{\underline{759.93}} & 8.60 & 
763.61 & 8.29 & 766.60 & 
-0.87\\CON8-6 & \bf{\underline{692.28}} & 10.01 & 
696.40 & 9.77 & 697.20 & 
-0.71\\CON8-7 & \bf{\underline{814.79}} & 7.78 & 
815.04 & 7.62 & 814.80 & 
-0.00\\CON8-8 & 784.77 & 9.54 & 
792.12 & 9.63 & \bf{771.30} & 
1.75\\CON8-9 & \bf{\underline{813.16}} & 9.68 & 
814.90 & 9.57 & 815.10 & 
-0.24\\[1ex]\hline
\end{tabular}
\label{table:nonlin}
\end{table} \clearpage
\begin{table}[ht]
\caption{Resultados de la ejecución de la metaheurística ACO, utilizando instancias de Dethloff con la configuración -n 22.0 -alpha 1.0 -beta 3.0 -q 0.1 -ro 0.015}
\centering
\small
\begin{tabular}{c c c c c c c}
\hline\hline
Instancia & Costo mínimo & Tiempo(seg.) & Costo promedio & Tiempo promedio(seg.) & Costo ACO & \%Gap \\ [0.5ex]
\hline
SCA3-0 & \bf{\underline{636.06}} & 15.48 & 
636.06 & 15.31 & 636.10 & 
-0.01\\SCA3-1 & \bf{\underline{697.84}} & 17.15 & 
697.84 & 17.09 & 700.10 & 
-0.32\\SCA3-2 & 659.34 & 14.68 & 
659.79 & 15.57 & \bf{659.30} & 
0.01\\SCA3-3 & 680.04 & 15.38 & 
680.04 & 14.95 & \bf{680.00} & 
0.01\\SCA3-4 & \bf{690.50} & 17.16 & 
690.50 & 16.68 & 690.50 & 0.00\\
SCA3-5 & \bf{\underline{659.90}} & 16.72 & 
662.04 & 16.46 & 671.10 & 
-1.67\\SCA3-6 & 652.94 & 15.71 & 
652.94 & 15.56 & \bf{651.10} & 
0.28\\SCA3-7 & 666.15 & 13.55 & 
666.15 & 13.92 & \bf{666.10} & 
0.01\\SCA3-8 & \bf{\underline{719.47}} & 16.93 & 
719.47 & 15.78 & 719.50 & 
-0.00\\SCA3-9 & \bf{681.00} & 14.53 & 
681.00 & 13.55 & 681.00 & 0.00\\
SCA8-0 & \bf{\underline{961.50}} & 16.76 & 
965.82 & 16.82 & 961.60 & 
-0.01\\SCA8-1 & \bf{\underline{1049.65}} & 15.16 & 
1052.22 & 14.87 & 1063.00 & 
-1.26\\SCA8-2 & 1043.79 & 13.50 & 
1045.56 & 12.72 & \bf{1040.60} & 
0.31\\SCA8-3 & \bf{\underline{983.34}} & 17.32 & 
991.05 & 16.55 & 985.90 & 
-0.26\\SCA8-4 & \bf{\underline{1065.49}} & 16.96 & 
1066.03 & 16.66 & 1071.00 & 
-0.51\\SCA8-5 & \bf{\underline{1034.74}} & 19.14 & 
1039.11 & 18.52 & 1054.30 & 
-1.86\\SCA8-6 & \bf{\underline{972.48}} & 16.32 & 
975.73 & 17.21 & 972.50 & 
-0.00\\SCA8-7 & 1067.20 & 18.30 & 
1067.31 & 18.41 & \bf{1059.70} & 
0.71\\SCA8-8 & \bf{\underline{1071.18}} & 17.79 & 
1071.18 & 17.90 & 1082.70 & 
-1.06\\SCA8-9 & \bf{\underline{1060.50}} & 14.10 & 
1064.38 & 14.60 & 1081.40 & 
-1.93\\CON3-0 & 616.52 & 17.10 & 
618.12 & 17.36 & \bf{616.50} & 
0.00\\CON3-1 & \bf{\underline{554.47}} & 17.35 & 
554.47 & 17.16 & 555.60 & 
-0.20\\CON3-2 & \bf{\underline{519.11}} & 16.48 & 
519.98 & 15.72 & 521.40 & 
-0.44\\CON3-3 & \bf{\underline{591.19}} & 18.08 & 
591.19 & 17.67 & 591.20 & 
-0.00\\CON3-4 & \bf{\underline{588.79}} & 16.14 & 
588.79 & 15.13 & 589.30 & 
-0.09\\CON3-5 & \bf{563.70} & 16.25 & 
564.29 & 16.41 & 563.70 & 0.00\\
CON3-6 & \bf{\underline{499.05}} & 19.90 & 
501.04 & 18.89 & 499.20 & 
-0.03\\CON3-7 & \bf{\underline{576.48}} & 13.97 & 
577.00 & 14.77 & 577.50 & 
-0.18\\CON3-8 & \bf{\underline{523.05}} & 15.38 & 
523.10 & 15.61 & 523.10 & 
-0.01\\CON3-9 & 578.25 & 15.72 & 
579.99 & 15.56 & \bf{578.20} & 
0.01\\CON8-0 & 867.47 & 16.22 & 
869.83 & 16.56 & \bf{858.90} & 
1.00\\CON8-1 & \bf{\underline{740.85}} & 16.18 & 
741.93 & 16.92 & 740.90 & 
-0.01\\CON8-2 & \bf{\underline{713.05}} & 19.35 & 
713.34 & 19.70 & 714.30 & 
-0.17\\CON8-3 & \bf{\underline{811.07}} & 16.82 & 
812.45 & 16.82 & 812.30 & 
-0.15\\CON8-4 & 776.37 & 15.14 & 
779.48 & 15.50 & \bf{770.10} & 
0.81\\CON8-5 & \bf{\underline{754.95}} & 16.76 & 
758.46 & 16.45 & 766.60 & 
-1.52\\CON8-6 & \bf{\underline{684.05}} & 18.61 & 
686.62 & 18.84 & 697.20 & 
-1.89\\CON8-7 & \bf{\underline{814.79}} & 15.31 & 
814.81 & 14.99 & 814.80 & 
-0.00\\CON8-8 & \bf{\underline{771.26}} & 19.96 & 
777.44 & 19.27 & 771.30 & 
-0.01\\CON8-9 & \bf{\underline{809.00}} & 19.05 & 
810.20 & 18.96 & 815.10 & 
-0.75\\[1ex]\hline
\end{tabular}
\label{table:nonlin}
\end{table} \clearpage
\begin{table}[ht]
\caption{Resultados de la ejecución de la metaheurística ACO, utilizando instancias de Dethloff con la configuración -n 22.0 -alpha 1.0 -beta 3.0 -q .2 -ro 0.015}
\centering
\small
\begin{tabular}{c c c c c c c}
\hline\hline
Instancia & Costo mínimo & Tiempo(seg.) & Costo promedio & Tiempo promedio(seg.) & Costo ACO & \%Gap \\ [0.5ex]
\hline
SCA3-0 & \bf{\underline{636.06}} & 15.93 & 
636.06 & 15.55 & 636.10 & 
-0.01\\SCA3-1 & \bf{\underline{697.84}} & 16.07 & 
697.84 & 16.66 & 700.10 & 
-0.32\\SCA3-2 & 659.34 & 15.38 & 
659.79 & 14.81 & \bf{659.30} & 
0.01\\SCA3-3 & 680.04 & 15.11 & 
680.04 & 14.99 & \bf{680.00} & 
0.01\\SCA3-4 & \bf{690.50} & 16.64 & 
690.50 & 16.67 & 690.50 & 0.00\\
SCA3-5 & \bf{\underline{659.90}} & 16.70 & 
661.62 & 16.90 & 671.10 & 
-1.67\\SCA3-6 & \bf{\underline{651.09}} & 15.23 & 
652.01 & 15.52 & 651.10 & 
-0.00\\SCA3-7 & \bf{\underline{659.17}} & 13.39 & 
664.40 & 13.74 & 666.10 & 
-1.04\\SCA3-8 & \bf{\underline{719.47}} & 15.12 & 
719.47 & 15.43 & 719.50 & 
-0.00\\SCA3-9 & \bf{681.00} & 13.27 & 
681.00 & 13.38 & 681.00 & 0.00\\
SCA8-0 & 968.79 & 16.62 & 
973.05 & 16.95 & \bf{961.60} & 
0.75\\SCA8-1 & \bf{\underline{1050.20}} & 14.69 & 
1051.99 & 13.97 & 1063.00 & 
-1.20\\SCA8-2 & 1045.64 & 13.12 & 
1047.98 & 13.05 & \bf{1040.60} & 
0.48\\SCA8-3 & 995.50 & 16.39 & 
999.30 & 16.19 & \bf{985.90} & 
0.97\\SCA8-4 & \bf{\underline{1065.49}} & 16.81 & 
1065.94 & 16.68 & 1071.00 & 
-0.51\\SCA8-5 & \bf{\underline{1029.95}} & 18.60 & 
1032.88 & 18.56 & 1054.30 & 
-2.31\\SCA8-6 & \bf{\underline{972.48}} & 17.88 & 
974.39 & 17.86 & 972.50 & 
-0.00\\SCA8-7 & 1066.65 & 33.76 & 
1067.99 & 21.61 & \bf{1059.70} & 
0.66\\SCA8-8 & \bf{\underline{1071.18}} & 17.69 & 
1071.18 & 18.07 & 1082.70 & 
-1.06\\SCA8-9 & \bf{\underline{1063.68}} & 14.12 & 
1066.03 & 14.31 & 1081.40 & 
-1.64\\CON3-0 & 616.52 & 17.03 & 
617.32 & 17.19 & \bf{616.50} & 
0.00\\CON3-1 & \bf{\underline{554.47}} & 16.24 & 
554.47 & 16.25 & 555.60 & 
-0.20\\CON3-2 & \bf{\underline{519.11}} & 14.86 & 
520.58 & 15.68 & 521.40 & 
-0.44\\CON3-3 & \bf{\underline{591.19}} & 16.48 & 
591.19 & 17.06 & 591.20 & 
-0.00\\CON3-4 & \bf{\underline{588.79}} & 15.12 & 
588.92 & 14.43 & 589.30 & 
-0.09\\CON3-5 & \bf{563.70} & 15.69 & 
564.00 & 16.29 & 563.70 & 0.00\\
CON3-6 & 500.37 & 17.83 & 
500.85 & 17.96 & \bf{499.20} & 
0.23\\CON3-7 & \bf{\underline{576.84}} & 13.39 & 
577.75 & 14.19 & 577.50 & 
-0.11\\CON3-8 & \bf{\underline{523.05}} & 14.11 & 
523.70 & 14.46 & 523.10 & 
-0.01\\CON3-9 & 578.25 & 14.12 & 
578.95 & 14.74 & \bf{578.20} & 
0.01\\CON8-0 & 867.52 & 16.10 & 
869.15 & 16.27 & \bf{858.90} & 
1.00\\CON8-1 & \bf{\underline{740.85}} & 16.32 & 
741.57 & 16.48 & 740.90 & 
-0.01\\CON8-2 & \bf{\underline{713.60}} & 19.88 & 
713.72 & 20.05 & 714.30 & 
-0.10\\CON8-3 & \bf{\underline{811.07}} & 17.11 & 
813.92 & 16.72 & 812.30 & 
-0.15\\CON8-4 & 775.42 & 16.26 & 
777.46 & 15.72 & \bf{770.10} & 
0.69\\CON8-5 & \bf{\underline{755.67}} & 16.12 & 
759.13 & 16.80 & 766.60 & 
-1.43\\CON8-6 & \bf{\underline{686.39}} & 18.54 & 
689.00 & 18.57 & 697.20 & 
-1.55\\CON8-7 & \bf{\underline{814.79}} & 13.83 & 
814.83 & 14.54 & 814.80 & 
-0.00\\CON8-8 & 782.93 & 18.16 & 
783.82 & 18.56 & \bf{771.30} & 
1.51\\CON8-9 & \bf{\underline{810.18}} & 18.34 & 
810.90 & 18.41 & 815.10 & 
-0.60\\[1ex]\hline
\end{tabular}
\label{table:nonlin}
\end{table} \clearpage
\begin{table}[ht]
\caption{Resultados de la ejecución de la metaheurística ACO, utilizando instancias de Dethloff con la configuración -n 22.0 -alpha 1.0 -beta 3.0 -q .3 -ro 0.015}
\centering
\small
\begin{tabular}{c c c c c c c}
\hline\hline
Instancia & Costo mínimo & Tiempo(seg.) & Costo promedio & Tiempo promedio(seg.) & Costo ACO & \%Gap \\ [0.5ex]
\hline
SCA3-0 & \bf{\underline{636.06}} & 16.44 & 
636.06 & 15.85 & 636.10 & 
-0.01\\SCA3-1 & \bf{\underline{697.84}} & 16.86 & 
697.84 & 16.75 & 700.10 & 
-0.32\\SCA3-2 & 659.34 & 13.98 & 
659.79 & 14.88 & \bf{659.30} & 
0.01\\SCA3-3 & 680.04 & 14.67 & 
680.04 & 14.91 & \bf{680.00} & 
0.01\\SCA3-4 & \bf{690.50} & 16.08 & 
690.50 & 16.27 & 690.50 & 0.00\\
SCA3-5 & \bf{\underline{659.90}} & 16.01 & 
661.62 & 16.46 & 671.10 & 
-1.67\\SCA3-6 & \bf{\underline{651.09}} & 15.74 & 
651.43 & 15.42 & 651.10 & 
-0.00\\SCA3-7 & 666.15 & 13.88 & 
666.15 & 14.03 & \bf{666.10} & 
0.01\\SCA3-8 & \bf{\underline{719.47}} & 16.09 & 
719.47 & 15.87 & 719.50 & 
-0.00\\SCA3-9 & \bf{681.00} & 12.20 & 
681.00 & 12.49 & 681.00 & 0.00\\
SCA8-0 & \bf{\underline{961.50}} & 17.43 & 
970.99 & 16.97 & 961.60 & 
-0.01\\SCA8-1 & \bf{\underline{1052.71}} & 14.44 & 
1055.06 & 14.09 & 1063.00 & 
-0.97\\SCA8-2 & 1045.78 & 12.12 & 
1049.17 & 12.45 & \bf{1040.60} & 
0.50\\SCA8-3 & \bf{\underline{985.47}} & 14.78 & 
1004.10 & 15.14 & 985.90 & 
-0.04\\SCA8-4 & \bf{\underline{1065.49}} & 17.60 & 
1066.58 & 16.85 & 1071.00 & 
-0.51\\SCA8-5 & \bf{\underline{1034.74}} & 16.84 & 
1040.49 & 18.00 & 1054.30 & 
-1.86\\SCA8-6 & \bf{\underline{972.48}} & 16.49 & 
974.59 & 16.93 & 972.50 & 
-0.00\\SCA8-7 & 1066.65 & 18.70 & 
1068.88 & 17.94 & \bf{1059.70} & 
0.66\\SCA8-8 & \bf{\underline{1071.18}} & 17.23 & 
1071.18 & 17.42 & 1082.70 & 
-1.06\\SCA8-9 & \bf{\underline{1063.68}} & 13.42 & 
1065.83 & 13.11 & 1081.40 & 
-1.64\\CON3-0 & 617.59 & 17.42 & 
618.34 & 17.23 & \bf{616.50} & 
0.18\\CON3-1 & \bf{\underline{554.47}} & 15.77 & 
555.32 & 16.65 & 555.60 & 
-0.20\\CON3-2 & \bf{\underline{519.61}} & 15.84 & 
520.94 & 15.43 & 521.40 & 
-0.34\\CON3-3 & \bf{\underline{591.19}} & 17.70 & 
591.20 & 17.20 & 591.20 & 
-0.00\\CON3-4 & \bf{\underline{588.79}} & 14.24 & 
588.79 & 14.50 & 589.30 & 
-0.09\\CON3-5 & \bf{563.70} & 15.18 & 
564.59 & 15.78 & 563.70 & 0.00\\
CON3-6 & \bf{\underline{499.05}} & 18.26 & 
500.39 & 21.41 & 499.20 & 
-0.03\\CON3-7 & \bf{\underline{576.48}} & 14.31 & 
577.10 & 14.26 & 577.50 & 
-0.18\\CON3-8 & 523.68 & 15.18 & 
524.36 & 14.52 & \bf{523.10} & 
0.11\\CON3-9 & 578.25 & 15.25 & 
580.45 & 14.86 & \bf{578.20} & 
0.01\\CON8-0 & 870.22 & 16.10 & 
871.78 & 16.46 & \bf{858.90} & 
1.32\\CON8-1 & \bf{\underline{740.85}} & 16.06 & 
741.42 & 16.55 & 740.90 & 
-0.01\\CON8-2 & \bf{\underline{713.05}} & 20.55 & 
713.46 & 20.41 & 714.30 & 
-0.17\\CON8-3 & \bf{\underline{811.07}} & 15.39 & 
811.73 & 15.89 & 812.30 & 
-0.15\\CON8-4 & 776.37 & 16.15 & 
779.71 & 15.54 & \bf{770.10} & 
0.81\\CON8-5 & \bf{\underline{758.12}} & 19.45 & 
759.47 & 16.33 & 766.60 & 
-1.11\\CON8-6 & \bf{\underline{683.83}} & 18.34 & 
689.10 & 18.49 & 697.20 & 
-1.92\\CON8-7 & \bf{\underline{814.77}} & 14.17 & 
814.82 & 14.55 & 814.80 & 
-0.00\\CON8-8 & 779.43 & 18.78 & 
782.30 & 18.95 & \bf{771.30} & 
1.05\\CON8-9 & \bf{\underline{810.18}} & 17.37 & 
810.94 & 18.00 & 815.10 & 
-0.60\\[1ex]\hline
\end{tabular}
\label{table:nonlin}
\end{table} \clearpage
\begin{table}[ht]
\caption{Resultados de la ejecución de la metaheurística ACO, utilizando instancias de Dethloff con la configuración -n 22.0 -alpha 1.0 -beta 3.0 -q .4 -ro 0.015}
\centering
\small
\begin{tabular}{c c c c c c c}
\hline\hline
Instancia & Costo mínimo & Tiempo(seg.) & Costo promedio & Tiempo promedio(seg.) & Costo ACO & \%Gap \\ [0.5ex]
\hline
SCA3-0 & \bf{\underline{636.06}} & 15.39 & 
636.06 & 15.92 & 636.10 & 
-0.01\\SCA3-1 & \bf{\underline{697.84}} & 16.85 & 
697.84 & 16.31 & 700.10 & 
-0.32\\SCA3-2 & 659.34 & 15.58 & 
659.34 & 14.61 & \bf{659.30} & 
0.01\\SCA3-3 & 680.04 & 14.68 & 
680.04 & 15.10 & \bf{680.00} & 
0.01\\SCA3-4 & \bf{690.50} & 16.70 & 
690.50 & 16.46 & 690.50 & 0.00\\
SCA3-5 & \bf{\underline{659.90}} & 16.14 & 
660.61 & 16.35 & 671.10 & 
-1.67\\SCA3-6 & \bf{\underline{651.09}} & 14.72 & 
652.36 & 15.49 & 651.10 & 
-0.00\\SCA3-7 & 666.15 & 13.09 & 
666.15 & 13.40 & \bf{666.10} & 
0.01\\SCA3-8 & \bf{\underline{719.47}} & 14.80 & 
719.54 & 15.21 & 719.50 & 
-0.00\\SCA3-9 & \bf{681.00} & 13.66 & 
681.00 & 12.75 & 681.00 & 0.00\\
SCA8-0 & 968.79 & 15.60 & 
972.14 & 16.18 & \bf{961.60} & 
0.75\\SCA8-1 & \bf{\underline{1055.60}} & 13.80 & 
1058.88 & 14.47 & 1063.00 & 
-0.70\\SCA8-2 & 1042.61 & 12.65 & 
1048.40 & 12.26 & \bf{1040.60} & 
0.19\\SCA8-3 & \bf{\underline{983.34}} & 15.74 & 
993.92 & 16.34 & 985.90 & 
-0.26\\SCA8-4 & \bf{\underline{1065.49}} & 16.54 & 
1066.77 & 16.30 & 1071.00 & 
-0.51\\SCA8-5 & \bf{\underline{1034.74}} & 17.27 & 
1039.22 & 17.77 & 1054.30 & 
-1.86\\SCA8-6 & \bf{\underline{972.48}} & 16.81 & 
973.62 & 17.22 & 972.50 & 
-0.00\\SCA8-7 & 1063.22 & 16.92 & 
1066.14 & 20.81 & \bf{1059.70} & 
0.33\\SCA8-8 & \bf{\underline{1071.18}} & 18.64 & 
1071.18 & 17.13 & 1082.70 & 
-1.06\\SCA8-9 & \bf{\underline{1067.42}} & 13.68 & 
1067.42 & 13.33 & 1081.40 & 
-1.29\\CON3-0 & 616.52 & 17.72 & 
619.17 & 17.30 & \bf{616.50} & 
0.00\\CON3-1 & \bf{\underline{554.47}} & 15.54 & 
554.92 & 16.10 & 555.60 & 
-0.20\\CON3-2 & \bf{\underline{519.11}} & 16.50 & 
520.37 & 15.32 & 521.40 & 
-0.44\\CON3-3 & \bf{\underline{591.19}} & 17.02 & 
591.19 & 17.13 & 591.20 & 
-0.00\\CON3-4 & \bf{\underline{588.79}} & 15.23 & 
588.92 & 15.07 & 589.30 & 
-0.09\\CON3-5 & \bf{563.70} & 16.17 & 
564.59 & 15.58 & 563.70 & 0.00\\
CON3-6 & \bf{\underline{499.05}} & 18.16 & 
500.36 & 18.13 & 499.20 & 
-0.03\\CON3-7 & \bf{\underline{576.48}} & 14.36 & 
578.54 & 14.16 & 577.50 & 
-0.18\\CON3-8 & \bf{\underline{523.05}} & 13.56 & 
523.39 & 14.38 & 523.10 & 
-0.01\\CON3-9 & 585.25 & 14.90 & 
587.07 & 15.32 & \bf{578.20} & 
1.22\\CON8-0 & 870.00 & 15.90 & 
872.23 & 15.97 & \bf{858.90} & 
1.29\\CON8-1 & \bf{\underline{740.85}} & 17.10 & 
740.87 & 16.23 & 740.90 & 
-0.01\\CON8-2 & \bf{\underline{713.44}} & 20.59 & 
713.44 & 20.77 & 714.30 & 
-0.12\\CON8-3 & \bf{\underline{811.07}} & 16.01 & 
813.38 & 16.34 & 812.30 & 
-0.15\\CON8-4 & 772.80 & 15.14 & 
777.52 & 14.82 & \bf{770.10} & 
0.35\\CON8-5 & \bf{\underline{755.14}} & 15.94 & 
758.77 & 16.36 & 766.60 & 
-1.49\\CON8-6 & \bf{\underline{684.05}} & 18.04 & 
689.86 & 17.95 & 697.20 & 
-1.89\\CON8-7 & \bf{\underline{814.77}} & 14.54 & 
814.89 & 14.12 & 814.80 & 
-0.00\\CON8-8 & 782.93 & 17.27 & 
786.83 & 17.99 & \bf{771.30} & 
1.51\\CON8-9 & \bf{\underline{810.18}} & 20.30 & 
811.83 & 19.49 & 815.10 & 
-0.60\\[1ex]\hline
\end{tabular}
\label{table:nonlin}
\end{table} \clearpage
\begin{table}[ht]
\caption{Resultados de la ejecución de la metaheurística ACO, utilizando instancias de Dethloff con la configuración -n 22.0 -alpha 1.0 -beta 3.0 -q .5 -ro 0.015}
\centering
\small
\begin{tabular}{c c c c c c c}
\hline\hline
Instancia & Costo mínimo & Tiempo(seg.) & Costo promedio & Tiempo promedio(seg.) & Costo ACO & \%Gap \\ [0.5ex]
\hline
SCA3-0 & \bf{\underline{636.06}} & 15.31 & 
636.06 & 15.34 & 636.10 & 
-0.01\\SCA3-1 & \bf{\underline{697.84}} & 15.82 & 
697.84 & 16.20 & 700.10 & 
-0.32\\SCA3-2 & 659.34 & 14.41 & 
659.34 & 14.67 & \bf{659.30} & 
0.01\\SCA3-3 & 680.04 & 15.07 & 
680.04 & 15.42 & \bf{680.00} & 
0.01\\SCA3-4 & \bf{690.50} & 15.91 & 
690.50 & 19.64 & 690.50 & 0.00\\
SCA3-5 & \bf{\underline{659.90}} & 16.17 & 
662.04 & 16.03 & 671.10 & 
-1.67\\SCA3-6 & \bf{\underline{651.09}} & 16.12 & 
652.48 & 15.53 & 651.10 & 
-0.00\\SCA3-7 & 666.15 & 13.40 & 
666.15 & 13.19 & \bf{666.10} & 
0.01\\SCA3-8 & \bf{\underline{719.47}} & 14.87 & 
719.47 & 14.47 & 719.50 & 
-0.00\\SCA3-9 & \bf{681.00} & 13.20 & 
681.00 & 12.93 & 681.00 & 0.00\\
SCA8-0 & 968.79 & 16.45 & 
976.21 & 16.90 & \bf{961.60} & 
0.75\\SCA8-1 & \bf{\underline{1051.23}} & 14.38 & 
1054.83 & 13.86 & 1063.00 & 
-1.11\\SCA8-2 & 1046.29 & 12.55 & 
1049.56 & 12.44 & \bf{1040.60} & 
0.55\\SCA8-3 & 1010.50 & 16.43 & 
1012.64 & 16.73 & \bf{985.90} & 
2.50\\SCA8-4 & \bf{\underline{1067.66}} & 16.09 & 
1072.57 & 15.68 & 1071.00 & 
-0.31\\SCA8-5 & \bf{\underline{1044.21}} & 18.13 & 
1050.53 & 17.43 & 1054.30 & 
-0.96\\SCA8-6 & \bf{\underline{972.48}} & 16.99 & 
973.62 & 17.21 & 972.50 & 
-0.00\\SCA8-7 & 1061.78 & 17.25 & 
1067.26 & 17.30 & \bf{1059.70} & 
0.20\\SCA8-8 & \bf{\underline{1071.18}} & 16.11 & 
1072.13 & 16.95 & 1082.70 & 
-1.06\\SCA8-9 & \bf{\underline{1067.42}} & 12.60 & 
1067.42 & 13.11 & 1081.40 & 
-1.29\\CON3-0 & 616.52 & 17.74 & 
617.32 & 17.40 & \bf{616.50} & 
0.00\\CON3-1 & \bf{\underline{554.47}} & 16.13 & 
554.47 & 16.80 & 555.60 & 
-0.20\\CON3-2 & \bf{\underline{519.61}} & 14.52 & 
520.56 & 15.18 & 521.40 & 
-0.34\\CON3-3 & \bf{\underline{591.19}} & 17.82 & 
591.19 & 17.33 & 591.20 & 
-0.00\\CON3-4 & \bf{\underline{588.79}} & 14.23 & 
588.79 & 14.87 & 589.30 & 
-0.09\\CON3-5 & \bf{563.70} & 16.25 & 
564.46 & 16.82 & 563.70 & 0.00\\
CON3-6 & \bf{\underline{499.05}} & 25.64 & 
500.25 & 20.74 & 499.20 & 
-0.03\\CON3-7 & 578.22 & 13.48 & 
578.27 & 13.80 & \bf{577.50} & 
0.12\\CON3-8 & \bf{\underline{523.05}} & 14.65 & 
523.21 & 14.31 & 523.10 & 
-0.01\\CON3-9 & 578.25 & 15.12 & 
580.97 & 14.76 & \bf{578.20} & 
0.01\\CON8-0 & 868.40 & 16.01 & 
868.94 & 16.06 & \bf{858.90} & 
1.11\\CON8-1 & \bf{\underline{740.85}} & 17.14 & 
741.21 & 16.21 & 740.90 & 
-0.01\\CON8-2 & \bf{\underline{712.89}} & 21.00 & 
713.40 & 20.83 & 714.30 & 
-0.20\\CON8-3 & \bf{\underline{811.07}} & 15.89 & 
814.63 & 16.22 & 812.30 & 
-0.15\\CON8-4 & 776.72 & 16.24 & 
779.57 & 15.75 & \bf{770.10} & 
0.86\\CON8-5 & \bf{\underline{758.12}} & 15.94 & 
760.88 & 15.71 & 766.60 & 
-1.11\\CON8-6 & \bf{\underline{691.34}} & 19.11 & 
693.42 & 18.79 & 697.20 & 
-0.84\\CON8-7 & \bf{\underline{814.79}} & 16.04 & 
814.89 & 14.59 & 814.80 & 
-0.00\\CON8-8 & \bf{\underline{771.26}} & 19.20 & 
783.43 & 18.71 & 771.30 & 
-0.01\\CON8-9 & \bf{\underline{811.43}} & 17.66 & 
812.42 & 17.52 & 815.10 & 
-0.45\\[1ex]\hline
\end{tabular}
\label{table:nonlin}
\end{table} \clearpage
\begin{table}[ht]
\caption{Resultados de la ejecución de la metaheurística ACO, utilizando instancias de Dethloff con la configuración -n 22.0 -alpha 1.0 -beta 3.0 -q .6 -ro 0.015}
\centering
\small
\begin{tabular}{c c c c c c c}
\hline\hline
Instancia & Costo mínimo & Tiempo(seg.) & Costo promedio & Tiempo promedio(seg.) & Costo ACO & \%Gap \\ [0.5ex]
\hline
SCA3-0 & \bf{\underline{636.06}} & 15.36 & 
636.06 & 15.53 & 636.10 & 
-0.01\\SCA3-1 & \bf{\underline{697.84}} & 16.58 & 
697.84 & 16.45 & 700.10 & 
-0.32\\SCA3-2 & 659.34 & 14.29 & 
659.79 & 14.61 & \bf{659.30} & 
0.01\\SCA3-3 & 680.04 & 14.76 & 
680.04 & 15.60 & \bf{680.00} & 
0.01\\SCA3-4 & \bf{690.50} & 15.30 & 
690.50 & 16.16 & 690.50 & 0.00\\
SCA3-5 & \bf{\underline{659.90}} & 16.50 & 
662.34 & 16.89 & 671.10 & 
-1.67\\SCA3-6 & \bf{\underline{651.09}} & 15.22 & 
652.70 & 14.80 & 651.10 & 
-0.00\\SCA3-7 & 666.15 & 13.60 & 
666.15 & 13.88 & \bf{666.10} & 
0.01\\SCA3-8 & \bf{\underline{719.47}} & 14.63 & 
719.47 & 14.29 & 719.50 & 
-0.00\\SCA3-9 & \bf{681.00} & 12.09 & 
681.00 & 12.67 & 681.00 & 0.00\\
SCA8-0 & \bf{\underline{961.50}} & 18.14 & 
971.63 & 17.29 & 961.60 & 
-0.01\\SCA8-1 & \bf{\underline{1056.87}} & 13.42 & 
1062.59 & 13.24 & 1063.00 & 
-0.58\\SCA8-2 & 1046.29 & 11.76 & 
1048.33 & 11.36 & \bf{1040.60} & 
0.55\\SCA8-3 & 995.50 & 17.63 & 
1007.65 & 17.15 & \bf{985.90} & 
0.97\\SCA8-4 & \bf{\underline{1065.49}} & 14.97 & 
1068.16 & 15.38 & 1071.00 & 
-0.51\\SCA8-5 & \bf{\underline{1035.53}} & 18.01 & 
1045.31 & 17.53 & 1054.30 & 
-1.78\\SCA8-6 & \bf{\underline{972.48}} & 17.86 & 
974.59 & 18.07 & 972.50 & 
-0.00\\SCA8-7 & 1067.11 & 17.28 & 
1069.28 & 18.11 & \bf{1059.70} & 
0.70\\SCA8-8 & \bf{\underline{1071.18}} & 16.15 & 
1071.18 & 17.24 & 1082.70 & 
-1.06\\SCA8-9 & \bf{\underline{1067.42}} & 13.37 & 
1067.42 & 13.45 & 1081.40 & 
-1.29\\CON3-0 & 616.52 & 18.12 & 
618.84 & 17.95 & \bf{616.50} & 
0.00\\CON3-1 & \bf{\underline{554.47}} & 16.93 & 
554.47 & 16.12 & 555.60 & 
-0.20\\CON3-2 & \bf{\underline{519.11}} & 14.54 & 
521.24 & 14.69 & 521.40 & 
-0.44\\CON3-3 & \bf{\underline{591.19}} & 16.36 & 
591.19 & 16.70 & 591.20 & 
-0.00\\CON3-4 & \bf{\underline{588.79}} & 14.12 & 
588.92 & 14.62 & 589.30 & 
-0.09\\CON3-5 & \bf{563.70} & 15.30 & 
564.59 & 15.81 & 563.70 & 0.00\\
CON3-6 & 500.37 & 19.13 & 
501.01 & 19.07 & \bf{499.20} & 
0.23\\CON3-7 & 577.68 & 14.52 & 
577.95 & 14.29 & \bf{577.50} & 
0.03\\CON3-8 & \bf{\underline{523.05}} & 13.47 & 
523.25 & 14.04 & 523.10 & 
-0.01\\CON3-9 & 580.05 & 14.14 & 
585.63 & 14.79 & \bf{578.20} & 
0.32\\CON8-0 & 869.43 & 17.18 & 
870.26 & 16.29 & \bf{858.90} & 
1.23\\CON8-1 & \bf{\underline{740.85}} & 15.73 & 
743.24 & 15.46 & 740.90 & 
-0.01\\CON8-2 & \bf{\underline{713.44}} & 21.22 & 
713.55 & 21.19 & 714.30 & 
-0.12\\CON8-3 & \bf{\underline{811.07}} & 15.41 & 
811.74 & 15.47 & 812.30 & 
-0.15\\CON8-4 & 776.37 & 16.52 & 
781.37 & 15.90 & \bf{770.10} & 
0.81\\CON8-5 & \bf{\underline{762.01}} & 15.40 & 
763.21 & 15.81 & 766.60 & 
-0.60\\CON8-6 & \bf{\underline{686.39}} & 17.71 & 
690.47 & 18.43 & 697.20 & 
-1.55\\CON8-7 & \bf{\underline{814.79}} & 14.40 & 
814.84 & 13.60 & 814.80 & 
-0.00\\CON8-8 & 782.93 & 18.66 & 
785.62 & 18.59 & \bf{771.30} & 
1.51\\CON8-9 & \bf{\underline{810.18}} & 19.65 & 
812.72 & 18.66 & 815.10 & 
-0.60\\[1ex]\hline
\end{tabular}
\label{table:nonlin}
\end{table} \clearpage
\begin{table}[ht]
\caption{Resultados de la ejecución de la metaheurística ACO, utilizando instancias de Dethloff con la configuración -n 22.0 -alpha 1.0 -beta 3.0 -q .7 -ro 0.015}
\centering
\small
\begin{tabular}{c c c c c c c}
\hline\hline
Instancia & Costo mínimo & Tiempo(seg.) & Costo promedio & Tiempo promedio(seg.) & Costo ACO & \%Gap \\ [0.5ex]
\hline
SCA3-0 & \bf{\underline{636.06}} & 15.31 & 
636.06 & 15.97 & 636.10 & 
-0.01\\SCA3-1 & \bf{\underline{697.84}} & 16.34 & 
697.84 & 16.62 & 700.10 & 
-0.32\\SCA3-2 & 659.34 & 15.76 & 
659.34 & 15.25 & \bf{659.30} & 
0.01\\SCA3-3 & 680.04 & 16.34 & 
680.04 & 16.03 & \bf{680.00} & 
0.01\\SCA3-4 & \bf{690.50} & 15.60 & 
690.50 & 15.67 & 690.50 & 0.00\\
SCA3-5 & \bf{\underline{659.90}} & 16.90 & 
660.61 & 16.84 & 671.10 & 
-1.67\\SCA3-6 & \bf{\underline{651.09}} & 14.30 & 
652.48 & 14.99 & 651.10 & 
-0.00\\SCA3-7 & \bf{\underline{664.88}} & 12.11 & 
665.51 & 12.48 & 666.10 & 
-0.18\\SCA3-8 & \bf{\underline{719.47}} & 14.75 & 
719.54 & 14.25 & 719.50 & 
-0.00\\SCA3-9 & \bf{681.00} & 12.38 & 
681.00 & 12.23 & 681.00 & 0.00\\
SCA8-0 & \bf{\underline{961.50}} & 16.70 & 
970.02 & 16.90 & 961.60 & 
-0.01\\SCA8-1 & \bf{\underline{1058.78}} & 13.54 & 
1060.24 & 13.47 & 1063.00 & 
-0.40\\SCA8-2 & 1046.29 & 11.28 & 
1049.77 & 12.15 & \bf{1040.60} & 
0.55\\SCA8-3 & 995.50 & 17.44 & 
1007.69 & 17.14 & \bf{985.90} & 
0.97\\SCA8-4 & \bf{\underline{1065.49}} & 16.52 & 
1070.59 & 16.54 & 1071.00 & 
-0.51\\SCA8-5 & \bf{\underline{1045.30}} & 17.69 & 
1047.02 & 17.99 & 1054.30 & 
-0.85\\SCA8-6 & \bf{\underline{972.48}} & 17.36 & 
974.97 & 18.30 & 972.50 & 
-0.00\\SCA8-7 & 1066.65 & 18.27 & 
1069.06 & 18.15 & \bf{1059.70} & 
0.66\\SCA8-8 & \bf{\underline{1071.18}} & 16.88 & 
1071.18 & 17.77 & 1082.70 & 
-1.06\\SCA8-9 & \bf{\underline{1067.42}} & 13.85 & 
1067.42 & 13.64 & 1081.40 & 
-1.29\\CON3-0 & 617.59 & 18.87 & 
621.03 & 17.37 & \bf{616.50} & 
0.18\\CON3-1 & \bf{\underline{554.47}} & 17.07 & 
555.20 & 15.96 & 555.60 & 
-0.20\\CON3-2 & \bf{\underline{519.11}} & 14.17 & 
520.81 & 14.09 & 521.40 & 
-0.44\\CON3-3 & \bf{\underline{591.19}} & 18.50 & 
591.20 & 17.06 & 591.20 & 
-0.00\\CON3-4 & \bf{\underline{588.79}} & 15.39 & 
588.79 & 14.00 & 589.30 & 
-0.09\\CON3-5 & \bf{563.70} & 14.48 & 
564.85 & 14.66 & 563.70 & 0.00\\
CON3-6 & 500.80 & 18.50 & 
501.29 & 18.76 & \bf{499.20} & 
0.32\\CON3-7 & 577.68 & 13.50 & 
578.23 & 13.40 & \bf{577.50} & 
0.03\\CON3-8 & \bf{\underline{523.05}} & 14.98 & 
523.25 & 14.28 & 523.10 & 
-0.01\\CON3-9 & 578.25 & 15.53 & 
585.51 & 14.77 & \bf{578.20} & 
0.01\\CON8-0 & 865.86 & 16.30 & 
871.94 & 15.63 & \bf{858.90} & 
0.81\\CON8-1 & \bf{\underline{740.85}} & 17.18 & 
741.21 & 15.85 & 740.90 & 
-0.01\\CON8-2 & \bf{\underline{712.89}} & 21.93 & 
713.42 & 21.42 & 714.30 & 
-0.20\\CON8-3 & \bf{\underline{811.23}} & 16.17 & 
814.45 & 15.78 & 812.30 & 
-0.13\\CON8-4 & 775.70 & 15.18 & 
785.19 & 15.69 & \bf{770.10} & 
0.73\\CON8-5 & \bf{\underline{758.99}} & 15.64 & 
760.09 & 15.77 & 766.60 & 
-0.99\\CON8-6 & \bf{\underline{683.83}} & 18.82 & 
691.79 & 18.31 & 697.20 & 
-1.92\\CON8-7 & \bf{\underline{814.79}} & 14.44 & 
814.81 & 13.74 & 814.80 & 
-0.00\\CON8-8 & 780.71 & 19.18 & 
782.98 & 18.39 & \bf{771.30} & 
1.22\\CON8-9 & \bf{\underline{810.18}} & 18.15 & 
812.00 & 17.38 & 815.10 & 
-0.60\\[1ex]\hline
\end{tabular}
\label{table:nonlin}
\end{table} \clearpage
\begin{table}[ht]
\caption{Resultados de la ejecución de la metaheurística ACO, utilizando instancias de Dethloff con la configuración -n 22.0 -alpha 1.0 -beta 3.0 -q .8 -ro 0.015}
\centering
\small
\begin{tabular}{c c c c c c c}
\hline\hline
Instancia & Costo mínimo & Tiempo(seg.) & Costo promedio & Tiempo promedio(seg.) & Costo ACO & \%Gap \\ [0.5ex]
\hline
SCA3-0 & \bf{\underline{636.06}} & 15.20 & 
636.06 & 15.48 & 636.10 & 
-0.01\\SCA3-1 & \bf{\underline{697.84}} & 17.21 & 
697.84 & 16.37 & 700.10 & 
-0.32\\SCA3-2 & 659.34 & 14.35 & 
659.79 & 15.04 & \bf{659.30} & 
0.01\\SCA3-3 & 680.04 & 15.18 & 
680.04 & 14.85 & \bf{680.00} & 
0.01\\SCA3-4 & \bf{690.50} & 15.66 & 
690.50 & 15.72 & 690.50 & 0.00\\
SCA3-5 & \bf{\underline{659.90}} & 17.44 & 
663.33 & 16.96 & 671.10 & 
-1.67\\SCA3-6 & \bf{\underline{651.09}} & 15.01 & 
652.70 & 15.31 & 651.10 & 
-0.00\\SCA3-7 & 666.15 & 12.40 & 
666.15 & 12.48 & \bf{666.10} & 
0.01\\SCA3-8 & \bf{\underline{719.47}} & 15.23 & 
719.54 & 14.12 & 719.50 & 
-0.00\\SCA3-9 & \bf{681.00} & 12.43 & 
681.00 & 12.28 & 681.00 & 0.00\\
SCA8-0 & 968.79 & 15.80 & 
970.76 & 16.09 & \bf{961.60} & 
0.75\\SCA8-1 & \bf{\underline{1053.44}} & 12.81 & 
1060.56 & 13.09 & 1063.00 & 
-0.90\\SCA8-2 & 1046.29 & 11.56 & 
1049.21 & 11.51 & \bf{1040.60} & 
0.55\\SCA8-3 & 999.10 & 17.18 & 
1010.33 & 17.02 & \bf{985.90} & 
1.34\\SCA8-4 & \bf{\underline{1065.49}} & 15.60 & 
1066.59 & 16.38 & 1071.00 & 
-0.51\\SCA8-5 & \bf{\underline{1052.64}} & 18.50 & 
1053.86 & 18.45 & 1054.30 & 
-0.16\\SCA8-6 & \bf{\underline{972.48}} & 19.03 & 
974.75 & 18.99 & 972.50 & 
-0.00\\SCA8-7 & 1067.20 & 18.04 & 
1069.26 & 18.08 & \bf{1059.70} & 
0.71\\SCA8-8 & \bf{\underline{1071.18}} & 16.52 & 
1071.18 & 16.03 & 1082.70 & 
-1.06\\SCA8-9 & \bf{\underline{1067.42}} & 13.42 & 
1067.42 & 13.24 & 1081.40 & 
-1.29\\CON3-0 & 617.59 & 17.41 & 
620.26 & 17.71 & \bf{616.50} & 
0.18\\CON3-1 & \bf{\underline{554.47}} & 15.29 & 
554.86 & 15.43 & 555.60 & 
-0.20\\CON3-2 & \bf{\underline{521.38}} & 14.24 & 
522.53 & 13.96 & 521.40 & 
-0.00\\CON3-3 & \bf{\underline{591.19}} & 17.00 & 
591.20 & 16.79 & 591.20 & 
-0.00\\CON3-4 & \bf{\underline{588.79}} & 14.58 & 
588.92 & 14.21 & 589.30 & 
-0.09\\CON3-5 & 564.88 & 16.08 & 
566.20 & 15.34 & \bf{563.70} & 
0.21\\CON3-6 & 500.80 & 19.13 & 
501.14 & 19.02 & \bf{499.20} & 
0.32\\CON3-7 & 578.22 & 13.98 & 
578.36 & 13.62 & \bf{577.50} & 
0.12\\CON3-8 & 523.68 & 14.02 & 
524.06 & 13.49 & \bf{523.10} & 
0.11\\CON3-9 & 584.52 & 14.67 & 
587.53 & 14.78 & \bf{578.20} & 
1.09\\CON8-0 & 866.22 & 16.19 & 
870.31 & 15.58 & \bf{858.90} & 
0.85\\CON8-1 & \bf{\underline{740.85}} & 15.46 & 
741.25 & 16.50 & 740.90 & 
-0.01\\CON8-2 & \bf{\underline{712.89}} & 22.30 & 
713.91 & 21.91 & 714.30 & 
-0.20\\CON8-3 & 812.54 & 15.03 & 
816.31 & 15.80 & \bf{812.30} & 
0.03\\CON8-4 & 776.37 & 16.15 & 
780.54 & 16.39 & \bf{770.10} & 
0.81\\CON8-5 & \bf{\underline{759.93}} & 14.48 & 
761.37 & 14.84 & 766.60 & 
-0.87\\CON8-6 & \bf{\underline{689.23}} & 17.70 & 
693.51 & 18.98 & 697.20 & 
-1.14\\CON8-7 & \bf{\underline{814.79}} & 13.34 & 
815.67 & 13.40 & 814.80 & 
-0.00\\CON8-8 & 781.69 & 18.54 & 
785.35 & 19.40 & \bf{771.30} & 
1.35\\CON8-9 & \bf{\underline{810.18}} & 16.74 & 
812.10 & 16.84 & 815.10 & 
-0.60\\[1ex]\hline
\end{tabular}
\label{table:nonlin}
\end{table} \clearpage
\begin{table}[ht]
\caption{Resultados de la ejecución de la metaheurística ACO, utilizando instancias de Dethloff con la configuración -n 22.0 -alpha 1.0 -beta 3.0 -q .9 -ro 0.015}
\centering
\small
\begin{tabular}{c c c c c c c}
\hline\hline
Instancia & Costo mínimo & Tiempo(seg.) & Costo promedio & Tiempo promedio(seg.) & Costo ACO & \%Gap \\ [0.5ex]
\hline
SCA3-0 & \bf{\underline{636.06}} & 14.33 & 
636.13 & 14.67 & 636.10 & 
-0.01\\SCA3-1 & \bf{\underline{697.84}} & 18.00 & 
697.84 & 16.50 & 700.10 & 
-0.32\\SCA3-2 & 659.34 & 13.98 & 
660.55 & 14.44 & \bf{659.30} & 
0.01\\SCA3-3 & 680.04 & 15.33 & 
680.04 & 15.14 & \bf{680.00} & 
0.01\\SCA3-4 & \bf{690.50} & 15.28 & 
690.50 & 15.51 & 690.50 & 0.00\\
SCA3-5 & \bf{\underline{662.75}} & 16.58 & 
664.20 & 16.82 & 671.10 & 
-1.24\\SCA3-6 & 652.94 & 16.47 & 
653.68 & 15.69 & \bf{651.10} & 
0.28\\SCA3-7 & 666.15 & 14.51 & 
666.15 & 13.05 & \bf{666.10} & 
0.01\\SCA3-8 & \bf{\underline{719.47}} & 14.18 & 
719.47 & 13.95 & 719.50 & 
-0.00\\SCA3-9 & \bf{681.00} & 12.33 & 
681.00 & 11.62 & 681.00 & 0.00\\
SCA8-0 & 965.26 & 16.49 & 
988.36 & 17.20 & \bf{961.60} & 
0.38\\SCA8-1 & 1063.66 & 13.65 & 
1066.01 & 13.66 & \bf{1063.00} & 
0.06\\SCA8-2 & 1046.29 & 11.94 & 
1048.33 & 11.86 & \bf{1040.60} & 
0.55\\SCA8-3 & 1011.98 & 18.20 & 
1014.58 & 16.98 & \bf{985.90} & 
2.65\\SCA8-4 & \bf{\underline{1067.66}} & 17.53 & 
1072.59 & 16.93 & 1071.00 & 
-0.31\\SCA8-5 & \bf{\underline{1047.55}} & 16.71 & 
1053.34 & 17.21 & 1054.30 & 
-0.64\\SCA8-6 & \bf{\underline{972.48}} & 19.68 & 
974.75 & 18.86 & 972.50 & 
-0.00\\SCA8-7 & 1067.20 & 18.28 & 
1067.20 & 18.04 & \bf{1059.70} & 
0.71\\SCA8-8 & \bf{\underline{1071.18}} & 17.31 & 
1071.18 & 16.70 & 1082.70 & 
-1.06\\SCA8-9 & \bf{\underline{1067.42}} & 13.82 & 
1067.42 & 13.41 & 1081.40 & 
-1.29\\CON3-0 & 621.22 & 17.88 & 
622.61 & 18.34 & \bf{616.50} & 
0.77\\CON3-1 & \bf{\underline{554.47}} & 16.90 & 
555.65 & 15.72 & 555.60 & 
-0.20\\CON3-2 & \bf{\underline{521.38}} & 13.01 & 
521.38 & 14.11 & 521.40 & 
-0.00\\CON3-3 & \bf{\underline{591.19}} & 17.77 & 
591.19 & 17.34 & 591.20 & 
-0.00\\CON3-4 & \bf{\underline{588.79}} & 12.68 & 
588.92 & 13.94 & 589.30 & 
-0.09\\CON3-5 & \bf{563.70} & 14.26 & 
566.06 & 14.97 & 563.70 & 0.00\\
CON3-6 & 500.37 & 18.73 & 
501.72 & 19.28 & \bf{499.20} & 
0.23\\CON3-7 & 578.22 & 13.72 & 
579.33 & 14.11 & \bf{577.50} & 
0.12\\CON3-8 & \bf{\underline{523.05}} & 14.25 & 
524.13 & 13.80 & 523.10 & 
-0.01\\CON3-9 & 578.25 & 14.24 & 
582.69 & 14.96 & \bf{578.20} & 
0.01\\CON8-0 & 868.49 & 16.37 & 
880.02 & 16.59 & \bf{858.90} & 
1.12\\CON8-1 & \bf{\underline{740.85}} & 13.47 & 
742.85 & 15.61 & 740.90 & 
-0.01\\CON8-2 & \bf{\underline{713.44}} & 21.30 & 
714.33 & 22.41 & 714.30 & 
-0.12\\CON8-3 & 817.57 & 16.58 & 
817.57 & 16.20 & \bf{812.30} & 
0.65\\CON8-4 & 776.37 & 17.22 & 
784.79 & 16.41 & \bf{770.10} & 
0.81\\CON8-5 & \bf{\underline{759.93}} & 16.71 & 
762.27 & 15.87 & 766.60 & 
-0.87\\CON8-6 & \bf{\underline{690.53}} & 18.08 & 
693.75 & 18.09 & 697.20 & 
-0.96\\CON8-7 & \bf{\underline{814.79}} & 14.02 & 
819.33 & 13.64 & 814.80 & 
-0.00\\CON8-8 & 784.82 & 17.93 & 
790.02 & 17.31 & \bf{771.30} & 
1.75\\CON8-9 & \bf{\underline{812.34}} & 17.04 & 
814.98 & 17.37 & 815.10 & 
-0.34\\[1ex]\hline
\end{tabular}
\label{table:nonlin}
\end{table} \clearpage
\begin{table}[ht]
\caption{Resultados de la ejecución de la metaheurística ACO, utilizando instancias de Dethloff con la configuración -n 32.0 -alpha 1.0 -beta 3.0 -q 0.1 -ro 0.015}
\centering
\small
\begin{tabular}{c c c c c c c}
\hline\hline
Instancia & Costo mínimo & Tiempo(seg.) & Costo promedio & Tiempo promedio(seg.) & Costo ACO & \%Gap \\ [0.5ex]
\hline
SCA3-0 & \bf{\underline{636.06}} & 23.74 & 
636.06 & 23.03 & 636.10 & 
-0.01\\SCA3-1 & \bf{\underline{697.84}} & 25.16 & 
697.84 & 25.37 & 700.10 & 
-0.32\\SCA3-2 & 659.34 & 21.97 & 
659.34 & 22.31 & \bf{659.30} & 
0.01\\SCA3-3 & 680.04 & 21.26 & 
680.04 & 22.31 & \bf{680.00} & 
0.01\\SCA3-4 & \bf{690.50} & 25.11 & 
690.50 & 25.04 & 690.50 & 0.00\\
SCA3-5 & \bf{\underline{659.90}} & 24.25 & 
659.90 & 23.76 & 671.10 & 
-1.67\\SCA3-6 & \bf{\underline{651.09}} & 23.25 & 
652.01 & 23.68 & 651.10 & 
-0.00\\SCA3-7 & 666.15 & 20.53 & 
666.15 & 20.32 & \bf{666.10} & 
0.01\\SCA3-8 & \bf{\underline{719.47}} & 22.63 & 
719.47 & 23.28 & 719.50 & 
-0.00\\SCA3-9 & \bf{681.00} & 21.22 & 
681.00 & 19.96 & 681.00 & 0.00\\
SCA8-0 & 968.79 & 25.39 & 
972.14 & 25.75 & \bf{961.60} & 
0.75\\SCA8-1 & \bf{\underline{1049.65}} & 23.79 & 
1052.94 & 22.43 & 1063.00 & 
-1.26\\SCA8-2 & 1045.64 & 19.08 & 
1048.87 & 18.70 & \bf{1040.60} & 
0.48\\SCA8-3 & 995.60 & 22.27 & 
1002.46 & 23.38 & \bf{985.90} & 
0.98\\SCA8-4 & \bf{\underline{1065.49}} & 25.63 & 
1065.49 & 24.69 & 1071.00 & 
-0.51\\SCA8-5 & \bf{\underline{1039.38}} & 26.02 & 
1044.62 & 26.47 & 1054.30 & 
-1.42\\SCA8-6 & \bf{\underline{972.48}} & 26.53 & 
974.97 & 25.48 & 972.50 & 
-0.00\\SCA8-7 & 1066.65 & 25.04 & 
1067.02 & 25.55 & \bf{1059.70} & 
0.66\\SCA8-8 & \bf{\underline{1071.18}} & 24.64 & 
1071.18 & 25.57 & 1082.70 & 
-1.06\\SCA8-9 & \bf{\underline{1066.61}} & 19.81 & 
1067.22 & 20.80 & 1081.40 & 
-1.37\\CON3-0 & 616.52 & 26.06 & 
617.78 & 25.82 & \bf{616.50} & 
0.00\\CON3-1 & \bf{\underline{554.47}} & 23.85 & 
554.47 & 24.09 & 555.60 & 
-0.20\\CON3-2 & \bf{\underline{518.00}} & 21.66 & 
518.83 & 22.95 & 521.40 & 
-0.65\\CON3-3 & \bf{\underline{591.19}} & 25.53 & 
591.19 & 25.08 & 591.20 & 
-0.00\\CON3-4 & \bf{\underline{588.79}} & 20.73 & 
588.79 & 21.73 & 589.30 & 
-0.09\\CON3-5 & \bf{563.70} & 23.18 & 
564.00 & 23.00 & 563.70 & 0.00\\
CON3-6 & \bf{\underline{499.07}} & 28.23 & 
500.55 & 26.68 & 499.20 & 
-0.03\\CON3-7 & \bf{\underline{576.48}} & 21.48 & 
577.30 & 20.64 & 577.50 & 
-0.18\\CON3-8 & \bf{\underline{523.05}} & 21.83 & 
523.07 & 22.34 & 523.10 & 
-0.01\\CON3-9 & 578.25 & 21.80 & 
580.51 & 22.16 & \bf{578.20} & 
0.01\\CON8-0 & 866.22 & 23.69 & 
868.77 & 24.18 & \bf{858.90} & 
0.85\\CON8-1 & \bf{\underline{740.85}} & 22.96 & 
740.85 & 25.95 & 740.90 & 
-0.01\\CON8-2 & \bf{\underline{712.89}} & 29.19 & 
713.30 & 28.61 & 714.30 & 
-0.20\\CON8-3 & \bf{\underline{811.07}} & 24.49 & 
811.28 & 24.61 & 812.30 & 
-0.15\\CON8-4 & 776.34 & 21.92 & 
776.45 & 22.33 & \bf{770.10} & 
0.81\\CON8-5 & \bf{\underline{758.12}} & 22.00 & 
758.78 & 23.61 & 766.60 & 
-1.11\\CON8-6 & \bf{\underline{686.39}} & 27.20 & 
687.90 & 27.62 & 697.20 & 
-1.55\\CON8-7 & \bf{\underline{814.79}} & 23.04 & 
814.81 & 21.71 & 814.80 & 
-0.00\\CON8-8 & \bf{\underline{771.26}} & 26.42 & 
775.58 & 27.17 & 771.30 & 
-0.01\\CON8-9 & \bf{\underline{809.00}} & 27.34 & 
811.10 & 27.77 & 815.10 & 
-0.75\\[1ex]\hline
\end{tabular}
\label{table:nonlin}
\end{table} \clearpage
\begin{table}[ht]
\caption{Resultados de la ejecución de la metaheurística ACO, utilizando instancias de Dethloff con la configuración -n 32.0 -alpha 1.0 -beta 3.0 -q .2 -ro 0.015}
\centering
\small
\begin{tabular}{c c c c c c c}
\hline\hline
Instancia & Costo mínimo & Tiempo(seg.) & Costo promedio & Tiempo promedio(seg.) & Costo ACO & \%Gap \\ [0.5ex]
\hline
SCA3-0 & \bf{\underline{636.06}} & 22.44 & 
636.06 & 22.41 & 636.10 & 
-0.01\\SCA3-1 & \bf{\underline{697.84}} & 23.97 & 
697.84 & 25.13 & 700.10 & 
-0.32\\SCA3-2 & 659.34 & 22.96 & 
659.34 & 23.43 & \bf{659.30} & 
0.01\\SCA3-3 & 680.04 & 21.39 & 
680.04 & 21.49 & \bf{680.00} & 
0.01\\SCA3-4 & \bf{690.50} & 23.93 & 
690.50 & 24.70 & 690.50 & 0.00\\
SCA3-5 & \bf{\underline{659.90}} & 24.22 & 
661.33 & 23.97 & 671.10 & 
-1.67\\SCA3-6 & \bf{\underline{651.09}} & 21.54 & 
651.09 & 23.12 & 651.10 & 
-0.00\\SCA3-7 & \bf{\underline{659.17}} & 19.20 & 
664.09 & 20.41 & 666.10 & 
-1.04\\SCA3-8 & \bf{\underline{719.47}} & 24.14 & 
719.47 & 23.61 & 719.50 & 
-0.00\\SCA3-9 & \bf{681.00} & 19.11 & 
681.00 & 20.13 & 681.00 & 0.00\\
SCA8-0 & \bf{\underline{961.50}} & 24.73 & 
969.44 & 25.05 & 961.60 & 
-0.01\\SCA8-1 & \bf{\underline{1049.65}} & 20.80 & 
1053.73 & 20.24 & 1063.00 & 
-1.26\\SCA8-2 & 1042.17 & 17.86 & 
1045.61 & 19.20 & \bf{1040.60} & 
0.15\\SCA8-3 & 991.84 & 24.06 & 
1000.35 & 23.34 & \bf{985.90} & 
0.60\\SCA8-4 & \bf{\underline{1065.49}} & 25.48 & 
1065.94 & 24.59 & 1071.00 & 
-0.51\\SCA8-5 & \bf{\underline{1034.74}} & 26.59 & 
1037.24 & 26.45 & 1054.30 & 
-1.86\\SCA8-6 & \bf{\underline{972.48}} & 24.19 & 
972.68 & 25.10 & 972.50 & 
-0.00\\SCA8-7 & 1066.65 & 24.76 & 
1068.34 & 26.00 & \bf{1059.70} & 
0.66\\SCA8-8 & \bf{\underline{1071.18}} & 26.03 & 
1071.18 & 25.76 & 1082.70 & 
-1.06\\SCA8-9 & \bf{\underline{1067.42}} & 20.29 & 
1067.42 & 19.83 & 1081.40 & 
-1.29\\CON3-0 & 616.52 & 26.27 & 
618.12 & 25.19 & \bf{616.50} & 
0.00\\CON3-1 & \bf{\underline{554.47}} & 24.08 & 
555.08 & 24.15 & 555.60 & 
-0.20\\CON3-2 & \bf{\underline{519.11}} & 21.69 & 
519.68 & 22.05 & 521.40 & 
-0.44\\CON3-3 & \bf{\underline{591.19}} & 26.22 & 
591.19 & 25.06 & 591.20 & 
-0.00\\CON3-4 & \bf{\underline{588.79}} & 21.54 & 
588.79 & 21.57 & 589.30 & 
-0.09\\CON3-5 & \bf{563.70} & 24.99 & 
564.29 & 23.68 & 563.70 & 0.00\\
CON3-6 & 500.80 & 26.10 & 
500.80 & 27.05 & \bf{499.20} & 
0.32\\CON3-7 & \bf{\underline{576.48}} & 19.96 & 
577.70 & 20.45 & 577.50 & 
-0.18\\CON3-8 & \bf{\underline{523.05}} & 22.54 & 
523.41 & 21.83 & 523.10 & 
-0.01\\CON3-9 & 578.25 & 21.17 & 
580.63 & 22.11 & \bf{578.20} & 
0.01\\CON8-0 & 867.76 & 24.64 & 
869.01 & 23.51 & \bf{858.90} & 
1.03\\CON8-1 & \bf{\underline{740.85}} & 24.07 & 
740.85 & 24.77 & 740.90 & 
-0.01\\CON8-2 & \bf{\underline{712.89}} & 28.63 & 
713.16 & 32.23 & 714.30 & 
-0.20\\CON8-3 & \bf{\underline{811.07}} & 24.46 & 
812.13 & 23.21 & 812.30 & 
-0.15\\CON8-4 & 772.76 & 22.63 & 
775.99 & 22.14 & \bf{770.10} & 
0.35\\CON8-5 & \bf{\underline{758.12}} & 22.56 & 
759.05 & 22.87 & 766.60 & 
-1.11\\CON8-6 & \bf{\underline{685.06}} & 27.00 & 
687.96 & 26.77 & 697.20 & 
-1.74\\CON8-7 & \bf{\underline{814.77}} & 21.95 & 
814.80 & 21.49 & 814.80 & 
-0.00\\CON8-8 & \bf{\underline{769.65}} & 27.38 & 
780.13 & 27.98 & 771.30 & 
-0.21\\CON8-9 & \bf{\underline{810.18}} & 26.97 & 
810.78 & 26.86 & 815.10 & 
-0.60\\[1ex]\hline
\end{tabular}
\label{table:nonlin}
\end{table} \clearpage
\begin{table}[ht]
\caption{Resultados de la ejecución de la metaheurística ACO, utilizando instancias de Dethloff con la configuración -n 32.0 -alpha 1.0 -beta 3.0 -q .3 -ro 0.015}
\centering
\small
\begin{tabular}{c c c c c c c}
\hline\hline
Instancia & Costo mínimo & Tiempo(seg.) & Costo promedio & Tiempo promedio(seg.) & Costo ACO & \%Gap \\ [0.5ex]
\hline
SCA3-0 & \bf{\underline{636.06}} & 22.09 & 
636.06 & 22.91 & 636.10 & 
-0.01\\SCA3-1 & \bf{\underline{697.84}} & 25.50 & 
697.84 & 25.27 & 700.10 & 
-0.32\\SCA3-2 & 659.34 & 21.60 & 
659.34 & 21.62 & \bf{659.30} & 
0.01\\SCA3-3 & 680.04 & 22.06 & 
680.04 & 22.07 & \bf{680.00} & 
0.01\\SCA3-4 & \bf{690.50} & 23.29 & 
690.50 & 24.07 & 690.50 & 0.00\\
SCA3-5 & \bf{\underline{659.90}} & 23.57 & 
662.76 & 23.19 & 671.10 & 
-1.67\\SCA3-6 & \bf{\underline{651.09}} & 23.44 & 
651.09 & 23.18 & 651.10 & 
-0.00\\SCA3-7 & 666.15 & 19.76 & 
666.15 & 20.09 & \bf{666.10} & 
0.01\\SCA3-8 & \bf{\underline{719.47}} & 22.23 & 
719.47 & 22.80 & 719.50 & 
-0.00\\SCA3-9 & \bf{681.00} & 20.40 & 
681.00 & 19.29 & 681.00 & 0.00\\
SCA8-0 & \bf{\underline{961.50}} & 24.55 & 
968.65 & 24.41 & 961.60 & 
-0.01\\SCA8-1 & \bf{\underline{1052.71}} & 20.67 & 
1053.75 & 20.53 & 1063.00 & 
-0.97\\SCA8-2 & 1045.64 & 18.43 & 
1048.07 & 18.89 & \bf{1040.60} & 
0.48\\SCA8-3 & 991.84 & 23.21 & 
1004.45 & 23.12 & \bf{985.90} & 
0.60\\SCA8-4 & \bf{\underline{1065.49}} & 24.14 & 
1065.49 & 24.77 & 1071.00 & 
-0.51\\SCA8-5 & \bf{\underline{1038.59}} & 29.56 & 
1042.56 & 26.64 & 1054.30 & 
-1.49\\SCA8-6 & \bf{\underline{971.82}} & 30.44 & 
974.19 & 26.20 & 972.50 & 
-0.07\\SCA8-7 & 1066.65 & 24.41 & 
1067.04 & 25.85 & \bf{1059.70} & 
0.66\\SCA8-8 & \bf{\underline{1071.18}} & 25.06 & 
1071.18 & 25.34 & 1082.70 & 
-1.06\\SCA8-9 & \bf{\underline{1066.61}} & 19.79 & 
1067.22 & 19.90 & 1081.40 & 
-1.37\\CON3-0 & 616.52 & 24.62 & 
617.32 & 25.15 & \bf{616.50} & 
0.00\\CON3-1 & \bf{\underline{554.47}} & 22.59 & 
554.92 & 23.30 & 555.60 & 
-0.20\\CON3-2 & \bf{\underline{519.11}} & 21.79 & 
519.69 & 22.52 & 521.40 & 
-0.44\\CON3-3 & \bf{\underline{591.19}} & 24.90 & 
591.19 & 25.38 & 591.20 & 
-0.00\\CON3-4 & \bf{\underline{588.79}} & 21.81 & 
588.79 & 21.30 & 589.30 & 
-0.09\\CON3-5 & \bf{563.70} & 23.34 & 
563.70 & 23.20 & 563.70 & 0.00\\
CON3-6 & \bf{\underline{499.07}} & 26.08 & 
501.01 & 27.89 & 499.20 & 
-0.03\\CON3-7 & \bf{\underline{576.48}} & 20.12 & 
576.84 & 20.11 & 577.50 & 
-0.18\\CON3-8 & \bf{\underline{523.05}} & 22.85 & 
523.12 & 22.22 & 523.10 & 
-0.01\\CON3-9 & 578.25 & 20.27 & 
584.59 & 20.94 & \bf{578.20} & 
0.01\\CON8-0 & 866.32 & 22.98 & 
869.02 & 23.06 & \bf{858.90} & 
0.86\\CON8-1 & \bf{\underline{740.85}} & 23.62 & 
740.85 & 23.71 & 740.90 & 
-0.01\\CON8-2 & \bf{\underline{712.89}} & 29.78 & 
713.16 & 29.57 & 714.30 & 
-0.20\\CON8-3 & \bf{\underline{811.07}} & 23.15 & 
812.45 & 23.05 & 812.30 & 
-0.15\\CON8-4 & 776.34 & 21.58 & 
779.22 & 22.36 & \bf{770.10} & 
0.81\\CON8-5 & \bf{\underline{754.95}} & 23.58 & 
758.98 & 22.79 & 766.60 & 
-1.52\\CON8-6 & \bf{\underline{683.83}} & 27.53 & 
690.14 & 27.09 & 697.20 & 
-1.92\\CON8-7 & \bf{\underline{814.77}} & 21.11 & 
814.78 & 20.45 & 814.80 & 
-0.00\\CON8-8 & 780.71 & 28.38 & 
782.59 & 28.40 & \bf{771.30} & 
1.22\\CON8-9 & \bf{\underline{810.18}} & 25.92 & 
811.78 & 26.82 & 815.10 & 
-0.60\\[1ex]\hline
\end{tabular}
\label{table:nonlin}
\end{table} \clearpage
\begin{table}[ht]
\caption{Resultados de la ejecución de la metaheurística ACO, utilizando instancias de Dethloff con la configuración -n 32.0 -alpha 1.0 -beta 3.0 -q .4 -ro 0.015}
\centering
\small
\begin{tabular}{c c c c c c c}
\hline\hline
Instancia & Costo mínimo & Tiempo(seg.) & Costo promedio & Tiempo promedio(seg.) & Costo ACO & \%Gap \\ [0.5ex]
\hline
SCA3-0 & \bf{\underline{636.06}} & 21.81 & 
636.06 & 21.95 & 636.10 & 
-0.01\\SCA3-1 & \bf{\underline{697.84}} & 24.90 & 
697.84 & 24.26 & 700.10 & 
-0.32\\SCA3-2 & 659.34 & 23.24 & 
659.34 & 22.55 & \bf{659.30} & 
0.01\\SCA3-3 & 680.04 & 21.70 & 
680.04 & 21.84 & \bf{680.00} & 
0.01\\SCA3-4 & \bf{690.50} & 25.05 & 
690.50 & 23.67 & 690.50 & 0.00\\
SCA3-5 & \bf{\underline{659.90}} & 23.72 & 
661.33 & 24.55 & 671.10 & 
-1.67\\SCA3-6 & \bf{\underline{651.09}} & 23.16 & 
652.01 & 23.01 & 651.10 & 
-0.00\\SCA3-7 & 666.15 & 19.28 & 
666.15 & 19.38 & \bf{666.10} & 
0.01\\SCA3-8 & \bf{\underline{719.47}} & 20.98 & 
719.47 & 21.69 & 719.50 & 
-0.00\\SCA3-9 & \bf{681.00} & 18.26 & 
681.00 & 18.37 & 681.00 & 0.00\\
SCA8-0 & \bf{\underline{961.50}} & 24.69 & 
965.14 & 25.75 & 961.60 & 
-0.01\\SCA8-1 & \bf{\underline{1052.71}} & 20.78 & 
1054.52 & 20.14 & 1063.00 & 
-0.97\\SCA8-2 & 1042.17 & 18.27 & 
1047.09 & 18.16 & \bf{1040.60} & 
0.15\\SCA8-3 & 995.12 & 24.10 & 
999.52 & 23.57 & \bf{985.90} & 
0.94\\SCA8-4 & \bf{\underline{1065.49}} & 22.98 & 
1066.58 & 24.73 & 1071.00 & 
-0.51\\SCA8-5 & \bf{\underline{1034.74}} & 24.89 & 
1041.03 & 26.24 & 1054.30 & 
-1.86\\SCA8-6 & \bf{\underline{972.48}} & 25.87 & 
972.68 & 25.26 & 972.50 & 
-0.00\\SCA8-7 & 1067.03 & 25.00 & 
1067.16 & 26.29 & \bf{1059.70} & 
0.69\\SCA8-8 & \bf{\underline{1071.18}} & 24.54 & 
1071.18 & 25.46 & 1082.70 & 
-1.06\\SCA8-9 & \bf{\underline{1065.60}} & 20.07 & 
1066.76 & 20.34 & 1081.40 & 
-1.46\\CON3-0 & 616.52 & 26.61 & 
617.05 & 26.01 & \bf{616.50} & 
0.00\\CON3-1 & \bf{\underline{554.47}} & 23.12 & 
554.47 & 23.85 & 555.60 & 
-0.20\\CON3-2 & \bf{\underline{519.11}} & 22.30 & 
520.31 & 22.76 & 521.40 & 
-0.44\\CON3-3 & \bf{\underline{591.19}} & 25.90 & 
591.19 & 25.00 & 591.20 & 
-0.00\\CON3-4 & \bf{\underline{588.79}} & 21.08 & 
588.79 & 20.59 & 589.30 & 
-0.09\\CON3-5 & \bf{563.70} & 21.92 & 
564.00 & 23.60 & 563.70 & 0.00\\
CON3-6 & 500.37 & 27.87 & 
500.58 & 27.06 & \bf{499.20} & 
0.23\\CON3-7 & \bf{\underline{576.87}} & 22.21 & 
577.98 & 20.96 & 577.50 & 
-0.11\\CON3-8 & \bf{\underline{523.05}} & 22.00 & 
523.41 & 21.55 & 523.10 & 
-0.01\\CON3-9 & 578.98 & 22.25 & 
583.46 & 22.88 & \bf{578.20} & 
0.13\\CON8-0 & 866.22 & 23.41 & 
868.39 & 23.87 & \bf{858.90} & 
0.85\\CON8-1 & \bf{\underline{740.85}} & 23.28 & 
741.23 & 23.25 & 740.90 & 
-0.01\\CON8-2 & \bf{\underline{713.44}} & 29.94 & 
713.48 & 30.78 & 714.30 & 
-0.12\\CON8-3 & \bf{\underline{811.07}} & 23.72 & 
813.27 & 24.49 & 812.30 & 
-0.15\\CON8-4 & 776.37 & 23.25 & 
778.86 & 22.16 & \bf{770.10} & 
0.81\\CON8-5 & \bf{\underline{754.95}} & 24.37 & 
759.17 & 21.43 & 766.60 & 
-1.52\\CON8-6 & \bf{\underline{686.39}} & 26.93 & 
689.13 & 26.67 & 697.20 & 
-1.55\\CON8-7 & \bf{\underline{814.77}} & 20.68 & 
814.78 & 20.54 & 814.80 & 
-0.00\\CON8-8 & 777.45 & 28.06 & 
781.95 & 26.70 & \bf{771.30} & 
0.80\\CON8-9 & \bf{\underline{810.61}} & 26.92 & 
812.77 & 26.27 & 815.10 & 
-0.55\\[1ex]\hline
\end{tabular}
\label{table:nonlin}
\end{table} \clearpage
\begin{table}[ht]
\caption{Resultados de la ejecución de la metaheurística ACO, utilizando instancias de Dethloff con la configuración -n 32.0 -alpha 1.0 -beta 3.0 -q .5 -ro 0.015}
\centering
\small
\begin{tabular}{c c c c c c c}
\hline\hline
Instancia & Costo mínimo & Tiempo(seg.) & Costo promedio & Tiempo promedio(seg.) & Costo ACO & \%Gap \\ [0.5ex]
\hline
SCA3-0 & \bf{\underline{636.06}} & 22.99 & 
636.06 & 22.62 & 636.10 & 
-0.01\\SCA3-1 & \bf{\underline{697.84}} & 23.57 & 
697.84 & 24.87 & 700.10 & 
-0.32\\SCA3-2 & 659.34 & 21.04 & 
659.34 & 21.84 & \bf{659.30} & 
0.01\\SCA3-3 & 680.04 & 21.36 & 
680.04 & 22.40 & \bf{680.00} & 
0.01\\SCA3-4 & \bf{690.50} & 22.86 & 
690.50 & 23.68 & 690.50 & 0.00\\
SCA3-5 & \bf{\underline{659.90}} & 23.24 & 
660.61 & 23.04 & 671.10 & 
-1.67\\SCA3-6 & \bf{\underline{651.09}} & 22.44 & 
652.48 & 24.68 & 651.10 & 
-0.00\\SCA3-7 & \bf{\underline{659.17}} & 20.84 & 
664.40 & 20.90 & 666.10 & 
-1.04\\SCA3-8 & \bf{\underline{719.47}} & 20.41 & 
719.47 & 21.63 & 719.50 & 
-0.00\\SCA3-9 & \bf{681.00} & 20.61 & 
681.00 & 19.42 & 681.00 & 0.00\\
SCA8-0 & 968.79 & 25.45 & 
976.80 & 24.92 & \bf{961.60} & 
0.75\\SCA8-1 & \bf{\underline{1052.71}} & 20.35 & 
1055.75 & 20.10 & 1063.00 & 
-0.97\\SCA8-2 & 1046.29 & 17.60 & 
1048.35 & 17.50 & \bf{1040.60} & 
0.55\\SCA8-3 & 995.50 & 24.85 & 
1007.44 & 25.10 & \bf{985.90} & 
0.97\\SCA8-4 & \bf{\underline{1065.49}} & 24.76 & 
1066.58 & 25.51 & 1071.00 & 
-0.51\\SCA8-5 & \bf{\underline{1034.74}} & 25.01 & 
1036.24 & 25.70 & 1054.30 & 
-1.86\\SCA8-6 & \bf{\underline{972.48}} & 26.59 & 
972.48 & 26.31 & 972.50 & 
-0.00\\SCA8-7 & 1066.65 & 24.68 & 
1067.06 & 26.47 & \bf{1059.70} & 
0.66\\SCA8-8 & \bf{\underline{1071.18}} & 24.22 & 
1071.18 & 24.51 & 1082.70 & 
-1.06\\SCA8-9 & \bf{\underline{1067.42}} & 18.98 & 
1067.42 & 19.57 & 1081.40 & 
-1.29\\CON3-0 & 616.52 & 27.55 & 
617.32 & 26.03 & \bf{616.50} & 
0.00\\CON3-1 & \bf{\underline{554.47}} & 23.70 & 
554.47 & 23.59 & 555.60 & 
-0.20\\CON3-2 & \bf{\underline{519.11}} & 23.02 & 
520.37 & 22.77 & 521.40 & 
-0.44\\CON3-3 & \bf{\underline{591.19}} & 24.22 & 
591.19 & 24.69 & 591.20 & 
-0.00\\CON3-4 & \bf{\underline{588.79}} & 20.39 & 
588.92 & 22.02 & 589.30 & 
-0.09\\CON3-5 & \bf{563.70} & 23.71 & 
564.59 & 23.06 & 563.70 & 0.00\\
CON3-6 & 500.80 & 29.33 & 
501.46 & 27.33 & \bf{499.20} & 
0.32\\CON3-7 & \bf{\underline{576.48}} & 20.59 & 
577.23 & 20.67 & 577.50 & 
-0.18\\CON3-8 & \bf{\underline{523.05}} & 21.10 & 
523.23 & 20.82 & 523.10 & 
-0.01\\CON3-9 & 578.25 & 20.34 & 
582.41 & 21.43 & \bf{578.20} & 
0.01\\CON8-0 & 865.86 & 24.48 & 
870.34 & 24.18 & \bf{858.90} & 
0.81\\CON8-1 & \bf{\underline{740.85}} & 23.96 & 
740.85 & 24.25 & 740.90 & 
-0.01\\CON8-2 & \bf{\underline{712.89}} & 33.00 & 
713.42 & 32.36 & 714.30 & 
-0.20\\CON8-3 & \bf{\underline{811.07}} & 23.08 & 
813.38 & 23.96 & 812.30 & 
-0.15\\CON8-4 & 776.72 & 20.57 & 
780.55 & 22.54 & \bf{770.10} & 
0.86\\CON8-5 & \bf{\underline{758.12}} & 23.31 & 
759.76 & 22.75 & 766.60 & 
-1.11\\CON8-6 & \bf{\underline{683.83}} & 27.63 & 
687.73 & 27.76 & 697.20 & 
-1.92\\CON8-7 & \bf{\underline{814.79}} & 20.26 & 
814.88 & 21.03 & 814.80 & 
-0.00\\CON8-8 & \bf{\underline{771.26}} & 26.82 & 
781.04 & 27.25 & 771.30 & 
-0.01\\CON8-9 & \bf{\underline{810.18}} & 26.81 & 
811.10 & 27.02 & 815.10 & 
-0.60\\[1ex]\hline
\end{tabular}
\label{table:nonlin}
\end{table} \clearpage
\begin{table}[ht]
\caption{Resultados de la ejecución de la metaheurística ACO, utilizando instancias de Dethloff con la configuración -n 32.0 -alpha 1.0 -beta 3.0 -q .6 -ro 0.015}
\centering
\small
\begin{tabular}{c c c c c c c}
\hline\hline
Instancia & Costo mínimo & Tiempo(seg.) & Costo promedio & Tiempo promedio(seg.) & Costo ACO & \%Gap \\ [0.5ex]
\hline
SCA3-0 & \bf{\underline{636.06}} & 22.96 & 
636.06 & 22.48 & 636.10 & 
-0.01\\SCA3-1 & \bf{\underline{697.84}} & 23.43 & 
697.84 & 23.64 & 700.10 & 
-0.32\\SCA3-2 & 659.34 & 21.62 & 
659.34 & 21.82 & \bf{659.30} & 
0.01\\SCA3-3 & 680.04 & 21.34 & 
680.04 & 21.79 & \bf{680.00} & 
0.01\\SCA3-4 & \bf{690.50} & 24.55 & 
690.50 & 23.62 & 690.50 & 0.00\\
SCA3-5 & \bf{\underline{659.90}} & 25.58 & 
660.90 & 24.93 & 671.10 & 
-1.67\\SCA3-6 & \bf{\underline{651.09}} & 20.73 & 
652.81 & 21.50 & 651.10 & 
-0.00\\SCA3-7 & 666.15 & 20.22 & 
666.15 & 19.13 & \bf{666.10} & 
0.01\\SCA3-8 & \bf{\underline{719.47}} & 20.68 & 
719.47 & 21.50 & 719.50 & 
-0.00\\SCA3-9 & \bf{681.00} & 19.46 & 
681.00 & 18.84 & 681.00 & 0.00\\
SCA8-0 & \bf{\underline{961.50}} & 25.03 & 
964.38 & 25.69 & 961.60 & 
-0.01\\SCA8-1 & \bf{\underline{1052.71}} & 18.34 & 
1055.43 & 19.62 & 1063.00 & 
-0.97\\SCA8-2 & 1050.37 & 17.30 & 
1050.63 & 17.80 & \bf{1040.60} & 
0.94\\SCA8-3 & 991.84 & 23.93 & 
1001.79 & 24.47 & \bf{985.90} & 
0.60\\SCA8-4 & \bf{\underline{1065.49}} & 26.26 & 
1067.14 & 23.88 & 1071.00 & 
-0.51\\SCA8-5 & \bf{\underline{1034.74}} & 26.30 & 
1041.97 & 25.62 & 1054.30 & 
-1.86\\SCA8-6 & \bf{\underline{972.48}} & 25.38 & 
972.68 & 26.06 & 972.50 & 
-0.00\\SCA8-7 & 1063.22 & 25.25 & 
1068.26 & 25.81 & \bf{1059.70} & 
0.33\\SCA8-8 & \bf{\underline{1071.18}} & 25.15 & 
1071.18 & 24.67 & 1082.70 & 
-1.06\\SCA8-9 & \bf{\underline{1060.50}} & 20.73 & 
1065.69 & 20.20 & 1081.40 & 
-1.93\\CON3-0 & 616.52 & 27.02 & 
618.83 & 25.85 & \bf{616.50} & 
0.00\\CON3-1 & \bf{\underline{554.47}} & 22.73 & 
554.86 & 24.14 & 555.60 & 
-0.20\\CON3-2 & \bf{\underline{519.11}} & 21.95 & 
520.81 & 22.06 & 521.40 & 
-0.44\\CON3-3 & \bf{\underline{591.19}} & 24.78 & 
591.19 & 24.77 & 591.20 & 
-0.00\\CON3-4 & \bf{\underline{588.79}} & 19.70 & 
588.79 & 20.61 & 589.30 & 
-0.09\\CON3-5 & \bf{563.70} & 23.66 & 
564.00 & 23.36 & 563.70 & 0.00\\
CON3-6 & 500.37 & 28.48 & 
500.83 & 27.62 & \bf{499.20} & 
0.23\\CON3-7 & \bf{\underline{576.48}} & 20.32 & 
577.50 & 20.79 & 577.50 & 
-0.18\\CON3-8 & \bf{\underline{523.05}} & 22.37 & 
523.25 & 20.38 & 523.10 & 
-0.01\\CON3-9 & 578.25 & 22.40 & 
581.12 & 21.26 & \bf{578.20} & 
0.01\\CON8-0 & 867.34 & 23.22 & 
869.05 & 23.32 & \bf{858.90} & 
0.98\\CON8-1 & \bf{\underline{740.85}} & 22.73 & 
741.21 & 22.89 & 740.90 & 
-0.01\\CON8-2 & \bf{\underline{713.44}} & 29.70 & 
713.44 & 31.51 & 714.30 & 
-0.12\\CON8-3 & \bf{\underline{812.11}} & 21.68 & 
815.44 & 22.51 & 812.30 & 
-0.02\\CON8-4 & 776.37 & 23.80 & 
777.50 & 23.12 & \bf{770.10} & 
0.81\\CON8-5 & \bf{\underline{755.14}} & 22.85 & 
759.26 & 22.73 & 766.60 & 
-1.49\\CON8-6 & \bf{\underline{686.39}} & 26.09 & 
690.99 & 26.86 & 697.20 & 
-1.55\\CON8-7 & \bf{\underline{814.79}} & 18.71 & 
814.81 & 19.75 & 814.80 & 
-0.00\\CON8-8 & 777.63 & 28.27 & 
783.08 & 28.06 & \bf{771.30} & 
0.82\\CON8-9 & \bf{\underline{809.00}} & 24.82 & 
811.22 & 25.42 & 815.10 & 
-0.75\\[1ex]\hline
\end{tabular}
\label{table:nonlin}
\end{table} \clearpage
\begin{table}[ht]
\caption{Resultados de la ejecución de la metaheurística ACO, utilizando instancias de Dethloff con la configuración -n 32.0 -alpha 1.0 -beta 3.0 -q .7 -ro 0.015}
\centering
\small
\begin{tabular}{c c c c c c c}
\hline\hline
Instancia & Costo mínimo & Tiempo(seg.) & Costo promedio & Tiempo promedio(seg.) & Costo ACO & \%Gap \\ [0.5ex]
\hline
SCA3-0 & \bf{\underline{636.06}} & 23.44 & 
636.06 & 21.78 & 636.10 & 
-0.01\\SCA3-1 & \bf{\underline{697.84}} & 23.44 & 
697.84 & 23.94 & 700.10 & 
-0.32\\SCA3-2 & 659.34 & 20.90 & 
659.34 & 21.20 & \bf{659.30} & 
0.01\\SCA3-3 & 680.04 & 21.77 & 
680.04 & 26.21 & \bf{680.00} & 
0.01\\SCA3-4 & \bf{690.50} & 23.50 & 
690.50 & 23.83 & 690.50 & 0.00\\
SCA3-5 & \bf{\underline{659.90}} & 26.14 & 
662.76 & 25.13 & 671.10 & 
-1.67\\SCA3-6 & \bf{\underline{651.09}} & 22.72 & 
651.90 & 21.86 & 651.10 & 
-0.00\\SCA3-7 & 666.15 & 18.13 & 
666.15 & 18.27 & \bf{666.10} & 
0.01\\SCA3-8 & \bf{\underline{719.47}} & 21.01 & 
719.47 & 20.99 & 719.50 & 
-0.00\\SCA3-9 & \bf{681.00} & 19.33 & 
681.00 & 18.31 & 681.00 & 0.00\\
SCA8-0 & \bf{\underline{961.50}} & 25.76 & 
973.36 & 24.84 & 961.60 & 
-0.01\\SCA8-1 & \bf{\underline{1052.36}} & 18.90 & 
1056.90 & 18.93 & 1063.00 & 
-1.00\\SCA8-2 & 1046.29 & 18.04 & 
1049.27 & 17.98 & \bf{1040.60} & 
0.55\\SCA8-3 & 1001.69 & 23.61 & 
1007.12 & 23.16 & \bf{985.90} & 
1.60\\SCA8-4 & \bf{\underline{1065.49}} & 24.81 & 
1067.09 & 24.29 & 1071.00 & 
-0.51\\SCA8-5 & \bf{\underline{1034.74}} & 24.57 & 
1041.06 & 25.61 & 1054.30 & 
-1.86\\SCA8-6 & \bf{\underline{972.48}} & 25.86 & 
972.68 & 26.03 & 972.50 & 
-0.00\\SCA8-7 & 1067.20 & 27.40 & 
1067.31 & 26.98 & \bf{1059.70} & 
0.71\\SCA8-8 & \bf{\underline{1071.18}} & 25.05 & 
1071.18 & 24.16 & 1082.70 & 
-1.06\\SCA8-9 & \bf{\underline{1063.68}} & 19.05 & 
1066.49 & 19.27 & 1081.40 & 
-1.64\\CON3-0 & 616.52 & 26.05 & 
617.32 & 27.73 & \bf{616.50} & 
0.00\\CON3-1 & \bf{\underline{554.47}} & 23.94 & 
555.53 & 24.02 & 555.60 & 
-0.20\\CON3-2 & \bf{\underline{519.11}} & 19.80 & 
520.81 & 20.68 & 521.40 & 
-0.44\\CON3-3 & \bf{\underline{591.19}} & 24.67 & 
591.19 & 25.06 & 591.20 & 
-0.00\\CON3-4 & \bf{\underline{588.79}} & 20.47 & 
588.92 & 21.26 & 589.30 & 
-0.09\\CON3-5 & \bf{563.70} & 24.33 & 
564.59 & 23.83 & 563.70 & 0.00\\
CON3-6 & 500.80 & 27.74 & 
501.68 & 27.41 & \bf{499.20} & 
0.32\\CON3-7 & \bf{\underline{576.48}} & 19.79 & 
577.70 & 20.21 & 577.50 & 
-0.18\\CON3-8 & \bf{\underline{523.05}} & 19.98 & 
523.46 & 19.46 & 523.10 & 
-0.01\\CON3-9 & 578.25 & 22.57 & 
580.45 & 21.41 & \bf{578.20} & 
0.01\\CON8-0 & 866.22 & 22.40 & 
869.20 & 23.13 & \bf{858.90} & 
0.85\\CON8-1 & \bf{\underline{740.85}} & 21.70 & 
741.23 & 23.11 & 740.90 & 
-0.01\\CON8-2 & \bf{\underline{712.89}} & 32.32 & 
713.36 & 31.20 & 714.30 & 
-0.20\\CON8-3 & \bf{\underline{811.07}} & 23.58 & 
812.51 & 22.86 & 812.30 & 
-0.15\\CON8-4 & 776.37 & 22.40 & 
783.55 & 23.33 & \bf{770.10} & 
0.81\\CON8-5 & \bf{\underline{757.53}} & 22.32 & 
759.64 & 22.23 & 766.60 & 
-1.18\\CON8-6 & \bf{\underline{690.53}} & 25.90 & 
693.39 & 25.51 & 697.20 & 
-0.96\\CON8-7 & \bf{\underline{814.77}} & 20.51 & 
814.89 & 19.94 & 814.80 & 
-0.00\\CON8-8 & 784.09 & 27.22 & 
786.39 & 26.68 & \bf{771.30} & 
1.66\\CON8-9 & \bf{\underline{811.16}} & 25.73 & 
813.04 & 24.80 & 815.10 & 
-0.48\\[1ex]\hline
\end{tabular}
\label{table:nonlin}
\end{table} \clearpage
\begin{table}[ht]
\caption{Resultados de la ejecución de la metaheurística ACO, utilizando instancias de Dethloff con la configuración -n 32.0 -alpha 1.0 -beta 3.0 -q .8 -ro 0.015}
\centering
\small
\begin{tabular}{c c c c c c c}
\hline\hline
Instancia & Costo mínimo & Tiempo(seg.) & Costo promedio & Tiempo promedio(seg.) & Costo ACO & \%Gap \\ [0.5ex]
\hline
SCA3-0 & \bf{\underline{636.06}} & 21.92 & 
636.06 & 20.99 & 636.10 & 
-0.01\\SCA3-1 & \bf{\underline{697.84}} & 23.37 & 
697.84 & 23.83 & 700.10 & 
-0.32\\SCA3-2 & 659.34 & 20.89 & 
659.79 & 21.80 & \bf{659.30} & 
0.01\\SCA3-3 & 680.04 & 21.74 & 
680.04 & 21.86 & \bf{680.00} & 
0.01\\SCA3-4 & \bf{690.50} & 21.88 & 
690.50 & 22.34 & 690.50 & 0.00\\
SCA3-5 & \bf{\underline{659.90}} & 24.55 & 
661.33 & 24.24 & 671.10 & 
-1.67\\SCA3-6 & 652.47 & 20.79 & 
652.82 & 22.22 & \bf{651.10} & 
0.21\\SCA3-7 & 666.15 & 18.21 & 
666.15 & 18.83 & \bf{666.10} & 
0.01\\SCA3-8 & \bf{\underline{719.47}} & 20.31 & 
719.47 & 19.53 & 719.50 & 
-0.00\\SCA3-9 & \bf{681.00} & 18.40 & 
681.00 & 18.09 & 681.00 & 0.00\\
SCA8-0 & 968.79 & 24.40 & 
982.68 & 24.23 & \bf{961.60} & 
0.75\\SCA8-1 & \bf{\underline{1052.71}} & 20.08 & 
1057.99 & 19.51 & 1063.00 & 
-0.97\\SCA8-2 & 1046.29 & 17.49 & 
1049.30 & 17.02 & \bf{1040.60} & 
0.55\\SCA8-3 & 1008.29 & 22.72 & 
1012.69 & 23.31 & \bf{985.90} & 
2.27\\SCA8-4 & \bf{\underline{1065.49}} & 25.26 & 
1073.27 & 23.76 & 1071.00 & 
-0.51\\SCA8-5 & \bf{\underline{1029.95}} & 25.06 & 
1046.43 & 25.61 & 1054.30 & 
-2.31\\SCA8-6 & \bf{\underline{972.48}} & 25.84 & 
974.97 & 26.39 & 972.50 & 
-0.00\\SCA8-7 & 1067.20 & 27.55 & 
1069.59 & 26.56 & \bf{1059.70} & 
0.71\\SCA8-8 & \bf{\underline{1071.18}} & 23.67 & 
1071.18 & 24.16 & 1082.70 & 
-1.06\\SCA8-9 & \bf{\underline{1067.42}} & 19.71 & 
1067.42 & 18.92 & 1081.40 & 
-1.29\\CON3-0 & 617.59 & 24.37 & 
620.31 & 25.22 & \bf{616.50} & 
0.18\\CON3-1 & \bf{\underline{554.47}} & 24.27 & 
555.65 & 23.54 & 555.60 & 
-0.20\\CON3-2 & \bf{\underline{519.61}} & 20.69 & 
520.94 & 19.98 & 521.40 & 
-0.34\\CON3-3 & \bf{\underline{591.19}} & 24.52 & 
591.19 & 23.79 & 591.20 & 
-0.00\\CON3-4 & \bf{\underline{588.79}} & 20.32 & 
589.87 & 20.69 & 589.30 & 
-0.09\\CON3-5 & 564.88 & 21.89 & 
564.88 & 22.20 & \bf{563.70} & 
0.21\\CON3-6 & 500.80 & 28.92 & 
501.68 & 27.48 & \bf{499.20} & 
0.32\\CON3-7 & 578.22 & 21.80 & 
578.32 & 20.08 & \bf{577.50} & 
0.12\\CON3-8 & \bf{\underline{523.05}} & 19.22 & 
523.52 & 19.80 & 523.10 & 
-0.01\\CON3-9 & 578.25 & 21.02 & 
582.04 & 20.56 & \bf{578.20} & 
0.01\\CON8-0 & 869.43 & 22.44 & 
872.18 & 22.44 & \bf{858.90} & 
1.23\\CON8-1 & \bf{\underline{740.85}} & 20.34 & 
741.21 & 22.16 & 740.90 & 
-0.01\\CON8-2 & \bf{\underline{713.44}} & 31.53 & 
713.55 & 32.22 & 714.30 & 
-0.12\\CON8-3 & \bf{\underline{811.92}} & 22.89 & 
816.16 & 22.42 & 812.30 & 
-0.05\\CON8-4 & 776.37 & 22.85 & 
779.28 & 23.05 & \bf{770.10} & 
0.81\\CON8-5 & \bf{\underline{759.93}} & 22.69 & 
762.23 & 22.17 & 766.60 & 
-0.87\\CON8-6 & \bf{\underline{685.69}} & 25.08 & 
691.88 & 25.72 & 697.20 & 
-1.65\\CON8-7 & \bf{\underline{814.79}} & 19.78 & 
814.88 & 20.21 & 814.80 & 
-0.00\\CON8-8 & 784.28 & 26.30 & 
786.51 & 26.79 & \bf{771.30} & 
1.68\\CON8-9 & \bf{\underline{812.60}} & 25.44 & 
813.32 & 25.70 & 815.10 & 
-0.31\\[1ex]\hline
\end{tabular}
\label{table:nonlin}
\end{table} \clearpage
\begin{table}[ht]
\caption{Resultados de la ejecución de la metaheurística ACO, utilizando instancias de Dethloff con la configuración -n 32.0 -alpha 1.0 -beta 3.0 -q .9 -ro 0.015}
\centering
\small
\begin{tabular}{c c c c c c c}
\hline\hline
Instancia & Costo mínimo & Tiempo(seg.) & Costo promedio & Tiempo promedio(seg.) & Costo ACO & \%Gap \\ [0.5ex]
\hline
SCA3-0 & \bf{\underline{636.06}} & 20.72 & 
636.06 & 21.92 & 636.10 & 
-0.01\\SCA3-1 & \bf{\underline{697.84}} & 23.12 & 
697.84 & 24.27 & 700.10 & 
-0.32\\SCA3-2 & 659.34 & 20.49 & 
661.00 & 21.40 & \bf{659.30} & 
0.01\\SCA3-3 & 680.04 & 22.72 & 
680.18 & 22.16 & \bf{680.00} & 
0.01\\SCA3-4 & \bf{690.50} & 22.80 & 
690.50 & 22.64 & 690.50 & 0.00\\
SCA3-5 & \bf{\underline{662.75}} & 25.00 & 
664.92 & 23.73 & 671.10 & 
-1.24\\SCA3-6 & 652.94 & 22.40 & 
653.49 & 22.61 & \bf{651.10} & 
0.28\\SCA3-7 & 666.15 & 18.64 & 
666.15 & 18.92 & \bf{666.10} & 
0.01\\SCA3-8 & \bf{\underline{719.47}} & 19.32 & 
719.54 & 19.25 & 719.50 & 
-0.00\\SCA3-9 & \bf{681.00} & 17.55 & 
681.00 & 17.39 & 681.00 & 0.00\\
SCA8-0 & \bf{\underline{961.50}} & 27.63 & 
981.28 & 24.68 & 961.60 & 
-0.01\\SCA8-1 & \bf{\underline{1052.71}} & 19.52 & 
1060.77 & 18.95 & 1063.00 & 
-0.97\\SCA8-2 & 1050.37 & 15.90 & 
1052.07 & 16.95 & \bf{1040.60} & 
0.94\\SCA8-3 & 1011.61 & 23.52 & 
1014.94 & 24.84 & \bf{985.90} & 
2.61\\SCA8-4 & \bf{\underline{1065.49}} & 24.24 & 
1074.89 & 24.92 & 1071.00 & 
-0.51\\SCA8-5 & \bf{\underline{1034.74}} & 28.50 & 
1050.70 & 27.17 & 1054.30 & 
-1.86\\SCA8-6 & \bf{\underline{972.48}} & 27.53 & 
976.70 & 27.23 & 972.50 & 
-0.00\\SCA8-7 & 1066.65 & 27.67 & 
1070.58 & 26.90 & \bf{1059.70} & 
0.66\\SCA8-8 & \bf{\underline{1071.18}} & 23.38 & 
1071.18 & 23.38 & 1082.70 & 
-1.06\\SCA8-9 & \bf{\underline{1067.42}} & 19.06 & 
1067.42 & 18.52 & 1081.40 & 
-1.29\\CON3-0 & 621.06 & 24.97 & 
623.20 & 26.28 & \bf{616.50} & 
0.74\\CON3-1 & \bf{\underline{554.47}} & 22.42 & 
554.47 & 22.32 & 555.60 & 
-0.20\\CON3-2 & \bf{\underline{521.38}} & 18.76 & 
521.79 & 20.70 & 521.40 & 
-0.00\\CON3-3 & \bf{\underline{591.19}} & 24.65 & 
591.19 & 24.16 & 591.20 & 
-0.00\\CON3-4 & \bf{\underline{588.79}} & 19.92 & 
589.05 & 21.02 & 589.30 & 
-0.09\\CON3-5 & \bf{563.70} & 20.44 & 
565.11 & 21.83 & 563.70 & 0.00\\
CON3-6 & 500.80 & 29.00 & 
501.14 & 27.30 & \bf{499.20} & 
0.32\\CON3-7 & 577.68 & 19.76 & 
578.23 & 20.02 & \bf{577.50} & 
0.03\\CON3-8 & 523.14 & 19.24 & 
523.70 & 18.78 & \bf{523.10} & 
0.01\\CON3-9 & 581.06 & 18.88 & 
586.06 & 20.20 & \bf{578.20} & 
0.49\\CON8-0 & 869.43 & 22.01 & 
870.59 & 23.00 & \bf{858.90} & 
1.23\\CON8-1 & 742.29 & 21.62 & 
743.02 & 21.17 & \bf{740.90} & 
0.19\\CON8-2 & \bf{\underline{713.44}} & 31.55 & 
714.44 & 32.35 & 714.30 & 
-0.12\\CON8-3 & 812.75 & 23.23 & 
815.92 & 22.71 & \bf{812.30} & 
0.06\\CON8-4 & 777.48 & 24.86 & 
784.77 & 23.07 & \bf{770.10} & 
0.96\\CON8-5 & \bf{\underline{759.93}} & 22.69 & 
760.78 & 22.13 & 766.60 & 
-0.87\\CON8-6 & \bf{\underline{685.68}} & 25.07 & 
692.53 & 26.18 & 697.20 & 
-1.65\\CON8-7 & 814.86 & 18.35 & 
817.01 & 19.62 & \bf{814.80} & 
0.01\\CON8-8 & 782.86 & 24.71 & 
788.27 & 25.80 & \bf{771.30} & 
1.50\\CON8-9 & \bf{\underline{812.60}} & 29.30 & 
814.74 & 26.64 & 815.10 & 
-0.31\\[1ex]\hline
\end{tabular}
\label{table:nonlin}
\end{table} \clearpage
\begin{table}[ht]
\caption{Resultados de la ejecución de la metaheurística ACO, utilizando instancias de Dethloff con la configuración -n 42.0 -alpha 1.0 -beta 3.0 -q 0.1 -ro 0.015}
\centering
\small
\begin{tabular}{c c c c c c c}
\hline\hline
Instancia & Costo mínimo & Tiempo(seg.) & Costo promedio & Tiempo promedio(seg.) & Costo ACO & \%Gap \\ [0.5ex]
\hline
SCA3-0 & \bf{\underline{636.06}} & 28.83 & 
636.06 & 31.07 & 636.10 & 
-0.01\\SCA3-1 & \bf{\underline{697.84}} & 31.67 & 
697.84 & 31.96 & 700.10 & 
-0.32\\SCA3-2 & 659.34 & 28.32 & 
659.34 & 28.32 & \bf{659.30} & 
0.01\\SCA3-3 & 680.04 & 26.66 & 
680.04 & 28.04 & \bf{680.00} & 
0.01\\SCA3-4 & \bf{690.50} & 32.31 & 
690.50 & 31.69 & 690.50 & 0.00\\
SCA3-5 & \bf{\underline{659.90}} & 32.46 & 
661.33 & 31.83 & 671.10 & 
-1.67\\SCA3-6 & \bf{\underline{651.09}} & 29.03 & 
651.09 & 30.69 & 651.10 & 
-0.00\\SCA3-7 & \bf{\underline{664.88}} & 25.84 & 
665.51 & 25.98 & 666.10 & 
-0.18\\SCA3-8 & \bf{\underline{719.47}} & 27.79 & 
719.47 & 29.07 & 719.50 & 
-0.00\\SCA3-9 & \bf{681.00} & 27.90 & 
681.00 & 26.38 & 681.00 & 0.00\\
SCA8-0 & \bf{\underline{961.50}} & 32.00 & 
967.28 & 31.91 & 961.60 & 
-0.01\\SCA8-1 & \bf{\underline{1052.71}} & 26.74 & 
1053.79 & 27.86 & 1063.00 & 
-0.97\\SCA8-2 & 1045.64 & 24.13 & 
1048.24 & 25.59 & \bf{1040.60} & 
0.48\\SCA8-3 & 995.50 & 30.74 & 
999.30 & 31.16 & \bf{985.90} & 
0.97\\SCA8-4 & \bf{\underline{1065.49}} & 31.88 & 
1065.49 & 31.91 & 1071.00 & 
-0.51\\SCA8-5 & \bf{\underline{1036.88}} & 37.59 & 
1037.83 & 36.40 & 1054.30 & 
-1.65\\SCA8-6 & \bf{\underline{972.48}} & 33.89 & 
972.68 & 33.51 & 972.50 & 
-0.00\\SCA8-7 & 1063.22 & 32.36 & 
1064.93 & 34.10 & \bf{1059.70} & 
0.33\\SCA8-8 & \bf{\underline{1071.18}} & 32.41 & 
1071.18 & 34.07 & 1082.70 & 
-1.06\\SCA8-9 & \bf{\underline{1063.68}} & 26.96 & 
1065.55 & 27.12 & 1081.40 & 
-1.64\\CON3-0 & 616.52 & 33.43 & 
617.05 & 33.66 & \bf{616.50} & 
0.00\\CON3-1 & \bf{\underline{554.47}} & 32.10 & 
554.47 & 32.10 & 555.60 & 
-0.20\\CON3-2 & \bf{\underline{519.11}} & 31.82 & 
519.80 & 32.85 & 521.40 & 
-0.44\\CON3-3 & \bf{\underline{591.19}} & 34.40 & 
591.19 & 33.45 & 591.20 & 
-0.00\\CON3-4 & \bf{\underline{588.79}} & 28.30 & 
588.79 & 28.95 & 589.30 & 
-0.09\\CON3-5 & \bf{563.70} & 29.62 & 
564.00 & 30.75 & 563.70 & 0.00\\
CON3-6 & \bf{\underline{499.05}} & 36.97 & 
500.36 & 35.97 & 499.20 & 
-0.03\\CON3-7 & \bf{\underline{576.48}} & 29.06 & 
576.96 & 28.15 & 577.50 & 
-0.18\\CON3-8 & \bf{\underline{523.05}} & 29.65 & 
523.10 & 29.53 & 523.10 & 
-0.01\\CON3-9 & 578.25 & 29.02 & 
579.88 & 28.91 & \bf{578.20} & 
0.01\\CON8-0 & 865.86 & 29.53 & 
867.28 & 30.59 & \bf{858.90} & 
0.81\\CON8-1 & \bf{\underline{740.85}} & 31.86 & 
740.87 & 31.39 & 740.90 & 
-0.01\\CON8-2 & \bf{\underline{712.89}} & 40.22 & 
713.11 & 38.27 & 714.30 & 
-0.20\\CON8-3 & \bf{\underline{811.07}} & 32.07 & 
811.33 & 30.81 & 812.30 & 
-0.15\\CON8-4 & 776.37 & 30.12 & 
777.24 & 29.45 & \bf{770.10} & 
0.81\\CON8-5 & \bf{\underline{754.95}} & 29.70 & 
757.78 & 29.96 & 766.60 & 
-1.52\\CON8-6 & \bf{\underline{684.05}} & 34.16 & 
687.17 & 35.55 & 697.20 & 
-1.89\\CON8-7 & \bf{\underline{814.77}} & 44.73 & 
814.78 & 31.71 & 814.80 & 
-0.00\\CON8-8 & 779.43 & 37.78 & 
781.48 & 35.79 & \bf{771.30} & 
1.05\\CON8-9 & \bf{\underline{811.43}} & 37.38 & 
811.61 & 37.10 & 815.10 & 
-0.45\\[1ex]\hline
\end{tabular}
\label{table:nonlin}
\end{table} \clearpage
\begin{table}[ht]
\caption{Resultados de la ejecución de la metaheurística ACO, utilizando instancias de Dethloff con la configuración -n 42.0 -alpha 1.0 -beta 3.0 -q .2 -ro 0.015}
\centering
\small
\begin{tabular}{c c c c c c c}
\hline\hline
Instancia & Costo mínimo & Tiempo(seg.) & Costo promedio & Tiempo promedio(seg.) & Costo ACO & \%Gap \\ [0.5ex]
\hline
SCA3-0 & \bf{\underline{636.06}} & 29.04 & 
636.06 & 29.30 & 636.10 & 
-0.01\\SCA3-1 & \bf{\underline{697.84}} & 34.81 & 
697.84 & 33.65 & 700.10 & 
-0.32\\SCA3-2 & 659.34 & 28.94 & 
659.34 & 29.57 & \bf{659.30} & 
0.01\\SCA3-3 & 680.04 & 28.40 & 
680.04 & 28.80 & \bf{680.00} & 
0.01\\SCA3-4 & \bf{690.50} & 32.91 & 
690.50 & 31.27 & 690.50 & 0.00\\
SCA3-5 & \bf{\underline{659.90}} & 31.28 & 
660.49 & 33.54 & 671.10 & 
-1.67\\SCA3-6 & \bf{\underline{651.09}} & 29.31 & 
651.55 & 30.06 & 651.10 & 
-0.00\\SCA3-7 & \bf{\underline{659.17}} & 26.20 & 
664.09 & 26.13 & 666.10 & 
-1.04\\SCA3-8 & \bf{\underline{719.47}} & 29.89 & 
719.47 & 30.22 & 719.50 & 
-0.00\\SCA3-9 & \bf{681.00} & 25.30 & 
681.00 & 24.85 & 681.00 & 0.00\\
SCA8-0 & \bf{\underline{961.50}} & 32.10 & 
970.13 & 32.45 & 961.60 & 
-0.01\\SCA8-1 & \bf{\underline{1052.71}} & 26.97 & 
1058.16 & 27.11 & 1063.00 & 
-0.97\\SCA8-2 & 1044.24 & 26.50 & 
1046.15 & 24.93 & \bf{1040.60} & 
0.35\\SCA8-3 & 997.63 & 32.11 & 
1001.04 & 32.05 & \bf{985.90} & 
1.19\\SCA8-4 & \bf{\underline{1065.49}} & 33.68 & 
1066.03 & 32.02 & 1071.00 & 
-0.51\\SCA8-5 & \bf{\underline{1034.74}} & 36.39 & 
1041.66 & 37.03 & 1054.30 & 
-1.86\\SCA8-6 & \bf{\underline{972.48}} & 33.97 & 
974.96 & 32.84 & 972.50 & 
-0.00\\SCA8-7 & 1063.22 & 32.36 & 
1066.07 & 33.26 & \bf{1059.70} & 
0.33\\SCA8-8 & \bf{\underline{1071.18}} & 34.71 & 
1071.18 & 34.08 & 1082.70 & 
-1.06\\SCA8-9 & \bf{\underline{1063.68}} & 26.89 & 
1066.03 & 26.99 & 1081.40 & 
-1.64\\CON3-0 & 617.59 & 34.71 & 
617.97 & 33.97 & \bf{616.50} & 
0.18\\CON3-1 & \bf{\underline{554.47}} & 30.37 & 
554.47 & 32.17 & 555.60 & 
-0.20\\CON3-2 & \bf{\underline{519.11}} & 28.96 & 
519.24 & 30.36 & 521.40 & 
-0.44\\CON3-3 & \bf{\underline{591.19}} & 31.63 & 
591.19 & 32.55 & 591.20 & 
-0.00\\CON3-4 & \bf{\underline{588.79}} & 28.20 & 
588.79 & 28.58 & 589.30 & 
-0.09\\CON3-5 & \bf{563.70} & 39.07 & 
563.70 & 32.72 & 563.70 & 0.00\\
CON3-6 & \bf{\underline{499.05}} & 33.86 & 
500.58 & 34.62 & 499.20 & 
-0.03\\CON3-7 & \bf{\underline{576.48}} & 25.97 & 
577.27 & 26.90 & 577.50 & 
-0.18\\CON3-8 & \bf{\underline{523.05}} & 30.62 & 
523.10 & 28.37 & 523.10 & 
-0.01\\CON3-9 & 578.25 & 26.22 & 
578.88 & 27.99 & \bf{578.20} & 
0.01\\CON8-0 & 868.49 & 30.38 & 
869.39 & 30.58 & \bf{858.90} & 
1.12\\CON8-1 & \bf{\underline{740.85}} & 32.72 & 
740.85 & 32.37 & 740.90 & 
-0.01\\CON8-2 & \bf{\underline{712.89}} & 39.27 & 
713.30 & 38.63 & 714.30 & 
-0.20\\CON8-3 & \bf{\underline{811.07}} & 32.05 & 
811.99 & 30.66 & 812.30 & 
-0.15\\CON8-4 & 776.37 & 29.09 & 
776.46 & 31.60 & \bf{770.10} & 
0.81\\CON8-5 & \bf{\underline{758.12}} & 35.21 & 
759.19 & 29.70 & 766.60 & 
-1.11\\CON8-6 & \bf{\underline{685.69}} & 33.61 & 
688.64 & 34.80 & 697.20 & 
-1.65\\CON8-7 & \bf{\underline{814.77}} & 25.84 & 
814.78 & 27.68 & 814.80 & 
-0.00\\CON8-8 & 779.21 & 35.16 & 
782.06 & 35.15 & \bf{771.30} & 
1.03\\CON8-9 & \bf{\underline{809.00}} & 34.40 & 
810.80 & 35.71 & 815.10 & 
-0.75\\[1ex]\hline
\end{tabular}
\label{table:nonlin}
\end{table} \clearpage
\begin{table}[ht]
\caption{Resultados de la ejecución de la metaheurística ACO, utilizando instancias de Dethloff con la configuración -n 42.0 -alpha 1.0 -beta 3.0 -q .3 -ro 0.015}
\centering
\small
\begin{tabular}{c c c c c c c}
\hline\hline
Instancia & Costo mínimo & Tiempo(seg.) & Costo promedio & Tiempo promedio(seg.) & Costo ACO & \%Gap \\ [0.5ex]
\hline
SCA3-0 & \bf{\underline{636.06}} & 29.32 & 
636.06 & 29.00 & 636.10 & 
-0.01\\SCA3-1 & \bf{\underline{697.84}} & 33.94 & 
697.84 & 31.99 & 700.10 & 
-0.32\\SCA3-2 & 659.34 & 29.47 & 
659.34 & 28.77 & \bf{659.30} & 
0.01\\SCA3-3 & 680.04 & 26.75 & 
680.04 & 28.37 & \bf{680.00} & 
0.01\\SCA3-4 & \bf{690.50} & 32.82 & 
690.50 & 31.91 & 690.50 & 0.00\\
SCA3-5 & \bf{\underline{659.90}} & 31.87 & 
661.62 & 31.89 & 671.10 & 
-1.67\\SCA3-6 & \bf{\underline{651.09}} & 27.86 & 
651.90 & 29.19 & 651.10 & 
-0.00\\SCA3-7 & 666.15 & 25.42 & 
666.15 & 24.98 & \bf{666.10} & 
0.01\\SCA3-8 & \bf{\underline{719.47}} & 29.30 & 
719.47 & 28.64 & 719.50 & 
-0.00\\SCA3-9 & \bf{681.00} & 26.60 & 
681.00 & 25.42 & 681.00 & 0.00\\
SCA8-0 & \bf{\underline{961.50}} & 31.53 & 
969.05 & 32.16 & 961.60 & 
-0.01\\SCA8-1 & \bf{\underline{1049.65}} & 27.16 & 
1053.03 & 26.20 & 1063.00 & 
-1.26\\SCA8-2 & 1043.79 & 24.52 & 
1046.29 & 23.87 & \bf{1040.60} & 
0.31\\SCA8-3 & 995.50 & 31.78 & 
1000.49 & 30.52 & \bf{985.90} & 
0.97\\SCA8-4 & \bf{\underline{1065.49}} & 33.46 & 
1067.44 & 32.74 & 1071.00 & 
-0.51\\SCA8-5 & \bf{\underline{1034.74}} & 34.41 & 
1039.67 & 34.93 & 1054.30 & 
-1.86\\SCA8-6 & \bf{\underline{972.48}} & 35.76 & 
973.83 & 33.45 & 972.50 & 
-0.00\\SCA8-7 & 1066.82 & 35.16 & 
1067.08 & 35.05 & \bf{1059.70} & 
0.67\\SCA8-8 & \bf{\underline{1071.18}} & 39.05 & 
1071.18 & 35.48 & 1082.70 & 
-1.06\\SCA8-9 & \bf{\underline{1065.60}} & 27.09 & 
1066.51 & 27.11 & 1081.40 & 
-1.46\\CON3-0 & 616.52 & 32.54 & 
617.43 & 32.80 & \bf{616.50} & 
0.00\\CON3-1 & \bf{\underline{554.47}} & 32.32 & 
554.47 & 31.88 & 555.60 & 
-0.20\\CON3-2 & \bf{\underline{519.61}} & 31.16 & 
520.26 & 30.40 & 521.40 & 
-0.34\\CON3-3 & \bf{\underline{591.19}} & 30.81 & 
591.19 & 32.17 & 591.20 & 
-0.00\\CON3-4 & \bf{\underline{588.79}} & 27.85 & 
588.79 & 27.22 & 589.30 & 
-0.09\\CON3-5 & \bf{563.70} & 30.06 & 
564.00 & 30.27 & 563.70 & 0.00\\
CON3-6 & \bf{\underline{499.05}} & 34.64 & 
500.15 & 34.31 & 499.20 & 
-0.03\\CON3-7 & \bf{\underline{576.84}} & 28.70 & 
577.19 & 27.99 & 577.50 & 
-0.11\\CON3-8 & \bf{\underline{523.05}} & 28.13 & 
523.23 & 27.45 & 523.10 & 
-0.01\\CON3-9 & 585.25 & 27.35 & 
586.82 & 27.61 & \bf{578.20} & 
1.22\\CON8-0 & 866.22 & 29.26 & 
868.05 & 29.96 & \bf{858.90} & 
0.85\\CON8-1 & \bf{\underline{740.85}} & 32.52 & 
741.21 & 30.99 & 740.90 & 
-0.01\\CON8-2 & \bf{\underline{712.89}} & 39.55 & 
713.07 & 39.67 & 714.30 & 
-0.20\\CON8-3 & \bf{\underline{811.07}} & 28.48 & 
811.74 & 30.29 & 812.30 & 
-0.15\\CON8-4 & 776.37 & 30.24 & 
780.60 & 31.19 & \bf{770.10} & 
0.81\\CON8-5 & \bf{\underline{758.12}} & 27.53 & 
759.00 & 29.88 & 766.60 & 
-1.11\\CON8-6 & \bf{\underline{686.36}} & 35.55 & 
688.57 & 35.47 & 697.20 & 
-1.55\\CON8-7 & \bf{\underline{814.50}} & 27.68 & 
814.75 & 27.48 & 814.80 & 
-0.04\\CON8-8 & \bf{\underline{771.26}} & 56.30 & 
778.50 & 41.98 & 771.30 & 
-0.01\\CON8-9 & \bf{\underline{812.60}} & 35.77 & 
813.30 & 35.59 & 815.10 & 
-0.31\\[1ex]\hline
\end{tabular}
\label{table:nonlin}
\end{table} \clearpage
\begin{table}[ht]
\caption{Resultados de la ejecución de la metaheurística ACO, utilizando instancias de Dethloff con la configuración -n 42.0 -alpha 1.0 -beta 3.0 -q .4 -ro 0.015}
\centering
\small
\begin{tabular}{c c c c c c c}
\hline\hline
Instancia & Costo mínimo & Tiempo(seg.) & Costo promedio & Tiempo promedio(seg.) & Costo ACO & \%Gap \\ [0.5ex]
\hline
SCA3-0 & \bf{\underline{636.06}} & 28.69 & 
636.06 & 30.45 & 636.10 & 
-0.01\\SCA3-1 & \bf{\underline{697.84}} & 31.93 & 
697.84 & 33.02 & 700.10 & 
-0.32\\SCA3-2 & 659.34 & 28.10 & 
659.34 & 27.93 & \bf{659.30} & 
0.01\\SCA3-3 & 680.04 & 30.87 & 
680.04 & 29.75 & \bf{680.00} & 
0.01\\SCA3-4 & \bf{690.50} & 31.04 & 
690.50 & 31.39 & 690.50 & 0.00\\
SCA3-5 & \bf{\underline{662.75}} & 33.35 & 
662.75 & 31.39 & 671.10 & 
-1.24\\SCA3-6 & \bf{\underline{651.09}} & 32.76 & 
651.09 & 31.38 & 651.10 & 
-0.00\\SCA3-7 & 666.15 & 26.53 & 
666.15 & 26.52 & \bf{666.10} & 
0.01\\SCA3-8 & \bf{\underline{719.47}} & 27.90 & 
719.47 & 28.32 & 719.50 & 
-0.00\\SCA3-9 & \bf{681.00} & 23.73 & 
681.00 & 23.93 & 681.00 & 0.00\\
SCA8-0 & 975.50 & 31.47 & 
977.93 & 32.92 & \bf{961.60} & 
1.45\\SCA8-1 & \bf{\underline{1052.71}} & 26.30 & 
1052.89 & 25.79 & 1063.00 & 
-0.97\\SCA8-2 & 1046.29 & 23.56 & 
1046.29 & 23.62 & \bf{1040.60} & 
0.55\\SCA8-3 & 995.50 & 31.91 & 
1000.18 & 32.55 & \bf{985.90} & 
0.97\\SCA8-4 & \bf{\underline{1065.49}} & 31.95 & 
1065.49 & 31.53 & 1071.00 & 
-0.51\\SCA8-5 & \bf{\underline{1034.74}} & 34.00 & 
1043.10 & 36.09 & 1054.30 & 
-1.86\\SCA8-6 & \bf{\underline{972.48}} & 34.05 & 
973.53 & 33.50 & 972.50 & 
-0.00\\SCA8-7 & 1067.03 & 35.55 & 
1068.33 & 34.61 & \bf{1059.70} & 
0.69\\SCA8-8 & \bf{\underline{1071.18}} & 34.33 & 
1071.18 & 33.38 & 1082.70 & 
-1.06\\SCA8-9 & \bf{\underline{1061.23}} & 25.45 & 
1065.42 & 26.35 & 1081.40 & 
-1.87\\CON3-0 & 616.52 & 32.14 & 
618.12 & 33.05 & \bf{616.50} & 
0.00\\CON3-1 & \bf{\underline{554.47}} & 30.54 & 
554.47 & 30.91 & 555.60 & 
-0.20\\CON3-2 & \bf{\underline{519.11}} & 31.33 & 
519.80 & 29.59 & 521.40 & 
-0.44\\CON3-3 & \bf{\underline{591.19}} & 31.28 & 
591.19 & 32.39 & 591.20 & 
-0.00\\CON3-4 & \bf{\underline{588.79}} & 26.66 & 
588.79 & 27.38 & 589.30 & 
-0.09\\CON3-5 & \bf{563.70} & 30.34 & 
563.70 & 31.79 & 563.70 & 0.00\\
CON3-6 & 500.37 & 34.18 & 
500.75 & 34.86 & \bf{499.20} & 
0.23\\CON3-7 & 577.68 & 29.39 & 
578.13 & 28.42 & \bf{577.50} & 
0.03\\CON3-8 & \bf{\underline{523.05}} & 30.55 & 
523.25 & 27.89 & 523.10 & 
-0.01\\CON3-9 & 578.25 & 29.59 & 
581.90 & 28.72 & \bf{578.20} & 
0.01\\CON8-0 & 864.56 & 32.38 & 
868.21 & 31.24 & \bf{858.90} & 
0.66\\CON8-1 & \bf{\underline{740.85}} & 29.87 & 
740.85 & 31.02 & 740.90 & 
-0.01\\CON8-2 & \bf{\underline{712.89}} & 39.26 & 
713.30 & 39.79 & 714.30 & 
-0.20\\CON8-3 & \bf{\underline{811.07}} & 30.44 & 
812.99 & 30.34 & 812.30 & 
-0.15\\CON8-4 & 775.22 & 29.13 & 
778.12 & 29.01 & \bf{770.10} & 
0.66\\CON8-5 & \bf{\underline{758.84}} & 28.49 & 
760.01 & 29.50 & 766.60 & 
-1.01\\CON8-6 & \bf{\underline{685.06}} & 34.83 & 
688.22 & 35.16 & 697.20 & 
-1.74\\CON8-7 & \bf{\underline{814.50}} & 25.50 & 
814.80 & 27.58 & 814.80 & 
-0.04\\CON8-8 & 779.50 & 34.63 & 
782.42 & 35.66 & \bf{771.30} & 
1.06\\CON8-9 & \bf{\underline{810.18}} & 36.43 & 
812.00 & 35.60 & 815.10 & 
-0.60\\[1ex]\hline
\end{tabular}
\label{table:nonlin}
\end{table} \clearpage
\begin{table}[ht]
\caption{Resultados de la ejecución de la metaheurística ACO, utilizando instancias de Dethloff con la configuración -n 42.0 -alpha 1.0 -beta 3.0 -q .5 -ro 0.015}
\centering
\small
\begin{tabular}{c c c c c c c}
\hline\hline
Instancia & Costo mínimo & Tiempo(seg.) & Costo promedio & Tiempo promedio(seg.) & Costo ACO & \%Gap \\ [0.5ex]
\hline
SCA3-0 & \bf{\underline{636.06}} & 28.22 & 
636.06 & 29.07 & 636.10 & 
-0.01\\SCA3-1 & \bf{\underline{697.84}} & 31.62 & 
697.84 & 31.62 & 700.10 & 
-0.32\\SCA3-2 & 659.34 & 28.37 & 
659.34 & 28.41 & \bf{659.30} & 
0.01\\SCA3-3 & 680.04 & 29.76 & 
680.04 & 29.01 & \bf{680.00} & 
0.01\\SCA3-4 & \bf{690.50} & 30.82 & 
690.50 & 32.30 & 690.50 & 0.00\\
SCA3-5 & \bf{\underline{659.90}} & 31.89 & 
660.61 & 30.67 & 671.10 & 
-1.67\\SCA3-6 & \bf{\underline{651.09}} & 29.78 & 
652.48 & 28.79 & 651.10 & 
-0.00\\SCA3-7 & 666.15 & 26.89 & 
666.15 & 25.44 & \bf{666.10} & 
0.01\\SCA3-8 & \bf{\underline{719.47}} & 27.88 & 
719.47 & 28.52 & 719.50 & 
-0.00\\SCA3-9 & \bf{681.00} & 23.38 & 
681.00 & 26.07 & 681.00 & 0.00\\
SCA8-0 & 973.03 & 32.51 & 
976.96 & 32.22 & \bf{961.60} & 
1.19\\SCA8-1 & \bf{\underline{1052.71}} & 26.54 & 
1054.87 & 25.93 & 1063.00 & 
-0.97\\SCA8-2 & 1042.17 & 22.96 & 
1047.25 & 26.71 & \bf{1040.60} & 
0.15\\SCA8-3 & 995.50 & 31.18 & 
1002.14 & 31.03 & \bf{985.90} & 
0.97\\SCA8-4 & \bf{\underline{1065.49}} & 33.56 & 
1066.48 & 30.92 & 1071.00 & 
-0.51\\SCA8-5 & \bf{\underline{1037.06}} & 35.37 & 
1048.80 & 34.15 & 1054.30 & 
-1.64\\SCA8-6 & \bf{\underline{972.48}} & 31.50 & 
972.89 & 33.66 & 972.50 & 
-0.00\\SCA8-7 & 1063.60 & 34.80 & 
1067.60 & 34.46 & \bf{1059.70} & 
0.37\\SCA8-8 & \bf{\underline{1071.18}} & 31.88 & 
1071.18 & 31.50 & 1082.70 & 
-1.06\\SCA8-9 & \bf{\underline{1067.42}} & 27.56 & 
1067.42 & 26.57 & 1081.40 & 
-1.29\\CON3-0 & 616.52 & 32.81 & 
618.49 & 33.41 & \bf{616.50} & 
0.00\\CON3-1 & \bf{\underline{554.47}} & 32.20 & 
554.47 & 32.58 & 555.60 & 
-0.20\\CON3-2 & \bf{\underline{519.11}} & 27.74 & 
519.93 & 28.40 & 521.40 & 
-0.44\\CON3-3 & \bf{\underline{591.19}} & 35.06 & 
591.19 & 33.60 & 591.20 & 
-0.00\\CON3-4 & \bf{\underline{588.79}} & 27.60 & 
588.92 & 27.64 & 589.30 & 
-0.09\\CON3-5 & \bf{563.70} & 29.71 & 
564.29 & 30.27 & 563.70 & 0.00\\
CON3-6 & 500.37 & 32.94 & 
500.93 & 35.64 & \bf{499.20} & 
0.23\\CON3-7 & \bf{\underline{576.48}} & 26.81 & 
577.28 & 26.75 & 577.50 & 
-0.18\\CON3-8 & \bf{\underline{523.05}} & 24.32 & 
523.23 & 26.05 & 523.10 & 
-0.01\\CON3-9 & 578.25 & 25.77 & 
583.67 & 28.25 & \bf{578.20} & 
0.01\\CON8-0 & 865.86 & 30.20 & 
868.50 & 30.09 & \bf{858.90} & 
0.81\\CON8-1 & \bf{\underline{740.85}} & 29.24 & 
740.85 & 30.22 & 740.90 & 
-0.01\\CON8-2 & \bf{\underline{712.89}} & 38.91 & 
713.34 & 38.34 & 714.30 & 
-0.20\\CON8-3 & \bf{\underline{811.07}} & 30.14 & 
812.66 & 30.23 & 812.30 & 
-0.15\\CON8-4 & 776.37 & 29.03 & 
778.72 & 28.82 & \bf{770.10} & 
0.81\\CON8-5 & \bf{\underline{759.93}} & 29.07 & 
760.18 & 29.45 & 766.60 & 
-0.87\\CON8-6 & \bf{\underline{683.83}} & 34.26 & 
684.43 & 34.32 & 697.20 & 
-1.92\\CON8-7 & \bf{\underline{814.77}} & 26.95 & 
814.80 & 25.84 & 814.80 & 
-0.00\\CON8-8 & 778.34 & 35.89 & 
782.61 & 35.09 & \bf{771.30} & 
0.91\\CON8-9 & \bf{\underline{810.18}} & 33.76 & 
811.39 & 33.47 & 815.10 & 
-0.60\\[1ex]\hline
\end{tabular}
\label{table:nonlin}
\end{table} \clearpage
\begin{table}[ht]
\caption{Resultados de la ejecución de la metaheurística ACO, utilizando instancias de Dethloff con la configuración -n 42.0 -alpha 1.0 -beta 3.0 -q .6 -ro 0.015}
\centering
\small
\begin{tabular}{c c c c c c c}
\hline\hline
Instancia & Costo mínimo & Tiempo(seg.) & Costo promedio & Tiempo promedio(seg.) & Costo ACO & \%Gap \\ [0.5ex]
\hline
SCA3-0 & \bf{\underline{636.06}} & 28.16 & 
636.06 & 29.76 & 636.10 & 
-0.01\\SCA3-1 & \bf{\underline{697.84}} & 31.07 & 
697.84 & 31.97 & 700.10 & 
-0.32\\SCA3-2 & 659.34 & 26.78 & 
659.34 & 28.60 & \bf{659.30} & 
0.01\\SCA3-3 & 680.04 & 29.98 & 
680.04 & 29.48 & \bf{680.00} & 
0.01\\SCA3-4 & \bf{690.50} & 31.91 & 
690.50 & 31.42 & 690.50 & 0.00\\
SCA3-5 & \bf{\underline{659.90}} & 30.98 & 
662.04 & 32.91 & 671.10 & 
-1.67\\SCA3-6 & 652.94 & 29.54 & 
652.94 & 29.61 & \bf{651.10} & 
0.28\\SCA3-7 & 666.15 & 24.16 & 
666.15 & 23.35 & \bf{666.10} & 
0.01\\SCA3-8 & \bf{\underline{719.47}} & 28.92 & 
719.47 & 27.70 & 719.50 & 
-0.00\\SCA3-9 & \bf{681.00} & 23.39 & 
681.00 & 23.63 & 681.00 & 0.00\\
SCA8-0 & \bf{\underline{961.50}} & 33.44 & 
967.28 & 32.21 & 961.60 & 
-0.01\\SCA8-1 & \bf{\underline{1052.71}} & 26.13 & 
1055.85 & 25.43 & 1063.00 & 
-0.97\\SCA8-2 & 1043.72 & 23.64 & 
1048.87 & 23.04 & \bf{1040.60} & 
0.30\\SCA8-3 & 1002.89 & 33.14 & 
1008.35 & 32.52 & \bf{985.90} & 
1.72\\SCA8-4 & \bf{\underline{1065.49}} & 36.38 & 
1066.93 & 32.72 & 1071.00 & 
-0.51\\SCA8-5 & \bf{\underline{1034.74}} & 31.81 & 
1041.87 & 33.12 & 1054.30 & 
-1.86\\SCA8-6 & \bf{\underline{972.48}} & 33.37 & 
974.79 & 33.50 & 972.50 & 
-0.00\\SCA8-7 & 1066.65 & 33.04 & 
1067.04 & 34.34 & \bf{1059.70} & 
0.66\\SCA8-8 & \bf{\underline{1071.18}} & 33.17 & 
1071.18 & 31.42 & 1082.70 & 
-1.06\\SCA8-9 & \bf{\underline{1067.42}} & 25.80 & 
1067.42 & 25.64 & 1081.40 & 
-1.29\\CON3-0 & 616.52 & 32.54 & 
619.07 & 32.28 & \bf{616.50} & 
0.00\\CON3-1 & \bf{\underline{554.47}} & 31.35 & 
554.86 & 31.80 & 555.60 & 
-0.20\\CON3-2 & \bf{\underline{519.61}} & 28.75 & 
520.05 & 28.48 & 521.40 & 
-0.34\\CON3-3 & \bf{\underline{591.19}} & 31.54 & 
591.19 & 32.41 & 591.20 & 
-0.00\\CON3-4 & \bf{\underline{588.79}} & 26.01 & 
588.79 & 26.80 & 589.30 & 
-0.09\\CON3-5 & \bf{563.70} & 28.94 & 
564.00 & 29.88 & 563.70 & 0.00\\
CON3-6 & \bf{\underline{499.05}} & 33.95 & 
500.50 & 34.52 & 499.20 & 
-0.03\\CON3-7 & \bf{\underline{576.84}} & 27.92 & 
577.24 & 26.93 & 577.50 & 
-0.11\\CON3-8 & \bf{\underline{523.05}} & 25.79 & 
523.07 & 25.84 & 523.10 & 
-0.01\\CON3-9 & 578.98 & 28.41 & 
582.60 & 27.22 & \bf{578.20} & 
0.13\\CON8-0 & 865.86 & 30.66 & 
867.79 & 31.66 & \bf{858.90} & 
0.81\\CON8-1 & \bf{\underline{740.85}} & 27.70 & 
740.85 & 28.93 & 740.90 & 
-0.01\\CON8-2 & \bf{\underline{712.89}} & 39.83 & 
713.79 & 41.93 & 714.30 & 
-0.20\\CON8-3 & \bf{\underline{811.07}} & 30.86 & 
813.10 & 30.19 & 812.30 & 
-0.15\\CON8-4 & 773.64 & 29.30 & 
779.36 & 30.38 & \bf{770.10} & 
0.46\\CON8-5 & \bf{\underline{754.95}} & 28.09 & 
758.82 & 27.96 & 766.60 & 
-1.52\\CON8-6 & \bf{\underline{688.00}} & 34.89 & 
693.19 & 34.81 & 697.20 & 
-1.32\\CON8-7 & \bf{\underline{814.50}} & 25.92 & 
814.73 & 26.90 & 814.80 & 
-0.04\\CON8-8 & \bf{\underline{771.26}} & 34.53 & 
780.58 & 35.44 & 771.30 & 
-0.01\\CON8-9 & \bf{\underline{809.00}} & 34.19 & 
811.70 & 34.81 & 815.10 & 
-0.75\\[1ex]\hline
\end{tabular}
\label{table:nonlin}
\end{table} \clearpage
\begin{table}[ht]
\caption{Resultados de la ejecución de la metaheurística ACO, utilizando instancias de Dethloff con la configuración -n 42.0 -alpha 1.0 -beta 3.0 -q .7 -ro 0.015}
\centering
\small
\begin{tabular}{c c c c c c c}
\hline\hline
Instancia & Costo mínimo & Tiempo(seg.) & Costo promedio & Tiempo promedio(seg.) & Costo ACO & \%Gap \\ [0.5ex]
\hline
SCA3-0 & \bf{\underline{636.06}} & 28.06 & 
636.06 & 28.68 & 636.10 & 
-0.01\\SCA3-1 & \bf{\underline{697.84}} & 33.10 & 
697.84 & 32.25 & 700.10 & 
-0.32\\SCA3-2 & 659.34 & 26.98 & 
660.55 & 27.19 & \bf{659.30} & 
0.01\\SCA3-3 & 680.04 & 30.03 & 
680.04 & 29.73 & \bf{680.00} & 
0.01\\SCA3-4 & \bf{690.50} & 32.47 & 
690.50 & 30.48 & 690.50 & 0.00\\
SCA3-5 & \bf{\underline{659.90}} & 28.37 & 
662.61 & 31.19 & 671.10 & 
-1.67\\SCA3-6 & \bf{\underline{651.09}} & 27.64 & 
652.01 & 28.73 & 651.10 & 
-0.00\\SCA3-7 & 666.15 & 26.08 & 
666.15 & 25.12 & \bf{666.10} & 
0.01\\SCA3-8 & \bf{\underline{719.47}} & 28.36 & 
719.54 & 27.40 & 719.50 & 
-0.00\\SCA3-9 & \bf{681.00} & 22.05 & 
681.00 & 23.09 & 681.00 & 0.00\\
SCA8-0 & 968.79 & 30.70 & 
971.34 & 33.15 & \bf{961.60} & 
0.75\\SCA8-1 & \bf{\underline{1053.09}} & 24.77 & 
1056.90 & 24.34 & 1063.00 & 
-0.93\\SCA8-2 & 1046.29 & 22.91 & 
1049.30 & 22.76 & \bf{1040.60} & 
0.55\\SCA8-3 & 991.84 & 30.18 & 
1004.59 & 32.96 & \bf{985.90} & 
0.60\\SCA8-4 & \bf{\underline{1065.49}} & 31.71 & 
1067.12 & 31.07 & 1071.00 & 
-0.51\\SCA8-5 & \bf{\underline{1034.74}} & 33.06 & 
1040.95 & 33.80 & 1054.30 & 
-1.86\\SCA8-6 & \bf{\underline{972.48}} & 36.94 & 
972.48 & 34.41 & 972.50 & 
-0.00\\SCA8-7 & 1060.77 & 34.28 & 
1067.03 & 33.94 & \bf{1059.70} & 
0.10\\SCA8-8 & \bf{\underline{1071.18}} & 30.58 & 
1071.18 & 31.76 & 1082.70 & 
-1.06\\SCA8-9 & \bf{\underline{1065.60}} & 25.22 & 
1066.97 & 26.03 & 1081.40 & 
-1.46\\CON3-0 & 616.52 & 33.13 & 
618.91 & 34.30 & \bf{616.50} & 
0.00\\CON3-1 & \bf{\underline{554.47}} & 30.15 & 
554.92 & 30.05 & 555.60 & 
-0.20\\CON3-2 & \bf{\underline{519.11}} & 30.70 & 
519.93 & 28.17 & 521.40 & 
-0.44\\CON3-3 & \bf{\underline{591.19}} & 31.99 & 
591.19 & 34.69 & 591.20 & 
-0.00\\CON3-4 & \bf{\underline{588.79}} & 26.06 & 
588.79 & 27.32 & 589.30 & 
-0.09\\CON3-5 & \bf{563.70} & 29.79 & 
564.59 & 30.55 & 563.70 & 0.00\\
CON3-6 & 500.80 & 35.38 & 
501.24 & 35.75 & \bf{499.20} & 
0.32\\CON3-7 & \bf{\underline{576.48}} & 26.08 & 
577.44 & 26.55 & 577.50 & 
-0.18\\CON3-8 & 523.14 & 23.42 & 
523.72 & 24.60 & \bf{523.10} & 
0.01\\CON3-9 & 580.78 & 25.40 & 
583.18 & 27.31 & \bf{578.20} & 
0.45\\CON8-0 & 866.22 & 31.03 & 
870.18 & 30.44 & \bf{858.90} & 
0.85\\CON8-1 & \bf{\underline{740.85}} & 30.58 & 
740.85 & 30.37 & 740.90 & 
-0.01\\CON8-2 & \bf{\underline{712.89}} & 39.19 & 
713.42 & 39.77 & 714.30 & 
-0.20\\CON8-3 & \bf{\underline{811.07}} & 30.06 & 
812.96 & 29.96 & 812.30 & 
-0.15\\CON8-4 & 772.76 & 30.58 & 
777.90 & 31.52 & \bf{770.10} & 
0.35\\CON8-5 & \bf{\underline{758.12}} & 30.81 & 
759.44 & 29.65 & 766.60 & 
-1.11\\CON8-6 & \bf{\underline{686.39}} & 32.53 & 
690.18 & 33.55 & 697.20 & 
-1.55\\CON8-7 & \bf{\underline{814.77}} & 25.20 & 
815.51 & 24.91 & 814.80 & 
-0.00\\CON8-8 & 781.54 & 35.15 & 
782.57 & 35.50 & \bf{771.30} & 
1.33\\CON8-9 & \bf{\underline{812.60}} & 37.32 & 
814.38 & 34.48 & 815.10 & 
-0.31\\[1ex]\hline
\end{tabular}
\label{table:nonlin}
\end{table} \clearpage
\begin{table}[ht]
\caption{Resultados de la ejecución de la metaheurística ACO, utilizando instancias de Dethloff con la configuración -n 42.0 -alpha 1.0 -beta 3.0 -q .8 -ro 0.015}
\centering
\small
\begin{tabular}{c c c c c c c}
\hline\hline
Instancia & Costo mínimo & Tiempo(seg.) & Costo promedio & Tiempo promedio(seg.) & Costo ACO & \%Gap \\ [0.5ex]
\hline
SCA3-0 & \bf{\underline{636.06}} & 28.81 & 
636.06 & 28.06 & 636.10 & 
-0.01\\SCA3-1 & \bf{\underline{697.84}} & 29.87 & 
697.84 & 32.02 & 700.10 & 
-0.32\\SCA3-2 & 659.34 & 31.10 & 
660.55 & 28.11 & \bf{659.30} & 
0.01\\SCA3-3 & 680.04 & 29.46 & 
680.04 & 30.53 & \bf{680.00} & 
0.01\\SCA3-4 & \bf{690.50} & 30.19 & 
690.50 & 30.91 & 690.50 & 0.00\\
SCA3-5 & \bf{\underline{665.04}} & 34.86 & 
665.19 & 32.19 & 671.10 & 
-0.90\\SCA3-6 & 652.47 & 30.12 & 
652.82 & 29.26 & \bf{651.10} & 
0.21\\SCA3-7 & 666.15 & 24.13 & 
666.15 & 24.45 & \bf{666.10} & 
0.01\\SCA3-8 & \bf{\underline{719.47}} & 25.51 & 
719.47 & 26.28 & 719.50 & 
-0.00\\SCA3-9 & \bf{681.00} & 22.58 & 
681.00 & 22.16 & 681.00 & 0.00\\
SCA8-0 & 968.79 & 33.07 & 
979.73 & 33.22 & \bf{961.60} & 
0.75\\SCA8-1 & \bf{\underline{1052.71}} & 22.96 & 
1059.20 & 23.89 & 1063.00 & 
-0.97\\SCA8-2 & 1046.15 & 23.48 & 
1047.95 & 22.45 & \bf{1040.60} & 
0.53\\SCA8-3 & 1012.50 & 30.76 & 
1015.44 & 31.12 & \bf{985.90} & 
2.70\\SCA8-4 & \bf{\underline{1065.49}} & 34.41 & 
1068.65 & 33.06 & 1071.00 & 
-0.51\\SCA8-5 & \bf{\underline{1034.74}} & 33.13 & 
1042.91 & 34.03 & 1054.30 & 
-1.86\\SCA8-6 & \bf{\underline{972.48}} & 31.54 & 
976.52 & 36.64 & 972.50 & 
-0.00\\SCA8-7 & 1067.20 & 35.65 & 
1067.20 & 36.27 & \bf{1059.70} & 
0.71\\SCA8-8 & \bf{\underline{1071.18}} & 32.46 & 
1073.91 & 30.51 & 1082.70 & 
-1.06\\SCA8-9 & \bf{\underline{1067.42}} & 24.54 & 
1067.42 & 25.39 & 1081.40 & 
-1.29\\CON3-0 & 620.76 & 33.77 & 
621.02 & 33.75 & \bf{616.50} & 
0.69\\CON3-1 & \bf{\underline{554.47}} & 30.35 & 
554.47 & 30.44 & 555.60 & 
-0.20\\CON3-2 & \bf{\underline{519.11}} & 25.36 & 
520.37 & 27.62 & 521.40 & 
-0.44\\CON3-3 & \bf{\underline{591.19}} & 30.94 & 
591.19 & 31.68 & 591.20 & 
-0.00\\CON3-4 & \bf{\underline{588.79}} & 28.32 & 
588.79 & 27.25 & 589.30 & 
-0.09\\CON3-5 & \bf{563.70} & 29.90 & 
564.59 & 27.95 & 563.70 & 0.00\\
CON3-6 & 500.37 & 35.16 & 
501.12 & 35.44 & \bf{499.20} & 
0.23\\CON3-7 & 577.54 & 27.79 & 
577.76 & 26.21 & \bf{577.50} & 
0.01\\CON3-8 & \bf{\underline{523.05}} & 24.66 & 
523.23 & 25.80 & 523.10 & 
-0.01\\CON3-9 & 578.25 & 25.55 & 
583.47 & 28.96 & \bf{578.20} & 
0.01\\CON8-0 & 869.43 & 32.17 & 
870.76 & 31.54 & \bf{858.90} & 
1.23\\CON8-1 & \bf{\underline{740.85}} & 40.95 & 
741.21 & 31.12 & 740.90 & 
-0.01\\CON8-2 & \bf{\underline{712.89}} & 40.80 & 
713.42 & 40.66 & 714.30 & 
-0.20\\CON8-3 & \bf{\underline{811.23}} & 28.20 & 
813.41 & 29.00 & 812.30 & 
-0.13\\CON8-4 & 776.72 & 30.16 & 
784.27 & 31.52 & \bf{770.10} & 
0.86\\CON8-5 & \bf{\underline{760.03}} & 29.17 & 
762.80 & 29.32 & 766.60 & 
-0.86\\CON8-6 & \bf{\underline{687.65}} & 34.75 & 
692.48 & 35.32 & 697.20 & 
-1.37\\CON8-7 & \bf{\underline{814.79}} & 25.68 & 
814.83 & 25.72 & 814.80 & 
-0.00\\CON8-8 & 782.93 & 33.83 & 
784.96 & 35.00 & \bf{771.30} & 
1.51\\CON8-9 & \bf{\underline{812.60}} & 31.93 & 
813.46 & 38.43 & 815.10 & 
-0.31\\[1ex]\hline
\end{tabular}
\label{table:nonlin}
\end{table} \clearpage
\begin{table}[ht]
\caption{Resultados de la ejecución de la metaheurística ACO, utilizando instancias de Dethloff con la configuración -n 42.0 -alpha 1.0 -beta 3.0 -q .9 -ro 0.015}
\centering
\small
\begin{tabular}{c c c c c c c}
\hline\hline
Instancia & Costo mínimo & Tiempo(seg.) & Costo promedio & Tiempo promedio(seg.) & Costo ACO & \%Gap \\ [0.5ex]
\hline
SCA3-0 & \bf{\underline{636.06}} & 28.89 & 
636.06 & 29.07 & 636.10 & 
-0.01\\SCA3-1 & \bf{\underline{697.84}} & 31.22 & 
697.84 & 31.80 & 700.10 & 
-0.32\\SCA3-2 & 659.34 & 27.07 & 
660.55 & 27.85 & \bf{659.30} & 
0.01\\SCA3-3 & 680.04 & 30.20 & 
680.04 & 29.27 & \bf{680.00} & 
0.01\\SCA3-4 & \bf{690.50} & 29.19 & 
690.50 & 30.71 & 690.50 & 0.00\\
SCA3-5 & \bf{\underline{659.90}} & 34.92 & 
664.05 & 32.64 & 671.10 & 
-1.67\\SCA3-6 & 652.94 & 27.60 & 
652.94 & 29.24 & \bf{651.10} & 
0.28\\SCA3-7 & 666.15 & 22.38 & 
666.15 & 23.69 & \bf{666.10} & 
0.01\\SCA3-8 & \bf{\underline{719.47}} & 24.16 & 
719.47 & 25.04 & 719.50 & 
-0.00\\SCA3-9 & \bf{681.00} & 22.93 & 
681.00 & 22.77 & 681.00 & 0.00\\
SCA8-0 & 965.26 & 30.33 & 
975.39 & 31.57 & \bf{961.60} & 
0.38\\SCA8-1 & \bf{\underline{1052.71}} & 24.86 & 
1061.05 & 24.77 & 1063.00 & 
-0.97\\SCA8-2 & 1046.29 & 21.75 & 
1050.99 & 21.66 & \bf{1040.60} & 
0.55\\SCA8-3 & 1002.86 & 30.18 & 
1011.73 & 31.20 & \bf{985.90} & 
1.72\\SCA8-4 & 1077.80 & 36.22 & 
1080.50 & 33.16 & \bf{1071.00} & 
0.63\\SCA8-5 & \bf{\underline{1047.55}} & 34.24 & 
1052.84 & 34.56 & 1054.30 & 
-0.64\\SCA8-6 & \bf{\underline{972.48}} & 34.79 & 
974.75 & 35.59 & 972.50 & 
-0.00\\SCA8-7 & 1067.20 & 36.41 & 
1069.37 & 34.89 & \bf{1059.70} & 
0.71\\SCA8-8 & \bf{\underline{1071.18}} & 31.00 & 
1071.18 & 31.82 & 1082.70 & 
-1.06\\SCA8-9 & \bf{\underline{1067.42}} & 25.95 & 
1067.42 & 26.64 & 1081.40 & 
-1.29\\CON3-0 & 616.52 & 34.66 & 
619.69 & 34.76 & \bf{616.50} & 
0.00\\CON3-1 & \bf{\underline{554.47}} & 28.83 & 
555.53 & 30.30 & 555.60 & 
-0.20\\CON3-2 & \bf{\underline{519.11}} & 27.04 & 
521.27 & 27.03 & 521.40 & 
-0.44\\CON3-3 & \bf{\underline{591.19}} & 33.76 & 
591.20 & 32.72 & 591.20 & 
-0.00\\CON3-4 & \bf{\underline{588.79}} & 26.13 & 
588.79 & 27.17 & 589.30 & 
-0.09\\CON3-5 & \bf{563.70} & 27.31 & 
565.35 & 28.87 & 563.70 & 0.00\\
CON3-6 & 500.80 & 37.48 & 
500.86 & 36.17 & \bf{499.20} & 
0.32\\CON3-7 & 578.22 & 25.62 & 
578.32 & 25.06 & \bf{577.50} & 
0.12\\CON3-8 & 523.14 & 25.43 & 
523.70 & 24.78 & \bf{523.10} & 
0.01\\CON3-9 & 588.48 & 28.32 & 
588.61 & 27.49 & \bf{578.20} & 
1.78\\CON8-0 & 871.80 & 28.30 & 
877.53 & 30.28 & \bf{858.90} & 
1.50\\CON8-1 & \bf{\underline{740.85}} & 30.14 & 
740.85 & 28.10 & 740.90 & 
-0.01\\CON8-2 & \bf{\underline{713.44}} & 40.70 & 
713.82 & 40.70 & 714.30 & 
-0.12\\CON8-3 & \bf{\underline{811.07}} & 31.08 & 
814.67 & 30.55 & 812.30 & 
-0.15\\CON8-4 & 785.73 & 30.97 & 
788.92 & 31.59 & \bf{770.10} & 
2.03\\CON8-5 & \bf{\underline{754.95}} & 30.07 & 
760.50 & 29.77 & 766.60 & 
-1.52\\CON8-6 & \bf{\underline{683.83}} & 36.09 & 
691.52 & 35.60 & 697.20 & 
-1.92\\CON8-7 & \bf{\underline{814.79}} & 25.61 & 
814.81 & 26.66 & 814.80 & 
-0.00\\CON8-8 & 782.86 & 32.10 & 
787.26 & 33.91 & \bf{771.30} & 
1.50\\CON8-9 & \bf{\underline{813.16}} & 32.23 & 
815.07 & 32.69 & 815.10 & 
-0.24\\[1ex]\hline
\end{tabular}
\label{table:nonlin}
\end{table} \clearpage
\begin{table}[ht]
\caption{Resultados de la ejecución de la metaheurística ACO, utilizando instancias de Dethloff con la configuración -n 3.0 -alpha 1.0 -beta 3.0 -q 0.1 -ro 0.015}
\centering
\small
\begin{tabular}{c c c c c c c}
\hline\hline
Instancia & Costo mínimo & Tiempo(seg.) & Costo promedio & Tiempo promedio(seg.) & Costo ACO & \%Gap \\ [0.5ex]
\hline
SCA3-0 & \bf{\underline{636.06}} & 2.15 & 
636.13 & 2.13 & 636.10 & 
-0.01\\SCA3-1 & \bf{\underline{697.84}} & 2.35 & 
697.84 & 2.34 & 700.10 & 
-0.32\\SCA3-2 & 659.34 & 2.24 & 
662.99 & 2.23 & \bf{659.30} & 
0.01\\SCA3-3 & 680.04 & 2.02 & 
680.32 & 2.13 & \bf{680.00} & 
0.01\\SCA3-4 & \bf{690.50} & 2.20 & 
690.50 & 2.33 & 690.50 & 0.00\\
SCA3-5 & \bf{\underline{659.90}} & 2.28 & 
662.61 & 2.26 & 671.10 & 
-1.67\\SCA3-6 & \bf{\underline{651.09}} & 2.24 & 
652.79 & 2.22 & 651.10 & 
-0.00\\SCA3-7 & 666.15 & 1.93 & 
667.80 & 1.93 & \bf{666.10} & 
0.01\\SCA3-8 & \bf{\underline{719.47}} & 2.22 & 
720.26 & 2.25 & 719.50 & 
-0.00\\SCA3-9 & \bf{681.00} & 2.03 & 
681.81 & 1.91 & 681.00 & 0.00\\
SCA8-0 & \bf{\underline{961.50}} & 2.24 & 
977.14 & 2.31 & 961.60 & 
-0.01\\SCA8-1 & \bf{\underline{1054.08}} & 2.16 & 
1067.20 & 2.10 & 1063.00 & 
-0.84\\SCA8-2 & 1050.35 & 1.90 & 
1051.86 & 1.95 & \bf{1040.60} & 
0.94\\SCA8-3 & 1000.72 & 2.11 & 
1007.16 & 2.21 & \bf{985.90} & 
1.50\\SCA8-4 & \bf{\underline{1067.55}} & 2.34 & 
1068.66 & 2.28 & 1071.00 & 
-0.32\\SCA8-5 & \bf{\underline{1052.18}} & 2.64 & 
1054.87 & 2.64 & 1054.30 & 
-0.20\\SCA8-6 & \bf{\underline{972.48}} & 2.60 & 
978.37 & 2.54 & 972.50 & 
-0.00\\SCA8-7 & 1066.65 & 2.56 & 
1070.78 & 2.50 & \bf{1059.70} & 
0.66\\SCA8-8 & \bf{\underline{1071.18}} & 2.55 & 
1081.24 & 2.50 & 1082.70 & 
-1.06\\SCA8-9 & \bf{\underline{1067.42}} & 2.10 & 
1068.15 & 2.13 & 1081.40 & 
-1.29\\CON3-0 & 620.76 & 2.38 & 
621.92 & 2.40 & \bf{616.50} & 
0.69\\CON3-1 & \bf{\underline{554.47}} & 2.19 & 
558.37 & 2.33 & 555.60 & 
-0.20\\CON3-2 & \bf{\underline{521.38}} & 2.25 & 
521.38 & 2.29 & 521.40 & 
-0.00\\CON3-3 & \bf{591.20} & 2.31 & 
591.24 & 2.35 & 591.20 & 0.00\\
CON3-4 & \bf{\underline{588.79}} & 2.15 & 
590.11 & 2.21 & 589.30 & 
-0.09\\CON3-5 & \bf{563.70} & 2.30 & 
567.50 & 2.18 & 563.70 & 0.00\\
CON3-6 & 500.80 & 2.64 & 
502.42 & 2.60 & \bf{499.20} & 
0.32\\CON3-7 & 578.22 & 2.06 & 
581.57 & 1.99 & \bf{577.50} & 
0.12\\CON3-8 & 524.38 & 2.01 & 
526.31 & 2.06 & \bf{523.10} & 
0.24\\CON3-9 & 578.25 & 2.25 & 
584.43 & 2.14 & \bf{578.20} & 
0.01\\CON8-0 & 871.77 & 2.38 & 
875.49 & 2.40 & \bf{858.90} & 
1.50\\CON8-1 & \bf{\underline{740.85}} & 2.40 & 
748.45 & 2.34 & 740.90 & 
-0.01\\CON8-2 & \bf{\underline{713.68}} & 3.16 & 
713.85 & 2.80 & 714.30 & 
-0.09\\CON8-3 & \bf{\underline{811.07}} & 2.36 & 
814.29 & 2.37 & 812.30 & 
-0.15\\CON8-4 & 776.34 & 2.27 & 
783.53 & 2.27 & \bf{770.10} & 
0.81\\CON8-5 & \bf{\underline{758.84}} & 2.29 & 
764.03 & 2.35 & 766.60 & 
-1.01\\CON8-6 & \bf{\underline{691.42}} & 2.69 & 
694.53 & 2.69 & 697.20 & 
-0.83\\CON8-7 & \bf{\underline{814.79}} & 1.99 & 
814.84 & 2.15 & 814.80 & 
-0.00\\CON8-8 & 778.40 & 2.62 & 
781.59 & 2.55 & \bf{771.30} & 
0.92\\CON8-9 & \bf{\underline{810.61}} & 2.68 & 
814.49 & 2.67 & 815.10 & 
-0.55\\[1ex]\hline
\end{tabular}
\label{table:nonlin}
\end{table} \clearpage
\begin{table}[ht]
\caption{Resultados de la ejecución de la metaheurística ACO, utilizando instancias de Dethloff con la configuración -n 3.0 -alpha 1.0 -beta 3.0 -q .2 -ro 0.015}
\centering
\small
\begin{tabular}{c c c c c c c}
\hline\hline
Instancia & Costo mínimo & Tiempo(seg.) & Costo promedio & Tiempo promedio(seg.) & Costo ACO & \%Gap \\ [0.5ex]
\hline
SCA3-0 & \bf{\underline{636.06}} & 2.00 & 
637.25 & 2.14 & 636.10 & 
-0.01\\SCA3-1 & \bf{\underline{697.84}} & 2.34 & 
697.84 & 2.44 & 700.10 & 
-0.32\\SCA3-2 & 659.34 & 2.08 & 
661.00 & 2.18 & \bf{659.30} & 
0.01\\SCA3-3 & 680.04 & 2.18 & 
680.04 & 2.11 & \bf{680.00} & 
0.01\\SCA3-4 & \bf{690.50} & 2.32 & 
690.50 & 2.21 & 690.50 & 0.00\\
SCA3-5 & \bf{\underline{661.07}} & 2.38 & 
663.31 & 2.32 & 671.10 & 
-1.49\\SCA3-6 & \bf{\underline{651.09}} & 2.22 & 
652.70 & 2.14 & 651.10 & 
-0.00\\SCA3-7 & 666.15 & 2.05 & 
666.15 & 1.96 & \bf{666.10} & 
0.01\\SCA3-8 & \bf{\underline{719.47}} & 2.22 & 
719.62 & 2.23 & 719.50 & 
-0.00\\SCA3-9 & \bf{681.00} & 2.00 & 
681.17 & 1.95 & 681.00 & 0.00\\
SCA8-0 & 983.71 & 2.24 & 
988.38 & 2.32 & \bf{961.60} & 
2.30\\SCA8-1 & \bf{\underline{1055.60}} & 2.10 & 
1059.97 & 2.07 & 1063.00 & 
-0.70\\SCA8-2 & 1050.37 & 1.93 & 
1051.62 & 1.93 & \bf{1040.60} & 
0.94\\SCA8-3 & 997.17 & 2.43 & 
1008.25 & 2.31 & \bf{985.90} & 
1.14\\SCA8-4 & \bf{\underline{1067.55}} & 2.21 & 
1068.16 & 2.32 & 1071.00 & 
-0.32\\SCA8-5 & \bf{\underline{1043.49}} & 2.79 & 
1050.80 & 2.55 & 1054.30 & 
-1.03\\SCA8-6 & \bf{\underline{971.82}} & 2.53 & 
976.33 & 2.50 & 972.50 & 
-0.07\\SCA8-7 & 1066.65 & 2.42 & 
1066.75 & 2.42 & \bf{1059.70} & 
0.66\\SCA8-8 & \bf{\underline{1071.18}} & 2.66 & 
1077.60 & 2.55 & 1082.70 & 
-1.06\\SCA8-9 & \bf{\underline{1067.42}} & 1.92 & 
1067.99 & 2.09 & 1081.40 & 
-1.29\\CON3-0 & 620.76 & 2.51 & 
622.74 & 2.46 & \bf{616.50} & 
0.69\\CON3-1 & 557.38 & 2.14 & 
559.46 & 2.34 & \bf{555.60} & 
0.32\\CON3-2 & \bf{\underline{519.11}} & 2.18 & 
521.27 & 2.22 & 521.40 & 
-0.44\\CON3-3 & \bf{\underline{591.19}} & 2.33 & 
591.92 & 2.39 & 591.20 & 
-0.00\\CON3-4 & \bf{\underline{588.79}} & 2.22 & 
589.58 & 2.16 & 589.30 & 
-0.09\\CON3-5 & 564.88 & 2.21 & 
567.94 & 2.22 & \bf{563.70} & 
0.21\\CON3-6 & 500.80 & 2.68 & 
502.81 & 2.62 & \bf{499.20} & 
0.32\\CON3-7 & 577.68 & 2.10 & 
579.93 & 2.12 & \bf{577.50} & 
0.03\\CON3-8 & \bf{\underline{523.05}} & 2.37 & 
523.54 & 2.17 & 523.10 & 
-0.01\\CON3-9 & 578.98 & 2.20 & 
582.76 & 2.27 & \bf{578.20} & 
0.13\\CON8-0 & 870.28 & 2.43 & 
881.68 & 2.35 & \bf{858.90} & 
1.32\\CON8-1 & 742.29 & 2.37 & 
745.75 & 2.35 & \bf{740.90} & 
0.19\\CON8-2 & \bf{\underline{713.05}} & 2.66 & 
715.75 & 2.82 & 714.30 & 
-0.17\\CON8-3 & \bf{\underline{811.07}} & 2.49 & 
818.81 & 2.49 & 812.30 & 
-0.15\\CON8-4 & 776.72 & 2.16 & 
784.31 & 2.23 & \bf{770.10} & 
0.86\\CON8-5 & \bf{\underline{760.91}} & 2.34 & 
762.81 & 2.35 & 766.60 & 
-0.74\\CON8-6 & \bf{\underline{688.87}} & 2.62 & 
693.72 & 2.63 & 697.20 & 
-1.19\\CON8-7 & \bf{\underline{814.79}} & 2.10 & 
814.81 & 2.15 & 814.80 & 
-0.00\\CON8-8 & 782.34 & 2.78 & 
784.82 & 2.68 & \bf{771.30} & 
1.43\\CON8-9 & \bf{\underline{813.96}} & 2.53 & 
815.09 & 2.66 & 815.10 & 
-0.14\\[1ex]\hline
\end{tabular}
\label{table:nonlin}
\end{table} \clearpage
\begin{table}[ht]
\caption{Resultados de la ejecución de la metaheurística ACO, utilizando instancias de Dethloff con la configuración -n 3.0 -alpha 1.0 -beta 3.0 -q .3 -ro 0.015}
\centering
\small
\begin{tabular}{c c c c c c c}
\hline\hline
Instancia & Costo mínimo & Tiempo(seg.) & Costo promedio & Tiempo promedio(seg.) & Costo ACO & \%Gap \\ [0.5ex]
\hline
SCA3-0 & \bf{\underline{636.06}} & 2.12 & 
636.20 & 2.18 & 636.10 & 
-0.01\\SCA3-1 & \bf{\underline{697.84}} & 2.34 & 
697.84 & 2.42 & 700.10 & 
-0.32\\SCA3-2 & 659.34 & 2.00 & 
659.79 & 2.07 & \bf{659.30} & 
0.01\\SCA3-3 & 680.04 & 2.15 & 
680.46 & 2.10 & \bf{680.00} & 
0.01\\SCA3-4 & \bf{690.50} & 2.32 & 
690.50 & 2.27 & 690.50 & 0.00\\
SCA3-5 & \bf{\underline{659.90}} & 2.18 & 
663.17 & 2.20 & 671.10 & 
-1.67\\SCA3-6 & \bf{\underline{651.09}} & 2.19 & 
652.01 & 2.31 & 651.10 & 
-0.00\\SCA3-7 & 666.15 & 1.88 & 
667.20 & 1.91 & \bf{666.10} & 
0.01\\SCA3-8 & \bf{\underline{719.47}} & 2.21 & 
721.88 & 2.33 & 719.50 & 
-0.00\\SCA3-9 & \bf{681.00} & 2.07 & 
681.64 & 1.86 & 681.00 & 0.00\\
SCA8-0 & \bf{\underline{961.50}} & 2.34 & 
974.22 & 2.35 & 961.60 & 
-0.01\\SCA8-1 & \bf{\underline{1058.98}} & 2.00 & 
1065.31 & 2.08 & 1063.00 & 
-0.38\\SCA8-2 & 1047.63 & 1.69 & 
1049.89 & 1.86 & \bf{1040.60} & 
0.68\\SCA8-3 & 991.84 & 2.08 & 
1007.50 & 2.18 & \bf{985.90} & 
0.60\\SCA8-4 & \bf{\underline{1067.28}} & 2.30 & 
1068.84 & 2.32 & 1071.00 & 
-0.35\\SCA8-5 & \bf{\underline{1051.47}} & 2.57 & 
1055.03 & 2.56 & 1054.30 & 
-0.27\\SCA8-6 & 977.83 & 2.26 & 
980.03 & 2.34 & \bf{972.50} & 
0.55\\SCA8-7 & 1067.20 & 2.48 & 
1073.25 & 2.48 & \bf{1059.70} & 
0.71\\SCA8-8 & \bf{\underline{1071.18}} & 2.53 & 
1079.95 & 2.49 & 1082.70 & 
-1.06\\SCA8-9 & \bf{\underline{1067.42}} & 2.47 & 
1067.42 & 2.10 & 1081.40 & 
-1.29\\CON3-0 & 619.09 & 2.37 & 
622.42 & 2.37 & \bf{616.50} & 
0.42\\CON3-1 & \bf{\underline{554.47}} & 2.09 & 
556.53 & 2.67 & 555.60 & 
-0.20\\CON3-2 & \bf{\underline{521.38}} & 2.22 & 
522.03 & 2.18 & 521.40 & 
-0.00\\CON3-3 & \bf{\underline{591.19}} & 2.40 & 
591.20 & 2.46 & 591.20 & 
-0.00\\CON3-4 & \bf{\underline{588.79}} & 2.06 & 
589.87 & 2.01 & 589.30 & 
-0.09\\CON3-5 & 564.88 & 2.28 & 
566.78 & 2.38 & \bf{563.70} & 
0.21\\CON3-6 & 502.88 & 2.50 & 
503.83 & 2.45 & \bf{499.20} & 
0.74\\CON3-7 & 578.22 & 1.92 & 
580.22 & 1.87 & \bf{577.50} & 
0.12\\CON3-8 & 523.14 & 2.15 & 
525.98 & 2.10 & \bf{523.10} & 
0.01\\CON3-9 & 578.98 & 2.27 & 
586.16 & 2.17 & \bf{578.20} & 
0.13\\CON8-0 & 867.28 & 2.35 & 
874.08 & 2.31 & \bf{858.90} & 
0.98\\CON8-1 & \bf{\underline{740.85}} & 2.31 & 
743.89 & 2.33 & 740.90 & 
-0.01\\CON8-2 & \bf{\underline{714.06}} & 2.75 & 
716.35 & 2.72 & 714.30 & 
-0.03\\CON8-3 & 816.45 & 2.40 & 
820.67 & 2.34 & \bf{812.30} & 
0.51\\CON8-4 & 780.48 & 2.10 & 
788.00 & 2.15 & \bf{770.10} & 
1.35\\CON8-5 & \bf{\underline{760.91}} & 2.28 & 
762.53 & 2.20 & 766.60 & 
-0.74\\CON8-6 & \bf{\underline{689.11}} & 3.21 & 
691.61 & 2.72 & 697.20 & 
-1.16\\CON8-7 & 814.86 & 2.24 & 
818.69 & 2.04 & \bf{814.80} & 
0.01\\CON8-8 & 791.31 & 2.54 & 
791.87 & 2.58 & \bf{771.30} & 
2.59\\CON8-9 & \bf{\underline{809.00}} & 2.72 & 
812.99 & 2.55 & 815.10 & 
-0.75\\[1ex]\hline
\end{tabular}
\label{table:nonlin}
\end{table} \clearpage
\begin{table}[ht]
\caption{Resultados de la ejecución de la metaheurística ACO, utilizando instancias de Dethloff con la configuración -n 3.0 -alpha 1.0 -beta 3.0 -q .4 -ro 0.015}
\centering
\small
\begin{tabular}{c c c c c c c}
\hline\hline
Instancia & Costo mínimo & Tiempo(seg.) & Costo promedio & Tiempo promedio(seg.) & Costo ACO & \%Gap \\ [0.5ex]
\hline
SCA3-0 & \bf{\underline{636.06}} & 2.05 & 
637.47 & 2.03 & 636.10 & 
-0.01\\SCA3-1 & \bf{\underline{697.84}} & 2.35 & 
698.76 & 2.29 & 700.10 & 
-0.32\\SCA3-2 & 659.34 & 2.09 & 
663.23 & 2.09 & \bf{659.30} & 
0.01\\SCA3-3 & 680.04 & 2.04 & 
680.64 & 2.13 & \bf{680.00} & 
0.01\\SCA3-4 & \bf{690.50} & 2.24 & 
690.50 & 2.26 & 690.50 & 0.00\\
SCA3-5 & \bf{\underline{665.04}} & 2.20 & 
666.47 & 2.22 & 671.10 & 
-0.90\\SCA3-6 & \bf{\underline{651.09}} & 2.05 & 
653.34 & 2.21 & 651.10 & 
-0.00\\SCA3-7 & 666.15 & 1.82 & 
667.31 & 1.84 & \bf{666.10} & 
0.01\\SCA3-8 & \bf{\underline{719.47}} & 2.11 & 
720.19 & 2.13 & 719.50 & 
-0.00\\SCA3-9 & \bf{681.00} & 1.74 & 
681.00 & 1.76 & 681.00 & 0.00\\
SCA8-0 & \bf{\underline{961.50}} & 2.51 & 
981.24 & 2.42 & 961.60 & 
-0.01\\SCA8-1 & \bf{\underline{1056.11}} & 1.84 & 
1061.02 & 1.89 & 1063.00 & 
-0.65\\SCA8-2 & 1050.17 & 1.84 & 
1050.53 & 1.84 & \bf{1040.60} & 
0.92\\SCA8-3 & 1000.96 & 2.20 & 
1010.46 & 2.15 & \bf{985.90} & 
1.53\\SCA8-4 & \bf{\underline{1065.49}} & 2.50 & 
1067.04 & 2.31 & 1071.00 & 
-0.51\\SCA8-5 & \bf{\underline{1038.59}} & 2.42 & 
1049.78 & 2.55 & 1054.30 & 
-1.49\\SCA8-6 & \bf{\underline{972.48}} & 2.37 & 
978.93 & 2.40 & 972.50 & 
-0.00\\SCA8-7 & 1067.20 & 2.48 & 
1069.38 & 2.59 & \bf{1059.70} & 
0.71\\SCA8-8 & \bf{\underline{1071.18}} & 2.47 & 
1081.82 & 2.45 & 1082.70 & 
-1.06\\SCA8-9 & \bf{\underline{1067.42}} & 2.12 & 
1068.36 & 2.02 & 1081.40 & 
-1.29\\CON3-0 & 616.52 & 2.28 & 
622.44 & 2.31 & \bf{616.50} & 
0.00\\CON3-1 & 557.21 & 2.34 & 
559.02 & 2.32 & \bf{555.60} & 
0.29\\CON3-2 & \bf{\underline{521.38}} & 2.43 & 
522.10 & 2.27 & 521.40 & 
-0.00\\CON3-3 & \bf{\underline{591.19}} & 2.44 & 
591.20 & 2.52 & 591.20 & 
-0.00\\CON3-4 & \bf{\underline{588.79}} & 1.98 & 
591.64 & 2.02 & 589.30 & 
-0.09\\CON3-5 & 564.88 & 2.07 & 
565.92 & 2.12 & \bf{563.70} & 
0.21\\CON3-6 & 500.80 & 2.42 & 
503.09 & 2.52 & \bf{499.20} & 
0.32\\CON3-7 & 578.22 & 2.04 & 
580.37 & 1.93 & \bf{577.50} & 
0.12\\CON3-8 & \bf{\underline{523.05}} & 2.19 & 
524.71 & 2.19 & 523.10 & 
-0.01\\CON3-9 & 585.25 & 1.89 & 
587.19 & 2.01 & \bf{578.20} & 
1.22\\CON8-0 & 869.15 & 2.10 & 
876.19 & 2.28 & \bf{858.90} & 
1.19\\CON8-1 & 742.44 & 2.47 & 
746.95 & 2.31 & \bf{740.90} & 
0.21\\CON8-2 & \bf{\underline{713.60}} & 2.78 & 
716.54 & 2.74 & 714.30 & 
-0.10\\CON8-3 & 812.54 & 2.30 & 
816.31 & 2.36 & \bf{812.30} & 
0.03\\CON8-4 & 784.87 & 2.10 & 
789.59 & 2.16 & \bf{770.10} & 
1.92\\CON8-5 & \bf{\underline{759.93}} & 2.23 & 
763.37 & 2.21 & 766.60 & 
-0.87\\CON8-6 & \bf{\underline{690.85}} & 2.65 & 
694.59 & 2.54 & 697.20 & 
-0.91\\CON8-7 & \bf{\underline{814.79}} & 1.94 & 
818.30 & 2.00 & 814.80 & 
-0.00\\CON8-8 & 785.30 & 2.38 & 
787.58 & 2.45 & \bf{771.30} & 
1.82\\CON8-9 & \bf{\underline{814.37}} & 2.45 & 
816.09 & 2.46 & 815.10 & 
-0.09\\[1ex]\hline
\end{tabular}
\label{table:nonlin}
\end{table} \clearpage
\begin{table}[ht]
\caption{Resultados de la ejecución de la metaheurística ACO, utilizando instancias de Dethloff con la configuración -n 3.0 -alpha 1.0 -beta 3.0 -q .5 -ro 0.015}
\centering
\small
\begin{tabular}{c c c c c c c}
\hline\hline
Instancia & Costo mínimo & Tiempo(seg.) & Costo promedio & Tiempo promedio(seg.) & Costo ACO & \%Gap \\ [0.5ex]
\hline
SCA3-0 & \bf{\underline{636.06}} & 2.03 & 
637.18 & 2.04 & 636.10 & 
-0.01\\SCA3-1 & \bf{\underline{697.84}} & 2.19 & 
697.84 & 2.25 & 700.10 & 
-0.32\\SCA3-2 & 659.34 & 2.51 & 
661.76 & 2.19 & \bf{659.30} & 
0.01\\SCA3-3 & 680.04 & 2.20 & 
680.32 & 2.00 & \bf{680.00} & 
0.01\\SCA3-4 & \bf{690.50} & 2.20 & 
690.50 & 2.23 & 690.50 & 0.00\\
SCA3-5 & \bf{\underline{661.07}} & 2.14 & 
664.90 & 2.15 & 671.10 & 
-1.49\\SCA3-6 & \bf{\underline{651.09}} & 2.31 & 
653.14 & 2.10 & 651.10 & 
-0.00\\SCA3-7 & 666.15 & 1.97 & 
666.15 & 1.88 & \bf{666.10} & 
0.01\\SCA3-8 & \bf{\underline{719.47}} & 2.05 & 
719.54 & 2.13 & 719.50 & 
-0.00\\SCA3-9 & \bf{681.00} & 1.67 & 
681.00 & 1.80 & 681.00 & 0.00\\
SCA8-0 & 971.49 & 2.36 & 
973.51 & 2.28 & \bf{961.60} & 
1.03\\SCA8-1 & \bf{\underline{1055.02}} & 1.78 & 
1057.25 & 1.83 & 1063.00 & 
-0.75\\SCA8-2 & 1048.78 & 1.88 & 
1050.95 & 1.74 & \bf{1040.60} & 
0.79\\SCA8-3 & 1009.39 & 2.12 & 
1014.38 & 2.23 & \bf{985.90} & 
2.38\\SCA8-4 & \bf{\underline{1065.49}} & 2.26 & 
1075.81 & 2.23 & 1071.00 & 
-0.51\\SCA8-5 & \bf{\underline{1041.29}} & 2.34 & 
1052.47 & 2.44 & 1054.30 & 
-1.23\\SCA8-6 & \bf{\underline{972.48}} & 2.54 & 
977.07 & 2.47 & 972.50 & 
-0.00\\SCA8-7 & 1067.20 & 2.47 & 
1072.79 & 2.38 & \bf{1059.70} & 
0.71\\SCA8-8 & \bf{\underline{1071.18}} & 2.35 & 
1074.87 & 2.44 & 1082.70 & 
-1.06\\SCA8-9 & \bf{\underline{1063.68}} & 2.03 & 
1066.49 & 1.97 & 1081.40 & 
-1.64\\CON3-0 & 617.59 & 2.53 & 
621.35 & 2.46 & \bf{616.50} & 
0.18\\CON3-1 & 556.92 & 2.09 & 
558.30 & 2.19 & \bf{555.60} & 
0.24\\CON3-2 & \bf{\underline{521.38}} & 2.03 & 
521.38 & 2.05 & 521.40 & 
-0.00\\CON3-3 & \bf{\underline{591.19}} & 2.38 & 
591.20 & 2.42 & 591.20 & 
-0.00\\CON3-4 & \bf{\underline{588.79}} & 2.15 & 
590.77 & 2.04 & 589.30 & 
-0.09\\CON3-5 & \bf{563.70} & 2.18 & 
566.51 & 2.07 & 563.70 & 0.00\\
CON3-6 & 500.80 & 2.60 & 
501.96 & 2.54 & \bf{499.20} & 
0.32\\CON3-7 & \bf{\underline{576.48}} & 1.82 & 
578.62 & 1.88 & 577.50 & 
-0.18\\CON3-8 & \bf{\underline{523.05}} & 2.08 & 
523.90 & 2.00 & 523.10 & 
-0.01\\CON3-9 & 583.35 & 2.05 & 
587.03 & 2.12 & \bf{578.20} & 
0.89\\CON8-0 & 865.86 & 2.10 & 
877.41 & 2.23 & \bf{858.90} & 
0.81\\CON8-1 & \bf{\underline{740.85}} & 2.25 & 
744.77 & 2.20 & 740.90 & 
-0.01\\CON8-2 & \bf{\underline{713.44}} & 2.71 & 
715.71 & 2.68 & 714.30 & 
-0.12\\CON8-3 & 817.45 & 2.27 & 
818.83 & 2.34 & \bf{812.30} & 
0.63\\CON8-4 & 785.41 & 2.36 & 
787.88 & 2.21 & \bf{770.10} & 
1.99\\CON8-5 & \bf{\underline{760.91}} & 2.18 & 
763.99 & 2.15 & 766.60 & 
-0.74\\CON8-6 & \bf{\underline{692.75}} & 2.54 & 
695.04 & 2.51 & 697.20 & 
-0.64\\CON8-7 & \bf{\underline{814.79}} & 1.88 & 
815.06 & 1.98 & 814.80 & 
-0.00\\CON8-8 & 785.89 & 2.48 & 
789.05 & 2.58 & \bf{771.30} & 
1.89\\CON8-9 & \bf{\underline{812.60}} & 2.52 & 
816.30 & 2.51 & 815.10 & 
-0.31\\[1ex]\hline
\end{tabular}
\label{table:nonlin}
\end{table} \clearpage
\begin{table}[ht]
\caption{Resultados de la ejecución de la metaheurística ACO, utilizando instancias de Dethloff con la configuración -n 3.0 -alpha 1.0 -beta 3.0 -q .6 -ro 0.015}
\centering
\small
\begin{tabular}{c c c c c c c}
\hline\hline
Instancia & Costo mínimo & Tiempo(seg.) & Costo promedio & Tiempo promedio(seg.) & Costo ACO & \%Gap \\ [0.5ex]
\hline
SCA3-0 & \bf{\underline{636.06}} & 2.08 & 
636.20 & 2.13 & 636.10 & 
-0.01\\SCA3-1 & \bf{\underline{697.84}} & 2.19 & 
697.84 & 2.24 & 700.10 & 
-0.32\\SCA3-2 & 659.34 & 1.99 & 
660.56 & 2.04 & \bf{659.30} & 
0.01\\SCA3-3 & 680.04 & 1.89 & 
680.18 & 2.04 & \bf{680.00} & 
0.01\\SCA3-4 & \bf{690.50} & 2.08 & 
690.50 & 2.17 & 690.50 & 0.00\\
SCA3-5 & \bf{\underline{661.07}} & 2.22 & 
664.90 & 2.17 & 671.10 & 
-1.49\\SCA3-6 & 652.94 & 1.93 & 
653.27 & 2.04 & \bf{651.10} & 
0.28\\SCA3-7 & 666.15 & 1.63 & 
667.09 & 1.85 & \bf{666.10} & 
0.01\\SCA3-8 & \bf{\underline{719.47}} & 2.00 & 
720.12 & 1.96 & 719.50 & 
-0.00\\SCA3-9 & \bf{681.00} & 1.64 & 
681.59 & 1.70 & 681.00 & 0.00\\
SCA8-0 & \bf{\underline{961.50}} & 2.36 & 
973.57 & 2.33 & 961.60 & 
-0.01\\SCA8-1 & \bf{\underline{1053.09}} & 1.88 & 
1057.22 & 1.85 & 1063.00 & 
-0.93\\SCA8-2 & 1050.37 & 1.66 & 
1051.61 & 1.60 & \bf{1040.60} & 
0.94\\SCA8-3 & 1015.31 & 2.36 & 
1018.94 & 2.27 & \bf{985.90} & 
2.98\\SCA8-4 & \bf{\underline{1065.49}} & 2.46 & 
1066.48 & 2.30 & 1071.00 & 
-0.51\\SCA8-5 & \bf{\underline{1034.74}} & 2.54 & 
1054.22 & 2.46 & 1054.30 & 
-1.86\\SCA8-6 & 977.87 & 2.51 & 
980.21 & 2.45 & \bf{972.50} & 
0.55\\SCA8-7 & 1067.20 & 2.32 & 
1077.40 & 2.36 & \bf{1059.70} & 
0.71\\SCA8-8 & \bf{\underline{1071.18}} & 2.46 & 
1079.52 & 2.42 & 1082.70 & 
-1.06\\SCA8-9 & \bf{\underline{1067.42}} & 1.72 & 
1067.42 & 1.86 & 1081.40 & 
-1.29\\CON3-0 & 617.59 & 2.29 & 
623.88 & 2.34 & \bf{616.50} & 
0.18\\CON3-1 & 556.04 & 2.28 & 
557.00 & 2.23 & \bf{555.60} & 
0.08\\CON3-2 & \bf{\underline{521.38}} & 2.11 & 
522.32 & 2.05 & 521.40 & 
-0.00\\CON3-3 & \bf{591.20} & 2.48 & 
592.02 & 2.35 & 591.20 & 0.00\\
CON3-4 & \bf{\underline{588.79}} & 1.98 & 
590.40 & 2.00 & 589.30 & 
-0.09\\CON3-5 & 568.69 & 2.28 & 
568.85 & 2.14 & \bf{563.70} & 
0.89\\CON3-6 & 502.16 & 2.65 & 
505.04 & 2.52 & \bf{499.20} & 
0.59\\CON3-7 & 578.41 & 2.08 & 
581.54 & 1.94 & \bf{577.50} & 
0.16\\CON3-8 & 524.31 & 1.90 & 
526.45 & 2.04 & \bf{523.10} & 
0.23\\CON3-9 & 578.25 & 2.01 & 
583.39 & 2.02 & \bf{578.20} & 
0.01\\CON8-0 & 873.62 & 2.11 & 
876.80 & 2.27 & \bf{858.90} & 
1.71\\CON8-1 & 742.29 & 2.24 & 
744.13 & 2.29 & \bf{740.90} & 
0.19\\CON8-2 & \bf{\underline{713.90}} & 3.03 & 
715.81 & 2.87 & 714.30 & 
-0.06\\CON8-3 & 817.22 & 2.07 & 
817.48 & 2.17 & \bf{812.30} & 
0.61\\CON8-4 & 780.46 & 2.11 & 
783.51 & 2.19 & \bf{770.10} & 
1.35\\CON8-5 & \bf{\underline{758.12}} & 2.30 & 
761.08 & 2.17 & 766.60 & 
-1.11\\CON8-6 & \bf{\underline{692.75}} & 2.29 & 
694.92 & 2.42 & 697.20 & 
-0.64\\CON8-7 & \bf{\underline{814.77}} & 2.01 & 
817.18 & 1.91 & 814.80 & 
-0.00\\CON8-8 & 781.47 & 2.50 & 
786.07 & 2.39 & \bf{771.30} & 
1.32\\CON8-9 & \bf{\underline{814.77}} & 2.60 & 
817.22 & 2.54 & 815.10 & 
-0.04\\[1ex]\hline
\end{tabular}
\label{table:nonlin}
\end{table} \clearpage
\begin{table}[ht]
\caption{Resultados de la ejecución de la metaheurística ACO, utilizando instancias de Dethloff con la configuración -n 3.0 -alpha 1.0 -beta 3.0 -q .7 -ro 0.015}
\centering
\small
\begin{tabular}{c c c c c c c}
\hline\hline
Instancia & Costo mínimo & Tiempo(seg.) & Costo promedio & Tiempo promedio(seg.) & Costo ACO & \%Gap \\ [0.5ex]
\hline
SCA3-0 & \bf{\underline{636.06}} & 2.09 & 
636.13 & 2.00 & 636.10 & 
-0.01\\SCA3-1 & \bf{\underline{697.84}} & 2.14 & 
697.84 & 2.21 & 700.10 & 
-0.32\\SCA3-2 & 661.13 & 2.12 & 
663.04 & 2.04 & \bf{659.30} & 
0.28\\SCA3-3 & 680.04 & 2.04 & 
680.32 & 2.01 & \bf{680.00} & 
0.01\\SCA3-4 & \bf{690.50} & 2.19 & 
690.50 & 2.17 & 690.50 & 0.00\\
SCA3-5 & \bf{\underline{662.75}} & 2.18 & 
666.86 & 2.20 & 671.10 & 
-1.24\\SCA3-6 & 652.94 & 2.12 & 
653.16 & 2.12 & \bf{651.10} & 
0.28\\SCA3-7 & 666.15 & 1.77 & 
666.15 & 1.70 & \bf{666.10} & 
0.01\\SCA3-8 & \bf{\underline{719.47}} & 2.18 & 
722.49 & 2.06 & 719.50 & 
-0.00\\SCA3-9 & \bf{681.00} & 1.64 & 
681.00 & 1.73 & 681.00 & 0.00\\
SCA8-0 & \bf{\underline{961.50}} & 2.47 & 
965.96 & 2.39 & 961.60 & 
-0.01\\SCA8-1 & \bf{\underline{1059.21}} & 1.87 & 
1065.09 & 1.88 & 1063.00 & 
-0.36\\SCA8-2 & 1046.29 & 1.90 & 
1048.93 & 1.68 & \bf{1040.60} & 
0.55\\SCA8-3 & 1011.16 & 2.32 & 
1013.96 & 2.31 & \bf{985.90} & 
2.56\\SCA8-4 & \bf{\underline{1067.66}} & 2.24 & 
1076.21 & 2.25 & 1071.00 & 
-0.31\\SCA8-5 & \bf{\underline{1047.55}} & 2.40 & 
1053.53 & 2.47 & 1054.30 & 
-0.64\\SCA8-6 & \bf{\underline{972.48}} & 2.34 & 
977.75 & 2.41 & 972.50 & 
-0.00\\SCA8-7 & 1067.20 & 2.34 & 
1075.36 & 2.36 & \bf{1059.70} & 
0.71\\SCA8-8 & \bf{\underline{1071.18}} & 2.22 & 
1077.66 & 2.21 & 1082.70 & 
-1.06\\SCA8-9 & \bf{\underline{1067.42}} & 1.95 & 
1067.42 & 1.92 & 1081.40 & 
-1.29\\CON3-0 & 617.59 & 2.33 & 
622.32 & 2.44 & \bf{616.50} & 
0.18\\CON3-1 & 557.21 & 2.20 & 
557.49 & 2.29 & \bf{555.60} & 
0.29\\CON3-2 & \bf{\underline{521.38}} & 1.84 & 
523.72 & 2.00 & 521.40 & 
-0.00\\CON3-3 & \bf{\underline{591.19}} & 2.34 & 
591.24 & 2.34 & 591.20 & 
-0.00\\CON3-4 & \bf{\underline{588.79}} & 1.99 & 
590.00 & 2.03 & 589.30 & 
-0.09\\CON3-5 & 564.88 & 2.15 & 
566.36 & 2.18 & \bf{563.70} & 
0.21\\CON3-6 & 503.97 & 2.57 & 
504.30 & 2.51 & \bf{499.20} & 
0.96\\CON3-7 & 578.22 & 2.00 & 
579.10 & 1.92 & \bf{577.50} & 
0.12\\CON3-8 & 523.14 & 1.72 & 
525.10 & 1.85 & \bf{523.10} & 
0.01\\CON3-9 & 587.78 & 1.88 & 
589.08 & 1.98 & \bf{578.20} & 
1.66\\CON8-0 & 869.43 & 2.11 & 
875.12 & 2.13 & \bf{858.90} & 
1.23\\CON8-1 & 742.44 & 2.05 & 
743.85 & 2.10 & \bf{740.90} & 
0.21\\CON8-2 & \bf{\underline{713.60}} & 2.75 & 
715.05 & 2.79 & 714.30 & 
-0.10\\CON8-3 & \bf{\underline{812.22}} & 2.26 & 
818.53 & 2.23 & 812.30 & 
-0.01\\CON8-4 & 789.84 & 2.12 & 
790.65 & 2.20 & \bf{770.10} & 
2.56\\CON8-5 & \bf{\underline{764.40}} & 2.12 & 
764.80 & 2.12 & 766.60 & 
-0.29\\CON8-6 & \bf{\underline{692.75}} & 2.61 & 
695.29 & 2.54 & 697.20 & 
-0.64\\CON8-7 & \bf{\underline{814.77}} & 1.83 & 
820.88 & 1.86 & 814.80 & 
-0.00\\CON8-8 & 782.86 & 2.43 & 
786.58 & 2.50 & \bf{771.30} & 
1.50\\CON8-9 & \bf{\underline{812.03}} & 2.38 & 
815.60 & 2.41 & 815.10 & 
-0.38\\[1ex]\hline
\end{tabular}
\label{table:nonlin}
\end{table} \clearpage
\begin{table}[ht]
\caption{Resultados de la ejecución de la metaheurística ACO, utilizando instancias de Dethloff con la configuración -n 3.0 -alpha 1.0 -beta 3.0 -q .8 -ro 0.015}
\centering
\small
\begin{tabular}{c c c c c c c}
\hline\hline
Instancia & Costo mínimo & Tiempo(seg.) & Costo promedio & Tiempo promedio(seg.) & Costo ACO & \%Gap \\ [0.5ex]
\hline
SCA3-0 & \bf{\underline{636.06}} & 2.03 & 
636.13 & 2.02 & 636.10 & 
-0.01\\SCA3-1 & \bf{\underline{697.84}} & 2.32 & 
697.84 & 2.27 & 700.10 & 
-0.32\\SCA3-2 & 659.34 & 1.97 & 
661.76 & 2.00 & \bf{659.30} & 
0.01\\SCA3-3 & 680.04 & 2.16 & 
680.64 & 2.11 & \bf{680.00} & 
0.01\\SCA3-4 & \bf{690.50} & 2.29 & 
690.50 & 2.15 & 690.50 & 0.00\\
SCA3-5 & \bf{\underline{661.07}} & 2.07 & 
663.77 & 2.20 & 671.10 & 
-1.49\\SCA3-6 & 652.94 & 1.90 & 
653.40 & 1.98 & \bf{651.10} & 
0.28\\SCA3-7 & 666.15 & 1.53 & 
666.38 & 1.84 & \bf{666.10} & 
0.01\\SCA3-8 & \bf{\underline{719.47}} & 2.00 & 
720.67 & 1.95 & 719.50 & 
-0.00\\SCA3-9 & \bf{681.00} & 1.58 & 
681.00 & 1.60 & 681.00 & 0.00\\
SCA8-0 & \bf{\underline{961.50}} & 2.43 & 
976.78 & 2.30 & 961.60 & 
-0.01\\SCA8-1 & \bf{\underline{1053.44}} & 1.75 & 
1061.12 & 1.95 & 1063.00 & 
-0.90\\SCA8-2 & 1046.29 & 1.66 & 
1049.83 & 1.61 & \bf{1040.60} & 
0.55\\SCA8-3 & 1006.29 & 2.28 & 
1014.56 & 2.34 & \bf{985.90} & 
2.07\\SCA8-4 & \bf{\underline{1065.49}} & 2.30 & 
1072.89 & 2.53 & 1071.00 & 
-0.51\\SCA8-5 & \bf{\underline{1047.55}} & 2.41 & 
1052.38 & 2.47 & 1054.30 & 
-0.64\\SCA8-6 & 981.41 & 2.42 & 
982.55 & 2.36 & \bf{972.50} & 
0.92\\SCA8-7 & 1067.20 & 2.35 & 
1067.20 & 2.47 & \bf{1059.70} & 
0.71\\SCA8-8 & \bf{\underline{1071.18}} & 2.32 & 
1075.92 & 2.27 & 1082.70 & 
-1.06\\SCA8-9 & \bf{\underline{1067.42}} & 1.83 & 
1067.42 & 1.87 & 1081.40 & 
-1.29\\CON3-0 & 621.82 & 2.52 & 
623.82 & 2.36 & \bf{616.50} & 
0.86\\CON3-1 & \bf{\underline{554.47}} & 2.33 & 
556.08 & 2.19 & 555.60 & 
-0.20\\CON3-2 & \bf{\underline{521.38}} & 2.01 & 
522.18 & 1.91 & 521.40 & 
-0.00\\CON3-3 & \bf{591.20} & 2.23 & 
591.20 & 2.23 & 591.20 & 0.00\\
CON3-4 & \bf{\underline{588.79}} & 1.89 & 
589.05 & 1.95 & 589.30 & 
-0.09\\CON3-5 & 568.66 & 2.01 & 
570.62 & 2.25 & \bf{563.70} & 
0.88\\CON3-6 & 502.16 & 2.62 & 
503.97 & 2.58 & \bf{499.20} & 
0.59\\CON3-7 & 578.41 & 2.02 & 
578.41 & 1.90 & \bf{577.50} & 
0.16\\CON3-8 & 524.59 & 2.05 & 
525.83 & 1.98 & \bf{523.10} & 
0.28\\CON3-9 & 584.05 & 1.88 & 
587.59 & 1.97 & \bf{578.20} & 
1.01\\CON8-0 & 879.00 & 2.18 & 
885.01 & 2.14 & \bf{858.90} & 
2.34\\CON8-1 & 742.44 & 2.23 & 
748.73 & 2.12 & \bf{740.90} & 
0.21\\CON8-2 & \bf{\underline{713.90}} & 2.92 & 
715.81 & 2.96 & 714.30 & 
-0.06\\CON8-3 & 815.80 & 2.13 & 
816.68 & 2.19 & \bf{812.30} & 
0.43\\CON8-4 & 778.60 & 2.26 & 
789.75 & 2.19 & \bf{770.10} & 
1.10\\CON8-5 & \bf{\underline{759.93}} & 2.01 & 
762.65 & 2.09 & 766.60 & 
-0.87\\CON8-6 & \bf{\underline{688.51}} & 2.51 & 
696.01 & 2.46 & 697.20 & 
-1.25\\CON8-7 & 823.36 & 1.93 & 
831.46 & 1.85 & \bf{814.80} & 
1.05\\CON8-8 & 788.77 & 2.32 & 
789.84 & 2.45 & \bf{771.30} & 
2.27\\CON8-9 & \bf{\underline{812.60}} & 2.47 & 
815.23 & 2.43 & 815.10 & 
-0.31\\[1ex]\hline
\end{tabular}
\label{table:nonlin}
\end{table} \clearpage
\begin{table}[ht]
\caption{Resultados de la ejecución de la metaheurística ACO, utilizando instancias de Dethloff con la configuración -n 3.0 -alpha 1.0 -beta 3.0 -q .9 -ro 0.015}
\centering
\small
\begin{tabular}{c c c c c c c}
\hline\hline
Instancia & Costo mínimo & Tiempo(seg.) & Costo promedio & Tiempo promedio(seg.) & Costo ACO & \%Gap \\ [0.5ex]
\hline
SCA3-0 & \bf{\underline{636.06}} & 2.00 & 
638.30 & 2.00 & 636.10 & 
-0.01\\SCA3-1 & \bf{\underline{697.84}} & 2.28 & 
697.84 & 2.22 & 700.10 & 
-0.32\\SCA3-2 & 664.18 & 1.96 & 
664.19 & 2.00 & \bf{659.30} & 
0.74\\SCA3-3 & 680.04 & 1.98 & 
680.36 & 2.09 & \bf{680.00} & 
0.01\\SCA3-4 & \bf{690.50} & 2.19 & 
690.50 & 2.11 & 690.50 & 0.00\\
SCA3-5 & \bf{\underline{665.04}} & 2.79 & 
666.61 & 2.35 & 671.10 & 
-0.90\\SCA3-6 & 652.94 & 2.06 & 
654.18 & 1.96 & \bf{651.10} & 
0.28\\SCA3-7 & \bf{\underline{659.17}} & 1.53 & 
664.40 & 1.57 & 666.10 & 
-1.04\\SCA3-8 & \bf{\underline{719.47}} & 1.69 & 
720.75 & 1.82 & 719.50 & 
-0.00\\SCA3-9 & \bf{681.00} & 1.48 & 
681.00 & 1.57 & 681.00 & 0.00\\
SCA8-0 & \bf{\underline{961.50}} & 2.26 & 
982.23 & 2.23 & 961.60 & 
-0.01\\SCA8-1 & 1070.43 & 2.00 & 
1074.08 & 1.85 & \bf{1063.00} & 
0.70\\SCA8-2 & 1051.21 & 1.53 & 
1054.00 & 1.61 & \bf{1040.60} & 
1.02\\SCA8-3 & 1021.51 & 2.21 & 
1024.02 & 2.19 & \bf{985.90} & 
3.61\\SCA8-4 & \bf{\underline{1065.49}} & 2.32 & 
1080.41 & 2.25 & 1071.00 & 
-0.51\\SCA8-5 & \bf{\underline{1053.09}} & 2.37 & 
1055.48 & 2.35 & 1054.30 & 
-0.11\\SCA8-6 & 980.13 & 2.44 & 
980.93 & 2.44 & \bf{972.50} & 
0.78\\SCA8-7 & 1067.20 & 2.40 & 
1068.53 & 2.45 & \bf{1059.70} & 
0.71\\SCA8-8 & \bf{\underline{1071.18}} & 2.28 & 
1078.38 & 2.19 & 1082.70 & 
-1.06\\SCA8-9 & \bf{\underline{1067.42}} & 1.76 & 
1067.42 & 1.81 & 1081.40 & 
-1.29\\CON3-0 & 617.59 & 2.33 & 
621.22 & 2.40 & \bf{616.50} & 
0.18\\CON3-1 & \bf{\underline{554.47}} & 2.31 & 
556.61 & 2.19 & 555.60 & 
-0.20\\CON3-2 & \bf{\underline{521.38}} & 1.88 & 
523.97 & 1.84 & 521.40 & 
-0.00\\CON3-3 & \bf{\underline{591.19}} & 2.36 & 
592.23 & 2.34 & 591.20 & 
-0.00\\CON3-4 & \bf{\underline{588.79}} & 2.50 & 
588.79 & 2.17 & 589.30 & 
-0.09\\CON3-5 & 564.88 & 2.17 & 
568.12 & 2.21 & \bf{563.70} & 
0.21\\CON3-6 & 500.80 & 2.54 & 
504.50 & 2.55 & \bf{499.20} & 
0.32\\CON3-7 & 578.41 & 1.75 & 
579.15 & 1.76 & \bf{577.50} & 
0.16\\CON3-8 & 524.59 & 1.92 & 
527.98 & 1.89 & \bf{523.10} & 
0.28\\CON3-9 & 586.17 & 2.05 & 
587.90 & 1.99 & \bf{578.20} & 
1.38\\CON8-0 & 874.76 & 2.23 & 
879.33 & 2.17 & \bf{858.90} & 
1.85\\CON8-1 & 742.29 & 2.05 & 
748.93 & 2.09 & \bf{740.90} & 
0.19\\CON8-2 & 714.94 & 3.00 & 
715.89 & 2.92 & \bf{714.30} & 
0.09\\CON8-3 & 817.57 & 2.07 & 
817.57 & 2.19 & \bf{812.30} & 
0.65\\CON8-4 & 784.87 & 2.33 & 
790.26 & 2.27 & \bf{770.10} & 
1.92\\CON8-5 & \bf{\underline{764.36}} & 2.04 & 
764.62 & 2.04 & 766.60 & 
-0.29\\CON8-6 & \bf{\underline{689.56}} & 2.80 & 
696.43 & 2.55 & 697.20 & 
-1.10\\CON8-7 & \bf{\underline{814.50}} & 1.84 & 
818.71 & 1.88 & 814.80 & 
-0.04\\CON8-8 & 791.45 & 2.38 & 
796.24 & 2.30 & \bf{771.30} & 
2.61\\CON8-9 & 815.75 & 2.29 & 
817.82 & 2.31 & \bf{815.10} & 
0.08\\[1ex]\hline
\end{tabular}
\label{table:nonlin}
\end{table} \clearpage
\begin{table}[ht]
\caption{Resultados de la ejecución de la metaheurística ACO, utilizando instancias de Dethloff con la configuración -n 4.0 -alpha 1.0 -beta 3.0 -q 0.1 -ro 0.015}
\centering
\small
\begin{tabular}{c c c c c c c}
\hline\hline
Instancia & Costo mínimo & Tiempo(seg.) & Costo promedio & Tiempo promedio(seg.) & Costo ACO & \%Gap \\ [0.5ex]
\hline
SCA3-0 & \bf{\underline{636.06}} & 2.97 & 
636.06 & 3.11 & 636.10 & 
-0.01\\SCA3-1 & \bf{\underline{697.84}} & 3.06 & 
697.84 & 3.23 & 700.10 & 
-0.32\\SCA3-2 & 659.34 & 2.79 & 
659.79 & 2.85 & \bf{659.30} & 
0.01\\SCA3-3 & 680.04 & 2.69 & 
680.36 & 2.81 & \bf{680.00} & 
0.01\\SCA3-4 & \bf{690.50} & 3.00 & 
690.50 & 3.02 & 690.50 & 0.00\\
SCA3-5 & \bf{\underline{662.75}} & 2.94 & 
663.47 & 3.08 & 671.10 & 
-1.24\\SCA3-6 & \bf{\underline{651.09}} & 3.12 & 
652.48 & 3.09 & 651.10 & 
-0.00\\SCA3-7 & \bf{\underline{664.88}} & 2.92 & 
665.95 & 2.66 & 666.10 & 
-0.18\\SCA3-8 & \bf{\underline{719.47}} & 2.92 & 
719.54 & 3.01 & 719.50 & 
-0.00\\SCA3-9 & \bf{681.00} & 2.65 & 
681.00 & 2.56 & 681.00 & 0.00\\
SCA8-0 & \bf{\underline{961.50}} & 2.93 & 
979.07 & 3.11 & 961.60 & 
-0.01\\SCA8-1 & \bf{\underline{1059.55}} & 2.83 & 
1067.80 & 2.72 & 1063.00 & 
-0.32\\SCA8-2 & 1050.17 & 2.52 & 
1050.47 & 2.48 & \bf{1040.60} & 
0.92\\SCA8-3 & 1005.22 & 3.11 & 
1009.93 & 2.97 & \bf{985.90} & 
1.96\\SCA8-4 & \bf{\underline{1065.49}} & 3.25 & 
1068.75 & 3.06 & 1071.00 & 
-0.51\\SCA8-5 & \bf{\underline{1034.74}} & 4.26 & 
1049.29 & 3.69 & 1054.30 & 
-1.86\\SCA8-6 & 979.28 & 3.24 & 
979.71 & 3.31 & \bf{972.50} & 
0.70\\SCA8-7 & 1066.65 & 3.28 & 
1070.93 & 3.10 & \bf{1059.70} & 
0.66\\SCA8-8 & \bf{\underline{1071.18}} & 3.22 & 
1081.12 & 3.32 & 1082.70 & 
-1.06\\SCA8-9 & \bf{\underline{1067.42}} & 2.76 & 
1067.42 & 3.00 & 1081.40 & 
-1.29\\CON3-0 & 616.52 & 3.01 & 
620.82 & 3.17 & \bf{616.50} & 
0.00\\CON3-1 & \bf{\underline{554.47}} & 3.29 & 
555.84 & 3.17 & 555.60 & 
-0.20\\CON3-2 & \bf{\underline{519.61}} & 2.90 & 
521.00 & 2.92 & 521.40 & 
-0.34\\CON3-3 & \bf{\underline{591.19}} & 3.48 & 
591.20 & 3.25 & 591.20 & 
-0.00\\CON3-4 & \bf{\underline{588.79}} & 2.93 & 
588.79 & 2.87 & 589.30 & 
-0.09\\CON3-5 & 564.88 & 2.89 & 
567.00 & 2.92 & \bf{563.70} & 
0.21\\CON3-6 & 500.80 & 3.23 & 
502.21 & 3.28 & \bf{499.20} & 
0.32\\CON3-7 & 578.22 & 2.58 & 
580.68 & 2.77 & \bf{577.50} & 
0.12\\CON3-8 & \bf{\underline{523.05}} & 3.25 & 
523.77 & 3.02 & 523.10 & 
-0.01\\CON3-9 & 588.40 & 3.06 & 
588.42 & 2.86 & \bf{578.20} & 
1.76\\CON8-0 & 870.22 & 3.16 & 
871.56 & 3.09 & \bf{858.90} & 
1.32\\CON8-1 & \bf{\underline{740.85}} & 3.05 & 
742.18 & 3.07 & 740.90 & 
-0.01\\CON8-2 & \bf{\underline{713.44}} & 3.72 & 
714.84 & 3.63 & 714.30 & 
-0.12\\CON8-3 & \bf{\underline{812.11}} & 3.08 & 
816.32 & 3.08 & 812.30 & 
-0.02\\CON8-4 & 780.09 & 3.06 & 
783.16 & 2.91 & \bf{770.10} & 
1.30\\CON8-5 & \bf{\underline{758.84}} & 3.26 & 
762.70 & 3.27 & 766.60 & 
-1.01\\CON8-6 & \bf{\underline{686.97}} & 3.40 & 
693.16 & 3.49 & 697.20 & 
-1.47\\CON8-7 & \bf{\underline{814.79}} & 2.74 & 
816.97 & 2.80 & 814.80 & 
-0.00\\CON8-8 & 782.09 & 3.40 & 
784.41 & 3.50 & \bf{771.30} & 
1.40\\CON8-9 & \bf{\underline{811.59}} & 3.50 & 
814.12 & 3.46 & 815.10 & 
-0.43\\[1ex]\hline
\end{tabular}
\label{table:nonlin}
\end{table} \clearpage
\begin{table}[ht]
\caption{Resultados de la ejecución de la metaheurística ACO, utilizando instancias de Dethloff con la configuración -n 4.0 -alpha 1.0 -beta 3.0 -q .2 -ro 0.015}
\centering
\small
\begin{tabular}{c c c c c c c}
\hline\hline
Instancia & Costo mínimo & Tiempo(seg.) & Costo promedio & Tiempo promedio(seg.) & Costo ACO & \%Gap \\ [0.5ex]
\hline
SCA3-0 & \bf{\underline{636.06}} & 2.89 & 
636.06 & 2.87 & 636.10 & 
-0.01\\SCA3-1 & \bf{\underline{697.84}} & 3.16 & 
697.84 & 3.19 & 700.10 & 
-0.32\\SCA3-2 & 659.34 & 3.04 & 
662.21 & 2.84 & \bf{659.30} & 
0.01\\SCA3-3 & 680.04 & 2.68 & 
680.18 & 2.74 & \bf{680.00} & 
0.01\\SCA3-4 & \bf{690.50} & 2.93 & 
690.50 & 2.97 & 690.50 & 0.00\\
SCA3-5 & \bf{\underline{662.75}} & 3.10 & 
664.04 & 3.02 & 671.10 & 
-1.24\\SCA3-6 & 652.94 & 2.86 & 
653.36 & 2.87 & \bf{651.10} & 
0.28\\SCA3-7 & 666.15 & 2.58 & 
666.15 & 2.57 & \bf{666.10} & 
0.01\\SCA3-8 & \bf{\underline{719.47}} & 2.95 & 
721.93 & 2.90 & 719.50 & 
-0.00\\SCA3-9 & \bf{681.00} & 2.53 & 
681.00 & 2.43 & 681.00 & 0.00\\
SCA8-0 & 968.79 & 2.98 & 
979.93 & 3.23 & \bf{961.60} & 
0.75\\SCA8-1 & \bf{\underline{1057.09}} & 2.74 & 
1065.92 & 2.81 & 1063.00 & 
-0.56\\SCA8-2 & 1045.64 & 2.56 & 
1048.22 & 2.36 & \bf{1040.60} & 
0.48\\SCA8-3 & \bf{\underline{983.34}} & 2.87 & 
1005.91 & 2.95 & 985.90 & 
-0.26\\SCA8-4 & \bf{\underline{1065.49}} & 2.87 & 
1066.50 & 3.04 & 1071.00 & 
-0.51\\SCA8-5 & \bf{\underline{1036.88}} & 3.46 & 
1051.27 & 3.46 & 1054.30 & 
-1.65\\SCA8-6 & 978.85 & 3.08 & 
983.37 & 3.24 & \bf{972.50} & 
0.65\\SCA8-7 & 1066.65 & 3.04 & 
1069.45 & 3.13 & \bf{1059.70} & 
0.66\\SCA8-8 & \bf{\underline{1071.18}} & 3.36 & 
1074.87 & 3.31 & 1082.70 & 
-1.06\\SCA8-9 & \bf{\underline{1067.42}} & 2.88 & 
1067.42 & 2.63 & 1081.40 & 
-1.29\\CON3-0 & 617.59 & 3.01 & 
621.02 & 3.14 & \bf{616.50} & 
0.18\\CON3-1 & \bf{\underline{554.47}} & 3.20 & 
555.77 & 3.12 & 555.60 & 
-0.20\\CON3-2 & \bf{\underline{521.38}} & 3.06 & 
521.44 & 2.91 & 521.40 & 
-0.00\\CON3-3 & \bf{\underline{591.19}} & 3.16 & 
591.20 & 3.21 & 591.20 & 
-0.00\\CON3-4 & 589.32 & 2.84 & 
591.69 & 2.74 & \bf{589.30} & 
0.00\\CON3-5 & \bf{563.70} & 3.03 & 
565.35 & 3.02 & 563.70 & 0.00\\
CON3-6 & 500.80 & 3.23 & 
502.02 & 3.35 & \bf{499.20} & 
0.32\\CON3-7 & 578.22 & 2.68 & 
578.36 & 2.61 & \bf{577.50} & 
0.12\\CON3-8 & \bf{\underline{523.05}} & 2.70 & 
523.82 & 2.87 & 523.10 & 
-0.01\\CON3-9 & 582.79 & 2.89 & 
585.79 & 2.92 & \bf{578.20} & 
0.79\\CON8-0 & 866.22 & 3.14 & 
877.24 & 3.11 & \bf{858.90} & 
0.85\\CON8-1 & \bf{\underline{740.85}} & 3.15 & 
743.90 & 3.20 & 740.90 & 
-0.01\\CON8-2 & \bf{\underline{713.60}} & 3.68 & 
714.29 & 3.64 & 714.30 & 
-0.10\\CON8-3 & \bf{\underline{812.22}} & 3.17 & 
815.42 & 3.06 & 812.30 & 
-0.01\\CON8-4 & 776.72 & 2.81 & 
783.47 & 2.87 & \bf{770.10} & 
0.86\\CON8-5 & \bf{\underline{760.03}} & 3.24 & 
762.32 & 3.03 & 766.60 & 
-0.86\\CON8-6 & \bf{\underline{688.56}} & 3.42 & 
694.51 & 3.53 & 697.20 & 
-1.24\\CON8-7 & \bf{\underline{814.50}} & 2.74 & 
814.74 & 2.87 & 814.80 & 
-0.04\\CON8-8 & 778.98 & 3.42 & 
784.25 & 3.37 & \bf{771.30} & 
1.00\\CON8-9 & \bf{\underline{811.59}} & 3.52 & 
813.07 & 3.48 & 815.10 & 
-0.43\\[1ex]\hline
\end{tabular}
\label{table:nonlin}
\end{table} \clearpage
\begin{table}[ht]
\caption{Resultados de la ejecución de la metaheurística ACO, utilizando instancias de Dethloff con la configuración -n 4.0 -alpha 1.0 -beta 3.0 -q .3 -ro 0.015}
\centering
\small
\begin{tabular}{c c c c c c c}
\hline\hline
Instancia & Costo mínimo & Tiempo(seg.) & Costo promedio & Tiempo promedio(seg.) & Costo ACO & \%Gap \\ [0.5ex]
\hline
SCA3-0 & \bf{\underline{636.06}} & 2.73 & 
636.06 & 2.79 & 636.10 & 
-0.01\\SCA3-1 & \bf{\underline{697.84}} & 3.10 & 
697.84 & 3.12 & 700.10 & 
-0.32\\SCA3-2 & 659.34 & 2.92 & 
661.76 & 2.82 & \bf{659.30} & 
0.01\\SCA3-3 & 680.04 & 2.65 & 
680.04 & 2.79 & \bf{680.00} & 
0.01\\SCA3-4 & \bf{690.50} & 3.13 & 
690.50 & 3.02 & 690.50 & 0.00\\
SCA3-5 & \bf{\underline{665.04}} & 3.04 & 
665.90 & 2.94 & 671.10 & 
-0.90\\SCA3-6 & \bf{\underline{651.09}} & 2.71 & 
652.01 & 2.83 & 651.10 & 
-0.00\\SCA3-7 & 666.15 & 2.50 & 
666.15 & 2.42 & \bf{666.10} & 
0.01\\SCA3-8 & \bf{\underline{719.47}} & 2.72 & 
720.12 & 2.84 & 719.50 & 
-0.00\\SCA3-9 & \bf{681.00} & 2.43 & 
681.00 & 2.41 & 681.00 & 0.00\\
SCA8-0 & 968.79 & 2.90 & 
980.97 & 3.06 & \bf{961.60} & 
0.75\\SCA8-1 & \bf{\underline{1060.41}} & 2.81 & 
1065.82 & 2.72 & 1063.00 & 
-0.24\\SCA8-2 & 1050.37 & 2.30 & 
1050.94 & 2.33 & \bf{1040.60} & 
0.94\\SCA8-3 & 998.59 & 2.86 & 
1007.59 & 2.90 & \bf{985.90} & 
1.29\\SCA8-4 & \bf{\underline{1067.28}} & 2.98 & 
1069.28 & 3.02 & 1071.00 & 
-0.35\\SCA8-5 & \bf{\underline{1045.15}} & 3.42 & 
1054.64 & 3.51 & 1054.30 & 
-0.87\\SCA8-6 & 977.03 & 3.09 & 
981.62 & 3.25 & \bf{972.50} & 
0.47\\SCA8-7 & 1067.20 & 3.14 & 
1071.31 & 3.20 & \bf{1059.70} & 
0.71\\SCA8-8 & \bf{\underline{1071.18}} & 3.34 & 
1072.13 & 3.34 & 1082.70 & 
-1.06\\SCA8-9 & \bf{\underline{1067.42}} & 2.76 & 
1068.24 & 2.63 & 1081.40 & 
-1.29\\CON3-0 & 617.59 & 3.25 & 
620.40 & 3.31 & \bf{616.50} & 
0.18\\CON3-1 & \bf{\underline{554.47}} & 2.94 & 
556.69 & 3.03 & 555.60 & 
-0.20\\CON3-2 & \bf{\underline{521.38}} & 2.63 & 
522.62 & 2.89 & 521.40 & 
-0.00\\CON3-3 & \bf{\underline{591.19}} & 3.24 & 
591.23 & 3.20 & 591.20 & 
-0.00\\CON3-4 & \bf{\underline{588.79}} & 2.78 & 
590.11 & 2.81 & 589.30 & 
-0.09\\CON3-5 & \bf{563.70} & 2.88 & 
565.24 & 2.88 & 563.70 & 0.00\\
CON3-6 & 502.16 & 3.20 & 
502.33 & 3.41 & \bf{499.20} & 
0.59\\CON3-7 & 578.41 & 2.56 & 
580.11 & 2.69 & \bf{577.50} & 
0.16\\CON3-8 & 523.14 & 2.80 & 
523.64 & 2.78 & \bf{523.10} & 
0.01\\CON3-9 & 578.25 & 2.82 & 
585.18 & 2.77 & \bf{578.20} & 
0.01\\CON8-0 & 869.43 & 3.07 & 
876.69 & 2.94 & \bf{858.90} & 
1.23\\CON8-1 & \bf{\underline{740.85}} & 3.05 & 
743.49 & 3.21 & 740.90 & 
-0.01\\CON8-2 & \bf{\underline{712.89}} & 3.62 & 
715.24 & 3.69 & 714.30 & 
-0.20\\CON8-3 & 815.14 & 3.04 & 
816.88 & 3.08 & \bf{812.30} & 
0.35\\CON8-4 & 776.72 & 2.98 & 
782.70 & 3.16 & \bf{770.10} & 
0.86\\CON8-5 & \bf{\underline{755.14}} & 3.14 & 
759.85 & 2.99 & 766.60 & 
-1.49\\CON8-6 & \bf{\underline{693.80}} & 3.26 & 
694.38 & 3.40 & 697.20 & 
-0.49\\CON8-7 & 814.86 & 2.67 & 
820.13 & 2.77 & \bf{814.80} & 
0.01\\CON8-8 & 783.02 & 3.18 & 
788.58 & 3.40 & \bf{771.30} & 
1.52\\CON8-9 & \bf{\underline{811.32}} & 3.37 & 
813.45 & 3.36 & 815.10 & 
-0.46\\[1ex]\hline
\end{tabular}
\label{table:nonlin}
\end{table} \clearpage
\begin{table}[ht]
\caption{Resultados de la ejecución de la metaheurística ACO, utilizando instancias de Dethloff con la configuración -n 4.0 -alpha 1.0 -beta 3.0 -q .4 -ro 0.015}
\centering
\small
\begin{tabular}{c c c c c c c}
\hline\hline
Instancia & Costo mínimo & Tiempo(seg.) & Costo promedio & Tiempo promedio(seg.) & Costo ACO & \%Gap \\ [0.5ex]
\hline
SCA3-0 & \bf{\underline{636.06}} & 3.03 & 
637.47 & 2.84 & 636.10 & 
-0.01\\SCA3-1 & \bf{\underline{697.84}} & 3.16 & 
697.84 & 3.17 & 700.10 & 
-0.32\\SCA3-2 & 659.34 & 2.95 & 
660.24 & 2.85 & \bf{659.30} & 
0.01\\SCA3-3 & 680.04 & 2.89 & 
680.04 & 2.80 & \bf{680.00} & 
0.01\\SCA3-4 & \bf{690.50} & 2.92 & 
690.50 & 2.90 & 690.50 & 0.00\\
SCA3-5 & \bf{\underline{659.90}} & 3.10 & 
662.05 & 2.97 & 671.10 & 
-1.67\\SCA3-6 & 652.94 & 2.80 & 
653.13 & 2.83 & \bf{651.10} & 
0.28\\SCA3-7 & 666.15 & 2.44 & 
666.15 & 2.45 & \bf{666.10} & 
0.01\\SCA3-8 & \bf{\underline{719.47}} & 2.82 & 
720.60 & 2.73 & 719.50 & 
-0.00\\SCA3-9 & \bf{681.00} & 2.45 & 
681.00 & 2.41 & 681.00 & 0.00\\
SCA8-0 & \bf{\underline{961.50}} & 3.11 & 
970.75 & 3.01 & 961.60 & 
-0.01\\SCA8-1 & \bf{\underline{1059.55}} & 2.65 & 
1062.33 & 2.72 & 1063.00 & 
-0.32\\SCA8-2 & 1045.64 & 2.34 & 
1049.73 & 2.35 & \bf{1040.60} & 
0.48\\SCA8-3 & 1011.52 & 2.98 & 
1015.71 & 2.88 & \bf{985.90} & 
2.60\\SCA8-4 & \bf{\underline{1065.49}} & 3.06 & 
1067.78 & 2.90 & 1071.00 & 
-0.51\\SCA8-5 & \bf{\underline{1034.74}} & 3.41 & 
1042.27 & 3.39 & 1054.30 & 
-1.86\\SCA8-6 & 977.03 & 3.44 & 
979.95 & 3.25 & \bf{972.50} & 
0.47\\SCA8-7 & 1067.11 & 3.19 & 
1069.92 & 3.22 & \bf{1059.70} & 
0.70\\SCA8-8 & \bf{\underline{1071.18}} & 3.16 & 
1076.64 & 3.25 & 1082.70 & 
-1.06\\SCA8-9 & \bf{\underline{1067.42}} & 2.76 & 
1067.42 & 2.59 & 1081.40 & 
-1.29\\CON3-0 & 616.52 & 3.37 & 
619.11 & 3.17 & \bf{616.50} & 
0.00\\CON3-1 & \bf{\underline{554.47}} & 2.83 & 
556.30 & 2.91 & 555.60 & 
-0.20\\CON3-2 & \bf{\underline{519.11}} & 2.86 & 
521.96 & 2.80 & 521.40 & 
-0.44\\CON3-3 & \bf{\underline{591.19}} & 3.19 & 
591.54 & 3.08 & 591.20 & 
-0.00\\CON3-4 & \bf{\underline{588.79}} & 2.65 & 
590.38 & 2.74 & 589.30 & 
-0.09\\CON3-5 & \bf{563.70} & 2.80 & 
566.49 & 2.93 & 563.70 & 0.00\\
CON3-6 & 501.05 & 3.37 & 
501.69 & 3.42 & \bf{499.20} & 
0.37\\CON3-7 & \bf{\underline{576.48}} & 2.68 & 
579.43 & 2.58 & 577.50 & 
-0.18\\CON3-8 & 523.14 & 2.90 & 
523.57 & 2.78 & \bf{523.10} & 
0.01\\CON3-9 & 588.48 & 3.09 & 
588.88 & 2.78 & \bf{578.20} & 
1.78\\CON8-0 & 866.22 & 3.07 & 
870.49 & 3.00 & \bf{858.90} & 
0.85\\CON8-1 & 742.29 & 3.14 & 
746.61 & 3.05 & \bf{740.90} & 
0.19\\CON8-2 & \bf{\underline{713.60}} & 3.40 & 
715.66 & 3.61 & 714.30 & 
-0.10\\CON8-3 & 815.80 & 2.92 & 
817.17 & 2.97 & \bf{812.30} & 
0.43\\CON8-4 & 781.56 & 2.90 & 
790.07 & 2.86 & \bf{770.10} & 
1.49\\CON8-5 & \bf{\underline{758.84}} & 2.94 & 
762.87 & 2.91 & 766.60 & 
-1.01\\CON8-6 & \bf{\underline{683.83}} & 3.29 & 
689.90 & 3.35 & 697.20 & 
-1.92\\CON8-7 & 814.86 & 2.71 & 
816.77 & 2.67 & \bf{814.80} & 
0.01\\CON8-8 & 779.43 & 3.17 & 
786.30 & 3.34 & \bf{771.30} & 
1.05\\CON8-9 & \bf{\underline{812.60}} & 3.71 & 
814.81 & 3.48 & 815.10 & 
-0.31\\[1ex]\hline
\end{tabular}
\label{table:nonlin}
\end{table} \clearpage
\begin{table}[ht]
\caption{Resultados de la ejecución de la metaheurística ACO, utilizando instancias de Dethloff con la configuración -n 4.0 -alpha 1.0 -beta 3.0 -q .5 -ro 0.015}
\centering
\small
\begin{tabular}{c c c c c c c}
\hline\hline
Instancia & Costo mínimo & Tiempo(seg.) & Costo promedio & Tiempo promedio(seg.) & Costo ACO & \%Gap \\ [0.5ex]
\hline
SCA3-0 & \bf{\underline{636.06}} & 2.73 & 
636.06 & 2.80 & 636.10 & 
-0.01\\SCA3-1 & \bf{\underline{697.84}} & 2.99 & 
697.84 & 3.05 & 700.10 & 
-0.32\\SCA3-2 & 659.34 & 2.75 & 
663.06 & 2.76 & \bf{659.30} & 
0.01\\SCA3-3 & 680.04 & 2.98 & 
680.32 & 2.77 & \bf{680.00} & 
0.01\\SCA3-4 & \bf{690.50} & 2.86 & 
690.50 & 3.04 & 690.50 & 0.00\\
SCA3-5 & \bf{\underline{665.04}} & 2.98 & 
666.77 & 2.99 & 671.10 & 
-0.90\\SCA3-6 & 652.94 & 2.94 & 
653.16 & 2.90 & \bf{651.10} & 
0.28\\SCA3-7 & 666.15 & 2.35 & 
666.15 & 2.45 & \bf{666.10} & 
0.01\\SCA3-8 & \bf{\underline{719.47}} & 2.82 & 
720.04 & 2.81 & 719.50 & 
-0.00\\SCA3-9 & \bf{681.00} & 2.16 & 
681.81 & 2.26 & 681.00 & 0.00\\
SCA8-0 & \bf{\underline{961.50}} & 3.11 & 
975.13 & 3.10 & 961.60 & 
-0.01\\SCA8-1 & \bf{\underline{1052.71}} & 2.54 & 
1063.33 & 2.58 & 1063.00 & 
-0.97\\SCA8-2 & 1045.64 & 2.19 & 
1049.40 & 2.42 & \bf{1040.60} & 
0.48\\SCA8-3 & 1002.24 & 2.76 & 
1007.93 & 3.05 & \bf{985.90} & 
1.66\\SCA8-4 & \bf{\underline{1065.49}} & 3.04 & 
1072.13 & 3.00 & 1071.00 & 
-0.51\\SCA8-5 & \bf{\underline{1050.60}} & 3.12 & 
1055.02 & 3.25 & 1054.30 & 
-0.35\\SCA8-6 & \bf{\underline{972.48}} & 3.35 & 
978.93 & 3.18 & 972.50 & 
-0.00\\SCA8-7 & 1067.03 & 3.34 & 
1069.28 & 3.47 & \bf{1059.70} & 
0.69\\SCA8-8 & \bf{\underline{1071.18}} & 3.04 & 
1075.44 & 3.20 & 1082.70 & 
-1.06\\SCA8-9 & \bf{\underline{1067.42}} & 2.56 & 
1067.42 & 2.45 & 1081.40 & 
-1.29\\CON3-0 & 616.52 & 3.32 & 
622.29 & 3.19 & \bf{616.50} & 
0.00\\CON3-1 & \bf{\underline{554.47}} & 2.96 & 
558.10 & 2.94 & 555.60 & 
-0.20\\CON3-2 & \bf{\underline{521.38}} & 2.87 & 
522.72 & 2.79 & 521.40 & 
-0.00\\CON3-3 & \bf{\underline{591.19}} & 3.06 & 
591.19 & 3.17 & 591.20 & 
-0.00\\CON3-4 & \bf{\underline{588.79}} & 2.71 & 
589.72 & 2.74 & 589.30 & 
-0.09\\CON3-5 & \bf{563.70} & 3.02 & 
564.59 & 2.91 & 563.70 & 0.00\\
CON3-6 & 502.16 & 3.19 & 
503.61 & 3.33 & \bf{499.20} & 
0.59\\CON3-7 & 578.22 & 2.51 & 
579.05 & 2.53 & \bf{577.50} & 
0.12\\CON3-8 & \bf{\underline{523.05}} & 2.65 & 
523.52 & 2.61 & 523.10 & 
-0.01\\CON3-9 & 586.17 & 2.95 & 
587.86 & 2.81 & \bf{578.20} & 
1.38\\CON8-0 & 866.43 & 3.00 & 
874.68 & 2.98 & \bf{858.90} & 
0.88\\CON8-1 & \bf{\underline{740.85}} & 3.07 & 
744.93 & 3.10 & 740.90 & 
-0.01\\CON8-2 & \bf{\underline{713.90}} & 3.72 & 
715.83 & 3.73 & 714.30 & 
-0.06\\CON8-3 & 812.32 & 2.82 & 
816.17 & 2.90 & \bf{812.30} & 
0.00\\CON8-4 & 776.37 & 2.87 & 
782.26 & 2.81 & \bf{770.10} & 
0.81\\CON8-5 & \bf{\underline{759.93}} & 2.93 & 
760.45 & 2.82 & 766.60 & 
-0.87\\CON8-6 & \bf{\underline{692.97}} & 3.37 & 
695.27 & 3.26 & 697.20 & 
-0.61\\CON8-7 & \bf{\underline{814.79}} & 2.54 & 
815.16 & 2.54 & 814.80 & 
-0.00\\CON8-8 & 784.25 & 3.26 & 
788.42 & 3.22 & \bf{771.30} & 
1.68\\CON8-9 & \bf{\underline{810.18}} & 3.28 & 
814.13 & 3.41 & 815.10 & 
-0.60\\[1ex]\hline
\end{tabular}
\label{table:nonlin}
\end{table} \clearpage
\begin{table}[ht]
\caption{Resultados de la ejecución de la metaheurística ACO, utilizando instancias de Dethloff con la configuración -n 4.0 -alpha 1.0 -beta 3.0 -q .6 -ro 0.015}
\centering
\small
\begin{tabular}{c c c c c c c}
\hline\hline
Instancia & Costo mínimo & Tiempo(seg.) & Costo promedio & Tiempo promedio(seg.) & Costo ACO & \%Gap \\ [0.5ex]
\hline
SCA3-0 & \bf{\underline{636.06}} & 2.74 & 
636.06 & 2.75 & 636.10 & 
-0.01\\SCA3-1 & \bf{\underline{697.84}} & 2.99 & 
697.84 & 2.90 & 700.10 & 
-0.32\\SCA3-2 & 659.34 & 2.65 & 
660.55 & 2.70 & \bf{659.30} & 
0.01\\SCA3-3 & 680.04 & 2.59 & 
680.18 & 2.70 & \bf{680.00} & 
0.01\\SCA3-4 & \bf{690.50} & 3.04 & 
690.50 & 3.05 & 690.50 & 0.00\\
SCA3-5 & \bf{\underline{659.90}} & 2.71 & 
662.76 & 2.79 & 671.10 & 
-1.67\\SCA3-6 & 652.94 & 2.68 & 
653.16 & 2.72 & \bf{651.10} & 
0.28\\SCA3-7 & 666.15 & 2.38 & 
666.15 & 2.30 & \bf{666.10} & 
0.01\\SCA3-8 & \bf{\underline{719.47}} & 2.51 & 
720.60 & 2.74 & 719.50 & 
-0.00\\SCA3-9 & \bf{681.00} & 2.29 & 
681.00 & 2.36 & 681.00 & 0.00\\
SCA8-0 & \bf{\underline{961.50}} & 3.29 & 
965.61 & 3.15 & 961.60 & 
-0.01\\SCA8-1 & 1064.33 & 2.33 & 
1067.51 & 2.47 & \bf{1063.00} & 
0.13\\SCA8-2 & 1049.22 & 2.47 & 
1051.17 & 2.27 & \bf{1040.60} & 
0.83\\SCA8-3 & 1008.29 & 3.24 & 
1015.00 & 3.04 & \bf{985.90} & 
2.27\\SCA8-4 & \bf{\underline{1067.66}} & 3.13 & 
1072.09 & 3.10 & 1071.00 & 
-0.31\\SCA8-5 & \bf{\underline{1034.74}} & 3.45 & 
1047.25 & 3.32 & 1054.30 & 
-1.86\\SCA8-6 & \bf{\underline{972.48}} & 3.26 & 
976.82 & 3.13 & 972.50 & 
-0.00\\SCA8-7 & 1067.20 & 3.25 & 
1071.82 & 3.18 & \bf{1059.70} & 
0.71\\SCA8-8 & \bf{\underline{1071.18}} & 3.20 & 
1079.24 & 3.16 & 1082.70 & 
-1.06\\SCA8-9 & \bf{\underline{1067.42}} & 2.66 & 
1067.42 & 2.53 & 1081.40 & 
-1.29\\CON3-0 & 624.96 & 2.91 & 
625.53 & 3.10 & \bf{616.50} & 
1.37\\CON3-1 & 556.04 & 2.97 & 
558.05 & 3.03 & \bf{555.60} & 
0.08\\CON3-2 & \bf{\underline{521.38}} & 2.61 & 
523.17 & 2.73 & 521.40 & 
-0.00\\CON3-3 & \bf{\underline{591.19}} & 3.21 & 
591.27 & 3.11 & 591.20 & 
-0.00\\CON3-4 & \bf{\underline{588.79}} & 2.86 & 
588.92 & 2.76 & 589.30 & 
-0.09\\CON3-5 & 566.45 & 3.08 & 
567.28 & 3.06 & \bf{563.70} & 
0.49\\CON3-6 & \bf{\underline{499.05}} & 3.42 & 
502.56 & 3.42 & 499.20 & 
-0.03\\CON3-7 & 578.22 & 2.45 & 
578.36 & 2.44 & \bf{577.50} & 
0.12\\CON3-8 & 524.59 & 2.77 & 
525.09 & 2.64 & \bf{523.10} & 
0.28\\CON3-9 & 583.32 & 2.56 & 
587.05 & 2.53 & \bf{578.20} & 
0.89\\CON8-0 & \bf{\underline{858.63}} & 3.12 & 
872.97 & 3.15 & 858.90 & 
-0.03\\CON8-1 & \bf{\underline{740.85}} & 3.01 & 
743.80 & 3.01 & 740.90 & 
-0.01\\CON8-2 & \bf{\underline{713.44}} & 3.82 & 
714.46 & 3.71 & 714.30 & 
-0.12\\CON8-3 & \bf{\underline{811.07}} & 2.90 & 
816.79 & 2.87 & 812.30 & 
-0.15\\CON8-4 & 785.73 & 2.99 & 
788.12 & 2.87 & \bf{770.10} & 
2.03\\CON8-5 & \bf{\underline{758.12}} & 2.79 & 
760.84 & 2.84 & 766.60 & 
-1.11\\CON8-6 & \bf{\underline{694.59}} & 3.36 & 
697.01 & 3.31 & 697.20 & 
-0.37\\CON8-7 & \bf{\underline{814.79}} & 2.70 & 
815.40 & 2.58 & 814.80 & 
-0.00\\CON8-8 & 788.09 & 3.07 & 
791.88 & 3.23 & \bf{771.30} & 
2.18\\CON8-9 & \bf{\underline{812.60}} & 3.18 & 
814.62 & 3.38 & 815.10 & 
-0.31\\[1ex]\hline
\end{tabular}
\label{table:nonlin}
\end{table} \clearpage
\begin{table}[ht]
\caption{Resultados de la ejecución de la metaheurística ACO, utilizando instancias de Dethloff con la configuración -n 4.0 -alpha 1.0 -beta 3.0 -q .7 -ro 0.015}
\centering
\small
\begin{tabular}{c c c c c c c}
\hline\hline
Instancia & Costo mínimo & Tiempo(seg.) & Costo promedio & Tiempo promedio(seg.) & Costo ACO & \%Gap \\ [0.5ex]
\hline
SCA3-0 & \bf{\underline{636.06}} & 2.65 & 
636.06 & 2.71 & 636.10 & 
-0.01\\SCA3-1 & \bf{\underline{697.84}} & 2.98 & 
697.84 & 3.01 & 700.10 & 
-0.32\\SCA3-2 & 664.18 & 2.53 & 
665.78 & 2.65 & \bf{659.30} & 
0.74\\SCA3-3 & 680.04 & 2.77 & 
680.18 & 2.71 & \bf{680.00} & 
0.01\\SCA3-4 & \bf{690.50} & 3.68 & 
690.50 & 3.00 & 690.50 & 0.00\\
SCA3-5 & \bf{\underline{661.07}} & 2.95 & 
666.72 & 2.97 & 671.10 & 
-1.49\\SCA3-6 & 652.94 & 2.69 & 
653.16 & 2.67 & \bf{651.10} & 
0.28\\SCA3-7 & 666.15 & 2.21 & 
666.15 & 2.19 & \bf{666.10} & 
0.01\\SCA3-8 & 719.77 & 2.60 & 
722.49 & 2.63 & \bf{719.50} & 
0.04\\SCA3-9 & \bf{681.00} & 2.21 & 
681.17 & 2.31 & 681.00 & 0.00\\
SCA8-0 & 987.04 & 2.98 & 
989.33 & 3.03 & \bf{961.60} & 
2.65\\SCA8-1 & \bf{\underline{1051.96}} & 2.61 & 
1060.79 & 2.51 & 1063.00 & 
-1.04\\SCA8-2 & 1052.56 & 2.38 & 
1053.46 & 2.23 & \bf{1040.60} & 
1.15\\SCA8-3 & 1015.31 & 3.06 & 
1017.23 & 3.10 & \bf{985.90} & 
2.98\\SCA8-4 & \bf{\underline{1065.49}} & 2.92 & 
1075.02 & 2.98 & 1071.00 & 
-0.51\\SCA8-5 & \bf{\underline{1034.74}} & 3.24 & 
1049.22 & 3.26 & 1054.30 & 
-1.86\\SCA8-6 & 980.91 & 3.14 & 
981.45 & 3.14 & \bf{972.50} & 
0.86\\SCA8-7 & 1067.20 & 3.28 & 
1069.26 & 3.23 & \bf{1059.70} & 
0.71\\SCA8-8 & \bf{\underline{1071.18}} & 2.98 & 
1080.39 & 3.02 & 1082.70 & 
-1.06\\SCA8-9 & \bf{\underline{1067.42}} & 2.38 & 
1067.42 & 2.50 & 1081.40 & 
-1.29\\CON3-0 & 620.76 & 2.86 & 
622.04 & 3.07 & \bf{616.50} & 
0.69\\CON3-1 & \bf{\underline{554.47}} & 2.69 & 
556.50 & 2.89 & 555.60 & 
-0.20\\CON3-2 & \bf{\underline{521.38}} & 2.58 & 
522.16 & 2.79 & 521.40 & 
-0.00\\CON3-3 & \bf{\underline{591.19}} & 3.04 & 
591.24 & 3.10 & 591.20 & 
-0.00\\CON3-4 & \bf{\underline{588.79}} & 2.70 & 
590.24 & 2.71 & 589.30 & 
-0.09\\CON3-5 & \bf{563.70} & 2.72 & 
567.33 & 2.78 & 563.70 & 0.00\\
CON3-6 & 503.97 & 3.30 & 
504.11 & 3.33 & \bf{499.20} & 
0.96\\CON3-7 & 578.41 & 2.37 & 
580.52 & 2.38 & \bf{577.50} & 
0.16\\CON3-8 & 524.59 & 2.34 & 
527.64 & 2.42 & \bf{523.10} & 
0.28\\CON3-9 & 580.78 & 2.88 & 
586.00 & 2.71 & \bf{578.20} & 
0.45\\CON8-0 & 871.19 & 3.14 & 
880.78 & 3.01 & \bf{858.90} & 
1.43\\CON8-1 & \bf{\underline{740.85}} & 3.22 & 
745.87 & 2.96 & 740.90 & 
-0.01\\CON8-2 & \bf{\underline{713.90}} & 3.84 & 
715.38 & 3.75 & 714.30 & 
-0.06\\CON8-3 & \bf{\underline{811.07}} & 3.04 & 
815.95 & 2.90 & 812.30 & 
-0.15\\CON8-4 & 776.37 & 3.12 & 
784.60 & 2.95 & \bf{770.10} & 
0.81\\CON8-5 & \bf{\underline{759.93}} & 2.89 & 
762.75 & 2.81 & 766.60 & 
-0.87\\CON8-6 & \bf{\underline{688.51}} & 3.46 & 
691.51 & 3.52 & 697.20 & 
-1.25\\CON8-7 & 814.86 & 2.58 & 
816.71 & 2.57 & \bf{814.80} & 
0.01\\CON8-8 & 792.09 & 3.70 & 
793.33 & 3.40 & \bf{771.30} & 
2.70\\CON8-9 & \bf{\underline{812.60}} & 3.29 & 
815.30 & 3.41 & 815.10 & 
-0.31\\[1ex]\hline
\end{tabular}
\label{table:nonlin}
\end{table} \clearpage
\begin{table}[ht]
\caption{Resultados de la ejecución de la metaheurística ACO, utilizando instancias de Dethloff con la configuración -n 4.0 -alpha 1.0 -beta 3.0 -q .8 -ro 0.015}
\centering
\small
\begin{tabular}{c c c c c c c}
\hline\hline
Instancia & Costo mínimo & Tiempo(seg.) & Costo promedio & Tiempo promedio(seg.) & Costo ACO & \%Gap \\ [0.5ex]
\hline
SCA3-0 & \bf{\underline{636.06}} & 2.70 & 
637.46 & 2.67 & 636.10 & 
-0.01\\SCA3-1 & \bf{\underline{697.84}} & 2.90 & 
697.84 & 2.99 & 700.10 & 
-0.32\\SCA3-2 & 659.34 & 2.56 & 
662.21 & 2.60 & \bf{659.30} & 
0.01\\SCA3-3 & 680.04 & 2.59 & 
680.50 & 2.71 & \bf{680.00} & 
0.01\\SCA3-4 & \bf{690.50} & 2.84 & 
690.50 & 2.84 & 690.50 & 0.00\\
SCA3-5 & \bf{\underline{662.75}} & 2.88 & 
665.29 & 2.98 & 671.10 & 
-1.24\\SCA3-6 & 652.94 & 2.51 & 
653.49 & 2.61 & \bf{651.10} & 
0.28\\SCA3-7 & \bf{\underline{664.88}} & 2.42 & 
665.83 & 2.23 & 666.10 & 
-0.18\\SCA3-8 & \bf{\underline{719.47}} & 2.48 & 
726.32 & 2.49 & 719.50 & 
-0.00\\SCA3-9 & \bf{681.00} & 2.29 & 
681.00 & 2.28 & 681.00 & 0.00\\
SCA8-0 & \bf{\underline{961.50}} & 2.88 & 
978.28 & 3.02 & 961.60 & 
-0.01\\SCA8-1 & \bf{\underline{1060.23}} & 2.54 & 
1069.82 & 2.46 & 1063.00 & 
-0.26\\SCA8-2 & 1046.29 & 2.20 & 
1049.94 & 2.14 & \bf{1040.60} & 
0.55\\SCA8-3 & 1000.96 & 3.18 & 
1013.58 & 3.15 & \bf{985.90} & 
1.53\\SCA8-4 & \bf{\underline{1065.49}} & 3.08 & 
1067.16 & 3.10 & 1071.00 & 
-0.51\\SCA8-5 & \bf{\underline{1045.30}} & 3.29 & 
1052.40 & 3.30 & 1054.30 & 
-0.85\\SCA8-6 & 979.28 & 3.32 & 
980.63 & 3.17 & \bf{972.50} & 
0.70\\SCA8-7 & 1067.20 & 3.03 & 
1071.26 & 3.19 & \bf{1059.70} & 
0.71\\SCA8-8 & \bf{\underline{1071.18}} & 3.03 & 
1075.65 & 2.98 & 1082.70 & 
-1.06\\SCA8-9 & \bf{\underline{1067.42}} & 2.62 & 
1067.42 & 2.38 & 1081.40 & 
-1.29\\CON3-0 & 623.60 & 4.08 & 
624.51 & 3.61 & \bf{616.50} & 
1.15\\CON3-1 & \bf{\underline{554.47}} & 2.64 & 
556.12 & 2.84 & 555.60 & 
-0.20\\CON3-2 & \bf{\underline{521.38}} & 2.50 & 
523.18 & 2.49 & 521.40 & 
-0.00\\CON3-3 & \bf{\underline{591.19}} & 3.25 & 
591.20 & 3.09 & 591.20 & 
-0.00\\CON3-4 & \bf{\underline{588.79}} & 2.80 & 
590.77 & 2.72 & 589.30 & 
-0.09\\CON3-5 & 568.69 & 2.68 & 
568.74 & 2.81 & \bf{563.70} & 
0.89\\CON3-6 & 503.20 & 3.14 & 
505.21 & 3.37 & \bf{499.20} & 
0.80\\CON3-7 & 578.22 & 2.55 & 
578.32 & 2.52 & \bf{577.50} & 
0.12\\CON3-8 & 524.30 & 2.54 & 
527.64 & 2.56 & \bf{523.10} & 
0.23\\CON3-9 & 586.17 & 2.70 & 
588.30 & 2.67 & \bf{578.20} & 
1.38\\CON8-0 & 878.44 & 2.76 & 
883.77 & 2.98 & \bf{858.90} & 
2.28\\CON8-1 & 742.29 & 2.71 & 
748.62 & 2.94 & \bf{740.90} & 
0.19\\CON8-2 & \bf{\underline{713.44}} & 4.06 & 
715.10 & 3.94 & 714.30 & 
-0.12\\CON8-3 & 812.54 & 3.11 & 
816.31 & 2.89 & \bf{812.30} & 
0.03\\CON8-4 & 789.98 & 3.20 & 
790.82 & 2.89 & \bf{770.10} & 
2.58\\CON8-5 & \bf{\underline{760.62}} & 2.92 & 
763.69 & 2.80 & 766.60 & 
-0.78\\CON8-6 & \bf{\underline{693.86}} & 3.27 & 
696.69 & 3.35 & 697.20 & 
-0.48\\CON8-7 & \bf{\underline{814.79}} & 2.61 & 
818.96 & 2.48 & 814.80 & 
-0.00\\CON8-8 & 788.09 & 3.41 & 
790.25 & 3.22 & \bf{771.30} & 
2.18\\CON8-9 & \bf{\underline{810.18}} & 2.96 & 
813.78 & 3.18 & 815.10 & 
-0.60\\[1ex]\hline
\end{tabular}
\label{table:nonlin}
\end{table} \clearpage
\begin{table}[ht]
\caption{Resultados de la ejecución de la metaheurística ACO, utilizando instancias de Dethloff con la configuración -n 4.0 -alpha 1.0 -beta 3.0 -q .9 -ro 0.015}
\centering
\small
\begin{tabular}{c c c c c c c}
\hline\hline
Instancia & Costo mínimo & Tiempo(seg.) & Costo promedio & Tiempo promedio(seg.) & Costo ACO & \%Gap \\ [0.5ex]
\hline
SCA3-0 & \bf{\underline{636.06}} & 4.85 & 
638.30 & 3.31 & 636.10 & 
-0.01\\SCA3-1 & \bf{\underline{697.84}} & 2.98 & 
697.84 & 3.02 & 700.10 & 
-0.32\\SCA3-2 & 664.18 & 2.46 & 
664.18 & 2.55 & \bf{659.30} & 
0.74\\SCA3-3 & 680.60 & 2.84 & 
680.96 & 2.83 & \bf{680.00} & 
0.09\\SCA3-4 & \bf{690.50} & 2.83 & 
690.50 & 2.81 & 690.50 & 0.00\\
SCA3-5 & \bf{\underline{665.64}} & 2.72 & 
665.64 & 2.75 & 671.10 & 
-0.81\\SCA3-6 & 652.94 & 2.79 & 
654.70 & 2.77 & \bf{651.10} & 
0.28\\SCA3-7 & 666.15 & 2.17 & 
666.15 & 2.27 & \bf{666.10} & 
0.01\\SCA3-8 & 719.77 & 2.25 & 
723.31 & 2.40 & \bf{719.50} & 
0.04\\SCA3-9 & \bf{681.00} & 2.00 & 
681.00 & 2.12 & 681.00 & 0.00\\
SCA8-0 & 968.79 & 3.04 & 
978.79 & 3.00 & \bf{961.60} & 
0.75\\SCA8-1 & \bf{\underline{1053.24}} & 2.27 & 
1065.07 & 2.33 & 1063.00 & 
-0.92\\SCA8-2 & 1049.22 & 2.12 & 
1052.80 & 2.00 & \bf{1040.60} & 
0.83\\SCA8-3 & 1013.17 & 2.94 & 
1018.20 & 3.00 & \bf{985.90} & 
2.77\\SCA8-4 & \bf{\underline{1065.49}} & 3.08 & 
1069.45 & 3.04 & 1071.00 & 
-0.51\\SCA8-5 & 1055.35 & 3.52 & 
1056.25 & 3.35 & \bf{1054.30} & 
0.10\\SCA8-6 & \bf{\underline{972.48}} & 3.26 & 
977.65 & 3.20 & 972.50 & 
-0.00\\SCA8-7 & 1067.20 & 3.15 & 
1070.32 & 3.30 & \bf{1059.70} & 
0.71\\SCA8-8 & \bf{\underline{1071.18}} & 3.11 & 
1074.87 & 2.98 & 1082.70 & 
-1.06\\SCA8-9 & \bf{\underline{1067.42}} & 2.23 & 
1067.42 & 2.33 & 1081.40 & 
-1.29\\CON3-0 & 624.96 & 3.29 & 
624.96 & 3.26 & \bf{616.50} & 
1.37\\CON3-1 & \bf{\underline{554.47}} & 2.86 & 
556.85 & 2.94 & 555.60 & 
-0.20\\CON3-2 & \bf{\underline{521.38}} & 2.34 & 
523.70 & 2.35 & 521.40 & 
-0.00\\CON3-3 & \bf{591.20} & 3.02 & 
591.27 & 3.08 & 591.20 & 0.00\\
CON3-4 & \bf{\underline{588.79}} & 2.51 & 
589.72 & 2.67 & 589.30 & 
-0.09\\CON3-5 & 567.94 & 2.64 & 
569.16 & 2.71 & \bf{563.70} & 
0.75\\CON3-6 & 500.80 & 3.28 & 
502.77 & 3.41 & \bf{499.20} & 
0.32\\CON3-7 & 578.22 & 2.29 & 
578.36 & 2.38 & \bf{577.50} & 
0.12\\CON3-8 & 523.14 & 2.41 & 
524.96 & 2.37 & \bf{523.10} & 
0.01\\CON3-9 & 583.35 & 2.63 & 
587.44 & 2.56 & \bf{578.20} & 
0.89\\CON8-0 & 879.00 & 2.73 & 
883.36 & 2.88 & \bf{858.90} & 
2.34\\CON8-1 & \bf{\underline{740.85}} & 2.91 & 
744.94 & 2.76 & 740.90 & 
-0.01\\CON8-2 & \bf{\underline{713.90}} & 3.83 & 
716.13 & 3.83 & 714.30 & 
-0.06\\CON8-3 & 812.54 & 2.84 & 
816.23 & 2.95 & \bf{812.30} & 
0.03\\CON8-4 & 781.64 & 3.13 & 
784.82 & 3.01 & \bf{770.10} & 
1.50\\CON8-5 & \bf{\underline{759.93}} & 2.72 & 
763.52 & 2.72 & 766.60 & 
-0.87\\CON8-6 & \bf{\underline{686.85}} & 3.36 & 
694.28 & 3.22 & 697.20 & 
-1.48\\CON8-7 & \bf{\underline{814.79}} & 2.56 & 
821.02 & 2.41 & 814.80 & 
-0.00\\CON8-8 & 784.74 & 3.17 & 
791.41 & 3.19 & \bf{771.30} & 
1.74\\CON8-9 & \bf{\underline{810.18}} & 3.24 & 
814.38 & 3.15 & 815.10 & 
-0.60\\[1ex]\hline
\end{tabular}
\label{table:nonlin}
\end{table} \clearpage
\begin{table}[ht]
\caption{Resultados de la ejecución de la metaheurística ACO, utilizando instancias de Dethloff con la configuración -n 5.0 -alpha 1.0 -beta 3.0 -q 0.1 -ro 0.015}
\centering
\small
\begin{tabular}{c c c c c c c}
\hline\hline
Instancia & Costo mínimo & Tiempo(seg.) & Costo promedio & Tiempo promedio(seg.) & Costo ACO & \%Gap \\ [0.5ex]
\hline
SCA3-0 & \bf{\underline{636.06}} & 3.38 & 
636.13 & 3.51 & 636.10 & 
-0.01\\SCA3-1 & \bf{\underline{697.84}} & 3.82 & 
697.84 & 3.92 & 700.10 & 
-0.32\\SCA3-2 & 659.34 & 3.50 & 
659.79 & 3.52 & \bf{659.30} & 
0.01\\SCA3-3 & 680.04 & 3.26 & 
680.04 & 3.59 & \bf{680.00} & 
0.01\\SCA3-4 & \bf{690.50} & 3.64 & 
690.50 & 3.73 & 690.50 & 0.00\\
SCA3-5 & \bf{\underline{659.90}} & 3.54 & 
664.73 & 3.66 & 671.10 & 
-1.67\\SCA3-6 & \bf{\underline{651.09}} & 4.07 & 
652.01 & 3.73 & 651.10 & 
-0.00\\SCA3-7 & \bf{\underline{664.88}} & 3.41 & 
666.11 & 3.41 & 666.10 & 
-0.18\\SCA3-8 & \bf{\underline{719.47}} & 3.74 & 
719.62 & 3.80 & 719.50 & 
-0.00\\SCA3-9 & \bf{681.00} & 3.31 & 
681.59 & 3.23 & 681.00 & 0.00\\
SCA8-0 & 968.79 & 4.09 & 
974.18 & 3.86 & \bf{961.60} & 
0.75\\SCA8-1 & \bf{\underline{1053.44}} & 3.17 & 
1058.36 & 3.36 & 1063.00 & 
-0.90\\SCA8-2 & 1046.29 & 3.25 & 
1049.30 & 3.41 & \bf{1040.60} & 
0.55\\SCA8-3 & 997.48 & 3.77 & 
1005.78 & 3.62 & \bf{985.90} & 
1.17\\SCA8-4 & \bf{\underline{1067.55}} & 3.86 & 
1069.46 & 3.78 & 1071.00 & 
-0.32\\SCA8-5 & \bf{\underline{1034.74}} & 4.23 & 
1041.65 & 4.76 & 1054.30 & 
-1.86\\SCA8-6 & 980.91 & 4.20 & 
980.91 & 4.98 & \bf{972.50} & 
0.86\\SCA8-7 & 1066.65 & 4.01 & 
1067.99 & 4.11 & \bf{1059.70} & 
0.66\\SCA8-8 & \bf{\underline{1071.18}} & 4.00 & 
1074.49 & 4.12 & 1082.70 & 
-1.06\\SCA8-9 & \bf{\underline{1067.42}} & 3.38 & 
1067.42 & 3.41 & 1081.40 & 
-1.29\\CON3-0 & 617.59 & 4.31 & 
619.17 & 4.03 & \bf{616.50} & 
0.18\\CON3-1 & \bf{\underline{554.47}} & 3.96 & 
555.15 & 3.85 & 555.60 & 
-0.20\\CON3-2 & \bf{\underline{519.11}} & 3.64 & 
520.25 & 3.69 & 521.40 & 
-0.44\\CON3-3 & \bf{\underline{591.19}} & 4.28 & 
591.24 & 4.01 & 591.20 & 
-0.00\\CON3-4 & \bf{\underline{588.79}} & 3.16 & 
589.58 & 3.36 & 589.30 & 
-0.09\\CON3-5 & 564.88 & 3.68 & 
565.64 & 3.75 & \bf{563.70} & 
0.21\\CON3-6 & 500.80 & 4.02 & 
501.49 & 4.32 & \bf{499.20} & 
0.32\\CON3-7 & 577.54 & 4.48 & 
578.14 & 3.64 & \bf{577.50} & 
0.01\\CON3-8 & \bf{\underline{523.05}} & 3.72 & 
523.10 & 3.43 & 523.10 & 
-0.01\\CON3-9 & 584.05 & 3.21 & 
586.96 & 3.48 & \bf{578.20} & 
1.01\\CON8-0 & 861.35 & 3.91 & 
876.04 & 4.00 & \bf{858.90} & 
0.29\\CON8-1 & 740.93 & 3.87 & 
741.95 & 3.92 & \bf{740.90} & 
0.00\\CON8-2 & \bf{\underline{713.21}} & 4.58 & 
714.34 & 4.54 & 714.30 & 
-0.15\\CON8-3 & \bf{\underline{811.07}} & 3.86 & 
814.00 & 3.96 & 812.30 & 
-0.15\\CON8-4 & 776.34 & 3.61 & 
783.93 & 3.70 & \bf{770.10} & 
0.81\\CON8-5 & \bf{\underline{758.12}} & 3.95 & 
759.02 & 3.90 & 766.60 & 
-1.11\\CON8-6 & \bf{\underline{693.10}} & 4.49 & 
696.34 & 4.38 & 697.20 & 
-0.59\\CON8-7 & \bf{\underline{814.79}} & 3.55 & 
815.13 & 3.34 & 814.80 & 
-0.00\\CON8-8 & 778.34 & 4.78 & 
782.71 & 4.63 & \bf{771.30} & 
0.91\\CON8-9 & \bf{\underline{812.60}} & 4.06 & 
815.19 & 4.17 & 815.10 & 
-0.31\\[1ex]\hline
\end{tabular}
\label{table:nonlin}
\end{table} \clearpage
\begin{table}[ht]
\caption{Resultados de la ejecución de la metaheurística ACO, utilizando instancias de Dethloff con la configuración -n 5.0 -alpha 1.0 -beta 3.0 -q .2 -ro 0.015}
\centering
\small
\begin{tabular}{c c c c c c c}
\hline\hline
Instancia & Costo mínimo & Tiempo(seg.) & Costo promedio & Tiempo promedio(seg.) & Costo ACO & \%Gap \\ [0.5ex]
\hline
SCA3-0 & \bf{\underline{636.06}} & 3.48 & 
636.06 & 3.60 & 636.10 & 
-0.01\\SCA3-1 & \bf{\underline{697.84}} & 4.18 & 
697.84 & 3.96 & 700.10 & 
-0.32\\SCA3-2 & 659.34 & 3.58 & 
659.79 & 3.69 & \bf{659.30} & 
0.01\\SCA3-3 & 680.04 & 3.56 & 
680.04 & 3.41 & \bf{680.00} & 
0.01\\SCA3-4 & \bf{690.50} & 3.60 & 
690.50 & 4.06 & 690.50 & 0.00\\
SCA3-5 & \bf{\underline{662.75}} & 3.82 & 
666.13 & 3.79 & 671.10 & 
-1.24\\SCA3-6 & \bf{\underline{651.09}} & 3.97 & 
652.48 & 3.71 & 651.10 & 
-0.00\\SCA3-7 & 666.15 & 3.20 & 
666.15 & 3.20 & \bf{666.10} & 
0.01\\SCA3-8 & \bf{\underline{719.47}} & 3.94 & 
721.97 & 3.69 & 719.50 & 
-0.00\\SCA3-9 & \bf{681.00} & 3.02 & 
681.00 & 3.06 & 681.00 & 0.00\\
SCA8-0 & 973.03 & 3.67 & 
976.30 & 3.98 & \bf{961.60} & 
1.19\\SCA8-1 & \bf{\underline{1049.65}} & 3.62 & 
1056.76 & 3.41 & 1063.00 & 
-1.26\\SCA8-2 & 1047.63 & 2.84 & 
1049.36 & 2.98 & \bf{1040.60} & 
0.68\\SCA8-3 & 991.84 & 3.62 & 
1008.70 & 3.59 & \bf{985.90} & 
0.60\\SCA8-4 & \bf{\underline{1065.49}} & 3.94 & 
1067.57 & 4.00 & 1071.00 & 
-0.51\\SCA8-5 & \bf{\underline{1034.74}} & 4.24 & 
1045.12 & 4.32 & 1054.30 & 
-1.86\\SCA8-6 & \bf{\underline{972.48}} & 4.12 & 
976.71 & 4.22 & 972.50 & 
-0.00\\SCA8-7 & 1067.20 & 4.18 & 
1071.22 & 4.10 & \bf{1059.70} & 
0.71\\SCA8-8 & \bf{\underline{1071.18}} & 4.35 & 
1071.18 & 4.17 & 1082.70 & 
-1.06\\SCA8-9 & \bf{\underline{1067.42}} & 3.18 & 
1067.42 & 3.26 & 1081.40 & 
-1.29\\CON3-0 & 617.59 & 4.02 & 
619.02 & 3.95 & \bf{616.50} & 
0.18\\CON3-1 & 556.79 & 4.22 & 
557.19 & 3.86 & \bf{555.60} & 
0.21\\CON3-2 & \bf{\underline{521.38}} & 3.92 & 
521.38 & 3.72 & 521.40 & 
-0.00\\CON3-3 & \bf{591.20} & 4.19 & 
591.27 & 4.02 & 591.20 & 0.00\\
CON3-4 & \bf{\underline{588.79}} & 3.32 & 
589.45 & 3.44 & 589.30 & 
-0.09\\CON3-5 & \bf{563.70} & 3.58 & 
566.23 & 3.65 & 563.70 & 0.00\\
CON3-6 & \bf{\underline{499.05}} & 3.95 & 
502.88 & 4.16 & 499.20 & 
-0.03\\CON3-7 & \bf{\underline{576.48}} & 3.15 & 
579.40 & 3.18 & 577.50 & 
-0.18\\CON3-8 & 523.14 & 3.64 & 
524.00 & 3.60 & \bf{523.10} & 
0.01\\CON3-9 & 578.98 & 3.46 & 
585.22 & 3.55 & \bf{578.20} & 
0.13\\CON8-0 & 866.22 & 3.83 & 
875.32 & 3.83 & \bf{858.90} & 
0.85\\CON8-1 & \bf{\underline{740.85}} & 3.78 & 
741.63 & 3.77 & 740.90 & 
-0.01\\CON8-2 & \bf{\underline{712.89}} & 4.52 & 
714.47 & 4.61 & 714.30 & 
-0.20\\CON8-3 & \bf{\underline{811.07}} & 4.06 & 
812.19 & 3.85 & 812.30 & 
-0.15\\CON8-4 & 779.41 & 3.84 & 
784.02 & 3.81 & \bf{770.10} & 
1.21\\CON8-5 & \bf{\underline{754.95}} & 3.66 & 
758.61 & 3.60 & 766.60 & 
-1.52\\CON8-6 & \bf{\underline{691.20}} & 4.48 & 
696.57 & 4.27 & 697.20 & 
-0.86\\CON8-7 & \bf{\underline{814.79}} & 3.17 & 
818.47 & 3.36 & 814.80 & 
-0.00\\CON8-8 & 783.15 & 4.20 & 
784.92 & 4.14 & \bf{771.30} & 
1.54\\CON8-9 & \bf{\underline{811.75}} & 4.36 & 
813.13 & 4.24 & 815.10 & 
-0.41\\[1ex]\hline
\end{tabular}
\label{table:nonlin}
\end{table} \clearpage
\begin{table}[ht]
\caption{Resultados de la ejecución de la metaheurística ACO, utilizando instancias de Dethloff con la configuración -n 5.0 -alpha 1.0 -beta 3.0 -q .3 -ro 0.015}
\centering
\small
\begin{tabular}{c c c c c c c}
\hline\hline
Instancia & Costo mínimo & Tiempo(seg.) & Costo promedio & Tiempo promedio(seg.) & Costo ACO & \%Gap \\ [0.5ex]
\hline
SCA3-0 & \bf{\underline{636.06}} & 3.52 & 
636.13 & 3.42 & 636.10 & 
-0.01\\SCA3-1 & \bf{\underline{697.84}} & 3.85 & 
697.84 & 3.85 & 700.10 & 
-0.32\\SCA3-2 & 659.34 & 3.57 & 
661.77 & 3.54 & \bf{659.30} & 
0.01\\SCA3-3 & 680.04 & 3.42 & 
680.04 & 3.50 & \bf{680.00} & 
0.01\\SCA3-4 & \bf{690.50} & 3.58 & 
690.50 & 3.78 & 690.50 & 0.00\\
SCA3-5 & \bf{\underline{659.90}} & 3.67 & 
662.76 & 3.84 & 671.10 & 
-1.67\\SCA3-6 & \bf{\underline{651.09}} & 3.53 & 
652.48 & 3.62 & 651.10 & 
-0.00\\SCA3-7 & 666.15 & 3.14 & 
667.09 & 3.05 & \bf{666.10} & 
0.01\\SCA3-8 & \bf{\underline{719.47}} & 3.80 & 
720.61 & 3.69 & 719.50 & 
-0.00\\SCA3-9 & \bf{681.00} & 2.93 & 
681.00 & 3.04 & 681.00 & 0.00\\
SCA8-0 & \bf{\underline{961.50}} & 3.75 & 
969.59 & 3.94 & 961.60 & 
-0.01\\SCA8-1 & \bf{\underline{1056.70}} & 3.34 & 
1061.04 & 3.27 & 1063.00 & 
-0.59\\SCA8-2 & 1046.29 & 3.11 & 
1049.35 & 3.06 & \bf{1040.60} & 
0.55\\SCA8-3 & 1015.17 & 4.01 & 
1016.97 & 3.74 & \bf{985.90} & 
2.97\\SCA8-4 & \bf{\underline{1067.28}} & 3.78 & 
1068.05 & 3.76 & 1071.00 & 
-0.35\\SCA8-5 & \bf{\underline{1045.30}} & 4.25 & 
1052.74 & 4.37 & 1054.30 & 
-0.85\\SCA8-6 & 977.87 & 4.15 & 
979.87 & 4.00 & \bf{972.50} & 
0.55\\SCA8-7 & 1065.61 & 4.10 & 
1071.63 & 4.20 & \bf{1059.70} & 
0.56\\SCA8-8 & \bf{\underline{1071.18}} & 4.10 & 
1073.91 & 4.03 & 1082.70 & 
-1.06\\SCA8-9 & \bf{\underline{1067.42}} & 3.34 & 
1067.42 & 3.29 & 1081.40 & 
-1.29\\CON3-0 & 616.52 & 3.78 & 
619.69 & 3.88 & \bf{616.50} & 
0.00\\CON3-1 & \bf{\underline{554.47}} & 3.93 & 
556.76 & 3.90 & 555.60 & 
-0.20\\CON3-2 & \bf{\underline{519.11}} & 3.66 & 
520.88 & 3.49 & 521.40 & 
-0.44\\CON3-3 & \bf{\underline{591.19}} & 3.98 & 
591.19 & 3.99 & 591.20 & 
-0.00\\CON3-4 & \bf{\underline{588.79}} & 3.47 & 
589.05 & 3.46 & 589.30 & 
-0.09\\CON3-5 & \bf{563.70} & 3.62 & 
565.54 & 3.56 & 563.70 & 0.00\\
CON3-6 & 500.80 & 4.39 & 
502.39 & 4.33 & \bf{499.20} & 
0.32\\CON3-7 & 578.22 & 3.42 & 
578.32 & 3.28 & \bf{577.50} & 
0.12\\CON3-8 & \bf{\underline{523.05}} & 3.31 & 
523.44 & 3.36 & 523.10 & 
-0.01\\CON3-9 & 578.25 & 3.72 & 
581.98 & 3.58 & \bf{578.20} & 
0.01\\CON8-0 & 866.43 & 3.87 & 
870.64 & 3.77 & \bf{858.90} & 
0.88\\CON8-1 & \bf{\underline{740.85}} & 3.83 & 
743.63 & 3.86 & 740.90 & 
-0.01\\CON8-2 & \bf{\underline{713.44}} & 4.60 & 
714.92 & 4.50 & 714.30 & 
-0.12\\CON8-3 & \bf{\underline{811.07}} & 3.55 & 
814.69 & 3.75 & 812.30 & 
-0.15\\CON8-4 & 772.25 & 3.45 & 
781.84 & 3.59 & \bf{770.10} & 
0.28\\CON8-5 & \bf{\underline{758.99}} & 3.64 & 
762.01 & 3.65 & 766.60 & 
-0.99\\CON8-6 & \bf{\underline{689.23}} & 4.31 & 
693.69 & 4.37 & 697.20 & 
-1.14\\CON8-7 & \bf{\underline{814.79}} & 3.74 & 
816.43 & 3.39 & 814.80 & 
-0.00\\CON8-8 & 778.22 & 4.16 & 
784.75 & 4.18 & \bf{771.30} & 
0.90\\CON8-9 & \bf{\underline{812.60}} & 4.07 & 
815.73 & 4.12 & 815.10 & 
-0.31\\[1ex]\hline
\end{tabular}
\label{table:nonlin}
\end{table} \clearpage
\begin{table}[ht]
\caption{Resultados de la ejecución de la metaheurística ACO, utilizando instancias de Dethloff con la configuración -n 5.0 -alpha 1.0 -beta 3.0 -q .4 -ro 0.015}
\centering
\small
\begin{tabular}{c c c c c c c}
\hline\hline
Instancia & Costo mínimo & Tiempo(seg.) & Costo promedio & Tiempo promedio(seg.) & Costo ACO & \%Gap \\ [0.5ex]
\hline
SCA3-0 & \bf{\underline{636.06}} & 3.47 & 
636.06 & 3.48 & 636.10 & 
-0.01\\SCA3-1 & \bf{\underline{697.84}} & 3.72 & 
697.84 & 3.73 & 700.10 & 
-0.32\\SCA3-2 & 659.34 & 3.58 & 
661.45 & 3.46 & \bf{659.30} & 
0.01\\SCA3-3 & 680.04 & 3.61 & 
680.04 & 3.60 & \bf{680.00} & 
0.01\\SCA3-4 & \bf{690.50} & 3.94 & 
690.50 & 3.79 & 690.50 & 0.00\\
SCA3-5 & \bf{\underline{659.90}} & 3.83 & 
660.61 & 3.67 & 671.10 & 
-1.67\\SCA3-6 & \bf{\underline{651.09}} & 3.23 & 
652.48 & 3.41 & 651.10 & 
-0.00\\SCA3-7 & \bf{\underline{664.88}} & 3.06 & 
665.83 & 3.08 & 666.10 & 
-0.18\\SCA3-8 & \bf{\underline{719.47}} & 3.46 & 
721.21 & 3.52 & 719.50 & 
-0.00\\SCA3-9 & \bf{681.00} & 2.94 & 
681.17 & 2.92 & 681.00 & 0.00\\
SCA8-0 & 968.79 & 3.76 & 
984.28 & 3.95 & \bf{961.60} & 
0.75\\SCA8-1 & \bf{\underline{1054.87}} & 2.98 & 
1062.53 & 3.14 & 1063.00 & 
-0.76\\SCA8-2 & 1046.29 & 2.90 & 
1049.35 & 2.88 & \bf{1040.60} & 
0.55\\SCA8-3 & \bf{\underline{983.34}} & 3.87 & 
1002.51 & 3.74 & 985.90 & 
-0.26\\SCA8-4 & \bf{\underline{1067.66}} & 3.42 & 
1068.73 & 3.62 & 1071.00 & 
-0.31\\SCA8-5 & \bf{\underline{1034.74}} & 4.32 & 
1047.76 & 4.26 & 1054.30 & 
-1.86\\SCA8-6 & 977.03 & 4.08 & 
980.36 & 4.08 & \bf{972.50} & 
0.47\\SCA8-7 & 1067.03 & 3.94 & 
1069.62 & 3.94 & \bf{1059.70} & 
0.69\\SCA8-8 & \bf{\underline{1071.18}} & 3.91 & 
1071.18 & 4.00 & 1082.70 & 
-1.06\\SCA8-9 & \bf{\underline{1067.42}} & 3.74 & 
1067.42 & 3.16 & 1081.40 & 
-1.29\\CON3-0 & 617.59 & 4.10 & 
621.09 & 3.95 & \bf{616.50} & 
0.18\\CON3-1 & 556.04 & 3.73 & 
557.39 & 3.71 & \bf{555.60} & 
0.08\\CON3-2 & \bf{\underline{519.11}} & 3.57 & 
520.90 & 3.61 & 521.40 & 
-0.44\\CON3-3 & \bf{\underline{591.19}} & 4.10 & 
591.19 & 3.92 & 591.20 & 
-0.00\\CON3-4 & \bf{\underline{588.79}} & 3.37 & 
590.24 & 3.26 & 589.30 & 
-0.09\\CON3-5 & 564.88 & 3.67 & 
566.79 & 3.63 & \bf{563.70} & 
0.21\\CON3-6 & 500.80 & 4.23 & 
502.13 & 4.21 & \bf{499.20} & 
0.32\\CON3-7 & 577.68 & 3.24 & 
578.89 & 3.31 & \bf{577.50} & 
0.03\\CON3-8 & \bf{\underline{523.05}} & 3.28 & 
524.06 & 3.36 & 523.10 & 
-0.01\\CON3-9 & 578.98 & 3.67 & 
584.51 & 3.64 & \bf{578.20} & 
0.13\\CON8-0 & 870.90 & 3.94 & 
872.29 & 3.94 & \bf{858.90} & 
1.40\\CON8-1 & \bf{\underline{740.85}} & 3.78 & 
742.69 & 3.79 & 740.90 & 
-0.01\\CON8-2 & \bf{\underline{713.44}} & 4.48 & 
714.19 & 4.52 & 714.30 & 
-0.12\\CON8-3 & \bf{\underline{811.07}} & 3.71 & 
814.57 & 3.73 & 812.30 & 
-0.15\\CON8-4 & 776.37 & 3.64 & 
780.99 & 3.64 & \bf{770.10} & 
0.81\\CON8-5 & \bf{\underline{754.95}} & 3.42 & 
759.46 & 3.54 & 766.60 & 
-1.52\\CON8-6 & \bf{\underline{688.00}} & 4.06 & 
692.85 & 4.04 & 697.20 & 
-1.32\\CON8-7 & \bf{\underline{814.77}} & 3.31 & 
816.39 & 3.33 & 814.80 & 
-0.00\\CON8-8 & 784.28 & 4.29 & 
787.47 & 4.24 & \bf{771.30} & 
1.68\\CON8-9 & \bf{\underline{810.18}} & 4.10 & 
812.71 & 4.03 & 815.10 & 
-0.60\\[1ex]\hline
\end{tabular}
\label{table:nonlin}
\end{table} \clearpage
\begin{table}[ht]
\caption{Resultados de la ejecución de la metaheurística ACO, utilizando instancias de Dethloff con la configuración -n 5.0 -alpha 1.0 -beta 3.0 -q .5 -ro 0.015}
\centering
\small
\begin{tabular}{c c c c c c c}
\hline\hline
Instancia & Costo mínimo & Tiempo(seg.) & Costo promedio & Tiempo promedio(seg.) & Costo ACO & \%Gap \\ [0.5ex]
\hline
SCA3-0 & \bf{\underline{636.06}} & 3.71 & 
636.06 & 3.58 & 636.10 & 
-0.01\\SCA3-1 & \bf{\underline{697.84}} & 3.71 & 
697.84 & 3.80 & 700.10 & 
-0.32\\SCA3-2 & 659.34 & 3.62 & 
659.79 & 3.41 & \bf{659.30} & 
0.01\\SCA3-3 & 680.04 & 3.29 & 
680.04 & 3.45 & \bf{680.00} & 
0.01\\SCA3-4 & \bf{690.50} & 3.50 & 
690.50 & 3.63 & 690.50 & 0.00\\
SCA3-5 & \bf{\underline{659.90}} & 3.79 & 
663.74 & 3.85 & 671.10 & 
-1.67\\SCA3-6 & \bf{\underline{651.09}} & 3.72 & 
653.01 & 3.48 & 651.10 & 
-0.00\\SCA3-7 & 666.15 & 3.04 & 
666.15 & 3.20 & \bf{666.10} & 
0.01\\SCA3-8 & \bf{\underline{719.47}} & 3.28 & 
719.54 & 3.36 & 719.50 & 
-0.00\\SCA3-9 & \bf{681.00} & 2.84 & 
681.00 & 2.91 & 681.00 & 0.00\\
SCA8-0 & \bf{\underline{961.50}} & 3.69 & 
965.82 & 3.83 & 961.60 & 
-0.01\\SCA8-1 & \bf{\underline{1062.39}} & 3.21 & 
1064.16 & 3.19 & 1063.00 & 
-0.06\\SCA8-2 & 1046.29 & 2.78 & 
1049.51 & 2.83 & \bf{1040.60} & 
0.55\\SCA8-3 & 998.98 & 3.72 & 
1009.93 & 3.77 & \bf{985.90} & 
1.33\\SCA8-4 & \bf{\underline{1065.49}} & 3.86 & 
1066.48 & 3.81 & 1071.00 & 
-0.51\\SCA8-5 & \bf{\underline{1039.12}} & 4.08 & 
1050.44 & 4.17 & 1054.30 & 
-1.44\\SCA8-6 & \bf{\underline{972.48}} & 3.83 & 
979.05 & 3.83 & 972.50 & 
-0.00\\SCA8-7 & 1067.20 & 4.08 & 
1072.89 & 4.01 & \bf{1059.70} & 
0.71\\SCA8-8 & \bf{\underline{1071.18}} & 3.90 & 
1076.64 & 3.89 & 1082.70 & 
-1.06\\SCA8-9 & \bf{\underline{1067.42}} & 3.12 & 
1067.42 & 3.09 & 1081.40 & 
-1.29\\CON3-0 & 619.86 & 3.90 & 
623.38 & 3.93 & \bf{616.50} & 
0.55\\CON3-1 & \bf{\underline{554.47}} & 3.96 & 
555.92 & 3.70 & 555.60 & 
-0.20\\CON3-2 & \bf{\underline{521.38}} & 3.37 & 
521.44 & 3.39 & 521.40 & 
-0.00\\CON3-3 & \bf{\underline{591.19}} & 4.04 & 
591.92 & 3.91 & 591.20 & 
-0.00\\CON3-4 & \bf{\underline{588.79}} & 3.49 & 
589.87 & 3.35 & 589.30 & 
-0.09\\CON3-5 & 564.88 & 3.64 & 
565.40 & 3.62 & \bf{563.70} & 
0.21\\CON3-6 & 502.16 & 4.26 & 
503.30 & 4.10 & \bf{499.20} & 
0.59\\CON3-7 & \bf{\underline{576.84}} & 3.08 & 
577.97 & 3.23 & 577.50 & 
-0.11\\CON3-8 & \bf{\underline{523.05}} & 3.45 & 
524.20 & 3.29 & 523.10 & 
-0.01\\CON3-9 & 583.80 & 3.22 & 
586.25 & 3.30 & \bf{578.20} & 
0.97\\CON8-0 & 860.28 & 3.65 & 
871.73 & 3.63 & \bf{858.90} & 
0.16\\CON8-1 & \bf{\underline{740.85}} & 3.82 & 
745.08 & 3.73 & 740.90 & 
-0.01\\CON8-2 & \bf{\underline{713.44}} & 5.18 & 
715.01 & 4.89 & 714.30 & 
-0.12\\CON8-3 & \bf{\underline{812.11}} & 3.50 & 
814.86 & 3.56 & 812.30 & 
-0.02\\CON8-4 & 777.24 & 3.50 & 
786.97 & 3.48 & \bf{770.10} & 
0.93\\CON8-5 & \bf{\underline{759.93}} & 3.48 & 
760.66 & 3.63 & 766.60 & 
-0.87\\CON8-6 & \bf{\underline{683.83}} & 4.13 & 
691.83 & 4.20 & 697.20 & 
-1.92\\CON8-7 & \bf{\underline{814.79}} & 3.42 & 
818.75 & 3.24 & 814.80 & 
-0.00\\CON8-8 & 780.78 & 4.04 & 
783.69 & 4.21 & \bf{771.30} & 
1.23\\CON8-9 & \bf{\underline{811.18}} & 4.19 & 
812.69 & 4.19 & 815.10 & 
-0.48\\[1ex]\hline
\end{tabular}
\label{table:nonlin}
\end{table} \clearpage
\begin{table}[ht]
\caption{Resultados de la ejecución de la metaheurística ACO, utilizando instancias de Dethloff con la configuración -n 5.0 -alpha 1.0 -beta 3.0 -q .6 -ro 0.015}
\centering
\small
\begin{tabular}{c c c c c c c}
\hline\hline
Instancia & Costo mínimo & Tiempo(seg.) & Costo promedio & Tiempo promedio(seg.) & Costo ACO & \%Gap \\ [0.5ex]
\hline
SCA3-0 & \bf{\underline{636.06}} & 3.30 & 
636.20 & 3.35 & 636.10 & 
-0.01\\SCA3-1 & \bf{\underline{697.84}} & 3.91 & 
697.84 & 3.81 & 700.10 & 
-0.32\\SCA3-2 & 659.34 & 3.36 & 
661.76 & 3.31 & \bf{659.30} & 
0.01\\SCA3-3 & 680.04 & 3.71 & 
680.32 & 3.48 & \bf{680.00} & 
0.01\\SCA3-4 & \bf{690.50} & 3.94 & 
690.50 & 3.65 & 690.50 & 0.00\\
SCA3-5 & \bf{\underline{659.90}} & 3.69 & 
663.48 & 3.80 & 671.10 & 
-1.67\\SCA3-6 & \bf{\underline{651.09}} & 3.40 & 
652.48 & 3.39 & 651.10 & 
-0.00\\SCA3-7 & \bf{\underline{664.88}} & 3.03 & 
665.83 & 2.87 & 666.10 & 
-0.18\\SCA3-8 & \bf{\underline{719.47}} & 3.46 & 
720.34 & 3.48 & 719.50 & 
-0.00\\SCA3-9 & \bf{681.00} & 2.82 & 
681.00 & 2.94 & 681.00 & 0.00\\
SCA8-0 & \bf{\underline{961.50}} & 3.85 & 
974.65 & 3.78 & 961.60 & 
-0.01\\SCA8-1 & 1063.12 & 3.25 & 
1066.62 & 3.15 & \bf{1063.00} & 
0.01\\SCA8-2 & 1046.29 & 2.85 & 
1049.51 & 2.69 & \bf{1040.60} & 
0.55\\SCA8-3 & 1016.59 & 3.82 & 
1017.12 & 3.83 & \bf{985.90} & 
3.11\\SCA8-4 & \bf{\underline{1065.49}} & 4.02 & 
1067.02 & 3.86 & 1071.00 & 
-0.51\\SCA8-5 & \bf{\underline{1039.12}} & 4.16 & 
1047.50 & 4.26 & 1054.30 & 
-1.44\\SCA8-6 & \bf{\underline{972.48}} & 3.99 & 
978.93 & 3.93 & 972.50 & 
-0.00\\SCA8-7 & 1067.20 & 4.42 & 
1071.38 & 4.14 & \bf{1059.70} & 
0.71\\SCA8-8 & \bf{\underline{1071.18}} & 4.04 & 
1073.57 & 3.88 & 1082.70 & 
-1.06\\SCA8-9 & \bf{\underline{1067.42}} & 2.93 & 
1067.42 & 3.06 & 1081.40 & 
-1.29\\CON3-0 & 616.52 & 4.11 & 
620.16 & 4.01 & \bf{616.50} & 
0.00\\CON3-1 & \bf{\underline{554.47}} & 3.67 & 
557.05 & 3.71 & 555.60 & 
-0.20\\CON3-2 & \bf{\underline{519.11}} & 3.48 & 
520.81 & 3.50 & 521.40 & 
-0.44\\CON3-3 & \bf{591.20} & 3.79 & 
591.20 & 3.98 & 591.20 & 0.00\\
CON3-4 & \bf{\underline{588.79}} & 3.44 & 
589.87 & 3.37 & 589.30 & 
-0.09\\CON3-5 & \bf{563.70} & 3.46 & 
566.08 & 4.27 & 563.70 & 0.00\\
CON3-6 & 500.80 & 4.26 & 
502.43 & 4.25 & \bf{499.20} & 
0.32\\CON3-7 & 578.22 & 3.10 & 
579.10 & 3.22 & \bf{577.50} & 
0.12\\CON3-8 & \bf{\underline{523.05}} & 3.22 & 
523.67 & 3.22 & 523.10 & 
-0.01\\CON3-9 & 588.48 & 3.42 & 
588.48 & 3.27 & \bf{578.20} & 
1.78\\CON8-0 & 869.43 & 3.80 & 
876.42 & 3.82 & \bf{858.90} & 
1.23\\CON8-1 & \bf{\underline{740.85}} & 3.59 & 
745.02 & 3.65 & 740.90 & 
-0.01\\CON8-2 & \bf{\underline{713.44}} & 4.92 & 
715.02 & 4.96 & 714.30 & 
-0.12\\CON8-3 & 817.22 & 3.71 & 
817.48 & 3.64 & \bf{812.30} & 
0.61\\CON8-4 & 782.89 & 3.79 & 
787.77 & 3.62 & \bf{770.10} & 
1.66\\CON8-5 & \bf{\underline{758.12}} & 3.78 & 
762.65 & 3.62 & 766.60 & 
-1.11\\CON8-6 & \bf{\underline{696.30}} & 4.08 & 
697.36 & 4.13 & 697.20 & 
-0.13\\CON8-7 & 814.86 & 3.47 & 
815.78 & 3.19 & \bf{814.80} & 
0.01\\CON8-8 & 783.88 & 4.20 & 
786.05 & 4.17 & \bf{771.30} & 
1.63\\CON8-9 & \bf{\underline{812.60}} & 4.27 & 
814.64 & 4.06 & 815.10 & 
-0.31\\[1ex]\hline
\end{tabular}
\label{table:nonlin}
\end{table} \clearpage
\begin{table}[ht]
\caption{Resultados de la ejecución de la metaheurística ACO, utilizando instancias de Dethloff con la configuración -n 5.0 -alpha 1.0 -beta 3.0 -q .7 -ro 0.015}
\centering
\small
\begin{tabular}{c c c c c c c}
\hline\hline
Instancia & Costo mínimo & Tiempo(seg.) & Costo promedio & Tiempo promedio(seg.) & Costo ACO & \%Gap \\ [0.5ex]
\hline
SCA3-0 & \bf{\underline{636.06}} & 3.36 & 
638.66 & 3.41 & 636.10 & 
-0.01\\SCA3-1 & \bf{\underline{697.84}} & 3.84 & 
697.84 & 3.93 & 700.10 & 
-0.32\\SCA3-2 & 659.34 & 3.20 & 
662.21 & 3.44 & \bf{659.30} & 
0.01\\SCA3-3 & 680.04 & 3.40 & 
680.32 & 3.40 & \bf{680.00} & 
0.01\\SCA3-4 & \bf{690.50} & 3.83 & 
690.50 & 3.83 & 690.50 & 0.00\\
SCA3-5 & \bf{\underline{659.90}} & 3.80 & 
664.60 & 3.64 & 671.10 & 
-1.67\\SCA3-6 & 652.94 & 3.64 & 
653.27 & 3.50 & \bf{651.10} & 
0.28\\SCA3-7 & 666.15 & 2.66 & 
666.42 & 2.78 & \bf{666.10} & 
0.01\\SCA3-8 & \bf{\underline{719.47}} & 3.30 & 
722.49 & 3.30 & 719.50 & 
-0.00\\SCA3-9 & \bf{681.00} & 2.90 & 
681.00 & 2.79 & 681.00 & 0.00\\
SCA8-0 & 968.79 & 4.14 & 
977.41 & 3.98 & \bf{961.60} & 
0.75\\SCA8-1 & \bf{\underline{1059.21}} & 3.18 & 
1065.28 & 3.09 & 1063.00 & 
-0.36\\SCA8-2 & 1046.29 & 2.88 & 
1050.31 & 2.77 & \bf{1040.60} & 
0.55\\SCA8-3 & 1009.39 & 3.69 & 
1016.16 & 3.79 & \bf{985.90} & 
2.38\\SCA8-4 & \bf{\underline{1065.49}} & 3.82 & 
1073.28 & 3.86 & 1071.00 & 
-0.51\\SCA8-5 & \bf{\underline{1034.74}} & 4.26 & 
1044.41 & 4.18 & 1054.30 & 
-1.86\\SCA8-6 & 977.03 & 4.03 & 
979.04 & 4.08 & \bf{972.50} & 
0.47\\SCA8-7 & 1067.20 & 4.24 & 
1067.20 & 4.17 & \bf{1059.70} & 
0.71\\SCA8-8 & \bf{\underline{1071.18}} & 4.00 & 
1077.60 & 3.85 & 1082.70 & 
-1.06\\SCA8-9 & \bf{\underline{1067.42}} & 3.24 & 
1067.42 & 3.07 & 1081.40 & 
-1.29\\CON3-0 & 616.52 & 4.06 & 
621.35 & 3.96 & \bf{616.50} & 
0.00\\CON3-1 & \bf{\underline{554.47}} & 3.94 & 
555.78 & 3.65 & 555.60 & 
-0.20\\CON3-2 & \bf{\underline{519.61}} & 3.31 & 
521.52 & 3.46 & 521.40 & 
-0.34\\CON3-3 & \bf{\underline{591.19}} & 4.69 & 
591.28 & 4.08 & 591.20 & 
-0.00\\CON3-4 & \bf{\underline{588.79}} & 3.65 & 
589.72 & 3.51 & 589.30 & 
-0.09\\CON3-5 & 564.89 & 3.22 & 
568.30 & 3.55 & \bf{563.70} & 
0.21\\CON3-6 & 500.37 & 4.29 & 
501.03 & 4.29 & \bf{499.20} & 
0.23\\CON3-7 & 577.68 & 3.24 & 
578.97 & 3.10 & \bf{577.50} & 
0.03\\CON3-8 & \bf{\underline{523.05}} & 3.13 & 
524.87 & 3.10 & 523.10 & 
-0.01\\CON3-9 & 578.98 & 3.55 & 
586.38 & 3.33 & \bf{578.20} & 
0.13\\CON8-0 & 870.22 & 3.52 & 
872.73 & 3.63 & \bf{858.90} & 
1.32\\CON8-1 & 740.93 & 3.53 & 
748.39 & 3.68 & \bf{740.90} & 
0.00\\CON8-2 & \bf{\underline{713.90}} & 4.46 & 
715.00 & 4.76 & 714.30 & 
-0.06\\CON8-3 & \bf{\underline{811.23}} & 3.44 & 
815.99 & 3.46 & 812.30 & 
-0.13\\CON8-4 & 790.23 & 3.71 & 
793.29 & 3.69 & \bf{770.10} & 
2.61\\CON8-5 & \bf{\underline{754.95}} & 3.59 & 
757.99 & 3.45 & 766.60 & 
-1.52\\CON8-6 & \bf{\underline{690.31}} & 4.20 & 
695.75 & 4.16 & 697.20 & 
-0.99\\CON8-7 & 814.86 & 3.17 & 
817.58 & 3.22 & \bf{814.80} & 
0.01\\CON8-8 & 779.43 & 3.89 & 
783.95 & 3.94 & \bf{771.30} & 
1.05\\CON8-9 & \bf{\underline{810.18}} & 4.01 & 
814.15 & 4.04 & 815.10 & 
-0.60\\[1ex]\hline
\end{tabular}
\label{table:nonlin}
\end{table} \clearpage
\begin{table}[ht]
\caption{Resultados de la ejecución de la metaheurística ACO, utilizando instancias de Dethloff con la configuración -n 5.0 -alpha 1.0 -beta 3.0 -q .8 -ro 0.015}
\centering
\small
\begin{tabular}{c c c c c c c}
\hline\hline
Instancia & Costo mínimo & Tiempo(seg.) & Costo promedio & Tiempo promedio(seg.) & Costo ACO & \%Gap \\ [0.5ex]
\hline
SCA3-0 & \bf{\underline{636.06}} & 3.50 & 
636.13 & 3.49 & 636.10 & 
-0.01\\SCA3-1 & \bf{\underline{697.84}} & 3.56 & 
698.76 & 3.75 & 700.10 & 
-0.32\\SCA3-2 & 664.18 & 3.23 & 
664.18 & 3.27 & \bf{659.30} & 
0.74\\SCA3-3 & 680.04 & 3.86 & 
680.18 & 3.50 & \bf{680.00} & 
0.01\\SCA3-4 & \bf{690.50} & 3.62 & 
690.50 & 3.55 & 690.50 & 0.00\\
SCA3-5 & \bf{\underline{659.90}} & 3.60 & 
664.25 & 3.67 & 671.10 & 
-1.67\\SCA3-6 & 652.94 & 3.22 & 
653.59 & 3.39 & \bf{651.10} & 
0.28\\SCA3-7 & 666.15 & 2.70 & 
666.15 & 2.70 & \bf{666.10} & 
0.01\\SCA3-8 & \bf{\underline{719.47}} & 3.10 & 
719.54 & 3.12 & 719.50 & 
-0.00\\SCA3-9 & \bf{681.00} & 2.75 & 
681.00 & 2.82 & 681.00 & 0.00\\
SCA8-0 & 979.12 & 3.72 & 
985.77 & 4.26 & \bf{961.60} & 
1.82\\SCA8-1 & \bf{\underline{1049.65}} & 3.08 & 
1059.21 & 3.23 & 1063.00 & 
-1.26\\SCA8-2 & 1051.21 & 2.68 & 
1052.28 & 2.65 & \bf{1040.60} & 
1.02\\SCA8-3 & 1011.56 & 3.88 & 
1018.01 & 3.97 & \bf{985.90} & 
2.60\\SCA8-4 & \bf{\underline{1065.49}} & 3.80 & 
1083.60 & 3.82 & 1071.00 & 
-0.51\\SCA8-5 & \bf{\underline{1034.74}} & 3.98 & 
1050.33 & 4.10 & 1054.30 & 
-1.86\\SCA8-6 & 980.91 & 3.91 & 
981.33 & 3.95 & \bf{972.50} & 
0.86\\SCA8-7 & 1067.20 & 3.92 & 
1072.15 & 3.92 & \bf{1059.70} & 
0.71\\SCA8-8 & \bf{\underline{1071.18}} & 3.95 & 
1076.64 & 3.98 & 1082.70 & 
-1.06\\SCA8-9 & \bf{\underline{1066.61}} & 3.04 & 
1067.22 & 3.09 & 1081.40 & 
-1.37\\CON3-0 & 617.59 & 3.99 & 
622.07 & 3.92 & \bf{616.50} & 
0.18\\CON3-1 & \bf{\underline{554.47}} & 3.79 & 
557.50 & 3.72 & 555.60 & 
-0.20\\CON3-2 & \bf{\underline{521.38}} & 3.87 & 
523.37 & 3.26 & 521.40 & 
-0.00\\CON3-3 & \bf{591.20} & 3.72 & 
592.65 & 3.76 & 591.20 & 0.00\\
CON3-4 & \bf{\underline{588.79}} & 3.25 & 
589.74 & 3.29 & 589.30 & 
-0.09\\CON3-5 & 564.88 & 3.81 & 
566.36 & 3.77 & \bf{563.70} & 
0.21\\CON3-6 & 500.88 & 4.44 & 
502.90 & 4.36 & \bf{499.20} & 
0.34\\CON3-7 & 578.41 & 3.02 & 
579.15 & 3.15 & \bf{577.50} & 
0.16\\CON3-8 & \bf{\underline{523.05}} & 3.48 & 
524.15 & 3.09 & 523.10 & 
-0.01\\CON3-9 & 581.06 & 3.10 & 
586.90 & 3.23 & \bf{578.20} & 
0.49\\CON8-0 & 871.92 & 3.62 & 
877.53 & 3.62 & \bf{858.90} & 
1.52\\CON8-1 & \bf{\underline{740.85}} & 3.79 & 
744.27 & 3.62 & 740.90 & 
-0.01\\CON8-2 & \bf{\underline{713.44}} & 4.84 & 
714.20 & 4.78 & 714.30 & 
-0.12\\CON8-3 & 817.57 & 3.93 & 
817.57 & 3.65 & \bf{812.30} & 
0.65\\CON8-4 & 776.37 & 3.43 & 
787.36 & 3.58 & \bf{770.10} & 
0.81\\CON8-5 & \bf{\underline{759.93}} & 3.30 & 
762.39 & 3.37 & 766.60 & 
-0.87\\CON8-6 & \bf{\underline{690.53}} & 3.67 & 
694.14 & 4.00 & 697.20 & 
-0.96\\CON8-7 & \bf{\underline{814.77}} & 3.13 & 
821.92 & 3.05 & 814.80 & 
-0.00\\CON8-8 & 785.48 & 4.02 & 
788.66 & 4.05 & \bf{771.30} & 
1.84\\CON8-9 & 815.49 & 3.72 & 
815.94 & 3.89 & \bf{815.10} & 
0.05\\[1ex]\hline
\end{tabular}
\label{table:nonlin}
\end{table} \clearpage
\begin{table}[ht]
\caption{Resultados de la ejecución de la metaheurística ACO, utilizando instancias de Dethloff con la configuración -n 5.0 -alpha 1.0 -beta 3.0 -q .9 -ro 0.015}
\centering
\small
\begin{tabular}{c c c c c c c}
\hline\hline
Instancia & Costo mínimo & Tiempo(seg.) & Costo promedio & Tiempo promedio(seg.) & Costo ACO & \%Gap \\ [0.5ex]
\hline
SCA3-0 & \bf{\underline{636.06}} & 5.03 & 
639.43 & 3.69 & 636.10 & 
-0.01\\SCA3-1 & \bf{\underline{697.84}} & 3.66 & 
697.84 & 3.73 & 700.10 & 
-0.32\\SCA3-2 & 659.34 & 3.32 & 
661.00 & 3.39 & \bf{659.30} & 
0.01\\SCA3-3 & 680.04 & 3.66 & 
680.82 & 3.60 & \bf{680.00} & 
0.01\\SCA3-4 & \bf{690.50} & 3.52 & 
690.50 & 3.56 & 690.50 & 0.00\\
SCA3-5 & \bf{\underline{661.07}} & 3.78 & 
664.90 & 3.59 & 671.10 & 
-1.49\\SCA3-6 & 652.94 & 3.32 & 
654.09 & 3.40 & \bf{651.10} & 
0.28\\SCA3-7 & 666.15 & 2.75 & 
666.15 & 2.79 & \bf{666.10} & 
0.01\\SCA3-8 & 721.45 & 3.12 & 
723.48 & 3.10 & \bf{719.50} & 
0.27\\SCA3-9 & \bf{681.00} & 3.65 & 
681.00 & 2.88 & 681.00 & 0.00\\
SCA8-0 & \bf{\underline{961.50}} & 3.62 & 
976.72 & 3.81 & 961.60 & 
-0.01\\SCA8-1 & \bf{\underline{1057.04}} & 2.97 & 
1067.80 & 2.98 & 1063.00 & 
-0.56\\SCA8-2 & 1049.22 & 2.41 & 
1052.92 & 2.46 & \bf{1040.60} & 
0.83\\SCA8-3 & 1023.09 & 3.74 & 
1029.02 & 3.69 & \bf{985.90} & 
3.77\\SCA8-4 & \bf{\underline{1065.49}} & 3.55 & 
1078.20 & 3.73 & 1071.00 & 
-0.51\\SCA8-5 & 1055.35 & 4.20 & 
1057.80 & 4.03 & \bf{1054.30} & 
0.10\\SCA8-6 & 977.03 & 4.22 & 
979.30 & 4.07 & \bf{972.50} & 
0.47\\SCA8-7 & 1067.20 & 4.37 & 
1067.31 & 4.12 & \bf{1059.70} & 
0.71\\SCA8-8 & \bf{\underline{1071.18}} & 3.70 & 
1076.64 & 3.69 & 1082.70 & 
-1.06\\SCA8-9 & \bf{\underline{1067.42}} & 3.02 & 
1067.42 & 2.99 & 1081.40 & 
-1.29\\CON3-0 & 624.96 & 3.93 & 
624.96 & 4.03 & \bf{616.50} & 
1.37\\CON3-1 & \bf{\underline{554.47}} & 3.78 & 
556.85 & 3.56 & 555.60 & 
-0.20\\CON3-2 & \bf{\underline{521.38}} & 3.08 & 
522.75 & 3.23 & 521.40 & 
-0.00\\CON3-3 & \bf{\underline{591.19}} & 3.96 & 
591.24 & 3.87 & 591.20 & 
-0.00\\CON3-4 & \bf{\underline{588.79}} & 3.08 & 
590.83 & 3.18 & 589.30 & 
-0.09\\CON3-5 & \bf{563.70} & 3.38 & 
568.73 & 3.54 & 563.70 & 0.00\\
CON3-6 & 503.97 & 4.16 & 
504.02 & 4.11 & \bf{499.20} & 
0.96\\CON3-7 & 578.22 & 3.16 & 
579.08 & 3.08 & \bf{577.50} & 
0.12\\CON3-8 & 524.30 & 3.21 & 
524.52 & 3.10 & \bf{523.10} & 
0.23\\CON3-9 & 588.48 & 3.44 & 
588.90 & 3.40 & \bf{578.20} & 
1.78\\CON8-0 & 873.87 & 3.78 & 
880.95 & 3.70 & \bf{858.90} & 
1.74\\CON8-1 & \bf{\underline{740.85}} & 3.20 & 
747.80 & 3.43 & 740.90 & 
-0.01\\CON8-2 & 716.07 & 4.90 & 
716.30 & 4.81 & \bf{714.30} & 
0.25\\CON8-3 & 812.75 & 3.83 & 
816.37 & 3.50 & \bf{812.30} & 
0.06\\CON8-4 & 781.64 & 3.78 & 
786.71 & 3.71 & \bf{770.10} & 
1.50\\CON8-5 & \bf{\underline{759.59}} & 3.46 & 
763.00 & 3.40 & 766.60 & 
-0.91\\CON8-6 & \bf{\underline{695.66}} & 4.08 & 
696.72 & 4.08 & 697.20 & 
-0.22\\CON8-7 & 814.86 & 2.99 & 
820.77 & 3.03 & \bf{814.80} & 
0.01\\CON8-8 & 791.15 & 3.90 & 
794.14 & 3.96 & \bf{771.30} & 
2.57\\CON8-9 & \bf{\underline{813.16}} & 3.90 & 
814.91 & 3.93 & 815.10 & 
-0.24\\[1ex]\hline
\end{tabular}
\label{table:nonlin}
\end{table} \clearpage
\begin{table}[ht]
\caption{Resultados de la ejecución de la metaheurística ACO, utilizando instancias de Dethloff con la configuración -n 6.0 -alpha 1.0 -beta 3.0 -q 0.1 -ro 0.015}
\centering
\small
\begin{tabular}{c c c c c c c}
\hline\hline
Instancia & Costo mínimo & Tiempo(seg.) & Costo promedio & Tiempo promedio(seg.) & Costo ACO & \%Gap \\ [0.5ex]
\hline
SCA3-0 & \bf{\underline{636.06}} & 4.31 & 
636.06 & 4.40 & 636.10 & 
-0.01\\SCA3-1 & \bf{\underline{697.84}} & 4.94 & 
697.84 & 4.83 & 700.10 & 
-0.32\\SCA3-2 & 659.34 & 3.93 & 
660.68 & 4.21 & \bf{659.30} & 
0.01\\SCA3-3 & 680.04 & 4.29 & 
680.04 & 4.28 & \bf{680.00} & 
0.01\\SCA3-4 & \bf{690.50} & 4.53 & 
690.50 & 4.62 & 690.50 & 0.00\\
SCA3-5 & \bf{\underline{662.75}} & 4.70 & 
663.32 & 4.51 & 671.10 & 
-1.24\\SCA3-6 & \bf{\underline{651.09}} & 4.46 & 
652.70 & 4.35 & 651.10 & 
-0.00\\SCA3-7 & 666.15 & 3.86 & 
666.15 & 3.90 & \bf{666.10} & 
0.01\\SCA3-8 & \bf{\underline{719.47}} & 4.31 & 
720.67 & 4.24 & 719.50 & 
-0.00\\SCA3-9 & \bf{681.00} & 3.94 & 
682.40 & 3.77 & 681.00 & 0.00\\
SCA8-0 & \bf{\underline{961.50}} & 4.64 & 
971.77 & 4.64 & 961.60 & 
-0.01\\SCA8-1 & \bf{\underline{1053.09}} & 3.91 & 
1060.39 & 4.07 & 1063.00 & 
-0.93\\SCA8-2 & 1050.37 & 3.63 & 
1050.68 & 3.58 & \bf{1040.60} & 
0.94\\SCA8-3 & \bf{\underline{983.34}} & 4.44 & 
995.78 & 4.33 & 985.90 & 
-0.26\\SCA8-4 & \bf{\underline{1065.49}} & 4.44 & 
1066.45 & 4.43 & 1071.00 & 
-0.51\\SCA8-5 & \bf{\underline{1050.46}} & 5.13 & 
1054.02 & 5.21 & 1054.30 & 
-0.36\\SCA8-6 & 977.03 & 4.83 & 
979.87 & 4.80 & \bf{972.50} & 
0.47\\SCA8-7 & 1067.03 & 4.85 & 
1069.09 & 4.92 & \bf{1059.70} & 
0.69\\SCA8-8 & \bf{\underline{1071.18}} & 5.06 & 
1076.50 & 4.98 & 1082.70 & 
-1.06\\SCA8-9 & \bf{\underline{1067.42}} & 4.07 & 
1067.42 & 4.15 & 1081.40 & 
-1.29\\CON3-0 & 620.29 & 5.10 & 
623.04 & 4.83 & \bf{616.50} & 
0.61\\CON3-1 & \bf{\underline{554.47}} & 4.43 & 
555.55 & 4.58 & 555.60 & 
-0.20\\CON3-2 & \bf{\underline{521.36}} & 4.25 & 
521.38 & 4.38 & 521.40 & 
-0.01\\CON3-3 & \bf{\underline{591.19}} & 4.66 & 
591.20 & 4.84 & 591.20 & 
-0.00\\CON3-4 & \bf{\underline{588.79}} & 4.10 & 
588.92 & 4.04 & 589.30 & 
-0.09\\CON3-5 & \bf{563.70} & 4.24 & 
565.24 & 4.43 & 563.70 & 0.00\\
CON3-6 & 502.09 & 4.96 & 
502.14 & 5.09 & \bf{499.20} & 
0.58\\CON3-7 & \bf{\underline{576.48}} & 3.97 & 
577.88 & 4.10 & 577.50 & 
-0.18\\CON3-8 & \bf{\underline{523.05}} & 4.42 & 
523.54 & 4.20 & 523.10 & 
-0.01\\CON3-9 & 582.79 & 4.15 & 
587.04 & 4.19 & \bf{578.20} & 
0.79\\CON8-0 & \bf{\underline{858.63}} & 4.76 & 
869.89 & 4.69 & 858.90 & 
-0.03\\CON8-1 & \bf{\underline{740.85}} & 4.49 & 
743.24 & 4.68 & 740.90 & 
-0.01\\CON8-2 & \bf{\underline{713.44}} & 5.11 & 
714.94 & 5.54 & 714.30 & 
-0.12\\CON8-3 & \bf{\underline{811.07}} & 4.73 & 
812.96 & 4.68 & 812.30 & 
-0.15\\CON8-4 & 780.48 & 4.34 & 
783.07 & 4.24 & \bf{770.10} & 
1.35\\CON8-5 & \bf{\underline{760.03}} & 4.65 & 
761.37 & 4.60 & 766.60 & 
-0.86\\CON8-6 & \bf{\underline{685.06}} & 5.33 & 
690.66 & 5.30 & 697.20 & 
-1.74\\CON8-7 & \bf{\underline{814.79}} & 4.19 & 
814.83 & 4.14 & 814.80 & 
-0.00\\CON8-8 & 779.76 & 5.30 & 
786.02 & 5.18 & \bf{771.30} & 
1.10\\CON8-9 & \bf{\underline{812.03}} & 5.07 & 
812.46 & 5.30 & 815.10 & 
-0.38\\[1ex]\hline
\end{tabular}
\label{table:nonlin}
\end{table} \clearpage
\begin{table}[ht]
\caption{Resultados de la ejecución de la metaheurística ACO, utilizando instancias de Dethloff con la configuración -n 6.0 -alpha 1.0 -beta 3.0 -q .2 -ro 0.015}
\centering
\small
\begin{tabular}{c c c c c c c}
\hline\hline
Instancia & Costo mínimo & Tiempo(seg.) & Costo promedio & Tiempo promedio(seg.) & Costo ACO & \%Gap \\ [0.5ex]
\hline
SCA3-0 & \bf{\underline{636.06}} & 4.48 & 
636.06 & 4.19 & 636.10 & 
-0.01\\SCA3-1 & \bf{\underline{697.84}} & 4.56 & 
697.84 & 4.62 & 700.10 & 
-0.32\\SCA3-2 & 659.34 & 4.51 & 
661.76 & 4.19 & \bf{659.30} & 
0.01\\SCA3-3 & 680.04 & 3.99 & 
680.04 & 4.54 & \bf{680.00} & 
0.01\\SCA3-4 & \bf{690.50} & 4.32 & 
690.50 & 4.51 & 690.50 & 0.00\\
SCA3-5 & \bf{\underline{659.90}} & 4.36 & 
661.90 & 4.58 & 671.10 & 
-1.67\\SCA3-6 & \bf{\underline{651.09}} & 4.34 & 
652.48 & 4.28 & 651.10 & 
-0.00\\SCA3-7 & 666.15 & 4.27 & 
666.42 & 3.99 & \bf{666.10} & 
0.01\\SCA3-8 & \bf{\underline{719.47}} & 4.20 & 
719.47 & 4.36 & 719.50 & 
-0.00\\SCA3-9 & \bf{681.00} & 4.34 & 
681.00 & 3.76 & 681.00 & 0.00\\
SCA8-0 & 975.50 & 5.02 & 
982.19 & 4.75 & \bf{961.60} & 
1.45\\SCA8-1 & \bf{\underline{1056.70}} & 4.01 & 
1062.89 & 3.98 & 1063.00 & 
-0.59\\SCA8-2 & 1045.64 & 3.52 & 
1048.87 & 3.60 & \bf{1040.60} & 
0.48\\SCA8-3 & 1006.94 & 4.64 & 
1009.85 & 4.61 & \bf{985.90} & 
2.13\\SCA8-4 & \bf{\underline{1065.49}} & 4.67 & 
1070.42 & 4.53 & 1071.00 & 
-0.51\\SCA8-5 & \bf{\underline{1047.30}} & 4.83 & 
1050.36 & 5.08 & 1054.30 & 
-0.66\\SCA8-6 & 977.03 & 4.50 & 
979.43 & 4.71 & \bf{972.50} & 
0.47\\SCA8-7 & 1067.20 & 4.51 & 
1070.07 & 4.89 & \bf{1059.70} & 
0.71\\SCA8-8 & \bf{\underline{1071.18}} & 4.84 & 
1073.97 & 4.92 & 1082.70 & 
-1.06\\SCA8-9 & \bf{\underline{1067.42}} & 3.94 & 
1067.42 & 4.00 & 1081.40 & 
-1.29\\CON3-0 & 617.59 & 4.60 & 
619.17 & 4.59 & \bf{616.50} & 
0.18\\CON3-1 & \bf{\underline{554.47}} & 4.54 & 
554.47 & 4.64 & 555.60 & 
-0.20\\CON3-2 & \bf{\underline{519.11}} & 4.48 & 
520.81 & 4.41 & 521.40 & 
-0.44\\CON3-3 & \bf{\underline{591.19}} & 4.69 & 
591.20 & 4.71 & 591.20 & 
-0.00\\CON3-4 & \bf{\underline{588.79}} & 3.90 & 
589.58 & 4.07 & 589.30 & 
-0.09\\CON3-5 & 564.88 & 4.73 & 
565.86 & 4.42 & \bf{563.70} & 
0.21\\CON3-6 & 500.37 & 5.30 & 
502.20 & 5.19 & \bf{499.20} & 
0.23\\CON3-7 & \bf{\underline{576.84}} & 4.06 & 
577.97 & 3.95 & 577.50 & 
-0.11\\CON3-8 & \bf{\underline{523.05}} & 4.18 & 
523.25 & 4.23 & 523.10 & 
-0.01\\CON3-9 & 581.79 & 4.24 & 
584.96 & 4.44 & \bf{578.20} & 
0.62\\CON8-0 & 869.43 & 4.54 & 
873.81 & 4.59 & \bf{858.90} & 
1.23\\CON8-1 & \bf{\underline{740.85}} & 4.75 & 
743.29 & 4.79 & 740.90 & 
-0.01\\CON8-2 & \bf{\underline{713.44}} & 5.52 & 
715.76 & 5.62 & 714.30 & 
-0.12\\CON8-3 & \bf{\underline{811.07}} & 4.31 & 
812.46 & 4.54 & 812.30 & 
-0.15\\CON8-4 & 776.37 & 4.08 & 
781.68 & 4.39 & \bf{770.10} & 
0.81\\CON8-5 & \bf{\underline{759.87}} & 4.90 & 
760.86 & 4.50 & 766.60 & 
-0.88\\CON8-6 & \bf{\underline{685.80}} & 5.23 & 
692.84 & 5.04 & 697.20 & 
-1.64\\CON8-7 & \bf{\underline{814.50}} & 4.46 & 
814.74 & 4.25 & 814.80 & 
-0.04\\CON8-8 & 781.74 & 5.58 & 
785.53 & 5.27 & \bf{771.30} & 
1.35\\CON8-9 & \bf{\underline{811.14}} & 5.08 & 
814.34 & 5.11 & 815.10 & 
-0.49\\[1ex]\hline
\end{tabular}
\label{table:nonlin}
\end{table} \clearpage
\begin{table}[ht]
\caption{Resultados de la ejecución de la metaheurística ACO, utilizando instancias de Dethloff con la configuración -n 6.0 -alpha 1.0 -beta 3.0 -q .3 -ro 0.015}
\centering
\small
\begin{tabular}{c c c c c c c}
\hline\hline
Instancia & Costo mínimo & Tiempo(seg.) & Costo promedio & Tiempo promedio(seg.) & Costo ACO & \%Gap \\ [0.5ex]
\hline
SCA3-0 & \bf{\underline{636.06}} & 4.46 & 
636.06 & 4.29 & 636.10 & 
-0.01\\SCA3-1 & \bf{\underline{697.84}} & 4.64 & 
697.84 & 4.50 & 700.10 & 
-0.32\\SCA3-2 & 659.34 & 4.11 & 
659.79 & 4.14 & \bf{659.30} & 
0.01\\SCA3-3 & 680.04 & 4.15 & 
680.04 & 4.05 & \bf{680.00} & 
0.01\\SCA3-4 & \bf{690.50} & 4.36 & 
690.50 & 4.51 & 690.50 & 0.00\\
SCA3-5 & \bf{\underline{662.75}} & 4.31 & 
664.20 & 4.44 & 671.10 & 
-1.24\\SCA3-6 & \bf{\underline{651.09}} & 4.37 & 
652.48 & 4.28 & 651.10 & 
-0.00\\SCA3-7 & 666.15 & 4.03 & 
666.15 & 4.01 & \bf{666.10} & 
0.01\\SCA3-8 & \bf{\underline{719.47}} & 4.25 & 
721.64 & 4.27 & 719.50 & 
-0.00\\SCA3-9 & \bf{681.00} & 3.43 & 
681.00 & 3.50 & 681.00 & 0.00\\
SCA8-0 & 968.79 & 4.59 & 
976.77 & 4.65 & \bf{961.60} & 
0.75\\SCA8-1 & \bf{\underline{1054.87}} & 3.68 & 
1057.44 & 3.75 & 1063.00 & 
-0.76\\SCA8-2 & 1050.17 & 3.66 & 
1050.74 & 3.46 & \bf{1040.60} & 
0.92\\SCA8-3 & 995.50 & 4.32 & 
1005.90 & 4.30 & \bf{985.90} & 
0.97\\SCA8-4 & \bf{\underline{1065.49}} & 4.68 & 
1065.94 & 4.54 & 1071.00 & 
-0.51\\SCA8-5 & \bf{\underline{1034.74}} & 4.70 & 
1046.68 & 4.95 & 1054.30 & 
-1.86\\SCA8-6 & \bf{\underline{972.48}} & 5.19 & 
978.93 & 4.96 & 972.50 & 
-0.00\\SCA8-7 & 1067.03 & 4.88 & 
1069.12 & 4.97 & \bf{1059.70} & 
0.69\\SCA8-8 & \bf{\underline{1071.18}} & 4.58 & 
1074.87 & 4.80 & 1082.70 & 
-1.06\\SCA8-9 & \bf{\underline{1065.60}} & 4.06 & 
1066.97 & 3.86 & 1081.40 & 
-1.46\\CON3-0 & 617.59 & 4.61 & 
620.93 & 4.73 & \bf{616.50} & 
0.18\\CON3-1 & \bf{\underline{554.47}} & 4.83 & 
556.34 & 4.74 & 555.60 & 
-0.20\\CON3-2 & \bf{\underline{519.11}} & 4.44 & 
519.80 & 4.49 & 521.40 & 
-0.44\\CON3-3 & \bf{\underline{591.19}} & 4.50 & 
591.50 & 4.64 & 591.20 & 
-0.00\\CON3-4 & \bf{\underline{588.79}} & 4.22 & 
589.72 & 4.08 & 589.30 & 
-0.09\\CON3-5 & \bf{563.70} & 4.42 & 
564.59 & 4.46 & 563.70 & 0.00\\
CON3-6 & 502.09 & 4.99 & 
503.14 & 5.03 & \bf{499.20} & 
0.58\\CON3-7 & 578.22 & 3.64 & 
578.36 & 3.81 & \bf{577.50} & 
0.12\\CON3-8 & \bf{\underline{523.05}} & 4.14 & 
523.36 & 4.21 & 523.10 & 
-0.01\\CON3-9 & 583.32 & 4.14 & 
587.32 & 4.21 & \bf{578.20} & 
0.89\\CON8-0 & 866.22 & 4.51 & 
869.91 & 4.53 & \bf{858.90} & 
0.85\\CON8-1 & \bf{\underline{740.85}} & 4.94 & 
742.95 & 4.95 & 740.90 & 
-0.01\\CON8-2 & \bf{\underline{713.44}} & 5.39 & 
713.86 & 5.38 & 714.30 & 
-0.12\\CON8-3 & \bf{\underline{811.07}} & 4.39 & 
814.40 & 4.45 & 812.30 & 
-0.15\\CON8-4 & 784.87 & 4.34 & 
787.99 & 4.45 & \bf{770.10} & 
1.92\\CON8-5 & \bf{\underline{759.93}} & 5.11 & 
760.97 & 4.84 & 766.60 & 
-0.87\\CON8-6 & \bf{\underline{692.75}} & 5.04 & 
693.80 & 5.09 & 697.20 & 
-0.64\\CON8-7 & \bf{\underline{814.79}} & 4.02 & 
818.25 & 4.03 & 814.80 & 
-0.00\\CON8-8 & 777.60 & 5.72 & 
786.37 & 5.29 & \bf{771.30} & 
0.82\\CON8-9 & \bf{\underline{810.18}} & 5.19 & 
812.96 & 5.25 & 815.10 & 
-0.60\\[1ex]\hline
\end{tabular}
\label{table:nonlin}
\end{table} \clearpage
\begin{table}[ht]
\caption{Resultados de la ejecución de la metaheurística ACO, utilizando instancias de Dethloff con la configuración -n 6.0 -alpha 1.0 -beta 3.0 -q .4 -ro 0.015}
\centering
\small
\begin{tabular}{c c c c c c c}
\hline\hline
Instancia & Costo mínimo & Tiempo(seg.) & Costo promedio & Tiempo promedio(seg.) & Costo ACO & \%Gap \\ [0.5ex]
\hline
SCA3-0 & \bf{\underline{636.06}} & 4.20 & 
636.06 & 4.09 & 636.10 & 
-0.01\\SCA3-1 & \bf{\underline{697.84}} & 4.35 & 
697.84 & 4.48 & 700.10 & 
-0.32\\SCA3-2 & 659.34 & 5.08 & 
661.45 & 4.36 & \bf{659.30} & 
0.01\\SCA3-3 & 680.04 & 4.13 & 
680.04 & 4.22 & \bf{680.00} & 
0.01\\SCA3-4 & \bf{690.50} & 4.36 & 
690.50 & 4.42 & 690.50 & 0.00\\
SCA3-5 & \bf{\underline{659.90}} & 5.17 & 
662.34 & 4.50 & 671.10 & 
-1.67\\SCA3-6 & 652.94 & 3.91 & 
653.19 & 4.63 & \bf{651.10} & 
0.28\\SCA3-7 & 666.15 & 3.58 & 
666.42 & 3.73 & \bf{666.10} & 
0.01\\SCA3-8 & \bf{\underline{719.47}} & 4.20 & 
719.47 & 4.14 & 719.50 & 
-0.00\\SCA3-9 & \bf{681.00} & 3.48 & 
681.00 & 3.60 & 681.00 & 0.00\\
SCA8-0 & 971.49 & 4.68 & 
981.11 & 4.66 & \bf{961.60} & 
1.03\\SCA8-1 & \bf{\underline{1054.87}} & 3.90 & 
1062.60 & 3.70 & 1063.00 & 
-0.76\\SCA8-2 & 1045.64 & 3.29 & 
1049.19 & 3.39 & \bf{1040.60} & 
0.48\\SCA8-3 & 991.84 & 4.74 & 
1000.33 & 4.52 & \bf{985.90} & 
0.60\\SCA8-4 & \bf{\underline{1065.49}} & 4.57 & 
1068.28 & 4.73 & 1071.00 & 
-0.51\\SCA8-5 & \bf{\underline{1034.74}} & 5.04 & 
1047.00 & 4.95 & 1054.30 & 
-1.86\\SCA8-6 & \bf{\underline{972.48}} & 4.54 & 
976.87 & 5.70 & 972.50 & 
-0.00\\SCA8-7 & 1067.03 & 5.14 & 
1070.41 & 4.82 & \bf{1059.70} & 
0.69\\SCA8-8 & \bf{\underline{1071.18}} & 4.76 & 
1073.91 & 4.82 & 1082.70 & 
-1.06\\SCA8-9 & \bf{\underline{1065.60}} & 3.85 & 
1066.97 & 3.89 & 1081.40 & 
-1.46\\CON3-0 & 616.52 & 4.88 & 
621.37 & 4.85 & \bf{616.50} & 
0.00\\CON3-1 & \bf{\underline{554.47}} & 4.50 & 
555.39 & 4.66 & 555.60 & 
-0.20\\CON3-2 & \bf{\underline{521.38}} & 4.55 & 
521.38 & 4.52 & 521.40 & 
-0.00\\CON3-3 & \bf{\underline{591.19}} & 4.86 & 
591.20 & 4.69 & 591.20 & 
-0.00\\CON3-4 & \bf{\underline{588.79}} & 3.92 & 
588.79 & 3.75 & 589.30 & 
-0.09\\CON3-5 & \bf{563.70} & 4.18 & 
564.59 & 4.40 & 563.70 & 0.00\\
CON3-6 & 502.16 & 5.12 & 
503.04 & 5.05 & \bf{499.20} & 
0.59\\CON3-7 & 578.41 & 4.26 & 
579.15 & 4.00 & \bf{577.50} & 
0.16\\CON3-8 & 523.14 & 3.88 & 
523.41 & 4.09 & \bf{523.10} & 
0.01\\CON3-9 & 578.25 & 4.16 & 
582.77 & 4.07 & \bf{578.20} & 
0.01\\CON8-0 & 865.86 & 4.61 & 
868.67 & 4.46 & \bf{858.90} & 
0.81\\CON8-1 & 740.93 & 4.61 & 
741.99 & 4.55 & \bf{740.90} & 
0.00\\CON8-2 & \bf{\underline{713.84}} & 5.55 & 
715.41 & 5.72 & 714.30 & 
-0.06\\CON8-3 & 815.14 & 4.30 & 
816.79 & 4.48 & \bf{812.30} & 
0.35\\CON8-4 & 772.32 & 4.64 & 
779.12 & 4.33 & \bf{770.10} & 
0.29\\CON8-5 & \bf{\underline{758.84}} & 4.30 & 
761.73 & 4.50 & 766.60 & 
-1.01\\CON8-6 & \bf{\underline{695.66}} & 5.07 & 
696.91 & 5.12 & 697.20 & 
-0.22\\CON8-7 & \bf{\underline{814.79}} & 4.01 & 
814.89 & 3.90 & 814.80 & 
-0.00\\CON8-8 & 782.86 & 4.84 & 
784.71 & 5.03 & \bf{771.30} & 
1.50\\CON8-9 & \bf{\underline{810.18}} & 5.19 & 
812.23 & 5.18 & 815.10 & 
-0.60\\[1ex]\hline
\end{tabular}
\label{table:nonlin}
\end{table} \clearpage
\begin{table}[ht]
\caption{Resultados de la ejecución de la metaheurística ACO, utilizando instancias de Dethloff con la configuración -n 6.0 -alpha 1.0 -beta 3.0 -q .5 -ro 0.015}
\centering
\small
\begin{tabular}{c c c c c c c}
\hline\hline
Instancia & Costo mínimo & Tiempo(seg.) & Costo promedio & Tiempo promedio(seg.) & Costo ACO & \%Gap \\ [0.5ex]
\hline
SCA3-0 & \bf{\underline{636.06}} & 4.00 & 
636.06 & 4.15 & 636.10 & 
-0.01\\SCA3-1 & \bf{\underline{697.84}} & 4.36 & 
697.84 & 4.58 & 700.10 & 
-0.32\\SCA3-2 & 659.34 & 3.92 & 
662.21 & 4.09 & \bf{659.30} & 
0.01\\SCA3-3 & 680.04 & 4.00 & 
680.04 & 4.12 & \bf{680.00} & 
0.01\\SCA3-4 & \bf{690.50} & 4.50 & 
690.50 & 4.31 & 690.50 & 0.00\\
SCA3-5 & \bf{\underline{659.90}} & 4.46 & 
663.59 & 4.47 & 671.10 & 
-1.67\\SCA3-6 & \bf{\underline{651.09}} & 3.92 & 
652.48 & 4.12 & 651.10 & 
-0.00\\SCA3-7 & 666.15 & 3.45 & 
666.15 & 3.58 & \bf{666.10} & 
0.01\\SCA3-8 & \bf{\underline{719.47}} & 4.14 & 
719.62 & 4.15 & 719.50 & 
-0.00\\SCA3-9 & \bf{681.00} & 3.46 & 
681.00 & 3.52 & 681.00 & 0.00\\
SCA8-0 & \bf{\underline{961.50}} & 4.58 & 
979.18 & 4.62 & 961.60 & 
-0.01\\SCA8-1 & \bf{\underline{1056.27}} & 3.79 & 
1059.36 & 3.65 & 1063.00 & 
-0.63\\SCA8-2 & 1047.63 & 3.46 & 
1049.63 & 3.54 & \bf{1040.60} & 
0.68\\SCA8-3 & 995.60 & 4.53 & 
1010.49 & 4.75 & \bf{985.90} & 
0.98\\SCA8-4 & \bf{\underline{1067.28}} & 4.64 & 
1069.91 & 4.72 & 1071.00 & 
-0.35\\SCA8-5 & \bf{\underline{1048.65}} & 5.08 & 
1055.07 & 5.17 & 1054.30 & 
-0.54\\SCA8-6 & \bf{\underline{972.48}} & 5.07 & 
977.49 & 4.83 & 972.50 & 
-0.00\\SCA8-7 & 1067.20 & 4.48 & 
1071.76 & 4.63 & \bf{1059.70} & 
0.71\\SCA8-8 & \bf{\underline{1071.18}} & 4.70 & 
1079.38 & 4.67 & 1082.70 & 
-1.06\\SCA8-9 & \bf{\underline{1067.42}} & 3.92 & 
1067.42 & 3.82 & 1081.40 & 
-1.29\\CON3-0 & 620.76 & 4.82 & 
621.80 & 4.83 & \bf{616.50} & 
0.69\\CON3-1 & \bf{\underline{554.47}} & 4.18 & 
555.39 & 4.48 & 555.60 & 
-0.20\\CON3-2 & \bf{\underline{519.11}} & 4.38 & 
520.81 & 4.19 & 521.40 & 
-0.44\\CON3-3 & \bf{\underline{591.19}} & 4.58 & 
591.24 & 4.67 & 591.20 & 
-0.00\\CON3-4 & \bf{\underline{588.79}} & 4.01 & 
588.79 & 4.08 & 589.30 & 
-0.09\\CON3-5 & \bf{563.70} & 4.14 & 
564.59 & 4.24 & 563.70 & 0.00\\
CON3-6 & 500.80 & 4.94 & 
501.87 & 5.04 & \bf{499.20} & 
0.32\\CON3-7 & 578.22 & 3.76 & 
578.32 & 3.81 & \bf{577.50} & 
0.12\\CON3-8 & 523.14 & 4.00 & 
523.43 & 3.88 & \bf{523.10} & 
0.01\\CON3-9 & 586.31 & 4.13 & 
588.19 & 4.12 & \bf{578.20} & 
1.40\\CON8-0 & 869.43 & 4.30 & 
876.29 & 4.42 & \bf{858.90} & 
1.23\\CON8-1 & \bf{\underline{740.85}} & 4.58 & 
742.91 & 4.55 & 740.90 & 
-0.01\\CON8-2 & \bf{\underline{713.44}} & 5.95 & 
714.18 & 5.64 & 714.30 & 
-0.12\\CON8-3 & 814.50 & 4.37 & 
816.47 & 4.57 & \bf{812.30} & 
0.27\\CON8-4 & 776.72 & 4.28 & 
787.58 & 4.27 & \bf{770.10} & 
0.86\\CON8-5 & \bf{\underline{758.12}} & 4.22 & 
761.67 & 4.20 & 766.60 & 
-1.11\\CON8-6 & \bf{\underline{684.95}} & 5.34 & 
691.45 & 5.08 & 697.20 & 
-1.76\\CON8-7 & 814.86 & 3.79 & 
814.86 & 3.86 & \bf{814.80} & 
0.01\\CON8-8 & 779.64 & 5.10 & 
784.05 & 5.00 & \bf{771.30} & 
1.08\\CON8-9 & \bf{\underline{810.18}} & 5.17 & 
812.28 & 5.10 & 815.10 & 
-0.60\\[1ex]\hline
\end{tabular}
\label{table:nonlin}
\end{table} \clearpage
\begin{table}[ht]
\caption{Resultados de la ejecución de la metaheurística ACO, utilizando instancias de Dethloff con la configuración -n 6.0 -alpha 1.0 -beta 3.0 -q .6 -ro 0.015}
\centering
\small
\begin{tabular}{c c c c c c c}
\hline\hline
Instancia & Costo mínimo & Tiempo(seg.) & Costo promedio & Tiempo promedio(seg.) & Costo ACO & \%Gap \\ [0.5ex]
\hline
SCA3-0 & \bf{\underline{636.06}} & 4.36 & 
636.06 & 4.17 & 636.10 & 
-0.01\\SCA3-1 & \bf{\underline{697.84}} & 4.40 & 
697.84 & 4.55 & 700.10 & 
-0.32\\SCA3-2 & 659.34 & 4.12 & 
661.76 & 4.07 & \bf{659.30} & 
0.01\\SCA3-3 & 680.04 & 3.58 & 
680.32 & 3.85 & \bf{680.00} & 
0.01\\SCA3-4 & \bf{690.50} & 4.35 & 
690.50 & 4.45 & 690.50 & 0.00\\
SCA3-5 & \bf{\underline{662.75}} & 4.39 & 
664.92 & 4.50 & 671.10 & 
-1.24\\SCA3-6 & \bf{\underline{651.09}} & 4.08 & 
652.66 & 4.18 & 651.10 & 
-0.00\\SCA3-7 & 666.15 & 3.80 & 
666.15 & 3.66 & \bf{666.10} & 
0.01\\SCA3-8 & \bf{\underline{719.47}} & 3.92 & 
720.04 & 3.94 & 719.50 & 
-0.00\\SCA3-9 & \bf{681.00} & 3.45 & 
681.00 & 3.44 & 681.00 & 0.00\\
SCA8-0 & 975.50 & 4.65 & 
987.29 & 4.58 & \bf{961.60} & 
1.45\\SCA8-1 & \bf{\underline{1054.87}} & 3.81 & 
1063.44 & 3.70 & 1063.00 & 
-0.76\\SCA8-2 & 1046.29 & 3.46 & 
1050.01 & 3.37 & \bf{1040.60} & 
0.55\\SCA8-3 & 1007.97 & 4.44 & 
1012.78 & 4.47 & \bf{985.90} & 
2.24\\SCA8-4 & \bf{\underline{1065.49}} & 5.05 & 
1069.98 & 4.55 & 1071.00 & 
-0.51\\SCA8-5 & \bf{\underline{1034.74}} & 4.89 & 
1048.97 & 4.79 & 1054.30 & 
-1.86\\SCA8-6 & 980.91 & 5.06 & 
981.03 & 4.88 & \bf{972.50} & 
0.86\\SCA8-7 & 1070.53 & 5.02 & 
1071.95 & 4.96 & \bf{1059.70} & 
1.02\\SCA8-8 & \bf{\underline{1071.18}} & 4.78 & 
1077.60 & 4.83 & 1082.70 & 
-1.06\\SCA8-9 & \bf{\underline{1067.26}} & 3.74 & 
1067.38 & 3.61 & 1081.40 & 
-1.31\\CON3-0 & 617.59 & 4.86 & 
622.73 & 4.70 & \bf{616.50} & 
0.18\\CON3-1 & 556.04 & 4.42 & 
557.62 & 4.56 & \bf{555.60} & 
0.08\\CON3-2 & \bf{\underline{521.38}} & 4.16 & 
521.38 & 4.17 & 521.40 & 
-0.00\\CON3-3 & \bf{\underline{591.19}} & 4.52 & 
591.20 & 4.62 & 591.20 & 
-0.00\\CON3-4 & \bf{\underline{588.79}} & 4.10 & 
590.40 & 4.05 & 589.30 & 
-0.09\\CON3-5 & 564.88 & 4.44 & 
566.41 & 4.25 & \bf{563.70} & 
0.21\\CON3-6 & 502.09 & 5.34 & 
503.33 & 5.57 & \bf{499.20} & 
0.58\\CON3-7 & 578.22 & 3.80 & 
579.08 & 3.85 & \bf{577.50} & 
0.12\\CON3-8 & \bf{\underline{523.05}} & 4.05 & 
524.21 & 3.93 & 523.10 & 
-0.01\\CON3-9 & 581.06 & 4.49 & 
586.45 & 4.11 & \bf{578.20} & 
0.49\\CON8-0 & 871.96 & 4.30 & 
875.76 & 4.34 & \bf{858.90} & 
1.52\\CON8-1 & \bf{\underline{740.85}} & 4.44 & 
741.25 & 4.43 & 740.90 & 
-0.01\\CON8-2 & \bf{\underline{713.90}} & 6.18 & 
715.12 & 5.88 & 714.30 & 
-0.06\\CON8-3 & 812.75 & 4.42 & 
814.83 & 4.46 & \bf{812.30} & 
0.06\\CON8-4 & 776.37 & 4.02 & 
787.78 & 4.34 & \bf{770.10} & 
0.81\\CON8-5 & \bf{\underline{758.12}} & 3.54 & 
761.03 & 3.98 & 766.60 & 
-1.11\\CON8-6 & \bf{\underline{695.31}} & 4.83 & 
697.05 & 4.87 & 697.20 & 
-0.27\\CON8-7 & 814.86 & 3.71 & 
816.50 & 3.70 & \bf{814.80} & 
0.01\\CON8-8 & 780.12 & 5.08 & 
785.72 & 5.02 & \bf{771.30} & 
1.14\\CON8-9 & 815.44 & 5.14 & 
817.03 & 4.92 & \bf{815.10} & 
0.04\\[1ex]\hline
\end{tabular}
\label{table:nonlin}
\end{table} \clearpage
\begin{table}[ht]
\caption{Resultados de la ejecución de la metaheurística ACO, utilizando instancias de Dethloff con la configuración -n 6.0 -alpha 1.0 -beta 3.0 -q .7 -ro 0.015}
\centering
\small
\begin{tabular}{c c c c c c c}
\hline\hline
Instancia & Costo mínimo & Tiempo(seg.) & Costo promedio & Tiempo promedio(seg.) & Costo ACO & \%Gap \\ [0.5ex]
\hline
SCA3-0 & \bf{\underline{636.06}} & 4.10 & 
636.06 & 4.17 & 636.10 & 
-0.01\\SCA3-1 & \bf{\underline{697.84}} & 4.65 & 
697.84 & 4.54 & 700.10 & 
-0.32\\SCA3-2 & 659.34 & 3.89 & 
660.60 & 3.91 & \bf{659.30} & 
0.01\\SCA3-3 & 680.04 & 4.03 & 
680.18 & 3.96 & \bf{680.00} & 
0.01\\SCA3-4 & \bf{690.50} & 4.20 & 
690.50 & 4.23 & 690.50 & 0.00\\
SCA3-5 & \bf{\underline{662.75}} & 4.37 & 
664.20 & 4.35 & 671.10 & 
-1.24\\SCA3-6 & 652.47 & 4.40 & 
652.82 & 4.26 & \bf{651.10} & 
0.21\\SCA3-7 & 666.15 & 3.02 & 
666.15 & 3.25 & \bf{666.10} & 
0.01\\SCA3-8 & \bf{\underline{719.47}} & 3.91 & 
719.47 & 3.92 & 719.50 & 
-0.00\\SCA3-9 & \bf{681.00} & 3.21 & 
681.00 & 3.45 & 681.00 & 0.00\\
SCA8-0 & 968.79 & 4.81 & 
974.86 & 4.72 & \bf{961.60} & 
0.75\\SCA8-1 & \bf{\underline{1061.71}} & 3.68 & 
1063.69 & 3.60 & 1063.00 & 
-0.12\\SCA8-2 & 1046.29 & 3.17 & 
1051.11 & 3.33 & \bf{1040.60} & 
0.55\\SCA8-3 & 1014.18 & 4.77 & 
1016.67 & 4.53 & \bf{985.90} & 
2.87\\SCA8-4 & \bf{\underline{1065.49}} & 4.56 & 
1071.44 & 4.41 & 1071.00 & 
-0.51\\SCA8-5 & \bf{\underline{1039.92}} & 4.92 & 
1048.98 & 4.84 & 1054.30 & 
-1.36\\SCA8-6 & \bf{\underline{972.48}} & 4.71 & 
977.76 & 4.83 & 972.50 & 
-0.00\\SCA8-7 & 1063.15 & 5.72 & 
1070.26 & 5.11 & \bf{1059.70} & 
0.33\\SCA8-8 & \bf{\underline{1071.18}} & 4.75 & 
1071.18 & 4.67 & 1082.70 & 
-1.06\\SCA8-9 & \bf{\underline{1067.42}} & 3.34 & 
1067.42 & 3.51 & 1081.40 & 
-1.29\\CON3-0 & 620.76 & 4.99 & 
621.78 & 4.86 & \bf{616.50} & 
0.69\\CON3-1 & \bf{\underline{554.47}} & 4.26 & 
555.20 & 4.36 & 555.60 & 
-0.20\\CON3-2 & \bf{\underline{521.38}} & 4.58 & 
522.10 & 4.03 & 521.40 & 
-0.00\\CON3-3 & \bf{\underline{591.19}} & 5.01 & 
591.20 & 4.67 & 591.20 & 
-0.00\\CON3-4 & \bf{\underline{588.79}} & 3.93 & 
589.58 & 3.92 & 589.30 & 
-0.09\\CON3-5 & 564.88 & 4.08 & 
564.88 & 4.45 & \bf{563.70} & 
0.21\\CON3-6 & 501.42 & 6.46 & 
502.88 & 5.38 & \bf{499.20} & 
0.44\\CON3-7 & \bf{\underline{576.87}} & 3.80 & 
580.25 & 3.72 & 577.50 & 
-0.11\\CON3-8 & 523.14 & 3.98 & 
526.43 & 3.88 & \bf{523.10} & 
0.01\\CON3-9 & 578.98 & 3.77 & 
585.32 & 3.78 & \bf{578.20} & 
0.13\\CON8-0 & 869.43 & 4.40 & 
879.20 & 4.41 & \bf{858.90} & 
1.23\\CON8-1 & \bf{\underline{740.85}} & 4.20 & 
742.41 & 4.22 & 740.90 & 
-0.01\\CON8-2 & \bf{\underline{713.90}} & 5.88 & 
715.09 & 5.71 & 714.30 & 
-0.06\\CON8-3 & 817.57 & 4.40 & 
817.57 & 4.37 & \bf{812.30} & 
0.65\\CON8-4 & 776.72 & 4.50 & 
783.97 & 4.45 & \bf{770.10} & 
0.86\\CON8-5 & \bf{\underline{758.99}} & 4.41 & 
762.67 & 4.25 & 766.60 & 
-0.99\\CON8-6 & \bf{\underline{683.83}} & 4.66 & 
693.54 & 4.86 & 697.20 & 
-1.92\\CON8-7 & 814.86 & 3.55 & 
820.42 & 3.75 & \bf{814.80} & 
0.01\\CON8-8 & 782.86 & 5.10 & 
786.72 & 4.85 & \bf{771.30} & 
1.50\\CON8-9 & \bf{\underline{812.44}} & 4.48 & 
814.76 & 4.61 & 815.10 & 
-0.33\\[1ex]\hline
\end{tabular}
\label{table:nonlin}
\end{table} \clearpage
\begin{table}[ht]
\caption{Resultados de la ejecución de la metaheurística ACO, utilizando instancias de Dethloff con la configuración -n 6.0 -alpha 1.0 -beta 3.0 -q .8 -ro 0.015}
\centering
\small
\begin{tabular}{c c c c c c c}
\hline\hline
Instancia & Costo mínimo & Tiempo(seg.) & Costo promedio & Tiempo promedio(seg.) & Costo ACO & \%Gap \\ [0.5ex]
\hline
SCA3-0 & \bf{\underline{636.06}} & 4.09 & 
636.20 & 4.10 & 636.10 & 
-0.01\\SCA3-1 & \bf{\underline{697.84}} & 4.40 & 
697.84 & 4.37 & 700.10 & 
-0.32\\SCA3-2 & 659.34 & 3.74 & 
661.76 & 3.90 & \bf{659.30} & 
0.01\\SCA3-3 & 680.04 & 4.29 & 
680.46 & 4.18 & \bf{680.00} & 
0.01\\SCA3-4 & \bf{690.50} & 4.10 & 
690.50 & 4.29 & 690.50 & 0.00\\
SCA3-5 & \bf{\underline{662.75}} & 4.36 & 
664.92 & 4.32 & 671.10 & 
-1.24\\SCA3-6 & 652.47 & 4.10 & 
654.56 & 4.09 & \bf{651.10} & 
0.21\\SCA3-7 & 666.15 & 3.26 & 
666.15 & 3.60 & \bf{666.10} & 
0.01\\SCA3-8 & \bf{\underline{719.47}} & 3.63 & 
724.16 & 3.73 & 719.50 & 
-0.00\\SCA3-9 & \bf{681.00} & 3.14 & 
681.00 & 3.32 & 681.00 & 0.00\\
SCA8-0 & 973.03 & 4.66 & 
978.25 & 4.56 & \bf{961.60} & 
1.19\\SCA8-1 & \bf{\underline{1049.65}} & 3.56 & 
1055.85 & 3.65 & 1063.00 & 
-1.26\\SCA8-2 & 1046.29 & 3.16 & 
1051.11 & 3.23 & \bf{1040.60} & 
0.55\\SCA8-3 & 1008.29 & 4.48 & 
1014.59 & 4.48 & \bf{985.90} & 
2.27\\SCA8-4 & \bf{\underline{1067.66}} & 4.38 & 
1079.64 & 4.51 & 1071.00 & 
-0.31\\SCA8-5 & 1055.35 & 5.75 & 
1057.25 & 5.10 & \bf{1054.30} & 
0.10\\SCA8-6 & 980.91 & 5.02 & 
983.43 & 4.95 & \bf{972.50} & 
0.86\\SCA8-7 & 1067.20 & 5.04 & 
1067.20 & 4.90 & \bf{1059.70} & 
0.71\\SCA8-8 & \bf{\underline{1071.18}} & 4.72 & 
1073.91 & 4.50 & 1082.70 & 
-1.06\\SCA8-9 & \bf{\underline{1067.42}} & 3.68 & 
1067.42 & 3.66 & 1081.40 & 
-1.29\\CON3-0 & 623.08 & 4.66 & 
624.48 & 4.66 & \bf{616.50} & 
1.07\\CON3-1 & \bf{\underline{554.47}} & 4.11 & 
555.84 & 4.15 & 555.60 & 
-0.20\\CON3-2 & \bf{\underline{521.38}} & 4.13 & 
522.80 & 3.92 & 521.40 & 
-0.00\\CON3-3 & \bf{\underline{591.19}} & 4.51 & 
591.20 & 4.71 & 591.20 & 
-0.00\\CON3-4 & \bf{\underline{588.79}} & 3.71 & 
588.92 & 3.68 & 589.30 & 
-0.09\\CON3-5 & 564.88 & 4.16 & 
565.92 & 4.33 & \bf{563.70} & 
0.21\\CON3-6 & 502.16 & 5.50 & 
503.23 & 5.26 & \bf{499.20} & 
0.59\\CON3-7 & 578.41 & 3.91 & 
580.40 & 3.81 & \bf{577.50} & 
0.16\\CON3-8 & 523.68 & 3.61 & 
526.34 & 3.73 & \bf{523.10} & 
0.11\\CON3-9 & 578.98 & 3.57 & 
582.05 & 3.93 & \bf{578.20} & 
0.13\\CON8-0 & 869.43 & 4.32 & 
877.39 & 4.33 & \bf{858.90} & 
1.23\\CON8-1 & \bf{\underline{740.85}} & 4.40 & 
744.33 & 4.44 & 740.90 & 
-0.01\\CON8-2 & \bf{\underline{713.44}} & 6.37 & 
715.68 & 5.84 & 714.30 & 
-0.12\\CON8-3 & 813.40 & 4.25 & 
816.53 & 4.30 & \bf{812.30} & 
0.14\\CON8-4 & 778.37 & 4.57 & 
786.75 & 4.55 & \bf{770.10} & 
1.07\\CON8-5 & \bf{\underline{761.40}} & 4.29 & 
763.90 & 4.35 & 766.60 & 
-0.68\\CON8-6 & \bf{\underline{695.66}} & 4.78 & 
696.03 & 5.05 & 697.20 & 
-0.22\\CON8-7 & \bf{\underline{814.79}} & 3.59 & 
815.27 & 3.71 & 814.80 & 
-0.00\\CON8-8 & 782.86 & 4.76 & 
787.70 & 4.88 & \bf{771.30} & 
1.50\\CON8-9 & \bf{\underline{814.37}} & 4.91 & 
816.33 & 4.78 & 815.10 & 
-0.09\\[1ex]\hline
\end{tabular}
\label{table:nonlin}
\end{table} \clearpage
\begin{table}[ht]
\caption{Resultados de la ejecución de la metaheurística ACO, utilizando instancias de Dethloff con la configuración -n 6.0 -alpha 1.0 -beta 3.0 -q .9 -ro 0.015}
\centering
\small
\begin{tabular}{c c c c c c c}
\hline\hline
Instancia & Costo mínimo & Tiempo(seg.) & Costo promedio & Tiempo promedio(seg.) & Costo ACO & \%Gap \\ [0.5ex]
\hline
SCA3-0 & \bf{\underline{636.06}} & 3.84 & 
636.13 & 3.96 & 636.10 & 
-0.01\\SCA3-1 & \bf{\underline{697.84}} & 4.56 & 
697.84 & 4.44 & 700.10 & 
-0.32\\SCA3-2 & 659.34 & 4.11 & 
661.00 & 4.11 & \bf{659.30} & 
0.01\\SCA3-3 & 680.04 & 5.42 & 
680.82 & 4.67 & \bf{680.00} & 
0.01\\SCA3-4 & \bf{690.50} & 4.40 & 
690.50 & 4.26 & 690.50 & 0.00\\
SCA3-5 & \bf{\underline{662.75}} & 3.98 & 
664.92 & 4.42 & 671.10 & 
-1.24\\SCA3-6 & 652.94 & 4.13 & 
653.69 & 4.15 & \bf{651.10} & 
0.28\\SCA3-7 & 666.15 & 3.04 & 
666.15 & 3.17 & \bf{666.10} & 
0.01\\SCA3-8 & \bf{\underline{719.47}} & 3.52 & 
721.17 & 3.52 & 719.50 & 
-0.00\\SCA3-9 & \bf{681.00} & 3.16 & 
681.00 & 3.20 & 681.00 & 0.00\\
SCA8-0 & 965.26 & 4.56 & 
980.50 & 4.42 & \bf{961.60} & 
0.38\\SCA8-1 & \bf{\underline{1059.21}} & 3.50 & 
1063.86 & 3.55 & 1063.00 & 
-0.36\\SCA8-2 & 1046.29 & 3.02 & 
1051.10 & 3.19 & \bf{1040.60} & 
0.55\\SCA8-3 & 995.50 & 4.34 & 
1015.66 & 4.58 & \bf{985.90} & 
0.97\\SCA8-4 & \bf{\underline{1065.49}} & 4.72 & 
1065.94 & 4.57 & 1071.00 & 
-0.51\\SCA8-5 & 1054.85 & 4.66 & 
1055.92 & 4.92 & \bf{1054.30} & 
0.05\\SCA8-6 & 977.03 & 4.79 & 
979.11 & 5.13 & \bf{972.50} & 
0.47\\SCA8-7 & 1067.20 & 5.01 & 
1067.20 & 4.92 & \bf{1059.70} & 
0.71\\SCA8-8 & \bf{\underline{1071.18}} & 4.46 & 
1077.94 & 4.51 & 1082.70 & 
-1.06\\SCA8-9 & \bf{\underline{1067.42}} & 3.39 & 
1067.42 & 3.50 & 1081.40 & 
-1.29\\CON3-0 & 620.76 & 5.07 & 
623.12 & 4.92 & \bf{616.50} & 
0.69\\CON3-1 & \bf{\underline{554.47}} & 4.91 & 
557.28 & 4.67 & 555.60 & 
-0.20\\CON3-2 & \bf{\underline{521.38}} & 3.70 & 
521.44 & 4.05 & 521.40 & 
-0.00\\CON3-3 & \bf{\underline{591.19}} & 4.47 & 
592.65 & 4.58 & 591.20 & 
-0.00\\CON3-4 & \bf{\underline{588.79}} & 4.03 & 
588.92 & 3.92 & 589.30 & 
-0.09\\CON3-5 & \bf{563.70} & 4.30 & 
567.48 & 3.97 & 563.70 & 0.00\\
CON3-6 & 502.16 & 5.10 & 
503.15 & 5.33 & \bf{499.20} & 
0.59\\CON3-7 & 578.41 & 3.59 & 
578.41 & 3.77 & \bf{577.50} & 
0.16\\CON3-8 & 524.30 & 3.66 & 
524.39 & 3.48 & \bf{523.10} & 
0.23\\CON3-9 & 578.25 & 4.05 & 
585.92 & 3.88 & \bf{578.20} & 
0.01\\CON8-0 & 869.43 & 4.29 & 
876.61 & 4.31 & \bf{858.90} & 
1.23\\CON8-1 & 742.29 & 4.55 & 
744.72 & 4.42 & \bf{740.90} & 
0.19\\CON8-2 & \bf{\underline{713.90}} & 6.21 & 
714.56 & 5.94 & 714.30 & 
-0.06\\CON8-3 & 817.57 & 4.24 & 
817.57 & 4.18 & \bf{812.30} & 
0.65\\CON8-4 & 781.64 & 4.34 & 
788.67 & 4.51 & \bf{770.10} & 
1.50\\CON8-5 & \bf{\underline{764.36}} & 4.00 & 
765.24 & 4.25 & 766.60 & 
-0.29\\CON8-6 & \bf{\underline{692.75}} & 5.30 & 
694.66 & 5.07 & 697.20 & 
-0.64\\CON8-7 & 816.14 & 3.41 & 
821.32 & 3.60 & \bf{814.80} & 
0.16\\CON8-8 & 791.18 & 4.68 & 
793.58 & 4.70 & \bf{771.30} & 
2.58\\CON8-9 & \bf{\underline{811.75}} & 4.84 & 
816.41 & 4.75 & 815.10 & 
-0.41\\[1ex]\hline
\end{tabular}
\label{table:nonlin}
\end{table} \clearpage
\begin{table}[ht]
\caption{Resultados de la ejecución de la metaheurística ACO, utilizando instancias de Dethloff con la configuración -n 7.0 -alpha 1.0 -beta 3.0 -q 0.1 -ro 0.015}
\centering
\small
\begin{tabular}{c c c c c c c}
\hline\hline
Instancia & Costo mínimo & Tiempo(seg.) & Costo promedio & Tiempo promedio(seg.) & Costo ACO & \%Gap \\ [0.5ex]
\hline
SCA3-0 & \bf{\underline{636.06}} & 5.26 & 
636.06 & 5.10 & 636.10 & 
-0.01\\SCA3-1 & \bf{\underline{697.84}} & 5.28 & 
697.84 & 5.48 & 700.10 & 
-0.32\\SCA3-2 & 659.34 & 4.99 & 
660.68 & 4.87 & \bf{659.30} & 
0.01\\SCA3-3 & 680.04 & 4.71 & 
680.04 & 4.78 & \bf{680.00} & 
0.01\\SCA3-4 & \bf{690.50} & 5.51 & 
690.50 & 5.34 & 690.50 & 0.00\\
SCA3-5 & \bf{\underline{659.90}} & 5.01 & 
662.04 & 5.06 & 671.10 & 
-1.67\\SCA3-6 & \bf{\underline{651.09}} & 4.98 & 
652.36 & 5.14 & 651.10 & 
-0.00\\SCA3-7 & \bf{\underline{659.17}} & 4.48 & 
664.40 & 4.64 & 666.10 & 
-1.04\\SCA3-8 & \bf{\underline{719.47}} & 4.98 & 
719.54 & 4.98 & 719.50 & 
-0.00\\SCA3-9 & \bf{681.00} & 4.24 & 
681.00 & 4.40 & 681.00 & 0.00\\
SCA8-0 & 972.04 & 5.75 & 
977.08 & 5.63 & \bf{961.60} & 
1.09\\SCA8-1 & \bf{\underline{1054.87}} & 4.62 & 
1063.74 & 5.12 & 1063.00 & 
-0.76\\SCA8-2 & 1045.64 & 4.34 & 
1049.19 & 4.36 & \bf{1040.60} & 
0.48\\SCA8-3 & 997.17 & 4.94 & 
1002.44 & 5.12 & \bf{985.90} & 
1.14\\SCA8-4 & \bf{\underline{1065.49}} & 5.05 & 
1066.58 & 5.28 & 1071.00 & 
-0.51\\SCA8-5 & \bf{\underline{1049.11}} & 5.76 & 
1051.40 & 5.92 & 1054.30 & 
-0.49\\SCA8-6 & \bf{\underline{972.48}} & 5.93 & 
977.79 & 5.72 & 972.50 & 
-0.00\\SCA8-7 & 1063.60 & 5.86 & 
1066.30 & 5.69 & \bf{1059.70} & 
0.37\\SCA8-8 & \bf{\underline{1071.18}} & 5.78 & 
1074.05 & 5.64 & 1082.70 & 
-1.06\\SCA8-9 & \bf{\underline{1063.68}} & 4.48 & 
1066.49 & 4.81 & 1081.40 & 
-1.64\\CON3-0 & 616.52 & 5.64 & 
619.83 & 5.56 & \bf{616.50} & 
0.00\\CON3-1 & \bf{\underline{554.47}} & 5.52 & 
555.59 & 5.48 & 555.60 & 
-0.20\\CON3-2 & \bf{\underline{520.46}} & 5.56 & 
521.15 & 5.35 & 521.40 & 
-0.18\\CON3-3 & \bf{591.20} & 5.66 & 
591.31 & 5.63 & 591.20 & 0.00\\
CON3-4 & \bf{\underline{588.79}} & 5.09 & 
589.06 & 5.04 & 589.30 & 
-0.09\\CON3-5 & 564.88 & 5.16 & 
565.84 & 5.16 & \bf{563.70} & 
0.21\\CON3-6 & 500.80 & 5.89 & 
501.87 & 6.04 & \bf{499.20} & 
0.32\\CON3-7 & \bf{\underline{576.48}} & 4.68 & 
577.83 & 4.67 & 577.50 & 
-0.18\\CON3-8 & \bf{\underline{523.05}} & 5.42 & 
523.48 & 5.08 & 523.10 & 
-0.01\\CON3-9 & 578.98 & 5.06 & 
585.15 & 4.92 & \bf{578.20} & 
0.13\\CON8-0 & 866.22 & 5.57 & 
876.92 & 5.40 & \bf{858.90} & 
0.85\\CON8-1 & 742.29 & 5.42 & 
742.39 & 5.72 & \bf{740.90} & 
0.19\\CON8-2 & \bf{\underline{712.89}} & 6.46 & 
713.40 & 6.16 & 714.30 & 
-0.20\\CON8-3 & \bf{\underline{811.07}} & 5.38 & 
811.85 & 5.35 & 812.30 & 
-0.15\\CON8-4 & 776.72 & 5.26 & 
777.90 & 5.19 & \bf{770.10} & 
0.86\\CON8-5 & \bf{\underline{758.12}} & 5.90 & 
760.63 & 5.37 & 766.60 & 
-1.11\\CON8-6 & \bf{\underline{687.92}} & 6.04 & 
690.53 & 6.06 & 697.20 & 
-1.33\\CON8-7 & 814.86 & 4.89 & 
814.86 & 4.84 & \bf{814.80} & 
0.01\\CON8-8 & 777.71 & 5.83 & 
784.59 & 6.02 & \bf{771.30} & 
0.83\\CON8-9 & \bf{\underline{810.18}} & 5.99 & 
811.48 & 5.98 & 815.10 & 
-0.60\\[1ex]\hline
\end{tabular}
\label{table:nonlin}
\end{table} \clearpage
\begin{table}[ht]
\caption{Resultados de la ejecución de la metaheurística ACO, utilizando instancias de Dethloff con la configuración -n 7.0 -alpha 1.0 -beta 3.0 -q .2 -ro 0.015}
\centering
\small
\begin{tabular}{c c c c c c c}
\hline\hline
Instancia & Costo mínimo & Tiempo(seg.) & Costo promedio & Tiempo promedio(seg.) & Costo ACO & \%Gap \\ [0.5ex]
\hline
SCA3-0 & \bf{\underline{636.06}} & 5.11 & 
636.06 & 5.02 & 636.10 & 
-0.01\\SCA3-1 & \bf{\underline{697.84}} & 5.50 & 
697.84 & 5.53 & 700.10 & 
-0.32\\SCA3-2 & 659.34 & 4.72 & 
660.55 & 4.81 & \bf{659.30} & 
0.01\\SCA3-3 & 680.04 & 5.16 & 
680.18 & 5.04 & \bf{680.00} & 
0.01\\SCA3-4 & \bf{690.50} & 4.93 & 
690.50 & 5.17 & 690.50 & 0.00\\
SCA3-5 & \bf{\underline{659.90}} & 5.15 & 
662.04 & 5.14 & 671.10 & 
-1.67\\SCA3-6 & 652.94 & 4.72 & 
652.94 & 5.01 & \bf{651.10} & 
0.28\\SCA3-7 & \bf{\underline{664.88}} & 4.56 & 
665.83 & 4.50 & 666.10 & 
-0.18\\SCA3-8 & \bf{\underline{719.47}} & 5.14 & 
720.04 & 5.04 & 719.50 & 
-0.00\\SCA3-9 & \bf{681.00} & 4.25 & 
681.00 & 4.25 & 681.00 & 0.00\\
SCA8-0 & \bf{\underline{961.50}} & 5.40 & 
970.31 & 5.44 & 961.60 & 
-0.01\\SCA8-1 & \bf{\underline{1052.71}} & 4.74 & 
1059.38 & 4.71 & 1063.00 & 
-0.97\\SCA8-2 & 1049.22 & 4.02 & 
1050.08 & 4.26 & \bf{1040.60} & 
0.83\\SCA8-3 & 991.84 & 5.09 & 
1006.87 & 5.12 & \bf{985.90} & 
0.60\\SCA8-4 & \bf{\underline{1065.49}} & 5.25 & 
1065.49 & 5.48 & 1071.00 & 
-0.51\\SCA8-5 & \bf{\underline{1043.65}} & 5.64 & 
1047.83 & 5.60 & 1054.30 & 
-1.01\\SCA8-6 & \bf{\underline{972.48}} & 6.80 & 
978.58 & 6.06 & 972.50 & 
-0.00\\SCA8-7 & 1067.03 & 5.72 & 
1071.44 & 5.67 & \bf{1059.70} & 
0.69\\SCA8-8 & \bf{\underline{1071.18}} & 5.66 & 
1073.91 & 5.75 & 1082.70 & 
-1.06\\SCA8-9 & \bf{\underline{1067.42}} & 4.48 & 
1067.42 & 4.38 & 1081.40 & 
-1.29\\CON3-0 & 620.49 & 5.72 & 
622.25 & 5.49 & \bf{616.50} & 
0.65\\CON3-1 & 556.28 & 5.46 & 
557.14 & 5.25 & \bf{555.60} & 
0.12\\CON3-2 & \bf{\underline{519.11}} & 5.12 & 
520.37 & 5.06 & 521.40 & 
-0.44\\CON3-3 & \bf{\underline{591.19}} & 5.60 & 
591.20 & 5.68 & 591.20 & 
-0.00\\CON3-4 & \bf{\underline{588.79}} & 4.75 & 
589.86 & 4.76 & 589.30 & 
-0.09\\CON3-5 & \bf{563.70} & 4.62 & 
566.09 & 4.80 & 563.70 & 0.00\\
CON3-6 & 502.09 & 6.30 & 
502.60 & 6.28 & \bf{499.20} & 
0.58\\CON3-7 & 577.54 & 4.82 & 
578.70 & 4.92 & \bf{577.50} & 
0.01\\CON3-8 & \bf{\underline{523.05}} & 4.38 & 
523.77 & 4.76 & 523.10 & 
-0.01\\CON3-9 & 580.05 & 4.96 & 
585.83 & 4.83 & \bf{578.20} & 
0.32\\CON8-0 & 866.57 & 5.75 & 
874.37 & 5.55 & \bf{858.90} & 
0.89\\CON8-1 & 742.29 & 5.52 & 
743.06 & 5.55 & \bf{740.90} & 
0.19\\CON8-2 & \bf{\underline{713.05}} & 6.03 & 
714.36 & 6.43 & 714.30 & 
-0.17\\CON8-3 & 812.54 & 5.52 & 
815.30 & 5.42 & \bf{812.30} & 
0.03\\CON8-4 & 776.60 & 5.70 & 
779.89 & 5.33 & \bf{770.10} & 
0.84\\CON8-5 & \bf{\underline{758.84}} & 5.22 & 
760.46 & 5.05 & 766.60 & 
-1.01\\CON8-6 & \bf{\underline{685.06}} & 6.13 & 
689.67 & 6.17 & 697.20 & 
-1.74\\CON8-7 & \bf{\underline{814.50}} & 4.57 & 
814.73 & 4.64 & 814.80 & 
-0.04\\CON8-8 & 782.86 & 5.97 & 
786.25 & 6.21 & \bf{771.30} & 
1.50\\CON8-9 & \bf{\underline{811.43}} & 5.90 & 
813.12 & 5.82 & 815.10 & 
-0.45\\[1ex]\hline
\end{tabular}
\label{table:nonlin}
\end{table} \clearpage
\begin{table}[ht]
\caption{Resultados de la ejecución de la metaheurística ACO, utilizando instancias de Dethloff con la configuración -n 7.0 -alpha 1.0 -beta 3.0 -q .3 -ro 0.015}
\centering
\small
\begin{tabular}{c c c c c c c}
\hline\hline
Instancia & Costo mínimo & Tiempo(seg.) & Costo promedio & Tiempo promedio(seg.) & Costo ACO & \%Gap \\ [0.5ex]
\hline
SCA3-0 & \bf{\underline{636.06}} & 5.26 & 
636.06 & 4.88 & 636.10 & 
-0.01\\SCA3-1 & \bf{\underline{697.84}} & 5.28 & 
697.84 & 5.38 & 700.10 & 
-0.32\\SCA3-2 & 659.34 & 4.95 & 
659.34 & 5.06 & \bf{659.30} & 
0.01\\SCA3-3 & 680.04 & 4.83 & 
680.18 & 4.78 & \bf{680.00} & 
0.01\\SCA3-4 & \bf{690.50} & 4.99 & 
690.50 & 5.17 & 690.50 & 0.00\\
SCA3-5 & \bf{\underline{659.90}} & 5.19 & 
663.48 & 5.18 & 671.10 & 
-1.67\\SCA3-6 & \bf{\underline{651.09}} & 5.16 & 
652.54 & 4.91 & 651.10 & 
-0.00\\SCA3-7 & \bf{\underline{659.17}} & 4.48 & 
664.40 & 4.43 & 666.10 & 
-1.04\\SCA3-8 & \bf{\underline{719.47}} & 5.06 & 
719.70 & 5.05 & 719.50 & 
-0.00\\SCA3-9 & \bf{681.00} & 4.33 & 
681.00 & 4.24 & 681.00 & 0.00\\
SCA8-0 & \bf{\underline{961.50}} & 5.43 & 
971.72 & 5.38 & 961.60 & 
-0.01\\SCA8-1 & \bf{\underline{1050.93}} & 4.66 & 
1055.33 & 4.63 & 1063.00 & 
-1.14\\SCA8-2 & 1044.23 & 4.20 & 
1048.62 & 4.05 & \bf{1040.60} & 
0.35\\SCA8-3 & 1001.69 & 5.20 & 
1009.41 & 5.28 & \bf{985.90} & 
1.60\\SCA8-4 & \bf{\underline{1065.49}} & 5.55 & 
1065.49 & 5.23 & 1071.00 & 
-0.51\\SCA8-5 & \bf{\underline{1045.30}} & 5.47 & 
1051.42 & 5.71 & 1054.30 & 
-0.85\\SCA8-6 & \bf{\underline{972.48}} & 5.93 & 
978.61 & 5.67 & 972.50 & 
-0.00\\SCA8-7 & 1060.77 & 5.36 & 
1068.52 & 5.49 & \bf{1059.70} & 
0.10\\SCA8-8 & \bf{\underline{1071.18}} & 5.31 & 
1073.09 & 5.62 & 1082.70 & 
-1.06\\SCA8-9 & \bf{\underline{1067.42}} & 4.39 & 
1067.42 & 4.36 & 1081.40 & 
-1.29\\CON3-0 & 617.59 & 5.62 & 
618.76 & 5.51 & \bf{616.50} & 
0.18\\CON3-1 & \bf{\underline{554.47}} & 5.19 & 
555.15 & 5.12 & 555.60 & 
-0.20\\CON3-2 & \bf{\underline{519.11}} & 4.89 & 
521.50 & 4.78 & 521.40 & 
-0.44\\CON3-3 & \bf{\underline{591.19}} & 5.39 & 
591.19 & 5.47 & 591.20 & 
-0.00\\CON3-4 & \bf{\underline{588.79}} & 4.71 & 
589.19 & 4.65 & 589.30 & 
-0.09\\CON3-5 & 564.88 & 5.04 & 
567.30 & 5.06 & \bf{563.70} & 
0.21\\CON3-6 & 501.05 & 6.51 & 
502.38 & 6.07 & \bf{499.20} & 
0.37\\CON3-7 & 578.22 & 4.46 & 
579.08 & 4.57 & \bf{577.50} & 
0.12\\CON3-8 & \bf{\underline{523.05}} & 4.71 & 
523.77 & 4.81 & 523.10 & 
-0.01\\CON3-9 & 578.25 & 4.52 & 
585.46 & 4.73 & \bf{578.20} & 
0.01\\CON8-0 & 870.24 & 5.45 & 
875.92 & 5.08 & \bf{858.90} & 
1.32\\CON8-1 & \bf{\underline{740.85}} & 5.30 & 
742.53 & 5.20 & 740.90 & 
-0.01\\CON8-2 & \bf{\underline{713.44}} & 7.04 & 
713.63 & 6.90 & 714.30 & 
-0.12\\CON8-3 & \bf{\underline{812.22}} & 5.00 & 
815.03 & 5.25 & 812.30 & 
-0.01\\CON8-4 & 776.72 & 5.34 & 
783.66 & 5.13 & \bf{770.10} & 
0.86\\CON8-5 & \bf{\underline{758.84}} & 5.84 & 
760.96 & 5.25 & 766.60 & 
-1.01\\CON8-6 & \bf{\underline{684.69}} & 5.95 & 
692.30 & 5.93 & 697.20 & 
-1.79\\CON8-7 & \bf{\underline{814.50}} & 4.78 & 
814.98 & 4.62 & 814.80 & 
-0.04\\CON8-8 & 778.39 & 6.06 & 
782.94 & 6.73 & \bf{771.30} & 
0.92\\CON8-9 & \bf{\underline{812.03}} & 5.91 & 
814.24 & 5.93 & 815.10 & 
-0.38\\[1ex]\hline
\end{tabular}
\label{table:nonlin}
\end{table} \clearpage
\begin{table}[ht]
\caption{Resultados de la ejecución de la metaheurística ACO, utilizando instancias de Dethloff con la configuración -n 7.0 -alpha 1.0 -beta 3.0 -q .4 -ro 0.015}
\centering
\small
\begin{tabular}{c c c c c c c}
\hline\hline
Instancia & Costo mínimo & Tiempo(seg.) & Costo promedio & Tiempo promedio(seg.) & Costo ACO & \%Gap \\ [0.5ex]
\hline
SCA3-0 & \bf{\underline{636.06}} & 4.75 & 
636.06 & 4.86 & 636.10 & 
-0.01\\SCA3-1 & \bf{\underline{697.84}} & 5.40 & 
697.84 & 5.41 & 700.10 & 
-0.32\\SCA3-2 & 659.34 & 4.58 & 
661.00 & 4.67 & \bf{659.30} & 
0.01\\SCA3-3 & 680.04 & 5.00 & 
680.04 & 4.82 & \bf{680.00} & 
0.01\\SCA3-4 & \bf{690.50} & 5.19 & 
690.50 & 5.09 & 690.50 & 0.00\\
SCA3-5 & \bf{\underline{659.90}} & 5.14 & 
662.34 & 5.21 & 671.10 & 
-1.67\\SCA3-6 & \bf{\underline{651.09}} & 4.90 & 
651.55 & 5.13 & 651.10 & 
-0.00\\SCA3-7 & 666.15 & 4.12 & 
666.42 & 4.41 & \bf{666.10} & 
0.01\\SCA3-8 & \bf{\underline{719.47}} & 4.78 & 
719.47 & 4.83 & 719.50 & 
-0.00\\SCA3-9 & \bf{681.00} & 4.07 & 
681.00 & 4.28 & 681.00 & 0.00\\
SCA8-0 & 968.79 & 5.56 & 
976.24 & 5.63 & \bf{961.60} & 
0.75\\SCA8-1 & \bf{\underline{1053.09}} & 4.65 & 
1057.49 & 4.59 & 1063.00 & 
-0.93\\SCA8-2 & 1043.79 & 4.25 & 
1048.44 & 4.11 & \bf{1040.60} & 
0.31\\SCA8-3 & 1005.59 & 5.60 & 
1010.76 & 5.36 & \bf{985.90} & 
2.00\\SCA8-4 & \bf{\underline{1065.49}} & 5.44 & 
1067.57 & 5.08 & 1071.00 & 
-0.51\\SCA8-5 & \bf{\underline{1036.88}} & 5.56 & 
1041.92 & 5.63 & 1054.30 & 
-1.65\\SCA8-6 & \bf{\underline{972.48}} & 5.52 & 
978.25 & 5.58 & 972.50 & 
-0.00\\SCA8-7 & 1067.20 & 5.76 & 
1068.66 & 5.77 & \bf{1059.70} & 
0.71\\SCA8-8 & \bf{\underline{1071.18}} & 5.59 & 
1071.18 & 5.50 & 1082.70 & 
-1.06\\SCA8-9 & \bf{\underline{1067.42}} & 4.23 & 
1067.42 & 4.41 & 1081.40 & 
-1.29\\CON3-0 & 617.59 & 5.41 & 
621.00 & 5.39 & \bf{616.50} & 
0.18\\CON3-1 & \bf{\underline{554.47}} & 5.12 & 
557.02 & 5.12 & 555.60 & 
-0.20\\CON3-2 & \bf{\underline{519.11}} & 5.05 & 
520.94 & 5.09 & 521.40 & 
-0.44\\CON3-3 & \bf{\underline{591.19}} & 5.13 & 
591.20 & 5.45 & 591.20 & 
-0.00\\CON3-4 & \bf{\underline{588.79}} & 4.62 & 
589.86 & 4.57 & 589.30 & 
-0.09\\CON3-5 & 564.88 & 4.99 & 
565.40 & 4.90 & \bf{563.70} & 
0.21\\CON3-6 & 500.80 & 5.80 & 
501.61 & 5.85 & \bf{499.20} & 
0.32\\CON3-7 & \bf{\underline{576.84}} & 4.51 & 
577.97 & 4.53 & 577.50 & 
-0.11\\CON3-8 & \bf{\underline{523.05}} & 4.60 & 
523.88 & 4.74 & 523.10 & 
-0.01\\CON3-9 & 578.25 & 5.08 & 
583.26 & 4.73 & \bf{578.20} & 
0.01\\CON8-0 & 868.49 & 5.29 & 
873.38 & 5.28 & \bf{858.90} & 
1.12\\CON8-1 & \bf{\underline{740.85}} & 5.34 & 
741.42 & 5.49 & 740.90 & 
-0.01\\CON8-2 & \bf{\underline{713.44}} & 6.72 & 
715.24 & 6.48 & 714.30 & 
-0.12\\CON8-3 & \bf{\underline{812.11}} & 5.39 & 
815.78 & 5.40 & 812.30 & 
-0.02\\CON8-4 & 777.81 & 4.83 & 
783.64 & 5.06 & \bf{770.10} & 
1.00\\CON8-5 & \bf{\underline{760.03}} & 5.30 & 
761.00 & 5.09 & 766.60 & 
-0.86\\CON8-6 & \bf{\underline{690.19}} & 6.87 & 
693.01 & 6.13 & 697.20 & 
-1.01\\CON8-7 & \bf{\underline{814.79}} & 5.43 & 
814.81 & 4.83 & 814.80 & 
-0.00\\CON8-8 & 783.15 & 5.82 & 
784.75 & 6.09 & \bf{771.30} & 
1.54\\CON8-9 & \bf{\underline{812.60}} & 5.48 & 
813.92 & 5.75 & 815.10 & 
-0.31\\[1ex]\hline
\end{tabular}
\label{table:nonlin}
\end{table} \clearpage
\begin{table}[ht]
\caption{Resultados de la ejecución de la metaheurística ACO, utilizando instancias de Dethloff con la configuración -n 7.0 -alpha 1.0 -beta 3.0 -q .5 -ro 0.015}
\centering
\small
\begin{tabular}{c c c c c c c}
\hline\hline
Instancia & Costo mínimo & Tiempo(seg.) & Costo promedio & Tiempo promedio(seg.) & Costo ACO & \%Gap \\ [0.5ex]
\hline
SCA3-0 & \bf{\underline{636.06}} & 4.96 & 
636.06 & 4.85 & 636.10 & 
-0.01\\SCA3-1 & \bf{\underline{697.84}} & 5.25 & 
697.84 & 5.34 & 700.10 & 
-0.32\\SCA3-2 & 659.34 & 4.54 & 
661.00 & 4.77 & \bf{659.30} & 
0.01\\SCA3-3 & 680.04 & 4.96 & 
680.18 & 4.74 & \bf{680.00} & 
0.01\\SCA3-4 & \bf{690.50} & 5.37 & 
690.50 & 5.21 & 690.50 & 0.00\\
SCA3-5 & \bf{\underline{659.90}} & 4.92 & 
660.61 & 5.17 & 671.10 & 
-1.67\\SCA3-6 & 652.94 & 4.60 & 
652.94 & 4.74 & \bf{651.10} & 
0.28\\SCA3-7 & 666.15 & 4.36 & 
666.15 & 4.29 & \bf{666.10} & 
0.01\\SCA3-8 & \bf{\underline{719.47}} & 4.79 & 
719.54 & 4.80 & 719.50 & 
-0.00\\SCA3-9 & \bf{681.00} & 4.24 & 
681.00 & 4.12 & 681.00 & 0.00\\
SCA8-0 & 968.79 & 5.70 & 
980.55 & 5.58 & \bf{961.60} & 
0.75\\SCA8-1 & \bf{\underline{1061.71}} & 4.42 & 
1063.66 & 4.49 & 1063.00 & 
-0.12\\SCA8-2 & 1043.79 & 3.53 & 
1048.72 & 3.87 & \bf{1040.60} & 
0.31\\SCA8-3 & 1010.50 & 4.86 & 
1015.06 & 5.26 & \bf{985.90} & 
2.50\\SCA8-4 & \bf{\underline{1065.49}} & 5.35 & 
1067.24 & 5.42 & 1071.00 & 
-0.51\\SCA8-5 & \bf{\underline{1034.74}} & 5.88 & 
1039.54 & 5.75 & 1054.30 & 
-1.86\\SCA8-6 & \bf{\underline{972.48}} & 5.35 & 
978.20 & 5.55 & 972.50 & 
-0.00\\SCA8-7 & 1067.20 & 5.60 & 
1070.12 & 5.71 & \bf{1059.70} & 
0.71\\SCA8-8 & \bf{\underline{1071.18}} & 5.68 & 
1071.18 & 5.71 & 1082.70 & 
-1.06\\SCA8-9 & \bf{\underline{1067.42}} & 4.50 & 
1067.42 & 4.30 & 1081.40 & 
-1.29\\CON3-0 & 617.59 & 5.85 & 
619.43 & 5.63 & \bf{616.50} & 
0.18\\CON3-1 & \bf{\underline{554.47}} & 5.29 & 
555.97 & 5.14 & 555.60 & 
-0.20\\CON3-2 & \bf{\underline{519.11}} & 5.02 & 
520.88 & 4.90 & 521.40 & 
-0.44\\CON3-3 & \bf{\underline{591.19}} & 5.38 & 
591.31 & 5.33 & 591.20 & 
-0.00\\CON3-4 & \bf{\underline{588.79}} & 4.45 & 
588.92 & 4.58 & 589.30 & 
-0.09\\CON3-5 & \bf{563.70} & 5.37 & 
564.59 & 5.22 & 563.70 & 0.00\\
CON3-6 & \bf{\underline{499.05}} & 6.02 & 
501.93 & 5.92 & 499.20 & 
-0.03\\CON3-7 & \bf{\underline{576.48}} & 4.28 & 
578.27 & 4.53 & 577.50 & 
-0.18\\CON3-8 & 523.14 & 4.52 & 
523.57 & 4.64 & \bf{523.10} & 
0.01\\CON3-9 & 578.25 & 5.04 & 
585.19 & 4.76 & \bf{578.20} & 
0.01\\CON8-0 & 870.00 & 5.33 & 
875.32 & 5.13 & \bf{858.90} & 
1.29\\CON8-1 & \bf{\underline{740.85}} & 4.87 & 
744.58 & 5.16 & 740.90 & 
-0.01\\CON8-2 & \bf{\underline{712.89}} & 6.84 & 
713.46 & 6.68 & 714.30 & 
-0.20\\CON8-3 & \bf{\underline{811.07}} & 5.02 & 
815.39 & 5.11 & 812.30 & 
-0.15\\CON8-4 & 776.37 & 5.08 & 
781.98 & 5.19 & \bf{770.10} & 
0.81\\CON8-5 & \bf{\underline{762.01}} & 4.77 & 
762.77 & 4.88 & 766.60 & 
-0.60\\CON8-6 & \bf{\underline{683.83}} & 6.12 & 
690.12 & 6.09 & 697.20 & 
-1.92\\CON8-7 & \bf{\underline{814.79}} & 4.46 & 
814.98 & 4.44 & 814.80 & 
-0.00\\CON8-8 & 781.81 & 5.54 & 
784.84 & 6.02 & \bf{771.30} & 
1.36\\CON8-9 & \bf{\underline{810.18}} & 5.81 & 
812.91 & 5.79 & 815.10 & 
-0.60\\[1ex]\hline
\end{tabular}
\label{table:nonlin}
\end{table} \clearpage
\begin{table}[ht]
\caption{Resultados de la ejecución de la metaheurística ACO, utilizando instancias de Dethloff con la configuración -n 7.0 -alpha 1.0 -beta 3.0 -q .6 -ro 0.015}
\centering
\small
\begin{tabular}{c c c c c c c}
\hline\hline
Instancia & Costo mínimo & Tiempo(seg.) & Costo promedio & Tiempo promedio(seg.) & Costo ACO & \%Gap \\ [0.5ex]
\hline
SCA3-0 & \bf{\underline{636.06}} & 4.93 & 
636.06 & 5.00 & 636.10 & 
-0.01\\SCA3-1 & \bf{\underline{697.84}} & 5.16 & 
697.84 & 5.24 & 700.10 & 
-0.32\\SCA3-2 & 659.34 & 4.32 & 
661.00 & 4.91 & \bf{659.30} & 
0.01\\SCA3-3 & 680.04 & 4.90 & 
680.18 & 4.80 & \bf{680.00} & 
0.01\\SCA3-4 & \bf{690.50} & 5.03 & 
690.50 & 5.10 & 690.50 & 0.00\\
SCA3-5 & \bf{\underline{662.75}} & 5.17 & 
664.04 & 5.33 & 671.10 & 
-1.24\\SCA3-6 & 652.94 & 5.10 & 
652.94 & 4.81 & \bf{651.10} & 
0.28\\SCA3-7 & 666.15 & 4.34 & 
666.15 & 4.69 & \bf{666.10} & 
0.01\\SCA3-8 & \bf{\underline{719.47}} & 4.94 & 
719.54 & 4.73 & 719.50 & 
-0.00\\SCA3-9 & \bf{681.00} & 4.04 & 
681.00 & 4.04 & 681.00 & 0.00\\
SCA8-0 & 968.79 & 5.02 & 
983.81 & 5.30 & \bf{961.60} & 
0.75\\SCA8-1 & \bf{\underline{1050.93}} & 4.44 & 
1058.96 & 4.25 & 1063.00 & 
-1.14\\SCA8-2 & 1049.22 & 3.88 & 
1050.84 & 3.89 & \bf{1040.60} & 
0.83\\SCA8-3 & 1002.86 & 5.27 & 
1013.74 & 5.34 & \bf{985.90} & 
1.72\\SCA8-4 & \bf{\underline{1065.49}} & 5.18 & 
1071.03 & 5.12 & 1071.00 & 
-0.51\\SCA8-5 & \bf{\underline{1048.31}} & 5.83 & 
1053.63 & 5.90 & 1054.30 & 
-0.57\\SCA8-6 & \bf{\underline{972.48}} & 5.73 & 
975.73 & 5.74 & 972.50 & 
-0.00\\SCA8-7 & 1067.20 & 5.48 & 
1069.26 & 5.56 & \bf{1059.70} & 
0.71\\SCA8-8 & \bf{\underline{1071.18}} & 5.18 & 
1071.18 & 5.46 & 1082.70 & 
-1.06\\SCA8-9 & \bf{\underline{1067.42}} & 3.97 & 
1067.42 & 4.25 & 1081.40 & 
-1.29\\CON3-0 & 616.52 & 5.72 & 
618.92 & 5.83 & \bf{616.50} & 
0.00\\CON3-1 & \bf{\underline{554.47}} & 5.06 & 
555.32 & 5.26 & 555.60 & 
-0.20\\CON3-2 & \bf{\underline{519.61}} & 4.79 & 
520.96 & 4.86 & 521.40 & 
-0.34\\CON3-3 & \bf{\underline{591.19}} & 5.42 & 
591.19 & 5.50 & 591.20 & 
-0.00\\CON3-4 & \bf{\underline{588.79}} & 4.97 & 
589.58 & 4.95 & 589.30 & 
-0.09\\CON3-5 & 564.88 & 5.20 & 
566.37 & 5.35 & \bf{563.70} & 
0.21\\CON3-6 & 500.80 & 5.71 & 
502.44 & 6.17 & \bf{499.20} & 
0.32\\CON3-7 & 578.22 & 4.35 & 
579.05 & 4.50 & \bf{577.50} & 
0.12\\CON3-8 & 523.68 & 4.49 & 
524.86 & 4.57 & \bf{523.10} & 
0.11\\CON3-9 & 580.05 & 4.72 & 
584.68 & 4.77 & \bf{578.20} & 
0.32\\CON8-0 & 870.24 & 5.16 & 
876.32 & 5.17 & \bf{858.90} & 
1.32\\CON8-1 & \bf{\underline{740.85}} & 4.73 & 
741.57 & 4.89 & 740.90 & 
-0.01\\CON8-2 & \bf{\underline{713.44}} & 6.83 & 
714.10 & 6.99 & 714.30 & 
-0.12\\CON8-3 & 814.50 & 4.70 & 
815.71 & 4.88 & \bf{812.30} & 
0.27\\CON8-4 & 776.72 & 5.01 & 
783.83 & 5.17 & \bf{770.10} & 
0.86\\CON8-5 & \bf{\underline{758.12}} & 4.94 & 
760.39 & 5.08 & 766.60 & 
-1.11\\CON8-6 & \bf{\underline{692.75}} & 6.28 & 
695.79 & 5.96 & 697.20 & 
-0.64\\CON8-7 & \bf{\underline{814.79}} & 4.14 & 
816.54 & 4.32 & 814.80 & 
-0.00\\CON8-8 & 777.24 & 5.78 & 
785.02 & 5.90 & \bf{771.30} & 
0.77\\CON8-9 & \bf{\underline{810.18}} & 5.72 & 
813.53 & 6.15 & 815.10 & 
-0.60\\[1ex]\hline
\end{tabular}
\label{table:nonlin}
\end{table} \clearpage
\begin{table}[ht]
\caption{Resultados de la ejecución de la metaheurística ACO, utilizando instancias de Dethloff con la configuración -n 7.0 -alpha 1.0 -beta 3.0 -q .7 -ro 0.015}
\centering
\small
\begin{tabular}{c c c c c c c}
\hline\hline
Instancia & Costo mínimo & Tiempo(seg.) & Costo promedio & Tiempo promedio(seg.) & Costo ACO & \%Gap \\ [0.5ex]
\hline
SCA3-0 & \bf{\underline{636.06}} & 4.83 & 
636.13 & 4.83 & 636.10 & 
-0.01\\SCA3-1 & \bf{\underline{697.84}} & 4.97 & 
697.84 & 5.91 & 700.10 & 
-0.32\\SCA3-2 & 659.34 & 4.66 & 
662.97 & 4.79 & \bf{659.30} & 
0.01\\SCA3-3 & 680.04 & 5.00 & 
680.04 & 5.01 & \bf{680.00} & 
0.01\\SCA3-4 & \bf{690.50} & 5.36 & 
690.50 & 5.06 & 690.50 & 0.00\\
SCA3-5 & \bf{\underline{662.75}} & 5.30 & 
663.32 & 5.16 & 671.10 & 
-1.24\\SCA3-6 & 652.94 & 4.65 & 
652.94 & 4.63 & \bf{651.10} & 
0.28\\SCA3-7 & \bf{\underline{659.17}} & 3.83 & 
664.40 & 3.94 & 666.10 & 
-1.04\\SCA3-8 & \bf{\underline{719.47}} & 4.88 & 
722.33 & 4.47 & 719.50 & 
-0.00\\SCA3-9 & \bf{681.00} & 3.81 & 
681.00 & 4.56 & 681.00 & 0.00\\
SCA8-0 & 968.79 & 5.20 & 
984.75 & 5.43 & \bf{961.60} & 
0.75\\SCA8-1 & \bf{\underline{1053.44}} & 4.03 & 
1062.32 & 4.21 & 1063.00 & 
-0.90\\SCA8-2 & 1046.29 & 3.67 & 
1049.97 & 3.73 & \bf{1040.60} & 
0.55\\SCA8-3 & 1006.62 & 5.33 & 
1011.62 & 5.21 & \bf{985.90} & 
2.10\\SCA8-4 & \bf{\underline{1065.49}} & 5.29 & 
1068.29 & 5.23 & 1071.00 & 
-0.51\\SCA8-5 & \bf{\underline{1029.95}} & 5.97 & 
1048.48 & 5.73 & 1054.30 & 
-2.31\\SCA8-6 & 976.69 & 5.42 & 
978.88 & 5.52 & \bf{972.50} & 
0.43\\SCA8-7 & 1067.20 & 5.64 & 
1071.36 & 5.55 & \bf{1059.70} & 
0.71\\SCA8-8 & \bf{\underline{1071.18}} & 5.61 & 
1071.18 & 5.38 & 1082.70 & 
-1.06\\SCA8-9 & \bf{\underline{1067.42}} & 4.25 & 
1067.42 & 4.16 & 1081.40 & 
-1.29\\CON3-0 & 620.76 & 5.76 & 
625.94 & 5.54 & \bf{616.50} & 
0.69\\CON3-1 & \bf{\underline{554.47}} & 5.21 & 
556.63 & 5.12 & 555.60 & 
-0.20\\CON3-2 & \bf{\underline{521.38}} & 4.56 & 
522.56 & 4.59 & 521.40 & 
-0.00\\CON3-3 & \bf{\underline{591.19}} & 5.72 & 
591.27 & 5.74 & 591.20 & 
-0.00\\CON3-4 & \bf{\underline{588.79}} & 4.71 & 
589.74 & 4.54 & 589.30 & 
-0.09\\CON3-5 & \bf{563.70} & 4.84 & 
566.34 & 4.95 & 563.70 & 0.00\\
CON3-6 & 501.42 & 5.98 & 
502.67 & 6.04 & \bf{499.20} & 
0.44\\CON3-7 & 578.41 & 4.30 & 
579.15 & 4.33 & \bf{577.50} & 
0.16\\CON3-8 & 523.14 & 4.22 & 
523.77 & 4.29 & \bf{523.10} & 
0.01\\CON3-9 & 578.98 & 4.64 & 
586.09 & 4.54 & \bf{578.20} & 
0.13\\CON8-0 & 869.43 & 5.13 & 
872.01 & 5.13 & \bf{858.90} & 
1.23\\CON8-1 & \bf{\underline{740.85}} & 5.26 & 
741.59 & 5.13 & 740.90 & 
-0.01\\CON8-2 & \bf{\underline{713.44}} & 6.60 & 
716.10 & 6.59 & 714.30 & 
-0.12\\CON8-3 & \bf{\underline{811.23}} & 5.09 & 
814.64 & 5.10 & 812.30 & 
-0.13\\CON8-4 & 779.01 & 4.91 & 
784.10 & 4.93 & \bf{770.10} & 
1.16\\CON8-5 & \bf{\underline{758.84}} & 4.43 & 
761.77 & 4.66 & 766.60 & 
-1.01\\CON8-6 & \bf{\underline{688.93}} & 5.86 & 
695.02 & 6.49 & 697.20 & 
-1.19\\CON8-7 & \bf{\underline{814.77}} & 4.58 & 
816.41 & 4.34 & 814.80 & 
-0.00\\CON8-8 & 777.98 & 6.19 & 
785.54 & 5.76 & \bf{771.30} & 
0.87\\CON8-9 & \bf{\underline{812.60}} & 5.48 & 
815.88 & 5.58 & 815.10 & 
-0.31\\[1ex]\hline
\end{tabular}
\label{table:nonlin}
\end{table} \clearpage
\begin{table}[ht]
\caption{Resultados de la ejecución de la metaheurística ACO, utilizando instancias de Dethloff con la configuración -n 7.0 -alpha 1.0 -beta 3.0 -q .8 -ro 0.015}
\centering
\small
\begin{tabular}{c c c c c c c}
\hline\hline
Instancia & Costo mínimo & Tiempo(seg.) & Costo promedio & Tiempo promedio(seg.) & Costo ACO & \%Gap \\ [0.5ex]
\hline
SCA3-0 & \bf{\underline{636.06}} & 4.53 & 
636.13 & 4.53 & 636.10 & 
-0.01\\SCA3-1 & \bf{\underline{697.84}} & 5.02 & 
697.84 & 5.09 & 700.10 & 
-0.32\\SCA3-2 & 661.13 & 4.67 & 
661.89 & 4.73 & \bf{659.30} & 
0.28\\SCA3-3 & 680.04 & 5.66 & 
680.18 & 5.16 & \bf{680.00} & 
0.01\\SCA3-4 & \bf{690.50} & 5.38 & 
690.50 & 5.24 & 690.50 & 0.00\\
SCA3-5 & \bf{\underline{659.90}} & 5.44 & 
663.48 & 5.42 & 671.10 & 
-1.67\\SCA3-6 & 652.94 & 4.56 & 
652.94 & 4.66 & \bf{651.10} & 
0.28\\SCA3-7 & 666.15 & 3.83 & 
666.15 & 4.00 & \bf{666.10} & 
0.01\\SCA3-8 & \bf{\underline{719.47}} & 4.48 & 
721.66 & 4.30 & 719.50 & 
-0.00\\SCA3-9 & \bf{681.00} & 3.72 & 
681.00 & 3.77 & 681.00 & 0.00\\
SCA8-0 & \bf{\underline{961.50}} & 5.67 & 
971.71 & 5.38 & 961.60 & 
-0.01\\SCA8-1 & \bf{\underline{1055.94}} & 4.60 & 
1062.03 & 4.40 & 1063.00 & 
-0.66\\SCA8-2 & 1050.37 & 3.68 & 
1051.49 & 3.79 & \bf{1040.60} & 
0.94\\SCA8-3 & 1008.29 & 5.53 & 
1015.63 & 5.29 & \bf{985.90} & 
2.27\\SCA8-4 & \bf{\underline{1065.49}} & 5.62 & 
1071.57 & 5.22 & 1071.00 & 
-0.51\\SCA8-5 & \bf{\underline{1052.18}} & 6.00 & 
1056.65 & 5.82 & 1054.30 & 
-0.20\\SCA8-6 & \bf{\underline{972.48}} & 5.70 & 
978.93 & 5.70 & 972.50 & 
-0.00\\SCA8-7 & 1067.20 & 5.77 & 
1071.19 & 5.82 & \bf{1059.70} & 
0.71\\SCA8-8 & \bf{\underline{1071.18}} & 6.00 & 
1074.87 & 5.48 & 1082.70 & 
-1.06\\SCA8-9 & \bf{\underline{1067.42}} & 4.20 & 
1067.42 & 4.18 & 1081.40 & 
-1.29\\CON3-0 & 617.59 & 5.59 & 
621.10 & 5.54 & \bf{616.50} & 
0.18\\CON3-1 & \bf{\underline{554.47}} & 4.90 & 
556.85 & 4.89 & 555.60 & 
-0.20\\CON3-2 & \bf{\underline{521.38}} & 4.50 & 
523.27 & 4.51 & 521.40 & 
-0.00\\CON3-3 & \bf{591.20} & 5.59 & 
591.95 & 5.38 & 591.20 & 0.00\\
CON3-4 & \bf{\underline{588.79}} & 4.30 & 
588.79 & 4.52 & 589.30 & 
-0.09\\CON3-5 & \bf{563.70} & 4.86 & 
566.78 & 5.07 & 563.70 & 0.00\\
CON3-6 & 502.16 & 5.90 & 
502.75 & 5.95 & \bf{499.20} & 
0.59\\CON3-7 & 578.22 & 4.44 & 
580.80 & 4.48 & \bf{577.50} & 
0.12\\CON3-8 & \bf{\underline{523.05}} & 4.54 & 
523.82 & 4.44 & 523.10 & 
-0.01\\CON3-9 & 586.31 & 4.41 & 
588.25 & 4.39 & \bf{578.20} & 
1.40\\CON8-0 & 870.22 & 5.26 & 
876.49 & 5.13 & \bf{858.90} & 
1.32\\CON8-1 & \bf{\underline{740.85}} & 4.66 & 
743.63 & 4.87 & 740.90 & 
-0.01\\CON8-2 & 716.07 & 6.43 & 
716.18 & 6.55 & \bf{714.30} & 
0.25\\CON8-3 & 817.57 & 4.96 & 
817.57 & 4.94 & \bf{812.30} & 
0.65\\CON8-4 & 789.55 & 4.66 & 
789.76 & 4.91 & \bf{770.10} & 
2.53\\CON8-5 & \bf{\underline{760.74}} & 4.70 & 
762.87 & 4.77 & 766.60 & 
-0.76\\CON8-6 & \bf{\underline{693.10}} & 5.69 & 
696.63 & 5.87 & 697.20 & 
-0.59\\CON8-7 & \bf{\underline{814.50}} & 4.46 & 
817.33 & 4.36 & 814.80 & 
-0.04\\CON8-8 & 782.86 & 5.80 & 
787.44 & 5.64 & \bf{771.30} & 
1.50\\CON8-9 & \bf{\underline{812.60}} & 5.41 & 
814.30 & 5.43 & 815.10 & 
-0.31\\[1ex]\hline
\end{tabular}
\label{table:nonlin}
\end{table} \clearpage
\begin{table}[ht]
\caption{Resultados de la ejecución de la metaheurística ACO, utilizando instancias de Dethloff con la configuración -n 7.0 -alpha 1.0 -beta 3.0 -q .9 -ro 0.015}
\centering
\small
\begin{tabular}{c c c c c c c}
\hline\hline
Instancia & Costo mínimo & Tiempo(seg.) & Costo promedio & Tiempo promedio(seg.) & Costo ACO & \%Gap \\ [0.5ex]
\hline
SCA3-0 & \bf{\underline{636.06}} & 5.22 & 
637.32 & 4.95 & 636.10 & 
-0.01\\SCA3-1 & \bf{\underline{697.84}} & 5.02 & 
697.84 & 5.16 & 700.10 & 
-0.32\\SCA3-2 & 659.34 & 4.76 & 
664.09 & 4.74 & \bf{659.30} & 
0.01\\SCA3-3 & 680.04 & 4.84 & 
680.32 & 5.00 & \bf{680.00} & 
0.01\\SCA3-4 & \bf{690.50} & 4.80 & 
690.50 & 5.01 & 690.50 & 0.00\\
SCA3-5 & \bf{\underline{659.90}} & 5.07 & 
663.48 & 5.22 & 671.10 & 
-1.67\\SCA3-6 & 652.94 & 4.46 & 
653.47 & 4.62 & \bf{651.10} & 
0.28\\SCA3-7 & 666.15 & 3.75 & 
666.15 & 3.78 & \bf{666.10} & 
0.01\\SCA3-8 & \bf{\underline{719.47}} & 4.07 & 
722.58 & 4.15 & 719.50 & 
-0.00\\SCA3-9 & \bf{681.00} & 3.72 & 
681.00 & 3.66 & 681.00 & 0.00\\
SCA8-0 & 968.79 & 5.20 & 
985.37 & 5.25 & \bf{961.60} & 
0.75\\SCA8-1 & \bf{\underline{1052.71}} & 4.02 & 
1064.22 & 4.40 & 1063.00 & 
-0.97\\SCA8-2 & 1046.29 & 3.68 & 
1049.26 & 3.62 & \bf{1040.60} & 
0.55\\SCA8-3 & 1012.89 & 5.34 & 
1017.44 & 5.47 & \bf{985.90} & 
2.74\\SCA8-4 & \bf{\underline{1067.66}} & 5.73 & 
1083.66 & 5.36 & 1071.00 & 
-0.31\\SCA8-5 & \bf{\underline{1052.64}} & 5.98 & 
1054.67 & 5.71 & 1054.30 & 
-0.16\\SCA8-6 & 979.99 & 6.00 & 
981.22 & 6.03 & \bf{972.50} & 
0.77\\SCA8-7 & 1067.20 & 5.72 & 
1069.26 & 5.71 & \bf{1059.70} & 
0.71\\SCA8-8 & \bf{\underline{1071.18}} & 5.74 & 
1077.94 & 5.33 & 1082.70 & 
-1.06\\SCA8-9 & \bf{\underline{1067.42}} & 4.60 & 
1067.42 & 4.13 & 1081.40 & 
-1.29\\CON3-0 & 616.52 & 5.52 & 
620.74 & 5.65 & \bf{616.50} & 
0.00\\CON3-1 & 557.21 & 4.88 & 
557.53 & 4.86 & \bf{555.60} & 
0.29\\CON3-2 & \bf{\underline{521.38}} & 4.34 & 
524.18 & 4.37 & 521.40 & 
-0.00\\CON3-3 & \bf{\underline{591.19}} & 5.50 & 
591.20 & 5.54 & 591.20 & 
-0.00\\CON3-4 & \bf{\underline{588.79}} & 5.00 & 
589.20 & 4.68 & 589.30 & 
-0.09\\CON3-5 & 564.89 & 4.96 & 
568.08 & 4.92 & \bf{563.70} & 
0.21\\CON3-6 & 501.33 & 6.18 & 
504.06 & 6.09 & \bf{499.20} & 
0.43\\CON3-7 & 578.41 & 4.88 & 
578.41 & 4.51 & \bf{577.50} & 
0.16\\CON3-8 & 524.30 & 4.16 & 
525.33 & 4.17 & \bf{523.10} & 
0.23\\CON3-9 & 581.06 & 4.29 & 
586.51 & 4.37 & \bf{578.20} & 
0.49\\CON8-0 & 870.22 & 4.86 & 
877.73 & 5.03 & \bf{858.90} & 
1.32\\CON8-1 & \bf{\underline{740.85}} & 4.76 & 
744.85 & 4.95 & 740.90 & 
-0.01\\CON8-2 & \bf{\underline{713.44}} & 6.81 & 
715.91 & 6.79 & 714.30 & 
-0.12\\CON8-3 & 816.27 & 4.90 & 
817.25 & 5.07 & \bf{812.30} & 
0.49\\CON8-4 & 778.60 & 5.42 & 
786.54 & 5.33 & \bf{770.10} & 
1.10\\CON8-5 & \bf{\underline{758.12}} & 4.67 & 
761.94 & 4.82 & 766.60 & 
-1.11\\CON8-6 & \bf{\underline{690.00}} & 5.86 & 
695.41 & 5.88 & 697.20 & 
-1.03\\CON8-7 & \bf{\underline{814.79}} & 4.55 & 
816.70 & 4.28 & 814.80 & 
-0.00\\CON8-8 & 791.17 & 5.50 & 
797.14 & 5.56 & \bf{771.30} & 
2.58\\CON8-9 & \bf{\underline{812.60}} & 5.62 & 
816.03 & 5.38 & 815.10 & 
-0.31\\[1ex]\hline
\end{tabular}
\label{table:nonlin}
\end{table} \clearpage
\begin{table}[ht]
\caption{Resultados de la ejecución de la metaheurística ACO, utilizando instancias de Dethloff con la configuración -n 8.0 -alpha 1.0 -beta 3.0 -q 0.1 -ro 0.015}
\centering
\small
\begin{tabular}{c c c c c c c}
\hline\hline
Instancia & Costo mínimo & Tiempo(seg.) & Costo promedio & Tiempo promedio(seg.) & Costo ACO & \%Gap \\ [0.5ex]
\hline
SCA3-0 & \bf{\underline{636.06}} & 5.72 & 
636.06 & 5.53 & 636.10 & 
-0.01\\SCA3-1 & \bf{\underline{697.84}} & 5.69 & 
697.84 & 6.26 & 700.10 & 
-0.32\\SCA3-2 & 659.34 & 5.24 & 
661.00 & 5.52 & \bf{659.30} & 
0.01\\SCA3-3 & 680.04 & 5.65 & 
680.04 & 5.86 & \bf{680.00} & 
0.01\\SCA3-4 & \bf{690.50} & 6.10 & 
690.50 & 6.16 & 690.50 & 0.00\\
SCA3-5 & \bf{\underline{659.90}} & 5.69 & 
662.04 & 5.94 & 671.10 & 
-1.67\\SCA3-6 & \bf{\underline{651.09}} & 6.17 & 
652.36 & 5.87 & 651.10 & 
-0.00\\SCA3-7 & \bf{\underline{659.17}} & 5.16 & 
664.40 & 5.08 & 666.10 & 
-1.04\\SCA3-8 & \bf{\underline{719.47}} & 5.86 & 
719.54 & 5.88 & 719.50 & 
-0.00\\SCA3-9 & \bf{681.00} & 4.72 & 
681.00 & 4.96 & 681.00 & 0.00\\
SCA8-0 & \bf{\underline{961.50}} & 6.37 & 
972.62 & 6.11 & 961.60 & 
-0.01\\SCA8-1 & \bf{\underline{1052.71}} & 5.46 & 
1060.03 & 5.52 & 1063.00 & 
-0.97\\SCA8-2 & 1046.29 & 4.84 & 
1048.88 & 5.00 & \bf{1040.60} & 
0.55\\SCA8-3 & 999.10 & 6.15 & 
1000.26 & 6.05 & \bf{985.90} & 
1.34\\SCA8-4 & \bf{\underline{1065.49}} & 5.77 & 
1067.02 & 6.12 & 1071.00 & 
-0.51\\SCA8-5 & \bf{\underline{1036.88}} & 7.16 & 
1048.50 & 6.95 & 1054.30 & 
-1.65\\SCA8-6 & \bf{\underline{972.48}} & 6.16 & 
976.70 & 6.39 & 972.50 & 
-0.00\\SCA8-7 & 1063.22 & 6.20 & 
1065.93 & 6.38 & \bf{1059.70} & 
0.33\\SCA8-8 & \bf{\underline{1071.18}} & 6.91 & 
1074.87 & 7.22 & 1082.70 & 
-1.06\\SCA8-9 & \bf{\underline{1067.42}} & 5.51 & 
1067.42 & 5.36 & 1081.40 & 
-1.29\\CON3-0 & 617.59 & 6.13 & 
620.95 & 6.43 & \bf{616.50} & 
0.18\\CON3-1 & 556.28 & 6.12 & 
557.67 & 6.09 & \bf{555.60} & 
0.12\\CON3-2 & \bf{\underline{519.11}} & 6.35 & 
520.37 & 5.90 & 521.40 & 
-0.44\\CON3-3 & \bf{\underline{591.19}} & 6.60 & 
591.24 & 6.49 & 591.20 & 
-0.00\\CON3-4 & 589.32 & 5.34 & 
590.38 & 5.43 & \bf{589.30} & 
0.00\\CON3-5 & \bf{563.70} & 5.98 & 
565.56 & 6.09 & 563.70 & 0.00\\
CON3-6 & 500.37 & 6.42 & 
501.44 & 6.66 & \bf{499.20} & 
0.23\\CON3-7 & \bf{\underline{576.48}} & 5.40 & 
577.88 & 5.52 & 577.50 & 
-0.18\\CON3-8 & \bf{\underline{523.05}} & 5.74 & 
523.46 & 5.64 & 523.10 & 
-0.01\\CON3-9 & 585.05 & 5.65 & 
587.60 & 5.82 & \bf{578.20} & 
1.18\\CON8-0 & 859.51 & 6.06 & 
865.55 & 5.92 & \bf{858.90} & 
0.07\\CON8-1 & \bf{\underline{740.85}} & 5.82 & 
741.25 & 6.10 & 740.90 & 
-0.01\\CON8-2 & \bf{\underline{713.05}} & 7.33 & 
713.38 & 7.57 & 714.30 & 
-0.17\\CON8-3 & \bf{\underline{812.11}} & 6.24 & 
813.78 & 6.59 & 812.30 & 
-0.02\\CON8-4 & 776.72 & 5.93 & 
778.10 & 5.76 & \bf{770.10} & 
0.86\\CON8-5 & \bf{\underline{760.03}} & 5.64 & 
760.65 & 5.81 & 766.60 & 
-0.86\\CON8-6 & \bf{\underline{690.16}} & 7.03 & 
693.95 & 6.81 & 697.20 & 
-1.01\\CON8-7 & \bf{\underline{814.79}} & 5.50 & 
814.83 & 5.61 & 814.80 & 
-0.00\\CON8-8 & 779.57 & 6.93 & 
786.47 & 6.87 & \bf{771.30} & 
1.07\\CON8-9 & \bf{\underline{812.03}} & 7.03 & 
813.97 & 7.02 & 815.10 & 
-0.38\\[1ex]\hline
\end{tabular}
\label{table:nonlin}
\end{table} \clearpage
\begin{table}[ht]
\caption{Resultados de la ejecución de la metaheurística ACO, utilizando instancias de Dethloff con la configuración -n 8.0 -alpha 1.0 -beta 3.0 -q .2 -ro 0.015}
\centering
\small
\begin{tabular}{c c c c c c c}
\hline\hline
Instancia & Costo mínimo & Tiempo(seg.) & Costo promedio & Tiempo promedio(seg.) & Costo ACO & \%Gap \\ [0.5ex]
\hline
SCA3-0 & \bf{\underline{636.06}} & 5.82 & 
636.06 & 5.73 & 636.10 & 
-0.01\\SCA3-1 & \bf{\underline{697.84}} & 6.68 & 
697.84 & 6.52 & 700.10 & 
-0.32\\SCA3-2 & 659.34 & 5.39 & 
661.45 & 5.66 & \bf{659.30} & 
0.01\\SCA3-3 & 680.04 & 5.64 & 
680.18 & 5.81 & \bf{680.00} & 
0.01\\SCA3-4 & \bf{690.50} & 5.85 & 
690.50 & 6.51 & 690.50 & 0.00\\
SCA3-5 & \bf{\underline{659.90}} & 6.06 & 
662.76 & 5.99 & 671.10 & 
-1.67\\SCA3-6 & \bf{\underline{651.09}} & 6.03 & 
652.01 & 5.89 & 651.10 & 
-0.00\\SCA3-7 & \bf{\underline{659.17}} & 5.00 & 
662.66 & 5.05 & 666.10 & 
-1.04\\SCA3-8 & \bf{\underline{719.47}} & 6.07 & 
719.47 & 5.92 & 719.50 & 
-0.00\\SCA3-9 & \bf{681.00} & 4.76 & 
681.00 & 5.08 & 681.00 & 0.00\\
SCA8-0 & 968.79 & 6.31 & 
980.35 & 6.38 & \bf{961.60} & 
0.75\\SCA8-1 & \bf{\underline{1053.44}} & 5.15 & 
1060.51 & 5.25 & 1063.00 & 
-0.90\\SCA8-2 & 1045.64 & 4.69 & 
1048.97 & 4.86 & \bf{1040.60} & 
0.48\\SCA8-3 & 999.35 & 6.09 & 
1003.63 & 5.96 & \bf{985.90} & 
1.36\\SCA8-4 & \bf{\underline{1065.49}} & 5.99 & 
1066.18 & 6.45 & 1071.00 & 
-0.51\\SCA8-5 & \bf{\underline{1034.74}} & 6.72 & 
1042.80 & 6.70 & 1054.30 & 
-1.86\\SCA8-6 & \bf{\underline{972.48}} & 6.55 & 
976.65 & 6.49 & 972.50 & 
-0.00\\SCA8-7 & 1067.20 & 6.24 & 
1069.85 & 6.39 & \bf{1059.70} & 
0.71\\SCA8-8 & \bf{\underline{1071.18}} & 6.51 & 
1076.64 & 6.58 & 1082.70 & 
-1.06\\SCA8-9 & \bf{\underline{1067.42}} & 5.17 & 
1067.42 & 5.16 & 1081.40 & 
-1.29\\CON3-0 & 617.59 & 6.48 & 
617.97 & 6.32 & \bf{616.50} & 
0.18\\CON3-1 & \bf{\underline{554.47}} & 5.84 & 
555.55 & 6.85 & 555.60 & 
-0.20\\CON3-2 & \bf{\underline{521.38}} & 6.87 & 
521.38 & 6.22 & 521.40 & 
-0.00\\CON3-3 & \bf{\underline{591.19}} & 6.53 & 
591.20 & 6.37 & 591.20 & 
-0.00\\CON3-4 & \bf{\underline{588.79}} & 5.65 & 
589.58 & 5.41 & 589.30 & 
-0.09\\CON3-5 & \bf{563.70} & 5.80 & 
565.24 & 5.82 & 563.70 & 0.00\\
CON3-6 & 500.80 & 6.63 & 
501.80 & 6.65 & \bf{499.20} & 
0.32\\CON3-7 & \bf{\underline{576.84}} & 5.36 & 
578.68 & 5.17 & 577.50 & 
-0.11\\CON3-8 & \bf{\underline{523.05}} & 5.54 & 
523.10 & 5.62 & 523.10 & 
-0.01\\CON3-9 & 578.98 & 5.74 & 
581.48 & 5.92 & \bf{578.20} & 
0.13\\CON8-0 & 866.11 & 6.74 & 
869.54 & 6.33 & \bf{858.90} & 
0.84\\CON8-1 & \bf{\underline{740.85}} & 6.20 & 
742.18 & 6.22 & 740.90 & 
-0.01\\CON8-2 & \bf{\underline{712.89}} & 7.02 & 
713.40 & 7.28 & 714.30 & 
-0.20\\CON8-3 & 816.27 & 5.86 & 
816.99 & 6.11 & \bf{812.30} & 
0.49\\CON8-4 & 776.37 & 6.11 & 
780.02 & 5.75 & \bf{770.10} & 
0.81\\CON8-5 & \bf{\underline{755.14}} & 6.28 & 
758.78 & 6.12 & 766.60 & 
-1.49\\CON8-6 & \bf{\underline{689.23}} & 6.42 & 
692.82 & 6.74 & 697.20 & 
-1.14\\CON8-7 & \bf{\underline{814.79}} & 5.44 & 
814.79 & 5.25 & 814.80 & 
-0.00\\CON8-8 & 779.43 & 6.52 & 
781.76 & 7.08 & \bf{771.30} & 
1.05\\CON8-9 & \bf{\underline{810.18}} & 6.54 & 
812.19 & 6.67 & 815.10 & 
-0.60\\[1ex]\hline
\end{tabular}
\label{table:nonlin}
\end{table} \clearpage
\begin{table}[ht]
\caption{Resultados de la ejecución de la metaheurística ACO, utilizando instancias de Dethloff con la configuración -n 8.0 -alpha 1.0 -beta 3.0 -q .3 -ro 0.015}
\centering
\small
\begin{tabular}{c c c c c c c}
\hline\hline
Instancia & Costo mínimo & Tiempo(seg.) & Costo promedio & Tiempo promedio(seg.) & Costo ACO & \%Gap \\ [0.5ex]
\hline
SCA3-0 & \bf{\underline{636.06}} & 5.96 & 
636.06 & 5.81 & 636.10 & 
-0.01\\SCA3-1 & \bf{\underline{697.84}} & 5.91 & 
697.84 & 6.11 & 700.10 & 
-0.32\\SCA3-2 & 659.34 & 5.63 & 
659.79 & 5.54 & \bf{659.30} & 
0.01\\SCA3-3 & 680.04 & 5.71 & 
680.04 & 5.51 & \bf{680.00} & 
0.01\\SCA3-4 & \bf{690.50} & 5.90 & 
690.50 & 6.05 & 690.50 & 0.00\\
SCA3-5 & \bf{\underline{659.90}} & 6.01 & 
661.90 & 6.08 & 671.10 & 
-1.67\\SCA3-6 & \bf{\underline{651.09}} & 5.80 & 
652.48 & 5.68 & 651.10 & 
-0.00\\SCA3-7 & \bf{\underline{659.17}} & 5.19 & 
664.09 & 5.05 & 666.10 & 
-1.04\\SCA3-8 & \bf{\underline{719.47}} & 5.64 & 
720.68 & 5.67 & 719.50 & 
-0.00\\SCA3-9 & \bf{681.00} & 4.71 & 
681.00 & 4.93 & 681.00 & 0.00\\
SCA8-0 & \bf{\underline{961.50}} & 6.02 & 
964.38 & 6.23 & 961.60 & 
-0.01\\SCA8-1 & \bf{\underline{1055.60}} & 10.20 & 
1060.22 & 6.52 & 1063.00 & 
-0.70\\SCA8-2 & 1050.17 & 4.90 & 
1050.53 & 4.64 & \bf{1040.60} & 
0.92\\SCA8-3 & 991.84 & 5.93 & 
1004.06 & 6.17 & \bf{985.90} & 
0.60\\SCA8-4 & \bf{\underline{1065.49}} & 6.04 & 
1066.03 & 5.96 & 1071.00 & 
-0.51\\SCA8-5 & \bf{\underline{1037.06}} & 6.58 & 
1046.95 & 6.60 & 1054.30 & 
-1.64\\SCA8-6 & \bf{\underline{972.48}} & 6.46 & 
976.70 & 6.42 & 972.50 & 
-0.00\\SCA8-7 & 1067.03 & 6.36 & 
1068.09 & 6.44 & \bf{1059.70} & 
0.69\\SCA8-8 & \bf{\underline{1071.18}} & 6.51 & 
1079.38 & 6.46 & 1082.70 & 
-1.06\\SCA8-9 & \bf{\underline{1065.60}} & 5.43 & 
1066.97 & 5.28 & 1081.40 & 
-1.46\\CON3-0 & 617.59 & 6.57 & 
619.86 & 6.34 & \bf{616.50} & 
0.18\\CON3-1 & \bf{\underline{554.47}} & 6.06 & 
556.45 & 6.17 & 555.60 & 
-0.20\\CON3-2 & \bf{\underline{519.61}} & 5.56 & 
521.00 & 5.64 & 521.40 & 
-0.34\\CON3-3 & \bf{\underline{591.19}} & 6.24 & 
591.24 & 6.23 & 591.20 & 
-0.00\\CON3-4 & \bf{\underline{588.79}} & 5.36 & 
589.05 & 5.53 & 589.30 & 
-0.09\\CON3-5 & \bf{563.70} & 5.50 & 
564.51 & 5.77 & 563.70 & 0.00\\
CON3-6 & \bf{\underline{499.05}} & 6.75 & 
501.42 & 7.12 & 499.20 & 
-0.03\\CON3-7 & \bf{\underline{576.84}} & 5.32 & 
577.79 & 5.24 & 577.50 & 
-0.11\\CON3-8 & \bf{\underline{523.05}} & 5.17 & 
523.82 & 5.55 & 523.10 & 
-0.01\\CON3-9 & 582.98 & 5.10 & 
585.89 & 5.49 & \bf{578.20} & 
0.83\\CON8-0 & 868.56 & 6.58 & 
870.19 & 6.16 & \bf{858.90} & 
1.12\\CON8-1 & \bf{\underline{740.85}} & 6.17 & 
743.39 & 6.06 & 740.90 & 
-0.01\\CON8-2 & \bf{\underline{713.44}} & 7.54 & 
713.91 & 7.36 & 714.30 & 
-0.12\\CON8-3 & \bf{\underline{811.07}} & 5.99 & 
811.64 & 6.02 & 812.30 & 
-0.15\\CON8-4 & 773.33 & 6.11 & 
778.61 & 5.91 & \bf{770.10} & 
0.42\\CON8-5 & \bf{\underline{760.03}} & 6.07 & 
761.51 & 6.02 & 766.60 & 
-0.86\\CON8-6 & \bf{\underline{690.53}} & 6.89 & 
693.31 & 6.83 & 697.20 & 
-0.96\\CON8-7 & \bf{\underline{814.79}} & 5.08 & 
814.97 & 5.16 & 814.80 & 
-0.00\\CON8-8 & 782.34 & 6.50 & 
784.95 & 6.69 & \bf{771.30} & 
1.43\\CON8-9 & \bf{\underline{812.60}} & 6.64 & 
813.05 & 6.76 & 815.10 & 
-0.31\\[1ex]\hline
\end{tabular}
\label{table:nonlin}
\end{table} \clearpage
\begin{table}[ht]
\caption{Resultados de la ejecución de la metaheurística ACO, utilizando instancias de Dethloff con la configuración -n 8.0 -alpha 1.0 -beta 3.0 -q .4 -ro 0.015}
\centering
\small
\begin{tabular}{c c c c c c c}
\hline\hline
Instancia & Costo mínimo & Tiempo(seg.) & Costo promedio & Tiempo promedio(seg.) & Costo ACO & \%Gap \\ [0.5ex]
\hline
SCA3-0 & \bf{\underline{636.06}} & 5.37 & 
636.06 & 5.61 & 636.10 & 
-0.01\\SCA3-1 & \bf{\underline{697.84}} & 6.15 & 
697.84 & 6.21 & 700.10 & 
-0.32\\SCA3-2 & 659.34 & 5.09 & 
659.79 & 5.37 & \bf{659.30} & 
0.01\\SCA3-3 & 680.04 & 5.73 & 
680.04 & 5.49 & \bf{680.00} & 
0.01\\SCA3-4 & \bf{690.50} & 6.33 & 
690.50 & 6.11 & 690.50 & 0.00\\
SCA3-5 & \bf{\underline{659.90}} & 5.59 & 
662.04 & 5.90 & 671.10 & 
-1.67\\SCA3-6 & 652.94 & 5.34 & 
653.80 & 5.66 & \bf{651.10} & 
0.28\\SCA3-7 & 666.15 & 4.90 & 
666.15 & 4.71 & \bf{666.10} & 
0.01\\SCA3-8 & \bf{\underline{719.47}} & 5.61 & 
719.97 & 5.50 & 719.50 & 
-0.00\\SCA3-9 & \bf{681.00} & 5.10 & 
681.00 & 5.03 & 681.00 & 0.00\\
SCA8-0 & 968.79 & 6.11 & 
973.96 & 6.33 & \bf{961.60} & 
0.75\\SCA8-1 & \bf{\underline{1056.87}} & 4.77 & 
1065.45 & 4.92 & 1063.00 & 
-0.58\\SCA8-2 & 1049.08 & 4.66 & 
1051.19 & 4.56 & \bf{1040.60} & 
0.81\\SCA8-3 & 995.50 & 5.36 & 
1007.29 & 5.80 & \bf{985.90} & 
0.97\\SCA8-4 & \bf{\underline{1065.49}} & 6.65 & 
1066.60 & 6.15 & 1071.00 & 
-0.51\\SCA8-5 & \bf{\underline{1038.59}} & 6.99 & 
1047.77 & 6.68 & 1054.30 & 
-1.49\\SCA8-6 & 977.87 & 6.85 & 
980.04 & 6.43 & \bf{972.50} & 
0.55\\SCA8-7 & 1067.03 & 6.45 & 
1067.16 & 6.50 & \bf{1059.70} & 
0.69\\SCA8-8 & \bf{\underline{1071.18}} & 6.59 & 
1072.13 & 6.42 & 1082.70 & 
-1.06\\SCA8-9 & \bf{\underline{1067.42}} & 5.04 & 
1067.42 & 4.99 & 1081.40 & 
-1.29\\CON3-0 & 617.59 & 6.00 & 
619.29 & 6.32 & \bf{616.50} & 
0.18\\CON3-1 & \bf{\underline{554.47}} & 5.64 & 
556.89 & 5.89 & 555.60 & 
-0.20\\CON3-2 & \bf{\underline{521.38}} & 5.96 & 
521.44 & 5.69 & 521.40 & 
-0.00\\CON3-3 & \bf{\underline{591.19}} & 6.06 & 
591.20 & 6.28 & 591.20 & 
-0.00\\CON3-4 & \bf{\underline{588.79}} & 5.32 & 
588.92 & 5.20 & 589.30 & 
-0.09\\CON3-5 & \bf{563.70} & 5.87 & 
564.00 & 5.79 & 563.70 & 0.00\\
CON3-6 & 500.80 & 6.25 & 
501.14 & 6.66 & \bf{499.20} & 
0.32\\CON3-7 & \bf{\underline{576.48}} & 5.08 & 
577.80 & 5.29 & 577.50 & 
-0.18\\CON3-8 & \bf{\underline{523.05}} & 5.43 & 
523.84 & 5.35 & 523.10 & 
-0.01\\CON3-9 & 586.17 & 5.65 & 
587.77 & 5.49 & \bf{578.20} & 
1.38\\CON8-0 & 866.11 & 5.67 & 
872.33 & 5.96 & \bf{858.90} & 
0.84\\CON8-1 & 740.93 & 6.30 & 
742.02 & 6.04 & \bf{740.90} & 
0.00\\CON8-2 & \bf{\underline{712.89}} & 6.81 & 
713.87 & 7.05 & 714.30 & 
-0.20\\CON8-3 & 814.50 & 5.89 & 
816.72 & 5.81 & \bf{812.30} & 
0.27\\CON8-4 & 776.34 & 5.88 & 
782.37 & 5.93 & \bf{770.10} & 
0.81\\CON8-5 & \bf{\underline{758.12}} & 5.61 & 
760.11 & 5.86 & 766.60 & 
-1.11\\CON8-6 & \bf{\underline{688.99}} & 6.95 & 
689.60 & 6.76 & 697.20 & 
-1.18\\CON8-7 & \bf{\underline{814.79}} & 5.36 & 
814.86 & 5.17 & 814.80 & 
-0.00\\CON8-8 & 781.96 & 6.47 & 
784.87 & 6.71 & \bf{771.30} & 
1.38\\CON8-9 & \bf{\underline{810.18}} & 6.61 & 
812.71 & 7.16 & 815.10 & 
-0.60\\[1ex]\hline
\end{tabular}
\label{table:nonlin}
\end{table} \clearpage
\begin{table}[ht]
\caption{Resultados de la ejecución de la metaheurística ACO, utilizando instancias de Dethloff con la configuración -n 8.0 -alpha 1.0 -beta 3.0 -q .5 -ro 0.015}
\centering
\small
\begin{tabular}{c c c c c c c}
\hline\hline
Instancia & Costo mínimo & Tiempo(seg.) & Costo promedio & Tiempo promedio(seg.) & Costo ACO & \%Gap \\ [0.5ex]
\hline
SCA3-0 & \bf{\underline{636.06}} & 5.47 & 
636.06 & 5.61 & 636.10 & 
-0.01\\SCA3-1 & \bf{\underline{697.84}} & 6.31 & 
697.84 & 6.13 & 700.10 & 
-0.32\\SCA3-2 & 659.34 & 5.44 & 
659.34 & 5.53 & \bf{659.30} & 
0.01\\SCA3-3 & 680.04 & 5.81 & 
680.04 & 5.53 & \bf{680.00} & 
0.01\\SCA3-4 & \bf{690.50} & 5.82 & 
690.50 & 5.90 & 690.50 & 0.00\\
SCA3-5 & \bf{\underline{662.75}} & 6.01 & 
664.92 & 5.91 & 671.10 & 
-1.24\\SCA3-6 & 652.94 & 5.39 & 
653.16 & 5.53 & \bf{651.10} & 
0.28\\SCA3-7 & 666.15 & 4.44 & 
666.15 & 4.66 & \bf{666.10} & 
0.01\\SCA3-8 & \bf{\underline{719.47}} & 5.46 & 
719.47 & 5.38 & 719.50 & 
-0.00\\SCA3-9 & \bf{681.00} & 4.61 & 
681.00 & 4.62 & 681.00 & 0.00\\
SCA8-0 & \bf{\underline{961.50}} & 6.27 & 
969.72 & 6.10 & 961.60 & 
-0.01\\SCA8-1 & \bf{\underline{1052.71}} & 5.22 & 
1060.66 & 4.97 & 1063.00 & 
-0.97\\SCA8-2 & 1050.17 & 4.36 & 
1050.53 & 4.51 & \bf{1040.60} & 
0.92\\SCA8-3 & 1007.33 & 6.16 & 
1013.04 & 6.23 & \bf{985.90} & 
2.17\\SCA8-4 & \bf{\underline{1065.49}} & 5.92 & 
1066.63 & 6.05 & 1071.00 & 
-0.51\\SCA8-5 & \bf{\underline{1034.74}} & 6.69 & 
1046.93 & 6.58 & 1054.30 & 
-1.86\\SCA8-6 & 977.87 & 6.45 & 
978.63 & 6.34 & \bf{972.50} & 
0.55\\SCA8-7 & 1067.20 & 6.82 & 
1069.26 & 6.49 & \bf{1059.70} & 
0.71\\SCA8-8 & \bf{\underline{1071.18}} & 6.44 & 
1073.91 & 6.30 & 1082.70 & 
-1.06\\SCA8-9 & \bf{\underline{1067.42}} & 5.18 & 
1067.42 & 5.20 & 1081.40 & 
-1.29\\CON3-0 & 617.59 & 5.96 & 
620.53 & 6.24 & \bf{616.50} & 
0.18\\CON3-1 & \bf{\underline{554.47}} & 5.61 & 
556.32 & 5.96 & 555.60 & 
-0.20\\CON3-2 & \bf{\underline{519.11}} & 5.92 & 
521.27 & 5.68 & 521.40 & 
-0.44\\CON3-3 & \bf{\underline{591.19}} & 6.65 & 
591.20 & 6.83 & 591.20 & 
-0.00\\CON3-4 & \bf{\underline{588.79}} & 5.50 & 
588.79 & 5.30 & 589.30 & 
-0.09\\CON3-5 & 568.66 & 5.52 & 
568.69 & 5.75 & \bf{563.70} & 
0.88\\CON3-6 & 500.37 & 6.51 & 
501.81 & 6.76 & \bf{499.20} & 
0.23\\CON3-7 & \bf{\underline{576.48}} & 5.20 & 
578.66 & 5.20 & 577.50 & 
-0.18\\CON3-8 & \bf{\underline{523.05}} & 5.46 & 
523.59 & 5.27 & 523.10 & 
-0.01\\CON3-9 & 580.05 & 5.79 & 
583.01 & 5.49 & \bf{578.20} & 
0.32\\CON8-0 & 875.11 & 5.86 & 
880.13 & 5.79 & \bf{858.90} & 
1.89\\CON8-1 & \bf{\underline{740.85}} & 5.88 & 
741.64 & 6.02 & 740.90 & 
-0.01\\CON8-2 & \bf{\underline{713.44}} & 7.20 & 
714.32 & 7.46 & 714.30 & 
-0.12\\CON8-3 & 812.54 & 5.82 & 
814.33 & 5.90 & \bf{812.30} & 
0.03\\CON8-4 & 776.37 & 5.76 & 
782.22 & 5.82 & \bf{770.10} & 
0.81\\CON8-5 & \bf{\underline{760.91}} & 5.91 & 
762.39 & 5.75 & 766.60 & 
-0.74\\CON8-6 & \bf{\underline{690.27}} & 6.66 & 
696.07 & 6.63 & 697.20 & 
-0.99\\CON8-7 & \bf{\underline{814.50}} & 5.27 & 
816.88 & 5.17 & 814.80 & 
-0.04\\CON8-8 & 783.43 & 7.12 & 
784.59 & 6.90 & \bf{771.30} & 
1.57\\CON8-9 & \bf{\underline{813.04}} & 6.66 & 
813.90 & 6.74 & 815.10 & 
-0.25\\[1ex]\hline
\end{tabular}
\label{table:nonlin}
\end{table} \clearpage
\begin{table}[ht]
\caption{Resultados de la ejecución de la metaheurística ACO, utilizando instancias de Dethloff con la configuración -n 8.0 -alpha 1.0 -beta 3.0 -q .6 -ro 0.015}
\centering
\small
\begin{tabular}{c c c c c c c}
\hline\hline
Instancia & Costo mínimo & Tiempo(seg.) & Costo promedio & Tiempo promedio(seg.) & Costo ACO & \%Gap \\ [0.5ex]
\hline
SCA3-0 & \bf{\underline{636.06}} & 5.40 & 
636.06 & 5.46 & 636.10 & 
-0.01\\SCA3-1 & \bf{\underline{697.84}} & 6.52 & 
697.84 & 6.15 & 700.10 & 
-0.32\\SCA3-2 & 659.34 & 5.34 & 
659.79 & 5.64 & \bf{659.30} & 
0.01\\SCA3-3 & 680.04 & 5.54 & 
680.04 & 5.43 & \bf{680.00} & 
0.01\\SCA3-4 & \bf{690.50} & 5.69 & 
690.50 & 5.77 & 690.50 & 0.00\\
SCA3-5 & \bf{\underline{662.75}} & 5.47 & 
664.77 & 5.88 & 671.10 & 
-1.24\\SCA3-6 & \bf{\underline{651.09}} & 5.94 & 
652.01 & 5.56 & 651.10 & 
-0.00\\SCA3-7 & \bf{\underline{659.17}} & 5.30 & 
664.40 & 4.87 & 666.10 & 
-1.04\\SCA3-8 & \bf{\underline{719.47}} & 5.63 & 
720.67 & 5.38 & 719.50 & 
-0.00\\SCA3-9 & \bf{681.00} & 4.23 & 
681.00 & 4.46 & 681.00 & 0.00\\
SCA8-0 & \bf{\underline{961.50}} & 6.06 & 
974.77 & 6.09 & 961.60 & 
-0.01\\SCA8-1 & \bf{\underline{1056.61}} & 4.85 & 
1061.38 & 4.93 & 1063.00 & 
-0.60\\SCA8-2 & 1045.49 & 4.18 & 
1048.68 & 4.18 & \bf{1040.60} & 
0.47\\SCA8-3 & 997.17 & 6.29 & 
1005.93 & 6.03 & \bf{985.90} & 
1.14\\SCA8-4 & \bf{\underline{1065.49}} & 5.74 & 
1071.34 & 6.17 & 1071.00 & 
-0.51\\SCA8-5 & \bf{\underline{1034.74}} & 5.81 & 
1049.14 & 6.24 & 1054.30 & 
-1.86\\SCA8-6 & \bf{\underline{972.48}} & 6.06 & 
978.25 & 6.46 & 972.50 & 
-0.00\\SCA8-7 & 1067.20 & 6.36 & 
1068.03 & 6.34 & \bf{1059.70} & 
0.71\\SCA8-8 & \bf{\underline{1071.18}} & 6.18 & 
1073.91 & 5.97 & 1082.70 & 
-1.06\\SCA8-9 & \bf{\underline{1063.68}} & 5.10 & 
1066.49 & 4.85 & 1081.40 & 
-1.64\\CON3-0 & 616.52 & 6.38 & 
620.23 & 6.30 & \bf{616.50} & 
0.00\\CON3-1 & \bf{\underline{554.47}} & 6.06 & 
555.20 & 6.08 & 555.60 & 
-0.20\\CON3-2 & \bf{\underline{519.11}} & 5.02 & 
520.81 & 5.58 & 521.40 & 
-0.44\\CON3-3 & \bf{\underline{591.19}} & 6.20 & 
591.20 & 6.20 & 591.20 & 
-0.00\\CON3-4 & \bf{\underline{588.79}} & 5.66 & 
589.45 & 5.30 & 589.30 & 
-0.09\\CON3-5 & 564.89 & 5.44 & 
567.33 & 5.72 & \bf{563.70} & 
0.21\\CON3-6 & 500.80 & 6.95 & 
502.09 & 6.76 & \bf{499.20} & 
0.32\\CON3-7 & \bf{\underline{576.84}} & 5.24 & 
577.62 & 5.14 & 577.50 & 
-0.11\\CON3-8 & 523.14 & 5.37 & 
523.87 & 5.24 & \bf{523.10} & 
0.01\\CON3-9 & 588.40 & 5.28 & 
588.71 & 5.34 & \bf{578.20} & 
1.76\\CON8-0 & 870.00 & 6.15 & 
874.38 & 6.00 & \bf{858.90} & 
1.29\\CON8-1 & \bf{\underline{740.85}} & 6.38 & 
741.63 & 6.01 & 740.90 & 
-0.01\\CON8-2 & \bf{\underline{713.44}} & 8.24 & 
714.21 & 7.79 & 714.30 & 
-0.12\\CON8-3 & \bf{\underline{811.07}} & 6.16 & 
813.48 & 5.88 & 812.30 & 
-0.15\\CON8-4 & 776.72 & 5.92 & 
781.49 & 5.84 & \bf{770.10} & 
0.86\\CON8-5 & \bf{\underline{758.12}} & 5.90 & 
762.18 & 5.63 & 766.60 & 
-1.11\\CON8-6 & \bf{\underline{694.81}} & 6.67 & 
695.77 & 6.43 & 697.20 & 
-0.34\\CON8-7 & \bf{\underline{814.50}} & 4.81 & 
815.05 & 4.92 & 814.80 & 
-0.04\\CON8-8 & 783.28 & 6.51 & 
786.78 & 6.53 & \bf{771.30} & 
1.55\\CON8-9 & \bf{\underline{809.00}} & 6.14 & 
813.60 & 6.58 & 815.10 & 
-0.75\\[1ex]\hline
\end{tabular}
\label{table:nonlin}
\end{table} \clearpage
\begin{table}[ht]
\caption{Resultados de la ejecución de la metaheurística ACO, utilizando instancias de Dethloff con la configuración -n 8.0 -alpha 1.0 -beta 3.0 -q .7 -ro 0.015}
\centering
\small
\begin{tabular}{c c c c c c c}
\hline\hline
Instancia & Costo mínimo & Tiempo(seg.) & Costo promedio & Tiempo promedio(seg.) & Costo ACO & \%Gap \\ [0.5ex]
\hline
SCA3-0 & \bf{\underline{636.06}} & 5.66 & 
636.06 & 5.66 & 636.10 & 
-0.01\\SCA3-1 & \bf{\underline{697.84}} & 6.18 & 
697.84 & 6.25 & 700.10 & 
-0.32\\SCA3-2 & 659.34 & 5.30 & 
660.55 & 5.39 & \bf{659.30} & 
0.01\\SCA3-3 & 680.04 & 5.93 & 
680.04 & 5.60 & \bf{680.00} & 
0.01\\SCA3-4 & \bf{690.50} & 5.69 & 
690.50 & 5.73 & 690.50 & 0.00\\
SCA3-5 & \bf{\underline{659.90}} & 5.86 & 
663.18 & 6.00 & 671.10 & 
-1.67\\SCA3-6 & 652.94 & 5.65 & 
653.16 & 5.66 & \bf{651.10} & 
0.28\\SCA3-7 & 666.15 & 4.85 & 
666.15 & 4.57 & \bf{666.10} & 
0.01\\SCA3-8 & \bf{\underline{719.47}} & 5.04 & 
720.26 & 5.20 & 719.50 & 
-0.00\\SCA3-9 & \bf{681.00} & 4.42 & 
681.00 & 4.69 & 681.00 & 0.00\\
SCA8-0 & \bf{\underline{961.50}} & 5.97 & 
970.32 & 5.92 & 961.60 & 
-0.01\\SCA8-1 & \bf{\underline{1049.65}} & 5.21 & 
1056.37 & 4.97 & 1063.00 & 
-1.26\\SCA8-2 & 1046.29 & 4.63 & 
1050.23 & 4.35 & \bf{1040.60} & 
0.55\\SCA8-3 & 997.17 & 5.56 & 
1008.63 & 5.99 & \bf{985.90} & 
1.14\\SCA8-4 & \bf{\underline{1067.66}} & 5.83 & 
1074.40 & 6.24 & 1071.00 & 
-0.31\\SCA8-5 & \bf{\underline{1034.74}} & 6.24 & 
1045.11 & 6.25 & 1054.30 & 
-1.86\\SCA8-6 & \bf{\underline{972.48}} & 6.22 & 
977.96 & 6.44 & 972.50 & 
-0.00\\SCA8-7 & 1067.20 & 6.74 & 
1069.20 & 6.70 & \bf{1059.70} & 
0.71\\SCA8-8 & \bf{\underline{1071.18}} & 6.39 & 
1072.14 & 6.10 & 1082.70 & 
-1.06\\SCA8-9 & \bf{\underline{1065.60}} & 4.41 & 
1066.93 & 4.71 & 1081.40 & 
-1.46\\CON3-0 & 616.52 & 6.00 & 
619.96 & 6.39 & \bf{616.50} & 
0.00\\CON3-1 & \bf{\underline{554.47}} & 6.14 & 
555.61 & 5.96 & 555.60 & 
-0.20\\CON3-2 & \bf{\underline{521.38}} & 5.28 & 
522.32 & 5.14 & 521.40 & 
-0.00\\CON3-3 & \bf{\underline{591.19}} & 6.00 & 
591.24 & 6.03 & 591.20 & 
-0.00\\CON3-4 & \bf{\underline{588.79}} & 5.41 & 
588.92 & 5.33 & 589.30 & 
-0.09\\CON3-5 & 567.94 & 5.67 & 
568.51 & 5.79 & \bf{563.70} & 
0.75\\CON3-6 & 502.09 & 7.22 & 
502.79 & 6.90 & \bf{499.20} & 
0.58\\CON3-7 & 578.22 & 4.90 & 
578.32 & 5.24 & \bf{577.50} & 
0.12\\CON3-8 & 523.14 & 5.14 & 
524.89 & 4.85 & \bf{523.10} & 
0.01\\CON3-9 & 578.25 & 5.87 & 
585.65 & 5.36 & \bf{578.20} & 
0.01\\CON8-0 & 868.49 & 6.14 & 
869.20 & 5.95 & \bf{858.90} & 
1.12\\CON8-1 & \bf{\underline{740.85}} & 5.81 & 
744.00 & 5.79 & 740.90 & 
-0.01\\CON8-2 & \bf{\underline{713.90}} & 7.62 & 
715.26 & 7.94 & 714.30 & 
-0.06\\CON8-3 & 812.54 & 5.80 & 
816.59 & 5.76 & \bf{812.30} & 
0.03\\CON8-4 & 776.37 & 6.48 & 
785.82 & 6.02 & \bf{770.10} & 
0.81\\CON8-5 & \bf{\underline{760.03}} & 5.28 & 
761.51 & 5.49 & 766.60 & 
-0.86\\CON8-6 & \bf{\underline{690.19}} & 6.13 & 
693.70 & 6.34 & 697.20 & 
-1.01\\CON8-7 & \bf{\underline{814.79}} & 4.68 & 
816.19 & 4.89 & 814.80 & 
-0.00\\CON8-8 & 782.86 & 6.87 & 
784.70 & 6.69 & \bf{771.30} & 
1.50\\CON8-9 & \bf{\underline{812.60}} & 6.61 & 
815.01 & 6.38 & 815.10 & 
-0.31\\[1ex]\hline
\end{tabular}
\label{table:nonlin}
\end{table} \clearpage
\begin{table}[ht]
\caption{Resultados de la ejecución de la metaheurística ACO, utilizando instancias de Dethloff con la configuración -n 8.0 -alpha 1.0 -beta 3.0 -q .8 -ro 0.015}
\centering
\small
\begin{tabular}{c c c c c c c}
\hline\hline
Instancia & Costo mínimo & Tiempo(seg.) & Costo promedio & Tiempo promedio(seg.) & Costo ACO & \%Gap \\ [0.5ex]
\hline
SCA3-0 & \bf{\underline{636.06}} & 5.46 & 
636.20 & 5.43 & 636.10 & 
-0.01\\SCA3-1 & \bf{\underline{697.84}} & 5.69 & 
697.84 & 5.80 & 700.10 & 
-0.32\\SCA3-2 & 659.34 & 5.18 & 
661.76 & 5.31 & \bf{659.30} & 
0.01\\SCA3-3 & 680.04 & 5.80 & 
680.18 & 5.79 & \bf{680.00} & 
0.01\\SCA3-4 & \bf{690.50} & 5.70 & 
690.50 & 6.25 & 690.50 & 0.00\\
SCA3-5 & \bf{\underline{659.90}} & 6.13 & 
662.61 & 6.12 & 671.10 & 
-1.67\\SCA3-6 & 652.94 & 5.75 & 
653.27 & 5.78 & \bf{651.10} & 
0.28\\SCA3-7 & 666.15 & 5.09 & 
666.15 & 4.68 & \bf{666.10} & 
0.01\\SCA3-8 & \bf{\underline{719.47}} & 5.62 & 
720.67 & 5.07 & 719.50 & 
-0.00\\SCA3-9 & \bf{681.00} & 4.84 & 
681.00 & 4.77 & 681.00 & 0.00\\
SCA8-0 & 968.79 & 5.94 & 
975.96 & 5.98 & \bf{961.60} & 
0.75\\SCA8-1 & \bf{\underline{1050.20}} & 4.88 & 
1061.12 & 4.91 & 1063.00 & 
-1.20\\SCA8-2 & 1050.37 & 4.16 & 
1050.84 & 4.27 & \bf{1040.60} & 
0.94\\SCA8-3 & 1008.29 & 6.54 & 
1015.24 & 6.21 & \bf{985.90} & 
2.27\\SCA8-4 & \bf{\underline{1065.49}} & 6.40 & 
1067.68 & 6.11 & 1071.00 & 
-0.51\\SCA8-5 & \bf{\underline{1053.09}} & 6.44 & 
1054.62 & 6.39 & 1054.30 & 
-0.11\\SCA8-6 & 977.03 & 6.89 & 
980.32 & 6.61 & \bf{972.50} & 
0.47\\SCA8-7 & 1067.20 & 6.34 & 
1068.13 & 6.36 & \bf{1059.70} & 
0.71\\SCA8-8 & \bf{\underline{1071.18}} & 6.17 & 
1072.13 & 6.00 & 1082.70 & 
-1.06\\SCA8-9 & \bf{\underline{1067.42}} & 4.82 & 
1067.42 & 4.89 & 1081.40 & 
-1.29\\CON3-0 & 621.22 & 7.60 & 
623.35 & 6.52 & \bf{616.50} & 
0.77\\CON3-1 & \bf{\underline{554.47}} & 5.81 & 
556.98 & 5.80 & 555.60 & 
-0.20\\CON3-2 & \bf{\underline{521.38}} & 5.33 & 
522.10 & 5.42 & 521.40 & 
-0.00\\CON3-3 & \bf{\underline{591.19}} & 6.02 & 
591.97 & 6.05 & 591.20 & 
-0.00\\CON3-4 & \bf{\underline{588.79}} & 5.29 & 
588.92 & 5.22 & 589.30 & 
-0.09\\CON3-5 & \bf{563.70} & 5.11 & 
566.51 & 5.24 & 563.70 & 0.00\\
CON3-6 & 500.80 & 6.74 & 
502.77 & 6.88 & \bf{499.20} & 
0.32\\CON3-7 & \bf{\underline{576.87}} & 4.92 & 
578.74 & 4.93 & 577.50 & 
-0.11\\CON3-8 & \bf{\underline{523.05}} & 4.53 & 
524.56 & 4.79 & 523.10 & 
-0.01\\CON3-9 & 588.40 & 5.50 & 
588.59 & 5.49 & \bf{578.20} & 
1.76\\CON8-0 & 869.43 & 5.59 & 
870.92 & 5.59 & \bf{858.90} & 
1.23\\CON8-1 & \bf{\underline{740.85}} & 5.25 & 
745.78 & 5.37 & 740.90 & 
-0.01\\CON8-2 & \bf{\underline{713.44}} & 8.22 & 
713.73 & 7.97 & 714.30 & 
-0.12\\CON8-3 & 817.57 & 5.72 & 
817.70 & 5.84 & \bf{812.30} & 
0.65\\CON8-4 & 782.89 & 5.58 & 
787.99 & 5.65 & \bf{770.10} & 
1.66\\CON8-5 & \bf{\underline{759.93}} & 5.38 & 
762.41 & 5.46 & 766.60 & 
-0.87\\CON8-6 & \bf{\underline{683.83}} & 7.59 & 
691.75 & 6.62 & 697.20 & 
-1.92\\CON8-7 & \bf{\underline{814.79}} & 4.62 & 
820.24 & 4.84 & 814.80 & 
-0.00\\CON8-8 & 778.50 & 6.74 & 
784.27 & 6.77 & \bf{771.30} & 
0.93\\CON8-9 & 815.49 & 6.18 & 
817.08 & 6.41 & \bf{815.10} & 
0.05\\[1ex]\hline
\end{tabular}
\label{table:nonlin}
\end{table} \clearpage
\begin{table}[ht]
\caption{Resultados de la ejecución de la metaheurística ACO, utilizando instancias de Dethloff con la configuración -n 8.0 -alpha 1.0 -beta 3.0 -q .9 -ro 0.015}
\centering
\small
\begin{tabular}{c c c c c c c}
\hline\hline
Instancia & Costo mínimo & Tiempo(seg.) & Costo promedio & Tiempo promedio(seg.) & Costo ACO & \%Gap \\ [0.5ex]
\hline
SCA3-0 & \bf{\underline{636.06}} & 5.36 & 
638.30 & 5.47 & 636.10 & 
-0.01\\SCA3-1 & \bf{\underline{697.84}} & 5.70 & 
697.84 & 5.95 & 700.10 & 
-0.32\\SCA3-2 & 661.13 & 5.25 & 
663.42 & 5.35 & \bf{659.30} & 
0.28\\SCA3-3 & 680.04 & 5.74 & 
680.64 & 5.64 & \bf{680.00} & 
0.01\\SCA3-4 & \bf{690.50} & 5.50 & 
690.50 & 5.87 & 690.50 & 0.00\\
SCA3-5 & \bf{\underline{665.04}} & 5.60 & 
665.49 & 5.82 & 671.10 & 
-0.90\\SCA3-6 & 652.94 & 5.93 & 
654.21 & 5.46 & \bf{651.10} & 
0.28\\SCA3-7 & 666.15 & 4.09 & 
666.15 & 4.46 & \bf{666.10} & 
0.01\\SCA3-8 & \bf{\underline{719.47}} & 4.72 & 
721.86 & 4.80 & 719.50 & 
-0.00\\SCA3-9 & \bf{681.00} & 4.96 & 
681.00 & 4.55 & 681.00 & 0.00\\
SCA8-0 & \bf{\underline{961.50}} & 6.17 & 
976.90 & 6.05 & 961.60 & 
-0.01\\SCA8-1 & \bf{\underline{1054.81}} & 5.09 & 
1062.36 & 4.86 & 1063.00 & 
-0.77\\SCA8-2 & 1050.37 & 4.44 & 
1053.52 & 4.24 & \bf{1040.60} & 
0.94\\SCA8-3 & 1007.99 & 6.38 & 
1014.72 & 6.20 & \bf{985.90} & 
2.24\\SCA8-4 & \bf{\underline{1065.49}} & 6.21 & 
1071.32 & 5.95 & 1071.00 & 
-0.51\\SCA8-5 & \bf{\underline{1047.55}} & 6.17 & 
1053.40 & 6.46 & 1054.30 & 
-0.64\\SCA8-6 & 977.03 & 6.93 & 
979.43 & 6.70 & \bf{972.50} & 
0.47\\SCA8-7 & 1067.20 & 6.36 & 
1070.49 & 6.62 & \bf{1059.70} & 
0.71\\SCA8-8 & \bf{\underline{1071.18}} & 6.18 & 
1077.94 & 5.95 & 1082.70 & 
-1.06\\SCA8-9 & \bf{\underline{1067.42}} & 4.49 & 
1067.42 & 4.78 & 1081.40 & 
-1.29\\CON3-0 & 617.59 & 6.12 & 
621.65 & 6.51 & \bf{616.50} & 
0.18\\CON3-1 & \bf{\underline{554.47}} & 5.62 & 
556.61 & 5.80 & 555.60 & 
-0.20\\CON3-2 & \bf{\underline{521.38}} & 5.41 & 
521.38 & 5.07 & 521.40 & 
-0.00\\CON3-3 & \bf{\underline{591.19}} & 6.02 & 
591.20 & 6.15 & 591.20 & 
-0.00\\CON3-4 & \bf{\underline{588.79}} & 4.83 & 
589.58 & 5.08 & 589.30 & 
-0.09\\CON3-5 & 564.88 & 5.50 & 
565.92 & 5.42 & \bf{563.70} & 
0.21\\CON3-6 & 502.16 & 6.90 & 
502.92 & 6.89 & \bf{499.20} & 
0.59\\CON3-7 & 578.22 & 4.75 & 
579.05 & 4.75 & \bf{577.50} & 
0.12\\CON3-8 & 523.68 & 4.96 & 
527.47 & 5.02 & \bf{523.10} & 
0.11\\CON3-9 & 578.25 & 4.95 & 
585.92 & 5.09 & \bf{578.20} & 
0.01\\CON8-0 & 868.49 & 5.92 & 
877.76 & 5.93 & \bf{858.90} & 
1.12\\CON8-1 & \bf{\underline{740.85}} & 6.45 & 
745.43 & 5.72 & 740.90 & 
-0.01\\CON8-2 & \bf{\underline{713.44}} & 8.06 & 
714.28 & 7.82 & 714.30 & 
-0.12\\CON8-3 & 817.57 & 5.64 & 
817.57 & 5.67 & \bf{812.30} & 
0.65\\CON8-4 & 778.60 & 6.22 & 
785.69 & 6.28 & \bf{770.10} & 
1.10\\CON8-5 & \bf{\underline{761.40}} & 5.64 & 
764.13 & 5.68 & 766.60 & 
-0.68\\CON8-6 & \bf{\underline{695.66}} & 6.44 & 
698.09 & 6.61 & 697.20 & 
-0.22\\CON8-7 & \bf{\underline{814.77}} & 4.85 & 
819.18 & 5.01 & 814.80 & 
-0.00\\CON8-8 & 788.09 & 6.39 & 
791.50 & 6.47 & \bf{771.30} & 
2.18\\CON8-9 & 815.44 & 6.19 & 
816.50 & 6.46 & \bf{815.10} & 
0.04\\[1ex]\hline
\end{tabular}
\label{table:nonlin}
\end{table} \clearpage
\begin{table}[ht]
\caption{Resultados de la ejecución de la metaheurística ACO, utilizando instancias de Dethloff con la configuración -n 9.0 -alpha 1.0 -beta 3.0 -q 0.1 -ro 0.015}
\centering
\small
\begin{tabular}{c c c c c c c}
\hline\hline
Instancia & Costo mínimo & Tiempo(seg.) & Costo promedio & Tiempo promedio(seg.) & Costo ACO & \%Gap \\ [0.5ex]
\hline
SCA3-0 & \bf{\underline{636.06}} & 6.26 & 
636.06 & 6.50 & 636.10 & 
-0.01\\SCA3-1 & \bf{\underline{697.84}} & 7.12 & 
697.84 & 7.15 & 700.10 & 
-0.32\\SCA3-2 & 659.34 & 5.97 & 
659.34 & 6.29 & \bf{659.30} & 
0.01\\SCA3-3 & 680.04 & 6.36 & 
680.04 & 6.42 & \bf{680.00} & 
0.01\\SCA3-4 & \bf{690.50} & 7.18 & 
690.50 & 6.97 & 690.50 & 0.00\\
SCA3-5 & \bf{\underline{662.75}} & 6.77 & 
662.75 & 6.81 & 671.10 & 
-1.24\\SCA3-6 & \bf{\underline{651.09}} & 6.50 & 
652.48 & 6.59 & 651.10 & 
-0.00\\SCA3-7 & \bf{\underline{659.17}} & 5.61 & 
664.40 & 5.79 & 666.10 & 
-1.04\\SCA3-8 & \bf{\underline{719.47}} & 6.17 & 
719.47 & 7.36 & 719.50 & 
-0.00\\SCA3-9 & \bf{681.00} & 5.97 & 
681.00 & 5.74 & 681.00 & 0.00\\
SCA8-0 & \bf{\underline{961.50}} & 7.42 & 
969.84 & 7.13 & 961.60 & 
-0.01\\SCA8-1 & \bf{\underline{1055.75}} & 6.29 & 
1060.03 & 6.13 & 1063.00 & 
-0.68\\SCA8-2 & 1050.37 & 5.36 & 
1050.58 & 5.53 & \bf{1040.60} & 
0.94\\SCA8-3 & 1002.86 & 6.62 & 
1006.43 & 6.62 & \bf{985.90} & 
1.72\\SCA8-4 & \bf{\underline{1065.49}} & 6.94 & 
1066.73 & 6.85 & 1071.00 & 
-0.51\\SCA8-5 & \bf{\underline{1029.95}} & 7.79 & 
1047.93 & 7.65 & 1054.30 & 
-2.31\\SCA8-6 & \bf{\underline{972.48}} & 7.31 & 
977.09 & 7.20 & 972.50 & 
-0.00\\SCA8-7 & 1067.20 & 7.67 & 
1068.75 & 7.35 & \bf{1059.70} & 
0.71\\SCA8-8 & \bf{\underline{1071.18}} & 7.48 & 
1075.07 & 8.14 & 1082.70 & 
-1.06\\SCA8-9 & \bf{\underline{1067.42}} & 6.41 & 
1067.42 & 6.11 & 1081.40 & 
-1.29\\CON3-0 & 616.52 & 6.76 & 
620.02 & 7.13 & \bf{616.50} & 
0.00\\CON3-1 & \bf{\underline{554.47}} & 6.65 & 
554.47 & 6.75 & 555.60 & 
-0.20\\CON3-2 & \bf{\underline{521.38}} & 6.75 & 
521.38 & 6.89 & 521.40 & 
-0.00\\CON3-3 & \bf{\underline{591.19}} & 7.61 & 
591.20 & 7.38 & 591.20 & 
-0.00\\CON3-4 & \bf{\underline{588.79}} & 6.32 & 
588.79 & 6.21 & 589.30 & 
-0.09\\CON3-5 & \bf{563.70} & 6.46 & 
564.29 & 6.71 & 563.70 & 0.00\\
CON3-6 & 500.37 & 7.56 & 
501.19 & 7.52 & \bf{499.20} & 
0.23\\CON3-7 & \bf{\underline{576.48}} & 6.46 & 
578.64 & 5.98 & 577.50 & 
-0.18\\CON3-8 & \bf{\underline{523.05}} & 6.76 & 
523.07 & 6.51 & 523.10 & 
-0.01\\CON3-9 & 578.98 & 6.43 & 
582.95 & 6.50 & \bf{578.20} & 
0.13\\CON8-0 & 870.28 & 7.05 & 
874.09 & 6.91 & \bf{858.90} & 
1.32\\CON8-1 & \bf{\underline{740.85}} & 6.85 & 
741.42 & 7.12 & 740.90 & 
-0.01\\CON8-2 & \bf{\underline{712.89}} & 8.73 & 
713.61 & 8.15 & 714.30 & 
-0.20\\CON8-3 & \bf{\underline{811.07}} & 6.84 & 
813.31 & 6.91 & 812.30 & 
-0.15\\CON8-4 & 772.80 & 6.84 & 
778.77 & 6.65 & \bf{770.10} & 
0.35\\CON8-5 & \bf{\underline{758.12}} & 6.84 & 
759.90 & 6.83 & 766.60 & 
-1.11\\CON8-6 & \bf{\underline{688.00}} & 7.48 & 
691.48 & 7.72 & 697.20 & 
-1.32\\CON8-7 & \bf{\underline{814.79}} & 6.05 & 
815.15 & 5.90 & 814.80 & 
-0.00\\CON8-8 & 773.89 & 7.39 & 
781.47 & 7.91 & \bf{771.30} & 
0.34\\CON8-9 & \bf{\underline{810.18}} & 7.38 & 
812.59 & 7.91 & 815.10 & 
-0.60\\[1ex]\hline
\end{tabular}
\label{table:nonlin}
\end{table} \clearpage
\begin{table}[ht]
\caption{Resultados de la ejecución de la metaheurística ACO, utilizando instancias de Dethloff con la configuración -n 9.0 -alpha 1.0 -beta 3.0 -q .2 -ro 0.015}
\centering
\small
\begin{tabular}{c c c c c c c}
\hline\hline
Instancia & Costo mínimo & Tiempo(seg.) & Costo promedio & Tiempo promedio(seg.) & Costo ACO & \%Gap \\ [0.5ex]
\hline
SCA3-0 & \bf{\underline{636.06}} & 6.51 & 
636.06 & 6.44 & 636.10 & 
-0.01\\SCA3-1 & \bf{\underline{697.84}} & 7.03 & 
697.84 & 7.16 & 700.10 & 
-0.32\\SCA3-2 & 659.34 & 6.05 & 
659.34 & 6.33 & \bf{659.30} & 
0.01\\SCA3-3 & 680.04 & 5.91 & 
680.18 & 6.24 & \bf{680.00} & 
0.01\\SCA3-4 & \bf{690.50} & 7.02 & 
690.50 & 6.83 & 690.50 & 0.00\\
SCA3-5 & \bf{\underline{659.90}} & 6.71 & 
662.61 & 6.67 & 671.10 & 
-1.67\\SCA3-6 & 652.94 & 6.20 & 
652.94 & 6.38 & \bf{651.10} & 
0.28\\SCA3-7 & 666.15 & 5.72 & 
666.15 & 5.61 & \bf{666.10} & 
0.01\\SCA3-8 & \bf{\underline{719.47}} & 7.00 & 
719.97 & 6.61 & 719.50 & 
-0.00\\SCA3-9 & \bf{681.00} & 5.33 & 
681.00 & 5.46 & 681.00 & 0.00\\
SCA8-0 & \bf{\underline{961.50}} & 7.26 & 
968.94 & 6.99 & 961.60 & 
-0.01\\SCA8-1 & \bf{\underline{1053.90}} & 6.17 & 
1058.20 & 6.03 & 1063.00 & 
-0.86\\SCA8-2 & 1044.24 & 5.21 & 
1048.79 & 5.33 & \bf{1040.60} & 
0.35\\SCA8-3 & \bf{\underline{985.60}} & 6.55 & 
998.75 & 6.54 & 985.90 & 
-0.03\\SCA8-4 & \bf{\underline{1065.49}} & 6.60 & 
1067.06 & 6.97 & 1071.00 & 
-0.51\\SCA8-5 & \bf{\underline{1034.74}} & 7.39 & 
1041.94 & 7.57 & 1054.30 & 
-1.86\\SCA8-6 & \bf{\underline{971.82}} & 7.24 & 
977.24 & 7.22 & 972.50 & 
-0.07\\SCA8-7 & 1067.03 & 6.99 & 
1067.16 & 7.07 & \bf{1059.70} & 
0.69\\SCA8-8 & \bf{\underline{1071.18}} & 7.35 & 
1071.18 & 7.25 & 1082.70 & 
-1.06\\SCA8-9 & \bf{\underline{1067.42}} & 5.19 & 
1067.42 & 5.55 & 1081.40 & 
-1.29\\CON3-0 & 616.52 & 7.23 & 
617.32 & 7.07 & \bf{616.50} & 
0.00\\CON3-1 & \bf{\underline{554.47}} & 6.90 & 
556.23 & 6.97 & 555.60 & 
-0.20\\CON3-2 & \bf{\underline{519.11}} & 7.15 & 
520.81 & 6.57 & 521.40 & 
-0.44\\CON3-3 & \bf{\underline{591.19}} & 7.49 & 
591.19 & 7.20 & 591.20 & 
-0.00\\CON3-4 & \bf{\underline{588.79}} & 6.02 & 
589.19 & 6.24 & 589.30 & 
-0.09\\CON3-5 & 564.88 & 6.86 & 
566.62 & 6.80 & \bf{563.70} & 
0.21\\CON3-6 & \bf{\underline{499.05}} & 7.63 & 
500.70 & 7.90 & 499.20 & 
-0.03\\CON3-7 & \bf{\underline{576.48}} & 6.12 & 
577.75 & 6.11 & 577.50 & 
-0.18\\CON3-8 & \bf{\underline{523.05}} & 6.62 & 
523.21 & 6.43 & 523.10 & 
-0.01\\CON3-9 & 578.25 & 6.25 & 
583.06 & 6.35 & \bf{578.20} & 
0.01\\CON8-0 & 869.43 & 6.99 & 
872.32 & 6.92 & \bf{858.90} & 
1.23\\CON8-1 & \bf{\underline{740.85}} & 7.54 & 
741.23 & 7.37 & 740.90 & 
-0.01\\CON8-2 & \bf{\underline{712.89}} & 8.23 & 
713.34 & 8.24 & 714.30 & 
-0.20\\CON8-3 & 815.14 & 6.93 & 
816.79 & 6.83 & \bf{812.30} & 
0.35\\CON8-4 & 776.37 & 6.64 & 
776.73 & 6.88 & \bf{770.10} & 
0.81\\CON8-5 & \bf{\underline{754.95}} & 6.34 & 
759.03 & 6.46 & 766.60 & 
-1.52\\CON8-6 & \bf{\underline{684.69}} & 7.39 & 
688.97 & 7.68 & 697.20 & 
-1.79\\CON8-7 & \bf{\underline{814.79}} & 6.20 & 
814.83 & 6.07 & 814.80 & 
-0.00\\CON8-8 & 782.86 & 7.37 & 
784.23 & 7.65 & \bf{771.30} & 
1.50\\CON8-9 & \bf{\underline{812.60}} & 7.49 & 
814.45 & 7.61 & 815.10 & 
-0.31\\[1ex]\hline
\end{tabular}
\label{table:nonlin}
\end{table} \clearpage
\begin{table}[ht]
\caption{Resultados de la ejecución de la metaheurística ACO, utilizando instancias de Dethloff con la configuración -n 9.0 -alpha 1.0 -beta 3.0 -q .3 -ro 0.015}
\centering
\small
\begin{tabular}{c c c c c c c}
\hline\hline
Instancia & Costo mínimo & Tiempo(seg.) & Costo promedio & Tiempo promedio(seg.) & Costo ACO & \%Gap \\ [0.5ex]
\hline
SCA3-0 & \bf{\underline{636.06}} & 6.68 & 
636.06 & 6.25 & 636.10 & 
-0.01\\SCA3-1 & \bf{\underline{697.84}} & 6.43 & 
697.84 & 6.90 & 700.10 & 
-0.32\\SCA3-2 & 659.34 & 6.19 & 
659.79 & 6.21 & \bf{659.30} & 
0.01\\SCA3-3 & 680.04 & 6.69 & 
680.04 & 6.32 & \bf{680.00} & 
0.01\\SCA3-4 & \bf{690.50} & 6.66 & 
690.50 & 6.76 & 690.50 & 0.00\\
SCA3-5 & \bf{\underline{662.75}} & 6.61 & 
663.47 & 6.44 & 671.10 & 
-1.24\\SCA3-6 & \bf{\underline{651.09}} & 6.23 & 
652.48 & 6.44 & 651.10 & 
-0.00\\SCA3-7 & 666.15 & 6.24 & 
666.15 & 5.94 & \bf{666.10} & 
0.01\\SCA3-8 & \bf{\underline{719.47}} & 6.61 & 
719.54 & 6.33 & 719.50 & 
-0.00\\SCA3-9 & \bf{681.00} & 5.46 & 
681.00 & 5.67 & 681.00 & 0.00\\
SCA8-0 & \bf{\underline{961.50}} & 6.61 & 
974.56 & 7.11 & 961.60 & 
-0.01\\SCA8-1 & \bf{\underline{1054.87}} & 6.11 & 
1059.68 & 5.95 & 1063.00 & 
-0.76\\SCA8-2 & 1049.22 & 5.10 & 
1050.08 & 5.29 & \bf{1040.60} & 
0.83\\SCA8-3 & 1002.00 & 6.28 & 
1010.13 & 6.59 & \bf{985.90} & 
1.63\\SCA8-4 & \bf{\underline{1065.49}} & 6.74 & 
1066.99 & 6.81 & 1071.00 & 
-0.51\\SCA8-5 & \bf{\underline{1038.26}} & 7.82 & 
1041.79 & 7.59 & 1054.30 & 
-1.52\\SCA8-6 & \bf{\underline{972.48}} & 6.93 & 
974.07 & 7.15 & 972.50 & 
-0.00\\SCA8-7 & 1066.65 & 7.19 & 
1067.94 & 7.18 & \bf{1059.70} & 
0.66\\SCA8-8 & \bf{\underline{1071.18}} & 7.43 & 
1071.18 & 7.33 & 1082.70 & 
-1.06\\SCA8-9 & \bf{\underline{1067.42}} & 5.68 & 
1067.42 & 5.73 & 1081.40 & 
-1.29\\CON3-0 & 616.52 & 7.08 & 
621.33 & 7.18 & \bf{616.50} & 
0.00\\CON3-1 & \bf{\underline{554.47}} & 6.62 & 
555.55 & 6.81 & 555.60 & 
-0.20\\CON3-2 & \bf{\underline{519.11}} & 6.34 & 
520.81 & 6.46 & 521.40 & 
-0.44\\CON3-3 & \bf{\underline{591.19}} & 6.53 & 
591.20 & 7.07 & 591.20 & 
-0.00\\CON3-4 & \bf{\underline{588.79}} & 5.71 & 
588.79 & 5.82 & 589.30 & 
-0.09\\CON3-5 & \bf{563.70} & 6.30 & 
565.24 & 6.53 & 563.70 & 0.00\\
CON3-6 & 500.80 & 7.87 & 
501.14 & 7.66 & \bf{499.20} & 
0.32\\CON3-7 & \bf{\underline{576.48}} & 6.19 & 
577.44 & 5.93 & 577.50 & 
-0.18\\CON3-8 & \bf{\underline{523.05}} & 6.12 & 
524.34 & 6.19 & 523.10 & 
-0.01\\CON3-9 & 582.79 & 6.22 & 
586.79 & 6.17 & \bf{578.20} & 
0.79\\CON8-0 & \bf{\underline{858.88}} & 6.53 & 
869.26 & 6.87 & 858.90 & 
-0.00\\CON8-1 & \bf{\underline{740.85}} & 6.84 & 
741.21 & 6.71 & 740.90 & 
-0.01\\CON8-2 & \bf{\underline{713.44}} & 7.87 & 
713.58 & 8.35 & 714.30 & 
-0.12\\CON8-3 & \bf{\underline{811.07}} & 7.12 & 
814.65 & 6.86 & 812.30 & 
-0.15\\CON8-4 & 776.37 & 6.31 & 
784.04 & 6.36 & \bf{770.10} & 
0.81\\CON8-5 & \bf{\underline{758.12}} & 6.59 & 
759.48 & 6.75 & 766.60 & 
-1.11\\CON8-6 & \bf{\underline{683.83}} & 7.40 & 
689.78 & 7.68 & 697.20 & 
-1.92\\CON8-7 & 814.86 & 5.64 & 
815.00 & 5.97 & \bf{814.80} & 
0.01\\CON8-8 & 779.57 & 7.24 & 
784.77 & 7.41 & \bf{771.30} & 
1.07\\CON8-9 & \bf{\underline{812.60}} & 7.59 & 
812.60 & 8.16 & 815.10 & 
-0.31\\[1ex]\hline
\end{tabular}
\label{table:nonlin}
\end{table} \clearpage
\begin{table}[ht]
\caption{Resultados de la ejecución de la metaheurística ACO, utilizando instancias de Dethloff con la configuración -n 9.0 -alpha 1.0 -beta 3.0 -q .4 -ro 0.015}
\centering
\small
\begin{tabular}{c c c c c c c}
\hline\hline
Instancia & Costo mínimo & Tiempo(seg.) & Costo promedio & Tiempo promedio(seg.) & Costo ACO & \%Gap \\ [0.5ex]
\hline
SCA3-0 & \bf{\underline{636.06}} & 6.62 & 
636.06 & 6.46 & 636.10 & 
-0.01\\SCA3-1 & \bf{\underline{697.84}} & 6.75 & 
697.84 & 6.80 & 700.10 & 
-0.32\\SCA3-2 & 659.34 & 6.20 & 
660.55 & 6.07 & \bf{659.30} & 
0.01\\SCA3-3 & 680.04 & 5.98 & 
680.04 & 6.11 & \bf{680.00} & 
0.01\\SCA3-4 & \bf{690.50} & 6.62 & 
690.50 & 6.78 & 690.50 & 0.00\\
SCA3-5 & \bf{\underline{662.75}} & 6.97 & 
662.75 & 6.93 & 671.10 & 
-1.24\\SCA3-6 & 652.94 & 6.27 & 
652.94 & 6.40 & \bf{651.10} & 
0.28\\SCA3-7 & \bf{\underline{664.88}} & 6.47 & 
665.83 & 5.70 & 666.10 & 
-0.18\\SCA3-8 & \bf{\underline{719.47}} & 5.74 & 
719.62 & 6.11 & 719.50 & 
-0.00\\SCA3-9 & \bf{681.00} & 5.60 & 
681.00 & 5.42 & 681.00 & 0.00\\
SCA8-0 & 971.49 & 6.58 & 
974.74 & 6.98 & \bf{961.60} & 
1.03\\SCA8-1 & \bf{\underline{1050.20}} & 5.97 & 
1058.32 & 5.90 & 1063.00 & 
-1.20\\SCA8-2 & 1046.29 & 5.32 & 
1049.61 & 5.22 & \bf{1040.60} & 
0.55\\SCA8-3 & 1004.25 & 7.03 & 
1010.26 & 6.66 & \bf{985.90} & 
1.86\\SCA8-4 & \bf{\underline{1065.49}} & 6.54 & 
1069.65 & 6.62 & 1071.00 & 
-0.51\\SCA8-5 & \bf{\underline{1036.88}} & 7.00 & 
1048.51 & 7.09 & 1054.30 & 
-1.65\\SCA8-6 & 977.87 & 7.26 & 
979.14 & 7.13 & \bf{972.50} & 
0.55\\SCA8-7 & 1067.11 & 7.13 & 
1069.09 & 7.22 & \bf{1059.70} & 
0.70\\SCA8-8 & \bf{\underline{1071.18}} & 7.18 & 
1076.64 & 7.20 & 1082.70 & 
-1.06\\SCA8-9 & \bf{\underline{1065.60}} & 5.50 & 
1066.97 & 5.48 & 1081.40 & 
-1.46\\CON3-0 & 617.59 & 7.01 & 
620.18 & 6.98 & \bf{616.50} & 
0.18\\CON3-1 & \bf{\underline{554.47}} & 6.94 & 
554.86 & 6.78 & 555.60 & 
-0.20\\CON3-2 & \bf{\underline{519.11}} & 6.56 & 
520.81 & 6.55 & 521.40 & 
-0.44\\CON3-3 & \bf{\underline{591.19}} & 7.21 & 
591.20 & 7.36 & 591.20 & 
-0.00\\CON3-4 & \bf{\underline{588.79}} & 5.90 & 
588.79 & 6.00 & 589.30 & 
-0.09\\CON3-5 & 564.88 & 6.72 & 
566.76 & 6.87 & \bf{563.70} & 
0.21\\CON3-6 & 500.37 & 7.44 & 
500.93 & 7.61 & \bf{499.20} & 
0.23\\CON3-7 & 578.22 & 5.66 & 
578.32 & 5.82 & \bf{577.50} & 
0.12\\CON3-8 & \bf{\underline{523.05}} & 6.21 & 
523.82 & 6.12 & 523.10 & 
-0.01\\CON3-9 & 581.06 & 5.68 & 
586.75 & 6.22 & \bf{578.20} & 
0.49\\CON8-0 & 867.63 & 6.95 & 
869.58 & 6.73 & \bf{858.90} & 
1.02\\CON8-1 & \bf{\underline{740.85}} & 7.29 & 
741.98 & 6.89 & 740.90 & 
-0.01\\CON8-2 & \bf{\underline{712.89}} & 8.22 & 
713.82 & 8.20 & 714.30 & 
-0.20\\CON8-3 & \bf{\underline{811.07}} & 6.74 & 
813.24 & 6.62 & 812.30 & 
-0.15\\CON8-4 & 776.34 & 6.29 & 
779.01 & 6.46 & \bf{770.10} & 
0.81\\CON8-5 & \bf{\underline{759.93}} & 6.45 & 
762.09 & 6.42 & 766.60 & 
-0.87\\CON8-6 & \bf{\underline{690.19}} & 7.40 & 
693.56 & 7.51 & 697.20 & 
-1.01\\CON8-7 & \bf{\underline{814.79}} & 5.89 & 
814.83 & 5.89 & 814.80 & 
-0.00\\CON8-8 & 771.32 & 7.62 & 
782.72 & 7.98 & \bf{771.30} & 
0.00\\CON8-9 & \bf{\underline{810.18}} & 7.32 & 
813.50 & 7.16 & 815.10 & 
-0.60\\[1ex]\hline
\end{tabular}
\label{table:nonlin}
\end{table} \clearpage
\begin{table}[ht]
\caption{Resultados de la ejecución de la metaheurística ACO, utilizando instancias de Dethloff con la configuración -n 9.0 -alpha 1.0 -beta 3.0 -q .5 -ro 0.015}
\centering
\small
\begin{tabular}{c c c c c c c}
\hline\hline
Instancia & Costo mínimo & Tiempo(seg.) & Costo promedio & Tiempo promedio(seg.) & Costo ACO & \%Gap \\ [0.5ex]
\hline
SCA3-0 & \bf{\underline{636.06}} & 5.80 & 
636.06 & 6.12 & 636.10 & 
-0.01\\SCA3-1 & \bf{\underline{697.84}} & 6.88 & 
697.84 & 6.83 & 700.10 & 
-0.32\\SCA3-2 & 659.34 & 6.12 & 
659.34 & 5.98 & \bf{659.30} & 
0.01\\SCA3-3 & 680.04 & 6.38 & 
680.04 & 6.27 & \bf{680.00} & 
0.01\\SCA3-4 & \bf{690.50} & 6.99 & 
690.50 & 6.75 & 690.50 & 0.00\\
SCA3-5 & \bf{\underline{659.90}} & 6.35 & 
661.90 & 6.63 & 671.10 & 
-1.67\\SCA3-6 & \bf{\underline{651.09}} & 6.06 & 
652.55 & 6.16 & 651.10 & 
-0.00\\SCA3-7 & 666.15 & 5.24 & 
666.15 & 5.32 & \bf{666.10} & 
0.01\\SCA3-8 & \bf{\underline{719.47}} & 5.54 & 
719.97 & 6.06 & 719.50 & 
-0.00\\SCA3-9 & \bf{681.00} & 5.41 & 
681.00 & 5.47 & 681.00 & 0.00\\
SCA8-0 & \bf{\underline{961.50}} & 7.06 & 
969.71 & 7.05 & 961.60 & 
-0.01\\SCA8-1 & \bf{\underline{1052.71}} & 5.57 & 
1058.32 & 5.57 & 1063.00 & 
-0.97\\SCA8-2 & 1050.37 & 5.48 & 
1050.84 & 5.20 & \bf{1040.60} & 
0.94\\SCA8-3 & 1009.30 & 6.46 & 
1011.66 & 6.72 & \bf{985.90} & 
2.37\\SCA8-4 & \bf{\underline{1065.49}} & 6.69 & 
1067.02 & 6.72 & 1071.00 & 
-0.51\\SCA8-5 & \bf{\underline{1034.74}} & 6.94 & 
1045.03 & 7.55 & 1054.30 & 
-1.86\\SCA8-6 & \bf{\underline{972.48}} & 7.02 & 
976.86 & 7.04 & 972.50 & 
-0.00\\SCA8-7 & 1066.65 & 7.32 & 
1068.53 & 7.37 & \bf{1059.70} & 
0.66\\SCA8-8 & \bf{\underline{1071.18}} & 7.45 & 
1071.18 & 7.27 & 1082.70 & 
-1.06\\SCA8-9 & \bf{\underline{1065.60}} & 5.65 & 
1066.97 & 5.49 & 1081.40 & 
-1.46\\CON3-0 & 617.59 & 7.24 & 
620.07 & 7.04 & \bf{616.50} & 
0.18\\CON3-1 & \bf{\underline{554.47}} & 6.46 & 
554.86 & 6.56 & 555.60 & 
-0.20\\CON3-2 & \bf{\underline{519.61}} & 6.60 & 
520.93 & 6.16 & 521.40 & 
-0.34\\CON3-3 & \bf{\underline{591.19}} & 7.16 & 
591.19 & 7.01 & 591.20 & 
-0.00\\CON3-4 & \bf{\underline{588.79}} & 6.02 & 
588.79 & 6.17 & 589.30 & 
-0.09\\CON3-5 & \bf{563.70} & 6.17 & 
566.49 & 6.33 & 563.70 & 0.00\\
CON3-6 & 500.80 & 8.01 & 
502.02 & 7.74 & \bf{499.20} & 
0.32\\CON3-7 & \bf{\underline{576.84}} & 5.50 & 
578.02 & 5.75 & 577.50 & 
-0.11\\CON3-8 & \bf{\underline{523.05}} & 5.71 & 
523.62 & 5.83 & 523.10 & 
-0.01\\CON3-9 & 578.25 & 6.41 & 
583.21 & 6.15 & \bf{578.20} & 
0.01\\CON8-0 & 869.43 & 6.75 & 
873.34 & 7.01 & \bf{858.90} & 
1.23\\CON8-1 & \bf{\underline{740.85}} & 6.70 & 
740.85 & 6.75 & 740.90 & 
-0.01\\CON8-2 & \bf{\underline{713.44}} & 8.73 & 
714.54 & 8.57 & 714.30 & 
-0.12\\CON8-3 & \bf{\underline{811.07}} & 6.79 & 
813.05 & 6.64 & 812.30 & 
-0.15\\CON8-4 & 783.88 & 6.37 & 
789.69 & 6.48 & \bf{770.10} & 
1.79\\CON8-5 & \bf{\underline{758.84}} & 6.20 & 
761.70 & 6.30 & 766.60 & 
-1.01\\CON8-6 & \bf{\underline{690.85}} & 7.63 & 
693.44 & 7.60 & 697.20 & 
-0.91\\CON8-7 & \bf{\underline{814.50}} & 5.88 & 
816.59 & 5.76 & 814.80 & 
-0.04\\CON8-8 & 785.30 & 7.52 & 
788.11 & 7.30 & \bf{771.30} & 
1.82\\CON8-9 & \bf{\underline{810.18}} & 7.18 & 
812.24 & 7.29 & 815.10 & 
-0.60\\[1ex]\hline
\end{tabular}
\label{table:nonlin}
\end{table} \clearpage
\begin{table}[ht]
\caption{Resultados de la ejecución de la metaheurística ACO, utilizando instancias de Dethloff con la configuración -n 9.0 -alpha 1.0 -beta 3.0 -q .6 -ro 0.015}
\centering
\small
\begin{tabular}{c c c c c c c}
\hline\hline
Instancia & Costo mínimo & Tiempo(seg.) & Costo promedio & Tiempo promedio(seg.) & Costo ACO & \%Gap \\ [0.5ex]
\hline
SCA3-0 & \bf{\underline{636.06}} & 5.83 & 
636.06 & 6.16 & 636.10 & 
-0.01\\SCA3-1 & \bf{\underline{697.84}} & 7.05 & 
697.84 & 6.83 & 700.10 & 
-0.32\\SCA3-2 & 659.34 & 6.00 & 
659.79 & 6.04 & \bf{659.30} & 
0.01\\SCA3-3 & 680.04 & 5.92 & 
680.04 & 6.01 & \bf{680.00} & 
0.01\\SCA3-4 & \bf{690.50} & 6.83 & 
690.50 & 6.70 & 690.50 & 0.00\\
SCA3-5 & \bf{\underline{661.07}} & 7.02 & 
663.77 & 6.80 & 671.10 & 
-1.49\\SCA3-6 & 652.94 & 6.00 & 
653.16 & 6.12 & \bf{651.10} & 
0.28\\SCA3-7 & 666.15 & 5.19 & 
666.15 & 5.33 & \bf{666.10} & 
0.01\\SCA3-8 & \bf{\underline{719.47}} & 5.83 & 
719.54 & 6.17 & 719.50 & 
-0.00\\SCA3-9 & \bf{681.00} & 4.86 & 
681.00 & 5.01 & 681.00 & 0.00\\
SCA8-0 & 982.79 & 7.12 & 
983.16 & 6.75 & \bf{961.60} & 
2.20\\SCA8-1 & \bf{\underline{1054.87}} & 5.67 & 
1058.26 & 5.58 & 1063.00 & 
-0.76\\SCA8-2 & 1046.29 & 5.00 & 
1049.38 & 5.10 & \bf{1040.60} & 
0.55\\SCA8-3 & 1007.97 & 6.91 & 
1012.69 & 7.08 & \bf{985.90} & 
2.24\\SCA8-4 & \bf{\underline{1065.49}} & 7.36 & 
1067.20 & 6.96 & 1071.00 & 
-0.51\\SCA8-5 & \bf{\underline{1041.29}} & 6.77 & 
1046.42 & 7.25 & 1054.30 & 
-1.23\\SCA8-6 & \bf{\underline{972.48}} & 6.95 & 
978.68 & 7.00 & 972.50 & 
-0.00\\SCA8-7 & 1067.20 & 7.60 & 
1069.26 & 7.47 & \bf{1059.70} & 
0.71\\SCA8-8 & \bf{\underline{1071.18}} & 7.64 & 
1071.18 & 7.21 & 1082.70 & 
-1.06\\SCA8-9 & \bf{\underline{1067.42}} & 5.84 & 
1067.42 & 5.62 & 1081.40 & 
-1.29\\CON3-0 & 616.52 & 7.06 & 
619.62 & 7.15 & \bf{616.50} & 
0.00\\CON3-1 & 556.04 & 6.63 & 
556.96 & 6.54 & \bf{555.60} & 
0.08\\CON3-2 & \bf{\underline{519.61}} & 6.30 & 
521.00 & 6.24 & 521.40 & 
-0.34\\CON3-3 & \bf{\underline{591.19}} & 7.21 & 
591.50 & 7.01 & 591.20 & 
-0.00\\CON3-4 & \bf{\underline{588.79}} & 6.03 & 
588.79 & 6.12 & 589.30 & 
-0.09\\CON3-5 & 564.88 & 6.94 & 
565.84 & 6.49 & \bf{563.70} & 
0.21\\CON3-6 & 500.80 & 7.34 & 
502.25 & 7.46 & \bf{499.20} & 
0.32\\CON3-7 & 578.22 & 5.65 & 
578.36 & 6.17 & \bf{577.50} & 
0.12\\CON3-8 & \bf{\underline{523.05}} & 6.30 & 
523.54 & 5.82 & 523.10 & 
-0.01\\CON3-9 & 586.17 & 5.93 & 
587.60 & 5.99 & \bf{578.20} & 
1.38\\CON8-0 & 867.97 & 6.64 & 
871.22 & 6.91 & \bf{858.90} & 
1.06\\CON8-1 & \bf{\underline{740.85}} & 6.09 & 
742.34 & 6.55 & 740.90 & 
-0.01\\CON8-2 & \bf{\underline{712.89}} & 7.73 & 
714.84 & 8.19 & 714.30 & 
-0.20\\CON8-3 & 812.32 & 6.34 & 
814.91 & 6.45 & \bf{812.30} & 
0.00\\CON8-4 & 776.34 & 5.86 & 
778.97 & 6.13 & \bf{770.10} & 
0.81\\CON8-5 & \bf{\underline{760.91}} & 6.42 & 
763.09 & 6.29 & 766.60 & 
-0.74\\CON8-6 & \bf{\underline{688.00}} & 7.51 & 
694.62 & 7.53 & 697.20 & 
-1.32\\CON8-7 & 814.86 & 5.62 & 
814.93 & 5.58 & \bf{814.80} & 
0.01\\CON8-8 & \bf{\underline{771.26}} & 7.13 & 
782.32 & 7.49 & 771.30 & 
-0.01\\CON8-9 & \bf{\underline{810.18}} & 7.78 & 
811.38 & 7.25 & 815.10 & 
-0.60\\[1ex]\hline
\end{tabular}
\label{table:nonlin}
\end{table} \clearpage
\begin{table}[ht]
\caption{Resultados de la ejecución de la metaheurística ACO, utilizando instancias de Dethloff con la configuración -n 9.0 -alpha 1.0 -beta 3.0 -q .7 -ro 0.015}
\centering
\small
\begin{tabular}{c c c c c c c}
\hline\hline
Instancia & Costo mínimo & Tiempo(seg.) & Costo promedio & Tiempo promedio(seg.) & Costo ACO & \%Gap \\ [0.5ex]
\hline
SCA3-0 & \bf{\underline{636.06}} & 5.83 & 
636.06 & 6.07 & 636.10 & 
-0.01\\SCA3-1 & \bf{\underline{697.84}} & 6.28 & 
697.84 & 6.66 & 700.10 & 
-0.32\\SCA3-2 & 659.34 & 6.08 & 
659.34 & 6.07 & \bf{659.30} & 
0.01\\SCA3-3 & 680.04 & 5.95 & 
680.04 & 6.12 & \bf{680.00} & 
0.01\\SCA3-4 & \bf{690.50} & 6.64 & 
690.50 & 6.50 & 690.50 & 0.00\\
SCA3-5 & \bf{\underline{659.90}} & 6.51 & 
664.05 & 6.79 & 671.10 & 
-1.67\\SCA3-6 & \bf{\underline{651.09}} & 6.24 & 
652.81 & 5.96 & 651.10 & 
-0.00\\SCA3-7 & \bf{\underline{659.17}} & 4.86 & 
664.40 & 4.96 & 666.10 & 
-1.04\\SCA3-8 & \bf{\underline{719.47}} & 5.26 & 
721.36 & 5.69 & 719.50 & 
-0.00\\SCA3-9 & \bf{681.00} & 5.05 & 
681.00 & 5.14 & 681.00 & 0.00\\
SCA8-0 & 973.03 & 7.14 & 
977.70 & 6.93 & \bf{961.60} & 
1.19\\SCA8-1 & \bf{\underline{1057.04}} & 5.57 & 
1059.29 & 5.42 & 1063.00 & 
-0.56\\SCA8-2 & 1045.64 & 4.98 & 
1050.33 & 4.93 & \bf{1040.60} & 
0.48\\SCA8-3 & 1010.50 & 7.26 & 
1014.89 & 7.04 & \bf{985.90} & 
2.50\\SCA8-4 & \bf{\underline{1067.66}} & 6.82 & 
1073.54 & 6.72 & 1071.00 & 
-0.31\\SCA8-5 & \bf{\underline{1034.74}} & 7.26 & 
1047.64 & 7.18 & 1054.30 & 
-1.86\\SCA8-6 & 973.30 & 6.89 & 
978.18 & 7.13 & \bf{972.50} & 
0.08\\SCA8-7 & 1067.20 & 7.56 & 
1069.23 & 7.22 & \bf{1059.70} & 
0.71\\SCA8-8 & \bf{\underline{1071.18}} & 6.97 & 
1071.18 & 6.96 & 1082.70 & 
-1.06\\SCA8-9 & \bf{\underline{1063.68}} & 5.45 & 
1066.03 & 5.28 & 1081.40 & 
-1.64\\CON3-0 & 619.09 & 7.35 & 
622.44 & 7.22 & \bf{616.50} & 
0.42\\CON3-1 & \bf{\underline{554.47}} & 7.03 & 
556.12 & 6.87 & 555.60 & 
-0.20\\CON3-2 & \bf{\underline{519.11}} & 6.30 & 
521.34 & 6.12 & 521.40 & 
-0.44\\CON3-3 & \bf{\underline{591.19}} & 7.17 & 
591.20 & 7.23 & 591.20 & 
-0.00\\CON3-4 & \bf{\underline{588.79}} & 5.62 & 
590.01 & 5.93 & 589.30 & 
-0.09\\CON3-5 & \bf{563.70} & 6.59 & 
565.36 & 6.69 & 563.70 & 0.00\\
CON3-6 & 500.80 & 7.50 & 
501.75 & 7.50 & \bf{499.20} & 
0.32\\CON3-7 & 578.22 & 6.03 & 
578.36 & 5.82 & \bf{577.50} & 
0.12\\CON3-8 & \bf{\underline{523.05}} & 6.21 & 
524.21 & 5.77 & 523.10 & 
-0.01\\CON3-9 & 582.79 & 5.70 & 
586.74 & 5.89 & \bf{578.20} & 
0.79\\CON8-0 & 871.82 & 6.47 & 
880.86 & 6.40 & \bf{858.90} & 
1.50\\CON8-1 & \bf{\underline{740.85}} & 5.73 & 
741.21 & 6.06 & 740.90 & 
-0.01\\CON8-2 & \bf{\underline{713.44}} & 8.63 & 
714.89 & 9.21 & 714.30 & 
-0.12\\CON8-3 & \bf{\underline{811.07}} & 6.70 & 
814.23 & 6.58 & 812.30 & 
-0.15\\CON8-4 & 776.37 & 5.90 & 
781.37 & 6.32 & \bf{770.10} & 
0.81\\CON8-5 & \bf{\underline{760.03}} & 5.98 & 
762.29 & 6.28 & 766.60 & 
-0.86\\CON8-6 & \bf{\underline{691.83}} & 7.40 & 
694.86 & 7.31 & 697.20 & 
-0.77\\CON8-7 & \bf{\underline{814.79}} & 5.72 & 
816.70 & 5.43 & 814.80 & 
-0.00\\CON8-8 & 782.86 & 7.09 & 
784.20 & 7.22 & \bf{771.30} & 
1.50\\CON8-9 & \bf{\underline{810.18}} & 7.92 & 
814.11 & 7.37 & 815.10 & 
-0.60\\[1ex]\hline
\end{tabular}
\label{table:nonlin}
\end{table} \clearpage
\begin{table}[ht]
\caption{Resultados de la ejecución de la metaheurística ACO, utilizando instancias de Dethloff con la configuración -n 9.0 -alpha 1.0 -beta 3.0 -q .8 -ro 0.015}
\centering
\small
\begin{tabular}{c c c c c c c}
\hline\hline
Instancia & Costo mínimo & Tiempo(seg.) & Costo promedio & Tiempo promedio(seg.) & Costo ACO & \%Gap \\ [0.5ex]
\hline
SCA3-0 & \bf{\underline{636.06}} & 5.75 & 
636.06 & 6.19 & 636.10 & 
-0.01\\SCA3-1 & \bf{\underline{697.84}} & 6.50 & 
697.84 & 6.79 & 700.10 & 
-0.32\\SCA3-2 & 659.34 & 5.78 & 
662.21 & 5.77 & \bf{659.30} & 
0.01\\SCA3-3 & 680.04 & 5.94 & 
680.04 & 6.15 & \bf{680.00} & 
0.01\\SCA3-4 & \bf{690.50} & 6.33 & 
690.50 & 6.43 & 690.50 & 0.00\\
SCA3-5 & \bf{\underline{662.75}} & 7.25 & 
664.60 & 7.02 & 671.10 & 
-1.24\\SCA3-6 & \bf{\underline{651.09}} & 6.20 & 
652.48 & 6.09 & 651.10 & 
-0.00\\SCA3-7 & 666.15 & 4.84 & 
666.15 & 5.07 & \bf{666.10} & 
0.01\\SCA3-8 & \bf{\underline{719.47}} & 6.04 & 
721.71 & 5.65 & 719.50 & 
-0.00\\SCA3-9 & \bf{681.00} & 4.84 & 
681.00 & 5.19 & 681.00 & 0.00\\
SCA8-0 & 975.84 & 7.02 & 
983.13 & 6.96 & \bf{961.60} & 
1.48\\SCA8-1 & \bf{\underline{1049.65}} & 5.28 & 
1059.73 & 5.29 & 1063.00 & 
-1.26\\SCA8-2 & 1049.22 & 4.86 & 
1052.91 & 4.74 & \bf{1040.60} & 
0.83\\SCA8-3 & 995.50 & 6.84 & 
1010.38 & 6.59 & \bf{985.90} & 
0.97\\SCA8-4 & 1081.05 & 7.16 & 
1081.83 & 6.72 & \bf{1071.00} & 
0.94\\SCA8-5 & \bf{\underline{1047.55}} & 7.32 & 
1052.26 & 7.19 & 1054.30 & 
-0.64\\SCA8-6 & 977.03 & 7.14 & 
979.18 & 7.33 & \bf{972.50} & 
0.47\\SCA8-7 & 1067.20 & 7.63 & 
1072.81 & 7.61 & \bf{1059.70} & 
0.71\\SCA8-8 & \bf{\underline{1071.18}} & 7.07 & 
1074.87 & 6.73 & 1082.70 & 
-1.06\\SCA8-9 & \bf{\underline{1067.42}} & 5.27 & 
1067.42 & 5.27 & 1081.40 & 
-1.29\\CON3-0 & 621.82 & 6.81 & 
624.24 & 6.96 & \bf{616.50} & 
0.86\\CON3-1 & \bf{\underline{554.47}} & 6.24 & 
555.81 & 6.51 & 555.60 & 
-0.20\\CON3-2 & \bf{\underline{521.38}} & 5.98 & 
521.50 & 5.83 & 521.40 & 
-0.00\\CON3-3 & \bf{\underline{591.19}} & 7.12 & 
591.50 & 6.95 & 591.20 & 
-0.00\\CON3-4 & \bf{\underline{588.79}} & 5.88 & 
589.05 & 5.97 & 589.30 & 
-0.09\\CON3-5 & 564.88 & 6.66 & 
565.83 & 6.16 & \bf{563.70} & 
0.21\\CON3-6 & 500.37 & 7.67 & 
502.41 & 7.57 & \bf{499.20} & 
0.23\\CON3-7 & 578.41 & 5.41 & 
578.41 & 5.47 & \bf{577.50} & 
0.16\\CON3-8 & 523.68 & 5.60 & 
525.02 & 5.33 & \bf{523.10} & 
0.11\\CON3-9 & 578.25 & 5.76 & 
586.48 & 5.83 & \bf{578.20} & 
0.01\\CON8-0 & 869.43 & 6.62 & 
872.86 & 6.55 & \bf{858.90} & 
1.23\\CON8-1 & \bf{\underline{740.85}} & 6.16 & 
745.38 & 6.36 & 740.90 & 
-0.01\\CON8-2 & \bf{\underline{713.44}} & 8.40 & 
714.44 & 8.64 & 714.30 & 
-0.12\\CON8-3 & \bf{\underline{811.23}} & 6.34 & 
814.51 & 6.53 & 812.30 & 
-0.13\\CON8-4 & 777.81 & 6.79 & 
783.08 & 6.64 & \bf{770.10} & 
1.00\\CON8-5 & \bf{\underline{760.73}} & 6.55 & 
762.72 & 6.43 & 766.60 & 
-0.77\\CON8-6 & \bf{\underline{693.19}} & 7.17 & 
695.42 & 7.24 & 697.20 & 
-0.58\\CON8-7 & \bf{\underline{814.50}} & 5.48 & 
814.82 & 5.55 & 814.80 & 
-0.04\\CON8-8 & 784.77 & 7.94 & 
790.69 & 7.35 & \bf{771.30} & 
1.75\\CON8-9 & \bf{\underline{810.18}} & 6.82 & 
813.92 & 6.94 & 815.10 & 
-0.60\\[1ex]\hline
\end{tabular}
\label{table:nonlin}
\end{table} \clearpage
\begin{table}[ht]
\caption{Resultados de la ejecución de la metaheurística ACO, utilizando instancias de Dethloff con la configuración -n 9.0 -alpha 1.0 -beta 3.0 -q .9 -ro 0.015}
\centering
\small
\begin{tabular}{c c c c c c c}
\hline\hline
Instancia & Costo mínimo & Tiempo(seg.) & Costo promedio & Tiempo promedio(seg.) & Costo ACO & \%Gap \\ [0.5ex]
\hline
SCA3-0 & \bf{\underline{636.06}} & 9.75 & 
637.25 & 7.07 & 636.10 & 
-0.01\\SCA3-1 & \bf{\underline{697.84}} & 6.72 & 
697.84 & 6.83 & 700.10 & 
-0.32\\SCA3-2 & 659.34 & 6.41 & 
662.97 & 6.29 & \bf{659.30} & 
0.01\\SCA3-3 & 680.04 & 6.66 & 
680.32 & 6.61 & \bf{680.00} & 
0.01\\SCA3-4 & \bf{690.50} & 6.44 & 
690.50 & 6.59 & 690.50 & 0.00\\
SCA3-5 & \bf{\underline{659.90}} & 6.84 & 
665.30 & 6.80 & 671.10 & 
-1.67\\SCA3-6 & \bf{\underline{651.09}} & 6.05 & 
652.70 & 6.46 & 651.10 & 
-0.00\\SCA3-7 & 666.15 & 5.12 & 
666.15 & 5.13 & \bf{666.10} & 
0.01\\SCA3-8 & \bf{\underline{719.47}} & 5.12 & 
724.81 & 5.21 & 719.50 & 
-0.00\\SCA3-9 & \bf{681.00} & 4.85 & 
681.00 & 4.81 & 681.00 & 0.00\\
SCA8-0 & 977.26 & 6.81 & 
987.04 & 6.88 & \bf{961.60} & 
1.63\\SCA8-1 & \bf{\underline{1053.09}} & 5.47 & 
1062.60 & 5.55 & 1063.00 & 
-0.93\\SCA8-2 & 1050.17 & 4.57 & 
1051.55 & 4.67 & \bf{1040.60} & 
0.92\\SCA8-3 & 1016.76 & 6.56 & 
1023.07 & 6.87 & \bf{985.90} & 
3.13\\SCA8-4 & 1081.05 & 6.38 & 
1088.18 & 6.71 & \bf{1071.00} & 
0.94\\SCA8-5 & \bf{\underline{1047.16}} & 7.55 & 
1052.86 & 7.26 & 1054.30 & 
-0.68\\SCA8-6 & \bf{\underline{972.48}} & 7.58 & 
977.11 & 7.51 & 972.50 & 
-0.00\\SCA8-7 & 1067.20 & 7.39 & 
1071.85 & 7.55 & \bf{1059.70} & 
0.71\\SCA8-8 & \bf{\underline{1071.18}} & 6.75 & 
1073.91 & 6.57 & 1082.70 & 
-1.06\\SCA8-9 & \bf{\underline{1067.42}} & 5.52 & 
1067.42 & 5.19 & 1081.40 & 
-1.29\\CON3-0 & 621.82 & 7.53 & 
623.93 & 7.20 & \bf{616.50} & 
0.86\\CON3-1 & \bf{\underline{554.47}} & 6.50 & 
558.29 & 6.59 & 555.60 & 
-0.20\\CON3-2 & \bf{\underline{521.38}} & 5.84 & 
523.89 & 6.16 & 521.40 & 
-0.00\\CON3-3 & \bf{\underline{591.19}} & 6.93 & 
591.81 & 6.94 & 591.20 & 
-0.00\\CON3-4 & \bf{\underline{588.79}} & 5.60 & 
589.58 & 5.75 & 589.30 & 
-0.09\\CON3-5 & 568.66 & 6.50 & 
569.05 & 6.29 & \bf{563.70} & 
0.88\\CON3-6 & 500.80 & 7.93 & 
503.06 & 7.70 & \bf{499.20} & 
0.32\\CON3-7 & 578.41 & 6.15 & 
578.95 & 5.71 & \bf{577.50} & 
0.16\\CON3-8 & 523.68 & 5.36 & 
526.43 & 5.32 & \bf{523.10} & 
0.11\\CON3-9 & 588.40 & 5.32 & 
588.46 & 5.95 & \bf{578.20} & 
1.76\\CON8-0 & 871.96 & 6.42 & 
876.32 & 6.54 & \bf{858.90} & 
1.52\\CON8-1 & \bf{\underline{740.85}} & 6.70 & 
741.94 & 6.31 & 740.90 & 
-0.01\\CON8-2 & \bf{\underline{713.44}} & 8.68 & 
714.85 & 8.90 & 714.30 & 
-0.12\\CON8-3 & 812.54 & 6.26 & 
815.87 & 6.46 & \bf{812.30} & 
0.03\\CON8-4 & 777.81 & 6.81 & 
787.57 & 6.85 & \bf{770.10} & 
1.00\\CON8-5 & \bf{\underline{760.91}} & 6.03 & 
764.10 & 6.06 & 766.60 & 
-0.74\\CON8-6 & \bf{\underline{695.66}} & 7.54 & 
696.22 & 7.45 & 697.20 & 
-0.22\\CON8-7 & 814.86 & 5.21 & 
815.03 & 5.24 & \bf{814.80} & 
0.01\\CON8-8 & 784.40 & 6.85 & 
791.14 & 7.27 & \bf{771.30} & 
1.70\\CON8-9 & 815.49 & 7.27 & 
815.80 & 7.26 & \bf{815.10} & 
0.05\\[1ex]\hline
\end{tabular}
\label{table:nonlin}
\end{table} \clearpage
\begin{table}[ht]
\caption{Resultados de la ejecución de la metaheurística ACO, utilizando instancias de Dethloff con la configuración -n 10.0 -alpha 1.0 -beta 3.0 -q 0.1 -ro 0.015}
\centering
\small
\begin{tabular}{c c c c c c c}
\hline\hline
Instancia & Costo mínimo & Tiempo(seg.) & Costo promedio & Tiempo promedio(seg.) & Costo ACO & \%Gap \\ [0.5ex]
\hline
SCA3-0 & \bf{\underline{636.06}} & 7.67 & 
636.06 & 7.45 & 636.10 & 
-0.01\\SCA3-1 & \bf{\underline{697.84}} & 7.94 & 
697.84 & 7.79 & 700.10 & 
-0.32\\SCA3-2 & 659.34 & 7.22 & 
659.34 & 6.85 & \bf{659.30} & 
0.01\\SCA3-3 & 680.04 & 7.22 & 
680.04 & 7.16 & \bf{680.00} & 
0.01\\SCA3-4 & \bf{690.50} & 7.67 & 
690.50 & 7.52 & 690.50 & 0.00\\
SCA3-5 & \bf{\underline{659.90}} & 7.36 & 
662.04 & 7.35 & 671.10 & 
-1.67\\SCA3-6 & \bf{\underline{651.09}} & 7.96 & 
652.01 & 7.40 & 651.10 & 
-0.00\\SCA3-7 & 666.15 & 6.81 & 
666.26 & 6.67 & \bf{666.10} & 
0.01\\SCA3-8 & \bf{\underline{719.47}} & 7.10 & 
719.47 & 7.12 & 719.50 & 
-0.00\\SCA3-9 & \bf{681.00} & 7.19 & 
681.00 & 6.57 & 681.00 & 0.00\\
SCA8-0 & \bf{\underline{961.50}} & 7.46 & 
976.69 & 7.82 & 961.60 & 
-0.01\\SCA8-1 & \bf{\underline{1049.65}} & 7.76 & 
1053.76 & 6.97 & 1063.00 & 
-1.26\\SCA8-2 & 1048.78 & 6.00 & 
1049.53 & 6.03 & \bf{1040.60} & 
0.79\\SCA8-3 & 995.50 & 7.62 & 
1000.39 & 7.26 & \bf{985.90} & 
0.97\\SCA8-4 & \bf{\underline{1065.49}} & 7.42 & 
1067.00 & 7.59 & 1071.00 & 
-0.51\\SCA8-5 & \bf{\underline{1034.74}} & 8.81 & 
1039.19 & 8.42 & 1054.30 & 
-1.86\\SCA8-6 & \bf{\underline{972.48}} & 8.20 & 
977.11 & 8.16 & 972.50 & 
-0.00\\SCA8-7 & 1066.65 & 8.04 & 
1067.06 & 8.19 & \bf{1059.70} & 
0.66\\SCA8-8 & \bf{\underline{1071.18}} & 7.80 & 
1071.18 & 8.07 & 1082.70 & 
-1.06\\SCA8-9 & \bf{\underline{1065.60}} & 6.62 & 
1066.51 & 6.83 & 1081.40 & 
-1.46\\CON3-0 & 617.59 & 7.96 & 
620.04 & 8.04 & \bf{616.50} & 
0.18\\CON3-1 & \bf{\underline{554.47}} & 7.91 & 
555.55 & 7.57 & 555.60 & 
-0.20\\CON3-2 & \bf{\underline{521.38}} & 7.88 & 
521.38 & 7.62 & 521.40 & 
-0.00\\CON3-3 & \bf{\underline{591.19}} & 7.59 & 
591.20 & 7.75 & 591.20 & 
-0.00\\CON3-4 & \bf{\underline{588.79}} & 6.80 & 
589.45 & 6.87 & 589.30 & 
-0.09\\CON3-5 & 564.88 & 7.54 & 
564.88 & 7.39 & \bf{563.70} & 
0.21\\CON3-6 & 500.80 & 8.82 & 
501.36 & 8.42 & \bf{499.20} & 
0.32\\CON3-7 & \bf{\underline{576.84}} & 6.49 & 
578.43 & 6.53 & 577.50 & 
-0.11\\CON3-8 & \bf{\underline{523.05}} & 6.67 & 
523.46 & 6.75 & 523.10 & 
-0.01\\CON3-9 & 578.25 & 7.18 & 
580.90 & 7.03 & \bf{578.20} & 
0.01\\CON8-0 & 864.56 & 7.80 & 
870.66 & 7.68 & \bf{858.90} & 
0.66\\CON8-1 & \bf{\underline{740.85}} & 7.99 & 
741.44 & 7.91 & 740.90 & 
-0.01\\CON8-2 & \bf{\underline{713.60}} & 9.49 & 
713.81 & 9.11 & 714.30 & 
-0.10\\CON8-3 & \bf{\underline{812.11}} & 7.69 & 
812.43 & 7.61 & 812.30 & 
-0.02\\CON8-4 & 776.34 & 7.04 & 
776.44 & 7.03 & \bf{770.10} & 
0.81\\CON8-5 & \bf{\underline{754.95}} & 8.37 & 
758.48 & 8.00 & 766.60 & 
-1.52\\CON8-6 & \bf{\underline{684.69}} & 8.58 & 
693.08 & 8.59 & 697.20 & 
-1.79\\CON8-7 & \bf{\underline{814.79}} & 6.58 & 
815.04 & 6.70 & 814.80 & 
-0.00\\CON8-8 & 779.43 & 8.38 & 
782.68 & 8.52 & \bf{771.30} & 
1.05\\CON8-9 & \bf{\underline{812.60}} & 8.20 & 
814.42 & 8.33 & 815.10 & 
-0.31\\[1ex]\hline
\end{tabular}
\label{table:nonlin}
\end{table} \clearpage
\begin{table}[ht]
\caption{Resultados de la ejecución de la metaheurística ACO, utilizando instancias de Dethloff con la configuración -n 10.0 -alpha 1.0 -beta 3.0 -q .2 -ro 0.015}
\centering
\small
\begin{tabular}{c c c c c c c}
\hline\hline
Instancia & Costo mínimo & Tiempo(seg.) & Costo promedio & Tiempo promedio(seg.) & Costo ACO & \%Gap \\ [0.5ex]
\hline
SCA3-0 & \bf{\underline{636.06}} & 7.51 & 
636.06 & 7.40 & 636.10 & 
-0.01\\SCA3-1 & \bf{\underline{697.84}} & 7.62 & 
697.84 & 7.68 & 700.10 & 
-0.32\\SCA3-2 & 659.34 & 6.89 & 
660.68 & 6.96 & \bf{659.30} & 
0.01\\SCA3-3 & 680.04 & 9.75 & 
680.04 & 7.39 & \bf{680.00} & 
0.01\\SCA3-4 & \bf{690.50} & 7.31 & 
690.50 & 7.44 & 690.50 & 0.00\\
SCA3-5 & \bf{\underline{659.90}} & 7.29 & 
662.04 & 7.33 & 671.10 & 
-1.67\\SCA3-6 & \bf{\underline{651.09}} & 7.38 & 
652.01 & 7.13 & 651.10 & 
-0.00\\SCA3-7 & 666.15 & 6.34 & 
666.15 & 6.46 & \bf{666.10} & 
0.01\\SCA3-8 & \bf{\underline{719.47}} & 7.38 & 
719.47 & 7.26 & 719.50 & 
-0.00\\SCA3-9 & \bf{681.00} & 5.93 & 
681.00 & 6.19 & 681.00 & 0.00\\
SCA8-0 & \bf{\underline{961.50}} & 7.93 & 
972.44 & 7.78 & 961.60 & 
-0.01\\SCA8-1 & \bf{\underline{1055.60}} & 9.33 & 
1060.55 & 7.20 & 1063.00 & 
-0.70\\SCA8-2 & 1047.63 & 5.64 & 
1049.68 & 5.91 & \bf{1040.60} & 
0.68\\SCA8-3 & 1008.60 & 7.24 & 
1010.75 & 7.41 & \bf{985.90} & 
2.30\\SCA8-4 & \bf{\underline{1065.49}} & 8.48 & 
1066.03 & 7.44 & 1071.00 & 
-0.51\\SCA8-5 & \bf{\underline{1035.32}} & 8.04 & 
1040.20 & 8.39 & 1054.30 & 
-1.80\\SCA8-6 & 977.03 & 7.61 & 
978.64 & 8.05 & \bf{972.50} & 
0.47\\SCA8-7 & 1067.20 & 8.22 & 
1071.03 & 8.02 & \bf{1059.70} & 
0.71\\SCA8-8 & \bf{\underline{1071.18}} & 7.87 & 
1074.87 & 8.01 & 1082.70 & 
-1.06\\SCA8-9 & \bf{\underline{1065.60}} & 6.66 & 
1066.97 & 6.65 & 1081.40 & 
-1.46\\CON3-0 & 616.52 & 7.63 & 
620.61 & 7.75 & \bf{616.50} & 
0.00\\CON3-1 & \bf{\underline{554.47}} & 7.18 & 
556.06 & 7.31 & 555.60 & 
-0.20\\CON3-2 & \bf{\underline{519.11}} & 7.00 & 
520.81 & 7.17 & 521.40 & 
-0.44\\CON3-3 & \bf{\underline{591.19}} & 7.74 & 
591.20 & 7.71 & 591.20 & 
-0.00\\CON3-4 & \bf{\underline{588.79}} & 6.68 & 
588.79 & 6.95 & 589.30 & 
-0.09\\CON3-5 & \bf{563.70} & 7.04 & 
565.54 & 6.98 & 563.70 & 0.00\\
CON3-6 & \bf{\underline{499.07}} & 8.39 & 
500.71 & 8.32 & 499.20 & 
-0.03\\CON3-7 & \bf{\underline{576.48}} & 6.68 & 
577.88 & 6.64 & 577.50 & 
-0.18\\CON3-8 & \bf{\underline{523.05}} & 6.61 & 
523.25 & 6.66 & 523.10 & 
-0.01\\CON3-9 & 578.25 & 6.59 & 
582.05 & 6.91 & \bf{578.20} & 
0.01\\CON8-0 & \bf{\underline{857.40}} & 7.60 & 
869.80 & 7.57 & 858.90 & 
-0.17\\CON8-1 & \bf{\underline{740.85}} & 7.36 & 
741.61 & 7.68 & 740.90 & 
-0.01\\CON8-2 & \bf{\underline{713.44}} & 9.24 & 
713.44 & 8.97 & 714.30 & 
-0.12\\CON8-3 & \bf{\underline{811.07}} & 7.64 & 
813.29 & 7.50 & 812.30 & 
-0.15\\CON8-4 & 772.32 & 7.00 & 
777.02 & 7.18 & \bf{770.10} & 
0.29\\CON8-5 & \bf{\underline{758.12}} & 7.44 & 
759.92 & 7.31 & 766.60 & 
-1.11\\CON8-6 & \bf{\underline{683.83}} & 8.27 & 
691.41 & 8.46 & 697.20 & 
-1.92\\CON8-7 & \bf{\underline{812.87}} & 6.26 & 
814.35 & 6.38 & 814.80 & 
-0.24\\CON8-8 & \bf{\underline{771.26}} & 8.48 & 
780.34 & 8.35 & 771.30 & 
-0.01\\CON8-9 & \bf{\underline{811.59}} & 8.31 & 
812.39 & 8.49 & 815.10 & 
-0.43\\[1ex]\hline
\end{tabular}
\label{table:nonlin}
\end{table} \clearpage
\begin{table}[ht]
\caption{Resultados de la ejecución de la metaheurística ACO, utilizando instancias de Dethloff con la configuración -n 10.0 -alpha 1.0 -beta 3.0 -q .3 -ro 0.015}
\centering
\small
\begin{tabular}{c c c c c c c}
\hline\hline
Instancia & Costo mínimo & Tiempo(seg.) & Costo promedio & Tiempo promedio(seg.) & Costo ACO & \%Gap \\ [0.5ex]
\hline
SCA3-0 & \bf{\underline{636.06}} & 7.47 & 
636.06 & 7.30 & 636.10 & 
-0.01\\SCA3-1 & \bf{\underline{697.84}} & 7.58 & 
697.84 & 7.58 & 700.10 & 
-0.32\\SCA3-2 & 659.34 & 6.40 & 
659.34 & 6.66 & \bf{659.30} & 
0.01\\SCA3-3 & 680.04 & 6.80 & 
680.04 & 6.83 & \bf{680.00} & 
0.01\\SCA3-4 & \bf{690.50} & 7.88 & 
690.50 & 7.64 & 690.50 & 0.00\\
SCA3-5 & \bf{\underline{662.75}} & 7.34 & 
663.47 & 7.35 & 671.10 & 
-1.24\\SCA3-6 & 652.47 & 6.79 & 
652.82 & 7.28 & \bf{651.10} & 
0.21\\SCA3-7 & 666.15 & 5.89 & 
666.15 & 6.46 & \bf{666.10} & 
0.01\\SCA3-8 & \bf{\underline{719.47}} & 6.95 & 
719.54 & 6.81 & 719.50 & 
-0.00\\SCA3-9 & \bf{681.00} & 6.66 & 
681.00 & 6.21 & 681.00 & 0.00\\
SCA8-0 & 968.79 & 7.70 & 
976.61 & 7.71 & \bf{961.60} & 
0.75\\SCA8-1 & \bf{\underline{1053.44}} & 6.09 & 
1059.97 & 6.28 & 1063.00 & 
-0.90\\SCA8-2 & 1050.17 & 5.56 & 
1051.03 & 5.64 & \bf{1040.60} & 
0.92\\SCA8-3 & 997.17 & 7.10 & 
1006.22 & 7.97 & \bf{985.90} & 
1.14\\SCA8-4 & \bf{\underline{1067.28}} & 7.99 & 
1069.29 & 7.59 & 1071.00 & 
-0.35\\SCA8-5 & \bf{\underline{1034.74}} & 7.80 & 
1042.35 & 8.00 & 1054.30 & 
-1.86\\SCA8-6 & \bf{\underline{972.48}} & 7.73 & 
975.21 & 7.91 & 972.50 & 
-0.00\\SCA8-7 & 1067.20 & 8.38 & 
1067.20 & 8.15 & \bf{1059.70} & 
0.71\\SCA8-8 & \bf{\underline{1071.18}} & 9.20 & 
1071.18 & 8.11 & 1082.70 & 
-1.06\\SCA8-9 & \bf{\underline{1065.60}} & 6.52 & 
1066.05 & 6.95 & 1081.40 & 
-1.46\\CON3-0 & 616.52 & 8.26 & 
619.47 & 8.06 & \bf{616.50} & 
0.00\\CON3-1 & \bf{\underline{554.47}} & 7.66 & 
555.48 & 7.43 & 555.60 & 
-0.20\\CON3-2 & \bf{\underline{519.11}} & 8.23 & 
520.25 & 7.60 & 521.40 & 
-0.44\\CON3-3 & \bf{591.20} & 7.65 & 
591.20 & 7.86 & 591.20 & 0.00\\
CON3-4 & \bf{\underline{588.79}} & 6.48 & 
588.79 & 6.61 & 589.30 & 
-0.09\\CON3-5 & \bf{563.70} & 7.73 & 
564.29 & 7.46 & 563.70 & 0.00\\
CON3-6 & 500.80 & 9.00 & 
501.59 & 8.44 & \bf{499.20} & 
0.32\\CON3-7 & 578.22 & 6.36 & 
578.32 & 6.35 & \bf{577.50} & 
0.12\\CON3-8 & \bf{\underline{523.05}} & 6.55 & 
523.84 & 6.62 & 523.10 & 
-0.01\\CON3-9 & 578.98 & 6.84 & 
584.29 & 6.83 & \bf{578.20} & 
0.13\\CON8-0 & 859.51 & 7.84 & 
869.38 & 7.43 & \bf{858.90} & 
0.07\\CON8-1 & \bf{\underline{740.85}} & 7.90 & 
740.87 & 7.59 & 740.90 & 
-0.01\\CON8-2 & \bf{\underline{713.05}} & 9.18 & 
713.34 & 9.13 & 714.30 & 
-0.17\\CON8-3 & \bf{\underline{811.07}} & 9.82 & 
812.91 & 7.84 & 812.30 & 
-0.15\\CON8-4 & 776.37 & 7.13 & 
782.98 & 7.04 & \bf{770.10} & 
0.81\\CON8-5 & \bf{\underline{758.12}} & 7.66 & 
759.72 & 7.51 & 766.60 & 
-1.11\\CON8-6 & \bf{\underline{689.23}} & 8.05 & 
692.96 & 8.11 & 697.20 & 
-1.14\\CON8-7 & \bf{\underline{814.79}} & 6.68 & 
814.91 & 6.37 & 814.80 & 
-0.00\\CON8-8 & 782.40 & 8.05 & 
785.64 & 8.64 & \bf{771.30} & 
1.44\\CON8-9 & \bf{\underline{810.18}} & 8.56 & 
811.29 & 8.43 & 815.10 & 
-0.60\\[1ex]\hline
\end{tabular}
\label{table:nonlin}
\end{table} \clearpage
\begin{table}[ht]
\caption{Resultados de la ejecución de la metaheurística ACO, utilizando instancias de Dethloff con la configuración -n 10.0 -alpha 1.0 -beta 3.0 -q .4 -ro 0.015}
\centering
\small
\begin{tabular}{c c c c c c c}
\hline\hline
Instancia & Costo mínimo & Tiempo(seg.) & Costo promedio & Tiempo promedio(seg.) & Costo ACO & \%Gap \\ [0.5ex]
\hline
SCA3-0 & \bf{\underline{636.06}} & 7.03 & 
636.06 & 6.98 & 636.10 & 
-0.01\\SCA3-1 & \bf{\underline{697.84}} & 7.38 & 
697.84 & 7.35 & 700.10 & 
-0.32\\SCA3-2 & 659.34 & 6.76 & 
661.00 & 6.79 & \bf{659.30} & 
0.01\\SCA3-3 & 680.04 & 6.47 & 
680.18 & 7.02 & \bf{680.00} & 
0.01\\SCA3-4 & \bf{690.50} & 7.36 & 
690.50 & 7.45 & 690.50 & 0.00\\
SCA3-5 & \bf{\underline{659.90}} & 7.25 & 
660.61 & 7.12 & 671.10 & 
-1.67\\SCA3-6 & \bf{\underline{651.09}} & 6.88 & 
652.01 & 6.94 & 651.10 & 
-0.00\\SCA3-7 & \bf{\underline{659.17}} & 6.28 & 
664.40 & 6.07 & 666.10 & 
-1.04\\SCA3-8 & \bf{\underline{719.47}} & 7.04 & 
719.47 & 6.95 & 719.50 & 
-0.00\\SCA3-9 & \bf{681.00} & 6.49 & 
681.00 & 6.51 & 681.00 & 0.00\\
SCA8-0 & 968.79 & 7.14 & 
976.88 & 7.62 & \bf{961.60} & 
0.75\\SCA8-1 & \bf{\underline{1052.71}} & 6.14 & 
1059.13 & 6.28 & 1063.00 & 
-0.97\\SCA8-2 & 1049.08 & 5.42 & 
1049.47 & 5.36 & \bf{1040.60} & 
0.81\\SCA8-3 & 995.50 & 7.38 & 
1002.96 & 7.48 & \bf{985.90} & 
0.97\\SCA8-4 & \bf{\underline{1065.49}} & 7.40 & 
1066.58 & 7.41 & 1071.00 & 
-0.51\\SCA8-5 & \bf{\underline{1037.06}} & 8.62 & 
1044.77 & 8.09 & 1054.30 & 
-1.64\\SCA8-6 & 977.03 & 7.47 & 
979.18 & 7.73 & \bf{972.50} & 
0.47\\SCA8-7 & 1067.03 & 8.37 & 
1068.86 & 8.15 & \bf{1059.70} & 
0.69\\SCA8-8 & \bf{\underline{1071.18}} & 8.29 & 
1071.18 & 8.03 & 1082.70 & 
-1.06\\SCA8-9 & \bf{\underline{1065.60}} & 6.30 & 
1066.51 & 6.19 & 1081.40 & 
-1.46\\CON3-0 & 617.59 & 8.07 & 
618.47 & 8.07 & \bf{616.50} & 
0.18\\CON3-1 & \bf{\underline{554.47}} & 8.73 & 
555.20 & 7.94 & 555.60 & 
-0.20\\CON3-2 & \bf{\underline{521.36}} & 7.14 & 
521.38 & 7.16 & 521.40 & 
-0.01\\CON3-3 & \bf{\underline{591.19}} & 7.48 & 
591.20 & 7.80 & 591.20 & 
-0.00\\CON3-4 & \bf{\underline{588.79}} & 7.05 & 
588.92 & 6.78 & 589.30 & 
-0.09\\CON3-5 & \bf{563.70} & 7.18 & 
565.72 & 7.26 & 563.70 & 0.00\\
CON3-6 & 500.80 & 8.60 & 
502.07 & 8.34 & \bf{499.20} & 
0.32\\CON3-7 & 578.22 & 6.93 & 
578.36 & 6.62 & \bf{577.50} & 
0.12\\CON3-8 & \bf{\underline{523.05}} & 6.62 & 
523.25 & 6.97 & 523.10 & 
-0.01\\CON3-9 & 578.25 & 6.27 & 
583.52 & 6.69 & \bf{578.20} & 
0.01\\CON8-0 & 869.43 & 7.42 & 
873.32 & 7.42 & \bf{858.90} & 
1.23\\CON8-1 & \bf{\underline{740.85}} & 7.85 & 
741.59 & 7.69 & 740.90 & 
-0.01\\CON8-2 & \bf{\underline{713.44}} & 9.50 & 
714.33 & 9.67 & 714.30 & 
-0.12\\CON8-3 & \bf{\underline{812.11}} & 7.60 & 
814.84 & 7.27 & 812.30 & 
-0.02\\CON8-4 & 776.37 & 6.80 & 
783.06 & 7.14 & \bf{770.10} & 
0.81\\CON8-5 & \bf{\underline{758.12}} & 7.44 & 
760.39 & 7.30 & 766.60 & 
-1.11\\CON8-6 & \bf{\underline{685.06}} & 8.12 & 
691.01 & 8.20 & 697.20 & 
-1.74\\CON8-7 & \bf{\underline{814.79}} & 6.22 & 
817.84 & 6.32 & 814.80 & 
-0.00\\CON8-8 & 783.02 & 8.75 & 
785.92 & 8.10 & \bf{771.30} & 
1.52\\CON8-9 & \bf{\underline{811.18}} & 8.40 & 
813.70 & 8.28 & 815.10 & 
-0.48\\[1ex]\hline
\end{tabular}
\label{table:nonlin}
\end{table} \clearpage
\begin{table}[ht]
\caption{Resultados de la ejecución de la metaheurística ACO, utilizando instancias de Dethloff con la configuración -n 10.0 -alpha 1.0 -beta 3.0 -q .5 -ro 0.015}
\centering
\small
\begin{tabular}{c c c c c c c}
\hline\hline
Instancia & Costo mínimo & Tiempo(seg.) & Costo promedio & Tiempo promedio(seg.) & Costo ACO & \%Gap \\ [0.5ex]
\hline
SCA3-0 & \bf{\underline{636.06}} & 6.78 & 
636.06 & 7.08 & 636.10 & 
-0.01\\SCA3-1 & \bf{\underline{697.84}} & 7.42 & 
697.84 & 7.50 & 700.10 & 
-0.32\\SCA3-2 & 659.34 & 6.38 & 
659.34 & 6.82 & \bf{659.30} & 
0.01\\SCA3-3 & 680.04 & 6.59 & 
680.18 & 7.02 & \bf{680.00} & 
0.01\\SCA3-4 & \bf{690.50} & 7.15 & 
690.50 & 7.37 & 690.50 & 0.00\\
SCA3-5 & \bf{\underline{661.07}} & 7.00 & 
663.05 & 7.28 & 671.10 & 
-1.49\\SCA3-6 & 652.94 & 7.06 & 
652.94 & 7.06 & \bf{651.10} & 
0.28\\SCA3-7 & \bf{\underline{659.17}} & 6.21 & 
664.40 & 6.00 & 666.10 & 
-1.04\\SCA3-8 & \bf{\underline{719.47}} & 6.88 & 
720.19 & 6.72 & 719.50 & 
-0.00\\SCA3-9 & \bf{681.00} & 5.91 & 
681.00 & 5.99 & 681.00 & 0.00\\
SCA8-0 & \bf{\underline{961.50}} & 7.88 & 
968.19 & 7.88 & 961.60 & 
-0.01\\SCA8-1 & \bf{\underline{1053.90}} & 6.12 & 
1060.30 & 6.13 & 1063.00 & 
-0.86\\SCA8-2 & 1046.29 & 5.75 & 
1049.35 & 6.00 & \bf{1040.60} & 
0.55\\SCA8-3 & 1004.25 & 8.17 & 
1010.01 & 7.77 & \bf{985.90} & 
1.86\\SCA8-4 & \bf{\underline{1065.49}} & 7.23 & 
1070.38 & 7.63 & 1071.00 & 
-0.51\\SCA8-5 & \bf{\underline{1034.74}} & 9.53 & 
1040.59 & 8.44 & 1054.30 & 
-1.86\\SCA8-6 & \bf{\underline{971.82}} & 8.17 & 
974.42 & 8.29 & 972.50 & 
-0.07\\SCA8-7 & 1067.20 & 7.90 & 
1069.13 & 8.10 & \bf{1059.70} & 
0.71\\SCA8-8 & \bf{\underline{1071.18}} & 7.66 & 
1071.18 & 7.62 & 1082.70 & 
-1.06\\SCA8-9 & \bf{\underline{1067.42}} & 6.04 & 
1067.42 & 6.30 & 1081.40 & 
-1.29\\CON3-0 & 616.52 & 7.74 & 
619.66 & 9.54 & \bf{616.50} & 
0.00\\CON3-1 & \bf{\underline{554.47}} & 7.18 & 
555.55 & 7.32 & 555.60 & 
-0.20\\CON3-2 & \bf{\underline{519.11}} & 7.06 & 
520.25 & 6.82 & 521.40 & 
-0.44\\CON3-3 & \bf{\underline{591.19}} & 8.07 & 
591.50 & 7.91 & 591.20 & 
-0.00\\CON3-4 & \bf{\underline{588.79}} & 6.83 & 
588.92 & 6.80 & 589.30 & 
-0.09\\CON3-5 & \bf{563.70} & 7.84 & 
564.59 & 7.30 & 563.70 & 0.00\\
CON3-6 & 500.37 & 8.54 & 
501.92 & 8.51 & \bf{499.20} & 
0.23\\CON3-7 & 578.22 & 6.54 & 
578.90 & 6.35 & \bf{577.50} & 
0.12\\CON3-8 & 523.14 & 6.34 & 
523.57 & 6.61 & \bf{523.10} & 
0.01\\CON3-9 & 583.32 & 6.65 & 
586.53 & 6.83 & \bf{578.20} & 
0.89\\CON8-0 & 866.22 & 7.15 & 
868.39 & 7.20 & \bf{858.90} & 
0.85\\CON8-1 & \bf{\underline{740.85}} & 7.43 & 
742.54 & 7.49 & 740.90 & 
-0.01\\CON8-2 & \bf{\underline{713.44}} & 9.46 & 
714.44 & 9.27 & 714.30 & 
-0.12\\CON8-3 & \bf{\underline{811.07}} & 7.26 & 
814.39 & 7.45 & 812.30 & 
-0.15\\CON8-4 & 776.72 & 7.40 & 
783.05 & 7.14 & \bf{770.10} & 
0.86\\CON8-5 & \bf{\underline{758.12}} & 7.04 & 
759.50 & 7.34 & 766.60 & 
-1.11\\CON8-6 & \bf{\underline{688.68}} & 7.90 & 
692.73 & 8.14 & 697.20 & 
-1.22\\CON8-7 & \bf{\underline{814.77}} & 6.23 & 
816.78 & 6.51 & 814.80 & 
-0.00\\CON8-8 & 777.45 & 8.06 & 
784.93 & 8.40 & \bf{771.30} & 
0.80\\CON8-9 & \bf{\underline{811.43}} & 8.18 & 
813.16 & 8.30 & 815.10 & 
-0.45\\[1ex]\hline
\end{tabular}
\label{table:nonlin}
\end{table} \clearpage
\begin{table}[ht]
\caption{Resultados de la ejecución de la metaheurística ACO, utilizando instancias de Dethloff con la configuración -n 10.0 -alpha 1.0 -beta 3.0 -q .6 -ro 0.015}
\centering
\small
\begin{tabular}{c c c c c c c}
\hline\hline
Instancia & Costo mínimo & Tiempo(seg.) & Costo promedio & Tiempo promedio(seg.) & Costo ACO & \%Gap \\ [0.5ex]
\hline
SCA3-0 & \bf{\underline{636.06}} & 7.55 & 
636.06 & 7.12 & 636.10 & 
-0.01\\SCA3-1 & \bf{\underline{697.84}} & 7.60 & 
697.84 & 7.51 & 700.10 & 
-0.32\\SCA3-2 & 659.34 & 9.45 & 
659.79 & 7.29 & \bf{659.30} & 
0.01\\SCA3-3 & 680.04 & 6.50 & 
680.04 & 6.70 & \bf{680.00} & 
0.01\\SCA3-4 & \bf{690.50} & 7.84 & 
690.50 & 7.29 & 690.50 & 0.00\\
SCA3-5 & \bf{\underline{659.90}} & 7.40 & 
662.04 & 7.09 & 671.10 & 
-1.67\\SCA3-6 & \bf{\underline{651.09}} & 6.82 & 
652.58 & 6.92 & 651.10 & 
-0.00\\SCA3-7 & 666.15 & 6.54 & 
666.15 & 6.25 & \bf{666.10} & 
0.01\\SCA3-8 & \bf{\underline{719.47}} & 6.62 & 
719.47 & 6.90 & 719.50 & 
-0.00\\SCA3-9 & \bf{681.00} & 6.19 & 
681.00 & 6.31 & 681.00 & 0.00\\
SCA8-0 & 968.79 & 7.52 & 
972.34 & 7.66 & \bf{961.60} & 
0.75\\SCA8-1 & \bf{\underline{1052.71}} & 6.16 & 
1059.90 & 6.16 & 1063.00 & 
-0.97\\SCA8-2 & 1046.29 & 4.98 & 
1049.51 & 5.42 & \bf{1040.60} & 
0.55\\SCA8-3 & 1002.38 & 7.69 & 
1013.17 & 7.50 & \bf{985.90} & 
1.67\\SCA8-4 & \bf{\underline{1065.49}} & 7.14 & 
1070.57 & 7.35 & 1071.00 & 
-0.51\\SCA8-5 & \bf{\underline{1034.74}} & 7.67 & 
1049.83 & 8.47 & 1054.30 & 
-1.86\\SCA8-6 & \bf{\underline{972.48}} & 8.09 & 
974.97 & 8.04 & 972.50 & 
-0.00\\SCA8-7 & 1067.20 & 7.99 & 
1069.72 & 8.15 & \bf{1059.70} & 
0.71\\SCA8-8 & \bf{\underline{1071.18}} & 7.47 & 
1073.91 & 7.64 & 1082.70 & 
-1.06\\SCA8-9 & \bf{\underline{1067.42}} & 6.02 & 
1067.42 & 6.18 & 1081.40 & 
-1.29\\CON3-0 & 617.59 & 7.52 & 
619.95 & 7.75 & \bf{616.50} & 
0.18\\CON3-1 & \bf{\underline{554.47}} & 8.41 & 
555.55 & 7.43 & 555.60 & 
-0.20\\CON3-2 & \bf{\underline{519.61}} & 6.36 & 
520.94 & 6.92 & 521.40 & 
-0.34\\CON3-3 & \bf{\underline{591.19}} & 8.04 & 
591.19 & 7.89 & 591.20 & 
-0.00\\CON3-4 & \bf{\underline{588.79}} & 6.68 & 
588.79 & 6.66 & 589.30 & 
-0.09\\CON3-5 & \bf{563.70} & 6.86 & 
566.48 & 7.01 & 563.70 & 0.00\\
CON3-6 & 502.16 & 8.38 & 
503.15 & 8.29 & \bf{499.20} & 
0.59\\CON3-7 & \bf{\underline{576.48}} & 6.38 & 
577.45 & 6.18 & 577.50 & 
-0.18\\CON3-8 & 524.30 & 6.37 & 
524.52 & 6.54 & \bf{523.10} & 
0.23\\CON3-9 & 586.31 & 6.51 & 
587.65 & 6.78 & \bf{578.20} & 
1.40\\CON8-0 & 859.51 & 7.23 & 
868.20 & 7.62 & \bf{858.90} & 
0.07\\CON8-1 & 740.93 & 7.72 & 
741.61 & 7.43 & \bf{740.90} & 
0.00\\CON8-2 & \bf{\underline{713.44}} & 9.41 & 
714.99 & 9.71 & 714.30 & 
-0.12\\CON8-3 & 812.75 & 7.13 & 
816.37 & 7.23 & \bf{812.30} & 
0.06\\CON8-4 & 776.37 & 7.45 & 
780.30 & 7.16 & \bf{770.10} & 
0.81\\CON8-5 & \bf{\underline{760.91}} & 6.87 & 
763.08 & 7.06 & 766.60 & 
-0.74\\CON8-6 & \bf{\underline{695.01}} & 7.78 & 
695.81 & 8.12 & 697.20 & 
-0.31\\CON8-7 & 814.86 & 5.93 & 
816.47 & 6.22 & \bf{814.80} & 
0.01\\CON8-8 & 783.63 & 8.38 & 
786.14 & 8.27 & \bf{771.30} & 
1.60\\CON8-9 & \bf{\underline{810.18}} & 8.14 & 
812.93 & 8.22 & 815.10 & 
-0.60\\[1ex]\hline
\end{tabular}
\label{table:nonlin}
\end{table} \clearpage
\begin{table}[ht]
\caption{Resultados de la ejecución de la metaheurística ACO, utilizando instancias de Dethloff con la configuración -n 10.0 -alpha 1.0 -beta 3.0 -q .7 -ro 0.015}
\centering
\small
\begin{tabular}{c c c c c c c}
\hline\hline
Instancia & Costo mínimo & Tiempo(seg.) & Costo promedio & Tiempo promedio(seg.) & Costo ACO & \%Gap \\ [0.5ex]
\hline
SCA3-0 & \bf{\underline{636.06}} & 6.68 & 
636.13 & 7.03 & 636.10 & 
-0.01\\SCA3-1 & \bf{\underline{697.84}} & 7.23 & 
697.84 & 7.66 & 700.10 & 
-0.32\\SCA3-2 & 659.34 & 9.87 & 
659.34 & 7.45 & \bf{659.30} & 
0.01\\SCA3-3 & 680.04 & 6.96 & 
680.04 & 7.09 & \bf{680.00} & 
0.01\\SCA3-4 & \bf{690.50} & 7.04 & 
690.50 & 7.09 & 690.50 & 0.00\\
SCA3-5 & \bf{\underline{659.90}} & 7.41 & 
662.76 & 7.43 & 671.10 & 
-1.67\\SCA3-6 & \bf{\underline{651.09}} & 7.31 & 
652.24 & 7.02 & 651.10 & 
-0.00\\SCA3-7 & 666.15 & 5.74 & 
666.15 & 5.85 & \bf{666.10} & 
0.01\\SCA3-8 & \bf{\underline{719.47}} & 6.37 & 
719.47 & 6.43 & 719.50 & 
-0.00\\SCA3-9 & \bf{681.00} & 5.85 & 
681.00 & 5.88 & 681.00 & 0.00\\
SCA8-0 & \bf{\underline{961.50}} & 8.11 & 
977.09 & 7.83 & 961.60 & 
-0.01\\SCA8-1 & \bf{\underline{1052.71}} & 6.08 & 
1059.00 & 6.22 & 1063.00 & 
-0.97\\SCA8-2 & 1046.29 & 6.70 & 
1048.54 & 5.74 & \bf{1040.60} & 
0.55\\SCA8-3 & 995.50 & 8.28 & 
1012.77 & 7.68 & \bf{985.90} & 
0.97\\SCA8-4 & \bf{\underline{1065.49}} & 7.60 & 
1071.13 & 7.56 & 1071.00 & 
-0.51\\SCA8-5 & \bf{\underline{1034.74}} & 7.35 & 
1048.78 & 7.77 & 1054.30 & 
-1.86\\SCA8-6 & \bf{\underline{972.48}} & 8.33 & 
978.39 & 8.24 & 972.50 & 
-0.00\\SCA8-7 & 1067.11 & 8.70 & 
1069.23 & 8.38 & \bf{1059.70} & 
0.70\\SCA8-8 & \bf{\underline{1071.18}} & 7.85 & 
1071.18 & 7.70 & 1082.70 & 
-1.06\\SCA8-9 & \bf{\underline{1067.42}} & 5.90 & 
1067.42 & 5.89 & 1081.40 & 
-1.29\\CON3-0 & 617.59 & 7.54 & 
619.26 & 7.64 & \bf{616.50} & 
0.18\\CON3-1 & 556.28 & 7.42 & 
556.95 & 7.14 & \bf{555.60} & 
0.12\\CON3-2 & \bf{\underline{519.11}} & 6.29 & 
521.50 & 6.54 & 521.40 & 
-0.44\\CON3-3 & \bf{\underline{591.19}} & 7.73 & 
591.20 & 7.83 & 591.20 & 
-0.00\\CON3-4 & \bf{\underline{588.79}} & 6.41 & 
588.79 & 6.64 & 589.30 & 
-0.09\\CON3-5 & \bf{563.70} & 6.98 & 
565.24 & 7.11 & 563.70 & 0.00\\
CON3-6 & 500.80 & 8.30 & 
501.64 & 8.62 & \bf{499.20} & 
0.32\\CON3-7 & \bf{\underline{576.48}} & 6.30 & 
577.71 & 6.20 & 577.50 & 
-0.18\\CON3-8 & \bf{\underline{523.05}} & 5.84 & 
523.70 & 6.23 & 523.10 & 
-0.01\\CON3-9 & 586.31 & 6.21 & 
587.92 & 6.48 & \bf{578.20} & 
1.40\\CON8-0 & 865.86 & 7.00 & 
876.25 & 7.19 & \bf{858.90} & 
0.81\\CON8-1 & \bf{\underline{740.85}} & 6.78 & 
741.57 & 7.00 & 740.90 & 
-0.01\\CON8-2 & \bf{\underline{713.44}} & 9.88 & 
714.19 & 9.37 & 714.30 & 
-0.12\\CON8-3 & \bf{\underline{811.07}} & 7.54 & 
813.59 & 7.22 & 812.30 & 
-0.15\\CON8-4 & 776.37 & 7.57 & 
781.71 & 7.18 & \bf{770.10} & 
0.81\\CON8-5 & \bf{\underline{758.12}} & 7.15 & 
760.14 & 6.74 & 766.60 & 
-1.11\\CON8-6 & \bf{\underline{695.20}} & 8.98 & 
696.46 & 8.33 & 697.20 & 
-0.29\\CON8-7 & 815.06 & 6.43 & 
816.80 & 6.38 & \bf{814.80} & 
0.03\\CON8-8 & 782.06 & 7.82 & 
785.11 & 8.60 & \bf{771.30} & 
1.40\\CON8-9 & \bf{\underline{812.60}} & 8.29 & 
814.18 & 7.97 & 815.10 & 
-0.31\\[1ex]\hline
\end{tabular}
\label{table:nonlin}
\end{table} \clearpage
\begin{table}[ht]
\caption{Resultados de la ejecución de la metaheurística ACO, utilizando instancias de Dethloff con la configuración -n 10.0 -alpha 1.0 -beta 3.0 -q .8 -ro 0.015}
\centering
\small
\begin{tabular}{c c c c c c c}
\hline\hline
Instancia & Costo mínimo & Tiempo(seg.) & Costo promedio & Tiempo promedio(seg.) & Costo ACO & \%Gap \\ [0.5ex]
\hline
SCA3-0 & \bf{\underline{636.06}} & 6.67 & 
636.13 & 6.61 & 636.10 & 
-0.01\\SCA3-1 & \bf{\underline{697.84}} & 7.00 & 
697.84 & 7.50 & 700.10 & 
-0.32\\SCA3-2 & 659.34 & 6.23 & 
661.76 & 6.71 & \bf{659.30} & 
0.01\\SCA3-3 & 680.04 & 7.07 & 
680.18 & 6.95 & \bf{680.00} & 
0.01\\SCA3-4 & \bf{690.50} & 7.39 & 
690.50 & 7.36 & 690.50 & 0.00\\
SCA3-5 & \bf{\underline{659.90}} & 7.68 & 
663.06 & 7.50 & 671.10 & 
-1.67\\SCA3-6 & 652.94 & 7.19 & 
652.94 & 6.87 & \bf{651.10} & 
0.28\\SCA3-7 & 666.15 & 5.68 & 
666.15 & 5.65 & \bf{666.10} & 
0.01\\SCA3-8 & \bf{\underline{719.47}} & 6.68 & 
721.36 & 6.25 & 719.50 & 
-0.00\\SCA3-9 & \bf{681.00} & 5.66 & 
681.86 & 5.57 & 681.00 & 0.00\\
SCA8-0 & \bf{\underline{961.50}} & 7.61 & 
971.98 & 7.50 & 961.60 & 
-0.01\\SCA8-1 & \bf{\underline{1059.89}} & 6.24 & 
1065.43 & 6.03 & 1063.00 & 
-0.29\\SCA8-2 & 1046.29 & 5.48 & 
1049.41 & 5.29 & \bf{1040.60} & 
0.55\\SCA8-3 & 1008.29 & 7.62 & 
1018.32 & 7.43 & \bf{985.90} & 
2.27\\SCA8-4 & \bf{\underline{1065.49}} & 7.00 & 
1069.38 & 7.53 & 1071.00 & 
-0.51\\SCA8-5 & \bf{\underline{1034.74}} & 7.97 & 
1043.00 & 8.02 & 1054.30 & 
-1.86\\SCA8-6 & \bf{\underline{972.48}} & 8.35 & 
977.11 & 8.26 & 972.50 & 
-0.00\\SCA8-7 & 1067.03 & 8.17 & 
1071.58 & 8.32 & \bf{1059.70} & 
0.69\\SCA8-8 & \bf{\underline{1071.18}} & 7.26 & 
1073.91 & 7.54 & 1082.70 & 
-1.06\\SCA8-9 & \bf{\underline{1067.42}} & 6.16 & 
1067.42 & 6.07 & 1081.40 & 
-1.29\\CON3-0 & 620.76 & 7.68 & 
623.21 & 7.75 & \bf{616.50} & 
0.69\\CON3-1 & \bf{\underline{554.47}} & 6.75 & 
556.00 & 7.36 & 555.60 & 
-0.20\\CON3-2 & \bf{\underline{519.11}} & 6.38 & 
520.37 & 6.34 & 521.40 & 
-0.44\\CON3-3 & \bf{\underline{591.19}} & 7.42 & 
591.50 & 7.73 & 591.20 & 
-0.00\\CON3-4 & \bf{\underline{588.79}} & 6.96 & 
589.45 & 6.71 & 589.30 & 
-0.09\\CON3-5 & 564.88 & 8.05 & 
567.89 & 7.48 & \bf{563.70} & 
0.21\\CON3-6 & 501.33 & 8.32 & 
503.16 & 8.37 & \bf{499.20} & 
0.43\\CON3-7 & \bf{\underline{576.48}} & 6.35 & 
577.88 & 6.08 & 577.50 & 
-0.18\\CON3-8 & 523.14 & 6.14 & 
523.87 & 7.09 & \bf{523.10} & 
0.01\\CON3-9 & 586.17 & 7.20 & 
587.30 & 6.45 & \bf{578.20} & 
1.38\\CON8-0 & 869.43 & 7.30 & 
874.30 & 7.18 & \bf{858.90} & 
1.23\\CON8-1 & \bf{\underline{740.85}} & 7.38 & 
744.32 & 7.38 & 740.90 & 
-0.01\\CON8-2 & \bf{\underline{713.44}} & 10.06 & 
714.04 & 10.04 & 714.30 & 
-0.12\\CON8-3 & 815.33 & 7.06 & 
816.68 & 7.18 & \bf{812.30} & 
0.37\\CON8-4 & 786.15 & 7.39 & 
789.02 & 7.00 & \bf{770.10} & 
2.08\\CON8-5 & \bf{\underline{758.12}} & 6.85 & 
762.55 & 6.97 & 766.60 & 
-1.11\\CON8-6 & \bf{\underline{683.83}} & 8.41 & 
691.86 & 8.56 & 697.20 & 
-1.92\\CON8-7 & 814.86 & 5.83 & 
817.90 & 6.14 & \bf{814.80} & 
0.01\\CON8-8 & 787.36 & 9.03 & 
793.53 & 8.35 & \bf{771.30} & 
2.08\\CON8-9 & \bf{\underline{810.18}} & 7.62 & 
813.21 & 7.87 & 815.10 & 
-0.60\\[1ex]\hline
\end{tabular}
\label{table:nonlin}
\end{table} \clearpage
\begin{table}[ht]
\caption{Resultados de la ejecución de la metaheurística ACO, utilizando instancias de Dethloff con la configuración -n 10.0 -alpha 1.0 -beta 3.0 -q .9 -ro 0.015}
\centering
\small
\begin{tabular}{c c c c c c c}
\hline\hline
Instancia & Costo mínimo & Tiempo(seg.) & Costo promedio & Tiempo promedio(seg.) & Costo ACO & \%Gap \\ [0.5ex]
\hline
SCA3-0 & \bf{\underline{636.06}} & 7.06 & 
636.20 & 8.03 & 636.10 & 
-0.01\\SCA3-1 & \bf{\underline{697.84}} & 7.83 & 
698.76 & 7.62 & 700.10 & 
-0.32\\SCA3-2 & 659.34 & 6.25 & 
662.97 & 6.49 & \bf{659.30} & 
0.01\\SCA3-3 & 680.04 & 6.92 & 
680.32 & 7.09 & \bf{680.00} & 
0.01\\SCA3-4 & \bf{690.50} & 7.54 & 
690.50 & 7.36 & 690.50 & 0.00\\
SCA3-5 & \bf{\underline{659.90}} & 7.42 & 
664.05 & 7.40 & 671.10 & 
-1.67\\SCA3-6 & 652.94 & 6.69 & 
653.80 & 6.33 & \bf{651.10} & 
0.28\\SCA3-7 & 666.15 & 5.42 & 
666.15 & 5.46 & \bf{666.10} & 
0.01\\SCA3-8 & \bf{\underline{719.47}} & 6.25 & 
722.99 & 6.17 & 719.50 & 
-0.00\\SCA3-9 & \bf{681.00} & 5.16 & 
681.00 & 5.58 & 681.00 & 0.00\\
SCA8-0 & 965.26 & 7.26 & 
982.65 & 7.53 & \bf{961.60} & 
0.38\\SCA8-1 & \bf{\underline{1052.71}} & 5.84 & 
1062.53 & 5.97 & 1063.00 & 
-0.97\\SCA8-2 & 1051.42 & 5.15 & 
1052.54 & 5.39 & \bf{1040.60} & 
1.04\\SCA8-3 & 1008.29 & 7.49 & 
1014.24 & 7.55 & \bf{985.90} & 
2.27\\SCA8-4 & \bf{\underline{1065.49}} & 7.23 & 
1067.51 & 7.60 & 1071.00 & 
-0.51\\SCA8-5 & \bf{\underline{1034.74}} & 8.11 & 
1050.33 & 8.43 & 1054.30 & 
-1.86\\SCA8-6 & 980.91 & 8.22 & 
980.91 & 8.36 & \bf{972.50} & 
0.86\\SCA8-7 & 1067.20 & 8.28 & 
1071.74 & 8.34 & \bf{1059.70} & 
0.71\\SCA8-8 & \bf{\underline{1071.18}} & 7.45 & 
1076.70 & 7.38 & 1082.70 & 
-1.06\\SCA8-9 & \bf{\underline{1067.42}} & 5.94 & 
1067.42 & 5.92 & 1081.40 & 
-1.29\\CON3-0 & 620.76 & 7.52 & 
622.88 & 7.83 & \bf{616.50} & 
0.69\\CON3-1 & \bf{\underline{554.47}} & 6.92 & 
556.65 & 7.04 & 555.60 & 
-0.20\\CON3-2 & \bf{\underline{519.11}} & 6.05 & 
522.61 & 6.08 & 521.40 & 
-0.44\\CON3-3 & \bf{591.20} & 7.64 & 
591.27 & 7.68 & 591.20 & 0.00\\
CON3-4 & \bf{\underline{588.79}} & 6.78 & 
588.79 & 6.49 & 589.30 & 
-0.09\\CON3-5 & 566.96 & 7.06 & 
567.86 & 6.83 & \bf{563.70} & 
0.58\\CON3-6 & 502.16 & 9.06 & 
502.81 & 8.72 & \bf{499.20} & 
0.59\\CON3-7 & 578.41 & 6.68 & 
578.95 & 6.29 & \bf{577.50} & 
0.16\\CON3-8 & 524.30 & 6.08 & 
526.84 & 5.93 & \bf{523.10} & 
0.23\\CON3-9 & 578.98 & 6.64 & 
586.09 & 6.62 & \bf{578.20} & 
0.13\\CON8-0 & 879.00 & 7.17 & 
883.14 & 7.36 & \bf{858.90} & 
2.34\\CON8-1 & \bf{\underline{740.85}} & 7.24 & 
744.45 & 6.96 & 740.90 & 
-0.01\\CON8-2 & \bf{\underline{713.44}} & 9.23 & 
713.88 & 9.84 & 714.30 & 
-0.12\\CON8-3 & \bf{\underline{811.07}} & 6.99 & 
818.01 & 7.00 & 812.30 & 
-0.15\\CON8-4 & 781.64 & 7.79 & 
788.76 & 7.58 & \bf{770.10} & 
1.50\\CON8-5 & \bf{\underline{758.84}} & 7.14 & 
763.24 & 8.17 & 766.60 & 
-1.01\\CON8-6 & \bf{\underline{696.08}} & 8.38 & 
696.84 & 8.48 & 697.20 & 
-0.16\\CON8-7 & 814.86 & 6.28 & 
816.52 & 6.20 & \bf{814.80} & 
0.01\\CON8-8 & 784.28 & 8.63 & 
787.99 & 8.17 & \bf{771.30} & 
1.68\\CON8-9 & \bf{\underline{811.43}} & 7.32 & 
814.97 & 7.94 & 815.10 & 
-0.45\\[1ex]\hline
\end{tabular}
\label{table:nonlin}
\end{table} \clearpage
\begin{table}[ht]
\caption{Resultados de la ejecución de la metaheurística ACO, utilizando instancias de Dethloff con la configuración -n 20 -alpha 1.0 -beta 3.0 -q 0.8 -ro 0.015}
\centering
\small
\begin{tabular}{c c c c c c c}
\hline\hline
Instancia & Costo mínimo & Tiempo(seg.) & Costo promedio & Tiempo promedio(seg.) & Costo ACO & \%Gap \\ [0.5ex]
\hline
SCA3-0 & \bf{\underline{636.06}} & 12.22 & 
636.06 & 13.19 & 636.10 & 
-0.01\\SCA3-1 & \bf{\underline{697.84}} & 12.97 & 
697.84 & 14.25 & 700.10 & 
-0.32\\SCA3-2 & 659.34 & 13.81 & 
659.79 & 12.97 & \bf{659.30} & 
0.01\\SCA3-3 & 680.04 & 13.25 & 
680.04 & 13.35 & \bf{680.00} & 
0.01\\SCA3-4 & \bf{690.50} & 13.72 & 
690.50 & 14.00 & 690.50 & 0.00\\
SCA3-5 & \bf{\underline{659.90}} & 13.30 & 
663.48 & 14.81 & 671.10 & 
-1.67\\SCA3-6 & \bf{\underline{651.09}} & 13.54 & 
653.20 & 13.28 & 651.10 & 
-0.00\\SCA3-7 & \bf{\underline{664.88}} & 12.30 & 
665.83 & 11.48 & 666.10 & 
-0.18\\SCA3-8 & \bf{\underline{719.47}} & 12.70 & 
719.47 & 13.02 & 719.50 & 
-0.00\\SCA3-9 & \bf{681.00} & 11.01 & 
681.00 & 10.68 & 681.00 & 0.00\\
SCA8-0 & 973.03 & 14.95 & 
979.38 & 14.51 & \bf{961.60} & 
1.19\\SCA8-1 & \bf{\underline{1052.71}} & 10.75 & 
1057.70 & 11.41 & 1063.00 & 
-0.97\\SCA8-2 & 1044.23 & 10.95 & 
1049.57 & 10.42 & \bf{1040.60} & 
0.35\\SCA8-3 & 995.50 & 16.02 & 
1011.12 & 14.94 & \bf{985.90} & 
0.97\\SCA8-4 & \bf{\underline{1067.66}} & 13.95 & 
1071.01 & 14.33 & 1071.00 & 
-0.31\\SCA8-5 & \bf{\underline{1039.12}} & 15.80 & 
1049.47 & 16.14 & 1054.30 & 
-1.44\\SCA8-6 & \bf{\underline{972.48}} & 15.99 & 
976.95 & 15.86 & 972.50 & 
-0.00\\SCA8-7 & 1067.11 & 16.20 & 
1069.23 & 16.11 & \bf{1059.70} & 
0.70\\SCA8-8 & \bf{\underline{1071.18}} & 13.65 & 
1071.18 & 14.05 & 1082.70 & 
-1.06\\SCA8-9 & \bf{\underline{1067.42}} & 11.71 & 
1067.42 & 11.67 & 1081.40 & 
-1.29\\CON3-0 & 617.59 & 14.82 & 
620.57 & 14.76 & \bf{616.50} & 
0.18\\CON3-1 & \bf{\underline{554.47}} & 14.11 & 
555.20 & 13.38 & 555.60 & 
-0.20\\CON3-2 & \bf{\underline{519.61}} & 13.47 & 
520.94 & 12.60 & 521.40 & 
-0.34\\CON3-3 & \bf{\underline{591.19}} & 15.23 & 
591.20 & 15.20 & 591.20 & 
-0.00\\CON3-4 & \bf{\underline{588.79}} & 13.01 & 
588.92 & 13.07 & 589.30 & 
-0.09\\CON3-5 & 564.88 & 13.06 & 
566.10 & 13.14 & \bf{563.70} & 
0.21\\CON3-6 & 502.16 & 16.94 & 
502.45 & 16.07 & \bf{499.20} & 
0.59\\CON3-7 & 577.54 & 11.84 & 
578.10 & 12.09 & \bf{577.50} & 
0.01\\CON3-8 & 523.68 & 11.28 & 
524.36 & 11.20 & \bf{523.10} & 
0.11\\CON3-9 & 586.31 & 12.48 & 
587.94 & 12.10 & \bf{578.20} & 
1.40\\CON8-0 & 866.22 & 15.12 & 
873.68 & 14.13 & \bf{858.90} & 
0.85\\CON8-1 & \bf{\underline{740.85}} & 13.79 & 
741.21 & 13.45 & 740.90 & 
-0.01\\CON8-2 & \bf{\underline{713.44}} & 18.15 & 
713.88 & 18.58 & 714.30 & 
-0.12\\CON8-3 & 817.23 & 14.72 & 
817.46 & 14.62 & \bf{812.30} & 
0.61\\CON8-4 & 772.25 & 14.50 & 
780.63 & 14.01 & \bf{770.10} & 
0.28\\CON8-5 & \bf{\underline{760.03}} & 13.25 & 
762.00 & 13.16 & 766.60 & 
-0.86\\CON8-6 & \bf{\underline{690.00}} & 16.56 & 
693.21 & 16.02 & 697.20 & 
-1.03\\CON8-7 & \bf{\underline{814.79}} & 11.28 & 
816.41 & 11.65 & 814.80 & 
-0.00\\CON8-8 & 788.80 & 16.99 & 
790.33 & 16.08 & \bf{771.30} & 
2.27\\CON8-9 & \bf{\underline{811.75}} & 14.98 & 
814.80 & 14.38 & 815.10 & 
-0.41\\[1ex]\hline
\end{tabular}
\label{table:nonlin}
\end{table} \clearpage
\begin{table}[ht]
\caption{Resultados de la ejecución de la metaheurística ACO, utilizando instancias de Dethloff con la configuración -n 20 -alpha 1.0 -beta 3.0 -q 0.8 -ro 0.05}
\centering
\small
\begin{tabular}{c c c c c c c}
\hline\hline
Instancia & Costo mínimo & Tiempo(seg.) & Costo promedio & Tiempo promedio(seg.) & Costo ACO & \%Gap \\ [0.5ex]
\hline
SCA3-0 & \bf{\underline{636.06}} & 12.24 & 
636.06 & 13.11 & 636.10 & 
-0.01\\SCA3-1 & \bf{\underline{697.84}} & 13.96 & 
697.84 & 14.57 & 700.10 & 
-0.32\\SCA3-2 & 659.34 & 12.88 & 
659.34 & 12.91 & \bf{659.30} & 
0.01\\SCA3-3 & 680.04 & 13.08 & 
680.18 & 13.15 & \bf{680.00} & 
0.01\\SCA3-4 & \bf{690.50} & 14.64 & 
690.50 & 13.94 & 690.50 & 0.00\\
SCA3-5 & \bf{\underline{659.90}} & 14.36 & 
664.21 & 14.54 & 671.10 & 
-1.67\\SCA3-6 & \bf{\underline{651.09}} & 14.29 & 
652.70 & 13.39 & 651.10 & 
-0.00\\SCA3-7 & 666.15 & 10.91 & 
666.15 & 10.80 & \bf{666.10} & 
0.01\\SCA3-8 & \bf{\underline{719.47}} & 11.23 & 
719.62 & 11.33 & 719.50 & 
-0.00\\SCA3-9 & \bf{681.00} & 11.35 & 
681.00 & 11.27 & 681.00 & 0.00\\
SCA8-0 & \bf{\underline{961.50}} & 14.02 & 
971.09 & 14.59 & 961.60 & 
-0.01\\SCA8-1 & \bf{\underline{1052.71}} & 11.74 & 
1060.27 & 11.37 & 1063.00 & 
-0.97\\SCA8-2 & 1046.29 & 9.70 & 
1049.56 & 10.16 & \bf{1040.60} & 
0.55\\SCA8-3 & 1015.31 & 13.82 & 
1016.00 & 14.65 & \bf{985.90} & 
2.98\\SCA8-4 & \bf{\underline{1069.45}} & 14.62 & 
1075.75 & 14.34 & 1071.00 & 
-0.14\\SCA8-5 & 1054.62 & 15.37 & 
1055.19 & 15.54 & \bf{1054.30} & 
0.03\\SCA8-6 & \bf{\underline{972.48}} & 16.20 & 
977.44 & 15.74 & 972.50 & 
-0.00\\SCA8-7 & 1067.20 & 15.77 & 
1068.87 & 15.72 & \bf{1059.70} & 
0.71\\SCA8-8 & \bf{\underline{1071.18}} & 14.69 & 
1074.05 & 14.03 & 1082.70 & 
-1.06\\SCA8-9 & \bf{\underline{1067.42}} & 10.62 & 
1067.42 & 11.30 & 1081.40 & 
-1.29\\CON3-0 & 616.52 & 14.24 & 
619.84 & 14.27 & \bf{616.50} & 
0.00\\CON3-1 & \bf{\underline{554.47}} & 13.94 & 
555.88 & 13.49 & 555.60 & 
-0.20\\CON3-2 & \bf{\underline{521.38}} & 12.15 & 
521.94 & 13.13 & 521.40 & 
-0.00\\CON3-3 & \bf{\underline{591.19}} & 15.50 & 
591.20 & 15.06 & 591.20 & 
-0.00\\CON3-4 & \bf{\underline{588.79}} & 13.44 & 
589.74 & 13.25 & 589.30 & 
-0.09\\CON3-5 & 564.88 & 13.96 & 
564.89 & 13.50 & \bf{563.70} & 
0.21\\CON3-6 & 500.80 & 16.27 & 
501.27 & 16.68 & \bf{499.20} & 
0.32\\CON3-7 & \bf{\underline{576.48}} & 12.65 & 
577.93 & 12.57 & 577.50 & 
-0.18\\CON3-8 & 523.68 & 12.62 & 
524.14 & 13.07 & \bf{523.10} & 
0.11\\CON3-9 & 578.98 & 13.32 & 
586.61 & 12.66 & \bf{578.20} & 
0.13\\CON8-0 & 866.22 & 13.46 & 
870.91 & 13.32 & \bf{858.90} & 
0.85\\CON8-1 & \bf{\underline{740.85}} & 12.80 & 
741.61 & 12.79 & 740.90 & 
-0.01\\CON8-2 & \bf{\underline{712.89}} & 18.43 & 
713.42 & 18.48 & 714.30 & 
-0.20\\CON8-3 & \bf{\underline{811.23}} & 13.30 & 
813.84 & 13.11 & 812.30 & 
-0.13\\CON8-4 & 775.73 & 14.77 & 
784.89 & 14.22 & \bf{770.10} & 
0.73\\CON8-5 & \bf{\underline{760.91}} & 12.85 & 
763.53 & 13.29 & 766.60 & 
-0.74\\CON8-6 & \bf{\underline{695.66}} & 15.83 & 
696.50 & 15.65 & 697.20 & 
-0.22\\CON8-7 & \bf{\underline{814.79}} & 11.58 & 
816.45 & 11.87 & 814.80 & 
-0.00\\CON8-8 & 784.56 & 16.00 & 
788.80 & 15.45 & \bf{771.30} & 
1.72\\CON8-9 & \bf{\underline{812.60}} & 16.15 & 
814.20 & 15.18 & 815.10 & 
-0.31\\[1ex]\hline
\end{tabular}
\label{table:nonlin}
\end{table} \clearpage
\begin{table}[ht]
\caption{Resultados de la ejecución de la metaheurística ACO, utilizando instancias de Dethloff con la configuración -n 20 -alpha 1.0 -beta 3.0 -q 0.8 -ro 0.1}
\centering
\small
\begin{tabular}{c c c c c c c}
\hline\hline
Instancia & Costo mínimo & Tiempo(seg.) & Costo promedio & Tiempo promedio(seg.) & Costo ACO & \%Gap \\ [0.5ex]
\hline
SCA3-0 & \bf{\underline{636.06}} & 13.72 & 
636.06 & 13.28 & 636.10 & 
-0.01\\SCA3-1 & \bf{\underline{697.84}} & 14.61 & 
697.84 & 14.59 & 700.10 & 
-0.32\\SCA3-2 & 659.34 & 13.45 & 
659.34 & 12.95 & \bf{659.30} & 
0.01\\SCA3-3 & 680.04 & 13.35 & 
680.04 & 13.49 & \bf{680.00} & 
0.01\\SCA3-4 & \bf{690.50} & 14.86 & 
690.50 & 14.01 & 690.50 & 0.00\\
SCA3-5 & \bf{\underline{662.75}} & 14.14 & 
664.77 & 14.55 & 671.10 & 
-1.24\\SCA3-6 & 652.94 & 14.54 & 
653.47 & 13.53 & \bf{651.10} & 
0.28\\SCA3-7 & 666.15 & 10.18 & 
666.15 & 10.77 & \bf{666.10} & 
0.01\\SCA3-8 & \bf{\underline{719.47}} & 11.82 & 
719.54 & 12.43 & 719.50 & 
-0.00\\SCA3-9 & \bf{681.00} & 10.40 & 
681.00 & 10.70 & 681.00 & 0.00\\
SCA8-0 & 975.84 & 15.31 & 
983.94 & 15.79 & \bf{961.60} & 
1.48\\SCA8-1 & \bf{\underline{1053.44}} & 10.97 & 
1058.04 & 11.51 & 1063.00 & 
-0.90\\SCA8-2 & 1050.17 & 10.82 & 
1050.53 & 10.43 & \bf{1040.60} & 
0.92\\SCA8-3 & 1008.29 & 14.50 & 
1013.80 & 14.28 & \bf{985.90} & 
2.27\\SCA8-4 & \bf{\underline{1065.49}} & 14.98 & 
1072.13 & 14.48 & 1071.00 & 
-0.51\\SCA8-5 & \bf{\underline{1034.74}} & 14.80 & 
1048.16 & 15.73 & 1054.30 & 
-1.86\\SCA8-6 & \bf{\underline{972.48}} & 15.01 & 
974.71 & 15.54 & 972.50 & 
-0.00\\SCA8-7 & 1067.20 & 15.64 & 
1071.31 & 15.30 & \bf{1059.70} & 
0.71\\SCA8-8 & \bf{\underline{1071.18}} & 14.54 & 
1074.05 & 14.21 & 1082.70 & 
-1.06\\SCA8-9 & \bf{\underline{1063.68}} & 11.17 & 
1066.49 & 11.54 & 1081.40 & 
-1.64\\CON3-0 & 617.59 & 15.32 & 
620.34 & 15.48 & \bf{616.50} & 
0.18\\CON3-1 & \bf{\underline{554.47}} & 13.71 & 
555.20 & 13.65 & 555.60 & 
-0.20\\CON3-2 & \bf{\underline{519.11}} & 12.77 & 
522.84 & 12.56 & 521.40 & 
-0.44\\CON3-3 & \bf{\underline{591.19}} & 15.34 & 
591.19 & 15.19 & 591.20 & 
-0.00\\CON3-4 & \bf{\underline{588.79}} & 11.46 & 
588.79 & 12.24 & 589.30 & 
-0.09\\CON3-5 & 564.88 & 13.21 & 
566.79 & 13.25 & \bf{563.70} & 
0.21\\CON3-6 & 500.80 & 16.31 & 
501.92 & 16.71 & \bf{499.20} & 
0.32\\CON3-7 & 578.41 & 12.38 & 
578.41 & 12.07 & \bf{577.50} & 
0.16\\CON3-8 & \bf{\underline{523.05}} & 10.69 & 
523.83 & 11.23 & 523.10 & 
-0.01\\CON3-9 & 578.98 & 12.04 & 
586.06 & 12.69 & \bf{578.20} & 
0.13\\CON8-0 & 872.05 & 14.68 & 
874.33 & 14.20 & \bf{858.90} & 
1.53\\CON8-1 & \bf{\underline{740.85}} & 14.35 & 
743.60 & 13.27 & 740.90 & 
-0.01\\CON8-2 & \bf{\underline{713.44}} & 18.82 & 
713.55 & 18.41 & 714.30 & 
-0.12\\CON8-3 & 812.54 & 12.89 & 
815.96 & 13.34 & \bf{812.30} & 
0.03\\CON8-4 & 777.59 & 14.16 & 
785.82 & 13.81 & \bf{770.10} & 
0.97\\CON8-5 & \bf{\underline{760.03}} & 14.42 & 
763.38 & 13.20 & 766.60 & 
-0.86\\CON8-6 & \bf{\underline{694.50}} & 16.34 & 
695.37 & 15.90 & 697.20 & 
-0.39\\CON8-7 & \bf{\underline{814.77}} & 11.24 & 
814.80 & 11.44 & 814.80 & 
-0.00\\CON8-8 & 782.61 & 16.21 & 
786.90 & 16.07 & \bf{771.30} & 
1.47\\CON8-9 & \bf{\underline{811.75}} & 15.54 & 
812.67 & 15.14 & 815.10 & 
-0.41\\[1ex]\hline
\end{tabular}
\label{table:nonlin}
\end{table} \clearpage
\begin{table}[ht]
\caption{Resultados de la ejecución de la metaheurística ACO, utilizando instancias de Dethloff con la configuración -n 20 -alpha 1.0 -beta 3.0 -q 0.8 -ro 0.2}
\centering
\small
\begin{tabular}{c c c c c c c}
\hline\hline
Instancia & Costo mínimo & Tiempo(seg.) & Costo promedio & Tiempo promedio(seg.) & Costo ACO & \%Gap \\ [0.5ex]
\hline
SCA3-0 & \bf{\underline{636.06}} & 12.76 & 
636.06 & 13.01 & 636.10 & 
-0.01\\SCA3-1 & \bf{\underline{697.84}} & 14.07 & 
697.84 & 14.31 & 700.10 & 
-0.32\\SCA3-2 & 659.34 & 12.86 & 
661.76 & 12.85 & \bf{659.30} & 
0.01\\SCA3-3 & 680.04 & 14.10 & 
680.04 & 13.65 & \bf{680.00} & 
0.01\\SCA3-4 & \bf{690.50} & 13.17 & 
690.50 & 13.80 & 690.50 & 0.00\\
SCA3-5 & \bf{\underline{662.75}} & 13.40 & 
663.47 & 13.93 & 671.10 & 
-1.24\\SCA3-6 & \bf{\underline{651.09}} & 13.74 & 
652.70 & 13.05 & 651.10 & 
-0.00\\SCA3-7 & 666.15 & 11.90 & 
666.15 & 11.20 & \bf{666.10} & 
0.01\\SCA3-8 & \bf{\underline{719.47}} & 12.70 & 
719.54 & 12.86 & 719.50 & 
-0.00\\SCA3-9 & \bf{681.00} & 11.36 & 
681.00 & 11.25 & 681.00 & 0.00\\
SCA8-0 & 970.64 & 14.35 & 
987.52 & 14.74 & \bf{961.60} & 
0.94\\SCA8-1 & \bf{\underline{1058.78}} & 11.88 & 
1064.74 & 11.84 & 1063.00 & 
-0.40\\SCA8-2 & 1049.22 & 10.16 & 
1050.63 & 10.62 & \bf{1040.60} & 
0.83\\SCA8-3 & 1014.86 & 14.66 & 
1015.72 & 14.30 & \bf{985.90} & 
2.94\\SCA8-4 & \bf{\underline{1065.49}} & 15.21 & 
1068.24 & 14.59 & 1071.00 & 
-0.51\\SCA8-5 & \bf{\underline{1034.74}} & 15.38 & 
1049.48 & 15.80 & 1054.30 & 
-1.86\\SCA8-6 & \bf{\underline{972.48}} & 15.19 & 
977.64 & 15.77 & 972.50 & 
-0.00\\SCA8-7 & 1067.20 & 16.39 & 
1069.63 & 16.14 & \bf{1059.70} & 
0.71\\SCA8-8 & \bf{\underline{1071.18}} & 13.45 & 
1073.91 & 14.55 & 1082.70 & 
-1.06\\SCA8-9 & \bf{\underline{1067.42}} & 11.93 & 
1067.42 & 11.69 & 1081.40 & 
-1.29\\CON3-0 & 617.59 & 14.57 & 
623.12 & 15.28 & \bf{616.50} & 
0.18\\CON3-1 & \bf{\underline{554.47}} & 12.97 & 
555.79 & 14.31 & 555.60 & 
-0.20\\CON3-2 & \bf{\underline{519.11}} & 13.04 & 
521.96 & 12.36 & 521.40 & 
-0.44\\CON3-3 & \bf{\underline{591.19}} & 14.61 & 
591.20 & 14.35 & 591.20 & 
-0.00\\CON3-4 & \bf{\underline{588.79}} & 12.62 & 
588.92 & 12.60 & 589.30 & 
-0.09\\CON3-5 & 564.88 & 13.50 & 
564.88 & 13.20 & \bf{563.70} & 
0.21\\CON3-6 & 500.80 & 15.59 & 
501.14 & 15.84 & \bf{499.20} & 
0.32\\CON3-7 & \bf{\underline{576.84}} & 12.61 & 
578.02 & 13.01 & 577.50 & 
-0.11\\CON3-8 & 523.68 & 10.53 & 
524.79 & 11.47 & \bf{523.10} & 
0.11\\CON3-9 & 588.40 & 12.24 & 
588.44 & 12.21 & \bf{578.20} & 
1.76\\CON8-0 & 865.86 & 12.97 & 
873.99 & 13.40 & \bf{858.90} & 
0.81\\CON8-1 & \bf{\underline{740.85}} & 13.92 & 
741.76 & 14.15 & 740.90 & 
-0.01\\CON8-2 & \bf{\underline{713.68}} & 18.20 & 
714.93 & 18.86 & 714.30 & 
-0.09\\CON8-3 & 813.40 & 13.90 & 
816.09 & 13.50 & \bf{812.30} & 
0.14\\CON8-4 & 776.37 & 13.12 & 
783.59 & 13.96 & \bf{770.10} & 
0.81\\CON8-5 & \bf{\underline{760.03}} & 13.57 & 
762.46 & 13.63 & 766.60 & 
-0.86\\CON8-6 & \bf{\underline{680.47}} & 15.67 & 
686.26 & 15.86 & 697.20 & 
-2.40\\CON8-7 & 814.86 & 12.52 & 
815.10 & 12.49 & \bf{814.80} & 
0.01\\CON8-8 & 784.34 & 16.09 & 
787.23 & 15.74 & \bf{771.30} & 
1.69\\CON8-9 & \bf{\underline{812.60}} & 14.64 & 
814.70 & 14.61 & 815.10 & 
-0.31\\[1ex]\hline
\end{tabular}
\label{table:nonlin}
\end{table} \clearpage
\begin{table}[ht]
\caption{Resultados de la ejecución de la metaheurística ACO, utilizando instancias de Dethloff con la configuración -n 20 -alpha 1.0 -beta 3.0 -q 0.8 -ro 0.3}
\centering
\small
\begin{tabular}{c c c c c c c}
\hline\hline
Instancia & Costo mínimo & Tiempo(seg.) & Costo promedio & Tiempo promedio(seg.) & Costo ACO & \%Gap \\ [0.5ex]
\hline
SCA3-0 & \bf{\underline{636.06}} & 13.26 & 
636.06 & 13.08 & 636.10 & 
-0.01\\SCA3-1 & \bf{\underline{697.84}} & 14.56 & 
697.84 & 14.58 & 700.10 & 
-0.32\\SCA3-2 & 659.34 & 13.03 & 
662.21 & 13.01 & \bf{659.30} & 
0.01\\SCA3-3 & 680.04 & 13.12 & 
680.04 & 13.16 & \bf{680.00} & 
0.01\\SCA3-4 & \bf{690.50} & 13.03 & 
690.50 & 13.74 & 690.50 & 0.00\\
SCA3-5 & \bf{\underline{659.90}} & 14.56 & 
662.05 & 14.44 & 671.10 & 
-1.67\\SCA3-6 & \bf{\underline{651.09}} & 14.56 & 
652.48 & 13.49 & 651.10 & 
-0.00\\SCA3-7 & 666.15 & 10.34 & 
666.15 & 10.62 & \bf{666.10} & 
0.01\\SCA3-8 & \bf{\underline{719.47}} & 12.58 & 
720.83 & 12.36 & 719.50 & 
-0.00\\SCA3-9 & \bf{681.00} & 10.72 & 
681.00 & 10.97 & 681.00 & 0.00\\
SCA8-0 & \bf{\underline{961.50}} & 15.50 & 
968.21 & 15.28 & 961.60 & 
-0.01\\SCA8-1 & \bf{\underline{1057.04}} & 11.60 & 
1062.82 & 11.67 & 1063.00 & 
-0.56\\SCA8-2 & 1049.22 & 10.35 & 
1050.47 & 10.24 & \bf{1040.60} & 
0.83\\SCA8-3 & 1013.77 & 15.29 & 
1015.38 & 14.97 & \bf{985.90} & 
2.83\\SCA8-4 & \bf{\underline{1067.66}} & 14.42 & 
1073.54 & 14.59 & 1071.00 & 
-0.31\\SCA8-5 & \bf{\underline{1034.74}} & 14.89 & 
1050.64 & 15.49 & 1054.30 & 
-1.86\\SCA8-6 & \bf{\underline{972.48}} & 15.92 & 
976.86 & 15.78 & 972.50 & 
-0.00\\SCA8-7 & 1067.20 & 16.92 & 
1069.90 & 16.19 & \bf{1059.70} & 
0.71\\SCA8-8 & \bf{\underline{1071.18}} & 13.91 & 
1072.14 & 14.55 & 1082.70 & 
-1.06\\SCA8-9 & \bf{\underline{1067.27}} & 10.98 & 
1067.38 & 11.15 & 1081.40 & 
-1.31\\CON3-0 & 616.52 & 14.41 & 
619.51 & 15.21 & \bf{616.50} & 
0.00\\CON3-1 & \bf{\underline{554.47}} & 14.72 & 
554.86 & 14.71 & 555.60 & 
-0.20\\CON3-2 & \bf{\underline{519.11}} & 13.00 & 
521.23 & 13.07 & 521.40 & 
-0.44\\CON3-3 & \bf{\underline{591.19}} & 14.74 & 
591.20 & 14.87 & 591.20 & 
-0.00\\CON3-4 & \bf{\underline{588.79}} & 12.54 & 
589.45 & 12.46 & 589.30 & 
-0.09\\CON3-5 & \bf{563.70} & 13.40 & 
565.24 & 13.62 & 563.70 & 0.00\\
CON3-6 & 500.80 & 15.20 & 
501.27 & 16.56 & \bf{499.20} & 
0.32\\CON3-7 & \bf{\underline{576.48}} & 12.10 & 
577.53 & 12.23 & 577.50 & 
-0.18\\CON3-8 & 523.68 & 12.06 & 
524.13 & 12.08 & \bf{523.10} & 
0.11\\CON3-9 & 588.40 & 12.72 & 
588.59 & 12.71 & \bf{578.20} & 
1.76\\CON8-0 & 869.43 & 14.04 & 
878.60 & 13.94 & \bf{858.90} & 
1.23\\CON8-1 & \bf{\underline{740.85}} & 14.52 & 
740.85 & 14.09 & 740.90 & 
-0.01\\CON8-2 & \bf{\underline{712.89}} & 19.70 & 
713.46 & 18.88 & 714.30 & 
-0.20\\CON8-3 & 816.27 & 13.55 & 
817.25 & 13.66 & \bf{812.30} & 
0.49\\CON8-4 & 776.37 & 13.84 & 
777.22 & 13.51 & \bf{770.10} & 
0.81\\CON8-5 & \bf{\underline{758.99}} & 13.79 & 
760.21 & 13.84 & 766.60 & 
-0.99\\CON8-6 & \bf{\underline{683.83}} & 15.76 & 
690.71 & 15.26 & 697.20 & 
-1.92\\CON8-7 & \bf{\underline{814.79}} & 11.84 & 
817.02 & 12.09 & 814.80 & 
-0.00\\CON8-8 & 782.86 & 15.92 & 
784.68 & 16.25 & \bf{771.30} & 
1.50\\CON8-9 & \bf{\underline{810.18}} & 15.22 & 
814.11 & 15.37 & 815.10 & 
-0.60\\[1ex]\hline
\end{tabular}
\label{table:nonlin}
\end{table} \clearpage
\begin{table}[ht]
\caption{Resultados de la ejecución de la metaheurística ACO, utilizando instancias de Dethloff con la configuración -n 20 -alpha 1.0 -beta 3.0 -q 0.8 -ro 0.4}
\centering
\small
\begin{tabular}{c c c c c c c}
\hline\hline
Instancia & Costo mínimo & Tiempo(seg.) & Costo promedio & Tiempo promedio(seg.) & Costo ACO & \%Gap \\ [0.5ex]
\hline
SCA3-0 & \bf{\underline{636.06}} & 16.99 & 
636.13 & 13.91 & 636.10 & 
-0.01\\SCA3-1 & \bf{\underline{697.84}} & 14.48 & 
697.84 & 14.13 & 700.10 & 
-0.32\\SCA3-2 & 659.34 & 13.10 & 
659.79 & 12.50 & \bf{659.30} & 
0.01\\SCA3-3 & 680.04 & 12.73 & 
680.04 & 13.38 & \bf{680.00} & 
0.01\\SCA3-4 & \bf{690.50} & 13.36 & 
690.50 & 13.89 & 690.50 & 0.00\\
SCA3-5 & \bf{\underline{659.90}} & 15.11 & 
662.04 & 16.83 & 671.10 & 
-1.67\\SCA3-6 & 652.94 & 13.46 & 
652.94 & 13.55 & \bf{651.10} & 
0.28\\SCA3-7 & 666.15 & 11.60 & 
666.15 & 11.49 & \bf{666.10} & 
0.01\\SCA3-8 & \bf{\underline{719.47}} & 13.14 & 
720.04 & 12.21 & 719.50 & 
-0.00\\SCA3-9 & \bf{681.00} & 10.83 & 
681.00 & 10.90 & 681.00 & 0.00\\
SCA8-0 & 968.79 & 13.89 & 
980.97 & 14.42 & \bf{961.60} & 
0.75\\SCA8-1 & \bf{\underline{1050.93}} & 12.27 & 
1060.62 & 11.76 & 1063.00 & 
-1.14\\SCA8-2 & 1044.24 & 12.50 & 
1049.57 & 10.78 & \bf{1040.60} & 
0.35\\SCA8-3 & 1008.52 & 14.06 & 
1013.63 & 14.49 & \bf{985.90} & 
2.29\\SCA8-4 & 1081.05 & 14.50 & 
1081.05 & 14.82 & \bf{1071.00} & 
0.94\\SCA8-5 & \bf{\underline{1034.74}} & 18.40 & 
1042.11 & 16.64 & 1054.30 & 
-1.86\\SCA8-6 & \bf{\underline{972.48}} & 16.42 & 
977.42 & 16.35 & 972.50 & 
-0.00\\SCA8-7 & 1067.03 & 16.64 & 
1068.26 & 15.76 & \bf{1059.70} & 
0.69\\SCA8-8 & \bf{\underline{1071.18}} & 14.86 & 
1071.18 & 15.04 & 1082.70 & 
-1.06\\SCA8-9 & \bf{\underline{1067.42}} & 10.69 & 
1067.42 & 11.22 & 1081.40 & 
-1.29\\CON3-0 & 617.59 & 14.50 & 
620.31 & 15.13 & \bf{616.50} & 
0.18\\CON3-1 & \bf{\underline{554.47}} & 14.20 & 
555.84 & 13.70 & 555.60 & 
-0.20\\CON3-2 & \bf{\underline{521.38}} & 14.04 & 
522.98 & 12.64 & 521.40 & 
-0.00\\CON3-3 & \bf{\underline{591.19}} & 15.22 & 
591.20 & 14.99 & 591.20 & 
-0.00\\CON3-4 & \bf{\underline{588.79}} & 14.05 & 
589.45 & 13.11 & 589.30 & 
-0.09\\CON3-5 & \bf{563.70} & 15.40 & 
564.76 & 14.92 & 563.70 & 0.00\\
CON3-6 & \bf{\underline{499.05}} & 17.58 & 
500.50 & 17.03 & 499.20 & 
-0.03\\CON3-7 & 577.54 & 11.69 & 
578.14 & 12.08 & \bf{577.50} & 
0.01\\CON3-8 & 523.14 & 11.38 & 
524.50 & 11.63 & \bf{523.10} & 
0.01\\CON3-9 & 587.78 & 12.62 & 
588.26 & 11.92 & \bf{578.20} & 
1.66\\CON8-0 & 865.86 & 14.35 & 
871.59 & 14.27 & \bf{858.90} & 
0.81\\CON8-1 & \bf{\underline{740.85}} & 14.27 & 
741.57 & 13.89 & 740.90 & 
-0.01\\CON8-2 & \bf{\underline{713.44}} & 19.53 & 
714.85 & 18.92 & 714.30 & 
-0.12\\CON8-3 & 812.54 & 12.56 & 
815.54 & 13.12 & \bf{812.30} & 
0.03\\CON8-4 & 776.37 & 14.47 & 
784.52 & 14.96 & \bf{770.10} & 
0.81\\CON8-5 & \bf{\underline{755.14}} & 13.79 & 
761.47 & 13.56 & 766.60 & 
-1.49\\CON8-6 & \bf{\underline{690.00}} & 15.12 & 
693.04 & 15.66 & 697.20 & 
-1.03\\CON8-7 & \bf{\underline{814.79}} & 12.54 & 
814.81 & 12.79 & 814.80 & 
-0.00\\CON8-8 & 784.34 & 15.02 & 
788.78 & 15.18 & \bf{771.30} & 
1.69\\CON8-9 & \bf{\underline{812.60}} & 14.50 & 
814.17 & 15.11 & 815.10 & 
-0.31\\[1ex]\hline
\end{tabular}
\label{table:nonlin}
\end{table} \clearpage
\begin{table}[ht]
\caption{Resultados de la ejecución de la metaheurística ACO, utilizando instancias de Dethloff con la configuración -n 20 -alpha 1.0 -beta 3.0 -q 0.8 -ro 0.5}
\centering
\small
\begin{tabular}{c c c c c c c}
\hline\hline
Instancia & Costo mínimo & Tiempo(seg.) & Costo promedio & Tiempo promedio(seg.) & Costo ACO & \%Gap \\ [0.5ex]
\hline
SCA3-0 & \bf{\underline{636.06}} & 13.62 & 
636.06 & 13.49 & 636.10 & 
-0.01\\SCA3-1 & \bf{\underline{697.84}} & 14.21 & 
697.84 & 14.39 & 700.10 & 
-0.32\\SCA3-2 & 659.34 & 12.53 & 
661.76 & 12.05 & \bf{659.30} & 
0.01\\SCA3-3 & 680.04 & 12.90 & 
680.18 & 13.01 & \bf{680.00} & 
0.01\\SCA3-4 & \bf{690.50} & 13.31 & 
690.50 & 13.54 & 690.50 & 0.00\\
SCA3-5 & \bf{\underline{662.75}} & 15.50 & 
663.32 & 14.98 & 671.10 & 
-1.24\\SCA3-6 & \bf{\underline{651.09}} & 13.82 & 
652.23 & 13.81 & 651.10 & 
-0.00\\SCA3-7 & 666.15 & 11.02 & 
666.15 & 11.64 & \bf{666.10} & 
0.01\\SCA3-8 & \bf{\underline{719.47}} & 11.69 & 
719.47 & 11.90 & 719.50 & 
-0.00\\SCA3-9 & \bf{681.00} & 10.57 & 
681.00 & 11.06 & 681.00 & 0.00\\
SCA8-0 & 968.79 & 15.33 & 
977.51 & 14.75 & \bf{961.60} & 
0.75\\SCA8-1 & \bf{\underline{1049.65}} & 11.20 & 
1059.57 & 11.29 & 1063.00 & 
-1.26\\SCA8-2 & 1046.29 & 10.38 & 
1049.00 & 10.42 & \bf{1040.60} & 
0.55\\SCA8-3 & 1002.89 & 15.53 & 
1011.56 & 15.01 & \bf{985.90} & 
1.72\\SCA8-4 & \bf{\underline{1065.49}} & 15.54 & 
1068.84 & 14.68 & 1071.00 & 
-0.51\\SCA8-5 & \bf{\underline{1052.96}} & 15.70 & 
1054.81 & 15.55 & 1054.30 & 
-0.13\\SCA8-6 & \bf{\underline{972.48}} & 16.66 & 
977.45 & 15.83 & 972.50 & 
-0.00\\SCA8-7 & 1067.20 & 15.72 & 
1068.12 & 15.86 & \bf{1059.70} & 
0.71\\SCA8-8 & \bf{\underline{1071.18}} & 15.33 & 
1071.18 & 15.41 & 1082.70 & 
-1.06\\SCA8-9 & \bf{\underline{1067.42}} & 11.88 & 
1067.42 & 11.65 & 1081.40 & 
-1.29\\CON3-0 & 617.59 & 14.66 & 
620.32 & 15.69 & \bf{616.50} & 
0.18\\CON3-1 & \bf{\underline{554.47}} & 15.36 & 
554.92 & 14.74 & 555.60 & 
-0.20\\CON3-2 & \bf{\underline{519.11}} & 13.48 & 
520.53 & 13.00 & 521.40 & 
-0.44\\CON3-3 & \bf{591.20} & 15.44 & 
591.50 & 14.91 & 591.20 & 0.00\\
CON3-4 & \bf{\underline{588.79}} & 13.35 & 
589.45 & 12.96 & 589.30 & 
-0.09\\CON3-5 & 564.88 & 13.12 & 
565.43 & 13.41 & \bf{563.70} & 
0.21\\CON3-6 & 500.80 & 16.18 & 
501.93 & 15.89 & \bf{499.20} & 
0.32\\CON3-7 & 577.54 & 12.73 & 
578.10 & 12.54 & \bf{577.50} & 
0.01\\CON3-8 & \bf{\underline{523.05}} & 12.66 & 
523.83 & 12.05 & 523.10 & 
-0.01\\CON3-9 & 580.05 & 12.97 & 
586.50 & 13.25 & \bf{578.20} & 
0.32\\CON8-0 & 867.34 & 13.16 & 
870.75 & 14.03 & \bf{858.90} & 
0.98\\CON8-1 & \bf{\underline{740.85}} & 11.76 & 
741.21 & 13.09 & 740.90 & 
-0.01\\CON8-2 & \bf{\underline{712.89}} & 20.62 & 
713.85 & 19.28 & 714.30 & 
-0.20\\CON8-3 & \bf{\underline{811.07}} & 14.13 & 
813.92 & 14.12 & 812.30 & 
-0.15\\CON8-4 & 784.57 & 15.55 & 
788.47 & 14.53 & \bf{770.10} & 
1.88\\CON8-5 & \bf{\underline{760.03}} & 13.63 & 
761.76 & 13.57 & 766.60 & 
-0.86\\CON8-6 & \bf{\underline{690.00}} & 16.62 & 
693.48 & 16.39 & 697.20 & 
-1.03\\CON8-7 & \bf{\underline{814.77}} & 11.68 & 
817.78 & 11.79 & 814.80 & 
-0.00\\CON8-8 & 784.70 & 15.85 & 
788.12 & 16.03 & \bf{771.30} & 
1.74\\CON8-9 & \bf{\underline{811.43}} & 15.05 & 
814.38 & 15.07 & 815.10 & 
-0.45\\[1ex]\hline
\end{tabular}
\label{table:nonlin}
\end{table} \clearpage
\begin{table}[ht]
\caption{Resultados de la ejecución de la metaheurística ACO, utilizando instancias de Dethloff con la configuración -n 20 -alpha 1.0 -beta 3.0 -q 0.8 -ro 0.6}
\centering
\small
\begin{tabular}{c c c c c c c}
\hline\hline
Instancia & Costo mínimo & Tiempo(seg.) & Costo promedio & Tiempo promedio(seg.) & Costo ACO & \%Gap \\ [0.5ex]
\hline
SCA3-0 & \bf{\underline{636.06}} & 13.68 & 
636.06 & 13.74 & 636.10 & 
-0.01\\SCA3-1 & \bf{\underline{697.84}} & 14.18 & 
697.84 & 14.10 & 700.10 & 
-0.32\\SCA3-2 & 659.34 & 20.53 & 
659.34 & 15.42 & \bf{659.30} & 
0.01\\SCA3-3 & 680.04 & 13.30 & 
680.04 & 13.32 & \bf{680.00} & 
0.01\\SCA3-4 & \bf{690.50} & 13.11 & 
690.50 & 13.35 & 690.50 & 0.00\\
SCA3-5 & \bf{\underline{659.90}} & 14.92 & 
661.62 & 15.61 & 671.10 & 
-1.67\\SCA3-6 & 652.94 & 12.84 & 
652.94 & 13.47 & \bf{651.10} & 
0.28\\SCA3-7 & 666.15 & 11.38 & 
666.15 & 11.63 & \bf{666.10} & 
0.01\\SCA3-8 & \bf{\underline{719.47}} & 12.46 & 
720.60 & 12.68 & 719.50 & 
-0.00\\SCA3-9 & \bf{681.00} & 11.95 & 
681.00 & 11.40 & 681.00 & 0.00\\
SCA8-0 & 968.79 & 14.09 & 
979.18 & 14.92 & \bf{961.60} & 
0.75\\SCA8-1 & \bf{\underline{1049.65}} & 11.30 & 
1062.86 & 11.25 & 1063.00 & 
-1.26\\SCA8-2 & 1050.17 & 9.99 & 
1050.99 & 10.51 & \bf{1040.60} & 
0.92\\SCA8-3 & 1008.29 & 15.04 & 
1011.07 & 15.83 & \bf{985.90} & 
2.27\\SCA8-4 & \bf{\underline{1067.66}} & 15.25 & 
1076.41 & 14.98 & 1071.00 & 
-0.31\\SCA8-5 & \bf{\underline{1039.12}} & 16.00 & 
1050.79 & 15.34 & 1054.30 & 
-1.44\\SCA8-6 & \bf{\underline{972.48}} & 16.81 & 
977.07 & 15.85 & 972.50 & 
-0.00\\SCA8-7 & 1067.20 & 16.48 & 
1069.10 & 16.79 & \bf{1059.70} & 
0.71\\SCA8-8 & \bf{\underline{1071.18}} & 15.10 & 
1072.82 & 15.17 & 1082.70 & 
-1.06\\SCA8-9 & \bf{\underline{1067.42}} & 10.45 & 
1067.42 & 11.50 & 1081.40 & 
-1.29\\CON3-0 & 617.59 & 14.72 & 
619.13 & 15.99 & \bf{616.50} & 
0.18\\CON3-1 & \bf{\underline{554.47}} & 14.26 & 
555.84 & 14.05 & 555.60 & 
-0.20\\CON3-2 & \bf{\underline{519.11}} & 12.16 & 
521.18 & 12.53 & 521.40 & 
-0.44\\CON3-3 & \bf{\underline{591.19}} & 15.59 & 
591.27 & 16.20 & 591.20 & 
-0.00\\CON3-4 & \bf{\underline{588.79}} & 12.56 & 
589.72 & 12.93 & 589.30 & 
-0.09\\CON3-5 & 564.88 & 15.55 & 
565.83 & 14.01 & \bf{563.70} & 
0.21\\CON3-6 & 502.16 & 17.69 & 
502.16 & 16.71 & \bf{499.20} & 
0.59\\CON3-7 & 578.22 & 11.65 & 
578.36 & 11.99 & \bf{577.50} & 
0.12\\CON3-8 & 523.14 & 11.66 & 
523.70 & 12.33 & \bf{523.10} & 
0.01\\CON3-9 & 578.25 & 13.99 & 
583.82 & 13.27 & \bf{578.20} & 
0.01\\CON8-0 & 869.43 & 13.47 & 
871.98 & 13.95 & \bf{858.90} & 
1.23\\CON8-1 & \bf{\underline{740.85}} & 13.63 & 
741.21 & 13.84 & 740.90 & 
-0.01\\CON8-2 & \bf{\underline{713.44}} & 17.61 & 
713.78 & 18.76 & 714.30 & 
-0.12\\CON8-3 & \bf{\underline{812.22}} & 13.78 & 
815.03 & 14.10 & 812.30 & 
-0.01\\CON8-4 & 776.37 & 14.36 & 
780.56 & 14.04 & \bf{770.10} & 
0.81\\CON8-5 & \bf{\underline{761.40}} & 12.80 & 
763.21 & 13.14 & 766.60 & 
-0.68\\CON8-6 & \bf{\underline{693.83}} & 14.98 & 
695.39 & 15.43 & 697.20 & 
-0.48\\CON8-7 & 814.86 & 12.02 & 
816.99 & 12.38 & \bf{814.80} & 
0.01\\CON8-8 & 782.86 & 15.04 & 
784.92 & 15.82 & \bf{771.30} & 
1.50\\CON8-9 & \bf{\underline{812.60}} & 14.39 & 
814.52 & 14.90 & 815.10 & 
-0.31\\[1ex]\hline
\end{tabular}
\label{table:nonlin}
\end{table} \clearpage
\begin{table}[ht]
\caption{Resultados de la ejecución de la metaheurística ACO, utilizando instancias de Dethloff con la configuración -n 20 -alpha 1.0 -beta 3.0 -q 0.8 -ro 0.7}
\centering
\small
\begin{tabular}{c c c c c c c}
\hline\hline
Instancia & Costo mínimo & Tiempo(seg.) & Costo promedio & Tiempo promedio(seg.) & Costo ACO & \%Gap \\ [0.5ex]
\hline
SCA3-0 & \bf{\underline{636.06}} & 13.58 & 
636.13 & 13.52 & 636.10 & 
-0.01\\SCA3-1 & \bf{\underline{697.84}} & 14.47 & 
697.84 & 14.53 & 700.10 & 
-0.32\\SCA3-2 & 659.34 & 12.56 & 
659.34 & 12.51 & \bf{659.30} & 
0.01\\SCA3-3 & 680.04 & 13.73 & 
680.18 & 13.65 & \bf{680.00} & 
0.01\\SCA3-4 & \bf{690.50} & 14.51 & 
690.50 & 14.63 & 690.50 & 0.00\\
SCA3-5 & \bf{\underline{659.90}} & 14.66 & 
662.76 & 15.02 & 671.10 & 
-1.67\\SCA3-6 & \bf{\underline{651.09}} & 12.42 & 
652.48 & 13.83 & 651.10 & 
-0.00\\SCA3-7 & 666.15 & 11.35 & 
666.15 & 11.65 & \bf{666.10} & 
0.01\\SCA3-8 & \bf{\underline{719.47}} & 13.14 & 
719.97 & 12.52 & 719.50 & 
-0.00\\SCA3-9 & \bf{681.00} & 10.86 & 
681.00 & 10.97 & 681.00 & 0.00\\
SCA8-0 & \bf{\underline{961.50}} & 14.40 & 
969.08 & 15.02 & 961.60 & 
-0.01\\SCA8-1 & \bf{\underline{1051.23}} & 12.24 & 
1058.90 & 11.78 & 1063.00 & 
-1.11\\SCA8-2 & 1050.37 & 10.33 & 
1051.93 & 10.13 & \bf{1040.60} & 
0.94\\SCA8-3 & 1008.29 & 13.32 & 
1014.90 & 14.51 & \bf{985.90} & 
2.27\\SCA8-4 & \bf{\underline{1065.49}} & 14.36 & 
1067.47 & 14.95 & 1071.00 & 
-0.51\\SCA8-5 & \bf{\underline{1034.74}} & 15.59 & 
1048.93 & 15.85 & 1054.30 & 
-1.86\\SCA8-6 & \bf{\underline{972.48}} & 16.30 & 
975.74 & 16.11 & 972.50 & 
-0.00\\SCA8-7 & 1067.20 & 15.36 & 
1067.31 & 15.64 & \bf{1059.70} & 
0.71\\SCA8-8 & \bf{\underline{1071.18}} & 15.39 & 
1071.18 & 14.86 & 1082.70 & 
-1.06\\SCA8-9 & \bf{\underline{1063.68}} & 11.24 & 
1066.49 & 11.48 & 1081.40 & 
-1.64\\CON3-0 & 621.09 & 15.26 & 
622.85 & 14.87 & \bf{616.50} & 
0.74\\CON3-1 & \bf{\underline{554.47}} & 13.53 & 
555.20 & 13.69 & 555.60 & 
-0.20\\CON3-2 & \bf{\underline{521.38}} & 13.68 & 
522.45 & 12.75 & 521.40 & 
-0.00\\CON3-3 & \bf{\underline{591.19}} & 15.06 & 
591.20 & 14.50 & 591.20 & 
-0.00\\CON3-4 & \bf{\underline{588.79}} & 12.53 & 
588.92 & 12.67 & 589.30 & 
-0.09\\CON3-5 & 564.88 & 12.65 & 
566.62 & 13.03 & \bf{563.70} & 
0.21\\CON3-6 & \bf{\underline{499.05}} & 16.87 & 
500.77 & 16.97 & 499.20 & 
-0.03\\CON3-7 & 578.41 & 12.93 & 
578.41 & 12.49 & \bf{577.50} & 
0.16\\CON3-8 & \bf{\underline{523.05}} & 12.88 & 
523.85 & 12.32 & 523.10 & 
-0.01\\CON3-9 & 578.25 & 12.62 & 
585.48 & 12.31 & \bf{578.20} & 
0.01\\CON8-0 & 865.86 & 13.90 & 
869.87 & 14.09 & \bf{858.90} & 
0.81\\CON8-1 & \bf{\underline{740.85}} & 14.13 & 
741.21 & 13.86 & 740.90 & 
-0.01\\CON8-2 & \bf{\underline{712.89}} & 19.29 & 
713.85 & 18.66 & 714.30 & 
-0.20\\CON8-3 & \bf{\underline{812.11}} & 13.50 & 
815.99 & 14.19 & 812.30 & 
-0.02\\CON8-4 & 780.51 & 14.62 & 
787.61 & 14.27 & \bf{770.10} & 
1.35\\CON8-5 & \bf{\underline{758.99}} & 15.11 & 
760.80 & 13.79 & 766.60 & 
-0.99\\CON8-6 & \bf{\underline{694.22}} & 16.18 & 
694.58 & 16.05 & 697.20 & 
-0.43\\CON8-7 & \bf{\underline{814.79}} & 11.68 & 
816.34 & 12.32 & 814.80 & 
-0.00\\CON8-8 & 780.96 & 15.98 & 
785.28 & 15.54 & \bf{771.30} & 
1.25\\CON8-9 & 815.44 & 13.81 & 
815.69 & 14.95 & \bf{815.10} & 
0.04\\[1ex]\hline
\end{tabular}
\label{table:nonlin}
\end{table} \clearpage
\begin{table}[ht]
\caption{Resultados de la ejecución de la metaheurística ACO, utilizando instancias de Dethloff con la configuración -n 20 -alpha 1.0 -beta 3.0 -q 0.8 -ro 0.8}
\centering
\small
\begin{tabular}{c c c c c c c}
\hline\hline
Instancia & Costo mínimo & Tiempo(seg.) & Costo promedio & Tiempo promedio(seg.) & Costo ACO & \%Gap \\ [0.5ex]
\hline
SCA3-0 & \bf{\underline{636.06}} & 13.87 & 
636.06 & 14.01 & 636.10 & 
-0.01\\SCA3-1 & \bf{\underline{697.84}} & 14.13 & 
697.84 & 14.19 & 700.10 & 
-0.32\\SCA3-2 & 659.34 & 12.88 & 
661.00 & 13.18 & \bf{659.30} & 
0.01\\SCA3-3 & 680.04 & 13.20 & 
680.04 & 13.48 & \bf{680.00} & 
0.01\\SCA3-4 & \bf{690.50} & 16.75 & 
690.50 & 15.17 & 690.50 & 0.00\\
SCA3-5 & \bf{\underline{659.90}} & 14.58 & 
662.76 & 15.30 & 671.10 & 
-1.67\\SCA3-6 & 652.94 & 13.82 & 
652.94 & 13.90 & \bf{651.10} & 
0.28\\SCA3-7 & 666.15 & 11.52 & 
666.15 & 11.19 & \bf{666.10} & 
0.01\\SCA3-8 & \bf{\underline{719.47}} & 12.18 & 
719.47 & 12.14 & 719.50 & 
-0.00\\SCA3-9 & \bf{681.00} & 10.78 & 
681.00 & 10.93 & 681.00 & 0.00\\
SCA8-0 & 977.26 & 15.19 & 
983.78 & 15.32 & \bf{961.60} & 
1.63\\SCA8-1 & \bf{\underline{1059.11}} & 10.51 & 
1062.91 & 10.98 & 1063.00 & 
-0.37\\SCA8-2 & 1049.22 & 11.59 & 
1050.03 & 10.70 & \bf{1040.60} & 
0.83\\SCA8-3 & 1008.29 & 15.07 & 
1012.26 & 15.29 & \bf{985.90} & 
2.27\\SCA8-4 & \bf{\underline{1067.82}} & 14.88 & 
1078.58 & 14.80 & 1071.00 & 
-0.30\\SCA8-5 & \bf{\underline{1047.16}} & 14.54 & 
1051.85 & 15.56 & 1054.30 & 
-0.68\\SCA8-6 & \bf{\underline{972.48}} & 16.74 & 
974.96 & 16.36 & 972.50 & 
-0.00\\SCA8-7 & 1067.20 & 15.70 & 
1072.55 & 15.45 & \bf{1059.70} & 
0.71\\SCA8-8 & \bf{\underline{1071.18}} & 15.51 & 
1071.18 & 14.51 & 1082.70 & 
-1.06\\SCA8-9 & \bf{\underline{1067.42}} & 11.31 & 
1067.42 & 11.76 & 1081.40 & 
-1.29\\CON3-0 & 620.76 & 15.38 & 
623.41 & 16.23 & \bf{616.50} & 
0.69\\CON3-1 & \bf{\underline{554.47}} & 12.76 & 
555.78 & 14.14 & 555.60 & 
-0.20\\CON3-2 & \bf{\underline{521.38}} & 13.96 & 
522.53 & 12.72 & 521.40 & 
-0.00\\CON3-3 & \bf{\underline{591.19}} & 15.02 & 
591.20 & 14.19 & 591.20 & 
-0.00\\CON3-4 & \bf{\underline{588.79}} & 12.56 & 
589.45 & 12.52 & 589.30 & 
-0.09\\CON3-5 & \bf{563.70} & 14.63 & 
564.81 & 14.27 & 563.70 & 0.00\\
CON3-6 & \bf{\underline{499.05}} & 15.47 & 
501.50 & 16.35 & 499.20 & 
-0.03\\CON3-7 & \bf{\underline{576.48}} & 11.87 & 
577.93 & 12.10 & 577.50 & 
-0.18\\CON3-8 & 523.68 & 10.84 & 
524.16 & 11.73 & \bf{523.10} & 
0.11\\CON3-9 & 587.78 & 12.22 & 
588.21 & 12.59 & \bf{578.20} & 
1.66\\CON8-0 & 869.43 & 13.49 & 
873.01 & 14.09 & \bf{858.90} & 
1.23\\CON8-1 & \bf{\underline{740.85}} & 13.50 & 
740.85 & 13.97 & 740.90 & 
-0.01\\CON8-2 & \bf{\underline{713.44}} & 18.52 & 
713.99 & 18.72 & 714.30 & 
-0.12\\CON8-3 & \bf{\underline{812.11}} & 13.55 & 
816.21 & 13.54 & 812.30 & 
-0.02\\CON8-4 & 776.34 & 14.71 & 
781.79 & 14.00 & \bf{770.10} & 
0.81\\CON8-5 & \bf{\underline{764.16}} & 13.48 & 
764.61 & 13.93 & 766.60 & 
-0.32\\CON8-6 & \bf{\underline{695.66}} & 15.88 & 
695.85 & 16.32 & 697.20 & 
-0.22\\CON8-7 & \bf{\underline{814.79}} & 12.04 & 
817.15 & 12.36 & 814.80 & 
-0.00\\CON8-8 & 782.86 & 16.09 & 
783.49 & 16.07 & \bf{771.30} & 
1.50\\CON8-9 & \bf{\underline{812.60}} & 14.71 & 
813.70 & 14.95 & 815.10 & 
-0.31\\[1ex]\hline
\end{tabular}
\label{table:nonlin}
\end{table} \clearpage
\begin{table}[ht]
\caption{Resultados de la ejecución de la metaheurística ACO, utilizando instancias de Dethloff con la configuración -n 20 -alpha 1.0 -beta 3.0 -q 0.8 -ro 0.9}
\centering
\small
\begin{tabular}{c c c c c c c}
\hline\hline
Instancia & Costo mínimo & Tiempo(seg.) & Costo promedio & Tiempo promedio(seg.) & Costo ACO & \%Gap \\ [0.5ex]
\hline
SCA3-0 & \bf{\underline{636.06}} & 13.80 & 
636.06 & 13.47 & 636.10 & 
-0.01\\SCA3-1 & \bf{\underline{697.84}} & 14.15 & 
697.84 & 14.79 & 700.10 & 
-0.32\\SCA3-2 & 659.34 & 13.09 & 
660.24 & 12.87 & \bf{659.30} & 
0.01\\SCA3-3 & 680.04 & 13.23 & 
680.32 & 13.33 & \bf{680.00} & 
0.01\\SCA3-4 & \bf{690.50} & 14.66 & 
690.50 & 14.05 & 690.50 & 0.00\\
SCA3-5 & \bf{\underline{659.90}} & 15.60 & 
663.48 & 15.30 & 671.10 & 
-1.67\\SCA3-6 & \bf{\underline{651.09}} & 12.85 & 
652.01 & 14.02 & 651.10 & 
-0.00\\SCA3-7 & \bf{\underline{659.17}} & 14.09 & 
664.40 & 12.25 & 666.10 & 
-1.04\\SCA3-8 & \bf{\underline{719.47}} & 12.42 & 
719.47 & 13.64 & 719.50 & 
-0.00\\SCA3-9 & \bf{681.00} & 11.67 & 
681.00 & 11.24 & 681.00 & 0.00\\
SCA8-0 & 975.50 & 15.31 & 
980.82 & 15.05 & \bf{961.60} & 
1.45\\SCA8-1 & \bf{\underline{1060.96}} & 13.16 & 
1066.07 & 11.82 & 1063.00 & 
-0.19\\SCA8-2 & 1046.29 & 10.73 & 
1049.56 & 10.65 & \bf{1040.60} & 
0.55\\SCA8-3 & 997.63 & 14.69 & 
1011.86 & 14.84 & \bf{985.90} & 
1.19\\SCA8-4 & \bf{\underline{1067.66}} & 14.44 & 
1071.18 & 14.64 & 1071.00 & 
-0.31\\SCA8-5 & \bf{\underline{1034.74}} & 16.76 & 
1049.02 & 15.76 & 1054.30 & 
-1.86\\SCA8-6 & \bf{\underline{972.48}} & 16.85 & 
977.83 & 15.90 & 972.50 & 
-0.00\\SCA8-7 & 1067.20 & 16.74 & 
1069.55 & 16.78 & \bf{1059.70} & 
0.71\\SCA8-8 & \bf{\underline{1071.18}} & 14.82 & 
1074.11 & 15.23 & 1082.70 & 
-1.06\\SCA8-9 & \bf{\underline{1067.42}} & 11.32 & 
1067.42 & 11.32 & 1081.40 & 
-1.29\\CON3-0 & 620.76 & 14.92 & 
623.73 & 15.03 & \bf{616.50} & 
0.69\\CON3-1 & \bf{\underline{554.47}} & 14.54 & 
554.47 & 15.23 & 555.60 & 
-0.20\\CON3-2 & \bf{\underline{519.61}} & 13.84 & 
521.43 & 13.07 & 521.40 & 
-0.34\\CON3-3 & \bf{\underline{591.19}} & 16.23 & 
591.20 & 15.10 & 591.20 & 
-0.00\\CON3-4 & \bf{\underline{588.79}} & 13.55 & 
589.87 & 13.86 & 589.30 & 
-0.09\\CON3-5 & \bf{563.70} & 13.46 & 
564.59 & 13.97 & 563.70 & 0.00\\
CON3-6 & \bf{\underline{499.05}} & 16.68 & 
500.61 & 17.36 & 499.20 & 
-0.03\\CON3-7 & \bf{\underline{576.48}} & 12.97 & 
577.26 & 12.43 & 577.50 & 
-0.18\\CON3-8 & 523.14 & 12.47 & 
523.93 & 12.13 & \bf{523.10} & 
0.01\\CON3-9 & 588.28 & 12.62 & 
588.53 & 12.93 & \bf{578.20} & 
1.74\\CON8-0 & 867.51 & 15.27 & 
869.50 & 14.15 & \bf{858.90} & 
1.00\\CON8-1 & \bf{\underline{740.85}} & 13.21 & 
743.58 & 13.24 & 740.90 & 
-0.01\\CON8-2 & \bf{\underline{713.44}} & 18.99 & 
713.67 & 18.97 & 714.30 & 
-0.12\\CON8-3 & 817.57 & 14.21 & 
817.57 & 14.03 & \bf{812.30} & 
0.65\\CON8-4 & 776.37 & 13.94 & 
783.16 & 14.57 & \bf{770.10} & 
0.81\\CON8-5 & \bf{\underline{760.73}} & 12.79 & 
763.32 & 13.43 & 766.60 & 
-0.77\\CON8-6 & \bf{\underline{688.46}} & 16.30 & 
693.55 & 18.89 & 697.20 & 
-1.25\\CON8-7 & \bf{\underline{814.50}} & 12.51 & 
816.97 & 11.98 & 814.80 & 
-0.04\\CON8-8 & 778.39 & 16.78 & 
783.28 & 18.95 & \bf{771.30} & 
0.92\\CON8-9 & \bf{\underline{812.60}} & 15.86 & 
813.82 & 15.60 & 815.10 & 
-0.31\\[1ex]\hline
\end{tabular}
\label{table:nonlin}
\end{table} \clearpage
\begin{table}[ht]
\caption{Resultados de la ejecución de la metaheurística ACO, utilizando instancias de Dethloff con la configuración -n 20 -alpha 1.0 -beta 3.0 -q 0.8 -ro 1.0}
\centering
\small
\begin{tabular}{c c c c c c c}
\hline\hline
Instancia & Costo mínimo & Tiempo(seg.) & Costo promedio & Tiempo promedio(seg.) & Costo ACO & \%Gap \\ [0.5ex]
\hline
SCA3-0 & \bf{\underline{636.06}} & 13.84 & 
636.06 & 13.62 & 636.10 & 
-0.01\\SCA3-1 & \bf{\underline{697.84}} & 14.28 & 
697.84 & 14.50 & 700.10 & 
-0.32\\SCA3-2 & 659.34 & 15.20 & 
661.00 & 13.78 & \bf{659.30} & 
0.01\\SCA3-3 & 680.04 & 14.05 & 
680.04 & 13.80 & \bf{680.00} & 
0.01\\SCA3-4 & \bf{690.50} & 17.99 & 
690.50 & 15.27 & 690.50 & 0.00\\
SCA3-5 & \bf{\underline{662.75}} & 16.62 & 
662.75 & 15.71 & 671.10 & 
-1.24\\SCA3-6 & \bf{\underline{651.09}} & 13.58 & 
652.48 & 13.89 & 651.10 & 
-0.00\\SCA3-7 & 666.15 & 11.74 & 
666.15 & 11.97 & \bf{666.10} & 
0.01\\SCA3-8 & \bf{\underline{719.47}} & 14.60 & 
719.47 & 13.80 & 719.50 & 
-0.00\\SCA3-9 & \bf{681.00} & 11.56 & 
681.00 & 11.83 & 681.00 & 0.00\\
SCA8-0 & 968.79 & 15.46 & 
974.17 & 15.49 & \bf{961.60} & 
0.75\\SCA8-1 & \bf{\underline{1049.65}} & 11.28 & 
1057.29 & 11.76 & 1063.00 & 
-1.26\\SCA8-2 & 1050.37 & 11.60 & 
1051.14 & 11.03 & \bf{1040.60} & 
0.94\\SCA8-3 & 1005.64 & 15.82 & 
1011.65 & 15.96 & \bf{985.90} & 
2.00\\SCA8-4 & \bf{\underline{1065.49}} & 14.48 & 
1067.83 & 14.90 & 1071.00 & 
-0.51\\SCA8-5 & \bf{\underline{1034.74}} & 15.97 & 
1044.13 & 17.88 & 1054.30 & 
-1.86\\SCA8-6 & \bf{\underline{972.48}} & 17.30 & 
976.14 & 16.61 & 972.50 & 
-0.00\\SCA8-7 & 1067.20 & 14.98 & 
1069.26 & 16.05 & \bf{1059.70} & 
0.71\\SCA8-8 & \bf{\underline{1071.18}} & 14.38 & 
1071.18 & 15.34 & 1082.70 & 
-1.06\\SCA8-9 & \bf{\underline{1067.42}} & 11.26 & 
1067.42 & 11.81 & 1081.40 & 
-1.29\\CON3-0 & 616.52 & 15.41 & 
619.70 & 15.72 & \bf{616.50} & 
0.00\\CON3-1 & \bf{\underline{554.47}} & 14.80 & 
555.59 & 14.79 & 555.60 & 
-0.20\\CON3-2 & \bf{\underline{521.38}} & 14.97 & 
521.38 & 15.07 & 521.40 & 
-0.00\\CON3-3 & \bf{\underline{591.19}} & 15.48 & 
591.20 & 15.04 & 591.20 & 
-0.00\\CON3-4 & \bf{\underline{588.79}} & 13.87 & 
589.05 & 13.39 & 589.30 & 
-0.09\\CON3-5 & \bf{563.70} & 14.43 & 
566.12 & 14.04 & 563.70 & 0.00\\
CON3-6 & \bf{\underline{499.05}} & 16.87 & 
500.83 & 17.13 & 499.20 & 
-0.03\\CON3-7 & \bf{\underline{576.48}} & 12.22 & 
577.88 & 12.59 & 577.50 & 
-0.18\\CON3-8 & \bf{\underline{523.05}} & 12.46 & 
523.52 & 12.38 & 523.10 & 
-0.01\\CON3-9 & 588.40 & 13.96 & 
588.42 & 13.47 & \bf{578.20} & 
1.76\\CON8-0 & 869.43 & 15.30 & 
871.15 & 15.13 & \bf{858.90} & 
1.23\\CON8-1 & \bf{\underline{740.85}} & 14.02 & 
741.21 & 15.12 & 740.90 & 
-0.01\\CON8-2 & \bf{\underline{712.89}} & 18.90 & 
713.85 & 19.48 & 714.30 & 
-0.20\\CON8-3 & 815.56 & 13.44 & 
816.54 & 14.07 & \bf{812.30} & 
0.40\\CON8-4 & 776.37 & 15.80 & 
779.45 & 14.95 & \bf{770.10} & 
0.81\\CON8-5 & \bf{\underline{758.84}} & 14.01 & 
762.16 & 14.19 & 766.60 & 
-1.01\\CON8-6 & \bf{\underline{686.39}} & 16.19 & 
692.95 & 16.02 & 697.20 & 
-1.55\\CON8-7 & \bf{\underline{814.79}} & 12.54 & 
815.25 & 12.96 & 814.80 & 
-0.00\\CON8-8 & 784.52 & 16.40 & 
788.28 & 16.36 & \bf{771.30} & 
1.71\\CON8-9 & \bf{\underline{811.18}} & 16.02 & 
814.18 & 15.28 & 815.10 & 
-0.48\\[1ex]\hline
\end{tabular}
\label{table:nonlin}
\end{table} \clearpage
\begin{table}[ht]
\caption{Resultados de la ejecución de la metaheurística ACO, utilizando instancias de Dethloff con la configuración -n 50 -alpha 1.0 -beta 3.0 -q 0.8 -ro 0.015}
\centering
\small
\begin{tabular}{c c c c c c c}
\hline\hline
Instancia & Costo mínimo & Tiempo(seg.) & Costo promedio & Tiempo promedio(seg.) & Costo ACO & \%Gap \\ [0.5ex]
\hline
SCA3-0 & \bf{\underline{636.06}} & 33.23 & 
636.06 & 32.89 & 636.10 & 
-0.01\\SCA3-1 & \bf{\underline{697.84}} & 34.81 & 
697.84 & 35.69 & 700.10 & 
-0.32\\SCA3-2 & 659.34 & 32.53 & 
660.55 & 40.07 & \bf{659.30} & 
0.01\\SCA3-3 & 680.04 & 32.31 & 
680.18 & 33.70 & \bf{680.00} & 
0.01\\SCA3-4 & \bf{690.50} & 42.26 & 
690.50 & 35.36 & 690.50 & 0.00\\
SCA3-5 & \bf{\underline{659.90}} & 39.55 & 
661.18 & 38.77 & 671.10 & 
-1.67\\SCA3-6 & \bf{\underline{651.09}} & 32.55 & 
651.55 & 33.87 & 651.10 & 
-0.00\\SCA3-7 & \bf{\underline{659.17}} & 27.58 & 
664.40 & 26.66 & 666.10 & 
-1.04\\SCA3-8 & \bf{\underline{719.47}} & 32.33 & 
720.60 & 31.09 & 719.50 & 
-0.00\\SCA3-9 & \bf{681.00} & 27.20 & 
681.00 & 28.25 & 681.00 & 0.00\\
SCA8-0 & \bf{\underline{961.50}} & 35.07 & 
966.21 & 37.04 & 961.60 & 
-0.01\\SCA8-1 & \bf{\underline{1052.36}} & 29.19 & 
1060.65 & 29.40 & 1063.00 & 
-1.00\\SCA8-2 & 1046.29 & 24.87 & 
1049.37 & 25.89 & \bf{1040.60} & 
0.55\\SCA8-3 & 995.50 & 37.54 & 
1003.20 & 37.52 & \bf{985.90} & 
0.97\\SCA8-4 & \bf{\underline{1065.49}} & 35.48 & 
1069.60 & 37.02 & 1071.00 & 
-0.51\\SCA8-5 & \bf{\underline{1034.74}} & 37.63 & 
1045.99 & 38.91 & 1054.30 & 
-1.86\\SCA8-6 & \bf{\underline{972.48}} & 38.32 & 
977.64 & 40.05 & 972.50 & 
-0.00\\SCA8-7 & 1066.65 & 41.30 & 
1068.57 & 40.88 & \bf{1059.70} & 
0.66\\SCA8-8 & \bf{\underline{1071.18}} & 38.71 & 
1071.18 & 37.62 & 1082.70 & 
-1.06\\SCA8-9 & \bf{\underline{1067.26}} & 29.18 & 
1067.38 & 28.87 & 1081.40 & 
-1.31\\CON3-0 & 616.52 & 39.08 & 
618.22 & 39.04 & \bf{616.50} & 
0.00\\CON3-1 & \bf{\underline{554.47}} & 35.77 & 
554.86 & 35.26 & 555.60 & 
-0.20\\CON3-2 & \bf{\underline{519.11}} & 31.29 & 
520.37 & 30.73 & 521.40 & 
-0.44\\CON3-3 & \bf{\underline{591.19}} & 38.78 & 
591.19 & 38.45 & 591.20 & 
-0.00\\CON3-4 & \bf{\underline{588.79}} & 31.37 & 
588.92 & 32.52 & 589.30 & 
-0.09\\CON3-5 & 564.88 & 35.60 & 
566.77 & 34.73 & \bf{563.70} & 
0.21\\CON3-6 & 500.37 & 42.51 & 
501.37 & 44.13 & \bf{499.20} & 
0.23\\CON3-7 & \bf{\underline{576.48}} & 29.38 & 
577.32 & 29.58 & 577.50 & 
-0.18\\CON3-8 & 523.14 & 26.98 & 
523.86 & 28.41 & \bf{523.10} & 
0.01\\CON3-9 & 578.98 & 30.78 & 
585.99 & 31.81 & \bf{578.20} & 
0.13\\CON8-0 & 869.43 & 34.57 & 
870.68 & 34.92 & \bf{858.90} & 
1.23\\CON8-1 & \bf{\underline{740.85}} & 34.53 & 
740.85 & 33.74 & 740.90 & 
-0.01\\CON8-2 & \bf{\underline{713.10}} & 50.67 & 
713.47 & 49.14 & 714.30 & 
-0.17\\CON8-3 & \bf{\underline{811.07}} & 35.05 & 
814.38 & 34.88 & 812.30 & 
-0.15\\CON8-4 & 776.37 & 39.09 & 
778.68 & 35.86 & \bf{770.10} & 
0.81\\CON8-5 & \bf{\underline{758.84}} & 33.33 & 
760.70 & 34.62 & 766.60 & 
-1.01\\CON8-6 & \bf{\underline{683.83}} & 39.38 & 
690.57 & 40.04 & 697.20 & 
-1.92\\CON8-7 & \bf{\underline{814.79}} & 28.36 & 
814.95 & 30.33 & 814.80 & 
-0.00\\CON8-8 & 772.11 & 39.95 & 
782.18 & 38.69 & \bf{771.30} & 
0.11\\CON8-9 & \bf{\underline{811.66}} & 41.98 & 
814.23 & 39.05 & 815.10 & 
-0.42\\[1ex]\hline
\end{tabular}
\label{table:nonlin}
\end{table} \clearpage
\begin{table}[ht]
\caption{Resultados de la ejecución de la metaheurística GTS, utilizando instancias de Dethloff con la configuración -mni 10 -lambda1 0.05 -lambda2 0.05 -tabu 20}
\centering
\small
\begin{tabular}{c c c c c c c}
\hline\hline
Instancia & Costo mínimo & Tiempo(seg.) & Costo promedio & Tiempo promedio(seg.) & Costo GTS & \%Gap \\ [0.5ex]
\hline
SCA3-0 & 682.47 & 0.09 & 
682.47 & 0.10 & \bf{636.06} & 
7.30\\SCA3-1 & 701.53 & 0.11 & 
742.89 & 0.10 & \bf{697.84} & 
0.53\\SCA3-2 & 664.91 & 0.09 & 
699.02 & 0.10 & \bf{659.34} & 
0.84\\SCA3-3 & 692.53 & 0.10 & 
694.23 & 0.10 & \bf{680.04} & 
1.84\\SCA3-4 & 719.12 & 0.10 & 
729.99 & 0.10 & \bf{690.50} & 
4.14\\SCA3-5 & 687.15 & 0.09 & 
695.29 & 0.11 & \bf{659.90} & 
4.13\\SCA3-6 & 717.87 & 0.07 & 
724.10 & 0.09 & \bf{651.09} & 
10.26\\SCA3-7 & 672.22 & 0.09 & 
695.07 & 0.10 & \bf{659.17} & 
1.98\\SCA3-8 & 732.24 & 0.10 & 
758.49 & 0.12 & \bf{719.47} & 
1.77\\SCA3-9 & 705.66 & 0.10 & 
713.82 & 0.12 & \bf{681.00} & 
3.62\\SCA8-0 & 1006.40 & 0.11 & 
1033.90 & 0.11 & \bf{961.50} & 
4.67\\SCA8-1 & 1131.92 & 0.09 & 
1141.95 & 0.10 & \bf{1050.20} & 
7.78\\SCA8-2 & 1095.61 & 0.08 & 
1122.75 & 0.10 & \bf{1039.64} & 
5.38\\SCA8-3 & 1043.12 & 0.13 & 
1054.37 & 0.11 & \bf{983.34} & 
6.08\\SCA8-4 & 1095.08 & 0.12 & 
1132.16 & 0.11 & \bf{1065.49} & 
2.78\\SCA8-5 & 1098.82 & 0.10 & 
1139.05 & 0.12 & \bf{1027.08} & 
6.98\\SCA8-6 & 1021.01 & 0.10 & 
1024.08 & 0.10 & \bf{971.82} & 
5.06\\SCA8-7 & 1119.94 & 0.08 & 
1145.67 & 0.09 & \bf{1052.17} & 
6.44\\SCA8-8 & 1092.96 & 0.10 & 
1122.14 & 0.10 & \bf{1071.18} & 
2.03\\SCA8-9 & 1269.84 & 0.10 & 
1280.40 & 0.13 & \bf{1060.50} & 
19.74\\CON3-0 & 632.30 & 0.09 & 
632.30 & 0.10 & \bf{616.52} & 
2.56\\CON3-1 & 579.18 & 0.09 & 
579.18 & 0.09 & \bf{554.47} & 
4.46\\CON3-2 & 532.56 & 0.10 & 
537.10 & 0.12 & \bf{519.26} & 
2.56\\CON3-3 & 633.10 & 0.08 & 
634.35 & 0.10 & \bf{591.19} & 
7.09\\CON3-4 & 614.86 & 0.09 & 
628.71 & 0.12 & \bf{589.32} & 
4.33\\CON3-5 & 591.09 & 0.10 & 
591.48 & 0.10 & \bf{563.70} & 
4.86\\CON3-6 & 516.22 & 0.15 & 
522.10 & 0.12 & \bf{500.80} & 
3.08\\CON3-7 & 607.00 & 0.10 & 
646.01 & 0.10 & \bf{576.48} & 
5.29\\CON3-8 & 530.25 & 0.11 & 
536.54 & 0.11 & \bf{523.05} & 
1.38\\CON3-9 & 588.99 & 0.10 & 
603.58 & 0.09 & \bf{580.05} & 
1.54\\CON8-0 & 917.90 & 0.14 & 
952.49 & 0.12 & \bf{857.17} & 
7.08\\CON8-1 & 799.58 & 0.11 & 
835.64 & 0.10 & \bf{740.85} & 
7.93\\CON8-2 & 734.06 & 0.10 & 
738.63 & 0.10 & \bf{713.44} & 
2.89\\CON8-3 & 853.90 & 0.09 & 
871.97 & 0.10 & \bf{811.07} & 
5.28\\CON8-4 & 792.19 & 0.17 & 
803.46 & 0.13 & \bf{772.25} & 
2.58\\CON8-5 & 771.44 & 0.10 & 
788.59 & 0.09 & \bf{756.91} & 
1.92\\CON8-6 & 710.99 & 0.10 & 
715.90 & 0.10 & \bf{678.92} & 
4.72\\CON8-7 & 821.21 & 0.09 & 
863.18 & 0.09 & \bf{811.96} & 
1.14\\CON8-8 & 806.45 & 0.11 & 
807.41 & 0.10 & \bf{767.53} & 
5.07\\CON8-9 & 850.25 & 0.13 & 
892.40 & 0.14 & \bf{809.00} & 
5.10\\[1ex]\hline
\end{tabular}
\label{table:nonlin}
\end{table} \clearpage
\begin{table}[ht]
\caption{Resultados de la ejecución de la metaheurística GTS, utilizando instancias de Dethloff con la configuración -mni 100 -lambda1 0.05 -lambda2 0.05 -tabu 20}
\centering
\small
\begin{tabular}{c c c c c c c}
\hline\hline
Instancia & Costo mínimo & Tiempo(seg.) & Costo promedio & Tiempo promedio(seg.) & Costo GTS & \%Gap \\ [0.5ex]
\hline
SCA3-0 & 643.15 & 0.13 & 
649.29 & 0.14 & \bf{636.06} & 
1.11\\SCA3-1 & 706.90 & 0.18 & 
717.11 & 0.14 & \bf{697.84} & 
1.30\\SCA3-2 & 664.21 & 0.20 & 
683.20 & 0.17 & \bf{659.34} & 
0.74\\SCA3-3 & 680.55 & 0.13 & 
689.07 & 0.15 & \bf{680.04} & 
0.07\\SCA3-4 & \bf{690.50} & 0.14 & 
703.46 & 0.13 & 690.50 & 0.00\\
SCA3-5 & 687.15 & 0.16 & 
687.15 & 0.14 & \bf{659.90} & 
4.13\\SCA3-6 & 661.11 & 0.12 & 
661.46 & 0.13 & \bf{651.09} & 
1.54\\SCA3-7 & 678.41 & 0.18 & 
678.41 & 0.14 & \bf{659.17} & 
2.92\\SCA3-8 & 724.28 & 0.18 & 
725.36 & 0.18 & \bf{719.47} & 
0.67\\SCA3-9 & 690.83 & 0.12 & 
707.34 & 0.14 & \bf{681.00} & 
1.44\\SCA8-0 & 1000.07 & 0.32 & 
1034.34 & 0.24 & \bf{961.50} & 
4.01\\SCA8-1 & 1067.36 & 0.25 & 
1067.55 & 0.22 & \bf{1050.20} & 
1.63\\SCA8-2 & 1056.68 & 0.12 & 
1068.39 & 0.17 & \bf{1039.64} & 
1.64\\SCA8-3 & 1002.63 & 0.11 & 
1017.23 & 0.20 & \bf{983.34} & 
1.96\\SCA8-4 & 1085.02 & 0.27 & 
1108.24 & 0.21 & \bf{1065.49} & 
1.83\\SCA8-5 & 1072.61 & 0.13 & 
1076.76 & 0.15 & \bf{1027.08} & 
4.43\\SCA8-6 & 993.58 & 0.14 & 
1029.77 & 0.13 & \bf{971.82} & 
2.24\\SCA8-7 & 1126.19 & 0.30 & 
1140.71 & 0.24 & \bf{1052.17} & 
7.03\\SCA8-8 & 1098.55 & 0.19 & 
1101.21 & 0.16 & \bf{1071.18} & 
2.56\\SCA8-9 & 1091.34 & 0.16 & 
1154.58 & 0.17 & \bf{1060.50} & 
2.91\\CON3-0 & 637.04 & 0.13 & 
643.91 & 0.17 & \bf{616.52} & 
3.33\\CON3-1 & 561.04 & 0.17 & 
575.63 & 0.19 & \bf{554.47} & 
1.18\\CON3-2 & 523.88 & 0.21 & 
535.20 & 0.20 & \bf{519.26} & 
0.89\\CON3-3 & 603.42 & 0.16 & 
614.21 & 0.15 & \bf{591.19} & 
2.07\\CON3-4 & 606.56 & 0.26 & 
612.89 & 0.22 & \bf{589.32} & 
2.93\\CON3-5 & \bf{563.70} & 0.17 & 
578.79 & 0.17 & 563.70 & 0.00\\
CON3-6 & 508.05 & 0.14 & 
518.00 & 0.20 & \bf{500.80} & 
1.45\\CON3-7 & 599.52 & 0.12 & 
603.01 & 0.13 & \bf{576.48} & 
4.00\\CON3-8 & 523.68 & 0.22 & 
549.98 & 0.17 & \bf{523.05} & 
0.12\\CON3-9 & 588.63 & 0.19 & 
590.20 & 0.19 & \bf{580.05} & 
1.48\\CON8-0 & 903.40 & 0.17 & 
918.80 & 0.24 & \bf{857.17} & 
5.39\\CON8-1 & 776.83 & 0.20 & 
794.66 & 0.26 & \bf{740.85} & 
4.86\\CON8-2 & 782.48 & 0.13 & 
797.30 & 0.14 & \bf{713.44} & 
9.68\\CON8-3 & 839.79 & 0.20 & 
843.74 & 0.18 & \bf{811.07} & 
3.54\\CON8-4 & 808.61 & 0.15 & 
829.08 & 0.14 & \bf{772.25} & 
4.71\\CON8-5 & 780.17 & 0.14 & 
805.59 & 0.16 & \bf{756.91} & 
3.07\\CON8-6 & 702.76 & 0.17 & 
715.16 & 0.23 & \bf{678.92} & 
3.51\\CON8-7 & 844.31 & 0.18 & 
853.59 & 0.20 & \bf{811.96} & 
3.98\\CON8-8 & 782.07 & 0.19 & 
787.93 & 0.18 & \bf{767.53} & 
1.89\\CON8-9 & 849.70 & 0.22 & 
893.31 & 0.17 & \bf{809.00} & 
5.03\\[1ex]\hline
\end{tabular}
\label{table:nonlin}
\end{table} \clearpage
\begin{table}[ht]
\caption{Resultados de la ejecución de la metaheurística GTS, utilizando instancias de Dethloff con la configuración -mni 1000 -lambda1 0.05 -lambda2 0.05 -tabu 20}
\centering
\small
\begin{tabular}{c c c c c c c}
\hline\hline
Instancia & Costo mínimo & Tiempo(seg.) & Costo promedio & Tiempo promedio(seg.) & Costo GTS & \%Gap \\ [0.5ex]
\hline
SCA3-0 & 640.55 & 0.49 & 
650.97 & 0.37 & \bf{636.06} & 
0.71\\SCA3-1 & \bf{697.84} & 0.46 & 
700.12 & 0.55 & 697.84 & 0.00\\
SCA3-2 & \bf{659.34} & 0.54 & 
666.15 & 0.65 & 659.34 & 0.00\\
SCA3-3 & \bf{680.04} & 0.86 & 
680.73 & 0.84 & 680.04 & 0.00\\
SCA3-4 & \bf{690.50} & 0.59 & 
696.64 & 0.94 & 690.50 & 0.00\\
SCA3-5 & \bf{659.90} & 1.70 & 
678.63 & 0.86 & 659.90 & 0.00\\
SCA3-6 & \bf{651.09} & 0.40 & 
659.40 & 0.59 & 651.09 & 0.00\\
SCA3-7 & 666.15 & 0.60 & 
668.91 & 0.49 & \bf{659.17} & 
1.06\\SCA3-8 & 729.18 & 0.66 & 
743.97 & 0.62 & \bf{719.47} & 
1.35\\SCA3-9 & \bf{681.00} & 0.90 & 
685.48 & 0.61 & 681.00 & 0.00\\
SCA8-0 & 987.26 & 0.46 & 
1007.88 & 0.58 & \bf{961.50} & 
2.68\\SCA8-1 & 1073.00 & 0.38 & 
1087.48 & 0.73 & \bf{1050.20} & 
2.17\\SCA8-2 & 1049.22 & 1.08 & 
1065.47 & 0.94 & \bf{1039.64} & 
0.92\\SCA8-3 & 1014.45 & 1.30 & 
1019.00 & 0.83 & \bf{983.34} & 
3.16\\SCA8-4 & 1067.55 & 1.65 & 
1075.79 & 1.14 & \bf{1065.49} & 
0.19\\SCA8-5 & \bf{1027.08} & 1.19 & 
1056.18 & 0.81 & 1027.08 & 0.00\\
SCA8-6 & 972.48 & 2.41 & 
977.98 & 1.16 & \bf{971.82} & 
0.07\\SCA8-7 & 1066.65 & 1.84 & 
1093.18 & 1.31 & \bf{1052.17} & 
1.38\\SCA8-8 & 1082.91 & 0.72 & 
1105.61 & 0.52 & \bf{1071.18} & 
1.10\\SCA8-9 & 1063.68 & 0.52 & 
1068.78 & 0.87 & \bf{1060.50} & 
0.30\\CON3-0 & 617.59 & 0.77 & 
631.26 & 0.62 & \bf{616.52} & 
0.17\\CON3-1 & \bf{554.47} & 0.68 & 
558.87 & 1.04 & 554.47 & 0.00\\
CON3-2 & 524.58 & 0.58 & 
530.72 & 0.70 & \bf{519.26} & 
1.02\\CON3-3 & \bf{591.19} & 0.68 & 
602.26 & 0.73 & 591.19 & 0.00\\
CON3-4 & 591.43 & 0.58 & 
600.26 & 0.78 & \bf{589.32} & 
0.36\\CON3-5 & \bf{563.70} & 0.86 & 
565.92 & 0.74 & 563.70 & 0.00\\
CON3-6 & \bf{\underline{499.05}} & 1.66 & 
503.43 & 0.87 & 500.80 & 
-0.35\\CON3-7 & 591.91 & 1.12 & 
599.19 & 1.01 & \bf{576.48} & 
2.68\\CON3-8 & \bf{523.05} & 1.25 & 
523.37 & 0.77 & 523.05 & 0.00\\
CON3-9 & 582.79 & 0.89 & 
587.71 & 0.58 & \bf{580.05} & 
0.47\\CON8-0 & 866.19 & 0.81 & 
896.81 & 0.88 & \bf{857.17} & 
1.05\\CON8-1 & 751.76 & 0.88 & 
756.31 & 0.82 & \bf{740.85} & 
1.47\\CON8-2 & 718.70 & 1.25 & 
726.91 & 1.27 & \bf{713.44} & 
0.74\\CON8-3 & 812.32 & 0.44 & 
830.94 & 0.57 & \bf{811.07} & 
0.15\\CON8-4 & 787.30 & 0.93 & 
837.58 & 0.67 & \bf{772.25} & 
1.95\\CON8-5 & 760.91 & 0.59 & 
773.11 & 0.78 & \bf{756.91} & 
0.53\\CON8-6 & 708.46 & 0.37 & 
711.25 & 1.04 & \bf{678.92} & 
4.35\\CON8-7 & 825.16 & 0.66 & 
829.72 & 0.81 & \bf{811.96} & 
1.63\\CON8-8 & 771.32 & 1.05 & 
787.22 & 1.09 & \bf{767.53} & 
0.49\\CON8-9 & \bf{809.00} & 0.83 & 
833.62 & 0.99 & 809.00 & 0.00\\
[1ex]\hline
\end{tabular}
\label{table:nonlin}
\end{table} \clearpage
\begin{table}[ht]
\caption{Resultados de la ejecución de la metaheurística GTS, utilizando instancias de Dethloff con la configuración -mni 2000 -lambda1 0.05 -lambda2 0.05 -tabu 20}
\centering
\small
\begin{tabular}{c c c c c c c}
\hline\hline
Instancia & Costo mínimo & Tiempo(seg.) & Costo promedio & Tiempo promedio(seg.) & Costo GTS & \%Gap \\ [0.5ex]
\hline
SCA3-0 & \bf{636.06} & 1.87 & 
637.18 & 1.21 & 636.06 & 0.00\\
SCA3-1 & \bf{697.84} & 1.31 & 
700.11 & 1.14 & 697.84 & 0.00\\
SCA3-2 & \bf{659.34} & 0.90 & 
659.34 & 1.36 & 659.34 & 0.00\\
SCA3-3 & \bf{680.04} & 2.62 & 
680.46 & 1.86 & 680.04 & 0.00\\
SCA3-4 & \bf{690.50} & 1.50 & 
690.50 & 1.45 & 690.50 & 0.00\\
SCA3-5 & \bf{659.90} & 0.60 & 
669.79 & 1.52 & 659.90 & 0.00\\
SCA3-6 & \bf{651.09} & 1.75 & 
656.79 & 1.44 & 651.09 & 0.00\\
SCA3-7 & 666.15 & 1.98 & 
666.15 & 1.58 & \bf{659.17} & 
1.06\\SCA3-8 & \bf{719.47} & 1.20 & 
728.43 & 1.00 & 719.47 & 0.00\\
SCA3-9 & \bf{681.00} & 0.62 & 
683.29 & 0.84 & 681.00 & 0.00\\
SCA8-0 & \bf{961.50} & 3.46 & 
973.28 & 2.76 & 961.50 & 0.00\\
SCA8-1 & 1050.72 & 1.23 & 
1065.87 & 1.49 & \bf{1050.20} & 
0.05\\SCA8-2 & \bf{1039.64} & 2.66 & 
1048.19 & 1.72 & 1039.64 & 0.00\\
SCA8-3 & \bf{983.34} & 2.50 & 
1010.34 & 1.72 & 983.34 & 0.00\\
SCA8-4 & 1067.55 & 2.56 & 
1081.20 & 1.94 & \bf{1065.49} & 
0.19\\SCA8-5 & 1047.55 & 2.32 & 
1057.59 & 2.02 & \bf{1027.08} & 
1.99\\SCA8-6 & 972.48 & 1.16 & 
982.70 & 0.96 & \bf{971.82} & 
0.07\\SCA8-7 & 1074.08 & 0.70 & 
1078.83 & 1.65 & \bf{1052.17} & 
2.08\\SCA8-8 & \bf{1071.18} & 1.23 & 
1073.53 & 1.24 & 1071.18 & 0.00\\
SCA8-9 & 1064.87 & 1.10 & 
1071.71 & 1.88 & \bf{1060.50} & 
0.41\\CON3-0 & \bf{616.52} & 1.80 & 
626.34 & 1.42 & 616.52 & 0.00\\
CON3-1 & 557.21 & 1.12 & 
558.98 & 1.41 & \bf{554.47} & 
0.49\\CON3-2 & 521.38 & 0.89 & 
523.39 & 1.75 & \bf{519.26} & 
0.41\\CON3-3 & 591.20 & 2.38 & 
613.88 & 1.28 & \bf{591.19} & 
0.00\\CON3-4 & 591.43 & 1.04 & 
601.17 & 1.37 & \bf{589.32} & 
0.36\\CON3-5 & \bf{563.70} & 1.52 & 
569.79 & 2.43 & 563.70 & 0.00\\
CON3-6 & \bf{\underline{499.05}} & 0.72 & 
501.89 & 1.62 & 500.80 & 
-0.35\\CON3-7 & \bf{576.48} & 3.84 & 
576.48 & 2.81 & 576.48 & 0.00\\
CON3-8 & \bf{523.05} & 0.96 & 
523.21 & 1.26 & 523.05 & 0.00\\
CON3-9 & 588.63 & 0.63 & 
598.21 & 1.48 & \bf{580.05} & 
1.48\\CON8-0 & 858.16 & 1.11 & 
896.03 & 1.44 & \bf{857.17} & 
0.12\\CON8-1 & 757.69 & 2.10 & 
760.33 & 1.73 & \bf{740.85} & 
2.27\\CON8-2 & \bf{\underline{712.94}} & 3.15 & 
725.25 & 2.29 & 713.44 & 
-0.07\\CON8-3 & 821.26 & 1.13 & 
836.00 & 2.38 & \bf{811.07} & 
1.26\\CON8-4 & \bf{772.25} & 1.77 & 
775.83 & 1.49 & 772.25 & 0.00\\
CON8-5 & 758.84 & 1.38 & 
760.30 & 1.46 & \bf{756.91} & 
0.25\\CON8-6 & 695.86 & 1.00 & 
702.55 & 1.96 & \bf{678.92} & 
2.50\\CON8-7 & 812.89 & 4.82 & 
817.07 & 2.38 & \bf{811.96} & 
0.11\\CON8-8 & 776.55 & 1.28 & 
782.13 & 2.26 & \bf{767.53} & 
1.18\\CON8-9 & \bf{809.00} & 2.58 & 
812.91 & 2.05 & 809.00 & 0.00\\
[1ex]\hline
\end{tabular}
\label{table:nonlin}
\end{table} \clearpage
\begin{table}[ht]
\caption{Resultados de la ejecución de la metaheurística GTS, utilizando instancias de Dethloff con la configuración -mni 3000 -lambda1 0.05 -lambda2 0.05 -tabu 20}
\centering
\small
\begin{tabular}{c c c c c c c}
\hline\hline
Instancia & Costo mínimo & Tiempo(seg.) & Costo promedio & Tiempo promedio(seg.) & Costo GTS & \%Gap \\ [0.5ex]
\hline
SCA3-0 & \bf{636.06} & 1.37 & 
638.30 & 1.86 & 636.06 & 0.00\\
SCA3-1 & \bf{697.84} & 2.38 & 
700.11 & 2.50 & 697.84 & 0.00\\
SCA3-2 & \bf{659.34} & 1.83 & 
666.01 & 2.28 & 659.34 & 0.00\\
SCA3-3 & 690.90 & 1.27 & 
691.31 & 2.84 & \bf{680.04} & 
1.60\\SCA3-4 & \bf{690.50} & 1.25 & 
690.50 & 2.32 & 690.50 & 0.00\\
SCA3-5 & \bf{659.90} & 2.76 & 
669.80 & 2.51 & 659.90 & 0.00\\
SCA3-6 & \bf{651.09} & 6.82 & 
655.40 & 2.88 & 651.09 & 0.00\\
SCA3-7 & 666.15 & 3.35 & 
669.22 & 1.92 & \bf{659.17} & 
1.06\\SCA3-8 & \bf{719.47} & 1.34 & 
720.59 & 2.94 & 719.47 & 0.00\\
SCA3-9 & \bf{681.00} & 3.05 & 
681.00 & 2.56 & 681.00 & 0.00\\
SCA8-0 & \bf{961.50} & 1.91 & 
982.58 & 1.51 & 961.50 & 0.00\\
SCA8-1 & \bf{\underline{1049.65}} & 2.89 & 
1071.59 & 2.17 & 1050.20 & 
-0.05\\SCA8-2 & 1042.17 & 1.25 & 
1052.93 & 1.60 & \bf{1039.64} & 
0.24\\SCA8-3 & \bf{983.34} & 2.04 & 
998.47 & 2.46 & 983.34 & 0.00\\
SCA8-4 & 1067.28 & 3.06 & 
1075.71 & 2.44 & \bf{1065.49} & 
0.17\\SCA8-5 & 1040.87 & 1.60 & 
1048.24 & 2.46 & \bf{1027.08} & 
1.34\\SCA8-6 & 972.48 & 1.90 & 
976.62 & 2.10 & \bf{971.82} & 
0.07\\SCA8-7 & \bf{\underline{1052.04}} & 2.02 & 
1083.87 & 1.86 & 1052.17 & 
-0.01\\SCA8-8 & 1082.12 & 2.80 & 
1082.99 & 3.44 & \bf{1071.18} & 
1.02\\SCA8-9 & \bf{1060.50} & 3.75 & 
1060.50 & 3.55 & 1060.50 & 0.00\\
CON3-0 & 628.47 & 2.57 & 
633.11 & 2.39 & \bf{616.52} & 
1.94\\CON3-1 & 556.79 & 1.56 & 
557.67 & 2.80 & \bf{554.47} & 
0.42\\CON3-2 & 522.86 & 2.47 & 
523.48 & 2.33 & \bf{519.26} & 
0.69\\CON3-3 & \bf{591.19} & 2.85 & 
591.19 & 2.62 & 591.19 & 0.00\\
CON3-4 & 591.43 & 4.17 & 
596.26 & 2.58 & \bf{589.32} & 
0.36\\CON3-5 & \bf{563.70} & 2.70 & 
566.19 & 1.80 & 563.70 & 0.00\\
CON3-6 & \bf{\underline{499.05}} & 1.50 & 
501.79 & 2.28 & 500.80 & 
-0.35\\CON3-7 & \bf{576.48} & 2.04 & 
584.70 & 2.26 & 576.48 & 0.00\\
CON3-8 & \bf{523.05} & 1.55 & 
523.05 & 2.33 & 523.05 & 0.00\\
CON3-9 & \bf{\underline{578.25}} & 4.82 & 
586.15 & 2.65 & 580.05 & 
-0.31\\CON8-0 & 857.40 & 3.03 & 
868.68 & 2.90 & \bf{857.17} & 
0.03\\CON8-1 & \bf{740.85} & 2.57 & 
744.98 & 2.18 & 740.85 & 0.00\\
CON8-2 & \bf{\underline{712.89}} & 3.94 & 
725.11 & 2.13 & 713.44 & 
-0.08\\CON8-3 & \bf{811.07} & 3.46 & 
821.55 & 2.83 & 811.07 & 0.00\\
CON8-4 & \bf{772.25} & 3.46 & 
781.68 & 2.50 & 772.25 & 0.00\\
CON8-5 & 759.93 & 1.77 & 
762.10 & 2.52 & \bf{756.91} & 
0.40\\CON8-6 & 685.49 & 6.12 & 
689.83 & 3.24 & \bf{678.92} & 
0.97\\CON8-7 & 814.50 & 1.60 & 
826.54 & 2.18 & \bf{811.96} & 
0.31\\CON8-8 & \bf{767.53} & 2.08 & 
778.72 & 2.21 & 767.53 & 0.00\\
CON8-9 & \bf{809.00} & 3.10 & 
812.00 & 2.79 & 809.00 & 0.00\\
[1ex]\hline
\end{tabular}
\label{table:nonlin}
\end{table} \clearpage
\begin{table}[ht]
\caption{Resultados de la ejecución de la metaheurística GTS, utilizando instancias de Dethloff con la configuración -mni 4000 -lambda1 0.05 -lambda2 0.05 -tabu 20}
\centering
\small
\begin{tabular}{c c c c c c c}
\hline\hline
Instancia & Costo mínimo & Tiempo(seg.) & Costo promedio & Tiempo promedio(seg.) & Costo GTS & \%Gap \\ [0.5ex]
\hline
SCA3-0 & \bf{636.06} & 2.33 & 
638.30 & 3.13 & 636.06 & 0.00\\
SCA3-1 & \bf{697.84} & 1.46 & 
698.50 & 2.68 & 697.84 & 0.00\\
SCA3-2 & \bf{659.34} & 2.59 & 
659.34 & 3.04 & 659.34 & 0.00\\
SCA3-3 & \bf{680.04} & 1.55 & 
685.34 & 1.73 & 680.04 & 0.00\\
SCA3-4 & \bf{690.50} & 3.96 & 
696.75 & 3.29 & 690.50 & 0.00\\
SCA3-5 & \bf{659.90} & 2.52 & 
663.16 & 4.43 & 659.90 & 0.00\\
SCA3-6 & \bf{651.09} & 1.84 & 
651.74 & 3.02 & 651.09 & 0.00\\
SCA3-7 & 666.15 & 3.01 & 
669.40 & 3.63 & \bf{659.17} & 
1.06\\SCA3-8 & \bf{719.47} & 2.36 & 
721.90 & 3.50 & 719.47 & 0.00\\
SCA3-9 & \bf{681.00} & 2.21 & 
681.00 & 2.94 & 681.00 & 0.00\\
SCA8-0 & \bf{961.50} & 2.81 & 
977.97 & 3.68 & 961.50 & 0.00\\
SCA8-1 & \bf{1050.20} & 2.76 & 
1063.35 & 2.79 & 1050.20 & 0.00\\
SCA8-2 & \bf{1039.64} & 1.60 & 
1049.21 & 2.82 & 1039.64 & 0.00\\
SCA8-3 & 1005.65 & 3.25 & 
1011.34 & 2.42 & \bf{983.34} & 
2.27\\SCA8-4 & 1067.55 & 2.86 & 
1079.83 & 2.10 & \bf{1065.49} & 
0.19\\SCA8-5 & 1039.64 & 3.22 & 
1056.53 & 3.73 & \bf{1027.08} & 
1.22\\SCA8-6 & \bf{971.82} & 1.23 & 
985.94 & 3.33 & 971.82 & 0.00\\
SCA8-7 & 1066.65 & 1.88 & 
1067.09 & 3.08 & \bf{1052.17} & 
1.38\\SCA8-8 & \bf{1071.18} & 1.28 & 
1077.22 & 2.29 & 1071.18 & 0.00\\
SCA8-9 & 1063.68 & 4.82 & 
1066.09 & 4.55 & \bf{1060.50} & 
0.30\\CON3-0 & \bf{616.52} & 3.06 & 
620.86 & 2.73 & 616.52 & 0.00\\
CON3-1 & 556.04 & 2.36 & 
558.36 & 2.44 & \bf{554.47} & 
0.28\\CON3-2 & \bf{\underline{518.00}} & 2.62 & 
522.15 & 3.86 & 519.26 & 
-0.24\\CON3-3 & \bf{591.19} & 4.10 & 
591.19 & 4.04 & 591.19 & 0.00\\
CON3-4 & 589.88 & 2.99 & 
597.50 & 3.54 & \bf{589.32} & 
0.10\\CON3-5 & \bf{563.70} & 1.28 & 
569.62 & 2.46 & 563.70 & 0.00\\
CON3-6 & \bf{\underline{499.05}} & 2.73 & 
500.04 & 3.12 & 500.80 & 
-0.35\\CON3-7 & \bf{576.48} & 2.52 & 
577.71 & 2.21 & 576.48 & 0.00\\
CON3-8 & \bf{523.05} & 3.98 & 
523.05 & 3.20 & 523.05 & 0.00\\
CON3-9 & 582.79 & 1.60 & 
584.12 & 3.01 & \bf{580.05} & 
0.47\\CON8-0 & \bf{857.17} & 4.98 & 
868.37 & 3.76 & 857.17 & 0.00\\
CON8-1 & \bf{740.85} & 4.52 & 
767.71 & 2.85 & 740.85 & 0.00\\
CON8-2 & 724.79 & 3.41 & 
730.59 & 2.84 & \bf{713.44} & 
1.59\\CON8-3 & \bf{811.07} & 7.84 & 
825.96 & 4.27 & 811.07 & 0.00\\
CON8-4 & \bf{772.25} & 1.67 & 
779.76 & 2.60 & 772.25 & 0.00\\
CON8-5 & \bf{756.91} & 1.88 & 
761.23 & 3.48 & 756.91 & 0.00\\
CON8-6 & 691.54 & 4.20 & 
695.74 & 3.60 & \bf{678.92} & 
1.86\\CON8-7 & 813.20 & 2.53 & 
813.96 & 3.31 & \bf{811.96} & 
0.15\\CON8-8 & 776.55 & 4.50 & 
781.49 & 3.19 & \bf{767.53} & 
1.18\\CON8-9 & \bf{809.00} & 3.48 & 
812.56 & 3.60 & 809.00 & 0.00\\
[1ex]\hline
\end{tabular}
\label{table:nonlin}
\end{table} \clearpage
\begin{table}[ht]
\caption{Resultados de la ejecución de la metaheurística GTS, utilizando instancias de Dethloff con la configuración -mni 5000 -lambda1 0.05 -lambda2 0.05 -tabu 20}
\centering
\small
\begin{tabular}{c c c c c c c}
\hline\hline
Instancia & Costo mínimo & Tiempo(seg.) & Costo promedio & Tiempo promedio(seg.) & Costo GTS & \%Gap \\ [0.5ex]
\hline
SCA3-0 & 639.22 & 3.00 & 
640.22 & 2.97 & \bf{636.06} & 
0.50\\SCA3-1 & \bf{697.84} & 2.13 & 
697.84 & 3.46 & 697.84 & 0.00\\
SCA3-2 & \bf{659.34} & 6.70 & 
659.34 & 5.79 & 659.34 & 0.00\\
SCA3-3 & \bf{680.04} & 5.19 & 
683.03 & 2.92 & 680.04 & 0.00\\
SCA3-4 & \bf{690.50} & 5.40 & 
690.50 & 5.67 & 690.50 & 0.00\\
SCA3-5 & \bf{659.90} & 3.65 & 
666.42 & 2.99 & 659.90 & 0.00\\
SCA3-6 & \bf{651.09} & 2.70 & 
651.09 & 4.64 & 651.09 & 0.00\\
SCA3-7 & 666.15 & 4.48 & 
667.09 & 3.57 & \bf{659.17} & 
1.06\\SCA3-8 & \bf{719.47} & 4.77 & 
721.90 & 4.18 & 719.47 & 0.00\\
SCA3-9 & \bf{681.00} & 2.77 & 
681.00 & 2.69 & 681.00 & 0.00\\
SCA8-0 & 970.64 & 2.26 & 
982.30 & 3.52 & \bf{961.50} & 
0.95\\SCA8-1 & 1067.45 & 7.10 & 
1070.03 & 4.21 & \bf{1050.20} & 
1.64\\SCA8-2 & 1050.17 & 2.33 & 
1050.32 & 3.34 & \bf{1039.64} & 
1.01\\SCA8-3 & \bf{983.34} & 12.05 & 
998.29 & 6.35 & 983.34 & 0.00\\
SCA8-4 & 1067.28 & 4.00 & 
1069.18 & 4.57 & \bf{1065.49} & 
0.17\\SCA8-5 & \bf{1027.08} & 4.41 & 
1047.91 & 4.53 & 1027.08 & 0.00\\
SCA8-6 & 972.48 & 2.54 & 
972.48 & 2.63 & \bf{971.82} & 
0.07\\SCA8-7 & 1084.27 & 7.78 & 
1090.52 & 4.17 & \bf{1052.17} & 
3.05\\SCA8-8 & \bf{1071.18} & 5.83 & 
1079.38 & 4.61 & 1071.18 & 0.00\\
SCA8-9 & \bf{1060.50} & 4.96 & 
1067.34 & 4.69 & 1060.50 & 0.00\\
CON3-0 & \bf{616.52} & 5.52 & 
624.56 & 5.88 & 616.52 & 0.00\\
CON3-1 & 557.21 & 6.81 & 
559.87 & 4.50 & \bf{554.47} & 
0.49\\CON3-2 & \bf{\underline{518.00}} & 2.74 & 
521.46 & 3.16 & 519.26 & 
-0.24\\CON3-3 & \bf{591.19} & 3.93 & 
591.19 & 6.27 & 591.19 & 0.00\\
CON3-4 & \bf{\underline{588.79}} & 3.50 & 
593.75 & 2.37 & 589.32 & 
-0.09\\CON3-5 & \bf{563.70} & 9.39 & 
565.92 & 6.43 & 563.70 & 0.00\\
CON3-6 & \bf{\underline{499.05}} & 6.42 & 
501.04 & 4.90 & 500.80 & 
-0.35\\CON3-7 & \bf{576.48} & 4.72 & 
577.40 & 3.70 & 576.48 & 0.00\\
CON3-8 & \bf{523.05} & 2.03 & 
523.37 & 3.27 & 523.05 & 0.00\\
CON3-9 & \bf{\underline{578.25}} & 4.26 & 
585.55 & 4.71 & 580.05 & 
-0.31\\CON8-0 & \bf{857.17} & 7.85 & 
889.74 & 4.06 & 857.17 & 0.00\\
CON8-1 & \bf{740.85} & 5.16 & 
749.59 & 3.91 & 740.85 & 0.00\\
CON8-2 & 718.64 & 5.57 & 
723.88 & 3.85 & \bf{713.44} & 
0.73\\CON8-3 & 821.26 & 3.23 & 
826.58 & 3.19 & \bf{811.07} & 
1.26\\CON8-4 & \bf{772.25} & 2.60 & 
777.61 & 3.30 & 772.25 & 0.00\\
CON8-5 & \bf{\underline{754.88}} & 4.77 & 
756.91 & 3.57 & 756.91 & 
-0.27\\CON8-6 & 684.69 & 10.52 & 
691.56 & 5.36 & \bf{678.92} & 
0.85\\CON8-7 & 814.79 & 2.66 & 
836.00 & 2.81 & \bf{811.96} & 
0.35\\CON8-8 & \bf{767.53} & 2.35 & 
778.71 & 3.02 & 767.53 & 0.00\\
CON8-9 & 811.16 & 4.45 & 
819.82 & 5.63 & \bf{809.00} & 
0.27\\[1ex]\hline
\end{tabular}
\label{table:nonlin}
\end{table} \clearpage
\begin{table}[ht]
\caption{Resultados de la ejecución de la metaheurística GTS, utilizando instancias de Dethloff con la configuración -mni 6000 -lambda1 0.05 -lambda2 0.05 -tabu 20}
\centering
\small
\begin{tabular}{c c c c c c c}
\hline\hline
Instancia & Costo mínimo & Tiempo(seg.) & Costo promedio & Tiempo promedio(seg.) & Costo GTS & \%Gap \\ [0.5ex]
\hline
SCA3-0 & \bf{\underline{635.67}} & 4.00 & 
638.21 & 3.84 & 636.06 & 
-0.06\\SCA3-1 & \bf{697.84} & 4.25 & 
697.84 & 4.21 & 697.84 & 0.00\\
SCA3-2 & \bf{659.34} & 3.53 & 
659.34 & 4.91 & 659.34 & 0.00\\
SCA3-3 & \bf{680.04} & 3.27 & 
682.94 & 4.18 & 680.04 & 0.00\\
SCA3-4 & \bf{690.50} & 6.28 & 
690.50 & 5.56 & 690.50 & 0.00\\
SCA3-5 & \bf{659.90} & 6.18 & 
659.90 & 7.24 & 659.90 & 0.00\\
SCA3-6 & \bf{651.09} & 5.75 & 
651.09 & 5.31 & 651.09 & 0.00\\
SCA3-7 & \bf{659.17} & 6.80 & 
664.40 & 5.93 & 659.17 & 0.00\\
SCA3-8 & \bf{719.47} & 7.80 & 
719.47 & 6.28 & 719.47 & 0.00\\
SCA3-9 & \bf{681.00} & 4.87 & 
681.00 & 4.48 & 681.00 & 0.00\\
SCA8-0 & \bf{961.50} & 3.97 & 
966.04 & 3.61 & 961.50 & 0.00\\
SCA8-1 & 1050.26 & 6.14 & 
1063.48 & 3.40 & \bf{1050.20} & 
0.01\\SCA8-2 & \bf{1039.64} & 7.01 & 
1050.02 & 5.63 & 1039.64 & 0.00\\
SCA8-3 & \bf{983.34} & 3.67 & 
996.27 & 9.07 & 983.34 & 0.00\\
SCA8-4 & \bf{1065.49} & 4.31 & 
1068.77 & 4.26 & 1065.49 & 0.00\\
SCA8-5 & \bf{1027.08} & 2.94 & 
1036.87 & 4.48 & 1027.08 & 0.00\\
SCA8-6 & \bf{971.82} & 2.27 & 
976.78 & 4.56 & 971.82 & 0.00\\
SCA8-7 & \bf{\underline{1051.28}} & 9.90 & 
1065.82 & 6.52 & 1052.17 & 
-0.08\\SCA8-8 & \bf{1071.18} & 2.18 & 
1076.65 & 4.42 & 1071.18 & 0.00\\
SCA8-9 & \bf{1060.50} & 3.80 & 
1060.50 & 6.41 & 1060.50 & 0.00\\
CON3-0 & \bf{616.52} & 6.90 & 
622.61 & 5.61 & 616.52 & 0.00\\
CON3-1 & \bf{554.47} & 7.86 & 
555.55 & 4.87 & 554.47 & 0.00\\
CON3-2 & \bf{\underline{519.11}} & 3.65 & 
522.41 & 4.37 & 519.26 & 
-0.03\\CON3-3 & \bf{591.19} & 8.44 & 
591.19 & 5.82 & 591.19 & 0.00\\
CON3-4 & \bf{\underline{588.79}} & 4.14 & 
596.02 & 3.26 & 589.32 & 
-0.09\\CON3-5 & \bf{563.70} & 9.44 & 
569.25 & 6.11 & 563.70 & 0.00\\
CON3-6 & \bf{\underline{499.05}} & 9.23 & 
499.49 & 7.12 & 500.80 & 
-0.35\\CON3-7 & \bf{576.48} & 5.15 & 
580.82 & 4.86 & 576.48 & 0.00\\
CON3-8 & \bf{523.05} & 4.16 & 
523.05 & 3.97 & 523.05 & 0.00\\
CON3-9 & \bf{\underline{578.25}} & 8.25 & 
582.99 & 6.81 & 580.05 & 
-0.31\\CON8-0 & \bf{857.17} & 5.64 & 
870.10 & 4.66 & 857.17 & 0.00\\
CON8-1 & 751.76 & 4.56 & 
764.48 & 5.84 & \bf{740.85} & 
1.47\\CON8-2 & \bf{713.44} & 5.08 & 
722.77 & 3.72 & 713.44 & 0.00\\
CON8-3 & 811.23 & 7.46 & 
822.74 & 5.00 & \bf{811.07} & 
0.02\\CON8-4 & \bf{772.25} & 2.72 & 
772.25 & 3.46 & 772.25 & 0.00\\
CON8-5 & \bf{\underline{754.88}} & 5.24 & 
758.43 & 4.03 & 756.91 & 
-0.27\\CON8-6 & 690.51 & 3.43 & 
692.78 & 4.13 & \bf{678.92} & 
1.71\\CON8-7 & 812.89 & 3.17 & 
813.70 & 3.42 & \bf{811.96} & 
0.11\\CON8-8 & \bf{767.53} & 2.80 & 
779.40 & 3.43 & 767.53 & 0.00\\
CON8-9 & 812.44 & 5.62 & 
840.59 & 4.89 & \bf{809.00} & 
0.43\\[1ex]\hline
\end{tabular}
\label{table:nonlin}
\end{table} \clearpage
\begin{table}[ht]
\caption{Resultados de la ejecución de la metaheurística GTS, utilizando instancias de Dethloff con la configuración -mni 7000 -lambda1 0.05 -lambda2 0.05 -tabu 20}
\centering
\small
\begin{tabular}{c c c c c c c}
\hline\hline
Instancia & Costo mínimo & Tiempo(seg.) & Costo promedio & Tiempo promedio(seg.) & Costo GTS & \%Gap \\ [0.5ex]
\hline
SCA3-0 & \bf{636.06} & 6.97 & 
638.30 & 5.07 & 636.06 & 0.00\\
SCA3-1 & \bf{697.84} & 3.77 & 
697.84 & 7.06 & 697.84 & 0.00\\
SCA3-2 & \bf{659.34} & 5.72 & 
659.34 & 4.75 & 659.34 & 0.00\\
SCA3-3 & \bf{680.04} & 3.60 & 
682.75 & 5.20 & 680.04 & 0.00\\
SCA3-4 & \bf{690.50} & 7.48 & 
690.50 & 7.39 & 690.50 & 0.00\\
SCA3-5 & \bf{659.90} & 11.75 & 
663.16 & 6.20 & 659.90 & 0.00\\
SCA3-6 & \bf{651.09} & 3.09 & 
651.09 & 5.79 & 651.09 & 0.00\\
SCA3-7 & 666.15 & 4.60 & 
666.26 & 6.17 & \bf{659.17} & 
1.06\\SCA3-8 & \bf{719.47} & 7.64 & 
719.47 & 6.79 & 719.47 & 0.00\\
SCA3-9 & \bf{681.00} & 5.86 & 
681.00 & 5.71 & 681.00 & 0.00\\
SCA8-0 & \bf{961.50} & 6.02 & 
972.05 & 5.11 & 961.50 & 0.00\\
SCA8-1 & \bf{\underline{1049.65}} & 7.50 & 
1054.38 & 8.34 & 1050.20 & 
-0.05\\SCA8-2 & \bf{1039.64} & 5.25 & 
1050.38 & 5.51 & 1039.64 & 0.00\\
SCA8-3 & 1005.65 & 4.10 & 
1010.15 & 3.69 & \bf{983.34} & 
2.27\\SCA8-4 & \bf{1065.49} & 3.56 & 
1069.67 & 6.00 & 1065.49 & 0.00\\
SCA8-5 & \bf{1027.08} & 5.72 & 
1037.69 & 5.22 & 1027.08 & 0.00\\
SCA8-6 & 972.48 & 4.41 & 
981.16 & 4.13 & \bf{971.82} & 
0.07\\SCA8-7 & \bf{1052.17} & 5.40 & 
1074.73 & 4.62 & 1052.17 & 0.00\\
SCA8-8 & \bf{1071.18} & 3.98 & 
1077.60 & 4.96 & 1071.18 & 0.00\\
SCA8-9 & \bf{1060.50} & 4.10 & 
1063.27 & 4.27 & 1060.50 & 0.00\\
CON3-0 & \bf{616.52} & 3.74 & 
625.28 & 5.43 & 616.52 & 0.00\\
CON3-1 & 556.04 & 4.05 & 
557.48 & 5.25 & \bf{554.47} & 
0.28\\CON3-2 & 522.86 & 5.18 & 
523.18 & 6.54 & \bf{519.26} & 
0.69\\CON3-3 & \bf{591.19} & 9.65 & 
591.92 & 7.88 & 591.19 & 0.00\\
CON3-4 & \bf{\underline{588.79}} & 5.66 & 
593.52 & 5.13 & 589.32 & 
-0.09\\CON3-5 & \bf{563.70} & 11.51 & 
563.70 & 8.00 & 563.70 & 0.00\\
CON3-6 & \bf{\underline{499.05}} & 15.22 & 
499.83 & 8.61 & 500.80 & 
-0.35\\CON3-7 & \bf{576.48} & 4.86 & 
585.30 & 4.82 & 576.48 & 0.00\\
CON3-8 & \bf{523.05} & 6.03 & 
523.05 & 5.29 & 523.05 & 0.00\\
CON3-9 & \bf{\underline{578.25}} & 7.53 & 
581.21 & 6.17 & 580.05 & 
-0.31\\CON8-0 & 857.40 & 6.02 & 
879.26 & 4.43 & \bf{857.17} & 
0.03\\CON8-1 & 751.84 & 9.56 & 
755.52 & 5.82 & \bf{740.85} & 
1.48\\CON8-2 & 718.64 & 6.51 & 
728.65 & 4.60 & \bf{713.44} & 
0.73\\CON8-3 & \bf{811.07} & 7.94 & 
846.17 & 5.26 & 811.07 & 0.00\\
CON8-4 & \bf{772.25} & 2.96 & 
777.61 & 4.55 & 772.25 & 0.00\\
CON8-5 & \bf{\underline{754.88}} & 8.68 & 
758.38 & 5.57 & 756.91 & 
-0.27\\CON8-6 & \bf{678.92} & 4.82 & 
687.68 & 6.32 & 678.92 & 0.00\\
CON8-7 & 813.20 & 6.16 & 
817.03 & 5.42 & \bf{811.96} & 
0.15\\CON8-8 & \bf{767.53} & 9.48 & 
769.78 & 6.42 & 767.53 & 0.00\\
CON8-9 & 815.02 & 7.45 & 
840.30 & 4.68 & \bf{809.00} & 
0.74\\[1ex]\hline
\end{tabular}
\label{table:nonlin}
\end{table} \clearpage
\begin{table}[ht]
\caption{Resultados de la ejecución de la metaheurística GTS, utilizando instancias de Dethloff con la configuración -mni 8000 -lambda1 0.05 -lambda2 0.05 -tabu 20}
\centering
\small
\begin{tabular}{c c c c c c c}
\hline\hline
Instancia & Costo mínimo & Tiempo(seg.) & Costo promedio & Tiempo promedio(seg.) & Costo GTS & \%Gap \\ [0.5ex]
\hline
SCA3-0 & \bf{636.06} & 7.52 & 
639.43 & 7.68 & 636.06 & 0.00\\
SCA3-1 & \bf{697.84} & 5.14 & 
697.84 & 5.12 & 697.84 & 0.00\\
SCA3-2 & \bf{659.34} & 4.79 & 
659.34 & 8.09 & 659.34 & 0.00\\
SCA3-3 & \bf{680.04} & 6.32 & 
680.18 & 5.57 & 680.04 & 0.00\\
SCA3-4 & \bf{690.50} & 6.85 & 
690.50 & 8.11 & 690.50 & 0.00\\
SCA3-5 & \bf{659.90} & 7.48 & 
659.90 & 4.79 & 659.90 & 0.00\\
SCA3-6 & \bf{651.09} & 7.07 & 
651.09 & 6.41 & 651.09 & 0.00\\
SCA3-7 & 666.15 & 4.89 & 
666.15 & 6.56 & \bf{659.17} & 
1.06\\SCA3-8 & \bf{719.47} & 14.76 & 
719.47 & 10.39 & 719.47 & 0.00\\
SCA3-9 & \bf{681.00} & 7.24 & 
681.00 & 6.35 & 681.00 & 0.00\\
SCA8-0 & \bf{961.50} & 8.67 & 
972.66 & 9.16 & 961.50 & 0.00\\
SCA8-1 & \bf{1050.20} & 5.06 & 
1056.63 & 6.07 & 1050.20 & 0.00\\
SCA8-2 & \bf{1039.64} & 5.70 & 
1046.40 & 8.29 & 1039.64 & 0.00\\
SCA8-3 & 1005.65 & 2.82 & 
1011.96 & 6.66 & \bf{983.34} & 
2.27\\SCA8-4 & 1067.28 & 4.26 & 
1071.15 & 4.87 & \bf{1065.49} & 
0.17\\SCA8-5 & \bf{1027.08} & 5.11 & 
1038.50 & 5.96 & 1027.08 & 0.00\\
SCA8-6 & 972.48 & 6.05 & 
972.48 & 6.29 & \bf{971.82} & 
0.07\\SCA8-7 & 1063.22 & 7.26 & 
1066.24 & 7.50 & \bf{1052.17} & 
1.05\\SCA8-8 & \bf{1071.18} & 7.06 & 
1079.38 & 5.13 & 1071.18 & 0.00\\
SCA8-9 & \bf{1060.50} & 4.30 & 
1066.31 & 6.56 & 1060.50 & 0.00\\
CON3-0 & \bf{616.52} & 3.46 & 
622.90 & 7.16 & 616.52 & 0.00\\
CON3-1 & \bf{554.47} & 12.00 & 
558.34 & 6.67 & 554.47 & 0.00\\
CON3-2 & 519.61 & 4.37 & 
520.94 & 4.82 & \bf{519.26} & 
0.07\\CON3-3 & \bf{591.19} & 3.77 & 
591.19 & 7.99 & 591.19 & 0.00\\
CON3-4 & 596.29 & 13.73 & 
596.29 & 8.13 & \bf{589.32} & 
1.18\\CON3-5 & \bf{563.70} & 4.15 & 
567.60 & 4.70 & 563.70 & 0.00\\
CON3-6 & \bf{\underline{499.05}} & 3.53 & 
500.60 & 6.01 & 500.80 & 
-0.35\\CON3-7 & \bf{576.48} & 9.50 & 
576.48 & 7.86 & 576.48 & 0.00\\
CON3-8 & \bf{523.05} & 5.66 & 
523.05 & 5.04 & 523.05 & 0.00\\
CON3-9 & \bf{\underline{578.25}} & 5.61 & 
580.76 & 7.70 & 580.05 & 
-0.31\\CON8-0 & \bf{857.17} & 5.17 & 
857.44 & 5.89 & 857.17 & 0.00\\
CON8-1 & \bf{740.85} & 5.92 & 
750.77 & 7.36 & 740.85 & 0.00\\
CON8-2 & 721.67 & 6.66 & 
729.95 & 5.10 & \bf{713.44} & 
1.15\\CON8-3 & 821.26 & 10.98 & 
834.79 & 7.66 & \bf{811.07} & 
1.26\\CON8-4 & \bf{772.25} & 4.84 & 
782.97 & 6.87 & 772.25 & 0.00\\
CON8-5 & \bf{\underline{755.67}} & 7.17 & 
757.35 & 7.25 & 756.91 & 
-0.16\\CON8-6 & 678.99 & 10.88 & 
685.49 & 9.09 & \bf{678.92} & 
0.01\\CON8-7 & 812.89 & 5.71 & 
813.84 & 6.98 & \bf{811.96} & 
0.11\\CON8-8 & \bf{767.53} & 3.21 & 
770.86 & 4.89 & 767.53 & 0.00\\
CON8-9 & \bf{809.00} & 4.32 & 
811.44 & 5.33 & 809.00 & 0.00\\
[1ex]\hline
\end{tabular}
\label{table:nonlin}
\end{table} \clearpage
\begin{table}[ht]
\caption{Resultados de la ejecución de la metaheurística GTS, utilizando instancias de Dethloff con la configuración -mni 9000 -lambda1 0.05 -lambda2 0.05 -tabu 20}
\centering
\small
\begin{tabular}{c c c c c c c}
\hline\hline
Instancia & Costo mínimo & Tiempo(seg.) & Costo promedio & Tiempo promedio(seg.) & Costo GTS & \%Gap \\ [0.5ex]
\hline
SCA3-0 & \bf{636.06} & 4.68 & 
637.18 & 9.28 & 636.06 & 0.00\\
SCA3-1 & \bf{697.84} & 5.10 & 
697.84 & 7.16 & 697.84 & 0.00\\
SCA3-2 & \bf{659.34} & 5.06 & 
659.34 & 6.12 & 659.34 & 0.00\\
SCA3-3 & \bf{680.04} & 14.30 & 
680.04 & 9.36 & 680.04 & 0.00\\
SCA3-4 & \bf{690.50} & 10.49 & 
690.50 & 10.12 & 690.50 & 0.00\\
SCA3-5 & \bf{659.90} & 6.61 & 
666.42 & 5.67 & 659.90 & 0.00\\
SCA3-6 & \bf{651.09} & 5.44 & 
651.09 & 6.91 & 651.09 & 0.00\\
SCA3-7 & 666.15 & 8.89 & 
666.15 & 5.86 & \bf{659.17} & 
1.06\\SCA3-8 & \bf{719.47} & 4.60 & 
719.47 & 7.12 & 719.47 & 0.00\\
SCA3-9 & \bf{681.00} & 9.60 & 
681.00 & 7.23 & 681.00 & 0.00\\
SCA8-0 & \bf{961.50} & 7.39 & 
971.41 & 13.93 & 961.50 & 0.00\\
SCA8-1 & \bf{\underline{1049.65}} & 5.95 & 
1054.24 & 10.23 & 1050.20 & 
-0.05\\SCA8-2 & \bf{1039.64} & 7.97 & 
1041.87 & 7.75 & 1039.64 & 0.00\\
SCA8-3 & \bf{983.34} & 8.66 & 
996.57 & 10.44 & 983.34 & 0.00\\
SCA8-4 & \bf{1065.49} & 7.85 & 
1069.36 & 7.46 & 1065.49 & 0.00\\
SCA8-5 & \bf{1027.08} & 8.74 & 
1042.78 & 6.57 & 1027.08 & 0.00\\
SCA8-6 & 972.48 & 5.08 & 
976.62 & 4.84 & \bf{971.82} & 
0.07\\SCA8-7 & 1063.22 & 12.23 & 
1065.16 & 8.53 & \bf{1052.17} & 
1.05\\SCA8-8 & \bf{1071.18} & 3.85 & 
1079.38 & 5.28 & 1071.18 & 0.00\\
SCA8-9 & \bf{1060.50} & 9.67 & 
1067.13 & 8.62 & 1060.50 & 0.00\\
CON3-0 & \bf{616.52} & 6.17 & 
616.52 & 5.65 & 616.52 & 0.00\\
CON3-1 & \bf{554.47} & 15.10 & 
556.22 & 7.69 & 554.47 & 0.00\\
CON3-2 & 519.61 & 5.84 & 
521.77 & 6.08 & \bf{519.26} & 
0.07\\CON3-3 & \bf{591.19} & 6.87 & 
591.19 & 10.06 & 591.19 & 0.00\\
CON3-4 & \bf{\underline{588.79}} & 18.32 & 
590.66 & 11.91 & 589.32 & 
-0.09\\CON3-5 & \bf{563.70} & 7.01 & 
565.92 & 6.35 & 563.70 & 0.00\\
CON3-6 & \bf{\underline{499.05}} & 5.45 & 
499.49 & 8.23 & 500.80 & 
-0.35\\CON3-7 & \bf{576.48} & 14.03 & 
576.48 & 8.29 & 576.48 & 0.00\\
CON3-8 & \bf{523.05} & 11.13 & 
523.05 & 8.59 & 523.05 & 0.00\\
CON3-9 & \bf{\underline{578.25}} & 11.08 & 
578.25 & 11.29 & 580.05 & 
-0.31\\CON8-0 & \bf{857.17} & 8.01 & 
876.45 & 6.47 & 857.17 & 0.00\\
CON8-1 & \bf{740.85} & 4.71 & 
753.13 & 7.99 & 740.85 & 0.00\\
CON8-2 & \bf{\underline{712.89}} & 5.48 & 
717.81 & 6.00 & 713.44 & 
-0.08\\CON8-3 & 811.23 & 6.51 & 
819.97 & 7.59 & \bf{811.07} & 
0.02\\CON8-4 & \bf{772.25} & 5.70 & 
786.11 & 4.82 & 772.25 & 0.00\\
CON8-5 & \bf{\underline{754.88}} & 3.61 & 
759.01 & 5.39 & 756.91 & 
-0.27\\CON8-6 & \bf{678.92} & 9.89 & 
684.93 & 6.78 & 678.92 & 0.00\\
CON8-7 & 812.26 & 8.87 & 
812.84 & 7.21 & \bf{811.96} & 
0.04\\CON8-8 & 776.55 & 5.75 & 
779.44 & 6.06 & \bf{767.53} & 
1.18\\CON8-9 & \bf{809.00} & 9.54 & 
811.73 & 8.00 & 809.00 & 0.00\\
[1ex]\hline
\end{tabular}
\label{table:nonlin}
\end{table} \clearpage
\begin{table}[ht]
\caption{Resultados de la ejecución de la metaheurística GTS, utilizando instancias de Dethloff con la configuración -mni 10000 -lambda1 0.05 -lambda2 0.05 -tabu 20}
\centering
\small
\begin{tabular}{c c c c c c c}
\hline\hline
Instancia & Costo mínimo & Tiempo(seg.) & Costo promedio & Tiempo promedio(seg.) & Costo GTS & \%Gap \\ [0.5ex]
\hline
SCA3-0 & \bf{636.06} & 4.69 & 
638.30 & 6.94 & 636.06 & 0.00\\
SCA3-1 & \bf{697.84} & 5.48 & 
698.50 & 9.18 & 697.84 & 0.00\\
SCA3-2 & \bf{659.34} & 12.42 & 
659.34 & 9.28 & 659.34 & 0.00\\
SCA3-3 & \bf{680.04} & 7.99 & 
680.32 & 8.06 & 680.04 & 0.00\\
SCA3-4 & \bf{690.50} & 7.01 & 
690.50 & 9.08 & 690.50 & 0.00\\
SCA3-5 & \bf{659.90} & 12.92 & 
666.42 & 6.96 & 659.90 & 0.00\\
SCA3-6 & \bf{651.09} & 9.38 & 
651.09 & 9.07 & 651.09 & 0.00\\
SCA3-7 & 666.15 & 10.53 & 
666.15 & 8.33 & \bf{659.17} & 
1.06\\SCA3-8 & \bf{719.47} & 4.13 & 
719.47 & 8.98 & 719.47 & 0.00\\
SCA3-9 & \bf{681.00} & 10.23 & 
681.00 & 12.35 & 681.00 & 0.00\\
SCA8-0 & \bf{961.50} & 14.85 & 
973.42 & 9.72 & 961.50 & 0.00\\
SCA8-1 & \bf{\underline{1049.65}} & 7.87 & 
1054.42 & 8.41 & 1050.20 & 
-0.05\\SCA8-2 & 1049.22 & 4.04 & 
1057.01 & 6.14 & \bf{1039.64} & 
0.92\\SCA8-3 & \bf{983.34} & 18.22 & 
989.47 & 12.55 & 983.34 & 0.00\\
SCA8-4 & 1065.83 & 5.00 & 
1068.49 & 6.75 & \bf{1065.49} & 
0.03\\SCA8-5 & \bf{1027.08} & 5.89 & 
1036.87 & 6.00 & 1027.08 & 0.00\\
SCA8-6 & 972.48 & 5.04 & 
977.02 & 5.04 & \bf{971.82} & 
0.07\\SCA8-7 & \bf{\underline{1051.28}} & 12.37 & 
1066.09 & 9.02 & 1052.17 & 
-0.08\\SCA8-8 & \bf{1071.18} & 9.12 & 
1078.12 & 5.80 & 1071.18 & 0.00\\
SCA8-9 & \bf{1060.50} & 9.25 & 
1062.23 & 7.89 & 1060.50 & 0.00\\
CON3-0 & \bf{616.52} & 9.71 & 
624.56 & 9.07 & 616.52 & 0.00\\
CON3-1 & \bf{554.47} & 8.20 & 
555.25 & 7.26 & 554.47 & 0.00\\
CON3-2 & 522.86 & 8.27 & 
523.10 & 12.07 & \bf{519.26} & 
0.69\\CON3-3 & \bf{591.19} & 6.98 & 
591.19 & 8.13 & 591.19 & 0.00\\
CON3-4 & \bf{\underline{588.79}} & 9.80 & 
594.41 & 10.26 & 589.32 & 
-0.09\\CON3-5 & \bf{563.70} & 8.06 & 
565.37 & 10.05 & 563.70 & 0.00\\
CON3-6 & \bf{\underline{499.05}} & 8.91 & 
499.51 & 13.45 & 500.80 & 
-0.35\\CON3-7 & \bf{576.48} & 7.57 & 
585.20 & 7.42 & 576.48 & 0.00\\
CON3-8 & \bf{523.05} & 12.82 & 
523.05 & 7.99 & 523.05 & 0.00\\
CON3-9 & 582.79 & 9.44 & 
585.37 & 8.74 & \bf{580.05} & 
0.47\\CON8-0 & \bf{857.17} & 6.37 & 
862.98 & 7.02 & 857.17 & 0.00\\
CON8-1 & \bf{740.85} & 5.70 & 
751.88 & 6.56 & 740.85 & 0.00\\
CON8-2 & \bf{\underline{712.89}} & 8.01 & 
720.67 & 7.74 & 713.44 & 
-0.08\\CON8-3 & \bf{811.07} & 7.95 & 
816.16 & 8.42 & 811.07 & 0.00\\
CON8-4 & \bf{772.25} & 9.74 & 
772.25 & 7.14 & 772.25 & 0.00\\
CON8-5 & \bf{756.91} & 5.53 & 
758.72 & 6.04 & 756.91 & 0.00\\
CON8-6 & 685.49 & 4.76 & 
695.63 & 5.45 & \bf{678.92} & 
0.97\\CON8-7 & 812.89 & 7.01 & 
812.89 & 8.28 & \bf{811.96} & 
0.11\\CON8-8 & \bf{767.53} & 13.24 & 
770.84 & 10.52 & 767.53 & 0.00\\
CON8-9 & 811.43 & 7.35 & 
831.56 & 8.72 & \bf{809.00} & 
0.30\\[1ex]\hline
\end{tabular}
\label{table:nonlin}
\end{table} \clearpage
\begin{table}[ht]
\caption{Resultados de la ejecución de la metaheurística GTS, utilizando instancias de Dethloff con la configuración -mni 6000 -lambda1 0.01 -lambda2 0.05 -tabu 20}
\centering
\small
\begin{tabular}{c c c c c c c}
\hline\hline
Instancia & Costo mínimo & Tiempo(seg.) & Costo promedio & Tiempo promedio(seg.) & Costo GTS & \%Gap \\ [0.5ex]
\hline
SCA3-0 & 640.55 & 2.74 & 
640.55 & 3.41 & \bf{636.06} & 
0.71\\SCA3-1 & \bf{697.84} & 3.08 & 
698.71 & 6.06 & 697.84 & 0.00\\
SCA3-2 & \bf{659.34} & 3.23 & 
659.34 & 3.87 & 659.34 & 0.00\\
SCA3-3 & \bf{680.04} & 11.28 & 
682.76 & 5.27 & 680.04 & 0.00\\
SCA3-4 & \bf{690.50} & 9.44 & 
690.50 & 6.34 & 690.50 & 0.00\\
SCA3-5 & \bf{659.90} & 2.92 & 
659.90 & 4.60 & 659.90 & 0.00\\
SCA3-6 & \bf{651.09} & 4.73 & 
651.09 & 4.66 & 651.09 & 0.00\\
SCA3-7 & 666.15 & 3.32 & 
667.09 & 4.75 & \bf{659.17} & 
1.06\\SCA3-8 & \bf{719.47} & 2.82 & 
719.47 & 3.16 & 719.47 & 0.00\\
SCA3-9 & \bf{681.00} & 2.92 & 
681.00 & 3.06 & 681.00 & 0.00\\
SCA8-0 & \bf{961.50} & 5.76 & 
978.00 & 4.97 & 961.50 & 0.00\\
SCA8-1 & 1050.93 & 3.45 & 
1064.52 & 3.42 & \bf{1050.20} & 
0.07\\SCA8-2 & \bf{1039.64} & 7.34 & 
1059.76 & 4.48 & 1039.64 & 0.00\\
SCA8-3 & \bf{983.34} & 4.56 & 
1006.58 & 4.61 & 983.34 & 0.00\\
SCA8-4 & \bf{1065.49} & 6.10 & 
1067.75 & 5.74 & 1065.49 & 0.00\\
SCA8-5 & \bf{1027.08} & 6.46 & 
1043.23 & 5.04 & 1027.08 & 0.00\\
SCA8-6 & 972.48 & 6.23 & 
977.02 & 4.25 & \bf{971.82} & 
0.07\\SCA8-7 & 1063.22 & 8.31 & 
1070.02 & 5.95 & \bf{1052.17} & 
1.05\\SCA8-8 & \bf{1071.18} & 2.84 & 
1079.38 & 2.64 & 1071.18 & 0.00\\
SCA8-9 & \bf{1060.50} & 4.58 & 
1065.79 & 4.28 & 1060.50 & 0.00\\
CON3-0 & \bf{616.52} & 4.29 & 
626.53 & 5.08 & 616.52 & 0.00\\
CON3-1 & \bf{554.47} & 4.29 & 
555.18 & 4.57 & 554.47 & 0.00\\
CON3-2 & 519.85 & 4.95 & 
522.50 & 5.19 & \bf{519.26} & 
0.11\\CON3-3 & \bf{591.19} & 9.18 & 
591.19 & 6.04 & 591.19 & 0.00\\
CON3-4 & \bf{\underline{588.79}} & 4.10 & 
589.45 & 4.11 & 589.32 & 
-0.09\\CON3-5 & \bf{563.70} & 4.00 & 
568.14 & 4.73 & 563.70 & 0.00\\
CON3-6 & \bf{\underline{499.05}} & 4.42 & 
501.43 & 8.31 & 500.80 & 
-0.35\\CON3-7 & \bf{576.48} & 4.09 & 
578.13 & 5.19 & 576.48 & 0.00\\
CON3-8 & \bf{523.05} & 3.60 & 
523.05 & 3.74 & 523.05 & 0.00\\
CON3-9 & 588.11 & 5.40 & 
592.84 & 4.43 & \bf{580.05} & 
1.39\\CON8-0 & \bf{857.17} & 4.57 & 
861.91 & 5.58 & 857.17 & 0.00\\
CON8-1 & \bf{740.85} & 5.05 & 
754.32 & 5.32 & 740.85 & 0.00\\
CON8-2 & 716.03 & 10.12 & 
720.29 & 5.04 & \bf{713.44} & 
0.36\\CON8-3 & 826.12 & 6.32 & 
834.48 & 4.80 & \bf{811.07} & 
1.86\\CON8-4 & \bf{772.25} & 5.86 & 
781.37 & 3.09 & 772.25 & 0.00\\
CON8-5 & \bf{\underline{754.95}} & 4.03 & 
757.24 & 4.58 & 756.91 & 
-0.26\\CON8-6 & 683.66 & 2.34 & 
691.97 & 3.91 & \bf{678.92} & 
0.70\\CON8-7 & 813.20 & 5.21 & 
819.68 & 3.94 & \bf{811.96} & 
0.15\\CON8-8 & \bf{767.53} & 4.53 & 
771.30 & 4.58 & 767.53 & 0.00\\
CON8-9 & 812.44 & 5.43 & 
827.55 & 4.48 & \bf{809.00} & 
0.43\\[1ex]\hline
\end{tabular}
\label{table:nonlin}
\end{table} \clearpage
\begin{table}[ht]
\caption{Resultados de la ejecución de la metaheurística GTS, utilizando instancias de Dethloff con la configuración -mni 6000 -lambda1 0.20 -lambda2 0.05 -tabu 20}
\centering
\small
\begin{tabular}{c c c c c c c}
\hline\hline
Instancia & Costo mínimo & Tiempo(seg.) & Costo promedio & Tiempo promedio(seg.) & Costo GTS & \%Gap \\ [0.5ex]
\hline
SCA3-0 & \bf{\underline{635.62}} & 4.59 & 
638.20 & 4.81 & 636.06 & 
-0.07\\SCA3-1 & \bf{697.84} & 2.78 & 
698.50 & 3.77 & 697.84 & 0.00\\
SCA3-2 & \bf{659.34} & 3.64 & 
659.34 & 3.80 & 659.34 & 0.00\\
SCA3-3 & \bf{680.04} & 10.39 & 
680.18 & 5.16 & 680.04 & 0.00\\
SCA3-4 & \bf{690.50} & 6.52 & 
690.50 & 5.76 & 690.50 & 0.00\\
SCA3-5 & \bf{659.90} & 3.88 & 
659.90 & 5.05 & 659.90 & 0.00\\
SCA3-6 & \bf{651.09} & 3.96 & 
651.09 & 4.12 & 651.09 & 0.00\\
SCA3-7 & 666.15 & 6.26 & 
666.15 & 6.23 & \bf{659.17} & 
1.06\\SCA3-8 & \bf{719.47} & 3.64 & 
719.47 & 4.17 & 719.47 & 0.00\\
SCA3-9 & \bf{681.00} & 3.99 & 
681.00 & 6.03 & 681.00 & 0.00\\
SCA8-0 & \bf{961.50} & 5.69 & 
968.36 & 4.87 & 961.50 & 0.00\\
SCA8-1 & 1067.45 & 7.24 & 
1071.79 & 4.43 & \bf{1050.20} & 
1.64\\SCA8-2 & 1042.10 & 5.02 & 
1057.89 & 5.43 & \bf{1039.64} & 
0.24\\SCA8-3 & \bf{983.34} & 6.30 & 
1005.81 & 4.38 & 983.34 & 0.00\\
SCA8-4 & 1067.28 & 6.68 & 
1075.83 & 3.45 & \bf{1065.49} & 
0.17\\SCA8-5 & \bf{1027.08} & 4.88 & 
1030.47 & 5.50 & 1027.08 & 0.00\\
SCA8-6 & 972.48 & 4.48 & 
972.48 & 5.44 & \bf{971.82} & 
0.07\\SCA8-7 & 1063.22 & 3.27 & 
1072.30 & 3.79 & \bf{1052.17} & 
1.05\\SCA8-8 & \bf{1071.18} & 4.10 & 
1080.53 & 3.37 & 1071.18 & 0.00\\
SCA8-9 & \bf{1060.50} & 8.82 & 
1064.86 & 4.97 & 1060.50 & 0.00\\
CON3-0 & \bf{616.52} & 5.07 & 
628.76 & 4.97 & 616.52 & 0.00\\
CON3-1 & \bf{554.47} & 4.18 & 
556.83 & 5.39 & 554.47 & 0.00\\
CON3-2 & 521.69 & 4.28 & 
522.75 & 4.39 & \bf{519.26} & 
0.47\\CON3-3 & \bf{591.19} & 2.70 & 
591.19 & 4.11 & 591.19 & 0.00\\
CON3-4 & \bf{589.32} & 6.00 & 
593.33 & 5.44 & 589.32 & 0.00\\
CON3-5 & \bf{563.70} & 3.84 & 
570.80 & 4.04 & 563.70 & 0.00\\
CON3-6 & \bf{\underline{499.05}} & 3.59 & 
500.61 & 5.14 & 500.80 & 
-0.35\\CON3-7 & \bf{576.48} & 5.01 & 
584.96 & 5.45 & 576.48 & 0.00\\
CON3-8 & \bf{523.05} & 4.76 & 
523.05 & 3.68 & 523.05 & 0.00\\
CON3-9 & \bf{\underline{578.25}} & 8.92 & 
585.81 & 5.92 & 580.05 & 
-0.31\\CON8-0 & \bf{857.17} & 8.70 & 
885.04 & 5.11 & 857.17 & 0.00\\
CON8-1 & 748.15 & 8.46 & 
755.45 & 3.92 & \bf{740.85} & 
0.99\\CON8-2 & 716.03 & 9.62 & 
722.02 & 5.13 & \bf{713.44} & 
0.36\\CON8-3 & \bf{811.07} & 4.72 & 
824.91 & 4.09 & 811.07 & 0.00\\
CON8-4 & \bf{772.25} & 5.42 & 
780.79 & 3.43 & 772.25 & 0.00\\
CON8-5 & 758.12 & 3.50 & 
759.24 & 4.72 & \bf{756.91} & 
0.16\\CON8-6 & 685.49 & 4.38 & 
689.75 & 5.80 & \bf{678.92} & 
0.97\\CON8-7 & \bf{811.96} & 5.99 & 
813.00 & 4.94 & 811.96 & 0.00\\
CON8-8 & \bf{767.53} & 5.67 & 
776.76 & 5.22 & 767.53 & 0.00\\
CON8-9 & \bf{809.00} & 6.55 & 
825.86 & 5.51 & 809.00 & 0.00\\
[1ex]\hline
\end{tabular}
\label{table:nonlin}
\end{table} \clearpage
\begin{table}[ht]
\caption{Resultados de la ejecución de la metaheurística GTS, utilizando instancias de Dethloff con la configuración -mni 6000 -lambda1 0.50 -lambda2 0.05 -tabu 20}
\centering
\small
\begin{tabular}{c c c c c c c}
\hline\hline
Instancia & Costo mínimo & Tiempo(seg.) & Costo promedio & Tiempo promedio(seg.) & Costo GTS & \%Gap \\ [0.5ex]
\hline
SCA3-0 & \bf{\underline{635.62}} & 5.97 & 
636.96 & 6.42 & 636.06 & 
-0.07\\SCA3-1 & \bf{697.84} & 4.43 & 
697.84 & 3.49 & 697.84 & 0.00\\
SCA3-2 & \bf{659.34} & 8.40 & 
659.34 & 5.14 & 659.34 & 0.00\\
SCA3-3 & \bf{680.04} & 7.64 & 
680.04 & 6.01 & 680.04 & 0.00\\
SCA3-4 & \bf{690.50} & 3.76 & 
690.50 & 5.54 & 690.50 & 0.00\\
SCA3-5 & \bf{659.90} & 6.32 & 
666.42 & 5.40 & 659.90 & 0.00\\
SCA3-6 & \bf{651.09} & 7.16 & 
651.09 & 4.73 & 651.09 & 0.00\\
SCA3-7 & \bf{659.17} & 2.56 & 
664.40 & 7.26 & 659.17 & 0.00\\
SCA3-8 & \bf{719.47} & 8.86 & 
721.90 & 6.17 & 719.47 & 0.00\\
SCA3-9 & \bf{681.00} & 8.96 & 
681.00 & 6.06 & 681.00 & 0.00\\
SCA8-0 & \bf{961.50} & 3.52 & 
981.99 & 4.54 & 961.50 & 0.00\\
SCA8-1 & \bf{\underline{1049.65}} & 6.81 & 
1065.77 & 4.72 & 1050.20 & 
-0.05\\SCA8-2 & \bf{1039.64} & 4.16 & 
1050.99 & 4.67 & 1039.64 & 0.00\\
SCA8-3 & \bf{983.34} & 4.77 & 
996.84 & 4.96 & 983.34 & 0.00\\
SCA8-4 & 1067.28 & 4.73 & 
1068.19 & 5.26 & \bf{1065.49} & 
0.17\\SCA8-5 & 1042.30 & 7.96 & 
1053.04 & 4.46 & \bf{1027.08} & 
1.48\\SCA8-6 & 972.48 & 5.49 & 
980.75 & 5.98 & \bf{971.82} & 
0.07\\SCA8-7 & 1063.60 & 6.92 & 
1071.35 & 5.30 & \bf{1052.17} & 
1.09\\SCA8-8 & \bf{1071.18} & 3.85 & 
1076.85 & 4.27 & 1071.18 & 0.00\\
SCA8-9 & \bf{1060.50} & 4.12 & 
1065.12 & 5.02 & 1060.50 & 0.00\\
CON3-0 & 617.59 & 11.24 & 
627.19 & 5.00 & \bf{616.52} & 
0.17\\CON3-1 & \bf{554.47} & 6.20 & 
556.47 & 5.09 & 554.47 & 0.00\\
CON3-2 & 519.85 & 2.96 & 
522.38 & 5.32 & \bf{519.26} & 
0.11\\CON3-3 & \bf{591.19} & 8.72 & 
591.19 & 5.67 & 591.19 & 0.00\\
CON3-4 & \bf{\underline{588.79}} & 3.98 & 
590.66 & 4.84 & 589.32 & 
-0.09\\CON3-5 & \bf{563.70} & 2.77 & 
563.70 & 3.43 & 563.70 & 0.00\\
CON3-6 & \bf{\underline{499.05}} & 6.76 & 
499.93 & 5.45 & 500.80 & 
-0.35\\CON3-7 & \bf{576.48} & 3.28 & 
580.34 & 6.25 & 576.48 & 0.00\\
CON3-8 & \bf{523.05} & 4.02 & 
523.07 & 3.62 & 523.05 & 0.00\\
CON3-9 & \bf{\underline{578.25}} & 3.47 & 
585.75 & 3.82 & 580.05 & 
-0.31\\CON8-0 & \bf{857.17} & 6.41 & 
870.66 & 3.93 & 857.17 & 0.00\\
CON8-1 & \bf{740.85} & 8.17 & 
753.10 & 5.80 & 740.85 & 0.00\\
CON8-2 & \bf{\underline{712.89}} & 4.68 & 
734.64 & 6.01 & 713.44 & 
-0.08\\CON8-3 & \bf{811.07} & 6.09 & 
847.11 & 5.06 & 811.07 & 0.00\\
CON8-4 & \bf{772.25} & 3.12 & 
780.75 & 4.68 & 772.25 & 0.00\\
CON8-5 & \bf{\underline{754.95}} & 5.69 & 
758.23 & 4.45 & 756.91 & 
-0.26\\CON8-6 & 691.20 & 4.49 & 
695.20 & 5.34 & \bf{678.92} & 
1.81\\CON8-7 & \bf{811.96} & 3.33 & 
813.76 & 4.21 & 811.96 & 0.00\\
CON8-8 & \bf{767.53} & 7.33 & 
775.39 & 4.98 & 767.53 & 0.00\\
CON8-9 & \bf{809.00} & 3.31 & 
811.40 & 4.59 & 809.00 & 0.00\\
[1ex]\hline
\end{tabular}
\label{table:nonlin}
\end{table} \clearpage
\begin{table}[ht]
\caption{Resultados de la ejecución de la metaheurística GTS, utilizando instancias de Dethloff con la configuración -mni 6000 -lambda1 0.80 -lambda2 0.05 -tabu 20}
\centering
\small
\begin{tabular}{c c c c c c c}
\hline\hline
Instancia & Costo mínimo & Tiempo(seg.) & Costo promedio & Tiempo promedio(seg.) & Costo GTS & \%Gap \\ [0.5ex]
\hline
SCA3-0 & \bf{636.06} & 4.27 & 
637.18 & 5.09 & 636.06 & 0.00\\
SCA3-1 & \bf{697.84} & 5.16 & 
697.84 & 4.55 & 697.84 & 0.00\\
SCA3-2 & \bf{659.34} & 2.95 & 
659.34 & 3.24 & 659.34 & 0.00\\
SCA3-3 & \bf{680.04} & 7.31 & 
680.46 & 5.78 & 680.04 & 0.00\\
SCA3-4 & \bf{690.50} & 9.19 & 
690.50 & 8.27 & 690.50 & 0.00\\
SCA3-5 & \bf{659.90} & 2.94 & 
659.90 & 5.32 & 659.90 & 0.00\\
SCA3-6 & \bf{651.09} & 4.13 & 
658.79 & 2.95 & 651.09 & 0.00\\
SCA3-7 & \bf{659.17} & 5.28 & 
665.34 & 5.13 & 659.17 & 0.00\\
SCA3-8 & \bf{719.47} & 4.25 & 
719.47 & 4.36 & 719.47 & 0.00\\
SCA3-9 & \bf{681.00} & 3.72 & 
681.00 & 3.64 & 681.00 & 0.00\\
SCA8-0 & 979.79 & 3.42 & 
987.68 & 2.84 & \bf{961.50} & 
1.90\\SCA8-1 & \bf{\underline{1049.65}} & 3.69 & 
1061.44 & 5.06 & 1050.20 & 
-0.05\\SCA8-2 & 1042.10 & 7.54 & 
1048.30 & 5.40 & \bf{1039.64} & 
0.24\\SCA8-3 & \bf{983.34} & 8.22 & 
990.13 & 8.76 & 983.34 & 0.00\\
SCA8-4 & 1067.28 & 8.51 & 
1069.04 & 6.76 & \bf{1065.49} & 
0.17\\SCA8-5 & \bf{1027.08} & 8.96 & 
1045.77 & 5.73 & 1027.08 & 0.00\\
SCA8-6 & \bf{971.82} & 3.58 & 
972.15 & 4.63 & 971.82 & 0.00\\
SCA8-7 & 1054.59 & 3.50 & 
1060.78 & 3.84 & \bf{1052.17} & 
0.23\\SCA8-8 & \bf{1071.18} & 3.54 & 
1080.53 & 3.50 & 1071.18 & 0.00\\
SCA8-9 & \bf{1060.50} & 6.02 & 
1063.78 & 6.03 & 1060.50 & 0.00\\
CON3-0 & \bf{616.52} & 5.42 & 
619.51 & 4.83 & 616.52 & 0.00\\
CON3-1 & \bf{554.47} & 3.48 & 
558.95 & 2.87 & 554.47 & 0.00\\
CON3-2 & \bf{\underline{519.11}} & 3.20 & 
519.73 & 3.55 & 519.26 & 
-0.03\\CON3-3 & \bf{591.19} & 3.73 & 
591.19 & 4.04 & 591.19 & 0.00\\
CON3-4 & 596.29 & 3.30 & 
599.11 & 3.73 & \bf{589.32} & 
1.18\\CON3-5 & \bf{563.70} & 2.94 & 
565.92 & 4.59 & 563.70 & 0.00\\
CON3-6 & \bf{\underline{499.05}} & 2.56 & 
501.54 & 3.77 & 500.80 & 
-0.35\\CON3-7 & \bf{576.48} & 6.59 & 
584.66 & 4.92 & 576.48 & 0.00\\
CON3-8 & \bf{523.05} & 4.33 & 
523.37 & 3.23 & 523.05 & 0.00\\
CON3-9 & 587.78 & 2.87 & 
588.15 & 4.93 & \bf{580.05} & 
1.33\\CON8-0 & 857.65 & 4.72 & 
869.52 & 5.13 & \bf{857.17} & 
0.06\\CON8-1 & \bf{740.85} & 4.72 & 
746.12 & 5.19 & 740.85 & 0.00\\
CON8-2 & \bf{\underline{712.89}} & 10.24 & 
718.88 & 4.78 & 713.44 & 
-0.08\\CON8-3 & \bf{811.07} & 9.16 & 
817.38 & 6.62 & 811.07 & 0.00\\
CON8-4 & \bf{772.25} & 6.54 & 
776.01 & 5.25 & 772.25 & 0.00\\
CON8-5 & 758.12 & 11.01 & 
761.99 & 6.58 & \bf{756.91} & 
0.16\\CON8-6 & \bf{678.92} & 9.67 & 
686.38 & 7.78 & 678.92 & 0.00\\
CON8-7 & 812.89 & 7.10 & 
813.37 & 5.34 & \bf{811.96} & 
0.11\\CON8-8 & \bf{767.53} & 2.26 & 
774.69 & 4.24 & 767.53 & 0.00\\
CON8-9 & \bf{809.00} & 3.51 & 
812.10 & 3.87 & 809.00 & 0.00\\
[1ex]\hline
\end{tabular}
\label{table:nonlin}
\end{table} \clearpage
\begin{table}[ht]
\caption{Resultados de la ejecución de la metaheurística GTS, utilizando instancias de Dethloff con la configuración -mni 6000 -lambda1 0.05 -lambda2 0.01 -tabu 20}
\centering
\small
\begin{tabular}{c c c c c c c}
\hline\hline
Instancia & Costo mínimo & Tiempo(seg.) & Costo promedio & Tiempo promedio(seg.) & Costo GTS & \%Gap \\ [0.5ex]
\hline
SCA3-0 & \bf{\underline{635.62}} & 3.35 & 
637.35 & 2.62 & 636.06 & 
-0.07\\SCA3-1 & \bf{697.84} & 4.79 & 
697.84 & 3.22 & 697.84 & 0.00\\
SCA3-2 & \bf{659.34} & 2.70 & 
659.34 & 2.69 & 659.34 & 0.00\\
SCA3-3 & 680.60 & 5.81 & 
689.63 & 2.89 & \bf{680.04} & 
0.08\\SCA3-4 & \bf{690.50} & 3.37 & 
705.41 & 2.47 & 690.50 & 0.00\\
SCA3-5 & \bf{659.90} & 2.56 & 
666.42 & 3.35 & 659.90 & 0.00\\
SCA3-6 & \bf{651.09} & 2.38 & 
654.16 & 2.82 & 651.09 & 0.00\\
SCA3-7 & 666.15 & 2.44 & 
667.09 & 2.93 & \bf{659.17} & 
1.06\\SCA3-8 & \bf{719.47} & 1.44 & 
721.90 & 2.06 & 719.47 & 0.00\\
SCA3-9 & \bf{681.00} & 2.94 & 
681.00 & 3.25 & 681.00 & 0.00\\
SCA8-0 & \bf{961.50} & 4.19 & 
978.00 & 3.73 & 961.50 & 0.00\\
SCA8-1 & \bf{1050.20} & 5.53 & 
1064.12 & 3.26 & 1050.20 & 0.00\\
SCA8-2 & 1050.37 & 2.06 & 
1059.28 & 3.30 & \bf{1039.64} & 
1.03\\SCA8-3 & 1007.85 & 3.06 & 
1011.57 & 6.09 & \bf{983.34} & 
2.49\\SCA8-4 & 1065.83 & 10.06 & 
1069.40 & 6.11 & \bf{1065.49} & 
0.03\\SCA8-5 & \bf{1027.08} & 3.62 & 
1048.68 & 3.49 & 1027.08 & 0.00\\
SCA8-6 & 972.48 & 2.68 & 
976.62 & 2.93 & \bf{971.82} & 
0.07\\SCA8-7 & 1057.29 & 3.07 & 
1065.01 & 3.52 & \bf{1052.17} & 
0.49\\SCA8-8 & \bf{1071.18} & 2.24 & 
1079.67 & 2.25 & 1071.18 & 0.00\\
SCA8-9 & \bf{1060.50} & 3.38 & 
1065.75 & 4.54 & 1060.50 & 0.00\\
CON3-0 & \bf{616.52} & 5.54 & 
621.66 & 3.33 & 616.52 & 0.00\\
CON3-1 & \bf{554.47} & 5.43 & 
558.00 & 4.45 & 554.47 & 0.00\\
CON3-2 & 523.23 & 2.36 & 
525.50 & 2.63 & \bf{519.26} & 
0.76\\CON3-3 & \bf{591.19} & 1.53 & 
594.58 & 2.02 & 591.19 & 0.00\\
CON3-4 & \bf{\underline{588.79}} & 3.26 & 
596.33 & 3.47 & 589.32 & 
-0.09\\CON3-5 & \bf{563.70} & 3.72 & 
565.91 & 2.94 & 563.70 & 0.00\\
CON3-6 & \bf{\underline{499.05}} & 4.01 & 
504.54 & 3.25 & 500.80 & 
-0.35\\CON3-7 & \bf{576.48} & 4.64 & 
577.26 & 3.39 & 576.48 & 0.00\\
CON3-8 & \bf{523.05} & 1.82 & 
523.37 & 2.89 & 523.05 & 0.00\\
CON3-9 & 588.11 & 7.22 & 
589.74 & 4.09 & \bf{580.05} & 
1.39\\CON8-0 & 857.40 & 6.87 & 
874.04 & 5.49 & \bf{857.17} & 
0.03\\CON8-1 & 751.76 & 5.28 & 
755.18 & 6.24 & \bf{740.85} & 
1.47\\CON8-2 & \bf{713.44} & 4.86 & 
722.95 & 4.84 & 713.44 & 0.00\\
CON8-3 & 821.26 & 3.62 & 
831.80 & 3.54 & \bf{811.07} & 
1.26\\CON8-4 & \bf{772.25} & 5.25 & 
782.97 & 4.23 & 772.25 & 0.00\\
CON8-5 & 759.93 & 12.19 & 
762.09 & 6.29 & \bf{756.91} & 
0.40\\CON8-6 & \bf{678.92} & 4.85 & 
691.30 & 2.94 & 678.92 & 0.00\\
CON8-7 & 813.20 & 6.37 & 
822.17 & 5.30 & \bf{811.96} & 
0.15\\CON8-8 & 776.55 & 4.26 & 
780.17 & 3.86 & \bf{767.53} & 
1.18\\CON8-9 & \bf{809.00} & 4.61 & 
813.99 & 5.11 & 809.00 & 0.00\\
[1ex]\hline
\end{tabular}
\label{table:nonlin}
\end{table} \clearpage
\begin{table}[ht]
\caption{Resultados de la ejecución de la metaheurística GTS, utilizando instancias de Dethloff con la configuración -mni 6000 -lambda1 0.05 -lambda2 0.20 -tabu 20}
\centering
\small
\begin{tabular}{c c c c c c c}
\hline\hline
Instancia & Costo mínimo & Tiempo(seg.) & Costo promedio & Tiempo promedio(seg.) & Costo GTS & \%Gap \\ [0.5ex]
\hline
SCA3-0 & \bf{636.06} & 3.12 & 
639.43 & 3.97 & 636.06 & 0.00\\
SCA3-1 & \bf{697.84} & 8.59 & 
698.50 & 6.52 & 697.84 & 0.00\\
SCA3-2 & \bf{659.34} & 5.97 & 
659.34 & 5.26 & 659.34 & 0.00\\
SCA3-3 & \bf{680.04} & 5.53 & 
680.76 & 4.34 & 680.04 & 0.00\\
SCA3-4 & \bf{690.50} & 6.70 & 
690.50 & 5.48 & 690.50 & 0.00\\
SCA3-5 & \bf{659.90} & 8.63 & 
659.90 & 5.97 & 659.90 & 0.00\\
SCA3-6 & \bf{651.09} & 3.89 & 
651.09 & 4.67 & 651.09 & 0.00\\
SCA3-7 & 666.15 & 3.69 & 
667.09 & 4.82 & \bf{659.17} & 
1.06\\SCA3-8 & \bf{719.47} & 6.52 & 
719.47 & 5.96 & 719.47 & 0.00\\
SCA3-9 & \bf{681.00} & 5.23 & 
681.64 & 6.12 & 681.00 & 0.00\\
SCA8-0 & \bf{961.50} & 4.97 & 
979.59 & 3.96 & 961.50 & 0.00\\
SCA8-1 & 1068.31 & 6.33 & 
1071.55 & 6.14 & \bf{1050.20} & 
1.72\\SCA8-2 & 1050.37 & 3.44 & 
1060.20 & 3.37 & \bf{1039.64} & 
1.03\\SCA8-3 & \bf{983.34} & 18.74 & 
983.34 & 9.85 & 983.34 & 0.00\\
SCA8-4 & 1067.28 & 5.32 & 
1067.77 & 5.59 & \bf{1065.49} & 
0.17\\SCA8-5 & \bf{1027.08} & 9.61 & 
1045.00 & 6.21 & 1027.08 & 0.00\\
SCA8-6 & 972.48 & 2.62 & 
976.67 & 3.83 & \bf{971.82} & 
0.07\\SCA8-7 & 1063.22 & 2.88 & 
1071.67 & 4.66 & \bf{1052.17} & 
1.05\\SCA8-8 & \bf{1071.18} & 6.00 & 
1081.12 & 3.97 & 1071.18 & 0.00\\
SCA8-9 & \bf{1060.50} & 5.06 & 
1062.23 & 6.38 & 1060.50 & 0.00\\
CON3-0 & 628.36 & 4.81 & 
632.10 & 5.39 & \bf{616.52} & 
1.92\\CON3-1 & \bf{554.47} & 7.29 & 
555.15 & 7.96 & 554.47 & 0.00\\
CON3-2 & 519.61 & 7.93 & 
522.32 & 5.94 & \bf{519.26} & 
0.07\\CON3-3 & \bf{591.19} & 4.32 & 
594.58 & 5.66 & 591.19 & 0.00\\
CON3-4 & \bf{\underline{588.79}} & 3.36 & 
592.54 & 7.00 & 589.32 & 
-0.09\\CON3-5 & \bf{563.70} & 3.15 & 
565.91 & 5.67 & 563.70 & 0.00\\
CON3-6 & \bf{\underline{499.05}} & 8.17 & 
499.62 & 5.14 & 500.80 & 
-0.35\\CON3-7 & \bf{576.48} & 4.96 & 
581.67 & 5.05 & 576.48 & 0.00\\
CON3-8 & \bf{523.05} & 3.69 & 
523.05 & 4.94 & 523.05 & 0.00\\
CON3-9 & \bf{\underline{578.25}} & 8.28 & 
585.77 & 7.33 & 580.05 & 
-0.31\\CON8-0 & 857.40 & 9.04 & 
870.91 & 6.79 & \bf{857.17} & 
0.03\\CON8-1 & \bf{740.85} & 7.88 & 
751.40 & 5.65 & 740.85 & 0.00\\
CON8-2 & 716.03 & 4.42 & 
723.08 & 3.79 & \bf{713.44} & 
0.36\\CON8-3 & 821.26 & 4.55 & 
823.69 & 3.79 & \bf{811.07} & 
1.26\\CON8-4 & \bf{772.25} & 2.40 & 
778.74 & 4.26 & 772.25 & 0.00\\
CON8-5 & \bf{756.91} & 3.55 & 
758.93 & 6.28 & 756.91 & 0.00\\
CON8-6 & \bf{678.92} & 3.85 & 
692.11 & 3.36 & 678.92 & 0.00\\
CON8-7 & 812.89 & 6.63 & 
813.32 & 5.84 & \bf{811.96} & 
0.11\\CON8-8 & \bf{767.53} & 10.23 & 
772.79 & 6.22 & 767.53 & 0.00\\
CON8-9 & \bf{809.00} & 7.80 & 
829.93 & 5.62 & 809.00 & 0.00\\
[1ex]\hline
\end{tabular}
\label{table:nonlin}
\end{table} \clearpage
\begin{table}[ht]
\caption{Resultados de la ejecución de la metaheurística GTS, utilizando instancias de Dethloff con la configuración -mni 6000 -lambda1 0.05 -lambda2 0.50 -tabu 20}
\centering
\small
\begin{tabular}{c c c c c c c}
\hline\hline
Instancia & Costo mínimo & Tiempo(seg.) & Costo promedio & Tiempo promedio(seg.) & Costo GTS & \%Gap \\ [0.5ex]
\hline
SCA3-0 & \bf{636.06} & 5.18 & 
637.18 & 5.21 & 636.06 & 0.00\\
SCA3-1 & \bf{697.84} & 3.98 & 
698.50 & 7.17 & 697.84 & 0.00\\
SCA3-2 & \bf{659.34} & 5.48 & 
659.34 & 6.00 & 659.34 & 0.00\\
SCA3-3 & \bf{680.04} & 4.28 & 
680.18 & 4.87 & 680.04 & 0.00\\
SCA3-4 & \bf{690.50} & 5.45 & 
690.50 & 5.33 & 690.50 & 0.00\\
SCA3-5 & \bf{659.90} & 4.12 & 
663.27 & 4.93 & 659.90 & 0.00\\
SCA3-6 & \bf{651.09} & 5.26 & 
652.01 & 5.67 & 651.09 & 0.00\\
SCA3-7 & 666.15 & 5.76 & 
667.09 & 5.39 & \bf{659.17} & 
1.06\\SCA3-8 & \bf{719.47} & 6.01 & 
719.47 & 6.68 & 719.47 & 0.00\\
SCA3-9 & \bf{681.00} & 5.62 & 
681.00 & 5.64 & 681.00 & 0.00\\
SCA8-0 & \bf{961.50} & 5.25 & 
973.28 & 5.97 & 961.50 & 0.00\\
SCA8-1 & \bf{\underline{1049.65}} & 10.64 & 
1067.45 & 5.71 & 1050.20 & 
-0.05\\SCA8-2 & \bf{1039.64} & 2.48 & 
1049.26 & 4.03 & 1039.64 & 0.00\\
SCA8-3 & 1011.22 & 3.86 & 
1012.78 & 4.09 & \bf{983.34} & 
2.84\\SCA8-4 & 1067.55 & 5.15 & 
1077.11 & 3.97 & \bf{1065.49} & 
0.19\\SCA8-5 & 1042.30 & 4.34 & 
1050.22 & 4.62 & \bf{1027.08} & 
1.48\\SCA8-6 & \bf{971.82} & 6.39 & 
972.32 & 5.99 & 971.82 & 0.00\\
SCA8-7 & 1062.63 & 3.82 & 
1073.57 & 3.46 & \bf{1052.17} & 
0.99\\SCA8-8 & \bf{1071.18} & 4.98 & 
1073.92 & 4.19 & 1071.18 & 0.00\\
SCA8-9 & \bf{1060.50} & 3.90 & 
1065.17 & 3.70 & 1060.50 & 0.00\\
CON3-0 & 628.47 & 3.82 & 
628.89 & 5.98 & \bf{616.52} & 
1.94\\CON3-1 & \bf{554.47} & 3.44 & 
556.43 & 6.64 & 554.47 & 0.00\\
CON3-2 & \bf{\underline{519.11}} & 4.65 & 
521.74 & 4.90 & 519.26 & 
-0.03\\CON3-3 & \bf{591.19} & 10.56 & 
591.19 & 7.15 & 591.19 & 0.00\\
CON3-4 & \bf{\underline{588.79}} & 6.17 & 
595.14 & 6.86 & 589.32 & 
-0.09\\CON3-5 & \bf{563.70} & 8.11 & 
570.72 & 4.78 & 563.70 & 0.00\\
CON3-6 & \bf{\underline{499.05}} & 10.15 & 
504.50 & 7.61 & 500.80 & 
-0.35\\CON3-7 & \bf{576.48} & 5.74 & 
576.48 & 5.47 & 576.48 & 0.00\\
CON3-8 & \bf{523.05} & 4.83 & 
523.05 & 4.40 & 523.05 & 0.00\\
CON3-9 & \bf{\underline{578.25}} & 8.21 & 
582.22 & 6.42 & 580.05 & 
-0.31\\CON8-0 & 857.40 & 3.50 & 
868.54 & 4.60 & \bf{857.17} & 
0.03\\CON8-1 & \bf{740.85} & 8.46 & 
749.65 & 5.79 & 740.85 & 0.00\\
CON8-2 & \bf{713.44} & 12.25 & 
717.39 & 7.11 & 713.44 & 0.00\\
CON8-3 & 811.23 & 6.02 & 
819.97 & 5.18 & \bf{811.07} & 
0.02\\CON8-4 & \bf{772.25} & 4.50 & 
778.86 & 3.64 & 772.25 & 0.00\\
CON8-5 & \bf{756.91} & 3.18 & 
773.82 & 3.49 & 756.91 & 0.00\\
CON8-6 & \bf{678.92} & 4.16 & 
688.03 & 6.30 & 678.92 & 0.00\\
CON8-7 & \bf{811.96} & 4.83 & 
813.05 & 5.61 & 811.96 & 0.00\\
CON8-8 & \bf{767.53} & 4.17 & 
770.84 & 4.61 & 767.53 & 0.00\\
CON8-9 & \bf{809.00} & 5.88 & 
810.09 & 6.25 & 809.00 & 0.00\\
[1ex]\hline
\end{tabular}
\label{table:nonlin}
\end{table} \clearpage
\begin{table}[ht]
\caption{Resultados de la ejecución de la metaheurística GTS, utilizando instancias de Dethloff con la configuración -mni 6000 -lambda1 0.05 -lambda2 0.80 -tabu 20}
\centering
\small
\begin{tabular}{c c c c c c c}
\hline\hline
Instancia & Costo mínimo & Tiempo(seg.) & Costo promedio & Tiempo promedio(seg.) & Costo GTS & \%Gap \\ [0.5ex]
\hline
SCA3-0 & \bf{\underline{635.62}} & 4.92 & 
638.20 & 5.05 & 636.06 & 
-0.07\\SCA3-1 & \bf{697.84} & 6.10 & 
697.84 & 5.73 & 697.84 & 0.00\\
SCA3-2 & \bf{659.34} & 7.67 & 
659.34 & 5.56 & 659.34 & 0.00\\
SCA3-3 & \bf{680.04} & 3.87 & 
682.76 & 5.53 & 680.04 & 0.00\\
SCA3-4 & \bf{690.50} & 6.68 & 
690.50 & 6.86 & 690.50 & 0.00\\
SCA3-5 & \bf{659.90} & 4.17 & 
659.90 & 6.37 & 659.90 & 0.00\\
SCA3-6 & \bf{651.09} & 3.62 & 
651.09 & 4.76 & 651.09 & 0.00\\
SCA3-7 & 666.15 & 9.67 & 
667.09 & 7.80 & \bf{659.17} & 
1.06\\SCA3-8 & \bf{719.47} & 4.92 & 
721.21 & 6.18 & 719.47 & 0.00\\
SCA3-9 & \bf{681.00} & 4.00 & 
681.00 & 5.38 & 681.00 & 0.00\\
SCA8-0 & \bf{961.50} & 3.62 & 
973.31 & 3.84 & 961.50 & 0.00\\
SCA8-1 & \bf{1050.20} & 8.47 & 
1056.12 & 5.32 & 1050.20 & 0.00\\
SCA8-2 & \bf{1039.64} & 6.91 & 
1044.60 & 5.92 & 1039.64 & 0.00\\
SCA8-3 & \bf{983.34} & 3.11 & 
996.62 & 4.69 & 983.34 & 0.00\\
SCA8-4 & 1067.55 & 5.53 & 
1069.06 & 4.40 & \bf{1065.49} & 
0.19\\SCA8-5 & \bf{1027.08} & 5.70 & 
1038.70 & 4.71 & 1027.08 & 0.00\\
SCA8-6 & \bf{971.82} & 4.19 & 
972.32 & 4.62 & 971.82 & 0.00\\
SCA8-7 & \bf{\underline{1052.04}} & 5.01 & 
1062.95 & 4.96 & 1052.17 & 
-0.01\\SCA8-8 & \bf{1071.18} & 3.06 & 
1077.22 & 5.09 & 1071.18 & 0.00\\
SCA8-9 & \bf{1060.50} & 4.93 & 
1065.33 & 3.75 & 1060.50 & 0.00\\
CON3-0 & \bf{616.52} & 5.03 & 
619.51 & 7.19 & 616.52 & 0.00\\
CON3-1 & \bf{554.47} & 7.33 & 
555.25 & 8.39 & 554.47 & 0.00\\
CON3-2 & 519.61 & 6.07 & 
522.26 & 6.34 & \bf{519.26} & 
0.07\\CON3-3 & \bf{591.19} & 6.22 & 
591.19 & 5.29 & 591.19 & 0.00\\
CON3-4 & \bf{\underline{588.79}} & 6.31 & 
594.77 & 6.53 & 589.32 & 
-0.09\\CON3-5 & \bf{563.70} & 4.47 & 
570.21 & 5.61 & 563.70 & 0.00\\
CON3-6 & \bf{\underline{499.05}} & 3.53 & 
499.05 & 4.36 & 500.80 & 
-0.35\\CON3-7 & \bf{576.48} & 7.16 & 
576.48 & 7.97 & 576.48 & 0.00\\
CON3-8 & \bf{523.05} & 3.62 & 
523.05 & 4.03 & 523.05 & 0.00\\
CON3-9 & \bf{\underline{578.25}} & 7.53 & 
586.84 & 5.13 & 580.05 & 
-0.31\\CON8-0 & \bf{857.17} & 9.73 & 
871.88 & 7.79 & 857.17 & 0.00\\
CON8-1 & \bf{740.85} & 5.95 & 
749.58 & 3.88 & 740.85 & 0.00\\
CON8-2 & \bf{\underline{712.89}} & 8.85 & 
719.30 & 5.97 & 713.44 & 
-0.08\\CON8-3 & 811.23 & 5.86 & 
820.35 & 5.39 & \bf{811.07} & 
0.02\\CON8-4 & \bf{772.25} & 4.84 & 
777.61 & 3.58 & 772.25 & 0.00\\
CON8-5 & \bf{756.91} & 3.22 & 
780.23 & 4.38 & 756.91 & 0.00\\
CON8-6 & 683.83 & 5.36 & 
693.34 & 4.92 & \bf{678.92} & 
0.72\\CON8-7 & 812.89 & 4.98 & 
812.92 & 6.05 & \bf{811.96} & 
0.11\\CON8-8 & 782.34 & 3.46 & 
786.45 & 3.17 & \bf{767.53} & 
1.93\\CON8-9 & 811.18 & 3.06 & 
824.36 & 3.70 & \bf{809.00} & 
0.27\\[1ex]\hline
\end{tabular}
\label{table:nonlin}
\end{table} \clearpage
\begin{table}[ht]
\caption{Resultados de la ejecución de la metaheurística GTS, utilizando instancias de Dethloff con la configuración -mni 6000 -lambda1 0.05 -lambda2 0.05 -tabu 2}
\centering
\small
\begin{tabular}{c c c c c c c}
\hline\hline
Instancia & Costo mínimo & Tiempo(seg.) & Costo promedio & Tiempo promedio(seg.) & Costo GTS & \%Gap \\ [0.5ex]
\hline
SCA3-0 & \bf{636.06} & 4.19 & 
639.90 & 3.27 & 636.06 & 0.00\\
SCA3-1 & \bf{697.84} & 9.10 & 
697.84 & 6.08 & 697.84 & 0.00\\
SCA3-2 & \bf{659.34} & 5.72 & 
659.34 & 3.65 & 659.34 & 0.00\\
SCA3-3 & \bf{680.04} & 2.00 & 
680.04 & 4.13 & 680.04 & 0.00\\
SCA3-4 & \bf{690.50} & 2.10 & 
690.50 & 4.43 & 690.50 & 0.00\\
SCA3-5 & \bf{659.90} & 3.01 & 
663.16 & 3.60 & 659.90 & 0.00\\
SCA3-6 & \bf{651.09} & 3.99 & 
654.29 & 5.94 & 651.09 & 0.00\\
SCA3-7 & \bf{659.17} & 6.59 & 
665.34 & 4.13 & 659.17 & 0.00\\
SCA3-8 & \bf{719.47} & 2.35 & 
719.47 & 3.83 & 719.47 & 0.00\\
SCA3-9 & \bf{681.00} & 3.68 & 
681.00 & 3.69 & 681.00 & 0.00\\
SCA8-0 & 968.18 & 2.74 & 
985.91 & 3.06 & \bf{961.50} & 
0.69\\SCA8-1 & 1054.42 & 4.46 & 
1065.18 & 4.35 & \bf{1050.20} & 
0.40\\SCA8-2 & 1042.25 & 4.32 & 
1053.45 & 4.20 & \bf{1039.64} & 
0.25\\SCA8-3 & \bf{983.34} & 3.51 & 
995.04 & 5.39 & 983.34 & 0.00\\
SCA8-4 & 1065.83 & 5.64 & 
1071.34 & 5.34 & \bf{1065.49} & 
0.03\\SCA8-5 & 1029.95 & 4.22 & 
1054.05 & 3.12 & \bf{1027.08} & 
0.28\\SCA8-6 & \bf{971.82} & 3.09 & 
976.86 & 3.03 & 971.82 & 0.00\\
SCA8-7 & 1070.61 & 5.08 & 
1075.71 & 3.23 & \bf{1052.17} & 
1.75\\SCA8-8 & \bf{1071.18} & 3.17 & 
1079.00 & 3.50 & 1071.18 & 0.00\\
SCA8-9 & 1063.68 & 2.31 & 
1067.25 & 2.87 & \bf{1060.50} & 
0.30\\CON3-0 & \bf{616.52} & 3.35 & 
627.20 & 3.54 & 616.52 & 0.00\\
CON3-1 & \bf{554.47} & 2.92 & 
556.43 & 3.58 & 554.47 & 0.00\\
CON3-2 & 519.61 & 4.92 & 
521.92 & 4.29 & \bf{519.26} & 
0.07\\CON3-3 & \bf{591.19} & 6.19 & 
594.58 & 4.88 & 591.19 & 0.00\\
CON3-4 & 589.88 & 3.21 & 
595.40 & 4.07 & \bf{589.32} & 
0.10\\CON3-5 & \bf{563.70} & 4.86 & 
565.91 & 4.11 & 563.70 & 0.00\\
CON3-6 & \bf{\underline{499.05}} & 5.82 & 
501.93 & 4.07 & 500.80 & 
-0.35\\CON3-7 & 577.54 & 4.24 & 
579.76 & 4.75 & \bf{576.48} & 
0.18\\CON3-8 & \bf{523.05} & 2.83 & 
537.14 & 2.94 & 523.05 & 0.00\\
CON3-9 & \bf{\underline{578.25}} & 2.84 & 
585.49 & 2.42 & 580.05 & 
-0.31\\CON8-0 & 857.40 & 4.47 & 
871.05 & 4.82 & \bf{857.17} & 
0.03\\CON8-1 & 740.93 & 4.78 & 
750.70 & 3.81 & \bf{740.85} & 
0.01\\CON8-2 & 718.64 & 5.81 & 
727.57 & 2.90 & \bf{713.44} & 
0.73\\CON8-3 & \bf{811.07} & 2.78 & 
848.97 & 4.67 & 811.07 & 0.00\\
CON8-4 & \bf{772.25} & 5.82 & 
784.54 & 3.96 & 772.25 & 0.00\\
CON8-5 & 758.12 & 4.63 & 
759.43 & 4.34 & \bf{756.91} & 
0.16\\CON8-6 & 691.20 & 5.29 & 
701.71 & 3.62 & \bf{678.92} & 
1.81\\CON8-7 & 814.50 & 5.44 & 
821.10 & 3.52 & \bf{811.96} & 
0.31\\CON8-8 & \bf{767.53} & 3.46 & 
774.62 & 3.97 & 767.53 & 0.00\\
CON8-9 & 817.97 & 7.10 & 
829.83 & 5.58 & \bf{809.00} & 
1.11\\[1ex]\hline
\end{tabular}
\label{table:nonlin}
\end{table} \clearpage
\begin{table}[ht]
\caption{Resultados de la ejecución de la metaheurística GTS, utilizando instancias de Dethloff con la configuración -mni 6000 -lambda1 0.05 -lambda2 0.05 -tabu 5}
\centering
\small
\begin{tabular}{c c c c c c c}
\hline\hline
Instancia & Costo mínimo & Tiempo(seg.) & Costo promedio & Tiempo promedio(seg.) & Costo GTS & \%Gap \\ [0.5ex]
\hline
SCA3-0 & \bf{636.06} & 2.20 & 
639.43 & 3.08 & 636.06 & 0.00\\
SCA3-1 & \bf{697.84} & 2.40 & 
699.17 & 3.56 & 697.84 & 0.00\\
SCA3-2 & \bf{659.34} & 4.74 & 
659.34 & 4.29 & 659.34 & 0.00\\
SCA3-3 & \bf{680.04} & 5.36 & 
680.32 & 3.51 & 680.04 & 0.00\\
SCA3-4 & \bf{690.50} & 5.36 & 
690.50 & 4.65 & 690.50 & 0.00\\
SCA3-5 & \bf{659.90} & 3.05 & 
663.16 & 3.23 & 659.90 & 0.00\\
SCA3-6 & \bf{651.09} & 3.10 & 
651.09 & 4.03 & 651.09 & 0.00\\
SCA3-7 & 666.15 & 3.69 & 
666.15 & 3.86 & \bf{659.17} & 
1.06\\SCA3-8 & \bf{719.47} & 6.09 & 
719.47 & 4.03 & 719.47 & 0.00\\
SCA3-9 & \bf{681.00} & 6.38 & 
681.00 & 4.63 & 681.00 & 0.00\\
SCA8-0 & \bf{961.50} & 4.56 & 
972.93 & 3.98 & 961.50 & 0.00\\
SCA8-1 & 1050.38 & 2.22 & 
1064.54 & 4.29 & \bf{1050.20} & 
0.02\\SCA8-2 & 1042.10 & 4.19 & 
1056.51 & 3.65 & \bf{1039.64} & 
0.24\\SCA8-3 & \bf{983.34} & 6.10 & 
1012.78 & 4.42 & 983.34 & 0.00\\
SCA8-4 & 1067.55 & 4.28 & 
1084.40 & 3.98 & \bf{1065.49} & 
0.19\\SCA8-5 & \bf{1027.08} & 3.20 & 
1041.36 & 4.49 & 1027.08 & 0.00\\
SCA8-6 & 972.48 & 3.84 & 
979.58 & 4.78 & \bf{971.82} & 
0.07\\SCA8-7 & 1060.98 & 5.66 & 
1075.92 & 4.27 & \bf{1052.17} & 
0.84\\SCA8-8 & 1082.12 & 8.40 & 
1083.26 & 3.92 & \bf{1071.18} & 
1.02\\SCA8-9 & \bf{1060.50} & 4.09 & 
1066.59 & 3.82 & 1060.50 & 0.00\\
CON3-0 & \bf{616.52} & 4.25 & 
625.85 & 3.68 & 616.52 & 0.00\\
CON3-1 & \bf{554.47} & 6.09 & 
558.00 & 3.50 & 554.47 & 0.00\\
CON3-2 & 522.86 & 8.30 & 
525.36 & 4.72 & \bf{519.26} & 
0.69\\CON3-3 & \bf{591.19} & 4.21 & 
595.50 & 4.56 & 591.19 & 0.00\\
CON3-4 & \bf{\underline{588.79}} & 3.42 & 
591.33 & 3.45 & 589.32 & 
-0.09\\CON3-5 & \bf{563.70} & 4.30 & 
572.89 & 5.15 & 563.70 & 0.00\\
CON3-6 & \bf{\underline{499.05}} & 2.94 & 
500.32 & 4.24 & 500.80 & 
-0.35\\CON3-7 & \bf{576.48} & 3.79 & 
589.06 & 4.25 & 576.48 & 0.00\\
CON3-8 & \bf{523.05} & 1.74 & 
523.05 & 2.88 & 523.05 & 0.00\\
CON3-9 & \bf{\underline{578.25}} & 6.64 & 
582.44 & 4.77 & 580.05 & 
-0.31\\CON8-0 & 857.40 & 7.26 & 
884.08 & 5.22 & \bf{857.17} & 
0.03\\CON8-1 & 751.76 & 1.83 & 
754.40 & 3.92 & \bf{740.85} & 
1.47\\CON8-2 & 718.64 & 2.63 & 
732.93 & 3.55 & \bf{713.44} & 
0.73\\CON8-3 & 821.26 & 5.15 & 
824.83 & 5.75 & \bf{811.07} & 
1.26\\CON8-4 & \bf{772.25} & 4.16 & 
775.39 & 3.77 & 772.25 & 0.00\\
CON8-5 & \bf{\underline{754.88}} & 5.98 & 
758.49 & 3.92 & 756.91 & 
-0.27\\CON8-6 & 683.83 & 3.03 & 
689.30 & 3.31 & \bf{678.92} & 
0.72\\CON8-7 & 813.20 & 10.97 & 
820.51 & 6.02 & \bf{811.96} & 
0.15\\CON8-8 & \bf{767.53} & 3.62 & 
774.75 & 5.58 & 767.53 & 0.00\\
CON8-9 & \bf{809.00} & 5.00 & 
822.11 & 6.89 & 809.00 & 0.00\\
[1ex]\hline
\end{tabular}
\label{table:nonlin}
\end{table} \clearpage
\begin{table}[ht]
\caption{Resultados de la ejecución de la metaheurística GTS, utilizando instancias de Dethloff con la configuración -mni 6000 -lambda1 0.05 -lambda2 0.05 -tabu 10}
\centering
\small
\begin{tabular}{c c c c c c c}
\hline\hline
Instancia & Costo mínimo & Tiempo(seg.) & Costo promedio & Tiempo promedio(seg.) & Costo GTS & \%Gap \\ [0.5ex]
\hline
SCA3-0 & \bf{636.06} & 5.62 & 
639.43 & 3.63 & 636.06 & 0.00\\
SCA3-1 & \bf{697.84} & 2.19 & 
697.84 & 4.51 & 697.84 & 0.00\\
SCA3-2 & \bf{659.34} & 2.67 & 
659.34 & 3.69 & 659.34 & 0.00\\
SCA3-3 & \bf{680.04} & 3.54 & 
683.03 & 3.15 & 680.04 & 0.00\\
SCA3-4 & \bf{690.50} & 4.21 & 
690.50 & 4.69 & 690.50 & 0.00\\
SCA3-5 & \bf{659.90} & 3.28 & 
659.90 & 4.99 & 659.90 & 0.00\\
SCA3-6 & \bf{651.09} & 5.46 & 
651.09 & 4.25 & 651.09 & 0.00\\
SCA3-7 & 666.15 & 4.24 & 
668.02 & 5.33 & \bf{659.17} & 
1.06\\SCA3-8 & \bf{719.47} & 3.16 & 
719.47 & 3.75 & 719.47 & 0.00\\
SCA3-9 & \bf{681.00} & 5.76 & 
681.00 & 4.07 & 681.00 & 0.00\\
SCA8-0 & 976.39 & 5.92 & 
991.67 & 5.79 & \bf{961.50} & 
1.55\\SCA8-1 & \bf{\underline{1049.65}} & 7.70 & 
1058.95 & 5.48 & 1050.20 & 
-0.05\\SCA8-2 & 1050.37 & 2.29 & 
1058.52 & 3.26 & \bf{1039.64} & 
1.03\\SCA8-3 & 1002.46 & 3.16 & 
1011.10 & 2.84 & \bf{983.34} & 
1.94\\SCA8-4 & 1067.28 & 7.49 & 
1069.66 & 5.01 & \bf{1065.49} & 
0.17\\SCA8-5 & \bf{1027.08} & 2.70 & 
1051.19 & 3.43 & 1027.08 & 0.00\\
SCA8-6 & 972.48 & 4.33 & 
977.02 & 2.61 & \bf{971.82} & 
0.07\\SCA8-7 & \bf{1052.17} & 5.42 & 
1063.26 & 3.94 & 1052.17 & 0.00\\
SCA8-8 & \bf{1071.18} & 4.44 & 
1081.18 & 4.33 & 1071.18 & 0.00\\
SCA8-9 & \bf{1060.50} & 6.52 & 
1062.19 & 5.23 & 1060.50 & 0.00\\
CON3-0 & 617.03 & 3.41 & 
633.91 & 4.49 & \bf{616.52} & 
0.08\\CON3-1 & \bf{554.47} & 2.32 & 
556.43 & 3.21 & 554.47 & 0.00\\
CON3-2 & \bf{\underline{518.00}} & 2.58 & 
520.49 & 6.07 & 519.26 & 
-0.24\\CON3-3 & \bf{591.19} & 5.51 & 
594.58 & 3.96 & 591.19 & 0.00\\
CON3-4 & \bf{\underline{588.79}} & 4.35 & 
594.41 & 4.39 & 589.32 & 
-0.09\\CON3-5 & \bf{563.70} & 2.06 & 
565.91 & 4.14 & 563.70 & 0.00\\
CON3-6 & \bf{\underline{499.05}} & 4.99 & 
501.83 & 5.16 & 500.80 & 
-0.35\\CON3-7 & \bf{576.48} & 6.80 & 
581.11 & 5.13 & 576.48 & 0.00\\
CON3-8 & \bf{523.05} & 3.07 & 
523.05 & 5.01 & 523.05 & 0.00\\
CON3-9 & \bf{\underline{578.25}} & 9.77 & 
581.77 & 7.29 & 580.05 & 
-0.31\\CON8-0 & 857.40 & 5.19 & 
873.59 & 4.48 & \bf{857.17} & 
0.03\\CON8-1 & \bf{740.85} & 5.03 & 
763.89 & 3.99 & 740.85 & 0.00\\
CON8-2 & \bf{713.44} & 3.76 & 
718.71 & 3.81 & 713.44 & 0.00\\
CON8-3 & \bf{811.07} & 3.86 & 
825.02 & 4.10 & 811.07 & 0.00\\
CON8-4 & \bf{772.25} & 3.31 & 
783.05 & 3.45 & 772.25 & 0.00\\
CON8-5 & \bf{\underline{755.67}} & 4.86 & 
756.60 & 5.35 & 756.91 & 
-0.16\\CON8-6 & \bf{678.92} & 4.64 & 
692.31 & 3.40 & 678.92 & 0.00\\
CON8-7 & 812.89 & 8.34 & 
814.38 & 5.75 & \bf{811.96} & 
0.11\\CON8-8 & 773.60 & 6.48 & 
783.81 & 5.10 & \bf{767.53} & 
0.79\\CON8-9 & 811.43 & 6.23 & 
820.41 & 6.52 & \bf{809.00} & 
0.30\\[1ex]\hline
\end{tabular}
\label{table:nonlin}
\end{table} \clearpage
\begin{table}[ht]
\caption{Resultados de la ejecución de la metaheurística GTS, utilizando instancias de Dethloff con la configuración -mni 6000 -lambda1 0.05 -lambda2 0.05 -tabu 20}
\centering
\small
\begin{tabular}{c c c c c c c}
\hline\hline
Instancia & Costo mínimo & Tiempo(seg.) & Costo promedio & Tiempo promedio(seg.) & Costo GTS & \%Gap \\ [0.5ex]
\hline
SCA3-0 & 640.55 & 4.90 & 
640.55 & 4.18 & \bf{636.06} & 
0.71\\SCA3-1 & \bf{697.84} & 4.70 & 
697.84 & 6.32 & 697.84 & 0.00\\
SCA3-2 & \bf{659.34} & 3.49 & 
659.34 & 4.70 & 659.34 & 0.00\\
SCA3-3 & \bf{680.04} & 6.68 & 
682.89 & 4.25 & 680.04 & 0.00\\
SCA3-4 & \bf{690.50} & 5.82 & 
690.50 & 7.51 & 690.50 & 0.00\\
SCA3-5 & \bf{659.90} & 4.60 & 
659.90 & 3.73 & 659.90 & 0.00\\
SCA3-6 & \bf{651.09} & 3.31 & 
651.09 & 3.65 & 651.09 & 0.00\\
SCA3-7 & 666.15 & 4.65 & 
666.15 & 5.23 & \bf{659.17} & 
1.06\\SCA3-8 & \bf{719.47} & 3.70 & 
719.47 & 7.30 & 719.47 & 0.00\\
SCA3-9 & \bf{681.00} & 4.63 & 
681.00 & 5.85 & 681.00 & 0.00\\
SCA8-0 & \bf{961.50} & 4.58 & 
976.50 & 4.64 & 961.50 & 0.00\\
SCA8-1 & \bf{1050.20} & 6.10 & 
1064.44 & 5.83 & 1050.20 & 0.00\\
SCA8-2 & \bf{1039.64} & 10.55 & 
1061.48 & 7.54 & 1039.64 & 0.00\\
SCA8-3 & \bf{983.34} & 15.75 & 
995.65 & 8.61 & 983.34 & 0.00\\
SCA8-4 & 1067.55 & 3.19 & 
1071.78 & 2.81 & \bf{1065.49} & 
0.19\\SCA8-5 & \bf{1027.08} & 7.54 & 
1042.26 & 4.54 & 1027.08 & 0.00\\
SCA8-6 & 972.48 & 4.28 & 
980.75 & 3.79 & \bf{971.82} & 
0.07\\SCA8-7 & \bf{\underline{1052.04}} & 6.26 & 
1064.90 & 5.74 & 1052.17 & 
-0.01\\SCA8-8 & \bf{1071.18} & 2.66 & 
1077.02 & 3.61 & 1071.18 & 0.00\\
SCA8-9 & \bf{1060.50} & 3.42 & 
1063.91 & 3.28 & 1060.50 & 0.00\\
CON3-0 & \bf{616.52} & 4.94 & 
625.42 & 5.01 & 616.52 & 0.00\\
CON3-1 & \bf{554.47} & 7.22 & 
555.25 & 5.74 & 554.47 & 0.00\\
CON3-2 & 519.33 & 2.89 & 
521.41 & 3.76 & \bf{519.26} & 
0.01\\CON3-3 & \bf{591.19} & 7.16 & 
594.58 & 4.56 & 591.19 & 0.00\\
CON3-4 & \bf{\underline{588.79}} & 4.89 & 
597.24 & 4.94 & 589.32 & 
-0.09\\CON3-5 & \bf{563.70} & 8.01 & 
573.17 & 4.49 & 563.70 & 0.00\\
CON3-6 & \bf{\underline{499.05}} & 5.57 & 
501.10 & 5.62 & 500.80 & 
-0.35\\CON3-7 & \bf{576.48} & 9.36 & 
577.89 & 5.74 & 576.48 & 0.00\\
CON3-8 & \bf{523.05} & 3.20 & 
523.05 & 3.94 & 523.05 & 0.00\\
CON3-9 & \bf{\underline{578.25}} & 6.59 & 
581.77 & 5.02 & 580.05 & 
-0.31\\CON8-0 & \bf{857.17} & 3.90 & 
871.18 & 6.28 & 857.17 & 0.00\\
CON8-1 & \bf{740.85} & 7.58 & 
751.59 & 5.91 & 740.85 & 0.00\\
CON8-2 & \bf{\underline{712.89}} & 7.02 & 
718.81 & 6.05 & 713.44 & 
-0.08\\CON8-3 & 811.98 & 4.46 & 
826.24 & 3.59 & \bf{811.07} & 
0.11\\CON8-4 & \bf{772.25} & 4.87 & 
796.66 & 3.97 & 772.25 & 0.00\\
CON8-5 & 758.12 & 5.12 & 
759.57 & 4.10 & \bf{756.91} & 
0.16\\CON8-6 & \bf{678.92} & 8.83 & 
687.50 & 5.92 & 678.92 & 0.00\\
CON8-7 & 812.26 & 4.09 & 
813.98 & 3.45 & \bf{811.96} & 
0.04\\CON8-8 & \bf{767.53} & 3.48 & 
787.97 & 3.44 & 767.53 & 0.00\\
CON8-9 & 811.18 & 4.70 & 
831.12 & 3.82 & \bf{809.00} & 
0.27\\[1ex]\hline
\end{tabular}
\label{table:nonlin}
\end{table} \clearpage
\begin{table}[ht]
\caption{Resultados de la ejecución de la metaheurística GTS, utilizando instancias de Dethloff con la configuración -mni 6000 -lambda1 0.05 -lambda2 0.05 -tabu 30}
\centering
\small
\begin{tabular}{c c c c c c c}
\hline\hline
Instancia & Costo mínimo & Tiempo(seg.) & Costo promedio & Tiempo promedio(seg.) & Costo GTS & \%Gap \\ [0.5ex]
\hline
SCA3-0 & \bf{\underline{635.62}} & 7.97 & 
639.32 & 6.08 & 636.06 & 
-0.07\\SCA3-1 & \bf{697.84} & 4.20 & 
697.84 & 8.94 & 697.84 & 0.00\\
SCA3-2 & \bf{659.34} & 4.97 & 
659.34 & 4.15 & 659.34 & 0.00\\
SCA3-3 & \bf{680.04} & 3.38 & 
680.46 & 3.48 & 680.04 & 0.00\\
SCA3-4 & \bf{690.50} & 14.28 & 
690.50 & 9.97 & 690.50 & 0.00\\
SCA3-5 & \bf{659.90} & 4.84 & 
663.16 & 5.04 & 659.90 & 0.00\\
SCA3-6 & \bf{651.09} & 4.51 & 
651.55 & 6.09 & 651.09 & 0.00\\
SCA3-7 & 666.15 & 7.86 & 
666.15 & 6.53 & \bf{659.17} & 
1.06\\SCA3-8 & \bf{719.47} & 4.44 & 
719.47 & 4.83 & 719.47 & 0.00\\
SCA3-9 & \bf{681.00} & 5.74 & 
684.57 & 4.69 & 681.00 & 0.00\\
SCA8-0 & \bf{961.50} & 10.19 & 
977.98 & 6.31 & 961.50 & 0.00\\
SCA8-1 & 1050.38 & 6.38 & 
1067.74 & 4.40 & \bf{1050.20} & 
0.02\\SCA8-2 & \bf{1039.64} & 10.30 & 
1054.63 & 6.71 & 1039.64 & 0.00\\
SCA8-3 & \bf{983.34} & 4.18 & 
997.02 & 6.50 & 983.34 & 0.00\\
SCA8-4 & 1067.55 & 4.44 & 
1070.87 & 3.52 & \bf{1065.49} & 
0.19\\SCA8-5 & \bf{1027.08} & 2.70 & 
1032.28 & 3.94 & 1027.08 & 0.00\\
SCA8-6 & 972.48 & 3.18 & 
976.62 & 3.89 & \bf{971.82} & 
0.07\\SCA8-7 & \bf{\underline{1051.28}} & 7.46 & 
1061.57 & 4.07 & 1052.17 & 
-0.08\\SCA8-8 & \bf{1071.18} & 4.59 & 
1073.97 & 3.27 & 1071.18 & 0.00\\
SCA8-9 & \bf{1060.50} & 3.25 & 
1062.09 & 2.95 & 1060.50 & 0.00\\
CON3-0 & \bf{616.52} & 5.98 & 
622.72 & 4.82 & 616.52 & 0.00\\
CON3-1 & 556.04 & 6.05 & 
556.67 & 7.89 & \bf{554.47} & 
0.28\\CON3-2 & 523.08 & 4.37 & 
523.40 & 4.49 & \bf{519.26} & 
0.74\\CON3-3 & \bf{591.19} & 4.81 & 
591.19 & 7.89 & 591.19 & 0.00\\
CON3-4 & \bf{\underline{588.79}} & 6.77 & 
591.33 & 4.50 & 589.32 & 
-0.09\\CON3-5 & \bf{563.70} & 6.05 & 
567.21 & 5.68 & 563.70 & 0.00\\
CON3-6 & \bf{\underline{499.05}} & 4.73 & 
502.24 & 5.40 & 500.80 & 
-0.35\\CON3-7 & \bf{576.48} & 5.34 & 
584.20 & 4.74 & 576.48 & 0.00\\
CON3-8 & \bf{523.05} & 2.60 & 
523.05 & 2.77 & 523.05 & 0.00\\
CON3-9 & \bf{\underline{578.25}} & 5.42 & 
586.85 & 4.91 & 580.05 & 
-0.31\\CON8-0 & \bf{857.17} & 4.44 & 
867.97 & 5.77 & 857.17 & 0.00\\
CON8-1 & \bf{740.85} & 11.85 & 
752.47 & 5.96 & 740.85 & 0.00\\
CON8-2 & \bf{\underline{712.89}} & 7.02 & 
718.89 & 5.95 & 713.44 & 
-0.08\\CON8-3 & 821.26 & 7.17 & 
823.69 & 6.17 & \bf{811.07} & 
1.26\\CON8-4 & 785.92 & 3.06 & 
788.62 & 4.06 & \bf{772.25} & 
1.77\\CON8-5 & \bf{\underline{754.88}} & 12.48 & 
759.87 & 6.78 & 756.91 & 
-0.27\\CON8-6 & 688.68 & 10.58 & 
692.13 & 6.79 & \bf{678.92} & 
1.44\\CON8-7 & 812.89 & 5.77 & 
813.92 & 5.61 & \bf{811.96} & 
0.11\\CON8-8 & \bf{767.53} & 4.39 & 
772.82 & 3.91 & 767.53 & 0.00\\
CON8-9 & \bf{809.00} & 3.72 & 
817.94 & 4.66 & 809.00 & 0.00\\
[1ex]\hline
\end{tabular}
\label{table:nonlin}
\end{table} \clearpage
\begin{table}[ht]
\caption{Resultados de la ejecución de la metaheurística GTS, utilizando instancias de Dethloff con la configuración -mni 6000 -lambda1 0.05 -lambda2 0.05 -tabu 40}
\centering
\small
\begin{tabular}{c c c c c c c}
\hline\hline
Instancia & Costo mínimo & Tiempo(seg.) & Costo promedio & Tiempo promedio(seg.) & Costo GTS & \%Gap \\ [0.5ex]
\hline
SCA3-0 & \bf{636.06} & 2.79 & 
638.30 & 5.90 & 636.06 & 0.00\\
SCA3-1 & \bf{697.84} & 6.52 & 
698.50 & 5.34 & 697.84 & 0.00\\
SCA3-2 & \bf{659.34} & 5.77 & 
659.34 & 6.79 & 659.34 & 0.00\\
SCA3-3 & \bf{680.04} & 6.59 & 
680.32 & 5.34 & 680.04 & 0.00\\
SCA3-4 & \bf{690.50} & 5.68 & 
690.50 & 5.88 & 690.50 & 0.00\\
SCA3-5 & \bf{659.90} & 4.34 & 
659.90 & 6.95 & 659.90 & 0.00\\
SCA3-6 & \bf{651.09} & 4.83 & 
651.09 & 5.18 & 651.09 & 0.00\\
SCA3-7 & 666.15 & 8.10 & 
666.15 & 6.52 & \bf{659.17} & 
1.06\\SCA3-8 & \bf{719.47} & 6.76 & 
719.47 & 8.05 & 719.47 & 0.00\\
SCA3-9 & \bf{681.00} & 10.83 & 
681.00 & 7.91 & 681.00 & 0.00\\
SCA8-0 & \bf{961.50} & 8.73 & 
961.50 & 7.08 & 961.50 & 0.00\\
SCA8-1 & \bf{1050.20} & 3.83 & 
1057.42 & 4.81 & 1050.20 & 0.00\\
SCA8-2 & 1049.22 & 7.24 & 
1058.29 & 4.72 & \bf{1039.64} & 
0.92\\SCA8-3 & \bf{983.34} & 9.94 & 
983.34 & 9.49 & 983.34 & 0.00\\
SCA8-4 & 1067.28 & 4.60 & 
1070.42 & 4.90 & \bf{1065.49} & 
0.17\\SCA8-5 & \bf{1027.08} & 2.38 & 
1045.04 & 3.90 & 1027.08 & 0.00\\
SCA8-6 & 972.48 & 2.81 & 
972.48 & 3.20 & \bf{971.82} & 
0.07\\SCA8-7 & \bf{\underline{1052.04}} & 4.55 & 
1059.87 & 7.73 & 1052.17 & 
-0.01\\SCA8-8 & \bf{1071.18} & 2.95 & 
1079.96 & 5.70 & 1071.18 & 0.00\\
SCA8-9 & \bf{1060.50} & 3.46 & 
1064.12 & 4.36 & 1060.50 & 0.00\\
CON3-0 & \bf{616.52} & 5.10 & 
622.66 & 5.02 & 616.52 & 0.00\\
CON3-1 & 555.59 & 4.60 & 
556.26 & 5.10 & \bf{554.47} & 
0.20\\CON3-2 & 523.23 & 5.00 & 
524.21 & 5.14 & \bf{519.26} & 
0.76\\CON3-3 & \bf{591.19} & 8.36 & 
591.19 & 6.76 & 591.19 & 0.00\\
CON3-4 & \bf{\underline{588.79}} & 7.01 & 
592.54 & 6.23 & 589.32 & 
-0.09\\CON3-5 & \bf{563.70} & 7.80 & 
575.93 & 5.05 & 563.70 & 0.00\\
CON3-6 & \bf{\underline{499.05}} & 6.84 & 
501.04 & 6.54 & 500.80 & 
-0.35\\CON3-7 & \bf{576.48} & 5.22 & 
576.96 & 7.25 & 576.48 & 0.00\\
CON3-8 & \bf{523.05} & 3.63 & 
523.05 & 4.09 & 523.05 & 0.00\\
CON3-9 & \bf{\underline{578.25}} & 5.30 & 
582.96 & 7.50 & 580.05 & 
-0.31\\CON8-0 & \bf{857.17} & 4.93 & 
869.03 & 5.58 & 857.17 & 0.00\\
CON8-1 & 751.76 & 5.82 & 
755.44 & 6.84 & \bf{740.85} & 
1.47\\CON8-2 & 718.70 & 3.51 & 
722.50 & 3.73 & \bf{713.44} & 
0.74\\CON8-3 & \bf{811.07} & 6.00 & 
813.62 & 7.95 & 811.07 & 0.00\\
CON8-4 & \bf{772.25} & 5.36 & 
782.95 & 4.92 & 772.25 & 0.00\\
CON8-5 & \bf{756.91} & 8.83 & 
758.42 & 5.96 & 756.91 & 0.00\\
CON8-6 & 688.47 & 3.50 & 
694.64 & 3.79 & \bf{678.92} & 
1.41\\CON8-7 & 812.89 & 5.29 & 
813.69 & 4.03 & \bf{811.96} & 
0.11\\CON8-8 & \bf{767.53} & 4.67 & 
777.86 & 3.51 & 767.53 & 0.00\\
CON8-9 & 811.18 & 4.85 & 
816.62 & 7.16 & \bf{809.00} & 
0.27\\[1ex]\hline
\end{tabular}
\label{table:nonlin}
\end{table} \clearpage
\begin{table}[ht]
\caption{Resultados de la ejecución de la metaheurística GTS, utilizando instancias de Dethloff con la configuración -mni 6000 -lambda1 0.05 -lambda2 0.05 -tabu 50}
\centering
\small
\begin{tabular}{c c c c c c c}
\hline\hline
Instancia & Costo mínimo & Tiempo(seg.) & Costo promedio & Tiempo promedio(seg.) & Costo GTS & \%Gap \\ [0.5ex]
\hline
SCA3-0 & \bf{636.06} & 6.02 & 
637.18 & 5.55 & 636.06 & 0.00\\
SCA3-1 & \bf{697.84} & 4.58 & 
697.84 & 5.97 & 697.84 & 0.00\\
SCA3-2 & \bf{659.34} & 5.22 & 
659.34 & 5.36 & 659.34 & 0.00\\
SCA3-3 & \bf{680.04} & 7.25 & 
682.75 & 8.71 & 680.04 & 0.00\\
SCA3-4 & \bf{690.50} & 13.40 & 
690.50 & 10.14 & 690.50 & 0.00\\
SCA3-5 & \bf{659.90} & 5.77 & 
659.90 & 5.25 & 659.90 & 0.00\\
SCA3-6 & \bf{651.09} & 8.59 & 
651.09 & 8.36 & 651.09 & 0.00\\
SCA3-7 & 666.15 & 3.99 & 
666.15 & 7.38 & \bf{659.17} & 
1.06\\SCA3-8 & \bf{719.47} & 7.25 & 
719.47 & 6.46 & 719.47 & 0.00\\
SCA3-9 & \bf{681.00} & 9.86 & 
681.00 & 5.80 & 681.00 & 0.00\\
SCA8-0 & \bf{961.50} & 6.44 & 
963.79 & 6.50 & 961.50 & 0.00\\
SCA8-1 & \bf{1050.20} & 5.64 & 
1067.15 & 5.12 & 1050.20 & 0.00\\
SCA8-2 & \bf{1039.64} & 4.81 & 
1051.85 & 4.84 & 1039.64 & 0.00\\
SCA8-3 & \bf{983.34} & 8.10 & 
990.89 & 8.05 & 983.34 & 0.00\\
SCA8-4 & 1068.97 & 4.08 & 
1071.79 & 3.62 & \bf{1065.49} & 
0.33\\SCA8-5 & \bf{1027.08} & 5.62 & 
1036.74 & 5.20 & 1027.08 & 0.00\\
SCA8-6 & 972.48 & 4.74 & 
972.48 & 6.04 & \bf{971.82} & 
0.07\\SCA8-7 & 1057.15 & 4.16 & 
1066.02 & 7.04 & \bf{1052.17} & 
0.47\\SCA8-8 & \bf{1071.18} & 7.97 & 
1079.91 & 5.47 & 1071.18 & 0.00\\
SCA8-9 & \bf{1060.50} & 6.10 & 
1065.81 & 5.08 & 1060.50 & 0.00\\
CON3-0 & \bf{616.52} & 3.89 & 
623.35 & 5.70 & 616.52 & 0.00\\
CON3-1 & \bf{554.47} & 9.93 & 
555.94 & 5.42 & 554.47 & 0.00\\
CON3-2 & 521.38 & 7.98 & 
522.73 & 10.42 & \bf{519.26} & 
0.41\\CON3-3 & \bf{591.19} & 5.21 & 
591.19 & 6.50 & 591.19 & 0.00\\
CON3-4 & \bf{\underline{588.79}} & 5.97 & 
593.75 & 6.46 & 589.32 & 
-0.09\\CON3-5 & \bf{563.70} & 8.18 & 
570.21 & 10.41 & 563.70 & 0.00\\
CON3-6 & \bf{\underline{499.05}} & 9.77 & 
499.83 & 7.33 & 500.80 & 
-0.35\\CON3-7 & \bf{576.48} & 3.76 & 
576.48 & 7.38 & 576.48 & 0.00\\
CON3-8 & \bf{523.05} & 3.98 & 
523.05 & 7.03 & 523.05 & 0.00\\
CON3-9 & \bf{\underline{578.25}} & 7.32 & 
582.99 & 7.58 & 580.05 & 
-0.31\\CON8-0 & 857.33 & 2.66 & 
870.03 & 3.52 & \bf{857.17} & 
0.02\\CON8-1 & \bf{740.85} & 8.48 & 
754.57 & 6.02 & 740.85 & 0.00\\
CON8-2 & \bf{713.44} & 7.24 & 
732.63 & 5.72 & 713.44 & 0.00\\
CON8-3 & \bf{811.07} & 8.93 & 
814.09 & 6.36 & 811.07 & 0.00\\
CON8-4 & \bf{772.25} & 3.08 & 
775.28 & 5.19 & 772.25 & 0.00\\
CON8-5 & \bf{\underline{754.88}} & 2.68 & 
757.46 & 5.33 & 756.91 & 
-0.27\\CON8-6 & \bf{678.92} & 10.36 & 
691.98 & 5.28 & 678.92 & 0.00\\
CON8-7 & 812.89 & 7.20 & 
825.91 & 5.21 & \bf{811.96} & 
0.11\\CON8-8 & \bf{767.53} & 12.24 & 
774.09 & 5.78 & 767.53 & 0.00\\
CON8-9 & 809.24 & 3.42 & 
821.22 & 4.11 & \bf{809.00} & 
0.03\\[1ex]\hline
\end{tabular}
\label{table:nonlin}
\end{table} \clearpage
\begin{table}[ht]
\caption{Resultados de la ejecución de la metaheurística GTS, utilizando instancias de Dethloff con la configuración -mni 3000 -lambda1 0.05 -lambda2 0.05 -tabu 5}
\centering
\small
\begin{tabular}{c c c c c c c}
\hline\hline
Instancia & Costo mínimo & Tiempo(seg.) & Costo promedio & Tiempo promedio(seg.) & Costo GTS & \%Gap \\ [0.5ex]
\hline
SCA3-0 & \bf{636.06} & 3.34 & 
643.00 & 1.97 & 636.06 & 0.00\\
SCA3-1 & \bf{697.84} & 4.79 & 
699.36 & 2.14 & 697.84 & 0.00\\
SCA3-2 & \bf{659.34} & 1.95 & 
659.34 & 2.34 & 659.34 & 0.00\\
SCA3-3 & \bf{680.04} & 1.12 & 
685.83 & 1.49 & 680.04 & 0.00\\
SCA3-4 & \bf{690.50} & 3.81 & 
693.33 & 2.21 & 690.50 & 0.00\\
SCA3-5 & \bf{659.90} & 1.59 & 
663.84 & 2.00 & 659.90 & 0.00\\
SCA3-6 & \bf{651.09} & 2.16 & 
654.37 & 1.67 & 651.09 & 0.00\\
SCA3-7 & \bf{659.17} & 1.94 & 
668.62 & 1.66 & 659.17 & 0.00\\
SCA3-8 & \bf{719.47} & 1.80 & 
721.41 & 2.04 & 719.47 & 0.00\\
SCA3-9 & \bf{681.00} & 1.32 & 
685.02 & 1.63 & 681.00 & 0.00\\
SCA8-0 & \bf{961.50} & 3.47 & 
988.07 & 2.59 & 961.50 & 0.00\\
SCA8-1 & \bf{1050.20} & 3.35 & 
1068.29 & 2.31 & 1050.20 & 0.00\\
SCA8-2 & \bf{1039.64} & 2.82 & 
1056.90 & 1.95 & 1039.64 & 0.00\\
SCA8-3 & \bf{983.34} & 2.27 & 
1010.68 & 2.18 & 983.34 & 0.00\\
SCA8-4 & \bf{1065.49} & 2.08 & 
1078.54 & 1.96 & 1065.49 & 0.00\\
SCA8-5 & \bf{1027.08} & 3.25 & 
1051.98 & 2.26 & 1027.08 & 0.00\\
SCA8-6 & \bf{971.82} & 1.44 & 
981.65 & 2.16 & 971.82 & 0.00\\
SCA8-7 & \bf{\underline{1051.28}} & 2.32 & 
1073.48 & 2.40 & 1052.17 & 
-0.08\\SCA8-8 & \bf{1071.18} & 1.55 & 
1087.40 & 1.67 & 1071.18 & 0.00\\
SCA8-9 & \bf{1060.50} & 2.33 & 
1070.37 & 2.55 & 1060.50 & 0.00\\
CON3-0 & \bf{616.52} & 2.83 & 
625.79 & 2.16 & 616.52 & 0.00\\
CON3-1 & \bf{554.47} & 2.27 & 
559.84 & 2.03 & 554.47 & 0.00\\
CON3-2 & \bf{\underline{519.11}} & 2.70 & 
523.25 & 1.96 & 519.26 & 
-0.03\\CON3-3 & \bf{591.19} & 5.21 & 
592.95 & 2.01 & 591.19 & 0.00\\
CON3-4 & \bf{\underline{588.79}} & 1.14 & 
596.71 & 1.62 & 589.32 & 
-0.09\\CON3-5 & \bf{563.70} & 2.33 & 
568.71 & 1.90 & 563.70 & 0.00\\
CON3-6 & \bf{\underline{499.05}} & 1.30 & 
502.77 & 1.99 & 500.80 & 
-0.35\\CON3-7 & \bf{576.48} & 1.18 & 
584.76 & 2.18 & 576.48 & 0.00\\
CON3-8 & \bf{523.05} & 0.90 & 
523.21 & 1.40 & 523.05 & 0.00\\
CON3-9 & \bf{\underline{578.25}} & 1.48 & 
587.64 & 2.08 & 580.05 & 
-0.31\\CON8-0 & \bf{857.17} & 2.32 & 
885.43 & 2.18 & 857.17 & 0.00\\
CON8-1 & \bf{740.85} & 3.56 & 
763.61 & 2.21 & 740.85 & 0.00\\
CON8-2 & \bf{713.44} & 4.24 & 
728.91 & 2.40 & 713.44 & 0.00\\
CON8-3 & 811.23 & 3.33 & 
834.86 & 2.54 & \bf{811.07} & 
0.02\\CON8-4 & \bf{772.25} & 3.34 & 
786.92 & 2.14 & 772.25 & 0.00\\
CON8-5 & \bf{\underline{754.88}} & 3.41 & 
764.78 & 2.05 & 756.91 & 
-0.27\\CON8-6 & \bf{678.92} & 2.46 & 
694.88 & 2.44 & 678.92 & 0.00\\
CON8-7 & \bf{811.96} & 3.30 & 
827.98 & 2.06 & 811.96 & 0.00\\
CON8-8 & \bf{767.53} & 2.34 & 
786.81 & 2.20 & 767.53 & 0.00\\
CON8-9 & \bf{809.00} & 1.98 & 
823.95 & 2.60 & 809.00 & 0.00\\
[1ex]\hline
\end{tabular}
\label{table:nonlin}
\end{table} \clearpage
\begin{table}[ht]
\caption{Resultados de la ejecución de la metaheurística GTS, utilizando instancias de Dethloff con la configuración -mni 3000 -lambda1 0.05 -lambda2 0.05 -tabu 5}
\centering
\small
\begin{tabular}{c c c c c c c}
\hline\hline
Instancia & Costo mínimo & Tiempo(seg.) & Costo promedio & Tiempo promedio(seg.) & Costo GTS & \%Gap \\ [0.5ex]
\hline
SCA3-0 & 640.55 & 6.08 & 
640.55 & 3.58 & \bf{636.06} & 
0.71\\SCA3-1 & \bf{697.84} & 2.35 & 
699.17 & 1.67 & 697.84 & 0.00\\
SCA3-2 & \bf{659.34} & 1.77 & 
659.34 & 2.47 & 659.34 & 0.00\\
SCA3-3 & \bf{680.04} & 1.16 & 
680.46 & 2.00 & 680.04 & 0.00\\
SCA3-4 & \bf{690.50} & 2.32 & 
699.27 & 1.84 & 690.50 & 0.00\\
SCA3-5 & \bf{659.90} & 1.78 & 
670.22 & 1.37 & 659.90 & 0.00\\
SCA3-6 & \bf{651.09} & 1.62 & 
654.81 & 2.02 & 651.09 & 0.00\\
SCA3-7 & 666.15 & 1.14 & 
667.83 & 1.45 & \bf{659.17} & 
1.06\\SCA3-8 & \bf{719.47} & 2.53 & 
719.47 & 1.88 & 719.47 & 0.00\\
SCA3-9 & \bf{681.00} & 1.62 & 
686.34 & 1.18 & 681.00 & 0.00\\
SCA8-0 & 970.64 & 3.35 & 
984.62 & 2.35 & \bf{961.50} & 
0.95\\SCA8-1 & 1067.45 & 2.80 & 
1072.91 & 1.99 & \bf{1050.20} & 
1.64\\SCA8-2 & 1042.17 & 2.52 & 
1053.30 & 1.98 & \bf{1039.64} & 
0.24\\SCA8-3 & \bf{983.34} & 2.00 & 
1004.75 & 1.61 & 983.34 & 0.00\\
SCA8-4 & 1067.29 & 2.74 & 
1072.30 & 1.89 & \bf{1065.49} & 
0.17\\SCA8-5 & 1048.65 & 3.40 & 
1075.27 & 1.88 & \bf{1027.08} & 
2.10\\SCA8-6 & 972.48 & 4.31 & 
987.13 & 2.27 & \bf{971.82} & 
0.07\\SCA8-7 & 1066.65 & 1.31 & 
1075.46 & 2.01 & \bf{1052.17} & 
1.38\\SCA8-8 & 1082.12 & 1.87 & 
1083.19 & 1.45 & \bf{1071.18} & 
1.02\\SCA8-9 & \bf{1060.50} & 2.94 & 
1065.52 & 2.39 & 1060.50 & 0.00\\
CON3-0 & \bf{616.52} & 1.94 & 
624.04 & 1.78 & 616.52 & 0.00\\
CON3-1 & 556.04 & 3.02 & 
558.39 & 1.62 & \bf{554.47} & 
0.28\\CON3-2 & 523.23 & 2.45 & 
526.01 & 1.59 & \bf{519.26} & 
0.76\\CON3-3 & \bf{591.19} & 1.66 & 
612.59 & 1.47 & 591.19 & 0.00\\
CON3-4 & 591.43 & 3.68 & 
598.15 & 1.78 & \bf{589.32} & 
0.36\\CON3-5 & \bf{563.70} & 2.17 & 
569.89 & 2.14 & 563.70 & 0.00\\
CON3-6 & \bf{\underline{499.05}} & 1.85 & 
501.86 & 1.98 & 500.80 & 
-0.35\\CON3-7 & \bf{576.48} & 1.18 & 
586.83 & 1.79 & 576.48 & 0.00\\
CON3-8 & \bf{523.05} & 2.60 & 
523.05 & 1.45 & 523.05 & 0.00\\
CON3-9 & 587.23 & 1.32 & 
592.85 & 1.70 & \bf{580.05} & 
1.24\\CON8-0 & 866.68 & 1.81 & 
890.44 & 2.18 & \bf{857.17} & 
1.11\\CON8-1 & \bf{740.85} & 1.73 & 
757.38 & 2.02 & 740.85 & 0.00\\
CON8-2 & 718.64 & 4.03 & 
737.71 & 3.58 & \bf{713.44} & 
0.73\\CON8-3 & \bf{811.07} & 2.22 & 
827.27 & 3.39 & 811.07 & 0.00\\
CON8-4 & \bf{772.25} & 2.06 & 
790.01 & 2.17 & 772.25 & 0.00\\
CON8-5 & \bf{\underline{754.88}} & 5.19 & 
759.27 & 2.79 & 756.91 & 
-0.27\\CON8-6 & 690.58 & 2.67 & 
700.02 & 2.13 & \bf{678.92} & 
1.72\\CON8-7 & 814.79 & 1.90 & 
832.91 & 1.52 & \bf{811.96} & 
0.35\\CON8-8 & 776.55 & 1.73 & 
790.69 & 1.92 & \bf{767.53} & 
1.18\\CON8-9 & \bf{809.00} & 1.78 & 
813.65 & 3.04 & 809.00 & 0.00\\
[1ex]\hline
\end{tabular}
\label{table:nonlin}
\end{table} \clearpage
\begin{table}[ht]
\caption{Resultados de la ejecución de la metaheurística GTS, utilizando instancias de Dethloff con la configuración -mni 3000 -lambda1 0.05 -lambda2 0.05 -tabu 7}
\centering
\small
\begin{tabular}{c c c c c c c}
\hline\hline
Instancia & Costo mínimo & Tiempo(seg.) & Costo promedio & Tiempo promedio(seg.) & Costo GTS & \%Gap \\ [0.5ex]
\hline
SCA3-0 & \bf{636.06} & 1.00 & 
638.78 & 1.51 & 636.06 & 0.00\\
SCA3-1 & \bf{697.84} & 0.90 & 
701.28 & 1.85 & 697.84 & 0.00\\
SCA3-2 & \bf{659.34} & 3.28 & 
659.34 & 2.09 & 659.34 & 0.00\\
SCA3-3 & 680.60 & 1.16 & 
683.51 & 1.27 & \bf{680.04} & 
0.08\\SCA3-4 & \bf{690.50} & 2.93 & 
690.50 & 2.12 & 690.50 & 0.00\\
SCA3-5 & \bf{659.90} & 1.41 & 
663.27 & 1.34 & 659.90 & 0.00\\
SCA3-6 & \bf{651.09} & 2.40 & 
656.49 & 2.38 & 651.09 & 0.00\\
SCA3-7 & 666.15 & 1.86 & 
670.15 & 2.24 & \bf{659.17} & 
1.06\\SCA3-8 & \bf{719.47} & 3.08 & 
719.47 & 2.23 & 719.47 & 0.00\\
SCA3-9 & \bf{681.00} & 0.85 & 
681.00 & 1.71 & 681.00 & 0.00\\
SCA8-0 & 979.79 & 1.75 & 
987.35 & 2.71 & \bf{961.50} & 
1.90\\SCA8-1 & 1068.14 & 1.28 & 
1071.57 & 2.06 & \bf{1050.20} & 
1.71\\SCA8-2 & 1050.37 & 5.45 & 
1075.07 & 3.21 & \bf{1039.64} & 
1.03\\SCA8-3 & \bf{983.34} & 1.94 & 
1005.61 & 1.77 & 983.34 & 0.00\\
SCA8-4 & 1071.16 & 2.25 & 
1081.81 & 2.21 & \bf{1065.49} & 
0.53\\SCA8-5 & 1048.65 & 2.28 & 
1060.52 & 1.68 & \bf{1027.08} & 
2.10\\SCA8-6 & 972.48 & 1.05 & 
977.09 & 1.48 & \bf{971.82} & 
0.07\\SCA8-7 & 1063.60 & 3.45 & 
1081.80 & 2.34 & \bf{1052.17} & 
1.09\\SCA8-8 & \bf{1071.18} & 0.89 & 
1080.72 & 1.58 & 1071.18 & 0.00\\
SCA8-9 & \bf{1060.50} & 1.75 & 
1074.75 & 1.97 & 1060.50 & 0.00\\
CON3-0 & \bf{616.52} & 3.95 & 
627.36 & 2.31 & 616.52 & 0.00\\
CON3-1 & \bf{554.47} & 3.42 & 
557.77 & 2.19 & 554.47 & 0.00\\
CON3-2 & 523.23 & 1.15 & 
527.39 & 1.69 & \bf{519.26} & 
0.76\\CON3-3 & \bf{591.19} & 0.83 & 
604.13 & 2.03 & 591.19 & 0.00\\
CON3-4 & \bf{\underline{588.79}} & 1.31 & 
596.30 & 1.91 & 589.32 & 
-0.09\\CON3-5 & \bf{563.70} & 1.10 & 
572.30 & 1.54 & 563.70 & 0.00\\
CON3-6 & 502.16 & 1.91 & 
508.25 & 1.99 & \bf{500.80} & 
0.27\\CON3-7 & \bf{576.48} & 2.38 & 
577.45 & 2.43 & 576.48 & 0.00\\
CON3-8 & \bf{523.05} & 0.97 & 
523.21 & 2.31 & 523.05 & 0.00\\
CON3-9 & 588.38 & 1.86 & 
589.32 & 1.56 & \bf{580.05} & 
1.44\\CON8-0 & 871.45 & 1.94 & 
876.40 & 1.88 & \bf{857.17} & 
1.67\\CON8-1 & 758.50 & 2.66 & 
759.93 & 2.43 & \bf{740.85} & 
2.38\\CON8-2 & 722.22 & 4.08 & 
736.16 & 2.69 & \bf{713.44} & 
1.23\\CON8-3 & \bf{811.07} & 2.29 & 
824.70 & 2.05 & 811.07 & 0.00\\
CON8-4 & \bf{772.25} & 3.03 & 
772.25 & 2.27 & 772.25 & 0.00\\
CON8-5 & \bf{756.91} & 2.88 & 
758.41 & 2.75 & 756.91 & 0.00\\
CON8-6 & 694.58 & 2.46 & 
703.98 & 3.00 & \bf{678.92} & 
2.31\\CON8-7 & 814.50 & 2.82 & 
820.40 & 2.46 & \bf{811.96} & 
0.31\\CON8-8 & 779.50 & 1.11 & 
794.03 & 1.52 & \bf{767.53} & 
1.56\\CON8-9 & 811.18 & 2.14 & 
822.20 & 2.73 & \bf{809.00} & 
0.27\\[1ex]\hline
\end{tabular}
\label{table:nonlin}
\end{table} \clearpage
\begin{table}[ht]
\caption{Resultados de la ejecución de la metaheurística GTS, utilizando instancias de Dethloff con la configuración -mni 3000 -lambda1 0.05 -lambda2 0.05 -tabu 9}
\centering
\small
\begin{tabular}{c c c c c c c}
\hline\hline
Instancia & Costo mínimo & Tiempo(seg.) & Costo promedio & Tiempo promedio(seg.) & Costo GTS & \%Gap \\ [0.5ex]
\hline
SCA3-0 & 640.55 & 2.36 & 
641.50 & 1.61 & \bf{636.06} & 
0.71\\SCA3-1 & \bf{697.84} & 1.30 & 
697.84 & 2.18 & 697.84 & 0.00\\
SCA3-2 & \bf{659.34} & 1.17 & 
659.34 & 2.42 & 659.34 & 0.00\\
SCA3-3 & 688.38 & 2.23 & 
693.15 & 2.23 & \bf{680.04} & 
1.23\\SCA3-4 & \bf{690.50} & 2.97 & 
690.50 & 1.98 & 690.50 & 0.00\\
SCA3-5 & \bf{659.90} & 2.50 & 
666.64 & 1.81 & 659.90 & 0.00\\
SCA3-6 & \bf{651.09} & 2.95 & 
652.85 & 2.04 & 651.09 & 0.00\\
SCA3-7 & 666.15 & 1.56 & 
670.15 & 2.19 & \bf{659.17} & 
1.06\\SCA3-8 & \bf{719.47} & 3.54 & 
719.47 & 2.22 & 719.47 & 0.00\\
SCA3-9 & \bf{681.00} & 2.28 & 
681.00 & 1.55 & 681.00 & 0.00\\
SCA8-0 & 970.64 & 5.03 & 
982.43 & 2.42 & \bf{961.50} & 
0.95\\SCA8-1 & \bf{1050.20} & 3.44 & 
1069.23 & 2.90 & 1050.20 & 0.00\\
SCA8-2 & 1050.37 & 1.57 & 
1062.56 & 2.62 & \bf{1039.64} & 
1.03\\SCA8-3 & \bf{983.34} & 1.36 & 
998.25 & 2.58 & 983.34 & 0.00\\
SCA8-4 & 1067.28 & 1.60 & 
1077.33 & 1.33 & \bf{1065.49} & 
0.17\\SCA8-5 & \bf{1027.08} & 3.67 & 
1038.09 & 2.09 & 1027.08 & 0.00\\
SCA8-6 & 972.48 & 2.35 & 
976.67 & 2.12 & \bf{971.82} & 
0.07\\SCA8-7 & 1063.22 & 3.38 & 
1065.41 & 2.81 & \bf{1052.17} & 
1.05\\SCA8-8 & \bf{1071.18} & 1.50 & 
1079.96 & 1.46 & 1071.18 & 0.00\\
SCA8-9 & 1063.68 & 3.70 & 
1074.19 & 2.71 & \bf{1060.50} & 
0.30\\CON3-0 & \bf{616.52} & 3.22 & 
625.54 & 2.12 & 616.52 & 0.00\\
CON3-1 & 556.64 & 1.08 & 
558.92 & 1.52 & \bf{554.47} & 
0.39\\CON3-2 & 522.86 & 1.22 & 
525.89 & 1.04 & \bf{519.26} & 
0.69\\CON3-3 & \bf{591.19} & 2.45 & 
598.54 & 2.00 & 591.19 & 0.00\\
CON3-4 & 591.43 & 1.82 & 
594.57 & 2.47 & \bf{589.32} & 
0.36\\CON3-5 & \bf{563.70} & 1.19 & 
567.25 & 1.86 & 563.70 & 0.00\\
CON3-6 & \bf{\underline{500.37}} & 2.76 & 
502.83 & 1.93 & 500.80 & 
-0.09\\CON3-7 & \bf{576.48} & 2.96 & 
588.15 & 1.89 & 576.48 & 0.00\\
CON3-8 & \bf{523.05} & 0.92 & 
523.37 & 1.34 & 523.05 & 0.00\\
CON3-9 & 582.79 & 2.83 & 
587.88 & 1.68 & \bf{580.05} & 
0.47\\CON8-0 & 868.63 & 1.62 & 
900.14 & 2.38 & \bf{857.17} & 
1.34\\CON8-1 & \bf{740.85} & 2.47 & 
749.62 & 2.76 & 740.85 & 0.00\\
CON8-2 & \bf{713.44} & 2.70 & 
731.34 & 2.10 & 713.44 & 0.00\\
CON8-3 & 815.14 & 1.17 & 
830.66 & 1.63 & \bf{811.07} & 
0.50\\CON8-4 & \bf{772.25} & 2.05 & 
776.01 & 2.38 & 772.25 & 0.00\\
CON8-5 & 758.12 & 3.09 & 
774.23 & 2.23 & \bf{756.91} & 
0.16\\CON8-6 & 690.58 & 4.13 & 
696.72 & 2.33 & \bf{678.92} & 
1.72\\CON8-7 & 812.89 & 3.40 & 
819.13 & 3.04 & \bf{811.96} & 
0.11\\CON8-8 & \bf{767.53} & 2.95 & 
780.28 & 2.69 & 767.53 & 0.00\\
CON8-9 & 812.03 & 3.10 & 
816.33 & 1.97 & \bf{809.00} & 
0.37\\[1ex]\hline
\end{tabular}
\label{table:nonlin}
\end{table} \clearpage
\begin{table}[ht]
\caption{Resultados de la ejecución de la metaheurística GTS, utilizando instancias de Dethloff con la configuración -mni 3000 -lambda1 0.05 -lambda2 0.05 -tabu 11}
\centering
\small
\begin{tabular}{c c c c c c c}
\hline\hline
Instancia & Costo mínimo & Tiempo(seg.) & Costo promedio & Tiempo promedio(seg.) & Costo GTS & \%Gap \\ [0.5ex]
\hline
SCA3-0 & \bf{636.06} & 1.40 & 
638.30 & 1.55 & 636.06 & 0.00\\
SCA3-1 & \bf{697.84} & 2.29 & 
699.17 & 2.65 & 697.84 & 0.00\\
SCA3-2 & \bf{659.34} & 5.30 & 
659.34 & 3.09 & 659.34 & 0.00\\
SCA3-3 & \bf{680.04} & 2.70 & 
682.76 & 1.71 & 680.04 & 0.00\\
SCA3-4 & \bf{690.50} & 2.10 & 
690.50 & 2.56 & 690.50 & 0.00\\
SCA3-5 & \bf{659.90} & 1.18 & 
672.38 & 1.47 & 659.90 & 0.00\\
SCA3-6 & \bf{651.09} & 2.31 & 
654.45 & 1.75 & 651.09 & 0.00\\
SCA3-7 & 666.15 & 6.45 & 
667.09 & 3.42 & \bf{659.17} & 
1.06\\SCA3-8 & \bf{719.47} & 2.05 & 
730.47 & 1.45 & 719.47 & 0.00\\
SCA3-9 & \bf{681.00} & 1.90 & 
681.00 & 2.07 & 681.00 & 0.00\\
SCA8-0 & 970.64 & 2.62 & 
992.86 & 2.28 & \bf{961.50} & 
0.95\\SCA8-1 & \bf{1050.20} & 2.18 & 
1066.30 & 2.18 & 1050.20 & 0.00\\
SCA8-2 & 1042.10 & 2.78 & 
1053.35 & 3.37 & \bf{1039.64} & 
0.24\\SCA8-3 & \bf{983.34} & 2.40 & 
1009.64 & 2.35 & 983.34 & 0.00\\
SCA8-4 & 1067.55 & 3.21 & 
1072.87 & 2.81 & \bf{1065.49} & 
0.19\\SCA8-5 & 1030.08 & 2.07 & 
1046.36 & 2.50 & \bf{1027.08} & 
0.29\\SCA8-6 & 972.48 & 2.26 & 
977.98 & 2.08 & \bf{971.82} & 
0.07\\SCA8-7 & 1071.72 & 1.86 & 
1073.41 & 2.48 & \bf{1052.17} & 
1.86\\SCA8-8 & \bf{1071.18} & 0.96 & 
1078.59 & 1.24 & 1071.18 & 0.00\\
SCA8-9 & \bf{1060.50} & 3.30 & 
1063.78 & 2.75 & 1060.50 & 0.00\\
CON3-0 & 628.47 & 2.27 & 
631.35 & 2.48 & \bf{616.52} & 
1.94\\CON3-1 & 556.04 & 1.20 & 
560.83 & 1.77 & \bf{554.47} & 
0.28\\CON3-2 & \bf{\underline{519.11}} & 5.58 & 
521.70 & 3.77 & 519.26 & 
-0.03\\CON3-3 & \bf{591.19} & 2.72 & 
591.23 & 2.81 & 591.19 & 0.00\\
CON3-4 & \bf{\underline{588.79}} & 1.17 & 
593.20 & 1.82 & 589.32 & 
-0.09\\CON3-5 & \bf{563.70} & 1.29 & 
571.11 & 1.60 & 563.70 & 0.00\\
CON3-6 & \bf{\underline{499.05}} & 2.57 & 
502.11 & 2.24 & 500.80 & 
-0.35\\CON3-7 & \bf{576.48} & 4.93 & 
578.13 & 3.81 & 576.48 & 0.00\\
CON3-8 & \bf{523.05} & 1.42 & 
523.21 & 2.51 & 523.05 & 0.00\\
CON3-9 & 588.11 & 2.64 & 
590.85 & 2.29 & \bf{580.05} & 
1.39\\CON8-0 & 857.40 & 1.51 & 
872.69 & 2.36 & \bf{857.17} & 
0.03\\CON8-1 & 751.76 & 2.97 & 
757.28 & 2.15 & \bf{740.85} & 
1.47\\CON8-2 & \bf{\underline{713.05}} & 2.06 & 
720.32 & 2.15 & 713.44 & 
-0.05\\CON8-3 & 811.23 & 4.70 & 
825.76 & 2.50 & \bf{811.07} & 
0.02\\CON8-4 & \bf{772.25} & 1.73 & 
779.16 & 1.90 & 772.25 & 0.00\\
CON8-5 & 758.12 & 2.34 & 
767.40 & 2.09 & \bf{756.91} & 
0.16\\CON8-6 & 693.61 & 1.47 & 
699.68 & 2.09 & \bf{678.92} & 
2.16\\CON8-7 & 812.89 & 3.99 & 
820.67 & 2.56 & \bf{811.96} & 
0.11\\CON8-8 & 782.07 & 1.75 & 
802.04 & 1.84 & \bf{767.53} & 
1.89\\CON8-9 & 810.61 & 2.04 & 
811.39 & 3.30 & \bf{809.00} & 
0.20\\[1ex]\hline
\end{tabular}
\label{table:nonlin}
\end{table} \clearpage
\begin{table}[ht]
\caption{Resultados de la ejecución de la metaheurística GTS, utilizando instancias de Dethloff con la configuración -mni 3000 -lambda1 0.05 -lambda2 0.05 -tabu 13}
\centering
\small
\begin{tabular}{c c c c c c c}
\hline\hline
Instancia & Costo mínimo & Tiempo(seg.) & Costo promedio & Tiempo promedio(seg.) & Costo GTS & \%Gap \\ [0.5ex]
\hline
SCA3-0 & \bf{636.06} & 1.68 & 
638.78 & 1.76 & 636.06 & 0.00\\
SCA3-1 & \bf{697.84} & 1.98 & 
697.84 & 1.75 & 697.84 & 0.00\\
SCA3-2 & \bf{659.34} & 3.00 & 
659.34 & 2.06 & 659.34 & 0.00\\
SCA3-3 & 680.60 & 1.30 & 
683.17 & 1.60 & \bf{680.04} & 
0.08\\SCA3-4 & \bf{690.50} & 3.15 & 
690.50 & 2.23 & 690.50 & 0.00\\
SCA3-5 & \bf{659.90} & 1.18 & 
666.85 & 1.69 & 659.90 & 0.00\\
SCA3-6 & \bf{651.09} & 2.22 & 
654.73 & 1.99 & 651.09 & 0.00\\
SCA3-7 & 664.88 & 1.29 & 
668.15 & 1.75 & \bf{659.17} & 
0.87\\SCA3-8 & \bf{719.47} & 2.70 & 
719.47 & 2.69 & 719.47 & 0.00\\
SCA3-9 & \bf{681.00} & 2.41 & 
681.00 & 2.34 & 681.00 & 0.00\\
SCA8-0 & \bf{961.50} & 3.28 & 
971.55 & 2.61 & 961.50 & 0.00\\
SCA8-1 & 1065.36 & 1.09 & 
1073.92 & 2.06 & \bf{1050.20} & 
1.44\\SCA8-2 & \bf{1039.64} & 2.16 & 
1049.75 & 2.19 & 1039.64 & 0.00\\
SCA8-3 & \bf{983.34} & 2.44 & 
1004.53 & 2.30 & 983.34 & 0.00\\
SCA8-4 & 1067.55 & 1.96 & 
1071.03 & 2.97 & \bf{1065.49} & 
0.19\\SCA8-5 & \bf{1027.08} & 3.95 & 
1051.81 & 2.18 & 1027.08 & 0.00\\
SCA8-6 & 972.48 & 2.15 & 
985.33 & 2.13 & \bf{971.82} & 
0.07\\SCA8-7 & 1063.22 & 1.59 & 
1082.05 & 2.04 & \bf{1052.17} & 
1.05\\SCA8-8 & \bf{1071.18} & 1.50 & 
1086.54 & 1.18 & 1071.18 & 0.00\\
SCA8-9 & 1063.68 & 2.76 & 
1076.48 & 1.80 & \bf{1060.50} & 
0.30\\CON3-0 & \bf{616.52} & 1.53 & 
625.33 & 2.27 & 616.52 & 0.00\\
CON3-1 & \bf{554.47} & 1.83 & 
558.11 & 2.15 & 554.47 & 0.00\\
CON3-2 & 519.61 & 2.32 & 
526.50 & 2.04 & \bf{519.26} & 
0.07\\CON3-3 & \bf{591.19} & 2.63 & 
591.19 & 1.70 & 591.19 & 0.00\\
CON3-4 & 591.43 & 0.89 & 
597.97 & 1.50 & \bf{589.32} & 
0.36\\CON3-5 & \bf{563.70} & 1.00 & 
574.71 & 1.49 & 563.70 & 0.00\\
CON3-6 & \bf{\underline{499.05}} & 2.83 & 
501.54 & 1.97 & 500.80 & 
-0.35\\CON3-7 & \bf{576.48} & 4.16 & 
582.58 & 2.89 & 576.48 & 0.00\\
CON3-8 & \bf{523.05} & 2.26 & 
523.05 & 2.05 & 523.05 & 0.00\\
CON3-9 & 582.79 & 2.04 & 
590.94 & 1.83 & \bf{580.05} & 
0.47\\CON8-0 & 857.40 & 6.12 & 
867.25 & 3.27 & \bf{857.17} & 
0.03\\CON8-1 & 752.18 & 3.52 & 
771.45 & 2.56 & \bf{740.85} & 
1.53\\CON8-2 & 718.64 & 4.48 & 
728.78 & 3.49 & \bf{713.44} & 
0.73\\CON8-3 & 811.23 & 1.46 & 
830.33 & 2.80 & \bf{811.07} & 
0.02\\CON8-4 & \bf{772.25} & 1.71 & 
775.62 & 1.80 & 772.25 & 0.00\\
CON8-5 & \bf{756.91} & 2.42 & 
762.33 & 2.10 & 756.91 & 0.00\\
CON8-6 & 688.68 & 1.92 & 
698.43 & 1.74 & \bf{678.92} & 
1.44\\CON8-7 & 814.79 & 5.72 & 
817.90 & 3.21 & \bf{811.96} & 
0.35\\CON8-8 & \bf{767.53} & 2.42 & 
774.29 & 2.81 & 767.53 & 0.00\\
CON8-9 & 812.23 & 3.96 & 
831.40 & 2.38 & \bf{809.00} & 
0.40\\[1ex]\hline
\end{tabular}
\label{table:nonlin}
\end{table} \clearpage
\begin{table}[ht]
\caption{Resultados de la ejecución de la metaheurística GTS, utilizando instancias de Dethloff con la configuración -mni 3000 -lambda1 0.05 -lambda2 0.05 -tabu 15}
\centering
\small
\begin{tabular}{c c c c c c c}
\hline\hline
Instancia & Costo mínimo & Tiempo(seg.) & Costo promedio & Tiempo promedio(seg.) & Costo GTS & \%Gap \\ [0.5ex]
\hline
SCA3-0 & \bf{636.06} & 2.48 & 
638.30 & 1.77 & 636.06 & 0.00\\
SCA3-1 & \bf{697.84} & 1.63 & 
702.72 & 2.17 & 697.84 & 0.00\\
SCA3-2 & \bf{659.34} & 1.38 & 
659.34 & 1.55 & 659.34 & 0.00\\
SCA3-3 & \bf{680.04} & 5.37 & 
680.46 & 3.27 & 680.04 & 0.00\\
SCA3-4 & \bf{690.50} & 1.96 & 
690.50 & 1.69 & 690.50 & 0.00\\
SCA3-5 & \bf{659.90} & 2.23 & 
659.90 & 2.40 & 659.90 & 0.00\\
SCA3-6 & \bf{651.09} & 2.62 & 
653.94 & 1.82 & 651.09 & 0.00\\
SCA3-7 & 666.15 & 2.15 & 
667.09 & 1.90 & \bf{659.17} & 
1.06\\SCA3-8 & \bf{719.47} & 3.26 & 
721.21 & 2.30 & 719.47 & 0.00\\
SCA3-9 & \bf{681.00} & 1.21 & 
681.00 & 1.62 & 681.00 & 0.00\\
SCA8-0 & 970.65 & 4.08 & 
996.95 & 2.17 & \bf{961.50} & 
0.95\\SCA8-1 & 1068.31 & 2.69 & 
1070.94 & 2.69 & \bf{1050.20} & 
1.72\\SCA8-2 & \bf{1039.64} & 2.70 & 
1046.28 & 2.07 & 1039.64 & 0.00\\
SCA8-3 & \bf{983.34} & 3.66 & 
1007.17 & 3.69 & 983.34 & 0.00\\
SCA8-4 & 1065.83 & 3.11 & 
1071.06 & 2.85 & \bf{1065.49} & 
0.03\\SCA8-5 & 1042.30 & 2.10 & 
1054.83 & 1.79 & \bf{1027.08} & 
1.48\\SCA8-6 & \bf{971.82} & 3.41 & 
976.86 & 2.88 & 971.82 & 0.00\\
SCA8-7 & 1063.22 & 1.64 & 
1081.16 & 1.87 & \bf{1052.17} & 
1.05\\SCA8-8 & \bf{1071.18} & 1.06 & 
1079.96 & 1.60 & 1071.18 & 0.00\\
SCA8-9 & \bf{1060.50} & 1.93 & 
1062.45 & 2.21 & 1060.50 & 0.00\\
CON3-0 & 617.59 & 1.82 & 
623.17 & 2.46 & \bf{616.52} & 
0.17\\CON3-1 & \bf{554.47} & 4.53 & 
557.01 & 2.78 & 554.47 & 0.00\\
CON3-2 & 521.38 & 1.70 & 
523.05 & 2.24 & \bf{519.26} & 
0.41\\CON3-3 & \bf{591.19} & 0.93 & 
594.58 & 2.31 & 591.19 & 0.00\\
CON3-4 & \bf{\underline{588.79}} & 2.28 & 
594.66 & 2.29 & 589.32 & 
-0.09\\CON3-5 & 572.58 & 3.16 & 
574.49 & 2.22 & \bf{563.70} & 
1.58\\CON3-6 & 502.16 & 3.46 & 
503.10 & 3.41 & \bf{500.80} & 
0.27\\CON3-7 & \bf{576.48} & 2.31 & 
586.11 & 1.68 & 576.48 & 0.00\\
CON3-8 & \bf{523.05} & 4.65 & 
523.21 & 2.16 & 523.05 & 0.00\\
CON3-9 & 582.79 & 1.45 & 
588.89 & 2.65 & \bf{580.05} & 
0.47\\CON8-0 & 857.40 & 2.07 & 
857.77 & 3.18 & \bf{857.17} & 
0.03\\CON8-1 & \bf{740.85} & 1.96 & 
751.13 & 1.66 & 740.85 & 0.00\\
CON8-2 & 718.64 & 1.05 & 
725.83 & 2.06 & \bf{713.44} & 
0.73\\CON8-3 & \bf{811.07} & 2.83 & 
824.09 & 2.70 & 811.07 & 0.00\\
CON8-4 & \bf{772.25} & 1.68 & 
800.57 & 2.22 & 772.25 & 0.00\\
CON8-5 & \bf{756.91} & 2.18 & 
761.87 & 1.70 & 756.91 & 0.00\\
CON8-6 & 690.66 & 2.71 & 
695.92 & 2.07 & \bf{678.92} & 
1.73\\CON8-7 & 812.89 & 3.54 & 
817.04 & 2.72 & \bf{811.96} & 
0.11\\CON8-8 & \bf{767.53} & 2.34 & 
777.90 & 1.81 & 767.53 & 0.00\\
CON8-9 & \bf{809.00} & 4.03 & 
818.06 & 3.20 & 809.00 & 0.00\\
[1ex]\hline
\end{tabular}
\label{table:nonlin}
\end{table} \clearpage
\begin{table}[ht]
\caption{Resultados de la ejecución de la metaheurística GTS, utilizando instancias de Dethloff con la configuración -mni 3500 -lambda1 0.05 -lambda2 0.05 -tabu 5}
\centering
\small
\begin{tabular}{c c c c c c c}
\hline\hline
Instancia & Costo mínimo & Tiempo(seg.) & Costo promedio & Tiempo promedio(seg.) & Costo GTS & \%Gap \\ [0.5ex]
\hline
SCA3-0 & \bf{636.06} & 2.77 & 
642.29 & 1.67 & 636.06 & 0.00\\
SCA3-1 & \bf{697.84} & 2.73 & 
698.50 & 2.11 & 697.84 & 0.00\\
SCA3-2 & \bf{659.34} & 2.13 & 
659.34 & 2.39 & 659.34 & 0.00\\
SCA3-3 & 680.60 & 1.67 & 
685.76 & 2.29 & \bf{680.04} & 
0.08\\SCA3-4 & \bf{690.50} & 1.44 & 
695.88 & 1.89 & 690.50 & 0.00\\
SCA3-5 & \bf{659.90} & 1.26 & 
669.80 & 2.20 & 659.90 & 0.00\\
SCA3-6 & \bf{651.09} & 1.32 & 
657.22 & 1.55 & 651.09 & 0.00\\
SCA3-7 & 666.15 & 2.45 & 
667.68 & 2.66 & \bf{659.17} & 
1.06\\SCA3-8 & \bf{719.47} & 1.52 & 
719.47 & 2.56 & 719.47 & 0.00\\
SCA3-9 & \bf{681.00} & 2.53 & 
681.00 & 2.50 & 681.00 & 0.00\\
SCA8-0 & 979.79 & 4.37 & 
991.07 & 2.67 & \bf{961.50} & 
1.90\\SCA8-1 & 1068.31 & 2.06 & 
1073.85 & 2.23 & \bf{1050.20} & 
1.72\\SCA8-2 & \bf{1039.64} & 2.09 & 
1048.39 & 2.69 & 1039.64 & 0.00\\
SCA8-3 & \bf{983.34} & 4.31 & 
1005.30 & 3.06 & 983.34 & 0.00\\
SCA8-4 & 1067.55 & 1.68 & 
1078.37 & 1.98 & \bf{1065.49} & 
0.19\\SCA8-5 & 1042.30 & 2.43 & 
1062.27 & 2.42 & \bf{1027.08} & 
1.48\\SCA8-6 & 972.48 & 2.24 & 
977.02 & 2.88 & \bf{971.82} & 
0.07\\SCA8-7 & \bf{\underline{1051.28}} & 5.22 & 
1072.36 & 3.17 & 1052.17 & 
-0.08\\SCA8-8 & \bf{1071.18} & 1.30 & 
1082.58 & 1.40 & 1071.18 & 0.00\\
SCA8-9 & \bf{1060.50} & 2.03 & 
1068.44 & 2.00 & 1060.50 & 0.00\\
CON3-0 & 629.03 & 1.75 & 
634.10 & 1.72 & \bf{616.52} & 
2.03\\CON3-1 & \bf{554.47} & 5.89 & 
560.41 & 2.71 & 554.47 & 0.00\\
CON3-2 & 523.23 & 1.39 & 
527.73 & 1.83 & \bf{519.26} & 
0.76\\CON3-3 & \bf{591.19} & 2.49 & 
591.19 & 2.38 & 591.19 & 0.00\\
CON3-4 & \bf{\underline{588.79}} & 2.16 & 
596.52 & 2.21 & 589.32 & 
-0.09\\CON3-5 & \bf{563.70} & 1.28 & 
573.65 & 2.31 & 563.70 & 0.00\\
CON3-6 & \bf{\underline{499.05}} & 3.36 & 
502.25 & 1.97 & 500.80 & 
-0.35\\CON3-7 & \bf{576.48} & 4.02 & 
580.43 & 2.88 & 576.48 & 0.00\\
CON3-8 & \bf{523.05} & 1.59 & 
523.05 & 1.56 & 523.05 & 0.00\\
CON3-9 & 582.79 & 4.33 & 
589.86 & 2.60 & \bf{580.05} & 
0.47\\CON8-0 & 857.38 & 3.09 & 
881.56 & 2.89 & \bf{857.17} & 
0.02\\CON8-1 & 752.95 & 1.94 & 
768.30 & 2.25 & \bf{740.85} & 
1.63\\CON8-2 & 718.64 & 5.42 & 
737.73 & 2.90 & \bf{713.44} & 
0.73\\CON8-3 & 835.04 & 2.89 & 
852.12 & 3.08 & \bf{811.07} & 
2.96\\CON8-4 & \bf{772.25} & 3.28 & 
795.88 & 2.61 & 772.25 & 0.00\\
CON8-5 & \bf{756.91} & 4.33 & 
763.75 & 2.55 & 756.91 & 0.00\\
CON8-6 & 695.96 & 2.00 & 
698.30 & 1.54 & \bf{678.92} & 
2.51\\CON8-7 & 814.18 & 2.68 & 
832.70 & 1.80 & \bf{811.96} & 
0.27\\CON8-8 & \bf{767.53} & 1.54 & 
792.21 & 2.18 & 767.53 & 0.00\\
CON8-9 & 817.60 & 2.08 & 
827.92 & 2.72 & \bf{809.00} & 
1.06\\[1ex]\hline
\end{tabular}
\label{table:nonlin}
\end{table} \clearpage
\begin{table}[ht]
\caption{Resultados de la ejecución de la metaheurística GTS, utilizando instancias de Dethloff con la configuración -mni 3500 -lambda1 0.05 -lambda2 0.05 -tabu 7}
\centering
\small
\begin{tabular}{c c c c c c c}
\hline\hline
Instancia & Costo mínimo & Tiempo(seg.) & Costo promedio & Tiempo promedio(seg.) & Costo GTS & \%Gap \\ [0.5ex]
\hline
SCA3-0 & \bf{636.06} & 2.37 & 
638.30 & 2.54 & 636.06 & 0.00\\
SCA3-1 & \bf{697.84} & 1.04 & 
697.84 & 1.51 & 697.84 & 0.00\\
SCA3-2 & \bf{659.34} & 4.59 & 
666.01 & 2.60 & 659.34 & 0.00\\
SCA3-3 & 680.60 & 1.79 & 
683.02 & 2.21 & \bf{680.04} & 
0.08\\SCA3-4 & \bf{690.50} & 1.78 & 
690.50 & 1.89 & 690.50 & 0.00\\
SCA3-5 & \bf{659.90} & 2.61 & 
663.27 & 2.31 & 659.90 & 0.00\\
SCA3-6 & \bf{651.09} & 2.46 & 
651.74 & 2.32 & 651.09 & 0.00\\
SCA3-7 & 666.15 & 3.88 & 
668.96 & 2.42 & \bf{659.17} & 
1.06\\SCA3-8 & \bf{719.47} & 1.88 & 
719.47 & 2.10 & 719.47 & 0.00\\
SCA3-9 & \bf{681.00} & 2.30 & 
681.00 & 2.83 & 681.00 & 0.00\\
SCA8-0 & 970.65 & 2.54 & 
988.65 & 3.90 & \bf{961.50} & 
0.95\\SCA8-1 & \bf{\underline{1049.65}} & 4.48 & 
1065.14 & 3.10 & 1050.20 & 
-0.05\\SCA8-2 & \bf{1039.64} & 5.69 & 
1046.42 & 4.70 & 1039.64 & 0.00\\
SCA8-3 & \bf{983.34} & 2.30 & 
1006.39 & 2.99 & 983.34 & 0.00\\
SCA8-4 & 1067.28 & 1.85 & 
1069.35 & 2.15 & \bf{1065.49} & 
0.17\\SCA8-5 & \bf{1027.08} & 4.10 & 
1052.75 & 3.43 & 1027.08 & 0.00\\
SCA8-6 & 972.48 & 5.12 & 
976.72 & 2.76 & \bf{971.82} & 
0.07\\SCA8-7 & 1066.65 & 2.73 & 
1081.50 & 2.76 & \bf{1052.17} & 
1.38\\SCA8-8 & 1082.12 & 1.13 & 
1093.33 & 2.14 & \bf{1071.18} & 
1.02\\SCA8-9 & 1067.42 & 3.36 & 
1080.05 & 2.41 & \bf{1060.50} & 
0.65\\CON3-0 & \bf{616.52} & 2.39 & 
630.57 & 2.57 & 616.52 & 0.00\\
CON3-1 & \bf{554.47} & 2.24 & 
558.95 & 2.81 & 554.47 & 0.00\\
CON3-2 & \bf{\underline{518.00}} & 2.53 & 
520.80 & 2.49 & 519.26 & 
-0.24\\CON3-3 & \bf{591.19} & 1.16 & 
601.78 & 2.00 & 591.19 & 0.00\\
CON3-4 & \bf{\underline{588.79}} & 2.54 & 
596.10 & 1.84 & 589.32 & 
-0.09\\CON3-5 & \bf{563.70} & 2.94 & 
568.13 & 2.68 & 563.70 & 0.00\\
CON3-6 & \bf{\underline{499.05}} & 1.21 & 
507.25 & 1.67 & 500.80 & 
-0.35\\CON3-7 & 576.87 & 3.50 & 
591.01 & 2.27 & \bf{576.48} & 
0.07\\CON3-8 & \bf{523.05} & 1.32 & 
523.05 & 1.61 & 523.05 & 0.00\\
CON3-9 & \bf{\underline{578.25}} & 1.87 & 
580.79 & 2.41 & 580.05 & 
-0.31\\CON8-0 & 857.40 & 1.81 & 
863.59 & 2.66 & \bf{857.17} & 
0.03\\CON8-1 & 758.29 & 4.37 & 
770.88 & 3.19 & \bf{740.85} & 
2.35\\CON8-2 & 722.22 & 2.36 & 
732.97 & 2.69 & \bf{713.44} & 
1.23\\CON8-3 & 811.74 & 2.05 & 
846.88 & 1.94 & \bf{811.07} & 
0.08\\CON8-4 & \bf{772.25} & 2.68 & 
781.01 & 2.17 & 772.25 & 0.00\\
CON8-5 & \bf{\underline{754.88}} & 3.22 & 
761.37 & 3.02 & 756.91 & 
-0.27\\CON8-6 & \bf{678.92} & 2.75 & 
697.47 & 2.75 & 678.92 & 0.00\\
CON8-7 & 814.86 & 5.54 & 
845.33 & 2.60 & \bf{811.96} & 
0.36\\CON8-8 & 776.55 & 2.76 & 
784.17 & 2.23 & \bf{767.53} & 
1.18\\CON8-9 & 818.50 & 2.30 & 
828.96 & 3.11 & \bf{809.00} & 
1.17\\[1ex]\hline
\end{tabular}
\label{table:nonlin}
\end{table} \clearpage
\begin{table}[ht]
\caption{Resultados de la ejecución de la metaheurística GTS, utilizando instancias de Dethloff con la configuración -mni 3500 -lambda1 0.05 -lambda2 0.05 -tabu 9}
\centering
\small
\begin{tabular}{c c c c c c c}
\hline\hline
Instancia & Costo mínimo & Tiempo(seg.) & Costo promedio & Tiempo promedio(seg.) & Costo GTS & \%Gap \\ [0.5ex]
\hline
SCA3-0 & 640.55 & 3.75 & 
641.02 & 2.54 & \bf{636.06} & 
0.71\\SCA3-1 & \bf{697.84} & 2.24 & 
698.50 & 2.73 & 697.84 & 0.00\\
SCA3-2 & \bf{659.34} & 3.35 & 
666.01 & 3.01 & 659.34 & 0.00\\
SCA3-3 & \bf{680.04} & 1.40 & 
680.18 & 1.54 & 680.04 & 0.00\\
SCA3-4 & \bf{690.50} & 3.28 & 
690.50 & 2.81 & 690.50 & 0.00\\
SCA3-5 & 672.94 & 1.06 & 
673.05 & 2.10 & \bf{659.90} & 
1.98\\SCA3-6 & 653.68 & 1.05 & 
655.58 & 1.76 & \bf{651.09} & 
0.40\\SCA3-7 & 666.15 & 3.71 & 
668.02 & 2.26 & \bf{659.17} & 
1.06\\SCA3-8 & \bf{719.47} & 2.44 & 
719.47 & 2.88 & 719.47 & 0.00\\
SCA3-9 & \bf{681.00} & 2.55 & 
681.00 & 2.06 & 681.00 & 0.00\\
SCA8-0 & 970.64 & 3.83 & 
985.70 & 2.41 & \bf{961.50} & 
0.95\\SCA8-1 & 1067.41 & 1.27 & 
1070.97 & 2.13 & \bf{1050.20} & 
1.64\\SCA8-2 & 1042.10 & 3.66 & 
1053.99 & 2.96 & \bf{1039.64} & 
0.24\\SCA8-3 & \bf{983.34} & 2.60 & 
1005.80 & 3.31 & 983.34 & 0.00\\
SCA8-4 & 1067.28 & 3.00 & 
1068.22 & 3.26 & \bf{1065.49} & 
0.17\\SCA8-5 & 1034.32 & 3.24 & 
1048.55 & 2.31 & \bf{1027.08} & 
0.70\\SCA8-6 & 972.48 & 1.75 & 
984.87 & 2.15 & \bf{971.82} & 
0.07\\SCA8-7 & 1063.22 & 1.35 & 
1078.48 & 2.29 & \bf{1052.17} & 
1.05\\SCA8-8 & \bf{1071.18} & 3.98 & 
1081.82 & 1.90 & 1071.18 & 0.00\\
SCA8-9 & \bf{1060.50} & 2.70 & 
1070.57 & 2.50 & 1060.50 & 0.00\\
CON3-0 & \bf{616.52} & 2.59 & 
626.53 & 2.74 & 616.52 & 0.00\\
CON3-1 & 556.04 & 1.58 & 
557.99 & 2.32 & \bf{554.47} & 
0.28\\CON3-2 & 521.38 & 2.64 & 
525.01 & 2.29 & \bf{519.26} & 
0.41\\CON3-3 & \bf{591.19} & 1.96 & 
594.58 & 2.50 & 591.19 & 0.00\\
CON3-4 & \bf{\underline{588.79}} & 3.20 & 
594.46 & 3.12 & 589.32 & 
-0.09\\CON3-5 & \bf{563.70} & 2.18 & 
570.93 & 1.85 & 563.70 & 0.00\\
CON3-6 & 502.16 & 2.89 & 
502.16 & 2.40 & \bf{500.80} & 
0.27\\CON3-7 & \bf{576.48} & 3.40 & 
587.85 & 2.96 & 576.48 & 0.00\\
CON3-8 & \bf{523.05} & 1.05 & 
523.05 & 2.11 & 523.05 & 0.00\\
CON3-9 & 582.79 & 3.50 & 
590.65 & 3.51 & \bf{580.05} & 
0.47\\CON8-0 & 858.03 & 2.68 & 
877.70 & 2.37 & \bf{857.17} & 
0.10\\CON8-1 & \bf{740.85} & 3.78 & 
753.97 & 2.75 & 740.85 & 0.00\\
CON8-2 & 721.81 & 3.79 & 
725.46 & 2.57 & \bf{713.44} & 
1.17\\CON8-3 & 821.26 & 2.65 & 
827.12 & 2.77 & \bf{811.07} & 
1.26\\CON8-4 & \bf{772.25} & 1.88 & 
804.08 & 1.78 & 772.25 & 0.00\\
CON8-5 & \bf{756.91} & 2.82 & 
758.27 & 2.24 & 756.91 & 0.00\\
CON8-6 & 690.63 & 2.16 & 
695.70 & 2.73 & \bf{678.92} & 
1.72\\CON8-7 & 812.89 & 1.98 & 
816.59 & 2.62 & \bf{811.96} & 
0.11\\CON8-8 & \bf{767.53} & 2.21 & 
773.90 & 1.81 & 767.53 & 0.00\\
CON8-9 & 818.93 & 3.50 & 
827.88 & 2.89 & \bf{809.00} & 
1.23\\[1ex]\hline
\end{tabular}
\label{table:nonlin}
\end{table} \clearpage
\begin{table}[ht]
\caption{Resultados de la ejecución de la metaheurística GTS, utilizando instancias de Dethloff con la configuración -mni 3500 -lambda1 0.05 -lambda2 0.05 -tabu 11}
\centering
\small
\begin{tabular}{c c c c c c c}
\hline\hline
Instancia & Costo mínimo & Tiempo(seg.) & Costo promedio & Tiempo promedio(seg.) & Costo GTS & \%Gap \\ [0.5ex]
\hline
SCA3-0 & \bf{636.06} & 3.20 & 
639.43 & 2.61 & 636.06 & 0.00\\
SCA3-1 & \bf{697.84} & 1.29 & 
699.17 & 1.90 & 697.84 & 0.00\\
SCA3-2 & \bf{659.34} & 1.62 & 
659.34 & 2.52 & 659.34 & 0.00\\
SCA3-3 & \bf{680.04} & 1.21 & 
682.76 & 2.15 & 680.04 & 0.00\\
SCA3-4 & \bf{690.50} & 2.22 & 
690.50 & 2.15 & 690.50 & 0.00\\
SCA3-5 & \bf{659.90} & 2.96 & 
666.54 & 3.06 & 659.90 & 0.00\\
SCA3-6 & \bf{651.09} & 2.14 & 
651.55 & 1.99 & 651.09 & 0.00\\
SCA3-7 & 666.15 & 3.06 & 
668.02 & 2.90 & \bf{659.17} & 
1.06\\SCA3-8 & \bf{719.47} & 3.14 & 
721.90 & 2.52 & 719.47 & 0.00\\
SCA3-9 & \bf{681.00} & 2.30 & 
681.00 & 2.65 & 681.00 & 0.00\\
SCA8-0 & 970.64 & 2.29 & 
995.09 & 3.08 & \bf{961.50} & 
0.95\\SCA8-1 & 1067.61 & 2.16 & 
1075.16 & 2.17 & \bf{1050.20} & 
1.66\\SCA8-2 & \bf{1039.64} & 2.14 & 
1052.55 & 2.46 & 1039.64 & 0.00\\
SCA8-3 & 1006.05 & 3.12 & 
1014.34 & 2.48 & \bf{983.34} & 
2.31\\SCA8-4 & 1067.28 & 3.36 & 
1075.58 & 3.86 & \bf{1065.49} & 
0.17\\SCA8-5 & 1042.30 & 3.21 & 
1050.85 & 4.00 & \bf{1027.08} & 
1.48\\SCA8-6 & 972.48 & 2.52 & 
981.34 & 1.99 & \bf{971.82} & 
0.07\\SCA8-7 & \bf{1052.17} & 4.76 & 
1083.09 & 3.02 & 1052.17 & 0.00\\
SCA8-8 & 1082.12 & 2.88 & 
1087.28 & 3.06 & \bf{1071.18} & 
1.02\\SCA8-9 & \bf{1060.50} & 6.88 & 
1068.41 & 3.83 & 1060.50 & 0.00\\
CON3-0 & \bf{616.52} & 2.75 & 
626.05 & 2.15 & 616.52 & 0.00\\
CON3-1 & 556.04 & 1.61 & 
558.51 & 2.14 & \bf{554.47} & 
0.28\\CON3-2 & 521.38 & 2.91 & 
522.99 & 2.35 & \bf{519.26} & 
0.41\\CON3-3 & \bf{591.19} & 2.79 & 
600.01 & 2.02 & 591.19 & 0.00\\
CON3-4 & 601.12 & 1.44 & 
602.03 & 2.21 & \bf{589.32} & 
2.00\\CON3-5 & \bf{563.70} & 3.18 & 
565.92 & 2.78 & 563.70 & 0.00\\
CON3-6 & \bf{\underline{500.37}} & 1.65 & 
502.21 & 2.44 & 500.80 & 
-0.09\\CON3-7 & \bf{576.48} & 3.41 & 
582.60 & 2.29 & 576.48 & 0.00\\
CON3-8 & \bf{523.05} & 1.63 & 
523.23 & 1.91 & 523.05 & 0.00\\
CON3-9 & \bf{\underline{578.25}} & 3.72 & 
586.76 & 3.12 & 580.05 & 
-0.31\\CON8-0 & 857.40 & 5.44 & 
875.10 & 4.12 & \bf{857.17} & 
0.03\\CON8-1 & 752.18 & 2.71 & 
760.60 & 3.21 & \bf{740.85} & 
1.53\\CON8-2 & \bf{713.44} & 0.96 & 
723.25 & 2.29 & 713.44 & 0.00\\
CON8-3 & 812.22 & 3.10 & 
828.99 & 3.72 & \bf{811.07} & 
0.14\\CON8-4 & \bf{772.25} & 4.38 & 
790.77 & 2.67 & 772.25 & 0.00\\
CON8-5 & 758.99 & 3.63 & 
761.68 & 2.52 & \bf{756.91} & 
0.27\\CON8-6 & 690.63 & 3.05 & 
695.56 & 2.48 & \bf{678.92} & 
1.72\\CON8-7 & 812.89 & 4.18 & 
813.86 & 3.01 & \bf{811.96} & 
0.11\\CON8-8 & \bf{767.53} & 2.95 & 
791.18 & 2.45 & 767.53 & 0.00\\
CON8-9 & \bf{809.00} & 4.43 & 
812.19 & 3.71 & 809.00 & 0.00\\
[1ex]\hline
\end{tabular}
\label{table:nonlin}
\end{table} \clearpage
\begin{table}[ht]
\caption{Resultados de la ejecución de la metaheurística GTS, utilizando instancias de Dethloff con la configuración -mni 3500 -lambda1 0.05 -lambda2 0.05 -tabu 13}
\centering
\small
\begin{tabular}{c c c c c c c}
\hline\hline
Instancia & Costo mínimo & Tiempo(seg.) & Costo promedio & Tiempo promedio(seg.) & Costo GTS & \%Gap \\ [0.5ex]
\hline
SCA3-0 & 640.55 & 1.67 & 
640.55 & 2.48 & \bf{636.06} & 
0.71\\SCA3-1 & \bf{697.84} & 4.70 & 
698.50 & 2.60 & 697.84 & 0.00\\
SCA3-2 & \bf{659.34} & 4.71 & 
659.34 & 3.04 & 659.34 & 0.00\\
SCA3-3 & 680.60 & 1.81 & 
687.74 & 2.02 & \bf{680.04} & 
0.08\\SCA3-4 & \bf{690.50} & 2.55 & 
690.50 & 2.62 & 690.50 & 0.00\\
SCA3-5 & \bf{659.90} & 5.80 & 
663.16 & 4.16 & 659.90 & 0.00\\
SCA3-6 & \bf{651.09} & 3.39 & 
654.10 & 2.21 & 651.09 & 0.00\\
SCA3-7 & 666.15 & 3.80 & 
667.09 & 2.91 & \bf{659.17} & 
1.06\\SCA3-8 & \bf{719.47} & 2.07 & 
719.47 & 1.81 & 719.47 & 0.00\\
SCA3-9 & \bf{681.00} & 2.27 & 
681.00 & 2.35 & 681.00 & 0.00\\
SCA8-0 & 982.13 & 4.39 & 
993.97 & 4.48 & \bf{961.50} & 
2.15\\SCA8-1 & \bf{1050.20} & 3.76 & 
1068.54 & 3.04 & 1050.20 & 0.00\\
SCA8-2 & 1064.23 & 3.73 & 
1070.00 & 2.99 & \bf{1039.64} & 
2.37\\SCA8-3 & \bf{983.34} & 6.38 & 
1004.52 & 3.31 & 983.34 & 0.00\\
SCA8-4 & 1067.55 & 1.69 & 
1087.65 & 2.25 & \bf{1065.49} & 
0.19\\SCA8-5 & 1042.30 & 1.44 & 
1049.68 & 1.89 & \bf{1027.08} & 
1.48\\SCA8-6 & 972.48 & 3.46 & 
976.25 & 3.52 & \bf{971.82} & 
0.07\\SCA8-7 & 1063.22 & 2.18 & 
1078.89 & 1.51 & \bf{1052.17} & 
1.05\\SCA8-8 & \bf{1071.18} & 2.14 & 
1085.86 & 1.93 & 1071.18 & 0.00\\
SCA8-9 & \bf{1060.50} & 1.75 & 
1068.78 & 1.75 & 1060.50 & 0.00\\
CON3-0 & 627.15 & 1.19 & 
628.05 & 2.37 & \bf{616.52} & 
1.72\\CON3-1 & \bf{554.47} & 2.90 & 
558.56 & 2.06 & 554.47 & 0.00\\
CON3-2 & 523.20 & 1.62 & 
525.28 & 2.06 & \bf{519.26} & 
0.76\\CON3-3 & \bf{591.19} & 2.98 & 
606.79 & 2.59 & 591.19 & 0.00\\
CON3-4 & \bf{\underline{588.79}} & 2.36 & 
593.20 & 3.19 & 589.32 & 
-0.09\\CON3-5 & \bf{563.70} & 3.26 & 
570.93 & 2.54 & 563.70 & 0.00\\
CON3-6 & \bf{\underline{499.05}} & 3.03 & 
500.32 & 2.42 & 500.80 & 
-0.35\\CON3-7 & \bf{576.48} & 1.79 & 
577.45 & 1.95 & 576.48 & 0.00\\
CON3-8 & \bf{523.05} & 2.56 & 
523.05 & 1.98 & 523.05 & 0.00\\
CON3-9 & \bf{\underline{578.25}} & 5.17 & 
586.37 & 2.84 & 580.05 & 
-0.31\\CON8-0 & 858.63 & 4.55 & 
900.76 & 3.31 & \bf{857.17} & 
0.17\\CON8-1 & \bf{740.85} & 4.70 & 
759.78 & 3.13 & 740.85 & 0.00\\
CON8-2 & \bf{713.44} & 2.38 & 
733.64 & 2.70 & 713.44 & 0.00\\
CON8-3 & \bf{811.07} & 2.96 & 
843.34 & 2.33 & 811.07 & 0.00\\
CON8-4 & \bf{772.25} & 2.08 & 
786.79 & 2.33 & 772.25 & 0.00\\
CON8-5 & 759.93 & 2.56 & 
773.49 & 1.97 & \bf{756.91} & 
0.40\\CON8-6 & 692.55 & 3.32 & 
697.51 & 3.27 & \bf{678.92} & 
2.01\\CON8-7 & 813.64 & 3.07 & 
820.38 & 3.15 & \bf{811.96} & 
0.21\\CON8-8 & \bf{767.53} & 1.57 & 
779.60 & 1.90 & 767.53 & 0.00\\
CON8-9 & 812.60 & 3.28 & 
842.57 & 3.33 & \bf{809.00} & 
0.44\\[1ex]\hline
\end{tabular}
\label{table:nonlin}
\end{table} \clearpage
\begin{table}[ht]
\caption{Resultados de la ejecución de la metaheurística GTS, utilizando instancias de Dethloff con la configuración -mni 3500 -lambda1 0.05 -lambda2 0.05 -tabu 15}
\centering
\small
\begin{tabular}{c c c c c c c}
\hline\hline
Instancia & Costo mínimo & Tiempo(seg.) & Costo promedio & Tiempo promedio(seg.) & Costo GTS & \%Gap \\ [0.5ex]
\hline
SCA3-0 & \bf{636.06} & 1.74 & 
640.08 & 1.54 & 636.06 & 0.00\\
SCA3-1 & \bf{697.84} & 3.35 & 
702.06 & 2.22 & 697.84 & 0.00\\
SCA3-2 & \bf{659.34} & 2.05 & 
659.34 & 2.79 & 659.34 & 0.00\\
SCA3-3 & \bf{680.04} & 2.02 & 
683.02 & 2.08 & 680.04 & 0.00\\
SCA3-4 & \bf{690.50} & 3.16 & 
690.50 & 2.81 & 690.50 & 0.00\\
SCA3-5 & \bf{659.90} & 1.32 & 
659.90 & 2.64 & 659.90 & 0.00\\
SCA3-6 & \bf{651.09} & 2.90 & 
651.55 & 2.73 & 651.09 & 0.00\\
SCA3-7 & 666.15 & 2.10 & 
667.64 & 2.26 & \bf{659.17} & 
1.06\\SCA3-8 & \bf{719.47} & 1.42 & 
719.47 & 1.51 & 719.47 & 0.00\\
SCA3-9 & \bf{681.00} & 2.15 & 
681.00 & 2.58 & 681.00 & 0.00\\
SCA8-0 & 968.63 & 4.11 & 
974.45 & 3.19 & \bf{961.50} & 
0.74\\SCA8-1 & \bf{\underline{1049.65}} & 2.55 & 
1062.07 & 2.88 & 1050.20 & 
-0.05\\SCA8-2 & \bf{1039.64} & 4.06 & 
1045.06 & 2.42 & 1039.64 & 0.00\\
SCA8-3 & 1005.65 & 2.45 & 
1012.27 & 3.12 & \bf{983.34} & 
2.27\\SCA8-4 & 1067.55 & 4.19 & 
1079.01 & 2.92 & \bf{1065.49} & 
0.19\\SCA8-5 & \bf{1027.08} & 2.05 & 
1033.87 & 2.54 & 1027.08 & 0.00\\
SCA8-6 & 972.48 & 1.38 & 
981.79 & 2.29 & \bf{971.82} & 
0.07\\SCA8-7 & \bf{\underline{1051.28}} & 5.03 & 
1079.51 & 3.31 & 1052.17 & 
-0.08\\SCA8-8 & \bf{1071.18} & 1.58 & 
1077.22 & 1.30 & 1071.18 & 0.00\\
SCA8-9 & \bf{1060.50} & 3.08 & 
1065.30 & 2.68 & 1060.50 & 0.00\\
CON3-0 & \bf{616.52} & 2.30 & 
629.41 & 2.92 & 616.52 & 0.00\\
CON3-1 & 556.04 & 1.63 & 
556.62 & 1.90 & \bf{554.47} & 
0.28\\CON3-2 & \bf{519.26} & 4.06 & 
521.77 & 3.19 & 519.26 & 0.00\\
CON3-3 & \bf{591.19} & 2.94 & 
591.19 & 3.40 & 591.19 & 0.00\\
CON3-4 & \bf{\underline{588.79}} & 2.35 & 
593.20 & 1.77 & 589.32 & 
-0.09\\CON3-5 & \bf{563.70} & 2.29 & 
570.39 & 2.44 & 563.70 & 0.00\\
CON3-6 & \bf{\underline{499.05}} & 3.07 & 
499.49 & 3.19 & 500.80 & 
-0.35\\CON3-7 & \bf{576.48} & 2.60 & 
584.36 & 2.70 & 576.48 & 0.00\\
CON3-8 & \bf{523.05} & 1.12 & 
523.05 & 2.59 & 523.05 & 0.00\\
CON3-9 & \bf{\underline{578.25}} & 1.71 & 
585.72 & 1.66 & 580.05 & 
-0.31\\CON8-0 & 884.00 & 1.77 & 
899.86 & 2.72 & \bf{857.17} & 
3.13\\CON8-1 & \bf{740.85} & 2.44 & 
769.89 & 2.33 & 740.85 & 0.00\\
CON8-2 & 718.70 & 2.02 & 
725.90 & 2.66 & \bf{713.44} & 
0.74\\CON8-3 & 841.64 & 2.04 & 
856.42 & 2.06 & \bf{811.07} & 
3.77\\CON8-4 & \bf{772.25} & 3.67 & 
779.28 & 3.14 & 772.25 & 0.00\\
CON8-5 & 758.12 & 4.74 & 
761.78 & 2.85 & \bf{756.91} & 
0.16\\CON8-6 & \bf{678.92} & 3.58 & 
692.07 & 2.55 & 678.92 & 0.00\\
CON8-7 & 812.89 & 3.51 & 
826.78 & 2.56 & \bf{811.96} & 
0.11\\CON8-8 & \bf{767.53} & 2.06 & 
776.28 & 3.44 & 767.53 & 0.00\\
CON8-9 & \bf{809.00} & 2.99 & 
813.02 & 2.28 & 809.00 & 0.00\\
[1ex]\hline
\end{tabular}
\label{table:nonlin}
\end{table} \clearpage
\begin{table}[ht]
\caption{Resultados de la ejecución de la metaheurística GTS, utilizando instancias de Dethloff con la configuración -mni 4000 -lambda1 0.05 -lambda2 0.05 -tabu 5}
\centering
\small
\begin{tabular}{c c c c c c c}
\hline\hline
Instancia & Costo mínimo & Tiempo(seg.) & Costo promedio & Tiempo promedio(seg.) & Costo GTS & \%Gap \\ [0.5ex]
\hline
SCA3-0 & \bf{636.06} & 3.91 & 
640.86 & 3.17 & 636.06 & 0.00\\
SCA3-1 & \bf{697.84} & 4.63 & 
706.98 & 2.50 & 697.84 & 0.00\\
SCA3-2 & \bf{659.34} & 2.29 & 
659.34 & 3.11 & 659.34 & 0.00\\
SCA3-3 & \bf{680.04} & 2.88 & 
685.07 & 2.37 & 680.04 & 0.00\\
SCA3-4 & \bf{690.50} & 3.72 & 
690.50 & 2.98 & 690.50 & 0.00\\
SCA3-5 & \bf{659.90} & 3.68 & 
659.90 & 2.17 & 659.90 & 0.00\\
SCA3-6 & \bf{651.09} & 4.58 & 
656.25 & 2.75 & 651.09 & 0.00\\
SCA3-7 & \bf{659.17} & 3.36 & 
665.78 & 2.47 & 659.17 & 0.00\\
SCA3-8 & \bf{719.47} & 4.52 & 
719.47 & 2.33 & 719.47 & 0.00\\
SCA3-9 & \bf{681.00} & 2.82 & 
681.00 & 1.98 & 681.00 & 0.00\\
SCA8-0 & \bf{961.50} & 2.24 & 
991.53 & 5.04 & 961.50 & 0.00\\
SCA8-1 & \bf{\underline{1049.65}} & 3.68 & 
1068.96 & 2.90 & 1050.20 & 
-0.05\\SCA8-2 & 1050.37 & 1.31 & 
1055.66 & 2.40 & \bf{1039.64} & 
1.03\\SCA8-3 & 1005.65 & 2.50 & 
1011.38 & 2.26 & \bf{983.34} & 
2.27\\SCA8-4 & 1067.29 & 2.43 & 
1074.83 & 2.62 & \bf{1065.49} & 
0.17\\SCA8-5 & 1029.95 & 3.09 & 
1046.48 & 2.95 & \bf{1027.08} & 
0.28\\SCA8-6 & 972.48 & 2.17 & 
977.98 & 1.87 & \bf{971.82} & 
0.07\\SCA8-7 & \bf{\underline{1051.28}} & 5.42 & 
1066.41 & 3.77 & 1052.17 & 
-0.08\\SCA8-8 & \bf{1071.18} & 1.77 & 
1078.40 & 2.17 & 1071.18 & 0.00\\
SCA8-9 & 1063.68 & 3.61 & 
1071.30 & 3.92 & \bf{1060.50} & 
0.30\\CON3-0 & \bf{616.52} & 2.11 & 
629.52 & 3.63 & 616.52 & 0.00\\
CON3-1 & \bf{554.47} & 1.92 & 
558.14 & 1.94 & 554.47 & 0.00\\
CON3-2 & 522.11 & 1.38 & 
525.72 & 1.92 & \bf{519.26} & 
0.55\\CON3-3 & \bf{591.19} & 3.31 & 
591.19 & 3.35 & 591.19 & 0.00\\
CON3-4 & 596.29 & 1.70 & 
598.69 & 2.09 & \bf{589.32} & 
1.18\\CON3-5 & \bf{563.70} & 1.44 & 
568.20 & 2.29 & 563.70 & 0.00\\
CON3-6 & \bf{\underline{499.05}} & 2.16 & 
501.54 & 1.51 & 500.80 & 
-0.35\\CON3-7 & \bf{576.48} & 4.60 & 
583.11 & 3.21 & 576.48 & 0.00\\
CON3-8 & \bf{523.05} & 3.26 & 
523.21 & 3.42 & 523.05 & 0.00\\
CON3-9 & \bf{\underline{578.25}} & 4.20 & 
585.17 & 2.48 & 580.05 & 
-0.31\\CON8-0 & 858.03 & 2.10 & 
870.97 & 2.50 & \bf{857.17} & 
0.10\\CON8-1 & 741.70 & 2.06 & 
766.58 & 2.94 & \bf{740.85} & 
0.11\\CON8-2 & 718.52 & 4.11 & 
723.36 & 3.45 & \bf{713.44} & 
0.71\\CON8-3 & 821.26 & 1.82 & 
823.15 & 3.33 & \bf{811.07} & 
1.26\\CON8-4 & \bf{772.25} & 1.97 & 
781.51 & 3.42 & 772.25 & 0.00\\
CON8-5 & \bf{756.91} & 3.68 & 
757.91 & 2.98 & 756.91 & 0.00\\
CON8-6 & 690.63 & 4.42 & 
697.35 & 3.11 & \bf{678.92} & 
1.72\\CON8-7 & \bf{811.96} & 4.44 & 
826.48 & 3.37 & 811.96 & 0.00\\
CON8-8 & \bf{767.53} & 2.56 & 
788.39 & 3.06 & 767.53 & 0.00\\
CON8-9 & 812.79 & 2.22 & 
815.09 & 3.07 & \bf{809.00} & 
0.47\\[1ex]\hline
\end{tabular}
\label{table:nonlin}
\end{table} \clearpage
\begin{table}[ht]
\caption{Resultados de la ejecución de la metaheurística GTS, utilizando instancias de Dethloff con la configuración -mni 4000 -lambda1 0.05 -lambda2 0.05 -tabu 7}
\centering
\small
\begin{tabular}{c c c c c c c}
\hline\hline
Instancia & Costo mínimo & Tiempo(seg.) & Costo promedio & Tiempo promedio(seg.) & Costo GTS & \%Gap \\ [0.5ex]
\hline
SCA3-0 & 640.55 & 4.74 & 
640.55 & 3.37 & \bf{636.06} & 
0.71\\SCA3-1 & \bf{697.84} & 2.25 & 
699.17 & 3.10 & 697.84 & 0.00\\
SCA3-2 & \bf{659.34} & 1.94 & 
659.34 & 2.12 & 659.34 & 0.00\\
SCA3-3 & 680.60 & 4.95 & 
683.19 & 2.63 & \bf{680.04} & 
0.08\\SCA3-4 & \bf{690.50} & 4.82 & 
690.50 & 3.39 & 690.50 & 0.00\\
SCA3-5 & \bf{659.90} & 1.80 & 
669.25 & 2.54 & 659.90 & 0.00\\
SCA3-6 & \bf{651.09} & 2.40 & 
653.70 & 2.45 & 651.09 & 0.00\\
SCA3-7 & 666.15 & 3.38 & 
668.02 & 2.99 & \bf{659.17} & 
1.06\\SCA3-8 & \bf{719.47} & 3.34 & 
721.90 & 3.65 & 719.47 & 0.00\\
SCA3-9 & \bf{681.00} & 2.75 & 
681.00 & 2.74 & 681.00 & 0.00\\
SCA8-0 & \bf{961.50} & 3.64 & 
986.28 & 3.40 & 961.50 & 0.00\\
SCA8-1 & 1068.14 & 2.10 & 
1068.36 & 3.51 & \bf{1050.20} & 
1.71\\SCA8-2 & 1046.08 & 2.54 & 
1054.43 & 3.31 & \bf{1039.64} & 
0.62\\SCA8-3 & \bf{983.34} & 5.45 & 
994.14 & 4.11 & 983.34 & 0.00\\
SCA8-4 & 1075.27 & 2.51 & 
1083.20 & 2.16 & \bf{1065.49} & 
0.92\\SCA8-5 & \bf{1027.08} & 5.20 & 
1050.20 & 3.10 & 1027.08 & 0.00\\
SCA8-6 & 972.48 & 2.82 & 
972.48 & 3.98 & \bf{971.82} & 
0.07\\SCA8-7 & 1063.60 & 3.15 & 
1072.69 & 4.85 & \bf{1052.17} & 
1.09\\SCA8-8 & \bf{1071.18} & 1.43 & 
1081.12 & 2.16 & 1071.18 & 0.00\\
SCA8-9 & \bf{1060.50} & 3.48 & 
1064.58 & 3.23 & 1060.50 & 0.00\\
CON3-0 & \bf{616.52} & 3.00 & 
616.79 & 2.74 & 616.52 & 0.00\\
CON3-1 & \bf{554.47} & 5.70 & 
557.04 & 3.25 & 554.47 & 0.00\\
CON3-2 & 522.86 & 5.86 & 
523.81 & 3.50 & \bf{519.26} & 
0.69\\CON3-3 & \bf{591.19} & 2.53 & 
591.19 & 2.98 & 591.19 & 0.00\\
CON3-4 & \bf{\underline{588.79}} & 2.42 & 
591.80 & 2.84 & 589.32 & 
-0.09\\CON3-5 & \bf{563.70} & 2.48 & 
565.37 & 2.67 & 563.70 & 0.00\\
CON3-6 & 502.16 & 2.75 & 
502.16 & 3.38 & \bf{500.80} & 
0.27\\CON3-7 & \bf{576.48} & 3.26 & 
580.75 & 2.32 & 576.48 & 0.00\\
CON3-8 & \bf{523.05} & 3.41 & 
523.21 & 1.96 & 523.05 & 0.00\\
CON3-9 & \bf{\underline{578.25}} & 5.62 & 
582.90 & 3.94 & 580.05 & 
-0.31\\CON8-0 & 867.04 & 3.40 & 
885.96 & 2.48 & \bf{857.17} & 
1.15\\CON8-1 & \bf{740.85} & 6.12 & 
748.01 & 3.12 & 740.85 & 0.00\\
CON8-2 & \bf{\underline{713.10}} & 3.43 & 
722.45 & 3.34 & 713.44 & 
-0.05\\CON8-3 & \bf{811.07} & 3.32 & 
853.88 & 2.50 & 811.07 & 0.00\\
CON8-4 & \bf{772.25} & 3.02 & 
781.42 & 3.31 & 772.25 & 0.00\\
CON8-5 & \bf{\underline{754.95}} & 3.78 & 
756.72 & 4.27 & 756.91 & 
-0.26\\CON8-6 & 692.33 & 1.27 & 
698.40 & 2.39 & \bf{678.92} & 
1.98\\CON8-7 & 814.07 & 1.67 & 
837.20 & 2.83 & \bf{811.96} & 
0.26\\CON8-8 & \bf{767.53} & 3.40 & 
788.35 & 3.22 & 767.53 & 0.00\\
CON8-9 & 811.43 & 4.58 & 
829.19 & 3.03 & \bf{809.00} & 
0.30\\[1ex]\hline
\end{tabular}
\label{table:nonlin}
\end{table} \clearpage
\begin{table}[ht]
\caption{Resultados de la ejecución de la metaheurística GTS, utilizando instancias de Dethloff con la configuración -mni 4000 -lambda1 0.05 -lambda2 0.05 -tabu 9}
\centering
\small
\begin{tabular}{c c c c c c c}
\hline\hline
Instancia & Costo mínimo & Tiempo(seg.) & Costo promedio & Tiempo promedio(seg.) & Costo GTS & \%Gap \\ [0.5ex]
\hline
SCA3-0 & 640.55 & 2.06 & 
640.55 & 1.84 & \bf{636.06} & 
0.71\\SCA3-1 & \bf{697.84} & 4.15 & 
698.76 & 3.27 & 697.84 & 0.00\\
SCA3-2 & \bf{659.34} & 4.45 & 
659.34 & 2.28 & 659.34 & 0.00\\
SCA3-3 & \bf{680.04} & 1.66 & 
685.34 & 2.65 & 680.04 & 0.00\\
SCA3-4 & \bf{690.50} & 3.31 & 
696.75 & 2.85 & 690.50 & 0.00\\
SCA3-5 & \bf{659.90} & 1.75 & 
663.16 & 1.83 & 659.90 & 0.00\\
SCA3-6 & \bf{651.09} & 2.14 & 
654.75 & 2.35 & 651.09 & 0.00\\
SCA3-7 & 666.15 & 2.62 & 
668.02 & 2.27 & \bf{659.17} & 
1.06\\SCA3-8 & \bf{719.47} & 2.08 & 
729.01 & 1.64 & 719.47 & 0.00\\
SCA3-9 & \bf{681.00} & 2.05 & 
681.00 & 2.62 & 681.00 & 0.00\\
SCA8-0 & 970.64 & 2.75 & 
984.55 & 3.01 & \bf{961.50} & 
0.95\\SCA8-1 & 1050.93 & 3.27 & 
1073.44 & 2.29 & \bf{1050.20} & 
0.07\\SCA8-2 & 1050.37 & 2.02 & 
1066.41 & 3.03 & \bf{1039.64} & 
1.03\\SCA8-3 & 1002.38 & 2.77 & 
1010.69 & 2.60 & \bf{983.34} & 
1.94\\SCA8-4 & 1067.29 & 1.85 & 
1071.11 & 2.06 & \bf{1065.49} & 
0.17\\SCA8-5 & 1039.64 & 3.75 & 
1051.23 & 2.62 & \bf{1027.08} & 
1.22\\SCA8-6 & \bf{971.82} & 2.66 & 
972.32 & 2.88 & 971.82 & 0.00\\
SCA8-7 & 1067.15 & 1.80 & 
1073.52 & 3.06 & \bf{1052.17} & 
1.42\\SCA8-8 & \bf{1071.18} & 1.57 & 
1080.83 & 2.11 & 1071.18 & 0.00\\
SCA8-9 & 1063.68 & 2.33 & 
1068.10 & 2.44 & \bf{1060.50} & 
0.30\\CON3-0 & 628.47 & 1.38 & 
631.19 & 2.67 & \bf{616.52} & 
1.94\\CON3-1 & 557.07 & 2.56 & 
561.41 & 1.79 & \bf{554.47} & 
0.47\\CON3-2 & 523.23 & 2.89 & 
524.93 & 2.61 & \bf{519.26} & 
0.76\\CON3-3 & \bf{591.19} & 5.55 & 
594.58 & 3.35 & 591.19 & 0.00\\
CON3-4 & 591.43 & 1.28 & 
595.78 & 3.30 & \bf{589.32} & 
0.36\\CON3-5 & \bf{563.70} & 1.65 & 
570.21 & 2.37 & 563.70 & 0.00\\
CON3-6 & \bf{500.80} & 2.54 & 
502.94 & 2.40 & 500.80 & 0.00\\
CON3-7 & \bf{576.48} & 1.99 & 
582.23 & 3.02 & 576.48 & 0.00\\
CON3-8 & \bf{523.05} & 4.08 & 
523.21 & 2.73 & 523.05 & 0.00\\
CON3-9 & 587.23 & 2.84 & 
591.53 & 3.06 & \bf{580.05} & 
1.24\\CON8-0 & 914.13 & 2.22 & 
923.23 & 2.67 & \bf{857.17} & 
6.65\\CON8-1 & 753.38 & 3.16 & 
763.12 & 3.27 & \bf{740.85} & 
1.69\\CON8-2 & \bf{\underline{712.89}} & 5.54 & 
720.67 & 3.00 & 713.44 & 
-0.08\\CON8-3 & \bf{811.07} & 2.66 & 
822.96 & 3.43 & 811.07 & 0.00\\
CON8-4 & \bf{772.25} & 3.33 & 
782.36 & 3.68 & 772.25 & 0.00\\
CON8-5 & \bf{756.91} & 4.59 & 
761.08 & 4.04 & 756.91 & 0.00\\
CON8-6 & 688.47 & 7.76 & 
699.91 & 4.21 & \bf{678.92} & 
1.41\\CON8-7 & 812.89 & 2.81 & 
832.39 & 2.96 & \bf{811.96} & 
0.11\\CON8-8 & 777.84 & 2.21 & 
786.58 & 3.25 & \bf{767.53} & 
1.34\\CON8-9 & \bf{809.00} & 5.17 & 
813.72 & 4.59 & 809.00 & 0.00\\
[1ex]\hline
\end{tabular}
\label{table:nonlin}
\end{table} \clearpage
\begin{table}[ht]
\caption{Resultados de la ejecución de la metaheurística GTS, utilizando instancias de Dethloff con la configuración -mni 4000 -lambda1 0.05 -lambda2 0.05 -tabu 11}
\centering
\small
\begin{tabular}{c c c c c c c}
\hline\hline
Instancia & Costo mínimo & Tiempo(seg.) & Costo promedio & Tiempo promedio(seg.) & Costo GTS & \%Gap \\ [0.5ex]
\hline
SCA3-0 & \bf{636.06} & 1.66 & 
639.43 & 2.19 & 636.06 & 0.00\\
SCA3-1 & \bf{697.84} & 2.78 & 
697.84 & 3.75 & 697.84 & 0.00\\
SCA3-2 & \bf{659.34} & 1.96 & 
659.34 & 1.90 & 659.34 & 0.00\\
SCA3-3 & \bf{680.04} & 1.99 & 
682.89 & 3.94 & 680.04 & 0.00\\
SCA3-4 & \bf{690.50} & 2.51 & 
690.50 & 3.21 & 690.50 & 0.00\\
SCA3-5 & \bf{659.90} & 2.98 & 
663.16 & 2.71 & 659.90 & 0.00\\
SCA3-6 & \bf{651.09} & 3.68 & 
653.53 & 2.73 & 651.09 & 0.00\\
SCA3-7 & 666.15 & 3.59 & 
667.09 & 3.19 & \bf{659.17} & 
1.06\\SCA3-8 & \bf{719.47} & 3.72 & 
719.47 & 2.99 & 719.47 & 0.00\\
SCA3-9 & \bf{681.00} & 2.07 & 
681.00 & 2.38 & 681.00 & 0.00\\
SCA8-0 & \bf{961.50} & 3.63 & 
973.49 & 4.21 & 961.50 & 0.00\\
SCA8-1 & 1059.42 & 3.55 & 
1071.72 & 2.29 & \bf{1050.20} & 
0.88\\SCA8-2 & 1042.17 & 2.61 & 
1062.95 & 2.63 & \bf{1039.64} & 
0.24\\SCA8-3 & \bf{983.34} & 8.05 & 
1010.27 & 3.98 & 983.34 & 0.00\\
SCA8-4 & 1068.97 & 1.50 & 
1071.76 & 4.38 & \bf{1065.49} & 
0.33\\SCA8-5 & 1042.30 & 1.54 & 
1044.74 & 2.90 & \bf{1027.08} & 
1.48\\SCA8-6 & 972.48 & 2.69 & 
981.53 & 2.54 & \bf{971.82} & 
0.07\\SCA8-7 & \bf{\underline{1051.28}} & 4.17 & 
1068.50 & 3.67 & 1052.17 & 
-0.08\\SCA8-8 & \bf{1071.18} & 3.02 & 
1080.14 & 2.58 & 1071.18 & 0.00\\
SCA8-9 & \bf{1060.50} & 5.29 & 
1061.81 & 2.84 & 1060.50 & 0.00\\
CON3-0 & \bf{616.52} & 3.38 & 
622.70 & 2.91 & 616.52 & 0.00\\
CON3-1 & \bf{554.47} & 3.16 & 
557.54 & 2.46 & 554.47 & 0.00\\
CON3-2 & \bf{\underline{518.00}} & 3.36 & 
521.89 & 2.47 & 519.26 & 
-0.24\\CON3-3 & \bf{591.19} & 2.21 & 
597.97 & 2.90 & 591.19 & 0.00\\
CON3-4 & 589.88 & 3.76 & 
597.50 & 3.37 & \bf{589.32} & 
0.10\\CON3-5 & \bf{563.70} & 7.14 & 
568.25 & 3.92 & 563.70 & 0.00\\
CON3-6 & \bf{\underline{499.05}} & 2.71 & 
499.83 & 2.92 & 500.80 & 
-0.35\\CON3-7 & \bf{576.48} & 2.81 & 
586.37 & 2.42 & 576.48 & 0.00\\
CON3-8 & \bf{523.05} & 3.24 & 
523.05 & 2.66 & 523.05 & 0.00\\
CON3-9 & \bf{\underline{578.25}} & 1.70 & 
582.74 & 3.30 & 580.05 & 
-0.31\\CON8-0 & 892.06 & 4.53 & 
905.70 & 3.42 & \bf{857.17} & 
4.07\\CON8-1 & 756.50 & 5.51 & 
762.68 & 2.61 & \bf{740.85} & 
2.11\\CON8-2 & \bf{\underline{712.89}} & 5.48 & 
720.66 & 3.30 & 713.44 & 
-0.08\\CON8-3 & 812.54 & 2.18 & 
824.39 & 2.42 & \bf{811.07} & 
0.18\\CON8-4 & \bf{772.25} & 2.84 & 
772.25 & 3.25 & 772.25 & 0.00\\
CON8-5 & \bf{756.91} & 2.66 & 
759.18 & 3.59 & 756.91 & 0.00\\
CON8-6 & 678.99 & 4.22 & 
696.00 & 3.00 & \bf{678.92} & 
0.01\\CON8-7 & 812.89 & 3.90 & 
813.87 & 3.54 & \bf{811.96} & 
0.11\\CON8-8 & \bf{767.53} & 3.59 & 
773.49 & 2.86 & 767.53 & 0.00\\
CON8-9 & 810.30 & 2.13 & 
834.45 & 3.62 & \bf{809.00} & 
0.16\\[1ex]\hline
\end{tabular}
\label{table:nonlin}
\end{table} \clearpage
\begin{table}[ht]
\caption{Resultados de la ejecución de la metaheurística GTS, utilizando instancias de Dethloff con la configuración -mni 4000 -lambda1 0.05 -lambda2 0.05 -tabu 13}
\centering
\small
\begin{tabular}{c c c c c c c}
\hline\hline
Instancia & Costo mínimo & Tiempo(seg.) & Costo promedio & Tiempo promedio(seg.) & Costo GTS & \%Gap \\ [0.5ex]
\hline
SCA3-0 & 640.55 & 2.06 & 
649.87 & 1.91 & \bf{636.06} & 
0.71\\SCA3-1 & \bf{697.84} & 1.56 & 
699.17 & 2.88 & 697.84 & 0.00\\
SCA3-2 & \bf{659.34} & 1.58 & 
659.34 & 2.02 & 659.34 & 0.00\\
SCA3-3 & \bf{680.04} & 5.56 & 
685.61 & 3.04 & 680.04 & 0.00\\
SCA3-4 & \bf{690.50} & 1.98 & 
690.50 & 3.67 & 690.50 & 0.00\\
SCA3-5 & \bf{659.90} & 1.81 & 
663.16 & 2.29 & 659.90 & 0.00\\
SCA3-6 & \bf{651.09} & 3.47 & 
653.94 & 2.44 & 651.09 & 0.00\\
SCA3-7 & 666.15 & 2.17 & 
666.15 & 1.86 & \bf{659.17} & 
1.06\\SCA3-8 & \bf{719.47} & 3.06 & 
719.47 & 2.67 & 719.47 & 0.00\\
SCA3-9 & \bf{681.00} & 2.21 & 
681.00 & 3.06 & 681.00 & 0.00\\
SCA8-0 & \bf{961.50} & 3.79 & 
988.47 & 2.19 & 961.50 & 0.00\\
SCA8-1 & \bf{\underline{1049.66}} & 2.93 & 
1050.11 & 2.99 & 1050.20 & 
-0.05\\SCA8-2 & 1042.17 & 6.41 & 
1061.08 & 3.98 & \bf{1039.64} & 
0.24\\SCA8-3 & 1005.65 & 2.95 & 
1010.88 & 2.80 & \bf{983.34} & 
2.27\\SCA8-4 & 1067.55 & 3.20 & 
1075.63 & 2.62 & \bf{1065.49} & 
0.19\\SCA8-5 & \bf{1027.08} & 2.55 & 
1042.99 & 2.37 & 1027.08 & 0.00\\
SCA8-6 & 972.48 & 3.05 & 
977.98 & 2.60 & \bf{971.82} & 
0.07\\SCA8-7 & 1066.65 & 8.41 & 
1079.03 & 3.75 & \bf{1052.17} & 
1.38\\SCA8-8 & \bf{1071.18} & 1.80 & 
1079.38 & 3.98 & 1071.18 & 0.00\\
SCA8-9 & 1063.68 & 1.55 & 
1069.69 & 2.44 & \bf{1060.50} & 
0.30\\CON3-0 & \bf{616.52} & 4.16 & 
622.50 & 4.13 & 616.52 & 0.00\\
CON3-1 & \bf{554.47} & 2.74 & 
557.41 & 2.94 & 554.47 & 0.00\\
CON3-2 & 522.86 & 1.22 & 
523.54 & 3.32 & \bf{519.26} & 
0.69\\CON3-3 & \bf{591.19} & 2.49 & 
591.19 & 3.53 & 591.19 & 0.00\\
CON3-4 & \bf{\underline{588.79}} & 1.70 & 
590.66 & 3.73 & 589.32 & 
-0.09\\CON3-5 & \bf{563.70} & 3.86 & 
571.65 & 2.71 & 563.70 & 0.00\\
CON3-6 & \bf{\underline{499.05}} & 2.66 & 
500.32 & 3.65 & 500.80 & 
-0.35\\CON3-7 & \bf{576.48} & 2.91 & 
577.45 & 3.16 & 576.48 & 0.00\\
CON3-8 & \bf{523.05} & 2.77 & 
523.05 & 3.65 & 523.05 & 0.00\\
CON3-9 & 582.79 & 1.91 & 
586.85 & 3.96 & \bf{580.05} & 
0.47\\CON8-0 & 881.02 & 3.12 & 
896.61 & 2.71 & \bf{857.17} & 
2.78\\CON8-1 & 757.34 & 3.61 & 
759.65 & 2.70 & \bf{740.85} & 
2.23\\CON8-2 & \bf{713.44} & 1.84 & 
728.00 & 1.78 & 713.44 & 0.00\\
CON8-3 & 826.12 & 2.18 & 
851.88 & 2.45 & \bf{811.07} & 
1.86\\CON8-4 & \bf{772.25} & 5.37 & 
779.83 & 3.77 & 772.25 & 0.00\\
CON8-5 & \bf{\underline{754.88}} & 4.65 & 
757.07 & 3.38 & 756.91 & 
-0.27\\CON8-6 & 680.11 & 3.26 & 
689.92 & 3.37 & \bf{678.92} & 
0.18\\CON8-7 & 812.89 & 2.99 & 
816.06 & 2.69 & \bf{811.96} & 
0.11\\CON8-8 & \bf{767.53} & 1.67 & 
779.65 & 3.37 & 767.53 & 0.00\\
CON8-9 & \bf{809.00} & 3.61 & 
810.92 & 3.98 & 809.00 & 0.00\\
[1ex]\hline
\end{tabular}
\label{table:nonlin}
\end{table} \clearpage
\begin{table}[ht]
\caption{Resultados de la ejecución de la metaheurística GTS, utilizando instancias de Dethloff con la configuración -mni 4000 -lambda1 0.05 -lambda2 0.05 -tabu 15}
\centering
\small
\begin{tabular}{c c c c c c c}
\hline\hline
Instancia & Costo mínimo & Tiempo(seg.) & Costo promedio & Tiempo promedio(seg.) & Costo GTS & \%Gap \\ [0.5ex]
\hline
SCA3-0 & \bf{\underline{635.67}} & 3.68 & 
637.09 & 3.18 & 636.06 & 
-0.06\\SCA3-1 & \bf{697.84} & 4.75 & 
698.50 & 2.92 & 697.84 & 0.00\\
SCA3-2 & \bf{659.34} & 3.10 & 
659.34 & 2.94 & 659.34 & 0.00\\
SCA3-3 & \bf{680.04} & 4.10 & 
684.61 & 2.79 & 680.04 & 0.00\\
SCA3-4 & \bf{690.50} & 1.68 & 
690.50 & 2.08 & 690.50 & 0.00\\
SCA3-5 & \bf{659.90} & 2.62 & 
659.90 & 2.10 & 659.90 & 0.00\\
SCA3-6 & \bf{651.09} & 2.39 & 
656.25 & 2.15 & 651.09 & 0.00\\
SCA3-7 & 666.15 & 2.16 & 
667.09 & 3.00 & \bf{659.17} & 
1.06\\SCA3-8 & \bf{719.47} & 5.30 & 
719.47 & 4.01 & 719.47 & 0.00\\
SCA3-9 & \bf{681.00} & 5.76 & 
681.00 & 4.28 & 681.00 & 0.00\\
SCA8-0 & \bf{961.50} & 2.57 & 
975.85 & 2.58 & 961.50 & 0.00\\
SCA8-1 & 1067.45 & 3.13 & 
1068.75 & 3.13 & \bf{1050.20} & 
1.64\\SCA8-2 & 1042.10 & 4.46 & 
1051.93 & 2.50 & \bf{1039.64} & 
0.24\\SCA8-3 & 1005.65 & 3.64 & 
1011.58 & 3.52 & \bf{983.34} & 
2.27\\SCA8-4 & 1067.28 & 2.60 & 
1073.03 & 2.79 & \bf{1065.49} & 
0.17\\SCA8-5 & \bf{1027.08} & 5.37 & 
1039.44 & 3.67 & 1027.08 & 0.00\\
SCA8-6 & 972.48 & 2.40 & 
982.12 & 2.28 & \bf{971.82} & 
0.07\\SCA8-7 & 1053.84 & 3.62 & 
1064.53 & 4.12 & \bf{1052.17} & 
0.16\\SCA8-8 & \bf{1071.18} & 1.72 & 
1083.21 & 2.73 & 1071.18 & 0.00\\
SCA8-9 & \bf{1060.50} & 3.14 & 
1067.79 & 3.35 & 1060.50 & 0.00\\
CON3-0 & \bf{616.52} & 1.32 & 
625.21 & 3.23 & 616.52 & 0.00\\
CON3-1 & \bf{554.47} & 1.84 & 
555.23 & 2.99 & 554.47 & 0.00\\
CON3-2 & \bf{\underline{518.00}} & 2.81 & 
520.80 & 2.03 & 519.26 & 
-0.24\\CON3-3 & \bf{591.19} & 5.48 & 
594.58 & 3.16 & 591.19 & 0.00\\
CON3-4 & \bf{\underline{588.79}} & 3.71 & 
594.47 & 2.61 & 589.32 & 
-0.09\\CON3-5 & \bf{563.70} & 1.56 & 
569.64 & 3.11 & 563.70 & 0.00\\
CON3-6 & \bf{\underline{499.05}} & 2.61 & 
501.46 & 2.98 & 500.80 & 
-0.35\\CON3-7 & \bf{576.48} & 1.68 & 
580.43 & 3.43 & 576.48 & 0.00\\
CON3-8 & \bf{523.05} & 1.75 & 
523.05 & 2.11 & 523.05 & 0.00\\
CON3-9 & 582.79 & 3.47 & 
587.85 & 3.00 & \bf{580.05} & 
0.47\\CON8-0 & 858.16 & 7.62 & 
902.48 & 4.26 & \bf{857.17} & 
0.12\\CON8-1 & 751.84 & 4.74 & 
757.21 & 3.77 & \bf{740.85} & 
1.48\\CON8-2 & \bf{713.44} & 3.42 & 
736.08 & 2.87 & 713.44 & 0.00\\
CON8-3 & \bf{811.07} & 4.22 & 
829.46 & 2.82 & 811.07 & 0.00\\
CON8-4 & \bf{772.25} & 3.30 & 
784.45 & 3.22 & 772.25 & 0.00\\
CON8-5 & \bf{\underline{754.88}} & 5.49 & 
759.40 & 3.31 & 756.91 & 
-0.27\\CON8-6 & 683.77 & 2.29 & 
690.51 & 3.25 & \bf{678.92} & 
0.71\\CON8-7 & 814.18 & 4.05 & 
828.50 & 2.81 & \bf{811.96} & 
0.27\\CON8-8 & \bf{767.53} & 3.52 & 
779.51 & 2.64 & 767.53 & 0.00\\
CON8-9 & 810.84 & 3.36 & 
825.11 & 3.14 & \bf{809.00} & 
0.23\\[1ex]\hline
\end{tabular}
\label{table:nonlin}
\end{table} \clearpage
\begin{table}[ht]
\caption{Resultados de la ejecución de la metaheurística GTS, utilizando instancias de Dethloff con la configuración -mni 4500 -lambda1 0.05 -lambda2 0.05 -tabu 5}
\centering
\small
\begin{tabular}{c c c c c c c}
\hline\hline
Instancia & Costo mínimo & Tiempo(seg.) & Costo promedio & Tiempo promedio(seg.) & Costo GTS & \%Gap \\ [0.5ex]
\hline
SCA3-0 & \bf{636.06} & 3.02 & 
639.43 & 2.21 & 636.06 & 0.00\\
SCA3-1 & \bf{697.84} & 4.56 & 
703.76 & 3.48 & 697.84 & 0.00\\
SCA3-2 & \bf{659.34} & 2.70 & 
659.34 & 2.42 & 659.34 & 0.00\\
SCA3-3 & \bf{680.04} & 2.21 & 
682.76 & 3.19 & 680.04 & 0.00\\
SCA3-4 & \bf{690.50} & 3.23 & 
690.50 & 4.21 & 690.50 & 0.00\\
SCA3-5 & \bf{659.90} & 1.90 & 
659.90 & 3.35 & 659.90 & 0.00\\
SCA3-6 & 660.86 & 2.20 & 
661.48 & 2.00 & \bf{651.09} & 
1.50\\SCA3-7 & 669.89 & 4.18 & 
670.34 & 2.64 & \bf{659.17} & 
1.63\\SCA3-8 & \bf{719.47} & 7.22 & 
722.50 & 3.70 & 719.47 & 0.00\\
SCA3-9 & \bf{681.00} & 2.72 & 
685.12 & 2.65 & 681.00 & 0.00\\
SCA8-0 & 970.64 & 2.22 & 
988.93 & 3.23 & \bf{961.50} & 
0.95\\SCA8-1 & 1068.31 & 2.75 & 
1077.80 & 2.47 & \bf{1050.20} & 
1.72\\SCA8-2 & 1042.10 & 2.73 & 
1049.59 & 2.08 & \bf{1039.64} & 
0.24\\SCA8-3 & \bf{983.34} & 8.41 & 
1006.00 & 4.11 & 983.34 & 0.00\\
SCA8-4 & 1067.28 & 4.84 & 
1071.07 & 3.10 & \bf{1065.49} & 
0.17\\SCA8-5 & 1039.64 & 4.89 & 
1044.92 & 3.30 & \bf{1027.08} & 
1.22\\SCA8-6 & 972.48 & 4.06 & 
985.99 & 2.37 & \bf{971.82} & 
0.07\\SCA8-7 & 1063.22 & 1.95 & 
1070.45 & 2.53 & \bf{1052.17} & 
1.05\\SCA8-8 & \bf{1071.18} & 3.34 & 
1081.71 & 2.19 & 1071.18 & 0.00\\
SCA8-9 & \bf{1060.50} & 3.25 & 
1069.02 & 2.74 & 1060.50 & 0.00\\
CON3-0 & 628.47 & 3.54 & 
635.22 & 3.33 & \bf{616.52} & 
1.94\\CON3-1 & \bf{554.47} & 3.04 & 
557.12 & 2.75 & 554.47 & 0.00\\
CON3-2 & \bf{\underline{518.00}} & 1.89 & 
520.66 & 2.75 & 519.26 & 
-0.24\\CON3-3 & \bf{591.19} & 4.52 & 
591.19 & 2.84 & 591.19 & 0.00\\
CON3-4 & \bf{\underline{588.79}} & 1.64 & 
590.66 & 2.59 & 589.32 & 
-0.09\\CON3-5 & \bf{563.70} & 3.35 & 
569.43 & 2.48 & 563.70 & 0.00\\
CON3-6 & \bf{\underline{499.05}} & 4.30 & 
500.28 & 3.05 & 500.80 & 
-0.35\\CON3-7 & \bf{576.48} & 3.67 & 
582.24 & 3.25 & 576.48 & 0.00\\
CON3-8 & \bf{523.05} & 2.12 & 
523.05 & 2.36 & 523.05 & 0.00\\
CON3-9 & 587.23 & 3.22 & 
592.51 & 3.36 & \bf{580.05} & 
1.24\\CON8-0 & 894.46 & 2.45 & 
911.61 & 2.17 & \bf{857.17} & 
4.35\\CON8-1 & 741.70 & 2.19 & 
758.88 & 2.77 & \bf{740.85} & 
0.11\\CON8-2 & 718.64 & 3.44 & 
724.55 & 2.88 & \bf{713.44} & 
0.73\\CON8-3 & \bf{811.07} & 2.82 & 
819.93 & 2.81 & 811.07 & 0.00\\
CON8-4 & \bf{772.25} & 3.00 & 
781.37 & 4.20 & 772.25 & 0.00\\
CON8-5 & \bf{\underline{754.88}} & 3.50 & 
761.67 & 3.84 & 756.91 & 
-0.27\\CON8-6 & 691.20 & 4.97 & 
698.12 & 3.44 & \bf{678.92} & 
1.81\\CON8-7 & 812.89 & 4.77 & 
823.40 & 4.12 & \bf{811.96} & 
0.11\\CON8-8 & \bf{767.53} & 3.59 & 
779.26 & 3.83 & 767.53 & 0.00\\
CON8-9 & 811.14 & 3.40 & 
814.94 & 3.14 & \bf{809.00} & 
0.26\\[1ex]\hline
\end{tabular}
\label{table:nonlin}
\end{table} \clearpage
\begin{table}[ht]
\caption{Resultados de la ejecución de la metaheurística GTS, utilizando instancias de Dethloff con la configuración -mni 4500 -lambda1 0.05 -lambda2 0.05 -tabu 7}
\centering
\small
\begin{tabular}{c c c c c c c}
\hline\hline
Instancia & Costo mínimo & Tiempo(seg.) & Costo promedio & Tiempo promedio(seg.) & Costo GTS & \%Gap \\ [0.5ex]
\hline
SCA3-0 & 640.55 & 3.60 & 
641.02 & 2.52 & \bf{636.06} & 
0.71\\SCA3-1 & \bf{697.84} & 2.01 & 
697.84 & 3.37 & 697.84 & 0.00\\
SCA3-2 & \bf{659.34} & 3.91 & 
659.34 & 3.83 & 659.34 & 0.00\\
SCA3-3 & 680.60 & 2.00 & 
687.05 & 4.55 & \bf{680.04} & 
0.08\\SCA3-4 & \bf{690.50} & 2.10 & 
690.50 & 3.86 & 690.50 & 0.00\\
SCA3-5 & \bf{659.90} & 2.38 & 
666.42 & 2.52 & 659.90 & 0.00\\
SCA3-6 & \bf{651.09} & 3.31 & 
654.93 & 2.24 & 651.09 & 0.00\\
SCA3-7 & \bf{659.17} & 1.45 & 
665.34 & 3.20 & 659.17 & 0.00\\
SCA3-8 & \bf{719.47} & 2.06 & 
721.90 & 2.81 & 719.47 & 0.00\\
SCA3-9 & \bf{681.00} & 2.16 & 
681.00 & 2.45 & 681.00 & 0.00\\
SCA8-0 & 970.64 & 4.80 & 
979.06 & 3.71 & \bf{961.50} & 
0.95\\SCA8-1 & 1050.69 & 2.02 & 
1062.32 & 2.95 & \bf{1050.20} & 
0.05\\SCA8-2 & 1050.17 & 2.26 & 
1058.59 & 2.74 & \bf{1039.64} & 
1.01\\SCA8-3 & 1012.77 & 3.83 & 
1016.30 & 3.09 & \bf{983.34} & 
2.99\\SCA8-4 & 1067.28 & 3.36 & 
1076.92 & 3.06 & \bf{1065.49} & 
0.17\\SCA8-5 & 1051.03 & 4.38 & 
1055.65 & 3.07 & \bf{1027.08} & 
2.33\\SCA8-6 & \bf{971.82} & 4.14 & 
972.32 & 2.96 & 971.82 & 0.00\\
SCA8-7 & \bf{\underline{1051.28}} & 5.76 & 
1074.61 & 3.81 & 1052.17 & 
-0.08\\SCA8-8 & \bf{1071.18} & 1.56 & 
1077.22 & 1.61 & 1071.18 & 0.00\\
SCA8-9 & 1065.60 & 3.21 & 
1077.42 & 2.80 & \bf{1060.50} & 
0.48\\CON3-0 & 628.47 & 5.98 & 
633.05 & 3.54 & \bf{616.52} & 
1.94\\CON3-1 & \bf{554.47} & 2.36 & 
557.61 & 2.60 & 554.47 & 0.00\\
CON3-2 & \bf{\underline{518.00}} & 1.98 & 
520.72 & 2.58 & 519.26 & 
-0.24\\CON3-3 & \bf{591.19} & 1.60 & 
597.89 & 2.42 & 591.19 & 0.00\\
CON3-4 & \bf{\underline{588.79}} & 4.82 & 
595.32 & 3.16 & 589.32 & 
-0.09\\CON3-5 & \bf{563.70} & 2.94 & 
567.42 & 2.55 & 563.70 & 0.00\\
CON3-6 & \bf{\underline{499.05}} & 10.96 & 
500.70 & 4.92 & 500.80 & 
-0.35\\CON3-7 & \bf{576.48} & 3.47 & 
581.32 & 2.77 & 576.48 & 0.00\\
CON3-8 & \bf{523.05} & 3.34 & 
532.95 & 2.27 & 523.05 & 0.00\\
CON3-9 & \bf{\underline{578.25}} & 5.73 & 
584.40 & 4.33 & 580.05 & 
-0.31\\CON8-0 & 857.40 & 5.21 & 
884.73 & 3.90 & \bf{857.17} & 
0.03\\CON8-1 & \bf{740.85} & 6.75 & 
750.57 & 4.71 & 740.85 & 0.00\\
CON8-2 & 718.64 & 3.88 & 
721.04 & 2.96 & \bf{713.44} & 
0.73\\CON8-3 & \bf{811.07} & 2.45 & 
819.45 & 2.53 & 811.07 & 0.00\\
CON8-4 & \bf{772.25} & 3.01 & 
780.75 & 3.45 & 772.25 & 0.00\\
CON8-5 & \bf{\underline{755.67}} & 3.50 & 
758.13 & 3.35 & 756.91 & 
-0.16\\CON8-6 & \bf{678.92} & 3.24 & 
691.85 & 2.58 & 678.92 & 0.00\\
CON8-7 & \bf{811.96} & 3.69 & 
831.42 & 3.36 & 811.96 & 0.00\\
CON8-8 & 776.55 & 1.71 & 
787.29 & 2.46 & \bf{767.53} & 
1.18\\CON8-9 & \bf{809.00} & 1.99 & 
827.76 & 4.58 & 809.00 & 0.00\\
[1ex]\hline
\end{tabular}
\label{table:nonlin}
\end{table} \clearpage
\begin{table}[ht]
\caption{Resultados de la ejecución de la metaheurística GTS, utilizando instancias de Dethloff con la configuración -mni 4500 -lambda1 0.05 -lambda2 0.05 -tabu 9}
\centering
\small
\begin{tabular}{c c c c c c c}
\hline\hline
Instancia & Costo mínimo & Tiempo(seg.) & Costo promedio & Tiempo promedio(seg.) & Costo GTS & \%Gap \\ [0.5ex]
\hline
SCA3-0 & \bf{636.06} & 4.74 & 
639.43 & 4.24 & 636.06 & 0.00\\
SCA3-1 & \bf{697.84} & 2.76 & 
698.50 & 2.73 & 697.84 & 0.00\\
SCA3-2 & \bf{659.34} & 3.13 & 
659.34 & 2.88 & 659.34 & 0.00\\
SCA3-3 & 680.60 & 2.82 & 
689.90 & 2.58 & \bf{680.04} & 
0.08\\SCA3-4 & \bf{690.50} & 3.58 & 
690.50 & 2.96 & 690.50 & 0.00\\
SCA3-5 & \bf{659.90} & 3.56 & 
669.79 & 2.62 & 659.90 & 0.00\\
SCA3-6 & \bf{651.09} & 4.37 & 
654.10 & 3.69 & 651.09 & 0.00\\
SCA3-7 & 666.15 & 1.47 & 
667.09 & 2.60 & \bf{659.17} & 
1.06\\SCA3-8 & \bf{719.47} & 2.65 & 
719.47 & 3.29 & 719.47 & 0.00\\
SCA3-9 & \bf{681.00} & 3.97 & 
681.00 & 3.14 & 681.00 & 0.00\\
SCA8-0 & \bf{961.50} & 4.36 & 
984.30 & 3.39 & 961.50 & 0.00\\
SCA8-1 & 1059.89 & 3.74 & 
1066.77 & 3.80 & \bf{1050.20} & 
0.92\\SCA8-2 & \bf{1039.64} & 2.87 & 
1051.78 & 3.68 & 1039.64 & 0.00\\
SCA8-3 & \bf{983.34} & 4.35 & 
997.68 & 4.96 & 983.34 & 0.00\\
SCA8-4 & 1067.55 & 1.98 & 
1072.19 & 2.87 & \bf{1065.49} & 
0.19\\SCA8-5 & 1040.66 & 6.54 & 
1049.64 & 4.80 & \bf{1027.08} & 
1.32\\SCA8-6 & \bf{971.82} & 2.28 & 
976.45 & 2.51 & 971.82 & 0.00\\
SCA8-7 & 1071.72 & 3.22 & 
1080.42 & 2.46 & \bf{1052.17} & 
1.86\\SCA8-8 & 1082.12 & 2.66 & 
1082.49 & 2.42 & \bf{1071.18} & 
1.02\\SCA8-9 & \bf{1060.50} & 2.50 & 
1070.98 & 2.20 & 1060.50 & 0.00\\
CON3-0 & \bf{616.52} & 1.77 & 
626.45 & 3.12 & 616.52 & 0.00\\
CON3-1 & \bf{554.47} & 5.27 & 
555.25 & 3.00 & 554.47 & 0.00\\
CON3-2 & 522.86 & 4.43 & 
523.14 & 3.29 & \bf{519.26} & 
0.69\\CON3-3 & \bf{591.19} & 2.16 & 
594.58 & 2.91 & 591.19 & 0.00\\
CON3-4 & \bf{\underline{588.79}} & 4.24 & 
591.33 & 3.84 & 589.32 & 
-0.09\\CON3-5 & 572.56 & 2.36 & 
575.58 & 2.70 & \bf{563.70} & 
1.57\\CON3-6 & \bf{\underline{499.05}} & 6.26 & 
500.61 & 3.88 & 500.80 & 
-0.35\\CON3-7 & \bf{576.48} & 8.51 & 
582.73 & 3.65 & 576.48 & 0.00\\
CON3-8 & \bf{523.05} & 1.72 & 
523.07 & 2.63 & 523.05 & 0.00\\
CON3-9 & 582.79 & 2.18 & 
587.87 & 2.21 & \bf{580.05} & 
0.47\\CON8-0 & 866.68 & 4.09 & 
892.12 & 4.08 & \bf{857.17} & 
1.11\\CON8-1 & 758.21 & 2.36 & 
761.22 & 3.10 & \bf{740.85} & 
2.34\\CON8-2 & 723.29 & 2.52 & 
730.50 & 3.33 & \bf{713.44} & 
1.38\\CON8-3 & 821.26 & 3.00 & 
866.45 & 4.20 & \bf{811.07} & 
1.26\\CON8-4 & \bf{772.25} & 2.83 & 
781.89 & 2.65 & 772.25 & 0.00\\
CON8-5 & \bf{\underline{754.88}} & 2.11 & 
759.39 & 2.90 & 756.91 & 
-0.27\\CON8-6 & 683.83 & 3.08 & 
690.98 & 3.57 & \bf{678.92} & 
0.72\\CON8-7 & 813.94 & 3.96 & 
814.41 & 3.58 & \bf{811.96} & 
0.24\\CON8-8 & 776.55 & 2.18 & 
782.60 & 3.01 & \bf{767.53} & 
1.18\\CON8-9 & 812.25 & 7.50 & 
840.35 & 4.00 & \bf{809.00} & 
0.40\\[1ex]\hline
\end{tabular}
\label{table:nonlin}
\end{table} \clearpage
\begin{table}[ht]
\caption{Resultados de la ejecución de la metaheurística GTS, utilizando instancias de Dethloff con la configuración -mni 4500 -lambda1 0.05 -lambda2 0.05 -tabu 11}
\centering
\small
\begin{tabular}{c c c c c c c}
\hline\hline
Instancia & Costo mínimo & Tiempo(seg.) & Costo promedio & Tiempo promedio(seg.) & Costo GTS & \%Gap \\ [0.5ex]
\hline
SCA3-0 & \bf{636.06} & 2.21 & 
639.90 & 3.65 & 636.06 & 0.00\\
SCA3-1 & \bf{697.84} & 5.46 & 
697.84 & 3.61 & 697.84 & 0.00\\
SCA3-2 & \bf{659.34} & 2.68 & 
659.34 & 3.46 & 659.34 & 0.00\\
SCA3-3 & 680.55 & 2.37 & 
683.35 & 3.08 & \bf{680.04} & 
0.07\\SCA3-4 & \bf{690.50} & 4.28 & 
690.50 & 4.87 & 690.50 & 0.00\\
SCA3-5 & \bf{659.90} & 5.12 & 
659.90 & 3.34 & 659.90 & 0.00\\
SCA3-6 & \bf{651.09} & 3.15 & 
651.74 & 2.98 & 651.09 & 0.00\\
SCA3-7 & 666.15 & 6.33 & 
668.02 & 3.41 & \bf{659.17} & 
1.06\\SCA3-8 & \bf{719.47} & 3.04 & 
728.84 & 3.10 & 719.47 & 0.00\\
SCA3-9 & \bf{681.00} & 3.01 & 
681.00 & 2.61 & 681.00 & 0.00\\
SCA8-0 & \bf{961.50} & 2.10 & 
978.27 & 4.78 & 961.50 & 0.00\\
SCA8-1 & \bf{\underline{1049.65}} & 5.22 & 
1070.28 & 3.69 & 1050.20 & 
-0.05\\SCA8-2 & \bf{1039.64} & 3.34 & 
1046.21 & 3.27 & 1039.64 & 0.00\\
SCA8-3 & \bf{983.34} & 3.61 & 
1005.66 & 3.27 & 983.34 & 0.00\\
SCA8-4 & 1067.29 & 3.18 & 
1069.19 & 3.25 & \bf{1065.49} & 
0.17\\SCA8-5 & \bf{1027.08} & 1.76 & 
1050.18 & 2.44 & 1027.08 & 0.00\\
SCA8-6 & 972.48 & 4.66 & 
978.14 & 2.86 & \bf{971.82} & 
0.07\\SCA8-7 & 1072.87 & 5.11 & 
1087.38 & 3.59 & \bf{1052.17} & 
1.97\\SCA8-8 & 1082.12 & 1.66 & 
1082.12 & 1.65 & \bf{1071.18} & 
1.02\\SCA8-9 & 1063.68 & 8.62 & 
1069.08 & 5.01 & \bf{1060.50} & 
0.30\\CON3-0 & \bf{616.52} & 2.70 & 
620.97 & 3.39 & 616.52 & 0.00\\
CON3-1 & \bf{554.47} & 3.41 & 
558.19 & 2.38 & 554.47 & 0.00\\
CON3-2 & 522.86 & 2.76 & 
523.14 & 3.81 & \bf{519.26} & 
0.69\\CON3-3 & \bf{591.19} & 1.72 & 
591.19 & 3.48 & 591.19 & 0.00\\
CON3-4 & \bf{\underline{588.79}} & 3.34 & 
595.12 & 2.21 & 589.32 & 
-0.09\\CON3-5 & \bf{563.70} & 3.04 & 
570.93 & 2.21 & 563.70 & 0.00\\
CON3-6 & \bf{\underline{499.05}} & 3.14 & 
500.28 & 3.65 & 500.80 & 
-0.35\\CON3-7 & \bf{576.48} & 5.18 & 
577.89 & 4.02 & 576.48 & 0.00\\
CON3-8 & \bf{523.05} & 2.33 & 
523.37 & 2.54 & 523.05 & 0.00\\
CON3-9 & \bf{\underline{578.25}} & 2.50 & 
586.20 & 2.92 & 580.05 & 
-0.31\\CON8-0 & 867.93 & 3.04 & 
890.83 & 3.15 & \bf{857.17} & 
1.26\\CON8-1 & 752.61 & 1.76 & 
757.46 & 3.48 & \bf{740.85} & 
1.59\\CON8-2 & 718.70 & 3.84 & 
726.50 & 2.81 & \bf{713.44} & 
0.74\\CON8-3 & 811.23 & 3.96 & 
820.41 & 3.96 & \bf{811.07} & 
0.02\\CON8-4 & \bf{772.25} & 1.66 & 
781.37 & 2.40 & 772.25 & 0.00\\
CON8-5 & \bf{\underline{754.88}} & 3.59 & 
761.42 & 2.79 & 756.91 & 
-0.27\\CON8-6 & \bf{678.92} & 1.86 & 
687.08 & 3.14 & 678.92 & 0.00\\
CON8-7 & 814.79 & 4.64 & 
835.55 & 3.12 & \bf{811.96} & 
0.35\\CON8-8 & \bf{767.53} & 9.42 & 
775.74 & 4.49 & 767.53 & 0.00\\
CON8-9 & \bf{809.00} & 3.54 & 
809.81 & 4.39 & 809.00 & 0.00\\
[1ex]\hline
\end{tabular}
\label{table:nonlin}
\end{table} \clearpage
\begin{table}[ht]
\caption{Resultados de la ejecución de la metaheurística GTS, utilizando instancias de Dethloff con la configuración -mni 4500 -lambda1 0.05 -lambda2 0.05 -tabu 13}
\centering
\small
\begin{tabular}{c c c c c c c}
\hline\hline
Instancia & Costo mínimo & Tiempo(seg.) & Costo promedio & Tiempo promedio(seg.) & Costo GTS & \%Gap \\ [0.5ex]
\hline
SCA3-0 & \bf{636.06} & 2.15 & 
639.43 & 2.04 & 636.06 & 0.00\\
SCA3-1 & \bf{697.84} & 3.16 & 
699.84 & 2.83 & 697.84 & 0.00\\
SCA3-2 & \bf{659.34} & 2.40 & 
659.34 & 3.37 & 659.34 & 0.00\\
SCA3-3 & 680.60 & 3.20 & 
685.75 & 3.35 & \bf{680.04} & 
0.08\\SCA3-4 & \bf{690.50} & 3.69 & 
690.50 & 3.10 & 690.50 & 0.00\\
SCA3-5 & \bf{659.90} & 3.19 & 
663.16 & 3.90 & 659.90 & 0.00\\
SCA3-6 & \bf{651.09} & 4.14 & 
651.74 & 3.10 & 651.09 & 0.00\\
SCA3-7 & \bf{659.17} & 2.44 & 
662.66 & 2.54 & 659.17 & 0.00\\
SCA3-8 & \bf{719.47} & 3.41 & 
719.47 & 3.95 & 719.47 & 0.00\\
SCA3-9 & \bf{681.00} & 3.30 & 
681.00 & 3.96 & 681.00 & 0.00\\
SCA8-0 & 970.65 & 4.50 & 
982.28 & 4.14 & \bf{961.50} & 
0.95\\SCA8-1 & 1050.38 & 2.73 & 
1061.36 & 2.62 & \bf{1050.20} & 
0.02\\SCA8-2 & \bf{1039.64} & 5.14 & 
1057.29 & 4.44 & 1039.64 & 0.00\\
SCA8-3 & \bf{983.34} & 7.08 & 
1005.60 & 3.58 & 983.34 & 0.00\\
SCA8-4 & 1068.97 & 2.40 & 
1073.13 & 3.40 & \bf{1065.49} & 
0.33\\SCA8-5 & \bf{1027.08} & 2.72 & 
1044.00 & 2.45 & 1027.08 & 0.00\\
SCA8-6 & 972.48 & 4.18 & 
976.62 & 3.18 & \bf{971.82} & 
0.07\\SCA8-7 & 1066.65 & 1.70 & 
1073.73 & 2.82 & \bf{1052.17} & 
1.38\\SCA8-8 & \bf{1071.18} & 2.20 & 
1077.22 & 2.96 & 1071.18 & 0.00\\
SCA8-9 & \bf{1060.50} & 3.44 & 
1070.81 & 2.71 & 1060.50 & 0.00\\
CON3-0 & 628.47 & 3.70 & 
630.62 & 2.86 & \bf{616.52} & 
1.94\\CON3-1 & 556.04 & 1.58 & 
559.88 & 3.53 & \bf{554.47} & 
0.28\\CON3-2 & \bf{519.26} & 1.89 & 
520.55 & 2.60 & 519.26 & 0.00\\
CON3-3 & \bf{591.19} & 4.67 & 
599.55 & 4.48 & 591.19 & 0.00\\
CON3-4 & \bf{\underline{588.79}} & 2.06 & 
589.45 & 2.23 & 589.32 & 
-0.09\\CON3-5 & \bf{563.70} & 4.07 & 
563.70 & 2.52 & 563.70 & 0.00\\
CON3-6 & \bf{\underline{499.05}} & 4.62 & 
500.16 & 4.22 & 500.80 & 
-0.35\\CON3-7 & \bf{576.48} & 6.17 & 
576.57 & 5.68 & 576.48 & 0.00\\
CON3-8 & \bf{523.05} & 3.52 & 
523.05 & 2.70 & 523.05 & 0.00\\
CON3-9 & \bf{\underline{578.25}} & 2.87 & 
586.98 & 3.48 & 580.05 & 
-0.31\\CON8-0 & \bf{857.17} & 3.10 & 
881.66 & 5.11 & 857.17 & 0.00\\
CON8-1 & \bf{740.85} & 4.51 & 
751.25 & 3.14 & 740.85 & 0.00\\
CON8-2 & 722.22 & 2.08 & 
743.42 & 2.38 & \bf{713.44} & 
1.23\\CON8-3 & 816.27 & 2.10 & 
825.50 & 1.97 & \bf{811.07} & 
0.64\\CON8-4 & \bf{772.25} & 2.81 & 
775.39 & 2.83 & 772.25 & 0.00\\
CON8-5 & 759.93 & 2.69 & 
762.61 & 2.69 & \bf{756.91} & 
0.40\\CON8-6 & \bf{678.92} & 4.30 & 
694.52 & 3.13 & 678.92 & 0.00\\
CON8-7 & 812.89 & 4.33 & 
815.07 & 4.16 & \bf{811.96} & 
0.11\\CON8-8 & 776.55 & 3.81 & 
790.91 & 4.26 & \bf{767.53} & 
1.18\\CON8-9 & \bf{809.00} & 6.09 & 
814.16 & 3.98 & 809.00 & 0.00\\
[1ex]\hline
\end{tabular}
\label{table:nonlin}
\end{table} \clearpage
\begin{table}[ht]
\caption{Resultados de la ejecución de la metaheurística GTS, utilizando instancias de Dethloff con la configuración -mni 4500 -lambda1 0.05 -lambda2 0.05 -tabu 15}
\centering
\small
\begin{tabular}{c c c c c c c}
\hline\hline
Instancia & Costo mínimo & Tiempo(seg.) & Costo promedio & Tiempo promedio(seg.) & Costo GTS & \%Gap \\ [0.5ex]
\hline
SCA3-0 & \bf{636.06} & 4.14 & 
639.43 & 3.01 & 636.06 & 0.00\\
SCA3-1 & \bf{697.84} & 4.02 & 
698.50 & 2.41 & 697.84 & 0.00\\
SCA3-2 & \bf{659.34} & 2.48 & 
659.34 & 2.98 & 659.34 & 0.00\\
SCA3-3 & \bf{680.04} & 4.74 & 
680.32 & 3.69 & 680.04 & 0.00\\
SCA3-4 & \bf{690.50} & 2.87 & 
690.50 & 2.61 & 690.50 & 0.00\\
SCA3-5 & \bf{659.90} & 2.06 & 
659.90 & 2.94 & 659.90 & 0.00\\
SCA3-6 & \bf{651.09} & 2.81 & 
651.09 & 3.98 & 651.09 & 0.00\\
SCA3-7 & 666.15 & 3.25 & 
667.09 & 3.31 & \bf{659.17} & 
1.06\\SCA3-8 & \bf{719.47} & 2.56 & 
726.01 & 2.75 & 719.47 & 0.00\\
SCA3-9 & \bf{681.00} & 2.00 & 
681.00 & 3.10 & 681.00 & 0.00\\
SCA8-0 & \bf{961.50} & 3.12 & 
991.43 & 2.72 & 961.50 & 0.00\\
SCA8-1 & \bf{\underline{1049.65}} & 5.13 & 
1054.35 & 5.83 & 1050.20 & 
-0.05\\SCA8-2 & \bf{1039.64} & 2.32 & 
1055.97 & 2.83 & 1039.64 & 0.00\\
SCA8-3 & 1005.65 & 3.86 & 
1017.73 & 3.18 & \bf{983.34} & 
2.27\\SCA8-4 & 1067.55 & 2.54 & 
1069.53 & 2.97 & \bf{1065.49} & 
0.19\\SCA8-5 & \bf{1027.08} & 4.96 & 
1050.43 & 3.40 & 1027.08 & 0.00\\
SCA8-6 & 972.48 & 3.88 & 
981.60 & 3.00 & \bf{971.82} & 
0.07\\SCA8-7 & 1066.65 & 3.37 & 
1077.28 & 4.12 & \bf{1052.17} & 
1.38\\SCA8-8 & 1082.12 & 4.54 & 
1082.32 & 3.79 & \bf{1071.18} & 
1.02\\SCA8-9 & \bf{1060.50} & 3.28 & 
1064.12 & 3.30 & 1060.50 & 0.00\\
CON3-0 & \bf{616.52} & 1.70 & 
627.55 & 2.66 & 616.52 & 0.00\\
CON3-1 & 556.04 & 3.99 & 
557.70 & 3.93 & \bf{554.47} & 
0.28\\CON3-2 & \bf{\underline{519.11}} & 4.86 & 
522.50 & 5.33 & 519.26 & 
-0.03\\CON3-3 & \bf{591.19} & 7.02 & 
591.19 & 5.43 & 591.19 & 0.00\\
CON3-4 & 590.85 & 2.42 & 
596.06 & 2.45 & \bf{589.32} & 
0.26\\CON3-5 & \bf{563.70} & 2.23 & 
563.70 & 2.25 & 563.70 & 0.00\\
CON3-6 & \bf{\underline{499.05}} & 2.84 & 
501.46 & 4.70 & 500.80 & 
-0.35\\CON3-7 & \bf{576.48} & 2.70 & 
576.48 & 4.53 & 576.48 & 0.00\\
CON3-8 & \bf{523.05} & 3.66 & 
523.21 & 2.49 & 523.05 & 0.00\\
CON3-9 & \bf{\underline{578.25}} & 4.78 & 
583.05 & 4.43 & 580.05 & 
-0.31\\CON8-0 & \bf{857.17} & 3.71 & 
870.72 & 4.13 & 857.17 & 0.00\\
CON8-1 & 751.76 & 5.12 & 
758.91 & 2.92 & \bf{740.85} & 
1.47\\CON8-2 & 717.67 & 3.75 & 
722.70 & 3.79 & \bf{713.44} & 
0.59\\CON8-3 & \bf{811.07} & 4.28 & 
832.56 & 4.92 & 811.07 & 0.00\\
CON8-4 & \bf{772.25} & 3.64 & 
788.43 & 4.16 & 772.25 & 0.00\\
CON8-5 & 758.99 & 4.74 & 
781.01 & 3.56 & \bf{756.91} & 
0.27\\CON8-6 & \bf{678.92} & 4.15 & 
689.51 & 3.36 & 678.92 & 0.00\\
CON8-7 & 812.87 & 2.70 & 
814.27 & 3.73 & \bf{811.96} & 
0.11\\CON8-8 & 776.55 & 4.78 & 
788.54 & 4.33 & \bf{767.53} & 
1.18\\CON8-9 & \bf{809.00} & 8.37 & 
812.32 & 5.05 & 809.00 & 0.00\\
[1ex]\hline
\end{tabular}
\label{table:nonlin}
\end{table} \clearpage
\begin{table}[ht]
\caption{Resultados de la ejecución de la metaheurística GTS, utilizando instancias de Dethloff con la configuración -mni 5000 -lambda1 0.05 -lambda2 0.05 -tabu 5}
\centering
\small
\begin{tabular}{c c c c c c c}
\hline\hline
Instancia & Costo mínimo & Tiempo(seg.) & Costo promedio & Tiempo promedio(seg.) & Costo GTS & \%Gap \\ [0.5ex]
\hline
SCA3-0 & 640.55 & 2.07 & 
640.55 & 2.85 & \bf{636.06} & 
0.71\\SCA3-1 & \bf{697.84} & 2.82 & 
698.50 & 2.74 & 697.84 & 0.00\\
SCA3-2 & \bf{659.34} & 2.20 & 
659.34 & 3.15 & 659.34 & 0.00\\
SCA3-3 & 680.60 & 1.63 & 
685.60 & 2.28 & \bf{680.04} & 
0.08\\SCA3-4 & \bf{690.50} & 4.27 & 
690.50 & 2.99 & 690.50 & 0.00\\
SCA3-5 & \bf{659.90} & 2.06 & 
659.90 & 3.88 & 659.90 & 0.00\\
SCA3-6 & \bf{651.09} & 4.20 & 
651.74 & 3.01 & 651.09 & 0.00\\
SCA3-7 & 666.15 & 2.30 & 
669.36 & 3.08 & \bf{659.17} & 
1.06\\SCA3-8 & \bf{719.47} & 7.99 & 
719.47 & 4.31 & 719.47 & 0.00\\
SCA3-9 & \bf{681.00} & 3.56 & 
681.00 & 3.64 & 681.00 & 0.00\\
SCA8-0 & 970.64 & 2.87 & 
981.08 & 6.10 & \bf{961.50} & 
0.95\\SCA8-1 & \bf{1050.20} & 2.29 & 
1060.43 & 3.10 & 1050.20 & 0.00\\
SCA8-2 & \bf{1039.64} & 4.25 & 
1044.67 & 4.50 & 1039.64 & 0.00\\
SCA8-3 & 1010.50 & 2.98 & 
1012.73 & 3.32 & \bf{983.34} & 
2.76\\SCA8-4 & 1067.55 & 4.62 & 
1073.34 & 2.87 & \bf{1065.49} & 
0.19\\SCA8-5 & \bf{1027.08} & 3.18 & 
1042.09 & 2.83 & 1027.08 & 0.00\\
SCA8-6 & \bf{971.82} & 2.34 & 
976.28 & 2.58 & 971.82 & 0.00\\
SCA8-7 & 1063.22 & 4.13 & 
1070.55 & 3.84 & \bf{1052.17} & 
1.05\\SCA8-8 & 1080.58 & 2.04 & 
1085.86 & 2.99 & \bf{1071.18} & 
0.88\\SCA8-9 & \bf{1060.50} & 3.79 & 
1065.19 & 4.46 & 1060.50 & 0.00\\
CON3-0 & 628.47 & 1.84 & 
635.11 & 2.37 & \bf{616.52} & 
1.94\\CON3-1 & \bf{554.47} & 4.78 & 
558.00 & 3.67 & 554.47 & 0.00\\
CON3-2 & \bf{519.26} & 3.53 & 
522.76 & 3.13 & 519.26 & 0.00\\
CON3-3 & \bf{591.19} & 2.99 & 
591.19 & 2.83 & 591.19 & 0.00\\
CON3-4 & \bf{\underline{588.79}} & 4.26 & 
590.11 & 3.20 & 589.32 & 
-0.09\\CON3-5 & \bf{563.70} & 3.47 & 
563.70 & 2.87 & 563.70 & 0.00\\
CON3-6 & 502.16 & 3.91 & 
502.66 & 3.15 & \bf{500.80} & 
0.27\\CON3-7 & 578.22 & 2.85 & 
580.22 & 3.16 & \bf{576.48} & 
0.30\\CON3-8 & \bf{523.05} & 4.25 & 
523.05 & 2.78 & 523.05 & 0.00\\
CON3-9 & \bf{\underline{578.25}} & 2.33 & 
584.18 & 3.99 & 580.05 & 
-0.31\\CON8-0 & 857.40 & 3.72 & 
866.99 & 3.43 & \bf{857.17} & 
0.03\\CON8-1 & \bf{740.85} & 2.51 & 
766.76 & 2.42 & 740.85 & 0.00\\
CON8-2 & 722.22 & 4.12 & 
734.71 & 2.76 & \bf{713.44} & 
1.23\\CON8-3 & \bf{811.07} & 3.62 & 
832.79 & 4.44 & 811.07 & 0.00\\
CON8-4 & \bf{772.25} & 5.57 & 
787.29 & 3.59 & 772.25 & 0.00\\
CON8-5 & \bf{\underline{755.67}} & 4.15 & 
762.24 & 3.93 & 756.91 & 
-0.16\\CON8-6 & 690.63 & 6.23 & 
697.87 & 3.95 & \bf{678.92} & 
1.72\\CON8-7 & \bf{811.96} & 2.72 & 
814.08 & 2.65 & 811.96 & 0.00\\
CON8-8 & \bf{767.53} & 3.09 & 
770.84 & 3.15 & 767.53 & 0.00\\
CON8-9 & \bf{809.00} & 7.80 & 
812.76 & 3.64 & 809.00 & 0.00\\
[1ex]\hline
\end{tabular}
\label{table:nonlin}
\end{table} \clearpage
\begin{table}[ht]
\caption{Resultados de la ejecución de la metaheurística GTS, utilizando instancias de Dethloff con la configuración -mni 5000 -lambda1 0.05 -lambda2 0.05 -tabu 7}
\centering
\small
\begin{tabular}{c c c c c c c}
\hline\hline
Instancia & Costo mínimo & Tiempo(seg.) & Costo promedio & Tiempo promedio(seg.) & Costo GTS & \%Gap \\ [0.5ex]
\hline
SCA3-0 & \bf{636.06} & 2.16 & 
638.30 & 2.26 & 636.06 & 0.00\\
SCA3-1 & \bf{697.84} & 3.17 & 
697.84 & 2.91 & 697.84 & 0.00\\
SCA3-2 & \bf{659.34} & 2.76 & 
659.34 & 3.85 & 659.34 & 0.00\\
SCA3-3 & \bf{680.04} & 3.93 & 
680.18 & 3.39 & 680.04 & 0.00\\
SCA3-4 & \bf{690.50} & 2.21 & 
690.50 & 2.81 & 690.50 & 0.00\\
SCA3-5 & \bf{659.90} & 2.11 & 
663.16 & 3.46 & 659.90 & 0.00\\
SCA3-6 & \bf{651.09} & 2.18 & 
653.70 & 3.88 & 651.09 & 0.00\\
SCA3-7 & 666.15 & 3.70 & 
668.42 & 2.88 & \bf{659.17} & 
1.06\\SCA3-8 & \bf{719.47} & 2.18 & 
719.47 & 3.26 & 719.47 & 0.00\\
SCA3-9 & \bf{681.00} & 1.58 & 
683.54 & 3.44 & 681.00 & 0.00\\
SCA8-0 & 970.64 & 3.33 & 
979.52 & 3.04 & \bf{961.50} & 
0.95\\SCA8-1 & 1050.93 & 3.92 & 
1063.74 & 3.95 & \bf{1050.20} & 
0.07\\SCA8-2 & \bf{1039.64} & 3.49 & 
1040.26 & 3.37 & 1039.64 & 0.00\\
SCA8-3 & 985.91 & 3.62 & 
1002.69 & 3.53 & \bf{983.34} & 
0.26\\SCA8-4 & 1067.55 & 3.26 & 
1076.84 & 3.94 & \bf{1065.49} & 
0.19\\SCA8-5 & 1045.82 & 3.52 & 
1055.43 & 4.25 & \bf{1027.08} & 
1.82\\SCA8-6 & 972.48 & 5.49 & 
980.86 & 4.42 & \bf{971.82} & 
0.07\\SCA8-7 & 1066.65 & 2.23 & 
1076.01 & 3.50 & \bf{1052.17} & 
1.38\\SCA8-8 & \bf{1071.18} & 1.71 & 
1079.00 & 2.52 & 1071.18 & 0.00\\
SCA8-9 & \bf{1060.50} & 2.05 & 
1064.85 & 2.50 & 1060.50 & 0.00\\
CON3-0 & \bf{616.52} & 4.76 & 
617.70 & 4.52 & 616.52 & 0.00\\
CON3-1 & \bf{554.47} & 6.45 & 
558.55 & 3.91 & 554.47 & 0.00\\
CON3-2 & \bf{\underline{519.11}} & 2.91 & 
520.37 & 3.55 & 519.26 & 
-0.03\\CON3-3 & \bf{591.19} & 5.44 & 
593.88 & 3.66 & 591.19 & 0.00\\
CON3-4 & 591.43 & 1.83 & 
596.27 & 4.16 & \bf{589.32} & 
0.36\\CON3-5 & \bf{563.70} & 2.46 & 
565.92 & 3.33 & 563.70 & 0.00\\
CON3-6 & \bf{\underline{499.05}} & 2.91 & 
500.16 & 3.69 & 500.80 & 
-0.35\\CON3-7 & \bf{576.48} & 2.09 & 
577.82 & 2.87 & 576.48 & 0.00\\
CON3-8 & \bf{523.05} & 1.82 & 
523.23 & 1.96 & 523.05 & 0.00\\
CON3-9 & \bf{\underline{578.25}} & 2.51 & 
585.84 & 2.06 & 580.05 & 
-0.31\\CON8-0 & 867.04 & 5.93 & 
884.13 & 4.21 & \bf{857.17} & 
1.15\\CON8-1 & \bf{740.85} & 3.62 & 
751.63 & 4.43 & 740.85 & 0.00\\
CON8-2 & 718.64 & 2.22 & 
731.31 & 2.73 & \bf{713.44} & 
0.73\\CON8-3 & 821.26 & 1.74 & 
846.46 & 4.01 & \bf{811.07} & 
1.26\\CON8-4 & \bf{772.25} & 6.70 & 
781.37 & 5.12 & 772.25 & 0.00\\
CON8-5 & \bf{\underline{755.67}} & 2.15 & 
768.65 & 2.94 & 756.91 & 
-0.16\\CON8-6 & 692.76 & 3.44 & 
700.54 & 2.56 & \bf{678.92} & 
2.04\\CON8-7 & 813.91 & 4.14 & 
814.42 & 3.27 & \bf{811.96} & 
0.24\\CON8-8 & \bf{767.53} & 3.74 & 
776.05 & 3.08 & 767.53 & 0.00\\
CON8-9 & \bf{809.00} & 3.58 & 
809.00 & 3.63 & 809.00 & 0.00\\
[1ex]\hline
\end{tabular}
\label{table:nonlin}
\end{table} \clearpage
\begin{table}[ht]
\caption{Resultados de la ejecución de la metaheurística GTS, utilizando instancias de Dethloff con la configuración -mni 5000 -lambda1 0.05 -lambda2 0.05 -tabu 9}
\centering
\small
\begin{tabular}{c c c c c c c}
\hline\hline
Instancia & Costo mínimo & Tiempo(seg.) & Costo promedio & Tiempo promedio(seg.) & Costo GTS & \%Gap \\ [0.5ex]
\hline
SCA3-0 & \bf{636.06} & 5.32 & 
639.74 & 3.48 & 636.06 & 0.00\\
SCA3-1 & \bf{697.84} & 2.31 & 
699.51 & 2.15 & 697.84 & 0.00\\
SCA3-2 & \bf{659.34} & 3.16 & 
659.34 & 4.29 & 659.34 & 0.00\\
SCA3-3 & \bf{680.04} & 1.87 & 
680.46 & 2.39 & 680.04 & 0.00\\
SCA3-4 & \bf{690.50} & 2.05 & 
690.50 & 4.00 & 690.50 & 0.00\\
SCA3-5 & \bf{659.90} & 3.98 & 
666.42 & 3.88 & 659.90 & 0.00\\
SCA3-6 & \bf{651.09} & 1.65 & 
651.74 & 2.55 & 651.09 & 0.00\\
SCA3-7 & 666.15 & 3.12 & 
667.36 & 3.13 & \bf{659.17} & 
1.06\\SCA3-8 & \bf{719.47} & 2.28 & 
724.15 & 3.04 & 719.47 & 0.00\\
SCA3-9 & \bf{681.00} & 3.39 & 
681.00 & 3.05 & 681.00 & 0.00\\
SCA8-0 & \bf{961.50} & 3.38 & 
976.97 & 5.08 & 961.50 & 0.00\\
SCA8-1 & 1050.38 & 5.71 & 
1060.24 & 4.11 & \bf{1050.20} & 
0.02\\SCA8-2 & \bf{1039.64} & 6.84 & 
1057.12 & 3.80 & 1039.64 & 0.00\\
SCA8-3 & 1005.75 & 5.55 & 
1015.20 & 3.96 & \bf{983.34} & 
2.28\\SCA8-4 & 1067.28 & 9.81 & 
1077.41 & 4.89 & \bf{1065.49} & 
0.17\\SCA8-5 & 1042.30 & 2.34 & 
1053.47 & 3.24 & \bf{1027.08} & 
1.48\\SCA8-6 & 972.48 & 4.02 & 
972.48 & 2.90 & \bf{971.82} & 
0.07\\SCA8-7 & \bf{\underline{1051.28}} & 9.82 & 
1069.79 & 5.25 & 1052.17 & 
-0.08\\SCA8-8 & \bf{1071.18} & 2.98 & 
1079.00 & 2.80 & 1071.18 & 0.00\\
SCA8-9 & \bf{1060.50} & 9.13 & 
1073.07 & 6.45 & 1060.50 & 0.00\\
CON3-0 & \bf{616.52} & 5.61 & 
627.30 & 4.17 & 616.52 & 0.00\\
CON3-1 & \bf{554.47} & 9.63 & 
558.00 & 4.89 & 554.47 & 0.00\\
CON3-2 & 519.61 & 3.21 & 
521.42 & 3.41 & \bf{519.26} & 
0.07\\CON3-3 & \bf{591.19} & 2.45 & 
594.58 & 3.31 & 591.19 & 0.00\\
CON3-4 & \bf{\underline{588.79}} & 2.77 & 
594.46 & 2.88 & 589.32 & 
-0.09\\CON3-5 & \bf{563.70} & 5.34 & 
563.70 & 4.23 & 563.70 & 0.00\\
CON3-6 & \bf{\underline{499.05}} & 3.58 & 
500.94 & 3.50 & 500.80 & 
-0.35\\CON3-7 & \bf{576.48} & 4.15 & 
584.20 & 3.83 & 576.48 & 0.00\\
CON3-8 & \bf{523.05} & 1.56 & 
523.05 & 3.26 & 523.05 & 0.00\\
CON3-9 & \bf{\underline{578.25}} & 3.09 & 
582.22 & 4.12 & 580.05 & 
-0.31\\CON8-0 & 860.28 & 4.21 & 
876.20 & 3.29 & \bf{857.17} & 
0.36\\CON8-1 & \bf{740.85} & 3.69 & 
755.42 & 4.92 & 740.85 & 0.00\\
CON8-2 & 718.64 & 3.57 & 
737.13 & 3.41 & \bf{713.44} & 
0.73\\CON8-3 & \bf{811.07} & 5.70 & 
831.01 & 4.16 & 811.07 & 0.00\\
CON8-4 & \bf{772.25} & 3.78 & 
775.39 & 4.29 & 772.25 & 0.00\\
CON8-5 & \bf{756.91} & 3.21 & 
758.42 & 3.39 & 756.91 & 0.00\\
CON8-6 & 688.68 & 2.84 & 
698.11 & 3.04 & \bf{678.92} & 
1.44\\CON8-7 & 814.79 & 3.89 & 
821.26 & 3.87 & \bf{811.96} & 
0.35\\CON8-8 & \bf{767.53} & 4.32 & 
776.59 & 3.04 & 767.53 & 0.00\\
CON8-9 & 812.35 & 3.64 & 
818.72 & 3.80 & \bf{809.00} & 
0.41\\[1ex]\hline
\end{tabular}
\label{table:nonlin}
\end{table} \clearpage
\begin{table}[ht]
\caption{Resultados de la ejecución de la metaheurística GTS, utilizando instancias de Dethloff con la configuración -mni 5000 -lambda1 0.05 -lambda2 0.05 -tabu 11}
\centering
\small
\begin{tabular}{c c c c c c c}
\hline\hline
Instancia & Costo mínimo & Tiempo(seg.) & Costo promedio & Tiempo promedio(seg.) & Costo GTS & \%Gap \\ [0.5ex]
\hline
SCA3-0 & \bf{636.06} & 2.82 & 
638.30 & 3.85 & 636.06 & 0.00\\
SCA3-1 & \bf{697.84} & 6.44 & 
698.50 & 4.87 & 697.84 & 0.00\\
SCA3-2 & \bf{659.34} & 2.68 & 
659.34 & 2.53 & 659.34 & 0.00\\
SCA3-3 & \bf{680.04} & 5.66 & 
680.18 & 5.21 & 680.04 & 0.00\\
SCA3-4 & \bf{690.50} & 2.67 & 
690.50 & 3.77 & 690.50 & 0.00\\
SCA3-5 & \bf{659.90} & 4.21 & 
669.68 & 2.96 & 659.90 & 0.00\\
SCA3-6 & \bf{651.09} & 2.51 & 
653.64 & 3.28 & 651.09 & 0.00\\
SCA3-7 & \bf{659.17} & 3.01 & 
665.34 & 3.41 & 659.17 & 0.00\\
SCA3-8 & \bf{719.47} & 7.34 & 
719.47 & 4.00 & 719.47 & 0.00\\
SCA3-9 & \bf{681.00} & 2.95 & 
681.00 & 3.24 & 681.00 & 0.00\\
SCA8-0 & \bf{961.50} & 5.41 & 
979.19 & 2.88 & 961.50 & 0.00\\
SCA8-1 & \bf{1050.20} & 5.25 & 
1061.27 & 3.63 & 1050.20 & 0.00\\
SCA8-2 & \bf{1039.64} & 6.07 & 
1048.00 & 3.33 & 1039.64 & 0.00\\
SCA8-3 & \bf{983.34} & 4.20 & 
999.94 & 3.69 & 983.34 & 0.00\\
SCA8-4 & \bf{1065.49} & 3.72 & 
1068.35 & 3.44 & 1065.49 & 0.00\\
SCA8-5 & 1037.20 & 2.70 & 
1060.20 & 2.67 & \bf{1027.08} & 
0.99\\SCA8-6 & 989.02 & 4.25 & 
989.85 & 3.50 & \bf{971.82} & 
1.77\\SCA8-7 & 1063.22 & 3.41 & 
1069.16 & 3.21 & \bf{1052.17} & 
1.05\\SCA8-8 & \bf{1071.18} & 2.28 & 
1079.00 & 3.26 & 1071.18 & 0.00\\
SCA8-9 & \bf{1060.50} & 4.49 & 
1066.24 & 4.97 & 1060.50 & 0.00\\
CON3-0 & 617.59 & 5.12 & 
627.42 & 5.42 & \bf{616.52} & 
0.17\\CON3-1 & 556.04 & 2.18 & 
557.70 & 2.47 & \bf{554.47} & 
0.28\\CON3-2 & 519.61 & 2.54 & 
522.75 & 2.66 & \bf{519.26} & 
0.07\\CON3-3 & \bf{591.19} & 4.34 & 
591.19 & 3.11 & 591.19 & 0.00\\
CON3-4 & 591.43 & 4.10 & 
599.10 & 3.19 & \bf{589.32} & 
0.36\\CON3-5 & \bf{563.70} & 1.68 & 
569.43 & 3.53 & 563.70 & 0.00\\
CON3-6 & \bf{\underline{499.05}} & 5.12 & 
500.61 & 4.12 & 500.80 & 
-0.35\\CON3-7 & \bf{576.48} & 4.49 & 
581.28 & 3.16 & 576.48 & 0.00\\
CON3-8 & \bf{523.05} & 4.09 & 
527.96 & 2.77 & 523.05 & 0.00\\
CON3-9 & \bf{\underline{578.25}} & 3.33 & 
584.36 & 3.75 & 580.05 & 
-0.31\\CON8-0 & 857.40 & 3.30 & 
892.04 & 3.40 & \bf{857.17} & 
0.03\\CON8-1 & \bf{740.85} & 4.25 & 
744.80 & 3.71 & 740.85 & 0.00\\
CON8-2 & 721.61 & 5.09 & 
732.24 & 3.69 & \bf{713.44} & 
1.15\\CON8-3 & 821.26 & 3.83 & 
832.56 & 2.68 & \bf{811.07} & 
1.26\\CON8-4 & \bf{772.25} & 2.20 & 
781.37 & 2.75 & 772.25 & 0.00\\
CON8-5 & 759.93 & 3.88 & 
764.52 & 2.87 & \bf{756.91} & 
0.40\\CON8-6 & \bf{678.92} & 3.36 & 
690.37 & 3.60 & 678.92 & 0.00\\
CON8-7 & \bf{811.96} & 9.92 & 
812.65 & 5.63 & 811.96 & 0.00\\
CON8-8 & \bf{767.53} & 4.72 & 
773.49 & 4.23 & 767.53 & 0.00\\
CON8-9 & \bf{809.00} & 3.66 & 
821.88 & 3.43 & 809.00 & 0.00\\
[1ex]\hline
\end{tabular}
\label{table:nonlin}
\end{table} \clearpage
\begin{table}[ht]
\caption{Resultados de la ejecución de la metaheurística GTS, utilizando instancias de Dethloff con la configuración -mni 5000 -lambda1 0.05 -lambda2 0.05 -tabu 13}
\centering
\small
\begin{tabular}{c c c c c c c}
\hline\hline
Instancia & Costo mínimo & Tiempo(seg.) & Costo promedio & Tiempo promedio(seg.) & Costo GTS & \%Gap \\ [0.5ex]
\hline
SCA3-0 & \bf{636.06} & 5.80 & 
637.18 & 3.06 & 636.06 & 0.00\\
SCA3-1 & \bf{697.84} & 6.81 & 
699.17 & 3.93 & 697.84 & 0.00\\
SCA3-2 & \bf{659.34} & 3.19 & 
659.34 & 4.96 & 659.34 & 0.00\\
SCA3-3 & \bf{680.04} & 2.87 & 
687.37 & 4.00 & 680.04 & 0.00\\
SCA3-4 & \bf{690.50} & 3.27 & 
690.50 & 3.82 & 690.50 & 0.00\\
SCA3-5 & \bf{659.90} & 4.34 & 
663.16 & 3.47 & 659.90 & 0.00\\
SCA3-6 & \bf{651.09} & 2.50 & 
651.09 & 2.76 & 651.09 & 0.00\\
SCA3-7 & 666.15 & 4.47 & 
666.15 & 4.84 & \bf{659.17} & 
1.06\\SCA3-8 & \bf{719.47} & 7.40 & 
722.50 & 3.81 & 719.47 & 0.00\\
SCA3-9 & \bf{681.00} & 3.28 & 
681.00 & 3.89 & 681.00 & 0.00\\
SCA8-0 & 970.64 & 3.92 & 
979.52 & 3.00 & \bf{961.50} & 
0.95\\SCA8-1 & \bf{1050.20} & 4.36 & 
1058.59 & 3.19 & 1050.20 & 0.00\\
SCA8-2 & 1042.10 & 6.18 & 
1064.62 & 4.29 & \bf{1039.64} & 
0.24\\SCA8-3 & 1007.85 & 4.88 & 
1013.70 & 3.55 & \bf{983.34} & 
2.49\\SCA8-4 & 1067.28 & 2.62 & 
1071.35 & 3.35 & \bf{1065.49} & 
0.17\\SCA8-5 & 1042.30 & 2.62 & 
1048.70 & 3.85 & \bf{1027.08} & 
1.48\\SCA8-6 & 972.48 & 1.84 & 
972.48 & 3.15 & \bf{971.82} & 
0.07\\SCA8-7 & 1070.72 & 7.45 & 
1072.75 & 4.70 & \bf{1052.17} & 
1.76\\SCA8-8 & \bf{1071.18} & 5.53 & 
1083.67 & 3.57 & 1071.18 & 0.00\\
SCA8-9 & \bf{1060.50} & 2.86 & 
1063.96 & 4.52 & 1060.50 & 0.00\\
CON3-0 & \bf{616.52} & 1.88 & 
619.75 & 2.71 & 616.52 & 0.00\\
CON3-1 & \bf{554.47} & 2.49 & 
557.61 & 3.95 & 554.47 & 0.00\\
CON3-2 & 523.23 & 3.93 & 
523.35 & 3.54 & \bf{519.26} & 
0.76\\CON3-3 & \bf{591.19} & 4.92 & 
594.58 & 3.25 & 591.19 & 0.00\\
CON3-4 & \bf{\underline{588.79}} & 2.34 & 
595.12 & 3.71 & 589.32 & 
-0.09\\CON3-5 & \bf{563.70} & 1.74 & 
565.92 & 2.94 & 563.70 & 0.00\\
CON3-6 & \bf{\underline{499.05}} & 3.84 & 
501.60 & 3.54 & 500.80 & 
-0.35\\CON3-7 & \bf{576.48} & 9.98 & 
581.75 & 5.44 & 576.48 & 0.00\\
CON3-8 & \bf{523.05} & 2.82 & 
523.05 & 2.22 & 523.05 & 0.00\\
CON3-9 & \bf{\underline{578.25}} & 5.48 & 
585.61 & 5.61 & 580.05 & 
-0.31\\CON8-0 & \bf{857.17} & 5.78 & 
864.60 & 4.71 & 857.17 & 0.00\\
CON8-1 & \bf{740.85} & 4.56 & 
766.76 & 3.18 & 740.85 & 0.00\\
CON8-2 & 718.64 & 9.61 & 
723.54 & 4.13 & \bf{713.44} & 
0.73\\CON8-3 & \bf{811.07} & 5.86 & 
811.11 & 3.66 & 811.07 & 0.00\\
CON8-4 & \bf{772.25} & 6.79 & 
776.01 & 4.61 & 772.25 & 0.00\\
CON8-5 & \bf{\underline{754.95}} & 4.72 & 
759.31 & 4.67 & 756.91 & 
-0.26\\CON8-6 & 685.45 & 2.10 & 
697.00 & 3.21 & \bf{678.92} & 
0.96\\CON8-7 & 813.31 & 4.18 & 
826.56 & 3.42 & \bf{811.96} & 
0.17\\CON8-8 & \bf{767.53} & 2.50 & 
785.13 & 3.16 & 767.53 & 0.00\\
CON8-9 & 811.14 & 4.01 & 
814.90 & 4.00 & \bf{809.00} & 
0.26\\[1ex]\hline
\end{tabular}
\label{table:nonlin}
\end{table} \clearpage
\begin{table}[ht]
\caption{Resultados de la ejecución de la metaheurística GTS, utilizando instancias de Dethloff con la configuración -mni 5000 -lambda1 0.05 -lambda2 0.05 -tabu 15}
\centering
\small
\begin{tabular}{c c c c c c c}
\hline\hline
Instancia & Costo mínimo & Tiempo(seg.) & Costo promedio & Tiempo promedio(seg.) & Costo GTS & \%Gap \\ [0.5ex]
\hline
SCA3-0 & \bf{636.06} & 2.10 & 
639.43 & 2.64 & 636.06 & 0.00\\
SCA3-1 & \bf{697.84} & 4.36 & 
697.84 & 2.98 & 697.84 & 0.00\\
SCA3-2 & \bf{659.34} & 3.46 & 
659.34 & 3.58 & 659.34 & 0.00\\
SCA3-3 & \bf{680.04} & 2.09 & 
680.32 & 3.75 & 680.04 & 0.00\\
SCA3-4 & \bf{690.50} & 3.86 & 
690.50 & 5.52 & 690.50 & 0.00\\
SCA3-5 & \bf{659.90} & 4.83 & 
663.16 & 4.14 & 659.90 & 0.00\\
SCA3-6 & \bf{651.09} & 3.52 & 
652.20 & 3.60 & 651.09 & 0.00\\
SCA3-7 & 659.20 & 2.14 & 
664.41 & 3.23 & \bf{659.17} & 
0.00\\SCA3-8 & \bf{719.47} & 2.63 & 
721.97 & 4.01 & 719.47 & 0.00\\
SCA3-9 & \bf{681.00} & 9.35 & 
681.00 & 5.62 & 681.00 & 0.00\\
SCA8-0 & \bf{961.50} & 3.99 & 
971.79 & 4.00 & 961.50 & 0.00\\
SCA8-1 & 1050.93 & 6.59 & 
1064.31 & 5.25 & \bf{1050.20} & 
0.07\\SCA8-2 & 1050.37 & 1.98 & 
1062.94 & 2.75 & \bf{1039.64} & 
1.03\\SCA8-3 & \bf{983.34} & 16.61 & 
1005.07 & 6.95 & 983.34 & 0.00\\
SCA8-4 & 1067.55 & 3.48 & 
1070.86 & 4.30 & \bf{1065.49} & 
0.19\\SCA8-5 & \bf{1027.08} & 3.73 & 
1038.52 & 3.62 & 1027.08 & 0.00\\
SCA8-6 & 972.48 & 4.12 & 
981.16 & 3.68 & \bf{971.82} & 
0.07\\SCA8-7 & 1066.65 & 6.09 & 
1072.49 & 3.25 & \bf{1052.17} & 
1.38\\SCA8-8 & \bf{1071.18} & 2.82 & 
1080.26 & 2.71 & 1071.18 & 0.00\\
SCA8-9 & 1063.68 & 3.86 & 
1064.62 & 4.18 & \bf{1060.50} & 
0.30\\CON3-0 & \bf{616.52} & 3.82 & 
627.88 & 2.89 & 616.52 & 0.00\\
CON3-1 & \bf{554.47} & 6.10 & 
558.13 & 4.09 & 554.47 & 0.00\\
CON3-2 & 521.38 & 5.73 & 
523.12 & 4.31 & \bf{519.26} & 
0.41\\CON3-3 & \bf{591.19} & 5.45 & 
600.01 & 4.95 & 591.19 & 0.00\\
CON3-4 & \bf{\underline{588.79}} & 4.23 & 
591.32 & 4.53 & 589.32 & 
-0.09\\CON3-5 & \bf{563.70} & 2.04 & 
563.70 & 3.12 & 563.70 & 0.00\\
CON3-6 & \bf{\underline{499.05}} & 2.92 & 
500.16 & 3.36 & 500.80 & 
-0.35\\CON3-7 & \bf{576.48} & 4.02 & 
576.48 & 6.51 & 576.48 & 0.00\\
CON3-8 & \bf{523.05} & 2.57 & 
523.05 & 2.60 & 523.05 & 0.00\\
CON3-9 & \bf{\underline{578.25}} & 6.02 & 
582.27 & 4.79 & 580.05 & 
-0.31\\CON8-0 & \bf{857.17} & 3.16 & 
873.90 & 5.02 & 857.17 & 0.00\\
CON8-1 & 751.76 & 2.29 & 
759.12 & 4.23 & \bf{740.85} & 
1.47\\CON8-2 & 718.64 & 2.82 & 
730.48 & 3.58 & \bf{713.44} & 
0.73\\CON8-3 & 811.23 & 4.81 & 
830.46 & 4.00 & \bf{811.07} & 
0.02\\CON8-4 & \bf{772.25} & 3.16 & 
772.25 & 3.00 & 772.25 & 0.00\\
CON8-5 & \bf{\underline{754.88}} & 7.79 & 
755.81 & 4.66 & 756.91 & 
-0.27\\CON8-6 & 686.23 & 5.03 & 
693.26 & 3.77 & \bf{678.92} & 
1.08\\CON8-7 & 812.89 & 8.13 & 
817.34 & 4.12 & \bf{811.96} & 
0.11\\CON8-8 & \bf{767.53} & 4.01 & 
773.10 & 4.13 & 767.53 & 0.00\\
CON8-9 & \bf{809.00} & 3.28 & 
815.07 & 3.73 & 809.00 & 0.00\\
[1ex]\hline
\end{tabular}
\label{table:nonlin}
\end{table} \clearpage
\begin{table}[ht]
\caption{Resultados de la ejecución de la metaheurística GTS, utilizando instancias de Dethloff con la configuración -mni 5500 -lambda1 0.05 -lambda2 0.05 -tabu 5}
\centering
\small
\begin{tabular}{c c c c c c c}
\hline\hline
Instancia & Costo mínimo & Tiempo(seg.) & Costo promedio & Tiempo promedio(seg.) & Costo GTS & \%Gap \\ [0.5ex]
\hline
SCA3-0 & 640.55 & 4.95 & 
640.55 & 4.58 & \bf{636.06} & 
0.71\\SCA3-1 & \bf{697.84} & 9.01 & 
698.50 & 4.84 & 697.84 & 0.00\\
SCA3-2 & \bf{659.34} & 2.94 & 
659.34 & 3.47 & 659.34 & 0.00\\
SCA3-3 & 680.55 & 4.23 & 
680.59 & 3.51 & \bf{680.04} & 
0.07\\SCA3-4 & \bf{690.50} & 2.30 & 
690.50 & 3.63 & 690.50 & 0.00\\
SCA3-5 & \bf{659.90} & 3.24 & 
666.53 & 2.80 & 659.90 & 0.00\\
SCA3-6 & \bf{651.09} & 2.93 & 
651.09 & 4.20 & 651.09 & 0.00\\
SCA3-7 & 666.15 & 6.24 & 
666.15 & 4.07 & \bf{659.17} & 
1.06\\SCA3-8 & \bf{719.47} & 2.81 & 
719.47 & 4.83 & 719.47 & 0.00\\
SCA3-9 & \bf{681.00} & 1.98 & 
681.00 & 3.10 & 681.00 & 0.00\\
SCA8-0 & \bf{961.50} & 3.14 & 
977.74 & 4.51 & 961.50 & 0.00\\
SCA8-1 & 1068.31 & 4.33 & 
1073.05 & 3.90 & \bf{1050.20} & 
1.72\\SCA8-2 & 1042.10 & 7.50 & 
1056.49 & 4.87 & \bf{1039.64} & 
0.24\\SCA8-3 & \bf{983.34} & 6.03 & 
1005.81 & 4.43 & 983.34 & 0.00\\
SCA8-4 & 1068.27 & 3.59 & 
1071.46 & 3.59 & \bf{1065.49} & 
0.26\\SCA8-5 & \bf{1027.08} & 3.70 & 
1043.55 & 3.69 & 1027.08 & 0.00\\
SCA8-6 & 972.48 & 1.92 & 
976.97 & 3.08 & \bf{971.82} & 
0.07\\SCA8-7 & 1066.65 & 3.63 & 
1078.64 & 3.86 & \bf{1052.17} & 
1.38\\SCA8-8 & \bf{1071.18} & 2.19 & 
1082.38 & 3.20 & 1071.18 & 0.00\\
SCA8-9 & \bf{1060.50} & 8.46 & 
1065.76 & 5.26 & 1060.50 & 0.00\\
CON3-0 & \bf{616.52} & 3.17 & 
626.86 & 2.69 & 616.52 & 0.00\\
CON3-1 & \bf{554.47} & 5.80 & 
558.29 & 4.03 & 554.47 & 0.00\\
CON3-2 & 523.23 & 3.65 & 
523.79 & 4.23 & \bf{519.26} & 
0.76\\CON3-3 & \bf{591.19} & 3.56 & 
591.19 & 4.34 & 591.19 & 0.00\\
CON3-4 & 589.88 & 3.86 & 
591.87 & 2.94 & \bf{589.32} & 
0.10\\CON3-5 & \bf{563.70} & 5.86 & 
563.70 & 3.20 & 563.70 & 0.00\\
CON3-6 & \bf{\underline{499.05}} & 4.40 & 
500.06 & 4.57 & 500.80 & 
-0.35\\CON3-7 & \bf{576.48} & 4.66 & 
578.78 & 4.22 & 576.48 & 0.00\\
CON3-8 & \bf{523.05} & 2.74 & 
523.37 & 3.22 & 523.05 & 0.00\\
CON3-9 & \bf{\underline{578.25}} & 3.67 & 
586.87 & 3.04 & 580.05 & 
-0.31\\CON8-0 & \bf{857.17} & 11.33 & 
866.61 & 7.79 & 857.17 & 0.00\\
CON8-1 & \bf{740.85} & 5.02 & 
760.40 & 3.38 & 740.85 & 0.00\\
CON8-2 & 723.29 & 4.44 & 
734.27 & 4.53 & \bf{713.44} & 
1.38\\CON8-3 & 821.26 & 7.39 & 
844.99 & 4.60 & \bf{811.07} & 
1.26\\CON8-4 & \bf{772.25} & 4.11 & 
789.21 & 4.63 & 772.25 & 0.00\\
CON8-5 & 758.84 & 7.28 & 
759.88 & 3.75 & \bf{756.91} & 
0.25\\CON8-6 & \bf{678.92} & 3.10 & 
687.65 & 3.40 & 678.92 & 0.00\\
CON8-7 & 815.43 & 3.31 & 
828.09 & 4.82 & \bf{811.96} & 
0.43\\CON8-8 & \bf{767.53} & 2.60 & 
779.76 & 3.04 & 767.53 & 0.00\\
CON8-9 & 811.58 & 5.40 & 
819.81 & 3.95 & \bf{809.00} & 
0.32\\[1ex]\hline
\end{tabular}
\label{table:nonlin}
\end{table} \clearpage
\begin{table}[ht]
\caption{Resultados de la ejecución de la metaheurística GTS, utilizando instancias de Dethloff con la configuración -mni 5500 -lambda1 0.05 -lambda2 0.05 -tabu 7}
\centering
\small
\begin{tabular}{c c c c c c c}
\hline\hline
Instancia & Costo mínimo & Tiempo(seg.) & Costo promedio & Tiempo promedio(seg.) & Costo GTS & \%Gap \\ [0.5ex]
\hline
SCA3-0 & \bf{636.06} & 6.20 & 
636.06 & 4.52 & 636.06 & 0.00\\
SCA3-1 & \bf{697.84} & 2.16 & 
699.43 & 3.63 & 697.84 & 0.00\\
SCA3-2 & \bf{659.34} & 2.74 & 
659.34 & 4.24 & 659.34 & 0.00\\
SCA3-3 & \bf{680.04} & 2.23 & 
680.32 & 3.29 & 680.04 & 0.00\\
SCA3-4 & \bf{690.50} & 3.15 & 
690.50 & 4.23 & 690.50 & 0.00\\
SCA3-5 & \bf{659.90} & 4.90 & 
666.42 & 4.39 & 659.90 & 0.00\\
SCA3-6 & \bf{651.09} & 2.44 & 
651.09 & 2.52 & 651.09 & 0.00\\
SCA3-7 & 666.15 & 3.80 & 
667.64 & 3.14 & \bf{659.17} & 
1.06\\SCA3-8 & \bf{719.47} & 2.36 & 
719.47 & 2.64 & 719.47 & 0.00\\
SCA3-9 & \bf{681.00} & 3.83 & 
681.00 & 3.48 & 681.00 & 0.00\\
SCA8-0 & \bf{961.50} & 5.41 & 
984.04 & 6.37 & 961.50 & 0.00\\
SCA8-1 & \bf{1050.20} & 4.56 & 
1064.58 & 3.95 & 1050.20 & 0.00\\
SCA8-2 & 1042.10 & 3.23 & 
1054.57 & 3.79 & \bf{1039.64} & 
0.24\\SCA8-3 & \bf{983.34} & 5.22 & 
998.25 & 5.22 & 983.34 & 0.00\\
SCA8-4 & 1067.28 & 4.32 & 
1083.53 & 3.91 & \bf{1065.49} & 
0.17\\SCA8-5 & 1034.32 & 2.81 & 
1046.76 & 3.20 & \bf{1027.08} & 
0.70\\SCA8-6 & 972.48 & 7.16 & 
972.48 & 4.25 & \bf{971.82} & 
0.07\\SCA8-7 & 1055.59 & 3.55 & 
1066.39 & 4.12 & \bf{1052.17} & 
0.33\\SCA8-8 & \bf{1071.18} & 2.23 & 
1079.96 & 2.96 & 1071.18 & 0.00\\
SCA8-9 & \bf{1060.50} & 6.99 & 
1066.10 & 6.73 & 1060.50 & 0.00\\
CON3-0 & \bf{616.52} & 2.65 & 
622.47 & 4.75 & 616.52 & 0.00\\
CON3-1 & \bf{554.47} & 3.35 & 
558.56 & 2.70 & 554.47 & 0.00\\
CON3-2 & 519.61 & 4.28 & 
521.40 & 3.48 & \bf{519.26} & 
0.07\\CON3-3 & \bf{591.19} & 2.39 & 
599.94 & 2.98 & 591.19 & 0.00\\
CON3-4 & \bf{\underline{588.79}} & 3.56 & 
594.75 & 3.09 & 589.32 & 
-0.09\\CON3-5 & \bf{563.70} & 2.59 & 
565.93 & 3.44 & 563.70 & 0.00\\
CON3-6 & \bf{\underline{499.05}} & 6.54 & 
501.18 & 5.87 & 500.80 & 
-0.35\\CON3-7 & \bf{576.48} & 4.92 & 
586.37 & 4.81 & 576.48 & 0.00\\
CON3-8 & \bf{523.05} & 3.56 & 
523.05 & 3.29 & 523.05 & 0.00\\
CON3-9 & \bf{\underline{578.25}} & 2.22 & 
580.78 & 4.33 & 580.05 & 
-0.31\\CON8-0 & 858.03 & 4.06 & 
864.90 & 4.92 & \bf{857.17} & 
0.10\\CON8-1 & 751.76 & 4.12 & 
755.62 & 3.11 & \bf{740.85} & 
1.47\\CON8-2 & 718.52 & 4.37 & 
725.49 & 3.12 & \bf{713.44} & 
0.71\\CON8-3 & \bf{811.07} & 3.95 & 
824.38 & 4.83 & 811.07 & 0.00\\
CON8-4 & \bf{772.25} & 3.16 & 
783.25 & 5.28 & 772.25 & 0.00\\
CON8-5 & \bf{756.91} & 4.23 & 
757.39 & 3.20 & 756.91 & 0.00\\
CON8-6 & 685.06 & 3.06 & 
691.98 & 4.76 & \bf{678.92} & 
0.90\\CON8-7 & 814.50 & 2.92 & 
817.30 & 3.66 & \bf{811.96} & 
0.31\\CON8-8 & \bf{767.53} & 3.26 & 
777.55 & 3.17 & 767.53 & 0.00\\
CON8-9 & 812.35 & 6.08 & 
820.34 & 4.34 & \bf{809.00} & 
0.41\\[1ex]\hline
\end{tabular}
\label{table:nonlin}
\end{table} \clearpage
\begin{table}[ht]
\caption{Resultados de la ejecución de la metaheurística GTS, utilizando instancias de Dethloff con la configuración -mni 5500 -lambda1 0.05 -lambda2 0.05 -tabu 9}
\centering
\small
\begin{tabular}{c c c c c c c}
\hline\hline
Instancia & Costo mínimo & Tiempo(seg.) & Costo promedio & Tiempo promedio(seg.) & Costo GTS & \%Gap \\ [0.5ex]
\hline
SCA3-0 & \bf{636.06} & 5.07 & 
638.30 & 4.34 & 636.06 & 0.00\\
SCA3-1 & \bf{697.84} & 5.22 & 
698.50 & 4.27 & 697.84 & 0.00\\
SCA3-2 & \bf{659.34} & 4.47 & 
659.34 & 3.27 & 659.34 & 0.00\\
SCA3-3 & 680.60 & 2.17 & 
683.17 & 3.84 & \bf{680.04} & 
0.08\\SCA3-4 & \bf{690.50} & 3.82 & 
690.50 & 4.82 & 690.50 & 0.00\\
SCA3-5 & \bf{659.90} & 4.63 & 
669.68 & 4.17 & 659.90 & 0.00\\
SCA3-6 & \bf{651.09} & 2.13 & 
651.55 & 3.21 & 651.09 & 0.00\\
SCA3-7 & 666.15 & 4.56 & 
666.15 & 4.90 & \bf{659.17} & 
1.06\\SCA3-8 & \bf{719.47} & 3.65 & 
719.47 & 4.57 & 719.47 & 0.00\\
SCA3-9 & \bf{681.00} & 2.54 & 
681.00 & 3.35 & 681.00 & 0.00\\
SCA8-0 & \bf{961.50} & 11.30 & 
972.25 & 7.28 & 961.50 & 0.00\\
SCA8-1 & \bf{1050.20} & 6.65 & 
1060.35 & 4.67 & 1050.20 & 0.00\\
SCA8-2 & \bf{1039.64} & 5.12 & 
1045.62 & 3.35 & 1039.64 & 0.00\\
SCA8-3 & \bf{983.34} & 5.61 & 
999.83 & 7.92 & 983.34 & 0.00\\
SCA8-4 & 1068.97 & 6.25 & 
1073.32 & 5.00 & \bf{1065.49} & 
0.33\\SCA8-5 & 1040.66 & 3.38 & 
1046.71 & 2.57 & \bf{1027.08} & 
1.32\\SCA8-6 & 972.48 & 3.49 & 
981.09 & 3.70 & \bf{971.82} & 
0.07\\SCA8-7 & 1052.60 & 7.79 & 
1060.00 & 6.09 & \bf{1052.17} & 
0.04\\SCA8-8 & \bf{1071.18} & 2.82 & 
1085.16 & 3.03 & 1071.18 & 0.00\\
SCA8-9 & \bf{1060.50} & 5.63 & 
1066.84 & 3.77 & 1060.50 & 0.00\\
CON3-0 & \bf{616.52} & 3.94 & 
622.62 & 3.61 & 616.52 & 0.00\\
CON3-1 & \bf{554.47} & 3.89 & 
559.17 & 2.78 & 554.47 & 0.00\\
CON3-2 & \bf{\underline{518.00}} & 4.98 & 
521.83 & 3.78 & 519.26 & 
-0.24\\CON3-3 & \bf{591.19} & 4.35 & 
591.19 & 3.70 & 591.19 & 0.00\\
CON3-4 & 589.88 & 3.54 & 
597.10 & 3.39 & \bf{589.32} & 
0.10\\CON3-5 & \bf{563.70} & 4.45 & 
565.88 & 4.28 & 563.70 & 0.00\\
CON3-6 & 502.16 & 5.71 & 
508.49 & 3.35 & \bf{500.80} & 
0.27\\CON3-7 & \bf{576.48} & 3.60 & 
580.34 & 4.38 & 576.48 & 0.00\\
CON3-8 & \bf{523.05} & 2.78 & 
523.05 & 3.29 & 523.05 & 0.00\\
CON3-9 & \bf{\underline{578.25}} & 6.88 & 
583.42 & 4.71 & 580.05 & 
-0.31\\CON8-0 & 863.92 & 3.90 & 
869.80 & 3.88 & \bf{857.17} & 
0.79\\CON8-1 & 752.61 & 2.84 & 
757.72 & 4.59 & \bf{740.85} & 
1.59\\CON8-2 & \bf{\underline{712.89}} & 5.67 & 
721.92 & 3.99 & 713.44 & 
-0.08\\CON8-3 & 826.12 & 9.42 & 
839.30 & 5.69 & \bf{811.07} & 
1.86\\CON8-4 & \bf{772.25} & 4.64 & 
772.25 & 5.46 & 772.25 & 0.00\\
CON8-5 & \bf{\underline{754.95}} & 3.21 & 
766.29 & 3.27 & 756.91 & 
-0.26\\CON8-6 & 688.68 & 2.94 & 
701.63 & 3.83 & \bf{678.92} & 
1.44\\CON8-7 & \bf{811.96} & 2.52 & 
815.65 & 4.14 & 811.96 & 0.00\\
CON8-8 & \bf{767.53} & 2.02 & 
769.78 & 4.49 & 767.53 & 0.00\\
CON8-9 & \bf{809.00} & 5.78 & 
831.22 & 3.94 & 809.00 & 0.00\\
[1ex]\hline
\end{tabular}
\label{table:nonlin}
\end{table} \clearpage
\begin{table}[ht]
\caption{Resultados de la ejecución de la metaheurística GTS, utilizando instancias de Dethloff con la configuración -mni 5500 -lambda1 0.05 -lambda2 0.05 -tabu 11}
\centering
\small
\begin{tabular}{c c c c c c c}
\hline\hline
Instancia & Costo mínimo & Tiempo(seg.) & Costo promedio & Tiempo promedio(seg.) & Costo GTS & \%Gap \\ [0.5ex]
\hline
SCA3-0 & \bf{\underline{635.62}} & 3.68 & 
637.07 & 4.57 & 636.06 & 
-0.07\\SCA3-1 & \bf{697.84} & 9.04 & 
697.84 & 6.25 & 697.84 & 0.00\\
SCA3-2 & \bf{659.34} & 8.19 & 
659.34 & 4.41 & 659.34 & 0.00\\
SCA3-3 & \bf{680.04} & 2.31 & 
682.89 & 2.51 & 680.04 & 0.00\\
SCA3-4 & \bf{690.50} & 4.92 & 
690.50 & 4.36 & 690.50 & 0.00\\
SCA3-5 & \bf{659.90} & 2.90 & 
663.16 & 3.52 & 659.90 & 0.00\\
SCA3-6 & \bf{651.09} & 3.10 & 
653.92 & 3.45 & 651.09 & 0.00\\
SCA3-7 & 666.15 & 3.09 & 
668.02 & 4.04 & \bf{659.17} & 
1.06\\SCA3-8 & \bf{719.47} & 2.52 & 
719.47 & 3.99 & 719.47 & 0.00\\
SCA3-9 & \bf{681.00} & 7.85 & 
681.00 & 5.14 & 681.00 & 0.00\\
SCA8-0 & 970.64 & 3.23 & 
981.02 & 4.71 & \bf{961.50} & 
0.95\\SCA8-1 & \bf{1050.20} & 4.10 & 
1056.91 & 3.65 & 1050.20 & 0.00\\
SCA8-2 & 1042.10 & 3.73 & 
1052.33 & 4.63 & \bf{1039.64} & 
0.24\\SCA8-3 & \bf{983.34} & 5.54 & 
999.17 & 4.77 & 983.34 & 0.00\\
SCA8-4 & 1067.28 & 5.12 & 
1071.04 & 5.11 & \bf{1065.49} & 
0.17\\SCA8-5 & 1036.88 & 6.80 & 
1050.87 & 3.88 & \bf{1027.08} & 
0.95\\SCA8-6 & \bf{971.82} & 4.15 & 
972.32 & 4.32 & 971.82 & 0.00\\
SCA8-7 & 1060.98 & 2.90 & 
1075.41 & 3.04 & \bf{1052.17} & 
0.84\\SCA8-8 & \bf{1071.18} & 1.57 & 
1078.40 & 2.77 & 1071.18 & 0.00\\
SCA8-9 & \bf{1060.50} & 3.11 & 
1064.50 & 3.13 & 1060.50 & 0.00\\
CON3-0 & \bf{616.52} & 10.07 & 
617.05 & 6.97 & 616.52 & 0.00\\
CON3-1 & \bf{554.47} & 6.05 & 
559.18 & 3.15 & 554.47 & 0.00\\
CON3-2 & 523.23 & 4.49 & 
523.57 & 3.39 & \bf{519.26} & 
0.76\\CON3-3 & \bf{591.19} & 2.83 & 
591.19 & 3.12 & 591.19 & 0.00\\
CON3-4 & \bf{\underline{588.79}} & 4.20 & 
591.99 & 2.83 & 589.32 & 
-0.09\\CON3-5 & \bf{563.70} & 4.86 & 
563.70 & 3.91 & 563.70 & 0.00\\
CON3-6 & \bf{\underline{499.05}} & 5.23 & 
499.49 & 3.79 & 500.80 & 
-0.35\\CON3-7 & \bf{576.48} & 4.36 & 
577.05 & 3.62 & 576.48 & 0.00\\
CON3-8 & \bf{523.05} & 5.26 & 
523.07 & 4.34 & 523.05 & 0.00\\
CON3-9 & 582.79 & 2.71 & 
588.14 & 3.32 & \bf{580.05} & 
0.47\\CON8-0 & 867.34 & 7.43 & 
891.15 & 4.43 & \bf{857.17} & 
1.19\\CON8-1 & \bf{740.85} & 4.19 & 
752.75 & 3.79 & 740.85 & 0.00\\
CON8-2 & 718.64 & 4.57 & 
724.43 & 4.01 & \bf{713.44} & 
0.73\\CON8-3 & \bf{811.07} & 5.06 & 
819.57 & 4.97 & 811.07 & 0.00\\
CON8-4 & \bf{772.25} & 3.88 & 
781.76 & 3.74 & 772.25 & 0.00\\
CON8-5 & \bf{756.91} & 3.34 & 
758.90 & 3.26 & 756.91 & 0.00\\
CON8-6 & 692.75 & 4.89 & 
696.59 & 4.50 & \bf{678.92} & 
2.04\\CON8-7 & 812.89 & 3.63 & 
814.40 & 3.71 & \bf{811.96} & 
0.11\\CON8-8 & 776.55 & 2.40 & 
784.22 & 3.40 & \bf{767.53} & 
1.18\\CON8-9 & 810.30 & 2.93 & 
827.38 & 4.37 & \bf{809.00} & 
0.16\\[1ex]\hline
\end{tabular}
\label{table:nonlin}
\end{table} \clearpage
\begin{table}[ht]
\caption{Resultados de la ejecución de la metaheurística GTS, utilizando instancias de Dethloff con la configuración -mni 5500 -lambda1 0.05 -lambda2 0.05 -tabu 13}
\centering
\small
\begin{tabular}{c c c c c c c}
\hline\hline
Instancia & Costo mínimo & Tiempo(seg.) & Costo promedio & Tiempo promedio(seg.) & Costo GTS & \%Gap \\ [0.5ex]
\hline
SCA3-0 & \bf{636.06} & 4.74 & 
639.43 & 3.14 & 636.06 & 0.00\\
SCA3-1 & \bf{697.84} & 8.77 & 
699.17 & 4.46 & 697.84 & 0.00\\
SCA3-2 & \bf{659.34} & 5.67 & 
659.34 & 4.64 & 659.34 & 0.00\\
SCA3-3 & \bf{680.04} & 5.93 & 
683.03 & 4.36 & 680.04 & 0.00\\
SCA3-4 & \bf{690.50} & 4.95 & 
690.50 & 3.40 & 690.50 & 0.00\\
SCA3-5 & \bf{659.90} & 2.94 & 
663.16 & 2.96 & 659.90 & 0.00\\
SCA3-6 & \bf{651.09} & 2.54 & 
656.49 & 3.11 & 651.09 & 0.00\\
SCA3-7 & 666.15 & 3.90 & 
666.15 & 3.91 & \bf{659.17} & 
1.06\\SCA3-8 & \bf{719.47} & 2.55 & 
719.47 & 3.81 & 719.47 & 0.00\\
SCA3-9 & \bf{681.00} & 2.82 & 
681.00 & 2.54 & 681.00 & 0.00\\
SCA8-0 & \bf{961.50} & 4.86 & 
967.51 & 4.08 & 961.50 & 0.00\\
SCA8-1 & 1068.14 & 2.22 & 
1070.81 & 3.79 & \bf{1050.20} & 
1.71\\SCA8-2 & \bf{1039.64} & 6.02 & 
1053.53 & 4.71 & 1039.64 & 0.00\\
SCA8-3 & 1005.60 & 3.26 & 
1015.20 & 4.80 & \bf{983.34} & 
2.26\\SCA8-4 & 1067.55 & 4.74 & 
1071.30 & 4.28 & \bf{1065.49} & 
0.19\\SCA8-5 & \bf{1027.08} & 3.95 & 
1034.28 & 3.08 & 1027.08 & 0.00\\
SCA8-6 & 972.48 & 3.24 & 
985.29 & 4.18 & \bf{971.82} & 
0.07\\SCA8-7 & 1066.65 & 4.48 & 
1080.03 & 3.93 & \bf{1052.17} & 
1.38\\SCA8-8 & \bf{1071.18} & 4.56 & 
1083.63 & 2.62 & 1071.18 & 0.00\\
SCA8-9 & \bf{1060.50} & 7.86 & 
1068.22 & 6.09 & 1060.50 & 0.00\\
CON3-0 & \bf{616.52} & 5.30 & 
626.31 & 4.04 & 616.52 & 0.00\\
CON3-1 & \bf{554.47} & 8.74 & 
556.43 & 4.81 & 554.47 & 0.00\\
CON3-2 & 519.61 & 3.92 & 
522.00 & 3.42 & \bf{519.26} & 
0.07\\CON3-3 & \bf{591.19} & 6.52 & 
600.01 & 4.54 & 591.19 & 0.00\\
CON3-4 & \bf{\underline{588.79}} & 2.76 & 
594.41 & 3.89 & 589.32 & 
-0.09\\CON3-5 & \bf{563.70} & 2.29 & 
568.25 & 3.46 & 563.70 & 0.00\\
CON3-6 & \bf{\underline{500.37}} & 5.28 & 
502.21 & 3.83 & 500.80 & 
-0.09\\CON3-7 & \bf{576.48} & 8.38 & 
580.84 & 5.58 & 576.48 & 0.00\\
CON3-8 & \bf{523.05} & 3.27 & 
532.28 & 3.24 & 523.05 & 0.00\\
CON3-9 & 582.79 & 4.55 & 
587.80 & 4.94 & \bf{580.05} & 
0.47\\CON8-0 & 857.40 & 3.08 & 
875.91 & 4.41 & \bf{857.17} & 
0.03\\CON8-1 & \bf{740.85} & 3.79 & 
750.60 & 4.82 & 740.85 & 0.00\\
CON8-2 & \bf{713.44} & 6.95 & 
718.81 & 5.00 & 713.44 & 0.00\\
CON8-3 & 821.26 & 7.38 & 
826.41 & 3.81 & \bf{811.07} & 
1.26\\CON8-4 & \bf{772.25} & 4.43 & 
779.16 & 4.29 & 772.25 & 0.00\\
CON8-5 & \bf{756.91} & 4.46 & 
758.18 & 4.00 & 756.91 & 0.00\\
CON8-6 & \bf{678.92} & 6.56 & 
689.66 & 5.19 & 678.92 & 0.00\\
CON8-7 & \bf{811.96} & 2.38 & 
824.49 & 2.99 & 811.96 & 0.00\\
CON8-8 & \bf{767.53} & 3.94 & 
779.69 & 4.87 & 767.53 & 0.00\\
CON8-9 & 810.61 & 7.13 & 
822.29 & 5.53 & \bf{809.00} & 
0.20\\[1ex]\hline
\end{tabular}
\label{table:nonlin}
\end{table} \clearpage
\begin{table}[ht]
\caption{Resultados de la ejecución de la metaheurística GTS, utilizando instancias de Dethloff con la configuración -mni 5500 -lambda1 0.05 -lambda2 0.05 -tabu 15}
\centering
\small
\begin{tabular}{c c c c c c c}
\hline\hline
Instancia & Costo mínimo & Tiempo(seg.) & Costo promedio & Tiempo promedio(seg.) & Costo GTS & \%Gap \\ [0.5ex]
\hline
SCA3-0 & \bf{636.06} & 3.14 & 
636.06 & 3.20 & 636.06 & 0.00\\
SCA3-1 & \bf{697.84} & 4.52 & 
702.06 & 4.77 & 697.84 & 0.00\\
SCA3-2 & \bf{659.34} & 2.41 & 
659.34 & 3.25 & 659.34 & 0.00\\
SCA3-3 & \bf{680.04} & 8.36 & 
680.18 & 5.55 & 680.04 & 0.00\\
SCA3-4 & \bf{690.50} & 3.30 & 
696.75 & 4.58 & 690.50 & 0.00\\
SCA3-5 & \bf{659.90} & 3.14 & 
663.16 & 3.55 & 659.90 & 0.00\\
SCA3-6 & \bf{651.09} & 3.98 & 
654.87 & 3.38 & 651.09 & 0.00\\
SCA3-7 & \bf{659.17} & 2.99 & 
665.34 & 3.21 & 659.17 & 0.00\\
SCA3-8 & \bf{719.47} & 2.43 & 
719.47 & 3.75 & 719.47 & 0.00\\
SCA3-9 & \bf{681.00} & 2.32 & 
681.00 & 3.96 & 681.00 & 0.00\\
SCA8-0 & 970.64 & 3.60 & 
981.04 & 5.90 & \bf{961.50} & 
0.95\\SCA8-1 & 1050.38 & 4.94 & 
1062.64 & 3.97 & \bf{1050.20} & 
0.02\\SCA8-2 & 1049.22 & 1.75 & 
1057.14 & 3.35 & \bf{1039.64} & 
0.92\\SCA8-3 & \bf{983.34} & 9.54 & 
1000.06 & 6.64 & 983.34 & 0.00\\
SCA8-4 & \bf{1065.49} & 5.14 & 
1068.29 & 4.83 & 1065.49 & 0.00\\
SCA8-5 & \bf{1027.08} & 2.18 & 
1039.72 & 3.51 & 1027.08 & 0.00\\
SCA8-6 & \bf{971.82} & 1.86 & 
976.86 & 2.75 & 971.82 & 0.00\\
SCA8-7 & 1063.22 & 3.42 & 
1070.41 & 2.93 & \bf{1052.17} & 
1.05\\SCA8-8 & \bf{1071.18} & 1.78 & 
1079.52 & 3.30 & 1071.18 & 0.00\\
SCA8-9 & \bf{1060.50} & 2.50 & 
1069.14 & 3.91 & 1060.50 & 0.00\\
CON3-0 & \bf{616.52} & 4.58 & 
625.48 & 4.79 & 616.52 & 0.00\\
CON3-1 & \bf{554.47} & 3.56 & 
557.12 & 3.76 & 554.47 & 0.00\\
CON3-2 & 522.86 & 1.94 & 
523.40 & 2.92 & \bf{519.26} & 
0.69\\CON3-3 & \bf{591.19} & 9.69 & 
591.19 & 4.71 & 591.19 & 0.00\\
CON3-4 & \bf{\underline{588.79}} & 3.14 & 
588.79 & 3.83 & 589.32 & 
-0.09\\CON3-5 & \bf{563.70} & 16.04 & 
567.42 & 7.33 & 563.70 & 0.00\\
CON3-6 & \bf{\underline{499.05}} & 7.39 & 
501.64 & 4.49 & 500.80 & 
-0.35\\CON3-7 & \bf{576.48} & 7.64 & 
576.48 & 5.21 & 576.48 & 0.00\\
CON3-8 & \bf{523.05} & 4.08 & 
523.05 & 3.46 & 523.05 & 0.00\\
CON3-9 & \bf{\underline{578.25}} & 2.87 & 
584.23 & 2.88 & 580.05 & 
-0.31\\CON8-0 & 870.92 & 5.62 & 
883.66 & 5.61 & \bf{857.17} & 
1.60\\CON8-1 & 751.76 & 4.21 & 
758.72 & 3.96 & \bf{740.85} & 
1.47\\CON8-2 & 718.64 & 2.03 & 
722.93 & 3.21 & \bf{713.44} & 
0.73\\CON8-3 & \bf{811.07} & 7.68 & 
821.11 & 4.19 & 811.07 & 0.00\\
CON8-4 & \bf{772.25} & 2.88 & 
784.02 & 4.25 & 772.25 & 0.00\\
CON8-5 & \bf{756.91} & 4.34 & 
758.94 & 3.18 & 756.91 & 0.00\\
CON8-6 & 691.20 & 3.50 & 
695.07 & 3.65 & \bf{678.92} & 
1.81\\CON8-7 & 812.89 & 4.47 & 
813.92 & 3.28 & \bf{811.96} & 
0.11\\CON8-8 & 777.27 & 4.26 & 
781.29 & 4.21 & \bf{767.53} & 
1.27\\CON8-9 & \bf{809.00} & 5.60 & 
809.90 & 4.89 & 809.00 & 0.00\\
[1ex]\hline
\end{tabular}
\label{table:nonlin}
\end{table} \clearpage
\begin{table}[ht]
\caption{Resultados de la ejecución de la metaheurística GTS, utilizando instancias de Dethloff con la configuración -mni 6000 -lambda1 0.05 -lambda2 0.05 -tabu 5}
\centering
\small
\begin{tabular}{c c c c c c c}
\hline\hline
Instancia & Costo mínimo & Tiempo(seg.) & Costo promedio & Tiempo promedio(seg.) & Costo GTS & \%Gap \\ [0.5ex]
\hline
SCA3-0 & \bf{636.06} & 6.14 & 
638.30 & 4.72 & 636.06 & 0.00\\
SCA3-1 & \bf{697.84} & 3.31 & 
697.84 & 3.30 & 697.84 & 0.00\\
SCA3-2 & \bf{659.34} & 2.13 & 
659.34 & 3.90 & 659.34 & 0.00\\
SCA3-3 & \bf{680.04} & 3.94 & 
680.46 & 3.44 & 680.04 & 0.00\\
SCA3-4 & \bf{690.50} & 5.48 & 
690.50 & 4.64 & 690.50 & 0.00\\
SCA3-5 & \bf{659.90} & 3.40 & 
663.16 & 3.29 & 659.90 & 0.00\\
SCA3-6 & 652.94 & 2.83 & 
655.70 & 2.91 & \bf{651.09} & 
0.28\\SCA3-7 & 666.15 & 6.94 & 
667.37 & 3.58 & \bf{659.17} & 
1.06\\SCA3-8 & \bf{719.47} & 6.07 & 
719.47 & 4.33 & 719.47 & 0.00\\
SCA3-9 & \bf{681.00} & 4.81 & 
681.00 & 4.42 & 681.00 & 0.00\\
SCA8-0 & 979.79 & 2.08 & 
986.82 & 3.16 & \bf{961.50} & 
1.90\\SCA8-1 & 1067.45 & 4.04 & 
1071.08 & 3.89 & \bf{1050.20} & 
1.64\\SCA8-2 & 1050.37 & 3.45 & 
1058.39 & 3.42 & \bf{1039.64} & 
1.03\\SCA8-3 & \bf{983.34} & 3.65 & 
995.85 & 4.50 & 983.34 & 0.00\\
SCA8-4 & 1067.28 & 5.15 & 
1069.53 & 4.86 & \bf{1065.49} & 
0.17\\SCA8-5 & 1047.55 & 6.97 & 
1056.38 & 3.98 & \bf{1027.08} & 
1.99\\SCA8-6 & 972.48 & 5.74 & 
977.09 & 4.93 & \bf{971.82} & 
0.07\\SCA8-7 & \bf{\underline{1051.28}} & 4.86 & 
1073.03 & 3.82 & 1052.17 & 
-0.08\\SCA8-8 & 1080.58 & 1.64 & 
1092.19 & 3.06 & \bf{1071.18} & 
0.88\\SCA8-9 & \bf{1060.50} & 8.60 & 
1068.74 & 4.58 & 1060.50 & 0.00\\
CON3-0 & 628.47 & 8.73 & 
634.36 & 5.28 & \bf{616.52} & 
1.94\\CON3-1 & \bf{554.47} & 5.21 & 
555.59 & 5.11 & 554.47 & 0.00\\
CON3-2 & \bf{\underline{518.00}} & 3.40 & 
521.27 & 3.56 & 519.26 & 
-0.24\\CON3-3 & \bf{591.19} & 3.77 & 
594.58 & 2.74 & 591.19 & 0.00\\
CON3-4 & 591.43 & 3.85 & 
595.07 & 3.19 & \bf{589.32} & 
0.36\\CON3-5 & \bf{563.70} & 3.11 & 
567.21 & 3.91 & 563.70 & 0.00\\
CON3-6 & \bf{\underline{499.05}} & 4.40 & 
501.04 & 4.01 & 500.80 & 
-0.35\\CON3-7 & \bf{576.48} & 3.70 & 
582.18 & 4.63 & 576.48 & 0.00\\
CON3-8 & \bf{523.05} & 2.60 & 
523.37 & 2.74 & 523.05 & 0.00\\
CON3-9 & \bf{\underline{578.25}} & 2.47 & 
581.65 & 6.38 & 580.05 & 
-0.31\\CON8-0 & 857.40 & 3.82 & 
881.57 & 4.95 & \bf{857.17} & 
0.03\\CON8-1 & \bf{740.85} & 3.02 & 
750.74 & 3.56 & 740.85 & 0.00\\
CON8-2 & 717.67 & 5.25 & 
733.24 & 4.58 & \bf{713.44} & 
0.59\\CON8-3 & \bf{811.07} & 2.27 & 
818.50 & 4.07 & 811.07 & 0.00\\
CON8-4 & \bf{772.25} & 5.86 & 
786.11 & 5.33 & 772.25 & 0.00\\
CON8-5 & \bf{\underline{754.88}} & 4.91 & 
757.67 & 3.28 & 756.91 & 
-0.27\\CON8-6 & 683.83 & 4.00 & 
691.56 & 3.41 & \bf{678.92} & 
0.72\\CON8-7 & 812.89 & 6.93 & 
825.80 & 4.42 & \bf{811.96} & 
0.11\\CON8-8 & 773.60 & 2.86 & 
779.75 & 3.91 & \bf{767.53} & 
0.79\\CON8-9 & 813.46 & 5.16 & 
828.73 & 6.18 & \bf{809.00} & 
0.55\\[1ex]\hline
\end{tabular}
\label{table:nonlin}
\end{table} \clearpage
\begin{table}[ht]
\caption{Resultados de la ejecución de la metaheurística GTS, utilizando instancias de Dethloff con la configuración -mni 6000 -lambda1 0.05 -lambda2 0.05 -tabu 7}
\centering
\small
\begin{tabular}{c c c c c c c}
\hline\hline
Instancia & Costo mínimo & Tiempo(seg.) & Costo promedio & Tiempo promedio(seg.) & Costo GTS & \%Gap \\ [0.5ex]
\hline
SCA3-0 & \bf{\underline{635.62}} & 3.87 & 
638.20 & 3.55 & 636.06 & 
-0.07\\SCA3-1 & \bf{697.84} & 2.30 & 
699.17 & 3.65 & 697.84 & 0.00\\
SCA3-2 & \bf{659.34} & 3.42 & 
659.34 & 3.81 & 659.34 & 0.00\\
SCA3-3 & \bf{680.04} & 2.95 & 
680.18 & 3.91 & 680.04 & 0.00\\
SCA3-4 & \bf{690.50} & 4.62 & 
690.50 & 4.50 & 690.50 & 0.00\\
SCA3-5 & \bf{659.90} & 3.63 & 
669.68 & 2.87 & 659.90 & 0.00\\
SCA3-6 & \bf{651.09} & 2.24 & 
651.74 & 3.23 & 651.09 & 0.00\\
SCA3-7 & \bf{659.17} & 4.13 & 
665.34 & 3.16 & 659.17 & 0.00\\
SCA3-8 & \bf{719.47} & 5.58 & 
719.47 & 4.81 & 719.47 & 0.00\\
SCA3-9 & \bf{681.00} & 2.37 & 
681.00 & 3.04 & 681.00 & 0.00\\
SCA8-0 & \bf{961.50} & 6.67 & 
972.34 & 5.36 & 961.50 & 0.00\\
SCA8-1 & \bf{1050.20} & 3.15 & 
1060.51 & 4.41 & 1050.20 & 0.00\\
SCA8-2 & 1050.17 & 1.98 & 
1057.38 & 3.83 & \bf{1039.64} & 
1.01\\SCA8-3 & 999.35 & 3.56 & 
1009.81 & 4.02 & \bf{983.34} & 
1.63\\SCA8-4 & \bf{1065.49} & 11.84 & 
1070.83 & 6.13 & 1065.49 & 0.00\\
SCA8-5 & 1040.66 & 4.47 & 
1053.26 & 4.28 & \bf{1027.08} & 
1.32\\SCA8-6 & \bf{971.82} & 5.48 & 
972.32 & 4.51 & 971.82 & 0.00\\
SCA8-7 & 1063.22 & 2.67 & 
1074.53 & 4.56 & \bf{1052.17} & 
1.05\\SCA8-8 & \bf{1071.18} & 1.61 & 
1083.98 & 3.15 & 1071.18 & 0.00\\
SCA8-9 & \bf{1060.50} & 8.61 & 
1065.29 & 6.21 & 1060.50 & 0.00\\
CON3-0 & \bf{616.52} & 5.49 & 
623.58 & 3.57 & 616.52 & 0.00\\
CON3-1 & \bf{554.47} & 3.27 & 
557.61 & 3.37 & 554.47 & 0.00\\
CON3-2 & 521.38 & 1.92 & 
522.99 & 3.86 & \bf{519.26} & 
0.41\\CON3-3 & \bf{591.19} & 3.67 & 
597.89 & 3.42 & 591.19 & 0.00\\
CON3-4 & \bf{\underline{588.79}} & 2.92 & 
591.33 & 4.05 & 589.32 & 
-0.09\\CON3-5 & \bf{563.70} & 6.16 & 
567.42 & 4.28 & 563.70 & 0.00\\
CON3-6 & \bf{\underline{499.05}} & 5.02 & 
501.04 & 5.85 & 500.80 & 
-0.35\\CON3-7 & \bf{576.48} & 4.84 & 
580.93 & 4.48 & 576.48 & 0.00\\
CON3-8 & \bf{523.05} & 2.81 & 
523.05 & 3.84 & 523.05 & 0.00\\
CON3-9 & \bf{\underline{578.25}} & 3.66 & 
584.51 & 5.11 & 580.05 & 
-0.31\\CON8-0 & 857.40 & 5.89 & 
872.77 & 4.39 & \bf{857.17} & 
0.03\\CON8-1 & \bf{740.85} & 3.05 & 
754.15 & 4.27 & 740.85 & 0.00\\
CON8-2 & 717.97 & 6.19 & 
737.89 & 4.22 & \bf{713.44} & 
0.63\\CON8-3 & \bf{811.07} & 3.19 & 
855.45 & 5.03 & 811.07 & 0.00\\
CON8-4 & \bf{772.25} & 4.33 & 
779.21 & 2.99 & 772.25 & 0.00\\
CON8-5 & \bf{\underline{754.95}} & 3.91 & 
758.45 & 4.27 & 756.91 & 
-0.26\\CON8-6 & 691.12 & 7.88 & 
695.10 & 4.39 & \bf{678.92} & 
1.80\\CON8-7 & 813.20 & 2.82 & 
820.30 & 4.14 & \bf{811.96} & 
0.15\\CON8-8 & \bf{767.53} & 6.38 & 
773.80 & 5.11 & 767.53 & 0.00\\
CON8-9 & \bf{809.00} & 3.71 & 
813.17 & 5.21 & 809.00 & 0.00\\
[1ex]\hline
\end{tabular}
\label{table:nonlin}
\end{table} \clearpage
\begin{table}[ht]
\caption{Resultados de la ejecución de la metaheurística GTS, utilizando instancias de Dethloff con la configuración -mni 6000 -lambda1 0.05 -lambda2 0.05 -tabu 9}
\centering
\small
\begin{tabular}{c c c c c c c}
\hline\hline
Instancia & Costo mínimo & Tiempo(seg.) & Costo promedio & Tiempo promedio(seg.) & Costo GTS & \%Gap \\ [0.5ex]
\hline
SCA3-0 & \bf{636.06} & 2.88 & 
639.90 & 2.71 & 636.06 & 0.00\\
SCA3-1 & \bf{697.84} & 4.39 & 
698.50 & 4.59 & 697.84 & 0.00\\
SCA3-2 & \bf{659.34} & 4.79 & 
659.34 & 4.19 & 659.34 & 0.00\\
SCA3-3 & \bf{680.04} & 2.20 & 
682.89 & 3.02 & 680.04 & 0.00\\
SCA3-4 & \bf{690.50} & 7.34 & 
690.50 & 6.55 & 690.50 & 0.00\\
SCA3-5 & \bf{659.90} & 5.18 & 
670.22 & 3.69 & 659.90 & 0.00\\
SCA3-6 & \bf{651.09} & 3.78 & 
651.09 & 3.97 & 651.09 & 0.00\\
SCA3-7 & 666.15 & 3.32 & 
667.09 & 4.26 & \bf{659.17} & 
1.06\\SCA3-8 & \bf{719.47} & 3.88 & 
719.47 & 4.32 & 719.47 & 0.00\\
SCA3-9 & \bf{681.00} & 8.19 & 
681.00 & 4.81 & 681.00 & 0.00\\
SCA8-0 & \bf{961.50} & 5.33 & 
978.09 & 3.94 & 961.50 & 0.00\\
SCA8-1 & \bf{1050.20} & 4.16 & 
1066.89 & 3.87 & 1050.20 & 0.00\\
SCA8-2 & \bf{1039.64} & 5.84 & 
1046.44 & 4.45 & 1039.64 & 0.00\\
SCA8-3 & \bf{983.34} & 4.83 & 
998.12 & 6.31 & 983.34 & 0.00\\
SCA8-4 & \bf{1065.49} & 4.02 & 
1066.97 & 4.73 & 1065.49 & 0.00\\
SCA8-5 & \bf{1027.08} & 10.19 & 
1043.93 & 5.97 & 1027.08 & 0.00\\
SCA8-6 & 972.48 & 2.81 & 
980.81 & 4.21 & \bf{971.82} & 
0.07\\SCA8-7 & 1074.38 & 2.77 & 
1076.21 & 2.79 & \bf{1052.17} & 
2.11\\SCA8-8 & \bf{1071.18} & 2.15 & 
1079.52 & 2.47 & 1071.18 & 0.00\\
SCA8-9 & \bf{1060.50} & 9.70 & 
1069.30 & 4.88 & 1060.50 & 0.00\\
CON3-0 & \bf{616.52} & 9.93 & 
616.52 & 7.32 & 616.52 & 0.00\\
CON3-1 & \bf{554.47} & 3.52 & 
556.03 & 4.90 & 554.47 & 0.00\\
CON3-2 & \bf{\underline{518.00}} & 4.55 & 
521.98 & 4.11 & 519.26 & 
-0.24\\CON3-3 & \bf{591.19} & 4.50 & 
594.58 & 4.15 & 591.19 & 0.00\\
CON3-4 & 589.88 & 4.26 & 
593.14 & 3.76 & \bf{589.32} & 
0.10\\CON3-5 & \bf{563.70} & 1.95 & 
568.06 & 2.67 & 563.70 & 0.00\\
CON3-6 & \bf{\underline{499.05}} & 2.29 & 
499.88 & 4.16 & 500.80 & 
-0.35\\CON3-7 & \bf{576.48} & 8.01 & 
576.48 & 6.82 & 576.48 & 0.00\\
CON3-8 & \bf{523.05} & 3.76 & 
523.05 & 3.75 & 523.05 & 0.00\\
CON3-9 & \bf{\underline{578.25}} & 5.50 & 
584.48 & 3.63 & 580.05 & 
-0.31\\CON8-0 & 857.40 & 3.14 & 
881.92 & 4.15 & \bf{857.17} & 
0.03\\CON8-1 & \bf{740.85} & 3.90 & 
752.49 & 3.85 & 740.85 & 0.00\\
CON8-2 & 716.03 & 3.53 & 
722.28 & 3.15 & \bf{713.44} & 
0.36\\CON8-3 & \bf{811.07} & 5.84 & 
832.43 & 5.56 & 811.07 & 0.00\\
CON8-4 & \bf{772.25} & 5.49 & 
772.25 & 4.82 & 772.25 & 0.00\\
CON8-5 & \bf{\underline{754.88}} & 4.70 & 
760.16 & 4.70 & 756.91 & 
-0.27\\CON8-6 & 688.47 & 5.58 & 
697.21 & 4.18 & \bf{678.92} & 
1.41\\CON8-7 & \bf{811.96} & 7.02 & 
813.55 & 7.18 & 811.96 & 0.00\\
CON8-8 & 771.76 & 5.40 & 
777.64 & 4.96 & \bf{767.53} & 
0.55\\CON8-9 & \bf{809.00} & 6.32 & 
824.97 & 5.80 & 809.00 & 0.00\\
[1ex]\hline
\end{tabular}
\label{table:nonlin}
\end{table} \clearpage
\begin{table}[ht]
\caption{Resultados de la ejecución de la metaheurística GTS, utilizando instancias de Dethloff con la configuración -mni 6000 -lambda1 0.05 -lambda2 0.05 -tabu 11}
\centering
\small
\begin{tabular}{c c c c c c c}
\hline\hline
Instancia & Costo mínimo & Tiempo(seg.) & Costo promedio & Tiempo promedio(seg.) & Costo GTS & \%Gap \\ [0.5ex]
\hline
SCA3-0 & 640.55 & 8.70 & 
640.55 & 5.97 & \bf{636.06} & 
0.71\\SCA3-1 & \bf{697.84} & 6.22 & 
697.84 & 5.16 & 697.84 & 0.00\\
SCA3-2 & \bf{659.34} & 7.22 & 
659.34 & 4.62 & 659.34 & 0.00\\
SCA3-3 & \bf{680.04} & 4.20 & 
682.89 & 3.73 & 680.04 & 0.00\\
SCA3-4 & \bf{690.50} & 5.10 & 
690.50 & 5.31 & 690.50 & 0.00\\
SCA3-5 & \bf{659.90} & 3.48 & 
663.16 & 3.90 & 659.90 & 0.00\\
SCA3-6 & \bf{651.09} & 4.86 & 
651.09 & 3.73 & 651.09 & 0.00\\
SCA3-7 & 666.15 & 2.23 & 
666.15 & 4.32 & \bf{659.17} & 
1.06\\SCA3-8 & \bf{719.47} & 2.32 & 
719.47 & 4.75 & 719.47 & 0.00\\
SCA3-9 & \bf{681.00} & 4.10 & 
681.00 & 6.30 & 681.00 & 0.00\\
SCA8-0 & \bf{961.50} & 4.67 & 
972.47 & 3.45 & 961.50 & 0.00\\
SCA8-1 & 1050.93 & 4.46 & 
1059.34 & 4.85 & \bf{1050.20} & 
0.07\\SCA8-2 & 1050.37 & 4.81 & 
1065.75 & 4.80 & \bf{1039.64} & 
1.03\\SCA8-3 & \bf{983.34} & 3.37 & 
1004.07 & 4.87 & 983.34 & 0.00\\
SCA8-4 & 1068.97 & 3.04 & 
1073.46 & 3.28 & \bf{1065.49} & 
0.33\\SCA8-5 & 1034.37 & 3.58 & 
1043.87 & 3.48 & \bf{1027.08} & 
0.71\\SCA8-6 & 972.48 & 2.57 & 
985.87 & 4.00 & \bf{971.82} & 
0.07\\SCA8-7 & 1060.98 & 3.45 & 
1082.38 & 3.32 & \bf{1052.17} & 
0.84\\SCA8-8 & 1082.12 & 4.81 & 
1082.99 & 4.52 & \bf{1071.18} & 
1.02\\SCA8-9 & \bf{1060.50} & 1.95 & 
1063.46 & 3.93 & 1060.50 & 0.00\\
CON3-0 & \bf{616.52} & 4.93 & 
626.57 & 3.85 & 616.52 & 0.00\\
CON3-1 & \bf{554.47} & 3.61 & 
557.21 & 3.97 & 554.47 & 0.00\\
CON3-2 & 521.38 & 2.23 & 
522.68 & 4.04 & \bf{519.26} & 
0.41\\CON3-3 & \bf{591.19} & 5.16 & 
594.58 & 4.24 & 591.19 & 0.00\\
CON3-4 & \bf{\underline{588.79}} & 8.40 & 
593.75 & 5.81 & 589.32 & 
-0.09\\CON3-5 & \bf{563.70} & 5.27 & 
563.70 & 3.88 & 563.70 & 0.00\\
CON3-6 & \bf{\underline{499.05}} & 3.50 & 
502.12 & 4.14 & 500.80 & 
-0.35\\CON3-7 & \bf{576.48} & 2.80 & 
577.23 & 3.93 & 576.48 & 0.00\\
CON3-8 & \bf{523.05} & 3.91 & 
523.05 & 4.32 & 523.05 & 0.00\\
CON3-9 & \bf{\underline{578.25}} & 4.91 & 
584.46 & 4.48 & 580.05 & 
-0.31\\CON8-0 & 858.63 & 5.01 & 
876.48 & 5.29 & \bf{857.17} & 
0.17\\CON8-1 & 751.76 & 5.76 & 
762.63 & 4.41 & \bf{740.85} & 
1.47\\CON8-2 & 721.61 & 3.83 & 
736.96 & 3.96 & \bf{713.44} & 
1.15\\CON8-3 & 821.26 & 2.68 & 
830.41 & 5.21 & \bf{811.07} & 
1.26\\CON8-4 & \bf{772.25} & 3.95 & 
775.45 & 3.75 & 772.25 & 0.00\\
CON8-5 & \bf{\underline{755.67}} & 3.36 & 
760.91 & 4.21 & 756.91 & 
-0.16\\CON8-6 & \bf{678.92} & 7.26 & 
693.97 & 4.12 & 678.92 & 0.00\\
CON8-7 & 814.50 & 2.64 & 
825.44 & 2.88 & \bf{811.96} & 
0.31\\CON8-8 & \bf{767.53} & 5.88 & 
772.57 & 4.12 & 767.53 & 0.00\\
CON8-9 & \bf{809.00} & 4.28 & 
811.52 & 4.71 & 809.00 & 0.00\\
[1ex]\hline
\end{tabular}
\label{table:nonlin}
\end{table} \clearpage
\begin{table}[ht]
\caption{Resultados de la ejecución de la metaheurística GTS, utilizando instancias de Dethloff con la configuración -mni 6000 -lambda1 0.05 -lambda2 0.05 -tabu 13}
\centering
\small
\begin{tabular}{c c c c c c c}
\hline\hline
Instancia & Costo mínimo & Tiempo(seg.) & Costo promedio & Tiempo promedio(seg.) & Costo GTS & \%Gap \\ [0.5ex]
\hline
SCA3-0 & \bf{636.06} & 3.31 & 
639.43 & 3.84 & 636.06 & 0.00\\
SCA3-1 & \bf{697.84} & 6.40 & 
697.84 & 3.60 & 697.84 & 0.00\\
SCA3-2 & \bf{659.34} & 5.47 & 
659.34 & 4.71 & 659.34 & 0.00\\
SCA3-3 & \bf{680.04} & 4.20 & 
680.46 & 4.42 & 680.04 & 0.00\\
SCA3-4 & \bf{690.50} & 3.81 & 
696.64 & 3.54 & 690.50 & 0.00\\
SCA3-5 & \bf{659.90} & 3.02 & 
663.16 & 5.50 & 659.90 & 0.00\\
SCA3-6 & \bf{651.09} & 2.96 & 
651.09 & 4.95 & 651.09 & 0.00\\
SCA3-7 & 666.15 & 3.31 & 
667.09 & 3.28 & \bf{659.17} & 
1.06\\SCA3-8 & \bf{719.47} & 4.30 & 
719.47 & 3.62 & 719.47 & 0.00\\
SCA3-9 & \bf{681.00} & 3.72 & 
681.00 & 3.77 & 681.00 & 0.00\\
SCA8-0 & 979.79 & 5.54 & 
986.95 & 4.91 & \bf{961.50} & 
1.90\\SCA8-1 & \bf{1050.20} & 5.98 & 
1063.57 & 4.35 & 1050.20 & 0.00\\
SCA8-2 & \bf{1039.64} & 2.59 & 
1049.10 & 3.69 & 1039.64 & 0.00\\
SCA8-3 & \bf{983.34} & 4.59 & 
989.85 & 4.54 & 983.34 & 0.00\\
SCA8-4 & 1067.55 & 3.92 & 
1069.84 & 4.85 & \bf{1065.49} & 
0.19\\SCA8-5 & \bf{1027.08} & 2.85 & 
1043.65 & 4.23 & 1027.08 & 0.00\\
SCA8-6 & \bf{971.82} & 9.55 & 
981.05 & 5.61 & 971.82 & 0.00\\
SCA8-7 & 1064.34 & 2.54 & 
1073.82 & 3.89 & \bf{1052.17} & 
1.16\\SCA8-8 & \bf{1071.18} & 2.98 & 
1079.39 & 3.79 & 1071.18 & 0.00\\
SCA8-9 & 1063.68 & 7.01 & 
1075.99 & 3.65 & \bf{1060.50} & 
0.30\\CON3-0 & \bf{616.52} & 4.92 & 
622.50 & 5.06 & 616.52 & 0.00\\
CON3-1 & \bf{554.47} & 6.50 & 
555.39 & 5.91 & 554.47 & 0.00\\
CON3-2 & 519.61 & 4.68 & 
521.60 & 4.43 & \bf{519.26} & 
0.07\\CON3-3 & \bf{591.19} & 3.38 & 
591.19 & 6.42 & 591.19 & 0.00\\
CON3-4 & \bf{\underline{588.79}} & 5.43 & 
604.85 & 4.35 & 589.32 & 
-0.09\\CON3-5 & \bf{563.70} & 3.12 & 
568.13 & 5.52 & 563.70 & 0.00\\
CON3-6 & \bf{\underline{499.05}} & 3.07 & 
500.16 & 5.46 & 500.80 & 
-0.35\\CON3-7 & \bf{576.48} & 8.46 & 
576.48 & 6.39 & 576.48 & 0.00\\
CON3-8 & \bf{523.05} & 4.59 & 
523.05 & 3.85 & 523.05 & 0.00\\
CON3-9 & 582.79 & 3.63 & 
586.95 & 4.22 & \bf{580.05} & 
0.47\\CON8-0 & 876.15 & 2.76 & 
886.57 & 5.36 & \bf{857.17} & 
2.21\\CON8-1 & 756.57 & 3.28 & 
759.24 & 2.91 & \bf{740.85} & 
2.12\\CON8-2 & \bf{\underline{713.05}} & 5.03 & 
720.43 & 5.42 & 713.44 & 
-0.05\\CON8-3 & 821.26 & 5.78 & 
828.29 & 5.00 & \bf{811.07} & 
1.26\\CON8-4 & \bf{772.25} & 3.89 & 
777.61 & 4.93 & 772.25 & 0.00\\
CON8-5 & 758.99 & 3.71 & 
759.94 & 4.54 & \bf{756.91} & 
0.27\\CON8-6 & 685.45 & 2.90 & 
692.76 & 3.66 & \bf{678.92} & 
0.96\\CON8-7 & 812.89 & 5.66 & 
813.14 & 4.44 & \bf{811.96} & 
0.11\\CON8-8 & 776.55 & 5.87 & 
783.10 & 4.22 & \bf{767.53} & 
1.18\\CON8-9 & 811.14 & 4.54 & 
838.29 & 3.91 & \bf{809.00} & 
0.26\\[1ex]\hline
\end{tabular}
\label{table:nonlin}
\end{table} \clearpage
\begin{table}[ht]
\caption{Resultados de la ejecución de la metaheurística GTS, utilizando instancias de Dethloff con la configuración -mni 6000 -lambda1 0.05 -lambda2 0.05 -tabu 15}
\centering
\small
\begin{tabular}{c c c c c c c}
\hline\hline
Instancia & Costo mínimo & Tiempo(seg.) & Costo promedio & Tiempo promedio(seg.) & Costo GTS & \%Gap \\ [0.5ex]
\hline
SCA3-0 & \bf{636.06} & 2.18 & 
639.99 & 3.37 & 636.06 & 0.00\\
SCA3-1 & \bf{697.84} & 4.76 & 
698.50 & 4.79 & 697.84 & 0.00\\
SCA3-2 & \bf{659.34} & 11.40 & 
660.29 & 5.92 & 659.34 & 0.00\\
SCA3-3 & \bf{680.04} & 6.06 & 
682.62 & 5.50 & 680.04 & 0.00\\
SCA3-4 & \bf{690.50} & 2.73 & 
690.50 & 4.22 & 690.50 & 0.00\\
SCA3-5 & \bf{659.90} & 4.00 & 
659.90 & 4.40 & 659.90 & 0.00\\
SCA3-6 & \bf{651.09} & 5.40 & 
651.74 & 4.11 & 651.09 & 0.00\\
SCA3-7 & 666.15 & 7.11 & 
666.15 & 5.75 & \bf{659.17} & 
1.06\\SCA3-8 & \bf{719.47} & 7.94 & 
728.43 & 5.86 & 719.47 & 0.00\\
SCA3-9 & \bf{681.00} & 6.87 & 
681.00 & 4.12 & 681.00 & 0.00\\
SCA8-0 & \bf{961.50} & 4.82 & 
965.22 & 7.11 & 961.50 & 0.00\\
SCA8-1 & 1050.72 & 5.68 & 
1055.22 & 4.61 & \bf{1050.20} & 
0.05\\SCA8-2 & \bf{1039.64} & 4.10 & 
1052.14 & 4.47 & 1039.64 & 0.00\\
SCA8-3 & \bf{983.34} & 7.05 & 
998.12 & 6.79 & 983.34 & 0.00\\
SCA8-4 & \bf{1065.49} & 4.70 & 
1079.15 & 3.41 & 1065.49 & 0.00\\
SCA8-5 & 1041.29 & 3.70 & 
1053.51 & 3.27 & \bf{1027.08} & 
1.38\\SCA8-6 & 972.48 & 4.34 & 
972.48 & 3.40 & \bf{971.82} & 
0.07\\SCA8-7 & 1063.22 & 5.21 & 
1076.74 & 4.19 & \bf{1052.17} & 
1.05\\SCA8-8 & \bf{1071.18} & 2.70 & 
1071.18 & 2.79 & 1071.18 & 0.00\\
SCA8-9 & \bf{1060.50} & 10.30 & 
1068.32 & 4.65 & 1060.50 & 0.00\\
CON3-0 & \bf{616.52} & 3.91 & 
619.66 & 3.79 & 616.52 & 0.00\\
CON3-1 & \bf{554.47} & 5.83 & 
555.44 & 3.91 & 554.47 & 0.00\\
CON3-2 & 521.38 & 2.38 & 
522.77 & 5.18 & \bf{519.26} & 
0.41\\CON3-3 & \bf{591.19} & 5.87 & 
591.19 & 5.92 & 591.19 & 0.00\\
CON3-4 & \bf{\underline{588.79}} & 4.23 & 
591.99 & 4.29 & 589.32 & 
-0.09\\CON3-5 & \bf{563.70} & 4.38 & 
571.38 & 4.16 & 563.70 & 0.00\\
CON3-6 & \bf{\underline{499.05}} & 4.53 & 
500.32 & 4.19 & 500.80 & 
-0.35\\CON3-7 & \bf{576.48} & 7.49 & 
581.32 & 6.11 & 576.48 & 0.00\\
CON3-8 & \bf{523.05} & 3.05 & 
523.05 & 3.81 & 523.05 & 0.00\\
CON3-9 & \bf{\underline{578.25}} & 5.54 & 
582.62 & 5.28 & 580.05 & 
-0.31\\CON8-0 & \bf{857.17} & 3.48 & 
867.33 & 6.14 & 857.17 & 0.00\\
CON8-1 & 756.50 & 4.65 & 
760.23 & 3.63 & \bf{740.85} & 
2.11\\CON8-2 & 718.64 & 4.35 & 
721.76 & 3.23 & \bf{713.44} & 
0.73\\CON8-3 & \bf{811.07} & 4.74 & 
823.00 & 4.24 & 811.07 & 0.00\\
CON8-4 & \bf{772.25} & 2.41 & 
779.64 & 3.23 & 772.25 & 0.00\\
CON8-5 & \bf{\underline{755.67}} & 9.00 & 
756.90 & 6.77 & 756.91 & 
-0.16\\CON8-6 & 684.69 & 3.06 & 
698.98 & 2.94 & \bf{678.92} & 
0.85\\CON8-7 & 812.89 & 7.16 & 
825.11 & 6.47 & \bf{811.96} & 
0.11\\CON8-8 & \bf{767.53} & 2.34 & 
778.86 & 3.08 & 767.53 & 0.00\\
CON8-9 & 812.35 & 8.71 & 
822.03 & 5.35 & \bf{809.00} & 
0.41\\[1ex]\hline
\end{tabular}
\label{table:nonlin}
\end{table} \clearpage
\begin{table}[ht]
\caption{Resultados de la ejecución de la metaheurística ILS, utilizando instancias de Dethloff con la configuración -n 15 -LS 30}
\centering
\small
\begin{tabular}{c c c c c c c}
\hline\hline
Instancia & Costo mínimo & Tiempo(seg.) & Costo promedio & Tiempo promedio(seg.) & Costo ILS & \%Gap \\ [0.5ex]
\hline
SCA3-0 & 100000 & 0 & 
nan & nan & \bf{635.62} & 
15632.67\\[1ex]\hline
\end{tabular}
\label{table:nonlin}
\end{table} \clearpage
\begin{table}[ht]
\caption{Resultados de la ejecución de la metaheurística ILS, utilizando instancias de Dethloff con la configuración -n 15 -LS 30}
\centering
\small
\begin{tabular}{c c c c c c c}
\hline\hline
Instancia & Costo mínimo & Tiempo(seg.) & Costo promedio & Tiempo promedio(seg.) & Costo ILS & \%Gap \\ [0.5ex]
\hline
SCA3-0 & 636.06 & 1.82 & 
647.61 & 2.21 & \bf{635.62} & 
0.07\\[1ex]\hline
\end{tabular}
\label{table:nonlin}
\end{table} \clearpage
\begin{table}[ht]
\caption{Resultados de la ejecución de la metaheurística ILS, utilizando instancias de Dethloff con la configuración -n 5.0 -LS 10.0}
\centering
\small
\begin{tabular}{c c c c c c c}
\hline\hline
Instancia & Costo mínimo & Tiempo(seg.) & Costo promedio & Tiempo promedio(seg.) & Costo ILS & \%Gap \\ [0.5ex]
\hline
SCA3-0 & 652.24 & 0.28 & 
657.32 & 0.33 & \bf{635.62} & 
2.61\\SCA3-1 & 743.53 & 0.40 & 
757.26 & 0.35 & \bf{697.84} & 
6.55\\SCA3-2 & 673.33 & 0.28 & 
693.99 & 0.26 & \bf{659.34} & 
2.12\\SCA3-3 & 682.46 & 0.36 & 
694.59 & 0.36 & \bf{680.04} & 
0.36\\SCA3-4 & 742.15 & 0.39 & 
744.96 & 0.31 & \bf{690.50} & 
7.48\\SCA3-5 & 673.56 & 0.32 & 
679.99 & 0.34 & \bf{659.90} & 
2.07\\SCA3-6 & 653.68 & 0.48 & 
662.46 & 0.43 & \bf{651.09} & 
0.40\\SCA3-7 & 674.48 & 0.37 & 
687.28 & 0.40 & \bf{659.17} & 
2.32\\SCA3-8 & 731.44 & 0.43 & 
732.35 & 0.32 & \bf{719.47} & 
1.66\\SCA3-9 & 685.14 & 0.29 & 
705.40 & 0.34 & \bf{681.00} & 
0.61\\SCA8-0 & 1030.42 & 0.27 & 
1060.64 & 0.29 & \bf{961.50} & 
7.17\\SCA8-1 & 1096.83 & 0.31 & 
1103.08 & 0.34 & \bf{1049.65} & 
4.49\\SCA8-2 & 1087.28 & 0.32 & 
1125.84 & 0.31 & \bf{1039.64} & 
4.58\\SCA8-3 & 1033.54 & 0.36 & 
1075.56 & 0.35 & \bf{983.34} & 
5.11\\SCA8-4 & 1132.15 & 0.41 & 
1141.84 & 0.34 & \bf{1065.49} & 
6.26\\SCA8-5 & 1099.96 & 0.32 & 
1109.64 & 0.30 & \bf{1027.08} & 
7.10\\SCA8-6 & 1024.08 & 0.22 & 
1041.35 & 0.23 & \bf{971.82} & 
5.38\\SCA8-7 & 1111.36 & 0.34 & 
1157.67 & 0.30 & \bf{1051.28} & 
5.71\\SCA8-8 & 1136.05 & 0.28 & 
1157.72 & 0.32 & \bf{1071.18} & 
6.06\\SCA8-9 & 1118.25 & 0.34 & 
1193.90 & 0.26 & \bf{1060.50} & 
5.45\\CON3-0 & 646.46 & 0.30 & 
665.44 & 0.25 & \bf{616.52} & 
4.86\\CON3-1 & 576.23 & 0.30 & 
576.38 & 0.28 & \bf{554.47} & 
3.92\\CON3-2 & 539.27 & 0.26 & 
547.32 & 0.33 & \bf{518.00} & 
4.11\\CON3-3 & 637.97 & 0.27 & 
643.17 & 0.35 & \bf{591.19} & 
7.91\\CON3-4 & 629.50 & 0.37 & 
635.81 & 0.33 & \bf{588.79} & 
6.91\\CON3-5 & 590.35 & 0.44 & 
595.56 & 0.40 & \bf{563.70} & 
4.73\\CON3-6 & 503.97 & 0.38 & 
511.72 & 0.36 & \bf{499.05} & 
0.99\\CON3-7 & 601.57 & 0.46 & 
612.11 & 0.40 & \bf{576.48} & 
4.35\\CON3-8 & 528.09 & 0.40 & 
539.71 & 0.44 & \bf{523.05} & 
0.96\\CON3-9 & 592.21 & 0.41 & 
598.00 & 0.35 & \bf{578.24} & 
2.42\\CON8-0 & 930.17 & 0.35 & 
960.39 & 0.41 & \bf{857.17} & 
8.52\\CON8-1 & 774.37 & 0.61 & 
846.80 & 0.32 & \bf{740.85} & 
4.52\\CON8-2 & 739.67 & 0.36 & 
744.25 & 0.38 & \bf{712.89} & 
3.76\\CON8-3 & 841.30 & 0.34 & 
879.91 & 0.30 & \bf{811.07} & 
3.73\\CON8-4 & 808.16 & 0.26 & 
827.34 & 0.27 & \bf{772.25} & 
4.65\\CON8-5 & 798.75 & 0.27 & 
833.17 & 0.33 & \bf{754.88} & 
5.81\\CON8-6 & 691.84 & 0.48 & 
710.57 & 0.42 & \bf{678.92} & 
1.90\\CON8-7 & 837.59 & 0.28 & 
838.11 & 0.28 & \bf{811.96} & 
3.16\\CON8-8 & 779.18 & 0.32 & 
819.25 & 0.32 & \bf{767.53} & 
1.52\\CON8-9 & 835.42 & 0.28 & 
838.95 & 0.32 & \bf{809.00} & 
3.27\\[1ex]\hline
\end{tabular}
\label{table:nonlin}
\end{table} \clearpage
\begin{table}[ht]
\caption{Resultados de la ejecución de la metaheurística ILS, utilizando instancias de SalhiNagy con la configuración -n 5.0 -LS 10.0}
\centering
\small
\begin{tabular}{c c c c c c c}
\hline\hline
Instancia & Costo mínimo & Tiempo(seg.) & Costo promedio & Tiempo promedio(seg.) & Costo ILS & \%Gap \\ [0.5ex]
\hline
CMT1X & 495.02 & 0.32 & 
497.64 & 0.34 & \bf{466.77} & 
6.05\\CMT1Y & 478.97 & 0.34 & 
503.26 & 0.26 & \bf{466.77} & 
2.61\\CMT2X & 705.67 & 0.62 & 
718.98 & 0.93 & \bf{684.21} & 
3.14\\CMT2Y & 729.71 & 0.86 & 
731.92 & 0.83 & \bf{684.21} & 
6.65\\CMT3X & 731.47 & 2.53 & 
750.87 & 2.38 & \bf{721.40} & 
1.40\\CMT3Y & 735.15 & 2.72 & 
744.11 & 2.69 & \bf{721.40} & 
1.91\\CMT4X & 909.93 & 6.64 & 
929.17 & 6.45 & \bf{852.83} & 
6.70\\CMT4Y & 916.32 & 10.42 & 
925.44 & 8.23 & \bf{852.46} & 
7.49\\CMT5X & 1091.29 & 16.88 & 
1134.56 & 15.26 & \bf{1030.55} & 
5.89\\CMT5Y & 1117.61 & 16.38 & 
1126.31 & 17.75 & \bf{1031.17} & 
8.38\\CMT11X & 888.65 & 4.80 & 
922.73 & 5.00 & \bf{839.39} & 
5.87\\CMT11Y & 904.89 & 4.66 & 
925.85 & 4.71 & \bf{841.88} & 
7.48\\CMT12X & 691.86 & 2.15 & 
710.36 & 1.83 & \bf{662.22} & 
4.48\\CMT12Y & 677.26 & 1.05 & 
697.30 & 1.54 & \bf{662.22} & 
2.27\\[1ex]\hline
\end{tabular}
\label{table:nonlin}
\end{table} \clearpage
\begin{table}[ht]
\caption{Resultados de la ejecución de la metaheurística IGA, utilizando instancias de Dethloff con la configuración -n 200 -p 40}
\centering
\small
\begin{tabular}{c c c c c c c}
\hline\hline
Instancia & Costo mínimo & Tiempo(seg.) & Costo promedio & Tiempo promedio(seg.) & Costo IGA & \%Gap \\ [0.5ex]
\hline
SCA3-0 & \bf{636.06} & 2.46 & 
640.59 & 2.55 & 636.06 & 0.00\\
SCA3-1 & \bf{697.84} & 2.50 & 
701.69 & 2.52 & 697.84 & 0.00\\
SCA3-2 & 664.21 & 2.29 & 
667.73 & 2.44 & \bf{659.34} & 
0.74\\SCA3-3 & \bf{680.04} & 2.30 & 
681.92 & 2.39 & 680.04 & 0.00\\
SCA3-4 & \bf{690.50} & 2.48 & 
690.50 & 2.50 & 690.50 & 0.00\\
SCA3-5 & 672.49 & 2.75 & 
676.12 & 2.44 & \bf{659.90} & 
1.91\\SCA3-6 & \bf{651.09} & 2.62 & 
654.72 & 2.51 & 651.09 & 0.00\\
SCA3-7 & 666.15 & 2.72 & 
669.75 & 2.50 & \bf{659.17} & 
1.06\\SCA3-8 & \bf{719.47} & 2.38 & 
726.21 & 2.43 & 719.47 & 0.00\\
SCA3-9 & \bf{681.00} & 2.32 & 
682.52 & 2.37 & 681.00 & 0.00\\
SCA8-0 & 980.00 & 2.20 & 
997.11 & 2.46 & \bf{961.50} & 
1.92\\SCA8-1 & 1066.39 & 2.51 & 
1078.66 & 2.51 & \bf{1050.20} & 
1.54\\SCA8-2 & 1050.17 & 2.70 & 
1057.76 & 2.59 & \bf{1039.64} & 
1.01\\SCA8-3 & 1021.07 & 2.42 & 
1027.34 & 2.56 & \bf{983.34} & 
3.84\\SCA8-4 & 1067.55 & 2.39 & 
1087.63 & 2.55 & \bf{1065.49} & 
0.19\\SCA8-5 & 1058.77 & 2.58 & 
1070.58 & 2.50 & \bf{1027.08} & 
3.09\\SCA8-6 & 982.57 & 2.57 & 
996.50 & 2.52 & \bf{971.82} & 
1.11\\SCA8-7 & 1066.65 & 2.56 & 
1080.63 & 2.33 & \bf{1052.17} & 
1.38\\SCA8-8 & 1075.00 & 2.73 & 
1091.74 & 2.43 & \bf{1071.18} & 
0.36\\SCA8-9 & 1066.15 & 2.81 & 
1074.00 & 2.49 & \bf{1060.50} & 
0.53\\CON3-0 & 620.49 & 2.46 & 
625.32 & 2.50 & \bf{616.52} & 
0.64\\CON3-1 & 559.25 & 2.39 & 
562.70 & 2.62 & \bf{554.47} & 
0.86\\CON3-2 & 521.38 & 3.06 & 
521.72 & 2.67 & \bf{519.26} & 
0.41\\CON3-3 & 591.20 & 1.99 & 
595.65 & 2.52 & \bf{591.19} & 
0.00\\CON3-4 & \bf{589.32} & 2.35 & 
594.65 & 2.47 & 589.32 & 0.00\\
CON3-5 & \bf{563.70} & 2.35 & 
570.75 & 2.48 & 563.70 & 0.00\\
CON3-6 & 502.16 & 3.12 & 
504.33 & 2.70 & \bf{500.80} & 
0.27\\CON3-7 & 578.22 & 2.43 & 
585.55 & 2.44 & \bf{576.48} & 
0.30\\CON3-8 & 523.14 & 2.55 & 
530.06 & 2.54 & \bf{523.05} & 
0.02\\CON3-9 & 582.79 & 2.50 & 
588.32 & 2.60 & \bf{580.05} & 
0.47\\CON8-0 & 865.86 & 2.91 & 
873.31 & 2.50 & \bf{857.17} & 
1.01\\CON8-1 & 761.56 & 2.52 & 
768.52 & 2.64 & \bf{740.85} & 
2.80\\CON8-2 & 717.68 & 3.17 & 
724.56 & 2.69 & \bf{713.44} & 
0.59\\CON8-3 & 820.43 & 2.58 & 
838.04 & 2.56 & \bf{811.07} & 
1.15\\CON8-4 & 780.48 & 2.90 & 
797.31 & 2.44 & \bf{772.25} & 
1.07\\CON8-5 & 762.61 & 2.38 & 
773.18 & 2.53 & \bf{756.91} & 
0.75\\CON8-6 & 691.28 & 2.61 & 
698.43 & 2.52 & \bf{678.92} & 
1.82\\CON8-7 & 815.54 & 1.81 & 
819.12 & 2.35 & \bf{811.96} & 
0.44\\CON8-8 & 778.38 & 2.20 & 
788.26 & 2.57 & \bf{767.53} & 
1.41\\CON8-9 & 819.31 & 2.42 & 
824.07 & 2.52 & \bf{809.00} & 
1.27\\[1ex]\hline
\end{tabular}
\label{table:nonlin}
\end{table} \clearpage
\begin{table}[ht]
\caption{Resultados de la ejecución de la metaheurística IGA, utilizando instancias de SalhiNagy con la configuración -n 200 -p 40}
\centering
\small
\begin{tabular}{c c c c c c c}
\hline\hline
Instancia & Costo mínimo & Tiempo(seg.) & Costo promedio & Tiempo promedio(seg.) & Costo IGA & \%Gap \\ [0.5ex]
\hline
CMT1X & 478.71 & 2.78 & 
478.77 & 2.75 & \bf{469.80} & 
1.90\\CMT1Y & 489.72 & 2.36 & 
490.92 & 2.27 & \bf{469.80} & 
4.24\\CMT2X & 100000 & 0 & 
nan & nan & \bf{684.21} & 
14515.40\\CMT2Y & 100000 & 0 & 
nan & nan & \bf{684.21} & 
14515.40\\CMT3X & 729.59 & 13.71 & 
738.13 & 13.02 & \bf{721.27} & 
1.15\\CMT3Y & 743.48 & 12.40 & 
744.60 & 11.74 & \bf{721.27} & 
3.08\\CMT4X & 909.73 & 33.30 & 
917.21 & 33.37 & \bf{852.46} & 
6.72\\CMT4Y & 911.25 & 33.22 & 
912.25 & 32.94 & \bf{852.46} & 
6.90\\CMT5X & \bf{\underline{86.00}} & Command & 
86.00 & 0.00 & 1030.55 & 
-91.65\\CMT5Y & 100000 & 0 & 
nan & nan & \bf{1030.55} & 
9603.56\\CMT11X & 925.92 & 19.32 & 
929.17 & 19.28 & \bf{838.66} & 
10.40\\CMT11Y & 917.77 & 17.99 & 
938.25 & 17.70 & \bf{837.08} & 
9.64\\CMT12X & 674.46 & 11.34 & 
677.50 & 11.36 & \bf{662.22} & 
1.85\\CMT12Y & 681.53 & 10.93 & 
685.53 & 11.02 & \bf{662.22} & 
2.92\\[1ex]\hline
\end{tabular}
\label{table:nonlin}
\end{table} \clearpage
\begin{table}[ht]
\caption{Resultados de la ejecución de la metaheurística ILS, utilizando instancias de Dethloff con la configuración -n 15 -LS 30}
\centering
\small
\begin{tabular}{c c c c c c c}
\hline\hline
Instancia & Costo mínimo & Tiempo(seg.) & Costo promedio & Tiempo promedio(seg.) & Costo ILS & \%Gap \\ [0.5ex]
\hline
SCA3-0 & 640.55 & 1.79 & 
655.48 & 1.90 & \bf{635.62} & 
0.78\\SCA3-1 & 712.59 & 1.94 & 
717.62 & 2.06 & \bf{697.84} & 
2.11\\SCA3-2 & 676.34 & 1.91 & 
680.20 & 1.67 & \bf{659.34} & 
2.58\\SCA3-3 & 682.46 & 1.90 & 
693.12 & 1.99 & \bf{680.04} & 
0.36\\SCA3-4 & 692.57 & 1.66 & 
700.19 & 1.68 & \bf{690.50} & 
0.30\\SCA3-5 & 673.39 & 1.49 & 
679.03 & 1.94 & \bf{659.90} & 
2.04\\SCA3-6 & 661.58 & 1.56 & 
663.83 & 1.70 & \bf{651.09} & 
1.61\\SCA3-7 & 671.67 & 1.65 & 
672.90 & 1.81 & \bf{659.17} & 
1.90\\SCA3-8 & 719.77 & 1.39 & 
739.89 & 1.58 & \bf{719.47} & 
0.04\\SCA3-9 & 687.61 & 1.76 & 
698.24 & 1.96 & \bf{681.00} & 
0.97\\SCA8-0 & 995.61 & 1.60 & 
1032.03 & 1.77 & \bf{961.50} & 
3.55\\SCA8-1 & 1090.03 & 1.12 & 
1125.75 & 1.52 & \bf{1049.65} & 
3.85\\SCA8-2 & 1065.69 & 2.24 & 
1072.71 & 1.90 & \bf{1039.64} & 
2.51\\SCA8-3 & 1028.92 & 2.14 & 
1042.08 & 1.65 & \bf{983.34} & 
4.64\\SCA8-4 & 1074.02 & 1.94 & 
1093.02 & 1.82 & \bf{1065.49} & 
0.80\\SCA8-5 & 1061.29 & 1.44 & 
1090.14 & 1.31 & \bf{1027.08} & 
3.33\\SCA8-6 & 993.60 & 1.77 & 
1013.36 & 1.53 & \bf{971.82} & 
2.24\\SCA8-7 & 1102.40 & 1.67 & 
1117.30 & 1.55 & \bf{1051.28} & 
4.86\\SCA8-8 & 1093.75 & 1.85 & 
1110.20 & 1.64 & \bf{1071.18} & 
2.11\\SCA8-9 & 1075.68 & 1.54 & 
1121.37 & 1.40 & \bf{1060.50} & 
1.43\\CON3-0 & 630.73 & 1.80 & 
637.16 & 2.00 & \bf{616.52} & 
2.30\\CON3-1 & 567.97 & 1.60 & 
572.47 & 1.97 & \bf{554.47} & 
2.43\\CON3-2 & 526.47 & 2.14 & 
533.49 & 2.00 & \bf{518.00} & 
1.64\\CON3-3 & 601.66 & 2.30 & 
614.22 & 1.98 & \bf{591.19} & 
1.77\\CON3-4 & 606.76 & 1.67 & 
610.19 & 1.79 & \bf{588.79} & 
3.05\\CON3-5 & 572.75 & 1.90 & 
576.27 & 1.84 & \bf{563.70} & 
1.61\\CON3-6 & 510.96 & 1.67 & 
518.67 & 1.70 & \bf{499.05} & 
2.39\\CON3-7 & 586.01 & 1.94 & 
595.43 & 2.04 & \bf{576.48} & 
1.65\\CON3-8 & 524.59 & 2.12 & 
530.62 & 2.17 & \bf{523.05} & 
0.29\\CON3-9 & 590.64 & 2.04 & 
592.45 & 1.93 & \bf{578.24} & 
2.14\\CON8-0 & 883.45 & 2.80 & 
903.88 & 1.96 & \bf{857.17} & 
3.07\\CON8-1 & 753.57 & 1.45 & 
778.75 & 1.69 & \bf{740.85} & 
1.72\\CON8-2 & 720.40 & 2.71 & 
730.51 & 2.02 & \bf{712.89} & 
1.05\\CON8-3 & 860.41 & 1.44 & 
881.89 & 1.60 & \bf{811.07} & 
6.08\\CON8-4 & 816.41 & 2.18 & 
832.50 & 1.78 & \bf{772.25} & 
5.72\\CON8-5 & 770.29 & 1.86 & 
806.44 & 1.71 & \bf{754.88} & 
2.04\\CON8-6 & 702.59 & 2.32 & 
717.16 & 1.82 & \bf{678.92} & 
3.49\\CON8-7 & 837.24 & 1.79 & 
849.75 & 1.53 & \bf{811.96} & 
3.11\\CON8-8 & 791.08 & 1.30 & 
809.73 & 1.95 & \bf{767.53} & 
3.07\\CON8-9 & 820.59 & 1.86 & 
840.19 & 1.88 & \bf{809.00} & 
1.43\\[1ex]\hline
\end{tabular}
\label{table:nonlin}
\end{table} \clearpage
\begin{table}[ht]
\caption{Resultados de la ejecución de la metaheurística ILS, utilizando instancias de Dethloff con la configuración -n 15 -LS 60}
\centering
\small
\begin{tabular}{c c c c c c c}
\hline\hline
Instancia & Costo mínimo & Tiempo(seg.) & Costo promedio & Tiempo promedio(seg.) & Costo ILS & \%Gap \\ [0.5ex]
\hline
SCA3-0 & 640.55 & 4.19 & 
642.50 & 3.29 & \bf{635.62} & 
0.78\\SCA3-1 & 701.53 & 3.52 & 
709.92 & 3.40 & \bf{697.84} & 
0.53\\SCA3-2 & 664.21 & 3.14 & 
669.80 & 3.42 & \bf{659.34} & 
0.74\\SCA3-3 & \bf{680.04} & 3.98 & 
687.77 & 3.74 & 680.04 & 0.00\\
SCA3-4 & \bf{690.50} & 2.74 & 
703.37 & 3.23 & 690.50 & 0.00\\
SCA3-5 & 673.39 & 2.34 & 
680.95 & 2.51 & \bf{659.90} & 
2.04\\SCA3-6 & \bf{651.09} & 3.46 & 
656.90 & 3.21 & 651.09 & 0.00\\
SCA3-7 & 671.77 & 2.82 & 
676.35 & 3.06 & \bf{659.17} & 
1.91\\SCA3-8 & 719.77 & 3.71 & 
723.85 & 3.37 & \bf{719.47} & 
0.04\\SCA3-9 & 685.00 & 3.48 & 
689.36 & 3.12 & \bf{681.00} & 
0.59\\SCA8-0 & 1001.23 & 2.43 & 
1016.81 & 2.56 & \bf{961.50} & 
4.13\\SCA8-1 & 1081.98 & 2.55 & 
1093.37 & 2.48 & \bf{1049.65} & 
3.08\\SCA8-2 & 1065.72 & 2.48 & 
1081.70 & 2.79 & \bf{1039.64} & 
2.51\\SCA8-3 & 1032.86 & 2.55 & 
1038.83 & 2.73 & \bf{983.34} & 
5.04\\SCA8-4 & 1094.49 & 3.08 & 
1123.04 & 2.66 & \bf{1065.49} & 
2.72\\SCA8-5 & 1042.30 & 3.39 & 
1073.68 & 3.39 & \bf{1027.08} & 
1.48\\SCA8-6 & 1003.20 & 3.24 & 
1012.78 & 2.96 & \bf{971.82} & 
3.23\\SCA8-7 & 1089.73 & 2.90 & 
1108.87 & 2.56 & \bf{1051.28} & 
3.66\\SCA8-8 & 1094.94 & 2.31 & 
1119.85 & 2.51 & \bf{1071.18} & 
2.22\\SCA8-9 & 1078.30 & 3.46 & 
1103.21 & 2.87 & \bf{1060.50} & 
1.68\\CON3-0 & 634.79 & 3.45 & 
638.49 & 3.69 & \bf{616.52} & 
2.96\\CON3-1 & 568.40 & 3.16 & 
571.22 & 3.35 & \bf{554.47} & 
2.51\\CON3-2 & 521.38 & 3.92 & 
526.15 & 4.04 & \bf{518.00} & 
0.65\\CON3-3 & 598.94 & 4.20 & 
608.76 & 3.36 & \bf{591.19} & 
1.31\\CON3-4 & 595.87 & 2.74 & 
599.10 & 3.32 & \bf{588.79} & 
1.20\\CON3-5 & 575.00 & 3.58 & 
579.38 & 3.02 & \bf{563.70} & 
2.00\\CON3-6 & 502.16 & 4.72 & 
505.20 & 3.95 & \bf{499.05} & 
0.62\\CON3-7 & 599.51 & 3.50 & 
607.73 & 3.40 & \bf{576.48} & 
3.99\\CON3-8 & 529.65 & 3.48 & 
536.17 & 3.69 & \bf{523.05} & 
1.26\\CON3-9 & 588.48 & 2.86 & 
588.80 & 3.56 & \bf{578.24} & 
1.77\\CON8-0 & 881.01 & 3.22 & 
899.66 & 2.77 & \bf{857.17} & 
2.78\\CON8-1 & 771.48 & 2.84 & 
793.79 & 2.71 & \bf{740.85} & 
4.13\\CON8-2 & 727.26 & 2.62 & 
741.25 & 2.85 & \bf{712.89} & 
2.02\\CON8-3 & 836.95 & 3.34 & 
852.60 & 3.22 & \bf{811.07} & 
3.19\\CON8-4 & 791.48 & 2.55 & 
816.20 & 2.62 & \bf{772.25} & 
2.49\\CON8-5 & 771.59 & 3.42 & 
781.93 & 3.00 & \bf{754.88} & 
2.21\\CON8-6 & 695.95 & 4.27 & 
717.71 & 3.70 & \bf{678.92} & 
2.51\\CON8-7 & 834.85 & 2.54 & 
839.75 & 2.83 & \bf{811.96} & 
2.82\\CON8-8 & 799.56 & 1.93 & 
806.33 & 2.35 & \bf{767.53} & 
4.17\\CON8-9 & 826.53 & 3.02 & 
848.34 & 2.59 & \bf{809.00} & 
2.17\\[1ex]\hline
\end{tabular}
\label{table:nonlin}
\end{table} \clearpage
\begin{table}[ht]
\caption{Resultados de la ejecución de la metaheurística ILS, utilizando instancias de Dethloff con la configuración -n 15 -LS 90}
\centering
\small
\begin{tabular}{c c c c c c c}
\hline\hline
Instancia & Costo mínimo & Tiempo(seg.) & Costo promedio & Tiempo promedio(seg.) & Costo ILS & \%Gap \\ [0.5ex]
\hline
SCA3-0 & 640.55 & 4.70 & 
641.77 & 4.64 & \bf{635.62} & 
0.78\\SCA3-1 & 701.53 & 5.03 & 
714.09 & 5.12 & \bf{697.84} & 
0.53\\SCA3-2 & 664.21 & 4.75 & 
667.49 & 4.28 & \bf{659.34} & 
0.74\\SCA3-3 & 680.60 & 5.80 & 
687.98 & 5.00 & \bf{680.04} & 
0.08\\SCA3-4 & \bf{690.50} & 4.72 & 
697.63 & 5.04 & 690.50 & 0.00\\
SCA3-5 & 675.81 & 5.56 & 
683.88 & 4.75 & \bf{659.90} & 
2.41\\SCA3-6 & 656.40 & 4.81 & 
660.88 & 4.44 & \bf{651.09} & 
0.82\\SCA3-7 & 671.77 & 5.10 & 
671.77 & 5.08 & \bf{659.17} & 
1.91\\SCA3-8 & 719.77 & 5.16 & 
724.59 & 4.87 & \bf{719.47} & 
0.04\\SCA3-9 & 685.00 & 4.59 & 
696.40 & 4.42 & \bf{681.00} & 
0.59\\SCA8-0 & 1018.38 & 4.30 & 
1025.53 & 4.17 & \bf{961.50} & 
5.92\\SCA8-1 & 1070.49 & 4.08 & 
1091.56 & 3.33 & \bf{1049.65} & 
1.99\\SCA8-2 & 1067.94 & 4.61 & 
1084.81 & 3.86 & \bf{1039.64} & 
2.72\\SCA8-3 & 1039.55 & 3.38 & 
1045.67 & 3.90 & \bf{983.34} & 
5.72\\SCA8-4 & 1072.75 & 3.60 & 
1091.08 & 3.65 & \bf{1065.49} & 
0.68\\SCA8-5 & 1060.40 & 3.97 & 
1068.40 & 3.88 & \bf{1027.08} & 
3.24\\SCA8-6 & 992.86 & 4.08 & 
1011.33 & 4.14 & \bf{971.82} & 
2.17\\SCA8-7 & 1076.68 & 5.16 & 
1101.60 & 4.53 & \bf{1051.28} & 
2.42\\SCA8-8 & 1094.43 & 4.30 & 
1106.16 & 4.55 & \bf{1071.18} & 
2.17\\SCA8-9 & 1092.40 & 3.14 & 
1107.92 & 3.29 & \bf{1060.50} & 
3.01\\CON3-0 & 628.47 & 6.44 & 
635.05 & 5.08 & \bf{616.52} & 
1.94\\CON3-1 & 563.27 & 4.84 & 
564.48 & 4.81 & \bf{554.47} & 
1.59\\CON3-2 & 521.38 & 5.42 & 
527.42 & 4.86 & \bf{518.00} & 
0.65\\CON3-3 & \bf{591.19} & 5.83 & 
595.52 & 5.08 & 591.19 & 0.00\\
CON3-4 & 594.59 & 4.84 & 
603.66 & 5.53 & \bf{588.79} & 
0.99\\CON3-5 & 569.88 & 4.75 & 
575.14 & 4.63 & \bf{563.70} & 
1.10\\CON3-6 & 503.97 & 4.19 & 
512.92 & 4.82 & \bf{499.05} & 
0.99\\CON3-7 & 586.01 & 5.14 & 
589.67 & 4.93 & \bf{576.48} & 
1.65\\CON3-8 & 532.86 & 4.46 & 
535.85 & 5.28 & \bf{523.05} & 
1.88\\CON3-9 & 589.57 & 5.48 & 
591.36 & 5.19 & \bf{578.24} & 
1.96\\CON8-0 & 861.11 & 3.79 & 
902.03 & 3.49 & \bf{857.17} & 
0.46\\CON8-1 & 780.09 & 3.71 & 
781.93 & 4.16 & \bf{740.85} & 
5.30\\CON8-2 & 714.44 & 4.98 & 
737.90 & 3.91 & \bf{712.89} & 
0.22\\CON8-3 & 825.59 & 3.94 & 
839.10 & 3.92 & \bf{811.07} & 
1.79\\CON8-4 & 813.26 & 4.22 & 
824.50 & 4.11 & \bf{772.25} & 
5.31\\CON8-5 & 774.59 & 3.05 & 
780.85 & 4.09 & \bf{754.88} & 
2.61\\CON8-6 & 698.73 & 3.92 & 
713.08 & 4.11 & \bf{678.92} & 
2.92\\CON8-7 & 836.82 & 3.79 & 
844.29 & 4.24 & \bf{811.96} & 
3.06\\CON8-8 & 769.65 & 4.06 & 
787.22 & 4.39 & \bf{767.53} & 
0.28\\CON8-9 & 828.15 & 5.67 & 
835.27 & 4.55 & \bf{809.00} & 
2.37\\[1ex]\hline
\end{tabular}
\label{table:nonlin}
\end{table} \clearpage
\begin{table}[ht]
\caption{Resultados de la ejecución de la metaheurística ILS, utilizando instancias de Dethloff con la configuración -n 15 -LS 120}
\centering
\small
\begin{tabular}{c c c c c c c}
\hline\hline
Instancia & Costo mínimo & Tiempo(seg.) & Costo promedio & Tiempo promedio(seg.) & Costo ILS & \%Gap \\ [0.5ex]
\hline
SCA3-0 & 640.55 & 6.77 & 
640.84 & 7.06 & \bf{635.62} & 
0.78\\SCA3-1 & \bf{697.84} & 7.08 & 
702.58 & 6.76 & 697.84 & 0.00\\
SCA3-2 & 664.25 & 6.07 & 
667.87 & 6.78 & \bf{659.34} & 
0.74\\SCA3-3 & 680.60 & 6.51 & 
681.21 & 6.64 & \bf{680.04} & 
0.08\\SCA3-4 & \bf{690.50} & 7.17 & 
691.18 & 6.64 & 690.50 & 0.00\\
SCA3-5 & 662.75 & 6.75 & 
673.53 & 6.15 & \bf{659.90} & 
0.43\\SCA3-6 & 653.68 & 7.22 & 
653.75 & 6.32 & \bf{651.09} & 
0.40\\SCA3-7 & 666.60 & 6.08 & 
671.66 & 6.38 & \bf{659.17} & 
1.13\\SCA3-8 & 719.77 & 5.17 & 
729.25 & 5.86 & \bf{719.47} & 
0.04\\SCA3-9 & \bf{681.00} & 6.40 & 
684.12 & 6.57 & 681.00 & 0.00\\
SCA8-0 & 1003.31 & 6.32 & 
1013.44 & 5.42 & \bf{961.50} & 
4.35\\SCA8-1 & 1085.86 & 5.16 & 
1100.67 & 4.71 & \bf{1049.65} & 
3.45\\SCA8-2 & 1065.79 & 4.29 & 
1067.69 & 4.95 & \bf{1039.64} & 
2.52\\SCA8-3 & 1021.24 & 5.12 & 
1034.69 & 5.32 & \bf{983.34} & 
3.85\\SCA8-4 & 1110.77 & 4.91 & 
1118.04 & 5.28 & \bf{1065.49} & 
4.25\\SCA8-5 & 1078.04 & 6.39 & 
1089.28 & 5.49 & \bf{1027.08} & 
4.96\\SCA8-6 & 995.43 & 4.80 & 
1010.72 & 4.99 & \bf{971.82} & 
2.43\\SCA8-7 & 1074.24 & 5.05 & 
1090.67 & 5.44 & \bf{1051.28} & 
2.18\\SCA8-8 & \bf{1071.18} & 5.46 & 
1088.94 & 5.14 & 1071.18 & 0.00\\
SCA8-9 & 1097.23 & 4.23 & 
1109.23 & 4.70 & \bf{1060.50} & 
3.46\\CON3-0 & 628.47 & 6.34 & 
638.43 & 6.47 & \bf{616.52} & 
1.94\\CON3-1 & 560.75 & 5.78 & 
565.05 & 6.88 & \bf{554.47} & 
1.13\\CON3-2 & 521.38 & 6.04 & 
526.72 & 6.04 & \bf{518.00} & 
0.65\\CON3-3 & 599.26 & 6.47 & 
601.99 & 6.96 & \bf{591.19} & 
1.37\\CON3-4 & 591.43 & 5.70 & 
599.82 & 6.47 & \bf{588.79} & 
0.45\\CON3-5 & 564.89 & 6.11 & 
577.07 & 6.43 & \bf{563.70} & 
0.21\\CON3-6 & 502.16 & 7.07 & 
509.27 & 6.69 & \bf{499.05} & 
0.62\\CON3-7 & 586.01 & 7.19 & 
591.93 & 6.25 & \bf{576.48} & 
1.65\\CON3-8 & 523.14 & 7.29 & 
524.17 & 6.97 & \bf{523.05} & 
0.02\\CON3-9 & 588.63 & 6.36 & 
589.27 & 6.74 & \bf{578.24} & 
1.80\\CON8-0 & 908.22 & 4.20 & 
917.81 & 4.99 & \bf{857.17} & 
5.96\\CON8-1 & 751.76 & 4.04 & 
763.42 & 4.79 & \bf{740.85} & 
1.47\\CON8-2 & 723.69 & 4.22 & 
733.68 & 4.45 & \bf{712.89} & 
1.51\\CON8-3 & 830.13 & 4.39 & 
840.80 & 5.45 & \bf{811.07} & 
2.35\\CON8-4 & 802.65 & 6.07 & 
822.98 & 5.59 & \bf{772.25} & 
3.94\\CON8-5 & 763.13 & 6.47 & 
782.62 & 5.59 & \bf{754.88} & 
1.09\\CON8-6 & 707.20 & 4.66 & 
709.92 & 5.21 & \bf{678.92} & 
4.17\\CON8-7 & 827.39 & 5.32 & 
837.38 & 5.76 & \bf{811.96} & 
1.90\\CON8-8 & 784.71 & 5.62 & 
800.52 & 5.30 & \bf{767.53} & 
2.24\\CON8-9 & 825.28 & 8.17 & 
834.76 & 6.72 & \bf{809.00} & 
2.01\\[1ex]\hline
\end{tabular}
\label{table:nonlin}
\end{table} \clearpage
\begin{table}[ht]
\caption{Resultados de la ejecución de la metaheurística ILS, utilizando instancias de Dethloff con la configuración -n 30 -LS 30}
\centering
\small
\begin{tabular}{c c c c c c c}
\hline\hline
Instancia & Costo mínimo & Tiempo(seg.) & Costo promedio & Tiempo promedio(seg.) & Costo ILS & \%Gap \\ [0.5ex]
\hline
SCA3-0 & 636.06 & 3.62 & 
640.77 & 3.79 & \bf{635.62} & 
0.07\\SCA3-1 & 712.59 & 4.61 & 
720.32 & 4.02 & \bf{697.84} & 
2.11\\SCA3-2 & \bf{659.34} & 3.25 & 
669.54 & 3.71 & 659.34 & 0.00\\
SCA3-3 & 680.60 & 4.20 & 
685.15 & 4.39 & \bf{680.04} & 
0.08\\SCA3-4 & \bf{690.50} & 3.64 & 
692.85 & 3.65 & 690.50 & 0.00\\
SCA3-5 & 670.10 & 4.92 & 
684.67 & 4.09 & \bf{659.90} & 
1.55\\SCA3-6 & 652.94 & 3.71 & 
663.07 & 3.38 & \bf{651.09} & 
0.28\\SCA3-7 & 671.77 & 3.42 & 
671.77 & 3.81 & \bf{659.17} & 
1.91\\SCA3-8 & 719.77 & 3.10 & 
729.23 & 3.36 & \bf{719.47} & 
0.04\\SCA3-9 & \bf{681.00} & 3.60 & 
685.81 & 3.75 & 681.00 & 0.00\\
SCA8-0 & 1033.81 & 3.26 & 
1046.88 & 3.31 & \bf{961.50} & 
7.52\\SCA8-1 & 1066.99 & 2.89 & 
1085.91 & 2.88 & \bf{1049.65} & 
1.65\\SCA8-2 & 1052.74 & 3.48 & 
1074.40 & 3.48 & \bf{1039.64} & 
1.26\\SCA8-3 & 1026.64 & 3.36 & 
1036.17 & 3.44 & \bf{983.34} & 
4.40\\SCA8-4 & 1110.30 & 2.75 & 
1131.09 & 3.13 & \bf{1065.49} & 
4.21\\SCA8-5 & 1066.35 & 3.03 & 
1074.41 & 3.19 & \bf{1027.08} & 
3.82\\SCA8-6 & 989.31 & 3.08 & 
1010.94 & 3.12 & \bf{971.82} & 
1.80\\SCA8-7 & 1097.69 & 3.02 & 
1107.98 & 3.00 & \bf{1051.28} & 
4.41\\SCA8-8 & 1092.86 & 3.52 & 
1100.42 & 2.96 & \bf{1071.18} & 
2.02\\SCA8-9 & 1068.61 & 2.87 & 
1099.26 & 3.15 & \bf{1060.50} & 
0.76\\CON3-0 & 626.38 & 4.20 & 
634.40 & 3.81 & \bf{616.52} & 
1.60\\CON3-1 & 560.75 & 4.01 & 
566.24 & 3.80 & \bf{554.47} & 
1.13\\CON3-2 & 521.38 & 3.22 & 
524.28 & 3.56 & \bf{518.00} & 
0.65\\CON3-3 & 591.20 & 3.02 & 
602.52 & 3.68 & \bf{591.19} & 
0.00\\CON3-4 & 591.43 & 4.02 & 
598.37 & 3.79 & \bf{588.79} & 
0.45\\CON3-5 & 581.82 & 3.96 & 
587.28 & 4.18 & \bf{563.70} & 
3.21\\CON3-6 & 502.16 & 3.56 & 
504.58 & 3.62 & \bf{499.05} & 
0.62\\CON3-7 & 578.41 & 3.00 & 
593.27 & 3.77 & \bf{576.48} & 
0.33\\CON3-8 & 523.14 & 4.04 & 
532.03 & 3.81 & \bf{523.05} & 
0.02\\CON3-9 & 588.99 & 3.79 & 
590.31 & 3.72 & \bf{578.24} & 
1.86\\CON8-0 & 881.09 & 3.62 & 
898.86 & 2.94 & \bf{857.17} & 
2.79\\CON8-1 & 761.91 & 3.76 & 
791.47 & 3.50 & \bf{740.85} & 
2.84\\CON8-2 & 722.58 & 4.10 & 
739.35 & 3.66 & \bf{712.89} & 
1.36\\CON8-3 & 848.09 & 3.49 & 
856.38 & 3.38 & \bf{811.07} & 
4.56\\CON8-4 & 806.27 & 2.76 & 
823.93 & 2.81 & \bf{772.25} & 
4.41\\CON8-5 & 768.53 & 3.97 & 
781.40 & 3.73 & \bf{754.88} & 
1.81\\CON8-6 & 695.12 & 3.22 & 
698.06 & 4.00 & \bf{678.92} & 
2.39\\CON8-7 & 820.92 & 2.99 & 
840.83 & 3.32 & \bf{811.96} & 
1.10\\CON8-8 & 791.67 & 3.97 & 
807.69 & 3.07 & \bf{767.53} & 
3.15\\CON8-9 & 856.27 & 3.17 & 
884.40 & 2.94 & \bf{809.00} & 
5.84\\[1ex]\hline
\end{tabular}
\label{table:nonlin}
\end{table} \clearpage
\begin{table}[ht]
\caption{Resultados de la ejecución de la metaheurística ILS, utilizando instancias de Dethloff con la configuración -n 60 -LS 30}
\centering
\small
\begin{tabular}{c c c c c c c}
\hline\hline
Instancia & Costo mínimo & Tiempo(seg.) & Costo promedio & Tiempo promedio(seg.) & Costo ILS & \%Gap \\ [0.5ex]
\hline
SCA3-0 & 640.55 & 6.33 & 
641.59 & 7.43 & \bf{635.62} & 
0.78\\SCA3-1 & \bf{697.84} & 7.09 & 
702.51 & 7.21 & 697.84 & 0.00\\
SCA3-2 & 661.13 & 6.96 & 
664.09 & 7.25 & \bf{659.34} & 
0.27\\SCA3-3 & 681.35 & 6.27 & 
683.29 & 7.44 & \bf{680.04} & 
0.19\\SCA3-4 & \bf{690.50} & 9.01 & 
696.73 & 7.51 & 690.50 & 0.00\\
SCA3-5 & \bf{659.90} & 7.28 & 
667.38 & 7.52 & 659.90 & 0.00\\
SCA3-6 & 655.19 & 7.64 & 
658.14 & 6.78 & \bf{651.09} & 
0.63\\SCA3-7 & 669.89 & 7.73 & 
674.02 & 7.16 & \bf{659.17} & 
1.63\\SCA3-8 & 722.05 & 8.20 & 
724.53 & 7.85 & \bf{719.47} & 
0.36\\SCA3-9 & \bf{681.00} & 7.00 & 
685.60 & 8.06 & 681.00 & 0.00\\
SCA8-0 & 980.64 & 5.90 & 
999.42 & 7.05 & \bf{961.50} & 
1.99\\SCA8-1 & 1070.29 & 6.41 & 
1081.95 & 6.13 & \bf{1049.65} & 
1.97\\SCA8-2 & 1064.87 & 5.83 & 
1073.47 & 6.00 & \bf{1039.64} & 
2.43\\SCA8-3 & 1024.45 & 6.72 & 
1032.76 & 6.18 & \bf{983.34} & 
4.18\\SCA8-4 & 1069.71 & 6.34 & 
1096.02 & 6.71 & \bf{1065.49} & 
0.40\\SCA8-5 & 1066.92 & 5.56 & 
1079.41 & 6.91 & \bf{1027.08} & 
3.88\\SCA8-6 & 998.94 & 7.54 & 
1018.07 & 6.22 & \bf{971.82} & 
2.79\\SCA8-7 & 1074.78 & 6.05 & 
1084.87 & 6.50 & \bf{1051.28} & 
2.24\\SCA8-8 & \bf{1071.18} & 6.22 & 
1091.97 & 6.80 & 1071.18 & 0.00\\
SCA8-9 & \bf{1060.50} & 6.11 & 
1082.26 & 5.75 & 1060.50 & 0.00\\
CON3-0 & 620.29 & 8.11 & 
632.31 & 7.49 & \bf{616.52} & 
0.61\\CON3-1 & 560.32 & 7.73 & 
563.30 & 8.26 & \bf{554.47} & 
1.06\\CON3-2 & 521.63 & 7.70 & 
525.69 & 7.41 & \bf{518.00} & 
0.70\\CON3-3 & 592.43 & 8.04 & 
594.55 & 8.05 & \bf{591.19} & 
0.21\\CON3-4 & 595.00 & 7.55 & 
597.31 & 7.67 & \bf{588.79} & 
1.05\\CON3-5 & 567.94 & 8.19 & 
570.37 & 7.38 & \bf{563.70} & 
0.75\\CON3-6 & 505.41 & 7.56 & 
507.06 & 7.59 & \bf{499.05} & 
1.27\\CON3-7 & 577.91 & 8.83 & 
588.48 & 8.79 & \bf{576.48} & 
0.25\\CON3-8 & 526.59 & 7.89 & 
532.21 & 7.83 & \bf{523.05} & 
0.68\\CON3-9 & 578.98 & 7.59 & 
588.52 & 7.89 & \bf{578.24} & 
0.13\\CON8-0 & 887.97 & 6.14 & 
899.45 & 6.34 & \bf{857.17} & 
3.59\\CON8-1 & 754.20 & 7.10 & 
768.77 & 6.81 & \bf{740.85} & 
1.80\\CON8-2 & 724.68 & 6.64 & 
734.25 & 6.33 & \bf{712.89} & 
1.65\\CON8-3 & 833.08 & 5.72 & 
839.68 & 6.43 & \bf{811.07} & 
2.71\\CON8-4 & 777.81 & 6.04 & 
803.24 & 6.25 & \bf{772.25} & 
0.72\\CON8-5 & 777.13 & 6.62 & 
780.91 & 6.39 & \bf{754.88} & 
2.95\\CON8-6 & 702.88 & 7.25 & 
707.37 & 6.83 & \bf{678.92} & 
3.53\\CON8-7 & 821.48 & 7.32 & 
830.02 & 6.45 & \bf{811.96} & 
1.17\\CON8-8 & 784.56 & 7.02 & 
794.87 & 6.26 & \bf{767.53} & 
2.22\\CON8-9 & 820.75 & 6.53 & 
828.42 & 6.73 & \bf{809.00} & 
1.45\\[1ex]\hline
\end{tabular}
\label{table:nonlin}
\end{table} \clearpage
\begin{table}[ht]
\caption{Resultados de la ejecución de la metaheurística ILS, utilizando instancias de Dethloff con la configuración -n 90 -LS 30}
\centering
\small
\begin{tabular}{c c c c c c c}
\hline\hline
Instancia & Costo mínimo & Tiempo(seg.) & Costo promedio & Tiempo promedio(seg.) & Costo ILS & \%Gap \\ [0.5ex]
\hline
SCA3-0 & 636.06 & 11.21 & 
640.77 & 10.92 & \bf{635.62} & 
0.07\\SCA3-1 & 701.53 & 11.23 & 
707.72 & 11.54 & \bf{697.84} & 
0.53\\SCA3-2 & \bf{659.34} & 10.72 & 
663.62 & 10.84 & 659.34 & 0.00\\
SCA3-3 & 680.60 & 12.51 & 
680.97 & 12.19 & \bf{680.04} & 
0.08\\SCA3-4 & \bf{690.50} & 11.83 & 
690.50 & 11.31 & 690.50 & 0.00\\
SCA3-5 & \bf{659.90} & 10.98 & 
672.12 & 11.10 & 659.90 & 0.00\\
SCA3-6 & \bf{651.09} & 11.04 & 
651.90 & 10.64 & 651.09 & 0.00\\
SCA3-7 & 671.67 & 10.09 & 
671.86 & 10.86 & \bf{659.17} & 
1.90\\SCA3-8 & \bf{719.47} & 10.26 & 
722.87 & 10.81 & 719.47 & 0.00\\
SCA3-9 & \bf{681.00} & 15.94 & 
684.51 & 11.97 & 681.00 & 0.00\\
SCA8-0 & 987.90 & 10.45 & 
1000.85 & 10.28 & \bf{961.50} & 
2.75\\SCA8-1 & 1070.29 & 9.62 & 
1078.51 & 9.58 & \bf{1049.65} & 
1.97\\SCA8-2 & 1066.25 & 9.46 & 
1073.31 & 9.67 & \bf{1039.64} & 
2.56\\SCA8-3 & 1015.61 & 10.99 & 
1024.14 & 9.85 & \bf{983.34} & 
3.28\\SCA8-4 & 1067.55 & 9.97 & 
1081.87 & 9.91 & \bf{1065.49} & 
0.19\\SCA8-5 & 1076.43 & 10.91 & 
1085.88 & 9.99 & \bf{1027.08} & 
4.80\\SCA8-6 & 972.48 & 11.08 & 
991.19 & 9.61 & \bf{971.82} & 
0.07\\SCA8-7 & 1073.05 & 9.08 & 
1086.76 & 9.28 & \bf{1051.28} & 
2.07\\SCA8-8 & \bf{1071.18} & 11.91 & 
1090.25 & 10.36 & 1071.18 & 0.00\\
SCA8-9 & 1068.10 & 9.62 & 
1083.19 & 8.78 & \bf{1060.50} & 
0.72\\CON3-0 & 630.73 & 11.42 & 
638.47 & 11.07 & \bf{616.52} & 
2.30\\CON3-1 & 556.04 & 11.89 & 
561.10 & 11.31 & \bf{554.47} & 
0.28\\CON3-2 & 521.38 & 11.87 & 
522.07 & 11.71 & \bf{518.00} & 
0.65\\CON3-3 & 591.48 & 11.50 & 
594.08 & 11.53 & \bf{591.19} & 
0.05\\CON3-4 & 597.39 & 13.06 & 
604.51 & 11.36 & \bf{588.79} & 
1.46\\CON3-5 & 568.79 & 12.30 & 
572.18 & 11.38 & \bf{563.70} & 
0.90\\CON3-6 & 502.16 & 13.02 & 
506.11 & 11.70 & \bf{499.05} & 
0.62\\CON3-7 & 578.41 & 10.87 & 
586.40 & 11.30 & \bf{576.48} & 
0.33\\CON3-8 & 523.14 & 14.50 & 
524.91 & 13.16 & \bf{523.05} & 
0.02\\CON3-9 & 588.48 & 11.86 & 
589.52 & 11.71 & \bf{578.24} & 
1.77\\CON8-0 & 873.05 & 9.95 & 
888.34 & 9.89 & \bf{857.17} & 
1.85\\CON8-1 & 755.73 & 10.38 & 
767.35 & 9.10 & \bf{740.85} & 
2.01\\CON8-2 & 728.73 & 10.37 & 
732.23 & 10.21 & \bf{712.89} & 
2.22\\CON8-3 & 833.74 & 10.16 & 
838.50 & 10.72 & \bf{811.07} & 
2.80\\CON8-4 & 794.54 & 9.48 & 
806.65 & 9.34 & \bf{772.25} & 
2.89\\CON8-5 & 769.78 & 17.65 & 
775.91 & 11.82 & \bf{754.88} & 
1.97\\CON8-6 & 698.93 & 9.32 & 
702.66 & 10.30 & \bf{678.92} & 
2.95\\CON8-7 & 823.68 & 11.48 & 
839.07 & 9.53 & \bf{811.96} & 
1.44\\CON8-8 & 777.60 & 10.89 & 
785.35 & 10.15 & \bf{767.53} & 
1.31\\CON8-9 & 816.10 & 10.04 & 
833.26 & 9.93 & \bf{809.00} & 
0.88\\[1ex]\hline
\end{tabular}
\label{table:nonlin}
\end{table} \clearpage
\begin{table}[ht]
\caption{Resultados de la ejecución de la metaheurística ILS, utilizando instancias de Dethloff con la configuración -n 120 -LS 30}
\centering
\small
\begin{tabular}{c c c c c c c}
\hline\hline
Instancia & Costo mínimo & Tiempo(seg.) & Costo promedio & Tiempo promedio(seg.) & Costo ILS & \%Gap \\ [0.5ex]
\hline
SCA3-0 & 640.55 & 15.54 & 
640.84 & 15.17 & \bf{635.62} & 
0.78\\SCA3-1 & \bf{697.84} & 16.66 & 
701.99 & 15.61 & 697.84 & 0.00\\
SCA3-2 & 661.13 & 14.13 & 
663.38 & 14.22 & \bf{659.34} & 
0.27\\SCA3-3 & \bf{680.04} & 15.33 & 
680.50 & 15.98 & 680.04 & 0.00\\
SCA3-4 & \bf{690.50} & 13.04 & 
690.50 & 14.90 & 690.50 & 0.00\\
SCA3-5 & 665.64 & 14.00 & 
669.92 & 15.20 & \bf{659.90} & 
0.87\\SCA3-6 & \bf{651.09} & 14.50 & 
655.17 & 13.97 & 651.09 & 0.00\\
SCA3-7 & 667.24 & 14.04 & 
670.64 & 14.80 & \bf{659.17} & 
1.22\\SCA3-8 & \bf{719.47} & 15.18 & 
723.74 & 14.38 & 719.47 & 0.00\\
SCA3-9 & \bf{681.00} & 14.55 & 
683.89 & 14.48 & 681.00 & 0.00\\
SCA8-0 & 987.90 & 14.53 & 
1005.19 & 14.78 & \bf{961.50} & 
2.75\\SCA8-1 & 1077.27 & 12.27 & 
1087.58 & 12.43 & \bf{1049.65} & 
2.63\\SCA8-2 & 1052.94 & 12.94 & 
1063.14 & 13.57 & \bf{1039.64} & 
1.28\\SCA8-3 & 1009.83 & 12.36 & 
1016.30 & 12.44 & \bf{983.34} & 
2.69\\SCA8-4 & 1069.80 & 13.99 & 
1084.88 & 12.65 & \bf{1065.49} & 
0.40\\SCA8-5 & 1058.85 & 15.92 & 
1073.06 & 13.89 & \bf{1027.08} & 
3.09\\SCA8-6 & 973.30 & 14.38 & 
992.66 & 13.86 & \bf{971.82} & 
0.15\\SCA8-7 & 1067.11 & 13.05 & 
1080.41 & 12.75 & \bf{1051.28} & 
1.51\\SCA8-8 & 1090.67 & 13.04 & 
1093.87 & 12.06 & \bf{1071.18} & 
1.82\\SCA8-9 & 1069.70 & 11.89 & 
1093.79 & 12.03 & \bf{1060.50} & 
0.87\\CON3-0 & 630.73 & 15.75 & 
633.61 & 15.21 & \bf{616.52} & 
2.30\\CON3-1 & 556.04 & 15.27 & 
559.57 & 15.10 & \bf{554.47} & 
0.28\\CON3-2 & 521.38 & 15.50 & 
522.95 & 15.69 & \bf{518.00} & 
0.65\\CON3-3 & 591.20 & 14.55 & 
595.86 & 16.17 & \bf{591.19} & 
0.00\\CON3-4 & 589.32 & 16.23 & 
592.68 & 15.28 & \bf{588.79} & 
0.09\\CON3-5 & 569.74 & 15.43 & 
572.97 & 13.85 & \bf{563.70} & 
1.07\\CON3-6 & 502.16 & 13.99 & 
503.76 & 15.10 & \bf{499.05} & 
0.62\\CON3-7 & 588.25 & 14.24 & 
595.73 & 14.79 & \bf{576.48} & 
2.04\\CON3-8 & 523.68 & 14.28 & 
525.83 & 15.13 & \bf{523.05} & 
0.12\\CON3-9 & 588.99 & 17.03 & 
589.14 & 15.68 & \bf{578.24} & 
1.86\\CON8-0 & 870.49 & 15.48 & 
904.50 & 13.10 & \bf{857.17} & 
1.55\\CON8-1 & 754.51 & 10.60 & 
766.97 & 12.68 & \bf{740.85} & 
1.84\\CON8-2 & 714.28 & 13.96 & 
723.58 & 14.42 & \bf{712.89} & 
0.19\\CON8-3 & 812.54 & 14.47 & 
829.24 & 13.26 & \bf{811.07} & 
0.18\\CON8-4 & 804.24 & 15.01 & 
810.30 & 13.16 & \bf{772.25} & 
4.14\\CON8-5 & 761.40 & 14.48 & 
769.83 & 15.48 & \bf{754.88} & 
0.86\\CON8-6 & 686.23 & 15.87 & 
705.50 & 13.91 & \bf{678.92} & 
1.08\\CON8-7 & 816.74 & 13.46 & 
822.83 & 13.57 & \bf{811.96} & 
0.59\\CON8-8 & 780.16 & 11.24 & 
789.83 & 13.20 & \bf{767.53} & 
1.65\\CON8-9 & 814.37 & 14.13 & 
824.70 & 13.30 & \bf{809.00} & 
0.66\\[1ex]\hline
\end{tabular}
\label{table:nonlin}
\end{table} \clearpage
\begin{table}[ht]
\caption{Resultados de la ejecución de la metaheurística ILS, utilizando instancias de Dethloff con la configuración -n 150 -LS 30}
\centering
\small
\begin{tabular}{c c c c c c c}
\hline\hline
Instancia & Costo mínimo & Tiempo(seg.) & Costo promedio & Tiempo promedio(seg.) & Costo ILS & \%Gap \\ [0.5ex]
\hline
SCA3-0 & 636.06 & 19.36 & 
640.12 & 18.79 & \bf{635.62} & 
0.07\\SCA3-1 & \bf{697.84} & 20.43 & 
701.07 & 19.89 & 697.84 & 0.00\\
SCA3-2 & 661.13 & 17.85 & 
663.62 & 17.64 & \bf{659.34} & 
0.27\\SCA3-3 & \bf{680.04} & 20.48 & 
680.78 & 20.00 & 680.04 & 0.00\\
SCA3-4 & \bf{690.50} & 18.96 & 
690.50 & 19.32 & 690.50 & 0.00\\
SCA3-5 & 662.75 & 16.98 & 
666.59 & 18.67 & \bf{659.90} & 
0.43\\SCA3-6 & \bf{651.09} & 19.20 & 
652.66 & 18.46 & 651.09 & 0.00\\
SCA3-7 & 669.89 & 17.52 & 
670.78 & 18.67 & \bf{659.17} & 
1.63\\SCA3-8 & \bf{719.47} & 16.78 & 
720.19 & 17.68 & 719.47 & 0.00\\
SCA3-9 & \bf{681.00} & 17.83 & 
682.81 & 18.61 & 681.00 & 0.00\\
SCA8-0 & 994.31 & 17.30 & 
1004.92 & 16.36 & \bf{961.50} & 
3.41\\SCA8-1 & 1067.56 & 16.76 & 
1071.93 & 15.81 & \bf{1049.65} & 
1.71\\SCA8-2 & 1049.71 & 16.72 & 
1057.40 & 15.93 & \bf{1039.64} & 
0.97\\SCA8-3 & 1014.82 & 15.95 & 
1021.72 & 15.68 & \bf{983.34} & 
3.20\\SCA8-4 & 1085.44 & 18.06 & 
1095.05 & 15.56 & \bf{1065.49} & 
1.87\\SCA8-5 & 1059.00 & 13.21 & 
1066.62 & 14.79 & \bf{1027.08} & 
3.11\\SCA8-6 & 982.57 & 17.37 & 
990.22 & 16.93 & \bf{971.82} & 
1.11\\SCA8-7 & 1067.11 & 17.20 & 
1076.60 & 16.16 & \bf{1051.28} & 
1.51\\SCA8-8 & \bf{1071.18} & 17.07 & 
1087.16 & 15.29 & 1071.18 & 0.00\\
SCA8-9 & 1093.20 & 14.09 & 
1099.84 & 14.72 & \bf{1060.50} & 
3.08\\CON3-0 & 619.86 & 19.59 & 
626.51 & 19.01 & \bf{616.52} & 
0.54\\CON3-1 & 560.75 & 18.08 & 
560.75 & 19.68 & \bf{554.47} & 
1.13\\CON3-2 & 521.38 & 20.37 & 
523.50 & 19.35 & \bf{518.00} & 
0.65\\CON3-3 & \bf{591.19} & 18.52 & 
593.77 & 18.51 & 591.19 & 0.00\\
CON3-4 & 593.78 & 17.32 & 
599.81 & 18.89 & \bf{588.79} & 
0.85\\CON3-5 & 567.94 & 17.66 & 
571.02 & 18.69 & \bf{563.70} & 
0.75\\CON3-6 & 503.95 & 20.47 & 
506.97 & 19.19 & \bf{499.05} & 
0.98\\CON3-7 & 578.41 & 18.48 & 
585.59 & 18.19 & \bf{576.48} & 
0.33\\CON3-8 & \bf{523.05} & 20.30 & 
526.28 & 19.57 & 523.05 & 0.00\\
CON3-9 & 588.40 & 20.00 & 
589.28 & 19.77 & \bf{578.24} & 
1.76\\CON8-0 & 880.40 & 13.80 & 
892.60 & 15.90 & \bf{857.17} & 
2.71\\CON8-1 & 762.37 & 14.90 & 
768.60 & 15.47 & \bf{740.85} & 
2.90\\CON8-2 & 720.68 & 15.57 & 
726.54 & 16.90 & \bf{712.89} & 
1.09\\CON8-3 & 826.14 & 14.31 & 
831.06 & 17.50 & \bf{811.07} & 
1.86\\CON8-4 & 785.09 & 16.16 & 
796.70 & 16.12 & \bf{772.25} & 
1.66\\CON8-5 & 767.09 & 16.34 & 
769.65 & 17.10 & \bf{754.88} & 
1.62\\CON8-6 & 680.47 & 16.97 & 
695.57 & 17.41 & \bf{678.92} & 
0.23\\CON8-7 & 814.50 & 14.77 & 
823.95 & 15.46 & \bf{811.96} & 
0.31\\CON8-8 & 784.05 & 15.92 & 
788.10 & 17.25 & \bf{767.53} & 
2.15\\CON8-9 & 822.42 & 16.24 & 
826.51 & 16.54 & \bf{809.00} & 
1.66\\[1ex]\hline
\end{tabular}
\label{table:nonlin}
\end{table} \clearpage
\begin{table}[ht]
\caption{Resultados de la ejecución de la metaheurística ILS, utilizando instancias de Dethloff con la configuración -n 180 -LS 30}
\centering
\small
\begin{tabular}{c c c c c c c}
\hline\hline
Instancia & Costo mínimo & Tiempo(seg.) & Costo promedio & Tiempo promedio(seg.) & Costo ILS & \%Gap \\ [0.5ex]
\hline
SCA3-0 & 640.55 & 23.84 & 
640.55 & 22.01 & \bf{635.62} & 
0.78\\SCA3-1 & \bf{697.84} & 21.62 & 
699.68 & 22.77 & 697.84 & 0.00\\
SCA3-2 & \bf{659.34} & 26.00 & 
662.99 & 21.82 & 659.34 & 0.00\\
SCA3-3 & \bf{680.04} & 22.84 & 
681.43 & 23.75 & 680.04 & 0.00\\
SCA3-4 & \bf{690.50} & 22.40 & 
691.18 & 21.57 & 690.50 & 0.00\\
SCA3-5 & \bf{659.90} & 21.30 & 
668.12 & 21.92 & 659.90 & 0.00\\
SCA3-6 & 652.94 & 20.74 & 
653.77 & 20.41 & \bf{651.09} & 
0.28\\SCA3-7 & 666.15 & 25.59 & 
670.34 & 22.83 & \bf{659.17} & 
1.06\\SCA3-8 & \bf{719.47} & 21.92 & 
719.97 & 22.36 & 719.47 & 0.00\\
SCA3-9 & \bf{681.00} & 25.50 & 
683.36 & 23.20 & 681.00 & 0.00\\
SCA8-0 & 970.64 & 19.39 & 
1003.17 & 20.59 & \bf{961.50} & 
0.95\\SCA8-1 & 1086.58 & 17.56 & 
1092.26 & 18.27 & \bf{1049.65} & 
3.52\\SCA8-2 & 1065.43 & 19.51 & 
1069.25 & 18.78 & \bf{1039.64} & 
2.48\\SCA8-3 & 1016.28 & 20.60 & 
1021.25 & 19.76 & \bf{983.34} & 
3.35\\SCA8-4 & 1074.02 & 18.12 & 
1078.81 & 19.40 & \bf{1065.49} & 
0.80\\SCA8-5 & 1063.32 & 19.58 & 
1067.78 & 19.19 & \bf{1027.08} & 
3.53\\SCA8-6 & 976.74 & 19.36 & 
983.40 & 18.22 & \bf{971.82} & 
0.51\\SCA8-7 & 1078.42 & 19.27 & 
1080.77 & 20.36 & \bf{1051.28} & 
2.58\\SCA8-8 & \bf{1071.18} & 17.98 & 
1083.55 & 19.26 & 1071.18 & 0.00\\
SCA8-9 & 1082.56 & 18.90 & 
1093.97 & 16.90 & \bf{1060.50} & 
2.08\\CON3-0 & 632.57 & 19.98 & 
634.91 & 21.33 & \bf{616.52} & 
2.60\\CON3-1 & 557.38 & 22.78 & 
560.19 & 22.30 & \bf{554.47} & 
0.52\\CON3-2 & 521.38 & 22.18 & 
522.93 & 22.09 & \bf{518.00} & 
0.65\\CON3-3 & 591.20 & 20.96 & 
592.35 & 21.83 & \bf{591.19} & 
0.00\\CON3-4 & 593.78 & 22.41 & 
594.59 & 21.99 & \bf{588.79} & 
0.85\\CON3-5 & \bf{563.70} & 21.64 & 
567.97 & 22.98 & 563.70 & 0.00\\
CON3-6 & 502.16 & 23.29 & 
504.62 & 24.43 & \bf{499.05} & 
0.62\\CON3-7 & 578.41 & 20.60 & 
580.31 & 22.85 & \bf{576.48} & 
0.33\\CON3-8 & 523.68 & 22.59 & 
524.36 & 23.96 & \bf{523.05} & 
0.12\\CON3-9 & 588.40 & 22.17 & 
588.96 & 23.05 & \bf{578.24} & 
1.76\\CON8-0 & 883.36 & 19.34 & 
888.35 & 19.53 & \bf{857.17} & 
3.06\\CON8-1 & 754.26 & 18.40 & 
758.65 & 20.39 & \bf{740.85} & 
1.81\\CON8-2 & 725.12 & 25.39 & 
728.10 & 22.07 & \bf{712.89} & 
1.72\\CON8-3 & 832.18 & 21.06 & 
837.14 & 19.73 & \bf{811.07} & 
2.60\\CON8-4 & 786.85 & 19.27 & 
802.70 & 18.27 & \bf{772.25} & 
1.89\\CON8-5 & 763.83 & 18.34 & 
776.13 & 19.17 & \bf{754.88} & 
1.19\\CON8-6 & 692.81 & 20.96 & 
701.89 & 20.33 & \bf{678.92} & 
2.05\\CON8-7 & 814.86 & 20.75 & 
815.60 & 19.18 & \bf{811.96} & 
0.36\\CON8-8 & 783.68 & 19.75 & 
787.94 & 20.15 & \bf{767.53} & 
2.10\\CON8-9 & 815.02 & 18.37 & 
820.63 & 19.63 & \bf{809.00} & 
0.74\\[1ex]\hline
\end{tabular}
\label{table:nonlin}
\end{table} \clearpage
\begin{table}[ht]
\caption{Resultados de la ejecución de la metaheurística ILS, utilizando instancias de Dethloff con la configuración -n 210 -LS 30}
\centering
\small
\begin{tabular}{c c c c c c c}
\hline\hline
Instancia & Costo mínimo & Tiempo(seg.) & Costo promedio & Tiempo promedio(seg.) & Costo ILS & \%Gap \\ [0.5ex]
\hline
SCA3-0 & 636.06 & 27.38 & 
639.43 & 27.09 & \bf{635.62} & 
0.07\\SCA3-1 & \bf{697.84} & 26.94 & 
701.74 & 25.96 & 697.84 & 0.00\\
SCA3-2 & 661.13 & 24.61 & 
663.61 & 24.69 & \bf{659.34} & 
0.27\\SCA3-3 & \bf{680.04} & 25.02 & 
680.65 & 26.44 & 680.04 & 0.00\\
SCA3-4 & \bf{690.50} & 27.99 & 
690.50 & 25.75 & 690.50 & 0.00\\
SCA3-5 & 668.48 & 22.86 & 
670.48 & 26.27 & \bf{659.90} & 
1.30\\SCA3-6 & \bf{651.09} & 23.22 & 
652.97 & 24.99 & 651.09 & 0.00\\
SCA3-7 & 669.89 & 25.20 & 
671.25 & 26.43 & \bf{659.17} & 
1.63\\SCA3-8 & \bf{719.47} & 30.24 & 
721.73 & 29.46 & 719.47 & 0.00\\
SCA3-9 & \bf{681.00} & 22.74 & 
683.03 & 23.45 & 681.00 & 0.00\\
SCA8-0 & \bf{961.50} & 24.96 & 
974.18 & 23.54 & 961.50 & 0.00\\
SCA8-1 & 1053.57 & 22.20 & 
1071.90 & 21.79 & \bf{1049.65} & 
0.37\\SCA8-2 & 1057.81 & 21.70 & 
1062.81 & 21.38 & \bf{1039.64} & 
1.75\\SCA8-3 & 1013.02 & 23.39 & 
1019.74 & 22.00 & \bf{983.34} & 
3.02\\SCA8-4 & 1069.87 & 20.46 & 
1078.88 & 21.98 & \bf{1065.49} & 
0.41\\SCA8-5 & 1056.74 & 20.18 & 
1063.91 & 21.82 & \bf{1027.08} & 
2.89\\SCA8-6 & 981.41 & 23.76 & 
987.14 & 22.76 & \bf{971.82} & 
0.99\\SCA8-7 & 1068.07 & 22.12 & 
1072.83 & 23.59 & \bf{1051.28} & 
1.60\\SCA8-8 & \bf{1071.18} & 24.54 & 
1081.86 & 23.47 & 1071.18 & 0.00\\
SCA8-9 & 1072.60 & 22.16 & 
1082.12 & 20.67 & \bf{1060.50} & 
1.14\\CON3-0 & 619.09 & 25.36 & 
627.51 & 23.87 & \bf{616.52} & 
0.42\\CON3-1 & 556.92 & 31.11 & 
560.23 & 28.08 & \bf{554.47} & 
0.44\\CON3-2 & 521.38 & 27.25 & 
522.13 & 26.74 & \bf{518.00} & 
0.65\\CON3-3 & \bf{591.19} & 23.73 & 
593.21 & 26.84 & 591.19 & 0.00\\
CON3-4 & 591.43 & 24.28 & 
593.94 & 25.05 & \bf{588.79} & 
0.45\\CON3-5 & \bf{563.70} & 25.76 & 
569.20 & 26.14 & 563.70 & 0.00\\
CON3-6 & 500.80 & 24.94 & 
502.32 & 24.40 & \bf{499.05} & 
0.35\\CON3-7 & 578.41 & 23.97 & 
582.77 & 25.22 & \bf{576.48} & 
0.33\\CON3-8 & 524.59 & 27.98 & 
528.27 & 29.92 & \bf{523.05} & 
0.29\\CON3-9 & 587.23 & 27.22 & 
588.96 & 26.05 & \bf{578.24} & 
1.55\\CON8-0 & 861.87 & 25.35 & 
882.42 & 22.41 & \bf{857.17} & 
0.55\\CON8-1 & 754.51 & 22.77 & 
761.47 & 22.58 & \bf{740.85} & 
1.84\\CON8-2 & 717.31 & 23.87 & 
724.56 & 23.66 & \bf{712.89} & 
0.62\\CON8-3 & 825.06 & 22.44 & 
829.35 & 24.23 & \bf{811.07} & 
1.72\\CON8-4 & 780.03 & 22.68 & 
787.30 & 21.77 & \bf{772.25} & 
1.01\\CON8-5 & 766.55 & 23.13 & 
770.80 & 23.71 & \bf{754.88} & 
1.55\\CON8-6 & 700.95 & 24.16 & 
701.93 & 22.90 & \bf{678.92} & 
3.24\\CON8-7 & 814.50 & 23.88 & 
825.47 & 22.64 & \bf{811.96} & 
0.31\\CON8-8 & 783.22 & 24.35 & 
789.45 & 24.73 & \bf{767.53} & 
2.04\\CON8-9 & 813.94 & 22.16 & 
820.31 & 23.11 & \bf{809.00} & 
0.61\\[1ex]\hline
\end{tabular}
\label{table:nonlin}
\end{table} \clearpage
\begin{table}[ht]
\caption{Resultados de la ejecución de la metaheurística ILS, utilizando instancias de Dethloff con la configuración -n 5.0 -LS 10.0}
\centering
\small
\begin{tabular}{c c c c c c c}
\hline\hline
Instancia & Costo mínimo & Tiempo(seg.) & Costo promedio & Tiempo promedio(seg.) & Costo ILS & \%Gap \\ [0.5ex]
\hline
SCA3-0 & 640.55 & 0.30 & 
641.97 & 0.29 & \bf{635.62} & 
0.78\\SCA3-1 & 708.40 & 0.38 & 
722.56 & 0.33 & \bf{697.84} & 
1.51\\SCA3-2 & 674.29 & 0.29 & 
697.58 & 0.27 & \bf{659.34} & 
2.27\\SCA3-3 & 712.47 & 0.36 & 
717.69 & 0.32 & \bf{680.04} & 
4.77\\SCA3-4 & \bf{690.50} & 0.30 & 
697.63 & 0.29 & 690.50 & 0.00\\
SCA3-5 & 689.59 & 0.37 & 
701.64 & 0.30 & \bf{659.90} & 
4.50\\SCA3-6 & 696.72 & 0.38 & 
709.08 & 0.26 & \bf{651.09} & 
7.01\\SCA3-7 & 669.89 & 0.39 & 
676.31 & 0.31 & \bf{659.17} & 
1.63\\SCA3-8 & 745.33 & 0.28 & 
754.95 & 0.31 & \bf{719.47} & 
3.59\\SCA3-9 & 711.93 & 0.27 & 
719.45 & 0.32 & \bf{681.00} & 
4.54\\SCA8-0 & 1001.33 & 0.28 & 
1015.17 & 0.28 & \bf{961.50} & 
4.14\\SCA8-1 & 1080.67 & 0.34 & 
1112.55 & 0.27 & \bf{1049.65} & 
2.96\\SCA8-2 & 1100.07 & 0.24 & 
1103.89 & 0.26 & \bf{1039.64} & 
5.81\\SCA8-3 & 1044.10 & 0.36 & 
1053.97 & 0.32 & \bf{983.34} & 
6.18\\SCA8-4 & 1138.23 & 0.40 & 
1153.97 & 0.37 & \bf{1065.49} & 
6.83\\SCA8-5 & 1088.81 & 0.31 & 
1106.23 & 0.39 & \bf{1027.08} & 
6.01\\SCA8-6 & 1021.99 & 0.27 & 
1036.04 & 0.30 & \bf{971.82} & 
5.16\\SCA8-7 & 1114.68 & 0.31 & 
1127.69 & 0.27 & \bf{1051.28} & 
6.03\\SCA8-8 & \bf{1071.18} & 0.24 & 
1130.81 & 0.26 & 1071.18 & 0.00\\
SCA8-9 & 1114.34 & 0.26 & 
1133.72 & 0.25 & \bf{1060.50} & 
5.08\\CON3-0 & 650.16 & 0.31 & 
669.65 & 0.33 & \bf{616.52} & 
5.46\\CON3-1 & 568.45 & 0.38 & 
573.28 & 0.28 & \bf{554.47} & 
2.52\\CON3-2 & 521.63 & 0.39 & 
540.90 & 0.38 & \bf{518.00} & 
0.70\\CON3-3 & 611.33 & 0.30 & 
635.93 & 0.27 & \bf{591.19} & 
3.41\\CON3-4 & 611.21 & 0.33 & 
628.71 & 0.33 & \bf{588.79} & 
3.81\\CON3-5 & 569.04 & 0.31 & 
572.26 & 0.27 & \bf{563.70} & 
0.95\\CON3-6 & 511.00 & 0.30 & 
529.03 & 0.39 & \bf{499.05} & 
2.39\\CON3-7 & 618.48 & 0.36 & 
630.72 & 0.30 & \bf{576.48} & 
7.29\\CON3-8 & 546.26 & 0.46 & 
566.07 & 0.39 & \bf{523.05} & 
4.44\\CON3-9 & 599.04 & 0.31 & 
599.84 & 0.30 & \bf{578.24} & 
3.60\\CON8-0 & 919.96 & 0.23 & 
934.16 & 0.23 & \bf{857.17} & 
7.33\\CON8-1 & 777.08 & 0.47 & 
815.24 & 0.37 & \bf{740.85} & 
4.89\\CON8-2 & 776.70 & 0.26 & 
787.67 & 0.26 & \bf{712.89} & 
8.95\\CON8-3 & 857.98 & 0.24 & 
858.49 & 0.27 & \bf{811.07} & 
5.78\\CON8-4 & 825.32 & 0.44 & 
848.55 & 0.36 & \bf{772.25} & 
6.87\\CON8-5 & 825.97 & 0.26 & 
832.06 & 0.33 & \bf{754.88} & 
9.42\\CON8-6 & 724.97 & 0.29 & 
741.21 & 0.27 & \bf{678.92} & 
6.78\\CON8-7 & 839.18 & 0.24 & 
868.98 & 0.34 & \bf{811.96} & 
3.35\\CON8-8 & 794.22 & 0.28 & 
809.92 & 0.29 & \bf{767.53} & 
3.48\\CON8-9 & 825.74 & 0.28 & 
851.28 & 0.40 & \bf{809.00} & 
2.07\\[1ex]\hline
\end{tabular}
\label{table:nonlin}
\end{table} \clearpage
\begin{table}[ht]
\caption{Resultados de la ejecución de la metaheurística ILS, utilizando instancias de SalhiNagy con la configuración -n 5.0 -LS 10.0}
\centering
\small
\begin{tabular}{c c c c c c c}
\hline\hline
Instancia & Costo mínimo & Tiempo(seg.) & Costo promedio & Tiempo promedio(seg.) & Costo ILS & \%Gap \\ [0.5ex]
\hline
CMT1X & 484.42 & 0.32 & 
490.96 & 0.30 & \bf{466.77} & 
3.78\\CMT1Y & 510.07 & 0.18 & 
516.21 & 0.22 & \bf{466.77} & 
9.28\\CMT2X & 718.33 & 1.06 & 
730.50 & 0.95 & \bf{684.21} & 
4.99\\CMT2Y & 718.80 & 0.91 & 
724.77 & 0.80 & \bf{684.21} & 
5.06\\CMT3X & 740.20 & 2.08 & 
747.31 & 2.30 & \bf{721.40} & 
2.61\\CMT3Y & 746.93 & 1.90 & 
753.51 & 2.12 & \bf{721.40} & 
3.54\\CMT4X & 925.51 & 5.44 & 
930.82 & 7.30 & \bf{852.83} & 
8.52\\CMT4Y & 914.77 & 6.05 & 
927.38 & 7.01 & \bf{852.46} & 
7.31\\CMT5X & 1105.30 & 23.41 & 
1140.11 & 16.27 & \bf{1030.55} & 
7.25\\CMT5Y & 1112.66 & 15.98 & 
1129.79 & 19.07 & \bf{1031.17} & 
7.90\\CMT11X & 894.90 & 4.28 & 
924.56 & 4.77 & \bf{839.39} & 
6.61\\CMT11Y & 907.63 & 5.65 & 
944.20 & 5.66 & \bf{841.88} & 
7.81\\CMT12X & 689.21 & 1.89 & 
708.75 & 1.97 & \bf{662.22} & 
4.08\\CMT12Y & 690.47 & 1.72 & 
697.54 & 1.91 & \bf{662.22} & 
4.27\\[1ex]\hline
\end{tabular}
\label{table:nonlin}
\end{table} \clearpage
\begin{table}[ht]
\caption{Resultados de la ejecución de la metaheurística ILS, utilizando instancias de Dethloff con la configuración -n 5.0 -LS 20.0}
\centering
\small
\begin{tabular}{c c c c c c c}
\hline\hline
Instancia & Costo mínimo & Tiempo(seg.) & Costo promedio & Tiempo promedio(seg.) & Costo ILS & \%Gap \\ [0.5ex]
\hline
SCA3-0 & 642.44 & 0.50 & 
642.88 & 0.49 & \bf{635.62} & 
1.07\\SCA3-1 & 714.92 & 0.53 & 
725.23 & 0.40 & \bf{697.84} & 
2.45\\SCA3-2 & 682.38 & 0.57 & 
697.93 & 0.55 & \bf{659.34} & 
3.49\\SCA3-3 & 688.71 & 0.43 & 
702.78 & 0.47 & \bf{680.04} & 
1.27\\SCA3-4 & 698.98 & 0.39 & 
725.90 & 0.38 & \bf{690.50} & 
1.23\\SCA3-5 & 686.47 & 0.58 & 
691.48 & 0.52 & \bf{659.90} & 
4.03\\SCA3-6 & 662.50 & 0.68 & 
694.59 & 0.43 & \bf{651.09} & 
1.75\\SCA3-7 & 671.77 & 0.33 & 
673.92 & 0.36 & \bf{659.17} & 
1.91\\SCA3-8 & \bf{719.47} & 0.50 & 
747.48 & 0.52 & 719.47 & 0.00\\
SCA3-9 & 700.27 & 0.39 & 
702.97 & 0.51 & \bf{681.00} & 
2.83\\SCA8-0 & 977.93 & 0.60 & 
1001.69 & 0.48 & \bf{961.50} & 
1.71\\SCA8-1 & 1071.03 & 0.56 & 
1103.83 & 0.45 & \bf{1049.65} & 
2.04\\SCA8-2 & 1064.69 & 0.42 & 
1064.93 & 0.41 & \bf{1039.64} & 
2.41\\SCA8-3 & 1043.01 & 0.46 & 
1061.13 & 0.46 & \bf{983.34} & 
6.07\\SCA8-4 & 1126.94 & 0.40 & 
1132.64 & 0.37 & \bf{1065.49} & 
5.77\\SCA8-5 & 1087.33 & 0.37 & 
1114.02 & 0.39 & \bf{1027.08} & 
5.87\\SCA8-6 & 1013.77 & 0.38 & 
1022.63 & 0.34 & \bf{971.82} & 
4.32\\SCA8-7 & 1077.46 & 0.38 & 
1141.27 & 0.41 & \bf{1051.28} & 
2.49\\SCA8-8 & 1095.65 & 0.57 & 
1101.53 & 0.41 & \bf{1071.18} & 
2.28\\SCA8-9 & 1114.21 & 0.38 & 
1120.16 & 0.35 & \bf{1060.50} & 
5.06\\CON3-0 & 632.57 & 0.62 & 
641.52 & 0.60 & \bf{616.52} & 
2.60\\CON3-1 & 580.49 & 0.44 & 
584.53 & 0.51 & \bf{554.47} & 
4.69\\CON3-2 & 541.17 & 0.38 & 
556.51 & 0.39 & \bf{518.00} & 
4.47\\CON3-3 & 603.99 & 0.49 & 
617.10 & 0.48 & \bf{591.19} & 
2.17\\CON3-4 & 605.94 & 0.42 & 
618.10 & 0.40 & \bf{588.79} & 
2.91\\CON3-5 & 600.88 & 0.46 & 
603.70 & 0.38 & \bf{563.70} & 
6.60\\CON3-6 & 504.15 & 0.54 & 
508.24 & 0.53 & \bf{499.05} & 
1.02\\CON3-7 & 606.84 & 0.56 & 
612.35 & 0.59 & \bf{576.48} & 
5.27\\CON3-8 & 535.32 & 0.39 & 
542.68 & 0.43 & \bf{523.05} & 
2.35\\CON3-9 & 589.85 & 0.65 & 
601.67 & 0.52 & \bf{578.24} & 
2.01\\CON8-0 & 968.67 & 0.34 & 
983.30 & 0.40 & \bf{857.17} & 
13.01\\CON8-1 & 773.56 & 1.08 & 
795.03 & 0.66 & \bf{740.85} & 
4.42\\CON8-2 & 768.27 & 0.50 & 
787.46 & 0.47 & \bf{712.89} & 
7.77\\CON8-3 & 824.90 & 0.40 & 
851.70 & 0.45 & \bf{811.07} & 
1.71\\CON8-4 & 808.96 & 0.33 & 
832.77 & 0.34 & \bf{772.25} & 
4.75\\CON8-5 & 788.25 & 0.46 & 
858.33 & 0.38 & \bf{754.88} & 
4.42\\CON8-6 & 723.09 & 0.52 & 
726.62 & 0.49 & \bf{678.92} & 
6.51\\CON8-7 & 857.41 & 0.32 & 
860.32 & 0.43 & \bf{811.96} & 
5.60\\CON8-8 & 799.04 & 0.35 & 
799.36 & 0.44 & \bf{767.53} & 
4.11\\CON8-9 & 851.34 & 0.67 & 
886.78 & 0.44 & \bf{809.00} & 
5.23\\[1ex]\hline
\end{tabular}
\label{table:nonlin}
\end{table} \clearpage
\begin{table}[ht]
\caption{Resultados de la ejecución de la metaheurística ILS, utilizando instancias de SalhiNagy con la configuración -n 5.0 -LS 20.0}
\centering
\small
\begin{tabular}{c c c c c c c}
\hline\hline
Instancia & Costo mínimo & Tiempo(seg.) & Costo promedio & Tiempo promedio(seg.) & Costo ILS & \%Gap \\ [0.5ex]
\hline
CMT1X & 485.80 & 0.43 & 
491.84 & 0.38 & \bf{466.77} & 
4.08\\CMT1Y & 503.24 & 0.58 & 
510.12 & 0.47 & \bf{466.77} & 
7.81\\CMT2X & 723.80 & 1.26 & 
743.86 & 1.13 & \bf{684.21} & 
5.79\\CMT2Y & 723.43 & 0.97 & 
742.33 & 0.88 & \bf{684.21} & 
5.73\\CMT3X & 747.26 & 2.68 & 
758.12 & 2.44 & \bf{721.40} & 
3.58\\CMT3Y & 737.36 & 3.39 & 
750.76 & 2.87 & \bf{721.40} & 
2.21\\CMT4X & 916.49 & 8.04 & 
931.40 & 8.40 & \bf{852.83} & 
7.46\\CMT4Y & 896.33 & 10.86 & 
913.93 & 8.70 & \bf{852.46} & 
5.15\\CMT5X & 1116.06 & 26.42 & 
1133.48 & 19.41 & \bf{1030.55} & 
8.30\\CMT5Y & 1128.26 & 23.79 & 
1141.40 & 18.84 & \bf{1031.17} & 
9.42\\CMT11X & 873.94 & 6.71 & 
921.84 & 6.45 & \bf{839.39} & 
4.12\\CMT11Y & 896.05 & 5.84 & 
931.10 & 5.58 & \bf{841.88} & 
6.43\\CMT12X & 695.13 & 2.80 & 
710.98 & 2.36 & \bf{662.22} & 
4.97\\CMT12Y & 697.79 & 2.64 & 
703.93 & 2.28 & \bf{662.22} & 
5.37\\[1ex]\hline
\end{tabular}
\label{table:nonlin}
\end{table} \clearpage
\begin{table}[ht]
\caption{Resultados de la ejecución de la metaheurística ILS, utilizando instancias de Dethloff con la configuración -n 5.0 -LS 30.0}
\centering
\small
\begin{tabular}{c c c c c c c}
\hline\hline
Instancia & Costo mínimo & Tiempo(seg.) & Costo promedio & Tiempo promedio(seg.) & Costo ILS & \%Gap \\ [0.5ex]
\hline
SCA3-0 & 640.55 & 0.72 & 
652.82 & 0.67 & \bf{635.62} & 
0.78\\SCA3-1 & 710.89 & 0.78 & 
719.29 & 0.64 & \bf{697.84} & 
1.87\\SCA3-2 & 666.72 & 0.59 & 
680.29 & 0.65 & \bf{659.34} & 
1.12\\SCA3-3 & 681.74 & 0.69 & 
693.16 & 0.61 & \bf{680.04} & 
0.25\\SCA3-4 & \bf{690.50} & 0.54 & 
701.02 & 0.62 & 690.50 & 0.00\\
SCA3-5 & 687.15 & 0.82 & 
701.75 & 0.66 & \bf{659.90} & 
4.13\\SCA3-6 & 654.26 & 0.72 & 
664.74 & 0.61 & \bf{651.09} & 
0.49\\SCA3-7 & 671.77 & 0.60 & 
677.45 & 0.59 & \bf{659.17} & 
1.91\\SCA3-8 & 719.77 & 0.70 & 
742.35 & 0.70 & \bf{719.47} & 
0.04\\SCA3-9 & 684.44 & 0.40 & 
684.72 & 0.48 & \bf{681.00} & 
0.51\\SCA8-0 & 1023.20 & 0.43 & 
1042.98 & 0.54 & \bf{961.50} & 
6.42\\SCA8-1 & 1090.41 & 0.47 & 
1114.91 & 0.55 & \bf{1049.65} & 
3.88\\SCA8-2 & 1072.45 & 0.69 & 
1119.59 & 0.55 & \bf{1039.64} & 
3.16\\SCA8-3 & 1062.82 & 0.38 & 
1073.51 & 0.46 & \bf{983.34} & 
8.08\\SCA8-4 & 1127.21 & 0.58 & 
1142.55 & 0.55 & \bf{1065.49} & 
5.79\\SCA8-5 & 1083.64 & 0.53 & 
1093.15 & 0.56 & \bf{1027.08} & 
5.51\\SCA8-6 & 1010.96 & 0.66 & 
1040.77 & 0.56 & \bf{971.82} & 
4.03\\SCA8-7 & 1098.54 & 0.43 & 
1116.15 & 0.56 & \bf{1051.28} & 
4.50\\SCA8-8 & 1094.03 & 0.72 & 
1133.61 & 0.68 & \bf{1071.18} & 
2.13\\SCA8-9 & 1105.19 & 0.39 & 
1119.37 & 0.42 & \bf{1060.50} & 
4.21\\CON3-0 & 642.11 & 0.61 & 
652.86 & 0.56 & \bf{616.52} & 
4.15\\CON3-1 & 567.39 & 0.65 & 
575.55 & 0.70 & \bf{554.47} & 
2.33\\CON3-2 & 521.63 & 0.65 & 
537.22 & 0.62 & \bf{518.00} & 
0.70\\CON3-3 & 602.16 & 0.80 & 
605.35 & 0.70 & \bf{591.19} & 
1.86\\CON3-4 & 598.21 & 0.61 & 
623.41 & 0.62 & \bf{588.79} & 
1.60\\CON3-5 & 568.69 & 0.98 & 
573.68 & 0.76 & \bf{563.70} & 
0.89\\CON3-6 & 512.74 & 0.60 & 
519.69 & 0.68 & \bf{499.05} & 
2.74\\CON3-7 & 586.01 & 0.74 & 
593.28 & 0.62 & \bf{576.48} & 
1.65\\CON3-8 & 530.94 & 1.01 & 
538.07 & 0.87 & \bf{523.05} & 
1.51\\CON3-9 & 592.60 & 0.62 & 
594.89 & 0.58 & \bf{578.24} & 
2.48\\CON8-0 & 913.70 & 0.84 & 
920.63 & 0.60 & \bf{857.17} & 
6.59\\CON8-1 & 770.15 & 0.45 & 
804.15 & 0.51 & \bf{740.85} & 
3.95\\CON8-2 & 746.90 & 0.51 & 
752.59 & 0.63 & \bf{712.89} & 
4.77\\CON8-3 & 846.19 & 1.09 & 
859.86 & 0.80 & \bf{811.07} & 
4.33\\CON8-4 & 807.88 & 0.67 & 
824.64 & 0.61 & \bf{772.25} & 
4.61\\CON8-5 & 777.34 & 0.84 & 
803.30 & 0.71 & \bf{754.88} & 
2.98\\CON8-6 & 727.18 & 0.43 & 
739.30 & 0.46 & \bf{678.92} & 
7.11\\CON8-7 & 860.80 & 0.63 & 
879.27 & 0.68 & \bf{811.96} & 
6.02\\CON8-8 & 779.43 & 0.47 & 
847.53 & 0.49 & \bf{767.53} & 
1.55\\CON8-9 & 833.37 & 1.11 & 
848.06 & 0.61 & \bf{809.00} & 
3.01\\[1ex]\hline
\end{tabular}
\label{table:nonlin}
\end{table} \clearpage
\begin{table}[ht]
\caption{Resultados de la ejecución de la metaheurística ILS, utilizando instancias de SalhiNagy con la configuración -n 5.0 -LS 30.0}
\centering
\small
\begin{tabular}{c c c c c c c}
\hline\hline
Instancia & Costo mínimo & Tiempo(seg.) & Costo promedio & Tiempo promedio(seg.) & Costo ILS & \%Gap \\ [0.5ex]
\hline
CMT1X & 497.31 & 0.38 & 
502.64 & 0.48 & \bf{466.77} & 
6.54\\CMT1Y & 481.74 & 0.42 & 
482.20 & 0.43 & \bf{466.77} & 
3.21\\CMT2X & 720.25 & 1.13 & 
728.34 & 1.22 & \bf{684.21} & 
5.27\\CMT2Y & 714.85 & 1.78 & 
719.85 & 1.47 & \bf{684.21} & 
4.48\\CMT3X & 755.44 & 2.71 & 
759.68 & 3.17 & \bf{721.40} & 
4.72\\CMT3Y & 738.34 & 3.85 & 
749.65 & 3.68 & \bf{721.40} & 
2.35\\CMT4X & 901.20 & 8.76 & 
929.04 & 9.96 & \bf{852.83} & 
5.67\\CMT4Y & 917.78 & 9.12 & 
938.28 & 8.74 & \bf{852.46} & 
7.66\\CMT5X & 1121.84 & 17.65 & 
1137.85 & 20.78 & \bf{1030.55} & 
8.86\\CMT5Y & 1111.86 & 30.30 & 
1129.17 & 23.96 & \bf{1031.17} & 
7.83\\CMT11X & 902.95 & 8.60 & 
932.91 & 7.99 & \bf{839.39} & 
7.57\\CMT11Y & 927.82 & 7.57 & 
949.87 & 8.16 & \bf{841.88} & 
10.21\\CMT12X & 691.07 & 2.85 & 
698.11 & 2.72 & \bf{662.22} & 
4.36\\CMT12Y & 678.58 & 3.19 & 
691.46 & 3.31 & \bf{662.22} & 
2.47\\[1ex]\hline
\end{tabular}
\label{table:nonlin}
\end{table} \clearpage
\begin{table}[ht]
\caption{Resultados de la ejecución de la metaheurística ILS, utilizando instancias de Dethloff con la configuración -n 5.0 -LS 40.0}
\centering
\small
\begin{tabular}{c c c c c c c}
\hline\hline
Instancia & Costo mínimo & Tiempo(seg.) & Costo promedio & Tiempo promedio(seg.) & Costo ILS & \%Gap \\ [0.5ex]
\hline
SCA3-0 & 640.55 & 1.25 & 
643.75 & 1.08 & \bf{635.62} & 
0.78\\SCA3-1 & 712.59 & 1.06 & 
729.38 & 0.87 & \bf{697.84} & 
2.11\\SCA3-2 & 680.00 & 0.84 & 
700.79 & 0.72 & \bf{659.34} & 
3.13\\SCA3-3 & 685.47 & 1.06 & 
687.22 & 0.94 & \bf{680.04} & 
0.80\\SCA3-4 & \bf{690.50} & 0.93 & 
706.32 & 0.84 & 690.50 & 0.00\\
SCA3-5 & 686.44 & 0.52 & 
686.53 & 0.78 & \bf{659.90} & 
4.02\\SCA3-6 & 654.79 & 0.91 & 
666.88 & 0.73 & \bf{651.09} & 
0.57\\SCA3-7 & 669.89 & 0.68 & 
672.23 & 0.76 & \bf{659.17} & 
1.63\\SCA3-8 & 726.44 & 0.78 & 
750.00 & 0.79 & \bf{719.47} & 
0.97\\SCA3-9 & 683.57 & 1.22 & 
699.63 & 0.96 & \bf{681.00} & 
0.38\\SCA8-0 & 1018.78 & 0.68 & 
1023.67 & 0.63 & \bf{961.50} & 
5.96\\SCA8-1 & 1104.59 & 0.61 & 
1141.33 & 0.62 & \bf{1049.65} & 
5.23\\SCA8-2 & 1066.96 & 0.79 & 
1105.54 & 0.57 & \bf{1039.64} & 
2.63\\SCA8-3 & 1049.88 & 0.80 & 
1087.59 & 0.66 & \bf{983.34} & 
6.77\\SCA8-4 & 1109.30 & 0.73 & 
1119.46 & 0.79 & \bf{1065.49} & 
4.11\\SCA8-5 & 1064.55 & 0.72 & 
1068.26 & 0.89 & \bf{1027.08} & 
3.65\\SCA8-6 & 1026.04 & 0.64 & 
1027.85 & 0.65 & \bf{971.82} & 
5.58\\SCA8-7 & 1110.98 & 1.02 & 
1138.58 & 0.74 & \bf{1051.28} & 
5.68\\SCA8-8 & 1103.10 & 0.53 & 
1139.92 & 0.53 & \bf{1071.18} & 
2.98\\SCA8-9 & 1084.48 & 0.74 & 
1137.56 & 0.60 & \bf{1060.50} & 
2.26\\CON3-0 & 661.33 & 0.72 & 
665.72 & 0.81 & \bf{616.52} & 
7.27\\CON3-1 & 561.63 & 1.04 & 
575.74 & 0.86 & \bf{554.47} & 
1.29\\CON3-2 & 523.99 & 0.87 & 
526.53 & 0.76 & \bf{518.00} & 
1.16\\CON3-3 & 591.20 & 1.18 & 
614.17 & 0.94 & \bf{591.19} & 
0.00\\CON3-4 & 620.15 & 0.79 & 
636.84 & 0.77 & \bf{588.79} & 
5.33\\CON3-5 & 571.63 & 0.83 & 
589.09 & 0.86 & \bf{563.70} & 
1.41\\CON3-6 & 513.13 & 0.85 & 
519.79 & 0.94 & \bf{499.05} & 
2.82\\CON3-7 & 604.95 & 1.23 & 
616.76 & 0.89 & \bf{576.48} & 
4.94\\CON3-8 & 529.65 & 0.88 & 
541.35 & 0.90 & \bf{523.05} & 
1.26\\CON3-9 & 590.64 & 1.01 & 
596.36 & 0.93 & \bf{578.24} & 
2.14\\CON8-0 & 923.06 & 0.68 & 
956.84 & 0.64 & \bf{857.17} & 
7.69\\CON8-1 & 777.60 & 1.00 & 
789.87 & 0.88 & \bf{740.85} & 
4.96\\CON8-2 & 729.26 & 1.09 & 
746.58 & 0.81 & \bf{712.89} & 
2.30\\CON8-3 & 851.97 & 0.99 & 
863.30 & 0.69 & \bf{811.07} & 
5.04\\CON8-4 & 824.36 & 0.71 & 
868.05 & 0.61 & \bf{772.25} & 
6.75\\CON8-5 & 768.63 & 0.55 & 
788.72 & 0.61 & \bf{754.88} & 
1.82\\CON8-6 & 699.30 & 0.63 & 
714.86 & 0.63 & \bf{678.92} & 
3.00\\CON8-7 & 857.87 & 0.61 & 
882.75 & 0.57 & \bf{811.96} & 
5.65\\CON8-8 & 800.37 & 0.64 & 
826.65 & 0.63 & \bf{767.53} & 
4.28\\CON8-9 & 851.07 & 0.72 & 
865.30 & 0.67 & \bf{809.00} & 
5.20\\[1ex]\hline
\end{tabular}
\label{table:nonlin}
\end{table} \clearpage
\begin{table}[ht]
\caption{Resultados de la ejecución de la metaheurística ILS, utilizando instancias de SalhiNagy con la configuración -n 5.0 -LS 40.0}
\centering
\small
\begin{tabular}{c c c c c c c}
\hline\hline
Instancia & Costo mínimo & Tiempo(seg.) & Costo promedio & Tiempo promedio(seg.) & Costo ILS & \%Gap \\ [0.5ex]
\hline
CMT1X & 479.75 & 0.42 & 
492.08 & 0.79 & \bf{466.77} & 
2.78\\CMT1Y & 478.23 & 0.78 & 
485.48 & 0.85 & \bf{466.77} & 
2.46\\CMT2X & 729.34 & 1.54 & 
747.66 & 1.50 & \bf{684.21} & 
6.60\\CMT2Y & 710.00 & 1.24 & 
719.57 & 1.47 & \bf{684.21} & 
3.77\\CMT3X & 743.31 & 4.07 & 
754.11 & 3.97 & \bf{721.40} & 
3.04\\CMT3Y & 746.77 & 3.76 & 
758.77 & 3.65 & \bf{721.40} & 
3.52\\CMT4X & 906.11 & 11.64 & 
922.71 & 12.06 & \bf{852.83} & 
6.25\\CMT4Y & 916.65 & 11.62 & 
928.42 & 10.91 & \bf{852.46} & 
7.53\\CMT5X & 1117.42 & 18.10 & 
1138.58 & 21.73 & \bf{1030.55} & 
8.43\\CMT5Y & 1106.24 & 20.25 & 
1113.81 & 22.92 & \bf{1031.17} & 
7.28\\CMT11X & 921.82 & 11.32 & 
934.40 & 9.31 & \bf{839.39} & 
9.82\\CMT11Y & 896.89 & 7.55 & 
951.83 & 7.92 & \bf{841.88} & 
6.53\\CMT12X & 678.73 & 3.55 & 
705.51 & 3.23 & \bf{662.22} & 
2.49\\CMT12Y & 696.28 & 3.99 & 
705.68 & 3.50 & \bf{662.22} & 
5.14\\[1ex]\hline
\end{tabular}
\label{table:nonlin}
\end{table} \clearpage
\begin{table}[ht]
\caption{Resultados de la ejecución de la metaheurística ILS, utilizando instancias de Dethloff con la configuración -n 5.0 -LS 50.0}
\centering
\small
\begin{tabular}{c c c c c c c}
\hline\hline
Instancia & Costo mínimo & Tiempo(seg.) & Costo promedio & Tiempo promedio(seg.) & Costo ILS & \%Gap \\ [0.5ex]
\hline
SCA3-0 & 640.55 & 1.17 & 
652.70 & 0.99 & \bf{635.62} & 
0.78\\SCA3-1 & 707.56 & 1.04 & 
710.52 & 0.96 & \bf{697.84} & 
1.39\\SCA3-2 & 668.65 & 1.20 & 
672.40 & 1.17 & \bf{659.34} & 
1.41\\SCA3-3 & 680.60 & 1.00 & 
689.88 & 0.92 & \bf{680.04} & 
0.08\\SCA3-4 & \bf{690.50} & 1.04 & 
697.92 & 1.15 & 690.50 & 0.00\\
SCA3-5 & 667.27 & 1.09 & 
694.59 & 1.05 & \bf{659.90} & 
1.12\\SCA3-6 & 652.94 & 0.89 & 
664.62 & 0.97 & \bf{651.09} & 
0.28\\SCA3-7 & 676.07 & 0.77 & 
685.58 & 0.81 & \bf{659.17} & 
2.56\\SCA3-8 & 729.48 & 0.73 & 
741.45 & 1.01 & \bf{719.47} & 
1.39\\SCA3-9 & 697.46 & 0.96 & 
703.26 & 0.96 & \bf{681.00} & 
2.42\\SCA8-0 & 1030.81 & 0.92 & 
1037.78 & 0.80 & \bf{961.50} & 
7.21\\SCA8-1 & 1082.41 & 0.90 & 
1106.43 & 0.82 & \bf{1049.65} & 
3.12\\SCA8-2 & 1071.89 & 1.00 & 
1093.26 & 0.84 & \bf{1039.64} & 
3.10\\SCA8-3 & 1037.53 & 0.64 & 
1046.89 & 0.79 & \bf{983.34} & 
5.51\\SCA8-4 & 1140.61 & 0.74 & 
1152.72 & 0.81 & \bf{1065.49} & 
7.05\\SCA8-5 & 1066.35 & 0.54 & 
1069.58 & 1.02 & \bf{1027.08} & 
3.82\\SCA8-6 & 1018.39 & 0.66 & 
1022.25 & 0.67 & \bf{971.82} & 
4.79\\SCA8-7 & 1096.59 & 0.85 & 
1123.87 & 0.76 & \bf{1051.28} & 
4.31\\SCA8-8 & 1089.91 & 0.56 & 
1104.27 & 0.70 & \bf{1071.18} & 
1.75\\SCA8-9 & 1099.00 & 0.76 & 
1118.58 & 0.80 & \bf{1060.50} & 
3.63\\CON3-0 & 632.57 & 1.22 & 
649.50 & 0.92 & \bf{616.52} & 
2.60\\CON3-1 & 560.75 & 0.97 & 
575.08 & 1.01 & \bf{554.47} & 
1.13\\CON3-2 & 523.93 & 0.80 & 
527.13 & 0.88 & \bf{518.00} & 
1.14\\CON3-3 & 616.13 & 1.02 & 
622.58 & 0.92 & \bf{591.19} & 
4.22\\CON3-4 & 621.06 & 0.96 & 
626.60 & 0.84 & \bf{588.79} & 
5.48\\CON3-5 & 579.41 & 0.72 & 
604.19 & 0.82 & \bf{563.70} & 
2.79\\CON3-6 & 505.41 & 1.17 & 
510.24 & 1.11 & \bf{499.05} & 
1.27\\CON3-7 & 602.08 & 1.08 & 
606.41 & 0.96 & \bf{576.48} & 
4.44\\CON3-8 & 539.17 & 1.02 & 
556.06 & 1.22 & \bf{523.05} & 
3.08\\CON3-9 & 588.99 & 1.08 & 
591.81 & 1.04 & \bf{578.24} & 
1.86\\CON8-0 & 927.09 & 1.17 & 
944.22 & 1.09 & \bf{857.17} & 
8.16\\CON8-1 & 767.10 & 1.03 & 
792.93 & 0.82 & \bf{740.85} & 
3.54\\CON8-2 & 737.95 & 0.82 & 
764.72 & 0.83 & \bf{712.89} & 
3.52\\CON8-3 & 850.89 & 0.73 & 
854.42 & 0.71 & \bf{811.07} & 
4.91\\CON8-4 & 812.88 & 0.78 & 
834.50 & 0.86 & \bf{772.25} & 
5.26\\CON8-5 & 767.03 & 0.78 & 
812.05 & 0.81 & \bf{754.88} & 
1.61\\CON8-6 & 707.65 & 0.82 & 
720.34 & 0.84 & \bf{678.92} & 
4.23\\CON8-7 & 816.26 & 0.62 & 
832.19 & 0.85 & \bf{811.96} & 
0.53\\CON8-8 & 791.03 & 0.74 & 
810.57 & 0.95 & \bf{767.53} & 
3.06\\CON8-9 & 853.51 & 1.04 & 
880.72 & 0.76 & \bf{809.00} & 
5.50\\[1ex]\hline
\end{tabular}
\label{table:nonlin}
\end{table} \clearpage
\begin{table}[ht]
\caption{Resultados de la ejecución de la metaheurística ILS, utilizando instancias de SalhiNagy con la configuración -n 5.0 -LS 50.0}
\centering
\small
\begin{tabular}{c c c c c c c}
\hline\hline
Instancia & Costo mínimo & Tiempo(seg.) & Costo promedio & Tiempo promedio(seg.) & Costo ILS & \%Gap \\ [0.5ex]
\hline
CMT1X & 492.44 & 1.05 & 
503.56 & 0.80 & \bf{466.77} & 
5.50\\CMT1Y & 484.06 & 0.93 & 
488.46 & 0.83 & \bf{466.77} & 
3.70\\CMT2X & 720.08 & 2.09 & 
726.70 & 1.96 & \bf{684.21} & 
5.24\\CMT2Y & 713.53 & 1.84 & 
725.79 & 1.85 & \bf{684.21} & 
4.29\\CMT3X & 734.07 & 4.62 & 
739.99 & 4.56 & \bf{721.40} & 
1.76\\CMT3Y & 739.72 & 5.01 & 
752.96 & 4.51 & \bf{721.40} & 
2.54\\CMT4X & 902.15 & 10.86 & 
933.16 & 10.86 & \bf{852.83} & 
5.78\\CMT4Y & 920.34 & 13.61 & 
934.35 & 12.62 & \bf{852.46} & 
7.96\\CMT5X & 1080.66 & 24.03 & 
1127.05 & 23.43 & \bf{1030.55} & 
4.86\\CMT5Y & 1112.18 & 28.95 & 
1135.44 & 25.26 & \bf{1031.17} & 
7.86\\CMT11X & 858.65 & 7.82 & 
914.60 & 10.10 & \bf{839.39} & 
2.29\\CMT11Y & 899.20 & 7.83 & 
924.64 & 10.22 & \bf{841.88} & 
6.81\\CMT12X & 677.87 & 3.74 & 
690.03 & 3.82 & \bf{662.22} & 
2.36\\CMT12Y & 691.95 & 5.57 & 
695.88 & 4.92 & \bf{662.22} & 
4.49\\[1ex]\hline
\end{tabular}
\label{table:nonlin}
\end{table} \clearpage
\begin{table}[ht]
\caption{Resultados de la ejecución de la metaheurística ILS, utilizando instancias de Dethloff con la configuración -n 5.0 -LS 60.0}
\centering
\small
\begin{tabular}{c c c c c c c}
\hline\hline
Instancia & Costo mínimo & Tiempo(seg.) & Costo promedio & Tiempo promedio(seg.) & Costo ILS & \%Gap \\ [0.5ex]
\hline
SCA3-0 & 641.69 & 1.13 & 
647.58 & 1.20 & \bf{635.62} & 
0.95\\SCA3-1 & 706.90 & 1.49 & 
716.36 & 1.23 & \bf{697.84} & 
1.30\\SCA3-2 & 669.06 & 1.23 & 
683.73 & 1.26 & \bf{659.34} & 
1.47\\SCA3-3 & 687.61 & 1.05 & 
695.33 & 1.01 & \bf{680.04} & 
1.11\\SCA3-4 & \bf{690.50} & 0.98 & 
713.85 & 1.06 & 690.50 & 0.00\\
SCA3-5 & 678.91 & 1.25 & 
688.99 & 1.22 & \bf{659.90} & 
2.88\\SCA3-6 & 653.69 & 1.11 & 
662.59 & 1.00 & \bf{651.09} & 
0.40\\SCA3-7 & 671.77 & 1.59 & 
685.15 & 1.32 & \bf{659.17} & 
1.91\\SCA3-8 & 724.28 & 0.98 & 
740.72 & 1.14 & \bf{719.47} & 
0.67\\SCA3-9 & 687.61 & 1.12 & 
692.96 & 1.07 & \bf{681.00} & 
0.97\\SCA8-0 & 1029.93 & 1.20 & 
1048.09 & 1.02 & \bf{961.50} & 
7.12\\SCA8-1 & 1090.49 & 0.92 & 
1110.47 & 0.91 & \bf{1049.65} & 
3.89\\SCA8-2 & 1054.95 & 1.59 & 
1078.49 & 1.10 & \bf{1039.64} & 
1.47\\SCA8-3 & 1014.71 & 1.21 & 
1043.08 & 1.04 & \bf{983.34} & 
3.19\\SCA8-4 & 1074.81 & 0.65 & 
1087.52 & 0.94 & \bf{1065.49} & 
0.87\\SCA8-5 & 1077.53 & 1.02 & 
1094.76 & 0.85 & \bf{1027.08} & 
4.91\\SCA8-6 & 1030.44 & 0.72 & 
1042.79 & 0.75 & \bf{971.82} & 
6.03\\SCA8-7 & 1090.61 & 1.11 & 
1113.31 & 1.20 & \bf{1051.28} & 
3.74\\SCA8-8 & 1097.05 & 0.94 & 
1105.87 & 1.17 & \bf{1071.18} & 
2.42\\SCA8-9 & 1095.02 & 1.00 & 
1111.54 & 1.16 & \bf{1060.50} & 
3.26\\CON3-0 & 643.28 & 1.00 & 
653.21 & 1.13 & \bf{616.52} & 
4.34\\CON3-1 & 567.40 & 0.94 & 
572.92 & 1.21 & \bf{554.47} & 
2.33\\CON3-2 & 526.55 & 1.52 & 
531.26 & 1.17 & \bf{518.00} & 
1.65\\CON3-3 & 591.20 & 1.24 & 
616.55 & 1.25 & \bf{591.19} & 
0.00\\CON3-4 & 591.43 & 1.04 & 
606.15 & 1.12 & \bf{588.79} & 
0.45\\CON3-5 & 589.19 & 0.98 & 
595.09 & 0.90 & \bf{563.70} & 
4.52\\CON3-6 & 512.50 & 1.06 & 
518.44 & 1.06 & \bf{499.05} & 
2.70\\CON3-7 & 589.93 & 1.03 & 
603.20 & 1.22 & \bf{576.48} & 
2.33\\CON3-8 & 524.59 & 0.89 & 
537.95 & 0.90 & \bf{523.05} & 
0.29\\CON3-9 & 590.17 & 1.29 & 
607.70 & 1.07 & \bf{578.24} & 
2.06\\CON8-0 & 908.39 & 0.67 & 
929.49 & 0.76 & \bf{857.17} & 
5.98\\CON8-1 & 755.77 & 0.87 & 
783.02 & 1.17 & \bf{740.85} & 
2.01\\CON8-2 & 729.70 & 0.81 & 
738.98 & 0.88 & \bf{712.89} & 
2.36\\CON8-3 & 821.22 & 1.05 & 
853.32 & 0.89 & \bf{811.07} & 
1.25\\CON8-4 & 838.60 & 0.66 & 
840.42 & 0.88 & \bf{772.25} & 
8.59\\CON8-5 & 765.74 & 1.18 & 
792.66 & 0.93 & \bf{754.88} & 
1.44\\CON8-6 & 693.76 & 0.84 & 
718.68 & 0.94 & \bf{678.92} & 
2.19\\CON8-7 & 825.08 & 1.40 & 
868.50 & 0.97 & \bf{811.96} & 
1.62\\CON8-8 & 799.89 & 0.92 & 
817.52 & 0.76 & \bf{767.53} & 
4.22\\CON8-9 & 814.72 & 0.95 & 
835.13 & 0.81 & \bf{809.00} & 
0.71\\[1ex]\hline
\end{tabular}
\label{table:nonlin}
\end{table} \clearpage
\begin{table}[ht]
\caption{Resultados de la ejecución de la metaheurística ILS, utilizando instancias de SalhiNagy con la configuración -n 5.0 -LS 60.0}
\centering
\small
\begin{tabular}{c c c c c c c}
\hline\hline
Instancia & Costo mínimo & Tiempo(seg.) & Costo promedio & Tiempo promedio(seg.) & Costo ILS & \%Gap \\ [0.5ex]
\hline
CMT1X & 486.69 & 0.71 & 
491.60 & 0.57 & \bf{466.77} & 
4.27\\CMT1Y & 500.22 & 0.79 & 
507.10 & 0.62 & \bf{466.77} & 
7.17\\CMT2X & 718.82 & 2.29 & 
723.13 & 2.14 & \bf{684.21} & 
5.06\\CMT2Y & 709.85 & 2.29 & 
725.71 & 2.31 & \bf{684.21} & 
3.75\\CMT3X & 743.59 & 4.75 & 
746.62 & 5.26 & \bf{721.40} & 
3.08\\CMT3Y & 737.60 & 3.99 & 
744.11 & 4.48 & \bf{721.40} & 
2.25\\CMT4X & 920.39 & 19.39 & 
930.29 & 16.55 & \bf{852.83} & 
7.92\\CMT4Y & 918.61 & 14.48 & 
929.34 & 13.33 & \bf{852.46} & 
7.76\\CMT5X & 1123.29 & 31.12 & 
1136.18 & 31.14 & \bf{1030.55} & 
9.00\\CMT5Y & 1118.87 & 34.52 & 
1137.55 & 31.46 & \bf{1031.17} & 
8.50\\CMT11X & 893.56 & 12.28 & 
909.56 & 9.79 & \bf{839.39} & 
6.45\\CMT11Y & 887.03 & 7.84 & 
924.81 & 9.56 & \bf{841.88} & 
5.36\\CMT12X & 689.94 & 5.40 & 
703.84 & 4.29 & \bf{662.22} & 
4.19\\CMT12Y & 681.38 & 4.99 & 
696.75 & 4.34 & \bf{662.22} & 
2.89\\[1ex]\hline
\end{tabular}
\label{table:nonlin}
\end{table} \clearpage
\begin{table}[ht]
\caption{Resultados de la ejecución de la metaheurística ILS, utilizando instancias de Dethloff con la configuración -n 5.0 -LS 70.0}
\centering
\small
\begin{tabular}{c c c c c c c}
\hline\hline
Instancia & Costo mínimo & Tiempo(seg.) & Costo promedio & Tiempo promedio(seg.) & Costo ILS & \%Gap \\ [0.5ex]
\hline
SCA3-0 & 640.55 & 1.36 & 
642.46 & 1.45 & \bf{635.62} & 
0.78\\SCA3-1 & 712.78 & 1.71 & 
718.16 & 1.40 & \bf{697.84} & 
2.14\\SCA3-2 & 677.10 & 1.18 & 
682.54 & 1.21 & \bf{659.34} & 
2.69\\SCA3-3 & 684.67 & 1.26 & 
696.03 & 1.09 & \bf{680.04} & 
0.68\\SCA3-4 & 693.23 & 1.35 & 
705.92 & 1.20 & \bf{690.50} & 
0.40\\SCA3-5 & 662.75 & 1.59 & 
677.01 & 1.83 & \bf{659.90} & 
0.43\\SCA3-6 & 661.28 & 1.29 & 
665.53 & 1.35 & \bf{651.09} & 
1.57\\SCA3-7 & 671.77 & 1.31 & 
680.79 & 1.15 & \bf{659.17} & 
1.91\\SCA3-8 & 726.86 & 2.04 & 
736.71 & 1.52 & \bf{719.47} & 
1.03\\SCA3-9 & 690.07 & 1.42 & 
692.78 & 1.23 & \bf{681.00} & 
1.33\\SCA8-0 & 1007.80 & 1.00 & 
1019.70 & 1.18 & \bf{961.50} & 
4.82\\SCA8-1 & 1101.06 & 1.14 & 
1127.76 & 0.90 & \bf{1049.65} & 
4.90\\SCA8-2 & 1091.71 & 1.00 & 
1119.01 & 0.84 & \bf{1039.64} & 
5.01\\SCA8-3 & 1036.79 & 0.84 & 
1048.63 & 0.95 & \bf{983.34} & 
5.44\\SCA8-4 & 1120.29 & 0.85 & 
1145.07 & 0.95 & \bf{1065.49} & 
5.14\\SCA8-5 & 1090.77 & 1.38 & 
1104.89 & 1.10 & \bf{1027.08} & 
6.20\\SCA8-6 & 1022.17 & 1.25 & 
1039.74 & 1.14 & \bf{971.82} & 
5.18\\SCA8-7 & 1081.50 & 0.96 & 
1095.93 & 1.33 & \bf{1051.28} & 
2.87\\SCA8-8 & 1099.22 & 1.46 & 
1124.08 & 1.13 & \bf{1071.18} & 
2.62\\SCA8-9 & 1126.82 & 1.39 & 
1145.95 & 1.06 & \bf{1060.50} & 
6.25\\CON3-0 & 633.24 & 1.22 & 
650.21 & 1.34 & \bf{616.52} & 
2.71\\CON3-1 & 568.89 & 1.62 & 
576.41 & 1.34 & \bf{554.47} & 
2.60\\CON3-2 & 521.38 & 1.32 & 
525.77 & 1.45 & \bf{518.00} & 
0.65\\CON3-3 & 594.31 & 1.28 & 
608.52 & 1.45 & \bf{591.19} & 
0.53\\CON3-4 & 605.94 & 1.79 & 
612.84 & 1.55 & \bf{588.79} & 
2.91\\CON3-5 & 568.69 & 1.00 & 
585.65 & 1.09 & \bf{563.70} & 
0.89\\CON3-6 & 516.86 & 1.35 & 
524.24 & 1.34 & \bf{499.05} & 
3.57\\CON3-7 & 599.51 & 1.00 & 
609.58 & 1.18 & \bf{576.48} & 
3.99\\CON3-8 & 526.59 & 1.52 & 
535.61 & 1.49 & \bf{523.05} & 
0.68\\CON3-9 & 591.24 & 1.74 & 
599.07 & 1.54 & \bf{578.24} & 
2.25\\CON8-0 & 907.51 & 1.12 & 
931.72 & 1.25 & \bf{857.17} & 
5.87\\CON8-1 & 786.81 & 0.98 & 
794.82 & 0.91 & \bf{740.85} & 
6.20\\CON8-2 & 727.20 & 1.18 & 
764.16 & 1.00 & \bf{712.89} & 
2.01\\CON8-3 & 850.32 & 1.03 & 
853.11 & 1.19 & \bf{811.07} & 
4.84\\CON8-4 & 827.58 & 0.72 & 
844.98 & 0.99 & \bf{772.25} & 
7.16\\CON8-5 & 785.71 & 0.92 & 
814.90 & 1.08 & \bf{754.88} & 
4.08\\CON8-6 & 702.18 & 1.22 & 
719.81 & 1.19 & \bf{678.92} & 
3.43\\CON8-7 & 842.21 & 1.27 & 
865.39 & 0.93 & \bf{811.96} & 
3.73\\CON8-8 & 790.87 & 1.02 & 
819.07 & 0.92 & \bf{767.53} & 
3.04\\CON8-9 & 815.58 & 1.51 & 
872.30 & 1.02 & \bf{809.00} & 
0.81\\[1ex]\hline
\end{tabular}
\label{table:nonlin}
\end{table} \clearpage
\begin{table}[ht]
\caption{Resultados de la ejecución de la metaheurística ILS, utilizando instancias de SalhiNagy con la configuración -n 5.0 -LS 70.0}
\centering
\small
\begin{tabular}{c c c c c c c}
\hline\hline
Instancia & Costo mínimo & Tiempo(seg.) & Costo promedio & Tiempo promedio(seg.) & Costo ILS & \%Gap \\ [0.5ex]
\hline
CMT1X & 475.71 & 1.28 & 
493.12 & 1.12 & \bf{466.77} & 
1.92\\CMT1Y & 484.83 & 0.84 & 
499.49 & 1.07 & \bf{466.77} & 
3.87\\CMT2X & 715.03 & 2.06 & 
723.79 & 2.53 & \bf{684.21} & 
4.50\\CMT2Y & 706.80 & 3.22 & 
719.04 & 2.85 & \bf{684.21} & 
3.30\\CMT3X & 749.50 & 6.07 & 
753.81 & 6.01 & \bf{721.40} & 
3.90\\CMT3Y & 738.80 & 7.47 & 
747.14 & 5.88 & \bf{721.40} & 
2.41\\CMT4X & 912.11 & 14.57 & 
922.41 & 13.18 & \bf{852.83} & 
6.95\\CMT4Y & 918.45 & 14.59 & 
926.63 & 14.45 & \bf{852.46} & 
7.74\\CMT5X & 1103.35 & 27.40 & 
1114.52 & 28.09 & \bf{1030.55} & 
7.06\\CMT5Y & 1112.39 & 32.62 & 
1121.66 & 34.88 & \bf{1031.17} & 
7.88\\CMT11X & 888.89 & 9.67 & 
949.62 & 9.13 & \bf{839.39} & 
5.90\\CMT11Y & 896.91 & 12.76 & 
933.68 & 12.34 & \bf{841.88} & 
6.54\\CMT12X & 685.84 & 4.89 & 
699.79 & 4.46 & \bf{662.22} & 
3.57\\CMT12Y & 683.03 & 4.83 & 
699.02 & 5.20 & \bf{662.22} & 
3.14\\[1ex]\hline
\end{tabular}
\label{table:nonlin}
\end{table} \clearpage
\begin{table}[ht]
\caption{Resultados de la ejecución de la metaheurística ILS, utilizando instancias de Dethloff con la configuración -n 5.0 -LS 80.0}
\centering
\small
\begin{tabular}{c c c c c c c}
\hline\hline
Instancia & Costo mínimo & Tiempo(seg.) & Costo promedio & Tiempo promedio(seg.) & Costo ILS & \%Gap \\ [0.5ex]
\hline
SCA3-0 & 644.06 & 1.54 & 
645.09 & 1.52 & \bf{635.62} & 
1.33\\SCA3-1 & 707.56 & 1.64 & 
714.15 & 1.53 & \bf{697.84} & 
1.39\\SCA3-2 & 664.18 & 1.47 & 
674.73 & 1.49 & \bf{659.34} & 
0.73\\SCA3-3 & 681.74 & 1.55 & 
690.89 & 1.70 & \bf{680.04} & 
0.25\\SCA3-4 & 693.23 & 1.24 & 
704.35 & 1.40 & \bf{690.50} & 
0.40\\SCA3-5 & \bf{659.90} & 1.24 & 
680.93 & 1.52 & 659.90 & 0.00\\
SCA3-6 & 661.52 & 1.58 & 
665.64 & 1.55 & \bf{651.09} & 
1.60\\SCA3-7 & 669.89 & 1.28 & 
673.65 & 1.35 & \bf{659.17} & 
1.63\\SCA3-8 & 723.99 & 1.69 & 
732.07 & 1.50 & \bf{719.47} & 
0.63\\SCA3-9 & 687.61 & 1.77 & 
699.06 & 1.48 & \bf{681.00} & 
0.97\\SCA8-0 & 1034.57 & 2.26 & 
1052.75 & 1.52 & \bf{961.50} & 
7.60\\SCA8-1 & 1085.51 & 1.08 & 
1102.78 & 1.17 & \bf{1049.65} & 
3.42\\SCA8-2 & 1086.84 & 0.94 & 
1116.02 & 1.12 & \bf{1039.64} & 
4.54\\SCA8-3 & 1013.59 & 1.36 & 
1043.61 & 1.19 & \bf{983.34} & 
3.08\\SCA8-4 & 1071.64 & 1.23 & 
1126.14 & 1.36 & \bf{1065.49} & 
0.58\\SCA8-5 & 1090.21 & 1.68 & 
1092.32 & 1.43 & \bf{1027.08} & 
6.15\\SCA8-6 & 1018.47 & 1.44 & 
1027.58 & 1.35 & \bf{971.82} & 
4.80\\SCA8-7 & 1099.62 & 0.89 & 
1111.05 & 1.32 & \bf{1051.28} & 
4.60\\SCA8-8 & 1097.30 & 1.65 & 
1114.59 & 1.49 & \bf{1071.18} & 
2.44\\SCA8-9 & 1101.47 & 1.20 & 
1104.92 & 1.23 & \bf{1060.50} & 
3.86\\CON3-0 & 640.11 & 1.52 & 
648.70 & 1.58 & \bf{616.52} & 
3.83\\CON3-1 & 560.75 & 2.17 & 
568.92 & 1.50 & \bf{554.47} & 
1.13\\CON3-2 & 521.63 & 1.37 & 
527.35 & 1.36 & \bf{518.00} & 
0.70\\CON3-3 & 599.26 & 1.22 & 
603.44 & 1.77 & \bf{591.19} & 
1.37\\CON3-4 & 605.94 & 1.91 & 
613.23 & 1.50 & \bf{588.79} & 
2.91\\CON3-5 & 564.88 & 1.15 & 
590.86 & 1.49 & \bf{563.70} & 
0.21\\CON3-6 & 517.70 & 1.49 & 
524.11 & 1.48 & \bf{499.05} & 
3.74\\CON3-7 & 586.01 & 1.25 & 
593.00 & 1.31 & \bf{576.48} & 
1.65\\CON3-8 & 535.32 & 1.60 & 
553.09 & 1.72 & \bf{523.05} & 
2.35\\CON3-9 & 594.95 & 1.21 & 
597.79 & 1.58 & \bf{578.24} & 
2.89\\CON8-0 & 879.18 & 0.83 & 
901.95 & 0.99 & \bf{857.17} & 
2.57\\CON8-1 & 772.74 & 2.88 & 
789.19 & 1.67 & \bf{740.85} & 
4.30\\CON8-2 & 722.68 & 1.02 & 
739.51 & 1.19 & \bf{712.89} & 
1.37\\CON8-3 & 831.49 & 1.02 & 
849.96 & 1.11 & \bf{811.07} & 
2.52\\CON8-4 & 819.51 & 1.23 & 
855.20 & 1.07 & \bf{772.25} & 
6.12\\CON8-5 & 783.36 & 1.31 & 
801.02 & 1.26 & \bf{754.88} & 
3.77\\CON8-6 & 710.73 & 1.37 & 
731.11 & 1.26 & \bf{678.92} & 
4.69\\CON8-7 & 828.54 & 1.09 & 
860.23 & 1.11 & \bf{811.96} & 
2.04\\CON8-8 & 779.43 & 2.26 & 
800.21 & 1.36 & \bf{767.53} & 
1.55\\CON8-9 & 821.72 & 1.88 & 
857.25 & 1.24 & \bf{809.00} & 
1.57\\[1ex]\hline
\end{tabular}
\label{table:nonlin}
\end{table} \clearpage
\begin{table}[ht]
\caption{Resultados de la ejecución de la metaheurística ILS, utilizando instancias de SalhiNagy con la configuración -n 5.0 -LS 80.0}
\centering
\small
\begin{tabular}{c c c c c c c}
\hline\hline
Instancia & Costo mínimo & Tiempo(seg.) & Costo promedio & Tiempo promedio(seg.) & Costo ILS & \%Gap \\ [0.5ex]
\hline
CMT1X & 481.61 & 1.36 & 
507.47 & 1.30 & \bf{466.77} & 
3.18\\CMT1Y & 490.30 & 2.12 & 
505.35 & 1.36 & \bf{466.77} & 
5.04\\CMT2X & 708.75 & 3.04 & 
720.83 & 2.89 & \bf{684.21} & 
3.59\\CMT2Y & 711.40 & 2.60 & 
714.64 & 2.49 & \bf{684.21} & 
3.97\\CMT3X & 734.99 & 6.30 & 
746.03 & 6.55 & \bf{721.40} & 
1.88\\CMT3Y & 732.79 & 8.46 & 
741.17 & 6.91 & \bf{721.40} & 
1.58\\CMT4X & 924.48 & 12.86 & 
942.26 & 15.70 & \bf{852.83} & 
8.40\\CMT4Y & 905.90 & 19.23 & 
919.52 & 16.14 & \bf{852.46} & 
6.27\\CMT5X & 1104.65 & 29.04 & 
1125.07 & 33.68 & \bf{1030.55} & 
7.19\\CMT5Y & 1080.50 & 36.47 & 
1112.76 & 37.15 & \bf{1031.17} & 
4.78\\CMT11X & 881.30 & 11.69 & 
919.11 & 11.04 & \bf{839.39} & 
4.99\\CMT11Y & 901.16 & 11.90 & 
947.28 & 12.01 & \bf{841.88} & 
7.04\\CMT12X & 674.46 & 6.45 & 
693.88 & 5.50 & \bf{662.22} & 
1.85\\CMT12Y & 684.83 & 7.58 & 
692.94 & 6.01 & \bf{662.22} & 
3.41\\[1ex]\hline
\end{tabular}
\label{table:nonlin}
\end{table} \clearpage
\begin{table}[ht]
\caption{Resultados de la ejecución de la metaheurística ILS, utilizando instancias de Dethloff con la configuración -n 15.0 -LS 10.0}
\centering
\small
\begin{tabular}{c c c c c c c}
\hline\hline
Instancia & Costo mínimo & Tiempo(seg.) & Costo promedio & Tiempo promedio(seg.) & Costo ILS & \%Gap \\ [0.5ex]
\hline
SCA3-0 & 642.80 & 1.06 & 
652.30 & 0.96 & \bf{635.62} & 
1.13\\SCA3-1 & 708.40 & 1.12 & 
728.92 & 1.05 & \bf{697.84} & 
1.51\\SCA3-2 & 670.08 & 1.14 & 
682.52 & 0.93 & \bf{659.34} & 
1.63\\SCA3-3 & 682.46 & 0.95 & 
692.43 & 1.06 & \bf{680.04} & 
0.36\\SCA3-4 & 720.66 & 0.97 & 
722.18 & 0.86 & \bf{690.50} & 
4.37\\SCA3-5 & 673.39 & 1.03 & 
683.40 & 0.97 & \bf{659.90} & 
2.04\\SCA3-6 & 656.23 & 1.12 & 
664.69 & 1.00 & \bf{651.09} & 
0.79\\SCA3-7 & 671.77 & 1.06 & 
674.22 & 0.99 & \bf{659.17} & 
1.91\\SCA3-8 & 726.86 & 0.99 & 
736.63 & 0.93 & \bf{719.47} & 
1.03\\SCA3-9 & 690.83 & 1.00 & 
697.85 & 0.85 & \bf{681.00} & 
1.44\\SCA8-0 & 1023.08 & 0.80 & 
1041.97 & 0.96 & \bf{961.50} & 
6.40\\SCA8-1 & 1096.04 & 1.34 & 
1106.20 & 0.93 & \bf{1049.65} & 
4.42\\SCA8-2 & 1082.63 & 1.30 & 
1156.26 & 0.96 & \bf{1039.64} & 
4.14\\SCA8-3 & 1015.27 & 0.86 & 
1042.83 & 0.85 & \bf{983.34} & 
3.25\\SCA8-4 & 1084.34 & 1.06 & 
1108.72 & 0.83 & \bf{1065.49} & 
1.77\\SCA8-5 & 1065.32 & 0.81 & 
1130.69 & 0.77 & \bf{1027.08} & 
3.72\\SCA8-6 & 982.49 & 0.98 & 
1011.27 & 0.89 & \bf{971.82} & 
1.10\\SCA8-7 & 1094.11 & 0.75 & 
1110.32 & 0.84 & \bf{1051.28} & 
4.07\\SCA8-8 & 1123.51 & 0.86 & 
1129.21 & 0.96 & \bf{1071.18} & 
4.89\\SCA8-9 & 1117.91 & 0.90 & 
1120.12 & 0.85 & \bf{1060.50} & 
5.41\\CON3-0 & 619.09 & 0.91 & 
646.35 & 0.85 & \bf{616.52} & 
0.42\\CON3-1 & 556.92 & 0.95 & 
570.77 & 0.83 & \bf{554.47} & 
0.44\\CON3-2 & 527.17 & 0.84 & 
532.64 & 0.85 & \bf{518.00} & 
1.77\\CON3-3 & 591.48 & 0.85 & 
596.91 & 0.97 & \bf{591.19} & 
0.05\\CON3-4 & 593.78 & 1.31 & 
612.62 & 0.93 & \bf{588.79} & 
0.85\\CON3-5 & 581.19 & 1.09 & 
587.28 & 1.04 & \bf{563.70} & 
3.10\\CON3-6 & 503.98 & 1.10 & 
524.96 & 0.99 & \bf{499.05} & 
0.99\\CON3-7 & 604.49 & 1.25 & 
613.06 & 0.88 & \bf{576.48} & 
4.86\\CON3-8 & 527.82 & 0.84 & 
536.82 & 1.02 & \bf{523.05} & 
0.91\\CON3-9 & 598.10 & 0.79 & 
601.45 & 0.85 & \bf{578.24} & 
3.43\\CON8-0 & 885.46 & 0.88 & 
905.12 & 0.78 & \bf{857.17} & 
3.30\\CON8-1 & 763.53 & 0.90 & 
785.00 & 0.95 & \bf{740.85} & 
3.06\\CON8-2 & 729.55 & 1.11 & 
740.91 & 1.02 & \bf{712.89} & 
2.34\\CON8-3 & 812.32 & 0.85 & 
832.98 & 0.81 & \bf{811.07} & 
0.15\\CON8-4 & 814.59 & 0.99 & 
834.41 & 0.92 & \bf{772.25} & 
5.48\\CON8-5 & 770.21 & 0.88 & 
808.48 & 0.88 & \bf{754.88} & 
2.03\\CON8-6 & 734.55 & 1.00 & 
742.34 & 0.86 & \bf{678.92} & 
8.19\\CON8-7 & 853.82 & 0.62 & 
856.22 & 0.76 & \bf{811.96} & 
5.16\\CON8-8 & 797.75 & 0.92 & 
806.02 & 0.89 & \bf{767.53} & 
3.94\\CON8-9 & 849.01 & 0.87 & 
875.06 & 0.94 & \bf{809.00} & 
4.95\\[1ex]\hline
\end{tabular}
\label{table:nonlin}
\end{table} \clearpage
\begin{table}[ht]
\caption{Resultados de la ejecución de la metaheurística ILS, utilizando instancias de SalhiNagy con la configuración -n 15.0 -LS 10.0}
\centering
\small
\begin{tabular}{c c c c c c c}
\hline\hline
Instancia & Costo mínimo & Tiempo(seg.) & Costo promedio & Tiempo promedio(seg.) & Costo ILS & \%Gap \\ [0.5ex]
\hline
CMT1X & 490.81 & 0.97 & 
501.76 & 0.83 & \bf{466.77} & 
5.15\\CMT1Y & 485.71 & 1.08 & 
498.10 & 0.71 & \bf{466.77} & 
4.06\\CMT2X & 722.89 & 2.41 & 
728.89 & 2.55 & \bf{684.21} & 
5.65\\CMT2Y & 722.28 & 2.32 & 
724.83 & 2.25 & \bf{684.21} & 
5.56\\CMT3X & 735.88 & 6.29 & 
746.94 & 6.08 & \bf{721.40} & 
2.01\\CMT3Y & 736.75 & 6.25 & 
742.77 & 6.00 & \bf{721.40} & 
2.13\\CMT4X & 891.65 & 19.14 & 
906.33 & 20.52 & \bf{852.83} & 
4.55\\CMT4Y & 910.90 & 20.03 & 
916.58 & 21.67 & \bf{852.46} & 
6.86\\CMT5X & 1109.35 & 68.60 & 
1111.90 & 51.84 & \bf{1030.55} & 
7.65\\CMT5Y & 1108.63 & 46.61 & 
1116.23 & 51.76 & \bf{1031.17} & 
7.51\\CMT11X & 908.96 & 14.70 & 
921.65 & 16.09 & \bf{839.39} & 
8.29\\CMT11Y & 885.84 & 14.03 & 
896.32 & 15.99 & \bf{841.88} & 
5.22\\CMT12X & 679.87 & 5.26 & 
690.46 & 5.60 & \bf{662.22} & 
2.67\\CMT12Y & 685.38 & 5.80 & 
686.17 & 6.38 & \bf{662.22} & 
3.50\\[1ex]\hline
\end{tabular}
\label{table:nonlin}
\end{table} \clearpage
\begin{table}[ht]
\caption{Resultados de la ejecución de la metaheurística ILS, utilizando instancias de Dethloff con la configuración -n 15.0 -LS 20.0}
\centering
\small
\begin{tabular}{c c c c c c c}
\hline\hline
Instancia & Costo mínimo & Tiempo(seg.) & Costo promedio & Tiempo promedio(seg.) & Costo ILS & \%Gap \\ [0.5ex]
\hline
SCA3-0 & 643.15 & 1.81 & 
651.04 & 1.59 & \bf{635.62} & 
1.18\\SCA3-1 & 701.53 & 1.44 & 
715.73 & 1.44 & \bf{697.84} & 
0.53\\SCA3-2 & 661.13 & 1.46 & 
671.13 & 1.41 & \bf{659.34} & 
0.27\\SCA3-3 & 682.44 & 1.75 & 
693.31 & 1.43 & \bf{680.04} & 
0.35\\SCA3-4 & \bf{690.50} & 2.00 & 
711.76 & 1.52 & 690.50 & 0.00\\
SCA3-5 & 678.28 & 1.94 & 
683.99 & 1.57 & \bf{659.90} & 
2.79\\SCA3-6 & 659.46 & 1.52 & 
681.51 & 1.07 & \bf{651.09} & 
1.29\\SCA3-7 & 671.77 & 1.57 & 
675.61 & 1.28 & \bf{659.17} & 
1.91\\SCA3-8 & \bf{719.47} & 1.26 & 
741.01 & 1.34 & 719.47 & 0.00\\
SCA3-9 & \bf{681.00} & 1.46 & 
687.77 & 1.56 & 681.00 & 0.00\\
SCA8-0 & 985.12 & 1.33 & 
1005.04 & 1.26 & \bf{961.50} & 
2.46\\SCA8-1 & 1079.51 & 1.34 & 
1102.48 & 1.12 & \bf{1049.65} & 
2.84\\SCA8-2 & 1056.87 & 1.03 & 
1093.94 & 1.08 & \bf{1039.64} & 
1.66\\SCA8-3 & 1030.20 & 1.19 & 
1044.39 & 1.41 & \bf{983.34} & 
4.77\\SCA8-4 & 1109.44 & 1.21 & 
1120.95 & 1.06 & \bf{1065.49} & 
4.12\\SCA8-5 & 1086.95 & 1.20 & 
1103.93 & 1.27 & \bf{1027.08} & 
5.83\\SCA8-6 & 1008.43 & 1.29 & 
1035.31 & 1.20 & \bf{971.82} & 
3.77\\SCA8-7 & 1070.92 & 1.31 & 
1116.42 & 1.27 & \bf{1051.28} & 
1.87\\SCA8-8 & 1103.44 & 1.16 & 
1131.56 & 1.18 & \bf{1071.18} & 
3.01\\SCA8-9 & 1085.11 & 1.29 & 
1120.71 & 1.10 & \bf{1060.50} & 
2.32\\CON3-0 & 634.04 & 1.36 & 
642.17 & 1.26 & \bf{616.52} & 
2.84\\CON3-1 & 560.61 & 1.54 & 
575.77 & 1.61 & \bf{554.47} & 
1.11\\CON3-2 & 521.38 & 1.20 & 
534.84 & 1.35 & \bf{518.00} & 
0.65\\CON3-3 & 594.31 & 1.44 & 
601.72 & 1.47 & \bf{591.19} & 
0.53\\CON3-4 & 603.94 & 1.54 & 
611.71 & 1.29 & \bf{588.79} & 
2.57\\CON3-5 & 569.74 & 1.39 & 
575.89 & 1.39 & \bf{563.70} & 
1.07\\CON3-6 & 507.65 & 1.41 & 
512.57 & 1.38 & \bf{499.05} & 
1.72\\CON3-7 & 578.41 & 1.21 & 
587.84 & 1.34 & \bf{576.48} & 
0.33\\CON3-8 & 534.71 & 1.58 & 
541.00 & 1.66 & \bf{523.05} & 
2.23\\CON3-9 & 589.72 & 1.65 & 
595.14 & 1.50 & \bf{578.24} & 
1.99\\CON8-0 & 899.01 & 1.17 & 
918.96 & 1.48 & \bf{857.17} & 
4.88\\CON8-1 & 756.80 & 1.08 & 
798.51 & 1.39 & \bf{740.85} & 
2.15\\CON8-2 & 716.58 & 1.19 & 
756.58 & 1.30 & \bf{712.89} & 
0.52\\CON8-3 & 842.72 & 1.34 & 
852.20 & 1.25 & \bf{811.07} & 
3.90\\CON8-4 & 797.67 & 1.05 & 
812.21 & 1.26 & \bf{772.25} & 
3.29\\CON8-5 & 781.27 & 1.16 & 
792.86 & 1.16 & \bf{754.88} & 
3.50\\CON8-6 & 706.57 & 1.37 & 
709.86 & 1.21 & \bf{678.92} & 
4.07\\CON8-7 & 822.26 & 1.60 & 
834.09 & 1.32 & \bf{811.96} & 
1.27\\CON8-8 & 773.63 & 1.44 & 
799.88 & 1.54 & \bf{767.53} & 
0.79\\CON8-9 & 835.01 & 2.26 & 
853.31 & 1.62 & \bf{809.00} & 
3.22\\[1ex]\hline
\end{tabular}
\label{table:nonlin}
\end{table} \clearpage
\begin{table}[ht]
\caption{Resultados de la ejecución de la metaheurística ILS, utilizando instancias de SalhiNagy con la configuración -n 15.0 -LS 20.0}
\centering
\small
\begin{tabular}{c c c c c c c}
\hline\hline
Instancia & Costo mínimo & Tiempo(seg.) & Costo promedio & Tiempo promedio(seg.) & Costo ILS & \%Gap \\ [0.5ex]
\hline
CMT1X & 486.26 & 1.05 & 
491.09 & 1.18 & \bf{466.77} & 
4.18\\CMT1Y & 485.04 & 1.66 & 
497.56 & 1.64 & \bf{466.77} & 
3.91\\CMT2X & 712.75 & 3.75 & 
719.04 & 3.54 & \bf{684.21} & 
4.17\\CMT2Y & 708.06 & 3.76 & 
713.36 & 3.29 & \bf{684.21} & 
3.49\\CMT3X & 737.20 & 8.68 & 
743.57 & 7.32 & \bf{721.40} & 
2.19\\CMT3Y & 736.49 & 8.11 & 
744.03 & 8.20 & \bf{721.40} & 
2.09\\CMT4X & 884.09 & 32.59 & 
907.08 & 24.07 & \bf{852.83} & 
3.67\\CMT4Y & 910.57 & 20.26 & 
919.45 & 21.61 & \bf{852.46} & 
6.82\\CMT5X & 1104.28 & 54.94 & 
1127.42 & 53.84 & \bf{1030.55} & 
7.15\\CMT5Y & 1104.90 & 58.52 & 
1119.76 & 60.12 & \bf{1031.17} & 
7.15\\CMT11X & 888.18 & 23.47 & 
902.48 & 18.95 & \bf{839.39} & 
5.81\\CMT11Y & 894.92 & 16.93 & 
910.85 & 16.28 & \bf{841.88} & 
6.30\\CMT12X & 676.42 & 6.88 & 
687.13 & 7.35 & \bf{662.22} & 
2.14\\CMT12Y & 682.07 & 6.44 & 
688.30 & 7.54 & \bf{662.22} & 
3.00\\[1ex]\hline
\end{tabular}
\label{table:nonlin}
\end{table} \clearpage
\begin{table}[ht]
\caption{Resultados de la ejecución de la metaheurística ILS, utilizando instancias de Dethloff con la configuración -n 15.0 -LS 30.0}
\centering
\small
\begin{tabular}{c c c c c c c}
\hline\hline
Instancia & Costo mínimo & Tiempo(seg.) & Costo promedio & Tiempo promedio(seg.) & Costo ILS & \%Gap \\ [0.5ex]
\hline
SCA3-0 & 640.55 & 2.17 & 
641.83 & 2.03 & \bf{635.62} & 
0.78\\SCA3-1 & 710.89 & 1.73 & 
719.42 & 1.80 & \bf{697.84} & 
1.87\\SCA3-2 & \bf{659.34} & 1.86 & 
669.22 & 1.90 & 659.34 & 0.00\\
SCA3-3 & 681.74 & 1.96 & 
689.78 & 2.04 & \bf{680.04} & 
0.25\\SCA3-4 & \bf{690.50} & 2.40 & 
714.41 & 2.08 & 690.50 & 0.00\\
SCA3-5 & 668.48 & 1.73 & 
678.48 & 1.89 & \bf{659.90} & 
1.30\\SCA3-6 & 653.69 & 2.32 & 
659.50 & 1.91 & \bf{651.09} & 
0.40\\SCA3-7 & 671.67 & 1.59 & 
672.82 & 1.80 & \bf{659.17} & 
1.90\\SCA3-8 & 731.95 & 2.47 & 
742.70 & 2.10 & \bf{719.47} & 
1.73\\SCA3-9 & \bf{681.00} & 2.16 & 
694.93 & 1.59 & 681.00 & 0.00\\
SCA8-0 & 1032.17 & 1.76 & 
1059.15 & 1.60 & \bf{961.50} & 
7.35\\SCA8-1 & 1097.27 & 1.80 & 
1107.28 & 1.57 & \bf{1049.65} & 
4.54\\SCA8-2 & 1053.94 & 1.84 & 
1092.38 & 1.52 & \bf{1039.64} & 
1.38\\SCA8-3 & 1037.15 & 1.48 & 
1048.16 & 1.52 & \bf{983.34} & 
5.47\\SCA8-4 & 1087.31 & 2.46 & 
1119.05 & 1.88 & \bf{1065.49} & 
2.05\\SCA8-5 & 1074.74 & 1.45 & 
1086.12 & 1.54 & \bf{1027.08} & 
4.64\\SCA8-6 & 1000.59 & 1.48 & 
1007.75 & 1.61 & \bf{971.82} & 
2.96\\SCA8-7 & 1091.44 & 1.73 & 
1104.60 & 1.79 & \bf{1051.28} & 
3.82\\SCA8-8 & 1098.24 & 1.99 & 
1113.80 & 1.85 & \bf{1071.18} & 
2.53\\SCA8-9 & 1075.30 & 1.73 & 
1134.80 & 1.40 & \bf{1060.50} & 
1.40\\CON3-0 & 633.24 & 1.52 & 
642.18 & 1.96 & \bf{616.52} & 
2.71\\CON3-1 & 560.75 & 2.36 & 
566.86 & 2.17 & \bf{554.47} & 
1.13\\CON3-2 & 528.47 & 2.12 & 
532.38 & 2.02 & \bf{518.00} & 
2.02\\CON3-3 & 594.31 & 1.92 & 
606.31 & 2.09 & \bf{591.19} & 
0.53\\CON3-4 & 597.75 & 2.02 & 
606.50 & 2.12 & \bf{588.79} & 
1.52\\CON3-5 & 566.96 & 2.36 & 
575.10 & 2.00 & \bf{563.70} & 
0.58\\CON3-6 & 510.38 & 1.67 & 
513.62 & 1.95 & \bf{499.05} & 
2.27\\CON3-7 & 595.90 & 1.90 & 
601.26 & 1.90 & \bf{576.48} & 
3.37\\CON3-8 & 534.28 & 2.10 & 
549.48 & 1.85 & \bf{523.05} & 
2.15\\CON3-9 & 590.64 & 1.84 & 
594.27 & 1.86 & \bf{578.24} & 
2.14\\CON8-0 & 870.35 & 2.13 & 
919.22 & 1.64 & \bf{857.17} & 
1.54\\CON8-1 & 761.69 & 2.10 & 
772.52 & 1.79 & \bf{740.85} & 
2.81\\CON8-2 & 722.68 & 1.78 & 
746.69 & 1.68 & \bf{712.89} & 
1.37\\CON8-3 & 849.44 & 1.95 & 
868.57 & 1.59 & \bf{811.07} & 
4.73\\CON8-4 & 815.15 & 1.97 & 
827.37 & 1.71 & \bf{772.25} & 
5.56\\CON8-5 & 776.17 & 1.89 & 
788.01 & 1.65 & \bf{754.88} & 
2.82\\CON8-6 & 708.92 & 1.45 & 
709.71 & 1.55 & \bf{678.92} & 
4.42\\CON8-7 & 827.08 & 1.35 & 
875.61 & 1.54 & \bf{811.96} & 
1.86\\CON8-8 & 787.95 & 1.99 & 
808.22 & 1.52 & \bf{767.53} & 
2.66\\CON8-9 & 829.64 & 2.13 & 
851.37 & 1.83 & \bf{809.00} & 
2.55\\[1ex]\hline
\end{tabular}
\label{table:nonlin}
\end{table} \clearpage
\begin{table}[ht]
\caption{Resultados de la ejecución de la metaheurística ILS, utilizando instancias de SalhiNagy con la configuración -n 15.0 -LS 30.0}
\centering
\small
\begin{tabular}{c c c c c c c}
\hline\hline
Instancia & Costo mínimo & Tiempo(seg.) & Costo promedio & Tiempo promedio(seg.) & Costo ILS & \%Gap \\ [0.5ex]
\hline
CMT1X & 485.80 & 1.56 & 
495.89 & 1.67 & \bf{466.77} & 
4.08\\CMT1Y & 476.43 & 1.69 & 
485.94 & 2.10 & \bf{466.77} & 
2.07\\CMT2X & 712.38 & 3.82 & 
719.23 & 3.69 & \bf{684.21} & 
4.12\\CMT2Y & 716.86 & 3.05 & 
721.10 & 3.47 & \bf{684.21} & 
4.77\\CMT3X & 733.82 & 9.36 & 
742.40 & 8.53 & \bf{721.40} & 
1.72\\CMT3Y & 736.53 & 8.22 & 
744.82 & 10.14 & \bf{721.40} & 
2.10\\CMT4X & 910.27 & 24.74 & 
914.12 & 26.60 & \bf{852.83} & 
6.74\\CMT4Y & 888.83 & 25.13 & 
905.68 & 25.77 & \bf{852.46} & 
4.27\\CMT5X & 1108.02 & 66.79 & 
1114.48 & 68.50 & \bf{1030.55} & 
7.52\\CMT5Y & 1102.41 & 59.63 & 
1115.61 & 71.81 & \bf{1031.17} & 
6.91\\CMT11X & 878.61 & 20.86 & 
897.95 & 24.50 & \bf{839.39} & 
4.67\\CMT11Y & 893.62 & 23.08 & 
904.36 & 19.47 & \bf{841.88} & 
6.15\\CMT12X & 692.89 & 7.64 & 
698.56 & 9.01 & \bf{662.22} & 
4.63\\CMT12Y & 685.12 & 8.11 & 
691.99 & 8.68 & \bf{662.22} & 
3.46\\[1ex]\hline
\end{tabular}
\label{table:nonlin}
\end{table} \clearpage
\begin{table}[ht]
\caption{Resultados de la ejecución de la metaheurística ILS, utilizando instancias de Dethloff con la configuración -n 15.0 -LS 40.0}
\centering
\small
\begin{tabular}{c c c c c c c}
\hline\hline
Instancia & Costo mínimo & Tiempo(seg.) & Costo promedio & Tiempo promedio(seg.) & Costo ILS & \%Gap \\ [0.5ex]
\hline
SCA3-0 & 641.69 & 2.64 & 
643.52 & 2.49 & \bf{635.62} & 
0.95\\SCA3-1 & 707.07 & 2.25 & 
711.97 & 2.40 & \bf{697.84} & 
1.32\\SCA3-2 & 661.13 & 2.69 & 
662.36 & 2.43 & \bf{659.34} & 
0.27\\SCA3-3 & 681.74 & 2.48 & 
687.22 & 2.57 & \bf{680.04} & 
0.25\\SCA3-4 & \bf{690.50} & 2.15 & 
699.55 & 2.29 & 690.50 & 0.00\\
SCA3-5 & 665.04 & 2.86 & 
675.96 & 2.64 & \bf{659.90} & 
0.78\\SCA3-6 & 652.94 & 2.00 & 
653.53 & 2.24 & \bf{651.09} & 
0.28\\SCA3-7 & 671.67 & 2.66 & 
676.13 & 2.36 & \bf{659.17} & 
1.90\\SCA3-8 & \bf{719.47} & 2.65 & 
724.19 & 2.48 & 719.47 & 0.00\\
SCA3-9 & \bf{681.00} & 2.62 & 
692.51 & 2.39 & 681.00 & 0.00\\
SCA8-0 & 992.95 & 2.65 & 
1018.17 & 2.04 & \bf{961.50} & 
3.27\\SCA8-1 & 1102.95 & 1.88 & 
1111.22 & 2.09 & \bf{1049.65} & 
5.08\\SCA8-2 & 1070.14 & 2.45 & 
1081.03 & 1.97 & \bf{1039.64} & 
2.93\\SCA8-3 & 1015.63 & 2.52 & 
1037.24 & 2.00 & \bf{983.34} & 
3.28\\SCA8-4 & 1076.95 & 1.71 & 
1118.78 & 1.80 & \bf{1065.49} & 
1.08\\SCA8-5 & 1092.28 & 1.99 & 
1094.41 & 1.99 & \bf{1027.08} & 
6.35\\SCA8-6 & 1004.39 & 1.54 & 
1020.80 & 1.79 & \bf{971.82} & 
3.35\\SCA8-7 & 1076.29 & 2.36 & 
1098.82 & 2.04 & \bf{1051.28} & 
2.38\\SCA8-8 & 1093.40 & 2.67 & 
1117.32 & 1.92 & \bf{1071.18} & 
2.07\\SCA8-9 & 1068.65 & 1.78 & 
1094.78 & 1.87 & \bf{1060.50} & 
0.77\\CON3-0 & 632.57 & 1.88 & 
642.15 & 2.48 & \bf{616.52} & 
2.60\\CON3-1 & 560.75 & 2.24 & 
563.04 & 2.69 & \bf{554.47} & 
1.13\\CON3-2 & 521.38 & 2.46 & 
527.60 & 2.68 & \bf{518.00} & 
0.65\\CON3-3 & 591.48 & 2.21 & 
606.53 & 2.68 & \bf{591.19} & 
0.05\\CON3-4 & 599.13 & 2.76 & 
621.49 & 2.40 & \bf{588.79} & 
1.76\\CON3-5 & 575.00 & 2.62 & 
578.44 & 2.77 & \bf{563.70} & 
2.00\\CON3-6 & 502.16 & 2.29 & 
512.80 & 2.47 & \bf{499.05} & 
0.62\\CON3-7 & 578.41 & 1.70 & 
592.69 & 2.10 & \bf{576.48} & 
0.33\\CON3-8 & 526.59 & 2.20 & 
534.42 & 2.85 & \bf{523.05} & 
0.68\\CON3-9 & 590.17 & 2.01 & 
599.41 & 2.18 & \bf{578.24} & 
2.06\\CON8-0 & 913.38 & 1.79 & 
939.51 & 1.88 & \bf{857.17} & 
6.56\\CON8-1 & 754.51 & 1.97 & 
776.48 & 2.13 & \bf{740.85} & 
1.84\\CON8-2 & 727.20 & 2.87 & 
737.40 & 2.13 & \bf{712.89} & 
2.01\\CON8-3 & 827.30 & 2.37 & 
842.71 & 2.21 & \bf{811.07} & 
2.00\\CON8-4 & 809.28 & 2.08 & 
827.58 & 2.17 & \bf{772.25} & 
4.80\\CON8-5 & 775.02 & 2.38 & 
790.22 & 2.05 & \bf{754.88} & 
2.67\\CON8-6 & 698.19 & 2.26 & 
702.74 & 2.20 & \bf{678.92} & 
2.84\\CON8-7 & 814.86 & 2.38 & 
853.53 & 1.81 & \bf{811.96} & 
0.36\\CON8-8 & 793.75 & 2.15 & 
803.37 & 2.22 & \bf{767.53} & 
3.42\\CON8-9 & 827.65 & 2.42 & 
857.12 & 1.96 & \bf{809.00} & 
2.31\\[1ex]\hline
\end{tabular}
\label{table:nonlin}
\end{table} \clearpage
\begin{table}[ht]
\caption{Resultados de la ejecución de la metaheurística ILS, utilizando instancias de SalhiNagy con la configuración -n 15.0 -LS 40.0}
\centering
\small
\begin{tabular}{c c c c c c c}
\hline\hline
Instancia & Costo mínimo & Tiempo(seg.) & Costo promedio & Tiempo promedio(seg.) & Costo ILS & \%Gap \\ [0.5ex]
\hline
CMT1X & 483.81 & 1.58 & 
487.36 & 1.95 & \bf{466.77} & 
3.65\\CMT1Y & 485.61 & 2.02 & 
492.14 & 1.81 & \bf{466.77} & 
4.04\\CMT2X & 719.57 & 4.65 & 
722.80 & 4.65 & \bf{684.21} & 
5.17\\CMT2Y & 707.29 & 4.18 & 
716.39 & 4.33 & \bf{684.21} & 
3.37\\CMT3X & 738.24 & 11.05 & 
745.11 & 11.86 & \bf{721.40} & 
2.33\\CMT3Y & 733.03 & 10.48 & 
737.23 & 10.92 & \bf{721.40} & 
1.61\\CMT4X & 886.86 & 30.14 & 
911.71 & 32.65 & \bf{852.83} & 
3.99\\CMT4Y & 906.85 & 31.16 & 
913.06 & 32.60 & \bf{852.46} & 
6.38\\CMT5X & 1096.95 & 64.51 & 
1117.64 & 66.93 & \bf{1030.55} & 
6.44\\CMT5Y & 1097.04 & 65.53 & 
1112.25 & 71.24 & \bf{1031.17} & 
6.39\\CMT11X & 885.76 & 23.56 & 
910.45 & 22.46 & \bf{839.39} & 
5.52\\CMT11Y & 890.29 & 30.85 & 
903.69 & 24.84 & \bf{841.88} & 
5.75\\CMT12X & 688.70 & 9.95 & 
691.88 & 9.90 & \bf{662.22} & 
4.00\\CMT12Y & 674.52 & 12.13 & 
680.92 & 9.64 & \bf{662.22} & 
1.86\\[1ex]\hline
\end{tabular}
\label{table:nonlin}
\end{table} \clearpage
\begin{table}[ht]
\caption{Resultados de la ejecución de la metaheurística ILS, utilizando instancias de Dethloff con la configuración -n 15.0 -LS 50.0}
\centering
\small
\begin{tabular}{c c c c c c c}
\hline\hline
Instancia & Costo mínimo & Tiempo(seg.) & Costo promedio & Tiempo promedio(seg.) & Costo ILS & \%Gap \\ [0.5ex]
\hline
SCA3-0 & 640.55 & 3.14 & 
643.89 & 3.06 & \bf{635.62} & 
0.78\\SCA3-1 & 701.53 & 3.91 & 
704.65 & 3.14 & \bf{697.84} & 
0.53\\SCA3-2 & 661.13 & 2.89 & 
665.61 & 2.52 & \bf{659.34} & 
0.27\\SCA3-3 & 680.60 & 2.88 & 
684.20 & 2.99 & \bf{680.04} & 
0.08\\SCA3-4 & 693.23 & 3.05 & 
698.42 & 2.97 & \bf{690.50} & 
0.40\\SCA3-5 & 680.58 & 2.34 & 
685.33 & 2.69 & \bf{659.90} & 
3.13\\SCA3-6 & \bf{651.09} & 3.25 & 
659.08 & 2.77 & 651.09 & 0.00\\
SCA3-7 & 671.77 & 2.62 & 
673.53 & 2.67 & \bf{659.17} & 
1.91\\SCA3-8 & 724.28 & 2.66 & 
736.83 & 2.40 & \bf{719.47} & 
0.67\\SCA3-9 & 685.00 & 1.78 & 
699.89 & 2.37 & \bf{681.00} & 
0.59\\SCA8-0 & 994.55 & 3.31 & 
1025.73 & 2.88 & \bf{961.50} & 
3.44\\SCA8-1 & 1099.37 & 2.85 & 
1109.90 & 2.50 & \bf{1049.65} & 
4.74\\SCA8-2 & 1058.07 & 2.44 & 
1070.44 & 2.57 & \bf{1039.64} & 
1.77\\SCA8-3 & 1022.89 & 1.78 & 
1038.36 & 2.29 & \bf{983.34} & 
4.02\\SCA8-4 & 1073.64 & 2.55 & 
1115.76 & 2.23 & \bf{1065.49} & 
0.76\\SCA8-5 & 1052.57 & 2.26 & 
1074.50 & 2.18 & \bf{1027.08} & 
2.48\\SCA8-6 & 1023.10 & 1.70 & 
1032.89 & 2.16 & \bf{971.82} & 
5.28\\SCA8-7 & 1076.01 & 1.69 & 
1107.86 & 2.21 & \bf{1051.28} & 
2.35\\SCA8-8 & 1082.11 & 2.45 & 
1087.91 & 2.56 & \bf{1071.18} & 
1.02\\SCA8-9 & 1101.27 & 2.40 & 
1114.80 & 2.37 & \bf{1060.50} & 
3.84\\CON3-0 & 637.07 & 3.18 & 
654.38 & 3.33 & \bf{616.52} & 
3.33\\CON3-1 & 560.75 & 2.82 & 
574.83 & 2.75 & \bf{554.47} & 
1.13\\CON3-2 & 526.07 & 2.64 & 
531.14 & 2.78 & \bf{518.00} & 
1.56\\CON3-3 & 594.31 & 2.61 & 
599.31 & 2.73 & \bf{591.19} & 
0.53\\CON3-4 & 605.10 & 2.88 & 
606.88 & 2.79 & \bf{588.79} & 
2.77\\CON3-5 & 575.98 & 2.87 & 
578.77 & 2.92 & \bf{563.70} & 
2.18\\CON3-6 & 504.00 & 2.71 & 
508.92 & 2.79 & \bf{499.05} & 
0.99\\CON3-7 & 586.01 & 2.56 & 
595.49 & 3.02 & \bf{576.48} & 
1.65\\CON3-8 & 526.59 & 3.52 & 
535.67 & 3.09 & \bf{523.05} & 
0.68\\CON3-9 & 588.55 & 3.59 & 
589.32 & 3.17 & \bf{578.24} & 
1.78\\CON8-0 & 875.52 & 1.86 & 
904.12 & 2.25 & \bf{857.17} & 
2.14\\CON8-1 & 764.40 & 2.16 & 
780.44 & 2.31 & \bf{740.85} & 
3.18\\CON8-2 & 727.69 & 2.84 & 
733.58 & 2.85 & \bf{712.89} & 
2.08\\CON8-3 & 835.64 & 1.78 & 
844.65 & 2.65 & \bf{811.07} & 
3.03\\CON8-4 & 813.23 & 2.45 & 
825.06 & 2.29 & \bf{772.25} & 
5.31\\CON8-5 & 778.59 & 2.31 & 
785.41 & 2.54 & \bf{754.88} & 
3.14\\CON8-6 & 699.80 & 3.13 & 
712.44 & 2.65 & \bf{678.92} & 
3.08\\CON8-7 & 863.21 & 2.11 & 
869.54 & 2.40 & \bf{811.96} & 
6.31\\CON8-8 & 793.52 & 2.39 & 
798.35 & 2.39 & \bf{767.53} & 
3.39\\CON8-9 & 827.60 & 3.14 & 
839.03 & 2.85 & \bf{809.00} & 
2.30\\[1ex]\hline
\end{tabular}
\label{table:nonlin}
\end{table} \clearpage
\begin{table}[ht]
\caption{Resultados de la ejecución de la metaheurística ILS, utilizando instancias de SalhiNagy con la configuración -n 15.0 -LS 50.0}
\centering
\small
\begin{tabular}{c c c c c c c}
\hline\hline
Instancia & Costo mínimo & Tiempo(seg.) & Costo promedio & Tiempo promedio(seg.) & Costo ILS & \%Gap \\ [0.5ex]
\hline
CMT1X & 482.58 & 2.42 & 
488.69 & 2.31 & \bf{466.77} & 
3.39\\CMT1Y & 484.35 & 3.17 & 
487.91 & 2.83 & \bf{466.77} & 
3.77\\CMT2X & 703.29 & 6.15 & 
715.84 & 5.59 & \bf{684.21} & 
2.79\\CMT2Y & 697.22 & 5.51 & 
710.04 & 5.49 & \bf{684.21} & 
1.90\\CMT3X & 738.81 & 14.16 & 
745.38 & 12.72 & \bf{721.40} & 
2.41\\CMT3Y & 736.83 & 14.25 & 
746.04 & 12.09 & \bf{721.40} & 
2.14\\CMT4X & 885.47 & 42.58 & 
908.59 & 35.96 & \bf{852.83} & 
3.83\\CMT4Y & 885.93 & 32.07 & 
901.18 & 33.30 & \bf{852.46} & 
3.93\\CMT5X & 1106.75 & 75.47 & 
1120.88 & 89.86 & \bf{1030.55} & 
7.39\\CMT5Y & 1106.52 & 78.43 & 
1113.08 & 73.41 & \bf{1031.17} & 
7.31\\CMT11X & 881.37 & 23.47 & 
905.50 & 26.91 & \bf{839.39} & 
5.00\\CMT11Y & 917.08 & 26.52 & 
927.80 & 27.64 & \bf{841.88} & 
8.93\\CMT12X & 675.84 & 11.72 & 
682.70 & 11.37 & \bf{662.22} & 
2.06\\CMT12Y & 681.45 & 11.44 & 
698.13 & 11.65 & \bf{662.22} & 
2.90\\[1ex]\hline
\end{tabular}
\label{table:nonlin}
\end{table} \clearpage
\begin{table}[ht]
\caption{Resultados de la ejecución de la metaheurística ILS, utilizando instancias de Dethloff con la configuración -n 15.0 -LS 60.0}
\centering
\small
\begin{tabular}{c c c c c c c}
\hline\hline
Instancia & Costo mínimo & Tiempo(seg.) & Costo promedio & Tiempo promedio(seg.) & Costo ILS & \%Gap \\ [0.5ex]
\hline
SCA3-0 & 640.55 & 3.60 & 
642.47 & 3.68 & \bf{635.62} & 
0.78\\SCA3-1 & \bf{697.84} & 3.80 & 
709.67 & 3.16 & 697.84 & 0.00\\
SCA3-2 & 664.21 & 3.61 & 
670.49 & 3.50 & \bf{659.34} & 
0.74\\SCA3-3 & 688.71 & 3.08 & 
691.89 & 3.21 & \bf{680.04} & 
1.27\\SCA3-4 & \bf{690.50} & 2.66 & 
696.44 & 3.25 & 690.50 & 0.00\\
SCA3-5 & 681.81 & 3.40 & 
682.83 & 3.28 & \bf{659.90} & 
3.32\\SCA3-6 & 652.94 & 2.89 & 
657.08 & 3.04 & \bf{651.09} & 
0.28\\SCA3-7 & 671.67 & 3.12 & 
673.98 & 2.99 & \bf{659.17} & 
1.90\\SCA3-8 & 722.05 & 2.24 & 
729.83 & 3.03 & \bf{719.47} & 
0.36\\SCA3-9 & 684.44 & 3.36 & 
688.17 & 2.78 & \bf{681.00} & 
0.51\\SCA8-0 & 1001.24 & 2.78 & 
1024.56 & 3.29 & \bf{961.50} & 
4.13\\SCA8-1 & 1080.34 & 3.33 & 
1098.20 & 3.09 & \bf{1049.65} & 
2.92\\SCA8-2 & 1054.69 & 3.70 & 
1076.45 & 2.96 & \bf{1039.64} & 
1.45\\SCA8-3 & 1013.56 & 3.68 & 
1034.87 & 2.83 & \bf{983.34} & 
3.07\\SCA8-4 & 1116.49 & 2.71 & 
1129.05 & 2.79 & \bf{1065.49} & 
4.79\\SCA8-5 & 1051.68 & 2.83 & 
1084.48 & 2.64 & \bf{1027.08} & 
2.40\\SCA8-6 & 986.87 & 2.86 & 
1008.29 & 2.65 & \bf{971.82} & 
1.55\\SCA8-7 & 1070.53 & 3.13 & 
1097.44 & 3.09 & \bf{1051.28} & 
1.83\\SCA8-8 & 1090.13 & 4.48 & 
1097.67 & 3.42 & \bf{1071.18} & 
1.77\\SCA8-9 & 1119.39 & 2.52 & 
1126.38 & 2.48 & \bf{1060.50} & 
5.55\\CON3-0 & 633.24 & 3.09 & 
641.35 & 3.44 & \bf{616.52} & 
2.71\\CON3-1 & 569.07 & 3.35 & 
569.72 & 3.54 & \bf{554.47} & 
2.63\\CON3-2 & 521.63 & 3.55 & 
528.02 & 3.47 & \bf{518.00} & 
0.70\\CON3-3 & \bf{591.19} & 3.68 & 
611.30 & 3.35 & 591.19 & 0.00\\
CON3-4 & 607.56 & 3.76 & 
613.12 & 3.20 & \bf{588.79} & 
3.19\\CON3-5 & 573.93 & 3.87 & 
584.76 & 4.33 & \bf{563.70} & 
1.81\\CON3-6 & 502.16 & 4.00 & 
509.28 & 3.17 & \bf{499.05} & 
0.62\\CON3-7 & 578.41 & 3.43 & 
590.63 & 3.54 & \bf{576.48} & 
0.33\\CON3-8 & 527.82 & 2.88 & 
529.06 & 3.41 & \bf{523.05} & 
0.91\\CON3-9 & 578.25 & 2.86 & 
588.07 & 3.39 & \bf{578.24} & 
0.00\\CON8-0 & 906.70 & 2.13 & 
924.12 & 2.36 & \bf{857.17} & 
5.78\\CON8-1 & 767.78 & 3.38 & 
782.16 & 2.98 & \bf{740.85} & 
3.64\\CON8-2 & 716.42 & 2.76 & 
730.41 & 3.08 & \bf{712.89} & 
0.50\\CON8-3 & 838.05 & 3.07 & 
847.90 & 2.93 & \bf{811.07} & 
3.33\\CON8-4 & 816.10 & 3.74 & 
829.37 & 2.74 & \bf{772.25} & 
5.68\\CON8-5 & 786.26 & 2.38 & 
799.15 & 2.63 & \bf{754.88} & 
4.16\\CON8-6 & 710.95 & 4.43 & 
716.80 & 3.35 & \bf{678.92} & 
4.72\\CON8-7 & 837.10 & 2.59 & 
848.37 & 2.63 & \bf{811.96} & 
3.10\\CON8-8 & 790.42 & 2.03 & 
818.38 & 2.36 & \bf{767.53} & 
2.98\\CON8-9 & 814.10 & 3.80 & 
851.04 & 3.00 & \bf{809.00} & 
0.63\\[1ex]\hline
\end{tabular}
\label{table:nonlin}
\end{table} \clearpage
\begin{table}[ht]
\caption{Resultados de la ejecución de la metaheurística ILS, utilizando instancias de SalhiNagy con la configuración -n 15.0 -LS 60.0}
\centering
\small
\begin{tabular}{c c c c c c c}
\hline\hline
Instancia & Costo mínimo & Tiempo(seg.) & Costo promedio & Tiempo promedio(seg.) & Costo ILS & \%Gap \\ [0.5ex]
\hline
CMT1X & 472.58 & 3.09 & 
479.25 & 3.19 & \bf{466.77} & 
1.24\\CMT1Y & 479.58 & 2.76 & 
491.56 & 2.46 & \bf{466.77} & 
2.74\\CMT2X & 709.75 & 6.40 & 
716.60 & 5.98 & \bf{684.21} & 
3.73\\CMT2Y & 706.79 & 7.23 & 
716.12 & 6.52 & \bf{684.21} & 
3.30\\CMT3X & 737.52 & 13.66 & 
741.51 & 14.49 & \bf{721.40} & 
2.23\\CMT3Y & 732.16 & 15.61 & 
740.49 & 13.80 & \bf{721.40} & 
1.49\\CMT4X & 892.10 & 35.67 & 
908.05 & 40.24 & \bf{852.83} & 
4.60\\CMT4Y & 907.31 & 31.36 & 
917.20 & 35.60 & \bf{852.46} & 
6.43\\CMT5X & 1094.53 & 87.30 & 
1115.61 & 94.43 & \bf{1030.55} & 
6.21\\CMT5Y & 1097.57 & 90.85 & 
1112.60 & 92.58 & \bf{1031.17} & 
6.44\\CMT11X & 906.83 & 28.35 & 
917.29 & 25.84 & \bf{839.39} & 
8.03\\CMT11Y & 903.57 & 28.24 & 
918.31 & 28.97 & \bf{841.88} & 
7.33\\CMT12X & 680.97 & 13.52 & 
684.69 & 14.19 & \bf{662.22} & 
2.83\\CMT12Y & 682.02 & 13.99 & 
689.30 & 12.93 & \bf{662.22} & 
2.99\\[1ex]\hline
\end{tabular}
\label{table:nonlin}
\end{table} \clearpage
\begin{table}[ht]
\caption{Resultados de la ejecución de la metaheurística ILS, utilizando instancias de Dethloff con la configuración -n 15.0 -LS 70.0}
\centering
\small
\begin{tabular}{c c c c c c c}
\hline\hline
Instancia & Costo mínimo & Tiempo(seg.) & Costo promedio & Tiempo promedio(seg.) & Costo ILS & \%Gap \\ [0.5ex]
\hline
SCA3-0 & 640.55 & 3.85 & 
640.55 & 4.34 & \bf{635.62} & 
0.78\\SCA3-1 & 701.74 & 3.85 & 
710.73 & 3.87 & \bf{697.84} & 
0.56\\SCA3-2 & 664.18 & 3.39 & 
668.87 & 3.85 & \bf{659.34} & 
0.73\\SCA3-3 & 680.60 & 3.79 & 
686.12 & 3.76 & \bf{680.04} & 
0.08\\SCA3-4 & \bf{690.50} & 4.08 & 
699.39 & 3.66 & 690.50 & 0.00\\
SCA3-5 & 670.02 & 3.58 & 
678.28 & 3.53 & \bf{659.90} & 
1.53\\SCA3-6 & 658.26 & 4.14 & 
661.70 & 3.60 & \bf{651.09} & 
1.10\\SCA3-7 & 666.15 & 3.54 & 
670.10 & 3.96 & \bf{659.17} & 
1.06\\SCA3-8 & 724.29 & 5.16 & 
733.03 & 4.09 & \bf{719.47} & 
0.67\\SCA3-9 & 684.44 & 4.22 & 
690.09 & 3.44 & \bf{681.00} & 
0.51\\SCA8-0 & 970.64 & 3.47 & 
992.63 & 3.31 & \bf{961.50} & 
0.95\\SCA8-1 & 1094.34 & 2.84 & 
1107.14 & 2.92 & \bf{1049.65} & 
4.26\\SCA8-2 & 1053.50 & 4.38 & 
1067.80 & 3.66 & \bf{1039.64} & 
1.33\\SCA8-3 & 1019.06 & 2.73 & 
1032.78 & 2.92 & \bf{983.34} & 
3.63\\SCA8-4 & 1067.55 & 3.26 & 
1103.88 & 2.86 & \bf{1065.49} & 
0.19\\SCA8-5 & 1094.60 & 3.55 & 
1097.64 & 3.17 & \bf{1027.08} & 
6.57\\SCA8-6 & 993.28 & 3.48 & 
1006.06 & 3.24 & \bf{971.82} & 
2.21\\SCA8-7 & 1081.46 & 2.83 & 
1092.44 & 2.88 & \bf{1051.28} & 
2.87\\SCA8-8 & \bf{1071.18} & 3.58 & 
1095.32 & 3.32 & 1071.18 & 0.00\\
SCA8-9 & 1078.21 & 2.87 & 
1098.01 & 3.41 & \bf{1060.50} & 
1.67\\CON3-0 & 633.24 & 3.47 & 
634.76 & 3.80 & \bf{616.52} & 
2.71\\CON3-1 & 562.52 & 3.99 & 
569.05 & 3.90 & \bf{554.47} & 
1.45\\CON3-2 & 519.11 & 3.83 & 
523.37 & 4.08 & \bf{518.00} & 
0.21\\CON3-3 & 599.26 & 4.78 & 
612.29 & 4.08 & \bf{591.19} & 
1.37\\CON3-4 & 593.78 & 4.34 & 
607.30 & 4.34 & \bf{588.79} & 
0.85\\CON3-5 & 569.04 & 4.46 & 
572.65 & 4.08 & \bf{563.70} & 
0.95\\CON3-6 & 505.99 & 3.60 & 
510.04 & 3.57 & \bf{499.05} & 
1.39\\CON3-7 & 583.65 & 3.78 & 
592.64 & 3.78 & \bf{576.48} & 
1.24\\CON3-8 & 524.59 & 3.70 & 
536.00 & 3.83 & \bf{523.05} & 
0.29\\CON3-9 & 578.25 & 2.48 & 
587.77 & 4.05 & \bf{578.24} & 
0.00\\CON8-0 & 869.87 & 2.94 & 
897.65 & 3.00 & \bf{857.17} & 
1.48\\CON8-1 & 777.27 & 6.56 & 
791.99 & 4.10 & \bf{740.85} & 
4.92\\CON8-2 & 735.75 & 3.06 & 
743.03 & 3.06 & \bf{712.89} & 
3.21\\CON8-3 & 826.64 & 3.68 & 
843.89 & 3.88 & \bf{811.07} & 
1.92\\CON8-4 & 823.60 & 3.17 & 
829.53 & 2.93 & \bf{772.25} & 
6.65\\CON8-5 & 772.95 & 3.80 & 
775.68 & 3.24 & \bf{754.88} & 
2.39\\CON8-6 & 700.72 & 3.51 & 
709.08 & 3.40 & \bf{678.92} & 
3.21\\CON8-7 & 814.50 & 3.10 & 
830.28 & 3.31 & \bf{811.96} & 
0.31\\CON8-8 & 781.53 & 4.82 & 
795.13 & 3.52 & \bf{767.53} & 
1.82\\CON8-9 & 820.41 & 4.06 & 
836.63 & 4.04 & \bf{809.00} & 
1.41\\[1ex]\hline
\end{tabular}
\label{table:nonlin}
\end{table} \clearpage
\begin{table}[ht]
\caption{Resultados de la ejecución de la metaheurística ILS, utilizando instancias de SalhiNagy con la configuración -n 15.0 -LS 70.0}
\centering
\small
\begin{tabular}{c c c c c c c}
\hline\hline
Instancia & Costo mínimo & Tiempo(seg.) & Costo promedio & Tiempo promedio(seg.) & Costo ILS & \%Gap \\ [0.5ex]
\hline
CMT1X & 475.26 & 4.34 & 
485.46 & 3.38 & \bf{466.77} & 
1.82\\CMT1Y & 481.35 & 3.20 & 
484.10 & 3.21 & \bf{466.77} & 
3.12\\CMT2X & 710.07 & 7.60 & 
716.92 & 7.26 & \bf{684.21} & 
3.78\\CMT2Y & 707.44 & 9.03 & 
714.15 & 7.36 & \bf{684.21} & 
3.40\\CMT3X & 731.70 & 16.21 & 
742.18 & 16.43 & \bf{721.40} & 
1.43\\CMT3Y & 740.17 & 15.13 & 
747.96 & 15.23 & \bf{721.40} & 
2.60\\CMT4X & 902.54 & 49.64 & 
916.48 & 44.45 & \bf{852.83} & 
5.83\\CMT4Y & 884.88 & 40.24 & 
904.55 & 40.63 & \bf{852.46} & 
3.80\\CMT5X & 1082.12 & 91.86 & 
1106.05 & 95.26 & \bf{1030.55} & 
5.00\\CMT5Y & 1092.17 & 86.99 & 
1095.57 & 102.14 & \bf{1031.17} & 
5.92\\CMT11X & 892.34 & 31.33 & 
903.21 & 32.70 & \bf{839.39} & 
6.31\\CMT11Y & 881.48 & 26.31 & 
895.35 & 32.22 & \bf{841.88} & 
4.70\\CMT12X & 683.86 & 13.20 & 
686.20 & 13.97 & \bf{662.22} & 
3.27\\CMT12Y & 677.36 & 13.16 & 
687.69 & 13.82 & \bf{662.22} & 
2.29\\[1ex]\hline
\end{tabular}
\label{table:nonlin}
\end{table} \clearpage
\begin{table}[ht]
\caption{Resultados de la ejecución de la metaheurística ILS, utilizando instancias de Dethloff con la configuración -n 15.0 -LS 80.0}
\centering
\small
\begin{tabular}{c c c c c c c}
\hline\hline
Instancia & Costo mínimo & Tiempo(seg.) & Costo promedio & Tiempo promedio(seg.) & Costo ILS & \%Gap \\ [0.5ex]
\hline
SCA3-0 & 641.69 & 4.44 & 
643.36 & 4.50 & \bf{635.62} & 
0.95\\SCA3-1 & \bf{697.84} & 4.31 & 
708.70 & 4.13 & 697.84 & 0.00\\
SCA3-2 & 664.21 & 4.50 & 
665.51 & 4.58 & \bf{659.34} & 
0.74\\SCA3-3 & \bf{680.04} & 4.42 & 
684.29 & 4.67 & 680.04 & 0.00\\
SCA3-4 & \bf{690.50} & 4.22 & 
697.00 & 4.45 & 690.50 & 0.00\\
SCA3-5 & 670.10 & 4.12 & 
676.84 & 4.44 & \bf{659.90} & 
1.55\\SCA3-6 & \bf{651.09} & 4.23 & 
653.80 & 4.57 & 651.09 & 0.00\\
SCA3-7 & 671.67 & 4.01 & 
673.34 & 4.09 & \bf{659.17} & 
1.90\\SCA3-8 & \bf{719.47} & 5.60 & 
729.26 & 4.71 & 719.47 & 0.00\\
SCA3-9 & \bf{681.00} & 4.16 & 
694.21 & 3.97 & 681.00 & 0.00\\
SCA8-0 & 1016.38 & 3.14 & 
1039.87 & 3.50 & \bf{961.50} & 
5.71\\SCA8-1 & 1081.69 & 2.83 & 
1098.72 & 2.83 & \bf{1049.65} & 
3.05\\SCA8-2 & 1066.96 & 2.91 & 
1072.10 & 3.45 & \bf{1039.64} & 
2.63\\SCA8-3 & 1023.04 & 3.25 & 
1036.71 & 3.63 & \bf{983.34} & 
4.04\\SCA8-4 & 1071.44 & 3.49 & 
1109.73 & 3.29 & \bf{1065.49} & 
0.56\\SCA8-5 & 1054.62 & 4.70 & 
1078.71 & 3.82 & \bf{1027.08} & 
2.68\\SCA8-6 & 992.79 & 4.07 & 
1004.05 & 4.06 & \bf{971.82} & 
2.16\\SCA8-7 & 1091.63 & 5.52 & 
1101.47 & 4.51 & \bf{1051.28} & 
3.84\\SCA8-8 & 1105.20 & 3.46 & 
1112.16 & 3.36 & \bf{1071.18} & 
3.18\\SCA8-9 & 1090.14 & 3.42 & 
1092.73 & 3.16 & \bf{1060.50} & 
2.79\\CON3-0 & 632.57 & 3.90 & 
641.70 & 4.06 & \bf{616.52} & 
2.60\\CON3-1 & 561.87 & 3.76 & 
567.50 & 4.65 & \bf{554.47} & 
1.33\\CON3-2 & 524.89 & 4.16 & 
535.70 & 3.97 & \bf{518.00} & 
1.33\\CON3-3 & 591.20 & 4.02 & 
601.83 & 4.33 & \bf{591.19} & 
0.00\\CON3-4 & 589.32 & 4.44 & 
591.16 & 4.05 & \bf{588.79} & 
0.09\\CON3-5 & 568.85 & 3.71 & 
574.24 & 4.29 & \bf{563.70} & 
0.91\\CON3-6 & 508.14 & 4.36 & 
512.75 & 3.81 & \bf{499.05} & 
1.82\\CON3-7 & 578.41 & 4.57 & 
588.59 & 4.34 & \bf{576.48} & 
0.33\\CON3-8 & 524.59 & 5.34 & 
528.55 & 4.99 & \bf{523.05} & 
0.29\\CON3-9 & 588.99 & 4.10 & 
591.18 & 4.39 & \bf{578.24} & 
1.86\\CON8-0 & 895.71 & 3.50 & 
908.88 & 3.33 & \bf{857.17} & 
4.50\\CON8-1 & 785.75 & 4.37 & 
792.75 & 3.68 & \bf{740.85} & 
6.06\\CON8-2 & 732.65 & 3.75 & 
736.25 & 3.37 & \bf{712.89} & 
2.77\\CON8-3 & 844.61 & 5.44 & 
856.23 & 4.09 & \bf{811.07} & 
4.14\\CON8-4 & 796.27 & 2.85 & 
809.30 & 3.82 & \bf{772.25} & 
3.11\\CON8-5 & 767.91 & 3.94 & 
776.24 & 3.81 & \bf{754.88} & 
1.73\\CON8-6 & 705.14 & 5.48 & 
714.47 & 4.24 & \bf{678.92} & 
3.86\\CON8-7 & 815.32 & 4.17 & 
835.56 & 3.72 & \bf{811.96} & 
0.41\\CON8-8 & 802.01 & 3.02 & 
809.11 & 3.44 & \bf{767.53} & 
4.49\\CON8-9 & 821.98 & 4.36 & 
835.27 & 3.97 & \bf{809.00} & 
1.60\\[1ex]\hline
\end{tabular}
\label{table:nonlin}
\end{table} \clearpage
\begin{table}[ht]
\caption{Resultados de la ejecución de la metaheurística ILS, utilizando instancias de SalhiNagy con la configuración -n 15.0 -LS 80.0}
\centering
\small
\begin{tabular}{c c c c c c c}
\hline\hline
Instancia & Costo mínimo & Tiempo(seg.) & Costo promedio & Tiempo promedio(seg.) & Costo ILS & \%Gap \\ [0.5ex]
\hline
CMT1X & 478.36 & 3.05 & 
481.78 & 3.39 & \bf{466.77} & 
2.48\\CMT1Y & 482.00 & 2.94 & 
484.41 & 4.03 & \bf{466.77} & 
3.26\\CMT2X & 701.91 & 6.38 & 
712.60 & 7.57 & \bf{684.21} & 
2.59\\CMT2Y & 704.88 & 10.15 & 
712.70 & 7.63 & \bf{684.21} & 
3.02\\CMT3X & 743.73 & 14.66 & 
747.17 & 17.75 & \bf{721.40} & 
3.10\\CMT3Y & 731.57 & 20.33 & 
739.73 & 19.57 & \bf{721.40} & 
1.41\\CMT4X & 892.92 & 49.28 & 
902.73 & 48.09 & \bf{852.83} & 
4.70\\CMT4Y & 901.33 & 47.54 & 
911.64 & 46.09 & \bf{852.46} & 
5.73\\CMT5X & 1110.97 & 116.63 & 
1118.00 & 102.77 & \bf{1030.55} & 
7.80\\CMT5Y & 1101.65 & 96.35 & 
1114.53 & 112.58 & \bf{1031.17} & 
6.83\\CMT11X & 883.45 & 34.88 & 
898.17 & 36.09 & \bf{839.39} & 
5.25\\CMT11Y & 878.31 & 38.18 & 
886.20 & 34.47 & \bf{841.88} & 
4.33\\CMT12X & 691.35 & 15.08 & 
693.81 & 15.86 & \bf{662.22} & 
4.40\\CMT12Y & 673.67 & 19.01 & 
679.04 & 17.70 & \bf{662.22} & 
1.73\\[1ex]\hline
\end{tabular}
\label{table:nonlin}
\end{table} \clearpage
\begin{table}[ht]
\caption{Resultados de la ejecución de la metaheurística ILS, utilizando instancias de Dethloff con la configuración -n 25.0 -LS 10.0}
\centering
\small
\begin{tabular}{c c c c c c c}
\hline\hline
Instancia & Costo mínimo & Tiempo(seg.) & Costo promedio & Tiempo promedio(seg.) & Costo ILS & \%Gap \\ [0.5ex]
\hline
SCA3-0 & 640.55 & 1.60 & 
643.19 & 1.53 & \bf{635.62} & 
0.78\\SCA3-1 & 700.50 & 2.12 & 
726.25 & 1.63 & \bf{697.84} & 
0.38\\SCA3-2 & 675.12 & 1.42 & 
683.29 & 1.53 & \bf{659.34} & 
2.39\\SCA3-3 & 682.46 & 1.54 & 
688.54 & 1.49 & \bf{680.04} & 
0.36\\SCA3-4 & \bf{690.50} & 1.38 & 
701.73 & 1.44 & 690.50 & 0.00\\
SCA3-5 & \bf{659.90} & 1.78 & 
673.82 & 1.62 & 659.90 & 0.00\\
SCA3-6 & 652.94 & 1.67 & 
655.67 & 1.55 & \bf{651.09} & 
0.28\\SCA3-7 & 672.85 & 1.42 & 
673.85 & 1.54 & \bf{659.17} & 
2.08\\SCA3-8 & \bf{719.47} & 1.52 & 
733.67 & 1.65 & 719.47 & 0.00\\
SCA3-9 & 690.07 & 1.58 & 
700.76 & 1.62 & \bf{681.00} & 
1.33\\SCA8-0 & 1015.61 & 1.98 & 
1027.39 & 1.87 & \bf{961.50} & 
5.63\\SCA8-1 & 1097.18 & 1.29 & 
1117.42 & 1.26 & \bf{1049.65} & 
4.53\\SCA8-2 & 1092.94 & 1.69 & 
1102.00 & 1.41 & \bf{1039.64} & 
5.13\\SCA8-3 & 1025.40 & 1.03 & 
1044.06 & 1.36 & \bf{983.34} & 
4.28\\SCA8-4 & 1121.03 & 1.63 & 
1156.94 & 1.47 & \bf{1065.49} & 
5.21\\SCA8-5 & 1071.46 & 1.36 & 
1091.91 & 1.50 & \bf{1027.08} & 
4.32\\SCA8-6 & 1013.70 & 1.62 & 
1025.38 & 1.25 & \bf{971.82} & 
4.31\\SCA8-7 & 1083.63 & 1.21 & 
1101.53 & 1.32 & \bf{1051.28} & 
3.08\\SCA8-8 & 1114.83 & 1.36 & 
1120.43 & 1.32 & \bf{1071.18} & 
4.07\\SCA8-9 & 1102.71 & 1.18 & 
1118.88 & 1.22 & \bf{1060.50} & 
3.98\\CON3-0 & 636.77 & 1.72 & 
639.11 & 1.68 & \bf{616.52} & 
3.28\\CON3-1 & 564.81 & 1.46 & 
566.50 & 1.46 & \bf{554.47} & 
1.86\\CON3-2 & 523.23 & 1.45 & 
529.21 & 1.70 & \bf{518.00} & 
1.01\\CON3-3 & 594.31 & 1.34 & 
613.26 & 1.34 & \bf{591.19} & 
0.53\\CON3-4 & 603.13 & 1.98 & 
609.41 & 1.61 & \bf{588.79} & 
2.44\\CON3-5 & 569.04 & 1.52 & 
586.28 & 1.57 & \bf{563.70} & 
0.95\\CON3-6 & 502.95 & 2.05 & 
507.77 & 1.80 & \bf{499.05} & 
0.78\\CON3-7 & 600.35 & 1.54 & 
607.71 & 1.48 & \bf{576.48} & 
4.14\\CON3-8 & 528.59 & 1.38 & 
533.48 & 1.61 & \bf{523.05} & 
1.06\\CON3-9 & 588.99 & 1.71 & 
596.03 & 1.38 & \bf{578.24} & 
1.86\\CON8-0 & 898.15 & 1.76 & 
928.87 & 1.43 & \bf{857.17} & 
4.78\\CON8-1 & 763.10 & 2.39 & 
772.95 & 1.82 & \bf{740.85} & 
3.00\\CON8-2 & 731.70 & 1.88 & 
737.10 & 1.66 & \bf{712.89} & 
2.64\\CON8-3 & 836.03 & 1.34 & 
850.09 & 1.31 & \bf{811.07} & 
3.08\\CON8-4 & 802.74 & 1.38 & 
824.89 & 1.43 & \bf{772.25} & 
3.95\\CON8-5 & 776.35 & 1.65 & 
802.90 & 1.55 & \bf{754.88} & 
2.84\\CON8-6 & 710.16 & 1.58 & 
718.62 & 1.39 & \bf{678.92} & 
4.60\\CON8-7 & 841.49 & 1.34 & 
844.21 & 1.36 & \bf{811.96} & 
3.64\\CON8-8 & 778.39 & 1.60 & 
798.16 & 1.52 & \bf{767.53} & 
1.41\\CON8-9 & 840.92 & 1.47 & 
865.62 & 1.57 & \bf{809.00} & 
3.95\\[1ex]\hline
\end{tabular}
\label{table:nonlin}
\end{table} \clearpage
\begin{table}[ht]
\caption{Resultados de la ejecución de la metaheurística ILS, utilizando instancias de SalhiNagy con la configuración -n 25.0 -LS 10.0}
\centering
\small
\begin{tabular}{c c c c c c c}
\hline\hline
Instancia & Costo mínimo & Tiempo(seg.) & Costo promedio & Tiempo promedio(seg.) & Costo ILS & \%Gap \\ [0.5ex]
\hline
CMT1X & 490.15 & 1.66 & 
494.18 & 1.33 & \bf{466.77} & 
5.01\\CMT1Y & 489.42 & 1.41 & 
492.26 & 1.55 & \bf{466.77} & 
4.85\\CMT2X & 704.22 & 3.82 & 
724.59 & 3.81 & \bf{684.21} & 
2.92\\CMT2Y & 703.82 & 3.96 & 
713.62 & 3.49 & \bf{684.21} & 
2.87\\CMT3X & 741.47 & 10.50 & 
745.74 & 10.06 & \bf{721.40} & 
2.78\\CMT3Y & 735.73 & 8.48 & 
739.81 & 10.60 & \bf{721.40} & 
1.99\\CMT4X & 892.45 & 43.86 & 
911.79 & 42.52 & \bf{852.83} & 
4.65\\CMT4Y & 902.31 & 31.57 & 
907.36 & 32.34 & \bf{852.46} & 
5.85\\CMT5X & 1102.21 & 73.59 & 
1109.08 & 85.14 & \bf{1030.55} & 
6.95\\CMT5Y & 1109.76 & 77.71 & 
1112.57 & 76.46 & \bf{1031.17} & 
7.62\\CMT11X & 890.23 & 24.01 & 
895.05 & 28.53 & \bf{839.39} & 
6.06\\CMT11Y & 888.47 & 24.83 & 
900.26 & 24.82 & \bf{841.88} & 
5.53\\CMT12X & 681.38 & 13.38 & 
688.92 & 11.01 & \bf{662.22} & 
2.89\\CMT12Y & 681.86 & 8.73 & 
691.88 & 9.89 & \bf{662.22} & 
2.97\\[1ex]\hline
\end{tabular}
\label{table:nonlin}
\end{table} \clearpage
\begin{table}[ht]
\caption{Resultados de la ejecución de la metaheurística ILS, utilizando instancias de Dethloff con la configuración -n 25.0 -LS 20.0}
\centering
\small
\begin{tabular}{c c c c c c c}
\hline\hline
Instancia & Costo mínimo & Tiempo(seg.) & Costo promedio & Tiempo promedio(seg.) & Costo ILS & \%Gap \\ [0.5ex]
\hline
SCA3-0 & 640.55 & 2.41 & 
642.67 & 2.39 & \bf{635.62} & 
0.78\\SCA3-1 & 712.59 & 2.59 & 
730.60 & 2.56 & \bf{697.84} & 
2.11\\SCA3-2 & 669.06 & 2.37 & 
674.86 & 2.09 & \bf{659.34} & 
1.47\\SCA3-3 & 680.60 & 2.16 & 
689.32 & 2.32 & \bf{680.04} & 
0.08\\SCA3-4 & \bf{690.50} & 2.26 & 
692.62 & 2.44 & 690.50 & 0.00\\
SCA3-5 & 670.02 & 2.22 & 
682.74 & 2.29 & \bf{659.90} & 
1.53\\SCA3-6 & 660.55 & 2.85 & 
666.62 & 2.71 & \bf{651.09} & 
1.45\\SCA3-7 & 671.77 & 2.36 & 
672.04 & 2.24 & \bf{659.17} & 
1.91\\SCA3-8 & 727.81 & 2.71 & 
735.28 & 2.27 & \bf{719.47} & 
1.16\\SCA3-9 & 689.95 & 2.20 & 
694.33 & 2.31 & \bf{681.00} & 
1.31\\SCA8-0 & 985.12 & 2.01 & 
1024.09 & 2.34 & \bf{961.50} & 
2.46\\SCA8-1 & 1081.37 & 2.43 & 
1090.61 & 2.08 & \bf{1049.65} & 
3.02\\SCA8-2 & 1070.14 & 2.04 & 
1082.03 & 1.80 & \bf{1039.64} & 
2.93\\SCA8-3 & 1010.82 & 1.96 & 
1026.39 & 1.88 & \bf{983.34} & 
2.79\\SCA8-4 & 1074.18 & 1.97 & 
1103.57 & 2.01 & \bf{1065.49} & 
0.82\\SCA8-5 & 1066.92 & 2.56 & 
1081.87 & 2.16 & \bf{1027.08} & 
3.88\\SCA8-6 & 1000.62 & 2.12 & 
1014.21 & 1.91 & \bf{971.82} & 
2.96\\SCA8-7 & 1121.88 & 1.84 & 
1132.35 & 2.06 & \bf{1051.28} & 
6.72\\SCA8-8 & \bf{1071.18} & 1.97 & 
1096.62 & 2.22 & 1071.18 & 0.00\\
SCA8-9 & 1103.55 & 1.97 & 
1109.80 & 1.96 & \bf{1060.50} & 
4.06\\CON3-0 & 630.73 & 1.99 & 
633.23 & 2.22 & \bf{616.52} & 
2.30\\CON3-1 & 562.52 & 2.66 & 
569.99 & 2.38 & \bf{554.47} & 
1.45\\CON3-2 & 528.77 & 2.18 & 
536.15 & 2.19 & \bf{518.00} & 
2.08\\CON3-3 & 594.10 & 2.06 & 
601.51 & 2.20 & \bf{591.19} & 
0.49\\CON3-4 & 603.60 & 2.55 & 
610.82 & 2.51 & \bf{588.79} & 
2.52\\CON3-5 & 575.81 & 2.30 & 
581.30 & 2.30 & \bf{563.70} & 
2.15\\CON3-6 & 510.81 & 2.63 & 
513.96 & 2.48 & \bf{499.05} & 
2.36\\CON3-7 & 592.77 & 2.26 & 
599.76 & 2.30 & \bf{576.48} & 
2.83\\CON3-8 & 532.86 & 2.29 & 
541.09 & 2.49 & \bf{523.05} & 
1.88\\CON3-9 & 588.48 & 2.42 & 
588.96 & 2.19 & \bf{578.24} & 
1.77\\CON8-0 & 890.52 & 2.21 & 
904.29 & 1.91 & \bf{857.17} & 
3.89\\CON8-1 & 767.37 & 2.63 & 
777.86 & 2.08 & \bf{740.85} & 
3.58\\CON8-2 & 727.95 & 2.30 & 
730.27 & 2.29 & \bf{712.89} & 
2.11\\CON8-3 & 845.95 & 2.03 & 
851.84 & 1.89 & \bf{811.07} & 
4.30\\CON8-4 & 772.73 & 2.24 & 
795.42 & 2.13 & \bf{772.25} & 
0.06\\CON8-5 & 778.93 & 1.76 & 
814.13 & 1.90 & \bf{754.88} & 
3.19\\CON8-6 & 701.44 & 1.87 & 
709.59 & 2.07 & \bf{678.92} & 
3.32\\CON8-7 & 816.67 & 2.19 & 
830.79 & 2.16 & \bf{811.96} & 
0.58\\CON8-8 & 793.89 & 2.22 & 
800.81 & 1.90 & \bf{767.53} & 
3.43\\CON8-9 & 835.56 & 2.34 & 
841.42 & 1.95 & \bf{809.00} & 
3.28\\[1ex]\hline
\end{tabular}
\label{table:nonlin}
\end{table} \clearpage
\begin{table}[ht]
\caption{Resultados de la ejecución de la metaheurística ILS, utilizando instancias de SalhiNagy con la configuración -n 25.0 -LS 20.0}
\centering
\small
\begin{tabular}{c c c c c c c}
\hline\hline
Instancia & Costo mínimo & Tiempo(seg.) & Costo promedio & Tiempo promedio(seg.) & Costo ILS & \%Gap \\ [0.5ex]
\hline
CMT1X & 480.08 & 1.68 & 
484.31 & 1.92 & \bf{466.77} & 
2.85\\CMT1Y & 488.44 & 2.14 & 
489.74 & 2.29 & \bf{466.77} & 
4.64\\CMT2X & 716.26 & 5.58 & 
723.76 & 5.10 & \bf{684.21} & 
4.68\\CMT2Y & 713.09 & 5.80 & 
717.26 & 5.53 & \bf{684.21} & 
4.22\\CMT3X & 731.72 & 12.10 & 
742.85 & 13.11 & \bf{721.40} & 
1.43\\CMT3Y & 738.54 & 12.26 & 
740.61 & 12.84 & \bf{721.40} & 
2.38\\CMT4X & 889.67 & 38.84 & 
901.94 & 40.45 & \bf{852.83} & 
4.32\\CMT4Y & 899.13 & 55.79 & 
903.86 & 45.58 & \bf{852.46} & 
5.47\\CMT5X & 1105.10 & 134.82 & 
1120.84 & 97.94 & \bf{1030.55} & 
7.23\\CMT5Y & 1107.39 & 81.12 & 
1117.60 & 94.31 & \bf{1031.17} & 
7.39\\CMT11X & 885.37 & 30.56 & 
907.00 & 32.92 & \bf{839.39} & 
5.48\\CMT11Y & 888.07 & 29.10 & 
906.77 & 31.89 & \bf{841.88} & 
5.49\\CMT12X & 681.36 & 10.46 & 
686.10 & 12.89 & \bf{662.22} & 
2.89\\CMT12Y & 677.05 & 11.45 & 
686.04 & 12.24 & \bf{662.22} & 
2.24\\[1ex]\hline
\end{tabular}
\label{table:nonlin}
\end{table} \clearpage
\begin{table}[ht]
\caption{Resultados de la ejecución de la metaheurística ILS, utilizando instancias de Dethloff con la configuración -n 25.0 -LS 30.0}
\centering
\small
\begin{tabular}{c c c c c c c}
\hline\hline
Instancia & Costo mínimo & Tiempo(seg.) & Costo promedio & Tiempo promedio(seg.) & Costo ILS & \%Gap \\ [0.5ex]
\hline
SCA3-0 & 641.69 & 3.38 & 
643.50 & 3.12 & \bf{635.62} & 
0.95\\SCA3-1 & 701.53 & 2.99 & 
706.08 & 3.09 & \bf{697.84} & 
0.53\\SCA3-2 & 666.05 & 3.72 & 
667.62 & 3.29 & \bf{659.34} & 
1.02\\SCA3-3 & 681.74 & 3.59 & 
686.59 & 3.43 & \bf{680.04} & 
0.25\\SCA3-4 & \bf{690.50} & 2.91 & 
691.18 & 3.05 & 690.50 & 0.00\\
SCA3-5 & 678.64 & 3.15 & 
682.84 & 3.47 & \bf{659.90} & 
2.84\\SCA3-6 & \bf{651.09} & 2.58 & 
653.41 & 3.33 & 651.09 & 0.00\\
SCA3-7 & 671.67 & 3.03 & 
672.78 & 2.92 & \bf{659.17} & 
1.90\\SCA3-8 & 721.45 & 3.58 & 
728.50 & 3.28 & \bf{719.47} & 
0.28\\SCA3-9 & \bf{681.00} & 2.78 & 
686.26 & 2.73 & 681.00 & 0.00\\
SCA8-0 & 1011.47 & 2.72 & 
1028.62 & 2.59 & \bf{961.50} & 
5.20\\SCA8-1 & 1068.14 & 2.45 & 
1088.68 & 2.79 & \bf{1049.65} & 
1.76\\SCA8-2 & 1065.44 & 2.63 & 
1076.08 & 2.85 & \bf{1039.64} & 
2.48\\SCA8-3 & 1024.48 & 2.21 & 
1033.79 & 2.46 & \bf{983.34} & 
4.18\\SCA8-4 & 1124.66 & 2.58 & 
1138.17 & 2.59 & \bf{1065.49} & 
5.55\\SCA8-5 & 1066.92 & 2.46 & 
1081.50 & 2.73 & \bf{1027.08} & 
3.88\\SCA8-6 & 996.13 & 3.22 & 
1004.44 & 2.75 & \bf{971.82} & 
2.50\\SCA8-7 & 1079.57 & 2.16 & 
1098.16 & 2.69 & \bf{1051.28} & 
2.69\\SCA8-8 & 1084.57 & 2.97 & 
1103.40 & 3.04 & \bf{1071.18} & 
1.25\\SCA8-9 & 1080.03 & 3.46 & 
1092.77 & 2.70 & \bf{1060.50} & 
1.84\\CON3-0 & 636.88 & 3.32 & 
639.89 & 3.37 & \bf{616.52} & 
3.30\\CON3-1 & 564.64 & 3.62 & 
568.80 & 3.52 & \bf{554.47} & 
1.83\\CON3-2 & 521.63 & 3.63 & 
522.88 & 3.28 & \bf{518.00} & 
0.70\\CON3-3 & 591.48 & 3.50 & 
601.76 & 3.58 & \bf{591.19} & 
0.05\\CON3-4 & 597.27 & 3.09 & 
606.82 & 3.07 & \bf{588.79} & 
1.44\\CON3-5 & 569.04 & 3.74 & 
570.70 & 3.23 & \bf{563.70} & 
0.95\\CON3-6 & 510.49 & 3.09 & 
515.14 & 3.06 & \bf{499.05} & 
2.29\\CON3-7 & 578.41 & 3.15 & 
584.40 & 2.96 & \bf{576.48} & 
0.33\\CON3-8 & 524.30 & 4.15 & 
526.02 & 3.53 & \bf{523.05} & 
0.24\\CON3-9 & 578.98 & 2.87 & 
587.66 & 3.41 & \bf{578.24} & 
0.13\\CON8-0 & 883.76 & 2.94 & 
902.27 & 2.90 & \bf{857.17} & 
3.10\\CON8-1 & 779.35 & 2.16 & 
788.96 & 2.80 & \bf{740.85} & 
5.20\\CON8-2 & 731.50 & 3.13 & 
736.60 & 2.78 & \bf{712.89} & 
2.61\\CON8-3 & 844.07 & 2.49 & 
865.07 & 2.58 & \bf{811.07} & 
4.07\\CON8-4 & 805.62 & 1.99 & 
818.17 & 2.90 & \bf{772.25} & 
4.32\\CON8-5 & 764.15 & 2.88 & 
782.40 & 2.78 & \bf{754.88} & 
1.23\\CON8-6 & 718.10 & 4.23 & 
726.65 & 3.31 & \bf{678.92} & 
5.77\\CON8-7 & 825.53 & 2.28 & 
836.62 & 2.60 & \bf{811.96} & 
1.67\\CON8-8 & 784.18 & 2.30 & 
791.27 & 2.54 & \bf{767.53} & 
2.17\\CON8-9 & 819.34 & 2.59 & 
842.11 & 2.79 & \bf{809.00} & 
1.28\\[1ex]\hline
\end{tabular}
\label{table:nonlin}
\end{table} \clearpage
\begin{table}[ht]
\caption{Resultados de la ejecución de la metaheurística ILS, utilizando instancias de SalhiNagy con la configuración -n 25.0 -LS 30.0}
\centering
\small
\begin{tabular}{c c c c c c c}
\hline\hline
Instancia & Costo mínimo & Tiempo(seg.) & Costo promedio & Tiempo promedio(seg.) & Costo ILS & \%Gap \\ [0.5ex]
\hline
CMT1X & 470.67 & 2.84 & 
475.77 & 2.94 & \bf{466.77} & 
0.84\\CMT1Y & 475.37 & 3.02 & 
488.05 & 3.22 & \bf{466.77} & 
1.84\\CMT2X & 710.84 & 6.51 & 
720.15 & 6.58 & \bf{684.21} & 
3.89\\CMT2Y & 705.23 & 7.23 & 
715.59 & 6.71 & \bf{684.21} & 
3.07\\CMT3X & 726.96 & 19.64 & 
731.96 & 17.45 & \bf{721.40} & 
0.77\\CMT3Y & 732.48 & 19.61 & 
742.38 & 18.11 & \bf{721.40} & 
1.54\\CMT4X & 900.50 & 41.08 & 
914.13 & 43.97 & \bf{852.83} & 
5.59\\CMT4Y & 899.98 & 44.90 & 
908.57 & 48.31 & \bf{852.46} & 
5.57\\CMT5X & 1092.74 & 101.37 & 
1103.72 & 117.66 & \bf{1030.55} & 
6.03\\CMT5Y & 1099.14 & 97.37 & 
1112.46 & 101.02 & \bf{1031.17} & 
6.59\\CMT11X & 882.82 & 30.56 & 
892.04 & 36.93 & \bf{839.39} & 
5.17\\CMT11Y & 880.05 & 44.79 & 
888.14 & 42.83 & \bf{841.88} & 
4.53\\CMT12X & 683.24 & 13.06 & 
687.65 & 13.37 & \bf{662.22} & 
3.17\\CMT12Y & 674.11 & 17.56 & 
681.28 & 15.33 & \bf{662.22} & 
1.80\\[1ex]\hline
\end{tabular}
\label{table:nonlin}
\end{table} \clearpage
\begin{table}[ht]
\caption{Resultados de la ejecución de la metaheurística ILS, utilizando instancias de Dethloff con la configuración -n 25.0 -LS 40.0}
\centering
\small
\begin{tabular}{c c c c c c c}
\hline\hline
Instancia & Costo mínimo & Tiempo(seg.) & Costo promedio & Tiempo promedio(seg.) & Costo ILS & \%Gap \\ [0.5ex]
\hline
SCA3-0 & 641.69 & 4.63 & 
642.63 & 4.25 & \bf{635.62} & 
0.95\\SCA3-1 & 700.50 & 4.10 & 
709.55 & 4.01 & \bf{697.84} & 
0.38\\SCA3-2 & 664.21 & 4.28 & 
672.85 & 3.95 & \bf{659.34} & 
0.74\\SCA3-3 & 681.74 & 3.90 & 
685.25 & 3.65 & \bf{680.04} & 
0.25\\SCA3-4 & \bf{690.50} & 3.94 & 
692.22 & 4.13 & 690.50 & 0.00\\
SCA3-5 & 680.80 & 4.25 & 
686.26 & 4.26 & \bf{659.90} & 
3.17\\SCA3-6 & 652.47 & 3.46 & 
654.19 & 3.58 & \bf{651.09} & 
0.21\\SCA3-7 & 671.77 & 3.50 & 
676.11 & 3.55 & \bf{659.17} & 
1.91\\SCA3-8 & \bf{719.47} & 3.61 & 
730.11 & 3.52 & 719.47 & 0.00\\
SCA3-9 & \bf{681.00} & 4.48 & 
690.88 & 4.03 & 681.00 & 0.00\\
SCA8-0 & 981.73 & 3.74 & 
1005.63 & 3.87 & \bf{961.50} & 
2.10\\SCA8-1 & 1085.54 & 3.06 & 
1086.60 & 3.27 & \bf{1049.65} & 
3.42\\SCA8-2 & 1062.62 & 5.10 & 
1067.91 & 3.54 & \bf{1039.64} & 
2.21\\SCA8-3 & 1032.69 & 3.24 & 
1040.32 & 3.33 & \bf{983.34} & 
5.02\\SCA8-4 & 1074.28 & 3.17 & 
1112.38 & 3.25 & \bf{1065.49} & 
0.82\\SCA8-5 & 1049.98 & 3.55 & 
1072.99 & 3.35 & \bf{1027.08} & 
2.23\\SCA8-6 & 991.49 & 3.53 & 
998.24 & 2.87 & \bf{971.82} & 
2.02\\SCA8-7 & 1066.65 & 4.56 & 
1093.26 & 3.54 & \bf{1051.28} & 
1.46\\SCA8-8 & 1088.20 & 3.50 & 
1104.36 & 3.13 & \bf{1071.18} & 
1.59\\SCA8-9 & 1119.56 & 3.74 & 
1130.42 & 3.09 & \bf{1060.50} & 
5.57\\CON3-0 & 617.59 & 4.37 & 
631.20 & 4.28 & \bf{616.52} & 
0.17\\CON3-1 & 560.61 & 3.27 & 
564.99 & 3.62 & \bf{554.47} & 
1.11\\CON3-2 & 524.13 & 4.51 & 
528.31 & 4.25 & \bf{518.00} & 
1.18\\CON3-3 & 591.48 & 4.85 & 
600.22 & 4.17 & \bf{591.19} & 
0.05\\CON3-4 & 599.13 & 4.19 & 
601.95 & 4.05 & \bf{588.79} & 
1.76\\CON3-5 & 569.04 & 4.48 & 
574.53 & 4.47 & \bf{563.70} & 
0.95\\CON3-6 & 505.14 & 4.36 & 
511.65 & 4.10 & \bf{499.05} & 
1.22\\CON3-7 & 586.53 & 3.84 & 
598.45 & 3.92 & \bf{576.48} & 
1.74\\CON3-8 & 524.59 & 4.96 & 
532.31 & 4.20 & \bf{523.05} & 
0.29\\CON3-9 & 589.57 & 3.64 & 
591.51 & 4.06 & \bf{578.24} & 
1.96\\CON8-0 & 872.79 & 3.52 & 
894.27 & 3.54 & \bf{857.17} & 
1.82\\CON8-1 & 773.48 & 3.21 & 
777.26 & 3.63 & \bf{740.85} & 
4.40\\CON8-2 & 722.56 & 3.74 & 
732.73 & 3.62 & \bf{712.89} & 
1.36\\CON8-3 & 845.27 & 3.21 & 
848.23 & 3.18 & \bf{811.07} & 
4.22\\CON8-4 & 813.27 & 2.85 & 
825.95 & 2.63 & \bf{772.25} & 
5.31\\CON8-5 & 767.45 & 4.07 & 
775.88 & 3.61 & \bf{754.88} & 
1.67\\CON8-6 & 703.00 & 3.05 & 
710.01 & 3.77 & \bf{678.92} & 
3.55\\CON8-7 & 816.07 & 4.33 & 
829.87 & 3.35 & \bf{811.96} & 
0.51\\CON8-8 & 787.24 & 4.46 & 
800.91 & 3.83 & \bf{767.53} & 
2.57\\CON8-9 & 823.94 & 4.56 & 
849.08 & 4.19 & \bf{809.00} & 
1.85\\[1ex]\hline
\end{tabular}
\label{table:nonlin}
\end{table} \clearpage
\begin{table}[ht]
\caption{Resultados de la ejecución de la metaheurística ILS, utilizando instancias de SalhiNagy con la configuración -n 25.0 -LS 40.0}
\centering
\small
\begin{tabular}{c c c c c c c}
\hline\hline
Instancia & Costo mínimo & Tiempo(seg.) & Costo promedio & Tiempo promedio(seg.) & Costo ILS & \%Gap \\ [0.5ex]
\hline
CMT1X & 474.87 & 3.20 & 
484.67 & 3.12 & \bf{466.77} & 
1.74\\CMT1Y & 483.62 & 3.31 & 
486.94 & 3.52 & \bf{466.77} & 
3.61\\CMT2X & 700.95 & 8.79 & 
712.74 & 9.11 & \bf{684.21} & 
2.45\\CMT2Y & 702.20 & 7.48 & 
712.05 & 7.17 & \bf{684.21} & 
2.63\\CMT3X & 731.81 & 18.62 & 
738.23 & 18.23 & \bf{721.40} & 
1.44\\CMT3Y & 728.82 & 16.50 & 
736.32 & 17.63 & \bf{721.40} & 
1.03\\CMT4X & 897.65 & 60.45 & 
907.02 & 58.60 & \bf{852.83} & 
5.26\\CMT4Y & 889.25 & 48.09 & 
898.76 & 47.65 & \bf{852.46} & 
4.32\\CMT5X & 1114.50 & 105.38 & 
1119.56 & 112.93 & \bf{1030.55} & 
8.15\\CMT5Y & 1113.74 & 115.56 & 
1118.84 & 126.28 & \bf{1031.17} & 
8.01\\CMT11X & 887.06 & 36.75 & 
894.50 & 39.01 & \bf{839.39} & 
5.68\\CMT11Y & 856.44 & 38.01 & 
889.32 & 37.06 & \bf{841.88} & 
1.73\\CMT12X & 676.14 & 16.58 & 
680.85 & 17.56 & \bf{662.22} & 
2.10\\CMT12Y & 680.75 & 15.81 & 
685.56 & 16.95 & \bf{662.22} & 
2.80\\[1ex]\hline
\end{tabular}
\label{table:nonlin}
\end{table} \clearpage
\begin{table}[ht]
\caption{Resultados de la ejecución de la metaheurística ILS, utilizando instancias de Dethloff con la configuración -n 25.0 -LS 50.0}
\centering
\small
\begin{tabular}{c c c c c c c}
\hline\hline
Instancia & Costo mínimo & Tiempo(seg.) & Costo promedio & Tiempo promedio(seg.) & Costo ILS & \%Gap \\ [0.5ex]
\hline
SCA3-0 & 640.55 & 5.43 & 
643.35 & 5.23 & \bf{635.62} & 
0.78\\SCA3-1 & 707.07 & 5.56 & 
709.41 & 4.61 & \bf{697.84} & 
1.32\\SCA3-2 & 661.13 & 5.54 & 
663.98 & 4.70 & \bf{659.34} & 
0.27\\SCA3-3 & 680.60 & 4.60 & 
686.99 & 4.58 & \bf{680.04} & 
0.08\\SCA3-4 & \bf{690.50} & 4.88 & 
710.73 & 4.95 & 690.50 & 0.00\\
SCA3-5 & 665.04 & 5.46 & 
681.04 & 4.74 & \bf{659.90} & 
0.78\\SCA3-6 & 652.47 & 3.88 & 
654.36 & 4.50 & \bf{651.09} & 
0.21\\SCA3-7 & 671.77 & 4.16 & 
674.56 & 4.67 & \bf{659.17} & 
1.91\\SCA3-8 & 719.77 & 4.04 & 
726.91 & 4.68 & \bf{719.47} & 
0.04\\SCA3-9 & 684.44 & 4.60 & 
689.21 & 4.41 & \bf{681.00} & 
0.51\\SCA8-0 & 980.30 & 4.69 & 
1010.69 & 4.41 & \bf{961.50} & 
1.96\\SCA8-1 & 1073.16 & 4.22 & 
1084.70 & 4.16 & \bf{1049.65} & 
2.24\\SCA8-2 & 1067.85 & 3.36 & 
1082.73 & 3.70 & \bf{1039.64} & 
2.71\\SCA8-3 & 1016.06 & 4.00 & 
1030.65 & 3.73 & \bf{983.34} & 
3.33\\SCA8-4 & 1069.87 & 5.32 & 
1092.81 & 4.15 & \bf{1065.49} & 
0.41\\SCA8-5 & 1063.11 & 3.66 & 
1091.71 & 3.60 & \bf{1027.08} & 
3.51\\SCA8-6 & 993.53 & 3.98 & 
1005.38 & 4.04 & \bf{971.82} & 
2.23\\SCA8-7 & 1102.85 & 3.62 & 
1118.36 & 3.78 & \bf{1051.28} & 
4.91\\SCA8-8 & 1087.01 & 4.80 & 
1094.19 & 3.77 & \bf{1071.18} & 
1.48\\SCA8-9 & 1105.18 & 3.10 & 
1120.10 & 3.15 & \bf{1060.50} & 
4.21\\CON3-0 & 630.73 & 4.58 & 
640.84 & 5.03 & \bf{616.52} & 
2.30\\CON3-1 & 562.52 & 4.99 & 
565.22 & 5.41 & \bf{554.47} & 
1.45\\CON3-2 & 523.23 & 5.80 & 
525.95 & 4.71 & \bf{518.00} & 
1.01\\CON3-3 & 610.74 & 4.13 & 
618.97 & 4.52 & \bf{591.19} & 
3.31\\CON3-4 & 593.78 & 4.98 & 
595.92 & 4.80 & \bf{588.79} & 
0.85\\CON3-5 & 570.22 & 4.79 & 
572.23 & 4.56 & \bf{563.70} & 
1.16\\CON3-6 & 504.15 & 5.34 & 
510.82 & 5.10 & \bf{499.05} & 
1.02\\CON3-7 & 585.94 & 4.68 & 
593.67 & 4.45 & \bf{576.48} & 
1.64\\CON3-8 & 523.68 & 6.61 & 
530.25 & 5.36 & \bf{523.05} & 
0.12\\CON3-9 & 588.99 & 4.44 & 
595.07 & 4.49 & \bf{578.24} & 
1.86\\CON8-0 & 874.47 & 4.94 & 
900.56 & 4.11 & \bf{857.17} & 
2.02\\CON8-1 & 761.33 & 3.48 & 
774.79 & 3.70 & \bf{740.85} & 
2.76\\CON8-2 & 733.10 & 4.47 & 
738.47 & 3.75 & \bf{712.89} & 
2.83\\CON8-3 & 842.69 & 4.79 & 
847.35 & 4.43 & \bf{811.07} & 
3.90\\CON8-4 & 808.09 & 3.86 & 
818.74 & 3.77 & \bf{772.25} & 
4.64\\CON8-5 & 770.56 & 4.09 & 
789.18 & 3.92 & \bf{754.88} & 
2.08\\CON8-6 & 708.69 & 4.20 & 
711.39 & 4.29 & \bf{678.92} & 
4.38\\CON8-7 & 835.96 & 3.40 & 
846.83 & 4.03 & \bf{811.96} & 
2.96\\CON8-8 & 784.43 & 4.49 & 
806.84 & 3.68 & \bf{767.53} & 
2.20\\CON8-9 & 815.11 & 4.61 & 
829.53 & 4.19 & \bf{809.00} & 
0.76\\[1ex]\hline
\end{tabular}
\label{table:nonlin}
\end{table} \clearpage
\begin{table}[ht]
\caption{Resultados de la ejecución de la metaheurística ILS, utilizando instancias de SalhiNagy con la configuración -n 25.0 -LS 50.0}
\centering
\small
\begin{tabular}{c c c c c c c}
\hline\hline
Instancia & Costo mínimo & Tiempo(seg.) & Costo promedio & Tiempo promedio(seg.) & Costo ILS & \%Gap \\ [0.5ex]
\hline
CMT1X & 475.58 & 3.55 & 
480.69 & 3.47 & \bf{466.77} & 
1.89\\CMT1Y & 478.54 & 4.95 & 
489.57 & 4.04 & \bf{466.77} & 
2.52\\CMT2X & 709.51 & 8.45 & 
716.02 & 9.20 & \bf{684.21} & 
3.70\\CMT2Y & 709.97 & 10.26 & 
714.93 & 8.85 & \bf{684.21} & 
3.76\\CMT3X & 728.82 & 19.54 & 
742.10 & 19.73 & \bf{721.40} & 
1.03\\CMT3Y & 729.45 & 24.63 & 
736.70 & 21.90 & \bf{721.40} & 
1.12\\CMT4X & 905.49 & 52.92 & 
912.29 & 62.55 & \bf{852.83} & 
6.17\\CMT4Y & 900.75 & 54.95 & 
909.07 & 60.89 & \bf{852.46} & 
5.66\\CMT5X & 1107.29 & 126.22 & 
1120.72 & 140.26 & \bf{1030.55} & 
7.45\\CMT5Y & 1114.55 & 128.69 & 
1117.46 & 133.17 & \bf{1031.17} & 
8.09\\CMT11X & 883.35 & 56.40 & 
901.88 & 45.45 & \bf{839.39} & 
5.24\\CMT11Y & 882.11 & 51.88 & 
890.33 & 46.38 & \bf{841.88} & 
4.78\\CMT12X & 676.38 & 18.48 & 
684.46 & 20.20 & \bf{662.22} & 
2.14\\CMT12Y & 679.40 & 19.25 & 
682.63 & 18.06 & \bf{662.22} & 
2.59\\[1ex]\hline
\end{tabular}
\label{table:nonlin}
\end{table} \clearpage
\begin{table}[ht]
\caption{Resultados de la ejecución de la metaheurística ILS, utilizando instancias de Dethloff con la configuración -n 25.0 -LS 60.0}
\centering
\small
\begin{tabular}{c c c c c c c}
\hline\hline
Instancia & Costo mínimo & Tiempo(seg.) & Costo promedio & Tiempo promedio(seg.) & Costo ILS & \%Gap \\ [0.5ex]
\hline
SCA3-0 & 640.55 & 5.06 & 
642.42 & 5.00 & \bf{635.62} & 
0.78\\SCA3-1 & \bf{697.84} & 6.75 & 
703.99 & 6.09 & 697.84 & 0.00\\
SCA3-2 & 664.18 & 4.83 & 
667.45 & 5.53 & \bf{659.34} & 
0.73\\SCA3-3 & 681.74 & 6.49 & 
688.12 & 6.00 & \bf{680.04} & 
0.25\\SCA3-4 & 692.57 & 5.18 & 
697.76 & 5.15 & \bf{690.50} & 
0.30\\SCA3-5 & 678.22 & 5.63 & 
679.89 & 5.78 & \bf{659.90} & 
2.78\\SCA3-6 & \bf{651.09} & 6.51 & 
657.82 & 5.66 & 651.09 & 0.00\\
SCA3-7 & 671.67 & 5.38 & 
673.50 & 5.26 & \bf{659.17} & 
1.90\\SCA3-8 & 724.29 & 5.23 & 
737.58 & 5.65 & \bf{719.47} & 
0.67\\SCA3-9 & \bf{681.00} & 5.22 & 
684.58 & 5.00 & 681.00 & 0.00\\
SCA8-0 & 1020.44 & 4.86 & 
1038.36 & 5.37 & \bf{961.50} & 
6.13\\SCA8-1 & 1071.23 & 4.20 & 
1086.69 & 4.45 & \bf{1049.65} & 
2.06\\SCA8-2 & 1062.04 & 3.49 & 
1065.39 & 3.88 & \bf{1039.64} & 
2.15\\SCA8-3 & 1010.76 & 5.34 & 
1030.97 & 5.03 & \bf{983.34} & 
2.79\\SCA8-4 & 1074.18 & 4.91 & 
1088.14 & 4.21 & \bf{1065.49} & 
0.82\\SCA8-5 & 1068.16 & 6.72 & 
1087.05 & 4.97 & \bf{1027.08} & 
4.00\\SCA8-6 & 992.65 & 5.05 & 
1006.57 & 5.25 & \bf{971.82} & 
2.14\\SCA8-7 & 1085.42 & 4.96 & 
1091.39 & 4.97 & \bf{1051.28} & 
3.25\\SCA8-8 & 1092.02 & 4.13 & 
1101.21 & 4.68 & \bf{1071.18} & 
1.95\\SCA8-9 & 1084.08 & 4.48 & 
1092.67 & 4.34 & \bf{1060.50} & 
2.22\\CON3-0 & 633.24 & 5.88 & 
635.48 & 6.09 & \bf{616.52} & 
2.71\\CON3-1 & 556.92 & 5.60 & 
562.29 & 5.99 & \bf{554.47} & 
0.44\\CON3-2 & 521.38 & 5.42 & 
527.18 & 5.72 & \bf{518.00} & 
0.65\\CON3-3 & 591.20 & 5.92 & 
592.36 & 5.86 & \bf{591.19} & 
0.00\\CON3-4 & 591.43 & 6.45 & 
596.05 & 5.59 & \bf{588.79} & 
0.45\\CON3-5 & 564.88 & 4.84 & 
572.09 & 5.50 & \bf{563.70} & 
0.21\\CON3-6 & 503.95 & 6.27 & 
515.74 & 5.24 & \bf{499.05} & 
0.98\\CON3-7 & 578.41 & 6.68 & 
589.28 & 6.76 & \bf{576.48} & 
0.33\\CON3-8 & 524.59 & 6.28 & 
530.68 & 5.95 & \bf{523.05} & 
0.29\\CON3-9 & 588.63 & 5.98 & 
590.36 & 5.66 & \bf{578.24} & 
1.80\\CON8-0 & 885.11 & 4.65 & 
906.38 & 5.03 & \bf{857.17} & 
3.26\\CON8-1 & 751.84 & 5.40 & 
761.27 & 6.37 & \bf{740.85} & 
1.48\\CON8-2 & 720.12 & 6.70 & 
728.41 & 5.30 & \bf{712.89} & 
1.01\\CON8-3 & 831.87 & 5.92 & 
845.35 & 4.86 & \bf{811.07} & 
2.56\\CON8-4 & 801.31 & 4.19 & 
816.39 & 4.47 & \bf{772.25} & 
3.76\\CON8-5 & 758.12 & 4.99 & 
773.88 & 4.99 & \bf{754.88} & 
0.43\\CON8-6 & 702.57 & 5.55 & 
711.82 & 5.56 & \bf{678.92} & 
3.48\\CON8-7 & 827.28 & 5.66 & 
844.78 & 5.17 & \bf{811.96} & 
1.89\\CON8-8 & 798.06 & 5.42 & 
803.16 & 4.81 & \bf{767.53} & 
3.98\\CON8-9 & 817.87 & 3.46 & 
832.90 & 4.33 & \bf{809.00} & 
1.10\\[1ex]\hline
\end{tabular}
\label{table:nonlin}
\end{table} \clearpage
\begin{table}[ht]
\caption{Resultados de la ejecución de la metaheurística ILS, utilizando instancias de SalhiNagy con la configuración -n 25.0 -LS 60.0}
\centering
\small
\begin{tabular}{c c c c c c c}
\hline\hline
Instancia & Costo mínimo & Tiempo(seg.) & Costo promedio & Tiempo promedio(seg.) & Costo ILS & \%Gap \\ [0.5ex]
\hline
CMT1X & 470.67 & 4.70 & 
475.33 & 4.46 & \bf{466.77} & 
0.84\\CMT1Y & 476.32 & 3.53 & 
485.00 & 3.94 & \bf{466.77} & 
2.05\\CMT2X & 719.00 & 9.02 & 
720.37 & 9.96 & \bf{684.21} & 
5.08\\CMT2Y & 704.49 & 10.22 & 
709.54 & 10.14 & \bf{684.21} & 
2.96\\CMT3X & 738.80 & 22.58 & 
744.17 & 23.48 & \bf{721.40} & 
2.41\\CMT3Y & 729.20 & 27.21 & 
737.64 & 25.27 & \bf{721.40} & 
1.08\\CMT4X & 900.67 & 61.23 & 
906.35 & 60.92 & \bf{852.83} & 
5.61\\CMT4Y & 900.03 & 85.55 & 
907.50 & 71.91 & \bf{852.46} & 
5.58\\CMT5X & 1104.11 & 150.61 & 
1110.77 & 150.23 & \bf{1030.55} & 
7.14\\CMT5Y & 1098.76 & 147.48 & 
1103.68 & 168.17 & \bf{1031.17} & 
6.55\\CMT11X & 880.32 & 55.62 & 
889.77 & 51.03 & \bf{839.39} & 
4.88\\CMT11Y & 890.05 & 60.99 & 
903.24 & 50.20 & \bf{841.88} & 
5.72\\CMT12X & 683.03 & 22.69 & 
685.32 & 22.77 & \bf{662.22} & 
3.14\\CMT12Y & 681.96 & 18.42 & 
683.26 & 21.20 & \bf{662.22} & 
2.98\\[1ex]\hline
\end{tabular}
\label{table:nonlin}
\end{table} \clearpage
\begin{table}[ht]
\caption{Resultados de la ejecución de la metaheurística ILS, utilizando instancias de Dethloff con la configuración -n 25.0 -LS 70.0}
\centering
\small
\begin{tabular}{c c c c c c c}
\hline\hline
Instancia & Costo mínimo & Tiempo(seg.) & Costo promedio & Tiempo promedio(seg.) & Costo ILS & \%Gap \\ [0.5ex]
\hline
SCA3-0 & 640.55 & 6.50 & 
642.00 & 6.48 & \bf{635.62} & 
0.78\\SCA3-1 & 700.50 & 5.67 & 
701.34 & 6.39 & \bf{697.84} & 
0.38\\SCA3-2 & 668.28 & 6.17 & 
676.70 & 6.05 & \bf{659.34} & 
1.36\\SCA3-3 & 681.74 & 6.30 & 
684.20 & 6.52 & \bf{680.04} & 
0.25\\SCA3-4 & \bf{690.50} & 6.95 & 
691.53 & 6.39 & 690.50 & 0.00\\
SCA3-5 & 661.07 & 6.88 & 
676.80 & 6.29 & \bf{659.90} & 
0.18\\SCA3-6 & \bf{651.09} & 5.87 & 
652.81 & 6.37 & 651.09 & 0.00\\
SCA3-7 & 671.77 & 5.99 & 
672.83 & 6.12 & \bf{659.17} & 
1.91\\SCA3-8 & 727.64 & 6.96 & 
731.02 & 6.23 & \bf{719.47} & 
1.14\\SCA3-9 & \bf{681.00} & 6.69 & 
686.16 & 6.22 & 681.00 & 0.00\\
SCA8-0 & 1004.83 & 6.18 & 
1015.99 & 5.32 & \bf{961.50} & 
4.51\\SCA8-1 & 1066.73 & 5.42 & 
1079.99 & 5.66 & \bf{1049.65} & 
1.63\\SCA8-2 & 1062.47 & 6.14 & 
1072.18 & 5.41 & \bf{1039.64} & 
2.20\\SCA8-3 & 1011.09 & 5.51 & 
1022.09 & 4.94 & \bf{983.34} & 
2.82\\SCA8-4 & 1096.60 & 6.17 & 
1102.88 & 5.68 & \bf{1065.49} & 
2.92\\SCA8-5 & 1065.60 & 5.54 & 
1073.88 & 5.49 & \bf{1027.08} & 
3.75\\SCA8-6 & 993.19 & 5.53 & 
1001.61 & 5.12 & \bf{971.82} & 
2.20\\SCA8-7 & 1089.45 & 4.63 & 
1102.95 & 5.19 & \bf{1051.28} & 
3.63\\SCA8-8 & 1092.82 & 6.61 & 
1107.96 & 5.33 & \bf{1071.18} & 
2.02\\SCA8-9 & 1081.51 & 5.34 & 
1093.93 & 5.40 & \bf{1060.50} & 
1.98\\CON3-0 & 620.76 & 5.70 & 
631.13 & 5.95 & \bf{616.52} & 
0.69\\CON3-1 & 556.04 & 6.20 & 
557.55 & 6.08 & \bf{554.47} & 
0.28\\CON3-2 & 521.63 & 6.66 & 
528.12 & 6.38 & \bf{518.00} & 
0.70\\CON3-3 & 599.26 & 6.28 & 
604.55 & 6.48 & \bf{591.19} & 
1.37\\CON3-4 & 591.43 & 6.60 & 
597.55 & 6.65 & \bf{588.79} & 
0.45\\CON3-5 & 573.23 & 7.94 & 
576.64 & 6.24 & \bf{563.70} & 
1.69\\CON3-6 & 504.20 & 7.02 & 
506.30 & 6.50 & \bf{499.05} & 
1.03\\CON3-7 & 591.91 & 5.98 & 
599.38 & 6.80 & \bf{576.48} & 
2.68\\CON3-8 & 524.59 & 5.91 & 
534.28 & 6.33 & \bf{523.05} & 
0.29\\CON3-9 & 590.50 & 7.40 & 
592.07 & 6.83 & \bf{578.24} & 
2.12\\CON8-0 & 888.34 & 5.03 & 
900.55 & 5.09 & \bf{857.17} & 
3.64\\CON8-1 & 761.69 & 4.48 & 
774.60 & 4.75 & \bf{740.85} & 
2.81\\CON8-2 & 725.72 & 6.17 & 
734.64 & 5.49 & \bf{712.89} & 
1.80\\CON8-3 & 822.12 & 5.98 & 
837.72 & 6.23 & \bf{811.07} & 
1.36\\CON8-4 & 781.78 & 6.84 & 
811.40 & 5.36 & \bf{772.25} & 
1.23\\CON8-5 & 769.55 & 4.99 & 
781.85 & 5.52 & \bf{754.88} & 
1.94\\CON8-6 & 699.04 & 5.97 & 
704.47 & 5.50 & \bf{678.92} & 
2.96\\CON8-7 & 817.70 & 5.43 & 
827.51 & 6.01 & \bf{811.96} & 
0.71\\CON8-8 & 783.97 & 5.09 & 
799.11 & 5.12 & \bf{767.53} & 
2.14\\CON8-9 & 815.52 & 7.19 & 
835.05 & 5.58 & \bf{809.00} & 
0.81\\[1ex]\hline
\end{tabular}
\label{table:nonlin}
\end{table} \clearpage
\begin{table}[ht]
\caption{Resultados de la ejecución de la metaheurística ILS, utilizando instancias de SalhiNagy con la configuración -n 25.0 -LS 70.0}
\centering
\small
\begin{tabular}{c c c c c c c}
\hline\hline
Instancia & Costo mínimo & Tiempo(seg.) & Costo promedio & Tiempo promedio(seg.) & Costo ILS & \%Gap \\ [0.5ex]
\hline
CMT1X & 473.58 & 5.30 & 
483.88 & 5.04 & \bf{466.77} & 
1.46\\CMT1Y & 470.67 & 6.72 & 
478.57 & 6.29 & \bf{466.77} & 
0.84\\CMT2X & 704.81 & 11.58 & 
710.85 & 12.01 & \bf{684.21} & 
3.01\\CMT2Y & 694.72 & 11.70 & 
707.78 & 12.37 & \bf{684.21} & 
1.54\\CMT3X & 733.81 & 26.76 & 
741.53 & 28.99 & \bf{721.40} & 
1.72\\CMT3Y & 727.29 & 26.07 & 
733.97 & 27.26 & \bf{721.40} & 
0.82\\CMT4X & 896.90 & 74.00 & 
908.06 & 75.66 & \bf{852.83} & 
5.17\\CMT4Y & 900.29 & 81.60 & 
902.37 & 77.46 & \bf{852.46} & 
5.61\\CMT5X & 1109.05 & 147.64 & 
1117.55 & 149.89 & \bf{1030.55} & 
7.62\\CMT5Y & 1094.29 & 162.55 & 
1105.45 & 175.39 & \bf{1031.17} & 
6.12\\CMT11X & 882.54 & 52.31 & 
896.37 & 54.47 & \bf{839.39} & 
5.14\\CMT11Y & 885.35 & 49.60 & 
895.38 & 53.16 & \bf{841.88} & 
5.16\\CMT12X & 686.96 & 25.22 & 
690.09 & 24.14 & \bf{662.22} & 
3.74\\CMT12Y & 678.17 & 29.93 & 
681.74 & 25.16 & \bf{662.22} & 
2.41\\[1ex]\hline
\end{tabular}
\label{table:nonlin}
\end{table} \clearpage
\begin{table}[ht]
\caption{Resultados de la ejecución de la metaheurística ILS, utilizando instancias de Dethloff con la configuración -n 25.0 -LS 80.0}
\centering
\small
\begin{tabular}{c c c c c c c}
\hline\hline
Instancia & Costo mínimo & Tiempo(seg.) & Costo promedio & Tiempo promedio(seg.) & Costo ILS & \%Gap \\ [0.5ex]
\hline
SCA3-0 & 640.55 & 8.06 & 
642.13 & 7.49 & \bf{635.62} & 
0.78\\SCA3-1 & \bf{697.84} & 6.63 & 
700.09 & 7.19 & 697.84 & 0.00\\
SCA3-2 & \bf{659.34} & 7.44 & 
665.46 & 7.26 & 659.34 & 0.00\\
SCA3-3 & \bf{680.04} & 7.46 & 
683.14 & 7.11 & 680.04 & 0.00\\
SCA3-4 & \bf{690.50} & 6.86 & 
691.87 & 7.46 & 690.50 & 0.00\\
SCA3-5 & 672.94 & 7.23 & 
678.00 & 7.11 & \bf{659.90} & 
1.98\\SCA3-6 & \bf{651.09} & 7.68 & 
655.40 & 7.56 & 651.09 & 0.00\\
SCA3-7 & 671.77 & 7.85 & 
673.17 & 6.93 & \bf{659.17} & 
1.91\\SCA3-8 & \bf{719.47} & 8.28 & 
725.71 & 6.75 & 719.47 & 0.00\\
SCA3-9 & \bf{681.00} & 6.98 & 
686.59 & 6.90 & 681.00 & 0.00\\
SCA8-0 & 998.31 & 5.97 & 
1020.21 & 5.70 & \bf{961.50} & 
3.83\\SCA8-1 & 1082.44 & 5.93 & 
1093.85 & 5.76 & \bf{1049.65} & 
3.12\\SCA8-2 & 1056.87 & 5.06 & 
1069.57 & 5.42 & \bf{1039.64} & 
1.66\\SCA8-3 & 1017.17 & 6.68 & 
1031.26 & 5.86 & \bf{983.34} & 
3.44\\SCA8-4 & 1069.87 & 5.14 & 
1078.14 & 5.88 & \bf{1065.49} & 
0.41\\SCA8-5 & 1073.57 & 7.71 & 
1081.99 & 6.50 & \bf{1027.08} & 
4.53\\SCA8-6 & 992.79 & 5.59 & 
1011.42 & 5.62 & \bf{971.82} & 
2.16\\SCA8-7 & 1067.11 & 5.17 & 
1090.26 & 5.34 & \bf{1051.28} & 
1.51\\SCA8-8 & 1085.93 & 6.23 & 
1095.58 & 6.00 & \bf{1071.18} & 
1.38\\SCA8-9 & 1083.62 & 6.98 & 
1103.38 & 5.88 & \bf{1060.50} & 
2.18\\CON3-0 & 619.09 & 6.09 & 
639.31 & 6.84 & \bf{616.52} & 
0.42\\CON3-1 & 560.75 & 7.82 & 
562.85 & 7.86 & \bf{554.47} & 
1.13\\CON3-2 & 521.38 & 8.26 & 
528.06 & 6.97 & \bf{518.00} & 
0.65\\CON3-3 & 591.48 & 7.66 & 
598.43 & 7.32 & \bf{591.19} & 
0.05\\CON3-4 & 591.43 & 7.80 & 
605.40 & 7.91 & \bf{588.79} & 
0.45\\CON3-5 & 567.94 & 6.57 & 
571.52 & 7.08 & \bf{563.70} & 
0.75\\CON3-6 & 502.16 & 6.99 & 
511.00 & 7.09 & \bf{499.05} & 
0.62\\CON3-7 & 586.01 & 6.44 & 
595.35 & 6.42 & \bf{576.48} & 
1.65\\CON3-8 & 523.14 & 7.63 & 
530.43 & 7.45 & \bf{523.05} & 
0.02\\CON3-9 & 588.99 & 6.81 & 
594.95 & 7.06 & \bf{578.24} & 
1.86\\CON8-0 & 871.92 & 7.28 & 
900.01 & 6.95 & \bf{857.17} & 
1.72\\CON8-1 & 754.51 & 5.98 & 
772.42 & 6.70 & \bf{740.85} & 
1.84\\CON8-2 & 716.89 & 6.10 & 
726.13 & 6.23 & \bf{712.89} & 
0.56\\CON8-3 & 823.68 & 5.92 & 
830.87 & 6.78 & \bf{811.07} & 
1.55\\CON8-4 & 799.14 & 6.19 & 
813.64 & 6.03 & \bf{772.25} & 
3.48\\CON8-5 & 763.13 & 6.07 & 
767.78 & 5.70 & \bf{754.88} & 
1.09\\CON8-6 & 696.73 & 6.39 & 
710.24 & 6.91 & \bf{678.92} & 
2.62\\CON8-7 & 828.76 & 5.33 & 
843.12 & 5.37 & \bf{811.96} & 
2.07\\CON8-8 & 783.75 & 5.43 & 
785.48 & 5.92 & \bf{767.53} & 
2.11\\CON8-9 & 817.79 & 8.02 & 
824.96 & 6.12 & \bf{809.00} & 
1.09\\[1ex]\hline
\end{tabular}
\label{table:nonlin}
\end{table} \clearpage
\begin{table}[ht]
\caption{Resultados de la ejecución de la metaheurística ILS, utilizando instancias de SalhiNagy con la configuración -n 25.0 -LS 80.0}
\centering
\small
\begin{tabular}{c c c c c c c}
\hline\hline
Instancia & Costo mínimo & Tiempo(seg.) & Costo promedio & Tiempo promedio(seg.) & Costo ILS & \%Gap \\ [0.5ex]
\hline
CMT1X & 479.19 & 5.09 & 
483.94 & 5.32 & \bf{466.77} & 
2.66\\CMT1Y & 477.72 & 5.56 & 
483.01 & 6.59 & \bf{466.77} & 
2.35\\CMT2X & 703.84 & 12.28 & 
714.18 & 13.04 & \bf{684.21} & 
2.87\\CMT2Y & 713.21 & 13.11 & 
715.29 & 12.09 & \bf{684.21} & 
4.24\\CMT3X & 737.09 & 30.42 & 
742.26 & 28.95 & \bf{721.40} & 
2.17\\CMT3Y & 735.96 & 30.71 & 
739.75 & 30.63 & \bf{721.40} & 
2.02\\CMT4X & 907.23 & 75.74 & 
911.97 & 75.89 & \bf{852.83} & 
6.38\\CMT4Y & 897.81 & 89.16 & 
906.46 & 78.70 & \bf{852.46} & 
5.32\\CMT5X & 1079.55 & 167.50 & 
1101.63 & 179.78 & \bf{1030.55} & 
4.75\\CMT5Y & 1095.85 & 165.92 & 
1105.98 & 182.75 & \bf{1031.17} & 
6.27\\CMT11X & 885.66 & 52.77 & 
893.21 & 57.49 & \bf{839.39} & 
5.51\\CMT11Y & 869.25 & 56.86 & 
907.09 & 59.48 & \bf{841.88} & 
3.25\\CMT12X & 673.17 & 29.22 & 
685.43 & 29.00 & \bf{662.22} & 
1.65\\CMT12Y & 675.11 & 25.01 & 
681.58 & 26.20 & \bf{662.22} & 
1.95\\[1ex]\hline
\end{tabular}
\label{table:nonlin}
\end{table} \clearpage
\begin{table}[ht]
\caption{Resultados de la ejecución de la metaheurística ILS, utilizando instancias de Dethloff con la configuración -n 35.0 -LS 10.0}
\centering
\small
\begin{tabular}{c c c c c c c}
\hline\hline
Instancia & Costo mínimo & Tiempo(seg.) & Costo promedio & Tiempo promedio(seg.) & Costo ILS & \%Gap \\ [0.5ex]
\hline
SCA3-0 & 636.06 & 2.12 & 
642.69 & 2.43 & \bf{635.62} & 
0.07\\SCA3-1 & 701.53 & 2.16 & 
709.75 & 2.36 & \bf{697.84} & 
0.53\\SCA3-2 & 673.67 & 2.55 & 
675.62 & 2.13 & \bf{659.34} & 
2.17\\SCA3-3 & 688.71 & 2.21 & 
690.93 & 1.96 & \bf{680.04} & 
1.27\\SCA3-4 & \bf{690.50} & 1.83 & 
703.32 & 2.03 & 690.50 & 0.00\\
SCA3-5 & 673.39 & 2.16 & 
677.60 & 2.30 & \bf{659.90} & 
2.04\\SCA3-6 & 654.79 & 2.53 & 
657.17 & 2.22 & \bf{651.09} & 
0.57\\SCA3-7 & 671.77 & 2.46 & 
679.53 & 2.15 & \bf{659.17} & 
1.91\\SCA3-8 & 719.77 & 2.04 & 
737.71 & 2.09 & \bf{719.47} & 
0.04\\SCA3-9 & 689.95 & 2.14 & 
697.43 & 2.04 & \bf{681.00} & 
1.31\\SCA8-0 & 1012.98 & 2.30 & 
1026.73 & 2.10 & \bf{961.50} & 
5.35\\SCA8-1 & 1094.39 & 1.83 & 
1104.61 & 1.93 & \bf{1049.65} & 
4.26\\SCA8-2 & 1053.94 & 2.16 & 
1071.04 & 2.07 & \bf{1039.64} & 
1.38\\SCA8-3 & 1033.38 & 1.92 & 
1055.79 & 1.97 & \bf{983.34} & 
5.09\\SCA8-4 & 1103.30 & 2.11 & 
1128.46 & 2.08 & \bf{1065.49} & 
3.55\\SCA8-5 & 1078.87 & 1.82 & 
1090.97 & 2.09 & \bf{1027.08} & 
5.04\\SCA8-6 & 998.05 & 1.70 & 
1010.96 & 1.84 & \bf{971.82} & 
2.70\\SCA8-7 & 1096.98 & 1.99 & 
1131.58 & 1.94 & \bf{1051.28} & 
4.35\\SCA8-8 & 1094.90 & 2.57 & 
1101.80 & 2.15 & \bf{1071.18} & 
2.21\\SCA8-9 & 1067.42 & 1.86 & 
1103.45 & 1.83 & \bf{1060.50} & 
0.65\\CON3-0 & 643.80 & 2.44 & 
652.65 & 2.19 & \bf{616.52} & 
4.42\\CON3-1 & 560.75 & 2.33 & 
564.78 & 2.28 & \bf{554.47} & 
1.13\\CON3-2 & 521.38 & 2.25 & 
529.52 & 2.15 & \bf{518.00} & 
0.65\\CON3-3 & 592.41 & 2.02 & 
598.13 & 2.19 & \bf{591.19} & 
0.21\\CON3-4 & 595.00 & 1.97 & 
600.81 & 2.33 & \bf{588.79} & 
1.05\\CON3-5 & 572.58 & 2.42 & 
581.09 & 2.13 & \bf{563.70} & 
1.58\\CON3-6 & 502.16 & 2.27 & 
506.08 & 2.28 & \bf{499.05} & 
0.62\\CON3-7 & 593.17 & 2.18 & 
598.88 & 2.14 & \bf{576.48} & 
2.90\\CON3-8 & 523.14 & 2.35 & 
530.05 & 2.17 & \bf{523.05} & 
0.02\\CON3-9 & 591.79 & 1.65 & 
594.72 & 1.85 & \bf{578.24} & 
2.34\\CON8-0 & 886.10 & 2.23 & 
929.66 & 2.00 & \bf{857.17} & 
3.38\\CON8-1 & 764.60 & 2.10 & 
778.48 & 2.35 & \bf{740.85} & 
3.21\\CON8-2 & 731.83 & 2.22 & 
734.05 & 2.12 & \bf{712.89} & 
2.66\\CON8-3 & 830.02 & 2.12 & 
856.57 & 2.06 & \bf{811.07} & 
2.34\\CON8-4 & 827.79 & 1.96 & 
833.80 & 1.81 & \bf{772.25} & 
7.19\\CON8-5 & 797.48 & 2.57 & 
810.74 & 2.12 & \bf{754.88} & 
5.64\\CON8-6 & 689.56 & 1.63 & 
701.23 & 2.02 & \bf{678.92} & 
1.57\\CON8-7 & 824.73 & 2.08 & 
843.89 & 2.06 & \bf{811.96} & 
1.57\\CON8-8 & 779.68 & 2.34 & 
789.87 & 2.19 & \bf{767.53} & 
1.58\\CON8-9 & 835.03 & 2.16 & 
848.48 & 2.29 & \bf{809.00} & 
3.22\\[1ex]\hline
\end{tabular}
\label{table:nonlin}
\end{table} \clearpage
\begin{table}[ht]
\caption{Resultados de la ejecución de la metaheurística ILS, utilizando instancias de SalhiNagy con la configuración -n 35.0 -LS 10.0}
\centering
\small
\begin{tabular}{c c c c c c c}
\hline\hline
Instancia & Costo mínimo & Tiempo(seg.) & Costo promedio & Tiempo promedio(seg.) & Costo ILS & \%Gap \\ [0.5ex]
\hline
CMT1X & 480.65 & 1.66 & 
487.08 & 1.76 & \bf{466.77} & 
2.97\\CMT1Y & 482.31 & 3.00 & 
485.87 & 2.19 & \bf{466.77} & 
3.33\\CMT2X & 702.99 & 7.47 & 
713.05 & 5.81 & \bf{684.21} & 
2.74\\CMT2Y & 711.53 & 5.03 & 
714.12 & 4.86 & \bf{684.21} & 
3.99\\CMT3X & 736.81 & 19.40 & 
740.75 & 15.01 & \bf{721.40} & 
2.14\\CMT3Y & 730.68 & 13.74 & 
735.83 & 16.21 & \bf{721.40} & 
1.29\\CMT4X & 893.70 & 42.45 & 
907.88 & 42.98 & \bf{852.83} & 
4.79\\CMT4Y & 905.89 & 39.80 & 
908.34 & 48.09 & \bf{852.46} & 
6.27\\CMT5X & 1108.43 & 106.01 & 
1116.91 & 105.17 & \bf{1030.55} & 
7.56\\CMT5Y & 1104.68 & 109.46 & 
1117.45 & 124.38 & \bf{1031.17} & 
7.13\\CMT11X & 896.15 & 54.29 & 
901.72 & 39.33 & \bf{839.39} & 
6.76\\CMT11Y & 888.02 & 31.38 & 
896.26 & 33.45 & \bf{841.88} & 
5.48\\CMT12X & 685.46 & 12.57 & 
688.43 & 14.85 & \bf{662.22} & 
3.51\\CMT12Y & 673.58 & 10.43 & 
680.89 & 11.84 & \bf{662.22} & 
1.72\\[1ex]\hline
\end{tabular}
\label{table:nonlin}
\end{table} \clearpage
\begin{table}[ht]
\caption{Resultados de la ejecución de la metaheurística ILS, utilizando instancias de Dethloff con la configuración -n 35.0 -LS 20.0}
\centering
\small
\begin{tabular}{c c c c c c c}
\hline\hline
Instancia & Costo mínimo & Tiempo(seg.) & Costo promedio & Tiempo promedio(seg.) & Costo ILS & \%Gap \\ [0.5ex]
\hline
SCA3-0 & 640.55 & 4.28 & 
641.85 & 3.72 & \bf{635.62} & 
0.78\\SCA3-1 & 710.84 & 2.78 & 
712.47 & 3.05 & \bf{697.84} & 
1.86\\SCA3-2 & 664.18 & 2.60 & 
666.53 & 3.10 & \bf{659.34} & 
0.73\\SCA3-3 & 680.60 & 3.80 & 
683.75 & 3.41 & \bf{680.04} & 
0.08\\SCA3-4 & \bf{690.50} & 3.53 & 
696.21 & 3.09 & 690.50 & 0.00\\
SCA3-5 & 672.94 & 3.30 & 
677.48 & 3.33 & \bf{659.90} & 
1.98\\SCA3-6 & 652.94 & 3.36 & 
661.55 & 2.94 & \bf{651.09} & 
0.28\\SCA3-7 & 669.89 & 2.96 & 
671.49 & 3.26 & \bf{659.17} & 
1.63\\SCA3-8 & 719.77 & 3.61 & 
728.55 & 3.31 & \bf{719.47} & 
0.04\\SCA3-9 & \bf{681.00} & 3.18 & 
689.09 & 3.25 & 681.00 & 0.00\\
SCA8-0 & 982.79 & 3.17 & 
1008.20 & 3.19 & \bf{961.50} & 
2.21\\SCA8-1 & 1075.20 & 3.80 & 
1087.64 & 2.74 & \bf{1049.65} & 
2.43\\SCA8-2 & 1064.72 & 2.62 & 
1076.79 & 3.04 & \bf{1039.64} & 
2.41\\SCA8-3 & 1014.10 & 2.98 & 
1047.93 & 2.99 & \bf{983.34} & 
3.13\\SCA8-4 & 1081.88 & 2.45 & 
1122.23 & 2.89 & \bf{1065.49} & 
1.54\\SCA8-5 & 1062.04 & 2.89 & 
1089.87 & 3.07 & \bf{1027.08} & 
3.40\\SCA8-6 & 1000.41 & 2.82 & 
1009.59 & 2.62 & \bf{971.82} & 
2.94\\SCA8-7 & 1094.76 & 2.69 & 
1110.06 & 2.72 & \bf{1051.28} & 
4.14\\SCA8-8 & 1096.75 & 3.06 & 
1114.24 & 2.98 & \bf{1071.18} & 
2.39\\SCA8-9 & 1072.10 & 2.35 & 
1080.63 & 2.39 & \bf{1060.50} & 
1.09\\CON3-0 & 633.24 & 3.82 & 
641.73 & 3.55 & \bf{616.52} & 
2.71\\CON3-1 & 564.15 & 3.57 & 
569.21 & 3.35 & \bf{554.47} & 
1.75\\CON3-2 & 524.89 & 3.18 & 
529.54 & 3.00 & \bf{518.00} & 
1.33\\CON3-3 & 591.48 & 3.48 & 
604.11 & 3.44 & \bf{591.19} & 
0.05\\CON3-4 & 591.43 & 4.11 & 
609.73 & 3.46 & \bf{588.79} & 
0.45\\CON3-5 & 568.76 & 4.01 & 
576.59 & 3.31 & \bf{563.70} & 
0.90\\CON3-6 & 503.97 & 3.59 & 
509.85 & 3.23 & \bf{499.05} & 
0.99\\CON3-7 & 578.41 & 3.52 & 
594.52 & 3.64 & \bf{576.48} & 
0.33\\CON3-8 & 524.59 & 4.45 & 
531.83 & 4.03 & \bf{523.05} & 
0.29\\CON3-9 & 588.99 & 3.30 & 
590.12 & 3.05 & \bf{578.24} & 
1.86\\CON8-0 & 884.71 & 2.53 & 
906.20 & 2.90 & \bf{857.17} & 
3.21\\CON8-1 & 764.29 & 2.82 & 
776.08 & 2.77 & \bf{740.85} & 
3.16\\CON8-2 & 717.31 & 3.02 & 
735.01 & 3.01 & \bf{712.89} & 
0.62\\CON8-3 & 846.12 & 3.34 & 
851.99 & 3.23 & \bf{811.07} & 
4.32\\CON8-4 & 810.87 & 3.30 & 
825.42 & 2.79 & \bf{772.25} & 
5.00\\CON8-5 & 777.01 & 2.96 & 
780.32 & 2.98 & \bf{754.88} & 
2.93\\CON8-6 & 700.00 & 3.34 & 
713.91 & 3.15 & \bf{678.92} & 
3.10\\CON8-7 & 829.57 & 3.16 & 
851.14 & 3.01 & \bf{811.96} & 
2.17\\CON8-8 & 802.55 & 2.73 & 
817.89 & 2.68 & \bf{767.53} & 
4.56\\CON8-9 & 836.08 & 3.66 & 
842.86 & 3.12 & \bf{809.00} & 
3.35\\[1ex]\hline
\end{tabular}
\label{table:nonlin}
\end{table} \clearpage
\begin{table}[ht]
\caption{Resultados de la ejecución de la metaheurística ILS, utilizando instancias de SalhiNagy con la configuración -n 35.0 -LS 20.0}
\centering
\small
\begin{tabular}{c c c c c c c}
\hline\hline
Instancia & Costo mínimo & Tiempo(seg.) & Costo promedio & Tiempo promedio(seg.) & Costo ILS & \%Gap \\ [0.5ex]
\hline
CMT1X & 475.20 & 2.30 & 
480.70 & 2.58 & \bf{466.77} & 
1.81\\CMT1Y & 475.91 & 3.30 & 
488.33 & 3.10 & \bf{466.77} & 
1.96\\CMT2X & 709.22 & 6.94 & 
714.07 & 7.55 & \bf{684.21} & 
3.66\\CMT2Y & 710.10 & 7.58 & 
714.74 & 7.23 & \bf{684.21} & 
3.78\\CMT3X & 736.97 & 17.89 & 
743.93 & 19.11 & \bf{721.40} & 
2.16\\CMT3Y & 726.94 & 17.33 & 
737.33 & 17.72 & \bf{721.40} & 
0.77\\CMT4X & 896.89 & 50.19 & 
907.10 & 56.79 & \bf{852.83} & 
5.17\\CMT4Y & 894.91 & 52.91 & 
904.43 & 51.61 & \bf{852.46} & 
4.98\\CMT5X & 1095.20 & 116.77 & 
1105.70 & 123.27 & \bf{1030.55} & 
6.27\\CMT5Y & 1100.66 & 121.22 & 
1108.82 & 132.37 & \bf{1031.17} & 
6.74\\CMT11X & 887.34 & 59.45 & 
895.92 & 46.10 & \bf{839.39} & 
5.71\\CMT11Y & 884.08 & 57.23 & 
896.14 & 49.60 & \bf{841.88} & 
5.01\\CMT12X & 680.02 & 21.09 & 
688.53 & 17.42 & \bf{662.22} & 
2.69\\CMT12Y & 674.10 & 15.09 & 
683.21 & 16.52 & \bf{662.22} & 
1.79\\[1ex]\hline
\end{tabular}
\label{table:nonlin}
\end{table} \clearpage
\begin{table}[ht]
\caption{Resultados de la ejecución de la metaheurística ILS, utilizando instancias de Dethloff con la configuración -n 35.0 -LS 30.0}
\centering
\small
\begin{tabular}{c c c c c c c}
\hline\hline
Instancia & Costo mínimo & Tiempo(seg.) & Costo promedio & Tiempo promedio(seg.) & Costo ILS & \%Gap \\ [0.5ex]
\hline
SCA3-0 & 641.69 & 4.48 & 
642.84 & 4.71 & \bf{635.62} & 
0.95\\SCA3-1 & 701.53 & 4.31 & 
708.07 & 4.03 & \bf{697.84} & 
0.53\\SCA3-2 & 664.21 & 3.67 & 
677.52 & 3.87 & \bf{659.34} & 
0.74\\SCA3-3 & 680.60 & 4.96 & 
682.08 & 4.88 & \bf{680.04} & 
0.08\\SCA3-4 & \bf{690.50} & 4.37 & 
691.53 & 4.44 & 690.50 & 0.00\\
SCA3-5 & 665.04 & 4.31 & 
674.81 & 4.67 & \bf{659.90} & 
0.78\\SCA3-6 & 654.26 & 4.02 & 
661.83 & 3.99 & \bf{651.09} & 
0.49\\SCA3-7 & 669.89 & 4.04 & 
672.48 & 4.09 & \bf{659.17} & 
1.63\\SCA3-8 & 724.29 & 3.50 & 
725.83 & 4.54 & \bf{719.47} & 
0.67\\SCA3-9 & \bf{681.00} & 4.30 & 
687.38 & 4.63 & 681.00 & 0.00\\
SCA8-0 & 1014.79 & 4.10 & 
1020.77 & 4.06 & \bf{961.50} & 
5.54\\SCA8-1 & 1070.87 & 4.00 & 
1084.72 & 4.15 & \bf{1049.65} & 
2.02\\SCA8-2 & 1069.37 & 3.86 & 
1075.78 & 3.07 & \bf{1039.64} & 
2.86\\SCA8-3 & 1030.39 & 3.78 & 
1038.74 & 3.49 & \bf{983.34} & 
4.78\\SCA8-4 & 1074.81 & 3.32 & 
1107.37 & 3.52 & \bf{1065.49} & 
0.87\\SCA8-5 & 1066.75 & 3.45 & 
1079.35 & 3.29 & \bf{1027.08} & 
3.86\\SCA8-6 & 1007.01 & 3.60 & 
1023.67 & 3.75 & \bf{971.82} & 
3.62\\SCA8-7 & 1070.92 & 4.41 & 
1087.35 & 3.95 & \bf{1051.28} & 
1.87\\SCA8-8 & 1087.01 & 3.76 & 
1101.49 & 3.94 & \bf{1071.18} & 
1.48\\SCA8-9 & 1111.43 & 4.04 & 
1115.54 & 3.19 & \bf{1060.50} & 
4.80\\CON3-0 & 633.24 & 4.23 & 
640.71 & 4.35 & \bf{616.52} & 
2.71\\CON3-1 & 564.15 & 4.10 & 
566.60 & 4.53 & \bf{554.47} & 
1.75\\CON3-2 & 521.38 & 5.06 & 
527.30 & 4.25 & \bf{518.00} & 
0.65\\CON3-3 & \bf{591.19} & 4.66 & 
608.76 & 4.43 & 591.19 & 0.00\\
CON3-4 & 593.78 & 3.72 & 
599.93 & 4.40 & \bf{588.79} & 
0.85\\CON3-5 & 566.45 & 5.22 & 
572.87 & 4.89 & \bf{563.70} & 
0.49\\CON3-6 & 502.16 & 4.16 & 
507.56 & 4.36 & \bf{499.05} & 
0.62\\CON3-7 & 589.64 & 3.74 & 
600.75 & 4.08 & \bf{576.48} & 
2.28\\CON3-8 & 524.30 & 4.61 & 
530.03 & 4.69 & \bf{523.05} & 
0.24\\CON3-9 & 588.99 & 3.89 & 
588.99 & 4.11 & \bf{578.24} & 
1.86\\CON8-0 & 887.50 & 3.04 & 
917.39 & 3.68 & \bf{857.17} & 
3.54\\CON8-1 & 756.05 & 3.95 & 
762.53 & 4.47 & \bf{740.85} & 
2.05\\CON8-2 & 723.13 & 3.96 & 
732.35 & 3.70 & \bf{712.89} & 
1.44\\CON8-3 & 835.19 & 3.61 & 
852.04 & 3.99 & \bf{811.07} & 
2.97\\CON8-4 & 796.77 & 4.55 & 
806.62 & 4.16 & \bf{772.25} & 
3.18\\CON8-5 & 771.14 & 4.10 & 
791.63 & 3.56 & \bf{754.88} & 
2.15\\CON8-6 & 700.85 & 4.89 & 
713.37 & 3.85 & \bf{678.92} & 
3.23\\CON8-7 & 826.97 & 3.64 & 
837.82 & 3.82 & \bf{811.96} & 
1.85\\CON8-8 & 784.28 & 3.53 & 
793.77 & 3.53 & \bf{767.53} & 
2.18\\CON8-9 & 820.15 & 3.48 & 
831.49 & 4.00 & \bf{809.00} & 
1.38\\[1ex]\hline
\end{tabular}
\label{table:nonlin}
\end{table} \clearpage
\begin{table}[ht]
\caption{Resultados de la ejecución de la metaheurística ILS, utilizando instancias de SalhiNagy con la configuración -n 35.0 -LS 30.0}
\centering
\small
\begin{tabular}{c c c c c c c}
\hline\hline
Instancia & Costo mínimo & Tiempo(seg.) & Costo promedio & Tiempo promedio(seg.) & Costo ILS & \%Gap \\ [0.5ex]
\hline
CMT1X & 471.25 & 3.64 & 
476.93 & 3.95 & \bf{466.77} & 
0.96\\CMT1Y & 480.50 & 3.70 & 
486.17 & 3.72 & \bf{466.77} & 
2.94\\CMT2X & 707.55 & 8.03 & 
713.78 & 9.37 & \bf{684.21} & 
3.41\\CMT2Y & 712.85 & 8.66 & 
718.33 & 8.49 & \bf{684.21} & 
4.19\\CMT3X & 734.04 & 20.23 & 
736.78 & 23.63 & \bf{721.40} & 
1.75\\CMT3Y & 732.60 & 21.84 & 
739.58 & 23.14 & \bf{721.40} & 
1.55\\CMT4X & 891.04 & 56.28 & 
906.68 & 68.34 & \bf{852.83} & 
4.48\\CMT4Y & 890.20 & 60.40 & 
900.87 & 69.36 & \bf{852.46} & 
4.43\\CMT5X & 1085.84 & 146.25 & 
1107.93 & 138.92 & \bf{1030.55} & 
5.37\\CMT5Y & 1105.80 & 143.08 & 
1112.36 & 151.21 & \bf{1031.17} & 
7.24\\CMT11X & 894.17 & 44.84 & 
896.62 & 44.33 & \bf{839.39} & 
6.53\\CMT11Y & 885.63 & 46.07 & 
903.72 & 50.86 & \bf{841.88} & 
5.20\\CMT12X & 682.37 & 18.88 & 
684.55 & 22.17 & \bf{662.22} & 
3.04\\CMT12Y & 673.57 & 24.15 & 
682.07 & 23.50 & \bf{662.22} & 
1.71\\[1ex]\hline
\end{tabular}
\label{table:nonlin}
\end{table} \clearpage
\begin{table}[ht]
\caption{Resultados de la ejecución de la metaheurística ILS, utilizando instancias de Dethloff con la configuración -n 35.0 -LS 40.0}
\centering
\small
\begin{tabular}{c c c c c c c}
\hline\hline
Instancia & Costo mínimo & Tiempo(seg.) & Costo promedio & Tiempo promedio(seg.) & Costo ILS & \%Gap \\ [0.5ex]
\hline
SCA3-0 & 641.69 & 5.98 & 
642.97 & 5.66 & \bf{635.62} & 
0.95\\SCA3-1 & \bf{697.84} & 5.78 & 
707.11 & 5.64 & 697.84 & 0.00\\
SCA3-2 & 665.71 & 6.28 & 
668.85 & 5.82 & \bf{659.34} & 
0.97\\SCA3-3 & 680.60 & 6.13 & 
685.34 & 5.57 & \bf{680.04} & 
0.08\\SCA3-4 & 693.23 & 5.25 & 
698.26 & 5.36 & \bf{690.50} & 
0.40\\SCA3-5 & 673.56 & 4.72 & 
678.71 & 5.08 & \bf{659.90} & 
2.07\\SCA3-6 & 652.47 & 5.10 & 
662.56 & 5.05 & \bf{651.09} & 
0.21\\SCA3-7 & 671.67 & 4.86 & 
672.09 & 5.29 & \bf{659.17} & 
1.90\\SCA3-8 & 719.77 & 5.29 & 
728.96 & 5.16 & \bf{719.47} & 
0.04\\SCA3-9 & \bf{681.00} & 5.30 & 
693.65 & 4.93 & 681.00 & 0.00\\
SCA8-0 & 997.74 & 4.40 & 
1009.72 & 4.98 & \bf{961.50} & 
3.77\\SCA8-1 & 1087.28 & 5.21 & 
1093.03 & 4.67 & \bf{1049.65} & 
3.59\\SCA8-2 & 1058.01 & 4.92 & 
1064.41 & 4.66 & \bf{1039.64} & 
1.77\\SCA8-3 & 1023.10 & 5.03 & 
1035.03 & 4.36 & \bf{983.34} & 
4.04\\SCA8-4 & 1069.71 & 4.36 & 
1099.14 & 4.53 & \bf{1065.49} & 
0.40\\SCA8-5 & 1067.85 & 3.85 & 
1083.93 & 4.76 & \bf{1027.08} & 
3.97\\SCA8-6 & 980.91 & 4.36 & 
1019.32 & 4.32 & \bf{971.82} & 
0.94\\SCA8-7 & 1076.01 & 4.93 & 
1084.00 & 4.94 & \bf{1051.28} & 
2.35\\SCA8-8 & \bf{1071.18} & 5.45 & 
1095.18 & 4.41 & 1071.18 & 0.00\\
SCA8-9 & 1091.14 & 4.25 & 
1101.87 & 4.50 & \bf{1060.50} & 
2.89\\CON3-0 & 624.91 & 5.81 & 
635.34 & 5.96 & \bf{616.52} & 
1.36\\CON3-1 & 557.21 & 6.18 & 
562.16 & 5.94 & \bf{554.47} & 
0.49\\CON3-2 & 521.38 & 6.20 & 
521.84 & 5.76 & \bf{518.00} & 
0.65\\CON3-3 & 591.48 & 6.41 & 
598.62 & 5.74 & \bf{591.19} & 
0.05\\CON3-4 & 593.78 & 5.24 & 
605.18 & 6.01 & \bf{588.79} & 
0.85\\CON3-5 & 568.76 & 5.74 & 
577.39 & 5.67 & \bf{563.70} & 
0.90\\CON3-6 & 502.16 & 6.74 & 
509.95 & 5.93 & \bf{499.05} & 
0.62\\CON3-7 & 578.41 & 4.85 & 
585.90 & 5.22 & \bf{576.48} & 
0.33\\CON3-8 & 530.94 & 5.85 & 
536.99 & 5.64 & \bf{523.05} & 
1.51\\CON3-9 & 589.57 & 5.25 & 
590.10 & 5.71 & \bf{578.24} & 
1.96\\CON8-0 & 878.47 & 5.70 & 
911.68 & 4.83 & \bf{857.17} & 
2.48\\CON8-1 & 760.41 & 5.80 & 
775.98 & 4.86 & \bf{740.85} & 
2.64\\CON8-2 & 713.44 & 4.34 & 
723.46 & 4.65 & \bf{712.89} & 
0.08\\CON8-3 & 827.92 & 5.18 & 
846.09 & 4.53 & \bf{811.07} & 
2.08\\CON8-4 & 792.33 & 4.02 & 
813.07 & 4.49 & \bf{772.25} & 
2.60\\CON8-5 & 771.35 & 5.21 & 
780.96 & 5.18 & \bf{754.88} & 
2.18\\CON8-6 & 705.38 & 4.99 & 
708.33 & 5.41 & \bf{678.92} & 
3.90\\CON8-7 & 825.09 & 4.58 & 
843.46 & 4.70 & \bf{811.96} & 
1.62\\CON8-8 & 786.38 & 4.24 & 
796.77 & 4.78 & \bf{767.53} & 
2.46\\CON8-9 & 817.60 & 3.96 & 
831.19 & 4.73 & \bf{809.00} & 
1.06\\[1ex]\hline
\end{tabular}
\label{table:nonlin}
\end{table} \clearpage
\begin{table}[ht]
\caption{Resultados de la ejecución de la metaheurística ILS, utilizando instancias de SalhiNagy con la configuración -n 35.0 -LS 40.0}
\centering
\small
\begin{tabular}{c c c c c c c}
\hline\hline
Instancia & Costo mínimo & Tiempo(seg.) & Costo promedio & Tiempo promedio(seg.) & Costo ILS & \%Gap \\ [0.5ex]
\hline
CMT1X & 479.19 & 4.59 & 
482.98 & 4.58 & \bf{466.77} & 
2.66\\CMT1Y & 478.67 & 3.63 & 
484.44 & 4.65 & \bf{466.77} & 
2.55\\CMT2X & 696.69 & 10.98 & 
712.02 & 11.52 & \bf{684.21} & 
1.82\\CMT2Y & 707.94 & 8.82 & 
711.63 & 10.74 & \bf{684.21} & 
3.47\\CMT3X & 730.15 & 27.22 & 
736.46 & 26.86 & \bf{721.40} & 
1.21\\CMT3Y & 737.76 & 24.24 & 
740.97 & 25.89 & \bf{721.40} & 
2.27\\CMT4X & 891.81 & 67.86 & 
901.27 & 72.48 & \bf{852.83} & 
4.57\\CMT4Y & 879.41 & 69.09 & 
897.97 & 68.81 & \bf{852.46} & 
3.16\\CMT5X & 1105.26 & 210.24 & 
1115.35 & 184.05 & \bf{1030.55} & 
7.25\\CMT5Y & 1101.83 & 160.60 & 
1110.40 & 156.52 & \bf{1031.17} & 
6.85\\CMT11X & 872.84 & 51.69 & 
889.76 & 60.92 & \bf{839.39} & 
3.99\\CMT11Y & 877.90 & 56.41 & 
891.41 & 57.51 & \bf{841.88} & 
4.28\\CMT12X & 678.19 & 22.57 & 
686.68 & 22.11 & \bf{662.22} & 
2.41\\CMT12Y & 679.25 & 22.55 & 
683.32 & 23.64 & \bf{662.22} & 
2.57\\[1ex]\hline
\end{tabular}
\label{table:nonlin}
\end{table} \clearpage
\begin{table}[ht]
\caption{Resultados de la ejecución de la metaheurística ILS, utilizando instancias de Dethloff con la configuración -n 35.0 -LS 50.0}
\centering
\small
\begin{tabular}{c c c c c c c}
\hline\hline
Instancia & Costo mínimo & Tiempo(seg.) & Costo promedio & Tiempo promedio(seg.) & Costo ILS & \%Gap \\ [0.5ex]
\hline
SCA3-0 & 636.06 & 7.92 & 
641.33 & 7.13 & \bf{635.62} & 
0.07\\SCA3-1 & 701.53 & 8.16 & 
708.67 & 6.96 & \bf{697.84} & 
0.53\\SCA3-2 & 664.18 & 6.87 & 
667.08 & 6.64 & \bf{659.34} & 
0.73\\SCA3-3 & 680.60 & 7.87 & 
682.97 & 7.40 & \bf{680.04} & 
0.08\\SCA3-4 & \bf{690.50} & 7.79 & 
696.44 & 7.20 & 690.50 & 0.00\\
SCA3-5 & 661.07 & 5.58 & 
672.96 & 6.05 & \bf{659.90} & 
0.18\\SCA3-6 & 652.47 & 6.73 & 
655.45 & 6.35 & \bf{651.09} & 
0.21\\SCA3-7 & 669.89 & 6.71 & 
674.64 & 6.47 & \bf{659.17} & 
1.63\\SCA3-8 & 723.99 & 6.31 & 
727.33 & 6.72 & \bf{719.47} & 
0.63\\SCA3-9 & \bf{681.00} & 7.48 & 
689.67 & 6.56 & 681.00 & 0.00\\
SCA8-0 & 990.48 & 5.54 & 
1002.84 & 5.67 & \bf{961.50} & 
3.01\\SCA8-1 & 1080.63 & 5.81 & 
1092.88 & 5.24 & \bf{1049.65} & 
2.95\\SCA8-2 & 1054.69 & 5.03 & 
1060.31 & 5.05 & \bf{1039.64} & 
1.45\\SCA8-3 & 1026.65 & 4.62 & 
1037.14 & 5.28 & \bf{983.34} & 
4.40\\SCA8-4 & 1077.80 & 5.34 & 
1102.56 & 5.79 & \bf{1065.49} & 
1.16\\SCA8-5 & 1085.33 & 6.14 & 
1087.41 & 5.96 & \bf{1027.08} & 
5.67\\SCA8-6 & 981.41 & 5.69 & 
1000.86 & 5.50 & \bf{971.82} & 
0.99\\SCA8-7 & 1093.27 & 5.04 & 
1111.16 & 5.25 & \bf{1051.28} & 
3.99\\SCA8-8 & 1097.82 & 6.17 & 
1106.07 & 5.42 & \bf{1071.18} & 
2.49\\SCA8-9 & 1084.38 & 5.55 & 
1091.41 & 6.14 & \bf{1060.50} & 
2.25\\CON3-0 & 633.24 & 7.22 & 
637.83 & 7.03 & \bf{616.52} & 
2.71\\CON3-1 & 560.75 & 5.50 & 
563.78 & 6.21 & \bf{554.47} & 
1.13\\CON3-2 & 521.38 & 5.66 & 
522.10 & 6.38 & \bf{518.00} & 
0.65\\CON3-3 & \bf{591.19} & 7.54 & 
594.55 & 7.00 & 591.19 & 0.00\\
CON3-4 & 593.78 & 6.96 & 
601.63 & 6.59 & \bf{588.79} & 
0.85\\CON3-5 & 574.44 & 6.09 & 
575.74 & 6.31 & \bf{563.70} & 
1.91\\CON3-6 & 505.82 & 5.68 & 
509.75 & 6.77 & \bf{499.05} & 
1.36\\CON3-7 & 577.54 & 6.31 & 
587.37 & 6.93 & \bf{576.48} & 
0.18\\CON3-8 & 529.65 & 6.95 & 
536.16 & 6.80 & \bf{523.05} & 
1.26\\CON3-9 & 588.99 & 6.56 & 
589.94 & 6.05 & \bf{578.24} & 
1.86\\CON8-0 & 872.97 & 6.40 & 
887.64 & 6.11 & \bf{857.17} & 
1.84\\CON8-1 & 752.61 & 5.08 & 
773.90 & 5.66 & \bf{740.85} & 
1.59\\CON8-2 & 717.68 & 7.47 & 
733.24 & 6.25 & \bf{712.89} & 
0.67\\CON8-3 & 824.75 & 5.73 & 
838.30 & 5.91 & \bf{811.07} & 
1.69\\CON8-4 & 794.34 & 5.84 & 
804.34 & 5.47 & \bf{772.25} & 
2.86\\CON8-5 & 776.74 & 5.22 & 
780.04 & 5.83 & \bf{754.88} & 
2.90\\CON8-6 & 706.28 & 6.46 & 
708.75 & 6.05 & \bf{678.92} & 
4.03\\CON8-7 & 815.32 & 6.24 & 
843.17 & 6.18 & \bf{811.96} & 
0.41\\CON8-8 & 783.44 & 5.82 & 
793.49 & 5.91 & \bf{767.53} & 
2.07\\CON8-9 & 820.69 & 6.66 & 
839.34 & 5.95 & \bf{809.00} & 
1.44\\[1ex]\hline
\end{tabular}
\label{table:nonlin}
\end{table} \clearpage
\begin{table}[ht]
\caption{Resultados de la ejecución de la metaheurística ILS, utilizando instancias de SalhiNagy con la configuración -n 35.0 -LS 50.0}
\centering
\small
\begin{tabular}{c c c c c c c}
\hline\hline
Instancia & Costo mínimo & Tiempo(seg.) & Costo promedio & Tiempo promedio(seg.) & Costo ILS & \%Gap \\ [0.5ex]
\hline
CMT1X & 472.58 & 7.24 & 
477.00 & 5.71 & \bf{466.77} & 
1.24\\CMT1Y & 486.09 & 5.95 & 
489.19 & 6.09 & \bf{466.77} & 
4.14\\CMT2X & 694.46 & 12.92 & 
714.31 & 12.18 & \bf{684.21} & 
1.50\\CMT2Y & 703.05 & 13.36 & 
708.58 & 12.71 & \bf{684.21} & 
2.75\\CMT3X & 723.97 & 33.97 & 
734.62 & 31.53 & \bf{721.40} & 
0.36\\CMT3Y & 729.13 & 33.52 & 
735.13 & 30.86 & \bf{721.40} & 
1.07\\CMT4X & 897.02 & 75.86 & 
904.77 & 89.86 & \bf{852.83} & 
5.18\\CMT4Y & 899.71 & 76.75 & 
909.27 & 80.83 & \bf{852.46} & 
5.54\\CMT5X & 1100.51 & 175.85 & 
1105.17 & 184.79 & \bf{1030.55} & 
6.79\\CMT5Y & 1096.60 & 215.68 & 
1105.88 & 178.39 & \bf{1031.17} & 
6.35\\CMT11X & 880.26 & 55.39 & 
892.43 & 55.45 & \bf{839.39} & 
4.87\\CMT11Y & 889.14 & 54.53 & 
895.15 & 54.77 & \bf{841.88} & 
5.61\\CMT12X & 677.78 & 26.35 & 
683.68 & 27.67 & \bf{662.22} & 
2.35\\CMT12Y & 675.21 & 25.42 & 
681.25 & 26.82 & \bf{662.22} & 
1.96\\[1ex]\hline
\end{tabular}
\label{table:nonlin}
\end{table} \clearpage
\begin{table}[ht]
\caption{Resultados de la ejecución de la metaheurística ILS, utilizando instancias de Dethloff con la configuración -n 35.0 -LS 60.0}
\centering
\small
\begin{tabular}{c c c c c c c}
\hline\hline
Instancia & Costo mínimo & Tiempo(seg.) & Costo promedio & Tiempo promedio(seg.) & Costo ILS & \%Gap \\ [0.5ex]
\hline
SCA3-0 & 640.55 & 7.48 & 
640.84 & 7.98 & \bf{635.62} & 
0.78\\SCA3-1 & \bf{697.84} & 6.96 & 
699.68 & 7.23 & 697.84 & 0.00\\
SCA3-2 & \bf{659.34} & 8.20 & 
664.19 & 8.00 & 659.34 & 0.00\\
SCA3-3 & 681.74 & 9.01 & 
685.80 & 8.38 & \bf{680.04} & 
0.25\\SCA3-4 & \bf{690.50} & 7.96 & 
691.02 & 7.92 & 690.50 & 0.00\\
SCA3-5 & 679.90 & 6.91 & 
681.14 & 7.42 & \bf{659.90} & 
3.03\\SCA3-6 & \bf{651.09} & 9.50 & 
653.48 & 8.16 & 651.09 & 0.00\\
SCA3-7 & 669.89 & 7.72 & 
671.63 & 7.68 & \bf{659.17} & 
1.63\\SCA3-8 & 719.77 & 7.54 & 
728.64 & 7.63 & \bf{719.47} & 
0.04\\SCA3-9 & \bf{681.00} & 6.84 & 
684.26 & 7.25 & 681.00 & 0.00\\
SCA8-0 & 984.73 & 5.78 & 
1000.13 & 6.35 & \bf{961.50} & 
2.42\\SCA8-1 & 1067.20 & 5.24 & 
1088.59 & 5.96 & \bf{1049.65} & 
1.67\\SCA8-2 & 1066.86 & 6.08 & 
1073.86 & 6.26 & \bf{1039.64} & 
2.62\\SCA8-3 & 1011.26 & 6.61 & 
1030.15 & 6.62 & \bf{983.34} & 
2.84\\SCA8-4 & 1069.71 & 5.34 & 
1089.06 & 6.34 & \bf{1065.49} & 
0.40\\SCA8-5 & 1064.19 & 6.95 & 
1069.62 & 7.34 & \bf{1027.08} & 
3.61\\SCA8-6 & 1004.17 & 6.34 & 
1013.17 & 6.41 & \bf{971.82} & 
3.33\\SCA8-7 & 1070.67 & 6.42 & 
1085.00 & 6.20 & \bf{1051.28} & 
1.84\\SCA8-8 & 1095.32 & 5.43 & 
1095.47 & 5.62 & \bf{1071.18} & 
2.25\\SCA8-9 & 1081.45 & 5.40 & 
1090.67 & 6.00 & \bf{1060.50} & 
1.98\\CON3-0 & 630.73 & 8.50 & 
637.38 & 8.49 & \bf{616.52} & 
2.30\\CON3-1 & 558.09 & 6.29 & 
563.53 & 7.41 & \bf{554.47} & 
0.65\\CON3-2 & 522.51 & 10.01 & 
526.19 & 8.31 & \bf{518.00} & 
0.87\\CON3-3 & \bf{591.19} & 8.47 & 
591.50 & 7.89 & 591.19 & 0.00\\
CON3-4 & 591.43 & 8.71 & 
598.84 & 8.02 & \bf{588.79} & 
0.45\\CON3-5 & 564.88 & 6.64 & 
571.35 & 7.69 & \bf{563.70} & 
0.21\\CON3-6 & 502.85 & 7.96 & 
506.50 & 7.64 & \bf{499.05} & 
0.76\\CON3-7 & 578.41 & 7.07 & 
578.41 & 7.59 & \bf{576.48} & 
0.33\\CON3-8 & 527.82 & 7.84 & 
534.90 & 8.44 & \bf{523.05} & 
0.91\\CON3-9 & 588.99 & 8.69 & 
589.81 & 8.13 & \bf{578.24} & 
1.86\\CON8-0 & 884.18 & 6.34 & 
906.64 & 6.37 & \bf{857.17} & 
3.15\\CON8-1 & 756.16 & 6.87 & 
764.92 & 6.82 & \bf{740.85} & 
2.07\\CON8-2 & 720.09 & 6.78 & 
728.82 & 6.54 & \bf{712.89} & 
1.01\\CON8-3 & 827.30 & 7.82 & 
844.69 & 7.37 & \bf{811.07} & 
2.00\\CON8-4 & 791.89 & 7.14 & 
800.74 & 6.33 & \bf{772.25} & 
2.54\\CON8-5 & 771.58 & 8.37 & 
783.58 & 7.07 & \bf{754.88} & 
2.21\\CON8-6 & 702.65 & 7.04 & 
705.13 & 6.58 & \bf{678.92} & 
3.50\\CON8-7 & 821.28 & 6.44 & 
831.36 & 6.55 & \bf{811.96} & 
1.15\\CON8-8 & 786.71 & 6.22 & 
792.47 & 6.44 & \bf{767.53} & 
2.50\\CON8-9 & 818.44 & 5.83 & 
827.92 & 6.39 & \bf{809.00} & 
1.17\\[1ex]\hline
\end{tabular}
\label{table:nonlin}
\end{table} \clearpage
\begin{table}[ht]
\caption{Resultados de la ejecución de la metaheurística ILS, utilizando instancias de SalhiNagy con la configuración -n 35.0 -LS 60.0}
\centering
\small
\begin{tabular}{c c c c c c c}
\hline\hline
Instancia & Costo mínimo & Tiempo(seg.) & Costo promedio & Tiempo promedio(seg.) & Costo ILS & \%Gap \\ [0.5ex]
\hline
CMT1X & 471.25 & 7.21 & 
478.02 & 7.39 & \bf{466.77} & 
0.96\\CMT1Y & 475.22 & 7.12 & 
479.76 & 6.74 & \bf{466.77} & 
1.81\\CMT2X & 704.62 & 15.17 & 
715.20 & 14.76 & \bf{684.21} & 
2.98\\CMT2Y & 707.57 & 15.16 & 
710.74 & 14.38 & \bf{684.21} & 
3.41\\CMT3X & 731.07 & 34.90 & 
735.83 & 34.72 & \bf{721.40} & 
1.34\\CMT3Y & 726.61 & 32.47 & 
738.35 & 36.31 & \bf{721.40} & 
0.72\\CMT4X & 898.24 & 77.33 & 
904.94 & 88.78 & \bf{852.83} & 
5.32\\CMT4Y & 872.20 & 82.90 & 
893.86 & 94.88 & \bf{852.46} & 
2.32\\CMT5X & 1109.17 & 176.76 & 
1116.39 & 184.71 & \bf{1030.55} & 
7.63\\CMT5Y & 1089.72 & 208.38 & 
1098.62 & 206.40 & \bf{1031.17} & 
5.68\\CMT11X & 852.51 & 65.71 & 
886.16 & 67.14 & \bf{839.39} & 
1.56\\CMT11Y & 883.44 & 55.95 & 
894.98 & 64.75 & \bf{841.88} & 
4.94\\CMT12X & 674.28 & 31.42 & 
678.00 & 30.08 & \bf{662.22} & 
1.82\\CMT12Y & 676.66 & 27.75 & 
681.34 & 28.00 & \bf{662.22} & 
2.18\\[1ex]\hline
\end{tabular}
\label{table:nonlin}
\end{table} \clearpage
\begin{table}[ht]
\caption{Resultados de la ejecución de la metaheurística ILS, utilizando instancias de Dethloff con la configuración -n 35.0 -LS 70.0}
\centering
\small
\begin{tabular}{c c c c c c c}
\hline\hline
Instancia & Costo mínimo & Tiempo(seg.) & Costo promedio & Tiempo promedio(seg.) & Costo ILS & \%Gap \\ [0.5ex]
\hline
SCA3-0 & 636.06 & 9.52 & 
640.69 & 9.36 & \bf{635.62} & 
0.07\\SCA3-1 & \bf{697.84} & 8.94 & 
702.05 & 8.96 & 697.84 & 0.00\\
SCA3-2 & 664.21 & 8.04 & 
668.45 & 8.73 & \bf{659.34} & 
0.74\\SCA3-3 & 681.31 & 8.90 & 
683.98 & 8.99 & \bf{680.04} & 
0.19\\SCA3-4 & \bf{690.50} & 8.64 & 
691.18 & 9.16 & 690.50 & 0.00\\
SCA3-5 & 665.04 & 7.95 & 
672.56 & 8.63 & \bf{659.90} & 
0.78\\SCA3-6 & 652.94 & 7.33 & 
655.47 & 8.88 & \bf{651.09} & 
0.28\\SCA3-7 & 671.67 & 8.16 & 
671.72 & 8.26 & \bf{659.17} & 
1.90\\SCA3-8 & 726.44 & 9.24 & 
729.02 & 8.84 & \bf{719.47} & 
0.97\\SCA3-9 & \bf{681.00} & 9.12 & 
685.71 & 8.53 & 681.00 & 0.00\\
SCA8-0 & 1001.55 & 8.00 & 
1012.47 & 7.54 & \bf{961.50} & 
4.17\\SCA8-1 & 1069.27 & 6.08 & 
1076.14 & 7.01 & \bf{1049.65} & 
1.87\\SCA8-2 & 1054.69 & 7.90 & 
1065.75 & 7.39 & \bf{1039.64} & 
1.45\\SCA8-3 & 1014.10 & 8.23 & 
1025.58 & 7.03 & \bf{983.34} & 
3.13\\SCA8-4 & 1067.55 & 6.56 & 
1088.81 & 7.19 & \bf{1065.49} & 
0.19\\SCA8-5 & 1072.42 & 9.39 & 
1085.94 & 8.71 & \bf{1027.08} & 
4.41\\SCA8-6 & 987.84 & 7.97 & 
998.42 & 7.18 & \bf{971.82} & 
1.65\\SCA8-7 & 1071.93 & 7.35 & 
1078.16 & 7.79 & \bf{1051.28} & 
1.96\\SCA8-8 & 1092.87 & 7.69 & 
1094.95 & 7.18 & \bf{1071.18} & 
2.02\\SCA8-9 & 1077.37 & 6.74 & 
1089.12 & 7.86 & \bf{1060.50} & 
1.59\\CON3-0 & 628.47 & 11.73 & 
630.59 & 9.75 & \bf{616.52} & 
1.94\\CON3-1 & 560.75 & 8.91 & 
564.36 & 9.40 & \bf{554.47} & 
1.13\\CON3-2 & 521.38 & 9.79 & 
523.16 & 8.91 & \bf{518.00} & 
0.65\\CON3-3 & 591.48 & 8.56 & 
600.26 & 8.99 & \bf{591.19} & 
0.05\\CON3-4 & 593.78 & 8.69 & 
594.86 & 9.04 & \bf{588.79} & 
0.85\\CON3-5 & 568.66 & 9.83 & 
571.82 & 8.91 & \bf{563.70} & 
0.88\\CON3-6 & 502.16 & 10.13 & 
508.56 & 9.45 & \bf{499.05} & 
0.62\\CON3-7 & 578.41 & 12.92 & 
586.58 & 9.84 & \bf{576.48} & 
0.33\\CON3-8 & 524.59 & 9.20 & 
529.36 & 9.89 & \bf{523.05} & 
0.29\\CON3-9 & 588.11 & 9.01 & 
589.51 & 8.58 & \bf{578.24} & 
1.71\\CON8-0 & 889.16 & 8.75 & 
904.27 & 7.38 & \bf{857.17} & 
3.73\\CON8-1 & 762.19 & 10.53 & 
764.17 & 9.01 & \bf{740.85} & 
2.88\\CON8-2 & 722.68 & 7.66 & 
732.87 & 8.17 & \bf{712.89} & 
1.37\\CON8-3 & 816.80 & 8.63 & 
834.08 & 7.41 & \bf{811.07} & 
0.71\\CON8-4 & 784.36 & 6.43 & 
806.51 & 7.43 & \bf{772.25} & 
1.57\\CON8-5 & 762.71 & 7.36 & 
777.91 & 7.04 & \bf{754.88} & 
1.04\\CON8-6 & 687.57 & 8.74 & 
698.66 & 8.15 & \bf{678.92} & 
1.27\\CON8-7 & 814.79 & 7.04 & 
830.91 & 7.38 & \bf{811.96} & 
0.35\\CON8-8 & 785.28 & 6.84 & 
794.44 & 6.79 & \bf{767.53} & 
2.31\\CON8-9 & 819.07 & 6.90 & 
822.63 & 7.78 & \bf{809.00} & 
1.24\\[1ex]\hline
\end{tabular}
\label{table:nonlin}
\end{table} \clearpage
\begin{table}[ht]
\caption{Resultados de la ejecución de la metaheurística ILS, utilizando instancias de SalhiNagy con la configuración -n 35.0 -LS 70.0}
\centering
\small
\begin{tabular}{c c c c c c c}
\hline\hline
Instancia & Costo mínimo & Tiempo(seg.) & Costo promedio & Tiempo promedio(seg.) & Costo ILS & \%Gap \\ [0.5ex]
\hline
CMT1X & 472.58 & 8.10 & 
478.69 & 7.31 & \bf{466.77} & 
1.24\\CMT1Y & 474.91 & 7.36 & 
479.20 & 7.27 & \bf{466.77} & 
1.74\\CMT2X & 702.26 & 17.80 & 
714.93 & 15.41 & \bf{684.21} & 
2.64\\CMT2Y & 696.36 & 17.91 & 
707.10 & 16.48 & \bf{684.21} & 
1.78\\CMT3X & 730.78 & 47.65 & 
735.71 & 38.80 & \bf{721.40} & 
1.30\\CMT3Y & 732.10 & 37.11 & 
736.29 & 36.57 & \bf{721.40} & 
1.48\\CMT4X & 890.31 & 111.61 & 
903.22 & 106.92 & \bf{852.83} & 
4.39\\CMT4Y & 901.78 & 103.32 & 
908.63 & 101.89 & \bf{852.46} & 
5.79\\CMT5X & 1068.98 & 216.80 & 
1096.78 & 241.09 & \bf{1030.55} & 
3.73\\CMT5Y & 1098.63 & 223.68 & 
1106.12 & 228.23 & \bf{1031.17} & 
6.54\\CMT11X & 880.79 & 71.90 & 
892.37 & 71.11 & \bf{839.39} & 
4.93\\CMT11Y & 860.42 & 71.38 & 
887.13 & 71.57 & \bf{841.88} & 
2.20\\CMT12X & 678.40 & 32.83 & 
681.12 & 34.23 & \bf{662.22} & 
2.44\\CMT12Y & 683.51 & 35.25 & 
686.01 & 32.00 & \bf{662.22} & 
3.21\\[1ex]\hline
\end{tabular}
\label{table:nonlin}
\end{table} \clearpage
\begin{table}[ht]
\caption{Resultados de la ejecución de la metaheurística ILS, utilizando instancias de Dethloff con la configuración -n 35.0 -LS 80.0}
\centering
\small
\begin{tabular}{c c c c c c c}
\hline\hline
Instancia & Costo mínimo & Tiempo(seg.) & Costo promedio & Tiempo promedio(seg.) & Costo ILS & \%Gap \\ [0.5ex]
\hline
SCA3-0 & 636.06 & 9.54 & 
640.65 & 9.99 & \bf{635.62} & 
0.07\\SCA3-1 & 700.50 & 10.51 & 
705.15 & 9.73 & \bf{697.84} & 
0.38\\SCA3-2 & 664.21 & 9.97 & 
670.65 & 9.84 & \bf{659.34} & 
0.74\\SCA3-3 & 680.60 & 11.54 & 
681.70 & 11.56 & \bf{680.04} & 
0.08\\SCA3-4 & \bf{690.50} & 10.81 & 
690.50 & 10.26 & 690.50 & 0.00\\
SCA3-5 & 670.02 & 10.48 & 
679.16 & 10.13 & \bf{659.90} & 
1.53\\SCA3-6 & 653.81 & 10.09 & 
655.17 & 9.95 & \bf{651.09} & 
0.42\\SCA3-7 & 669.89 & 8.61 & 
671.27 & 9.35 & \bf{659.17} & 
1.63\\SCA3-8 & \bf{719.47} & 9.20 & 
724.37 & 9.64 & 719.47 & 0.00\\
SCA3-9 & \bf{681.00} & 9.06 & 
687.71 & 9.43 & 681.00 & 0.00\\
SCA8-0 & 977.93 & 9.21 & 
1000.08 & 8.83 & \bf{961.50} & 
1.71\\SCA8-1 & 1088.59 & 7.57 & 
1092.20 & 8.29 & \bf{1049.65} & 
3.71\\SCA8-2 & 1056.87 & 8.16 & 
1070.47 & 7.99 & \bf{1039.64} & 
1.66\\SCA8-3 & 1016.01 & 6.82 & 
1027.95 & 7.46 & \bf{983.34} & 
3.32\\SCA8-4 & 1075.71 & 8.92 & 
1094.47 & 8.79 & \bf{1065.49} & 
0.96\\SCA8-5 & 1060.32 & 7.98 & 
1074.32 & 8.25 & \bf{1027.08} & 
3.24\\SCA8-6 & 986.90 & 9.13 & 
998.02 & 8.68 & \bf{971.82} & 
1.55\\SCA8-7 & 1070.92 & 9.91 & 
1080.79 & 8.29 & \bf{1051.28} & 
1.87\\SCA8-8 & 1089.07 & 8.62 & 
1096.24 & 7.50 & \bf{1071.18} & 
1.67\\SCA8-9 & 1091.16 & 9.76 & 
1102.44 & 8.06 & \bf{1060.50} & 
2.89\\CON3-0 & 623.15 & 10.86 & 
632.64 & 10.83 & \bf{616.52} & 
1.08\\CON3-1 & 560.75 & 11.67 & 
562.40 & 10.80 & \bf{554.47} & 
1.13\\CON3-2 & 521.63 & 9.99 & 
524.77 & 9.33 & \bf{518.00} & 
0.70\\CON3-3 & \bf{591.19} & 10.43 & 
593.82 & 9.55 & 591.19 & 0.00\\
CON3-4 & 591.43 & 9.88 & 
598.93 & 9.91 & \bf{588.79} & 
0.45\\CON3-5 & 564.88 & 10.15 & 
569.86 & 10.39 & \bf{563.70} & 
0.21\\CON3-6 & 502.16 & 10.78 & 
505.96 & 10.27 & \bf{499.05} & 
0.62\\CON3-7 & 577.68 & 9.94 & 
591.69 & 9.97 & \bf{576.48} & 
0.21\\CON3-8 & 524.59 & 10.16 & 
529.58 & 10.39 & \bf{523.05} & 
0.29\\CON3-9 & 588.38 & 10.60 & 
589.99 & 9.87 & \bf{578.24} & 
1.75\\CON8-0 & 863.53 & 8.25 & 
887.43 & 8.83 & \bf{857.17} & 
0.74\\CON8-1 & 769.67 & 6.81 & 
780.66 & 7.75 & \bf{740.85} & 
3.89\\CON8-2 & 725.92 & 8.92 & 
730.16 & 8.84 & \bf{712.89} & 
1.83\\CON8-3 & 827.60 & 8.65 & 
833.90 & 8.89 & \bf{811.07} & 
2.04\\CON8-4 & 794.38 & 9.82 & 
812.92 & 8.46 & \bf{772.25} & 
2.87\\CON8-5 & 758.84 & 7.50 & 
768.60 & 7.58 & \bf{754.88} & 
0.52\\CON8-6 & 696.89 & 8.10 & 
706.53 & 8.64 & \bf{678.92} & 
2.65\\CON8-7 & 816.00 & 7.76 & 
823.25 & 8.09 & \bf{811.96} & 
0.50\\CON8-8 & 785.63 & 8.78 & 
793.87 & 8.43 & \bf{767.53} & 
2.36\\CON8-9 & 816.00 & 11.15 & 
831.96 & 9.44 & \bf{809.00} & 
0.87\\[1ex]\hline
\end{tabular}
\label{table:nonlin}
\end{table} \clearpage
\begin{table}[ht]
\caption{Resultados de la ejecución de la metaheurística ILS, utilizando instancias de SalhiNagy con la configuración -n 35.0 -LS 80.0}
\centering
\small
\begin{tabular}{c c c c c c c}
\hline\hline
Instancia & Costo mínimo & Tiempo(seg.) & Costo promedio & Tiempo promedio(seg.) & Costo ILS & \%Gap \\ [0.5ex]
\hline
CMT1X & 476.66 & 7.34 & 
478.83 & 7.03 & \bf{466.77} & 
2.12\\CMT1Y & 475.72 & 8.42 & 
481.51 & 8.33 & \bf{466.77} & 
1.92\\CMT2X & 712.07 & 16.75 & 
718.12 & 18.12 & \bf{684.21} & 
4.07\\CMT2Y & 690.24 & 20.17 & 
705.73 & 19.14 & \bf{684.21} & 
0.88\\CMT3X & 730.80 & 39.83 & 
738.58 & 39.59 & \bf{721.40} & 
1.30\\CMT3Y & 731.74 & 41.82 & 
733.20 & 44.49 & \bf{721.40} & 
1.43\\CMT4X & 895.51 & 97.50 & 
900.15 & 108.87 & \bf{852.83} & 
5.00\\CMT4Y & 893.75 & 99.69 & 
904.84 & 97.93 & \bf{852.46} & 
4.84\\CMT5X & 1103.34 & 274.41 & 
1106.00 & 237.80 & \bf{1030.55} & 
7.06\\CMT5Y & 1095.28 & 231.64 & 
1099.38 & 246.88 & \bf{1031.17} & 
6.22\\CMT11X & 874.94 & 106.31 & 
884.56 & 88.19 & \bf{839.39} & 
4.24\\CMT11Y & 884.56 & 79.52 & 
889.35 & 78.45 & \bf{841.88} & 
5.07\\CMT12X & 675.24 & 32.27 & 
680.92 & 35.58 & \bf{662.22} & 
1.97\\CMT12Y & 674.59 & 39.05 & 
683.78 & 36.30 & \bf{662.22} & 
1.87\\[1ex]\hline
\end{tabular}
\label{table:nonlin}
\end{table} \clearpage
\begin{table}[ht]
\caption{Resultados de la ejecución de la metaheurística ILS, utilizando instancias de Dethloff con la configuración -n 45.0 -LS 10.0}
\centering
\small
\begin{tabular}{c c c c c c c}
\hline\hline
Instancia & Costo mínimo & Tiempo(seg.) & Costo promedio & Tiempo promedio(seg.) & Costo ILS & \%Gap \\ [0.5ex]
\hline
SCA3-0 & 640.55 & 2.74 & 
641.49 & 2.85 & \bf{635.62} & 
0.78\\SCA3-1 & \bf{697.84} & 2.92 & 
703.38 & 3.02 & 697.84 & 0.00\\
SCA3-2 & 661.13 & 2.53 & 
667.91 & 2.42 & \bf{659.34} & 
0.27\\SCA3-3 & 681.74 & 2.63 & 
685.85 & 2.93 & \bf{680.04} & 
0.25\\SCA3-4 & 693.23 & 3.16 & 
707.32 & 2.87 & \bf{690.50} & 
0.40\\SCA3-5 & 681.81 & 3.05 & 
683.70 & 3.06 & \bf{659.90} & 
3.32\\SCA3-6 & 653.83 & 2.81 & 
664.35 & 2.66 & \bf{651.09} & 
0.42\\SCA3-7 & 671.67 & 2.77 & 
673.86 & 2.67 & \bf{659.17} & 
1.90\\SCA3-8 & 726.44 & 2.80 & 
731.76 & 2.88 & \bf{719.47} & 
0.97\\SCA3-9 & \bf{681.00} & 2.61 & 
689.94 & 2.45 & 681.00 & 0.00\\
SCA8-0 & 998.24 & 3.96 & 
1032.58 & 2.91 & \bf{961.50} & 
3.82\\SCA8-1 & 1080.99 & 2.37 & 
1097.00 & 2.27 & \bf{1049.65} & 
2.99\\SCA8-2 & 1054.78 & 2.13 & 
1075.96 & 2.40 & \bf{1039.64} & 
1.46\\SCA8-3 & 1015.31 & 3.03 & 
1025.31 & 2.82 & \bf{983.34} & 
3.25\\SCA8-4 & 1076.42 & 2.58 & 
1093.86 & 2.60 & \bf{1065.49} & 
1.03\\SCA8-5 & 1061.75 & 2.03 & 
1087.66 & 2.69 & \bf{1027.08} & 
3.38\\SCA8-6 & 1003.41 & 2.18 & 
1014.68 & 2.63 & \bf{971.82} & 
3.25\\SCA8-7 & 1085.45 & 2.09 & 
1103.55 & 2.40 & \bf{1051.28} & 
3.25\\SCA8-8 & 1088.19 & 2.79 & 
1099.10 & 2.61 & \bf{1071.18} & 
1.59\\SCA8-9 & 1106.11 & 1.96 & 
1115.38 & 2.50 & \bf{1060.50} & 
4.30\\CON3-0 & 633.24 & 3.59 & 
642.84 & 2.83 & \bf{616.52} & 
2.71\\CON3-1 & 561.87 & 2.86 & 
564.63 & 2.76 & \bf{554.47} & 
1.33\\CON3-2 & 526.95 & 2.42 & 
531.67 & 2.58 & \bf{518.00} & 
1.73\\CON3-3 & 592.43 & 2.59 & 
599.60 & 2.90 & \bf{591.19} & 
0.21\\CON3-4 & 596.43 & 2.74 & 
611.27 & 2.68 & \bf{588.79} & 
1.30\\CON3-5 & 569.88 & 2.08 & 
578.08 & 2.79 & \bf{563.70} & 
1.10\\CON3-6 & 502.64 & 4.04 & 
509.89 & 3.26 & \bf{499.05} & 
0.72\\CON3-7 & 591.04 & 2.78 & 
594.70 & 2.50 & \bf{576.48} & 
2.53\\CON3-8 & 526.59 & 3.10 & 
530.47 & 2.94 & \bf{523.05} & 
0.68\\CON3-9 & 588.40 & 2.69 & 
590.47 & 2.77 & \bf{578.24} & 
1.76\\CON8-0 & 881.36 & 3.05 & 
891.89 & 2.65 & \bf{857.17} & 
2.82\\CON8-1 & 745.98 & 2.66 & 
767.87 & 2.58 & \bf{740.85} & 
0.69\\CON8-2 & 723.29 & 3.24 & 
735.17 & 2.94 & \bf{712.89} & 
1.46\\CON8-3 & 824.87 & 3.06 & 
847.92 & 2.80 & \bf{811.07} & 
1.70\\CON8-4 & 807.10 & 2.73 & 
815.39 & 2.87 & \bf{772.25} & 
4.51\\CON8-5 & 779.59 & 3.24 & 
788.66 & 3.15 & \bf{754.88} & 
3.27\\CON8-6 & 706.94 & 2.98 & 
709.70 & 2.83 & \bf{678.92} & 
4.13\\CON8-7 & 820.92 & 2.92 & 
842.38 & 2.60 & \bf{811.96} & 
1.10\\CON8-8 & 799.89 & 2.56 & 
804.71 & 2.38 & \bf{767.53} & 
4.22\\CON8-9 & 830.43 & 3.35 & 
852.90 & 2.69 & \bf{809.00} & 
2.65\\[1ex]\hline
\end{tabular}
\label{table:nonlin}
\end{table} \clearpage
\begin{table}[ht]
\caption{Resultados de la ejecución de la metaheurística ILS, utilizando instancias de SalhiNagy con la configuración -n 45.0 -LS 10.0}
\centering
\small
\begin{tabular}{c c c c c c c}
\hline\hline
Instancia & Costo mínimo & Tiempo(seg.) & Costo promedio & Tiempo promedio(seg.) & Costo ILS & \%Gap \\ [0.5ex]
\hline
CMT1X & 472.58 & 2.90 & 
483.95 & 2.30 & \bf{466.77} & 
1.24\\CMT1Y & 486.98 & 2.28 & 
488.73 & 2.59 & \bf{466.77} & 
4.33\\CMT2X & 712.31 & 7.48 & 
719.32 & 6.84 & \bf{684.21} & 
4.11\\CMT2Y & 712.10 & 6.62 & 
715.02 & 6.90 & \bf{684.21} & 
4.08\\CMT3X & 738.36 & 18.26 & 
740.20 & 17.47 & \bf{721.40} & 
2.35\\CMT3Y & 727.83 & 18.83 & 
737.43 & 19.70 & \bf{721.40} & 
0.89\\CMT4X & 898.88 & 57.44 & 
904.49 & 64.51 & \bf{852.83} & 
5.40\\CMT4Y & 890.39 & 54.91 & 
902.96 & 55.83 & \bf{852.46} & 
4.45\\CMT5X & 1107.23 & 132.19 & 
1109.94 & 153.86 & \bf{1030.55} & 
7.44\\CMT5Y & 1084.64 & 208.57 & 
1102.72 & 156.24 & \bf{1031.17} & 
5.19\\CMT11X & 888.61 & 42.21 & 
895.32 & 43.45 & \bf{839.39} & 
5.86\\CMT11Y & 877.15 & 42.23 & 
902.29 & 43.63 & \bf{841.88} & 
4.19\\CMT12X & 681.01 & 17.82 & 
685.77 & 16.09 & \bf{662.22} & 
2.84\\CMT12Y & 676.34 & 15.62 & 
681.71 & 18.20 & \bf{662.22} & 
2.13\\[1ex]\hline
\end{tabular}
\label{table:nonlin}
\end{table} \clearpage
\begin{table}[ht]
\caption{Resultados de la ejecución de la metaheurística ILS, utilizando instancias de Dethloff con la configuración -n 45.0 -LS 20.0}
\centering
\small
\begin{tabular}{c c c c c c c}
\hline\hline
Instancia & Costo mínimo & Tiempo(seg.) & Costo promedio & Tiempo promedio(seg.) & Costo ILS & \%Gap \\ [0.5ex]
\hline
SCA3-0 & 640.55 & 4.63 & 
641.12 & 4.44 & \bf{635.62} & 
0.78\\SCA3-1 & \bf{697.84} & 4.27 & 
704.32 & 4.33 & 697.84 & 0.00\\
SCA3-2 & 664.92 & 4.30 & 
669.83 & 4.22 & \bf{659.34} & 
0.85\\SCA3-3 & 681.74 & 4.04 & 
682.10 & 4.32 & \bf{680.04} & 
0.25\\SCA3-4 & 692.57 & 4.48 & 
700.50 & 4.32 & \bf{690.50} & 
0.30\\SCA3-5 & 675.81 & 4.89 & 
681.57 & 4.40 & \bf{659.90} & 
2.41\\SCA3-6 & 655.41 & 4.23 & 
658.48 & 3.89 & \bf{651.09} & 
0.66\\SCA3-7 & 669.89 & 4.10 & 
670.80 & 4.21 & \bf{659.17} & 
1.63\\SCA3-8 & \bf{719.47} & 4.47 & 
727.65 & 4.35 & 719.47 & 0.00\\
SCA3-9 & 685.19 & 3.66 & 
691.86 & 3.56 & \bf{681.00} & 
0.62\\SCA8-0 & 1004.86 & 4.76 & 
1018.33 & 4.57 & \bf{961.50} & 
4.51\\SCA8-1 & 1083.79 & 3.74 & 
1091.69 & 3.99 & \bf{1049.65} & 
3.25\\SCA8-2 & 1054.47 & 4.10 & 
1072.46 & 3.54 & \bf{1039.64} & 
1.43\\SCA8-3 & 1035.16 & 4.19 & 
1042.10 & 3.87 & \bf{983.34} & 
5.27\\SCA8-4 & 1095.53 & 3.31 & 
1111.81 & 3.31 & \bf{1065.49} & 
2.82\\SCA8-5 & 1070.69 & 4.37 & 
1076.72 & 3.62 & \bf{1027.08} & 
4.25\\SCA8-6 & 997.87 & 5.06 & 
1011.21 & 3.69 & \bf{971.82} & 
2.68\\SCA8-7 & 1076.22 & 3.93 & 
1092.18 & 3.65 & \bf{1051.28} & 
2.37\\SCA8-8 & 1090.67 & 3.17 & 
1094.86 & 3.21 & \bf{1071.18} & 
1.82\\SCA8-9 & 1105.14 & 3.29 & 
1111.71 & 3.50 & \bf{1060.50} & 
4.21\\CON3-0 & 619.09 & 3.60 & 
629.91 & 4.17 & \bf{616.52} & 
0.42\\CON3-1 & 560.75 & 3.86 & 
565.05 & 4.30 & \bf{554.47} & 
1.13\\CON3-2 & 521.38 & 4.45 & 
523.49 & 4.25 & \bf{518.00} & 
0.65\\CON3-3 & 591.20 & 5.19 & 
599.75 & 5.05 & \bf{591.19} & 
0.00\\CON3-4 & 593.78 & 4.64 & 
602.95 & 4.52 & \bf{588.79} & 
0.85\\CON3-5 & 572.42 & 3.24 & 
575.76 & 3.89 & \bf{563.70} & 
1.55\\CON3-6 & 502.75 & 4.29 & 
507.19 & 4.38 & \bf{499.05} & 
0.74\\CON3-7 & 586.84 & 4.18 & 
591.97 & 3.94 & \bf{576.48} & 
1.80\\CON3-8 & 527.82 & 4.41 & 
532.93 & 4.67 & \bf{523.05} & 
0.91\\CON3-9 & 587.97 & 3.64 & 
592.50 & 4.20 & \bf{578.24} & 
1.68\\CON8-0 & 900.47 & 4.35 & 
916.84 & 3.71 & \bf{857.17} & 
5.05\\CON8-1 & 755.21 & 4.94 & 
769.86 & 3.89 & \bf{740.85} & 
1.94\\CON8-2 & 727.11 & 4.56 & 
734.75 & 4.27 & \bf{712.89} & 
1.99\\CON8-3 & 836.92 & 3.62 & 
845.15 & 3.93 & \bf{811.07} & 
3.19\\CON8-4 & 817.14 & 3.59 & 
820.09 & 3.78 & \bf{772.25} & 
5.81\\CON8-5 & 768.42 & 4.40 & 
776.64 & 4.45 & \bf{754.88} & 
1.79\\CON8-6 & 699.47 & 3.41 & 
709.08 & 4.07 & \bf{678.92} & 
3.03\\CON8-7 & 815.43 & 4.05 & 
836.11 & 3.90 & \bf{811.96} & 
0.43\\CON8-8 & 798.02 & 4.59 & 
805.63 & 4.29 & \bf{767.53} & 
3.97\\CON8-9 & 836.38 & 3.91 & 
842.14 & 3.84 & \bf{809.00} & 
3.38\\[1ex]\hline
\end{tabular}
\label{table:nonlin}
\end{table} \clearpage
\begin{table}[ht]
\caption{Resultados de la ejecución de la metaheurística ILS, utilizando instancias de SalhiNagy con la configuración -n 45.0 -LS 20.0}
\centering
\small
\begin{tabular}{c c c c c c c}
\hline\hline
Instancia & Costo mínimo & Tiempo(seg.) & Costo promedio & Tiempo promedio(seg.) & Costo ILS & \%Gap \\ [0.5ex]
\hline
CMT1X & 475.95 & 3.99 & 
480.81 & 3.59 & \bf{466.77} & 
1.97\\CMT1Y & 473.62 & 2.84 & 
487.29 & 3.56 & \bf{466.77} & 
1.47\\CMT2X & 709.63 & 8.68 & 
713.97 & 8.79 & \bf{684.21} & 
3.72\\CMT2Y & 702.28 & 9.67 & 
707.51 & 9.51 & \bf{684.21} & 
2.64\\CMT3X & 727.54 & 21.90 & 
738.73 & 22.73 & \bf{721.40} & 
0.85\\CMT3Y & 732.84 & 22.30 & 
739.32 & 24.58 & \bf{721.40} & 
1.59\\CMT4X & 892.30 & 63.59 & 
899.92 & 71.36 & \bf{852.83} & 
4.63\\CMT4Y & 889.41 & 62.53 & 
904.05 & 65.02 & \bf{852.46} & 
4.33\\CMT5X & 1104.84 & 216.95 & 
1109.93 & 176.45 & \bf{1030.55} & 
7.21\\CMT5Y & 1088.07 & 220.81 & 
1106.46 & 175.90 & \bf{1031.17} & 
5.52\\CMT11X & 888.82 & 52.90 & 
893.28 & 57.02 & \bf{839.39} & 
5.89\\CMT11Y & 881.49 & 51.88 & 
897.75 & 56.74 & \bf{841.88} & 
4.70\\CMT12X & 671.94 & 19.77 & 
678.29 & 19.82 & \bf{662.22} & 
1.47\\CMT12Y & 675.22 & 20.73 & 
684.42 & 23.53 & \bf{662.22} & 
1.96\\[1ex]\hline
\end{tabular}
\label{table:nonlin}
\end{table} \clearpage
\begin{table}[ht]
\caption{Resultados de la ejecución de la metaheurística ILS, utilizando instancias de Dethloff con la configuración -n 45.0 -LS 30.0}
\centering
\small
\begin{tabular}{c c c c c c c}
\hline\hline
Instancia & Costo mínimo & Tiempo(seg.) & Costo promedio & Tiempo promedio(seg.) & Costo ILS & \%Gap \\ [0.5ex]
\hline
SCA3-0 & 640.55 & 6.32 & 
642.00 & 5.85 & \bf{635.62} & 
0.78\\SCA3-1 & 706.90 & 6.04 & 
709.27 & 6.24 & \bf{697.84} & 
1.30\\SCA3-2 & \bf{659.34} & 4.90 & 
668.26 & 5.23 & 659.34 & 0.00\\
SCA3-3 & 680.60 & 5.86 & 
683.00 & 5.25 & \bf{680.04} & 
0.08\\SCA3-4 & \bf{690.50} & 6.91 & 
692.22 & 5.92 & 690.50 & 0.00\\
SCA3-5 & \bf{659.90} & 5.84 & 
674.07 & 6.00 & 659.90 & 0.00\\
SCA3-6 & \bf{651.09} & 6.22 & 
655.02 & 5.96 & 651.09 & 0.00\\
SCA3-7 & 669.89 & 6.12 & 
672.57 & 5.55 & \bf{659.17} & 
1.63\\SCA3-8 & 719.77 & 7.00 & 
728.07 & 5.63 & \bf{719.47} & 
0.04\\SCA3-9 & 683.57 & 5.40 & 
689.65 & 5.19 & \bf{681.00} & 
0.38\\SCA8-0 & 996.34 & 5.07 & 
1012.96 & 5.33 & \bf{961.50} & 
3.62\\SCA8-1 & 1067.83 & 3.85 & 
1083.89 & 4.51 & \bf{1049.65} & 
1.73\\SCA8-2 & 1071.06 & 4.77 & 
1078.68 & 4.65 & \bf{1039.64} & 
3.02\\SCA8-3 & 1015.61 & 4.16 & 
1023.27 & 5.03 & \bf{983.34} & 
3.28\\SCA8-4 & 1067.55 & 5.41 & 
1097.38 & 5.37 & \bf{1065.49} & 
0.19\\SCA8-5 & 1070.99 & 4.90 & 
1081.51 & 5.18 & \bf{1027.08} & 
4.28\\SCA8-6 & 1003.87 & 4.96 & 
1011.25 & 4.58 & \bf{971.82} & 
3.30\\SCA8-7 & 1078.87 & 5.39 & 
1093.10 & 5.17 & \bf{1051.28} & 
2.62\\SCA8-8 & 1088.65 & 5.13 & 
1109.46 & 4.54 & \bf{1071.18} & 
1.63\\SCA8-9 & 1069.22 & 4.32 & 
1097.94 & 4.60 & \bf{1060.50} & 
0.82\\CON3-0 & 632.14 & 5.82 & 
642.20 & 5.19 & \bf{616.52} & 
2.53\\CON3-1 & 560.75 & 5.94 & 
566.82 & 5.85 & \bf{554.47} & 
1.13\\CON3-2 & 521.38 & 6.35 & 
527.11 & 5.92 & \bf{518.00} & 
0.65\\CON3-3 & 592.41 & 5.30 & 
612.41 & 5.22 & \bf{591.19} & 
0.21\\CON3-4 & 593.78 & 6.13 & 
597.65 & 6.06 & \bf{588.79} & 
0.85\\CON3-5 & 568.76 & 5.32 & 
577.52 & 5.37 & \bf{563.70} & 
0.90\\CON3-6 & 504.28 & 5.62 & 
507.04 & 5.55 & \bf{499.05} & 
1.05\\CON3-7 & 590.75 & 6.03 & 
598.84 & 5.72 & \bf{576.48} & 
2.48\\CON3-8 & \bf{523.05} & 6.33 & 
523.85 & 6.25 & 523.05 & 0.00\\
CON3-9 & 578.25 & 6.95 & 
587.65 & 6.08 & \bf{578.24} & 
0.00\\CON8-0 & 905.34 & 4.61 & 
913.34 & 4.46 & \bf{857.17} & 
5.62\\CON8-1 & 754.22 & 5.31 & 
766.27 & 5.58 & \bf{740.85} & 
1.80\\CON8-2 & 721.13 & 4.89 & 
729.39 & 5.52 & \bf{712.89} & 
1.16\\CON8-3 & 824.90 & 5.68 & 
836.35 & 4.86 & \bf{811.07} & 
1.71\\CON8-4 & 797.82 & 5.20 & 
816.64 & 5.07 & \bf{772.25} & 
3.31\\CON8-5 & 768.50 & 5.80 & 
775.54 & 5.33 & \bf{754.88} & 
1.80\\CON8-6 & 693.57 & 4.43 & 
705.75 & 4.86 & \bf{678.92} & 
2.16\\CON8-7 & 816.07 & 4.45 & 
820.64 & 4.73 & \bf{811.96} & 
0.51\\CON8-8 & 792.59 & 5.40 & 
798.29 & 5.64 & \bf{767.53} & 
3.27\\CON8-9 & 816.02 & 4.59 & 
836.36 & 4.65 & \bf{809.00} & 
0.87\\[1ex]\hline
\end{tabular}
\label{table:nonlin}
\end{table} \clearpage
\begin{table}[ht]
\caption{Resultados de la ejecución de la metaheurística ILS, utilizando instancias de SalhiNagy con la configuración -n 45.0 -LS 30.0}
\centering
\small
\begin{tabular}{c c c c c c c}
\hline\hline
Instancia & Costo mínimo & Tiempo(seg.) & Costo promedio & Tiempo promedio(seg.) & Costo ILS & \%Gap \\ [0.5ex]
\hline
CMT1X & 478.54 & 5.41 & 
484.64 & 4.99 & \bf{466.77} & 
2.52\\CMT1Y & 481.74 & 4.28 & 
488.26 & 4.88 & \bf{466.77} & 
3.21\\CMT2X & 704.47 & 11.77 & 
709.52 & 11.16 & \bf{684.21} & 
2.96\\CMT2Y & 706.88 & 12.47 & 
709.90 & 11.32 & \bf{684.21} & 
3.31\\CMT3X & 729.25 & 27.58 & 
733.91 & 27.29 & \bf{721.40} & 
1.09\\CMT3Y & 725.87 & 35.78 & 
734.03 & 28.96 & \bf{721.40} & 
0.62\\CMT4X & 884.84 & 76.26 & 
893.29 & 75.05 & \bf{852.83} & 
3.75\\CMT4Y & 885.76 & 79.02 & 
898.56 & 92.55 & \bf{852.46} & 
3.91\\CMT5X & 1088.14 & 175.41 & 
1103.87 & 196.63 & \bf{1030.55} & 
5.59\\CMT5Y & 1095.85 & 184.22 & 
1100.96 & 206.82 & \bf{1031.17} & 
6.27\\CMT11X & 885.18 & 59.06 & 
891.29 & 59.30 & \bf{839.39} & 
5.46\\CMT11Y & 867.28 & 54.30 & 
883.41 & 58.33 & \bf{841.88} & 
3.02\\CMT12X & 679.44 & 24.39 & 
680.31 & 25.76 & \bf{662.22} & 
2.60\\CMT12Y & 678.28 & 30.81 & 
686.12 & 26.38 & \bf{662.22} & 
2.43\\[1ex]\hline
\end{tabular}
\label{table:nonlin}
\end{table} \clearpage
\begin{table}[ht]
\caption{Resultados de la ejecución de la metaheurística ILS, utilizando instancias de Dethloff con la configuración -n 45.0 -LS 40.0}
\centering
\small
\begin{tabular}{c c c c c c c}
\hline\hline
Instancia & Costo mínimo & Tiempo(seg.) & Costo promedio & Tiempo promedio(seg.) & Costo ILS & \%Gap \\ [0.5ex]
\hline
SCA3-0 & 640.55 & 7.36 & 
641.12 & 7.80 & \bf{635.62} & 
0.78\\SCA3-1 & 701.53 & 7.38 & 
705.91 & 7.37 & \bf{697.84} & 
0.53\\SCA3-2 & 666.01 & 6.82 & 
667.16 & 6.91 & \bf{659.34} & 
1.01\\SCA3-3 & \bf{680.04} & 7.46 & 
683.00 & 7.24 & 680.04 & 0.00\\
SCA3-4 & \bf{690.50} & 7.84 & 
691.02 & 7.13 & 690.50 & 0.00\\
SCA3-5 & 665.04 & 7.35 & 
674.42 & 8.64 & \bf{659.90} & 
0.78\\SCA3-6 & \bf{651.09} & 6.97 & 
652.88 & 7.00 & 651.09 & 0.00\\
SCA3-7 & 671.67 & 6.13 & 
674.83 & 6.84 & \bf{659.17} & 
1.90\\SCA3-8 & \bf{719.47} & 7.30 & 
721.55 & 6.93 & 719.47 & 0.00\\
SCA3-9 & \bf{681.00} & 6.10 & 
684.50 & 6.55 & 681.00 & 0.00\\
SCA8-0 & 995.57 & 6.04 & 
1015.42 & 6.37 & \bf{961.50} & 
3.54\\SCA8-1 & 1062.35 & 5.55 & 
1069.39 & 5.63 & \bf{1049.65} & 
1.21\\SCA8-2 & 1053.51 & 5.58 & 
1063.59 & 5.92 & \bf{1039.64} & 
1.33\\SCA8-3 & 1024.57 & 6.36 & 
1030.34 & 6.07 & \bf{983.34} & 
4.19\\SCA8-4 & 1081.35 & 6.17 & 
1087.51 & 5.91 & \bf{1065.49} & 
1.49\\SCA8-5 & 1062.34 & 5.61 & 
1083.94 & 5.94 & \bf{1027.08} & 
3.43\\SCA8-6 & 991.57 & 6.08 & 
1001.83 & 5.75 & \bf{971.82} & 
2.03\\SCA8-7 & 1067.11 & 7.13 & 
1077.29 & 6.25 & \bf{1051.28} & 
1.51\\SCA8-8 & 1095.14 & 5.60 & 
1105.64 & 5.35 & \bf{1071.18} & 
2.24\\SCA8-9 & 1078.23 & 6.22 & 
1089.07 & 6.20 & \bf{1060.50} & 
1.67\\CON3-0 & 629.19 & 8.11 & 
632.49 & 7.00 & \bf{616.52} & 
2.06\\CON3-1 & 562.83 & 8.01 & 
565.63 & 7.43 & \bf{554.47} & 
1.51\\CON3-2 & 521.38 & 5.61 & 
524.26 & 6.82 & \bf{518.00} & 
0.65\\CON3-3 & 591.20 & 10.47 & 
595.37 & 7.87 & \bf{591.19} & 
0.00\\CON3-4 & 593.78 & 7.67 & 
595.46 & 7.00 & \bf{588.79} & 
0.85\\CON3-5 & 569.88 & 6.62 & 
571.58 & 6.80 & \bf{563.70} & 
1.10\\CON3-6 & 502.16 & 6.72 & 
504.56 & 6.78 & \bf{499.05} & 
0.62\\CON3-7 & \bf{576.48} & 6.60 & 
588.74 & 6.64 & 576.48 & 0.00\\
CON3-8 & 524.59 & 7.54 & 
530.33 & 7.45 & \bf{523.05} & 
0.29\\CON3-9 & 588.18 & 6.47 & 
590.09 & 6.42 & \bf{578.24} & 
1.72\\CON8-0 & 873.62 & 6.84 & 
908.50 & 6.25 & \bf{857.17} & 
1.92\\CON8-1 & 759.18 & 7.29 & 
770.33 & 6.72 & \bf{740.85} & 
2.47\\CON8-2 & 727.20 & 7.57 & 
733.46 & 6.78 & \bf{712.89} & 
2.01\\CON8-3 & 836.46 & 6.38 & 
842.99 & 6.24 & \bf{811.07} & 
3.13\\CON8-4 & 792.76 & 5.14 & 
810.67 & 5.96 & \bf{772.25} & 
2.66\\CON8-5 & 769.55 & 5.44 & 
776.99 & 6.70 & \bf{754.88} & 
1.94\\CON8-6 & 699.30 & 8.09 & 
703.11 & 6.44 & \bf{678.92} & 
3.00\\CON8-7 & 816.07 & 6.59 & 
838.78 & 5.82 & \bf{811.96} & 
0.51\\CON8-8 & 791.06 & 5.32 & 
798.75 & 6.02 & \bf{767.53} & 
3.07\\CON8-9 & 827.71 & 8.96 & 
837.66 & 6.88 & \bf{809.00} & 
2.31\\[1ex]\hline
\end{tabular}
\label{table:nonlin}
\end{table} \clearpage
\begin{table}[ht]
\caption{Resultados de la ejecución de la metaheurística ILS, utilizando instancias de SalhiNagy con la configuración -n 45.0 -LS 40.0}
\centering
\small
\begin{tabular}{c c c c c c c}
\hline\hline
Instancia & Costo mínimo & Tiempo(seg.) & Costo promedio & Tiempo promedio(seg.) & Costo ILS & \%Gap \\ [0.5ex]
\hline
CMT1X & 471.25 & 4.64 & 
476.48 & 5.66 & \bf{466.77} & 
0.96\\CMT1Y & 478.39 & 6.99 & 
485.67 & 6.93 & \bf{466.77} & 
2.49\\CMT2X & 707.59 & 11.61 & 
715.32 & 13.54 & \bf{684.21} & 
3.42\\CMT2Y & 707.83 & 14.63 & 
711.33 & 13.83 & \bf{684.21} & 
3.45\\CMT3X & 730.46 & 39.50 & 
734.32 & 34.94 & \bf{721.40} & 
1.26\\CMT3Y & 731.91 & 33.50 & 
733.70 & 34.05 & \bf{721.40} & 
1.46\\CMT4X & 871.26 & 84.44 & 
898.88 & 92.08 & \bf{852.83} & 
2.16\\CMT4Y & 887.72 & 85.79 & 
901.17 & 92.97 & \bf{852.46} & 
4.14\\CMT5X & 1093.28 & 201.98 & 
1106.19 & 202.75 & \bf{1030.55} & 
6.09\\CMT5Y & 1107.75 & 193.46 & 
1115.09 & 194.77 & \bf{1031.17} & 
7.43\\CMT11X & 885.66 & 64.61 & 
889.90 & 78.42 & \bf{839.39} & 
5.51\\CMT11Y & 880.85 & 65.32 & 
892.69 & 71.95 & \bf{841.88} & 
4.63\\CMT12X & 677.06 & 27.70 & 
680.90 & 29.34 & \bf{662.22} & 
2.24\\CMT12Y & 674.15 & 30.19 & 
680.73 & 31.36 & \bf{662.22} & 
1.80\\[1ex]\hline
\end{tabular}
\label{table:nonlin}
\end{table} \clearpage
\begin{table}[ht]
\caption{Resultados de la ejecución de la metaheurística ILS, utilizando instancias de Dethloff con la configuración -n 45.0 -LS 50.0}
\centering
\small
\begin{tabular}{c c c c c c c}
\hline\hline
Instancia & Costo mínimo & Tiempo(seg.) & Costo promedio & Tiempo promedio(seg.) & Costo ILS & \%Gap \\ [0.5ex]
\hline
SCA3-0 & 640.55 & 8.54 & 
641.96 & 8.81 & \bf{635.62} & 
0.78\\SCA3-1 & \bf{697.84} & 9.50 & 
709.19 & 9.06 & 697.84 & 0.00\\
SCA3-2 & 664.18 & 7.73 & 
665.41 & 8.44 & \bf{659.34} & 
0.73\\SCA3-3 & \bf{680.04} & 9.95 & 
681.38 & 8.96 & 680.04 & 0.00\\
SCA3-4 & \bf{690.50} & 9.59 & 
693.82 & 9.09 & 690.50 & 0.00\\
SCA3-5 & 662.75 & 7.54 & 
674.17 & 7.89 & \bf{659.90} & 
0.43\\SCA3-6 & 652.94 & 9.84 & 
654.28 & 9.11 & \bf{651.09} & 
0.28\\SCA3-7 & 671.77 & 8.36 & 
671.77 & 8.31 & \bf{659.17} & 
1.91\\SCA3-8 & \bf{719.47} & 7.66 & 
735.78 & 7.79 & 719.47 & 0.00\\
SCA3-9 & \bf{681.00} & 7.20 & 
683.25 & 8.52 & 681.00 & 0.00\\
SCA8-0 & 974.62 & 7.69 & 
1010.10 & 7.31 & \bf{961.50} & 
1.36\\SCA8-1 & 1069.65 & 8.84 & 
1077.53 & 7.50 & \bf{1049.65} & 
1.91\\SCA8-2 & 1054.47 & 6.75 & 
1068.50 & 7.25 & \bf{1039.64} & 
1.43\\SCA8-3 & 1013.02 & 7.59 & 
1018.40 & 7.07 & \bf{983.34} & 
3.02\\SCA8-4 & 1082.87 & 7.60 & 
1092.04 & 7.38 & \bf{1065.49} & 
1.63\\SCA8-5 & 1058.56 & 10.55 & 
1071.11 & 8.28 & \bf{1027.08} & 
3.06\\SCA8-6 & 1001.38 & 6.92 & 
1011.31 & 6.67 & \bf{971.82} & 
3.04\\SCA8-7 & 1085.83 & 7.84 & 
1098.00 & 6.71 & \bf{1051.28} & 
3.29\\SCA8-8 & 1082.12 & 7.34 & 
1091.25 & 6.71 & \bf{1071.18} & 
1.02\\SCA8-9 & 1077.44 & 6.25 & 
1092.76 & 7.18 & \bf{1060.50} & 
1.60\\CON3-0 & 632.57 & 7.13 & 
634.73 & 8.11 & \bf{616.52} & 
2.60\\CON3-1 & 560.75 & 8.74 & 
564.83 & 8.99 & \bf{554.47} & 
1.13\\CON3-2 & 521.63 & 8.02 & 
524.42 & 8.13 & \bf{518.00} & 
0.70\\CON3-3 & 592.41 & 8.96 & 
601.88 & 8.70 & \bf{591.19} & 
0.21\\CON3-4 & 594.59 & 8.78 & 
599.31 & 8.64 & \bf{588.79} & 
0.99\\CON3-5 & 564.88 & 8.28 & 
571.11 & 8.37 & \bf{563.70} & 
0.21\\CON3-6 & 502.16 & 9.38 & 
506.80 & 9.40 & \bf{499.05} & 
0.62\\CON3-7 & 578.41 & 8.28 & 
585.87 & 8.53 & \bf{576.48} & 
0.33\\CON3-8 & 524.59 & 10.55 & 
528.86 & 9.23 & \bf{523.05} & 
0.29\\CON3-9 & 584.52 & 7.57 & 
588.16 & 8.20 & \bf{578.24} & 
1.09\\CON8-0 & 870.49 & 6.72 & 
897.61 & 6.55 & \bf{857.17} & 
1.55\\CON8-1 & 759.65 & 7.41 & 
771.95 & 7.11 & \bf{740.85} & 
2.54\\CON8-2 & 719.90 & 6.67 & 
725.53 & 8.22 & \bf{712.89} & 
0.98\\CON8-3 & 827.08 & 6.46 & 
843.69 & 7.59 & \bf{811.07} & 
1.97\\CON8-4 & 799.56 & 6.64 & 
810.19 & 6.67 & \bf{772.25} & 
3.54\\CON8-5 & 758.12 & 7.17 & 
786.46 & 8.65 & \bf{754.88} & 
0.43\\CON8-6 & 704.00 & 7.20 & 
713.08 & 6.87 & \bf{678.92} & 
3.69\\CON8-7 & 814.79 & 7.36 & 
824.26 & 6.87 & \bf{811.96} & 
0.35\\CON8-8 & 783.63 & 6.83 & 
792.56 & 7.20 & \bf{767.53} & 
2.10\\CON8-9 & 820.45 & 8.60 & 
831.34 & 8.03 & \bf{809.00} & 
1.42\\[1ex]\hline
\end{tabular}
\label{table:nonlin}
\end{table} \clearpage
\begin{table}[ht]
\caption{Resultados de la ejecución de la metaheurística ILS, utilizando instancias de SalhiNagy con la configuración -n 45.0 -LS 50.0}
\centering
\small
\begin{tabular}{c c c c c c c}
\hline\hline
Instancia & Costo mínimo & Tiempo(seg.) & Costo promedio & Tiempo promedio(seg.) & Costo ILS & \%Gap \\ [0.5ex]
\hline
CMT1X & 478.41 & 6.76 & 
482.16 & 7.97 & \bf{466.77} & 
2.49\\CMT1Y & 470.67 & 7.86 & 
478.98 & 6.75 & \bf{466.77} & 
0.84\\CMT2X & 708.97 & 15.84 & 
714.20 & 16.29 & \bf{684.21} & 
3.62\\CMT2Y & 705.28 & 15.65 & 
709.29 & 16.45 & \bf{684.21} & 
3.08\\CMT3X & 727.57 & 36.17 & 
735.28 & 38.44 & \bf{721.40} & 
0.86\\CMT3Y & 734.44 & 37.77 & 
735.90 & 38.87 & \bf{721.40} & 
1.81\\CMT4X & 897.37 & 95.09 & 
905.72 & 95.37 & \bf{852.83} & 
5.22\\CMT4Y & 892.42 & 97.52 & 
901.66 & 99.05 & \bf{852.46} & 
4.69\\CMT5X & 1097.00 & 224.34 & 
1104.52 & 237.39 & \bf{1030.55} & 
6.45\\CMT5Y & 1090.74 & 218.41 & 
1105.05 & 241.63 & \bf{1031.17} & 
5.78\\CMT11X & 884.30 & 96.73 & 
896.86 & 82.06 & \bf{839.39} & 
5.35\\CMT11Y & 871.64 & 74.98 & 
880.69 & 81.72 & \bf{841.88} & 
3.53\\CMT12X & 674.04 & 33.91 & 
678.68 & 34.95 & \bf{662.22} & 
1.78\\CMT12Y & 677.38 & 38.88 & 
680.94 & 35.94 & \bf{662.22} & 
2.29\\[1ex]\hline
\end{tabular}
\label{table:nonlin}
\end{table} \clearpage
\begin{table}[ht]
\caption{Resultados de la ejecución de la metaheurística ILS, utilizando instancias de Dethloff con la configuración -n 45.0 -LS 60.0}
\centering
\small
\begin{tabular}{c c c c c c c}
\hline\hline
Instancia & Costo mínimo & Tiempo(seg.) & Costo promedio & Tiempo promedio(seg.) & Costo ILS & \%Gap \\ [0.5ex]
\hline
SCA3-0 & 640.55 & 9.08 & 
641.49 & 10.60 & \bf{635.62} & 
0.78\\SCA3-1 & \bf{697.84} & 11.24 & 
702.33 & 10.77 & 697.84 & 0.00\\
SCA3-2 & 664.21 & 9.75 & 
666.45 & 9.97 & \bf{659.34} & 
0.74\\SCA3-3 & 680.60 & 10.93 & 
681.35 & 10.82 & \bf{680.04} & 
0.08\\SCA3-4 & \bf{690.50} & 10.88 & 
691.02 & 10.46 & 690.50 & 0.00\\
SCA3-5 & 661.07 & 10.86 & 
666.70 & 10.28 & \bf{659.90} & 
0.18\\SCA3-6 & \bf{651.09} & 11.29 & 
653.74 & 10.03 & 651.09 & 0.00\\
SCA3-7 & 667.24 & 9.31 & 
673.57 & 9.36 & \bf{659.17} & 
1.22\\SCA3-8 & \bf{719.47} & 9.00 & 
728.76 & 9.78 & 719.47 & 0.00\\
SCA3-9 & \bf{681.00} & 8.84 & 
685.32 & 9.53 & 681.00 & 0.00\\
SCA8-0 & 984.48 & 8.78 & 
1010.62 & 8.56 & \bf{961.50} & 
2.39\\SCA8-1 & 1073.93 & 6.70 & 
1083.73 & 7.60 & \bf{1049.65} & 
2.31\\SCA8-2 & 1064.91 & 8.14 & 
1068.18 & 8.55 & \bf{1039.64} & 
2.43\\SCA8-3 & 1004.78 & 8.02 & 
1030.04 & 7.91 & \bf{983.34} & 
2.18\\SCA8-4 & 1080.34 & 7.56 & 
1089.94 & 8.13 & \bf{1065.49} & 
1.39\\SCA8-5 & 1049.44 & 8.34 & 
1072.51 & 8.43 & \bf{1027.08} & 
2.18\\SCA8-6 & 991.27 & 7.73 & 
1003.27 & 8.10 & \bf{971.82} & 
2.00\\SCA8-7 & 1071.53 & 8.00 & 
1084.18 & 9.24 & \bf{1051.28} & 
1.93\\SCA8-8 & 1085.91 & 7.16 & 
1094.09 & 7.25 & \bf{1071.18} & 
1.38\\SCA8-9 & 1079.22 & 8.24 & 
1097.08 & 8.04 & \bf{1060.50} & 
1.77\\CON3-0 & 619.09 & 9.68 & 
631.68 & 9.90 & \bf{616.52} & 
0.42\\CON3-1 & 560.75 & 10.59 & 
562.06 & 10.38 & \bf{554.47} & 
1.13\\CON3-2 & 521.38 & 11.33 & 
525.88 & 10.57 & \bf{518.00} & 
0.65\\CON3-3 & 591.20 & 11.45 & 
599.15 & 10.84 & \bf{591.19} & 
0.00\\CON3-4 & 591.43 & 11.38 & 
594.84 & 10.16 & \bf{588.79} & 
0.45\\CON3-5 & 567.94 & 9.20 & 
572.42 & 10.10 & \bf{563.70} & 
0.75\\CON3-6 & 502.16 & 10.08 & 
504.49 & 10.86 & \bf{499.05} & 
0.62\\CON3-7 & 586.01 & 11.76 & 
591.83 & 9.80 & \bf{576.48} & 
1.65\\CON3-8 & 524.59 & 10.78 & 
525.64 & 10.50 & \bf{523.05} & 
0.29\\CON3-9 & 590.17 & 11.08 & 
590.57 & 10.04 & \bf{578.24} & 
2.06\\CON8-0 & 873.31 & 8.30 & 
894.91 & 8.10 & \bf{857.17} & 
1.88\\CON8-1 & 763.29 & 8.84 & 
769.31 & 8.16 & \bf{740.85} & 
3.03\\CON8-2 & 720.09 & 8.70 & 
731.88 & 9.08 & \bf{712.89} & 
1.01\\CON8-3 & 830.53 & 8.76 & 
837.94 & 8.81 & \bf{811.07} & 
2.40\\CON8-4 & 779.97 & 7.62 & 
786.59 & 7.79 & \bf{772.25} & 
1.00\\CON8-5 & 758.84 & 8.70 & 
772.86 & 8.34 & \bf{754.88} & 
0.52\\CON8-6 & 699.30 & 9.81 & 
706.80 & 9.28 & \bf{678.92} & 
3.00\\CON8-7 & 824.75 & 8.12 & 
842.00 & 8.61 & \bf{811.96} & 
1.58\\CON8-8 & 791.03 & 8.46 & 
793.64 & 8.64 & \bf{767.53} & 
3.06\\CON8-9 & 816.55 & 9.31 & 
833.28 & 8.93 & \bf{809.00} & 
0.93\\[1ex]\hline
\end{tabular}
\label{table:nonlin}
\end{table} \clearpage
\begin{table}[ht]
\caption{Resultados de la ejecución de la metaheurística ILS, utilizando instancias de SalhiNagy con la configuración -n 45.0 -LS 60.0}
\centering
\small
\begin{tabular}{c c c c c c c}
\hline\hline
Instancia & Costo mínimo & Tiempo(seg.) & Costo promedio & Tiempo promedio(seg.) & Costo ILS & \%Gap \\ [0.5ex]
\hline
CMT1X & 475.37 & 8.24 & 
481.20 & 7.81 & \bf{466.77} & 
1.84\\CMT1Y & 473.62 & 9.70 & 
482.98 & 9.09 & \bf{466.77} & 
1.47\\CMT2X & 703.62 & 20.52 & 
707.26 & 20.84 & \bf{684.21} & 
2.84\\CMT2Y & 698.05 & 16.05 & 
709.31 & 18.54 & \bf{684.21} & 
2.02\\CMT3X & 729.48 & 43.02 & 
732.36 & 43.17 & \bf{721.40} & 
1.12\\CMT3Y & 729.30 & 51.33 & 
735.71 & 50.73 & \bf{721.40} & 
1.10\\CMT4X & 889.21 & 103.43 & 
898.44 & 124.19 & \bf{852.83} & 
4.27\\CMT4Y & 879.41 & 110.01 & 
889.53 & 107.47 & \bf{852.46} & 
3.16\\CMT5X & 1101.64 & 282.21 & 
1108.62 & 255.22 & \bf{1030.55} & 
6.90\\CMT5Y & 1086.52 & 258.09 & 
1099.92 & 271.40 & \bf{1031.17} & 
5.37\\CMT11X & 882.77 & 87.68 & 
893.27 & 85.80 & \bf{839.39} & 
5.17\\CMT11Y & 875.29 & 102.34 & 
884.62 & 84.93 & \bf{841.88} & 
3.97\\CMT12X & 680.73 & 36.53 & 
682.34 & 39.59 & \bf{662.22} & 
2.80\\CMT12Y & 680.21 & 64.25 & 
681.03 & 49.13 & \bf{662.22} & 
2.72\\[1ex]\hline
\end{tabular}
\label{table:nonlin}
\end{table} \clearpage
\begin{table}[ht]
\caption{Resultados de la ejecución de la metaheurística ILS, utilizando instancias de Dethloff con la configuración -n 45.0 -LS 70.0}
\centering
\small
\begin{tabular}{c c c c c c c}
\hline\hline
Instancia & Costo mínimo & Tiempo(seg.) & Costo promedio & Tiempo promedio(seg.) & Costo ILS & \%Gap \\ [0.5ex]
\hline
SCA3-0 & 636.06 & 12.22 & 
639.43 & 11.75 & \bf{635.62} & 
0.07\\SCA3-1 & 701.53 & 11.64 & 
705.30 & 11.28 & \bf{697.84} & 
0.53\\SCA3-2 & 664.21 & 10.81 & 
665.50 & 11.51 & \bf{659.34} & 
0.74\\SCA3-3 & \bf{680.04} & 12.15 & 
680.92 & 12.59 & 680.04 & 0.00\\
SCA3-4 & \bf{690.50} & 12.02 & 
692.55 & 11.01 & 690.50 & 0.00\\
SCA3-5 & \bf{659.90} & 13.17 & 
669.14 & 11.34 & 659.90 & 0.00\\
SCA3-6 & 653.93 & 11.13 & 
654.87 & 10.96 & \bf{651.09} & 
0.44\\SCA3-7 & 669.89 & 10.90 & 
670.94 & 10.91 & \bf{659.17} & 
1.63\\SCA3-8 & \bf{719.47} & 13.42 & 
723.54 & 11.63 & 719.47 & 0.00\\
SCA3-9 & 684.25 & 10.12 & 
688.47 & 11.47 & \bf{681.00} & 
0.48\\SCA8-0 & 982.58 & 8.32 & 
1003.51 & 9.16 & \bf{961.50} & 
2.19\\SCA8-1 & 1067.92 & 8.54 & 
1085.79 & 9.95 & \bf{1049.65} & 
1.74\\SCA8-2 & 1065.76 & 9.68 & 
1075.82 & 8.63 & \bf{1039.64} & 
2.51\\SCA8-3 & 1004.78 & 11.08 & 
1021.59 & 9.99 & \bf{983.34} & 
2.18\\SCA8-4 & 1077.80 & 8.01 & 
1085.46 & 8.90 & \bf{1065.49} & 
1.16\\SCA8-5 & 1054.03 & 9.29 & 
1069.97 & 10.47 & \bf{1027.08} & 
2.62\\SCA8-6 & 979.28 & 10.73 & 
997.70 & 9.91 & \bf{971.82} & 
0.77\\SCA8-7 & 1064.45 & 9.72 & 
1078.18 & 9.61 & \bf{1051.28} & 
1.25\\SCA8-8 & 1086.77 & 9.04 & 
1088.86 & 9.27 & \bf{1071.18} & 
1.46\\SCA8-9 & 1094.93 & 7.82 & 
1103.21 & 9.10 & \bf{1060.50} & 
3.25\\CON3-0 & 617.59 & 11.05 & 
630.25 & 11.19 & \bf{616.52} & 
0.17\\CON3-1 & 560.75 & 12.77 & 
562.47 & 12.56 & \bf{554.47} & 
1.13\\CON3-2 & 521.38 & 11.78 & 
521.96 & 12.14 & \bf{518.00} & 
0.65\\CON3-3 & \bf{591.19} & 12.46 & 
593.78 & 11.81 & 591.19 & 0.00\\
CON3-4 & 591.43 & 12.70 & 
595.25 & 10.74 & \bf{588.79} & 
0.45\\CON3-5 & 569.57 & 11.28 & 
574.96 & 11.78 & \bf{563.70} & 
1.04\\CON3-6 & 502.16 & 9.98 & 
505.14 & 11.28 & \bf{499.05} & 
0.62\\CON3-7 & 578.41 & 12.78 & 
584.40 & 11.55 & \bf{576.48} & 
0.33\\CON3-8 & 524.59 & 16.88 & 
526.36 & 13.27 & \bf{523.05} & 
0.29\\CON3-9 & 589.72 & 12.00 & 
591.80 & 11.75 & \bf{578.24} & 
1.99\\CON8-0 & 878.94 & 9.56 & 
887.88 & 9.95 & \bf{857.17} & 
2.54\\CON8-1 & 754.51 & 10.05 & 
765.19 & 10.44 & \bf{740.85} & 
1.84\\CON8-2 & 718.78 & 8.96 & 
726.68 & 9.86 & \bf{712.89} & 
0.83\\CON8-3 & 825.62 & 9.84 & 
836.72 & 9.65 & \bf{811.07} & 
1.79\\CON8-4 & 795.72 & 10.24 & 
806.65 & 9.16 & \bf{772.25} & 
3.04\\CON8-5 & 762.36 & 8.72 & 
778.55 & 9.29 & \bf{754.88} & 
0.99\\CON8-6 & 693.19 & 8.08 & 
702.67 & 9.52 & \bf{678.92} & 
2.10\\CON8-7 & 815.14 & 10.62 & 
828.24 & 10.05 & \bf{811.96} & 
0.39\\CON8-8 & 782.09 & 8.70 & 
786.27 & 9.80 & \bf{767.53} & 
1.90\\CON8-9 & 814.57 & 11.46 & 
822.56 & 9.77 & \bf{809.00} & 
0.69\\[1ex]\hline
\end{tabular}
\label{table:nonlin}
\end{table} \clearpage
\begin{table}[ht]
\caption{Resultados de la ejecución de la metaheurística ILS, utilizando instancias de SalhiNagy con la configuración -n 45.0 -LS 70.0}
\centering
\small
\begin{tabular}{c c c c c c c}
\hline\hline
Instancia & Costo mínimo & Tiempo(seg.) & Costo promedio & Tiempo promedio(seg.) & Costo ILS & \%Gap \\ [0.5ex]
\hline
CMT1X & 476.41 & 11.52 & 
479.61 & 10.35 & \bf{466.77} & 
2.07\\CMT1Y & 479.25 & 10.62 & 
482.76 & 9.47 & \bf{466.77} & 
2.67\\CMT2X & 697.29 & 22.22 & 
704.10 & 21.70 & \bf{684.21} & 
1.91\\CMT2Y & 699.75 & 21.46 & 
706.00 & 20.80 & \bf{684.21} & 
2.27\\CMT3X & 727.38 & 45.54 & 
730.67 & 47.32 & \bf{721.40} & 
0.83\\CMT3Y & 733.41 & 47.49 & 
736.27 & 48.55 & \bf{721.40} & 
1.66\\CMT4X & 886.76 & 121.26 & 
891.80 & 133.63 & \bf{852.83} & 
3.98\\CMT4Y & 896.68 & 124.91 & 
900.97 & 140.72 & \bf{852.46} & 
5.19\\CMT5X & 1102.12 & 337.96 & 
1109.63 & 291.88 & \bf{1030.55} & 
6.94\\CMT5Y & 1091.12 & 340.11 & 
1102.90 & 312.98 & \bf{1031.17} & 
5.81\\CMT11X & 882.96 & 109.13 & 
889.97 & 102.02 & \bf{839.39} & 
5.19\\CMT11Y & 868.75 & 121.28 & 
885.03 & 97.53 & \bf{841.88} & 
3.19\\CMT12X & 681.60 & 52.57 & 
684.32 & 49.90 & \bf{662.22} & 
2.93\\CMT12Y & 675.06 & 51.52 & 
680.42 & 44.93 & \bf{662.22} & 
1.94\\[1ex]\hline
\end{tabular}
\label{table:nonlin}
\end{table} \clearpage
\begin{table}[ht]
\caption{Resultados de la ejecución de la metaheurística ILS, utilizando instancias de Dethloff con la configuración -n 45.0 -LS 80.0}
\centering
\small
\begin{tabular}{c c c c c c c}
\hline\hline
Instancia & Costo mínimo & Tiempo(seg.) & Costo promedio & Tiempo promedio(seg.) & Costo ILS & \%Gap \\ [0.5ex]
\hline
SCA3-0 & 636.06 & 13.90 & 
639.43 & 13.64 & \bf{635.62} & 
0.07\\SCA3-1 & \bf{697.84} & 14.45 & 
699.68 & 13.27 & 697.84 & 0.00\\
SCA3-2 & 661.13 & 12.22 & 
663.43 & 14.00 & \bf{659.34} & 
0.27\\SCA3-3 & 680.60 & 14.63 & 
681.24 & 14.40 & \bf{680.04} & 
0.08\\SCA3-4 & \bf{690.50} & 14.24 & 
690.50 & 13.84 & 690.50 & 0.00\\
SCA3-5 & \bf{659.90} & 13.53 & 
673.98 & 12.82 & 659.90 & 0.00\\
SCA3-6 & \bf{651.09} & 13.59 & 
653.56 & 13.28 & 651.09 & 0.00\\
SCA3-7 & 669.89 & 13.18 & 
671.49 & 13.48 & \bf{659.17} & 
1.63\\SCA3-8 & \bf{719.47} & 11.30 & 
720.84 & 12.36 & 719.47 & 0.00\\
SCA3-9 & \bf{681.00} & 15.84 & 
687.78 & 13.09 & 681.00 & 0.00\\
SCA8-0 & 970.64 & 11.96 & 
990.79 & 11.40 & \bf{961.50} & 
0.95\\SCA8-1 & 1072.23 & 11.15 & 
1088.81 & 10.21 & \bf{1049.65} & 
2.15\\SCA8-2 & 1042.69 & 11.85 & 
1052.74 & 10.86 & \bf{1039.64} & 
0.29\\SCA8-3 & 1013.07 & 10.60 & 
1033.68 & 10.48 & \bf{983.34} & 
3.02\\SCA8-4 & 1098.11 & 11.72 & 
1103.43 & 11.13 & \bf{1065.49} & 
3.06\\SCA8-5 & 1063.83 & 10.88 & 
1078.22 & 10.75 & \bf{1027.08} & 
3.58\\SCA8-6 & 989.31 & 11.56 & 
998.29 & 10.10 & \bf{971.82} & 
1.80\\SCA8-7 & 1089.02 & 11.89 & 
1096.99 & 11.48 & \bf{1051.28} & 
3.59\\SCA8-8 & 1084.41 & 13.34 & 
1090.98 & 11.50 & \bf{1071.18} & 
1.24\\SCA8-9 & 1067.42 & 11.77 & 
1081.58 & 10.45 & \bf{1060.50} & 
0.65\\CON3-0 & 628.47 & 14.26 & 
634.22 & 13.52 & \bf{616.52} & 
1.94\\CON3-1 & 560.75 & 13.66 & 
561.19 & 13.48 & \bf{554.47} & 
1.13\\CON3-2 & 521.38 & 13.78 & 
521.50 & 13.43 & \bf{518.00} & 
0.65\\CON3-3 & 591.20 & 13.93 & 
593.90 & 13.35 & \bf{591.19} & 
0.00\\CON3-4 & 593.78 & 12.10 & 
599.46 & 13.41 & \bf{588.79} & 
0.85\\CON3-5 & 569.88 & 14.42 & 
572.54 & 12.90 & \bf{563.70} & 
1.10\\CON3-6 & 502.16 & 14.34 & 
505.21 & 13.36 & \bf{499.05} & 
0.62\\CON3-7 & 578.41 & 11.72 & 
580.52 & 13.94 & \bf{576.48} & 
0.33\\CON3-8 & 524.59 & 13.58 & 
525.90 & 13.04 & \bf{523.05} & 
0.29\\CON3-9 & 588.11 & 12.90 & 
590.05 & 13.66 & \bf{578.24} & 
1.71\\CON8-0 & 865.86 & 9.88 & 
882.43 & 10.67 & \bf{857.17} & 
1.01\\CON8-1 & 754.51 & 10.07 & 
763.86 & 10.86 & \bf{740.85} & 
1.84\\CON8-2 & 719.56 & 11.57 & 
724.98 & 11.09 & \bf{712.89} & 
0.94\\CON8-3 & 835.95 & 10.66 & 
839.78 & 11.17 & \bf{811.07} & 
3.07\\CON8-4 & 798.09 & 10.61 & 
807.96 & 10.22 & \bf{772.25} & 
3.35\\CON8-5 & 758.99 & 12.20 & 
768.40 & 11.55 & \bf{754.88} & 
0.54\\CON8-6 & 685.45 & 11.93 & 
700.51 & 11.02 & \bf{678.92} & 
0.96\\CON8-7 & 820.92 & 10.09 & 
829.16 & 10.92 & \bf{811.96} & 
1.10\\CON8-8 & 785.14 & 12.57 & 
793.96 & 10.43 & \bf{767.53} & 
2.29\\CON8-9 & 838.04 & 10.08 & 
844.40 & 11.49 & \bf{809.00} & 
3.59\\[1ex]\hline
\end{tabular}
\label{table:nonlin}
\end{table} \clearpage
\begin{table}[ht]
\caption{Resultados de la ejecución de la metaheurística ILS, utilizando instancias de SalhiNagy con la configuración -n 45.0 -LS 80.0}
\centering
\small
\begin{tabular}{c c c c c c c}
\hline\hline
Instancia & Costo mínimo & Tiempo(seg.) & Costo promedio & Tiempo promedio(seg.) & Costo ILS & \%Gap \\ [0.5ex]
\hline
CMT1X & 470.67 & 9.30 & 
477.31 & 11.55 & \bf{466.77} & 
0.84\\CMT1Y & 480.08 & 12.01 & 
483.02 & 11.14 & \bf{466.77} & 
2.85\\CMT2X & 698.55 & 22.04 & 
707.85 & 23.34 & \bf{684.21} & 
2.10\\CMT2Y & 699.32 & 22.54 & 
707.05 & 22.33 & \bf{684.21} & 
2.21\\CMT3X & 737.18 & 51.76 & 
740.95 & 56.45 & \bf{721.40} & 
2.19\\CMT3Y & 734.76 & 61.43 & 
738.94 & 54.94 & \bf{721.40} & 
1.85\\CMT4X & 894.77 & 131.80 & 
897.18 & 142.61 & \bf{852.83} & 
4.92\\CMT4Y & 889.98 & 159.95 & 
894.48 & 142.74 & \bf{852.46} & 
4.40\\CMT5X & 1079.84 & 301.11 & 
1100.29 & 307.53 & \bf{1030.55} & 
4.78\\CMT5Y & 1090.28 & 354.27 & 
1100.26 & 321.39 & \bf{1031.17} & 
5.73\\CMT11X & 861.14 & 123.81 & 
881.90 & 111.82 & \bf{839.39} & 
2.59\\CMT11Y & 897.87 & 102.12 & 
905.93 & 107.41 & \bf{841.88} & 
6.65\\CMT12X & 682.01 & 52.28 & 
685.15 & 52.12 & \bf{662.22} & 
2.99\\CMT12Y & 673.59 & 57.44 & 
680.91 & 51.83 & \bf{662.22} & 
1.72\\[1ex]\hline
\end{tabular}
\label{table:nonlin}
\end{table} \clearpage
\begin{table}[ht]
\caption{Resultados de la ejecución de la metaheurística ILS, utilizando instancias de Dethloff con la configuración -n 55.0 -LS 10.0}
\centering
\small
\begin{tabular}{c c c c c c c}
\hline\hline
Instancia & Costo mínimo & Tiempo(seg.) & Costo promedio & Tiempo promedio(seg.) & Costo ILS & \%Gap \\ [0.5ex]
\hline
SCA3-0 & 640.55 & 4.11 & 
644.64 & 3.42 & \bf{635.62} & 
0.78\\SCA3-1 & \bf{697.84} & 3.50 & 
703.73 & 3.81 & 697.84 & 0.00\\
SCA3-2 & 661.13 & 3.57 & 
667.09 & 3.41 & \bf{659.34} & 
0.27\\SCA3-3 & 680.60 & 3.58 & 
681.16 & 3.68 & \bf{680.04} & 
0.08\\SCA3-4 & \bf{690.50} & 4.93 & 
700.65 & 3.58 & 690.50 & 0.00\\
SCA3-5 & 673.39 & 3.49 & 
677.86 & 3.65 & \bf{659.90} & 
2.04\\SCA3-6 & 652.94 & 2.92 & 
656.54 & 3.15 & \bf{651.09} & 
0.28\\SCA3-7 & 671.67 & 3.56 & 
671.86 & 3.32 & \bf{659.17} & 
1.90\\SCA3-8 & 719.77 & 3.36 & 
723.80 & 3.38 & \bf{719.47} & 
0.04\\SCA3-9 & 685.00 & 3.43 & 
688.84 & 3.44 & \bf{681.00} & 
0.59\\SCA8-0 & 977.93 & 3.46 & 
1015.19 & 3.50 & \bf{961.50} & 
1.71\\SCA8-1 & 1081.72 & 3.30 & 
1091.99 & 3.13 & \bf{1049.65} & 
3.06\\SCA8-2 & 1065.43 & 3.08 & 
1071.13 & 2.87 & \bf{1039.64} & 
2.48\\SCA8-3 & 1030.78 & 3.44 & 
1056.09 & 3.18 & \bf{983.34} & 
4.82\\SCA8-4 & 1095.76 & 2.78 & 
1127.76 & 2.86 & \bf{1065.49} & 
2.84\\SCA8-5 & 1078.65 & 3.86 & 
1092.81 & 3.31 & \bf{1027.08} & 
5.02\\SCA8-6 & 1005.01 & 3.79 & 
1022.01 & 3.42 & \bf{971.82} & 
3.42\\SCA8-7 & 1088.59 & 2.77 & 
1100.14 & 2.92 & \bf{1051.28} & 
3.55\\SCA8-8 & 1097.68 & 3.15 & 
1104.15 & 3.08 & \bf{1071.18} & 
2.47\\SCA8-9 & 1104.83 & 2.53 & 
1124.07 & 2.92 & \bf{1060.50} & 
4.18\\CON3-0 & 625.35 & 3.64 & 
633.19 & 3.58 & \bf{616.52} & 
1.43\\CON3-1 & 561.63 & 3.56 & 
563.12 & 3.64 & \bf{554.47} & 
1.29\\CON3-2 & 521.38 & 4.09 & 
528.34 & 3.50 & \bf{518.00} & 
0.65\\CON3-3 & 594.31 & 3.43 & 
606.44 & 3.39 & \bf{591.19} & 
0.53\\CON3-4 & 591.43 & 3.64 & 
608.94 & 3.85 & \bf{588.79} & 
0.45\\CON3-5 & 569.04 & 4.24 & 
572.44 & 3.81 & \bf{563.70} & 
0.95\\CON3-6 & 511.05 & 3.74 & 
514.37 & 3.45 & \bf{499.05} & 
2.40\\CON3-7 & 576.87 & 3.21 & 
584.77 & 3.25 & \bf{576.48} & 
0.07\\CON3-8 & \bf{523.05} & 3.39 & 
531.55 & 3.42 & 523.05 & 0.00\\
CON3-9 & 590.50 & 4.64 & 
592.45 & 3.74 & \bf{578.24} & 
2.12\\CON8-0 & 893.61 & 2.79 & 
903.69 & 2.86 & \bf{857.17} & 
4.25\\CON8-1 & 763.99 & 3.64 & 
768.45 & 3.40 & \bf{740.85} & 
3.12\\CON8-2 & 731.85 & 3.85 & 
739.77 & 3.60 & \bf{712.89} & 
2.66\\CON8-3 & 833.78 & 3.62 & 
840.26 & 3.41 & \bf{811.07} & 
2.80\\CON8-4 & 795.93 & 3.24 & 
808.00 & 3.21 & \bf{772.25} & 
3.07\\CON8-5 & 772.89 & 3.51 & 
776.65 & 3.23 & \bf{754.88} & 
2.39\\CON8-6 & 699.79 & 3.42 & 
707.08 & 3.23 & \bf{678.92} & 
3.07\\CON8-7 & 815.60 & 2.99 & 
839.10 & 3.34 & \bf{811.96} & 
0.45\\CON8-8 & 784.36 & 3.70 & 
801.54 & 3.12 & \bf{767.53} & 
2.19\\CON8-9 & 838.80 & 3.33 & 
842.97 & 3.44 & \bf{809.00} & 
3.68\\[1ex]\hline
\end{tabular}
\label{table:nonlin}
\end{table} \clearpage
\begin{table}[ht]
\caption{Resultados de la ejecución de la metaheurística ILS, utilizando instancias de SalhiNagy con la configuración -n 55.0 -LS 10.0}
\centering
\small
\begin{tabular}{c c c c c c c}
\hline\hline
Instancia & Costo mínimo & Tiempo(seg.) & Costo promedio & Tiempo promedio(seg.) & Costo ILS & \%Gap \\ [0.5ex]
\hline
CMT1X & 477.47 & 2.30 & 
486.59 & 2.96 & \bf{466.77} & 
2.29\\CMT1Y & 486.63 & 2.46 & 
489.68 & 3.05 & \bf{466.77} & 
4.25\\CMT2X & 705.69 & 8.20 & 
710.01 & 8.43 & \bf{684.21} & 
3.14\\CMT2Y & 709.71 & 9.21 & 
716.71 & 8.08 & \bf{684.21} & 
3.73\\CMT3X & 727.76 & 21.76 & 
733.67 & 21.55 & \bf{721.40} & 
0.88\\CMT3Y & 726.14 & 21.24 & 
734.23 & 21.77 & \bf{721.40} & 
0.66\\CMT4X & 883.82 & 100.83 & 
897.66 & 83.64 & \bf{852.83} & 
3.63\\CMT4Y & 885.78 & 67.68 & 
896.36 & 83.11 & \bf{852.46} & 
3.91\\CMT5X & 1089.68 & 161.79 & 
1107.14 & 169.18 & \bf{1030.55} & 
5.74\\CMT5Y & 1083.29 & 175.29 & 
1097.57 & 194.01 & \bf{1031.17} & 
5.05\\CMT11X & 881.85 & 88.17 & 
891.13 & 68.36 & \bf{839.39} & 
5.06\\CMT11Y & 855.88 & 53.37 & 
889.37 & 61.41 & \bf{841.88} & 
1.66\\CMT12X & 679.60 & 26.58 & 
688.33 & 20.86 & \bf{662.22} & 
2.62\\CMT12Y & 676.06 & 28.26 & 
682.80 & 23.64 & \bf{662.22} & 
2.09\\[1ex]\hline
\end{tabular}
\label{table:nonlin}
\end{table} \clearpage
\begin{table}[ht]
\caption{Resultados de la ejecución de la metaheurística ILS, utilizando instancias de Dethloff con la configuración -n 55.0 -LS 20.0}
\centering
\small
\begin{tabular}{c c c c c c c}
\hline\hline
Instancia & Costo mínimo & Tiempo(seg.) & Costo promedio & Tiempo promedio(seg.) & Costo ILS & \%Gap \\ [0.5ex]
\hline
SCA3-0 & 640.55 & 5.47 & 
641.42 & 5.44 & \bf{635.62} & 
0.78\\SCA3-1 & 700.50 & 5.26 & 
704.36 & 5.19 & \bf{697.84} & 
0.38\\SCA3-2 & \bf{659.34} & 3.44 & 
663.75 & 4.50 & 659.34 & 0.00\\
SCA3-3 & 680.60 & 5.82 & 
681.24 & 5.53 & \bf{680.04} & 
0.08\\SCA3-4 & \bf{690.50} & 5.52 & 
697.56 & 4.97 & 690.50 & 0.00\\
SCA3-5 & \bf{659.90} & 6.86 & 
679.01 & 5.42 & 659.90 & 0.00\\
SCA3-6 & \bf{651.09} & 5.50 & 
652.35 & 4.92 & 651.09 & 0.00\\
SCA3-7 & 667.24 & 4.70 & 
670.51 & 5.16 & \bf{659.17} & 
1.22\\SCA3-8 & 723.99 & 9.27 & 
728.17 & 6.05 & \bf{719.47} & 
0.63\\SCA3-9 & \bf{681.00} & 5.68 & 
689.72 & 5.22 & 681.00 & 0.00\\
SCA8-0 & 1006.52 & 5.79 & 
1032.53 & 5.54 & \bf{961.50} & 
4.68\\SCA8-1 & 1080.32 & 4.30 & 
1093.20 & 4.41 & \bf{1049.65} & 
2.92\\SCA8-2 & 1056.87 & 5.04 & 
1068.80 & 4.63 & \bf{1039.64} & 
1.66\\SCA8-3 & 1019.35 & 4.35 & 
1036.38 & 4.56 & \bf{983.34} & 
3.66\\SCA8-4 & 1088.71 & 3.94 & 
1110.86 & 4.30 & \bf{1065.49} & 
2.18\\SCA8-5 & 1064.02 & 4.90 & 
1071.52 & 5.00 & \bf{1027.08} & 
3.60\\SCA8-6 & 1001.19 & 3.08 & 
1011.74 & 4.36 & \bf{971.82} & 
3.02\\SCA8-7 & 1070.53 & 4.31 & 
1090.37 & 4.74 & \bf{1051.28} & 
1.83\\SCA8-8 & 1097.63 & 5.26 & 
1098.88 & 4.67 & \bf{1071.18} & 
2.47\\SCA8-9 & 1083.28 & 4.59 & 
1093.26 & 4.04 & \bf{1060.50} & 
2.15\\CON3-0 & 630.73 & 5.95 & 
635.34 & 5.18 & \bf{616.52} & 
2.30\\CON3-1 & 560.75 & 6.38 & 
560.75 & 6.10 & \bf{554.47} & 
1.13\\CON3-2 & 521.38 & 6.72 & 
526.02 & 5.67 & \bf{518.00} & 
0.65\\CON3-3 & \bf{591.19} & 6.44 & 
594.50 & 5.23 & 591.19 & 0.00\\
CON3-4 & 593.78 & 5.40 & 
601.78 & 4.94 & \bf{588.79} & 
0.85\\CON3-5 & 575.00 & 6.17 & 
577.63 & 4.87 & \bf{563.70} & 
2.00\\CON3-6 & 502.85 & 5.77 & 
507.73 & 5.37 & \bf{499.05} & 
0.76\\CON3-7 & 578.41 & 5.40 & 
587.29 & 5.08 & \bf{576.48} & 
0.33\\CON3-8 & 527.82 & 5.44 & 
531.75 & 5.44 & \bf{523.05} & 
0.91\\CON3-9 & 588.40 & 5.29 & 
590.61 & 5.21 & \bf{578.24} & 
1.76\\CON8-0 & 905.02 & 4.44 & 
921.03 & 4.33 & \bf{857.17} & 
5.58\\CON8-1 & 754.51 & 4.96 & 
786.11 & 4.53 & \bf{740.85} & 
1.84\\CON8-2 & 730.92 & 5.13 & 
735.01 & 5.24 & \bf{712.89} & 
2.53\\CON8-3 & 829.24 & 5.10 & 
846.52 & 4.72 & \bf{811.07} & 
2.24\\CON8-4 & 807.41 & 4.15 & 
819.18 & 4.79 & \bf{772.25} & 
4.55\\CON8-5 & 765.73 & 5.44 & 
776.59 & 5.22 & \bf{754.88} & 
1.44\\CON8-6 & 697.71 & 5.24 & 
704.56 & 5.09 & \bf{678.92} & 
2.77\\CON8-7 & 815.14 & 5.34 & 
835.88 & 4.62 & \bf{811.96} & 
0.39\\CON8-8 & 791.13 & 4.86 & 
802.87 & 5.07 & \bf{767.53} & 
3.07\\CON8-9 & 835.01 & 5.50 & 
847.93 & 4.66 & \bf{809.00} & 
3.22\\[1ex]\hline
\end{tabular}
\label{table:nonlin}
\end{table} \clearpage
\begin{table}[ht]
\caption{Resultados de la ejecución de la metaheurística ILS, utilizando instancias de SalhiNagy con la configuración -n 55.0 -LS 20.0}
\centering
\small
\begin{tabular}{c c c c c c c}
\hline\hline
Instancia & Costo mínimo & Tiempo(seg.) & Costo promedio & Tiempo promedio(seg.) & Costo ILS & \%Gap \\ [0.5ex]
\hline
CMT1X & 476.71 & 2.34 & 
486.17 & 4.37 & \bf{466.77} & 
2.13\\CMT1Y & 479.20 & 3.24 & 
483.69 & 4.39 & \bf{466.77} & 
2.66\\CMT2X & 695.35 & 14.24 & 
707.83 & 12.64 & \bf{684.21} & 
1.63\\CMT2Y & 709.60 & 12.39 & 
713.93 & 11.27 & \bf{684.21} & 
3.71\\CMT3X & 729.00 & 35.65 & 
737.48 & 29.73 & \bf{721.40} & 
1.05\\CMT3Y & 733.56 & 36.58 & 
736.72 & 31.85 & \bf{721.40} & 
1.69\\CMT4X & 896.39 & 109.45 & 
899.25 & 95.81 & \bf{852.83} & 
5.11\\CMT4Y & 897.05 & 117.95 & 
905.39 & 100.22 & \bf{852.46} & 
5.23\\CMT5X & 1098.51 & 277.19 & 
1111.41 & 238.47 & \bf{1030.55} & 
6.59\\CMT5Y & 1103.06 & 285.77 & 
1108.43 & 215.52 & \bf{1031.17} & 
6.97\\CMT11X & 880.50 & 62.37 & 
897.81 & 61.97 & \bf{839.39} & 
4.90\\CMT11Y & 882.97 & 64.30 & 
890.93 & 62.15 & \bf{841.88} & 
4.88\\CMT12X & 675.42 & 27.60 & 
685.18 & 26.97 & \bf{662.22} & 
1.99\\CMT12Y & 680.50 & 25.25 & 
684.41 & 26.45 & \bf{662.22} & 
2.76\\[1ex]\hline
\end{tabular}
\label{table:nonlin}
\end{table} \clearpage
\begin{table}[ht]
\caption{Resultados de la ejecución de la metaheurística ILS, utilizando instancias de Dethloff con la configuración -n 55.0 -LS 30.0}
\centering
\small
\begin{tabular}{c c c c c c c}
\hline\hline
Instancia & Costo mínimo & Tiempo(seg.) & Costo promedio & Tiempo promedio(seg.) & Costo ILS & \%Gap \\ [0.5ex]
\hline
SCA3-0 & 640.55 & 7.97 & 
641.89 & 7.56 & \bf{635.62} & 
0.78\\SCA3-1 & 701.53 & 6.86 & 
705.25 & 7.32 & \bf{697.84} & 
0.53\\SCA3-2 & \bf{659.34} & 7.30 & 
664.85 & 6.95 & 659.34 & 0.00\\
SCA3-3 & 681.16 & 7.56 & 
683.93 & 7.66 & \bf{680.04} & 
0.16\\SCA3-4 & \bf{690.50} & 7.87 & 
692.38 & 6.91 & 690.50 & 0.00\\
SCA3-5 & 673.39 & 5.64 & 
681.28 & 6.39 & \bf{659.90} & 
2.04\\SCA3-6 & 653.93 & 8.48 & 
658.38 & 6.42 & \bf{651.09} & 
0.44\\SCA3-7 & 671.67 & 6.65 & 
671.83 & 6.79 & \bf{659.17} & 
1.90\\SCA3-8 & 721.45 & 6.12 & 
727.19 & 6.71 & \bf{719.47} & 
0.28\\SCA3-9 & \bf{681.00} & 7.01 & 
688.71 & 6.57 & 681.00 & 0.00\\
SCA8-0 & 985.54 & 5.99 & 
1002.95 & 6.47 & \bf{961.50} & 
2.50\\SCA8-1 & 1077.98 & 5.79 & 
1085.13 & 5.71 & \bf{1049.65} & 
2.70\\SCA8-2 & 1064.50 & 4.90 & 
1072.19 & 5.86 & \bf{1039.64} & 
2.39\\SCA8-3 & 1025.28 & 5.53 & 
1041.36 & 5.29 & \bf{983.34} & 
4.27\\SCA8-4 & 1071.86 & 7.95 & 
1092.26 & 6.39 & \bf{1065.49} & 
0.60\\SCA8-5 & 1065.27 & 6.38 & 
1075.92 & 6.32 & \bf{1027.08} & 
3.72\\SCA8-6 & 993.42 & 6.02 & 
997.07 & 6.35 & \bf{971.82} & 
2.22\\SCA8-7 & 1074.37 & 7.42 & 
1087.78 & 6.53 & \bf{1051.28} & 
2.20\\SCA8-8 & 1080.58 & 4.89 & 
1097.07 & 6.32 & \bf{1071.18} & 
0.88\\SCA8-9 & 1079.39 & 7.70 & 
1095.62 & 6.51 & \bf{1060.50} & 
1.78\\CON3-0 & 637.14 & 6.36 & 
644.06 & 6.43 & \bf{616.52} & 
3.34\\CON3-1 & 560.75 & 7.06 & 
567.07 & 6.78 & \bf{554.47} & 
1.13\\CON3-2 & 521.63 & 6.40 & 
527.90 & 6.99 & \bf{518.00} & 
0.70\\CON3-3 & 597.34 & 8.67 & 
604.41 & 7.74 & \bf{591.19} & 
1.04\\CON3-4 & \bf{588.79} & 5.98 & 
600.12 & 6.06 & 588.79 & 0.00\\
CON3-5 & 564.88 & 7.69 & 
572.37 & 7.26 & \bf{563.70} & 
0.21\\CON3-6 & 503.97 & 6.97 & 
510.48 & 6.91 & \bf{499.05} & 
0.99\\CON3-7 & 578.22 & 6.19 & 
582.58 & 6.49 & \bf{576.48} & 
0.30\\CON3-8 & 524.30 & 7.18 & 
527.67 & 6.79 & \bf{523.05} & 
0.24\\CON3-9 & 588.40 & 6.99 & 
590.86 & 6.84 & \bf{578.24} & 
1.76\\CON8-0 & 887.49 & 6.06 & 
905.71 & 6.08 & \bf{857.17} & 
3.54\\CON8-1 & 754.65 & 7.29 & 
768.85 & 6.36 & \bf{740.85} & 
1.86\\CON8-2 & 729.05 & 6.39 & 
736.52 & 6.40 & \bf{712.89} & 
2.27\\CON8-3 & 818.41 & 6.13 & 
830.07 & 5.92 & \bf{811.07} & 
0.90\\CON8-4 & 803.08 & 6.90 & 
814.53 & 6.54 & \bf{772.25} & 
3.99\\CON8-5 & 762.34 & 6.60 & 
777.25 & 6.06 & \bf{754.88} & 
0.99\\CON8-6 & 694.03 & 5.56 & 
708.62 & 5.98 & \bf{678.92} & 
2.23\\CON8-7 & 826.35 & 6.37 & 
839.89 & 5.96 & \bf{811.96} & 
1.77\\CON8-8 & 789.64 & 7.10 & 
806.24 & 5.93 & \bf{767.53} & 
2.88\\CON8-9 & 823.63 & 5.42 & 
830.94 & 6.11 & \bf{809.00} & 
1.81\\[1ex]\hline
\end{tabular}
\label{table:nonlin}
\end{table} \clearpage
\begin{table}[ht]
\caption{Resultados de la ejecución de la metaheurística ILS, utilizando instancias de SalhiNagy con la configuración -n 55.0 -LS 30.0}
\centering
\small
\begin{tabular}{c c c c c c c}
\hline\hline
Instancia & Costo mínimo & Tiempo(seg.) & Costo promedio & Tiempo promedio(seg.) & Costo ILS & \%Gap \\ [0.5ex]
\hline
CMT1X & 479.89 & 6.14 & 
487.87 & 7.58 & \bf{466.77} & 
2.81\\CMT1Y & 471.25 & 7.08 & 
476.19 & 6.40 & \bf{466.77} & 
0.96\\CMT2X & 706.04 & 16.57 & 
714.08 & 15.54 & \bf{684.21} & 
3.19\\CMT2Y & 699.95 & 14.60 & 
707.48 & 14.03 & \bf{684.21} & 
2.30\\CMT3X & 734.44 & 32.81 & 
739.19 & 32.78 & \bf{721.40} & 
1.81\\CMT3Y & 730.16 & 33.31 & 
734.64 & 34.28 & \bf{721.40} & 
1.21\\CMT4X & 904.04 & 95.32 & 
906.48 & 92.12 & \bf{852.83} & 
6.00\\CMT4Y & 884.89 & 97.95 & 
895.42 & 103.42 & \bf{852.46} & 
3.80\\CMT5X & 1092.00 & 213.90 & 
1098.79 & 218.65 & \bf{1030.55} & 
5.96\\CMT5Y & 1082.54 & 207.84 & 
1099.09 & 219.28 & \bf{1031.17} & 
4.98\\CMT11X & 860.58 & 77.82 & 
882.38 & 81.09 & \bf{839.39} & 
2.52\\CMT11Y & 882.11 & 78.78 & 
891.53 & 82.31 & \bf{841.88} & 
4.78\\CMT12X & 677.55 & 31.75 & 
684.18 & 32.12 & \bf{662.22} & 
2.31\\CMT12Y & 675.33 & 30.36 & 
680.02 & 32.14 & \bf{662.22} & 
1.98\\[1ex]\hline
\end{tabular}
\label{table:nonlin}
\end{table} \clearpage
\begin{table}[ht]
\caption{Resultados de la ejecución de la metaheurística ILS, utilizando instancias de Dethloff con la configuración -n 55.0 -LS 40.0}
\centering
\small
\begin{tabular}{c c c c c c c}
\hline\hline
Instancia & Costo mínimo & Tiempo(seg.) & Costo promedio & Tiempo promedio(seg.) & Costo ILS & \%Gap \\ [0.5ex]
\hline
SCA3-0 & 640.55 & 8.78 & 
643.15 & 8.81 & \bf{635.62} & 
0.78\\SCA3-1 & 700.50 & 9.12 & 
704.36 & 8.38 & \bf{697.84} & 
0.38\\SCA3-2 & 664.18 & 10.02 & 
668.66 & 8.59 & \bf{659.34} & 
0.73\\SCA3-3 & \bf{680.04} & 7.87 & 
681.21 & 7.90 & 680.04 & 0.00\\
SCA3-4 & \bf{690.50} & 7.56 & 
697.88 & 7.89 & 690.50 & 0.00\\
SCA3-5 & 666.67 & 8.84 & 
673.11 & 9.06 & \bf{659.90} & 
1.03\\SCA3-6 & \bf{651.09} & 8.83 & 
654.19 & 8.28 & 651.09 & 0.00\\
SCA3-7 & 671.67 & 8.18 & 
671.72 & 8.79 & \bf{659.17} & 
1.90\\SCA3-8 & 719.77 & 7.76 & 
723.08 & 8.57 & \bf{719.47} & 
0.04\\SCA3-9 & \bf{681.00} & 9.16 & 
682.65 & 8.60 & 681.00 & 0.00\\
SCA8-0 & 1013.72 & 6.68 & 
1024.13 & 7.82 & \bf{961.50} & 
5.43\\SCA8-1 & 1062.35 & 6.65 & 
1083.20 & 6.86 & \bf{1049.65} & 
1.21\\SCA8-2 & 1055.65 & 6.85 & 
1066.67 & 6.75 & \bf{1039.64} & 
1.54\\SCA8-3 & 1017.97 & 6.73 & 
1025.57 & 7.32 & \bf{983.34} & 
3.52\\SCA8-4 & 1081.36 & 6.94 & 
1089.46 & 6.54 & \bf{1065.49} & 
1.49\\SCA8-5 & 1082.58 & 7.09 & 
1088.39 & 7.20 & \bf{1027.08} & 
5.40\\SCA8-6 & 981.41 & 6.97 & 
994.52 & 7.22 & \bf{971.82} & 
0.99\\SCA8-7 & 1082.76 & 7.40 & 
1090.41 & 7.30 & \bf{1051.28} & 
2.99\\SCA8-8 & \bf{1071.18} & 8.02 & 
1088.61 & 7.58 & 1071.18 & 0.00\\
SCA8-9 & 1079.82 & 6.39 & 
1101.23 & 6.53 & \bf{1060.50} & 
1.82\\CON3-0 & 633.86 & 8.36 & 
636.21 & 9.04 & \bf{616.52} & 
2.81\\CON3-1 & 558.16 & 8.55 & 
560.54 & 9.83 & \bf{554.47} & 
0.67\\CON3-2 & 521.38 & 8.41 & 
522.10 & 7.96 & \bf{518.00} & 
0.65\\CON3-3 & 591.20 & 9.60 & 
598.81 & 9.09 & \bf{591.19} & 
0.00\\CON3-4 & 595.40 & 7.70 & 
604.52 & 8.34 & \bf{588.79} & 
1.12\\CON3-5 & 567.94 & 8.63 & 
573.34 & 9.02 & \bf{563.70} & 
0.75\\CON3-6 & 504.09 & 9.22 & 
508.40 & 8.63 & \bf{499.05} & 
1.01\\CON3-7 & 592.52 & 8.98 & 
594.85 & 8.53 & \bf{576.48} & 
2.78\\CON3-8 & 523.14 & 9.52 & 
528.90 & 9.29 & \bf{523.05} & 
0.02\\CON3-9 & 588.99 & 7.36 & 
589.75 & 8.24 & \bf{578.24} & 
1.86\\CON8-0 & 876.12 & 7.94 & 
902.92 & 7.37 & \bf{857.17} & 
2.21\\CON8-1 & 754.51 & 7.00 & 
769.29 & 7.60 & \bf{740.85} & 
1.84\\CON8-2 & 727.18 & 8.20 & 
735.67 & 7.56 & \bf{712.89} & 
2.00\\CON8-3 & 824.03 & 8.44 & 
836.62 & 7.91 & \bf{811.07} & 
1.60\\CON8-4 & 781.78 & 7.02 & 
802.40 & 7.04 & \bf{772.25} & 
1.23\\CON8-5 & 758.84 & 7.16 & 
771.14 & 7.08 & \bf{754.88} & 
0.52\\CON8-6 & 701.20 & 9.42 & 
707.88 & 7.79 & \bf{678.92} & 
3.28\\CON8-7 & 822.22 & 7.32 & 
841.55 & 7.63 & \bf{811.96} & 
1.26\\CON8-8 & 782.40 & 8.67 & 
795.62 & 7.43 & \bf{767.53} & 
1.94\\CON8-9 & 826.47 & 6.77 & 
831.01 & 7.45 & \bf{809.00} & 
2.16\\[1ex]\hline
\end{tabular}
\label{table:nonlin}
\end{table} \clearpage
\begin{table}[ht]
\caption{Resultados de la ejecución de la metaheurística ILS, utilizando instancias de SalhiNagy con la configuración -n 55.0 -LS 40.0}
\centering
\small
\begin{tabular}{c c c c c c c}
\hline\hline
Instancia & Costo mínimo & Tiempo(seg.) & Costo promedio & Tiempo promedio(seg.) & Costo ILS & \%Gap \\ [0.5ex]
\hline
CMT1X & 470.67 & 6.01 & 
479.78 & 6.93 & \bf{466.77} & 
0.84\\CMT1Y & 475.71 & 7.17 & 
481.28 & 6.81 & \bf{466.77} & 
1.92\\CMT2X & 693.78 & 19.71 & 
706.89 & 21.07 & \bf{684.21} & 
1.40\\CMT2Y & 687.95 & 17.48 & 
704.28 & 17.59 & \bf{684.21} & 
0.55\\CMT3X & 732.97 & 38.95 & 
737.03 & 41.02 & \bf{721.40} & 
1.60\\CMT3Y & 725.60 & 50.40 & 
731.08 & 42.28 & \bf{721.40} & 
0.58\\CMT4X & 894.80 & 109.61 & 
898.52 & 105.34 & \bf{852.83} & 
4.92\\CMT4Y & 896.83 & 106.96 & 
905.00 & 108.51 & \bf{852.46} & 
5.20\\CMT5X & 1094.12 & 231.54 & 
1108.03 & 235.13 & \bf{1030.55} & 
6.17\\CMT5Y & 1089.38 & 232.81 & 
1101.41 & 266.93 & \bf{1031.17} & 
5.65\\CMT11X & 870.61 & 82.22 & 
879.31 & 82.70 & \bf{839.39} & 
3.72\\CMT11Y & 895.30 & 79.86 & 
896.97 & 93.92 & \bf{841.88} & 
6.35\\CMT12X & 685.31 & 46.27 & 
689.60 & 41.01 & \bf{662.22} & 
3.49\\CMT12Y & 674.92 & 35.18 & 
681.37 & 34.77 & \bf{662.22} & 
1.92\\[1ex]\hline
\end{tabular}
\label{table:nonlin}
\end{table} \clearpage
\begin{table}[ht]
\caption{Resultados de la ejecución de la metaheurística ILS, utilizando instancias de Dethloff con la configuración -n 55.0 -LS 50.0}
\centering
\small
\begin{tabular}{c c c c c c c}
\hline\hline
Instancia & Costo mínimo & Tiempo(seg.) & Costo promedio & Tiempo promedio(seg.) & Costo ILS & \%Gap \\ [0.5ex]
\hline
SCA3-0 & 636.06 & 10.10 & 
640.22 & 10.74 & \bf{635.62} & 
0.07\\SCA3-1 & \bf{697.84} & 9.73 & 
701.19 & 10.60 & 697.84 & 0.00\\
SCA3-2 & 664.18 & 10.53 & 
666.25 & 10.26 & \bf{659.34} & 
0.73\\SCA3-3 & \bf{680.04} & 11.49 & 
680.65 & 10.58 & 680.04 & 0.00\\
SCA3-4 & \bf{690.50} & 10.00 & 
691.18 & 10.50 & 690.50 & 0.00\\
SCA3-5 & 673.39 & 10.95 & 
677.82 & 10.55 & \bf{659.90} & 
2.04\\SCA3-6 & \bf{651.09} & 11.07 & 
653.19 & 11.13 & 651.09 & 0.00\\
SCA3-7 & 671.67 & 10.86 & 
671.72 & 10.06 & \bf{659.17} & 
1.90\\SCA3-8 & \bf{719.47} & 10.30 & 
721.21 & 11.45 & 719.47 & 0.00\\
SCA3-9 & \bf{681.00} & 10.93 & 
682.00 & 9.40 & 681.00 & 0.00\\
SCA8-0 & 975.32 & 11.08 & 
991.12 & 9.68 & \bf{961.50} & 
1.44\\SCA8-1 & 1054.81 & 9.39 & 
1073.76 & 7.95 & \bf{1049.65} & 
0.49\\SCA8-2 & 1063.96 & 9.21 & 
1067.40 & 8.95 & \bf{1039.64} & 
2.34\\SCA8-3 & 1026.39 & 11.20 & 
1034.57 & 8.74 & \bf{983.34} & 
4.38\\SCA8-4 & \bf{1065.49} & 8.71 & 
1090.01 & 7.61 & 1065.49 & 0.00\\
SCA8-5 & 1071.34 & 9.38 & 
1077.42 & 8.61 & \bf{1027.08} & 
4.31\\SCA8-6 & 986.90 & 11.76 & 
998.96 & 9.02 & \bf{971.82} & 
1.55\\SCA8-7 & 1075.36 & 8.38 & 
1090.14 & 8.86 & \bf{1051.28} & 
2.29\\SCA8-8 & 1082.11 & 9.59 & 
1097.35 & 8.37 & \bf{1071.18} & 
1.02\\SCA8-9 & 1087.05 & 7.41 & 
1093.12 & 7.78 & \bf{1060.50} & 
2.50\\CON3-0 & 629.51 & 10.11 & 
634.29 & 9.87 & \bf{616.52} & 
2.11\\CON3-1 & 560.75 & 10.54 & 
564.29 & 10.02 & \bf{554.47} & 
1.13\\CON3-2 & 521.38 & 9.42 & 
522.86 & 10.15 & \bf{518.00} & 
0.65\\CON3-3 & 591.20 & 10.25 & 
602.09 & 10.74 & \bf{591.19} & 
0.00\\CON3-4 & 591.43 & 11.83 & 
598.80 & 10.91 & \bf{588.79} & 
0.45\\CON3-5 & \bf{563.70} & 10.27 & 
569.73 & 10.79 & 563.70 & 0.00\\
CON3-6 & 502.16 & 10.32 & 
507.19 & 9.93 & \bf{499.05} & 
0.62\\CON3-7 & 577.68 & 10.35 & 
585.18 & 11.73 & \bf{576.48} & 
0.21\\CON3-8 & 524.59 & 11.52 & 
527.77 & 11.14 & \bf{523.05} & 
0.29\\CON3-9 & 588.40 & 10.00 & 
589.30 & 10.97 & \bf{578.24} & 
1.76\\CON8-0 & 866.11 & 8.04 & 
902.32 & 7.87 & \bf{857.17} & 
1.04\\CON8-1 & 766.65 & 11.36 & 
776.47 & 9.36 & \bf{740.85} & 
3.48\\CON8-2 & 722.80 & 8.11 & 
728.42 & 8.71 & \bf{712.89} & 
1.39\\CON8-3 & 826.26 & 10.98 & 
834.29 & 9.15 & \bf{811.07} & 
1.87\\CON8-4 & 794.75 & 8.81 & 
798.76 & 8.60 & \bf{772.25} & 
2.91\\CON8-5 & 763.13 & 9.94 & 
766.25 & 8.94 & \bf{754.88} & 
1.09\\CON8-6 & 696.41 & 9.64 & 
701.60 & 8.96 & \bf{678.92} & 
2.58\\CON8-7 & 814.79 & 9.10 & 
830.68 & 8.74 & \bf{811.96} & 
0.35\\CON8-8 & 783.01 & 8.63 & 
792.99 & 8.62 & \bf{767.53} & 
2.02\\CON8-9 & 827.05 & 7.53 & 
833.80 & 7.96 & \bf{809.00} & 
2.23\\[1ex]\hline
\end{tabular}
\label{table:nonlin}
\end{table} \clearpage
\begin{table}[ht]
\caption{Resultados de la ejecución de la metaheurística ILS, utilizando instancias de SalhiNagy con la configuración -n 55.0 -LS 50.0}
\centering
\small
\begin{tabular}{c c c c c c c}
\hline\hline
Instancia & Costo mínimo & Tiempo(seg.) & Costo promedio & Tiempo promedio(seg.) & Costo ILS & \%Gap \\ [0.5ex]
\hline
CMT1X & 470.67 & 5.59 & 
476.00 & 8.55 & \bf{466.77} & 
0.84\\CMT1Y & 472.85 & 6.00 & 
485.75 & 7.31 & \bf{466.77} & 
1.30\\CMT2X & 707.76 & 19.60 & 
711.20 & 20.16 & \bf{684.21} & 
3.44\\CMT2Y & 702.53 & 25.52 & 
708.70 & 20.82 & \bf{684.21} & 
2.68\\CMT3X & 730.13 & 56.76 & 
733.41 & 47.42 & \bf{721.40} & 
1.21\\CMT3Y & 729.62 & 56.16 & 
734.07 & 50.69 & \bf{721.40} & 
1.14\\CMT4X & 901.98 & 128.84 & 
907.36 & 120.86 & \bf{852.83} & 
5.76\\CMT4Y & 902.15 & 152.15 & 
906.81 & 128.03 & \bf{852.46} & 
5.83\\CMT5X & 1079.14 & 344.34 & 
1100.13 & 304.98 & \bf{1030.55} & 
4.71\\CMT5Y & 1092.86 & 361.21 & 
1099.81 & 298.55 & \bf{1031.17} & 
5.98\\CMT11X & 881.54 & 88.44 & 
886.79 & 93.00 & \bf{839.39} & 
5.02\\CMT11Y & 879.31 & 94.88 & 
887.50 & 97.19 & \bf{841.88} & 
4.45\\CMT12X & 676.87 & 43.21 & 
679.14 & 45.80 & \bf{662.22} & 
2.21\\CMT12Y & 677.39 & 53.95 & 
683.28 & 47.40 & \bf{662.22} & 
2.29\\[1ex]\hline
\end{tabular}
\label{table:nonlin}
\end{table} \clearpage
\begin{table}[ht]
\caption{Resultados de la ejecución de la metaheurística ILS, utilizando instancias de Dethloff con la configuración -n 55.0 -LS 60.0}
\centering
\small
\begin{tabular}{c c c c c c c}
\hline\hline
Instancia & Costo mínimo & Tiempo(seg.) & Costo promedio & Tiempo promedio(seg.) & Costo ILS & \%Gap \\ [0.5ex]
\hline
SCA3-0 & 636.06 & 10.86 & 
639.43 & 11.96 & \bf{635.62} & 
0.07\\SCA3-1 & \bf{697.84} & 13.18 & 
700.15 & 13.08 & 697.84 & 0.00\\
SCA3-2 & 664.21 & 12.73 & 
666.46 & 12.29 & \bf{659.34} & 
0.74\\SCA3-3 & \bf{680.04} & 12.31 & 
680.64 & 13.44 & 680.04 & 0.00\\
SCA3-4 & \bf{690.50} & 11.88 & 
691.18 & 11.36 & 690.50 & 0.00\\
SCA3-5 & \bf{659.90} & 12.92 & 
664.28 & 12.44 & 659.90 & 0.00\\
SCA3-6 & 652.94 & 12.50 & 
655.13 & 12.20 & \bf{651.09} & 
0.28\\SCA3-7 & \bf{659.17} & 13.66 & 
668.10 & 12.36 & 659.17 & 0.00\\
SCA3-8 & 722.05 & 11.83 & 
725.49 & 13.44 & \bf{719.47} & 
0.36\\SCA3-9 & \bf{681.00} & 12.13 & 
683.81 & 11.88 & 681.00 & 0.00\\
SCA8-0 & 990.52 & 9.94 & 
1013.45 & 10.56 & \bf{961.50} & 
3.02\\SCA8-1 & 1074.03 & 9.22 & 
1079.92 & 10.09 & \bf{1049.65} & 
2.32\\SCA8-2 & 1066.60 & 10.31 & 
1073.02 & 9.99 & \bf{1039.64} & 
2.59\\SCA8-3 & 1023.68 & 9.58 & 
1029.70 & 9.56 & \bf{983.34} & 
4.10\\SCA8-4 & 1107.90 & 9.34 & 
1113.82 & 9.54 & \bf{1065.49} & 
3.98\\SCA8-5 & 1060.43 & 9.80 & 
1064.70 & 10.30 & \bf{1027.08} & 
3.25\\SCA8-6 & 983.38 & 10.10 & 
998.47 & 10.23 & \bf{971.82} & 
1.19\\SCA8-7 & 1083.34 & 10.30 & 
1087.56 & 10.65 & \bf{1051.28} & 
3.05\\SCA8-8 & \bf{1071.18} & 8.90 & 
1082.11 & 10.26 & 1071.18 & 0.00\\
SCA8-9 & 1077.21 & 9.24 & 
1087.34 & 8.76 & \bf{1060.50} & 
1.58\\CON3-0 & 625.35 & 12.45 & 
630.31 & 12.37 & \bf{616.52} & 
1.43\\CON3-1 & 556.92 & 12.47 & 
561.46 & 12.18 & \bf{554.47} & 
0.44\\CON3-2 & 521.38 & 13.27 & 
524.48 & 11.73 & \bf{518.00} & 
0.65\\CON3-3 & 591.20 & 13.86 & 
591.92 & 13.18 & \bf{591.19} & 
0.00\\CON3-4 & \bf{588.79} & 13.48 & 
592.96 & 12.93 & 588.79 & 0.00\\
CON3-5 & 570.22 & 10.16 & 
572.96 & 11.78 & \bf{563.70} & 
1.16\\CON3-6 & 502.16 & 10.46 & 
505.13 & 12.55 & \bf{499.05} & 
0.62\\CON3-7 & 577.54 & 12.74 & 
580.67 & 12.45 & \bf{576.48} & 
0.18\\CON3-8 & 523.14 & 13.85 & 
524.73 & 12.46 & \bf{523.05} & 
0.02\\CON3-9 & 578.25 & 11.91 & 
586.42 & 12.33 & \bf{578.24} & 
0.00\\CON8-0 & 883.39 & 11.31 & 
897.05 & 10.39 & \bf{857.17} & 
3.06\\CON8-1 & 754.51 & 10.52 & 
764.58 & 10.13 & \bf{740.85} & 
1.84\\CON8-2 & 723.29 & 9.81 & 
730.03 & 10.30 & \bf{712.89} & 
1.46\\CON8-3 & 820.31 & 10.47 & 
832.65 & 10.36 & \bf{811.07} & 
1.14\\CON8-4 & 799.14 & 12.26 & 
808.67 & 10.70 & \bf{772.25} & 
3.48\\CON8-5 & 776.52 & 10.78 & 
778.13 & 10.36 & \bf{754.88} & 
2.87\\CON8-6 & 692.51 & 11.17 & 
697.33 & 10.54 & \bf{678.92} & 
2.00\\CON8-7 & 816.67 & 9.20 & 
821.47 & 10.27 & \bf{811.96} & 
0.58\\CON8-8 & 785.82 & 10.70 & 
791.37 & 10.49 & \bf{767.53} & 
2.38\\CON8-9 & 813.68 & 9.66 & 
828.39 & 9.85 & \bf{809.00} & 
0.58\\[1ex]\hline
\end{tabular}
\label{table:nonlin}
\end{table} \clearpage
\begin{table}[ht]
\caption{Resultados de la ejecución de la metaheurística ILS, utilizando instancias de SalhiNagy con la configuración -n 55.0 -LS 60.0}
\centering
\small
\begin{tabular}{c c c c c c c}
\hline\hline
Instancia & Costo mínimo & Tiempo(seg.) & Costo promedio & Tiempo promedio(seg.) & Costo ILS & \%Gap \\ [0.5ex]
\hline
CMT1X & 478.39 & 9.21 & 
485.06 & 9.69 & \bf{466.77} & 
2.49\\CMT1Y & 473.62 & 10.59 & 
476.21 & 10.13 & \bf{466.77} & 
1.47\\CMT2X & 703.21 & 23.15 & 
710.00 & 23.20 & \bf{684.21} & 
2.78\\CMT2Y & 698.65 & 25.38 & 
706.81 & 23.43 & \bf{684.21} & 
2.11\\CMT3X & 729.64 & 50.76 & 
737.43 & 54.39 & \bf{721.40} & 
1.14\\CMT3Y & 727.65 & 49.41 & 
734.35 & 57.92 & \bf{721.40} & 
0.87\\CMT4X & 883.95 & 133.22 & 
894.33 & 146.65 & \bf{852.83} & 
3.65\\CMT4Y & 897.89 & 125.88 & 
902.98 & 131.37 & \bf{852.46} & 
5.33\\CMT5X & 1092.25 & 306.73 & 
1103.41 & 297.02 & \bf{1030.55} & 
5.99\\CMT5Y & 1082.45 & 312.29 & 
1092.26 & 320.70 & \bf{1031.17} & 
4.97\\CMT11X & 867.53 & 129.11 & 
881.14 & 116.82 & \bf{839.39} & 
3.35\\CMT11Y & 871.04 & 93.81 & 
886.33 & 111.26 & \bf{841.88} & 
3.46\\CMT12X & 673.99 & 55.75 & 
675.62 & 49.33 & \bf{662.22} & 
1.78\\CMT12Y & 674.09 & 44.94 & 
679.17 & 45.63 & \bf{662.22} & 
1.79\\[1ex]\hline
\end{tabular}
\label{table:nonlin}
\end{table} \clearpage
\begin{table}[ht]
\caption{Resultados de la ejecución de la metaheurística ILS, utilizando instancias de Dethloff con la configuración -n 55.0 -LS 70.0}
\centering
\small
\begin{tabular}{c c c c c c c}
\hline\hline
Instancia & Costo mínimo & Tiempo(seg.) & Costo promedio & Tiempo promedio(seg.) & Costo ILS & \%Gap \\ [0.5ex]
\hline
SCA3-0 & 636.06 & 13.19 & 
639.43 & 13.64 & \bf{635.62} & 
0.07\\SCA3-1 & \bf{697.84} & 14.05 & 
703.68 & 13.92 & 697.84 & 0.00\\
SCA3-2 & 661.13 & 12.55 & 
664.80 & 13.95 & \bf{659.34} & 
0.27\\SCA3-3 & 680.60 & 14.15 & 
680.88 & 13.36 & \bf{680.04} & 
0.08\\SCA3-4 & \bf{690.50} & 16.02 & 
691.18 & 13.95 & 690.50 & 0.00\\
SCA3-5 & 666.67 & 14.45 & 
673.10 & 12.99 & \bf{659.90} & 
1.03\\SCA3-6 & 652.94 & 14.01 & 
655.38 & 13.03 & \bf{651.09} & 
0.28\\SCA3-7 & 669.89 & 13.67 & 
671.93 & 14.20 & \bf{659.17} & 
1.63\\SCA3-8 & \bf{719.47} & 15.05 & 
723.79 & 13.82 & 719.47 & 0.00\\
SCA3-9 & 684.44 & 11.65 & 
685.36 & 14.11 & \bf{681.00} & 
0.51\\SCA8-0 & 990.21 & 11.79 & 
1000.62 & 12.32 & \bf{961.50} & 
2.99\\SCA8-1 & 1070.29 & 10.80 & 
1078.63 & 10.87 & \bf{1049.65} & 
1.97\\SCA8-2 & 1054.47 & 11.95 & 
1061.10 & 11.15 & \bf{1039.64} & 
1.43\\SCA8-3 & 1015.23 & 11.72 & 
1022.01 & 10.76 & \bf{983.34} & 
3.24\\SCA8-4 & 1069.87 & 10.34 & 
1092.29 & 11.25 & \bf{1065.49} & 
0.41\\SCA8-5 & 1052.02 & 14.64 & 
1062.53 & 12.94 & \bf{1027.08} & 
2.43\\SCA8-6 & 990.61 & 13.74 & 
997.28 & 12.02 & \bf{971.82} & 
1.93\\SCA8-7 & 1067.11 & 12.20 & 
1084.92 & 12.30 & \bf{1051.28} & 
1.51\\SCA8-8 & \bf{1071.18} & 12.54 & 
1092.09 & 11.62 & 1071.18 & 0.00\\
SCA8-9 & 1069.83 & 13.35 & 
1078.73 & 11.76 & \bf{1060.50} & 
0.88\\CON3-0 & 620.76 & 13.82 & 
630.55 & 14.34 & \bf{616.52} & 
0.69\\CON3-1 & 556.04 & 13.81 & 
558.62 & 15.12 & \bf{554.47} & 
0.28\\CON3-2 & 521.38 & 15.15 & 
521.57 & 14.56 & \bf{518.00} & 
0.65\\CON3-3 & \bf{591.19} & 15.58 & 
595.30 & 14.44 & 591.19 & 0.00\\
CON3-4 & 592.58 & 13.94 & 
594.49 & 14.19 & \bf{588.79} & 
0.64\\CON3-5 & 569.04 & 13.60 & 
569.35 & 13.67 & \bf{563.70} & 
0.95\\CON3-6 & 502.16 & 13.08 & 
504.39 & 20.38 & \bf{499.05} & 
0.62\\CON3-7 & 578.41 & 13.98 & 
587.20 & 14.02 & \bf{576.48} & 
0.33\\CON3-8 & 523.68 & 16.10 & 
524.51 & 14.98 & \bf{523.05} & 
0.12\\CON3-9 & 588.40 & 14.70 & 
589.51 & 14.59 & \bf{578.24} & 
1.76\\CON8-0 & 875.60 & 11.29 & 
889.83 & 11.27 & \bf{857.17} & 
2.15\\CON8-1 & 746.81 & 10.68 & 
763.05 & 11.83 & \bf{740.85} & 
0.80\\CON8-2 & 723.85 & 14.17 & 
730.02 & 13.56 & \bf{712.89} & 
1.54\\CON8-3 & 813.40 & 13.54 & 
828.90 & 12.70 & \bf{811.07} & 
0.29\\CON8-4 & 803.90 & 11.58 & 
813.55 & 11.63 & \bf{772.25} & 
4.10\\CON8-5 & 762.61 & 12.82 & 
769.04 & 12.23 & \bf{754.88} & 
1.02\\CON8-6 & 683.16 & 11.64 & 
692.35 & 11.73 & \bf{678.92} & 
0.62\\CON8-7 & 815.91 & 10.76 & 
824.77 & 11.31 & \bf{811.96} & 
0.49\\CON8-8 & 779.43 & 11.27 & 
784.75 & 11.41 & \bf{767.53} & 
1.55\\CON8-9 & 812.35 & 13.80 & 
821.01 & 12.31 & \bf{809.00} & 
0.41\\[1ex]\hline
\end{tabular}
\label{table:nonlin}
\end{table} \clearpage
\begin{table}[ht]
\caption{Resultados de la ejecución de la metaheurística ILS, utilizando instancias de SalhiNagy con la configuración -n 55.0 -LS 70.0}
\centering
\small
\begin{tabular}{c c c c c c c}
\hline\hline
Instancia & Costo mínimo & Tiempo(seg.) & Costo promedio & Tiempo promedio(seg.) & Costo ILS & \%Gap \\ [0.5ex]
\hline
CMT1X & 477.20 & 11.25 & 
480.61 & 11.51 & \bf{466.77} & 
2.23\\CMT1Y & 477.73 & 10.94 & 
480.75 & 11.92 & \bf{466.77} & 
2.35\\CMT2X & 705.58 & 23.12 & 
709.04 & 25.59 & \bf{684.21} & 
3.12\\CMT2Y & 698.03 & 26.12 & 
701.49 & 28.02 & \bf{684.21} & 
2.02\\CMT3X & 726.68 & 53.21 & 
732.48 & 55.62 & \bf{721.40} & 
0.73\\CMT3Y & 729.23 & 65.42 & 
733.51 & 59.69 & \bf{721.40} & 
1.09\\CMT4X & 897.23 & 147.05 & 
903.36 & 151.91 & \bf{852.83} & 
5.21\\CMT4Y & 891.29 & 145.44 & 
898.71 & 159.12 & \bf{852.46} & 
4.56\\CMT5X & 1087.16 & 388.24 & 
1097.24 & 346.76 & \bf{1030.55} & 
5.49\\CMT5Y & 1093.74 & 340.83 & 
1103.84 & 335.81 & \bf{1031.17} & 
6.07\\CMT11X & 870.20 & 137.53 & 
879.29 & 125.19 & \bf{839.39} & 
3.67\\CMT11Y & 890.71 & 109.09 & 
895.59 & 115.56 & \bf{841.88} & 
5.80\\CMT12X & 676.36 & 48.51 & 
678.41 & 52.97 & \bf{662.22} & 
2.14\\CMT12Y & 674.38 & 50.70 & 
678.05 & 53.15 & \bf{662.22} & 
1.84\\[1ex]\hline
\end{tabular}
\label{table:nonlin}
\end{table} \clearpage
\begin{table}[ht]
\caption{Resultados de la ejecución de la metaheurística ILS, utilizando instancias de Dethloff con la configuración -n 55.0 -LS 80.0}
\centering
\small
\begin{tabular}{c c c c c c c}
\hline\hline
Instancia & Costo mínimo & Tiempo(seg.) & Costo promedio & Tiempo promedio(seg.) & Costo ILS & \%Gap \\ [0.5ex]
\hline
SCA3-0 & 640.55 & 17.98 & 
642.18 & 16.09 & \bf{635.62} & 
0.78\\SCA3-1 & 701.53 & 14.73 & 
701.53 & 16.10 & \bf{697.84} & 
0.53\\SCA3-2 & \bf{659.34} & 15.01 & 
661.53 & 15.87 & 659.34 & 0.00\\
SCA3-3 & \bf{680.04} & 17.35 & 
683.79 & 16.95 & 680.04 & 0.00\\
SCA3-4 & \bf{690.50} & 16.72 & 
690.50 & 16.24 & 690.50 & 0.00\\
SCA3-5 & \bf{659.90} & 16.29 & 
663.32 & 16.14 & 659.90 & 0.00\\
SCA3-6 & \bf{651.09} & 15.14 & 
655.00 & 15.22 & 651.09 & 0.00\\
SCA3-7 & 667.75 & 16.85 & 
669.75 & 15.72 & \bf{659.17} & 
1.30\\SCA3-8 & \bf{719.47} & 14.32 & 
722.79 & 15.18 & 719.47 & 0.00\\
SCA3-9 & \bf{681.00} & 15.03 & 
683.67 & 14.84 & 681.00 & 0.00\\
SCA8-0 & 989.54 & 11.30 & 
1001.47 & 13.29 & \bf{961.50} & 
2.92\\SCA8-1 & 1066.80 & 11.10 & 
1080.70 & 11.43 & \bf{1049.65} & 
1.63\\SCA8-2 & 1060.09 & 10.30 & 
1069.61 & 12.43 & \bf{1039.64} & 
1.97\\SCA8-3 & 1014.45 & 11.97 & 
1026.39 & 11.82 & \bf{983.34} & 
3.16\\SCA8-4 & 1071.64 & 12.17 & 
1079.84 & 12.81 & \bf{1065.49} & 
0.58\\SCA8-5 & 1038.59 & 12.03 & 
1062.07 & 12.52 & \bf{1027.08} & 
1.12\\SCA8-6 & 987.08 & 13.47 & 
1001.40 & 12.61 & \bf{971.82} & 
1.57\\SCA8-7 & 1073.57 & 16.25 & 
1085.10 & 13.81 & \bf{1051.28} & 
2.12\\SCA8-8 & 1089.91 & 11.45 & 
1096.03 & 11.86 & \bf{1071.18} & 
1.75\\SCA8-9 & 1087.65 & 11.46 & 
1093.37 & 12.03 & \bf{1060.50} & 
2.56\\CON3-0 & 629.09 & 16.05 & 
632.49 & 15.80 & \bf{616.52} & 
2.04\\CON3-1 & 560.75 & 16.00 & 
560.75 & 16.27 & \bf{554.47} & 
1.13\\CON3-2 & 521.38 & 17.00 & 
524.70 & 16.46 & \bf{518.00} & 
0.65\\CON3-3 & \bf{591.19} & 15.08 & 
595.86 & 16.10 & 591.19 & 0.00\\
CON3-4 & 591.43 & 14.68 & 
592.91 & 15.17 & \bf{588.79} & 
0.45\\CON3-5 & 569.57 & 16.06 & 
573.99 & 15.46 & \bf{563.70} & 
1.04\\CON3-6 & 500.80 & 16.77 & 
501.82 & 15.76 & \bf{499.05} & 
0.35\\CON3-7 & 583.65 & 15.84 & 
588.52 & 16.26 & \bf{576.48} & 
1.24\\CON3-8 & 523.68 & 16.84 & 
524.36 & 16.09 & \bf{523.05} & 
0.12\\CON3-9 & 588.40 & 18.68 & 
589.36 & 16.82 & \bf{578.24} & 
1.76\\CON8-0 & 861.87 & 13.24 & 
882.43 & 12.71 & \bf{857.17} & 
0.55\\CON8-1 & 755.23 & 14.04 & 
762.93 & 13.44 & \bf{740.85} & 
1.94\\CON8-2 & 719.17 & 12.69 & 
726.14 & 12.71 & \bf{712.89} & 
0.88\\CON8-3 & 824.04 & 15.20 & 
835.21 & 14.60 & \bf{811.07} & 
1.60\\CON8-4 & 789.20 & 11.21 & 
795.85 & 12.62 & \bf{772.25} & 
2.19\\CON8-5 & 762.61 & 12.37 & 
769.27 & 12.59 & \bf{754.88} & 
1.02\\CON8-6 & 697.44 & 12.73 & 
702.50 & 13.86 & \bf{678.92} & 
2.73\\CON8-7 & 815.72 & 13.00 & 
820.12 & 13.11 & \bf{811.96} & 
0.46\\CON8-8 & 783.39 & 12.36 & 
788.39 & 12.07 & \bf{767.53} & 
2.07\\CON8-9 & 822.18 & 14.30 & 
836.10 & 12.86 & \bf{809.00} & 
1.63\\[1ex]\hline
\end{tabular}
\label{table:nonlin}
\end{table} \clearpage
\begin{table}[ht]
\caption{Resultados de la ejecución de la metaheurística ILS, utilizando instancias de SalhiNagy con la configuración -n 55.0 -LS 80.0}
\centering
\small
\begin{tabular}{c c c c c c c}
\hline\hline
Instancia & Costo mínimo & Tiempo(seg.) & Costo promedio & Tiempo promedio(seg.) & Costo ILS & \%Gap \\ [0.5ex]
\hline
CMT1X & 474.85 & 13.10 & 
478.73 & 12.40 & \bf{466.77} & 
1.73\\CMT1Y & 475.22 & 13.60 & 
477.55 & 13.68 & \bf{466.77} & 
1.81\\CMT2X & 704.58 & 26.99 & 
709.19 & 27.97 & \bf{684.21} & 
2.98\\CMT2Y & 692.37 & 30.62 & 
700.51 & 29.95 & \bf{684.21} & 
1.19\\CMT3X & 727.10 & 63.73 & 
734.75 & 65.50 & \bf{721.40} & 
0.79\\CMT3Y & 728.12 & 63.90 & 
731.20 & 66.71 & \bf{721.40} & 
0.93\\CMT4X & 884.17 & 161.42 & 
894.02 & 160.37 & \bf{852.83} & 
3.67\\CMT4Y & 881.71 & 199.47 & 
897.42 & 173.26 & \bf{852.46} & 
3.43\\CMT5X & 1094.05 & 335.46 & 
1097.41 & 378.99 & \bf{1030.55} & 
6.16\\CMT5Y & 1078.81 & 373.06 & 
1089.79 & 365.23 & \bf{1031.17} & 
4.62\\CMT11X & 878.41 & 153.82 & 
889.95 & 137.27 & \bf{839.39} & 
4.65\\CMT11Y & 880.39 & 121.95 & 
888.88 & 118.90 & \bf{841.88} & 
4.57\\CMT12X & 677.54 & 57.06 & 
680.55 & 57.05 & \bf{662.22} & 
2.31\\CMT12Y & 675.95 & 62.77 & 
679.94 & 64.66 & \bf{662.22} & 
2.07\\[1ex]\hline
\end{tabular}
\label{table:nonlin}
\end{table} \clearpage
\begin{table}[ht]
\caption{Resultados de la ejecución de la metaheurística ILS, utilizando instancias de Dethloff con la configuración -n 65.0 -LS 10.0}
\centering
\small
\begin{tabular}{c c c c c c c}
\hline\hline
Instancia & Costo mínimo & Tiempo(seg.) & Costo promedio & Tiempo promedio(seg.) & Costo ILS & \%Gap \\ [0.5ex]
\hline
SCA3-0 & 640.55 & 4.63 & 
641.89 & 4.11 & \bf{635.62} & 
0.78\\SCA3-1 & 706.90 & 4.20 & 
712.04 & 4.12 & \bf{697.84} & 
1.30\\SCA3-2 & \bf{659.34} & 3.57 & 
667.53 & 3.78 & 659.34 & 0.00\\
SCA3-3 & 680.60 & 3.73 & 
682.36 & 4.03 & \bf{680.04} & 
0.08\\SCA3-4 & \bf{690.50} & 4.40 & 
706.38 & 3.85 & 690.50 & 0.00\\
SCA3-5 & 673.39 & 4.42 & 
677.70 & 3.94 & \bf{659.90} & 
2.04\\SCA3-6 & 652.47 & 4.32 & 
656.18 & 3.85 & \bf{651.09} & 
0.21\\SCA3-7 & 666.15 & 4.15 & 
677.20 & 3.87 & \bf{659.17} & 
1.06\\SCA3-8 & \bf{719.47} & 3.35 & 
724.31 & 4.01 & 719.47 & 0.00\\
SCA3-9 & \bf{681.00} & 3.44 & 
687.06 & 3.90 & 681.00 & 0.00\\
SCA8-0 & 1004.86 & 4.48 & 
1019.69 & 4.37 & \bf{961.50} & 
4.51\\SCA8-1 & 1078.10 & 3.25 & 
1084.90 & 3.76 & \bf{1049.65} & 
2.71\\SCA8-2 & 1064.72 & 3.70 & 
1086.05 & 3.48 & \bf{1039.64} & 
2.41\\SCA8-3 & 1012.77 & 4.21 & 
1032.53 & 3.61 & \bf{983.34} & 
2.99\\SCA8-4 & 1074.63 & 3.95 & 
1098.36 & 3.62 & \bf{1065.49} & 
0.86\\SCA8-5 & 1066.35 & 4.31 & 
1087.46 & 3.75 & \bf{1027.08} & 
3.82\\SCA8-6 & 995.52 & 3.44 & 
1002.61 & 3.58 & \bf{971.82} & 
2.44\\SCA8-7 & 1070.92 & 4.65 & 
1082.83 & 4.14 & \bf{1051.28} & 
1.87\\SCA8-8 & 1096.45 & 3.45 & 
1109.39 & 3.50 & \bf{1071.18} & 
2.36\\SCA8-9 & 1072.76 & 3.93 & 
1087.85 & 3.45 & \bf{1060.50} & 
1.16\\CON3-0 & 632.57 & 3.74 & 
639.32 & 3.71 & \bf{616.52} & 
2.60\\CON3-1 & 560.41 & 4.08 & 
566.12 & 4.09 & \bf{554.47} & 
1.07\\CON3-2 & 521.63 & 4.99 & 
525.05 & 4.08 & \bf{518.00} & 
0.70\\CON3-3 & 610.17 & 3.96 & 
620.03 & 3.90 & \bf{591.19} & 
3.21\\CON3-4 & 593.78 & 4.06 & 
599.50 & 4.12 & \bf{588.79} & 
0.85\\CON3-5 & 569.74 & 4.35 & 
573.81 & 4.38 & \bf{563.70} & 
1.07\\CON3-6 & 502.16 & 4.18 & 
505.48 & 4.30 & \bf{499.05} & 
0.62\\CON3-7 & \bf{576.48} & 3.98 & 
591.05 & 4.01 & 576.48 & 0.00\\
CON3-8 & \bf{523.05} & 4.41 & 
530.57 & 4.32 & 523.05 & 0.00\\
CON3-9 & 590.64 & 3.97 & 
591.90 & 4.07 & \bf{578.24} & 
2.14\\CON8-0 & 904.21 & 4.39 & 
920.71 & 3.69 & \bf{857.17} & 
5.49\\CON8-1 & 769.67 & 3.70 & 
783.31 & 3.79 & \bf{740.85} & 
3.89\\CON8-2 & 720.47 & 4.14 & 
727.52 & 3.93 & \bf{712.89} & 
1.06\\CON8-3 & 830.96 & 4.37 & 
840.90 & 3.70 & \bf{811.07} & 
2.45\\CON8-4 & 802.57 & 3.52 & 
816.54 & 3.62 & \bf{772.25} & 
3.93\\CON8-5 & 780.16 & 3.81 & 
789.60 & 3.69 & \bf{754.88} & 
3.35\\CON8-6 & 689.21 & 4.11 & 
704.38 & 3.92 & \bf{678.92} & 
1.52\\CON8-7 & 815.82 & 3.22 & 
833.12 & 3.54 & \bf{811.96} & 
0.48\\CON8-8 & 769.65 & 3.76 & 
794.77 & 4.00 & \bf{767.53} & 
0.28\\CON8-9 & 817.56 & 3.79 & 
827.68 & 4.15 & \bf{809.00} & 
1.06\\[1ex]\hline
\end{tabular}
\label{table:nonlin}
\end{table} \clearpage
\begin{table}[ht]
\caption{Resultados de la ejecución de la metaheurística ILS, utilizando instancias de SalhiNagy con la configuración -n 65.0 -LS 10.0}
\centering
\small
\begin{tabular}{c c c c c c c}
\hline\hline
Instancia & Costo mínimo & Tiempo(seg.) & Costo promedio & Tiempo promedio(seg.) & Costo ILS & \%Gap \\ [0.5ex]
\hline
CMT1X & 479.05 & 3.54 & 
481.76 & 3.62 & \bf{466.77} & 
2.63\\CMT1Y & 481.39 & 3.01 & 
488.85 & 3.37 & \bf{466.77} & 
3.13\\CMT2X & 706.75 & 10.18 & 
712.11 & 10.01 & \bf{684.21} & 
3.29\\CMT2Y & 709.03 & 10.24 & 
712.02 & 9.66 & \bf{684.21} & 
3.63\\CMT3X & 730.96 & 36.30 & 
734.78 & 29.99 & \bf{721.40} & 
1.33\\CMT3Y & 728.95 & 25.27 & 
732.84 & 28.29 & \bf{721.40} & 
1.05\\CMT4X & 890.45 & 81.63 & 
905.43 & 91.34 & \bf{852.83} & 
4.41\\CMT4Y & 899.25 & 81.83 & 
903.17 & 83.64 & \bf{852.46} & 
5.49\\CMT5X & 1095.74 & 211.25 & 
1107.51 & 229.18 & \bf{1030.55} & 
6.33\\CMT5Y & 1096.09 & 209.48 & 
1104.99 & 251.43 & \bf{1031.17} & 
6.30\\CMT11X & 890.88 & 99.36 & 
895.66 & 81.39 & \bf{839.39} & 
6.13\\CMT11Y & 874.78 & 65.28 & 
879.99 & 64.55 & \bf{841.88} & 
3.91\\CMT12X & 674.73 & 31.60 & 
684.62 & 28.29 & \bf{662.22} & 
1.89\\CMT12Y & 674.34 & 24.04 & 
677.30 & 27.95 & \bf{662.22} & 
1.83\\[1ex]\hline
\end{tabular}
\label{table:nonlin}
\end{table} \clearpage
\begin{table}[ht]
\caption{Resultados de la ejecución de la metaheurística ILS, utilizando instancias de Dethloff con la configuración -n 65.0 -LS 20.0}
\centering
\small
\begin{tabular}{c c c c c c c}
\hline\hline
Instancia & Costo mínimo & Tiempo(seg.) & Costo promedio & Tiempo promedio(seg.) & Costo ILS & \%Gap \\ [0.5ex]
\hline
SCA3-0 & 636.06 & 6.36 & 
641.10 & 6.43 & \bf{635.62} & 
0.07\\SCA3-1 & 701.53 & 7.00 & 
704.21 & 6.88 & \bf{697.84} & 
0.53\\SCA3-2 & \bf{659.34} & 5.45 & 
662.98 & 5.70 & 659.34 & 0.00\\
SCA3-3 & 680.60 & 6.06 & 
685.48 & 6.35 & \bf{680.04} & 
0.08\\SCA3-4 & \bf{690.50} & 6.11 & 
697.60 & 6.24 & 690.50 & 0.00\\
SCA3-5 & 666.67 & 6.43 & 
678.73 & 6.12 & \bf{659.90} & 
1.03\\SCA3-6 & 652.47 & 5.20 & 
654.05 & 5.32 & \bf{651.09} & 
0.21\\SCA3-7 & 671.67 & 4.90 & 
672.07 & 5.95 & \bf{659.17} & 
1.90\\SCA3-8 & 724.29 & 5.08 & 
731.99 & 6.20 & \bf{719.47} & 
0.67\\SCA3-9 & \bf{681.00} & 4.98 & 
684.46 & 5.20 & 681.00 & 0.00\\
SCA8-0 & 1001.48 & 5.66 & 
1009.08 & 6.04 & \bf{961.50} & 
4.16\\SCA8-1 & 1079.07 & 5.02 & 
1083.23 & 5.17 & \bf{1049.65} & 
2.80\\SCA8-2 & 1064.13 & 4.42 & 
1070.38 & 5.08 & \bf{1039.64} & 
2.36\\SCA8-3 & 1022.58 & 5.76 & 
1033.38 & 5.42 & \bf{983.34} & 
3.99\\SCA8-4 & 1074.10 & 5.17 & 
1085.85 & 5.10 & \bf{1065.49} & 
0.81\\SCA8-5 & 1063.39 & 5.78 & 
1073.72 & 5.09 & \bf{1027.08} & 
3.54\\SCA8-6 & 999.81 & 6.43 & 
1006.26 & 5.57 & \bf{971.82} & 
2.88\\SCA8-7 & 1067.49 & 5.37 & 
1082.08 & 5.08 & \bf{1051.28} & 
1.54\\SCA8-8 & \bf{1071.18} & 5.57 & 
1088.37 & 5.14 & 1071.18 & 0.00\\
SCA8-9 & 1104.99 & 5.35 & 
1115.62 & 4.87 & \bf{1060.50} & 
4.20\\CON3-0 & 623.84 & 6.35 & 
631.22 & 6.06 & \bf{616.52} & 
1.19\\CON3-1 & 564.55 & 7.29 & 
569.67 & 6.24 & \bf{554.47} & 
1.82\\CON3-2 & 521.38 & 6.83 & 
525.95 & 5.85 & \bf{518.00} & 
0.65\\CON3-3 & \bf{591.19} & 6.58 & 
595.86 & 6.45 & 591.19 & 0.00\\
CON3-4 & 598.36 & 5.85 & 
601.48 & 5.86 & \bf{588.79} & 
1.63\\CON3-5 & 573.41 & 7.16 & 
580.55 & 6.33 & \bf{563.70} & 
1.72\\CON3-6 & 502.16 & 6.82 & 
507.49 & 6.50 & \bf{499.05} & 
0.62\\CON3-7 & 578.41 & 5.03 & 
590.57 & 6.26 & \bf{576.48} & 
0.33\\CON3-8 & 523.14 & 7.31 & 
524.81 & 6.92 & \bf{523.05} & 
0.02\\CON3-9 & 588.99 & 5.76 & 
589.14 & 6.01 & \bf{578.24} & 
1.86\\CON8-0 & 884.44 & 5.97 & 
889.60 & 5.75 & \bf{857.17} & 
3.18\\CON8-1 & 763.07 & 6.04 & 
767.07 & 5.77 & \bf{740.85} & 
3.00\\CON8-2 & 729.85 & 6.38 & 
739.75 & 5.44 & \bf{712.89} & 
2.38\\CON8-3 & 830.41 & 5.00 & 
841.91 & 5.78 & \bf{811.07} & 
2.38\\CON8-4 & 791.67 & 6.21 & 
806.60 & 5.78 & \bf{772.25} & 
2.51\\CON8-5 & 763.52 & 5.24 & 
780.23 & 5.90 & \bf{754.88} & 
1.14\\CON8-6 & 693.99 & 5.97 & 
705.47 & 5.76 & \bf{678.92} & 
2.22\\CON8-7 & 825.69 & 5.94 & 
832.83 & 5.21 & \bf{811.96} & 
1.69\\CON8-8 & 793.10 & 5.32 & 
797.38 & 5.66 & \bf{767.53} & 
3.33\\CON8-9 & 819.81 & 5.77 & 
833.73 & 5.83 & \bf{809.00} & 
1.34\\[1ex]\hline
\end{tabular}
\label{table:nonlin}
\end{table} \clearpage
\begin{table}[ht]
\caption{Resultados de la ejecución de la metaheurística ILS, utilizando instancias de SalhiNagy con la configuración -n 65.0 -LS 20.0}
\centering
\small
\begin{tabular}{c c c c c c c}
\hline\hline
Instancia & Costo mínimo & Tiempo(seg.) & Costo promedio & Tiempo promedio(seg.) & Costo ILS & \%Gap \\ [0.5ex]
\hline
CMT1X & 471.25 & 6.62 & 
476.19 & 5.47 & \bf{466.77} & 
0.96\\CMT1Y & 480.02 & 5.60 & 
486.50 & 5.25 & \bf{466.77} & 
2.84\\CMT2X & 706.32 & 13.36 & 
708.96 & 13.35 & \bf{684.21} & 
3.23\\CMT2Y & 693.13 & 13.50 & 
708.24 & 13.96 & \bf{684.21} & 
1.30\\CMT3X & 727.53 & 32.18 & 
737.80 & 32.35 & \bf{721.40} & 
0.85\\CMT3Y & 728.14 & 35.31 & 
732.01 & 36.15 & \bf{721.40} & 
0.93\\CMT4X & 895.02 & 94.16 & 
901.76 & 108.97 & \bf{852.83} & 
4.95\\CMT4Y & 883.33 & 94.71 & 
888.16 & 103.77 & \bf{852.46} & 
3.62\\CMT5X & 1098.91 & 212.67 & 
1104.01 & 270.07 & \bf{1030.55} & 
6.63\\CMT5Y & 1100.28 & 227.69 & 
1107.65 & 254.31 & \bf{1031.17} & 
6.70\\CMT11X & 881.34 & 110.89 & 
889.10 & 83.97 & \bf{839.39} & 
5.00\\CMT11Y & 885.02 & 109.11 & 
889.31 & 102.83 & \bf{841.88} & 
5.12\\CMT12X & 674.91 & 38.01 & 
679.95 & 34.05 & \bf{662.22} & 
1.92\\CMT12Y & 674.87 & 27.80 & 
682.06 & 28.51 & \bf{662.22} & 
1.91\\[1ex]\hline
\end{tabular}
\label{table:nonlin}
\end{table} \clearpage
\begin{table}[ht]
\caption{Resultados de la ejecución de la metaheurística ILS, utilizando instancias de Dethloff con la configuración -n 65.0 -LS 30.0}
\centering
\small
\begin{tabular}{c c c c c c c}
\hline\hline
Instancia & Costo mínimo & Tiempo(seg.) & Costo promedio & Tiempo promedio(seg.) & Costo ILS & \%Gap \\ [0.5ex]
\hline
SCA3-0 & 640.55 & 8.51 & 
641.81 & 8.51 & \bf{635.62} & 
0.78\\SCA3-1 & \bf{697.84} & 8.68 & 
705.28 & 8.69 & 697.84 & 0.00\\
SCA3-2 & 664.18 & 7.96 & 
665.78 & 8.03 & \bf{659.34} & 
0.73\\SCA3-3 & 680.60 & 8.15 & 
683.67 & 8.17 & \bf{680.04} & 
0.08\\SCA3-4 & \bf{690.50} & 9.32 & 
691.18 & 8.29 & 690.50 & 0.00\\
SCA3-5 & 670.10 & 9.13 & 
678.20 & 8.46 & \bf{659.90} & 
1.55\\SCA3-6 & 652.94 & 8.74 & 
654.39 & 8.03 & \bf{651.09} & 
0.28\\SCA3-7 & 671.77 & 7.52 & 
672.95 & 7.85 & \bf{659.17} & 
1.91\\SCA3-8 & \bf{719.47} & 7.51 & 
727.72 & 7.67 & 719.47 & 0.00\\
SCA3-9 & \bf{681.00} & 8.45 & 
686.52 & 7.71 & 681.00 & 0.00\\
SCA8-0 & 993.77 & 7.10 & 
1008.65 & 7.03 & \bf{961.50} & 
3.36\\SCA8-1 & 1071.11 & 6.81 & 
1083.19 & 7.09 & \bf{1049.65} & 
2.04\\SCA8-2 & 1055.32 & 6.83 & 
1061.02 & 6.75 & \bf{1039.64} & 
1.51\\SCA8-3 & 1033.18 & 8.51 & 
1039.60 & 7.14 & \bf{983.34} & 
5.07\\SCA8-4 & 1090.66 & 6.82 & 
1105.82 & 6.83 & \bf{1065.49} & 
2.36\\SCA8-5 & 1063.25 & 6.63 & 
1071.76 & 7.35 & \bf{1027.08} & 
3.52\\SCA8-6 & 987.58 & 6.46 & 
1011.76 & 6.22 & \bf{971.82} & 
1.62\\SCA8-7 & 1067.49 & 6.41 & 
1080.68 & 6.99 & \bf{1051.28} & 
1.54\\SCA8-8 & 1092.06 & 7.08 & 
1097.72 & 6.98 & \bf{1071.18} & 
1.95\\SCA8-9 & 1082.53 & 6.22 & 
1095.69 & 6.47 & \bf{1060.50} & 
2.08\\CON3-0 & 632.82 & 8.69 & 
635.74 & 8.10 & \bf{616.52} & 
2.64\\CON3-1 & 560.61 & 8.42 & 
561.16 & 8.57 & \bf{554.47} & 
1.11\\CON3-2 & 521.38 & 7.92 & 
524.87 & 8.39 & \bf{518.00} & 
0.65\\CON3-3 & \bf{591.19} & 8.23 & 
598.98 & 8.22 & 591.19 & 0.00\\
CON3-4 & 591.43 & 7.51 & 
600.67 & 7.60 & \bf{588.79} & 
0.45\\CON3-5 & 570.22 & 7.86 & 
574.32 & 8.48 & \bf{563.70} & 
1.16\\CON3-6 & 502.16 & 8.41 & 
504.78 & 8.99 & \bf{499.05} & 
0.62\\CON3-7 & 578.22 & 8.94 & 
590.56 & 8.61 & \bf{576.48} & 
0.30\\CON3-8 & 523.68 & 7.65 & 
528.33 & 8.30 & \bf{523.05} & 
0.12\\CON3-9 & 588.99 & 8.08 & 
590.54 & 8.34 & \bf{578.24} & 
1.86\\CON8-0 & 865.86 & 7.50 & 
884.53 & 7.24 & \bf{857.17} & 
1.01\\CON8-1 & 764.41 & 7.21 & 
782.42 & 7.07 & \bf{740.85} & 
3.18\\CON8-2 & 716.53 & 7.49 & 
729.55 & 7.09 & \bf{712.89} & 
0.51\\CON8-3 & 832.18 & 7.88 & 
840.20 & 7.45 & \bf{811.07} & 
2.60\\CON8-4 & 797.50 & 7.44 & 
821.52 & 6.13 & \bf{772.25} & 
3.27\\CON8-5 & 779.68 & 6.23 & 
781.12 & 7.12 & \bf{754.88} & 
3.29\\CON8-6 & 695.12 & 7.79 & 
704.05 & 7.59 & \bf{678.92} & 
2.39\\CON8-7 & 817.98 & 7.00 & 
841.08 & 6.67 & \bf{811.96} & 
0.74\\CON8-8 & 785.81 & 8.39 & 
793.53 & 7.45 & \bf{767.53} & 
2.38\\CON8-9 & 825.19 & 8.04 & 
831.84 & 7.70 & \bf{809.00} & 
2.00\\[1ex]\hline
\end{tabular}
\label{table:nonlin}
\end{table} \clearpage
\begin{table}[ht]
\caption{Resultados de la ejecución de la metaheurística ILS, utilizando instancias de SalhiNagy con la configuración -n 65.0 -LS 30.0}
\centering
\small
\begin{tabular}{c c c c c c c}
\hline\hline
Instancia & Costo mínimo & Tiempo(seg.) & Costo promedio & Tiempo promedio(seg.) & Costo ILS & \%Gap \\ [0.5ex]
\hline
CMT1X & 470.67 & 7.89 & 
477.46 & 7.09 & \bf{466.77} & 
0.84\\CMT1Y & 475.35 & 8.14 & 
480.63 & 6.94 & \bf{466.77} & 
1.84\\CMT2X & 708.80 & 14.92 & 
714.60 & 16.67 & \bf{684.21} & 
3.59\\CMT2Y & 706.08 & 20.24 & 
714.54 & 17.36 & \bf{684.21} & 
3.20\\CMT3X & 739.24 & 51.03 & 
740.60 & 45.03 & \bf{721.40} & 
2.47\\CMT3Y & 731.43 & 42.70 & 
734.20 & 42.44 & \bf{721.40} & 
1.39\\CMT4X & 896.26 & 107.20 & 
904.59 & 107.51 & \bf{852.83} & 
5.09\\CMT4Y & 896.58 & 111.40 & 
902.41 & 117.45 & \bf{852.46} & 
5.18\\CMT5X & 1090.69 & 257.56 & 
1102.18 & 287.56 & \bf{1030.55} & 
5.84\\CMT5Y & 1088.29 & 259.91 & 
1097.49 & 279.70 & \bf{1031.17} & 
5.54\\CMT11X & 851.65 & 83.62 & 
879.16 & 94.43 & \bf{839.39} & 
1.46\\CMT11Y & 874.86 & 92.12 & 
892.72 & 84.15 & \bf{841.88} & 
3.92\\CMT12X & 680.03 & 35.06 & 
683.26 & 35.70 & \bf{662.22} & 
2.69\\CMT12Y & 674.65 & 33.70 & 
678.00 & 35.28 & \bf{662.22} & 
1.88\\[1ex]\hline
\end{tabular}
\label{table:nonlin}
\end{table} \clearpage
\begin{table}[ht]
\caption{Resultados de la ejecución de la metaheurística ILS, utilizando instancias de Dethloff con la configuración -n 65.0 -LS 40.0}
\centering
\small
\begin{tabular}{c c c c c c c}
\hline\hline
Instancia & Costo mínimo & Tiempo(seg.) & Costo promedio & Tiempo promedio(seg.) & Costo ILS & \%Gap \\ [0.5ex]
\hline
SCA3-0 & 640.55 & 9.70 & 
641.12 & 10.33 & \bf{635.62} & 
0.78\\SCA3-1 & 701.53 & 9.48 & 
703.87 & 9.99 & \bf{697.84} & 
0.53\\SCA3-2 & 661.13 & 9.39 & 
663.99 & 10.07 & \bf{659.34} & 
0.27\\SCA3-3 & 680.60 & 10.85 & 
685.19 & 10.37 & \bf{680.04} & 
0.08\\SCA3-4 & \bf{690.50} & 10.27 & 
691.87 & 10.35 & 690.50 & 0.00\\
SCA3-5 & 673.39 & 9.28 & 
680.84 & 10.16 & \bf{659.90} & 
2.04\\SCA3-6 & 652.94 & 11.96 & 
653.97 & 10.97 & \bf{651.09} & 
0.28\\SCA3-7 & 666.60 & 9.53 & 
671.50 & 9.69 & \bf{659.17} & 
1.13\\SCA3-8 & 719.77 & 11.51 & 
725.67 & 10.23 & \bf{719.47} & 
0.04\\SCA3-9 & 684.25 & 9.05 & 
687.93 & 9.31 & \bf{681.00} & 
0.48\\SCA8-0 & 1000.13 & 8.09 & 
1007.94 & 9.66 & \bf{961.50} & 
4.02\\SCA8-1 & 1086.74 & 8.44 & 
1087.48 & 8.00 & \bf{1049.65} & 
3.53\\SCA8-2 & 1050.37 & 8.07 & 
1058.13 & 8.74 & \bf{1039.64} & 
1.03\\SCA8-3 & 1030.71 & 7.87 & 
1040.55 & 8.13 & \bf{983.34} & 
4.82\\SCA8-4 & 1082.87 & 7.93 & 
1090.85 & 8.64 & \bf{1065.49} & 
1.63\\SCA8-5 & 1074.49 & 8.11 & 
1083.70 & 8.51 & \bf{1027.08} & 
4.62\\SCA8-6 & 987.54 & 9.27 & 
1001.21 & 9.00 & \bf{971.82} & 
1.62\\SCA8-7 & 1070.53 & 8.53 & 
1088.26 & 9.54 & \bf{1051.28} & 
1.83\\SCA8-8 & 1092.82 & 8.73 & 
1096.81 & 8.95 & \bf{1071.18} & 
2.02\\SCA8-9 & 1082.56 & 7.70 & 
1092.38 & 9.02 & \bf{1060.50} & 
2.08\\CON3-0 & 632.00 & 9.88 & 
637.53 & 9.52 & \bf{616.52} & 
2.51\\CON3-1 & 559.52 & 10.41 & 
563.05 & 10.38 & \bf{554.47} & 
0.91\\CON3-2 & 521.38 & 9.56 & 
522.93 & 9.65 & \bf{518.00} & 
0.65\\CON3-3 & \bf{591.19} & 11.17 & 
594.17 & 11.05 & 591.19 & 0.00\\
CON3-4 & 593.78 & 11.04 & 
596.31 & 10.47 & \bf{588.79} & 
0.85\\CON3-5 & 569.04 & 11.20 & 
572.12 & 10.85 & \bf{563.70} & 
0.95\\CON3-6 & 502.16 & 8.59 & 
505.75 & 10.27 & \bf{499.05} & 
0.62\\CON3-7 & 578.41 & 10.49 & 
588.24 & 9.80 & \bf{576.48} & 
0.33\\CON3-8 & 524.59 & 8.76 & 
525.90 & 9.75 & \bf{523.05} & 
0.29\\CON3-9 & 578.25 & 10.38 & 
587.98 & 10.07 & \bf{578.24} & 
0.00\\CON8-0 & 890.01 & 8.35 & 
894.68 & 8.92 & \bf{857.17} & 
3.83\\CON8-1 & 752.72 & 10.16 & 
763.66 & 9.11 & \bf{740.85} & 
1.60\\CON8-2 & 716.69 & 8.83 & 
725.38 & 9.15 & \bf{712.89} & 
0.53\\CON8-3 & 822.09 & 10.38 & 
839.79 & 9.29 & \bf{811.07} & 
1.36\\CON8-4 & 802.11 & 8.31 & 
809.93 & 8.03 & \bf{772.25} & 
3.87\\CON8-5 & 766.55 & 7.46 & 
783.10 & 7.55 & \bf{754.88} & 
1.55\\CON8-6 & 699.10 & 9.91 & 
708.58 & 9.48 & \bf{678.92} & 
2.97\\CON8-7 & 814.86 & 9.07 & 
833.77 & 8.89 & \bf{811.96} & 
0.36\\CON8-8 & 789.40 & 9.68 & 
792.23 & 9.12 & \bf{767.53} & 
2.85\\CON8-9 & 816.47 & 8.39 & 
837.99 & 8.64 & \bf{809.00} & 
0.92\\[1ex]\hline
\end{tabular}
\label{table:nonlin}
\end{table} \clearpage
\begin{table}[ht]
\caption{Resultados de la ejecución de la metaheurística ILS, utilizando instancias de SalhiNagy con la configuración -n 65.0 -LS 40.0}
\centering
\small
\begin{tabular}{c c c c c c c}
\hline\hline
Instancia & Costo mínimo & Tiempo(seg.) & Costo promedio & Tiempo promedio(seg.) & Costo ILS & \%Gap \\ [0.5ex]
\hline
CMT1X & 475.37 & 8.09 & 
481.59 & 8.25 & \bf{466.77} & 
1.84\\CMT1Y & 476.66 & 7.33 & 
479.63 & 8.38 & \bf{466.77} & 
2.12\\CMT2X & 705.23 & 19.06 & 
710.59 & 20.09 & \bf{684.21} & 
3.07\\CMT2Y & 697.78 & 20.57 & 
704.37 & 21.76 & \bf{684.21} & 
1.98\\CMT3X & 724.87 & 56.08 & 
732.34 & 50.64 & \bf{721.40} & 
0.48\\CMT3Y & 722.63 & 46.40 & 
729.83 & 48.79 & \bf{721.40} & 
0.17\\CMT4X & 896.24 & 132.03 & 
905.08 & 145.01 & \bf{852.83} & 
5.09\\CMT4Y & 878.80 & 123.82 & 
900.19 & 142.41 & \bf{852.46} & 
3.09\\CMT5X & 1093.44 & 385.22 & 
1103.22 & 361.29 & \bf{1030.55} & 
6.10\\CMT5Y & 1079.55 & 297.11 & 
1099.29 & 297.75 & \bf{1031.17} & 
4.69\\CMT11X & 874.27 & 92.55 & 
886.63 & 102.22 & \bf{839.39} & 
4.16\\CMT11Y & 879.58 & 129.26 & 
887.58 & 126.23 & \bf{841.88} & 
4.48\\CMT12X & 673.92 & 51.33 & 
679.75 & 44.42 & \bf{662.22} & 
1.77\\CMT12Y & 680.35 & 46.29 & 
683.94 & 42.12 & \bf{662.22} & 
2.74\\[1ex]\hline
\end{tabular}
\label{table:nonlin}
\end{table} \clearpage
\begin{table}[ht]
\caption{Resultados de la ejecución de la metaheurística ILS, utilizando instancias de Dethloff con la configuración -n 65.0 -LS 50.0}
\centering
\small
\begin{tabular}{c c c c c c c}
\hline\hline
Instancia & Costo mínimo & Tiempo(seg.) & Costo promedio & Tiempo promedio(seg.) & Costo ILS & \%Gap \\ [0.5ex]
\hline
SCA3-0 & 636.06 & 11.33 & 
640.18 & 12.12 & \bf{635.62} & 
0.07\\SCA3-1 & \bf{697.84} & 13.89 & 
704.01 & 13.49 & 697.84 & 0.00\\
SCA3-2 & \bf{659.34} & 12.61 & 
662.67 & 12.75 & 659.34 & 0.00\\
SCA3-3 & \bf{680.04} & 12.32 & 
682.18 & 11.97 & 680.04 & 0.00\\
SCA3-4 & \bf{690.50} & 13.46 & 
690.50 & 12.46 & 690.50 & 0.00\\
SCA3-5 & 665.64 & 12.29 & 
673.45 & 12.54 & \bf{659.90} & 
0.87\\SCA3-6 & \bf{651.09} & 10.69 & 
655.30 & 11.60 & 651.09 & 0.00\\
SCA3-7 & 671.67 & 12.61 & 
671.72 & 12.96 & \bf{659.17} & 
1.90\\SCA3-8 & 724.66 & 11.53 & 
725.92 & 12.40 & \bf{719.47} & 
0.72\\SCA3-9 & \bf{681.00} & 12.71 & 
683.27 & 12.47 & 681.00 & 0.00\\
SCA8-0 & 987.08 & 11.30 & 
1006.67 & 10.68 & \bf{961.50} & 
2.66\\SCA8-1 & 1082.51 & 9.62 & 
1087.66 & 9.14 & \bf{1049.65} & 
3.13\\SCA8-2 & 1064.69 & 10.15 & 
1067.63 & 10.01 & \bf{1039.64} & 
2.41\\SCA8-3 & 1022.76 & 9.22 & 
1027.16 & 9.94 & \bf{983.34} & 
4.01\\SCA8-4 & 1069.71 & 10.44 & 
1084.03 & 11.24 & \bf{1065.49} & 
0.40\\SCA8-5 & 1059.39 & 11.02 & 
1063.96 & 9.91 & \bf{1027.08} & 
3.15\\SCA8-6 & 981.41 & 12.04 & 
994.55 & 10.86 & \bf{971.82} & 
0.99\\SCA8-7 & 1071.53 & 9.64 & 
1078.66 & 10.18 & \bf{1051.28} & 
1.93\\SCA8-8 & \bf{1071.18} & 10.31 & 
1090.75 & 10.45 & 1071.18 & 0.00\\
SCA8-9 & 1078.30 & 8.29 & 
1083.36 & 8.77 & \bf{1060.50} & 
1.68\\CON3-0 & 629.03 & 11.62 & 
632.92 & 12.30 & \bf{616.52} & 
2.03\\CON3-1 & 560.61 & 13.78 & 
561.44 & 13.13 & \bf{554.47} & 
1.11\\CON3-2 & 521.38 & 12.76 & 
521.38 & 12.49 & \bf{518.00} & 
0.65\\CON3-3 & 594.10 & 12.25 & 
596.39 & 12.17 & \bf{591.19} & 
0.49\\CON3-4 & 592.58 & 11.64 & 
593.68 & 12.81 & \bf{588.79} & 
0.64\\CON3-5 & 567.94 & 12.23 & 
570.30 & 12.23 & \bf{563.70} & 
0.75\\CON3-6 & 502.16 & 12.90 & 
507.96 & 12.48 & \bf{499.05} & 
0.62\\CON3-7 & 578.41 & 12.56 & 
586.52 & 12.88 & \bf{576.48} & 
0.33\\CON3-8 & 523.68 & 12.85 & 
525.36 & 12.83 & \bf{523.05} & 
0.12\\CON3-9 & 578.25 & 12.29 & 
587.24 & 12.28 & \bf{578.24} & 
0.00\\CON8-0 & 865.86 & 10.71 & 
894.88 & 10.36 & \bf{857.17} & 
1.01\\CON8-1 & 754.22 & 11.20 & 
764.98 & 10.59 & \bf{740.85} & 
1.80\\CON8-2 & 722.29 & 9.68 & 
725.76 & 10.37 & \bf{712.89} & 
1.32\\CON8-3 & 830.79 & 10.62 & 
839.78 & 10.17 & \bf{811.07} & 
2.43\\CON8-4 & 775.49 & 10.03 & 
793.73 & 10.00 & \bf{772.25} & 
0.42\\CON8-5 & 763.13 & 10.49 & 
773.30 & 10.78 & \bf{754.88} & 
1.09\\CON8-6 & 696.58 & 10.44 & 
707.41 & 10.11 & \bf{678.92} & 
2.60\\CON8-7 & 815.72 & 10.69 & 
825.53 & 11.10 & \bf{811.96} & 
0.46\\CON8-8 & 786.86 & 11.10 & 
788.54 & 11.00 & \bf{767.53} & 
2.52\\CON8-9 & 822.87 & 10.82 & 
825.74 & 11.00 & \bf{809.00} & 
1.71\\[1ex]\hline
\end{tabular}
\label{table:nonlin}
\end{table} \clearpage
\begin{table}[ht]
\caption{Resultados de la ejecución de la metaheurística ILS, utilizando instancias de SalhiNagy con la configuración -n 65.0 -LS 50.0}
\centering
\small
\begin{tabular}{c c c c c c c}
\hline\hline
Instancia & Costo mínimo & Tiempo(seg.) & Costo promedio & Tiempo promedio(seg.) & Costo ILS & \%Gap \\ [0.5ex]
\hline
CMT1X & 473.96 & 9.28 & 
478.63 & 10.21 & \bf{466.77} & 
1.54\\CMT1Y & 470.48 & 9.61 & 
476.88 & 9.40 & \bf{466.77} & 
0.79\\CMT2X & 695.73 & 24.81 & 
709.80 & 23.98 & \bf{684.21} & 
1.68\\CMT2Y & 708.90 & 22.98 & 
711.76 & 23.97 & \bf{684.21} & 
3.61\\CMT3X & 731.39 & 64.80 & 
739.55 & 57.27 & \bf{721.40} & 
1.38\\CMT3Y & 729.39 & 54.10 & 
736.56 & 57.27 & \bf{721.40} & 
1.11\\CMT4X & 890.85 & 171.63 & 
899.41 & 146.97 & \bf{852.83} & 
4.46\\CMT4Y & 893.46 & 143.00 & 
902.28 & 158.32 & \bf{852.46} & 
4.81\\CMT5X & 1079.78 & 325.49 & 
1093.13 & 330.57 & \bf{1030.55} & 
4.78\\CMT5Y & 1080.33 & 319.15 & 
1098.41 & 351.53 & \bf{1031.17} & 
4.77\\CMT11X & 880.18 & 106.44 & 
890.03 & 116.41 & \bf{839.39} & 
4.86\\CMT11Y & 880.79 & 104.33 & 
889.73 & 106.72 & \bf{841.88} & 
4.62\\CMT12X & 674.63 & 45.71 & 
678.88 & 52.44 & \bf{662.22} & 
1.87\\CMT12Y & 674.52 & 59.62 & 
678.53 & 53.71 & \bf{662.22} & 
1.86\\[1ex]\hline
\end{tabular}
\label{table:nonlin}
\end{table} \clearpage
\begin{table}[ht]
\caption{Resultados de la ejecución de la metaheurística ILS, utilizando instancias de Dethloff con la configuración -n 65.0 -LS 60.0}
\centering
\small
\begin{tabular}{c c c c c c c}
\hline\hline
Instancia & Costo mínimo & Tiempo(seg.) & Costo promedio & Tiempo promedio(seg.) & Costo ILS & \%Gap \\ [0.5ex]
\hline
SCA3-0 & 636.06 & 14.80 & 
639.43 & 14.60 & \bf{635.62} & 
0.07\\SCA3-1 & \bf{697.84} & 15.74 & 
700.35 & 14.72 & 697.84 & 0.00\\
SCA3-2 & \bf{659.34} & 13.62 & 
661.95 & 14.40 & 659.34 & 0.00\\
SCA3-3 & 680.60 & 15.71 & 
681.71 & 15.10 & \bf{680.04} & 
0.08\\SCA3-4 & \bf{690.50} & 14.47 & 
692.17 & 16.54 & 690.50 & 0.00\\
SCA3-5 & 665.64 & 15.64 & 
674.53 & 15.04 & \bf{659.90} & 
0.87\\SCA3-6 & 652.47 & 13.39 & 
654.00 & 14.73 & \bf{651.09} & 
0.21\\SCA3-7 & 671.67 & 14.50 & 
671.75 & 14.68 & \bf{659.17} & 
1.90\\SCA3-8 & \bf{719.47} & 14.01 & 
721.88 & 15.48 & 719.47 & 0.00\\
SCA3-9 & \bf{681.00} & 12.94 & 
683.75 & 14.16 & 681.00 & 0.00\\
SCA8-0 & 977.93 & 14.64 & 
1003.11 & 13.68 & \bf{961.50} & 
1.71\\SCA8-1 & 1072.05 & 10.01 & 
1083.67 & 11.58 & \bf{1049.65} & 
2.13\\SCA8-2 & 1066.63 & 10.58 & 
1068.79 & 12.05 & \bf{1039.64} & 
2.60\\SCA8-3 & 1005.33 & 12.14 & 
1020.12 & 11.20 & \bf{983.34} & 
2.24\\SCA8-4 & 1074.63 & 12.27 & 
1090.36 & 12.11 & \bf{1065.49} & 
0.86\\SCA8-5 & 1048.25 & 14.14 & 
1067.30 & 13.39 & \bf{1027.08} & 
2.06\\SCA8-6 & 985.96 & 13.37 & 
992.51 & 12.88 & \bf{971.82} & 
1.46\\SCA8-7 & 1067.49 & 11.78 & 
1079.65 & 12.11 & \bf{1051.28} & 
1.54\\SCA8-8 & 1082.91 & 11.90 & 
1092.47 & 11.34 & \bf{1071.18} & 
1.10\\SCA8-9 & 1081.53 & 11.34 & 
1087.37 & 11.65 & \bf{1060.50} & 
1.98\\CON3-0 & 619.09 & 14.98 & 
626.08 & 14.76 & \bf{616.52} & 
0.42\\CON3-1 & 560.75 & 13.38 & 
563.42 & 15.38 & \bf{554.47} & 
1.13\\CON3-2 & 521.38 & 13.61 & 
522.89 & 15.12 & \bf{518.00} & 
0.65\\CON3-3 & 591.48 & 15.22 & 
593.32 & 14.77 & \bf{591.19} & 
0.05\\CON3-4 & 591.43 & 14.28 & 
593.38 & 13.79 & \bf{588.79} & 
0.45\\CON3-5 & 564.88 & 12.72 & 
572.18 & 14.69 & \bf{563.70} & 
0.21\\CON3-6 & 502.64 & 15.05 & 
508.10 & 14.80 & \bf{499.05} & 
0.72\\CON3-7 & 578.41 & 14.44 & 
583.40 & 14.78 & \bf{576.48} & 
0.33\\CON3-8 & 524.59 & 16.43 & 
526.21 & 15.61 & \bf{523.05} & 
0.29\\CON3-9 & 587.97 & 14.18 & 
588.51 & 15.08 & \bf{578.24} & 
1.68\\CON8-0 & 889.90 & 10.68 & 
899.58 & 11.32 & \bf{857.17} & 
3.82\\CON8-1 & 755.71 & 14.99 & 
764.11 & 12.47 & \bf{740.85} & 
2.01\\CON8-2 & 722.51 & 12.07 & 
728.61 & 12.59 & \bf{712.89} & 
1.35\\CON8-3 & 822.00 & 13.54 & 
830.68 & 13.45 & \bf{811.07} & 
1.35\\CON8-4 & 805.03 & 11.00 & 
808.08 & 10.70 & \bf{772.25} & 
4.24\\CON8-5 & 764.17 & 12.72 & 
773.07 & 13.31 & \bf{754.88} & 
1.23\\CON8-6 & 698.58 & 12.53 & 
701.60 & 13.13 & \bf{678.92} & 
2.90\\CON8-7 & 814.50 & 11.73 & 
821.64 & 11.75 & \bf{811.96} & 
0.31\\CON8-8 & 782.61 & 11.55 & 
787.34 & 11.87 & \bf{767.53} & 
1.96\\CON8-9 & 812.03 & 12.44 & 
825.00 & 12.05 & \bf{809.00} & 
0.37\\[1ex]\hline
\end{tabular}
\label{table:nonlin}
\end{table} \clearpage
\begin{table}[ht]
\caption{Resultados de la ejecución de la metaheurística ILS, utilizando instancias de SalhiNagy con la configuración -n 65.0 -LS 60.0}
\centering
\small
\begin{tabular}{c c c c c c c}
\hline\hline
Instancia & Costo mínimo & Tiempo(seg.) & Costo promedio & Tiempo promedio(seg.) & Costo ILS & \%Gap \\ [0.5ex]
\hline
CMT1X & 472.58 & 13.85 & 
476.54 & 12.17 & \bf{466.77} & 
1.24\\CMT1Y & 477.73 & 11.42 & 
480.37 & 10.22 & \bf{466.77} & 
2.35\\CMT2X & 693.67 & 27.40 & 
705.13 & 27.63 & \bf{684.21} & 
1.38\\CMT2Y & 703.54 & 27.23 & 
707.02 & 26.39 & \bf{684.21} & 
2.83\\CMT3X & 729.80 & 60.68 & 
732.32 & 59.83 & \bf{721.40} & 
1.16\\CMT3Y & 728.60 & 73.70 & 
735.05 & 66.97 & \bf{721.40} & 
1.00\\CMT4X & 896.40 & 151.07 & 
903.64 & 153.20 & \bf{852.83} & 
5.11\\CMT4Y & 894.03 & 149.53 & 
900.76 & 155.46 & \bf{852.46} & 
4.88\\CMT5X & 1089.90 & 362.26 & 
1104.00 & 352.00 & \bf{1030.55} & 
5.76\\CMT5Y & 1094.83 & 368.60 & 
1098.26 & 426.10 & \bf{1031.17} & 
6.17\\CMT11X & 878.86 & 112.92 & 
888.00 & 118.16 & \bf{839.39} & 
4.70\\CMT11Y & 878.11 & 154.68 & 
890.23 & 135.71 & \bf{841.88} & 
4.30\\CMT12X & 676.16 & 53.73 & 
680.49 & 56.48 & \bf{662.22} & 
2.11\\CMT12Y & 681.02 & 54.42 & 
682.50 & 55.65 & \bf{662.22} & 
2.84\\[1ex]\hline
\end{tabular}
\label{table:nonlin}
\end{table} \clearpage
\begin{table}[ht]
\caption{Resultados de la ejecución de la metaheurística ILS, utilizando instancias de Dethloff con la configuración -n 65.0 -LS 70.0}
\centering
\small
\begin{tabular}{c c c c c c c}
\hline\hline
Instancia & Costo mínimo & Tiempo(seg.) & Costo promedio & Tiempo promedio(seg.) & Costo ILS & \%Gap \\ [0.5ex]
\hline
SCA3-0 & 636.06 & 16.90 & 
639.43 & 17.51 & \bf{635.62} & 
0.07\\SCA3-1 & \bf{697.84} & 18.88 & 
701.99 & 16.93 & 697.84 & 0.00\\
SCA3-2 & \bf{659.34} & 16.74 & 
661.01 & 16.98 & 659.34 & 0.00\\
SCA3-3 & \bf{680.04} & 18.57 & 
681.39 & 16.30 & 680.04 & 0.00\\
SCA3-4 & \bf{690.50} & 15.18 & 
690.50 & 16.09 & 690.50 & 0.00\\
SCA3-5 & 670.02 & 16.93 & 
674.43 & 17.68 & \bf{659.90} & 
1.53\\SCA3-6 & \bf{651.09} & 15.76 & 
653.27 & 16.70 & 651.09 & 0.00\\
SCA3-7 & 669.89 & 16.03 & 
670.80 & 16.39 & \bf{659.17} & 
1.63\\SCA3-8 & \bf{719.47} & 16.92 & 
723.93 & 16.88 & 719.47 & 0.00\\
SCA3-9 & \bf{681.00} & 16.01 & 
684.40 & 16.03 & 681.00 & 0.00\\
SCA8-0 & 981.19 & 14.91 & 
993.13 & 15.56 & \bf{961.50} & 
2.05\\SCA8-1 & 1074.98 & 11.87 & 
1084.69 & 12.97 & \bf{1049.65} & 
2.41\\SCA8-2 & 1054.30 & 14.36 & 
1060.10 & 14.90 & \bf{1039.64} & 
1.41\\SCA8-3 & 1002.63 & 12.55 & 
1019.11 & 13.00 & \bf{983.34} & 
1.96\\SCA8-4 & 1069.71 & 15.56 & 
1081.55 & 13.63 & \bf{1065.49} & 
0.40\\SCA8-5 & 1043.65 & 14.58 & 
1061.36 & 14.31 & \bf{1027.08} & 
1.61\\SCA8-6 & 991.11 & 13.23 & 
993.87 & 13.82 & \bf{971.82} & 
1.98\\SCA8-7 & 1071.53 & 14.63 & 
1086.40 & 13.70 & \bf{1051.28} & 
1.93\\SCA8-8 & 1087.01 & 13.30 & 
1090.57 & 13.73 & \bf{1071.18} & 
1.48\\SCA8-9 & 1067.42 & 13.73 & 
1083.48 & 12.77 & \bf{1060.50} & 
0.65\\CON3-0 & 632.57 & 16.22 & 
635.98 & 16.27 & \bf{616.52} & 
2.60\\CON3-1 & 556.92 & 18.99 & 
559.79 & 17.96 & \bf{554.47} & 
0.44\\CON3-2 & 521.38 & 17.40 & 
522.32 & 16.40 & \bf{518.00} & 
0.65\\CON3-3 & 591.20 & 17.81 & 
592.36 & 17.05 & \bf{591.19} & 
0.00\\CON3-4 & 591.43 & 15.55 & 
594.45 & 16.61 & \bf{588.79} & 
0.45\\CON3-5 & 569.57 & 15.81 & 
570.16 & 17.48 & \bf{563.70} & 
1.04\\CON3-6 & 502.16 & 17.10 & 
505.64 & 16.05 & \bf{499.05} & 
0.62\\CON3-7 & 578.41 & 15.65 & 
583.66 & 16.04 & \bf{576.48} & 
0.33\\CON3-8 & \bf{523.05} & 19.62 & 
524.78 & 17.91 & 523.05 & 0.00\\
CON3-9 & 588.40 & 17.25 & 
588.55 & 17.77 & \bf{578.24} & 
1.76\\CON8-0 & 889.13 & 11.98 & 
897.17 & 13.59 & \bf{857.17} & 
3.73\\CON8-1 & 754.51 & 13.38 & 
761.72 & 13.11 & \bf{740.85} & 
1.84\\CON8-2 & 726.54 & 16.29 & 
731.47 & 15.34 & \bf{712.89} & 
1.91\\CON8-3 & 826.27 & 13.41 & 
832.63 & 14.35 & \bf{811.07} & 
1.87\\CON8-4 & 780.54 & 14.93 & 
802.58 & 14.23 & \bf{772.25} & 
1.07\\CON8-5 & 771.87 & 15.16 & 
773.28 & 16.16 & \bf{754.88} & 
2.25\\CON8-6 & 698.31 & 14.02 & 
703.16 & 14.08 & \bf{678.92} & 
2.86\\CON8-7 & 815.32 & 14.39 & 
818.69 & 14.88 & \bf{811.96} & 
0.41\\CON8-8 & 791.37 & 13.71 & 
793.73 & 13.35 & \bf{767.53} & 
3.11\\CON8-9 & 820.94 & 17.81 & 
828.65 & 15.40 & \bf{809.00} & 
1.48\\[1ex]\hline
\end{tabular}
\label{table:nonlin}
\end{table} \clearpage
\begin{table}[ht]
\caption{Resultados de la ejecución de la metaheurística ILS, utilizando instancias de SalhiNagy con la configuración -n 65.0 -LS 70.0}
\centering
\small
\begin{tabular}{c c c c c c c}
\hline\hline
Instancia & Costo mínimo & Tiempo(seg.) & Costo promedio & Tiempo promedio(seg.) & Costo ILS & \%Gap \\ [0.5ex]
\hline
CMT1X & 476.93 & 14.61 & 
479.96 & 12.77 & \bf{466.77} & 
2.18\\CMT1Y & 478.84 & 13.00 & 
480.76 & 14.21 & \bf{466.77} & 
2.59\\CMT2X & 707.42 & 31.58 & 
711.80 & 30.91 & \bf{684.21} & 
3.39\\CMT2Y & 700.38 & 29.33 & 
707.35 & 31.00 & \bf{684.21} & 
2.36\\CMT3X & 736.54 & 80.94 & 
737.75 & 71.56 & \bf{721.40} & 
2.10\\CMT3Y & 726.79 & 68.60 & 
732.32 & 72.69 & \bf{721.40} & 
0.75\\CMT4X & 882.70 & 208.51 & 
893.84 & 189.25 & \bf{852.83} & 
3.50\\CMT4Y & 883.43 & 175.48 & 
894.68 & 178.16 & \bf{852.46} & 
3.63\\CMT5X & 1087.99 & 611.04 & 
1099.57 & 462.85 & \bf{1030.55} & 
5.57\\CMT5Y & 1092.31 & 506.29 & 
1097.11 & 449.00 & \bf{1031.17} & 
5.93\\CMT11X & 878.78 & 129.08 & 
890.98 & 146.91 & \bf{839.39} & 
4.69\\CMT11Y & 888.71 & 126.54 & 
893.79 & 129.77 & \bf{841.88} & 
5.56\\CMT12X & 675.41 & 60.79 & 
680.32 & 60.84 & \bf{662.22} & 
1.99\\CMT12Y & 674.21 & 56.85 & 
679.05 & 66.18 & \bf{662.22} & 
1.81\\[1ex]\hline
\end{tabular}
\label{table:nonlin}
\end{table} \clearpage
\begin{table}[ht]
\caption{Resultados de la ejecución de la metaheurística ILS, utilizando instancias de Dethloff con la configuración -n 65.0 -LS 80.0}
\centering
\small
\begin{tabular}{c c c c c c c}
\hline\hline
Instancia & Costo mínimo & Tiempo(seg.) & Costo promedio & Tiempo promedio(seg.) & Costo ILS & \%Gap \\ [0.5ex]
\hline
SCA3-0 & 640.55 & 17.55 & 
640.84 & 18.79 & \bf{635.62} & 
0.78\\SCA3-1 & \bf{697.84} & 19.22 & 
703.44 & 18.54 & 697.84 & 0.00\\
SCA3-2 & \bf{659.34} & 19.33 & 
662.61 & 19.23 & 659.34 & 0.00\\
SCA3-3 & \bf{680.04} & 20.44 & 
680.18 & 20.28 & 680.04 & 0.00\\
SCA3-4 & \bf{690.50} & 16.04 & 
690.50 & 17.27 & 690.50 & 0.00\\
SCA3-5 & 662.75 & 18.50 & 
670.50 & 18.00 & \bf{659.90} & 
0.43\\SCA3-6 & \bf{651.09} & 18.14 & 
652.48 & 18.28 & 651.09 & 0.00\\
SCA3-7 & 671.67 & 19.08 & 
671.72 & 19.02 & \bf{659.17} & 
1.90\\SCA3-8 & \bf{719.47} & 19.59 & 
723.93 & 18.97 & 719.47 & 0.00\\
SCA3-9 & \bf{681.00} & 19.35 & 
686.56 & 18.14 & 681.00 & 0.00\\
SCA8-0 & \bf{961.50} & 19.61 & 
977.64 & 16.09 & 961.50 & 0.00\\
SCA8-1 & 1058.33 & 15.96 & 
1069.45 & 14.38 & \bf{1049.65} & 
0.83\\SCA8-2 & 1054.95 & 16.85 & 
1059.74 & 14.91 & \bf{1039.64} & 
1.47\\SCA8-3 & 1016.62 & 16.68 & 
1025.42 & 14.95 & \bf{983.34} & 
3.38\\SCA8-4 & 1069.71 & 15.93 & 
1072.51 & 15.61 & \bf{1065.49} & 
0.40\\SCA8-5 & 1059.41 & 15.68 & 
1073.54 & 14.94 & \bf{1027.08} & 
3.15\\SCA8-6 & 985.48 & 15.86 & 
995.88 & 15.03 & \bf{971.82} & 
1.41\\SCA8-7 & 1067.11 & 14.92 & 
1070.51 & 15.57 & \bf{1051.28} & 
1.51\\SCA8-8 & 1089.55 & 16.32 & 
1092.01 & 15.74 & \bf{1071.18} & 
1.71\\SCA8-9 & 1067.42 & 11.80 & 
1086.37 & 14.70 & \bf{1060.50} & 
0.65\\CON3-0 & 619.09 & 19.15 & 
628.52 & 19.48 & \bf{616.52} & 
0.42\\CON3-1 & 560.32 & 19.38 & 
560.64 & 20.04 & \bf{554.47} & 
1.06\\CON3-2 & 521.38 & 18.15 & 
521.38 & 18.00 & \bf{518.00} & 
0.65\\CON3-3 & \bf{591.19} & 21.46 & 
594.81 & 20.59 & 591.19 & 0.00\\
CON3-4 & 591.43 & 17.24 & 
592.61 & 18.71 & \bf{588.79} & 
0.45\\CON3-5 & 564.88 & 17.32 & 
570.58 & 18.57 & \bf{563.70} & 
0.21\\CON3-6 & 501.33 & 21.25 & 
505.56 & 19.08 & \bf{499.05} & 
0.46\\CON3-7 & 578.41 & 19.21 & 
586.88 & 18.95 & \bf{576.48} & 
0.33\\CON3-8 & 524.59 & 19.50 & 
527.16 & 19.43 & \bf{523.05} & 
0.29\\CON3-9 & 588.40 & 20.92 & 
589.79 & 19.20 & \bf{578.24} & 
1.76\\CON8-0 & 883.19 & 14.66 & 
892.24 & 15.38 & \bf{857.17} & 
3.04\\CON8-1 & 754.95 & 13.73 & 
762.99 & 15.51 & \bf{740.85} & 
1.90\\CON8-2 & 720.12 & 15.59 & 
727.61 & 15.02 & \bf{712.89} & 
1.01\\CON8-3 & 815.38 & 16.06 & 
832.18 & 17.51 & \bf{811.07} & 
0.53\\CON8-4 & 780.48 & 12.66 & 
799.21 & 14.27 & \bf{772.25} & 
1.07\\CON8-5 & 760.72 & 16.05 & 
769.60 & 15.36 & \bf{754.88} & 
0.77\\CON8-6 & 700.32 & 15.40 & 
703.84 & 16.60 & \bf{678.92} & 
3.15\\CON8-7 & 825.09 & 13.42 & 
827.32 & 15.16 & \bf{811.96} & 
1.62\\CON8-8 & 779.43 & 16.42 & 
790.90 & 15.21 & \bf{767.53} & 
1.55\\CON8-9 & 818.23 & 14.39 & 
823.59 & 14.62 & \bf{809.00} & 
1.14\\[1ex]\hline
\end{tabular}
\label{table:nonlin}
\end{table} \clearpage
\begin{table}[ht]
\caption{Resultados de la ejecución de la metaheurística ILS, utilizando instancias de SalhiNagy con la configuración -n 65.0 -LS 80.0}
\centering
\small
\begin{tabular}{c c c c c c c}
\hline\hline
Instancia & Costo mínimo & Tiempo(seg.) & Costo promedio & Tiempo promedio(seg.) & Costo ILS & \%Gap \\ [0.5ex]
\hline
CMT1X & 475.22 & 14.47 & 
480.72 & 13.95 & \bf{466.77} & 
1.81\\CMT1Y & 475.26 & 16.03 & 
479.17 & 16.13 & \bf{466.77} & 
1.82\\CMT2X & 702.48 & 31.34 & 
707.45 & 33.75 & \bf{684.21} & 
2.67\\CMT2Y & 698.74 & 36.73 & 
708.63 & 32.64 & \bf{684.21} & 
2.12\\CMT3X & 727.97 & 74.77 & 
731.61 & 75.04 & \bf{721.40} & 
0.91\\CMT3Y & 730.44 & 89.16 & 
734.41 & 80.67 & \bf{721.40} & 
1.25\\CMT4X & 896.89 & 188.41 & 
900.54 & 210.01 & \bf{852.83} & 
5.17\\CMT4Y & 888.04 & 194.10 & 
900.04 & 197.07 & \bf{852.46} & 
4.17\\CMT5X & 1075.89 & 445.93 & 
1096.47 & 467.52 & \bf{1030.55} & 
4.40\\CMT5Y & 1073.56 & 537.95 & 
1090.10 & 476.30 & \bf{1031.17} & 
4.11\\CMT11X & 870.22 & 136.41 & 
880.75 & 142.70 & \bf{839.39} & 
3.67\\CMT11Y & 871.52 & 181.71 & 
882.20 & 151.97 & \bf{841.88} & 
3.52\\CMT12X & 676.05 & 70.74 & 
679.97 & 71.23 & \bf{662.22} & 
2.09\\CMT12Y & 674.87 & 65.52 & 
677.47 & 69.69 & \bf{662.22} & 
1.91\\[1ex]\hline
\end{tabular}
\label{table:nonlin}
\end{table} \clearpage
\begin{table}[ht]
\caption{Resultados de la ejecución de la metaheurística SCA, utilizando instancias de Dethloff con la configuración -n 50.0 -b 10 -y 0.1}
\centering
\small
\begin{tabular}{c c c c c c c}
\hline\hline
Instancia & Costo mínimo & Tiempo(seg.) & Costo promedio & Tiempo promedio(seg.) & Costo SCA & \%Gap \\ [0.5ex]
\hline
SCA3-0 & 640.55 & 14.32 & 
640.55 & 14.48 & \bf{636.06} & 
0.71\\SCA3-1 & \bf{697.84} & 10.07 & 
698.76 & 11.62 & 697.84 & 0.00\\
SCA3-2 & 661.13 & 13.29 & 
664.82 & 14.20 & \bf{659.34} & 
0.27\\SCA3-3 & 680.60 & 9.86 & 
680.96 & 13.81 & \bf{680.04} & 
0.08\\SCA3-4 & \bf{690.50} & 19.16 & 
691.53 & 10.88 & 690.50 & 0.00\\
SCA3-5 & 670.10 & 12.90 & 
673.49 & 7.99 & \bf{659.90} & 
1.55\\SCA3-6 & 652.94 & 10.56 & 
652.94 & 13.09 & \bf{651.09} & 
0.28\\SCA3-7 & 671.21 & 17.33 & 
671.55 & 12.24 & \bf{659.17} & 
1.83\\SCA3-8 & \bf{719.47} & 10.33 & 
720.12 & 11.04 & 719.47 & 0.00\\
SCA3-9 & \bf{681.00} & 12.77 & 
682.03 & 12.71 & 681.00 & 0.00\\
SCA8-0 & 973.22 & 51.53 & 
989.72 & 39.96 & \bf{961.50} & 
1.22\\SCA8-1 & 1062.88 & 49.42 & 
1069.50 & 34.15 & \bf{1050.20} & 
1.21\\SCA8-2 & 1051.42 & 40.26 & 
1052.40 & 44.93 & \bf{1039.64} & 
1.13\\SCA8-3 & 1021.31 & 22.58 & 
1023.88 & 30.54 & \bf{983.34} & 
3.86\\SCA8-4 & \bf{1065.49} & 38.28 & 
1077.07 & 41.09 & 1065.49 & 0.00\\
SCA8-5 & 1040.18 & 42.54 & 
1050.17 & 36.56 & \bf{1027.08} & 
1.28\\SCA8-6 & 972.48 & 49.21 & 
978.82 & 58.23 & \bf{971.82} & 
0.07\\SCA8-7 & 1064.06 & 30.38 & 
1065.59 & 31.41 & \bf{1052.17} & 
1.13\\SCA8-8 & \bf{1071.18} & 42.21 & 
1087.02 & 40.00 & 1071.18 & 0.00\\
SCA8-9 & 1072.10 & 54.28 & 
1078.92 & 51.05 & \bf{1060.50} & 
1.09\\CON3-0 & 617.59 & 4.56 & 
620.33 & 7.83 & \bf{616.52} & 
0.17\\CON3-1 & 560.75 & 11.40 & 
560.75 & 10.12 & \bf{554.47} & 
1.13\\CON3-2 & 521.38 & 27.03 & 
521.38 & 15.75 & \bf{519.26} & 
0.41\\CON3-3 & \bf{591.19} & 10.80 & 
591.20 & 12.13 & 591.19 & 0.00\\
CON3-4 & \bf{\underline{588.79}} & 10.38 & 
589.45 & 12.44 & 589.32 & 
-0.09\\CON3-5 & 564.88 & 5.69 & 
564.88 & 12.30 & \bf{563.70} & 
0.21\\CON3-6 & \bf{\underline{499.05}} & 7.90 & 
501.38 & 9.72 & 500.80 & 
-0.35\\CON3-7 & 578.41 & 9.32 & 
582.49 & 9.90 & \bf{576.48} & 
0.33\\CON3-8 & \bf{523.05} & 6.99 & 
523.82 & 10.22 & 523.05 & 0.00\\
CON3-9 & 588.40 & 8.90 & 
588.40 & 7.83 & \bf{580.05} & 
1.44\\CON8-0 & 879.90 & 18.18 & 
885.24 & 26.66 & \bf{857.17} & 
2.65\\CON8-1 & \bf{740.85} & 29.25 & 
750.50 & 40.45 & 740.85 & 0.00\\
CON8-2 & 714.44 & 35.92 & 
716.98 & 42.59 & \bf{713.44} & 
0.14\\CON8-3 & 823.06 & 40.54 & 
829.67 & 41.35 & \bf{811.07} & 
1.48\\CON8-4 & 780.03 & 37.72 & 
786.98 & 29.98 & \bf{772.25} & 
1.01\\CON8-5 & 760.41 & 30.13 & 
762.49 & 31.32 & \bf{756.91} & 
0.46\\CON8-6 & 689.23 & 39.01 & 
692.66 & 34.26 & \bf{678.92} & 
1.52\\CON8-7 & 814.79 & 79.99 & 
817.82 & 49.43 & \bf{811.96} & 
0.35\\CON8-8 & 779.96 & 43.22 & 
786.02 & 45.56 & \bf{767.53} & 
1.62\\CON8-9 & 815.09 & 38.91 & 
823.54 & 45.53 & \bf{809.00} & 
0.75\\[1ex]\hline
\end{tabular}
\label{table:nonlin}
\end{table} \clearpage
\begin{table}[ht]
\caption{Resultados de la ejecución de la metaheurística SCA, utilizando instancias de SalhiNagy con la configuración -n 50.0 -b 10 -y 0.1}
\centering
\small
\begin{tabular}{c c c c c c c}
\hline\hline
Instancia & Costo mínimo & Tiempo(seg.) & Costo promedio & Tiempo promedio(seg.) & Costo SCA & \%Gap \\ [0.5ex]
\hline
[1ex]\hline
\end{tabular}
\label{table:nonlin}
\end{table} \clearpage
\begin{table}[ht]
\caption{Resultados de la ejecución de la metaheurística SCA, utilizando instancias de Dethloff con la configuración -n 50.0 -b 10 -y .2}
\centering
\small
\begin{tabular}{c c c c c c c}
\hline\hline
Instancia & Costo mínimo & Tiempo(seg.) & Costo promedio & Tiempo promedio(seg.) & Costo SCA & \%Gap \\ [0.5ex]
\hline
SCA3-0 & 640.55 & 15.38 & 
640.55 & 13.54 & \bf{636.06} & 
0.71\\SCA3-1 & 700.50 & 12.39 & 
700.76 & 11.93 & \bf{697.84} & 
0.38\\SCA3-2 & 661.13 & 18.58 & 
666.28 & 15.76 & \bf{659.34} & 
0.27\\SCA3-3 & 680.60 & 11.91 & 
680.88 & 11.16 & \bf{680.04} & 
0.08\\SCA3-4 & \bf{690.50} & 15.01 & 
691.02 & 14.34 & 690.50 & 0.00\\
SCA3-5 & 670.10 & 11.84 & 
673.53 & 11.26 & \bf{659.90} & 
1.55\\SCA3-6 & \bf{651.09} & 11.75 & 
652.26 & 12.92 & 651.09 & 0.00\\
SCA3-7 & 666.60 & 17.59 & 
669.96 & 12.87 & \bf{659.17} & 
1.13\\SCA3-8 & \bf{719.47} & 9.50 & 
719.62 & 11.38 & 719.47 & 0.00\\
SCA3-9 & \bf{681.00} & 12.26 & 
682.21 & 14.49 & 681.00 & 0.00\\
SCA8-0 & 984.75 & 40.68 & 
996.60 & 32.06 & \bf{961.50} & 
2.42\\SCA8-1 & 1058.43 & 32.95 & 
1071.32 & 42.10 & \bf{1050.20} & 
0.78\\SCA8-2 & 1051.55 & 35.46 & 
1053.22 & 36.61 & \bf{1039.64} & 
1.15\\SCA8-3 & 1015.27 & 51.16 & 
1021.73 & 35.41 & \bf{983.34} & 
3.25\\SCA8-4 & 1071.64 & 33.47 & 
1078.29 & 36.90 & \bf{1065.49} & 
0.58\\SCA8-5 & 1043.05 & 33.45 & 
1050.81 & 46.76 & \bf{1027.08} & 
1.55\\SCA8-6 & 972.48 & 36.74 & 
977.35 & 48.68 & \bf{971.82} & 
0.07\\SCA8-7 & 1064.06 & 27.22 & 
1066.54 & 39.40 & \bf{1052.17} & 
1.13\\SCA8-8 & 1089.28 & 27.68 & 
1094.48 & 42.44 & \bf{1071.18} & 
1.69\\SCA8-9 & 1067.42 & 44.02 & 
1074.45 & 41.91 & \bf{1060.50} & 
0.65\\CON3-0 & 619.09 & 13.28 & 
621.88 & 8.38 & \bf{616.52} & 
0.42\\CON3-1 & 560.75 & 9.20 & 
560.75 & 12.86 & \bf{554.47} & 
1.13\\CON3-2 & 521.38 & 15.30 & 
523.24 & 11.83 & \bf{519.26} & 
0.41\\CON3-3 & 591.20 & 11.02 & 
591.20 & 13.51 & \bf{591.19} & 
0.00\\CON3-4 & 591.43 & 10.85 & 
591.72 & 12.98 & \bf{589.32} & 
0.36\\CON3-5 & 564.88 & 11.74 & 
567.15 & 11.04 & \bf{563.70} & 
0.21\\CON3-6 & 502.16 & 10.94 & 
503.02 & 13.45 & \bf{500.80} & 
0.27\\CON3-7 & 578.79 & 14.42 & 
584.21 & 11.84 & \bf{576.48} & 
0.40\\CON3-8 & \bf{523.05} & 8.84 & 
523.49 & 12.88 & 523.05 & 0.00\\
CON3-9 & 588.40 & 10.37 & 
588.40 & 8.77 & \bf{580.05} & 
1.44\\CON8-0 & 860.72 & 25.30 & 
871.95 & 28.05 & \bf{857.17} & 
0.41\\CON8-1 & 741.70 & 27.16 & 
747.79 & 35.99 & \bf{740.85} & 
0.11\\CON8-2 & 713.60 & 35.42 & 
716.01 & 36.75 & \bf{713.44} & 
0.02\\CON8-3 & 824.69 & 37.27 & 
831.27 & 36.19 & \bf{811.07} & 
1.68\\CON8-4 & 781.90 & 58.77 & 
785.96 & 40.17 & \bf{772.25} & 
1.25\\CON8-5 & \bf{\underline{754.95}} & 31.45 & 
761.30 & 43.12 & 756.91 & 
-0.26\\CON8-6 & 691.74 & 40.72 & 
696.12 & 42.32 & \bf{678.92} & 
1.89\\CON8-7 & 815.43 & 54.91 & 
815.59 & 45.87 & \bf{811.96} & 
0.43\\CON8-8 & 784.16 & 37.49 & 
786.43 & 33.55 & \bf{767.53} & 
2.17\\CON8-9 & 818.66 & 22.65 & 
824.19 & 34.31 & \bf{809.00} & 
1.19\\[1ex]\hline
\end{tabular}
\label{table:nonlin}
\end{table} \clearpage
\begin{table}[ht]
\caption{Resultados de la ejecución de la metaheurística SCA, utilizando instancias de SalhiNagy con la configuración -n 50.0 -b 10 -y .2}
\centering
\small
\begin{tabular}{c c c c c c c}
\hline\hline
Instancia & Costo mínimo & Tiempo(seg.) & Costo promedio & Tiempo promedio(seg.) & Costo SCA & \%Gap \\ [0.5ex]
\hline
[1ex]\hline
\end{tabular}
\label{table:nonlin}
\end{table} \clearpage
\begin{table}[ht]
\caption{Resultados de la ejecución de la metaheurística SCA, utilizando instancias de Dethloff con la configuración -n 50.0 -b 10 -y .3}
\centering
\small
\begin{tabular}{c c c c c c c}
\hline\hline
Instancia & Costo mínimo & Tiempo(seg.) & Costo promedio & Tiempo promedio(seg.) & Costo SCA & \%Gap \\ [0.5ex]
\hline
SCA3-0 & \bf{636.06} & 10.13 & 
639.43 & 17.09 & 636.06 & 0.00\\
SCA3-1 & \bf{697.84} & 8.75 & 
698.76 & 9.08 & 697.84 & 0.00\\
SCA3-2 & \bf{659.34} & 11.91 & 
665.53 & 19.70 & 659.34 & 0.00\\
SCA3-3 & \bf{680.04} & 17.78 & 
680.78 & 14.15 & 680.04 & 0.00\\
SCA3-4 & \bf{690.50} & 24.38 & 
692.05 & 16.15 & 690.50 & 0.00\\
SCA3-5 & 670.10 & 13.45 & 
673.03 & 12.38 & \bf{659.90} & 
1.55\\SCA3-6 & 652.47 & 24.99 & 
653.01 & 17.97 & \bf{651.09} & 
0.21\\SCA3-7 & 667.24 & 13.49 & 
668.35 & 13.68 & \bf{659.17} & 
1.22\\SCA3-8 & \bf{719.47} & 10.94 & 
719.47 & 13.14 & 719.47 & 0.00\\
SCA3-9 & \bf{681.00} & 16.52 & 
681.00 & 17.43 & 681.00 & 0.00\\
SCA8-0 & 983.13 & 30.99 & 
994.50 & 36.81 & \bf{961.50} & 
2.25\\SCA8-1 & 1062.08 & 27.71 & 
1064.86 & 40.84 & \bf{1050.20} & 
1.13\\SCA8-2 & 1047.63 & 39.31 & 
1050.45 & 38.09 & \bf{1039.64} & 
0.77\\SCA8-3 & 1013.77 & 34.51 & 
1020.12 & 31.38 & \bf{983.34} & 
3.09\\SCA8-4 & 1071.86 & 33.04 & 
1082.69 & 31.52 & \bf{1065.49} & 
0.60\\SCA8-5 & 1045.89 & 105.03 & 
1049.26 & 58.78 & \bf{1027.08} & 
1.83\\SCA8-6 & \bf{971.82} & 35.83 & 
973.91 & 38.55 & 971.82 & 0.00\\
SCA8-7 & 1063.22 & 32.80 & 
1068.31 & 40.88 & \bf{1052.17} & 
1.05\\SCA8-8 & 1084.41 & 49.40 & 
1088.62 & 40.93 & \bf{1071.18} & 
1.24\\SCA8-9 & 1073.62 & 35.13 & 
1077.70 & 39.83 & \bf{1060.50} & 
1.24\\CON3-0 & 620.76 & 10.31 & 
627.11 & 11.78 & \bf{616.52} & 
0.69\\CON3-1 & 560.75 & 19.47 & 
560.75 & 15.87 & \bf{554.47} & 
1.13\\CON3-2 & 521.38 & 15.50 & 
521.38 & 10.34 & \bf{519.26} & 
0.41\\CON3-3 & \bf{591.19} & 15.24 & 
591.20 & 16.20 & 591.19 & 0.00\\
CON3-4 & 591.43 & 10.47 & 
591.43 & 13.98 & \bf{589.32} & 
0.36\\CON3-5 & \bf{563.70} & 8.51 & 
564.00 & 14.87 & 563.70 & 0.00\\
CON3-6 & 502.16 & 7.96 & 
502.16 & 9.07 & \bf{500.80} & 
0.27\\CON3-7 & 580.40 & 8.00 & 
582.77 & 9.84 & \bf{576.48} & 
0.68\\CON3-8 & \bf{523.05} & 9.67 & 
523.07 & 9.57 & 523.05 & 0.00\\
CON3-9 & 583.86 & 7.92 & 
587.26 & 10.00 & \bf{580.05} & 
0.66\\CON8-0 & 876.12 & 31.48 & 
881.12 & 41.49 & \bf{857.17} & 
2.21\\CON8-1 & 741.70 & 41.21 & 
747.20 & 43.35 & \bf{740.85} & 
0.11\\CON8-2 & \bf{\underline{713.10}} & 35.53 & 
716.73 & 42.20 & 713.44 & 
-0.05\\CON8-3 & 812.70 & 60.29 & 
826.86 & 59.08 & \bf{811.07} & 
0.20\\CON8-4 & 786.91 & 43.47 & 
788.21 & 40.24 & \bf{772.25} & 
1.90\\CON8-5 & 758.12 & 47.89 & 
760.97 & 44.09 & \bf{756.91} & 
0.16\\CON8-6 & 693.87 & 41.11 & 
698.46 & 37.55 & \bf{678.92} & 
2.20\\CON8-7 & 816.07 & 33.01 & 
818.25 & 44.49 & \bf{811.96} & 
0.51\\CON8-8 & 782.09 & 20.92 & 
786.07 & 33.62 & \bf{767.53} & 
1.90\\CON8-9 & 819.64 & 26.96 & 
823.87 & 39.46 & \bf{809.00} & 
1.32\\[1ex]\hline
\end{tabular}
\label{table:nonlin}
\end{table} \clearpage
\begin{table}[ht]
\caption{Resultados de la ejecución de la metaheurística SCA, utilizando instancias de SalhiNagy con la configuración -n 50.0 -b 10 -y .3}
\centering
\small
\begin{tabular}{c c c c c c c}
\hline\hline
Instancia & Costo mínimo & Tiempo(seg.) & Costo promedio & Tiempo promedio(seg.) & Costo SCA & \%Gap \\ [0.5ex]
\hline
[1ex]\hline
\end{tabular}
\label{table:nonlin}
\end{table} \clearpage
\begin{table}[ht]
\caption{Resultados de la ejecución de la metaheurística SCA, utilizando instancias de Dethloff con la configuración -n 50.0 -b 10 -y .4}
\centering
\small
\begin{tabular}{c c c c c c c}
\hline\hline
Instancia & Costo mínimo & Tiempo(seg.) & Costo promedio & Tiempo promedio(seg.) & Costo SCA & \%Gap \\ [0.5ex]
\hline
SCA3-0 & 640.55 & 22.46 & 
640.55 & 16.31 & \bf{636.06} & 
0.71\\SCA3-1 & \bf{697.84} & 28.08 & 
698.76 & 16.91 & 697.84 & 0.00\\
SCA3-2 & \bf{659.34} & 18.07 & 
660.24 & 20.24 & 659.34 & 0.00\\
SCA3-3 & \bf{680.04} & 16.06 & 
680.64 & 10.91 & 680.04 & 0.00\\
SCA3-4 & \bf{690.50} & 16.22 & 
691.02 & 14.87 & 690.50 & 0.00\\
SCA3-5 & 661.07 & 9.26 & 
668.05 & 9.16 & \bf{659.90} & 
0.18\\SCA3-6 & 652.94 & 17.61 & 
653.60 & 18.14 & \bf{651.09} & 
0.28\\SCA3-7 & 666.60 & 11.16 & 
669.96 & 12.58 & \bf{659.17} & 
1.13\\SCA3-8 & \bf{719.47} & 8.84 & 
719.47 & 16.20 & 719.47 & 0.00\\
SCA3-9 & \bf{681.00} & 16.98 & 
684.33 & 16.83 & 681.00 & 0.00\\
SCA8-0 & 976.79 & 32.01 & 
992.08 & 38.64 & \bf{961.50} & 
1.59\\SCA8-1 & 1060.62 & 49.42 & 
1076.76 & 32.15 & \bf{1050.20} & 
0.99\\SCA8-2 & 1050.37 & 49.72 & 
1051.22 & 45.08 & \bf{1039.64} & 
1.03\\SCA8-3 & 1024.37 & 62.38 & 
1028.48 & 44.51 & \bf{983.34} & 
4.17\\SCA8-4 & 1074.63 & 34.14 & 
1082.83 & 33.63 & \bf{1065.49} & 
0.86\\SCA8-5 & 1030.08 & 40.04 & 
1041.43 & 44.72 & \bf{1027.08} & 
0.29\\SCA8-6 & 972.48 & 44.25 & 
978.68 & 50.09 & \bf{971.82} & 
0.07\\SCA8-7 & 1066.65 & 49.32 & 
1069.76 & 39.48 & \bf{1052.17} & 
1.38\\SCA8-8 & 1085.98 & 49.50 & 
1093.79 & 45.08 & \bf{1071.18} & 
1.38\\SCA8-9 & 1069.70 & 48.87 & 
1073.41 & 36.54 & \bf{1060.50} & 
0.87\\CON3-0 & 622.11 & 11.05 & 
630.45 & 9.01 & \bf{616.52} & 
0.91\\CON3-1 & 560.75 & 12.72 & 
561.38 & 9.66 & \bf{554.47} & 
1.13\\CON3-2 & 521.38 & 11.70 & 
523.06 & 11.71 & \bf{519.26} & 
0.41\\CON3-3 & 591.20 & 7.44 & 
591.20 & 9.08 & \bf{591.19} & 
0.00\\CON3-4 & 591.43 & 13.39 & 
591.43 & 11.19 & \bf{589.32} & 
0.36\\CON3-5 & \bf{563.70} & 8.33 & 
564.00 & 8.16 & 563.70 & 0.00\\
CON3-6 & 502.16 & 8.35 & 
502.65 & 10.32 & \bf{500.80} & 
0.27\\CON3-7 & 578.22 & 15.48 & 
583.86 & 12.43 & \bf{576.48} & 
0.30\\CON3-8 & \bf{523.05} & 6.95 & 
523.24 & 12.60 & 523.05 & 0.00\\
CON3-9 & \bf{\underline{578.25}} & 16.94 & 
583.33 & 13.68 & 580.05 & 
-0.31\\CON8-0 & 874.49 & 25.98 & 
877.32 & 29.24 & \bf{857.17} & 
2.02\\CON8-1 & 742.47 & 43.24 & 
749.62 & 44.61 & \bf{740.85} & 
0.22\\CON8-2 & \bf{713.44} & 62.54 & 
715.01 & 53.02 & 713.44 & 0.00\\
CON8-3 & 818.51 & 50.38 & 
830.23 & 37.95 & \bf{811.07} & 
0.92\\CON8-4 & 772.32 & 26.54 & 
784.52 & 44.62 & \bf{772.25} & 
0.01\\CON8-5 & \bf{\underline{755.14}} & 37.04 & 
759.82 & 37.90 & 756.91 & 
-0.23\\CON8-6 & 684.05 & 25.99 & 
693.77 & 46.38 & \bf{678.92} & 
0.76\\CON8-7 & 814.79 & 50.27 & 
815.55 & 43.72 & \bf{811.96} & 
0.35\\CON8-8 & 782.68 & 29.63 & 
788.28 & 33.05 & \bf{767.53} & 
1.97\\CON8-9 & 813.68 & 25.08 & 
819.28 & 31.09 & \bf{809.00} & 
0.58\\[1ex]\hline
\end{tabular}
\label{table:nonlin}
\end{table} \clearpage
\begin{table}[ht]
\caption{Resultados de la ejecución de la metaheurística SCA, utilizando instancias de SalhiNagy con la configuración -n 50.0 -b 10 -y .4}
\centering
\small
\begin{tabular}{c c c c c c c}
\hline\hline
Instancia & Costo mínimo & Tiempo(seg.) & Costo promedio & Tiempo promedio(seg.) & Costo SCA & \%Gap \\ [0.5ex]
\hline
[1ex]\hline
\end{tabular}
\label{table:nonlin}
\end{table} \clearpage
\begin{table}[ht]
\caption{Resultados de la ejecución de la metaheurística SCA, utilizando instancias de Dethloff con la configuración -n 50.0 -b 10 -y .5}
\centering
\small
\begin{tabular}{c c c c c c c}
\hline\hline
Instancia & Costo mínimo & Tiempo(seg.) & Costo promedio & Tiempo promedio(seg.) & Costo SCA & \%Gap \\ [0.5ex]
\hline
SCA3-0 & 640.55 & 16.21 & 
640.55 & 11.96 & \bf{636.06} & 
0.71\\SCA3-1 & \bf{697.84} & 13.43 & 
697.84 & 12.44 & 697.84 & 0.00\\
SCA3-2 & 661.13 & 17.11 & 
663.89 & 17.92 & \bf{659.34} & 
0.27\\SCA3-3 & \bf{680.04} & 12.03 & 
680.50 & 11.79 & 680.04 & 0.00\\
SCA3-4 & \bf{690.50} & 11.38 & 
691.02 & 12.47 & 690.50 & 0.00\\
SCA3-5 & \bf{659.90} & 10.23 & 
665.69 & 9.60 & 659.90 & 0.00\\
SCA3-6 & 652.94 & 23.12 & 
655.07 & 17.83 & \bf{651.09} & 
0.28\\SCA3-7 & 666.60 & 14.70 & 
669.29 & 13.96 & \bf{659.17} & 
1.13\\SCA3-8 & \bf{719.47} & 20.07 & 
719.47 & 16.21 & 719.47 & 0.00\\
SCA3-9 & \bf{681.00} & 10.34 & 
682.03 & 14.86 & 681.00 & 0.00\\
SCA8-0 & 973.22 & 56.93 & 
991.37 & 38.71 & \bf{961.50} & 
1.22\\SCA8-1 & 1050.38 & 32.53 & 
1058.97 & 47.84 & \bf{1050.20} & 
0.02\\SCA8-2 & 1051.21 & 35.72 & 
1052.27 & 42.75 & \bf{1039.64} & 
1.11\\SCA8-3 & 1008.89 & 34.95 & 
1021.97 & 34.89 & \bf{983.34} & 
2.60\\SCA8-4 & 1073.78 & 35.44 & 
1082.86 & 32.34 & \bf{1065.49} & 
0.78\\SCA8-5 & 1050.06 & 36.43 & 
1051.84 & 43.11 & \bf{1027.08} & 
2.24\\SCA8-6 & 972.48 & 36.51 & 
978.96 & 32.15 & \bf{971.82} & 
0.07\\SCA8-7 & 1063.60 & 45.51 & 
1067.59 & 41.28 & \bf{1052.17} & 
1.09\\SCA8-8 & 1082.12 & 28.12 & 
1085.84 & 33.27 & \bf{1071.18} & 
1.02\\SCA8-9 & 1072.10 & 54.41 & 
1085.23 & 42.74 & \bf{1060.50} & 
1.09\\CON3-0 & 621.22 & 9.54 & 
629.59 & 9.03 & \bf{616.52} & 
0.76\\CON3-1 & 560.75 & 8.46 & 
560.75 & 11.43 & \bf{554.47} & 
1.13\\CON3-2 & 521.38 & 6.90 & 
521.38 & 8.26 & \bf{519.26} & 
0.41\\CON3-3 & 591.20 & 12.90 & 
591.20 & 12.26 & \bf{591.19} & 
0.00\\CON3-4 & \bf{\underline{588.79}} & 11.06 & 
590.77 & 15.59 & 589.32 & 
-0.09\\CON3-5 & \bf{563.70} & 9.70 & 
564.59 & 9.06 & 563.70 & 0.00\\
CON3-6 & 502.16 & 8.69 & 
502.65 & 12.96 & \bf{500.80} & 
0.27\\CON3-7 & 578.41 & 14.26 & 
584.11 & 14.33 & \bf{576.48} & 
0.33\\CON3-8 & \bf{523.05} & 10.98 & 
525.05 & 9.56 & 523.05 & 0.00\\
CON3-9 & 587.23 & 8.39 & 
587.96 & 10.08 & \bf{580.05} & 
1.24\\CON8-0 & 858.63 & 35.51 & 
872.16 & 32.77 & \bf{857.17} & 
0.17\\CON8-1 & 741.70 & 52.39 & 
746.01 & 42.99 & \bf{740.85} & 
0.11\\CON8-2 & \bf{\underline{713.05}} & 38.96 & 
715.14 & 45.13 & 713.44 & 
-0.05\\CON8-3 & 817.44 & 31.09 & 
829.03 & 43.44 & \bf{811.07} & 
0.79\\CON8-4 & 780.10 & 55.99 & 
788.96 & 44.69 & \bf{772.25} & 
1.02\\CON8-5 & \bf{\underline{755.86}} & 67.58 & 
759.84 & 52.42 & 756.91 & 
-0.14\\CON8-6 & 687.70 & 63.07 & 
693.41 & 38.40 & \bf{678.92} & 
1.29\\CON8-7 & 815.43 & 37.80 & 
818.25 & 44.81 & \bf{811.96} & 
0.43\\CON8-8 & 782.06 & 50.52 & 
784.77 & 33.15 & \bf{767.53} & 
1.89\\CON8-9 & 817.71 & 57.98 & 
821.70 & 41.04 & \bf{809.00} & 
1.08\\[1ex]\hline
\end{tabular}
\label{table:nonlin}
\end{table} \clearpage
\begin{table}[ht]
\caption{Resultados de la ejecución de la metaheurística SCA, utilizando instancias de SalhiNagy con la configuración -n 50.0 -b 10 -y .5}
\centering
\small
\begin{tabular}{c c c c c c c}
\hline\hline
Instancia & Costo mínimo & Tiempo(seg.) & Costo promedio & Tiempo promedio(seg.) & Costo SCA & \%Gap \\ [0.5ex]
\hline
[1ex]\hline
\end{tabular}
\label{table:nonlin}
\end{table} \clearpage
\begin{table}[ht]
\caption{Resultados de la ejecución de la metaheurística SCA, utilizando instancias de Dethloff con la configuración -n 75.0 -b 10 -y 0.1}
\centering
\small
\begin{tabular}{c c c c c c c}
\hline\hline
Instancia & Costo mínimo & Tiempo(seg.) & Costo promedio & Tiempo promedio(seg.) & Costo SCA & \%Gap \\ [0.5ex]
\hline
SCA3-0 & 640.55 & 11.25 & 
640.55 & 14.37 & \bf{636.06} & 
0.71\\SCA3-1 & 700.50 & 10.39 & 
701.27 & 9.14 & \bf{697.84} & 
0.38\\SCA3-2 & 664.21 & 8.43 & 
664.21 & 8.71 & \bf{659.34} & 
0.74\\SCA3-3 & \bf{680.04} & 11.74 & 
680.61 & 17.29 & 680.04 & 0.00\\
SCA3-4 & \bf{690.50} & 15.71 & 
691.53 & 13.61 & 690.50 & 0.00\\
SCA3-5 & 672.49 & 13.31 & 
675.13 & 10.05 & \bf{659.90} & 
1.91\\SCA3-6 & 652.94 & 21.50 & 
654.38 & 16.67 & \bf{651.09} & 
0.28\\SCA3-7 & 671.67 & 10.55 & 
671.67 & 10.03 & \bf{659.17} & 
1.90\\SCA3-8 & \bf{719.47} & 8.96 & 
719.70 & 10.21 & 719.47 & 0.00\\
SCA3-9 & \bf{681.00} & 9.68 & 
681.17 & 9.36 & 681.00 & 0.00\\
SCA8-0 & 983.04 & 31.91 & 
992.63 & 34.07 & \bf{961.50} & 
2.24\\SCA8-1 & 1072.18 & 29.76 & 
1078.59 & 29.75 & \bf{1050.20} & 
2.09\\SCA8-2 & 1050.37 & 42.05 & 
1052.32 & 35.80 & \bf{1039.64} & 
1.03\\SCA8-3 & 1010.01 & 32.39 & 
1013.11 & 28.80 & \bf{983.34} & 
2.71\\SCA8-4 & 1067.29 & 50.67 & 
1073.31 & 37.88 & \bf{1065.49} & 
0.17\\SCA8-5 & 1050.64 & 26.01 & 
1058.75 & 26.52 & \bf{1027.08} & 
2.29\\SCA8-6 & 972.48 & 57.22 & 
978.54 & 45.55 & \bf{971.82} & 
0.07\\SCA8-7 & 1070.92 & 39.91 & 
1073.18 & 42.98 & \bf{1052.17} & 
1.78\\SCA8-8 & 1082.12 & 31.66 & 
1083.16 & 34.62 & \bf{1071.18} & 
1.02\\SCA8-9 & 1078.80 & 58.62 & 
1087.30 & 47.25 & \bf{1060.50} & 
1.73\\CON3-0 & 620.49 & 11.28 & 
624.10 & 7.47 & \bf{616.52} & 
0.64\\CON3-1 & 560.13 & 6.72 & 
560.28 & 11.85 & \bf{554.47} & 
1.02\\CON3-2 & 521.38 & 12.74 & 
521.38 & 13.07 & \bf{519.26} & 
0.41\\CON3-3 & \bf{591.19} & 8.66 & 
591.20 & 10.30 & 591.19 & 0.00\\
CON3-4 & \bf{\underline{588.79}} & 17.63 & 
590.77 & 13.71 & 589.32 & 
-0.09\\CON3-5 & 564.88 & 13.03 & 
564.99 & 10.49 & \bf{563.70} & 
0.21\\CON3-6 & 502.16 & 15.85 & 
502.16 & 13.32 & \bf{500.80} & 
0.27\\CON3-7 & 578.79 & 16.59 & 
583.91 & 13.41 & \bf{576.48} & 
0.40\\CON3-8 & 523.19 & 17.56 & 
524.27 & 15.06 & \bf{523.05} & 
0.03\\CON3-9 & 588.28 & 12.61 & 
589.35 & 11.63 & \bf{580.05} & 
1.42\\CON8-0 & 871.92 & 23.62 & 
880.47 & 28.18 & \bf{857.17} & 
1.72\\CON8-1 & 742.28 & 82.98 & 
747.81 & 50.10 & \bf{740.85} & 
0.19\\CON8-2 & \bf{\underline{713.05}} & 34.82 & 
715.27 & 35.33 & 713.44 & 
-0.05\\CON8-3 & 824.71 & 33.47 & 
833.11 & 44.65 & \bf{811.07} & 
1.68\\CON8-4 & 773.27 & 28.24 & 
778.08 & 41.43 & \bf{772.25} & 
0.13\\CON8-5 & 759.87 & 71.44 & 
762.09 & 53.11 & \bf{756.91} & 
0.39\\CON8-6 & 688.77 & 34.08 & 
695.24 & 38.44 & \bf{678.92} & 
1.45\\CON8-7 & 814.79 & 38.38 & 
819.58 & 37.88 & \bf{811.96} & 
0.35\\CON8-8 & 779.43 & 28.57 & 
782.63 & 30.08 & \bf{767.53} & 
1.55\\CON8-9 & 817.79 & 24.48 & 
819.89 & 38.25 & \bf{809.00} & 
1.09\\[1ex]\hline
\end{tabular}
\label{table:nonlin}
\end{table} \clearpage
\begin{table}[ht]
\caption{Resultados de la ejecución de la metaheurística SCA, utilizando instancias de SalhiNagy con la configuración -n 75.0 -b 10 -y 0.1}
\centering
\small
\begin{tabular}{c c c c c c c}
\hline\hline
Instancia & Costo mínimo & Tiempo(seg.) & Costo promedio & Tiempo promedio(seg.) & Costo SCA & \%Gap \\ [0.5ex]
\hline
[1ex]\hline
\end{tabular}
\label{table:nonlin}
\end{table} \clearpage
\begin{table}[ht]
\caption{Resultados de la ejecución de la metaheurística SCA, utilizando instancias de Dethloff con la configuración -n 75.0 -b 10 -y .2}
\centering
\small
\begin{tabular}{c c c c c c c}
\hline\hline
Instancia & Costo mínimo & Tiempo(seg.) & Costo promedio & Tiempo promedio(seg.) & Costo SCA & \%Gap \\ [0.5ex]
\hline
SCA3-0 & 640.55 & 11.71 & 
640.55 & 13.82 & \bf{636.06} & 
0.71\\SCA3-1 & \bf{697.84} & 14.59 & 
697.84 & 11.69 & 697.84 & 0.00\\
SCA3-2 & 661.13 & 12.46 & 
663.11 & 16.04 & \bf{659.34} & 
0.27\\SCA3-3 & 680.60 & 12.08 & 
680.88 & 12.22 & \bf{680.04} & 
0.08\\SCA3-4 & 692.57 & 9.73 & 
693.07 & 9.98 & \bf{690.50} & 
0.30\\SCA3-5 & 665.64 & 13.28 & 
674.38 & 10.22 & \bf{659.90} & 
0.87\\SCA3-6 & 652.94 & 7.98 & 
652.94 & 9.71 & \bf{651.09} & 
0.28\\SCA3-7 & 667.34 & 8.34 & 
670.59 & 12.38 & \bf{659.17} & 
1.24\\SCA3-8 & \bf{719.47} & 10.94 & 
719.47 & 10.04 & 719.47 & 0.00\\
SCA3-9 & \bf{681.00} & 18.57 & 
683.08 & 15.18 & 681.00 & 0.00\\
SCA8-0 & 980.51 & 53.60 & 
990.81 & 41.47 & \bf{961.50} & 
1.98\\SCA8-1 & 1064.55 & 41.60 & 
1074.64 & 33.59 & \bf{1050.20} & 
1.37\\SCA8-2 & 1051.21 & 49.63 & 
1053.02 & 42.48 & \bf{1039.64} & 
1.11\\SCA8-3 & 1010.01 & 22.90 & 
1019.29 & 25.56 & \bf{983.34} & 
2.71\\SCA8-4 & 1072.06 & 33.66 & 
1078.79 & 37.24 & \bf{1065.49} & 
0.62\\SCA8-5 & 1049.44 & 36.51 & 
1056.46 & 69.40 & \bf{1027.08} & 
2.18\\SCA8-6 & 972.48 & 34.45 & 
972.48 & 51.11 & \bf{971.82} & 
0.07\\SCA8-7 & 1070.67 & 27.77 & 
1072.52 & 29.13 & \bf{1052.17} & 
1.76\\SCA8-8 & 1075.00 & 56.25 & 
1087.65 & 41.40 & \bf{1071.18} & 
0.36\\SCA8-9 & 1085.18 & 77.98 & 
1087.69 & 47.08 & \bf{1060.50} & 
2.33\\CON3-0 & 620.29 & 7.97 & 
628.48 & 10.17 & \bf{616.52} & 
0.61\\CON3-1 & 560.75 & 17.72 & 
560.75 & 18.33 & \bf{554.47} & 
1.13\\CON3-2 & 521.36 & 11.14 & 
521.38 & 14.15 & \bf{519.26} & 
0.40\\CON3-3 & \bf{591.19} & 7.74 & 
591.19 & 7.47 & 591.19 & 0.00\\
CON3-4 & 591.43 & 9.20 & 
592.02 & 14.65 & \bf{589.32} & 
0.36\\CON3-5 & \bf{563.70} & 10.81 & 
563.70 & 11.06 & 563.70 & 0.00\\
CON3-6 & 502.16 & 10.43 & 
502.31 & 9.46 & \bf{500.80} & 
0.27\\CON3-7 & 578.22 & 11.76 & 
584.06 & 9.79 & \bf{576.48} & 
0.30\\CON3-8 & \bf{523.05} & 8.44 & 
523.61 & 10.36 & 523.05 & 0.00\\
CON3-9 & 588.40 & 10.12 & 
588.40 & 10.28 & \bf{580.05} & 
1.44\\CON8-0 & 871.19 & 23.04 & 
874.23 & 26.51 & \bf{857.17} & 
1.64\\CON8-1 & \bf{740.85} & 40.91 & 
748.19 & 40.20 & 740.85 & 0.00\\
CON8-2 & \bf{713.44} & 60.11 & 
717.76 & 44.61 & 713.44 & 0.00\\
CON8-3 & 814.52 & 40.39 & 
820.00 & 37.47 & \bf{811.07} & 
0.43\\CON8-4 & 783.47 & 45.34 & 
785.87 & 49.26 & \bf{772.25} & 
1.45\\CON8-5 & \bf{\underline{754.95}} & 32.54 & 
761.14 & 36.01 & 756.91 & 
-0.26\\CON8-6 & 687.70 & 28.64 & 
696.55 & 33.25 & \bf{678.92} & 
1.29\\CON8-7 & 815.43 & 41.46 & 
818.21 & 45.11 & \bf{811.96} & 
0.43\\CON8-8 & 783.65 & 36.53 & 
789.99 & 29.30 & \bf{767.53} & 
2.10\\CON8-9 & 814.77 & 36.81 & 
821.26 & 28.16 & \bf{809.00} & 
0.71\\[1ex]\hline
\end{tabular}
\label{table:nonlin}
\end{table} \clearpage
\begin{table}[ht]
\caption{Resultados de la ejecución de la metaheurística SCA, utilizando instancias de SalhiNagy con la configuración -n 75.0 -b 10 -y .2}
\centering
\small
\begin{tabular}{c c c c c c c}
\hline\hline
Instancia & Costo mínimo & Tiempo(seg.) & Costo promedio & Tiempo promedio(seg.) & Costo SCA & \%Gap \\ [0.5ex]
\hline
[1ex]\hline
\end{tabular}
\label{table:nonlin}
\end{table} \clearpage
\begin{table}[ht]
\caption{Resultados de la ejecución de la metaheurística SCA, utilizando instancias de Dethloff con la configuración -n 75.0 -b 10 -y .3}
\centering
\small
\begin{tabular}{c c c c c c c}
\hline\hline
Instancia & Costo mínimo & Tiempo(seg.) & Costo promedio & Tiempo promedio(seg.) & Costo SCA & \%Gap \\ [0.5ex]
\hline
SCA3-0 & 640.55 & 12.42 & 
640.55 & 13.77 & \bf{636.06} & 
0.71\\SCA3-1 & \bf{697.84} & 8.77 & 
698.85 & 10.55 & 697.84 & 0.00\\
SCA3-2 & 664.21 & 14.66 & 
664.84 & 18.33 & \bf{659.34} & 
0.74\\SCA3-3 & \bf{680.04} & 10.94 & 
680.47 & 12.13 & 680.04 & 0.00\\
SCA3-4 & \bf{690.50} & 14.60 & 
692.05 & 12.34 & 690.50 & 0.00\\
SCA3-5 & 662.75 & 12.14 & 
662.75 & 14.40 & \bf{659.90} & 
0.43\\SCA3-6 & 652.94 & 10.84 & 
653.16 & 14.76 & \bf{651.09} & 
0.28\\SCA3-7 & 666.15 & 11.18 & 
667.64 & 15.70 & \bf{659.17} & 
1.06\\SCA3-8 & \bf{719.47} & 9.87 & 
720.19 & 10.64 & 719.47 & 0.00\\
SCA3-9 & \bf{681.00} & 15.34 & 
681.06 & 12.97 & 681.00 & 0.00\\
SCA8-0 & 982.18 & 30.32 & 
991.33 & 36.02 & \bf{961.50} & 
2.15\\SCA8-1 & 1056.87 & 24.77 & 
1063.27 & 34.19 & \bf{1050.20} & 
0.64\\SCA8-2 & 1051.95 & 51.33 & 
1053.40 & 45.52 & \bf{1039.64} & 
1.18\\SCA8-3 & 1022.69 & 41.59 & 
1028.02 & 38.19 & \bf{983.34} & 
4.00\\SCA8-4 & 1068.27 & 28.76 & 
1071.05 & 29.79 & \bf{1065.49} & 
0.26\\SCA8-5 & 1050.44 & 36.53 & 
1054.51 & 40.23 & \bf{1027.08} & 
2.27\\SCA8-6 & 972.48 & 45.42 & 
978.57 & 60.34 & \bf{971.82} & 
0.07\\SCA8-7 & 1063.22 & 40.55 & 
1069.19 & 35.28 & \bf{1052.17} & 
1.05\\SCA8-8 & 1088.20 & 45.55 & 
1089.62 & 43.08 & \bf{1071.18} & 
1.59\\SCA8-9 & 1070.71 & 52.17 & 
1083.38 & 34.66 & \bf{1060.50} & 
0.96\\CON3-0 & 619.09 & 4.46 & 
619.09 & 5.87 & \bf{616.52} & 
0.42\\CON3-1 & 560.75 & 11.78 & 
561.19 & 12.30 & \bf{554.47} & 
1.13\\CON3-2 & 521.38 & 22.92 & 
521.38 & 12.36 & \bf{519.26} & 
0.41\\CON3-3 & \bf{591.19} & 12.76 & 
591.20 & 16.69 & 591.19 & 0.00\\
CON3-4 & 591.43 & 15.60 & 
591.43 & 14.57 & \bf{589.32} & 
0.36\\CON3-5 & 564.88 & 11.72 & 
564.88 & 11.25 & \bf{563.70} & 
0.21\\CON3-6 & 502.16 & 8.78 & 
502.16 & 14.63 & \bf{500.80} & 
0.27\\CON3-7 & 578.41 & 15.65 & 
582.06 & 15.22 & \bf{576.48} & 
0.33\\CON3-8 & \bf{523.05} & 13.63 & 
523.67 & 9.22 & 523.05 & 0.00\\
CON3-9 & 581.06 & 15.12 & 
586.83 & 18.84 & \bf{580.05} & 
0.17\\CON8-0 & 873.17 & 20.55 & 
880.85 & 30.33 & \bf{857.17} & 
1.87\\CON8-1 & 741.70 & 23.63 & 
746.65 & 38.21 & \bf{740.85} & 
0.11\\CON8-2 & \bf{713.44} & 43.87 & 
715.53 & 42.67 & 713.44 & 0.00\\
CON8-3 & 824.71 & 30.19 & 
830.92 & 35.50 & \bf{811.07} & 
1.68\\CON8-4 & 778.31 & 52.74 & 
787.34 & 39.36 & \bf{772.25} & 
0.78\\CON8-5 & 758.12 & 32.79 & 
762.43 & 40.66 & \bf{756.91} & 
0.16\\CON8-6 & 693.83 & 42.12 & 
697.82 & 44.56 & \bf{678.92} & 
2.20\\CON8-7 & 815.43 & 31.30 & 
816.37 & 42.66 & \bf{811.96} & 
0.43\\CON8-8 & 775.56 & 56.99 & 
785.34 & 35.65 & \bf{767.53} & 
1.05\\CON8-9 & 819.78 & 27.88 & 
822.67 & 31.37 & \bf{809.00} & 
1.33\\[1ex]\hline
\end{tabular}
\label{table:nonlin}
\end{table} \clearpage
\begin{table}[ht]
\caption{Resultados de la ejecución de la metaheurística SCA, utilizando instancias de SalhiNagy con la configuración -n 75.0 -b 10 -y .3}
\centering
\small
\begin{tabular}{c c c c c c c}
\hline\hline
Instancia & Costo mínimo & Tiempo(seg.) & Costo promedio & Tiempo promedio(seg.) & Costo SCA & \%Gap \\ [0.5ex]
\hline
[1ex]\hline
\end{tabular}
\label{table:nonlin}
\end{table} \clearpage
\begin{table}[ht]
\caption{Resultados de la ejecución de la metaheurística SCA, utilizando instancias de Dethloff con la configuración -n 75.0 -b 10 -y .4}
\centering
\small
\begin{tabular}{c c c c c c c}
\hline\hline
Instancia & Costo mínimo & Tiempo(seg.) & Costo promedio & Tiempo promedio(seg.) & Costo SCA & \%Gap \\ [0.5ex]
\hline
SCA3-0 & 640.55 & 18.56 & 
640.55 & 14.73 & \bf{636.06} & 
0.71\\SCA3-1 & \bf{697.84} & 11.92 & 
699.68 & 12.86 & 697.84 & 0.00\\
SCA3-2 & 661.13 & 13.41 & 
662.08 & 16.29 & \bf{659.34} & 
0.27\\SCA3-3 & \bf{680.04} & 17.22 & 
680.61 & 16.59 & 680.04 & 0.00\\
SCA3-4 & \bf{690.50} & 10.24 & 
690.50 & 11.41 & 690.50 & 0.00\\
SCA3-5 & 665.04 & 9.79 & 
670.50 & 9.04 & \bf{659.90} & 
0.78\\SCA3-6 & \bf{651.09} & 9.19 & 
651.55 & 13.76 & 651.09 & 0.00\\
SCA3-7 & 666.60 & 13.28 & 
668.69 & 12.93 & \bf{659.17} & 
1.13\\SCA3-8 & \bf{719.47} & 13.56 & 
719.47 & 13.81 & 719.47 & 0.00\\
SCA3-9 & \bf{681.00} & 12.36 & 
681.00 & 10.86 & 681.00 & 0.00\\
SCA8-0 & 982.58 & 42.73 & 
989.09 & 42.96 & \bf{961.50} & 
2.19\\SCA8-1 & 1057.18 & 31.28 & 
1060.64 & 36.30 & \bf{1050.20} & 
0.66\\SCA8-2 & 1050.17 & 74.92 & 
1052.88 & 52.51 & \bf{1039.64} & 
1.01\\SCA8-3 & 1014.18 & 32.03 & 
1019.44 & 37.43 & \bf{983.34} & 
3.14\\SCA8-4 & 1067.82 & 35.25 & 
1075.37 & 37.37 & \bf{1065.49} & 
0.22\\SCA8-5 & 1036.97 & 35.87 & 
1048.13 & 37.91 & \bf{1027.08} & 
0.96\\SCA8-6 & 977.87 & 62.27 & 
981.97 & 42.42 & \bf{971.82} & 
0.62\\SCA8-7 & 1063.22 & 30.42 & 
1073.81 & 34.35 & \bf{1052.17} & 
1.05\\SCA8-8 & 1085.22 & 32.87 & 
1091.83 & 32.28 & \bf{1071.18} & 
1.31\\SCA8-9 & 1067.42 & 27.43 & 
1080.05 & 41.54 & \bf{1060.50} & 
0.65\\CON3-0 & 619.09 & 8.30 & 
625.82 & 7.59 & \bf{616.52} & 
0.42\\CON3-1 & 556.79 & 8.99 & 
559.76 & 12.53 & \bf{554.47} & 
0.42\\CON3-2 & 521.38 & 9.57 & 
521.38 & 10.32 & \bf{519.26} & 
0.41\\CON3-3 & \bf{591.19} & 10.05 & 
591.20 & 10.59 & 591.19 & 0.00\\
CON3-4 & 589.88 & 14.60 & 
590.65 & 13.91 & \bf{589.32} & 
0.10\\CON3-5 & \bf{563.70} & 11.61 & 
564.59 & 9.06 & 563.70 & 0.00\\
CON3-6 & 502.16 & 10.04 & 
502.16 & 11.55 & \bf{500.80} & 
0.27\\CON3-7 & 578.22 & 18.66 & 
582.12 & 16.52 & \bf{576.48} & 
0.30\\CON3-8 & \bf{523.05} & 9.86 & 
523.12 & 11.08 & 523.05 & 0.00\\
CON3-9 & 582.79 & 14.06 & 
586.70 & 12.49 & \bf{580.05} & 
0.47\\CON8-0 & 872.20 & 21.89 & 
877.76 & 25.58 & \bf{857.17} & 
1.75\\CON8-1 & 742.47 & 50.56 & 
745.74 & 44.99 & \bf{740.85} & 
0.22\\CON8-2 & 713.60 & 29.84 & 
714.89 & 34.27 & \bf{713.44} & 
0.02\\CON8-3 & 828.01 & 47.69 & 
833.01 & 48.49 & \bf{811.07} & 
2.09\\CON8-4 & 777.64 & 49.64 & 
785.77 & 40.74 & \bf{772.25} & 
0.70\\CON8-5 & \bf{\underline{754.95}} & 41.73 & 
762.13 & 42.55 & 756.91 & 
-0.26\\CON8-6 & 684.29 & 29.80 & 
694.24 & 30.34 & \bf{678.92} & 
0.79\\CON8-7 & 814.77 & 45.60 & 
815.20 & 41.87 & \bf{811.96} & 
0.35\\CON8-8 & 779.69 & 26.21 & 
783.59 & 31.50 & \bf{767.53} & 
1.58\\CON8-9 & 813.57 & 28.50 & 
822.41 & 41.56 & \bf{809.00} & 
0.56\\[1ex]\hline
\end{tabular}
\label{table:nonlin}
\end{table} \clearpage
\begin{table}[ht]
\caption{Resultados de la ejecución de la metaheurística SCA, utilizando instancias de SalhiNagy con la configuración -n 75.0 -b 10 -y .4}
\centering
\small
\begin{tabular}{c c c c c c c}
\hline\hline
Instancia & Costo mínimo & Tiempo(seg.) & Costo promedio & Tiempo promedio(seg.) & Costo SCA & \%Gap \\ [0.5ex]
\hline
[1ex]\hline
\end{tabular}
\label{table:nonlin}
\end{table} \clearpage
\begin{table}[ht]
\caption{Resultados de la ejecución de la metaheurística SCA, utilizando instancias de Dethloff con la configuración -n 75.0 -b 10 -y .5}
\centering
\small
\begin{tabular}{c c c c c c c}
\hline\hline
Instancia & Costo mínimo & Tiempo(seg.) & Costo promedio & Tiempo promedio(seg.) & Costo SCA & \%Gap \\ [0.5ex]
\hline
SCA3-0 & 640.55 & 14.05 & 
640.55 & 14.27 & \bf{636.06} & 
0.71\\SCA3-1 & \bf{697.84} & 13.37 & 
698.76 & 13.53 & 697.84 & 0.00\\
SCA3-2 & 661.13 & 13.00 & 
661.90 & 15.36 & \bf{659.34} & 
0.27\\SCA3-3 & 680.60 & 11.42 & 
680.98 & 9.19 & \bf{680.04} & 
0.08\\SCA3-4 & \bf{690.50} & 8.11 & 
691.18 & 15.18 & 690.50 & 0.00\\
SCA3-5 & 665.64 & 9.05 & 
668.92 & 13.54 & \bf{659.90} & 
0.87\\SCA3-6 & 652.94 & 11.66 & 
655.04 & 16.75 & \bf{651.09} & 
0.28\\SCA3-7 & 666.60 & 11.76 & 
670.40 & 10.56 & \bf{659.17} & 
1.13\\SCA3-8 & \bf{719.47} & 17.22 & 
719.47 & 14.33 & 719.47 & 0.00\\
SCA3-9 & 685.00 & 17.96 & 
685.29 & 15.69 & \bf{681.00} & 
0.59\\SCA8-0 & 980.51 & 35.83 & 
981.68 & 35.35 & \bf{961.50} & 
1.98\\SCA8-1 & 1060.62 & 48.94 & 
1068.86 & 45.97 & \bf{1050.20} & 
0.99\\SCA8-2 & 1053.90 & 29.24 & 
1053.93 & 35.64 & \bf{1039.64} & 
1.37\\SCA8-3 & 1011.28 & 35.01 & 
1018.67 & 32.16 & \bf{983.34} & 
2.84\\SCA8-4 & 1074.19 & 34.41 & 
1084.19 & 31.81 & \bf{1065.49} & 
0.82\\SCA8-5 & 1049.44 & 34.27 & 
1051.96 & 39.12 & \bf{1027.08} & 
2.18\\SCA8-6 & 979.28 & 70.33 & 
984.75 & 47.79 & \bf{971.82} & 
0.77\\SCA8-7 & 1070.53 & 25.05 & 
1073.34 & 40.33 & \bf{1052.17} & 
1.74\\SCA8-8 & 1085.34 & 50.66 & 
1089.00 & 41.47 & \bf{1071.18} & 
1.32\\SCA8-9 & 1073.62 & 45.85 & 
1078.95 & 38.58 & \bf{1060.50} & 
1.24\\CON3-0 & 620.76 & 8.12 & 
622.30 & 8.03 & \bf{616.52} & 
0.69\\CON3-1 & 558.09 & 11.04 & 
560.09 & 13.56 & \bf{554.47} & 
0.65\\CON3-2 & 521.38 & 10.18 & 
521.38 & 11.86 & \bf{519.26} & 
0.41\\CON3-3 & 591.20 & 17.41 & 
591.20 & 15.95 & \bf{591.19} & 
0.00\\CON3-4 & 591.43 & 16.96 & 
591.43 & 15.52 & \bf{589.32} & 
0.36\\CON3-5 & \bf{563.70} & 12.25 & 
564.76 & 11.32 & 563.70 & 0.00\\
CON3-6 & 502.16 & 8.54 & 
502.16 & 9.64 & \bf{500.80} & 
0.27\\CON3-7 & 577.54 & 13.68 & 
581.48 & 16.44 & \bf{576.48} & 
0.18\\CON3-8 & 523.19 & 13.02 & 
524.98 & 15.26 & \bf{523.05} & 
0.03\\CON3-9 & 588.38 & 10.66 & 
588.39 & 10.47 & \bf{580.05} & 
1.44\\CON8-0 & 871.77 & 47.47 & 
880.84 & 37.56 & \bf{857.17} & 
1.70\\CON8-1 & 742.47 & 41.69 & 
744.07 & 53.23 & \bf{740.85} & 
0.22\\CON8-2 & \bf{\underline{712.89}} & 54.77 & 
713.38 & 42.53 & 713.44 & 
-0.08\\CON8-3 & 826.08 & 37.31 & 
833.29 & 36.97 & \bf{811.07} & 
1.85\\CON8-4 & 780.46 & 26.08 & 
785.52 & 42.84 & \bf{772.25} & 
1.06\\CON8-5 & \bf{\underline{754.95}} & 34.51 & 
757.44 & 40.90 & 756.91 & 
-0.26\\CON8-6 & 694.76 & 44.92 & 
698.83 & 42.85 & \bf{678.92} & 
2.33\\CON8-7 & 814.79 & 47.96 & 
815.45 & 42.77 & \bf{811.96} & 
0.35\\CON8-8 & 785.30 & 21.08 & 
788.46 & 27.19 & \bf{767.53} & 
2.32\\CON8-9 & 822.42 & 35.63 & 
826.25 & 37.50 & \bf{809.00} & 
1.66\\[1ex]\hline
\end{tabular}
\label{table:nonlin}
\end{table} \clearpage
\begin{table}[ht]
\caption{Resultados de la ejecución de la metaheurística SCA, utilizando instancias de SalhiNagy con la configuración -n 75.0 -b 10 -y .5}
\centering
\small
\begin{tabular}{c c c c c c c}
\hline\hline
Instancia & Costo mínimo & Tiempo(seg.) & Costo promedio & Tiempo promedio(seg.) & Costo SCA & \%Gap \\ [0.5ex]
\hline
[1ex]\hline
\end{tabular}
\label{table:nonlin}
\end{table} \clearpage
\begin{table}[ht]
\caption{Resultados de la ejecución de la metaheurística SCA, utilizando instancias de Dethloff con la configuración -n 100.0 -b 10 -y 0.1}
\centering
\small
\begin{tabular}{c c c c c c c}
\hline\hline
Instancia & Costo mínimo & Tiempo(seg.) & Costo promedio & Tiempo promedio(seg.) & Costo SCA & \%Gap \\ [0.5ex]
\hline
SCA3-0 & 640.55 & 13.06 & 
640.55 & 16.09 & \bf{636.06} & 
0.71\\SCA3-1 & 700.50 & 5.45 & 
700.50 & 6.45 & \bf{697.84} & 
0.38\\SCA3-2 & 664.18 & 10.43 & 
665.33 & 14.60 & \bf{659.34} & 
0.73\\SCA3-3 & 680.60 & 12.77 & 
680.60 & 10.29 & \bf{680.04} & 
0.08\\SCA3-4 & \bf{690.50} & 14.16 & 
690.50 & 15.05 & 690.50 & 0.00\\
SCA3-5 & 668.48 & 9.26 & 
668.59 & 13.33 & \bf{659.90} & 
1.30\\SCA3-6 & \bf{651.09} & 11.48 & 
651.09 & 10.54 & 651.09 & 0.00\\
SCA3-7 & 666.15 & 7.74 & 
666.15 & 9.77 & \bf{659.17} & 
1.06\\SCA3-8 & \bf{719.47} & 11.94 & 
719.47 & 10.95 & 719.47 & 0.00\\
SCA3-9 & \bf{681.00} & 15.59 & 
683.69 & 13.52 & 681.00 & 0.00\\
SCA8-0 & 982.58 & 39.81 & 
986.35 & 29.21 & \bf{961.50} & 
2.19\\SCA8-1 & \bf{1050.20} & 30.27 & 
1065.41 & 32.12 & 1050.20 & 0.00\\
SCA8-2 & 1050.37 & 56.93 & 
1052.74 & 45.61 & \bf{1039.64} & 
1.03\\SCA8-3 & 1010.82 & 27.93 & 
1011.69 & 31.54 & \bf{983.34} & 
2.79\\SCA8-4 & 1068.48 & 30.33 & 
1071.97 & 32.77 & \bf{1065.49} & 
0.28\\SCA8-5 & 1043.05 & 67.36 & 
1048.34 & 45.79 & \bf{1027.08} & 
1.55\\SCA8-6 & 972.48 & 29.57 & 
977.22 & 29.23 & \bf{971.82} & 
0.07\\SCA8-7 & 1070.53 & 25.17 & 
1073.87 & 27.12 & \bf{1052.17} & 
1.74\\SCA8-8 & \bf{1071.18} & 54.43 & 
1081.34 & 48.02 & 1071.18 & 0.00\\
SCA8-9 & 1067.42 & 22.67 & 
1073.62 & 31.48 & \bf{1060.50} & 
0.65\\CON3-0 & 617.59 & 9.40 & 
621.93 & 12.48 & \bf{616.52} & 
0.17\\CON3-1 & 560.75 & 14.13 & 
560.75 & 14.48 & \bf{554.47} & 
1.13\\CON3-2 & 521.38 & 24.20 & 
521.38 & 15.36 & \bf{519.26} & 
0.41\\CON3-3 & 591.20 & 11.29 & 
591.20 & 12.21 & \bf{591.19} & 
0.00\\CON3-4 & 591.43 & 9.10 & 
591.43 & 12.57 & \bf{589.32} & 
0.36\\CON3-5 & 564.89 & 6.21 & 
564.89 & 6.53 & \bf{563.70} & 
0.21\\CON3-6 & 502.16 & 11.02 & 
502.16 & 11.16 & \bf{500.80} & 
0.27\\CON3-7 & 578.79 & 12.01 & 
582.25 & 11.71 & \bf{576.48} & 
0.40\\CON3-8 & \bf{523.05} & 8.70 & 
524.32 & 11.88 & 523.05 & 0.00\\
CON3-9 & 588.40 & 7.77 & 
588.40 & 11.25 & \bf{580.05} & 
1.44\\CON8-0 & 863.53 & 29.00 & 
874.45 & 32.90 & \bf{857.17} & 
0.74\\CON8-1 & 742.47 & 43.57 & 
748.43 & 34.79 & \bf{740.85} & 
0.22\\CON8-2 & \bf{\underline{713.05}} & 45.22 & 
715.81 & 48.02 & 713.44 & 
-0.05\\CON8-3 & 826.06 & 44.00 & 
828.76 & 37.33 & \bf{811.07} & 
1.85\\CON8-4 & 776.72 & 26.60 & 
783.43 & 29.30 & \bf{772.25} & 
0.58\\CON8-5 & \bf{\underline{754.95}} & 36.88 & 
760.23 & 35.12 & 756.91 & 
-0.26\\CON8-6 & 688.00 & 24.66 & 
689.99 & 27.53 & \bf{678.92} & 
1.34\\CON8-7 & 814.77 & 30.82 & 
815.18 & 43.87 & \bf{811.96} & 
0.35\\CON8-8 & 784.82 & 23.84 & 
789.49 & 31.65 & \bf{767.53} & 
2.25\\CON8-9 & 816.07 & 33.90 & 
821.27 & 29.36 & \bf{809.00} & 
0.87\\[1ex]\hline
\end{tabular}
\label{table:nonlin}
\end{table} \clearpage
\begin{table}[ht]
\caption{Resultados de la ejecución de la metaheurística SCA, utilizando instancias de SalhiNagy con la configuración -n 100.0 -b 10 -y 0.1}
\centering
\small
\begin{tabular}{c c c c c c c}
\hline\hline
Instancia & Costo mínimo & Tiempo(seg.) & Costo promedio & Tiempo promedio(seg.) & Costo SCA & \%Gap \\ [0.5ex]
\hline
[1ex]\hline
\end{tabular}
\label{table:nonlin}
\end{table} \clearpage
\begin{table}[ht]
\caption{Resultados de la ejecución de la metaheurística SCA, utilizando instancias de Dethloff con la configuración -n 100.0 -b 10 -y .2}
\centering
\small
\begin{tabular}{c c c c c c c}
\hline\hline
Instancia & Costo mínimo & Tiempo(seg.) & Costo promedio & Tiempo promedio(seg.) & Costo SCA & \%Gap \\ [0.5ex]
\hline
SCA3-0 & 640.55 & 8.18 & 
640.84 & 11.89 & \bf{636.06} & 
0.71\\SCA3-1 & \bf{697.84} & 14.91 & 
699.43 & 15.33 & 697.84 & 0.00\\
SCA3-2 & 666.01 & 14.93 & 
668.20 & 14.91 & \bf{659.34} & 
1.01\\SCA3-3 & 680.60 & 16.77 & 
681.13 & 12.38 & \bf{680.04} & 
0.08\\SCA3-4 & 692.57 & 14.46 & 
692.57 & 14.64 & \bf{690.50} & 
0.30\\SCA3-5 & 665.04 & 12.50 & 
669.64 & 12.27 & \bf{659.90} & 
0.78\\SCA3-6 & 653.81 & 9.20 & 
654.43 & 13.88 & \bf{651.09} & 
0.42\\SCA3-7 & 666.60 & 6.54 & 
668.25 & 7.64 & \bf{659.17} & 
1.13\\SCA3-8 & \bf{719.47} & 12.14 & 
719.54 & 15.48 & 719.47 & 0.00\\
SCA3-9 & \bf{681.00} & 9.98 & 
681.00 & 10.80 & 681.00 & 0.00\\
SCA8-0 & 987.33 & 33.62 & 
998.16 & 26.16 & \bf{961.50} & 
2.69\\SCA8-1 & 1063.58 & 33.33 & 
1073.78 & 40.24 & \bf{1050.20} & 
1.27\\SCA8-2 & 1050.37 & 32.75 & 
1051.45 & 34.45 & \bf{1039.64} & 
1.03\\SCA8-3 & 995.60 & 22.20 & 
1014.75 & 28.63 & \bf{983.34} & 
1.25\\SCA8-4 & 1074.91 & 22.44 & 
1078.71 & 26.56 & \bf{1065.49} & 
0.88\\SCA8-5 & 1049.44 & 40.25 & 
1052.34 & 41.43 & \bf{1027.08} & 
2.18\\SCA8-6 & 972.48 & 53.54 & 
974.88 & 42.30 & \bf{971.82} & 
0.07\\SCA8-7 & 1063.22 & 37.91 & 
1068.95 & 27.86 & \bf{1052.17} & 
1.05\\SCA8-8 & 1082.12 & 28.13 & 
1089.04 & 29.50 & \bf{1071.18} & 
1.02\\SCA8-9 & 1072.91 & 62.32 & 
1075.61 & 43.10 & \bf{1060.50} & 
1.17\\CON3-0 & 617.98 & 7.33 & 
623.23 & 8.74 & \bf{616.52} & 
0.24\\CON3-1 & 560.75 & 8.78 & 
560.75 & 10.81 & \bf{554.47} & 
1.13\\CON3-2 & 521.38 & 14.84 & 
522.68 & 12.32 & \bf{519.26} & 
0.41\\CON3-3 & \bf{591.19} & 11.90 & 
591.19 & 16.61 & 591.19 & 0.00\\
CON3-4 & 591.43 & 7.65 & 
591.43 & 10.46 & \bf{589.32} & 
0.36\\CON3-5 & \bf{563.70} & 8.47 & 
563.70 & 9.38 & 563.70 & 0.00\\
CON3-6 & 501.34 & 13.67 & 
503.44 & 9.80 & \bf{500.80} & 
0.11\\CON3-7 & 578.22 & 18.94 & 
581.14 & 14.30 & \bf{576.48} & 
0.30\\CON3-8 & 523.68 & 9.03 & 
523.68 & 9.49 & \bf{523.05} & 
0.12\\CON3-9 & 588.40 & 8.76 & 
588.96 & 9.03 & \bf{580.05} & 
1.44\\CON8-0 & 871.84 & 23.38 & 
879.49 & 33.48 & \bf{857.17} & 
1.71\\CON8-1 & 742.47 & 46.22 & 
743.95 & 41.46 & \bf{740.85} & 
0.22\\CON8-2 & 714.06 & 25.69 & 
715.66 & 29.50 & \bf{713.44} & 
0.09\\CON8-3 & 823.24 & 55.48 & 
829.10 & 40.83 & \bf{811.07} & 
1.50\\CON8-4 & 777.81 & 24.05 & 
778.93 & 31.83 & \bf{772.25} & 
0.72\\CON8-5 & \bf{\underline{755.86}} & 35.93 & 
762.37 & 41.38 & 756.91 & 
-0.14\\CON8-6 & 681.11 & 24.54 & 
688.87 & 26.36 & \bf{678.92} & 
0.32\\CON8-7 & 814.86 & 37.35 & 
820.43 & 42.32 & \bf{811.96} & 
0.36\\CON8-8 & 784.31 & 42.73 & 
789.22 & 40.16 & \bf{767.53} & 
2.19\\CON8-9 & 810.18 & 62.18 & 
818.71 & 43.20 & \bf{809.00} & 
0.15\\[1ex]\hline
\end{tabular}
\label{table:nonlin}
\end{table} \clearpage
\begin{table}[ht]
\caption{Resultados de la ejecución de la metaheurística SCA, utilizando instancias de SalhiNagy con la configuración -n 100.0 -b 10 -y .2}
\centering
\small
\begin{tabular}{c c c c c c c}
\hline\hline
Instancia & Costo mínimo & Tiempo(seg.) & Costo promedio & Tiempo promedio(seg.) & Costo SCA & \%Gap \\ [0.5ex]
\hline
[1ex]\hline
\end{tabular}
\label{table:nonlin}
\end{table} \clearpage
\begin{table}[ht]
\caption{Resultados de la ejecución de la metaheurística SCA, utilizando instancias de Dethloff con la configuración -n 100.0 -b 10 -y .3}
\centering
\small
\begin{tabular}{c c c c c c c}
\hline\hline
Instancia & Costo mínimo & Tiempo(seg.) & Costo promedio & Tiempo promedio(seg.) & Costo SCA & \%Gap \\ [0.5ex]
\hline
SCA3-0 & 640.55 & 13.34 & 
640.55 & 11.45 & \bf{636.06} & 
0.71\\SCA3-1 & \bf{697.84} & 9.84 & 
700.09 & 12.13 & 697.84 & 0.00\\
SCA3-2 & \bf{659.34} & 16.50 & 
662.99 & 15.65 & 659.34 & 0.00\\
SCA3-3 & \bf{680.04} & 21.05 & 
680.60 & 13.45 & 680.04 & 0.00\\
SCA3-4 & \bf{690.50} & 21.74 & 
691.02 & 17.75 & 690.50 & 0.00\\
SCA3-5 & 672.49 & 5.90 & 
672.49 & 8.46 & \bf{659.90} & 
1.91\\SCA3-6 & 652.94 & 11.64 & 
653.76 & 11.64 & \bf{651.09} & 
0.28\\SCA3-7 & 666.15 & 11.27 & 
666.81 & 8.81 & \bf{659.17} & 
1.06\\SCA3-8 & 719.77 & 14.14 & 
719.77 & 13.95 & \bf{719.47} & 
0.04\\SCA3-9 & \bf{681.00} & 11.34 & 
682.03 & 11.81 & 681.00 & 0.00\\
SCA8-0 & 984.75 & 27.91 & 
987.61 & 26.94 & \bf{961.50} & 
2.42\\SCA8-1 & 1050.93 & 44.93 & 
1072.99 & 46.67 & \bf{1050.20} & 
0.07\\SCA8-2 & 1051.42 & 45.65 & 
1051.59 & 39.98 & \bf{1039.64} & 
1.13\\SCA8-3 & \bf{983.34} & 24.59 & 
1008.88 & 32.21 & 983.34 & 0.00\\
SCA8-4 & 1075.55 & 26.42 & 
1079.24 & 29.59 & \bf{1065.49} & 
0.94\\SCA8-5 & 1049.44 & 47.48 & 
1053.05 & 34.48 & \bf{1027.08} & 
2.18\\SCA8-6 & 972.48 & 51.58 & 
976.85 & 54.85 & \bf{971.82} & 
0.07\\SCA8-7 & 1067.11 & 49.53 & 
1073.33 & 39.05 & \bf{1052.17} & 
1.42\\SCA8-8 & 1080.58 & 27.95 & 
1086.11 & 42.12 & \bf{1071.18} & 
0.88\\SCA8-9 & 1080.96 & 62.27 & 
1090.32 & 43.46 & \bf{1060.50} & 
1.93\\CON3-0 & 620.76 & 8.97 & 
625.78 & 9.10 & \bf{616.52} & 
0.69\\CON3-1 & \bf{554.47} & 10.48 & 
557.50 & 11.06 & 554.47 & 0.00\\
CON3-2 & 521.38 & 13.73 & 
521.38 & 10.84 & \bf{519.26} & 
0.41\\CON3-3 & 591.20 & 22.66 & 
591.20 & 17.64 & \bf{591.19} & 
0.00\\CON3-4 & 591.43 & 20.94 & 
591.43 & 13.91 & \bf{589.32} & 
0.36\\CON3-5 & 564.89 & 19.25 & 
567.02 & 14.35 & \bf{563.70} & 
0.21\\CON3-6 & 502.16 & 12.53 & 
502.16 & 11.94 & \bf{500.80} & 
0.27\\CON3-7 & 578.41 & 12.68 & 
582.61 & 13.98 & \bf{576.48} & 
0.33\\CON3-8 & 523.19 & 9.95 & 
523.19 & 12.63 & \bf{523.05} & 
0.03\\CON3-9 & 588.28 & 18.17 & 
588.37 & 12.15 & \bf{580.05} & 
1.42\\CON8-0 & 858.63 & 25.28 & 
871.73 & 31.45 & \bf{857.17} & 
0.17\\CON8-1 & 742.28 & 60.53 & 
749.00 & 41.74 & \bf{740.85} & 
0.19\\CON8-2 & \bf{\underline{713.05}} & 59.75 & 
714.94 & 40.22 & 713.44 & 
-0.05\\CON8-3 & 817.22 & 36.25 & 
826.28 & 48.29 & \bf{811.07} & 
0.76\\CON8-4 & 781.56 & 23.14 & 
784.98 & 32.85 & \bf{772.25} & 
1.21\\CON8-5 & \bf{\underline{754.95}} & 46.54 & 
757.36 & 40.77 & 756.91 & 
-0.26\\CON8-6 & 683.06 & 25.23 & 
684.25 & 32.04 & \bf{678.92} & 
0.61\\CON8-7 & 814.86 & 44.93 & 
816.37 & 39.62 & \bf{811.96} & 
0.36\\CON8-8 & 784.80 & 40.13 & 
788.26 & 39.24 & \bf{767.53} & 
2.25\\CON8-9 & 814.65 & 28.73 & 
817.07 & 34.49 & \bf{809.00} & 
0.70\\[1ex]\hline
\end{tabular}
\label{table:nonlin}
\end{table} \clearpage
\begin{table}[ht]
\caption{Resultados de la ejecución de la metaheurística SCA, utilizando instancias de SalhiNagy con la configuración -n 100.0 -b 10 -y .3}
\centering
\small
\begin{tabular}{c c c c c c c}
\hline\hline
Instancia & Costo mínimo & Tiempo(seg.) & Costo promedio & Tiempo promedio(seg.) & Costo SCA & \%Gap \\ [0.5ex]
\hline
[1ex]\hline
\end{tabular}
\label{table:nonlin}
\end{table} \clearpage
\begin{table}[ht]
\caption{Resultados de la ejecución de la metaheurística SCA, utilizando instancias de Dethloff con la configuración -n 100.0 -b 10 -y .4}
\centering
\small
\begin{tabular}{c c c c c c c}
\hline\hline
Instancia & Costo mínimo & Tiempo(seg.) & Costo promedio & Tiempo promedio(seg.) & Costo SCA & \%Gap \\ [0.5ex]
\hline
SCA3-0 & 640.55 & 27.58 & 
640.55 & 16.49 & \bf{636.06} & 
0.71\\SCA3-1 & \bf{697.84} & 10.33 & 
700.61 & 11.75 & 697.84 & 0.00\\
SCA3-2 & 661.13 & 9.42 & 
661.13 & 10.24 & \bf{659.34} & 
0.27\\SCA3-3 & 680.60 & 14.04 & 
680.60 & 17.00 & \bf{680.04} & 
0.08\\SCA3-4 & \bf{690.50} & 10.80 & 
692.55 & 12.56 & 690.50 & 0.00\\
SCA3-5 & 665.04 & 9.56 & 
668.70 & 15.19 & \bf{659.90} & 
0.78\\SCA3-6 & 652.94 & 19.57 & 
653.31 & 16.47 & \bf{651.09} & 
0.28\\SCA3-7 & 666.15 & 17.06 & 
666.81 & 12.59 & \bf{659.17} & 
1.06\\SCA3-8 & \bf{719.47} & 20.26 & 
719.47 & 17.04 & 719.47 & 0.00\\
SCA3-9 & \bf{681.00} & 16.97 & 
682.22 & 14.64 & 681.00 & 0.00\\
SCA8-0 & 970.64 & 41.61 & 
977.92 & 36.33 & \bf{961.50} & 
0.95\\SCA8-1 & 1060.19 & 34.70 & 
1074.67 & 29.33 & \bf{1050.20} & 
0.95\\SCA8-2 & 1050.37 & 48.84 & 
1051.92 & 41.01 & \bf{1039.64} & 
1.03\\SCA8-3 & 1014.19 & 26.84 & 
1022.00 & 31.87 & \bf{983.34} & 
3.14\\SCA8-4 & \bf{1065.49} & 44.25 & 
1084.76 & 37.67 & 1065.49 & 0.00\\
SCA8-5 & 1043.52 & 39.75 & 
1048.47 & 49.41 & \bf{1027.08} & 
1.60\\SCA8-6 & 977.87 & 33.38 & 
980.46 & 35.23 & \bf{971.82} & 
0.62\\SCA8-7 & 1067.67 & 30.14 & 
1071.49 & 33.22 & \bf{1052.17} & 
1.47\\SCA8-8 & 1088.65 & 34.27 & 
1089.65 & 46.78 & \bf{1071.18} & 
1.63\\SCA8-9 & 1069.70 & 28.10 & 
1075.93 & 33.94 & \bf{1060.50} & 
0.87\\CON3-0 & 620.76 & 8.22 & 
624.77 & 8.82 & \bf{616.52} & 
0.69\\CON3-1 & 560.75 & 11.29 & 
560.75 & 11.08 & \bf{554.47} & 
1.13\\CON3-2 & 521.38 & 10.48 & 
521.38 & 10.49 & \bf{519.26} & 
0.41\\CON3-3 & 591.20 & 8.13 & 
591.20 & 9.79 & \bf{591.19} & 
0.00\\CON3-4 & 591.43 & 16.15 & 
591.43 & 11.94 & \bf{589.32} & 
0.36\\CON3-5 & \bf{563.70} & 13.81 & 
563.70 & 9.99 & 563.70 & 0.00\\
CON3-6 & 502.16 & 16.77 & 
502.16 & 13.72 & \bf{500.80} & 
0.27\\CON3-7 & 578.41 & 17.58 & 
581.47 & 15.78 & \bf{576.48} & 
0.33\\CON3-8 & \bf{523.05} & 9.80 & 
523.28 & 10.48 & 523.05 & 0.00\\
CON3-9 & 588.40 & 7.80 & 
588.84 & 7.78 & \bf{580.05} & 
1.44\\CON8-0 & 869.68 & 49.25 & 
871.53 & 32.75 & \bf{857.17} & 
1.46\\CON8-1 & 742.47 & 74.39 & 
751.32 & 50.14 & \bf{740.85} & 
0.22\\CON8-2 & 716.56 & 28.55 & 
718.03 & 33.55 & \bf{713.44} & 
0.44\\CON8-3 & 828.85 & 30.56 & 
830.75 & 41.42 & \bf{811.07} & 
2.19\\CON8-4 & 781.83 & 25.33 & 
784.38 & 36.58 & \bf{772.25} & 
1.24\\CON8-5 & 762.61 & 39.74 & 
763.85 & 38.22 & \bf{756.91} & 
0.75\\CON8-6 & 695.85 & 36.59 & 
700.17 & 41.61 & \bf{678.92} & 
2.49\\CON8-7 & 815.72 & 27.32 & 
817.11 & 36.16 & \bf{811.96} & 
0.46\\CON8-8 & 776.55 & 40.61 & 
779.55 & 33.15 & \bf{767.53} & 
1.18\\CON8-9 & 812.25 & 45.99 & 
821.18 & 57.33 & \bf{809.00} & 
0.40\\[1ex]\hline
\end{tabular}
\label{table:nonlin}
\end{table} \clearpage
\begin{table}[ht]
\caption{Resultados de la ejecución de la metaheurística SCA, utilizando instancias de SalhiNagy con la configuración -n 100.0 -b 10 -y .4}
\centering
\small
\begin{tabular}{c c c c c c c}
\hline\hline
Instancia & Costo mínimo & Tiempo(seg.) & Costo promedio & Tiempo promedio(seg.) & Costo SCA & \%Gap \\ [0.5ex]
\hline
[1ex]\hline
\end{tabular}
\label{table:nonlin}
\end{table} \clearpage
\begin{table}[ht]
\caption{Resultados de la ejecución de la metaheurística SCA, utilizando instancias de Dethloff con la configuración -n 100.0 -b 10 -y .5}
\centering
\small
\begin{tabular}{c c c c c c c}
\hline\hline
Instancia & Costo mínimo & Tiempo(seg.) & Costo promedio & Tiempo promedio(seg.) & Costo SCA & \%Gap \\ [0.5ex]
\hline
SCA3-0 & 640.55 & 14.77 & 
640.55 & 15.20 & \bf{636.06} & 
0.71\\SCA3-1 & \bf{697.84} & 12.24 & 
698.76 & 14.47 & 697.84 & 0.00\\
SCA3-2 & 664.21 & 14.55 & 
665.56 & 18.03 & \bf{659.34} & 
0.74\\SCA3-3 & 680.60 & 12.98 & 
680.60 & 10.95 & \bf{680.04} & 
0.08\\SCA3-4 & \bf{690.50} & 23.41 & 
691.02 & 15.04 & 690.50 & 0.00\\
SCA3-5 & 673.56 & 10.50 & 
678.76 & 9.86 & \bf{659.90} & 
2.07\\SCA3-6 & 652.94 & 9.88 & 
653.93 & 13.16 & \bf{651.09} & 
0.28\\SCA3-7 & 666.60 & 9.51 & 
666.60 & 11.51 & \bf{659.17} & 
1.13\\SCA3-8 & \bf{719.47} & 12.81 & 
719.70 & 12.72 & 719.47 & 0.00\\
SCA3-9 & \bf{681.00} & 11.64 & 
682.03 & 16.19 & 681.00 & 0.00\\
SCA8-0 & 974.40 & 33.32 & 
986.23 & 37.33 & \bf{961.50} & 
1.34\\SCA8-1 & 1061.47 & 40.32 & 
1068.70 & 32.09 & \bf{1050.20} & 
1.07\\SCA8-2 & 1051.21 & 37.37 & 
1053.26 & 43.08 & \bf{1039.64} & 
1.11\\SCA8-3 & 1022.69 & 45.34 & 
1023.98 & 34.27 & \bf{983.34} & 
4.00\\SCA8-4 & 1074.87 & 29.39 & 
1076.86 & 33.31 & \bf{1065.49} & 
0.88\\SCA8-5 & 1049.44 & 60.21 & 
1049.44 & 48.94 & \bf{1027.08} & 
2.18\\SCA8-6 & 972.48 & 71.24 & 
979.43 & 65.90 & \bf{971.82} & 
0.07\\SCA8-7 & 1067.49 & 32.21 & 
1074.17 & 35.75 & \bf{1052.17} & 
1.46\\SCA8-8 & 1082.91 & 35.69 & 
1085.58 & 40.56 & \bf{1071.18} & 
1.10\\SCA8-9 & 1072.10 & 32.11 & 
1075.93 & 38.67 & \bf{1060.50} & 
1.09\\CON3-0 & 617.59 & 11.80 & 
623.78 & 9.70 & \bf{616.52} & 
0.17\\CON3-1 & 560.75 & 21.67 & 
560.75 & 15.77 & \bf{554.47} & 
1.13\\CON3-2 & 521.38 & 19.85 & 
521.70 & 15.02 & \bf{519.26} & 
0.41\\CON3-3 & \bf{591.19} & 12.84 & 
591.20 & 14.06 & 591.19 & 0.00\\
CON3-4 & \bf{\underline{588.79}} & 20.24 & 
590.38 & 14.21 & 589.32 & 
-0.09\\CON3-5 & \bf{563.70} & 8.46 & 
564.29 & 13.21 & 563.70 & 0.00\\
CON3-6 & 502.16 & 12.92 & 
502.16 & 12.81 & \bf{500.80} & 
0.27\\CON3-7 & 585.42 & 15.29 & 
585.86 & 11.80 & \bf{576.48} & 
1.55\\CON3-8 & \bf{523.05} & 11.16 & 
523.15 & 9.09 & 523.05 & 0.00\\
CON3-9 & 588.40 & 15.62 & 
588.40 & 10.80 & \bf{580.05} & 
1.44\\CON8-0 & 867.76 & 30.24 & 
876.05 & 26.85 & \bf{857.17} & 
1.24\\CON8-1 & 740.93 & 31.10 & 
745.58 & 40.13 & \bf{740.85} & 
0.01\\CON8-2 & \bf{\underline{712.89}} & 62.79 & 
719.77 & 42.40 & 713.44 & 
-0.08\\CON8-3 & 816.27 & 23.03 & 
822.21 & 31.84 & \bf{811.07} & 
0.64\\CON8-4 & 784.83 & 34.32 & 
786.07 & 34.55 & \bf{772.25} & 
1.63\\CON8-5 & 760.41 & 35.31 & 
768.68 & 43.35 & \bf{756.91} & 
0.46\\CON8-6 & 697.24 & 32.83 & 
699.91 & 40.99 & \bf{678.92} & 
2.70\\CON8-7 & 815.72 & 30.27 & 
819.00 & 30.58 & \bf{811.96} & 
0.46\\CON8-8 & 782.68 & 25.51 & 
786.85 & 35.79 & \bf{767.53} & 
1.97\\CON8-9 & 819.56 & 56.31 & 
824.24 & 36.94 & \bf{809.00} & 
1.31\\[1ex]\hline
\end{tabular}
\label{table:nonlin}
\end{table} \clearpage
\begin{table}[ht]
\caption{Resultados de la ejecución de la metaheurística SCA, utilizando instancias de SalhiNagy con la configuración -n 100.0 -b 10 -y .5}
\centering
\small
\begin{tabular}{c c c c c c c}
\hline\hline
Instancia & Costo mínimo & Tiempo(seg.) & Costo promedio & Tiempo promedio(seg.) & Costo SCA & \%Gap \\ [0.5ex]
\hline
[1ex]\hline
\end{tabular}
\label{table:nonlin}
\end{table} \clearpage
\begin{table}[ht]
\caption{Resultados de la ejecución de la metaheurística SCA, utilizando instancias de Dethloff con la configuración -n 125.0 -b 10 -y 0.1}
\centering
\small
\begin{tabular}{c c c c c c c}
\hline\hline
Instancia & Costo mínimo & Tiempo(seg.) & Costo promedio & Tiempo promedio(seg.) & Costo SCA & \%Gap \\ [0.5ex]
\hline
SCA3-0 & 640.55 & 14.30 & 
640.55 & 13.81 & \bf{636.06} & 
0.71\\SCA3-1 & \bf{697.84} & 13.98 & 
698.50 & 13.89 & 697.84 & 0.00\\
SCA3-2 & 661.13 & 27.50 & 
662.08 & 15.54 & \bf{659.34} & 
0.27\\SCA3-3 & 680.60 & 12.98 & 
680.60 & 10.59 & \bf{680.04} & 
0.08\\SCA3-4 & \bf{690.50} & 9.44 & 
692.05 & 9.99 & 690.50 & 0.00\\
SCA3-5 & 665.04 & 11.55 & 
665.04 & 11.61 & \bf{659.90} & 
0.78\\SCA3-6 & 653.81 & 11.69 & 
654.46 & 13.21 & \bf{651.09} & 
0.42\\SCA3-7 & 666.15 & 16.36 & 
667.75 & 17.64 & \bf{659.17} & 
1.06\\SCA3-8 & \bf{719.47} & 9.65 & 
719.54 & 11.61 & 719.47 & 0.00\\
SCA3-9 & 681.68 & 15.82 & 
683.42 & 14.42 & \bf{681.00} & 
0.10\\SCA8-0 & \bf{961.50} & 36.81 & 
977.13 & 29.41 & 961.50 & 0.00\\
SCA8-1 & 1062.88 & 30.80 & 
1068.81 & 42.00 & \bf{1050.20} & 
1.21\\SCA8-2 & 1050.37 & 31.33 & 
1051.85 & 39.39 & \bf{1039.64} & 
1.03\\SCA8-3 & 1017.03 & 40.11 & 
1020.26 & 31.48 & \bf{983.34} & 
3.43\\SCA8-4 & 1071.35 & 40.02 & 
1079.47 & 39.85 & \bf{1065.49} & 
0.55\\SCA8-5 & 1047.41 & 37.79 & 
1049.36 & 34.77 & \bf{1027.08} & 
1.98\\SCA8-6 & 972.48 & 63.51 & 
975.17 & 43.56 & \bf{971.82} & 
0.07\\SCA8-7 & 1067.03 & 28.63 & 
1070.31 & 41.23 & \bf{1052.17} & 
1.41\\SCA8-8 & \bf{1071.18} & 35.06 & 
1088.73 & 39.08 & 1071.18 & 0.00\\
SCA8-9 & 1070.71 & 31.81 & 
1082.07 & 40.69 & \bf{1060.50} & 
0.96\\CON3-0 & 624.84 & 8.66 & 
627.68 & 11.35 & \bf{616.52} & 
1.35\\CON3-1 & 560.75 & 15.95 & 
560.75 & 13.00 & \bf{554.47} & 
1.13\\CON3-2 & 521.38 & 10.18 & 
521.38 & 13.10 & \bf{519.26} & 
0.41\\CON3-3 & 591.20 & 13.43 & 
591.20 & 10.73 & \bf{591.19} & 
0.00\\CON3-4 & 591.43 & 14.43 & 
591.43 & 10.31 & \bf{589.32} & 
0.36\\CON3-5 & \bf{563.70} & 12.68 & 
564.29 & 11.30 & 563.70 & 0.00\\
CON3-6 & 502.16 & 12.16 & 
502.16 & 10.54 & \bf{500.80} & 
0.27\\CON3-7 & 585.42 & 13.43 & 
585.42 & 14.26 & \bf{576.48} & 
1.55\\CON3-8 & 523.14 & 8.48 & 
523.29 & 8.55 & \bf{523.05} & 
0.02\\CON3-9 & 588.40 & 12.20 & 
588.40 & 12.03 & \bf{580.05} & 
1.44\\CON8-0 & 874.79 & 25.49 & 
874.79 & 23.41 & \bf{857.17} & 
2.06\\CON8-1 & 748.85 & 40.54 & 
753.87 & 38.94 & \bf{740.85} & 
1.08\\CON8-2 & 713.60 & 69.54 & 
717.12 & 44.87 & \bf{713.44} & 
0.02\\CON8-3 & 822.65 & 40.17 & 
825.73 & 34.16 & \bf{811.07} & 
1.43\\CON8-4 & 785.86 & 38.50 & 
788.92 & 40.62 & \bf{772.25} & 
1.76\\CON8-5 & 759.82 & 39.80 & 
762.95 & 48.09 & \bf{756.91} & 
0.38\\CON8-6 & 692.81 & 23.10 & 
694.36 & 34.27 & \bf{678.92} & 
2.05\\CON8-7 & 814.50 & 50.23 & 
816.00 & 40.48 & \bf{811.96} & 
0.31\\CON8-8 & 783.13 & 40.45 & 
790.93 & 32.82 & \bf{767.53} & 
2.03\\CON8-9 & 821.42 & 32.55 & 
823.05 & 36.68 & \bf{809.00} & 
1.54\\[1ex]\hline
\end{tabular}
\label{table:nonlin}
\end{table} \clearpage
\begin{table}[ht]
\caption{Resultados de la ejecución de la metaheurística SCA, utilizando instancias de SalhiNagy con la configuración -n 125.0 -b 10 -y 0.1}
\centering
\small
\begin{tabular}{c c c c c c c}
\hline\hline
Instancia & Costo mínimo & Tiempo(seg.) & Costo promedio & Tiempo promedio(seg.) & Costo SCA & \%Gap \\ [0.5ex]
\hline
[1ex]\hline
\end{tabular}
\label{table:nonlin}
\end{table} \clearpage
\begin{table}[ht]
\caption{Resultados de la ejecución de la metaheurística SCA, utilizando instancias de Dethloff con la configuración -n 125.0 -b 10 -y .2}
\centering
\small
\begin{tabular}{c c c c c c c}
\hline\hline
Instancia & Costo mínimo & Tiempo(seg.) & Costo promedio & Tiempo promedio(seg.) & Costo SCA & \%Gap \\ [0.5ex]
\hline
SCA3-0 & 640.55 & 27.14 & 
640.55 & 17.44 & \bf{636.06} & 
0.71\\SCA3-1 & 701.78 & 13.93 & 
701.84 & 10.93 & \bf{697.84} & 
0.56\\SCA3-2 & 661.13 & 17.48 & 
664.44 & 20.01 & \bf{659.34} & 
0.27\\SCA3-3 & 681.74 & 17.03 & 
681.91 & 11.58 & \bf{680.04} & 
0.25\\SCA3-4 & \bf{690.50} & 19.20 & 
691.70 & 16.11 & 690.50 & 0.00\\
SCA3-5 & 665.04 & 12.79 & 
668.84 & 11.18 & \bf{659.90} & 
0.78\\SCA3-6 & 652.94 & 11.70 & 
653.16 & 12.10 & \bf{651.09} & 
0.28\\SCA3-7 & 667.24 & 10.34 & 
668.35 & 10.28 & \bf{659.17} & 
1.22\\SCA3-8 & \bf{719.47} & 16.42 & 
720.44 & 16.21 & 719.47 & 0.00\\
SCA3-9 & \bf{681.00} & 15.35 & 
681.06 & 11.95 & 681.00 & 0.00\\
SCA8-0 & 995.46 & 32.28 & 
999.43 & 36.02 & \bf{961.50} & 
3.53\\SCA8-1 & 1058.59 & 40.34 & 
1070.04 & 40.22 & \bf{1050.20} & 
0.80\\SCA8-2 & 1053.78 & 30.36 & 
1053.88 & 36.05 & \bf{1039.64} & 
1.36\\SCA8-3 & 1002.63 & 43.35 & 
1008.15 & 35.12 & \bf{983.34} & 
1.96\\SCA8-4 & 1067.55 & 32.74 & 
1071.33 & 28.52 & \bf{1065.49} & 
0.19\\SCA8-5 & 1042.51 & 45.30 & 
1046.45 & 51.40 & \bf{1027.08} & 
1.50\\SCA8-6 & \bf{971.82} & 33.56 & 
978.65 & 32.88 & 971.82 & 0.00\\
SCA8-7 & 1070.89 & 51.21 & 
1074.02 & 34.99 & \bf{1052.17} & 
1.78\\SCA8-8 & \bf{1071.18} & 24.26 & 
1081.14 & 28.45 & 1071.18 & 0.00\\
SCA8-9 & 1072.10 & 34.11 & 
1078.19 & 39.95 & \bf{1060.50} & 
1.09\\CON3-0 & 617.59 & 6.23 & 
618.34 & 7.07 & \bf{616.52} & 
0.17\\CON3-1 & 560.75 & 11.69 & 
560.75 & 13.75 & \bf{554.47} & 
1.13\\CON3-2 & 521.38 & 12.40 & 
521.38 & 8.91 & \bf{519.26} & 
0.41\\CON3-3 & \bf{591.19} & 13.98 & 
591.19 & 12.64 & 591.19 & 0.00\\
CON3-4 & 591.43 & 8.88 & 
591.43 & 9.15 & \bf{589.32} & 
0.36\\CON3-5 & 564.88 & 11.38 & 
564.89 & 9.21 & \bf{563.70} & 
0.21\\CON3-6 & 502.16 & 11.38 & 
502.16 & 10.71 & \bf{500.80} & 
0.27\\CON3-7 & 578.41 & 16.86 & 
578.41 & 13.42 & \bf{576.48} & 
0.33\\CON3-8 & 523.19 & 13.98 & 
523.19 & 13.56 & \bf{523.05} & 
0.03\\CON3-9 & 588.40 & 14.83 & 
589.00 & 11.32 & \bf{580.05} & 
1.44\\CON8-0 & 873.77 & 37.66 & 
876.41 & 33.73 & \bf{857.17} & 
1.94\\CON8-1 & 748.85 & 35.51 & 
751.35 & 33.22 & \bf{740.85} & 
1.08\\CON8-2 & 713.60 & 31.68 & 
717.57 & 42.83 & \bf{713.44} & 
0.02\\CON8-3 & 816.87 & 26.90 & 
817.97 & 28.07 & \bf{811.07} & 
0.72\\CON8-4 & 776.98 & 33.77 & 
776.98 & 25.56 & \bf{772.25} & 
0.61\\CON8-5 & \bf{756.91} & 38.51 & 
768.97 & 35.74 & 756.91 & 0.00\\
CON8-6 & 698.41 & 34.13 & 
703.06 & 26.19 & \bf{678.92} & 
2.87\\CON8-7 & 815.79 & 20.99 & 
819.13 & 26.19 & \bf{811.96} & 
0.47\\CON8-8 & 781.42 & 45.17 & 
784.49 & 40.33 & \bf{767.53} & 
1.81\\CON8-9 & 821.09 & 25.52 & 
822.78 & 34.47 & \bf{809.00} & 
1.49\\[1ex]\hline
\end{tabular}
\label{table:nonlin}
\end{table} \clearpage
\begin{table}[ht]
\caption{Resultados de la ejecución de la metaheurística SCA, utilizando instancias de SalhiNagy con la configuración -n 125.0 -b 10 -y .2}
\centering
\small
\begin{tabular}{c c c c c c c}
\hline\hline
Instancia & Costo mínimo & Tiempo(seg.) & Costo promedio & Tiempo promedio(seg.) & Costo SCA & \%Gap \\ [0.5ex]
\hline
[1ex]\hline
\end{tabular}
\label{table:nonlin}
\end{table} \clearpage
\begin{table}[ht]
\caption{Resultados de la ejecución de la metaheurística SCA, utilizando instancias de Dethloff con la configuración -n 125.0 -b 10 -y .3}
\centering
\small
\begin{tabular}{c c c c c c c}
\hline\hline
Instancia & Costo mínimo & Tiempo(seg.) & Costo promedio & Tiempo promedio(seg.) & Costo SCA & \%Gap \\ [0.5ex]
\hline
SCA3-0 & 640.55 & 7.97 & 
640.55 & 13.95 & \bf{636.06} & 
0.71\\SCA3-1 & \bf{697.84} & 13.83 & 
698.76 & 11.76 & 697.84 & 0.00\\
SCA3-2 & 664.92 & 15.75 & 
665.12 & 17.70 & \bf{659.34} & 
0.85\\SCA3-3 & 680.60 & 14.57 & 
680.60 & 10.91 & \bf{680.04} & 
0.08\\SCA3-4 & \bf{690.50} & 11.08 & 
690.50 & 13.05 & 690.50 & 0.00\\
SCA3-5 & 665.64 & 9.61 & 
667.77 & 14.46 & \bf{659.90} & 
0.87\\SCA3-6 & \bf{651.09} & 25.45 & 
652.88 & 21.14 & 651.09 & 0.00\\
SCA3-7 & 671.67 & 9.20 & 
671.67 & 8.70 & \bf{659.17} & 
1.90\\SCA3-8 & \bf{719.47} & 26.74 & 
719.54 & 15.40 & 719.47 & 0.00\\
SCA3-9 & \bf{681.00} & 13.72 & 
682.03 & 13.08 & 681.00 & 0.00\\
SCA8-0 & 979.79 & 27.58 & 
980.41 & 32.00 & \bf{961.50} & 
1.90\\SCA8-1 & 1050.93 & 39.55 & 
1059.98 & 40.08 & \bf{1050.20} & 
0.07\\SCA8-2 & 1051.42 & 61.29 & 
1052.56 & 42.79 & \bf{1039.64} & 
1.13\\SCA8-3 & 1021.22 & 22.78 & 
1021.22 & 24.25 & \bf{983.34} & 
3.85\\SCA8-4 & 1071.61 & 41.97 & 
1078.99 & 33.33 & \bf{1065.49} & 
0.57\\SCA8-5 & 1038.59 & 34.81 & 
1048.83 & 37.85 & \bf{1027.08} & 
1.12\\SCA8-6 & 977.87 & 34.59 & 
984.33 & 34.21 & \bf{971.82} & 
0.62\\SCA8-7 & 1063.60 & 69.02 & 
1074.69 & 52.28 & \bf{1052.17} & 
1.09\\SCA8-8 & 1090.71 & 26.15 & 
1091.05 & 29.29 & \bf{1071.18} & 
1.82\\SCA8-9 & 1068.65 & 26.58 & 
1080.24 & 35.06 & \bf{1060.50} & 
0.77\\CON3-0 & 617.59 & 11.57 & 
622.61 & 9.88 & \bf{616.52} & 
0.17\\CON3-1 & 556.04 & 11.63 & 
558.39 & 11.23 & \bf{554.47} & 
0.28\\CON3-2 & 521.38 & 8.48 & 
521.38 & 9.11 & \bf{519.26} & 
0.41\\CON3-3 & \bf{591.19} & 10.45 & 
591.19 & 11.52 & 591.19 & 0.00\\
CON3-4 & 591.43 & 10.60 & 
591.43 & 11.34 & \bf{589.32} & 
0.36\\CON3-5 & \bf{563.70} & 12.60 & 
564.70 & 12.94 & 563.70 & 0.00\\
CON3-6 & 502.16 & 7.41 & 
502.16 & 11.09 & \bf{500.80} & 
0.27\\CON3-7 & 578.41 & 11.57 & 
579.79 & 12.26 & \bf{576.48} & 
0.33\\CON3-8 & \bf{523.05} & 14.81 & 
523.67 & 17.22 & 523.05 & 0.00\\
CON3-9 & 582.79 & 7.53 & 
582.79 & 10.18 & \bf{580.05} & 
0.47\\CON8-0 & 870.61 & 19.70 & 
874.31 & 28.30 & \bf{857.17} & 
1.57\\CON8-1 & 742.47 & 29.58 & 
745.72 & 42.20 & \bf{740.85} & 
0.22\\CON8-2 & 713.60 & 33.18 & 
714.65 & 39.52 & \bf{713.44} & 
0.02\\CON8-3 & 823.06 & 27.92 & 
825.69 & 27.97 & \bf{811.07} & 
1.48\\CON8-4 & 777.24 & 27.42 & 
783.94 & 36.00 & \bf{772.25} & 
0.65\\CON8-5 & 763.13 & 34.76 & 
766.91 & 32.40 & \bf{756.91} & 
0.82\\CON8-6 & 693.83 & 29.78 & 
696.01 & 45.40 & \bf{678.92} & 
2.20\\CON8-7 & 814.79 & 73.33 & 
815.53 & 49.73 & \bf{811.96} & 
0.35\\CON8-8 & 777.28 & 27.25 & 
777.28 & 28.05 & \bf{767.53} & 
1.27\\CON8-9 & 825.61 & 30.44 & 
829.43 & 38.61 & \bf{809.00} & 
2.05\\[1ex]\hline
\end{tabular}
\label{table:nonlin}
\end{table} \clearpage
\begin{table}[ht]
\caption{Resultados de la ejecución de la metaheurística SCA, utilizando instancias de SalhiNagy con la configuración -n 125.0 -b 10 -y .3}
\centering
\small
\begin{tabular}{c c c c c c c}
\hline\hline
Instancia & Costo mínimo & Tiempo(seg.) & Costo promedio & Tiempo promedio(seg.) & Costo SCA & \%Gap \\ [0.5ex]
\hline
[1ex]\hline
\end{tabular}
\label{table:nonlin}
\end{table} \clearpage
\begin{table}[ht]
\caption{Resultados de la ejecución de la metaheurística SCA, utilizando instancias de Dethloff con la configuración -n 125.0 -b 10 -y .4}
\centering
\small
\begin{tabular}{c c c c c c c}
\hline\hline
Instancia & Costo mínimo & Tiempo(seg.) & Costo promedio & Tiempo promedio(seg.) & Costo SCA & \%Gap \\ [0.5ex]
\hline
SCA3-0 & 640.55 & 12.56 & 
640.55 & 13.18 & \bf{636.06} & 
0.71\\SCA3-1 & \bf{697.84} & 10.68 & 
699.68 & 13.60 & 697.84 & 0.00\\
SCA3-2 & \bf{659.34} & 15.29 & 
663.16 & 15.89 & 659.34 & 0.00\\
SCA3-3 & 681.74 & 12.43 & 
681.74 & 11.32 & \bf{680.04} & 
0.25\\SCA3-4 & \bf{690.50} & 14.30 & 
690.50 & 12.67 & 690.50 & 0.00\\
SCA3-5 & 670.02 & 9.41 & 
675.91 & 15.20 & \bf{659.90} & 
1.53\\SCA3-6 & 652.94 & 13.09 & 
653.35 & 15.43 & \bf{651.09} & 
0.28\\SCA3-7 & 666.60 & 9.92 & 
670.40 & 15.48 & \bf{659.17} & 
1.13\\SCA3-8 & \bf{719.47} & 28.27 & 
719.47 & 16.43 & 719.47 & 0.00\\
SCA3-9 & \bf{681.00} & 13.99 & 
682.03 & 14.73 & 681.00 & 0.00\\
SCA8-0 & 989.06 & 38.17 & 
998.03 & 32.22 & \bf{961.50} & 
2.87\\SCA8-1 & 1054.07 & 34.24 & 
1064.55 & 44.39 & \bf{1050.20} & 
0.37\\SCA8-2 & 1051.95 & 33.64 & 
1053.96 & 42.83 & \bf{1039.64} & 
1.18\\SCA8-3 & 1012.89 & 35.51 & 
1021.92 & 25.79 & \bf{983.34} & 
3.01\\SCA8-4 & 1074.19 & 36.37 & 
1076.33 & 30.39 & \bf{1065.49} & 
0.82\\SCA8-5 & 1032.79 & 54.27 & 
1046.16 & 39.14 & \bf{1027.08} & 
0.56\\SCA8-6 & 972.48 & 39.51 & 
974.66 & 38.09 & \bf{971.82} & 
0.07\\SCA8-7 & 1066.65 & 29.25 & 
1068.20 & 45.20 & \bf{1052.17} & 
1.38\\SCA8-8 & 1085.34 & 64.51 & 
1093.42 & 38.54 & \bf{1071.18} & 
1.32\\SCA8-9 & 1067.27 & 42.32 & 
1067.62 & 35.39 & \bf{1060.50} & 
0.64\\CON3-0 & 619.09 & 8.96 & 
619.09 & 9.20 & \bf{616.52} & 
0.42\\CON3-1 & 556.92 & 9.40 & 
556.92 & 9.18 & \bf{554.47} & 
0.44\\CON3-2 & 521.38 & 13.57 & 
521.38 & 15.74 & \bf{519.26} & 
0.41\\CON3-3 & 591.20 & 10.45 & 
591.20 & 11.86 & \bf{591.19} & 
0.00\\CON3-4 & 591.43 & 8.49 & 
591.43 & 8.89 & \bf{589.32} & 
0.36\\CON3-5 & \bf{563.70} & 10.04 & 
563.70 & 9.26 & 563.70 & 0.00\\
CON3-6 & 502.16 & 13.44 & 
502.16 & 15.40 & \bf{500.80} & 
0.27\\CON3-7 & 581.39 & 14.72 & 
581.75 & 16.21 & \bf{576.48} & 
0.85\\CON3-8 & 523.19 & 7.42 & 
523.19 & 7.91 & \bf{523.05} & 
0.03\\CON3-9 & 582.79 & 12.72 & 
585.60 & 9.78 & \bf{580.05} & 
0.47\\CON8-0 & 873.79 & 41.99 & 
876.65 & 35.25 & \bf{857.17} & 
1.94\\CON8-1 & 742.47 & 30.15 & 
746.87 & 45.78 & \bf{740.85} & 
0.22\\CON8-2 & 716.07 & 31.12 & 
717.49 & 32.16 & \bf{713.44} & 
0.37\\CON8-3 & 816.87 & 43.03 & 
826.73 & 51.82 & \bf{811.07} & 
0.72\\CON8-4 & 784.99 & 30.82 & 
789.71 & 33.76 & \bf{772.25} & 
1.65\\CON8-5 & 765.06 & 37.45 & 
771.59 & 31.11 & \bf{756.91} & 
1.08\\CON8-6 & 693.82 & 60.59 & 
697.85 & 46.62 & \bf{678.92} & 
2.19\\CON8-7 & 814.79 & 67.48 & 
816.78 & 53.78 & \bf{811.96} & 
0.35\\CON8-8 & 781.96 & 27.25 & 
787.05 & 36.10 & \bf{767.53} & 
1.88\\CON8-9 & 817.29 & 43.57 & 
817.60 & 36.53 & \bf{809.00} & 
1.02\\[1ex]\hline
\end{tabular}
\label{table:nonlin}
\end{table} \clearpage
\begin{table}[ht]
\caption{Resultados de la ejecución de la metaheurística SCA, utilizando instancias de SalhiNagy con la configuración -n 125.0 -b 10 -y .4}
\centering
\small
\begin{tabular}{c c c c c c c}
\hline\hline
Instancia & Costo mínimo & Tiempo(seg.) & Costo promedio & Tiempo promedio(seg.) & Costo SCA & \%Gap \\ [0.5ex]
\hline
[1ex]\hline
\end{tabular}
\label{table:nonlin}
\end{table} \clearpage
\begin{table}[ht]
\caption{Resultados de la ejecución de la metaheurística SCA, utilizando instancias de Dethloff con la configuración -n 125.0 -b 10 -y .5}
\centering
\small
\begin{tabular}{c c c c c c c}
\hline\hline
Instancia & Costo mínimo & Tiempo(seg.) & Costo promedio & Tiempo promedio(seg.) & Costo SCA & \%Gap \\ [0.5ex]
\hline
SCA3-0 & 640.55 & 17.72 & 
640.55 & 14.93 & \bf{636.06} & 
0.71\\SCA3-1 & \bf{697.84} & 16.38 & 
697.84 & 12.84 & 697.84 & 0.00\\
SCA3-2 & \bf{659.34} & 22.54 & 
662.99 & 22.30 & 659.34 & 0.00\\
SCA3-3 & 680.60 & 9.62 & 
680.60 & 12.11 & \bf{680.04} & 
0.08\\SCA3-4 & \bf{690.50} & 22.30 & 
690.50 & 18.62 & 690.50 & 0.00\\
SCA3-5 & 668.48 & 15.88 & 
669.25 & 13.64 & \bf{659.90} & 
1.30\\SCA3-6 & \bf{651.09} & 11.76 & 
651.09 & 11.79 & 651.09 & 0.00\\
SCA3-7 & 666.15 & 9.85 & 
669.28 & 12.01 & \bf{659.17} & 
1.06\\SCA3-8 & \bf{719.47} & 10.62 & 
719.47 & 13.48 & 719.47 & 0.00\\
SCA3-9 & \bf{681.00} & 12.71 & 
683.07 & 12.16 & 681.00 & 0.00\\
SCA8-0 & 970.64 & 38.21 & 
973.81 & 32.73 & \bf{961.50} & 
0.95\\SCA8-1 & 1063.83 & 30.92 & 
1064.85 & 32.00 & \bf{1050.20} & 
1.30\\SCA8-2 & 1052.94 & 45.49 & 
1052.94 & 34.68 & \bf{1039.64} & 
1.28\\SCA8-3 & 1013.56 & 45.11 & 
1026.33 & 38.95 & \bf{983.34} & 
3.07\\SCA8-4 & 1068.97 & 37.37 & 
1072.12 & 38.37 & \bf{1065.49} & 
0.33\\SCA8-5 & 1044.70 & 27.70 & 
1052.76 & 32.67 & \bf{1027.08} & 
1.72\\SCA8-6 & 972.48 & 49.77 & 
974.66 & 57.11 & \bf{971.82} & 
0.07\\SCA8-7 & 1063.22 & 35.83 & 
1067.60 & 32.55 & \bf{1052.17} & 
1.05\\SCA8-8 & 1095.67 & 37.78 & 
1096.11 & 33.38 & \bf{1071.18} & 
2.29\\SCA8-9 & 1070.71 & 51.93 & 
1078.14 & 43.81 & \bf{1060.50} & 
0.96\\CON3-0 & 619.09 & 8.78 & 
624.49 & 11.49 & \bf{616.52} & 
0.42\\CON3-1 & 560.75 & 15.51 & 
560.75 & 17.04 & \bf{554.47} & 
1.13\\CON3-2 & 521.38 & 8.93 & 
521.38 & 14.44 & \bf{519.26} & 
0.41\\CON3-3 & \bf{591.19} & 12.48 & 
591.19 & 12.41 & 591.19 & 0.00\\
CON3-4 & 591.43 & 11.12 & 
591.43 & 11.67 & \bf{589.32} & 
0.36\\CON3-5 & 564.89 & 15.53 & 
564.89 & 11.09 & \bf{563.70} & 
0.21\\CON3-6 & 502.16 & 9.90 & 
502.16 & 10.80 & \bf{500.80} & 
0.27\\CON3-7 & 578.79 & 16.79 & 
581.76 & 14.40 & \bf{576.48} & 
0.40\\CON3-8 & 523.14 & 14.04 & 
523.93 & 13.41 & \bf{523.05} & 
0.02\\CON3-9 & 582.79 & 8.06 & 
586.99 & 11.55 & \bf{580.05} & 
0.47\\CON8-0 & 864.56 & 25.87 & 
866.91 & 36.63 & \bf{857.17} & 
0.86\\CON8-1 & 742.47 & 45.29 & 
749.22 & 43.07 & \bf{740.85} & 
0.22\\CON8-2 & 714.06 & 38.85 & 
721.48 & 36.14 & \bf{713.44} & 
0.09\\CON8-3 & 821.24 & 55.57 & 
823.86 & 43.55 & \bf{811.07} & 
1.25\\CON8-4 & \bf{772.25} & 28.63 & 
780.77 & 31.62 & 772.25 & 0.00\\
CON8-5 & 761.27 & 44.83 & 
764.37 & 46.01 & \bf{756.91} & 
0.58\\CON8-6 & 698.87 & 34.82 & 
702.97 & 33.63 & \bf{678.92} & 
2.94\\CON8-7 & 815.43 & 38.67 & 
817.13 & 48.87 & \bf{811.96} & 
0.43\\CON8-8 & 784.98 & 20.23 & 
785.84 & 27.33 & \bf{767.53} & 
2.27\\CON8-9 & 812.35 & 29.61 & 
817.00 & 27.62 & \bf{809.00} & 
0.41\\[1ex]\hline
\end{tabular}
\label{table:nonlin}
\end{table} \clearpage
\begin{table}[ht]
\caption{Resultados de la ejecución de la metaheurística SCA, utilizando instancias de SalhiNagy con la configuración -n 125.0 -b 10 -y .5}
\centering
\small
\begin{tabular}{c c c c c c c}
\hline\hline
Instancia & Costo mínimo & Tiempo(seg.) & Costo promedio & Tiempo promedio(seg.) & Costo SCA & \%Gap \\ [0.5ex]
\hline
[1ex]\hline
\end{tabular}
\label{table:nonlin}
\end{table} \clearpage
\begin{table}[ht]
\caption{Resultados de la ejecución de la metaheurística SCA, utilizando instancias de SalhiNagy con la configuración -n 50.0 -b 10 -y 0.1}
\centering
\small
\begin{tabular}{c c c c c c c}
\hline\hline
Instancia & Costo mínimo & Tiempo(seg.) & Costo promedio & Tiempo promedio(seg.) & Costo SCA & \%Gap \\ [0.5ex]
\hline
CMT1X & 472.37 & 12.56 & 
476.39 & 12.46 & \bf{469.80} & 
0.55\\CMT1Y & 472.37 & 12.73 & 
477.37 & 10.11 & \bf{469.80} & 
0.55\\CMT2X & 100000 & 0 & 
nan & nan & \bf{684.21} & 
14515.40\\CMT2Y & 100000 & 0 & 
nan & nan & \bf{684.21} & 
14515.40\\CMT3X & 731.48 & 95.85 & 
737.00 & 114.52 & \bf{721.27} & 
1.42\\CMT3Y & 736.83 & 371.85 & 
739.14 & 213.60 & \bf{721.27} & 
2.16\\CMT4X & 899.91 & 764.69 & 
909.99 & 882.99 & \bf{852.46} & 
5.57\\CMT4Y & 899.14 & 737.02 & 
907.64 & 956.04 & \bf{852.46} & 
5.48\\CMT5X & 100000 & 0 & 
nan & nan & \bf{1030.55} & 
9603.56\\CMT5Y & 100000 & 0 & 
nan & nan & \bf{1030.55} & 
9603.56\\CMT11X & 895.32 & 311.88 & 
902.62 & 170.62 & \bf{838.66} & 
6.76\\CMT11Y & 889.98 & 123.11 & 
899.97 & 125.33 & \bf{837.08} & 
6.32\\CMT12X & 682.07 & 183.88 & 
686.00 & 268.03 & \bf{662.22} & 
3.00\\CMT12Y & 673.32 & 180.50 & 
680.46 & 276.88 & \bf{662.22} & 
1.68\\[1ex]\hline
\end{tabular}
\label{table:nonlin}
\end{table} \clearpage
\begin{table}[ht]
\caption{Resultados de la ejecución de la metaheurística SCA, utilizando instancias de SalhiNagy con la configuración -n 50.0 -b 10 -y .2}
\centering
\small
\begin{tabular}{c c c c c c c}
\hline\hline
Instancia & Costo mínimo & Tiempo(seg.) & Costo promedio & Tiempo promedio(seg.) & Costo SCA & \%Gap \\ [0.5ex]
\hline
CMT1X & 472.37 & 8.47 & 
474.45 & 9.21 & \bf{469.80} & 
0.55\\CMT1Y & 474.72 & 3.20 & 
474.72 & 6.17 & \bf{469.80} & 
1.05\\CMT2X & 100000 & 0 & 
nan & nan & \bf{684.21} & 
14515.40\\CMT2Y & 100000 & 0 & 
nan & nan & \bf{684.21} & 
14515.40\\CMT3X & 734.88 & 85.40 & 
740.12 & 106.17 & \bf{721.27} & 
1.89\\CMT3Y & 728.81 & 385.62 & 
737.13 & 200.08 & \bf{721.27} & 
1.05\\CMT4X & 885.31 & 757.27 & 
896.28 & 812.98 & \bf{852.46} & 
3.85\\CMT4Y & 895.35 & 1078.63 & 
903.97 & 1140.83 & \bf{852.46} & 
5.03\\CMT5X & 100000 & 0 & 
nan & nan & \bf{1030.55} & 
9603.56\\CMT5Y & 100000 & 0 & 
nan & nan & \bf{1030.55} & 
9603.56\\CMT11X & 897.43 & 122.23 & 
899.57 & 166.69 & \bf{838.66} & 
7.01\\CMT11Y & 896.81 & 109.67 & 
916.04 & 179.71 & \bf{837.08} & 
7.14\\CMT12X & 678.01 & 225.44 & 
683.83 & 199.98 & \bf{662.22} & 
2.38\\CMT12Y & 673.49 & 456.12 & 
678.52 & 384.33 & \bf{662.22} & 
1.70\\[1ex]\hline
\end{tabular}
\label{table:nonlin}
\end{table} \clearpage
\begin{table}[ht]
\caption{Resultados de la ejecución de la metaheurística SCA, utilizando instancias de SalhiNagy con la configuración -n 50.0 -b 10 -y .3}
\centering
\small
\begin{tabular}{c c c c c c c}
\hline\hline
Instancia & Costo mínimo & Tiempo(seg.) & Costo promedio & Tiempo promedio(seg.) & Costo SCA & \%Gap \\ [0.5ex]
\hline
CMT1X & 473.96 & 4.88 & 
479.57 & 4.91 & \bf{469.80} & 
0.89\\CMT1Y & 472.37 & 18.07 & 
473.25 & 10.52 & \bf{469.80} & 
0.55\\CMT2X & 100000 & 0 & 
nan & nan & \bf{684.21} & 
14515.40\\CMT2Y & 100000 & 0 & 
nan & nan & \bf{684.21} & 
14515.40\\CMT3X & 730.67 & 128.38 & 
732.67 & 164.91 & \bf{721.27} & 
1.30\\CMT3Y & 732.08 & 96.15 & 
737.00 & 147.54 & \bf{721.27} & 
1.50\\CMT4X & 883.12 & 1200.77 & 
903.07 & 1093.23 & \bf{852.46} & 
3.60\\CMT4Y & 882.12 & 1860.04 & 
902.95 & 1197.42 & \bf{852.46} & 
3.48\\CMT5X & 100000 & 0 & 
nan & nan & \bf{1030.55} & 
9603.56\\CMT5Y & 100000 & 0 & 
nan & nan & \bf{1030.55} & 
9603.56\\CMT11X & 889.09 & 147.15 & 
903.61 & 137.99 & \bf{838.66} & 
6.01\\CMT11Y & 900.73 & 134.04 & 
905.80 & 181.85 & \bf{837.08} & 
7.60\\CMT12X & 684.60 & 235.19 & 
686.91 & 263.97 & \bf{662.22} & 
3.38\\CMT12Y & 680.56 & 184.19 & 
685.06 & 245.53 & \bf{662.22} & 
2.77\\[1ex]\hline
\end{tabular}
\label{table:nonlin}
\end{table} \clearpage
\begin{table}[ht]
\caption{Resultados de la ejecución de la metaheurística SCA, utilizando instancias de SalhiNagy con la configuración -n 50.0 -b 10 -y .4}
\centering
\small
\begin{tabular}{c c c c c c c}
\hline\hline
Instancia & Costo mínimo & Tiempo(seg.) & Costo promedio & Tiempo promedio(seg.) & Costo SCA & \%Gap \\ [0.5ex]
\hline
CMT1X & 472.37 & 9.82 & 
476.32 & 7.84 & \bf{469.80} & 
0.55\\CMT1Y & 473.01 & 4.67 & 
474.37 & 9.62 & \bf{469.80} & 
0.68\\CMT2X & 100000 & 0 & 
nan & nan & \bf{684.21} & 
14515.40\\CMT2Y & 100000 & 0 & 
nan & nan & \bf{684.21} & 
14515.40\\CMT3X & 733.13 & 276.84 & 
738.10 & 150.42 & \bf{721.27} & 
1.64\\CMT3Y & 730.73 & 148.30 & 
735.58 & 136.20 & \bf{721.27} & 
1.31\\CMT4X & 885.57 & 786.92 & 
895.17 & 833.33 & \bf{852.46} & 
3.88\\CMT4Y & 903.26 & 899.07 & 
908.51 & 975.58 & \bf{852.46} & 
5.96\\CMT5X & 100000 & 0 & 
nan & nan & \bf{1030.55} & 
9603.56\\CMT5Y & 100000 & 0 & 
nan & nan & \bf{1030.55} & 
9603.56\\CMT11X & 893.66 & 347.97 & 
897.29 & 256.02 & \bf{838.66} & 
6.56\\CMT11Y & 894.23 & 101.84 & 
906.45 & 109.51 & \bf{837.08} & 
6.83\\CMT12X & 675.74 & 174.11 & 
684.41 & 308.81 & \bf{662.22} & 
2.04\\CMT12Y & 674.16 & 281.33 & 
676.76 & 291.47 & \bf{662.22} & 
1.80\\[1ex]\hline
\end{tabular}
\label{table:nonlin}
\end{table} \clearpage
\begin{table}[ht]
\caption{Resultados de la ejecución de la metaheurística SCA, utilizando instancias de SalhiNagy con la configuración -n 50.0 -b 10 -y .5}
\centering
\small
\begin{tabular}{c c c c c c c}
\hline\hline
Instancia & Costo mínimo & Tiempo(seg.) & Costo promedio & Tiempo promedio(seg.) & Costo SCA & \%Gap \\ [0.5ex]
\hline
CMT1X & 475.22 & 6.91 & 
476.66 & 8.31 & \bf{469.80} & 
1.15\\CMT1Y & 472.37 & 9.78 & 
473.59 & 10.51 & \bf{469.80} & 
0.55\\CMT2X & 100000 & 0 & 
nan & nan & \bf{684.21} & 
14515.40\\CMT2Y & 100000 & 0 & 
nan & nan & \bf{684.21} & 
14515.40\\CMT3X & 735.45 & 102.97 & 
738.43 & 173.65 & \bf{721.27} & 
1.97\\CMT3Y & 734.04 & 123.37 & 
739.63 & 146.46 & \bf{721.27} & 
1.77\\CMT4X & 866.27 & 905.86 & 
896.16 & 981.30 & \bf{852.46} & 
1.62\\CMT4Y & 903.46 & 1050.13 & 
908.80 & 1006.39 & \bf{852.46} & 
5.98\\CMT5X & 100000 & 0 & 
nan & nan & \bf{1030.55} & 
9603.56\\CMT5Y & 100000 & 0 & 
nan & nan & \bf{1030.55} & 
9603.56\\CMT11X & 889.03 & 118.64 & 
904.22 & 254.75 & \bf{838.66} & 
6.01\\CMT11Y & 896.65 & 354.33 & 
901.27 & 238.11 & \bf{837.08} & 
7.12\\CMT12X & 686.54 & 486.04 & 
688.73 & 315.61 & \bf{662.22} & 
3.67\\CMT12Y & 677.84 & 211.35 & 
681.32 & 189.87 & \bf{662.22} & 
2.36\\[1ex]\hline
\end{tabular}
\label{table:nonlin}
\end{table} \clearpage
\begin{table}[ht]
\caption{Resultados de la ejecución de la metaheurística GTS, utilizando instancias de Dethloff con la configuración -mni 100 -lambda1 0.05 -lambda2 0.05 -tabu 10}
\centering
\small
\begin{tabular}{c c c c c c c}
\hline\hline
Instancia & Costo mínimo & Tiempo(seg.) & Costo promedio & Tiempo promedio(seg.) & Costo GTS & \%Gap \\ [0.5ex]
\hline
SCA3-0 & 643.15 & 0.16 & 
643.15 & 0.16 & \bf{636.06} & 
1.11\\SCA3-1 & 712.77 & 0.12 & 
712.77 & 0.12 & \bf{697.84} & 
2.14\\SCA3-2 & \bf{659.34} & 0.13 & 
659.34 & 0.13 & 659.34 & 0.00\\
SCA3-3 & 699.25 & 0.15 & 
699.25 & 0.15 & \bf{680.04} & 
2.82\\SCA3-4 & 742.02 & 0.18 & 
742.02 & 0.18 & \bf{690.50} & 
7.46\\SCA3-5 & 702.62 & 0.12 & 
702.62 & 0.12 & \bf{659.90} & 
6.47\\SCA3-6 & 652.94 & 0.12 & 
652.94 & 0.12 & \bf{651.09} & 
0.28\\SCA3-7 & 672.22 & 0.11 & 
672.22 & 0.11 & \bf{659.17} & 
1.98\\SCA3-8 & 744.22 & 0.17 & 
744.22 & 0.17 & \bf{719.47} & 
3.44\\SCA3-9 & 697.46 & 0.12 & 
697.46 & 0.12 & \bf{681.00} & 
2.42\\SCA8-0 & 1087.33 & 0.18 & 
1087.33 & 0.18 & \bf{961.50} & 
13.09\\SCA8-1 & 1073.44 & 0.19 & 
1073.44 & 0.19 & \bf{1050.20} & 
2.21\\SCA8-2 & 1050.37 & 0.21 & 
1050.37 & 0.21 & \bf{1039.64} & 
1.03\\SCA8-3 & 1022.99 & 0.15 & 
1022.99 & 0.15 & \bf{983.34} & 
4.03\\SCA8-4 & 1069.33 & 0.21 & 
1069.33 & 0.21 & \bf{1065.49} & 
0.36\\SCA8-5 & 1101.62 & 0.15 & 
1101.62 & 0.15 & \bf{1027.08} & 
7.26\\SCA8-6 & 973.12 & 0.12 & 
973.12 & 0.12 & \bf{971.82} & 
0.13\\SCA8-7 & 1179.58 & 0.13 & 
1179.58 & 0.13 & \bf{1052.17} & 
12.11\\SCA8-8 & 1116.76 & 0.12 & 
1116.76 & 0.12 & \bf{1071.18} & 
4.26\\SCA8-9 & 1092.60 & 0.18 & 
1092.60 & 0.18 & \bf{1060.50} & 
3.03\\CON3-0 & 651.53 & 0.12 & 
651.53 & 0.12 & \bf{616.52} & 
5.68\\CON3-1 & 581.04 & 0.19 & 
581.04 & 0.19 & \bf{554.47} & 
4.79\\CON3-2 & 542.97 & 0.12 & 
542.97 & 0.12 & \bf{519.26} & 
4.57\\CON3-3 & 641.80 & 0.13 & 
641.80 & 0.13 & \bf{591.19} & 
8.56\\CON3-4 & 609.94 & 0.11 & 
609.94 & 0.11 & \bf{589.32} & 
3.50\\CON3-5 & 594.26 & 0.15 & 
594.26 & 0.15 & \bf{563.70} & 
5.42\\CON3-6 & 531.32 & 0.14 & 
531.32 & 0.14 & \bf{500.80} & 
6.09\\CON3-7 & 586.01 & 0.16 & 
586.01 & 0.16 & \bf{576.48} & 
1.65\\CON3-8 & \bf{523.05} & 0.11 & 
523.05 & 0.11 & 523.05 & 0.00\\
CON3-9 & 634.44 & 0.16 & 
634.44 & 0.16 & \bf{580.05} & 
9.38\\CON8-0 & 914.83 & 0.13 & 
914.83 & 0.13 & \bf{857.17} & 
6.73\\CON8-1 & 764.76 & 0.24 & 
764.76 & 0.24 & \bf{740.85} & 
3.23\\CON8-2 & 721.67 & 0.18 & 
721.67 & 0.18 & \bf{713.44} & 
1.15\\CON8-3 & 871.69 & 0.30 & 
871.69 & 0.30 & \bf{811.07} & 
7.47\\CON8-4 & 799.05 & 0.15 & 
799.05 & 0.15 & \bf{772.25} & 
3.47\\CON8-5 & 774.29 & 0.14 & 
774.29 & 0.14 & \bf{756.91} & 
2.30\\CON8-6 & 727.19 & 0.13 & 
727.19 & 0.13 & \bf{678.92} & 
7.11\\CON8-7 & 871.13 & 0.11 & 
871.13 & 0.11 & \bf{811.96} & 
7.29\\CON8-8 & 794.84 & 0.14 & 
794.84 & 0.14 & \bf{767.53} & 
3.56\\CON8-9 & 878.36 & 0.23 & 
878.36 & 0.23 & \bf{809.00} & 
8.57\\\bf{TOTAL & 792.682 & 0.154 & 792.682 & 0.154 & 758.782 & -4.49884}\\
[1ex]\hline
\end{tabular}
\label{table:nonlin}
\end{table} \clearpage
\begin{table}[ht]
\caption{Resultados de la ejecución de la metaheurística GTS, utilizando instancias de Dethloff con la configuración -mni 100 -lambda1 0.05 -lambda2 0.05 -tabu 10}
\centering
\small
\begin{tabular}{c c c c c c c}
\hline\hline
Instancia & Costo mínimo & Tiempo(seg.) & Costo promedio & Tiempo promedio(seg.) & Costo GTS & \%Gap \\ [0.5ex]
\hline
[1ex]\hline
\end{tabular}
\label{table:nonlin}
\end{table} \clearpage
\begin{table}[ht]
\caption{Resultados de la ejecución de la metaheurística GTS, utilizando instancias de Dethloff con la configuración -mni 100 -lambda1 0.05 -lambda2 0.05 -tabu 10}
\centering
\small
\begin{tabular}{c c c c c c c}
\hline\hline
Instancia & Costo mínimo & Tiempo(seg.) & Costo promedio & Tiempo promedio(seg.) & Costo GTS & \%Gap \\ [0.5ex]
\hline
[1ex]\hline
\end{tabular}
\label{table:nonlin}
\end{table} \clearpage
\begin{table}[ht]
\caption{Resultados de la ejecución de la metaheurística GTS, utilizando instancias de Dethloff con la configuración -mni 100 -lambda1 0.05 -lambda2 0.05 -tabu 10}
\centering
\small
\begin{tabular}{c c c c c c c}
\hline\hline
Instancia & Costo mínimo & Tiempo(seg.) & Costo promedio & Tiempo promedio(seg.) & Costo GTS & \%Gap \\ [0.5ex]
\hline
SCA3-0 & 643.15 & 0.15 & 
643.15 & 0.15 & \bf{636.06} & 
1.11\\SCA3-1 & \bf{697.84} & 0.14 & 
697.84 & 0.14 & 697.84 & 0.00\\
SCA3-2 & 686.20 & 0.15 & 
686.20 & 0.15 & \bf{659.34} & 
4.07\\SCA3-3 & 709.03 & 0.16 & 
709.03 & 0.16 & \bf{680.04} & 
4.26\\SCA3-4 & \bf{690.50} & 0.15 & 
690.50 & 0.15 & 690.50 & 0.00\\
SCA3-5 & 681.81 & 0.17 & 
681.81 & 0.17 & \bf{659.90} & 
3.32\\SCA3-6 & 678.35 & 0.16 & 
678.35 & 0.16 & \bf{651.09} & 
4.19\\SCA3-7 & 677.38 & 0.10 & 
677.38 & 0.10 & \bf{659.17} & 
2.76\\SCA3-8 & 719.77 & 0.13 & 
719.77 & 0.13 & \bf{719.47} & 
0.04\\SCA3-9 & 710.24 & 0.13 & 
710.24 & 0.13 & \bf{681.00} & 
4.29\\SCA8-0 & 1037.22 & 0.24 & 
1037.22 & 0.24 & \bf{961.50} & 
7.88\\SCA8-1 & 1083.90 & 0.15 & 
1083.90 & 0.15 & \bf{1050.20} & 
3.21\\SCA8-2 & 1058.65 & 0.20 & 
1058.65 & 0.20 & \bf{1039.64} & 
1.83\\SCA8-3 & 1026.92 & 0.22 & 
1026.92 & 0.22 & \bf{983.34} & 
4.43\\SCA8-4 & 1131.42 & 0.10 & 
1131.42 & 0.10 & \bf{1065.49} & 
6.19\\SCA8-5 & 1068.73 & 0.20 & 
1068.73 & 0.20 & \bf{1027.08} & 
4.06\\SCA8-6 & 1003.06 & 0.15 & 
1003.06 & 0.15 & \bf{971.82} & 
3.21\\SCA8-7 & 1091.44 & 0.34 & 
1091.44 & 0.34 & \bf{1052.17} & 
3.73\\SCA8-8 & 1098.39 & 0.20 & 
1098.39 & 0.20 & \bf{1071.18} & 
2.54\\SCA8-9 & 1088.34 & 0.26 & 
1088.34 & 0.26 & \bf{1060.50} & 
2.63\\CON3-0 & 661.31 & 0.13 & 
661.31 & 0.13 & \bf{616.52} & 
7.26\\CON3-1 & 567.40 & 0.16 & 
567.40 & 0.16 & \bf{554.47} & 
2.33\\CON3-2 & 528.41 & 0.19 & 
528.41 & 0.19 & \bf{519.26} & 
1.76\\CON3-3 & 591.20 & 0.15 & 
591.20 & 0.15 & \bf{591.19} & 
0.00\\CON3-4 & 591.43 & 0.12 & 
591.43 & 0.12 & \bf{589.32} & 
0.36\\CON3-5 & 576.49 & 0.15 & 
576.49 & 0.15 & \bf{563.70} & 
2.27\\CON3-6 & 504.14 & 0.12 & 
504.14 & 0.12 & \bf{500.80} & 
0.67\\CON3-7 & 589.93 & 0.13 & 
589.93 & 0.13 & \bf{576.48} & 
2.33\\CON3-8 & 545.31 & 0.17 & 
545.31 & 0.17 & \bf{523.05} & 
4.26\\CON3-9 & 603.33 & 0.18 & 
603.33 & 0.18 & \bf{580.05} & 
4.01\\CON8-0 & 998.98 & 0.21 & 
998.98 & 0.21 & \bf{857.17} & 
16.54\\CON8-1 & 764.29 & 0.20 & 
764.29 & 0.20 & \bf{740.85} & 
3.16\\CON8-2 & 797.43 & 0.20 & 
797.43 & 0.20 & \bf{713.44} & 
11.77\\CON8-3 & 852.50 & 0.16 & 
852.50 & 0.16 & \bf{811.07} & 
5.11\\CON8-4 & 862.20 & 0.18 & 
862.20 & 0.18 & \bf{772.25} & 
11.65\\CON8-5 & 764.40 & 0.27 & 
764.40 & 0.27 & \bf{756.91} & 
0.99\\CON8-6 & 717.46 & 0.19 & 
717.46 & 0.19 & \bf{678.92} & 
5.68\\CON8-7 & 894.58 & 0.20 & 
894.58 & 0.20 & \bf{811.96} & 
10.18\\CON8-8 & 786.19 & 0.18 & 
786.19 & 0.18 & \bf{767.53} & 
2.43\\CON8-9 & 838.18 & 0.13 & 
838.18 & 0.13 & \bf{809.00} & 
3.61\\\bf{TOTAL & 
790.44 & 0.17 & 790.44 & 0.17 & 758.78 & -4.14}\\[1ex]\hline
\end{tabular}
\label{table:nonlin}
\end{table} \clearpage
\begin{table}[ht]
\caption{Resultados de la ejecución de la metaheurística GTS, utilizando instancias de Dethloff con la configuración -mni 100 -lambda1 0.05 -lambda2 0.05 -tabu 10}
\centering
\small
\begin{tabular}{c c c c c c c}
\hline\hline
Instancia & Costo mínimo & Tiempo(seg.) & Costo promedio & Tiempo promedio(seg.) & Costo GTS & \%Gap \\ [0.5ex]
\hline
SCA3-0 & 661.16 & 0.13 & 
661.16 & 0.13 & \bf{636.06} & 
3.95\\SCA3-1 & 701.53 & 0.21 & 
701.53 & 0.21 & \bf{697.84} & 
0.53\\SCA3-2 & 726.40 & 0.10 & 
726.40 & 0.10 & \bf{659.34} & 
10.17\\SCA3-3 & 682.46 & 0.14 & 
682.46 & 0.14 & \bf{680.04} & 
0.36\\SCA3-4 & \bf{690.50} & 0.20 & 
690.50 & 0.20 & 690.50 & 0.00\\
SCA3-5 & 734.03 & 0.18 & 
734.03 & 0.18 & \bf{659.90} & 
11.23\\SCA3-6 & 655.05 & 0.15 & 
655.05 & 0.15 & \bf{651.09} & 
0.61\\SCA3-7 & \bf{659.17} & 0.13 & 
659.17 & 0.13 & 659.17 & 0.00\\
SCA3-8 & 745.62 & 0.19 & 
745.62 & 0.19 & \bf{719.47} & 
3.63\\SCA3-9 & 689.95 & 0.13 & 
689.95 & 0.13 & \bf{681.00} & 
1.31\\SCA8-0 & 977.93 & 0.23 & 
977.93 & 0.23 & \bf{961.50} & 
1.71\\SCA8-1 & 1074.02 & 0.19 & 
1074.02 & 0.19 & \bf{1050.20} & 
2.27\\SCA8-2 & 1196.64 & 0.21 & 
1196.64 & 0.21 & \bf{1039.64} & 
15.10\\SCA8-3 & 1074.55 & 0.19 & 
1074.55 & 0.19 & \bf{983.34} & 
9.28\\SCA8-4 & 1130.78 & 0.22 & 
1130.78 & 0.22 & \bf{1065.49} & 
6.13\\SCA8-5 & 1061.29 & 0.18 & 
1061.29 & 0.18 & \bf{1027.08} & 
3.33\\SCA8-6 & 1008.55 & 0.22 & 
1008.55 & 0.22 & \bf{971.82} & 
3.78\\SCA8-7 & 1135.44 & 0.20 & 
1135.44 & 0.20 & \bf{1052.17} & 
7.91\\SCA8-8 & 1099.10 & 0.26 & 
1099.10 & 0.26 & \bf{1071.18} & 
2.61\\SCA8-9 & 1108.19 & 0.19 & 
1108.19 & 0.19 & \bf{1060.50} & 
4.50\\CON3-0 & 633.86 & 0.12 & 
633.86 & 0.12 & \bf{616.52} & 
2.81\\CON3-1 & 569.81 & 0.12 & 
569.81 & 0.12 & \bf{554.47} & 
2.77\\CON3-2 & 538.53 & 0.17 & 
538.53 & 0.17 & \bf{519.26} & 
3.71\\CON3-3 & 609.02 & 0.14 & 
609.02 & 0.14 & \bf{591.19} & 
3.02\\CON3-4 & 602.36 & 0.12 & 
602.36 & 0.12 & \bf{589.32} & 
2.21\\CON3-5 & 564.36 & 0.21 & 
564.36 & 0.21 & \bf{563.70} & 
0.12\\CON3-6 & 520.82 & 0.19 & 
520.82 & 0.19 & \bf{500.80} & 
4.00\\CON3-7 & 599.02 & 0.15 & 
599.02 & 0.15 & \bf{576.48} & 
3.91\\CON3-8 & 523.68 & 0.15 & 
523.68 & 0.15 & \bf{523.05} & 
0.12\\CON3-9 & 593.87 & 0.16 & 
593.87 & 0.16 & \bf{580.05} & 
2.38\\CON8-0 & 924.37 & 0.25 & 
924.37 & 0.25 & \bf{857.17} & 
7.84\\CON8-1 & 776.54 & 0.18 & 
776.54 & 0.18 & \bf{740.85} & 
4.82\\CON8-2 & 760.40 & 0.26 & 
760.40 & 0.26 & \bf{713.44} & 
6.58\\CON8-3 & 883.65 & 0.17 & 
883.65 & 0.17 & \bf{811.07} & 
8.95\\CON8-4 & 787.57 & 0.34 & 
787.57 & 0.34 & \bf{772.25} & 
1.98\\CON8-5 & 796.18 & 0.17 & 
796.18 & 0.17 & \bf{756.91} & 
5.19\\CON8-6 & 740.78 & 0.14 & 
740.78 & 0.14 & \bf{678.92} & 
9.11\\CON8-7 & 826.20 & 0.14 & 
826.20 & 0.14 & \bf{811.96} & 
1.75\\CON8-8 & 809.25 & 0.22 & 
809.25 & 0.22 & \bf{767.53} & 
5.44\\CON8-9 & 826.73 & 0.22 & 
826.73 & 0.22 & \bf{809.00} & 
2.19\\\bf{TOTAL} & 
\bf{792.48} & \bf{0.18} & \bf{792.48} & \bf{0.18} & \bf{758.78} & \bf{4.21}\\[1ex]\hline
\end{tabular}
\label{table:nonlin}
\end{table} \clearpage
\begin{table}[ht]
\caption{Resultados de la ejecución de la metaheurística GTS, utilizando instancias de Dethloff con la configuración -mni 6000 -lambda1 0.05 -lambda2 0.05 -tabu 30}
\centering
\small
\begin{tabular}{c c c c c c c}
\hline\hline
Instancia & Costo mínimo & Tiempo(seg.) & Costo promedio & Tiempo promedio(seg.) & Costo GTS & \%Gap \\ [0.5ex]
\hline
SCA3-0 & \bf{636.06} & 3.23 & 
636.06 & 4.53 & 636.06 & 0.00\\
SCA3-1 & \bf{697.84} & 7.00 & 
699.17 & 5.79 & 697.84 & 0.00\\
SCA3-2 & \bf{659.34} & 3.17 & 
659.34 & 3.70 & 659.34 & 0.00\\
SCA3-3 & \bf{680.04} & 13.59 & 
685.16 & 8.16 & 680.04 & 0.00\\
SCA3-4 & \bf{690.50} & 5.90 & 
690.50 & 9.01 & 690.50 & 0.00\\
SCA3-5 & \bf{659.90} & 9.53 & 
659.90 & 6.28 & 659.90 & 0.00\\
SCA3-6 & \bf{651.09} & 3.33 & 
652.38 & 6.20 & 651.09 & 0.00\\
SCA3-7 & 666.15 & 3.87 & 
666.15 & 3.88 & \bf{659.17} & 
1.06\\SCA3-8 & \bf{719.47} & 7.69 & 
719.47 & 5.84 & 719.47 & 0.00\\
SCA3-9 & \bf{681.00} & 3.14 & 
681.00 & 4.09 & 681.00 & 0.00\\
SCA8-0 & 970.65 & 2.98 & 
975.99 & 3.77 & \bf{961.50} & 
0.95\\SCA8-1 & 1068.14 & 3.84 & 
1068.22 & 4.18 & \bf{1050.20} & 
1.71\\SCA8-2 & 1050.37 & 8.59 & 
1057.30 & 6.38 & \bf{1039.64} & 
1.03\\SCA8-3 & 1005.75 & 4.36 & 
1009.65 & 5.21 & \bf{983.34} & 
2.28\\SCA8-4 & 1067.55 & 3.52 & 
1087.51 & 3.64 & \bf{1065.49} & 
0.19\\SCA8-5 & 1042.30 & 8.65 & 
1042.37 & 6.05 & \bf{1027.08} & 
1.48\\SCA8-6 & \bf{971.82} & 3.88 & 
972.15 & 3.37 & 971.82 & 0.00\\
SCA8-7 & \bf{\underline{1052.04}} & 15.40 & 
1056.51 & 10.29 & 1052.17 & 
-0.01\\SCA8-8 & 1082.12 & 2.08 & 
1083.26 & 2.60 & \bf{1071.18} & 
1.02\\SCA8-9 & \bf{1060.50} & 5.84 & 
1060.50 & 5.61 & 1060.50 & 0.00\\
CON3-0 & 628.47 & 5.77 & 
630.57 & 5.49 & \bf{616.52} & 
1.94\\CON3-1 & \bf{554.47} & 4.90 & 
555.38 & 4.92 & 554.47 & 0.00\\
CON3-2 & \bf{519.26} & 4.52 & 
520.32 & 3.81 & 519.26 & 0.00\\
CON3-3 & \bf{591.19} & 4.85 & 
591.19 & 4.80 & 591.19 & 0.00\\
CON3-4 & \bf{\underline{588.79}} & 5.70 & 
596.77 & 4.55 & 589.32 & 
-0.09\\CON3-5 & \bf{563.70} & 3.20 & 
571.14 & 3.15 & 563.70 & 0.00\\
CON3-6 & \bf{\underline{499.05}} & 14.60 & 
500.61 & 8.71 & 500.80 & 
-0.35\\CON3-7 & \bf{576.48} & 3.47 & 
576.48 & 5.76 & 576.48 & 0.00\\
CON3-8 & \bf{523.05} & 3.01 & 
523.05 & 3.10 & 523.05 & 0.00\\
CON3-9 & 582.79 & 13.03 & 
583.17 & 9.68 & \bf{580.05} & 
0.47\\CON8-0 & \bf{857.17} & 4.27 & 
861.70 & 3.88 & 857.17 & 0.00\\
CON8-1 & \bf{740.85} & 5.96 & 
740.85 & 6.61 & 740.85 & 0.00\\
CON8-2 & 721.67 & 3.35 & 
748.06 & 3.60 & \bf{713.44} & 
1.15\\CON8-3 & \bf{811.07} & 5.49 & 
816.16 & 7.22 & 811.07 & 0.00\\
CON8-4 & \bf{772.25} & 7.57 & 
782.97 & 6.44 & 772.25 & 0.00\\
CON8-5 & \bf{\underline{754.95}} & 4.54 & 
756.53 & 4.75 & 756.91 & 
-0.26\\CON8-6 & 688.47 & 6.69 & 
690.80 & 5.58 & \bf{678.92} & 
1.41\\CON8-7 & 812.89 & 4.48 & 
812.89 & 4.20 & \bf{811.96} & 
0.11\\CON8-8 & \bf{767.53} & 7.15 & 
767.53 & 6.67 & 767.53 & 0.00\\
CON8-9 & \bf{809.00} & 3.83 & 
810.62 & 3.58 & 809.00 & 0.00\\
\bf{TOTAL} & 
\bf{761.89} & \bf{5.90} & \bf{764.98} & \bf{5.38} & \bf{758.78} & \bf{0.76}\\[1ex]\hline
\end{tabular}
\label{table:nonlin}
\end{table} \clearpage
\begin{table}[ht]
\caption{Resultados de la ejecución de la metaheurística GTS, utilizando instancias de Dethloff con la configuración -mni 6000 -lambda1 0.05 -lambda2 0.05 -tabu 30}
\centering
\small
\begin{tabular}{c c c c c c c}
\hline\hline
Instancia & Costo mínimo & Tiempo(seg.) & Costo promedio & Tiempo promedio(seg.) & Costo GTS & \%Gap \\ [0.5ex]
\hline
SCA3-0 & \bf{636.06} & 4.62 & 
638.30 & 4.32 & 636.06 & 0.00\\
SCA3-1 & \bf{697.84} & 7.53 & 
699.17 & 5.14 & 697.84 & 0.00\\
SCA3-2 & \bf{659.34} & 4.32 & 
659.34 & 3.56 & 659.34 & 0.00\\
SCA3-3 & \bf{680.04} & 10.22 & 
680.04 & 7.41 & 680.04 & 0.00\\
SCA3-4 & \bf{690.50} & 3.40 & 
690.50 & 3.58 & 690.50 & 0.00\\
SCA3-5 & \bf{659.90} & 4.82 & 
659.90 & 5.07 & 659.90 & 0.00\\
SCA3-6 & \bf{651.09} & 5.16 & 
651.09 & 5.80 & 651.09 & 0.00\\
SCA3-7 & 666.15 & 5.34 & 
666.15 & 4.59 & \bf{659.17} & 
1.06\\SCA3-8 & \bf{719.47} & 4.30 & 
719.47 & 5.07 & 719.47 & 0.00\\
SCA3-9 & \bf{681.00} & 3.45 & 
681.00 & 3.40 & 681.00 & 0.00\\
\bf{TOTAL} & 
\bf{674.14} & \bf{5.32} & \bf{674.50} & \bf{4.79} & \bf{673.44} & \bf{0.32}\\[1ex]\hline
\end{tabular}
\label{table:nonlin}
\end{table} \clearpage
\begin{table}[ht]
\caption{Resultados de la ejecución de la metaheurística GTS, utilizando instancias de Dethloff con la configuración -mni 6000 -lambda1 0.05 -lambda2 0.05 -tabu 30}
\centering
\small
\begin{tabular}{c c c c c c c}
\hline\hline
Instancia & Costo mínimo & Tiempo(seg.) & Costo promedio & Tiempo promedio(seg.) & Costo GTS & \%Gap \\ [0.5ex]
\hline
SCA3-0 & \bf{\underline{635.62}} & 5.58 & 
638.09 & 8.37 & 636.06 & 
\bf{-0.07}\\SCA3-1 & \bf{697.84} & 5.42 & 
699.17 & 6.33 & 697.84 & 0.00\\
SCA3-2 & \bf{659.34} & 13.41 & 
659.34 & 10.10 & 659.34 & 0.00\\
SCA3-3 & \bf{680.04} & 10.28 & 
684.92 & 7.37 & 680.04 & 0.00\\
SCA3-4 & \bf{690.50} & 7.10 & 
690.50 & 8.41 & 690.50 & 0.00\\
SCA3-5 & \bf{659.90} & 4.01 & 
659.90 & 4.70 & 659.90 & 0.00\\
SCA3-6 & \bf{651.09} & 4.55 & 
651.09 & 4.25 & 651.09 & 0.00\\
SCA3-7 & \bf{659.17} & 2.74 & 
664.53 & 3.26 & 659.17 & 0.00\\
SCA3-8 & \bf{719.47} & 5.73 & 
719.47 & 5.40 & 719.47 & 0.00\\
SCA3-9 & \bf{681.00} & 5.13 & 
681.00 & 6.75 & 681.00 & 0.00\\
\bf{TOTAL} & 
\bf{673.40} & \bf{6.39} & \bf{674.80} & \bf{6.49} & \bf{673.44} & \bf{-0.01}\\[1ex]\hline
\end{tabular}
\label{table:nonlin}
\end{table} \clearpage
\begin{table}[ht]
\caption{Resultados de la ejecución de la metaheurística GTS, utilizando instancias de Dethloff con la configuración -mni 6000 -lambda1 0.05 -lambda2 0.05 -tabu 40}
\centering
\small
\begin{tabular}{c c c c c c c}
\hline\hline
Instancia & Costo mínimo & Tiempo(seg.) & Costo promedio & Tiempo promedio(seg.) & Costo GTS & \%Gap \\ [0.5ex]
\hline
SCA3-0 & \bf{\underline{635.62}} & 6.20 & 
638.20 & 5.16 & 636.06 & 
\bf{-0.07}\\SCA3-1 & \bf{697.84} & 4.60 & 
698.50 & 6.04 & 697.84 & 0.00\\
SCA3-2 & \bf{659.34} & 3.63 & 
659.34 & 4.66 & 659.34 & 0.00\\
SCA3-3 & \bf{680.04} & 7.21 & 
680.18 & 5.39 & 680.04 & 0.00\\
SCA3-4 & \bf{690.50} & 7.80 & 
690.50 & 8.25 & 690.50 & 0.00\\
SCA3-5 & \bf{659.90} & 8.52 & 
659.90 & 6.09 & 659.90 & 0.00\\
SCA3-6 & \bf{651.09} & 3.40 & 
652.01 & 3.62 & 651.09 & 0.00\\
SCA3-7 & 666.15 & 4.79 & 
666.15 & 5.06 & \bf{659.17} & 
1.06\\SCA3-8 & \bf{719.47} & 5.76 & 
719.47 & 6.84 & 719.47 & 0.00\\
SCA3-9 & \bf{681.00} & 9.84 & 
681.00 & 6.58 & 681.00 & 0.00\\
SCA8-0 & \bf{961.50} & 3.94 & 
974.95 & 4.83 & 961.50 & 0.00\\
SCA8-1 & \bf{\underline{1049.65}} & 7.75 & 
1054.63 & 5.45 & 1050.20 & 
\bf{-0.05}\\SCA8-2 & 1049.22 & 5.14 & 
1058.00 & 4.74 & \bf{1039.64} & 
0.92\\SCA8-3 & \bf{983.34} & 4.92 & 
993.45 & 6.13 & 983.34 & 0.00\\
SCA8-4 & 1067.28 & 11.84 & 
1069.10 & 6.44 & \bf{1065.49} & 
0.17\\SCA8-5 & \bf{1027.08} & 6.33 & 
1042.05 & 5.29 & 1027.08 & 0.00\\
SCA8-6 & 972.48 & 7.80 & 
980.75 & 5.36 & \bf{971.82} & 
0.07\\SCA8-7 & 1060.98 & 6.46 & 
1071.81 & 5.73 & \bf{1052.17} & 
0.84\\SCA8-8 & \bf{1071.18} & 4.45 & 
1076.65 & 4.04 & 1071.18 & 0.00\\
SCA8-9 & 1063.68 & 5.36 & 
1065.06 & 4.61 & \bf{1060.50} & 
0.30\\CON3-0 & \bf{616.52} & 8.40 & 
626.86 & 6.86 & 616.52 & 0.00\\
CON3-1 & \bf{554.47} & 9.61 & 
556.83 & 4.85 & 554.47 & 0.00\\
CON3-2 & 519.61 & 12.75 & 
522.44 & 8.04 & \bf{519.26} & 
0.07\\CON3-3 & \bf{591.19} & 9.64 & 
591.19 & 7.79 & 591.19 & 0.00\\
CON3-4 & \bf{\underline{588.79}} & 4.64 & 
596.53 & 5.72 & 589.32 & 
\bf{-0.09}\\CON3-5 & \bf{563.70} & 5.10 & 
563.70 & 6.28 & 563.70 & 0.00\\
CON3-6 & \bf{\underline{499.05}} & 5.92 & 
499.83 & 5.08 & 500.80 & 
\bf{-0.35}\\CON3-7 & \bf{576.48} & 6.83 & 
583.01 & 6.89 & 576.48 & 0.00\\
CON3-8 & \bf{523.05} & 10.39 & 
523.05 & 5.79 & 523.05 & 0.00\\
CON3-9 & \bf{\underline{578.25}} & 5.08 & 
580.63 & 4.34 & 580.05 & 
\bf{-0.31}\\CON8-0 & \bf{857.17} & 4.44 & 
869.59 & 5.93 & 857.17 & 0.00\\
CON8-1 & \bf{740.85} & 4.54 & 
743.60 & 5.90 & 740.85 & 0.00\\
CON8-2 & 722.22 & 7.34 & 
726.75 & 6.86 & \bf{713.44} & 
1.23\\CON8-3 & \bf{811.07} & 5.08 & 
813.62 & 6.58 & 811.07 & 0.00\\
CON8-4 & 784.36 & 4.24 & 
794.46 & 4.16 & \bf{772.25} & 
1.57\\CON8-5 & \bf{\underline{754.88}} & 8.44 & 
756.65 & 6.61 & 756.91 & 
\bf{-0.27}\\CON8-6 & 688.47 & 7.14 & 
693.67 & 5.97 & \bf{678.92} & 
1.41\\CON8-7 & 812.89 & 9.65 & 
814.25 & 7.53 & \bf{811.96} & 
0.11\\CON8-8 & \bf{767.53} & 7.04 & 
776.14 & 5.55 & 767.53 & 0.00\\
CON8-9 & \bf{809.00} & 2.68 & 
826.89 & 5.14 & 809.00 & 0.00\\
\bf{TOTAL} & 
\bf{760.17} & \bf{6.62} & \bf{764.78} & \bf{5.80} & \bf{758.78} & \bf{0.17}\\[1ex]\hline
\end{tabular}
\label{table:nonlin}
\end{table} \clearpage
\begin{table}[ht]
\caption{Resultados de la ejecución de la metaheurística SCA, utilizando instancias de SalhiNagy con la configuración -n 50.0 -b 10 -y 0.1}
\centering
\small
\begin{tabular}{c c c c c c c}
\hline\hline
Instancia & Costo mínimo & Tiempo(seg.) & Costo promedio & Tiempo promedio(seg.) & Costo SCA & \%Gap \\ [0.5ex]
\hline
CMT1X & 472.37 & 11.10 & 
477.24 & 8.21 & \bf{469.80} & 
0.55\\CMT1Y & 474.72 & 5.12 & 
475.79 & 7.38 & \bf{469.80} & 
1.05\\CMT2X & 706.42 & 45.60 & 
710.09 & 60.66 & \bf{684.21} & 
3.25\\CMT2Y & 700.61 & 128.19 & 
709.28 & 109.25 & \bf{684.21} & 
2.40\\CMT3X & 727.62 & 136.88 & 
734.00 & 158.17 & \bf{721.27} & 
0.88\\CMT3Y & 733.20 & 125.67 & 
738.26 & 128.16 & \bf{721.27} & 
1.65\\CMT4X & 897.96 & 1009.94 & 
904.25 & 880.14 & \bf{852.46} & 
5.34\\CMT4Y & 895.89 & 889.57 & 
902.38 & 1302.33 & \bf{852.46} & 
5.09\\CMT5X & 1103.71 & 3512.99 & 
1115.62 & 4002.39 & \bf{1030.55} & 
7.10\\CMT5Y & 1091.22 & 2565.57 & 
1106.72 & 4247.15 & \bf{1030.55} & 
5.89\\CMT11X & 892.07 & 345.90 & 
901.08 & 181.88 & \bf{838.66} & 
6.37\\CMT11Y & 879.21 & 227.98 & 
903.41 & 178.84 & \bf{837.08} & 
5.03\\CMT12X & 676.05 & 204.97 & 
682.66 & 336.77 & \bf{662.22} & 
2.09\\CMT12Y & 675.77 & 302.95 & 
681.80 & 303.47 & \bf{662.22} & 
2.05\\\bf{PROM.} & 
\bf{780.49} & \bf{679.46} & \bf{788.76} & \bf{850.34} & \bf{751.20} & \bf{3.48}\\[1ex]\hline
\end{tabular}
\label{table:nonlin}
\end{table} \clearpage
\begin{table}[ht]
\caption{Resultados de la ejecución de la metaheurística SCA, utilizando instancias de SalhiNagy con la configuración -n 50.0 -b 10 -y .2}
\centering
\small
\begin{tabular}{c c c c c c c}
\hline\hline
Instancia & Costo mínimo & Tiempo(seg.) & Costo promedio & Tiempo promedio(seg.) & Costo SCA & \%Gap \\ [0.5ex]
\hline
CMT1X & 472.37 & 7.29 & 
476.38 & 7.92 & \bf{469.80} & 
0.55\\CMT1Y & 473.01 & 10.00 & 
473.82 & 8.99 & \bf{469.80} & 
0.68\\CMT2X & 705.00 & 57.44 & 
710.77 & 72.07 & \bf{684.21} & 
3.04\\CMT2Y & 708.04 & 101.42 & 
714.75 & 101.90 & \bf{684.21} & 
3.48\\CMT3X & 727.88 & 148.63 & 
734.83 & 121.80 & \bf{721.27} & 
0.92\\CMT3Y & 730.10 & 96.04 & 
738.43 & 153.71 & \bf{721.27} & 
1.22\\CMT4X & 884.59 & 1103.76 & 
894.38 & 1489.52 & \bf{852.46} & 
3.77\\CMT4Y & 884.28 & 968.81 & 
899.26 & 1038.38 & \bf{852.46} & 
3.73\\CMT5X & 1101.36 & 5181.28 & 
1111.45 & 4716.54 & \bf{1030.55} & 
6.87\\CMT5Y & 1102.95 & 2855.27 & 
1124.71 & 3868.89 & \bf{1030.55} & 
7.03\\CMT11X & 900.12 & 349.42 & 
910.18 & 226.47 & \bf{838.66} & 
7.33\\CMT11Y & 900.67 & 145.80 & 
903.58 & 131.09 & \bf{837.08} & 
7.60\\CMT12X & 680.25 & 555.57 & 
686.22 & 377.02 & \bf{662.22} & 
2.72\\CMT12Y & 671.00 & 282.65 & 
682.42 & 253.34 & \bf{662.22} & 
1.33\\\bf{PROM.} & 
\bf{781.54} & \bf{847.38} & \bf{790.08} & \bf{897.69} & \bf{751.20} & \bf{3.59}\\[1ex]\hline
\end{tabular}
\label{table:nonlin}
\end{table} \clearpage
\begin{table}[ht]
\caption{Resultados de la ejecución de la metaheurística SCA, utilizando instancias de SalhiNagy con la configuración -n 50.0 -b 10 -y .3}
\centering
\small
\begin{tabular}{c c c c c c c}
\hline\hline
Instancia & Costo mínimo & Tiempo(seg.) & Costo promedio & Tiempo promedio(seg.) & Costo SCA & \%Gap \\ [0.5ex]
\hline
CMT1X & 472.37 & 6.75 & 
473.21 & 10.88 & \bf{469.80} & 
0.55\\CMT1Y & 472.37 & 15.58 & 
473.67 & 14.25 & \bf{469.80} & 
0.55\\CMT2X & 702.96 & 103.95 & 
707.70 & 92.75 & \bf{684.21} & 
2.74\\CMT2Y & 712.58 & 158.42 & 
715.31 & 86.06 & \bf{684.21} & 
4.15\\CMT3X & 735.60 & 146.76 & 
738.91 & 132.66 & \bf{721.27} & 
1.99\\CMT3Y & 733.37 & 72.99 & 
735.86 & 131.09 & \bf{721.27} & 
1.68\\CMT4X & 901.59 & 1255.72 & 
909.40 & 980.09 & \bf{852.46} & 
5.76\\CMT4Y & 904.78 & 963.33 & 
908.32 & 1034.99 & \bf{852.46} & 
6.14\\CMT5X & 1091.99 & 5979.58 & 
1112.42 & 4324.88 & \bf{1030.55} & 
5.96\\CMT5Y & 1089.67 & 3990.18 & 
1115.27 & 3964.31 & \bf{1030.55} & 
5.74\\CMT11X & 887.25 & 116.10 & 
899.51 & 176.97 & \bf{838.66} & 
5.79\\CMT11Y & 892.88 & 321.13 & 
903.71 & 233.61 & \bf{837.08} & 
6.67\\CMT12X & 687.16 & 212.13 & 
690.83 & 330.78 & \bf{662.22} & 
3.77\\CMT12Y & 677.05 & 483.04 & 
684.40 & 372.72 & \bf{662.22} & 
2.24\\\bf{PROM.} & 
\bf{782.97} & \bf{987.55} & \bf{790.61} & \bf{849.00} & \bf{751.20} & \bf{3.84}\\[1ex]\hline
\end{tabular}
\label{table:nonlin}
\end{table} \clearpage
\begin{table}[ht]
\caption{Resultados de la ejecución de la metaheurística SCA, utilizando instancias de SalhiNagy con la configuración -n 50.0 -b 10 -y .4}
\centering
\small
\begin{tabular}{c c c c c c c}
\hline\hline
Instancia & Costo mínimo & Tiempo(seg.) & Costo promedio & Tiempo promedio(seg.) & Costo SCA & \%Gap \\ [0.5ex]
\hline
CMT1X & 474.41 & 6.04 & 
475.56 & 8.44 & \bf{469.80} & 
0.98\\CMT1Y & 472.37 & 6.71 & 
474.06 & 8.77 & \bf{469.80} & 
0.55\\CMT2X & 697.70 & 70.58 & 
704.66 & 69.58 & \bf{684.21} & 
1.97\\CMT2Y & 704.17 & 66.54 & 
705.49 & 69.91 & \bf{684.21} & 
2.92\\CMT3X & 733.47 & 157.17 & 
739.26 & 139.47 & \bf{721.27} & 
1.69\\CMT3Y & 729.10 & 161.87 & 
734.41 & 152.90 & \bf{721.27} & 
1.09\\CMT4X & 910.95 & 1123.18 & 
911.94 & 1189.45 & \bf{852.46} & 
6.86\\CMT4Y & 897.27 & 1005.95 & 
911.59 & 1131.84 & \bf{852.46} & 
5.26\\CMT5X & 1108.56 & 4659.77 & 
1113.17 & 5656.24 & \bf{1030.55} & 
7.57\\CMT5Y & 1094.86 & 3281.61 & 
1110.00 & 3475.35 & \bf{1030.55} & 
6.24\\CMT11X & 884.73 & 270.62 & 
897.64 & 225.68 & \bf{838.66} & 
5.49\\CMT11Y & 889.63 & 169.30 & 
900.37 & 158.12 & \bf{837.08} & 
6.28\\CMT12X & 676.38 & 288.54 & 
682.17 & 277.64 & \bf{662.22} & 
2.14\\CMT12Y & 681.15 & 462.31 & 
683.37 & 310.64 & \bf{662.22} & 
2.86\\\bf{PROM.} & 
\bf{782.48} & \bf{837.87} & \bf{788.84} & \bf{919.57} & \bf{751.20} & \bf{3.71}\\[1ex]\hline
\end{tabular}
\label{table:nonlin}
\end{table} \clearpage
\begin{table}[ht]
\caption{Resultados de la ejecución de la metaheurística SCA, utilizando instancias de SalhiNagy con la configuración -n 50.0 -b 10 -y .5}
\centering
\small
\begin{tabular}{c c c c c c c}
\hline\hline
Instancia & Costo mínimo & Tiempo(seg.) & Costo promedio & Tiempo promedio(seg.) & Costo SCA & \%Gap \\ [0.5ex]
\hline
CMT1X & 474.72 & 14.82 & 
478.11 & 9.55 & \bf{469.80} & 
1.05\\CMT1Y & 474.41 & 13.07 & 
477.83 & 8.18 & \bf{469.80} & 
0.98\\CMT2X & 708.06 & 112.45 & 
709.77 & 104.76 & \bf{684.21} & 
3.49\\CMT2Y & 703.42 & 69.18 & 
708.38 & 76.66 & \bf{684.21} & 
2.81\\CMT3X & 727.10 & 99.03 & 
734.13 & 167.23 & \bf{721.27} & 
0.81\\CMT3Y & 727.11 & 259.92 & 
733.40 & 202.17 & \bf{721.27} & 
0.81\\CMT4X & 904.84 & 883.64 & 
909.54 & 867.03 & \bf{852.46} & 
6.14\\CMT4Y & 898.86 & 651.66 & 
909.91 & 688.47 & \bf{852.46} & 
5.44\\CMT5X & 1097.98 & 5023.08 & 
1110.59 & 4394.23 & \bf{1030.55} & 
6.54\\CMT5Y & 1101.67 & 6204.91 & 
1113.13 & 4882.31 & \bf{1030.55} & 
6.90\\CMT11X & 867.27 & 289.24 & 
890.55 & 189.01 & \bf{838.66} & 
3.41\\CMT11Y & 896.09 & 152.96 & 
907.32 & 230.94 & \bf{837.08} & 
7.05\\CMT12X & 674.46 & 305.25 & 
684.68 & 378.53 & \bf{662.22} & 
1.85\\CMT12Y & 674.72 & 175.99 & 
682.40 & 284.60 & \bf{662.22} & 
1.89\\\bf{PROM.} & 
\bf{780.77} & \bf{1018.23} & \bf{789.27} & \bf{891.69} & \bf{751.20} & \bf{3.51}\\[1ex]\hline
\end{tabular}
\label{table:nonlin}
\end{table} \clearpage
\begin{table}[ht]
\caption{Resultados de la ejecución de la metaheurística SCA, utilizando instancias de SalhiNagy con la configuración -n 75.0 -b 10 -y 0.1}
\centering
\small
\begin{tabular}{c c c c c c c}
\hline\hline
Instancia & Costo mínimo & Tiempo(seg.) & Costo promedio & Tiempo promedio(seg.) & Costo SCA & \%Gap \\ [0.5ex]
\hline
CMT1X & 472.37 & 8.80 & 
474.88 & 10.05 & \bf{469.80} & 
0.55 & 1.08\\CMT1Y & 472.37 & 10.02 & 
473.72 & 8.70 & \bf{469.80} & 
0.55 & 0.83\\CMT2X & 708.23 & 55.77 & 
712.69 & 70.34 & \bf{684.21} & 
3.51 & 4.16\\CMT2Y & 707.37 & 99.39 & 
712.07 & 70.56 & \bf{684.21} & 
3.38 & 4.07\\CMT3X & 732.82 & 204.62 & 
739.05 & 154.96 & \bf{721.27} & 
1.60 & 2.47\\CMT3Y & 732.22 & 244.74 & 
739.35 & 183.86 & \bf{721.27} & 
1.52 & 2.51\\CMT4X & 890.84 & 1317.73 & 
903.66 & 1120.64 & \bf{852.46} & 
4.50 & 6.01\\CMT4Y & 894.27 & 975.36 & 
901.38 & 874.21 & \bf{852.46} & 
4.90 & 5.74\\CMT5X & 1084.91 & 3866.52 & 
1101.49 & 3825.87 & \bf{1030.55} & 
5.27 & 6.88\\CMT5Y & 1093.15 & 3260.79 & 
1104.79 & 3272.98 & \bf{1030.55} & 
6.07 & 7.20\\CMT11X & 867.24 & 150.60 & 
887.23 & 177.40 & \bf{838.66} & 
3.41 & 5.79\\CMT11Y & 890.33 & 211.00 & 
901.68 & 160.62 & \bf{837.08} & 
6.36 & 7.72\\CMT12X & 674.46 & 236.33 & 
682.78 & 198.34 & \bf{662.22} & 
1.85 & 3.10\\CMT12Y & 675.62 & 213.41 & 
681.64 & 203.78 & \bf{662.22} & 
2.02 & 2.93\\\bf{PROM.} & 
\bf{778.30} & \bf{775.36} & \bf{786.89} & \bf{738.02} & \bf{751.20} & \bf{3.25} & \bf{4.32}\\[1ex]\hline
\end{tabular}
\label{table:nonlin}
\end{table} \clearpage
\begin{table}[ht]
\caption{Resultados de la ejecución de la metaheurística SCA, utilizando instancias de SalhiNagy con la configuración -n 75.0 -b 10 -y .2}
\centering
\small
\begin{tabular}{c c c c c c c c}
\hline\hline
Instancia & Costo mínimo & Tiempo(seg.) & Costo promedio & Tiempo promedio(seg.) & CME & \%G & \%GP \\ [0.5ex]
\hline
CMT1X & 472.87 & 7.74 & 
477.36 & 9.73 & \bf{469.80} & 
0.65 & 1.61\\CMT1Y & 476.75 & 4.30 & 
476.75 & 4.64 & \bf{469.80} & 
1.48 & 1.48\\CMT2X & 706.58 & 71.44 & 
711.34 & 71.11 & \bf{684.21} & 
3.27 & 3.96\\CMT2Y & 694.44 & 63.39 & 
707.66 & 102.27 & \bf{684.21} & 
1.50 & 3.43\\CMT3X & 727.74 & 115.95 & 
733.53 & 171.82 & \bf{721.27} & 
0.90 & 1.70\\CMT3Y & 734.23 & 195.55 & 
737.79 & 198.73 & \bf{721.27} & 
1.80 & 2.29\\CMT4X & 897.79 & 1543.87 & 
905.62 & 1110.94 & \bf{852.46} & 
5.32 & 6.24\\CMT4Y & 895.77 & 1471.41 & 
902.16 & 1130.19 & \bf{852.46} & 
5.08 & 5.83\\CMT5X & 1103.02 & 3472.35 & 
1110.06 & 3839.15 & \bf{1030.55} & 
7.03 & 7.72\\CMT5Y & 1101.94 & 3219.53 & 
1106.67 & 4335.72 & \bf{1030.55} & 
6.93 & 7.39\\CMT11X & 896.20 & 259.28 & 
904.10 & 180.30 & \bf{838.66} & 
6.86 & 7.80\\CMT11Y & 900.83 & 282.74 & 
911.49 & 174.79 & \bf{837.08} & 
7.62 & 8.89\\CMT12X & 683.98 & 168.30 & 
687.97 & 317.64 & \bf{662.22} & 
3.29 & 3.89\\CMT12Y & 674.81 & 358.75 & 
680.08 & 408.95 & \bf{662.22} & 
1.90 & 2.70\\\bf{PROM.} & 
\bf{783.35} & \bf{802.47} & \bf{789.47} & \bf{861.14} & \bf{751.20} & \bf{3.83} & \bf{4.64}\\[1ex]\hline
\end{tabular}
\label{table:nonlin}
\end{table} \clearpage
\begin{table}[ht]
\caption{Resultados de la ejecución de la metaheurística GTS, utilizando instancias de SalhiNagy con la configuración -mni 3000 -lambda1 0.05 -lambda2 0.05 -tabu 5}
\centering
\small
\begin{tabular}{c c c c c c c c}
\hline\hline
Instancia & Costo mínimo & Tiempo(seg.) & Costo promedio & Tiempo promedio(seg.) & CME & \%G & \%GP \\ [0.5ex]
\hline
CMT1X & \bf{470.48} & 2.39 & 
476.37 & 2.30 & 470.48 & 0.00
 & 1.25\\CMT1Y & \bf{470.48} & 1.45 & 
471.43 & 1.86 & 470.48 & 0.00
 & 0.20\\CMT2X & 690.50 & 6.81 & 
695.57 & 5.34 & \bf{682.39} & 
1.19 & 1.93\\CMT2Y & 684.33 & 8.14 & 
690.32 & 8.03 & \bf{682.39} & 
0.28 & 1.16\\CMT3X & 731.11 & 7.23 & 
734.15 & 5.92 & \bf{719.06} & 
1.68 & 2.10\\CMT3Y & 728.31 & 13.55 & 
731.19 & 10.36 & \bf{719.06} & 
1.29 & 1.69\\CMT4X & 860.05 & 37.86 & 
873.55 & 31.84 & \bf{854.21} & 
0.68 & 2.26\\CMT4Y & 875.32 & 26.85 & 
881.91 & 19.92 & \bf{852.46} & 
2.68 & 3.46\\CMT5X & 1056.38 & 55.30 & 
1072.32 & 73.74 & \bf{1030.56} & 
2.51 & 4.05\\CMT5Y & 1046.01 & 76.00 & 
1058.52 & 58.35 & \bf{1031.69} & 
1.39 & 2.60\\CMT11X & 886.55 & 15.48 & 
904.07 & 20.26 & \bf{831.09} & 
6.67 & 8.78\\CMT11Y & 876.59 & 60.21 & 
922.32 & 32.83 & \bf{829.85} & 
5.63 & 11.14\\CMT12X & 670.73 & 11.38 & 
688.26 & 8.30 & \bf{658.83} & 
1.81 & 4.47\\CMT12Y & 673.64 & 15.00 & 
675.70 & 10.71 & \bf{660.47} & 
1.99 & 2.31\\\bf{PROM.} & 
\bf{765.75} & \bf{24.12} & \bf{776.83} & \bf{20.70} & \bf{749.50} & \bf{1.99} & \bf{3.39}\\[1ex]\hline
\end{tabular}
\label{table:nonlin}
\end{table} \clearpage
\begin{table}[ht]
\caption{Resultados de la ejecución de la metaheurística GTS, utilizando instancias de SalhiNagy con la configuración -mni 3000 -lambda1 0.05 -lambda2 0.05 -tabu 7}
\centering
\small
\begin{tabular}{c c c c c c c c}
\hline\hline
Instancia & Costo mínimo & Tiempo(seg.) & Costo promedio & Tiempo promedio(seg.) & CME & \%G & \%GP \\ [0.5ex]
\hline
CMT1X & \bf{470.48} & 2.03 & 
474.77 & 2.47 & 470.48 & 0.00
 & 0.91\\CMT1Y & \bf{470.48} & 5.08 & 
473.66 & 3.73 & 470.48 & 0.00
 & 0.68\\CMT2X & 684.22 & 4.98 & 
688.19 & 5.28 & \bf{682.39} & 
0.27 & 0.85\\CMT2Y & 685.79 & 3.48 & 
688.98 & 5.06 & \bf{682.39} & 
0.50 & 0.97\\CMT3X & 728.64 & 6.80 & 
735.12 & 7.15 & \bf{719.06} & 
1.33 & 2.23\\CMT3Y & 729.32 & 8.73 & 
734.41 & 6.60 & \bf{719.06} & 
1.43 & 2.13\\CMT4X & 869.73 & 36.17 & 
885.47 & 29.21 & \bf{854.21} & 
1.82 & 3.66\\CMT4Y & 869.56 & 25.01 & 
878.91 & 24.61 & \bf{852.46} & 
2.01 & 3.10\\CMT5X & 1067.24 & 42.42 & 
1087.41 & 68.11 & \bf{1030.56} & 
3.56 & 5.52\\CMT5Y & 1066.20 & 71.41 & 
1084.29 & 48.61 & \bf{1031.69} & 
3.34 & 5.10\\CMT11X & 884.89 & 11.02 & 
922.86 & 24.09 & \bf{831.09} & 
6.47 & 11.04\\CMT11Y & 883.77 & 31.08 & 
908.62 & 31.32 & \bf{829.85} & 
6.50 & 9.49\\CMT12X & 671.20 & 6.63 & 
676.14 & 6.98 & \bf{658.83} & 
1.88 & 2.63\\CMT12Y & 674.44 & 11.00 & 
681.58 & 7.48 & \bf{660.47} & 
2.12 & 3.20\\\bf{PROM.} & 
\bf{768.28} & \bf{18.99} & \bf{780.03} & \bf{19.34} & \bf{749.50} & \bf{2.23} & \bf{3.68}\\[1ex]\hline
\end{tabular}
\label{table:nonlin}
\end{table} \clearpage
\begin{table}[ht]
\caption{Resultados de la ejecución de la metaheurística GTS, utilizando instancias de SalhiNagy con la configuración -mni 3000 -lambda1 0.05 -lambda2 0.05 -tabu 9}
\centering
\small
\begin{tabular}{c c c c c c c c}
\hline\hline
Instancia & Costo mínimo & Tiempo(seg.) & Costo promedio & Tiempo promedio(seg.) & CME & \%G & \%GP \\ [0.5ex]
\hline
CMT1X & \bf{470.48} & 5.62 & 
472.98 & 3.68 & 470.48 & 0.00
 & 0.53\\CMT1Y & 472.37 & 5.05 & 
474.29 & 2.77 & \bf{470.48} & 
0.40 & 0.81\\CMT2X & \bf{682.39} & 5.82 & 
689.50 & 3.89 & 682.39 & 0.00
 & 1.04\\CMT2Y & 684.24 & 6.31 & 
687.14 & 6.57 & \bf{682.39} & 
0.27 & 0.70\\CMT3X & 724.57 & 13.69 & 
730.53 & 8.91 & \bf{719.06} & 
0.77 & 1.59\\CMT3Y & 722.79 & 11.81 & 
730.67 & 8.66 & \bf{719.06} & 
0.52 & 1.62\\CMT4X & 855.29 & 28.58 & 
867.89 & 23.52 & \bf{854.21} & 
0.13 & 1.60\\CMT4Y & 877.35 & 24.66 & 
880.67 & 27.75 & \bf{852.46} & 
2.92 & 3.31\\CMT5X & 1046.64 & 73.74 & 
1073.74 & 60.48 & \bf{1030.56} & 
1.56 & 4.19\\CMT5Y & 1061.81 & 38.88 & 
1082.89 & 42.77 & \bf{1031.69} & 
2.92 & 4.96\\CMT11X & 876.61 & 38.18 & 
920.91 & 25.24 & \bf{831.09} & 
5.48 & 10.81\\CMT11Y & 885.13 & 30.67 & 
922.84 & 30.19 & \bf{829.85} & 
6.66 & 11.21\\CMT12X & 673.32 & 8.67 & 
679.36 & 6.93 & \bf{658.83} & 
2.20 & 3.12\\CMT12Y & 674.91 & 8.58 & 
687.42 & 6.46 & \bf{660.47} & 
2.19 & 4.08\\\bf{PROM.} & 
\bf{764.85} & \bf{21.45} & \bf{778.63} & \bf{18.42} & \bf{749.50} & \bf{1.86} & \bf{3.54}\\[1ex]\hline
\end{tabular}
\label{table:nonlin}
\end{table} \clearpage
\begin{table}[ht]
\caption{Resultados de la ejecución de la metaheurística GTS, utilizando instancias de SalhiNagy con la configuración -mni 3000 -lambda1 0.05 -lambda2 0.05 -tabu 11}
\centering
\small
\begin{tabular}{c c c c c c c c}
\hline\hline
Instancia & Costo mínimo & Tiempo(seg.) & Costo promedio & Tiempo promedio(seg.) & CME & \%G & \%GP \\ [0.5ex]
\hline
CMT1X & \bf{470.48} & 2.56 & 
471.47 & 2.86 & 470.48 & 0.00
 & 0.21\\CMT1Y & \bf{470.48} & 2.41 & 
471.47 & 3.97 & 470.48 & 0.00
 & 0.21\\CMT2X & 687.04 & 11.63 & 
689.59 & 6.68 & \bf{682.39} & 
0.68 & 1.05\\CMT2Y & 687.92 & 3.22 & 
695.22 & 4.17 & \bf{682.39} & 
0.81 & 1.88\\CMT3X & 723.67 & 12.98 & 
732.23 & 7.43 & \bf{719.06} & 
0.64 & 1.83\\CMT3Y & 726.09 & 9.41 & 
732.08 & 7.82 & \bf{719.06} & 
0.98 & 1.81\\CMT4X & 864.36 & 47.96 & 
871.32 & 39.09 & \bf{854.21} & 
1.19 & 2.00\\CMT4Y & 871.57 & 21.97 & 
881.65 & 24.22 & \bf{852.46} & 
2.24 & 3.42\\CMT5X & 1052.38 & 58.45 & 
1058.57 & 64.71 & \bf{1030.56} & 
2.12 & 2.72\\CMT5Y & 1050.95 & 59.78 & 
1078.51 & 51.32 & \bf{1031.69} & 
1.87 & 4.54\\CMT11X & 902.76 & 63.93 & 
922.91 & 34.97 & \bf{831.09} & 
8.62 & 11.05\\CMT11Y & 893.70 & 18.24 & 
924.35 & 13.82 & \bf{829.85} & 
7.69 & 11.39\\CMT12X & 669.63 & 8.39 & 
673.67 & 10.35 & \bf{658.83} & 
1.64 & 2.25\\CMT12Y & 674.44 & 9.53 & 
682.89 & 9.12 & \bf{660.47} & 
2.12 & 3.40\\\bf{PROM.} & 
\bf{767.53} & \bf{23.60} & \bf{777.57} & \bf{20.04} & \bf{749.50} & \bf{2.19} & \bf{3.41}\\[1ex]\hline
\end{tabular}
\label{table:nonlin}
\end{table} \clearpage
\begin{table}[ht]
\caption{Resultados de la ejecución de la metaheurística GTS, utilizando instancias de SalhiNagy con la configuración -mni 3000 -lambda1 0.05 -lambda2 0.05 -tabu 13}
\centering
\small
\begin{tabular}{c c c c c c c c}
\hline\hline
Instancia & Costo mínimo & Tiempo(seg.) & Costo promedio & Tiempo promedio(seg.) & CME & \%G & \%GP \\ [0.5ex]
\hline
CMT1X & \bf{470.48} & 2.49 & 
474.49 & 2.33 & 470.48 & 0.00
 & 0.85\\CMT1Y & \bf{470.48} & 1.14 & 
474.14 & 2.53 & 470.48 & 0.00
 & 0.78\\CMT2X & 682.97 & 9.30 & 
689.71 & 6.97 & \bf{682.39} & 
0.08 & 1.07\\CMT2Y & 685.83 & 3.53 & 
692.28 & 4.19 & \bf{682.39} & 
0.50 & 1.45\\CMT3X & 727.10 & 5.56 & 
731.30 & 4.82 & \bf{719.06} & 
1.12 & 1.70\\CMT3Y & 725.24 & 3.82 & 
734.83 & 6.14 & \bf{719.06} & 
0.86 & 2.19\\CMT4X & 857.10 & 44.51 & 
880.01 & 29.54 & \bf{854.21} & 
0.34 & 3.02\\CMT4Y & 864.65 & 30.39 & 
878.79 & 21.78 & \bf{852.46} & 
1.43 & 3.09\\CMT5X & 1042.41 & 59.10 & 
1076.56 & 51.16 & \bf{1030.56} & 
1.15 & 4.46\\CMT5Y & 1051.63 & 112.91 & 
1064.77 & 62.12 & \bf{1031.69} & 
1.93 & 3.21\\CMT11X & 874.16 & 30.02 & 
927.10 & 17.56 & \bf{831.09} & 
5.18 & 11.55\\CMT11Y & 877.73 & 13.89 & 
890.64 & 20.76 & \bf{829.85} & 
5.77 & 7.33\\CMT12X & 670.63 & 11.53 & 
682.35 & 7.75 & \bf{658.83} & 
1.79 & 3.57\\CMT12Y & 673.80 & 10.35 & 
677.40 & 8.68 & \bf{660.47} & 
2.02 & 2.56\\\bf{PROM.} & 
\bf{762.44} & \bf{24.18} & \bf{776.74} & \bf{17.60} & \bf{749.50} & \bf{1.58} & \bf{3.35}\\[1ex]\hline
\end{tabular}
\label{table:nonlin}
\end{table} \clearpage
\begin{table}[ht]
\caption{Resultados de la ejecución de la metaheurística GTS, utilizando instancias de SalhiNagy con la configuración -mni 3000 -lambda1 0.05 -lambda2 0.05 -tabu 15}
\centering
\small
\begin{tabular}{c c c c c c c c}
\hline\hline
Instancia & Costo mínimo & Tiempo(seg.) & Costo promedio & Tiempo promedio(seg.) & CME & \%G & \%GP \\ [0.5ex]
\hline
CMT1X & 472.37 & 2.56 & 
477.53 & 2.28 & \bf{470.48} & 
0.40 & 1.50\\CMT1Y & 472.37 & 3.96 & 
472.44 & 2.26 & \bf{470.48} & 
0.40 & 0.42\\CMT2X & 684.40 & 7.54 & 
689.27 & 5.72 & \bf{682.39} & 
0.29 & 1.01\\CMT2Y & 683.73 & 6.74 & 
688.16 & 5.71 & \bf{682.39} & 
0.20 & 0.85\\CMT3X & 729.59 & 5.19 & 
731.65 & 7.85 & \bf{719.06} & 
1.46 & 1.75\\CMT3Y & 726.28 & 7.32 & 
732.12 & 8.05 & \bf{719.06} & 
1.00 & 1.82\\CMT4X & 863.20 & 26.74 & 
873.50 & 25.85 & \bf{854.21} & 
1.05 & 2.26\\CMT4Y & 881.85 & 15.88 & 
884.73 & 24.55 & \bf{852.46} & 
3.45 & 3.79\\CMT5X & 1041.91 & 117.48 & 
1060.57 & 82.27 & \bf{1030.56} & 
1.10 & 2.91\\CMT5Y & 1053.30 & 44.89 & 
1070.55 & 65.21 & \bf{1031.69} & 
2.09 & 3.77\\CMT11X & 916.05 & 26.93 & 
923.23 & 15.67 & \bf{831.09} & 
10.22 & 11.09\\CMT11Y & 882.86 & 23.12 & 
902.59 & 26.96 & \bf{829.85} & 
6.39 & 8.77\\CMT12X & 673.13 & 9.69 & 
690.93 & 8.98 & \bf{658.83} & 
2.17 & 4.87\\CMT12Y & 680.19 & 4.04 & 
700.43 & 8.52 & \bf{660.47} & 
2.99 & 6.05\\\bf{PROM.} & 
\bf{768.66} & \bf{21.58} & \bf{778.41} & \bf{20.70} & \bf{749.50} & \bf{2.37} & \bf{3.63}\\[1ex]\hline
\end{tabular}
\label{table:nonlin}
\end{table} \clearpage
\begin{table}[ht]
\caption{Resultados de la ejecución de la metaheurística GTS, utilizando instancias de SalhiNagy con la configuración -mni 3500 -lambda1 0.05 -lambda2 0.05 -tabu 5}
\centering
\small
\begin{tabular}{c c c c c c c c}
\hline\hline
Instancia & Costo mínimo & Tiempo(seg.) & Costo promedio & Tiempo promedio(seg.) & CME & \%G & \%GP \\ [0.5ex]
\hline
CMT1X & 472.37 & 2.44 & 
472.37 & 2.89 & \bf{470.48} & 
0.40 & 0.40\\CMT1Y & \bf{470.48} & 4.64 & 
474.81 & 3.10 & 470.48 & 0.00
 & 0.92\\CMT2X & 685.20 & 5.83 & 
689.73 & 4.93 & \bf{682.39} & 
0.41 & 1.08\\CMT2Y & 691.00 & 5.97 & 
695.24 & 4.07 & \bf{682.39} & 
1.26 & 1.88\\CMT3X & 726.61 & 10.96 & 
729.08 & 8.97 & \bf{719.06} & 
1.05 & 1.39\\CMT3Y & 731.78 & 7.17 & 
735.98 & 9.68 & \bf{719.06} & 
1.77 & 2.35\\CMT4X & 869.72 & 41.16 & 
882.75 & 29.32 & \bf{854.21} & 
1.82 & 3.34\\CMT4Y & 881.11 & 36.19 & 
894.57 & 26.36 & \bf{852.46} & 
3.36 & 4.94\\CMT5X & 1061.98 & 143.79 & 
1082.32 & 67.61 & \bf{1030.56} & 
3.05 & 5.02\\CMT5Y & 1043.60 & 71.61 & 
1057.84 & 72.68 & \bf{1031.69} & 
1.15 & 2.53\\CMT11X & 946.11 & 20.24 & 
960.66 & 31.95 & \bf{831.09} & 
13.84 & 15.59\\CMT11Y & 882.45 & 47.51 & 
904.72 & 22.02 & \bf{829.85} & 
6.34 & 9.02\\CMT12X & 670.54 & 6.78 & 
678.17 & 9.24 & \bf{658.83} & 
1.78 & 2.94\\CMT12Y & 670.35 & 7.26 & 
675.91 & 8.95 & \bf{660.47} & 
1.50 & 2.34\\\bf{PROM.} & 
\bf{771.66} & \bf{29.40} & \bf{781.01} & \bf{21.55} & \bf{749.50} & \bf{2.69} & \bf{3.84}\\[1ex]\hline
\end{tabular}
\label{table:nonlin}
\end{table} \clearpage
\begin{table}[ht]
\caption{Resultados de la ejecución de la metaheurística GTS, utilizando instancias de SalhiNagy con la configuración -mni 3500 -lambda1 0.05 -lambda2 0.05 -tabu 7}
\centering
\small
\begin{tabular}{c c c c c c c c}
\hline\hline
Instancia & Costo mínimo & Tiempo(seg.) & Costo promedio & Tiempo promedio(seg.) & CME & \%G & \%GP \\ [0.5ex]
\hline
CMT1X & \bf{470.48} & 2.01 & 
471.43 & 3.79 & 470.48 & 0.00
 & 0.20\\CMT1Y & \bf{470.48} & 1.98 & 
476.62 & 2.71 & 470.48 & 0.00
 & 1.31\\CMT2X & 683.97 & 5.83 & 
688.78 & 5.71 & \bf{682.39} & 
0.23 & 0.94\\CMT2Y & 684.09 & 10.46 & 
691.68 & 5.79 & \bf{682.39} & 
0.25 & 1.36\\CMT3X & 724.07 & 7.02 & 
729.64 & 10.35 & \bf{719.06} & 
0.70 & 1.47\\CMT3Y & 726.28 & 6.42 & 
735.60 & 13.41 & \bf{719.06} & 
1.00 & 2.30\\CMT4X & 876.41 & 54.08 & 
883.40 & 35.14 & \bf{854.21} & 
2.60 & 3.42\\CMT4Y & 857.38 & 48.04 & 
876.79 & 31.43 & \bf{852.46} & 
0.58 & 2.85\\CMT5X & 1061.16 & 48.18 & 
1081.68 & 60.97 & \bf{1030.56} & 
2.97 & 4.96\\CMT5Y & 1047.44 & 56.19 & 
1075.80 & 53.12 & \bf{1031.69} & 
1.53 & 4.28\\CMT11X & 879.46 & 19.07 & 
948.98 & 29.30 & \bf{831.09} & 
5.82 & 14.18\\CMT11Y & 868.93 & 22.48 & 
935.27 & 25.68 & \bf{829.85} & 
4.71 & 12.70\\CMT12X & 669.27 & 27.09 & 
673.16 & 12.12 & \bf{658.83} & 
1.58 & 2.17\\CMT12Y & 674.77 & 7.02 & 
681.83 & 8.41 & \bf{660.47} & 
2.17 & 3.23\\\bf{PROM.} & 
\bf{763.87} & \bf{22.56} & \bf{782.19} & \bf{21.28} & \bf{749.50} & \bf{1.72} & \bf{3.96}\\[1ex]\hline
\end{tabular}
\label{table:nonlin}
\end{table} \clearpage
\begin{table}[ht]
\caption{Resultados de la ejecución de la metaheurística GTS, utilizando instancias de SalhiNagy con la configuración -mni 3500 -lambda1 0.05 -lambda2 0.05 -tabu 9}
\centering
\small
\begin{tabular}{c c c c c c c c}
\hline\hline
Instancia & Costo mínimo & Tiempo(seg.) & Costo promedio & Tiempo promedio(seg.) & CME & \%G & \%GP \\ [0.5ex]
\hline
CMT1X & \bf{470.48} & 3.32 & 
471.43 & 3.21 & 470.48 & 0.00
 & 0.20\\CMT1Y & 470.67 & 3.73 & 
475.73 & 2.47 & \bf{470.48} & 
0.04 & 1.12\\CMT2X & 684.68 & 12.52 & 
688.17 & 6.15 & \bf{682.39} & 
0.34 & 0.85\\CMT2Y & 685.71 & 10.78 & 
691.43 & 7.10 & \bf{682.39} & 
0.49 & 1.33\\CMT3X & 726.28 & 8.06 & 
730.76 & 8.46 & \bf{719.06} & 
1.00 & 1.63\\CMT3Y & 724.57 & 14.34 & 
725.28 & 12.09 & \bf{719.06} & 
0.77 & 0.87\\CMT4X & 859.16 & 52.47 & 
875.56 & 47.95 & \bf{854.21} & 
0.58 & 2.50\\CMT4Y & 873.33 & 18.61 & 
880.39 & 27.67 & \bf{852.46} & 
2.45 & 3.28\\CMT5X & 1052.60 & 79.87 & 
1063.09 & 86.17 & \bf{1030.56} & 
2.14 & 3.16\\CMT5Y & 1065.55 & 59.37 & 
1071.49 & 46.20 & \bf{1031.69} & 
3.28 & 3.86\\CMT11X & 888.68 & 18.10 & 
903.05 & 21.86 & \bf{831.09} & 
6.93 & 8.66\\CMT11Y & 850.29 & 16.36 & 
888.42 & 18.25 & \bf{829.85} & 
2.46 & 7.06\\CMT12X & 673.13 & 8.12 & 
677.46 & 11.75 & \bf{658.83} & 
2.17 & 2.83\\CMT12Y & 674.11 & 10.82 & 
680.17 & 9.09 & \bf{660.47} & 
2.07 & 2.98\\\bf{PROM.} & 
\bf{764.23} & \bf{22.61} & \bf{773.03} & \bf{22.03} & \bf{749.50} & \bf{1.76} & \bf{2.88}\\[1ex]\hline
\end{tabular}
\label{table:nonlin}
\end{table} \clearpage
\begin{table}[ht]
\caption{Resultados de la ejecución de la metaheurística GTS, utilizando instancias de SalhiNagy con la configuración -mni 3500 -lambda1 0.05 -lambda2 0.05 -tabu 11}
\centering
\small
\begin{tabular}{c c c c c c c c}
\hline\hline
Instancia & Costo mínimo & Tiempo(seg.) & Costo promedio & Tiempo promedio(seg.) & CME & \%G & \%GP \\ [0.5ex]
\hline
CMT1X & 470.67 & 3.99 & 
476.90 & 2.96 & \bf{470.48} & 
0.04 & 1.36\\CMT1Y & \bf{470.48} & 4.10 & 
471.62 & 3.37 & 470.48 & 0.00
 & 0.24\\CMT2X & 685.70 & 5.66 & 
687.99 & 5.88 & \bf{682.39} & 
0.49 & 0.82\\CMT2Y & 686.60 & 4.02 & 
689.85 & 5.11 & \bf{682.39} & 
0.62 & 1.09\\CMT3X & 719.93 & 14.90 & 
729.82 & 9.63 & \bf{719.06} & 
0.12 & 1.50\\CMT3Y & 726.69 & 7.61 & 
729.03 & 7.08 & \bf{719.06} & 
1.06 & 1.39\\CMT4X & 861.17 & 70.38 & 
876.86 & 41.43 & \bf{854.21} & 
0.81 & 2.65\\CMT4Y & 873.84 & 25.86 & 
878.20 & 28.79 & \bf{852.46} & 
2.51 & 3.02\\CMT5X & 1056.40 & 44.93 & 
1069.55 & 70.49 & \bf{1030.56} & 
2.51 & 3.78\\CMT5Y & 1057.23 & 85.78 & 
1066.28 & 78.16 & \bf{1031.69} & 
2.48 & 3.35\\CMT11X & 873.79 & 13.08 & 
934.61 & 36.55 & \bf{831.09} & 
5.14 & 12.46\\CMT11Y & 900.86 & 22.66 & 
932.14 & 18.66 & \bf{829.85} & 
8.56 & 12.33\\CMT12X & 665.12 & 6.94 & 
676.98 & 10.39 & \bf{658.83} & 
0.95 & 2.75\\CMT12Y & 673.77 & 7.04 & 
681.31 & 8.97 & \bf{660.47} & 
2.01 & 3.16\\\bf{PROM.} & 
\bf{765.88} & \bf{22.64} & \bf{778.65} & \bf{23.39} & \bf{749.50} & \bf{1.95} & \bf{3.56}\\[1ex]\hline
\end{tabular}
\label{table:nonlin}
\end{table} \clearpage
\begin{table}[ht]
\caption{Resultados de la ejecución de la metaheurística GTS, utilizando instancias de SalhiNagy con la configuración -mni 3500 -lambda1 0.05 -lambda2 0.05 -tabu 13}
\centering
\small
\begin{tabular}{c c c c c c c c}
\hline\hline
Instancia & Costo mínimo & Tiempo(seg.) & Costo promedio & Tiempo promedio(seg.) & CME & \%G & \%GP \\ [0.5ex]
\hline
CMT1X & \bf{470.48} & 2.37 & 
471.43 & 3.45 & 470.48 & 0.00
 & 0.20\\CMT1Y & \bf{470.48} & 2.60 & 
472.12 & 2.09 & 470.48 & 0.00
 & 0.35\\CMT2X & 688.95 & 4.04 & 
690.37 & 4.79 & \bf{682.39} & 
0.96 & 1.17\\CMT2Y & 684.40 & 5.05 & 
688.42 & 4.72 & \bf{682.39} & 
0.29 & 0.88\\CMT3X & 723.85 & 6.09 & 
725.36 & 10.32 & \bf{719.06} & 
0.67 & 0.88\\CMT3Y & 724.57 & 6.45 & 
728.97 & 7.86 & \bf{719.06} & 
0.77 & 1.38\\CMT4X & 870.94 & 27.72 & 
885.72 & 21.00 & \bf{854.21} & 
1.96 & 3.69\\CMT4Y & 867.61 & 39.11 & 
878.12 & 37.02 & \bf{852.46} & 
1.78 & 3.01\\CMT5X & 1065.98 & 49.22 & 
1069.76 & 67.41 & \bf{1030.56} & 
3.44 & 3.80\\CMT5Y & 1062.34 & 108.05 & 
1077.29 & 81.06 & \bf{1031.69} & 
2.97 & 4.42\\CMT11X & 882.41 & 33.04 & 
915.61 & 23.40 & \bf{831.09} & 
6.18 & 10.17\\CMT11Y & 906.52 & 13.12 & 
932.83 & 23.75 & \bf{829.85} & 
9.24 & 12.41\\CMT12X & 663.30 & 9.32 & 
673.34 & 9.84 & \bf{658.83} & 
0.68 & 2.20\\CMT12Y & 673.77 & 11.26 & 
674.36 & 11.58 & \bf{660.47} & 
2.01 & 2.10\\\bf{PROM.} & 
\bf{768.26} & \bf{22.67} & \bf{777.41} & \bf{22.02} & \bf{749.50} & \bf{2.21} & \bf{3.33}\\[1ex]\hline
\end{tabular}
\label{table:nonlin}
\end{table} \clearpage
\begin{table}[ht]
\caption{Resultados de la ejecución de la metaheurística GTS, utilizando instancias de SalhiNagy con la configuración -mni 3500 -lambda1 0.05 -lambda2 0.05 -tabu 15}
\centering
\small
\begin{tabular}{c c c c c c c c}
\hline\hline
Instancia & Costo mínimo & Tiempo(seg.) & Costo promedio & Tiempo promedio(seg.) & CME & \%G & \%GP \\ [0.5ex]
\hline
CMT1X & \bf{470.48} & 1.68 & 
474.13 & 2.37 & 470.48 & 0.00
 & 0.78\\CMT1Y & \bf{470.48} & 4.79 & 
474.71 & 2.90 & 470.48 & 0.00
 & 0.90\\CMT2X & 687.23 & 4.24 & 
692.71 & 4.77 & \bf{682.39} & 
0.71 & 1.51\\CMT2Y & 684.39 & 5.78 & 
687.79 & 6.49 & \bf{682.39} & 
0.29 & 0.79\\CMT3X & 726.01 & 15.10 & 
730.15 & 9.73 & \bf{719.06} & 
0.97 & 1.54\\CMT3Y & 724.58 & 9.25 & 
732.90 & 6.46 & \bf{719.06} & 
0.77 & 1.92\\CMT4X & 866.47 & 24.16 & 
870.92 & 28.30 & \bf{854.21} & 
1.44 & 1.96\\CMT4Y & 861.94 & 16.22 & 
886.36 & 32.76 & \bf{852.46} & 
1.11 & 3.98\\CMT5X & 1076.17 & 113.88 & 
1091.48 & 60.63 & \bf{1030.56} & 
4.43 & 5.91\\CMT5Y & 1041.99 & 119.04 & 
1057.64 & 85.09 & \bf{1031.69} & 
1.00 & 2.52\\CMT11X & 945.39 & 10.19 & 
951.45 & 26.95 & \bf{831.09} & 
13.75 & 14.48\\CMT11Y & 890.61 & 25.23 & 
929.10 & 25.95 & \bf{829.85} & 
7.32 & 11.96\\CMT12X & 665.31 & 8.17 & 
671.33 & 9.13 & \bf{658.83} & 
0.98 & 1.90\\CMT12Y & 674.28 & 11.46 & 
683.85 & 12.68 & \bf{660.47} & 
2.09 & 3.54\\\bf{PROM.} & 
\bf{770.38} & \bf{26.37} & \bf{781.04} & \bf{22.44} & \bf{749.50} & \bf{2.49} & \bf{3.83}\\[1ex]\hline
\end{tabular}
\label{table:nonlin}
\end{table} \clearpage
\begin{table}[ht]
\caption{Resultados de la ejecución de la metaheurística GTS, utilizando instancias de SalhiNagy con la configuración -mni 4000 -lambda1 0.05 -lambda2 0.05 -tabu 5}
\centering
\small
\begin{tabular}{c c c c c c c c}
\hline\hline
Instancia & Costo mínimo & Tiempo(seg.) & Costo promedio & Tiempo promedio(seg.) & CME & \%G & \%GP \\ [0.5ex]
\hline
CMT1X & \bf{470.48} & 3.44 & 
475.69 & 2.38 & 470.48 & 0.00
 & 1.11\\CMT1Y & \bf{470.48} & 3.87 & 
474.14 & 3.93 & 470.48 & 0.00
 & 0.78\\CMT2X & 690.10 & 4.84 & 
693.05 & 5.72 & \bf{682.39} & 
1.13 & 1.56\\CMT2Y & 686.30 & 6.69 & 
689.07 & 5.21 & \bf{682.39} & 
0.57 & 0.98\\CMT3X & 724.40 & 16.42 & 
727.40 & 10.78 & \bf{719.06} & 
0.74 & 1.16\\CMT3Y & 728.79 & 9.78 & 
732.64 & 8.85 & \bf{719.06} & 
1.35 & 1.89\\CMT4X & 861.15 & 57.63 & 
884.01 & 41.10 & \bf{854.21} & 
0.81 & 3.49\\CMT4Y & 874.61 & 61.27 & 
893.23 & 41.89 & \bf{852.46} & 
2.60 & 4.78\\CMT5X & 1047.68 & 85.58 & 
1067.25 & 70.00 & \bf{1030.56} & 
1.66 & 3.56\\CMT5Y & 1059.42 & 59.50 & 
1069.40 & 81.55 & \bf{1031.69} & 
2.69 & 3.66\\CMT11X & 879.01 & 26.50 & 
896.05 & 29.10 & \bf{831.09} & 
5.77 & 7.82\\CMT11Y & 885.30 & 30.06 & 
909.18 & 25.61 & \bf{829.85} & 
6.68 & 9.56\\CMT12X & 664.45 & 9.28 & 
674.45 & 21.18 & \bf{658.83} & 
0.85 & 2.37\\CMT12Y & 664.00 & 20.78 & 
679.61 & 14.72 & \bf{660.47} & 
0.53 & 2.90\\\bf{PROM.} & 
\bf{764.73} & \bf{28.26} & \bf{776.08} & \bf{25.86} & \bf{749.50} & \bf{1.81} & \bf{3.26}\\[1ex]\hline
\end{tabular}
\label{table:nonlin}
\end{table} \clearpage
\begin{table}[ht]
\caption{Resultados de la ejecución de la metaheurística GTS, utilizando instancias de SalhiNagy con la configuración -mni 4000 -lambda1 0.05 -lambda2 0.05 -tabu 7}
\centering
\small
\begin{tabular}{c c c c c c c c}
\hline\hline
Instancia & Costo mínimo & Tiempo(seg.) & Costo promedio & Tiempo promedio(seg.) & CME & \%G & \%GP \\ [0.5ex]
\hline
CMT1X & \bf{470.48} & 3.54 & 
472.02 & 3.00 & 470.48 & 0.00
 & 0.33\\CMT1Y & 472.37 & 3.16 & 
475.08 & 2.81 & \bf{470.48} & 
0.40 & 0.98\\CMT2X & 685.71 & 13.97 & 
691.41 & 7.80 & \bf{682.39} & 
0.49 & 1.32\\CMT2Y & 684.29 & 6.23 & 
687.28 & 5.98 & \bf{682.39} & 
0.28 & 0.72\\CMT3X & 725.72 & 21.02 & 
728.05 & 13.01 & \bf{719.06} & 
0.93 & 1.25\\CMT3Y & 723.86 & 10.30 & 
728.86 & 9.21 & \bf{719.06} & 
0.67 & 1.36\\CMT4X & 867.90 & 47.43 & 
882.21 & 45.70 & \bf{854.21} & 
1.60 & 3.28\\CMT4Y & 861.25 & 35.85 & 
872.14 & 30.93 & \bf{852.46} & 
1.03 & 2.31\\CMT5X & 1047.60 & 156.06 & 
1061.73 & 103.15 & \bf{1030.56} & 
1.65 & 3.02\\CMT5Y & 1056.59 & 72.76 & 
1073.75 & 69.97 & \bf{1031.69} & 
2.41 & 4.08\\CMT11X & 877.89 & 22.90 & 
917.00 & 25.96 & \bf{831.09} & 
5.63 & 10.34\\CMT11Y & 886.11 & 40.59 & 
900.69 & 37.39 & \bf{829.85} & 
6.78 & 8.54\\CMT12X & 686.18 & 13.47 & 
692.84 & 10.97 & \bf{658.83} & 
4.15 & 5.16\\CMT12Y & 674.75 & 10.31 & 
682.54 & 9.29 & \bf{660.47} & 
2.16 & 3.34\\\bf{PROM.} & 
\bf{765.76} & \bf{32.68} & \bf{776.12} & \bf{26.80} & \bf{749.50} & \bf{2.01} & \bf{3.29}\\[1ex]\hline
\end{tabular}
\label{table:nonlin}
\end{table} \clearpage
\begin{table}[ht]
\caption{Resultados de la ejecución de la metaheurística GTS, utilizando instancias de SalhiNagy con la configuración -mni 4000 -lambda1 0.05 -lambda2 0.05 -tabu 9}
\centering
\small
\begin{tabular}{c c c c c c c c}
\hline\hline
Instancia & Costo mínimo & Tiempo(seg.) & Costo promedio & Tiempo promedio(seg.) & CME & \%G & \%GP \\ [0.5ex]
\hline
CMT1X & \bf{470.48} & 2.24 & 
471.90 & 3.76 & 470.48 & 0.00
 & 0.30\\CMT1Y & \bf{470.48} & 1.97 & 
472.01 & 3.08 & 470.48 & 0.00
 & 0.33\\CMT2X & 687.04 & 8.79 & 
692.28 & 8.43 & \bf{682.39} & 
0.68 & 1.45\\CMT2Y & 684.33 & 3.57 & 
690.77 & 4.91 & \bf{682.39} & 
0.28 & 1.23\\CMT3X & 721.74 & 7.20 & 
729.71 & 8.47 & \bf{719.06} & 
0.37 & 1.48\\CMT3Y & 726.69 & 10.00 & 
731.17 & 12.87 & \bf{719.06} & 
1.06 & 1.68\\CMT4X & 868.50 & 42.55 & 
887.78 & 32.02 & \bf{854.21} & 
1.67 & 3.93\\CMT4Y & 876.98 & 33.01 & 
886.75 & 42.09 & \bf{852.46} & 
2.88 & 4.02\\CMT5X & 1065.37 & 65.42 & 
1093.76 & 67.42 & \bf{1030.56} & 
3.38 & 6.13\\CMT5Y & 1042.89 & 103.95 & 
1061.62 & 66.61 & \bf{1031.69} & 
1.09 & 2.90\\CMT11X & 957.61 & 25.92 & 
967.61 & 26.48 & \bf{831.09} & 
15.22 & 16.43\\CMT11Y & 885.83 & 25.20 & 
918.35 & 25.55 & \bf{829.85} & 
6.75 & 10.66\\CMT12X & 673.59 & 7.56 & 
677.88 & 14.65 & \bf{658.83} & 
2.24 & 2.89\\CMT12Y & 675.22 & 7.58 & 
681.43 & 13.60 & \bf{660.47} & 
2.23 & 3.17\\\bf{PROM.} & 
\bf{771.91} & \bf{24.64} & \bf{783.07} & \bf{23.57} & \bf{749.50} & \bf{2.70} & \bf{4.04}\\[1ex]\hline
\end{tabular}
\label{table:nonlin}
\end{table} \clearpage
\begin{table}[ht]
\caption{Resultados de la ejecución de la metaheurística GTS, utilizando instancias de SalhiNagy con la configuración -mni 4000 -lambda1 0.05 -lambda2 0.05 -tabu 11}
\centering
\small
\begin{tabular}{c c c c c c c c}
\hline\hline
Instancia & Costo mínimo & Tiempo(seg.) & Costo promedio & Tiempo promedio(seg.) & CME & \%G & \%GP \\ [0.5ex]
\hline
CMT1X & \bf{470.48} & 5.26 & 
471.90 & 3.28 & 470.48 & 0.00
 & 0.30\\CMT1Y & 470.58 & 1.65 & 
474.35 & 3.00 & \bf{470.48} & 
0.02 & 0.82\\CMT2X & 684.40 & 6.77 & 
687.37 & 6.84 & \bf{682.39} & 
0.29 & 0.73\\CMT2Y & \bf{682.39} & 5.94 & 
686.22 & 4.29 & 682.39 & 0.00
 & 0.56\\CMT3X & 729.05 & 13.14 & 
731.76 & 11.21 & \bf{719.06} & 
1.39 & 1.77\\CMT3Y & \bf{\underline{718.40}} & 13.57 & 
724.09 & 14.30 & 719.06 & 
\bf{-0.09} & 0.70\\CMT4X & 865.17 & 98.07 & 
879.86 & 48.98 & \bf{854.21} & 
1.28 & 3.00\\CMT4Y & 862.92 & 30.79 & 
866.83 & 35.14 & \bf{852.46} & 
1.23 & 1.69\\CMT5X & 1070.29 & 79.19 & 
1079.06 & 58.72 & \bf{1030.56} & 
3.86 & 4.71\\CMT5Y & 1047.73 & 80.88 & 
1068.25 & 65.20 & \bf{1031.69} & 
1.55 & 3.54\\CMT11X & 898.16 & 23.65 & 
939.05 & 38.73 & \bf{831.09} & 
8.07 & 12.99\\CMT11Y & 881.52 & 21.94 & 
891.21 & 32.58 & \bf{829.85} & 
6.23 & 7.39\\CMT12X & 678.25 & 15.78 & 
680.15 & 11.28 & \bf{658.83} & 
2.95 & 3.24\\CMT12Y & 674.11 & 10.71 & 
679.59 & 15.07 & \bf{660.47} & 
2.07 & 2.89\\\bf{PROM.} & 
\bf{766.68} & \bf{29.10} & \bf{775.69} & \bf{24.90} & \bf{749.50} & \bf{2.06} & \bf{3.17}\\[1ex]\hline
\end{tabular}
\label{table:nonlin}
\end{table} \clearpage
\begin{table}[ht]
\caption{Resultados de la ejecución de la metaheurística GTS, utilizando instancias de SalhiNagy con la configuración -mni 4000 -lambda1 0.05 -lambda2 0.05 -tabu 13}
\centering
\small
\begin{tabular}{c c c c c c c c}
\hline\hline
Instancia & Costo mínimo & Tiempo(seg.) & Costo promedio & Tiempo promedio(seg.) & CME & \%G & \%GP \\ [0.5ex]
\hline
CMT1X & \bf{470.48} & 1.76 & 
470.53 & 3.81 & 470.48 & 0.00
 & 0.01\\CMT1Y & \bf{470.48} & 2.77 & 
474.61 & 3.75 & 470.48 & 0.00
 & 0.88\\CMT2X & \bf{682.39} & 7.82 & 
687.79 & 6.12 & 682.39 & 0.00
 & 0.79\\CMT2Y & 689.15 & 3.82 & 
693.45 & 5.44 & \bf{682.39} & 
0.99 & 1.62\\CMT3X & 721.95 & 6.68 & 
728.42 & 9.52 & \bf{719.06} & 
0.40 & 1.30\\CMT3Y & 728.26 & 12.66 & 
729.87 & 11.02 & \bf{719.06} & 
1.28 & 1.50\\CMT4X & 864.55 & 39.37 & 
874.61 & 41.56 & \bf{854.21} & 
1.21 & 2.39\\CMT4Y & 863.12 & 28.41 & 
867.22 & 43.16 & \bf{852.46} & 
1.25 & 1.73\\CMT5X & 1038.01 & 69.30 & 
1063.00 & 74.47 & \bf{1030.56} & 
0.72 & 3.15\\CMT5Y & 1064.78 & 81.50 & 
1071.88 & 107.81 & \bf{1031.69} & 
3.21 & 3.90\\CMT11X & 876.80 & 18.92 & 
932.45 & 23.95 & \bf{831.09} & 
5.50 & 12.20\\CMT11Y & 871.87 & 27.88 & 
940.12 & 25.32 & \bf{829.85} & 
5.06 & 13.29\\CMT12X & 672.40 & 13.33 & 
673.34 & 14.08 & \bf{658.83} & 
2.06 & 2.20\\CMT12Y & 674.11 & 8.47 & 
680.16 & 9.22 & \bf{660.47} & 
2.07 & 2.98\\\bf{PROM.} & 
\bf{763.45} & \bf{23.05} & \bf{777.67} & \bf{27.09} & \bf{749.50} & \bf{1.70} & \bf{3.42}\\[1ex]\hline
\end{tabular}
\label{table:nonlin}
\end{table} \clearpage
\begin{table}[ht]
\caption{Resultados de la ejecución de la metaheurística GTS, utilizando instancias de SalhiNagy con la configuración -mni 4000 -lambda1 0.05 -lambda2 0.05 -tabu 15}
\centering
\small
\begin{tabular}{c c c c c c c c}
\hline\hline
Instancia & Costo mínimo & Tiempo(seg.) & Costo promedio & Tiempo promedio(seg.) & CME & \%G & \%GP \\ [0.5ex]
\hline
CMT1X & \bf{470.48} & 3.52 & 
470.95 & 3.02 & 470.48 & 0.00
 & 0.10\\CMT1Y & \bf{470.48} & 3.80 & 
471.00 & 4.16 & 470.48 & 0.00
 & 0.11\\CMT2X & 683.64 & 4.64 & 
689.61 & 5.46 & \bf{682.39} & 
0.18 & 1.06\\CMT2Y & 684.68 & 10.37 & 
688.66 & 6.27 & \bf{682.39} & 
0.34 & 0.92\\CMT3X & 726.12 & 11.46 & 
729.61 & 8.45 & \bf{719.06} & 
0.98 & 1.47\\CMT3Y & \bf{\underline{718.40}} & 12.77 & 
728.16 & 10.66 & 719.06 & 
\bf{-0.09} & 1.27\\CMT4X & 869.90 & 59.42 & 
878.73 & 45.28 & \bf{854.21} & 
1.84 & 2.87\\CMT4Y & 875.39 & 27.20 & 
895.16 & 26.55 & \bf{852.46} & 
2.69 & 5.01\\CMT5X & 1054.48 & 100.03 & 
1071.06 & 85.13 & \bf{1030.56} & 
2.32 & 3.93\\CMT5Y & 1049.78 & 54.02 & 
1070.98 & 67.89 & \bf{1031.69} & 
1.75 & 3.81\\CMT11X & 885.67 & 33.14 & 
927.08 & 34.49 & \bf{831.09} & 
6.57 & 11.55\\CMT11Y & 863.11 & 36.41 & 
932.22 & 26.25 & \bf{829.85} & 
4.01 & 12.34\\CMT12X & 670.19 & 5.74 & 
674.81 & 9.74 & \bf{658.83} & 
1.72 & 2.43\\CMT12Y & 669.56 & 21.28 & 
676.69 & 14.97 & \bf{660.47} & 
1.38 & 2.46\\\bf{PROM.} & 
\bf{763.71} & \bf{27.41} & \bf{778.91} & \bf{24.88} & \bf{749.50} & \bf{1.69} & \bf{3.52}\\[1ex]\hline
\end{tabular}
\label{table:nonlin}
\end{table} \clearpage
\begin{table}[ht]
\caption{Resultados de la ejecución de la metaheurística GTS, utilizando instancias de SalhiNagy con la configuración -mni 4500 -lambda1 0.05 -lambda2 0.05 -tabu 5}
\centering
\small
\begin{tabular}{c c c c c c c c}
\hline\hline
Instancia & Costo mínimo & Tiempo(seg.) & Costo promedio & Tiempo promedio(seg.) & CME & \%G & \%GP \\ [0.5ex]
\hline
CMT1X & \bf{470.48} & 2.33 & 
471.43 & 3.46 & 470.48 & 0.00
 & 0.20\\CMT1Y & \bf{470.48} & 2.87 & 
473.66 & 3.39 & 470.48 & 0.00
 & 0.68\\CMT2X & 685.91 & 9.35 & 
688.82 & 7.81 & \bf{682.39} & 
0.52 & 0.94\\CMT2Y & 688.86 & 12.91 & 
691.46 & 11.40 & \bf{682.39} & 
0.95 & 1.33\\CMT3X & 719.42 & 10.89 & 
727.20 & 10.66 & \bf{719.06} & 
0.05 & 1.13\\CMT3Y & 728.79 & 9.35 & 
732.93 & 11.64 & \bf{719.06} & 
1.35 & 1.93\\CMT4X & 868.23 & 41.46 & 
882.35 & 33.40 & \bf{854.21} & 
1.64 & 3.29\\CMT4Y & 863.65 & 40.67 & 
875.58 & 29.56 & \bf{852.46} & 
1.31 & 2.71\\CMT5X & 1067.82 & 74.15 & 
1076.10 & 73.39 & \bf{1030.56} & 
3.62 & 4.42\\CMT5Y & 1053.83 & 69.34 & 
1059.39 & 79.14 & \bf{1031.69} & 
2.15 & 2.68\\CMT11X & 877.69 & 32.09 & 
908.77 & 31.63 & \bf{831.09} & 
5.61 & 9.35\\CMT11Y & 889.10 & 31.38 & 
923.82 & 37.29 & \bf{829.85} & 
7.14 & 11.32\\CMT12X & 673.59 & 12.83 & 
687.81 & 8.82 & \bf{658.83} & 
2.24 & 4.40\\CMT12Y & 673.59 & 11.99 & 
681.15 & 9.01 & \bf{660.47} & 
1.99 & 3.13\\\bf{PROM.} & 
\bf{766.53} & \bf{25.83} & \bf{777.18} & \bf{25.04} & \bf{749.50} & \bf{2.04} & \bf{3.39}\\[1ex]\hline
\end{tabular}
\label{table:nonlin}
\end{table} \clearpage
\begin{table}[ht]
\caption{Resultados de la ejecución de la metaheurística GTS, utilizando instancias de SalhiNagy con la configuración -mni 4500 -lambda1 0.05 -lambda2 0.05 -tabu 7}
\centering
\small
\begin{tabular}{c c c c c c c c}
\hline\hline
Instancia & Costo mínimo & Tiempo(seg.) & Costo promedio & Tiempo promedio(seg.) & CME & \%G & \%GP \\ [0.5ex]
\hline
CMT1X & \bf{470.48} & 2.96 & 
471.47 & 4.67 & 470.48 & 0.00
 & 0.21\\CMT1Y & \bf{470.48} & 6.94 & 
471.00 & 4.33 & 470.48 & 0.00
 & 0.11\\CMT2X & 684.40 & 7.38 & 
687.53 & 5.15 & \bf{682.39} & 
0.29 & 0.75\\CMT2Y & 687.63 & 7.38 & 
688.17 & 6.96 & \bf{682.39} & 
0.77 & 0.85\\CMT3X & 722.97 & 26.97 & 
730.64 & 14.60 & \bf{719.06} & 
0.54 & 1.61\\CMT3Y & \bf{\underline{718.40}} & 19.72 & 
724.44 & 12.46 & 719.06 & 
\bf{-0.09} & 0.75\\CMT4X & 856.21 & 47.99 & 
876.90 & 56.68 & \bf{854.21} & 
0.23 & 2.66\\CMT4Y & 866.42 & 57.68 & 
870.37 & 49.19 & \bf{852.46} & 
1.64 & 2.10\\CMT5X & 1043.80 & 100.96 & 
1055.47 & 102.42 & \bf{1030.56} & 
1.28 & 2.42\\CMT5Y & 1049.91 & 102.59 & 
1072.80 & 80.53 & \bf{1031.69} & 
1.77 & 3.98\\CMT11X & 892.97 & 17.48 & 
939.88 & 25.72 & \bf{831.09} & 
7.45 & 13.09\\CMT11Y & 880.87 & 26.20 & 
920.57 & 37.03 & \bf{829.85} & 
6.15 & 10.93\\CMT12X & 663.61 & 11.18 & 
673.44 & 13.28 & \bf{658.83} & 
0.73 & 2.22\\CMT12Y & 674.62 & 8.34 & 
676.65 & 18.23 & \bf{660.47} & 
2.14 & 2.45\\\bf{PROM.} & 
\bf{763.06} & \bf{31.70} & \bf{775.67} & \bf{30.80} & \bf{749.50} & \bf{1.64} & \bf{3.15}\\[1ex]\hline
\end{tabular}
\label{table:nonlin}
\end{table} \clearpage
\begin{table}[ht]
\caption{Resultados de la ejecución de la metaheurística GTS, utilizando instancias de SalhiNagy con la configuración -mni 4500 -lambda1 0.05 -lambda2 0.05 -tabu 9}
\centering
\small
\begin{tabular}{c c c c c c c c}
\hline\hline
Instancia & Costo mínimo & Tiempo(seg.) & Costo promedio & Tiempo promedio(seg.) & CME & \%G & \%GP \\ [0.5ex]
\hline
CMT1X & \bf{470.48} & 2.15 & 
470.53 & 3.49 & 470.48 & 0.00
 & 0.01\\CMT1Y & \bf{470.48} & 5.51 & 
471.45 & 4.10 & 470.48 & 0.00
 & 0.21\\CMT2X & \bf{682.39} & 9.83 & 
688.36 & 7.56 & 682.39 & 0.00
 & 0.87\\CMT2Y & 685.12 & 8.71 & 
688.59 & 7.94 & \bf{682.39} & 
0.40 & 0.91\\CMT3X & 721.09 & 9.78 & 
728.75 & 11.32 & \bf{719.06} & 
0.28 & 1.35\\CMT3Y & \bf{719.06} & 14.79 & 
723.48 & 11.28 & 719.06 & 0.00
 & 0.61\\CMT4X & 879.08 & 52.11 & 
886.71 & 45.82 & \bf{854.21} & 
2.91 & 3.80\\CMT4Y & 865.70 & 66.86 & 
878.67 & 40.85 & \bf{852.46} & 
1.55 & 3.08\\CMT5X & 1058.36 & 71.07 & 
1065.22 & 80.38 & \bf{1030.56} & 
2.70 & 3.36\\CMT5Y & 1047.45 & 103.22 & 
1065.23 & 89.58 & \bf{1031.69} & 
1.53 & 3.25\\CMT11X & 941.80 & 26.65 & 
950.72 & 28.00 & \bf{831.09} & 
13.32 & 14.39\\CMT11Y & 878.16 & 29.35 & 
929.95 & 22.70 & \bf{829.85} & 
5.82 & 12.06\\CMT12X & 663.69 & 7.83 & 
673.66 & 10.87 & \bf{658.83} & 
0.74 & 2.25\\CMT12Y & 674.11 & 14.73 & 
678.45 & 15.64 & \bf{660.47} & 
2.07 & 2.72\\\bf{PROM.} & 
\bf{768.35} & \bf{30.18} & \bf{778.55} & \bf{27.11} & \bf{749.50} & \bf{2.24} & \bf{3.49}\\[1ex]\hline
\end{tabular}
\label{table:nonlin}
\end{table} \clearpage
\begin{table}[ht]
\caption{Resultados de la ejecución de la metaheurística GTS, utilizando instancias de SalhiNagy con la configuración -mni 4500 -lambda1 0.05 -lambda2 0.05 -tabu 11}
\centering
\small
\begin{tabular}{c c c c c c c c}
\hline\hline
Instancia & Costo mínimo & Tiempo(seg.) & Costo promedio & Tiempo promedio(seg.) & CME & \%G & \%GP \\ [0.5ex]
\hline
CMT1X & \bf{470.48} & 3.43 & 
471.43 & 3.94 & 470.48 & 0.00
 & 0.20\\CMT1Y & \bf{470.48} & 4.52 & 
472.56 & 4.11 & 470.48 & 0.00
 & 0.44\\CMT2X & 689.10 & 5.26 & 
693.80 & 4.46 & \bf{682.39} & 
0.98 & 1.67\\CMT2Y & \bf{682.39} & 6.74 & 
687.69 & 6.34 & 682.39 & 0.00
 & 0.78\\CMT3X & 721.37 & 11.06 & 
727.68 & 14.45 & \bf{719.06} & 
0.32 & 1.20\\CMT3Y & 724.57 & 20.29 & 
728.16 & 14.25 & \bf{719.06} & 
0.77 & 1.27\\CMT4X & 872.93 & 32.58 & 
884.45 & 30.40 & \bf{854.21} & 
2.19 & 3.54\\CMT4Y & 876.10 & 27.30 & 
884.67 & 34.55 & \bf{852.46} & 
2.77 & 3.78\\CMT5X & 1070.80 & 75.01 & 
1080.12 & 67.55 & \bf{1030.56} & 
3.90 & 4.81\\CMT5Y & 1047.44 & 39.44 & 
1063.11 & 107.78 & \bf{1031.69} & 
1.53 & 3.05\\CMT11X & 881.03 & 29.03 & 
913.05 & 35.90 & \bf{831.09} & 
6.01 & 9.86\\CMT11Y & 877.46 & 22.51 & 
906.40 & 27.59 & \bf{829.85} & 
5.74 & 9.22\\CMT12X & 669.34 & 8.33 & 
674.20 & 8.79 & \bf{658.83} & 
1.60 & 2.33\\CMT12Y & 673.59 & 12.50 & 
678.12 & 18.38 & \bf{660.47} & 
1.99 & 2.67\\\bf{PROM.} & 
\bf{766.22} & \bf{21.29} & \bf{776.10} & \bf{27.03} & \bf{749.50} & \bf{1.99} & \bf{3.20}\\[1ex]\hline
\end{tabular}
\label{table:nonlin}
\end{table} \clearpage
\begin{table}[ht]
\caption{Resultados de la ejecución de la metaheurística GTS, utilizando instancias de SalhiNagy con la configuración -mni 4500 -lambda1 0.05 -lambda2 0.05 -tabu 13}
\centering
\small
\begin{tabular}{c c c c c c c c}
\hline\hline
Instancia & Costo mínimo & Tiempo(seg.) & Costo promedio & Tiempo promedio(seg.) & CME & \%G & \%GP \\ [0.5ex]
\hline
CMT1X & \bf{470.48} & 5.44 & 
471.43 & 3.47 & 470.48 & 0.00
 & 0.20\\CMT1Y & \bf{470.48} & 3.21 & 
474.32 & 3.62 & 470.48 & 0.00
 & 0.82\\CMT2X & 692.10 & 7.20 & 
694.06 & 5.28 & \bf{682.39} & 
1.42 & 1.71\\CMT2Y & 686.75 & 7.38 & 
689.56 & 10.29 & \bf{682.39} & 
0.64 & 1.05\\CMT3X & 720.18 & 14.91 & 
725.53 & 10.70 & \bf{719.06} & 
0.16 & 0.90\\CMT3Y & 721.85 & 17.70 & 
726.68 & 17.01 & \bf{719.06} & 
0.39 & 1.06\\CMT4X & 867.97 & 29.76 & 
881.00 & 29.73 & \bf{854.21} & 
1.61 & 3.14\\CMT4Y & 865.66 & 91.75 & 
877.26 & 54.12 & \bf{852.46} & 
1.55 & 2.91\\CMT5X & 1061.05 & 70.50 & 
1077.73 & 88.74 & \bf{1030.56} & 
2.96 & 4.58\\CMT5Y & 1045.78 & 83.34 & 
1060.07 & 82.19 & \bf{1031.69} & 
1.37 & 2.75\\CMT11X & 879.24 & 11.53 & 
900.79 & 23.32 & \bf{831.09} & 
5.79 & 8.39\\CMT11Y & 882.99 & 56.31 & 
928.17 & 36.05 & \bf{829.85} & 
6.40 & 11.85\\CMT12X & 672.97 & 10.21 & 
692.06 & 10.02 & \bf{658.83} & 
2.15 & 5.04\\CMT12Y & 673.13 & 11.60 & 
680.26 & 8.21 & \bf{660.47} & 
1.92 & 3.00\\\bf{PROM.} & 
\bf{765.04} & \bf{30.06} & \bf{777.07} & \bf{27.34} & \bf{749.50} & \bf{1.88} & \bf{3.38}\\[1ex]\hline
\end{tabular}
\label{table:nonlin}
\end{table} \clearpage
\begin{table}[ht]
\caption{Resultados de la ejecución de la metaheurística GTS, utilizando instancias de SalhiNagy con la configuración -mni 4500 -lambda1 0.05 -lambda2 0.05 -tabu 15}
\centering
\small
\begin{tabular}{c c c c c c c c}
\hline\hline
Instancia & Costo mínimo & Tiempo(seg.) & Costo promedio & Tiempo promedio(seg.) & CME & \%G & \%GP \\ [0.5ex]
\hline
CMT1X & \bf{470.48} & 1.73 & 
470.48 & 4.54 & 470.48 & 0.00
 & 0.00\\
CMT1Y & \bf{470.48} & 7.55 & 
471.00 & 5.42 & 470.48 & 0.00
 & 0.11\\CMT2X & 684.40 & 8.33 & 
687.85 & 5.46 & \bf{682.39} & 
0.29 & 0.80\\CMT2Y & 683.95 & 10.92 & 
687.28 & 7.51 & \bf{682.39} & 
0.23 & 0.72\\CMT3X & 727.10 & 16.99 & 
731.89 & 15.06 & \bf{719.06} & 
1.12 & 1.78\\CMT3Y & 720.04 & 6.17 & 
727.08 & 11.84 & \bf{719.06} & 
0.14 & 1.12\\CMT4X & 867.43 & 63.91 & 
884.36 & 36.62 & \bf{854.21} & 
1.55 & 3.53\\CMT4Y & 862.38 & 21.70 & 
868.50 & 34.31 & \bf{852.46} & 
1.16 & 1.88\\CMT5X & 1071.02 & 106.22 & 
1075.72 & 80.47 & \bf{1030.56} & 
3.93 & 4.38\\CMT5Y & 1055.17 & 46.90 & 
1076.13 & 75.68 & \bf{1031.69} & 
2.28 & 4.31\\CMT11X & 920.00 & 24.07 & 
933.22 & 34.81 & \bf{831.09} & 
10.70 & 12.29\\CMT11Y & 880.49 & 47.39 & 
906.09 & 28.09 & \bf{829.85} & 
6.10 & 9.19\\CMT12X & 663.84 & 11.94 & 
673.92 & 15.53 & \bf{658.83} & 
0.76 & 2.29\\CMT12Y & 674.70 & 17.85 & 
679.79 & 14.45 & \bf{660.47} & 
2.15 & 2.93\\\bf{PROM.} & 
\bf{767.96} & \bf{27.98} & \bf{776.67} & \bf{26.41} & \bf{749.50} & \bf{2.17} & \bf{3.24}\\[1ex]\hline
\end{tabular}
\label{table:nonlin}
\end{table} \clearpage
\begin{table}[ht]
\caption{Resultados de la ejecución de la metaheurística GTS, utilizando instancias de SalhiNagy con la configuración -mni 5000 -lambda1 0.05 -lambda2 0.05 -tabu 5}
\centering
\small
\begin{tabular}{c c c c c c c c}
\hline\hline
Instancia & Costo mínimo & Tiempo(seg.) & Costo promedio & Tiempo promedio(seg.) & CME & \%G & \%GP \\ [0.5ex]
\hline
CMT1X & \bf{470.48} & 3.24 & 
474.14 & 3.19 & 470.48 & 0.00
 & 0.78\\CMT1Y & \bf{470.48} & 5.62 & 
472.26 & 5.56 & 470.48 & 0.00
 & 0.38\\CMT2X & 683.52 & 12.75 & 
688.38 & 8.01 & \bf{682.39} & 
0.17 & 0.88\\CMT2Y & \bf{682.39} & 7.22 & 
693.66 & 9.75 & 682.39 & 0.00
 & 1.65\\CMT3X & \bf{\underline{718.40}} & 14.93 & 
724.80 & 13.71 & 719.06 & 
\bf{-0.09} & 0.80\\CMT3Y & 726.28 & 11.27 & 
729.32 & 12.80 & \bf{719.06} & 
1.00 & 1.43\\CMT4X & 857.77 & 48.73 & 
880.70 & 43.04 & \bf{854.21} & 
0.42 & 3.10\\CMT4Y & 871.01 & 28.47 & 
883.92 & 31.14 & \bf{852.46} & 
2.18 & 3.69\\CMT5X & 1053.60 & 78.61 & 
1073.71 & 75.47 & \bf{1030.56} & 
2.24 & 4.19\\CMT5Y & 1065.33 & 89.55 & 
1084.73 & 86.14 & \bf{1031.69} & 
3.26 & 5.14\\CMT11X & 885.58 & 55.43 & 
914.03 & 37.53 & \bf{831.09} & 
6.56 & 9.98\\CMT11Y & 864.51 & 16.59 & 
897.43 & 25.27 & \bf{829.85} & 
4.18 & 8.14\\CMT12X & 672.47 & 22.52 & 
673.72 & 12.43 & \bf{658.83} & 
2.07 & 2.26\\CMT12Y & 673.59 & 14.16 & 
681.59 & 11.20 & \bf{660.47} & 
1.99 & 3.20\\\bf{PROM.} & 
\bf{763.96} & \bf{29.22} & \bf{776.60} & \bf{26.80} & \bf{749.50} & \bf{1.71} & \bf{3.26}\\[1ex]\hline
\end{tabular}
\label{table:nonlin}
\end{table} \clearpage
\begin{table}[ht]
\caption{Resultados de la ejecución de la metaheurística GTS, utilizando instancias de SalhiNagy con la configuración -mni 5000 -lambda1 0.05 -lambda2 0.05 -tabu 7}
\centering
\small
\begin{tabular}{c c c c c c c c}
\hline\hline
Instancia & Costo mínimo & Tiempo(seg.) & Costo promedio & Tiempo promedio(seg.) & CME & \%G & \%GP \\ [0.5ex]
\hline
CMT1X & \bf{470.48} & 3.61 & 
471.90 & 3.35 & 470.48 & 0.00
 & 0.30\\CMT1Y & \bf{470.48} & 4.54 & 
475.32 & 4.70 & 470.48 & 0.00
 & 1.03\\CMT2X & 683.95 & 8.94 & 
691.22 & 8.79 & \bf{682.39} & 
0.23 & 1.29\\CMT2Y & \bf{682.39} & 7.65 & 
687.70 & 8.89 & 682.39 & 0.00
 & 0.78\\CMT3X & 726.82 & 8.68 & 
728.42 & 10.80 & \bf{719.06} & 
1.08 & 1.30\\CMT3Y & 724.98 & 12.67 & 
729.70 & 14.69 & \bf{719.06} & 
0.82 & 1.48\\CMT4X & 858.31 & 63.20 & 
867.64 & 53.02 & \bf{854.21} & 
0.48 & 1.57\\CMT4Y & 860.72 & 57.22 & 
871.04 & 52.59 & \bf{852.46} & 
0.97 & 2.18\\CMT5X & 1058.11 & 118.02 & 
1067.01 & 115.84 & \bf{1030.56} & 
2.67 & 3.54\\CMT5Y & 1067.78 & 123.29 & 
1085.47 & 86.41 & \bf{1031.69} & 
3.50 & 5.21\\CMT11X & 886.98 & 22.30 & 
925.41 & 29.69 & \bf{831.09} & 
6.72 & 11.35\\CMT11Y & 874.65 & 24.79 & 
889.91 & 38.23 & \bf{829.85} & 
5.40 & 7.24\\CMT12X & 663.61 & 7.44 & 
673.43 & 8.89 & \bf{658.83} & 
0.73 & 2.22\\CMT12Y & 674.11 & 15.71 & 
704.88 & 11.14 & \bf{660.47} & 
2.07 & 6.72\\\bf{PROM.} & 
\bf{764.53} & \bf{34.15} & \bf{776.36} & \bf{31.93} & \bf{749.50} & \bf{1.76} & \bf{3.30}\\[1ex]\hline
\end{tabular}
\label{table:nonlin}
\end{table} \clearpage
\begin{table}[ht]
\caption{Resultados de la ejecución de la metaheurística GTS, utilizando instancias de SalhiNagy con la configuración -mni 5000 -lambda1 0.05 -lambda2 0.05 -tabu 9}
\centering
\small
\begin{tabular}{c c c c c c c c}
\hline\hline
Instancia & Costo mínimo & Tiempo(seg.) & Costo promedio & Tiempo promedio(seg.) & CME & \%G & \%GP \\ [0.5ex]
\hline
CMT1X & \bf{470.48} & 6.08 & 
471.43 & 3.79 & 470.48 & 0.00
 & 0.20\\CMT1Y & \bf{470.48} & 3.94 & 
471.43 & 3.73 & 470.48 & 0.00
 & 0.20\\CMT2X & 689.66 & 10.72 & 
693.38 & 6.88 & \bf{682.39} & 
1.07 & 1.61\\CMT2Y & 684.29 & 7.86 & 
688.11 & 8.63 & \bf{682.39} & 
0.28 & 0.84\\CMT3X & 725.87 & 20.51 & 
729.34 & 15.71 & \bf{719.06} & 
0.95 & 1.43\\CMT3Y & 731.36 & 6.14 & 
734.00 & 10.69 & \bf{719.06} & 
1.71 & 2.08\\CMT4X & 857.01 & 27.98 & 
872.93 & 36.82 & \bf{854.21} & 
0.33 & 2.19\\CMT4Y & 860.75 & 40.57 & 
869.14 & 48.82 & \bf{852.46} & 
0.97 & 1.96\\CMT5X & 1051.74 & 120.86 & 
1073.20 & 111.00 & \bf{1030.56} & 
2.06 & 4.14\\CMT5Y & 1043.91 & 83.04 & 
1066.48 & 80.86 & \bf{1031.69} & 
1.18 & 3.37\\CMT11X & 878.19 & 88.14 & 
904.35 & 43.62 & \bf{831.09} & 
5.67 & 8.81\\CMT11Y & 886.02 & 37.40 & 
924.81 & 30.47 & \bf{829.85} & 
6.77 & 11.44\\CMT12X & 669.86 & 17.71 & 
672.95 & 14.73 & \bf{658.83} & 
1.67 & 2.14\\CMT12Y & 673.64 & 22.67 & 
686.10 & 17.00 & \bf{660.47} & 
1.99 & 3.88\\\bf{PROM.} & 
\bf{763.80} & \bf{35.26} & \bf{775.55} & \bf{30.91} & \bf{749.50} & \bf{1.76} & \bf{3.16}\\[1ex]\hline
\end{tabular}
\label{table:nonlin}
\end{table} \clearpage
\begin{table}[ht]
\caption{Resultados de la ejecución de la metaheurística GTS, utilizando instancias de SalhiNagy con la configuración -mni 5000 -lambda1 0.05 -lambda2 0.05 -tabu 11}
\centering
\small
\begin{tabular}{c c c c c c c c}
\hline\hline
Instancia & Costo mínimo & Tiempo(seg.) & Costo promedio & Tiempo promedio(seg.) & CME & \%G & \%GP \\ [0.5ex]
\hline
CMT1X & 470.67 & 3.38 & 
471.62 & 3.53 & \bf{470.48} & 
0.04 & 0.24\\CMT1Y & \bf{470.48} & 2.27 & 
471.90 & 4.33 & 470.48 & 0.00
 & 0.30\\CMT2X & 687.27 & 7.34 & 
690.68 & 9.05 & \bf{682.39} & 
0.72 & 1.21\\CMT2Y & 685.67 & 4.62 & 
691.51 & 5.85 & \bf{682.39} & 
0.48 & 1.34\\CMT3X & 723.52 & 9.73 & 
728.49 & 10.79 & \bf{719.06} & 
0.62 & 1.31\\CMT3Y & 723.86 & 5.84 & 
729.86 & 9.50 & \bf{719.06} & 
0.67 & 1.50\\CMT4X & 869.67 & 81.35 & 
876.31 & 44.54 & \bf{854.21} & 
1.81 & 2.59\\CMT4Y & 867.60 & 32.76 & 
884.29 & 36.26 & \bf{852.46} & 
1.78 & 3.73\\CMT5X & 1051.65 & 86.53 & 
1063.93 & 81.55 & \bf{1030.56} & 
2.05 & 3.24\\CMT5Y & 1067.87 & 67.29 & 
1079.59 & 70.28 & \bf{1031.69} & 
3.51 & 4.64\\CMT11X & 871.41 & 21.20 & 
891.71 & 36.91 & \bf{831.09} & 
4.85 & 7.29\\CMT11Y & 875.05 & 55.65 & 
880.41 & 35.77 & \bf{829.85} & 
5.45 & 6.09\\CMT12X & 669.38 & 8.96 & 
672.69 & 10.43 & \bf{658.83} & 
1.60 & 2.10\\CMT12Y & 663.61 & 25.07 & 
673.86 & 13.85 & \bf{660.47} & 
0.48 & 2.03\\\bf{PROM.} & 
\bf{764.12} & \bf{29.43} & \bf{771.92} & \bf{26.62} & \bf{749.50} & \bf{1.72} & \bf{2.69}\\[1ex]\hline
\end{tabular}
\label{table:nonlin}
\end{table} \clearpage
\begin{table}[ht]
\caption{Resultados de la ejecución de la metaheurística GTS, utilizando instancias de SalhiNagy con la configuración -mni 5000 -lambda1 0.05 -lambda2 0.05 -tabu 13}
\centering
\small
\begin{tabular}{c c c c c c c c}
\hline\hline
Instancia & Costo mínimo & Tiempo(seg.) & Costo promedio & Tiempo promedio(seg.) & CME & \%G & \%GP \\ [0.5ex]
\hline
CMT1X & 470.67 & 5.28 & 
474.65 & 4.79 & \bf{470.48} & 
0.04 & 0.89\\CMT1Y & \bf{470.48} & 5.17 & 
471.90 & 4.96 & 470.48 & 0.00
 & 0.30\\CMT2X & 687.04 & 10.30 & 
688.10 & 7.61 & \bf{682.39} & 
0.68 & 0.84\\CMT2Y & 683.95 & 8.23 & 
687.21 & 6.09 & \bf{682.39} & 
0.23 & 0.71\\CMT3X & 720.85 & 19.41 & 
727.44 & 13.26 & \bf{719.06} & 
0.25 & 1.17\\CMT3Y & 727.67 & 23.26 & 
730.82 & 16.68 & \bf{719.06} & 
1.20 & 1.64\\CMT4X & 856.96 & 53.35 & 
875.27 & 53.15 & \bf{854.21} & 
0.32 & 2.47\\CMT4Y & 861.87 & 42.58 & 
873.45 & 43.27 & \bf{852.46} & 
1.10 & 2.46\\CMT5X & 1044.32 & 67.27 & 
1061.08 & 86.57 & \bf{1030.56} & 
1.34 & 2.96\\CMT5Y & 1070.09 & 67.93 & 
1075.29 & 88.73 & \bf{1031.69} & 
3.72 & 4.23\\CMT11X & 874.78 & 24.59 & 
929.53 & 35.62 & \bf{831.09} & 
5.26 & 11.85\\CMT11Y & 877.95 & 35.18 & 
929.10 & 26.55 & \bf{829.85} & 
5.80 & 11.96\\CMT12X & 669.67 & 15.38 & 
675.60 & 13.96 & \bf{658.83} & 
1.65 & 2.55\\CMT12Y & 674.38 & 16.92 & 
682.40 & 11.98 & \bf{660.47} & 
2.11 & 3.32\\\bf{PROM.} & 
\bf{763.62} & \bf{28.20} & \bf{777.27} & \bf{29.52} & \bf{749.50} & \bf{1.69} & \bf{3.38}\\[1ex]\hline
\end{tabular}
\label{table:nonlin}
\end{table} \clearpage
\begin{table}[ht]
\caption{Resultados de la ejecución de la metaheurística GTS, utilizando instancias de SalhiNagy con la configuración -mni 5000 -lambda1 0.05 -lambda2 0.05 -tabu 15}
\centering
\small
\begin{tabular}{c c c c c c c c}
\hline\hline
Instancia & Costo mínimo & Tiempo(seg.) & Costo promedio & Tiempo promedio(seg.) & CME & \%G & \%GP \\ [0.5ex]
\hline
CMT1X & \bf{470.48} & 2.45 & 
471.55 & 3.98 & 470.48 & 0.00
 & 0.23\\CMT1Y & \bf{470.48} & 6.14 & 
471.90 & 4.93 & 470.48 & 0.00
 & 0.30\\CMT2X & 683.52 & 11.10 & 
686.81 & 7.83 & \bf{682.39} & 
0.17 & 0.65\\CMT2Y & 683.97 & 8.98 & 
689.06 & 6.64 & \bf{682.39} & 
0.23 & 0.98\\CMT3X & 726.27 & 7.12 & 
726.91 & 12.43 & \bf{719.06} & 
1.00 & 1.09\\CMT3Y & 719.24 & 22.21 & 
728.07 & 12.65 & \bf{719.06} & 
0.03 & 1.25\\CMT4X & 858.88 & 57.39 & 
886.55 & 43.93 & \bf{854.21} & 
0.55 & 3.79\\CMT4Y & 865.01 & 44.80 & 
872.39 & 42.17 & \bf{852.46} & 
1.47 & 2.34\\CMT5X & 1049.41 & 134.65 & 
1077.02 & 112.53 & \bf{1030.56} & 
1.83 & 4.51\\CMT5Y & 1048.92 & 144.62 & 
1064.13 & 106.15 & \bf{1031.69} & 
1.67 & 3.14\\CMT11X & 880.38 & 28.51 & 
928.70 & 24.09 & \bf{831.09} & 
5.93 & 11.74\\CMT11Y & 882.86 & 45.00 & 
911.50 & 34.93 & \bf{829.85} & 
6.39 & 9.84\\CMT12X & 673.45 & 13.12 & 
676.90 & 11.03 & \bf{658.83} & 
2.22 & 2.74\\CMT12Y & 674.44 & 7.92 & 
680.17 & 8.31 & \bf{660.47} & 
2.12 & 2.98\\\bf{PROM.} & 
\bf{763.38} & \bf{38.14} & \bf{776.55} & \bf{30.83} & \bf{749.50} & \bf{1.69} & \bf{3.26}\\[1ex]\hline
\end{tabular}
\label{table:nonlin}
\end{table} \clearpage
\begin{table}[ht]
\caption{Resultados de la ejecución de la metaheurística GTS, utilizando instancias de SalhiNagy con la configuración -mni 5500 -lambda1 0.05 -lambda2 0.05 -tabu 5}
\centering
\small
\begin{tabular}{c c c c c c c c}
\hline\hline
Instancia & Costo mínimo & Tiempo(seg.) & Costo promedio & Tiempo promedio(seg.) & CME & \%G & \%GP \\ [0.5ex]
\hline
CMT1X & 472.37 & 2.44 & 
472.37 & 4.25 & \bf{470.48} & 
0.40 & 0.40\\CMT1Y & 470.67 & 4.80 & 
472.38 & 3.56 & \bf{470.48} & 
0.04 & 0.40\\CMT2X & 683.64 & 21.83 & 
687.24 & 12.72 & \bf{682.39} & 
0.18 & 0.71\\CMT2Y & 684.24 & 5.85 & 
688.34 & 6.04 & \bf{682.39} & 
0.27 & 0.87\\CMT3X & 723.23 & 8.52 & 
728.66 & 12.53 & \bf{719.06} & 
0.58 & 1.34\\CMT3Y & \bf{\underline{718.40}} & 14.21 & 
725.79 & 11.38 & 719.06 & 
\bf{-0.09} & 0.94\\CMT4X & 860.28 & 62.14 & 
876.56 & 44.79 & \bf{854.21} & 
0.71 & 2.62\\CMT4Y & 865.57 & 29.64 & 
871.74 & 35.16 & \bf{852.46} & 
1.54 & 2.26\\CMT5X & 1047.07 & 156.90 & 
1070.18 & 95.11 & \bf{1030.56} & 
1.60 & 3.84\\CMT5Y & 1057.70 & 106.50 & 
1076.70 & 113.37 & \bf{1031.69} & 
2.52 & 4.36\\CMT11X & 878.89 & 63.53 & 
914.88 & 41.45 & \bf{831.09} & 
5.75 & 10.08\\CMT11Y & 865.78 & 39.13 & 
920.05 & 38.20 & \bf{829.85} & 
4.33 & 10.87\\CMT12X & 675.38 & 8.34 & 
683.51 & 13.12 & \bf{658.83} & 
2.51 & 3.75\\CMT12Y & 673.71 & 8.45 & 
674.50 & 13.48 & \bf{660.47} & 
2.00 & 2.12\\\bf{PROM.} & 
\bf{762.64} & \bf{38.02} & \bf{775.92} & \bf{31.80} & \bf{749.50} & \bf{1.60} & \bf{3.18}\\[1ex]\hline
\end{tabular}
\label{table:nonlin}
\end{table} \clearpage
\begin{table}[ht]
\caption{Resultados de la ejecución de la metaheurística GTS, utilizando instancias de SalhiNagy con la configuración -mni 5500 -lambda1 0.05 -lambda2 0.05 -tabu 7}
\centering
\small
\begin{tabular}{c c c c c c c c}
\hline\hline
Instancia & Costo mínimo & Tiempo(seg.) & Costo promedio & Tiempo promedio(seg.) & CME & \%G & \%GP \\ [0.5ex]
\hline
CMT1X & \bf{470.48} & 3.60 & 
471.90 & 3.67 & 470.48 & 0.00
 & 0.30\\CMT1Y & \bf{470.48} & 3.58 & 
471.43 & 5.79 & 470.48 & 0.00
 & 0.20\\CMT2X & \bf{682.39} & 8.92 & 
685.81 & 11.69 & 682.39 & 0.00
 & 0.50\\CMT2Y & 684.40 & 6.95 & 
687.10 & 8.30 & \bf{682.39} & 
0.29 & 0.69\\CMT3X & 723.18 & 22.43 & 
731.96 & 12.99 & \bf{719.06} & 
0.57 & 1.79\\CMT3Y & 719.93 & 12.69 & 
729.08 & 12.98 & \bf{719.06} & 
0.12 & 1.39\\CMT4X & 871.68 & 34.53 & 
879.43 & 36.52 & \bf{854.21} & 
2.05 & 2.95\\CMT4Y & 867.59 & 40.81 & 
881.13 & 39.41 & \bf{852.46} & 
1.77 & 3.36\\CMT5X & 1046.16 & 69.26 & 
1066.14 & 85.11 & \bf{1030.56} & 
1.51 & 3.45\\CMT5Y & 1051.24 & 111.83 & 
1070.14 & 132.00 & \bf{1031.69} & 
1.89 & 3.73\\CMT11X & 880.68 & 64.33 & 
908.46 & 44.34 & \bf{831.09} & 
5.97 & 9.31\\CMT11Y & 885.61 & 33.56 & 
927.37 & 29.70 & \bf{829.85} & 
6.72 & 11.75\\CMT12X & 670.79 & 18.20 & 
673.66 & 12.81 & \bf{658.83} & 
1.82 & 2.25\\CMT12Y & 663.72 & 13.88 & 
676.88 & 11.16 & \bf{660.47} & 
0.49 & 2.49\\\bf{PROM.} & 
\bf{763.45} & \bf{31.75} & \bf{775.75} & \bf{31.89} & \bf{749.50} & \bf{1.66} & \bf{3.16}\\[1ex]\hline
\end{tabular}
\label{table:nonlin}
\end{table} \clearpage
\begin{table}[ht]
\caption{Resultados de la ejecución de la metaheurística GTS, utilizando instancias de SalhiNagy con la configuración -mni 5500 -lambda1 0.05 -lambda2 0.05 -tabu 9}
\centering
\small
\begin{tabular}{c c c c c c c c}
\hline\hline
Instancia & Costo mínimo & Tiempo(seg.) & Costo promedio & Tiempo promedio(seg.) & CME & \%G & \%GP \\ [0.5ex]
\hline
CMT1X & \bf{470.48} & 3.20 & 
472.05 & 4.29 & 470.48 & 0.00
 & 0.33\\CMT1Y & \bf{470.48} & 7.30 & 
471.00 & 6.01 & 470.48 & 0.00
 & 0.11\\CMT2X & 684.24 & 6.62 & 
689.63 & 6.37 & \bf{682.39} & 
0.27 & 1.06\\CMT2Y & \bf{682.39} & 14.43 & 
685.94 & 10.15 & 682.39 & 0.00
 & 0.52\\CMT3X & \bf{\underline{718.40}} & 12.86 & 
724.97 & 14.07 & 719.06 & 
\bf{-0.09} & 0.82\\CMT3Y & 723.78 & 9.17 & 
729.15 & 8.99 & \bf{719.06} & 
0.66 & 1.40\\CMT4X & 855.11 & 37.75 & 
877.04 & 43.70 & \bf{854.21} & 
0.11 & 2.67\\CMT4Y & 868.03 & 39.33 & 
879.85 & 46.31 & \bf{852.46} & 
1.83 & 3.21\\CMT5X & 1064.64 & 63.72 & 
1069.83 & 62.98 & \bf{1030.56} & 
3.31 & 3.81\\CMT5Y & 1046.09 & 97.80 & 
1065.31 & 100.99 & \bf{1031.69} & 
1.40 & 3.26\\CMT11X & 881.17 & 41.68 & 
911.04 & 36.44 & \bf{831.09} & 
6.03 & 9.62\\CMT11Y & 877.70 & 44.31 & 
924.58 & 34.49 & \bf{829.85} & 
5.77 & 11.41\\CMT12X & 674.35 & 12.44 & 
682.80 & 14.48 & \bf{658.83} & 
2.36 & 3.64\\CMT12Y & 673.64 & 29.43 & 
676.00 & 14.46 & \bf{660.47} & 
1.99 & 2.35\\\bf{PROM.} & 
\bf{763.61} & \bf{30.00} & \bf{775.66} & \bf{28.84} & \bf{749.50} & \bf{1.69} & \bf{3.16}\\[1ex]\hline
\end{tabular}
\label{table:nonlin}
\end{table} \clearpage
\begin{table}[ht]
\caption{Resultados de la ejecución de la metaheurística GTS, utilizando instancias de SalhiNagy con la configuración -mni 5500 -lambda1 0.05 -lambda2 0.05 -tabu 11}
\centering
\small
\begin{tabular}{c c c c c c c c}
\hline\hline
Instancia & Costo mínimo & Tiempo(seg.) & Costo promedio & Tiempo promedio(seg.) & CME & \%G & \%GP \\ [0.5ex]
\hline
CMT1X & \bf{470.48} & 5.81 & 
471.00 & 5.76 & 470.48 & 0.00
 & 0.11\\CMT1Y & \bf{470.48} & 6.78 & 
471.00 & 5.29 & 470.48 & 0.00
 & 0.11\\CMT2X & \bf{682.39} & 8.26 & 
687.26 & 9.03 & 682.39 & 0.00
 & 0.71\\CMT2Y & \bf{682.39} & 9.22 & 
683.31 & 10.63 & 682.39 & 0.00
 & 0.13\\CMT3X & 724.40 & 25.90 & 
731.49 & 15.68 & \bf{719.06} & 
0.74 & 1.73\\CMT3Y & 725.34 & 15.90 & 
728.43 & 15.18 & \bf{719.06} & 
0.87 & 1.30\\CMT4X & 858.01 & 48.37 & 
869.90 & 37.31 & \bf{854.21} & 
0.44 & 1.84\\CMT4Y & 857.27 & 47.40 & 
880.55 & 52.69 & \bf{852.46} & 
0.56 & 3.30\\CMT5X & 1040.91 & 146.90 & 
1071.59 & 124.35 & \bf{1030.56} & 
1.00 & 3.98\\CMT5Y & 1054.10 & 71.84 & 
1074.07 & 91.59 & \bf{1031.69} & 
2.17 & 4.11\\CMT11X & 862.64 & 36.57 & 
897.02 & 63.87 & \bf{831.09} & 
3.80 & 7.93\\CMT11Y & 883.76 & 25.96 & 
931.40 & 26.63 & \bf{829.85} & 
6.50 & 12.24\\CMT12X & 669.93 & 17.88 & 
675.67 & 12.67 & \bf{658.83} & 
1.68 & 2.56\\CMT12Y & 673.81 & 6.65 & 
675.81 & 10.68 & \bf{660.47} & 
2.02 & 2.32\\\bf{PROM.} & 
\bf{761.14} & \bf{33.82} & \bf{774.89} & \bf{34.38} & \bf{749.50} & \bf{1.41} & \bf{3.03}\\[1ex]\hline
\end{tabular}
\label{table:nonlin}
\end{table} \clearpage
\begin{table}[ht]
\caption{Resultados de la ejecución de la metaheurística GTS, utilizando instancias de SalhiNagy con la configuración -mni 5500 -lambda1 0.05 -lambda2 0.05 -tabu 13}
\centering
\small
\begin{tabular}{c c c c c c c c}
\hline\hline
Instancia & Costo mínimo & Tiempo(seg.) & Costo promedio & Tiempo promedio(seg.) & CME & \%G & \%GP \\ [0.5ex]
\hline
CMT1X & \bf{470.48} & 2.54 & 
471.90 & 4.07 & 470.48 & 0.00
 & 0.30\\CMT1Y & \bf{470.48} & 4.96 & 
474.13 & 3.94 & 470.48 & 0.00
 & 0.78\\CMT2X & 685.69 & 7.94 & 
689.45 & 8.48 & \bf{682.39} & 
0.48 & 1.03\\CMT2Y & 684.06 & 14.78 & 
685.82 & 8.01 & \bf{682.39} & 
0.24 & 0.50\\CMT3X & 725.03 & 12.76 & 
728.80 & 11.11 & \bf{719.06} & 
0.83 & 1.35\\CMT3Y & 719.93 & 15.74 & 
724.92 & 13.18 & \bf{719.06} & 
0.12 & 0.82\\CMT4X & 864.16 & 52.44 & 
879.20 & 40.44 & \bf{854.21} & 
1.16 & 2.93\\CMT4Y & 858.41 & 77.48 & 
870.40 & 49.25 & \bf{852.46} & 
0.70 & 2.10\\CMT5X & 1054.33 & 57.12 & 
1064.52 & 99.54 & \bf{1030.56} & 
2.31 & 3.30\\CMT5Y & 1051.80 & 146.99 & 
1066.45 & 104.88 & \bf{1031.69} & 
1.95 & 3.37\\CMT11X & 859.84 & 54.48 & 
907.96 & 45.87 & \bf{831.09} & 
3.46 & 9.25\\CMT11Y & 877.96 & 54.34 & 
924.92 & 44.73 & \bf{829.85} & 
5.80 & 11.46\\CMT12X & 673.49 & 16.51 & 
679.25 & 16.20 & \bf{658.83} & 
2.23 & 3.10\\CMT12Y & 673.74 & 10.59 & 
679.98 & 13.55 & \bf{660.47} & 
2.01 & 2.95\\\bf{PROM.} & 
\bf{762.10} & \bf{37.76} & \bf{774.84} & \bf{33.09} & \bf{749.50} & \bf{1.52} & \bf{3.09}\\[1ex]\hline
\end{tabular}
\label{table:nonlin}
\end{table} \clearpage
\begin{table}[ht]
\caption{Resultados de la ejecución de la metaheurística GTS, utilizando instancias de SalhiNagy con la configuración -mni 5500 -lambda1 0.05 -lambda2 0.05 -tabu 15}
\centering
\small
\begin{tabular}{c c c c c c c c}
\hline\hline
Instancia & Costo mínimo & Tiempo(seg.) & Costo promedio & Tiempo promedio(seg.) & CME & \%G & \%GP \\ [0.5ex]
\hline
CMT1X & \bf{470.48} & 5.48 & 
471.43 & 4.25 & 470.48 & 0.00
 & 0.20\\CMT1Y & \bf{470.48} & 6.05 & 
473.66 & 5.64 & 470.48 & 0.00
 & 0.68\\CMT2X & 687.92 & 8.50 & 
692.16 & 9.02 & \bf{682.39} & 
0.81 & 1.43\\CMT2Y & 683.64 & 6.83 & 
685.75 & 6.36 & \bf{682.39} & 
0.18 & 0.49\\CMT3X & 724.57 & 15.10 & 
727.42 & 14.71 & \bf{719.06} & 
0.77 & 1.16\\CMT3Y & 723.64 & 21.38 & 
725.36 & 17.65 & \bf{719.06} & 
0.64 & 0.88\\CMT4X & 856.62 & 18.68 & 
874.87 & 29.77 & \bf{854.21} & 
0.28 & 2.42\\CMT4Y & 865.22 & 58.41 & 
873.13 & 52.03 & \bf{852.46} & 
1.50 & 2.42\\CMT5X & 1048.41 & 86.68 & 
1066.39 & 92.86 & \bf{1030.56} & 
1.73 & 3.48\\CMT5Y & 1043.48 & 52.52 & 
1081.41 & 69.85 & \bf{1031.69} & 
1.14 & 4.82\\CMT11X & 938.28 & 57.31 & 
960.04 & 36.74 & \bf{831.09} & 
12.90 & 15.52\\CMT11Y & 849.61 & 17.62 & 
923.51 & 28.71 & \bf{829.85} & 
2.38 & 11.29\\CMT12X & 663.53 & 26.86 & 
669.86 & 14.19 & \bf{658.83} & 
0.71 & 1.67\\CMT12Y & 674.11 & 14.57 & 
684.15 & 15.00 & \bf{660.47} & 
2.07 & 3.59\\\bf{PROM.} & 
\bf{764.29} & \bf{28.29} & \bf{779.22} & \bf{28.34} & \bf{749.50} & \bf{1.79} & \bf{3.57}\\[1ex]\hline
\end{tabular}
\label{table:nonlin}
\end{table} \clearpage
\begin{table}[ht]
\caption{Resultados de la ejecución de la metaheurística GTS, utilizando instancias de SalhiNagy con la configuración -mni 6000 -lambda1 0.05 -lambda2 0.05 -tabu 5}
\centering
\small
\begin{tabular}{c c c c c c c c}
\hline\hline
Instancia & Costo mínimo & Tiempo(seg.) & Costo promedio & Tiempo promedio(seg.) & CME & \%G & \%GP \\ [0.5ex]
\hline
CMT1X & \bf{470.48} & 3.80 & 
470.95 & 3.93 & 470.48 & 0.00
 & 0.10\\CMT1Y & \bf{470.48} & 7.44 & 
470.95 & 6.68 & 470.48 & 0.00
 & 0.10\\CMT2X & 688.06 & 9.45 & 
689.90 & 8.47 & \bf{682.39} & 
0.83 & 1.10\\CMT2Y & 684.29 & 8.03 & 
690.79 & 8.82 & \bf{682.39} & 
0.28 & 1.23\\CMT3X & 720.17 & 14.80 & 
727.20 & 12.06 & \bf{719.06} & 
0.15 & 1.13\\CMT3Y & 725.02 & 15.33 & 
730.14 & 16.30 & \bf{719.06} & 
0.83 & 1.54\\CMT4X & 863.01 & 68.36 & 
879.09 & 59.50 & \bf{854.21} & 
1.03 & 2.91\\CMT4Y & 861.32 & 67.47 & 
868.18 & 47.98 & \bf{852.46} & 
1.04 & 1.84\\CMT5X & 1055.39 & 118.95 & 
1068.93 & 83.02 & \bf{1030.56} & 
2.41 & 3.72\\CMT5Y & 1049.57 & 79.66 & 
1070.86 & 105.15 & \bf{1031.69} & 
1.73 & 3.80\\CMT11X & 896.87 & 48.70 & 
944.09 & 40.97 & \bf{831.09} & 
7.91 & 13.60\\CMT11Y & 875.47 & 41.09 & 
895.22 & 41.19 & \bf{829.85} & 
5.50 & 7.88\\CMT12X & 667.62 & 11.11 & 
671.05 & 13.10 & \bf{658.83} & 
1.33 & 1.86\\CMT12Y & 674.44 & 19.16 & 
682.28 & 21.88 & \bf{660.47} & 
2.12 & 3.30\\\bf{PROM.} & 
\bf{764.44} & \bf{36.67} & \bf{775.69} & \bf{33.50} & \bf{749.50} & \bf{1.80} & \bf{3.15}\\[1ex]\hline
\end{tabular}
\label{table:nonlin}
\end{table} \clearpage
\begin{table}[ht]
\caption{Resultados de la ejecución de la metaheurística GTS, utilizando instancias de SalhiNagy con la configuración -mni 6000 -lambda1 0.05 -lambda2 0.05 -tabu 7}
\centering
\small
\begin{tabular}{c c c c c c c c}
\hline\hline
Instancia & Costo mínimo & Tiempo(seg.) & Costo promedio & Tiempo promedio(seg.) & CME & \%G & \%GP \\ [0.5ex]
\hline
CMT1X & \bf{470.48} & 2.60 & 
471.90 & 4.31 & 470.48 & 0.00
 & 0.30\\CMT1Y & \bf{470.48} & 3.93 & 
470.95 & 4.56 & 470.48 & 0.00
 & 0.10\\CMT2X & 686.60 & 7.45 & 
690.98 & 7.48 & \bf{682.39} & 
0.62 & 1.26\\CMT2Y & \bf{682.39} & 10.53 & 
684.04 & 13.38 & 682.39 & 0.00
 & 0.24\\CMT3X & 720.85 & 23.26 & 
729.39 & 19.95 & \bf{719.06} & 
0.25 & 1.44\\CMT3Y & 723.58 & 19.94 & 
726.71 & 19.58 & \bf{719.06} & 
0.63 & 1.06\\CMT4X & 859.31 & 91.84 & 
879.54 & 59.30 & \bf{854.21} & 
0.60 & 2.97\\CMT4Y & 864.67 & 41.37 & 
877.22 & 68.30 & \bf{852.46} & 
1.43 & 2.90\\CMT5X & 1044.88 & 144.98 & 
1068.21 & 125.83 & \bf{1030.56} & 
1.39 & 3.65\\CMT5Y & 1047.01 & 141.66 & 
1053.57 & 116.84 & \bf{1031.69} & 
1.48 & 2.12\\CMT11X & 879.64 & 30.13 & 
907.49 & 32.08 & \bf{831.09} & 
5.84 & 9.19\\CMT11Y & 877.21 & 26.68 & 
919.60 & 40.13 & \bf{829.85} & 
5.71 & 10.82\\CMT12X & 663.94 & 9.10 & 
674.51 & 11.66 & \bf{658.83} & 
0.78 & 2.38\\CMT12Y & 673.49 & 12.34 & 
688.04 & 15.60 & \bf{660.47} & 
1.97 & 4.17\\\bf{PROM.} & 
\bf{761.75} & \bf{40.41} & \bf{774.44} & \bf{38.50} & \bf{749.50} & \bf{1.48} & \bf{3.04}\\[1ex]\hline
\end{tabular}
\label{table:nonlin}
\end{table} \clearpage
\begin{table}[ht]
\caption{Resultados de la ejecución de la metaheurística GTS, utilizando instancias de SalhiNagy con la configuración -mni 6000 -lambda1 0.05 -lambda2 0.05 -tabu 9}
\centering
\small
\begin{tabular}{c c c c c c c c}
\hline\hline
Instancia & Costo mínimo & Tiempo(seg.) & Costo promedio & Tiempo promedio(seg.) & CME & \%G & \%GP \\ [0.5ex]
\hline
CMT1X & \bf{470.48} & 3.08 & 
471.90 & 3.99 & 470.48 & 0.00
 & 0.30\\CMT1Y & \bf{470.48} & 6.63 & 
473.20 & 3.73 & 470.48 & 0.00
 & 0.58\\CMT2X & \bf{682.39} & 13.01 & 
693.22 & 9.51 & 682.39 & 0.00
 & 1.59\\CMT2Y & 687.04 & 11.24 & 
689.56 & 6.97 & \bf{682.39} & 
0.68 & 1.05\\CMT3X & 720.66 & 13.36 & 
728.45 & 22.45 & \bf{719.06} & 
0.22 & 1.31\\CMT3Y & \bf{\underline{718.40}} & 12.57 & 
725.54 & 15.07 & 719.06 & 
\bf{-0.09} & 0.90\\CMT4X & 864.21 & 33.40 & 
882.69 & 45.73 & \bf{854.21} & 
1.17 & 3.33\\CMT4Y & 857.35 & 47.47 & 
863.17 & 54.01 & \bf{852.46} & 
0.57 & 1.26\\CMT5X & 1035.01 & 239.87 & 
1059.36 & 108.99 & \bf{1030.56} & 
0.43 & 2.79\\CMT5Y & 1035.26 & 95.09 & 
1060.06 & 97.55 & \bf{1031.69} & 
0.35 & 2.75\\CMT11X & 881.96 & 36.03 & 
920.89 & 35.30 & \bf{831.09} & 
6.12 & 10.81\\CMT11Y & 880.13 & 115.67 & 
913.65 & 51.82 & \bf{829.85} & 
6.06 & 10.10\\CMT12X & 664.80 & 8.16 & 
675.04 & 12.48 & \bf{658.83} & 
0.91 & 2.46\\CMT12Y & 674.70 & 14.40 & 
682.53 & 11.87 & \bf{660.47} & 
2.15 & 3.34\\\bf{PROM.} & 
\bf{760.21} & \bf{46.43} & \bf{774.23} & \bf{34.25} & \bf{749.50} & \bf{1.33} & \bf{3.04}\\[1ex]\hline
\end{tabular}
\label{table:nonlin}
\end{table} \clearpage
\begin{table}[ht]
\caption{Resultados de la ejecución de la metaheurística GTS, utilizando instancias de SalhiNagy con la configuración -mni 6000 -lambda1 0.05 -lambda2 0.05 -tabu 11}
\centering
\small
\begin{tabular}{c c c c c c c c}
\hline\hline
Instancia & Costo mínimo & Tiempo(seg.) & Costo promedio & Tiempo promedio(seg.) & CME & \%G & \%GP \\ [0.5ex]
\hline
CMT1X & \bf{470.48} & 5.65 & 
471.41 & 4.18 & 470.48 & 0.00
 & 0.20\\CMT1Y & \bf{470.48} & 9.00 & 
471.00 & 6.33 & 470.48 & 0.00
 & 0.11\\CMT2X & 682.94 & 18.59 & 
687.12 & 10.97 & \bf{682.39} & 
0.08 & 0.69\\CMT2Y & 682.94 & 6.48 & 
686.79 & 7.16 & \bf{682.39} & 
0.08 & 0.65\\CMT3X & 725.72 & 8.22 & 
729.91 & 11.64 & \bf{719.06} & 
0.93 & 1.51\\CMT3Y & 726.69 & 26.26 & 
729.28 & 14.38 & \bf{719.06} & 
1.06 & 1.42\\CMT4X & 873.55 & 47.00 & 
885.71 & 36.26 & \bf{854.21} & 
2.26 & 3.69\\CMT4Y & 861.16 & 32.43 & 
872.17 & 44.57 & \bf{852.46} & 
1.02 & 2.31\\CMT5X & 1059.76 & 67.61 & 
1073.24 & 73.17 & \bf{1030.56} & 
2.83 & 4.14\\CMT5Y & 1054.25 & 139.59 & 
1057.86 & 135.59 & \bf{1031.69} & 
2.19 & 2.54\\CMT11X & 883.76 & 50.57 & 
927.68 & 30.62 & \bf{831.09} & 
6.34 & 11.62\\CMT11Y & 871.22 & 19.06 & 
930.37 & 30.20 & \bf{829.85} & 
4.99 & 12.11\\CMT12X & 663.61 & 16.65 & 
672.62 & 12.72 & \bf{658.83} & 
0.73 & 2.09\\CMT12Y & 671.79 & 21.82 & 
677.12 & 17.48 & \bf{660.47} & 
1.71 & 2.52\\\bf{PROM.} & 
\bf{764.17} & \bf{33.49} & \bf{776.59} & \bf{31.09} & \bf{749.50} & \bf{1.73} & \bf{3.26}\\[1ex]\hline
\end{tabular}
\label{table:nonlin}
\end{table} \clearpage
\begin{table}[ht]
\caption{Resultados de la ejecución de la metaheurística GTS, utilizando instancias de SalhiNagy con la configuración -mni 6000 -lambda1 0.05 -lambda2 0.05 -tabu 13}
\centering
\small
\begin{tabular}{c c c c c c c c}
\hline\hline
Instancia & Costo mínimo & Tiempo(seg.) & Costo promedio & Tiempo promedio(seg.) & CME & \%G & \%GP \\ [0.5ex]
\hline
CMT1X & \bf{470.48} & 3.54 & 
470.95 & 5.05 & 470.48 & 0.00
 & 0.10\\CMT1Y & \bf{470.48} & 6.14 & 
470.95 & 3.78 & 470.48 & 0.00
 & 0.10\\CMT2X & 683.95 & 11.29 & 
688.55 & 9.14 & \bf{682.39} & 
0.23 & 0.90\\CMT2Y & 684.29 & 7.24 & 
685.10 & 7.54 & \bf{682.39} & 
0.28 & 0.40\\CMT3X & 723.52 & 10.92 & 
727.06 & 21.06 & \bf{719.06} & 
0.62 & 1.11\\CMT3Y & 720.08 & 40.20 & 
723.09 & 19.58 & \bf{719.06} & 
0.14 & 0.56\\CMT4X & 861.31 & 113.35 & 
865.77 & 61.74 & \bf{854.21} & 
0.83 & 1.35\\CMT4Y & 860.99 & 44.63 & 
872.36 & 46.59 & \bf{852.46} & 
1.00 & 2.33\\CMT5X & 1046.69 & 205.14 & 
1061.31 & 129.06 & \bf{1030.56} & 
1.57 & 2.98\\CMT5Y & 1048.16 & 105.49 & 
1061.84 & 82.57 & \bf{1031.69} & 
1.60 & 2.92\\CMT11X & 869.87 & 32.10 & 
916.02 & 54.55 & \bf{831.09} & 
4.67 & 10.22\\CMT11Y & 873.14 & 33.83 & 
880.33 & 36.66 & \bf{829.85} & 
5.22 & 6.08\\CMT12X & 664.07 & 15.73 & 
680.90 & 20.53 & \bf{658.83} & 
0.80 & 3.35\\CMT12Y & 674.40 & 22.88 & 
682.37 & 12.76 & \bf{660.47} & 
2.11 & 3.32\\\bf{PROM.} & 
\bf{760.82} & \bf{46.61} & \bf{770.47} & \bf{36.47} & \bf{749.50} & \bf{1.36} & \bf{2.55}\\[1ex]\hline
\end{tabular}
\label{table:nonlin}
\end{table} \clearpage
\begin{table}[ht]
\caption{Resultados de la ejecución de la metaheurística GTS, utilizando instancias de SalhiNagy con la configuración -mni 6000 -lambda1 0.05 -lambda2 0.05 -tabu 15}
\centering
\small
\begin{tabular}{c c c c c c c c}
\hline\hline
Instancia & Costo mínimo & Tiempo(seg.) & Costo promedio & Tiempo promedio(seg.) & CME & \%G & \%GP \\ [0.5ex]
\hline
CMT1X & \bf{470.48} & 2.35 & 
470.95 & 4.64 & 470.48 & 0.00
 & 0.10\\CMT1Y & \bf{470.48} & 4.34 & 
471.90 & 5.11 & 470.48 & 0.00
 & 0.30\\CMT2X & 684.40 & 5.93 & 
688.02 & 8.56 & \bf{682.39} & 
0.29 & 0.83\\CMT2Y & 685.96 & 7.21 & 
687.98 & 6.01 & \bf{682.39} & 
0.52 & 0.82\\CMT3X & 719.17 & 19.77 & 
726.99 & 22.40 & \bf{719.06} & 
0.02 & 1.10\\CMT3Y & 724.07 & 11.69 & 
726.30 & 16.00 & \bf{719.06} & 
0.70 & 1.01\\CMT4X & 878.38 & 62.28 & 
882.53 & 51.95 & \bf{854.21} & 
2.83 & 3.32\\CMT4Y & 855.55 & 70.61 & 
860.58 & 68.84 & \bf{852.46} & 
0.36 & 0.95\\CMT5X & 1059.05 & 141.34 & 
1066.39 & 120.15 & \bf{1030.56} & 
2.76 & 3.48\\CMT5Y & 1051.30 & 163.99 & 
1073.98 & 110.38 & \bf{1031.69} & 
1.90 & 4.10\\CMT11X & 899.42 & 37.12 & 
934.67 & 28.89 & \bf{831.09} & 
8.22 & 12.46\\CMT11Y & 844.21 & 56.89 & 
885.82 & 48.29 & \bf{829.85} & 
1.73 & 6.74\\CMT12X & 669.80 & 12.86 & 
678.92 & 14.90 & \bf{658.83} & 
1.67 & 3.05\\CMT12Y & 670.30 & 22.42 & 
673.60 & 19.32 & \bf{660.47} & 
1.49 & 1.99\\\bf{PROM.} & 
\bf{763.04} & \bf{44.20} & \bf{773.47} & \bf{37.53} & \bf{749.50} & \bf{1.61} & \bf{2.87}\\[1ex]\hline
\end{tabular}
\label{table:nonlin}
\end{table} \clearpage
\begin{table}[ht]
\caption{Resultados de la ejecución de la metaheurística GTS, utilizando instancias de Dethloff con la configuración -mni 3000 -lambda1 0.05 -lambda2 0.05 -tabu 17}
\centering
\small
\begin{tabular}{c c c c c c c c}
\hline\hline
Instancia & Costo mínimo & Tiempo(seg.) & Costo promedio & Tiempo promedio(seg.) & CME & \%G & \%GP \\ [0.5ex]
\hline
SCA3-0 & 636.34 & 1.74 & 
639.50 & 1.80 & \bf{635.62} & 
0.11 & 0.61\\SCA3-1 & \bf{697.84} & 2.10 & 
700.11 & 2.32 & 697.84 & 0.00
 & 0.32\\SCA3-2 & \bf{659.34} & 4.18 & 
672.68 & 2.36 & 659.34 & 0.00
 & 2.02\\SCA3-3 & 680.60 & 3.62 & 
683.63 & 1.81 & \bf{680.04} & 
0.08 & 0.53\\SCA3-4 & \bf{690.50} & 1.36 & 
690.50 & 2.05 & 690.50 & 0.00
 & 0.00\\
SCA3-5 & \bf{659.90} & 2.81 & 
663.16 & 2.56 & 659.90 & 0.00
 & 0.49\\SCA3-6 & \bf{651.09} & 1.84 & 
651.74 & 1.91 & 651.09 & 0.00
 & 0.10\\SCA3-7 & 666.15 & 2.00 & 
666.15 & 1.56 & \bf{659.17} & 
1.06 & 1.06\\SCA3-8 & \bf{719.47} & 1.30 & 
721.90 & 2.40 & 719.47 & 0.00
 & 0.34\\SCA3-9 & \bf{681.00} & 1.27 & 
681.00 & 1.64 & 681.00 & 0.00
 & 0.00\\
SCA8-0 & \bf{961.50} & 4.28 & 
984.30 & 2.58 & 961.50 & 0.00
 & 2.37\\SCA8-1 & 1050.38 & 4.98 & 
1059.53 & 3.05 & \bf{1049.65} & 
0.07 & 0.94\\SCA8-2 & 1064.23 & 1.88 & 
1073.47 & 1.88 & \bf{1039.64} & 
2.37 & 3.25\\SCA8-3 & \bf{983.34} & 7.36 & 
998.25 & 4.54 & 983.34 & 0.00
 & 1.52\\SCA8-4 & 1068.97 & 2.21 & 
1078.57 & 2.21 & \bf{1065.49} & 
0.33 & 1.23\\SCA8-5 & 1047.93 & 1.44 & 
1059.08 & 1.50 & \bf{1027.08} & 
2.03 & 3.12\\SCA8-6 & 972.48 & 4.08 & 
972.48 & 2.75 & \bf{971.82} & 
0.07 & 0.07\\SCA8-7 & 1066.65 & 1.60 & 
1079.26 & 1.54 & \bf{1051.28} & 
1.46 & 2.66\\SCA8-8 & 1082.12 & 1.70 & 
1083.70 & 1.50 & \bf{1071.18} & 
1.02 & 1.17\\SCA8-9 & \bf{1060.50} & 1.21 & 
1065.06 & 1.76 & 1060.50 & 0.00
 & 0.43\\CON3-0 & \bf{616.52} & 3.70 & 
619.22 & 2.80 & 616.52 & 0.00
 & 0.44\\CON3-1 & 558.16 & 4.74 & 
561.05 & 2.74 & \bf{554.47} & 
0.67 & 1.19\\CON3-2 & 521.38 & 2.85 & 
524.54 & 1.67 & \bf{518.00} & 
0.65 & 1.26\\CON3-3 & \bf{591.19} & 4.28 & 
603.85 & 2.56 & 591.19 & 0.00
 & 2.14\\CON3-4 & 591.43 & 2.38 & 
595.78 & 1.64 & \bf{588.79} & 
0.45 & 1.19\\CON3-5 & \bf{563.70} & 1.13 & 
569.17 & 2.37 & 563.70 & 0.00
 & 0.97\\CON3-6 & \bf{499.05} & 2.12 & 
500.84 & 2.39 & 499.05 & 0.00
 & 0.36\\CON3-7 & 578.41 & 1.06 & 
579.95 & 1.59 & \bf{576.48} & 
0.33 & 0.60\\CON3-8 & \bf{523.05} & 2.07 & 
533.18 & 2.14 & 523.05 & 0.00
 & 1.94\\CON3-9 & 582.79 & 1.48 & 
586.75 & 1.64 & \bf{578.24} & 
0.79 & 1.47\\CON8-0 & 866.68 & 3.13 & 
915.97 & 2.32 & \bf{857.17} & 
1.11 & 6.86\\CON8-1 & \bf{740.85} & 1.36 & 
752.50 & 2.08 & 740.85 & 0.00
 & 1.57\\CON8-2 & 718.64 & 2.88 & 
733.02 & 2.60 & \bf{712.89} & 
0.81 & 2.82\\CON8-3 & 811.92 & 2.25 & 
827.73 & 2.14 & \bf{811.07} & 
0.10 & 2.05\\CON8-4 & \bf{772.25} & 1.26 & 
786.73 & 2.10 & 772.25 & 0.00
 & 1.87\\CON8-5 & 756.91 & 1.86 & 
759.20 & 3.07 & \bf{754.88} & 
0.27 & 0.57\\CON8-6 & 692.74 & 2.88 & 
696.77 & 2.73 & \bf{678.92} & 
2.04 & 2.63\\CON8-7 & 812.89 & 3.47 & 
818.52 & 2.98 & \bf{811.96} & 
0.11 & 0.81\\CON8-8 & \bf{767.53} & 3.13 & 
780.74 & 1.88 & 767.53 & 0.00
 & 1.72\\CON8-9 & \bf{809.00} & 1.99 & 
826.58 & 2.53 & 809.00 & 0.00
 & 2.17\\\bf{PROM.} & 
\bf{761.89} & \bf{2.57} & \bf{769.90} & \bf{2.24} & \bf{758.54} & \bf{0.40} & \bf{1.42}\\[1ex]\hline
\end{tabular}
\label{table:nonlin}
\end{table} \clearpage
\begin{table}[ht]
\caption{Resultados de la ejecución de la metaheurística GTS, utilizando instancias de SalhiNagy con la configuración -mni 3000 -lambda1 0.05 -lambda2 0.05 -tabu 17}
\centering
\small
\begin{tabular}{c c c c c c c c}
\hline\hline
Instancia & Costo mínimo & Tiempo(seg.) & Costo promedio & Tiempo promedio(seg.) & CME & \%G & \%GP \\ [0.5ex]
\hline
CMT1X & 472.37 & 2.78 & 
473.62 & 2.53 & \bf{470.48} & 
0.40 & 0.67\\CMT1Y & \bf{470.48} & 2.10 & 
474.85 & 2.25 & 470.48 & 0.00
 & 0.93\\CMT2X & 682.94 & 5.54 & 
689.12 & 4.56 & \bf{682.39} & 
0.08 & 0.99\\CMT2Y & 687.58 & 5.52 & 
690.49 & 5.84 & \bf{682.39} & 
0.76 & 1.19\\CMT3X & 731.78 & 5.86 & 
734.97 & 7.37 & \bf{719.06} & 
1.77 & 2.21\\CMT3Y & 726.69 & 13.74 & 
728.46 & 9.11 & \bf{719.06} & 
1.06 & 1.31\\CMT4X & 860.53 & 37.64 & 
866.04 & 36.49 & \bf{854.21} & 
0.74 & 1.38\\CMT4Y & 858.88 & 38.41 & 
866.66 & 27.91 & \bf{852.46} & 
0.75 & 1.67\\CMT5X & 1063.68 & 53.53 & 
1074.61 & 71.58 & \bf{1030.56} & 
3.21 & 4.27\\CMT5Y & 1048.00 & 79.17 & 
1067.23 & 57.25 & \bf{1031.69} & 
1.58 & 3.44\\CMT11X & 885.18 & 16.18 & 
918.48 & 18.53 & \bf{831.09} & 
6.51 & 10.52\\CMT11Y & 879.18 & 22.40 & 
907.66 & 22.85 & \bf{829.85} & 
5.94 & 9.38\\CMT12X & 672.15 & 9.16 & 
678.30 & 9.35 & \bf{658.83} & 
2.02 & 2.96\\CMT12Y & 676.42 & 13.69 & 
684.16 & 8.53 & \bf{660.47} & 
2.41 & 3.59\\\bf{PROM.} & 
\bf{765.42} & \bf{21.84} & \bf{775.33} & \bf{20.30} & \bf{749.50} & \bf{1.95} & \bf{3.18}\\[1ex]\hline
\end{tabular}
\label{table:nonlin}
\end{table} \clearpage
\begin{table}[ht]
\caption{Resultados de la ejecución de la metaheurística GTS, utilizando instancias de Dethloff con la configuración -mni 3000 -lambda1 0.05 -lambda2 0.05 -tabu 21}
\centering
\small
\begin{tabular}{c c c c c c c c}
\hline\hline
Instancia & Costo mínimo & Tiempo(seg.) & Costo promedio & Tiempo promedio(seg.) & CME & \%G & \%GP \\ [0.5ex]
\hline
SCA3-0 & 636.06 & 1.27 & 
638.78 & 1.67 & \bf{635.62} & 
0.07 & 0.50\\SCA3-1 & \bf{697.84} & 1.49 & 
700.77 & 1.80 & 697.84 & 0.00
 & 0.42\\SCA3-2 & \bf{659.34} & 2.62 & 
659.34 & 2.21 & 659.34 & 0.00
 & 0.00\\
SCA3-3 & \bf{680.04} & 2.17 & 
680.32 & 1.89 & 680.04 & 0.00
 & 0.04\\SCA3-4 & \bf{690.50} & 2.87 & 
690.50 & 3.25 & 690.50 & 0.00
 & 0.00\\
SCA3-5 & \bf{659.90} & 2.62 & 
659.90 & 2.34 & 659.90 & 0.00
 & 0.00\\
SCA3-6 & \bf{651.09} & 2.41 & 
654.10 & 1.66 & 651.09 & 0.00
 & 0.46\\SCA3-7 & 666.15 & 1.39 & 
671.09 & 1.69 & \bf{659.17} & 
1.06 & 1.81\\SCA3-8 & \bf{719.47} & 2.35 & 
719.54 & 2.67 & 719.47 & 0.00
 & 0.01\\SCA3-9 & \bf{681.00} & 1.88 & 
681.00 & 1.84 & 681.00 & 0.00
 & 0.00\\
SCA8-0 & 979.79 & 1.75 & 
990.46 & 1.85 & \bf{961.50} & 
1.90 & 3.01\\SCA8-1 & \bf{1049.65} & 2.65 & 
1066.41 & 1.99 & 1049.65 & 0.00
 & 1.60\\SCA8-2 & \bf{1039.64} & 1.70 & 
1050.34 & 2.03 & 1039.64 & 0.00
 & 1.03\\SCA8-3 & \bf{983.34} & 1.34 & 
1000.62 & 2.23 & 983.34 & 0.00
 & 1.76\\SCA8-4 & 1067.28 & 3.40 & 
1069.31 & 2.23 & \bf{1065.49} & 
0.17 & 0.36\\SCA8-5 & 1034.32 & 5.23 & 
1047.26 & 2.49 & \bf{1027.08} & 
0.70 & 1.96\\SCA8-6 & 972.48 & 2.70 & 
981.50 & 2.13 & \bf{971.82} & 
0.07 & 1.00\\SCA8-7 & 1063.22 & 3.19 & 
1085.34 & 2.09 & \bf{1051.28} & 
1.14 & 3.24\\SCA8-8 & \bf{1071.18} & 5.20 & 
1076.27 & 2.29 & 1071.18 & 0.00
 & 0.48\\SCA8-9 & \bf{1060.50} & 2.24 & 
1062.23 & 3.25 & 1060.50 & 0.00
 & 0.16\\CON3-0 & \bf{616.52} & 1.38 & 
626.88 & 2.00 & 616.52 & 0.00
 & 1.68\\CON3-1 & \bf{554.47} & 2.16 & 
557.33 & 2.00 & 554.47 & 0.00
 & 0.51\\CON3-2 & 521.36 & 1.02 & 
523.33 & 2.18 & \bf{518.00} & 
0.65 & 1.03\\CON3-3 & \bf{591.19} & 1.53 & 
602.94 & 2.59 & 591.19 & 0.00
 & 1.99\\CON3-4 & 596.29 & 1.25 & 
597.50 & 2.65 & \bf{588.79} & 
1.27 & 1.48\\CON3-5 & \bf{563.70} & 4.00 & 
565.92 & 2.48 & 563.70 & 0.00
 & 0.39\\CON3-6 & \bf{499.05} & 2.96 & 
499.71 & 2.14 & 499.05 & 0.00
 & 0.13\\CON3-7 & 578.41 & 1.32 & 
586.16 & 3.33 & \bf{576.48} & 
0.33 & 1.68\\CON3-8 & \bf{523.05} & 4.23 & 
523.21 & 2.25 & 523.05 & 0.00
 & 0.03\\CON3-9 & 588.40 & 1.34 & 
589.94 & 1.92 & \bf{578.24} & 
1.76 & 2.02\\CON8-0 & 876.08 & 2.59 & 
899.59 & 2.87 & \bf{857.17} & 
2.21 & 4.95\\CON8-1 & \bf{740.85} & 2.78 & 
745.21 & 3.54 & 740.85 & 0.00
 & 0.59\\CON8-2 & 718.70 & 2.94 & 
722.86 & 3.50 & \bf{712.89} & 
0.81 & 1.40\\CON8-3 & \bf{811.07} & 3.40 & 
836.69 & 2.76 & 811.07 & 0.00
 & 3.16\\CON8-4 & 784.83 & 2.16 & 
787.88 & 2.74 & \bf{772.25} & 
1.63 & 2.02\\CON8-5 & 759.93 & 3.58 & 
759.98 & 3.31 & \bf{754.88} & 
0.67 & 0.68\\CON8-6 & 688.68 & 2.71 & 
694.01 & 2.57 & \bf{678.92} & 
1.44 & 2.22\\CON8-7 & 812.89 & 3.12 & 
820.15 & 3.45 & \bf{811.96} & 
0.11 & 1.01\\CON8-8 & 771.32 & 2.68 & 
779.91 & 1.74 & \bf{767.53} & 
0.49 & 1.61\\CON8-9 & \bf{809.00} & 1.52 & 
809.84 & 2.31 & 809.00 & 0.00
 & 0.10\\\bf{PROM.} & 
\bf{761.71} & \bf{2.48} & \bf{767.85} & \bf{2.40} & \bf{758.54} & \bf{0.41} & \bf{1.16}\\[1ex]\hline
\end{tabular}
\label{table:nonlin}
\end{table} \clearpage
\begin{table}[ht]
\caption{Resultados de la ejecución de la metaheurística GTS, utilizando instancias de SalhiNagy con la configuración -mni 3000 -lambda1 0.05 -lambda2 0.05 -tabu 21}
\centering
\small
\begin{tabular}{c c c c c c c c}
\hline\hline
Instancia & Costo mínimo & Tiempo(seg.) & Costo promedio & Tiempo promedio(seg.) & CME & \%G & \%GP \\ [0.5ex]
\hline
CMT1X & \bf{470.48} & 1.69 & 
474.13 & 1.76 & 470.48 & 0.00
 & 0.78\\CMT1Y & 470.67 & 4.20 & 
474.18 & 3.64 & \bf{470.48} & 
0.04 & 0.79\\CMT2X & 683.95 & 8.23 & 
690.17 & 4.96 & \bf{682.39} & 
0.23 & 1.14\\CMT2Y & 688.78 & 5.05 & 
693.01 & 5.00 & \bf{682.39} & 
0.94 & 1.56\\CMT3X & 732.00 & 6.48 & 
739.20 & 8.74 & \bf{719.06} & 
1.80 & 2.80\\CMT3Y & 724.74 & 4.74 & 
732.65 & 6.85 & \bf{719.06} & 
0.79 & 1.89\\CMT4X & 866.79 & 30.60 & 
875.36 & 23.89 & \bf{854.21} & 
1.47 & 2.48\\CMT4Y & 864.37 & 18.30 & 
875.55 & 23.20 & \bf{852.46} & 
1.40 & 2.71\\CMT5X & 1053.14 & 57.09 & 
1067.41 & 56.53 & \bf{1030.56} & 
2.19 & 3.58\\CMT5Y & 1082.49 & 33.53 & 
1096.12 & 28.27 & \bf{1031.69} & 
4.92 & 6.25\\CMT11X & 921.99 & 13.82 & 
930.57 & 16.04 & \bf{831.09} & 
10.94 & 11.97\\CMT11Y & 862.36 & 21.46 & 
900.31 & 19.86 & \bf{829.85} & 
3.92 & 8.49\\CMT12X & 673.26 & 7.09 & 
676.16 & 7.19 & \bf{658.83} & 
2.19 & 2.63\\CMT12Y & 673.80 & 6.66 & 
683.91 & 7.01 & \bf{660.47} & 
2.02 & 3.55\\\bf{PROM.} & 
\bf{769.20} & \bf{15.64} & \bf{779.19} & \bf{15.21} & \bf{749.50} & \bf{2.35} & \bf{3.61}\\[1ex]\hline
\end{tabular}
\label{table:nonlin}
\end{table} \clearpage
\begin{table}[ht]
\caption{Resultados de la ejecución de la metaheurística GTS, utilizando instancias de Dethloff con la configuración -mni 3000 -lambda1 0.05 -lambda2 0.05 -tabu 25}
\centering
\small
\begin{tabular}{c c c c c c c c}
\hline\hline
Instancia & Costo mínimo & Tiempo(seg.) & Costo promedio & Tiempo promedio(seg.) & CME & \%G & \%GP \\ [0.5ex]
\hline
SCA3-0 & 636.06 & 1.77 & 
639.43 & 2.54 & \bf{635.62} & 
0.07 & 0.60\\SCA3-1 & \bf{697.84} & 2.80 & 
699.17 & 2.57 & 697.84 & 0.00
 & 0.19\\SCA3-2 & \bf{659.34} & 1.54 & 
659.34 & 2.83 & 659.34 & 0.00
 & 0.00\\
SCA3-3 & \bf{680.04} & 7.16 & 
687.03 & 2.73 & 680.04 & 0.00
 & 1.03\\SCA3-4 & \bf{690.50} & 4.17 & 
699.27 & 2.48 & 690.50 & 0.00
 & 1.27\\SCA3-5 & \bf{659.90} & 1.14 & 
659.90 & 2.24 & 659.90 & 0.00
 & 0.00\\
SCA3-6 & \bf{651.09} & 1.84 & 
651.09 & 2.45 & 651.09 & 0.00
 & 0.00\\
SCA3-7 & 666.15 & 2.58 & 
666.15 & 2.37 & \bf{659.17} & 
1.06 & 1.06\\SCA3-8 & \bf{719.47} & 2.34 & 
719.47 & 3.21 & 719.47 & 0.00
 & 0.00\\
SCA3-9 & \bf{681.00} & 2.40 & 
681.00 & 2.66 & 681.00 & 0.00
 & 0.00\\
SCA8-0 & \bf{961.50} & 5.15 & 
969.79 & 3.70 & 961.50 & 0.00
 & 0.86\\SCA8-1 & 1050.38 & 4.37 & 
1071.57 & 2.33 & \bf{1049.65} & 
0.07 & 2.09\\SCA8-2 & 1042.10 & 2.11 & 
1051.72 & 2.44 & \bf{1039.64} & 
0.24 & 1.16\\SCA8-3 & \bf{983.34} & 5.44 & 
1004.21 & 2.84 & 983.34 & 0.00
 & 2.12\\SCA8-4 & 1067.28 & 2.53 & 
1073.79 & 2.77 & \bf{1065.49} & 
0.17 & 0.78\\SCA8-5 & 1042.30 & 1.53 & 
1057.34 & 1.91 & \bf{1027.08} & 
1.48 & 2.95\\SCA8-6 & 972.48 & 2.08 & 
978.33 & 2.13 & \bf{971.82} & 
0.07 & 0.67\\SCA8-7 & 1052.17 & 4.50 & 
1062.87 & 3.33 & \bf{1051.28} & 
0.08 & 1.10\\SCA8-8 & \bf{1071.18} & 1.10 & 
1079.38 & 1.34 & 1071.18 & 0.00
 & 0.77\\SCA8-9 & \bf{1060.50} & 2.38 & 
1066.06 & 3.32 & 1060.50 & 0.00
 & 0.52\\CON3-0 & \bf{616.52} & 1.45 & 
622.84 & 3.10 & 616.52 & 0.00
 & 1.03\\CON3-1 & 556.78 & 1.27 & 
562.10 & 2.05 & \bf{554.47} & 
0.42 & 1.38\\CON3-2 & 523.23 & 4.24 & 
523.77 & 2.98 & \bf{518.00} & 
1.01 & 1.11\\CON3-3 & \bf{591.19} & 1.22 & 
597.89 & 2.90 & 591.19 & 0.00
 & 1.13\\CON3-4 & \bf{588.79} & 4.11 & 
590.66 & 3.15 & 588.79 & 0.00
 & 0.32\\CON3-5 & \bf{563.70} & 1.53 & 
570.25 & 1.55 & 563.70 & 0.00
 & 1.16\\CON3-6 & \bf{499.05} & 2.38 & 
501.72 & 1.99 & 499.05 & 0.00
 & 0.54\\CON3-7 & \bf{576.48} & 1.75 & 
587.35 & 1.57 & 576.48 & 0.00
 & 1.89\\CON3-8 & \bf{523.05} & 1.41 & 
523.09 & 2.84 & 523.05 & 0.00
 & 0.01\\CON3-9 & 578.25 & 1.57 & 
582.48 & 2.04 & \bf{578.24} & 
0.00 & 0.73\\CON8-0 & 867.93 & 3.22 & 
895.85 & 3.23 & \bf{857.17} & 
1.26 & 4.51\\CON8-1 & \bf{740.85} & 1.84 & 
754.74 & 2.69 & 740.85 & 0.00
 & 1.88\\CON8-2 & 713.44 & 1.74 & 
728.58 & 1.96 & \bf{712.89} & 
0.08 & 2.20\\CON8-3 & 821.26 & 1.74 & 
833.53 & 1.96 & \bf{811.07} & 
1.26 & 2.77\\CON8-4 & \bf{772.25} & 1.17 & 
782.69 & 2.63 & 772.25 & 0.00
 & 1.35\\CON8-5 & 754.95 & 2.08 & 
757.70 & 2.18 & \bf{754.88} & 
0.01 & 0.37\\CON8-6 & \bf{678.92} & 3.64 & 
687.61 & 4.25 & 678.92 & 0.00
 & 1.28\\CON8-7 & 812.26 & 1.48 & 
812.94 & 2.47 & \bf{811.96} & 
0.04 & 0.12\\CON8-8 & \bf{767.53} & 5.36 & 
778.43 & 4.16 & 767.53 & 0.00
 & 1.42\\CON8-9 & 814.96 & 3.11 & 
823.55 & 2.29 & \bf{809.00} & 
0.74 & 1.80\\\bf{PROM.} & 
\bf{760.15} & \bf{2.63} & \bf{767.37} & \bf{2.60} & \bf{758.54} & \bf{0.20} & \bf{1.10}\\[1ex]\hline
\end{tabular}
\label{table:nonlin}
\end{table} \clearpage
\begin{table}[ht]
\caption{Resultados de la ejecución de la metaheurística GTS, utilizando instancias de SalhiNagy con la configuración -mni 3000 -lambda1 0.05 -lambda2 0.05 -tabu 25}
\centering
\small
\begin{tabular}{c c c c c c c c}
\hline\hline
Instancia & Costo mínimo & Tiempo(seg.) & Costo promedio & Tiempo promedio(seg.) & CME & \%G & \%GP \\ [0.5ex]
\hline
CMT1X & \bf{470.48} & 3.36 & 
471.43 & 2.91 & 470.48 & 0.00
 & 0.20\\CMT1Y & \bf{470.48} & 2.50 & 
473.81 & 3.06 & 470.48 & 0.00
 & 0.71\\CMT2X & 687.70 & 8.31 & 
690.85 & 5.82 & \bf{682.39} & 
0.78 & 1.24\\CMT2Y & 684.40 & 3.29 & 
687.76 & 6.57 & \bf{682.39} & 
0.29 & 0.79\\CMT3X & 723.78 & 15.18 & 
727.21 & 11.70 & \bf{719.06} & 
0.66 & 1.13\\CMT3Y & 724.47 & 7.48 & 
726.34 & 12.04 & \bf{719.06} & 
0.75 & 1.01\\CMT4X & 863.20 & 35.67 & 
873.74 & 33.28 & \bf{854.21} & 
1.05 & 2.29\\CMT4Y & 870.80 & 29.46 & 
881.56 & 26.14 & \bf{852.46} & 
2.15 & 3.41\\CMT5X & 1054.05 & 83.57 & 
1062.51 & 70.77 & \bf{1030.56} & 
2.28 & 3.10\\CMT5Y & 1067.91 & 50.87 & 
1072.82 & 59.76 & \bf{1031.69} & 
3.51 & 3.99\\CMT11X & 872.71 & 47.58 & 
888.68 & 27.84 & \bf{831.09} & 
5.01 & 6.93\\CMT11Y & 901.30 & 22.71 & 
941.16 & 17.09 & \bf{829.85} & 
8.61 & 13.41\\CMT12X & 673.32 & 6.06 & 
674.65 & 8.40 & \bf{658.83} & 
2.20 & 2.40\\CMT12Y & 664.46 & 20.91 & 
676.68 & 12.47 & \bf{660.47} & 
0.60 & 2.45\\\bf{PROM.} & 
\bf{766.36} & \bf{24.07} & \bf{774.94} & \bf{21.28} & \bf{749.50} & \bf{1.99} & \bf{3.08}\\[1ex]\hline
\end{tabular}
\label{table:nonlin}
\end{table} \clearpage
\begin{table}[ht]
\caption{Resultados de la ejecución de la metaheurística GTS, utilizando instancias de Dethloff con la configuración -mni 3000 -lambda1 0.05 -lambda2 0.05 -tabu 29}
\centering
\small
\begin{tabular}{c c c c c c c c}
\hline\hline
Instancia & Costo mínimo & Tiempo(seg.) & Costo promedio & Tiempo promedio(seg.) & CME & \%G & \%GP \\ [0.5ex]
\hline
SCA3-0 & 640.55 & 1.53 & 
640.55 & 2.19 & \bf{635.62} & 
0.78 & 0.78\\SCA3-1 & \bf{697.84} & 2.10 & 
699.17 & 2.77 & 697.84 & 0.00
 & 0.19\\SCA3-2 & \bf{659.34} & 1.53 & 
659.34 & 2.44 & 659.34 & 0.00
 & 0.00\\
SCA3-3 & \bf{680.04} & 1.32 & 
680.32 & 2.09 & 680.04 & 0.00
 & 0.04\\SCA3-4 & \bf{690.50} & 2.52 & 
690.50 & 2.82 & 690.50 & 0.00
 & 0.00\\
SCA3-5 & \bf{659.90} & 3.39 & 
666.42 & 2.09 & 659.90 & 0.00
 & 0.99\\SCA3-6 & \bf{651.09} & 2.53 & 
651.37 & 3.03 & 651.09 & 0.00
 & 0.04\\SCA3-7 & 666.15 & 1.97 & 
667.09 & 2.08 & \bf{659.17} & 
1.06 & 1.20\\SCA3-8 & \bf{719.47} & 2.39 & 
719.47 & 2.67 & 719.47 & 0.00
 & 0.00\\
SCA3-9 & \bf{681.00} & 2.46 & 
681.00 & 1.76 & 681.00 & 0.00
 & 0.00\\
SCA8-0 & \bf{961.50} & 2.70 & 
981.89 & 2.04 & 961.50 & 0.00
 & 2.12\\SCA8-1 & 1050.38 & 1.64 & 
1060.47 & 2.48 & \bf{1049.65} & 
0.07 & 1.03\\SCA8-2 & 1039.71 & 1.45 & 
1042.99 & 1.62 & \bf{1039.64} & 
0.01 & 0.32\\SCA8-3 & 1002.38 & 1.44 & 
1012.59 & 1.50 & \bf{983.34} & 
1.94 & 2.97\\SCA8-4 & 1067.28 & 1.58 & 
1068.64 & 2.40 & \bf{1065.49} & 
0.17 & 0.30\\SCA8-5 & 1042.43 & 1.48 & 
1058.44 & 1.93 & \bf{1027.08} & 
1.49 & 3.05\\SCA8-6 & 972.48 & 1.66 & 
977.43 & 2.14 & \bf{971.82} & 
0.07 & 0.58\\SCA8-7 & 1053.84 & 1.65 & 
1067.64 & 3.18 & \bf{1051.28} & 
0.24 & 1.56\\SCA8-8 & 1082.12 & 3.28 & 
1091.10 & 1.88 & \bf{1071.18} & 
1.02 & 1.86\\SCA8-9 & \bf{1060.50} & 8.53 & 
1070.11 & 3.91 & 1060.50 & 0.00
 & 0.91\\CON3-0 & \bf{616.52} & 2.40 & 
626.18 & 1.75 & 616.52 & 0.00
 & 1.57\\CON3-1 & 556.28 & 4.71 & 
558.74 & 2.79 & \bf{554.47} & 
0.33 & 0.77\\CON3-2 & 519.61 & 2.08 & 
523.94 & 1.96 & \bf{518.00} & 
0.31 & 1.15\\CON3-3 & \bf{591.19} & 4.78 & 
603.40 & 3.71 & 591.19 & 0.00
 & 2.06\\CON3-4 & \bf{588.79} & 2.42 & 
595.61 & 2.73 & 588.79 & 0.00
 & 1.16\\CON3-5 & \bf{563.70} & 1.09 & 
569.64 & 1.26 & 563.70 & 0.00
 & 1.05\\CON3-6 & \bf{499.05} & 2.41 & 
500.40 & 2.32 & 499.05 & 0.00
 & 0.27\\CON3-7 & \bf{576.48} & 2.18 & 
586.02 & 2.38 & 576.48 & 0.00
 & 1.65\\CON3-8 & \bf{523.05} & 2.11 & 
523.21 & 1.93 & 523.05 & 0.00
 & 0.03\\CON3-9 & 578.25 & 3.23 & 
586.41 & 2.55 & \bf{578.24} & 
0.00 & 1.41\\CON8-0 & 857.40 & 1.97 & 
868.43 & 5.08 & \bf{857.17} & 
0.03 & 1.31\\CON8-1 & \bf{740.85} & 3.10 & 
756.38 & 3.14 & 740.85 & 0.00
 & 2.10\\CON8-2 & \bf{712.89} & 2.00 & 
727.86 & 3.02 & 712.89 & 0.00
 & 2.10\\CON8-3 & 826.12 & 3.29 & 
828.89 & 1.98 & \bf{811.07} & 
1.86 & 2.20\\CON8-4 & \bf{772.25} & 2.58 & 
778.42 & 3.17 & 772.25 & 0.00
 & 0.80\\CON8-5 & 759.93 & 2.26 & 
760.17 & 2.80 & \bf{754.88} & 
0.67 & 0.70\\CON8-6 & \bf{678.92} & 1.33 & 
689.98 & 2.01 & 678.92 & 0.00
 & 1.63\\CON8-7 & 813.91 & 3.88 & 
837.75 & 2.33 & \bf{811.96} & 
0.24 & 3.18\\CON8-8 & \bf{767.53} & 2.30 & 
772.67 & 1.95 & 767.53 & 0.00
 & 0.67\\CON8-9 & \bf{809.00} & 4.25 & 
817.08 & 2.88 & 809.00 & 0.00
 & 1.00\\\bf{PROM.} & 
\bf{760.76} & \bf{2.54} & \bf{767.44} & \bf{2.47} & \bf{758.54} & \bf{0.26} & \bf{1.12}\\[1ex]\hline
\end{tabular}
\label{table:nonlin}
\end{table} \clearpage
\begin{table}[ht]
\caption{Resultados de la ejecución de la metaheurística GTS, utilizando instancias de SalhiNagy con la configuración -mni 3000 -lambda1 0.05 -lambda2 0.05 -tabu 29}
\centering
\small
\begin{tabular}{c c c c c c c c}
\hline\hline
Instancia & Costo mínimo & Tiempo(seg.) & Costo promedio & Tiempo promedio(seg.) & CME & \%G & \%GP \\ [0.5ex]
\hline
CMT1X & \bf{470.48} & 1.87 & 
470.92 & 2.22 & 470.48 & 0.00
 & 0.09\\CMT1Y & 472.37 & 2.26 & 
476.12 & 1.91 & \bf{470.48} & 
0.40 & 1.20\\CMT2X & 683.24 & 3.37 & 
688.90 & 3.79 & \bf{682.39} & 
0.12 & 0.95\\CMT2Y & 684.24 & 4.74 & 
686.00 & 6.74 & \bf{682.39} & 
0.27 & 0.53\\CMT3X & 725.25 & 9.50 & 
730.82 & 7.51 & \bf{719.06} & 
0.86 & 1.64\\CMT3Y & 723.67 & 3.33 & 
726.16 & 8.43 & \bf{719.06} & 
0.64 & 0.99\\CMT4X & 861.29 & 19.30 & 
874.26 & 24.07 & \bf{854.21} & 
0.83 & 2.35\\CMT4Y & 857.86 & 44.85 & 
870.08 & 40.63 & \bf{852.46} & 
0.63 & 2.07\\CMT5X & 1055.29 & 71.62 & 
1066.52 & 58.18 & \bf{1030.56} & 
2.40 & 3.49\\CMT5Y & 1054.81 & 81.07 & 
1068.03 & 51.84 & \bf{1031.69} & 
2.24 & 3.52\\CMT11X & 883.60 & 27.35 & 
923.70 & 25.00 & \bf{831.09} & 
6.32 & 11.14\\CMT11Y & 879.29 & 53.60 & 
903.25 & 26.61 & \bf{829.85} & 
5.96 & 8.85\\CMT12X & 673.13 & 16.37 & 
677.44 & 9.21 & \bf{658.83} & 
2.17 & 2.82\\CMT12Y & 670.27 & 18.54 & 
676.88 & 10.79 & \bf{660.47} & 
1.48 & 2.48\\\bf{PROM.} & 
\bf{763.91} & \bf{25.56} & \bf{774.22} & \bf{19.78} & \bf{749.50} & \bf{1.74} & \bf{3.01}\\[1ex]\hline
\end{tabular}
\label{table:nonlin}
\end{table} \clearpage
\begin{table}[ht]
\caption{Resultados de la ejecución de la metaheurística GTS, utilizando instancias de Dethloff con la configuración -mni 3000 -lambda1 0.05 -lambda2 0.05 -tabu 33}
\centering
\small
\begin{tabular}{c c c c c c c c}
\hline\hline
Instancia & Costo mínimo & Tiempo(seg.) & Costo promedio & Tiempo promedio(seg.) & CME & \%G & \%GP \\ [0.5ex]
\hline
SCA3-0 & 636.06 & 1.61 & 
639.43 & 2.99 & \bf{635.62} & 
0.07 & 0.60\\SCA3-1 & \bf{697.84} & 2.88 & 
697.84 & 1.82 & 697.84 & 0.00
 & 0.00\\
SCA3-2 & \bf{659.34} & 1.42 & 
659.34 & 2.55 & 659.34 & 0.00
 & 0.00\\
SCA3-3 & \bf{680.04} & 1.36 & 
682.62 & 2.51 & 680.04 & 0.00
 & 0.38\\SCA3-4 & \bf{690.50} & 4.27 & 
696.75 & 2.55 & 690.50 & 0.00
 & 0.91\\SCA3-5 & \bf{659.90} & 1.79 & 
663.16 & 2.20 & 659.90 & 0.00
 & 0.49\\SCA3-6 & \bf{651.09} & 3.40 & 
651.55 & 2.83 & 651.09 & 0.00
 & 0.07\\SCA3-7 & 666.15 & 3.12 & 
667.20 & 3.14 & \bf{659.17} & 
1.06 & 1.22\\SCA3-8 & \bf{719.47} & 3.51 & 
726.01 & 2.86 & 719.47 & 0.00
 & 0.91\\SCA3-9 & \bf{681.00} & 1.87 & 
681.00 & 2.62 & 681.00 & 0.00
 & 0.00\\
SCA8-0 & 970.64 & 2.44 & 
980.29 & 2.73 & \bf{961.50} & 
0.95 & 1.95\\SCA8-1 & \bf{1049.65} & 2.80 & 
1063.81 & 2.01 & 1049.65 & 0.00
 & 1.35\\SCA8-2 & 1050.37 & 1.58 & 
1064.34 & 1.89 & \bf{1039.64} & 
1.03 & 2.38\\SCA8-3 & 1007.33 & 1.41 & 
1011.28 & 2.04 & \bf{983.34} & 
2.44 & 2.84\\SCA8-4 & 1067.55 & 1.85 & 
1078.68 & 2.17 & \bf{1065.49} & 
0.19 & 1.24\\SCA8-5 & 1040.66 & 1.98 & 
1043.70 & 2.69 & \bf{1027.08} & 
1.32 & 1.62\\SCA8-6 & \bf{971.82} & 1.42 & 
972.32 & 1.61 & 971.82 & 0.00
 & 0.05\\SCA8-7 & 1054.73 & 2.90 & 
1081.03 & 3.47 & \bf{1051.28} & 
0.33 & 2.83\\SCA8-8 & 1082.12 & 1.69 & 
1085.64 & 1.65 & \bf{1071.18} & 
1.02 & 1.35\\SCA8-9 & \bf{1060.50} & 1.76 & 
1065.33 & 2.73 & 1060.50 & 0.00
 & 0.46\\CON3-0 & \bf{616.52} & 2.80 & 
625.33 & 2.23 & 616.52 & 0.00
 & 1.43\\CON3-1 & \bf{554.47} & 2.90 & 
556.73 & 2.60 & 554.47 & 0.00
 & 0.41\\CON3-2 & 522.86 & 2.57 & 
523.48 & 3.14 & \bf{518.00} & 
0.94 & 1.06\\CON3-3 & \bf{591.19} & 3.35 & 
608.83 & 2.57 & 591.19 & 0.00
 & 2.98\\CON3-4 & 591.43 & 1.07 & 
599.20 & 1.26 & \bf{588.79} & 
0.45 & 1.77\\CON3-5 & \bf{563.70} & 3.86 & 
572.00 & 2.98 & 563.70 & 0.00
 & 1.47\\CON3-6 & \bf{499.05} & 3.86 & 
500.61 & 3.17 & 499.05 & 0.00
 & 0.31\\CON3-7 & \bf{576.48} & 6.15 & 
578.72 & 2.85 & 576.48 & 0.00
 & 0.39\\CON3-8 & \bf{523.05} & 2.13 & 
523.07 & 2.40 & 523.05 & 0.00
 & 0.00\\CON3-9 & 578.25 & 1.37 & 
580.87 & 2.92 & \bf{578.24} & 
0.00 & 0.46\\CON8-0 & 857.40 & 1.98 & 
865.12 & 2.07 & \bf{857.17} & 
0.03 & 0.93\\CON8-1 & \bf{740.85} & 4.06 & 
749.36 & 2.95 & 740.85 & 0.00
 & 1.15\\CON8-2 & 718.64 & 4.04 & 
732.41 & 3.21 & \bf{712.89} & 
0.81 & 2.74\\CON8-3 & 827.25 & 1.54 & 
851.76 & 3.14 & \bf{811.07} & 
1.99 & 5.02\\CON8-4 & \bf{772.25} & 2.71 & 
791.76 & 2.02 & 772.25 & 0.00
 & 2.53\\CON8-5 & 756.91 & 2.31 & 
758.27 & 2.81 & \bf{754.88} & 
0.27 & 0.45\\CON8-6 & 692.75 & 2.06 & 
697.58 & 2.15 & \bf{678.92} & 
2.04 & 2.75\\CON8-7 & 814.79 & 1.93 & 
815.04 & 3.60 & \bf{811.96} & 
0.35 & 0.38\\CON8-8 & \bf{767.53} & 1.72 & 
782.99 & 2.55 & 767.53 & 0.00
 & 2.01\\CON8-9 & \bf{809.00} & 3.03 & 
811.67 & 2.47 & 809.00 & 0.00
 & 0.33\\\bf{PROM.} & 
\bf{761.78} & \bf{2.51} & \bf{768.40} & \bf{2.55} & \bf{758.54} & \bf{0.38} & \bf{1.23}\\[1ex]\hline
\end{tabular}
\label{table:nonlin}
\end{table} \clearpage
\begin{table}[ht]
\caption{Resultados de la ejecución de la metaheurística GTS, utilizando instancias de SalhiNagy con la configuración -mni 3000 -lambda1 0.05 -lambda2 0.05 -tabu 33}
\centering
\small
\begin{tabular}{c c c c c c c c}
\hline\hline
Instancia & Costo mínimo & Tiempo(seg.) & Costo promedio & Tiempo promedio(seg.) & CME & \%G & \%GP \\ [0.5ex]
\hline
CMT1X & \bf{470.48} & 1.35 & 
473.66 & 2.29 & 470.48 & 0.00
 & 0.68\\CMT1Y & \bf{470.48} & 2.56 & 
471.20 & 2.84 & 470.48 & 0.00
 & 0.15\\CMT2X & 683.95 & 16.84 & 
687.26 & 8.15 & \bf{682.39} & 
0.23 & 0.71\\CMT2Y & 685.92 & 5.11 & 
690.39 & 4.03 & \bf{682.39} & 
0.52 & 1.17\\CMT3X & 724.86 & 6.18 & 
727.73 & 7.62 & \bf{719.06} & 
0.81 & 1.21\\CMT3Y & 725.72 & 9.59 & 
726.74 & 9.15 & \bf{719.06} & 
0.93 & 1.07\\CMT4X & 874.88 & 51.67 & 
877.42 & 31.06 & \bf{854.21} & 
2.42 & 2.72\\CMT4Y & 874.19 & 27.60 & 
883.05 & 22.06 & \bf{852.46} & 
2.55 & 3.59\\CMT5X & 1055.94 & 82.23 & 
1068.28 & 60.84 & \bf{1030.56} & 
2.46 & 3.66\\CMT5Y & 1051.04 & 48.39 & 
1078.16 & 52.87 & \bf{1031.69} & 
1.88 & 4.50\\CMT11X & 875.99 & 39.22 & 
904.68 & 34.76 & \bf{831.09} & 
5.40 & 8.85\\CMT11Y & 879.37 & 33.26 & 
904.97 & 22.09 & \bf{829.85} & 
5.97 & 9.05\\CMT12X & 671.88 & 5.78 & 
675.67 & 8.11 & \bf{658.83} & 
1.98 & 2.56\\CMT12Y & 673.59 & 6.10 & 
680.76 & 5.53 & \bf{660.47} & 
1.99 & 3.07\\\bf{PROM.} & 
\bf{765.59} & \bf{23.99} & \bf{775.00} & \bf{19.38} & \bf{749.50} & \bf{1.94} & \bf{3.07}\\[1ex]\hline
\end{tabular}
\label{table:nonlin}
\end{table} \clearpage
\begin{table}[ht]
\caption{Resultados de la ejecución de la metaheurística GTS, utilizando instancias de Dethloff con la configuración -mni 3000 -lambda1 0.05 -lambda2 0.05 -tabu 37}
\centering
\small
\begin{tabular}{c c c c c c c c}
\hline\hline
Instancia & Costo mínimo & Tiempo(seg.) & Costo promedio & Tiempo promedio(seg.) & CME & \%G & \%GP \\ [0.5ex]
\hline
SCA3-0 & 636.45 & 1.81 & 
638.59 & 2.32 & \bf{635.62} & 
0.13 & 0.47\\SCA3-1 & \bf{697.84} & 4.87 & 
700.09 & 2.52 & 697.84 & 0.00
 & 0.32\\SCA3-2 & \bf{659.34} & 1.75 & 
666.01 & 1.71 & 659.34 & 0.00
 & 1.01\\SCA3-3 & \bf{680.04} & 3.23 & 
682.94 & 3.17 & 680.04 & 0.00
 & 0.43\\SCA3-4 & \bf{690.50} & 1.58 & 
690.50 & 3.13 & 690.50 & 0.00
 & 0.00\\
SCA3-5 & \bf{659.90} & 2.10 & 
663.16 & 2.10 & 659.90 & 0.00
 & 0.49\\SCA3-6 & \bf{651.09} & 1.11 & 
656.65 & 1.31 & 651.09 & 0.00
 & 0.85\\SCA3-7 & 666.15 & 2.02 & 
667.09 & 2.38 & \bf{659.17} & 
1.06 & 1.20\\SCA3-8 & \bf{719.47} & 3.56 & 
726.01 & 2.58 & 719.47 & 0.00
 & 0.91\\SCA3-9 & \bf{681.00} & 3.92 & 
681.00 & 3.08 & 681.00 & 0.00
 & 0.00\\
SCA8-0 & \bf{961.50} & 4.69 & 
982.80 & 2.83 & 961.50 & 0.00
 & 2.22\\SCA8-1 & \bf{1049.65} & 7.26 & 
1066.14 & 3.63 & 1049.65 & 0.00
 & 1.57\\SCA8-2 & 1050.37 & 2.40 & 
1062.19 & 2.53 & \bf{1039.64} & 
1.03 & 2.17\\SCA8-3 & \bf{983.34} & 3.56 & 
1007.40 & 2.59 & 983.34 & 0.00
 & 2.45\\SCA8-4 & \bf{1065.49} & 4.02 & 
1068.96 & 3.12 & 1065.49 & 0.00
 & 0.33\\SCA8-5 & \bf{1027.08} & 1.75 & 
1041.95 & 1.62 & 1027.08 & 0.00
 & 1.45\\SCA8-6 & 972.48 & 3.38 & 
972.48 & 2.22 & \bf{971.82} & 
0.07 & 0.07\\SCA8-7 & 1052.17 & 4.70 & 
1064.93 & 3.14 & \bf{1051.28} & 
0.08 & 1.30\\SCA8-8 & 1082.12 & 3.95 & 
1083.40 & 1.96 & \bf{1071.18} & 
1.02 & 1.14\\SCA8-9 & \bf{1060.50} & 5.33 & 
1065.17 & 3.69 & 1060.50 & 0.00
 & 0.44\\CON3-0 & 617.59 & 2.02 & 
626.84 & 1.90 & \bf{616.52} & 
0.17 & 1.67\\CON3-1 & \bf{554.47} & 1.44 & 
555.36 & 1.95 & 554.47 & 0.00
 & 0.16\\CON3-2 & 519.61 & 2.59 & 
522.07 & 2.56 & \bf{518.00} & 
0.31 & 0.79\\CON3-3 & \bf{591.19} & 3.58 & 
597.89 & 2.99 & 591.19 & 0.00
 & 1.13\\CON3-4 & \bf{588.79} & 2.63 & 
594.81 & 2.04 & 588.79 & 0.00
 & 1.02\\CON3-5 & \bf{563.70} & 2.42 & 
567.38 & 2.29 & 563.70 & 0.00
 & 0.65\\CON3-6 & 502.16 & 2.77 & 
505.10 & 2.11 & \bf{499.05} & 
0.62 & 1.21\\CON3-7 & \bf{576.48} & 2.47 & 
576.96 & 3.83 & 576.48 & 0.00
 & 0.08\\CON3-8 & \bf{523.05} & 1.14 & 
523.05 & 2.38 & 523.05 & 0.00
 & 0.00\\
CON3-9 & 578.25 & 2.48 & 
579.57 & 3.41 & \bf{578.24} & 
0.00 & 0.23\\CON8-0 & 867.04 & 2.39 & 
878.76 & 2.84 & \bf{857.17} & 
1.15 & 2.52\\CON8-1 & \bf{740.85} & 4.23 & 
750.94 & 2.62 & 740.85 & 0.00
 & 1.36\\CON8-2 & 716.03 & 5.23 & 
725.38 & 3.61 & \bf{712.89} & 
0.44 & 1.75\\CON8-3 & \bf{811.07} & 4.02 & 
825.47 & 3.39 & 811.07 & 0.00
 & 1.77\\CON8-4 & \bf{772.25} & 3.40 & 
780.75 & 2.90 & 772.25 & 0.00
 & 1.10\\CON8-5 & 755.14 & 6.35 & 
761.58 & 3.08 & \bf{754.88} & 
0.03 & 0.89\\CON8-6 & 692.75 & 2.76 & 
700.62 & 2.30 & \bf{678.92} & 
2.04 & 3.20\\CON8-7 & 812.89 & 4.42 & 
817.78 & 3.16 & \bf{811.96} & 
0.11 & 0.72\\CON8-8 & \bf{767.53} & 3.50 & 
778.01 & 2.54 & 767.53 & 0.00
 & 1.37\\CON8-9 & 812.03 & 2.64 & 
815.42 & 2.50 & \bf{809.00} & 
0.37 & 0.79\\\bf{PROM.} & 
\bf{760.23} & \bf{3.24} & \bf{766.78} & \bf{2.65} & \bf{758.54} & \bf{0.22} & \bf{1.03}\\[1ex]\hline
\end{tabular}
\label{table:nonlin}
\end{table} \clearpage
\begin{table}[ht]
\caption{Resultados de la ejecución de la metaheurística GTS, utilizando instancias de SalhiNagy con la configuración -mni 3000 -lambda1 0.05 -lambda2 0.05 -tabu 37}
\centering
\small
\begin{tabular}{c c c c c c c c}
\hline\hline
Instancia & Costo mínimo & Tiempo(seg.) & Costo promedio & Tiempo promedio(seg.) & CME & \%G & \%GP \\ [0.5ex]
\hline
CMT1X & \bf{470.48} & 1.92 & 
471.27 & 3.78 & 470.48 & 0.00
 & 0.17\\CMT1Y & \bf{470.48} & 2.93 & 
470.53 & 3.10 & 470.48 & 0.00
 & 0.01\\CMT2X & 687.47 & 2.84 & 
691.24 & 3.85 & \bf{682.39} & 
0.74 & 1.30\\CMT2Y & 683.64 & 5.18 & 
692.29 & 4.39 & \bf{682.39} & 
0.18 & 1.45\\CMT3X & 726.56 & 6.52 & 
730.58 & 8.29 & \bf{719.06} & 
1.04 & 1.60\\CMT3Y & \bf{\underline{718.40}} & 9.59 & 
726.25 & 9.36 & 719.06 & 
\bf{-0.09} & 1.00\\CMT4X & 860.34 & 62.94 & 
871.92 & 36.80 & \bf{854.21} & 
0.72 & 2.07\\CMT4Y & 871.94 & 15.54 & 
876.66 & 24.44 & \bf{852.46} & 
2.29 & 2.84\\CMT5X & 1049.05 & 73.98 & 
1063.42 & 61.17 & \bf{1030.56} & 
1.79 & 3.19\\CMT5Y & 1063.25 & 58.15 & 
1075.56 & 49.43 & \bf{1031.69} & 
3.06 & 4.25\\CMT11X & 862.52 & 17.70 & 
908.98 & 35.20 & \bf{831.09} & 
3.78 & 9.37\\CMT11Y & 877.99 & 41.80 & 
922.88 & 25.49 & \bf{829.85} & 
5.80 & 11.21\\CMT12X & 673.32 & 7.85 & 
675.05 & 9.27 & \bf{658.83} & 
2.20 & 2.46\\CMT12Y & 674.00 & 9.58 & 
680.58 & 8.52 & \bf{660.47} & 
2.05 & 3.05\\\bf{PROM.} & 
\bf{763.53} & \bf{22.61} & \bf{775.51} & \bf{20.22} & \bf{749.50} & \bf{1.68} & \bf{3.14}\\[1ex]\hline
\end{tabular}
\label{table:nonlin}
\end{table} \clearpage
\begin{table}[ht]
\caption{Resultados de la ejecución de la metaheurística GTS, utilizando instancias de Dethloff con la configuración -mni 3500 -lambda1 0.05 -lambda2 0.05 -tabu 17}
\centering
\small
\begin{tabular}{c c c c c c c c}
\hline\hline
Instancia & Costo mínimo & Tiempo(seg.) & Costo promedio & Tiempo promedio(seg.) & CME & \%G & \%GP \\ [0.5ex]
\hline
SCA3-0 & 640.55 & 1.36 & 
640.55 & 1.89 & \bf{635.62} & 
0.78 & 0.78\\SCA3-1 & \bf{697.84} & 1.75 & 
697.84 & 2.23 & 697.84 & 0.00
 & 0.00\\
SCA3-2 & \bf{659.34} & 3.49 & 
659.34 & 3.34 & 659.34 & 0.00
 & 0.00\\
SCA3-3 & \bf{680.04} & 3.45 & 
683.03 & 3.01 & 680.04 & 0.00
 & 0.44\\SCA3-4 & \bf{690.50} & 4.71 & 
696.64 & 2.81 & 690.50 & 0.00
 & 0.89\\SCA3-5 & \bf{659.90} & 1.88 & 
666.42 & 2.46 & 659.90 & 0.00
 & 0.99\\SCA3-6 & \bf{651.09} & 1.96 & 
654.35 & 2.35 & 651.09 & 0.00
 & 0.50\\SCA3-7 & 666.15 & 4.48 & 
666.15 & 2.58 & \bf{659.17} & 
1.06 & 1.06\\SCA3-8 & \bf{719.47} & 1.60 & 
719.47 & 2.48 & 719.47 & 0.00
 & 0.00\\
SCA3-9 & \bf{681.00} & 2.86 & 
683.24 & 2.53 & 681.00 & 0.00
 & 0.33\\SCA8-0 & \bf{961.50} & 4.04 & 
975.83 & 3.30 & 961.50 & 0.00
 & 1.49\\SCA8-1 & \bf{1049.65} & 3.45 & 
1064.16 & 2.54 & 1049.65 & 0.00
 & 1.38\\SCA8-2 & \bf{1039.64} & 6.02 & 
1051.80 & 2.65 & 1039.64 & 0.00
 & 1.17\\SCA8-3 & 1007.85 & 3.84 & 
1012.92 & 3.53 & \bf{983.34} & 
2.49 & 3.01\\SCA8-4 & 1065.83 & 3.59 & 
1075.34 & 3.75 & \bf{1065.49} & 
0.03 & 0.92\\SCA8-5 & 1042.30 & 1.62 & 
1054.92 & 2.30 & \bf{1027.08} & 
1.48 & 2.71\\SCA8-6 & 972.48 & 1.72 & 
977.09 & 2.38 & \bf{971.82} & 
0.07 & 0.54\\SCA8-7 & 1060.98 & 1.64 & 
1067.04 & 2.26 & \bf{1051.28} & 
0.92 & 1.50\\SCA8-8 & \bf{1071.18} & 2.28 & 
1079.52 & 1.87 & 1071.18 & 0.00
 & 0.78\\SCA8-9 & \bf{1060.50} & 2.60 & 
1070.36 & 3.12 & 1060.50 & 0.00
 & 0.93\\CON3-0 & \bf{616.52} & 1.83 & 
628.62 & 1.92 & 616.52 & 0.00
 & 1.96\\CON3-1 & \bf{554.47} & 4.68 & 
556.83 & 3.87 & 554.47 & 0.00
 & 0.42\\CON3-2 & \bf{518.00} & 2.32 & 
521.22 & 3.15 & 518.00 & 0.00
 & 0.62\\CON3-3 & \bf{591.19} & 2.08 & 
597.89 & 3.34 & 591.19 & 0.00
 & 1.13\\CON3-4 & \bf{588.79} & 4.68 & 
591.99 & 2.63 & 588.79 & 0.00
 & 0.54\\CON3-5 & \bf{563.70} & 1.23 & 
572.77 & 2.95 & 563.70 & 0.00
 & 1.61\\CON3-6 & \bf{499.05} & 1.94 & 
501.27 & 3.12 & 499.05 & 0.00
 & 0.45\\CON3-7 & \bf{576.48} & 3.72 & 
581.80 & 3.09 & 576.48 & 0.00
 & 0.92\\CON3-8 & \bf{523.05} & 2.60 & 
523.05 & 1.67 & 523.05 & 0.00
 & 0.00\\
CON3-9 & 588.38 & 4.14 & 
589.12 & 2.59 & \bf{578.24} & 
1.75 & 1.88\\CON8-0 & 871.45 & 1.30 & 
876.33 & 1.95 & \bf{857.17} & 
1.67 & 2.23\\CON8-1 & 751.76 & 2.08 & 
758.51 & 2.29 & \bf{740.85} & 
1.47 & 2.38\\CON8-2 & 713.44 & 2.12 & 
735.27 & 2.44 & \bf{712.89} & 
0.08 & 3.14\\CON8-3 & \bf{811.07} & 2.10 & 
830.67 & 2.27 & 811.07 & 0.00
 & 2.42\\CON8-4 & \bf{772.25} & 3.16 & 
776.01 & 2.52 & 772.25 & 0.00
 & 0.49\\CON8-5 & 756.91 & 2.85 & 
758.45 & 2.25 & \bf{754.88} & 
0.27 & 0.47\\CON8-6 & 688.47 & 3.21 & 
693.54 & 3.63 & \bf{678.92} & 
1.41 & 2.15\\CON8-7 & 813.95 & 3.30 & 
838.98 & 3.86 & \bf{811.96} & 
0.25 & 3.33\\CON8-8 & \bf{767.53} & 3.96 & 
770.53 & 2.60 & 767.53 & 0.00
 & 0.39\\CON8-9 & \bf{809.00} & 2.78 & 
818.77 & 2.58 & 809.00 & 0.00
 & 1.21\\\bf{PROM.} & 
\bf{761.33} & \bf{2.86} & \bf{767.94} & \bf{2.70} & \bf{758.54} & \bf{0.34} & \bf{1.18}\\[1ex]\hline
\end{tabular}
\label{table:nonlin}
\end{table} \clearpage
\begin{table}[ht]
\caption{Resultados de la ejecución de la metaheurística GTS, utilizando instancias de SalhiNagy con la configuración -mni 3500 -lambda1 0.05 -lambda2 0.05 -tabu 17}
\centering
\small
\begin{tabular}{c c c c c c c c}
\hline\hline
Instancia & Costo mínimo & Tiempo(seg.) & Costo promedio & Tiempo promedio(seg.) & CME & \%G & \%GP \\ [0.5ex]
\hline
CMT1X & 472.37 & 1.38 & 
473.72 & 2.22 & \bf{470.48} & 
0.40 & 0.69\\CMT1Y & \bf{470.48} & 5.23 & 
471.34 & 4.35 & 470.48 & 0.00
 & 0.18\\CMT2X & 684.09 & 6.16 & 
688.00 & 5.02 & \bf{682.39} & 
0.25 & 0.82\\CMT2Y & 686.11 & 4.85 & 
688.45 & 4.70 & \bf{682.39} & 
0.55 & 0.89\\CMT3X & 723.52 & 13.59 & 
732.66 & 9.61 & \bf{719.06} & 
0.62 & 1.89\\CMT3Y & 724.57 & 17.35 & 
730.86 & 11.28 & \bf{719.06} & 
0.77 & 1.64\\CMT4X & 860.26 & 13.63 & 
874.27 & 26.77 & \bf{854.21} & 
0.71 & 2.35\\CMT4Y & 855.22 & 72.51 & 
873.87 & 38.93 & \bf{852.46} & 
0.32 & 2.51\\CMT5X & 1058.23 & 74.69 & 
1072.64 & 70.35 & \bf{1030.56} & 
2.68 & 4.08\\CMT5Y & 1073.57 & 113.47 & 
1082.05 & 91.97 & \bf{1031.69} & 
4.06 & 4.88\\CMT11X & 878.44 & 13.01 & 
902.79 & 14.71 & \bf{831.09} & 
5.70 & 8.63\\CMT11Y & 874.53 & 18.46 & 
916.19 & 20.20 & \bf{829.85} & 
5.38 & 10.40\\CMT12X & 671.31 & 10.62 & 
673.41 & 9.27 & \bf{658.83} & 
1.89 & 2.21\\CMT12Y & 674.11 & 11.10 & 
674.40 & 11.53 & \bf{660.47} & 
2.07 & 2.11\\\bf{PROM.} & 
\bf{764.77} & \bf{26.86} & \bf{775.33} & \bf{22.92} & \bf{749.50} & \bf{1.81} & \bf{3.09}\\[1ex]\hline
\end{tabular}
\label{table:nonlin}
\end{table} \clearpage
\begin{table}[ht]
\caption{Resultados de la ejecución de la metaheurística GTS, utilizando instancias de Dethloff con la configuración -mni 3500 -lambda1 0.05 -lambda2 0.05 -tabu 21}
\centering
\small
\begin{tabular}{c c c c c c c c}
\hline\hline
Instancia & Costo mínimo & Tiempo(seg.) & Costo promedio & Tiempo promedio(seg.) & CME & \%G & \%GP \\ [0.5ex]
\hline
SCA3-0 & 636.06 & 1.80 & 
639.43 & 2.59 & \bf{635.62} & 
0.07 & 0.60\\SCA3-1 & \bf{697.84} & 5.02 & 
697.84 & 3.23 & 697.84 & 0.00
 & 0.00\\
SCA3-2 & \bf{659.34} & 2.49 & 
659.34 & 2.57 & 659.34 & 0.00
 & 0.00\\
SCA3-3 & \bf{680.04} & 1.35 & 
684.47 & 2.31 & 680.04 & 0.00
 & 0.65\\SCA3-4 & \bf{690.50} & 2.74 & 
690.50 & 2.32 & 690.50 & 0.00
 & 0.00\\
SCA3-5 & \bf{659.90} & 4.24 & 
669.68 & 3.10 & 659.90 & 0.00
 & 1.48\\SCA3-6 & \bf{651.09} & 2.11 & 
653.70 & 2.03 & 651.09 & 0.00
 & 0.40\\SCA3-7 & 666.15 & 1.59 & 
667.09 & 2.46 & \bf{659.17} & 
1.06 & 1.20\\SCA3-8 & \bf{719.47} & 1.14 & 
719.47 & 3.15 & 719.47 & 0.00
 & 0.00\\
SCA3-9 & \bf{681.00} & 1.43 & 
681.00 & 2.23 & 681.00 & 0.00
 & 0.00\\
SCA8-0 & \bf{961.50} & 3.79 & 
975.96 & 3.61 & 961.50 & 0.00
 & 1.50\\SCA8-1 & 1050.20 & 2.64 & 
1067.34 & 3.19 & \bf{1049.65} & 
0.05 & 1.69\\SCA8-2 & 1040.23 & 2.89 & 
1048.22 & 2.48 & \bf{1039.64} & 
0.06 & 0.83\\SCA8-3 & \bf{983.34} & 5.16 & 
1007.38 & 4.98 & 983.34 & 0.00
 & 2.44\\SCA8-4 & 1067.28 & 3.72 & 
1067.84 & 3.56 & \bf{1065.49} & 
0.17 & 0.22\\SCA8-5 & 1042.43 & 3.06 & 
1054.65 & 2.88 & \bf{1027.08} & 
1.49 & 2.68\\SCA8-6 & 972.48 & 4.84 & 
985.70 & 2.51 & \bf{971.82} & 
0.07 & 1.43\\SCA8-7 & 1052.17 & 4.45 & 
1067.61 & 2.90 & \bf{1051.28} & 
0.08 & 1.55\\SCA8-8 & \bf{1071.18} & 2.48 & 
1082.27 & 1.55 & 1071.18 & 0.00
 & 1.04\\SCA8-9 & \bf{1060.50} & 2.06 & 
1066.95 & 2.48 & 1060.50 & 0.00
 & 0.61\\CON3-0 & \bf{616.52} & 8.65 & 
625.15 & 3.59 & 616.52 & 0.00
 & 1.40\\CON3-1 & 556.04 & 1.67 & 
558.69 & 2.34 & \bf{554.47} & 
0.28 & 0.76\\CON3-2 & 522.86 & 3.75 & 
526.07 & 2.61 & \bf{518.00} & 
0.94 & 1.56\\CON3-3 & \bf{591.19} & 1.76 & 
597.97 & 2.29 & 591.19 & 0.00
 & 1.15\\CON3-4 & \bf{588.79} & 4.58 & 
591.65 & 2.92 & 588.79 & 0.00
 & 0.49\\CON3-5 & \bf{563.70} & 2.40 & 
570.46 & 2.34 & 563.70 & 0.00
 & 1.20\\CON3-6 & \bf{499.05} & 2.72 & 
504.08 & 2.17 & 499.05 & 0.00
 & 1.01\\CON3-7 & 576.84 & 3.26 & 
586.26 & 2.33 & \bf{576.48} & 
0.06 & 1.70\\CON3-8 & \bf{523.05} & 3.49 & 
523.21 & 2.53 & 523.05 & 0.00
 & 0.03\\CON3-9 & 578.25 & 4.91 & 
585.63 & 2.60 & \bf{578.24} & 
0.00 & 1.28\\CON8-0 & 857.40 & 3.49 & 
872.37 & 3.23 & \bf{857.17} & 
0.03 & 1.77\\CON8-1 & \bf{740.85} & 6.32 & 
766.00 & 3.36 & 740.85 & 0.00
 & 3.39\\CON8-2 & 718.70 & 2.46 & 
729.54 & 1.70 & \bf{712.89} & 
0.81 & 2.34\\CON8-3 & 811.23 & 2.41 & 
817.82 & 3.10 & \bf{811.07} & 
0.02 & 0.83\\CON8-4 & \bf{772.25} & 5.73 & 
793.09 & 3.33 & 772.25 & 0.00
 & 2.70\\CON8-5 & \bf{754.88} & 3.40 & 
758.86 & 2.90 & 754.88 & 0.00
 & 0.53\\CON8-6 & 688.68 & 2.58 & 
690.41 & 2.98 & \bf{678.92} & 
1.44 & 1.69\\CON8-7 & 814.50 & 4.92 & 
831.45 & 2.70 & \bf{811.96} & 
0.31 & 2.40\\CON8-8 & \bf{767.53} & 3.11 & 
777.23 & 2.01 & 767.53 & 0.00
 & 1.26\\CON8-9 & 812.53 & 2.97 & 
828.37 & 2.71 & \bf{809.00} & 
0.44 & 2.39\\\bf{PROM.} & 
\bf{759.94} & \bf{3.34} & \bf{768.02} & \bf{2.75} & \bf{758.54} & \bf{0.18} & \bf{1.20}\\[1ex]\hline
\end{tabular}
\label{table:nonlin}
\end{table} \clearpage
\begin{table}[ht]
\caption{Resultados de la ejecución de la metaheurística GTS, utilizando instancias de SalhiNagy con la configuración -mni 3500 -lambda1 0.05 -lambda2 0.05 -tabu 21}
\centering
\small
\begin{tabular}{c c c c c c c c}
\hline\hline
Instancia & Costo mínimo & Tiempo(seg.) & Costo promedio & Tiempo promedio(seg.) & CME & \%G & \%GP \\ [0.5ex]
\hline
CMT1X & \bf{470.48} & 5.88 & 
471.08 & 3.68 & 470.48 & 0.00
 & 0.13\\CMT1Y & \bf{470.48} & 4.18 & 
471.55 & 3.56 & 470.48 & 0.00
 & 0.23\\CMT2X & 685.88 & 3.68 & 
691.41 & 4.25 & \bf{682.39} & 
0.51 & 1.32\\CMT2Y & \bf{682.39} & 4.61 & 
686.84 & 4.97 & 682.39 & 0.00
 & 0.65\\CMT3X & 725.72 & 7.03 & 
732.49 & 6.22 & \bf{719.06} & 
0.93 & 1.87\\CMT3Y & 725.72 & 9.10 & 
732.26 & 9.94 & \bf{719.06} & 
0.93 & 1.84\\CMT4X & 880.81 & 50.17 & 
885.98 & 33.36 & \bf{854.21} & 
3.11 & 3.72\\CMT4Y & 855.92 & 32.82 & 
867.44 & 35.16 & \bf{852.46} & 
0.41 & 1.76\\CMT5X & 1054.14 & 72.27 & 
1078.44 & 75.12 & \bf{1030.56} & 
2.29 & 4.65\\CMT5Y & 1043.57 & 79.96 & 
1074.87 & 65.30 & \bf{1031.69} & 
1.15 & 4.19\\CMT11X & 887.66 & 42.95 & 
900.38 & 33.99 & \bf{831.09} & 
6.81 & 8.34\\CMT11Y & 878.65 & 20.81 & 
908.29 & 28.72 & \bf{829.85} & 
5.88 & 9.45\\CMT12X & \bf{658.83} & 34.11 & 
672.94 & 14.70 & 658.83 & 0.00
 & 2.14\\CMT12Y & 675.66 & 10.94 & 
678.88 & 9.01 & \bf{660.47} & 
2.30 & 2.79\\\bf{PROM.} & 
\bf{763.99} & \bf{27.04} & \bf{775.20} & \bf{23.43} & \bf{749.50} & \bf{1.74} & \bf{3.08}\\[1ex]\hline
\end{tabular}
\label{table:nonlin}
\end{table} \clearpage
\begin{table}[ht]
\caption{Resultados de la ejecución de la metaheurística GTS, utilizando instancias de Dethloff con la configuración -mni 3500 -lambda1 0.05 -lambda2 0.05 -tabu 25}
\centering
\small
\begin{tabular}{c c c c c c c c}
\hline\hline
Instancia & Costo mínimo & Tiempo(seg.) & Costo promedio & Tiempo promedio(seg.) & CME & \%G & \%GP \\ [0.5ex]
\hline
SCA3-0 & 639.34 & 2.64 & 
640.25 & 2.76 & \bf{635.62} & 
0.59 & 0.73\\SCA3-1 & \bf{697.84} & 1.44 & 
698.50 & 2.65 & 697.84 & 0.00
 & 0.10\\SCA3-2 & \bf{659.34} & 2.51 & 
659.34 & 3.37 & 659.34 & 0.00
 & 0.00\\
SCA3-3 & 680.60 & 1.72 & 
688.05 & 2.42 & \bf{680.04} & 
0.08 & 1.18\\SCA3-4 & \bf{690.50} & 3.65 & 
690.50 & 4.11 & 690.50 & 0.00
 & 0.00\\
SCA3-5 & \bf{659.90} & 1.77 & 
659.90 & 2.32 & 659.90 & 0.00
 & 0.00\\
SCA3-6 & \bf{651.09} & 2.58 & 
654.16 & 2.47 & 651.09 & 0.00
 & 0.47\\SCA3-7 & 666.15 & 1.88 & 
666.15 & 2.31 & \bf{659.17} & 
1.06 & 1.06\\SCA3-8 & \bf{719.47} & 3.34 & 
719.47 & 3.82 & 719.47 & 0.00
 & 0.00\\
SCA3-9 & \bf{681.00} & 4.17 & 
681.00 & 3.14 & 681.00 & 0.00
 & 0.00\\
SCA8-0 & \bf{961.50} & 2.40 & 
978.90 & 2.33 & 961.50 & 0.00
 & 1.81\\SCA8-1 & 1067.45 & 2.41 & 
1069.53 & 2.55 & \bf{1049.65} & 
1.70 & 1.89\\SCA8-2 & \bf{1039.64} & 4.15 & 
1053.14 & 3.55 & 1039.64 & 0.00
 & 1.30\\SCA8-3 & 1002.20 & 2.44 & 
1010.01 & 2.88 & \bf{983.34} & 
1.92 & 2.71\\SCA8-4 & \bf{1065.49} & 4.02 & 
1069.67 & 3.23 & 1065.49 & 0.00
 & 0.39\\SCA8-5 & \bf{1027.08} & 3.02 & 
1048.39 & 2.44 & 1027.08 & 0.00
 & 2.07\\SCA8-6 & 972.48 & 3.03 & 
978.54 & 2.39 & \bf{971.82} & 
0.07 & 0.69\\SCA8-7 & 1052.04 & 1.79 & 
1067.04 & 3.14 & \bf{1051.28} & 
0.07 & 1.50\\SCA8-8 & \bf{1071.18} & 1.67 & 
1081.84 & 3.32 & 1071.18 & 0.00
 & 1.00\\SCA8-9 & \bf{1060.50} & 4.09 & 
1064.18 & 2.35 & 1060.50 & 0.00
 & 0.35\\CON3-0 & \bf{616.52} & 6.59 & 
619.51 & 4.00 & 616.52 & 0.00
 & 0.48\\CON3-1 & \bf{554.47} & 5.59 & 
557.35 & 4.35 & 554.47 & 0.00
 & 0.52\\CON3-2 & 522.86 & 1.30 & 
523.20 & 2.58 & \bf{518.00} & 
0.94 & 1.00\\CON3-3 & \bf{591.19} & 5.06 & 
591.19 & 4.88 & 591.19 & 0.00
 & 0.00\\
CON3-4 & \bf{588.79} & 1.48 & 
595.62 & 2.77 & 588.79 & 0.00
 & 1.16\\CON3-5 & \bf{563.70} & 1.65 & 
563.70 & 3.05 & 563.70 & 0.00
 & 0.00\\
CON3-6 & \bf{499.05} & 4.26 & 
501.86 & 3.86 & 499.05 & 0.00
 & 0.56\\CON3-7 & \bf{576.48} & 4.52 & 
586.09 & 3.00 & 576.48 & 0.00
 & 1.67\\CON3-8 & \bf{523.05} & 2.05 & 
523.05 & 3.27 & 523.05 & 0.00
 & 0.00\\
CON3-9 & 578.25 & 4.90 & 
584.17 & 3.18 & \bf{578.24} & 
0.00 & 1.03\\CON8-0 & 858.03 & 5.78 & 
871.17 & 4.55 & \bf{857.17} & 
0.10 & 1.63\\CON8-1 & \bf{740.85} & 2.28 & 
753.27 & 2.07 & 740.85 & 0.00
 & 1.68\\CON8-2 & 725.31 & 5.28 & 
730.85 & 4.20 & \bf{712.89} & 
1.74 & 2.52\\CON8-3 & 821.26 & 2.12 & 
827.38 & 4.18 & \bf{811.07} & 
1.26 & 2.01\\CON8-4 & \bf{772.25} & 2.12 & 
781.37 & 2.56 & 772.25 & 0.00
 & 1.18\\CON8-5 & 754.95 & 6.45 & 
756.16 & 4.26 & \bf{754.88} & 
0.01 & 0.17\\CON8-6 & 688.47 & 4.18 & 
706.01 & 2.68 & \bf{678.92} & 
1.41 & 3.99\\CON8-7 & 812.89 & 3.63 & 
814.14 & 3.06 & \bf{811.96} & 
0.11 & 0.27\\CON8-8 & \bf{767.53} & 2.57 & 
778.63 & 3.00 & 767.53 & 0.00
 & 1.45\\CON8-9 & 811.43 & 2.48 & 
814.89 & 2.86 & \bf{809.00} & 
0.30 & 0.73\\\bf{PROM.} & 
\bf{760.80} & \bf{3.23} & \bf{766.45} & \bf{3.15} & \bf{758.54} & \bf{0.28} & \bf{0.98}\\[1ex]\hline
\end{tabular}
\label{table:nonlin}
\end{table} \clearpage
\begin{table}[ht]
\caption{Resultados de la ejecución de la metaheurística GTS, utilizando instancias de SalhiNagy con la configuración -mni 3500 -lambda1 0.05 -lambda2 0.05 -tabu 25}
\centering
\small
\begin{tabular}{c c c c c c c c}
\hline\hline
Instancia & Costo mínimo & Tiempo(seg.) & Costo promedio & Tiempo promedio(seg.) & CME & \%G & \%GP \\ [0.5ex]
\hline
CMT1X & \bf{470.48} & 3.08 & 
471.47 & 2.93 & 470.48 & 0.00
 & 0.21\\CMT1Y & \bf{470.48} & 6.11 & 
471.55 & 3.74 & 470.48 & 0.00
 & 0.23\\CMT2X & 684.14 & 7.34 & 
692.23 & 4.71 & \bf{682.39} & 
0.26 & 1.44\\CMT2Y & 686.13 & 7.28 & 
688.38 & 5.17 & \bf{682.39} & 
0.55 & 0.88\\CMT3X & 724.38 & 15.01 & 
729.52 & 9.68 & \bf{719.06} & 
0.74 & 1.46\\CMT3Y & 723.88 & 11.28 & 
728.51 & 11.26 & \bf{719.06} & 
0.67 & 1.31\\CMT4X & 863.35 & 27.68 & 
866.76 & 29.93 & \bf{854.21} & 
1.07 & 1.47\\CMT4Y & 856.54 & 32.89 & 
887.47 & 20.63 & \bf{852.46} & 
0.48 & 4.11\\CMT5X & 1045.34 & 48.01 & 
1068.01 & 68.34 & \bf{1030.56} & 
1.43 & 3.63\\CMT5Y & 1055.14 & 64.32 & 
1073.39 & 52.47 & \bf{1031.69} & 
2.27 & 4.04\\CMT11X & 881.17 & 49.97 & 
905.00 & 31.72 & \bf{831.09} & 
6.03 & 8.89\\CMT11Y & 867.11 & 14.17 & 
880.74 & 24.31 & \bf{829.85} & 
4.49 & 6.13\\CMT12X & 662.07 & 17.66 & 
688.50 & 12.79 & \bf{658.83} & 
0.49 & 4.50\\CMT12Y & 673.49 & 7.26 & 
681.01 & 8.00 & \bf{660.47} & 
1.97 & 3.11\\\bf{PROM.} & 
\bf{761.69} & \bf{22.29} & \bf{773.75} & \bf{20.40} & \bf{749.50} & \bf{1.46} & \bf{2.96}\\[1ex]\hline
\end{tabular}
\label{table:nonlin}
\end{table} \clearpage
\begin{table}[ht]
\caption{Resultados de la ejecución de la metaheurística GTS, utilizando instancias de Dethloff con la configuración -mni 3500 -lambda1 0.05 -lambda2 0.05 -tabu 29}
\centering
\small
\begin{tabular}{c c c c c c c c}
\hline\hline
Instancia & Costo mínimo & Tiempo(seg.) & Costo promedio & Tiempo promedio(seg.) & CME & \%G & \%GP \\ [0.5ex]
\hline
SCA3-0 & 640.55 & 1.37 & 
640.55 & 2.92 & \bf{635.62} & 
0.78 & 0.78\\SCA3-1 & \bf{697.84} & 5.03 & 
702.06 & 3.14 & 697.84 & 0.00
 & 0.60\\SCA3-2 & \bf{659.34} & 2.12 & 
659.34 & 2.69 & 659.34 & 0.00
 & 0.00\\
SCA3-3 & \bf{680.04} & 2.68 & 
680.32 & 2.46 & 680.04 & 0.00
 & 0.04\\SCA3-4 & \bf{690.50} & 3.55 & 
705.52 & 2.89 & 690.50 & 0.00
 & 2.18\\SCA3-5 & \bf{659.90} & 4.99 & 
663.16 & 3.43 & 659.90 & 0.00
 & 0.49\\SCA3-6 & \bf{651.09} & 1.98 & 
651.74 & 3.23 & 651.09 & 0.00
 & 0.10\\SCA3-7 & 666.15 & 1.93 & 
667.09 & 3.21 & \bf{659.17} & 
1.06 & 1.20\\SCA3-8 & \bf{719.47} & 3.18 & 
719.47 & 2.61 & 719.47 & 0.00
 & 0.00\\
SCA3-9 & \bf{681.00} & 3.66 & 
681.00 & 3.61 & 681.00 & 0.00
 & 0.00\\
SCA8-0 & 970.64 & 2.43 & 
974.73 & 3.80 & \bf{961.50} & 
0.95 & 1.38\\SCA8-1 & 1068.31 & 3.11 & 
1070.48 & 2.08 & \bf{1049.65} & 
1.78 & 1.98\\SCA8-2 & 1050.37 & 2.21 & 
1065.29 & 2.25 & \bf{1039.64} & 
1.03 & 2.47\\SCA8-3 & \bf{983.34} & 5.54 & 
1003.83 & 3.47 & 983.34 & 0.00
 & 2.08\\SCA8-4 & 1067.28 & 3.44 & 
1072.25 & 2.04 & \bf{1065.49} & 
0.17 & 0.63\\SCA8-5 & 1042.30 & 1.80 & 
1060.99 & 1.99 & \bf{1027.08} & 
1.48 & 3.30\\SCA8-6 & 972.48 & 1.54 & 
972.48 & 2.35 & \bf{971.82} & 
0.07 & 0.07\\SCA8-7 & 1054.73 & 2.12 & 
1075.72 & 3.33 & \bf{1051.28} & 
0.33 & 2.32\\SCA8-8 & 1082.12 & 2.23 & 
1083.26 & 2.35 & \bf{1071.18} & 
1.02 & 1.13\\SCA8-9 & \bf{1060.50} & 4.98 & 
1073.87 & 2.83 & 1060.50 & 0.00
 & 1.26\\CON3-0 & 628.47 & 3.10 & 
630.12 & 2.38 & \bf{616.52} & 
1.94 & 2.21\\CON3-1 & \bf{554.47} & 4.96 & 
562.03 & 3.04 & 554.47 & 0.00
 & 1.36\\CON3-2 & 519.61 & 4.92 & 
522.28 & 3.83 & \bf{518.00} & 
0.31 & 0.83\\CON3-3 & \bf{591.19} & 2.34 & 
598.89 & 3.54 & 591.19 & 0.00
 & 1.30\\CON3-4 & \bf{588.79} & 3.33 & 
591.33 & 3.66 & 588.79 & 0.00
 & 0.43\\CON3-5 & \bf{563.70} & 2.42 & 
569.43 & 2.52 & 563.70 & 0.00
 & 1.02\\CON3-6 & \bf{499.05} & 2.04 & 
500.94 & 2.44 & 499.05 & 0.00
 & 0.38\\CON3-7 & \bf{576.48} & 3.28 & 
581.66 & 5.10 & 576.48 & 0.00
 & 0.90\\CON3-8 & \bf{523.05} & 1.35 & 
523.05 & 1.90 & 523.05 & 0.00
 & 0.00\\
CON3-9 & 582.79 & 3.65 & 
587.05 & 3.47 & \bf{578.24} & 
0.79 & 1.52\\CON8-0 & 857.38 & 3.55 & 
875.04 & 4.05 & \bf{857.17} & 
0.02 & 2.08\\CON8-1 & \bf{740.85} & 5.42 & 
753.32 & 4.79 & 740.85 & 0.00
 & 1.68\\CON8-2 & 713.44 & 5.74 & 
724.05 & 3.61 & \bf{712.89} & 
0.08 & 1.57\\CON8-3 & 813.51 & 2.57 & 
838.77 & 2.67 & \bf{811.07} & 
0.30 & 3.42\\CON8-4 & \bf{772.25} & 1.41 & 
784.65 & 2.33 & 772.25 & 0.00
 & 1.61\\CON8-5 & 756.91 & 2.54 & 
759.14 & 2.30 & \bf{754.88} & 
0.27 & 0.56\\CON8-6 & 685.45 & 2.39 & 
716.68 & 2.09 & \bf{678.92} & 
0.96 & 5.56\\CON8-7 & 814.79 & 3.11 & 
814.94 & 3.22 & \bf{811.96} & 
0.35 & 0.37\\CON8-8 & 776.55 & 2.60 & 
778.38 & 2.96 & \bf{767.53} & 
1.18 & 1.41\\CON8-9 & 809.15 & 2.28 & 
812.70 & 2.66 & \bf{809.00} & 
0.02 & 0.46\\\bf{PROM.} & 
\bf{761.65} & \bf{3.07} & \bf{768.69} & \bf{2.98} & \bf{758.54} & \bf{0.37} & \bf{1.27}\\[1ex]\hline
\end{tabular}
\label{table:nonlin}
\end{table} \clearpage
\begin{table}[ht]
\caption{Resultados de la ejecución de la metaheurística GTS, utilizando instancias de SalhiNagy con la configuración -mni 3500 -lambda1 0.05 -lambda2 0.05 -tabu 29}
\centering
\small
\begin{tabular}{c c c c c c c c}
\hline\hline
Instancia & Costo mínimo & Tiempo(seg.) & Costo promedio & Tiempo promedio(seg.) & CME & \%G & \%GP \\ [0.5ex]
\hline
CMT1X & \bf{470.48} & 3.28 & 
471.43 & 4.64 & 470.48 & 0.00
 & 0.20\\CMT1Y & \bf{470.48} & 4.29 & 
474.13 & 3.71 & 470.48 & 0.00
 & 0.78\\CMT2X & 683.60 & 4.46 & 
687.05 & 7.36 & \bf{682.39} & 
0.18 & 0.68\\CMT2Y & 687.04 & 7.00 & 
689.52 & 7.09 & \bf{682.39} & 
0.68 & 1.05\\CMT3X & 724.57 & 14.96 & 
726.35 & 12.87 & \bf{719.06} & 
0.77 & 1.01\\CMT3Y & 724.86 & 9.55 & 
728.50 & 8.48 & \bf{719.06} & 
0.81 & 1.31\\CMT4X & 855.93 & 40.73 & 
872.01 & 29.40 & \bf{854.21} & 
0.20 & 2.08\\CMT4Y & 867.06 & 36.95 & 
879.36 & 24.23 & \bf{852.46} & 
1.71 & 3.16\\CMT5X & 1053.02 & 127.95 & 
1080.65 & 72.75 & \bf{1030.56} & 
2.18 & 4.86\\CMT5Y & 1043.37 & 60.98 & 
1069.40 & 53.38 & \bf{1031.69} & 
1.13 & 3.66\\CMT11X & 918.80 & 73.59 & 
935.36 & 33.77 & \bf{831.09} & 
10.55 & 12.55\\CMT11Y & 884.30 & 15.72 & 
917.65 & 30.39 & \bf{829.85} & 
6.56 & 10.58\\CMT12X & 670.16 & 20.45 & 
678.50 & 14.13 & \bf{658.83} & 
1.72 & 2.99\\CMT12Y & 674.07 & 4.38 & 
674.71 & 6.27 & \bf{660.47} & 
2.06 & 2.16\\\bf{PROM.} & 
\bf{766.27} & \bf{30.31} & \bf{777.47} & \bf{22.03} & \bf{749.50} & \bf{2.04} & \bf{3.36}\\[1ex]\hline
\end{tabular}
\label{table:nonlin}
\end{table} \clearpage
\begin{table}[ht]
\caption{Resultados de la ejecución de la metaheurística GTS, utilizando instancias de Dethloff con la configuración -mni 3500 -lambda1 0.05 -lambda2 0.05 -tabu 33}
\centering
\small
\begin{tabular}{c c c c c c c c}
\hline\hline
Instancia & Costo mínimo & Tiempo(seg.) & Costo promedio & Tiempo promedio(seg.) & CME & \%G & \%GP \\ [0.5ex]
\hline
SCA3-0 & 640.55 & 1.86 & 
640.55 & 3.99 & \bf{635.62} & 
0.78 & 0.78\\SCA3-1 & \bf{697.84} & 2.12 & 
698.50 & 3.14 & 697.84 & 0.00
 & 0.10\\SCA3-2 & \bf{659.34} & 2.79 & 
659.34 & 3.10 & 659.34 & 0.00
 & 0.00\\
SCA3-3 & \bf{680.04} & 2.31 & 
682.62 & 2.27 & 680.04 & 0.00
 & 0.38\\SCA3-4 & \bf{690.50} & 3.86 & 
702.89 & 4.33 & 690.50 & 0.00
 & 1.79\\SCA3-5 & \bf{659.90} & 2.52 & 
669.68 & 2.53 & 659.90 & 0.00
 & 1.48\\SCA3-6 & \bf{651.09} & 1.45 & 
651.09 & 3.82 & 651.09 & 0.00
 & 0.00\\
SCA3-7 & 666.15 & 3.32 & 
666.15 & 4.16 & \bf{659.17} & 
1.06 & 1.06\\SCA3-8 & \bf{719.47} & 3.56 & 
719.47 & 3.37 & 719.47 & 0.00
 & 0.00\\
SCA3-9 & \bf{681.00} & 2.60 & 
681.00 & 3.25 & 681.00 & 0.00
 & 0.00\\
SCA8-0 & \bf{961.50} & 2.01 & 
971.00 & 3.08 & 961.50 & 0.00
 & 0.99\\SCA8-1 & \bf{1049.65} & 5.54 & 
1059.08 & 3.43 & 1049.65 & 0.00
 & 0.90\\SCA8-2 & 1050.17 & 2.64 & 
1057.84 & 2.08 & \bf{1039.64} & 
1.01 & 1.75\\SCA8-3 & 1008.92 & 2.18 & 
1012.88 & 2.41 & \bf{983.34} & 
2.60 & 3.00\\SCA8-4 & 1074.33 & 1.70 & 
1081.95 & 2.86 & \bf{1065.49} & 
0.83 & 1.54\\SCA8-5 & \bf{1027.08} & 2.33 & 
1044.99 & 2.58 & 1027.08 & 0.00
 & 1.74\\SCA8-6 & 972.48 & 3.13 & 
976.62 & 2.96 & \bf{971.82} & 
0.07 & 0.49\\SCA8-7 & 1063.22 & 4.20 & 
1078.78 & 2.84 & \bf{1051.28} & 
1.14 & 2.62\\SCA8-8 & \bf{1071.18} & 2.32 & 
1078.39 & 2.35 & 1071.18 & 0.00
 & 0.67\\SCA8-9 & \bf{1060.50} & 4.26 & 
1063.51 & 2.91 & 1060.50 & 0.00
 & 0.28\\CON3-0 & \bf{616.52} & 1.96 & 
626.71 & 2.69 & 616.52 & 0.00
 & 1.65\\CON3-1 & \bf{554.47} & 4.37 & 
555.94 & 3.96 & 554.47 & 0.00
 & 0.27\\CON3-2 & 521.38 & 1.87 & 
522.92 & 2.58 & \bf{518.00} & 
0.65 & 0.95\\CON3-3 & \bf{591.19} & 4.48 & 
600.65 & 3.71 & 591.19 & 0.00
 & 1.60\\CON3-4 & \bf{588.79} & 3.98 & 
595.62 & 3.05 & 588.79 & 0.00
 & 1.16\\CON3-5 & \bf{563.70} & 5.25 & 
563.70 & 3.39 & 563.70 & 0.00
 & 0.00\\
CON3-6 & \bf{499.05} & 4.05 & 
501.88 & 3.65 & 499.05 & 0.00
 & 0.57\\CON3-7 & \bf{576.48} & 3.27 & 
580.93 & 3.00 & 576.48 & 0.00
 & 0.77\\CON3-8 & \bf{523.05} & 2.43 & 
523.05 & 2.38 & 523.05 & 0.00
 & 0.00\\
CON3-9 & 588.40 & 2.38 & 
588.81 & 2.80 & \bf{578.24} & 
1.76 & 1.83\\CON8-0 & 857.40 & 3.59 & 
885.33 & 3.30 & \bf{857.17} & 
0.03 & 3.28\\CON8-1 & \bf{740.85} & 2.21 & 
746.30 & 3.30 & 740.85 & 0.00
 & 0.74\\CON8-2 & 716.03 & 5.28 & 
735.71 & 3.27 & \bf{712.89} & 
0.44 & 3.20\\CON8-3 & \bf{811.07} & 2.31 & 
827.80 & 2.13 & 811.07 & 0.00
 & 2.06\\CON8-4 & \bf{772.25} & 5.01 & 
781.50 & 2.96 & 772.25 & 0.00
 & 1.20\\CON8-5 & 756.91 & 2.64 & 
759.62 & 2.70 & \bf{754.88} & 
0.27 & 0.63\\CON8-6 & 684.69 & 3.65 & 
695.51 & 3.39 & \bf{678.92} & 
0.85 & 2.44\\CON8-7 & 812.89 & 2.73 & 
822.22 & 3.16 & \bf{811.96} & 
0.11 & 1.26\\CON8-8 & \bf{767.53} & 8.50 & 
769.78 & 5.06 & 767.53 & 0.00
 & 0.29\\CON8-9 & 811.16 & 3.26 & 
814.39 & 2.41 & \bf{809.00} & 
0.27 & 0.67\\\bf{PROM.} & 
\bf{760.97} & \bf{3.25} & \bf{767.37} & \bf{3.11} & \bf{758.54} & \bf{0.30} & \bf{1.10}\\[1ex]\hline
\end{tabular}
\label{table:nonlin}
\end{table} \clearpage
\begin{table}[ht]
\caption{Resultados de la ejecución de la metaheurística GTS, utilizando instancias de SalhiNagy con la configuración -mni 3500 -lambda1 0.05 -lambda2 0.05 -tabu 33}
\centering
\small
\begin{tabular}{c c c c c c c c}
\hline\hline
Instancia & Costo mínimo & Tiempo(seg.) & Costo promedio & Tiempo promedio(seg.) & CME & \%G & \%GP \\ [0.5ex]
\hline
CMT1X & \bf{470.48} & 3.56 & 
471.62 & 2.97 & 470.48 & 0.00
 & 0.24\\CMT1Y & 472.37 & 2.22 & 
473.90 & 3.06 & \bf{470.48} & 
0.40 & 0.73\\CMT2X & 686.94 & 3.86 & 
691.20 & 4.68 & \bf{682.39} & 
0.67 & 1.29\\CMT2Y & 684.24 & 12.20 & 
690.54 & 6.02 & \bf{682.39} & 
0.27 & 1.19\\CMT3X & 726.69 & 8.22 & 
728.56 & 12.07 & \bf{719.06} & 
1.06 & 1.32\\CMT3Y & 724.07 & 10.55 & 
729.91 & 9.74 & \bf{719.06} & 
0.70 & 1.51\\CMT4X & 873.99 & 32.97 & 
877.94 & 31.23 & \bf{854.21} & 
2.32 & 2.78\\CMT4Y & 865.88 & 19.18 & 
872.34 & 21.90 & \bf{852.46} & 
1.57 & 2.33\\CMT5X & 1048.12 & 87.29 & 
1068.53 & 113.45 & \bf{1030.56} & 
1.70 & 3.68\\CMT5Y & 1063.27 & 125.10 & 
1073.92 & 70.94 & \bf{1031.69} & 
3.06 & 4.09\\CMT11X & 870.00 & 22.98 & 
891.32 & 28.80 & \bf{831.09} & 
4.68 & 7.25\\CMT11Y & 878.33 & 52.03 & 
926.56 & 33.32 & \bf{829.85} & 
5.84 & 11.65\\CMT12X & 664.23 & 12.94 & 
676.50 & 11.13 & \bf{658.83} & 
0.82 & 2.68\\CMT12Y & 679.75 & 10.20 & 
683.79 & 23.37 & \bf{660.47} & 
2.92 & 3.53\\\bf{PROM.} & 
\bf{764.88} & \bf{28.81} & \bf{775.47} & \bf{26.62} & \bf{749.50} & \bf{1.86} & \bf{3.16}\\[1ex]\hline
\end{tabular}
\label{table:nonlin}
\end{table} \clearpage
\begin{table}[ht]
\caption{Resultados de la ejecución de la metaheurística GTS, utilizando instancias de Dethloff con la configuración -mni 3500 -lambda1 0.05 -lambda2 0.05 -tabu 37}
\centering
\small
\begin{tabular}{c c c c c c c c}
\hline\hline
Instancia & Costo mínimo & Tiempo(seg.) & Costo promedio & Tiempo promedio(seg.) & CME & \%G & \%GP \\ [0.5ex]
\hline
SCA3-0 & 640.55 & 1.51 & 
641.20 & 2.85 & \bf{635.62} & 
0.78 & 0.88\\SCA3-1 & \bf{697.84} & 2.59 & 
698.50 & 2.64 & 697.84 & 0.00
 & 0.10\\SCA3-2 & \bf{659.34} & 3.23 & 
659.34 & 3.94 & 659.34 & 0.00
 & 0.00\\
SCA3-3 & 680.60 & 2.43 & 
682.90 & 2.65 & \bf{680.04} & 
0.08 & 0.42\\SCA3-4 & \bf{690.50} & 3.10 & 
690.50 & 3.30 & 690.50 & 0.00
 & 0.00\\
SCA3-5 & \bf{659.90} & 2.70 & 
659.90 & 3.26 & 659.90 & 0.00
 & 0.00\\
SCA3-6 & \bf{651.09} & 2.22 & 
653.70 & 2.84 & 651.09 & 0.00
 & 0.40\\SCA3-7 & 666.15 & 2.16 & 
668.02 & 2.42 & \bf{659.17} & 
1.06 & 1.34\\SCA3-8 & \bf{719.47} & 4.16 & 
719.47 & 3.70 & 719.47 & 0.00
 & 0.00\\
SCA3-9 & \bf{681.00} & 1.94 & 
681.00 & 3.25 & 681.00 & 0.00
 & 0.00\\
SCA8-0 & 970.64 & 4.22 & 
979.04 & 2.54 & \bf{961.50} & 
0.95 & 1.82\\SCA8-1 & 1053.38 & 1.34 & 
1067.94 & 2.48 & \bf{1049.65} & 
0.36 & 1.74\\SCA8-2 & \bf{1039.64} & 3.80 & 
1054.08 & 2.42 & 1039.64 & 0.00
 & 1.39\\SCA8-3 & \bf{983.34} & 2.70 & 
998.59 & 3.81 & 983.34 & 0.00
 & 1.55\\SCA8-4 & 1067.55 & 2.55 & 
1072.78 & 3.40 & \bf{1065.49} & 
0.19 & 0.68\\SCA8-5 & 1042.30 & 3.50 & 
1056.17 & 2.59 & \bf{1027.08} & 
1.48 & 2.83\\SCA8-6 & 972.48 & 2.98 & 
985.29 & 2.19 & \bf{971.82} & 
0.07 & 1.39\\SCA8-7 & 1052.04 & 3.08 & 
1065.37 & 3.11 & \bf{1051.28} & 
0.07 & 1.34\\SCA8-8 & \bf{1071.18} & 3.92 & 
1078.39 & 2.40 & 1071.18 & 0.00
 & 0.67\\SCA8-9 & \bf{1060.50} & 6.28 & 
1066.31 & 3.98 & 1060.50 & 0.00
 & 0.55\\CON3-0 & \bf{616.52} & 3.61 & 
627.74 & 3.50 & 616.52 & 0.00
 & 1.82\\CON3-1 & \bf{554.47} & 2.88 & 
557.01 & 2.42 & 554.47 & 0.00
 & 0.46\\CON3-2 & 519.61 & 1.71 & 
521.92 & 3.02 & \bf{518.00} & 
0.31 & 0.76\\CON3-3 & \bf{591.19} & 3.96 & 
591.19 & 4.46 & 591.19 & 0.00
 & 0.00\\
CON3-4 & \bf{588.79} & 3.00 & 
589.45 & 3.17 & 588.79 & 0.00
 & 0.11\\CON3-5 & \bf{563.70} & 3.70 & 
565.92 & 3.53 & 563.70 & 0.00
 & 0.39\\CON3-6 & \bf{499.05} & 2.47 & 
501.79 & 4.04 & 499.05 & 0.00
 & 0.55\\CON3-7 & \bf{576.48} & 5.51 & 
580.96 & 2.77 & 576.48 & 0.00
 & 0.78\\CON3-8 & \bf{523.05} & 1.42 & 
523.05 & 2.20 & 523.05 & 0.00
 & 0.00\\
CON3-9 & 578.25 & 7.54 & 
587.17 & 4.19 & \bf{578.24} & 
0.00 & 1.54\\CON8-0 & 857.42 & 3.87 & 
889.53 & 3.30 & \bf{857.17} & 
0.03 & 3.78\\CON8-1 & 752.47 & 3.35 & 
756.63 & 2.97 & \bf{740.85} & 
1.57 & 2.13\\CON8-2 & 717.97 & 5.81 & 
736.81 & 3.13 & \bf{712.89} & 
0.71 & 3.36\\CON8-3 & 821.26 & 2.24 & 
834.10 & 3.49 & \bf{811.07} & 
1.26 & 2.84\\CON8-4 & \bf{772.25} & 2.46 & 
781.42 & 3.02 & 772.25 & 0.00
 & 1.19\\CON8-5 & 758.12 & 2.01 & 
762.22 & 2.80 & \bf{754.88} & 
0.43 & 0.97\\CON8-6 & 688.47 & 2.66 & 
693.87 & 3.27 & \bf{678.92} & 
1.41 & 2.20\\CON8-7 & 813.00 & 6.59 & 
826.64 & 3.62 & \bf{811.96} & 
0.13 & 1.81\\CON8-8 & \bf{767.53} & 2.27 & 
771.30 & 4.68 & 767.53 & 0.00
 & 0.49\\CON8-9 & \bf{809.00} & 3.39 & 
812.58 & 3.28 & 809.00 & 0.00
 & 0.44\\\bf{PROM.} & 
\bf{760.70} & \bf{3.27} & \bf{767.25} & \bf{3.17} & \bf{758.54} & \bf{0.27} & \bf{1.07}\\[1ex]\hline
\end{tabular}
\label{table:nonlin}
\end{table} \clearpage
\begin{table}[ht]
\caption{Resultados de la ejecución de la metaheurística GTS, utilizando instancias de SalhiNagy con la configuración -mni 3500 -lambda1 0.05 -lambda2 0.05 -tabu 37}
\centering
\small
\begin{tabular}{c c c c c c c c}
\hline\hline
Instancia & Costo mínimo & Tiempo(seg.) & Costo promedio & Tiempo promedio(seg.) & CME & \%G & \%GP \\ [0.5ex]
\hline
CMT1X & \bf{470.48} & 2.79 & 
471.72 & 2.62 & 470.48 & 0.00
 & 0.26\\CMT1Y & 471.25 & 2.80 & 
479.70 & 2.46 & \bf{470.48} & 
0.16 & 1.96\\CMT2X & 685.96 & 7.74 & 
688.25 & 5.69 & \bf{682.39} & 
0.52 & 0.86\\CMT2Y & 688.53 & 4.72 & 
691.60 & 5.42 & \bf{682.39} & 
0.90 & 1.35\\CMT3X & 724.57 & 10.17 & 
730.07 & 8.03 & \bf{719.06} & 
0.77 & 1.53\\CMT3Y & 724.57 & 21.51 & 
728.37 & 10.54 & \bf{719.06} & 
0.77 & 1.30\\CMT4X & 866.59 & 34.05 & 
885.17 & 31.48 & \bf{854.21} & 
1.45 & 3.62\\CMT4Y & 854.50 & 57.98 & 
875.44 & 34.17 & \bf{852.46} & 
0.24 & 2.70\\CMT5X & 1043.31 & 72.02 & 
1056.75 & 64.62 & \bf{1030.56} & 
1.24 & 2.54\\CMT5Y & 1048.89 & 57.59 & 
1066.76 & 94.91 & \bf{1031.69} & 
1.67 & 3.40\\CMT11X & 870.94 & 15.51 & 
912.57 & 23.20 & \bf{831.09} & 
4.79 & 9.80\\CMT11Y & 882.48 & 37.21 & 
926.61 & 29.86 & \bf{829.85} & 
6.34 & 11.66\\CMT12X & 670.76 & 8.47 & 
677.81 & 9.70 & \bf{658.83} & 
1.81 & 2.88\\CMT12Y & 673.92 & 8.36 & 
681.57 & 8.91 & \bf{660.47} & 
2.04 & 3.19\\\bf{PROM.} & 
\bf{762.62} & \bf{24.35} & \bf{776.60} & \bf{23.69} & \bf{749.50} & \bf{1.62} & \bf{3.36}\\[1ex]\hline
\end{tabular}
\label{table:nonlin}
\end{table} \clearpage
\begin{table}[ht]
\caption{Resultados de la ejecución de la metaheurística GTS, utilizando instancias de Dethloff con la configuración -mni 4000 -lambda1 0.05 -lambda2 0.05 -tabu 17}
\centering
\small
\begin{tabular}{c c c c c c c c}
\hline\hline
Instancia & Costo mínimo & Tiempo(seg.) & Costo promedio & Tiempo promedio(seg.) & CME & \%G & \%GP \\ [0.5ex]
\hline
SCA3-0 & 636.34 & 3.94 & 
639.50 & 4.07 & \bf{635.62} & 
0.11 & 0.61\\SCA3-1 & \bf{697.84} & 3.76 & 
697.84 & 3.16 & 697.84 & 0.00
 & 0.00\\
SCA3-2 & \bf{659.34} & 2.64 & 
659.34 & 3.41 & 659.34 & 0.00
 & 0.00\\
SCA3-3 & \bf{680.04} & 3.89 & 
682.89 & 2.33 & 680.04 & 0.00
 & 0.42\\SCA3-4 & \bf{690.50} & 5.14 & 
690.50 & 3.60 & 690.50 & 0.00
 & 0.00\\
SCA3-5 & \bf{659.90} & 4.96 & 
659.90 & 3.24 & 659.90 & 0.00
 & 0.00\\
SCA3-6 & \bf{651.09} & 2.27 & 
651.09 & 2.81 & 651.09 & 0.00
 & 0.00\\
SCA3-7 & 666.15 & 4.76 & 
667.09 & 2.67 & \bf{659.17} & 
1.06 & 1.20\\SCA3-8 & \bf{719.47} & 2.70 & 
719.47 & 3.25 & 719.47 & 0.00
 & 0.00\\
SCA3-9 & \bf{681.00} & 2.82 & 
681.00 & 2.50 & 681.00 & 0.00
 & 0.00\\
SCA8-0 & 968.49 & 2.96 & 
972.39 & 3.54 & \bf{961.50} & 
0.73 & 1.13\\SCA8-1 & 1050.20 & 3.48 & 
1055.96 & 3.19 & \bf{1049.65} & 
0.05 & 0.60\\SCA8-2 & \bf{1039.64} & 4.63 & 
1055.08 & 3.29 & 1039.64 & 0.00
 & 1.49\\SCA8-3 & \bf{983.34} & 2.41 & 
1004.59 & 2.79 & 983.34 & 0.00
 & 2.16\\SCA8-4 & \bf{1065.49} & 3.22 & 
1084.81 & 2.97 & 1065.49 & 0.00
 & 1.81\\SCA8-5 & \bf{1027.08} & 6.42 & 
1055.33 & 3.00 & 1027.08 & 0.00
 & 2.75\\SCA8-6 & 972.48 & 3.81 & 
977.02 & 2.49 & \bf{971.82} & 
0.07 & 0.54\\SCA8-7 & 1066.65 & 2.78 & 
1077.55 & 3.69 & \bf{1051.28} & 
1.46 & 2.50\\SCA8-8 & \bf{1071.18} & 2.53 & 
1076.65 & 4.01 & 1071.18 & 0.00
 & 0.51\\SCA8-9 & \bf{1060.50} & 2.66 & 
1073.44 & 2.35 & 1060.50 & 0.00
 & 1.22\\CON3-0 & 617.59 & 1.66 & 
625.57 & 2.46 & \bf{616.52} & 
0.17 & 1.47\\CON3-1 & \bf{554.47} & 3.06 & 
558.56 & 3.10 & 554.47 & 0.00
 & 0.74\\CON3-2 & 523.16 & 2.07 & 
523.21 & 2.65 & \bf{518.00} & 
1.00 & 1.01\\CON3-3 & \bf{591.19} & 5.53 & 
591.19 & 3.62 & 591.19 & 0.00
 & 0.00\\
CON3-4 & \bf{588.79} & 2.49 & 
595.62 & 1.77 & 588.79 & 0.00
 & 1.16\\CON3-5 & \bf{563.70} & 4.72 & 
570.21 & 3.20 & 563.70 & 0.00
 & 1.15\\CON3-6 & \bf{499.05} & 2.03 & 
501.38 & 2.60 & 499.05 & 0.00
 & 0.47\\CON3-7 & \bf{576.48} & 3.82 & 
584.84 & 3.54 & 576.48 & 0.00
 & 1.45\\CON3-8 & \bf{523.05} & 1.83 & 
523.23 & 1.97 & 523.05 & 0.00
 & 0.03\\CON3-9 & 578.25 & 2.34 & 
584.91 & 2.29 & \bf{578.24} & 
0.00 & 1.15\\CON8-0 & 857.40 & 3.29 & 
874.62 & 2.94 & \bf{857.17} & 
0.03 & 2.04\\CON8-1 & \bf{740.85} & 1.82 & 
768.16 & 2.62 & 740.85 & 0.00
 & 3.69\\CON8-2 & 717.36 & 4.11 & 
733.53 & 3.88 & \bf{712.89} & 
0.63 & 2.89\\CON8-3 & \bf{811.07} & 3.91 & 
820.85 & 3.85 & 811.07 & 0.00
 & 1.21\\CON8-4 & \bf{772.25} & 4.21 & 
775.65 & 4.02 & 772.25 & 0.00
 & 0.44\\CON8-5 & 754.95 & 2.74 & 
758.41 & 3.23 & \bf{754.88} & 
0.01 & 0.47\\CON8-6 & 688.47 & 2.30 & 
694.43 & 2.41 & \bf{678.92} & 
1.41 & 2.28\\CON8-7 & 812.89 & 4.05 & 
816.03 & 2.66 & \bf{811.96} & 
0.11 & 0.50\\CON8-8 & \bf{767.53} & 4.35 & 
772.50 & 3.27 & 767.53 & 0.00
 & 0.65\\CON8-9 & 810.19 & 6.03 & 
819.27 & 3.56 & \bf{809.00} & 
0.15 & 1.27\\\bf{PROM.} & 
\bf{759.89} & \bf{3.45} & \bf{766.84} & \bf{3.05} & \bf{758.54} & \bf{0.17} & \bf{1.03}\\[1ex]\hline
\end{tabular}
\label{table:nonlin}
\end{table} \clearpage
\begin{table}[ht]
\caption{Resultados de la ejecución de la metaheurística GTS, utilizando instancias de SalhiNagy con la configuración -mni 4000 -lambda1 0.05 -lambda2 0.05 -tabu 17}
\centering
\small
\begin{tabular}{c c c c c c c c}
\hline\hline
Instancia & Costo mínimo & Tiempo(seg.) & Costo promedio & Tiempo promedio(seg.) & CME & \%G & \%GP \\ [0.5ex]
\hline
CMT1X & \bf{470.48} & 4.12 & 
471.90 & 3.44 & 470.48 & 0.00
 & 0.30\\CMT1Y & \bf{470.48} & 2.28 & 
471.58 & 2.64 & 470.48 & 0.00
 & 0.23\\CMT2X & 683.52 & 4.34 & 
686.39 & 6.11 & \bf{682.39} & 
0.17 & 0.59\\CMT2Y & \bf{682.39} & 7.90 & 
688.79 & 5.79 & 682.39 & 0.00
 & 0.94\\CMT3X & 726.28 & 16.57 & 
730.88 & 10.95 & \bf{719.06} & 
1.00 & 1.64\\CMT3Y & 729.94 & 13.13 & 
732.88 & 10.01 & \bf{719.06} & 
1.51 & 1.92\\CMT4X & 857.79 & 34.58 & 
878.78 & 30.41 & \bf{854.21} & 
0.42 & 2.88\\CMT4Y & 862.84 & 25.81 & 
872.02 & 23.88 & \bf{852.46} & 
1.22 & 2.29\\CMT5X & 1073.55 & 68.30 & 
1083.08 & 67.58 & \bf{1030.56} & 
4.17 & 5.10\\CMT5Y & 1040.76 & 44.98 & 
1061.37 & 61.57 & \bf{1031.69} & 
0.88 & 2.88\\CMT11X & 876.47 & 21.97 & 
906.31 & 23.36 & \bf{831.09} & 
5.46 & 9.05\\CMT11Y & 912.63 & 30.80 & 
947.79 & 62.72 & \bf{829.85} & 
9.98 & 14.21\\CMT12X & 673.59 & 7.35 & 
679.34 & 8.05 & \bf{658.83} & 
2.24 & 3.11\\CMT12Y & 673.77 & 7.72 & 
675.93 & 10.87 & \bf{660.47} & 
2.01 & 2.34\\\bf{PROM.} & 
\bf{766.75} & \bf{20.70} & \bf{777.65} & \bf{23.38} & \bf{749.50} & \bf{2.08} & \bf{3.39}\\[1ex]\hline
\end{tabular}
\label{table:nonlin}
\end{table} \clearpage
\begin{table}[ht]
\caption{Resultados de la ejecución de la metaheurística GTS, utilizando instancias de Dethloff con la configuración -mni 4000 -lambda1 0.05 -lambda2 0.05 -tabu 21}
\centering
\small
\begin{tabular}{c c c c c c c c}
\hline\hline
Instancia & Costo mínimo & Tiempo(seg.) & Costo promedio & Tiempo promedio(seg.) & CME & \%G & \%GP \\ [0.5ex]
\hline
SCA3-0 & 640.55 & 2.65 & 
640.55 & 3.24 & \bf{635.62} & 
0.78 & 0.78\\SCA3-1 & \bf{697.84} & 2.27 & 
699.17 & 2.61 & 697.84 & 0.00
 & 0.19\\SCA3-2 & \bf{659.34} & 2.10 & 
659.34 & 2.78 & 659.34 & 0.00
 & 0.00\\
SCA3-3 & \bf{680.04} & 6.77 & 
685.20 & 3.88 & 680.04 & 0.00
 & 0.76\\SCA3-4 & \bf{690.50} & 3.58 & 
690.50 & 3.86 & 690.50 & 0.00
 & 0.00\\
SCA3-5 & \bf{659.90} & 3.13 & 
663.16 & 3.85 & 659.90 & 0.00
 & 0.49\\SCA3-6 & \bf{651.09} & 3.46 & 
651.74 & 2.58 & 651.09 & 0.00
 & 0.10\\SCA3-7 & 666.15 & 1.41 & 
667.09 & 3.31 & \bf{659.17} & 
1.06 & 1.20\\SCA3-8 & \bf{719.47} & 5.28 & 
726.01 & 4.29 & 719.47 & 0.00
 & 0.91\\SCA3-9 & \bf{681.00} & 1.59 & 
681.00 & 2.71 & 681.00 & 0.00
 & 0.00\\
SCA8-0 & \bf{961.50} & 4.68 & 
969.79 & 4.69 & 961.50 & 0.00
 & 0.86\\SCA8-1 & \bf{1049.65} & 3.59 & 
1064.63 & 3.43 & 1049.65 & 0.00
 & 1.43\\SCA8-2 & \bf{1039.64} & 4.40 & 
1054.60 & 4.11 & 1039.64 & 0.00
 & 1.44\\SCA8-3 & \bf{983.34} & 6.82 & 
1005.61 & 3.67 & 983.34 & 0.00
 & 2.26\\SCA8-4 & 1067.55 & 6.68 & 
1071.04 & 4.33 & \bf{1065.49} & 
0.19 & 0.52\\SCA8-5 & \bf{1027.08} & 3.03 & 
1050.42 & 2.49 & 1027.08 & 0.00
 & 2.27\\SCA8-6 & \bf{971.82} & 3.38 & 
972.32 & 2.86 & 971.82 & 0.00
 & 0.05\\SCA8-7 & \bf{1051.28} & 3.91 & 
1066.81 & 2.88 & 1051.28 & 0.00
 & 1.48\\SCA8-8 & \bf{1071.18} & 2.09 & 
1080.15 & 2.52 & 1071.18 & 0.00
 & 0.84\\SCA8-9 & \bf{1060.50} & 3.89 & 
1063.97 & 4.21 & 1060.50 & 0.00
 & 0.33\\CON3-0 & \bf{616.52} & 5.01 & 
624.13 & 4.01 & 616.52 & 0.00
 & 1.23\\CON3-1 & \bf{554.47} & 2.59 & 
556.42 & 3.27 & 554.47 & 0.00
 & 0.35\\CON3-2 & 523.23 & 2.59 & 
523.79 & 2.67 & \bf{518.00} & 
1.01 & 1.12\\CON3-3 & \bf{591.19} & 1.58 & 
591.92 & 3.83 & 591.19 & 0.00
 & 0.12\\CON3-4 & 589.32 & 1.92 & 
595.75 & 2.44 & \bf{588.79} & 
0.09 & 1.18\\CON3-5 & \bf{563.70} & 3.29 & 
568.13 & 2.80 & 563.70 & 0.00
 & 0.79\\CON3-6 & \bf{499.05} & 3.02 & 
501.10 & 3.17 & 499.05 & 0.00
 & 0.41\\CON3-7 & \bf{576.48} & 6.48 & 
581.75 & 4.58 & 576.48 & 0.00
 & 0.91\\CON3-8 & \bf{523.05} & 2.28 & 
523.21 & 2.48 & 523.05 & 0.00
 & 0.03\\CON3-9 & 583.35 & 5.12 & 
587.52 & 3.21 & \bf{578.24} & 
0.88 & 1.60\\CON8-0 & 866.68 & 2.76 & 
883.23 & 3.06 & \bf{857.17} & 
1.11 & 3.04\\CON8-1 & \bf{740.85} & 2.14 & 
755.33 & 2.55 & 740.85 & 0.00
 & 1.95\\CON8-2 & 713.44 & 5.96 & 
724.44 & 4.13 & \bf{712.89} & 
0.08 & 1.62\\CON8-3 & \bf{811.07} & 2.12 & 
822.66 & 2.40 & 811.07 & 0.00
 & 1.43\\CON8-4 & \bf{772.25} & 4.09 & 
777.72 & 3.25 & 772.25 & 0.00
 & 0.71\\CON8-5 & 756.91 & 1.90 & 
769.55 & 2.17 & \bf{754.88} & 
0.27 & 1.94\\CON8-6 & \bf{678.92} & 3.24 & 
687.12 & 4.30 & 678.92 & 0.00
 & 1.21\\CON8-7 & 812.89 & 3.33 & 
834.64 & 3.04 & \bf{811.96} & 
0.11 & 2.79\\CON8-8 & \bf{767.53} & 1.86 & 
771.30 & 3.25 & 767.53 & 0.00
 & 0.49\\CON8-9 & \bf{809.00} & 5.39 & 
812.90 & 3.47 & 809.00 & 0.00
 & 0.48\\\bf{PROM.} & 
\bf{759.48} & \bf{3.53} & \bf{766.39} & \bf{3.31} & \bf{758.54} & \bf{0.14} & \bf{0.98}\\[1ex]\hline
\end{tabular}
\label{table:nonlin}
\end{table} \clearpage
\begin{table}[ht]
\caption{Resultados de la ejecución de la metaheurística GTS, utilizando instancias de SalhiNagy con la configuración -mni 4000 -lambda1 0.05 -lambda2 0.05 -tabu 21}
\centering
\small
\begin{tabular}{c c c c c c c c}
\hline\hline
Instancia & Costo mínimo & Tiempo(seg.) & Costo promedio & Tiempo promedio(seg.) & CME & \%G & \%GP \\ [0.5ex]
\hline
CMT1X & \bf{470.48} & 2.97 & 
474.13 & 3.80 & 470.48 & 0.00
 & 0.78\\CMT1Y & \bf{470.48} & 3.19 & 
474.26 & 4.10 & 470.48 & 0.00
 & 0.80\\CMT2X & 685.31 & 6.24 & 
688.40 & 6.80 & \bf{682.39} & 
0.43 & 0.88\\CMT2Y & 682.94 & 9.65 & 
686.53 & 7.08 & \bf{682.39} & 
0.08 & 0.61\\CMT3X & 722.97 & 14.50 & 
727.30 & 12.62 & \bf{719.06} & 
0.54 & 1.15\\CMT3Y & 724.57 & 13.38 & 
728.63 & 10.17 & \bf{719.06} & 
0.77 & 1.33\\CMT4X & 868.82 & 43.48 & 
878.53 & 31.31 & \bf{854.21} & 
1.71 & 2.85\\CMT4Y & 867.65 & 26.42 & 
878.34 & 32.33 & \bf{852.46} & 
1.78 & 3.04\\CMT5X & 1068.45 & 76.28 & 
1081.16 & 62.98 & \bf{1030.56} & 
3.68 & 4.91\\CMT5Y & 1050.10 & 106.36 & 
1067.18 & 76.51 & \bf{1031.69} & 
1.78 & 3.44\\CMT11X & 881.19 & 36.93 & 
937.63 & 29.08 & \bf{831.09} & 
6.03 & 12.82\\CMT11Y & 891.76 & 36.85 & 
904.20 & 33.72 & \bf{829.85} & 
7.46 & 8.96\\CMT12X & 663.30 & 14.29 & 
676.59 & 11.96 & \bf{658.83} & 
0.68 & 2.70\\CMT12Y & 673.64 & 13.69 & 
680.99 & 9.81 & \bf{660.47} & 
1.99 & 3.11\\\bf{PROM.} & 
\bf{765.83} & \bf{28.87} & \bf{777.42} & \bf{23.73} & \bf{749.50} & \bf{1.92} & \bf{3.38}\\[1ex]\hline
\end{tabular}
\label{table:nonlin}
\end{table} \clearpage
\begin{table}[ht]
\caption{Resultados de la ejecución de la metaheurística GTS, utilizando instancias de Dethloff con la configuración -mni 4000 -lambda1 0.05 -lambda2 0.05 -tabu 25}
\centering
\small
\begin{tabular}{c c c c c c c c}
\hline\hline
Instancia & Costo mínimo & Tiempo(seg.) & Costo promedio & Tiempo promedio(seg.) & CME & \%G & \%GP \\ [0.5ex]
\hline
SCA3-0 & 636.06 & 3.49 & 
639.43 & 3.73 & \bf{635.62} & 
0.07 & 0.60\\SCA3-1 & \bf{697.84} & 2.41 & 
698.50 & 2.98 & 697.84 & 0.00
 & 0.10\\SCA3-2 & \bf{659.34} & 2.22 & 
659.34 & 2.78 & 659.34 & 0.00
 & 0.00\\
SCA3-3 & \bf{680.04} & 3.30 & 
685.61 & 2.31 & 680.04 & 0.00
 & 0.82\\SCA3-4 & \bf{690.50} & 1.97 & 
690.50 & 3.57 & 690.50 & 0.00
 & 0.00\\
SCA3-5 & \bf{659.90} & 4.24 & 
663.16 & 3.94 & 659.90 & 0.00
 & 0.49\\SCA3-6 & \bf{651.09} & 4.31 & 
651.09 & 2.59 & 651.09 & 0.00
 & 0.00\\
SCA3-7 & 664.88 & 3.26 & 
666.88 & 2.64 & \bf{659.17} & 
0.87 & 1.17\\SCA3-8 & \bf{719.47} & 3.77 & 
719.47 & 3.74 & 719.47 & 0.00
 & 0.00\\
SCA3-9 & \bf{681.00} & 3.95 & 
681.00 & 2.74 & 681.00 & 0.00
 & 0.00\\
SCA8-0 & \bf{961.50} & 3.50 & 
981.05 & 5.19 & 961.50 & 0.00
 & 2.03\\SCA8-1 & 1050.38 & 5.02 & 
1060.06 & 2.85 & \bf{1049.65} & 
0.07 & 0.99\\SCA8-2 & \bf{1039.64} & 3.78 & 
1049.30 & 3.48 & 1039.64 & 0.00
 & 0.93\\SCA8-3 & 1011.27 & 1.91 & 
1013.51 & 2.81 & \bf{983.34} & 
2.84 & 3.07\\SCA8-4 & 1067.55 & 2.76 & 
1073.89 & 2.46 & \bf{1065.49} & 
0.19 & 0.79\\SCA8-5 & \bf{1027.08} & 4.42 & 
1044.83 & 4.84 & 1027.08 & 0.00
 & 1.73\\SCA8-6 & 972.48 & 4.11 & 
972.48 & 3.21 & \bf{971.82} & 
0.07 & 0.07\\SCA8-7 & 1068.16 & 2.28 & 
1075.49 & 2.78 & \bf{1051.28} & 
1.61 & 2.30\\SCA8-8 & \bf{1071.18} & 2.35 & 
1079.52 & 2.79 & 1071.18 & 0.00
 & 0.78\\SCA8-9 & \bf{1060.50} & 2.20 & 
1067.57 & 2.93 & 1060.50 & 0.00
 & 0.67\\CON3-0 & \bf{616.52} & 3.07 & 
617.05 & 3.23 & 616.52 & 0.00
 & 0.09\\CON3-1 & \bf{554.47} & 2.68 & 
557.36 & 2.91 & 554.47 & 0.00
 & 0.52\\CON3-2 & 522.86 & 1.80 & 
523.33 & 2.14 & \bf{518.00} & 
0.94 & 1.03\\CON3-3 & \bf{591.19} & 4.32 & 
600.01 & 3.23 & 591.19 & 0.00
 & 1.49\\CON3-4 & \bf{588.79} & 3.69 & 
593.91 & 3.28 & 588.79 & 0.00
 & 0.87\\CON3-5 & \bf{563.70} & 2.80 & 
565.92 & 4.09 & 563.70 & 0.00
 & 0.39\\CON3-6 & \bf{499.05} & 2.06 & 
499.83 & 2.79 & 499.05 & 0.00
 & 0.16\\CON3-7 & \bf{576.48} & 5.39 & 
584.20 & 4.27 & 576.48 & 0.00
 & 1.34\\CON3-8 & \bf{523.05} & 1.98 & 
523.05 & 2.27 & 523.05 & 0.00
 & 0.00\\
CON3-9 & 578.25 & 6.44 & 
584.00 & 3.63 & \bf{578.24} & 
0.00 & 1.00\\CON8-0 & 858.03 & 2.97 & 
868.10 & 2.90 & \bf{857.17} & 
0.10 & 1.27\\CON8-1 & 751.84 & 2.30 & 
755.73 & 4.72 & \bf{740.85} & 
1.48 & 2.01\\CON8-2 & 716.03 & 2.12 & 
718.09 & 4.39 & \bf{712.89} & 
0.44 & 0.73\\CON8-3 & \bf{811.07} & 2.00 & 
825.02 & 4.40 & 811.07 & 0.00
 & 1.72\\CON8-4 & \bf{772.25} & 2.05 & 
786.11 & 2.75 & 772.25 & 0.00
 & 1.79\\CON8-5 & 758.12 & 1.77 & 
761.88 & 2.98 & \bf{754.88} & 
0.43 & 0.93\\CON8-6 & 683.66 & 2.53 & 
689.23 & 4.52 & \bf{678.92} & 
0.70 & 1.52\\CON8-7 & 812.89 & 4.10 & 
817.47 & 2.80 & \bf{811.96} & 
0.11 & 0.68\\CON8-8 & \bf{767.53} & 1.92 & 
773.49 & 2.45 & 767.53 & 0.00
 & 0.78\\CON8-9 & \bf{809.00} & 4.82 & 
816.32 & 4.61 & 809.00 & 0.00
 & 0.90\\\bf{PROM.} & 
\bf{760.62} & \bf{3.15} & \bf{765.82} & \bf{3.32} & \bf{758.54} & \bf{0.25} & \bf{0.89}\\[1ex]\hline
\end{tabular}
\label{table:nonlin}
\end{table} \clearpage
\begin{table}[ht]
\caption{Resultados de la ejecución de la metaheurística GTS, utilizando instancias de SalhiNagy con la configuración -mni 4000 -lambda1 0.05 -lambda2 0.05 -tabu 25}
\centering
\small
\begin{tabular}{c c c c c c c c}
\hline\hline
Instancia & Costo mínimo & Tiempo(seg.) & Costo promedio & Tiempo promedio(seg.) & CME & \%G & \%GP \\ [0.5ex]
\hline
CMT1X & \bf{470.48} & 4.11 & 
470.60 & 4.67 & 470.48 & 0.00
 & 0.03\\CMT1Y & 470.77 & 3.98 & 
474.92 & 2.64 & \bf{470.48} & 
0.06 & 0.94\\CMT2X & 687.78 & 4.12 & 
690.36 & 5.20 & \bf{682.39} & 
0.79 & 1.17\\CMT2Y & 684.11 & 6.22 & 
693.19 & 5.80 & \bf{682.39} & 
0.25 & 1.58\\CMT3X & 720.96 & 9.24 & 
727.09 & 11.29 & \bf{719.06} & 
0.26 & 1.12\\CMT3Y & 724.40 & 12.65 & 
728.66 & 8.22 & \bf{719.06} & 
0.74 & 1.34\\CMT4X & 859.75 & 30.15 & 
874.01 & 30.95 & \bf{854.21} & 
0.65 & 2.32\\CMT4Y & 862.55 & 17.70 & 
886.77 & 22.68 & \bf{852.46} & 
1.18 & 4.02\\CMT5X & 1046.01 & 50.09 & 
1056.94 & 72.31 & \bf{1030.56} & 
1.50 & 2.56\\CMT5Y & 1042.81 & 97.61 & 
1050.70 & 82.20 & \bf{1031.69} & 
1.08 & 1.84\\CMT11X & 876.38 & 28.62 & 
907.83 & 33.86 & \bf{831.09} & 
5.45 & 9.23\\CMT11Y & 883.34 & 32.46 & 
907.85 & 29.17 & \bf{829.85} & 
6.45 & 9.40\\CMT12X & 673.32 & 21.38 & 
675.98 & 10.00 & \bf{658.83} & 
2.20 & 2.60\\CMT12Y & 673.59 & 11.57 & 
677.93 & 11.44 & \bf{660.47} & 
1.99 & 2.64\\\bf{PROM.} & 
\bf{762.59} & \bf{23.56} & \bf{773.06} & \bf{23.60} & \bf{749.50} & \bf{1.61} & \bf{2.91}\\[1ex]\hline
\end{tabular}
\label{table:nonlin}
\end{table} \clearpage
\begin{table}[ht]
\caption{Resultados de la ejecución de la metaheurística GTS, utilizando instancias de Dethloff con la configuración -mni 4000 -lambda1 0.05 -lambda2 0.05 -tabu 29}
\centering
\small
\begin{tabular}{c c c c c c c c}
\hline\hline
Instancia & Costo mínimo & Tiempo(seg.) & Costo promedio & Tiempo promedio(seg.) & CME & \%G & \%GP \\ [0.5ex]
\hline
SCA3-0 & 636.06 & 4.74 & 
638.30 & 3.71 & \bf{635.62} & 
0.07 & 0.42\\SCA3-1 & \bf{697.84} & 3.99 & 
697.84 & 3.89 & 697.84 & 0.00
 & 0.00\\
SCA3-2 & \bf{659.34} & 3.56 & 
659.34 & 2.81 & 659.34 & 0.00
 & 0.00\\
SCA3-3 & \bf{680.04} & 3.99 & 
683.03 & 3.44 & 680.04 & 0.00
 & 0.44\\SCA3-4 & \bf{690.50} & 3.32 & 
690.50 & 4.12 & 690.50 & 0.00
 & 0.00\\
SCA3-5 & \bf{659.90} & 2.56 & 
659.90 & 2.91 & 659.90 & 0.00
 & 0.00\\
SCA3-6 & \bf{651.09} & 4.73 & 
651.55 & 3.03 & 651.09 & 0.00
 & 0.07\\SCA3-7 & 664.88 & 3.66 & 
665.83 & 3.73 & \bf{659.17} & 
0.87 & 1.01\\SCA3-8 & \bf{719.47} & 4.12 & 
719.47 & 3.05 & 719.47 & 0.00
 & 0.00\\
SCA3-9 & \bf{681.00} & 4.86 & 
681.00 & 3.65 & 681.00 & 0.00
 & 0.00\\
SCA8-0 & 979.79 & 4.17 & 
990.31 & 3.35 & \bf{961.50} & 
1.90 & 3.00\\SCA8-1 & \bf{1049.65} & 4.35 & 
1054.32 & 3.81 & 1049.65 & 0.00
 & 0.44\\SCA8-2 & 1050.17 & 1.62 & 
1059.37 & 2.19 & \bf{1039.64} & 
1.01 & 1.90\\SCA8-3 & 1005.75 & 2.94 & 
1012.13 & 3.22 & \bf{983.34} & 
2.28 & 2.93\\SCA8-4 & 1067.28 & 5.38 & 
1068.38 & 4.38 & \bf{1065.49} & 
0.17 & 0.27\\SCA8-5 & \bf{1027.08} & 3.91 & 
1047.28 & 3.21 & 1027.08 & 0.00
 & 1.97\\SCA8-6 & 972.48 & 2.60 & 
977.02 & 3.58 & \bf{971.82} & 
0.07 & 0.54\\SCA8-7 & 1063.22 & 4.35 & 
1069.80 & 4.27 & \bf{1051.28} & 
1.14 & 1.76\\SCA8-8 & \bf{1071.18} & 1.43 & 
1073.91 & 2.60 & 1071.18 & 0.00
 & 0.25\\SCA8-9 & 1072.91 & 3.39 & 
1074.51 & 2.67 & \bf{1060.50} & 
1.17 & 1.32\\CON3-0 & 628.36 & 4.80 & 
629.53 & 3.43 & \bf{616.52} & 
1.92 & 2.11\\CON3-1 & \bf{554.47} & 1.61 & 
557.45 & 1.94 & 554.47 & 0.00
 & 0.54\\CON3-2 & 523.23 & 5.99 & 
523.60 & 4.05 & \bf{518.00} & 
1.01 & 1.08\\CON3-3 & \bf{591.19} & 3.52 & 
597.89 & 3.40 & 591.19 & 0.00
 & 1.13\\CON3-4 & \bf{588.79} & 7.40 & 
595.62 & 4.04 & 588.79 & 0.00
 & 1.16\\CON3-5 & \bf{563.70} & 4.86 & 
569.76 & 3.84 & 563.70 & 0.00
 & 1.08\\CON3-6 & \bf{499.05} & 3.30 & 
500.42 & 2.73 & 499.05 & 0.00
 & 0.27\\CON3-7 & \bf{576.48} & 2.80 & 
577.89 & 3.13 & 576.48 & 0.00
 & 0.24\\CON3-8 & \bf{523.05} & 3.44 & 
523.05 & 6.08 & 523.05 & 0.00
 & 0.00\\
CON3-9 & 578.25 & 4.47 & 
587.97 & 3.59 & \bf{578.24} & 
0.00 & 1.68\\CON8-0 & \bf{857.17} & 4.51 & 
870.80 & 3.63 & 857.17 & 0.00
 & 1.59\\CON8-1 & 741.70 & 3.56 & 
752.60 & 3.54 & \bf{740.85} & 
0.11 & 1.59\\CON8-2 & 717.67 & 5.53 & 
730.96 & 2.89 & \bf{712.89} & 
0.67 & 2.53\\CON8-3 & 823.10 & 1.88 & 
830.62 & 3.11 & \bf{811.07} & 
1.48 & 2.41\\CON8-4 & \bf{772.25} & 3.22 & 
775.39 & 2.40 & 772.25 & 0.00
 & 0.41\\CON8-5 & \bf{754.88} & 2.94 & 
757.46 & 2.29 & 754.88 & 0.00
 & 0.34\\CON8-6 & \bf{678.92} & 5.07 & 
695.25 & 3.70 & 678.92 & 0.00
 & 2.40\\CON8-7 & 812.26 & 2.62 & 
813.16 & 4.02 & \bf{811.96} & 
0.04 & 0.15\\CON8-8 & 776.55 & 5.21 & 
777.01 & 3.10 & \bf{767.53} & 
1.18 & 1.24\\CON8-9 & \bf{809.00} & 3.40 & 
820.28 & 3.17 & 809.00 & 0.00
 & 1.39\\\bf{PROM.} & 
\bf{761.74} & \bf{3.84} & \bf{766.51} & \bf{3.39} & \bf{758.54} & \bf{0.38} & \bf{0.99}\\[1ex]\hline
\end{tabular}
\label{table:nonlin}
\end{table} \clearpage
\begin{table}[ht]
\caption{Resultados de la ejecución de la metaheurística GTS, utilizando instancias de SalhiNagy con la configuración -mni 4000 -lambda1 0.05 -lambda2 0.05 -tabu 29}
\centering
\small
\begin{tabular}{c c c c c c c c}
\hline\hline
Instancia & Costo mínimo & Tiempo(seg.) & Costo promedio & Tiempo promedio(seg.) & CME & \%G & \%GP \\ [0.5ex]
\hline
CMT1X & \bf{470.48} & 3.33 & 
470.95 & 3.92 & 470.48 & 0.00
 & 0.10\\CMT1Y & \bf{470.48} & 6.44 & 
471.47 & 5.04 & 470.48 & 0.00
 & 0.21\\CMT2X & 687.91 & 8.30 & 
691.80 & 6.64 & \bf{682.39} & 
0.81 & 1.38\\CMT2Y & 683.64 & 6.55 & 
688.98 & 6.04 & \bf{682.39} & 
0.18 & 0.96\\CMT3X & 723.86 & 14.35 & 
728.78 & 11.42 & \bf{719.06} & 
0.67 & 1.35\\CMT3Y & 722.63 & 4.10 & 
733.55 & 8.23 & \bf{719.06} & 
0.50 & 2.01\\CMT4X & 875.78 & 37.33 & 
880.71 & 32.28 & \bf{854.21} & 
2.53 & 3.10\\CMT4Y & 870.43 & 43.85 & 
874.97 & 29.39 & \bf{852.46} & 
2.11 & 2.64\\CMT5X & 1060.22 & 72.50 & 
1069.76 & 53.93 & \bf{1030.56} & 
2.88 & 3.80\\CMT5Y & 1049.77 & 54.06 & 
1063.03 & 64.77 & \bf{1031.69} & 
1.75 & 3.04\\CMT11X & 876.48 & 28.47 & 
951.53 & 27.43 & \bf{831.09} & 
5.46 & 14.49\\CMT11Y & 849.79 & 17.30 & 
900.68 & 20.84 & \bf{829.85} & 
2.40 & 8.54\\CMT12X & 664.05 & 11.08 & 
675.48 & 14.47 & \bf{658.83} & 
0.79 & 2.53\\CMT12Y & 681.53 & 15.78 & 
683.52 & 11.88 & \bf{660.47} & 
3.19 & 3.49\\\bf{PROM.} & 
\bf{763.36} & \bf{23.10} & \bf{777.51} & \bf{21.16} & \bf{749.50} & \bf{1.66} & \bf{3.40}\\[1ex]\hline
\end{tabular}
\label{table:nonlin}
\end{table} \clearpage
\begin{table}[ht]
\caption{Resultados de la ejecución de la metaheurística GTS, utilizando instancias de Dethloff con la configuración -mni 4000 -lambda1 0.05 -lambda2 0.05 -tabu 33}
\centering
\small
\begin{tabular}{c c c c c c c c}
\hline\hline
Instancia & Costo mínimo & Tiempo(seg.) & Costo promedio & Tiempo promedio(seg.) & CME & \%G & \%GP \\ [0.5ex]
\hline
SCA3-0 & 640.55 & 3.38 & 
640.55 & 3.46 & \bf{635.62} & 
0.78 & 0.78\\SCA3-1 & \bf{697.84} & 1.86 & 
698.50 & 2.83 & 697.84 & 0.00
 & 0.10\\SCA3-2 & \bf{659.34} & 2.04 & 
659.34 & 3.13 & 659.34 & 0.00
 & 0.00\\
SCA3-3 & \bf{680.04} & 4.34 & 
682.62 & 3.15 & 680.04 & 0.00
 & 0.38\\SCA3-4 & \bf{690.50} & 4.88 & 
690.50 & 5.05 & 690.50 & 0.00
 & 0.00\\
SCA3-5 & \bf{659.90} & 4.43 & 
659.90 & 3.41 & 659.90 & 0.00
 & 0.00\\
SCA3-6 & \bf{651.09} & 2.44 & 
653.70 & 2.63 & 651.09 & 0.00
 & 0.40\\SCA3-7 & 666.15 & 4.63 & 
666.15 & 4.38 & \bf{659.17} & 
1.06 & 1.06\\SCA3-8 & \bf{719.47} & 3.31 & 
719.47 & 4.33 & 719.47 & 0.00
 & 0.00\\
SCA3-9 & \bf{681.00} & 2.27 & 
681.00 & 3.15 & 681.00 & 0.00
 & 0.00\\
SCA8-0 & \bf{961.50} & 5.36 & 
973.58 & 4.56 & 961.50 & 0.00
 & 1.26\\SCA8-1 & \bf{1049.65} & 3.12 & 
1060.42 & 3.38 & 1049.65 & 0.00
 & 1.03\\SCA8-2 & 1042.10 & 3.04 & 
1056.38 & 3.08 & \bf{1039.64} & 
0.24 & 1.61\\SCA8-3 & \bf{983.34} & 4.44 & 
997.49 & 4.29 & 983.34 & 0.00
 & 1.44\\SCA8-4 & 1067.28 & 7.85 & 
1071.93 & 4.49 & \bf{1065.49} & 
0.17 & 0.60\\SCA8-5 & \bf{1027.08} & 2.00 & 
1041.09 & 2.79 & 1027.08 & 0.00
 & 1.36\\SCA8-6 & 972.48 & 3.34 & 
977.02 & 3.23 & \bf{971.82} & 
0.07 & 0.54\\SCA8-7 & 1063.22 & 3.50 & 
1089.03 & 2.68 & \bf{1051.28} & 
1.14 & 3.59\\SCA8-8 & \bf{1071.18} & 1.72 & 
1077.80 & 1.96 & 1071.18 & 0.00
 & 0.62\\SCA8-9 & \bf{1060.50} & 4.82 & 
1061.30 & 4.80 & 1060.50 & 0.00
 & 0.08\\CON3-0 & \bf{616.52} & 5.29 & 
620.43 & 3.10 & 616.52 & 0.00
 & 0.63\\CON3-1 & \bf{554.47} & 8.69 & 
556.75 & 4.38 & 554.47 & 0.00
 & 0.41\\CON3-2 & 519.61 & 4.56 & 
522.44 & 4.99 & \bf{518.00} & 
0.31 & 0.86\\CON3-3 & \bf{591.19} & 3.90 & 
591.19 & 3.55 & 591.19 & 0.00
 & 0.00\\
CON3-4 & \bf{588.79} & 8.94 & 
591.38 & 5.08 & 588.79 & 0.00
 & 0.44\\CON3-5 & \bf{563.70} & 4.31 & 
563.70 & 4.74 & 563.70 & 0.00
 & 0.00\\
CON3-6 & \bf{499.05} & 3.20 & 
501.04 & 4.67 & 499.05 & 0.00
 & 0.40\\CON3-7 & \bf{576.48} & 4.38 & 
580.84 & 4.86 & 576.48 & 0.00
 & 0.76\\CON3-8 & \bf{523.05} & 2.57 & 
523.28 & 2.60 & 523.05 & 0.00
 & 0.04\\CON3-9 & 578.25 & 6.72 & 
580.52 & 4.66 & \bf{578.24} & 
0.00 & 0.39\\CON8-0 & 857.40 & 3.00 & 
877.22 & 4.55 & \bf{857.17} & 
0.03 & 2.34\\CON8-1 & \bf{740.85} & 9.88 & 
752.87 & 5.86 & 740.85 & 0.00
 & 1.62\\CON8-2 & 713.44 & 4.15 & 
726.62 & 3.00 & \bf{712.89} & 
0.08 & 1.93\\CON8-3 & 812.32 & 8.83 & 
830.61 & 5.23 & \bf{811.07} & 
0.15 & 2.41\\CON8-4 & \bf{772.25} & 1.98 & 
780.02 & 2.66 & 772.25 & 0.00
 & 1.01\\CON8-5 & \bf{754.88} & 2.50 & 
758.82 & 2.98 & 754.88 & 0.00
 & 0.52\\CON8-6 & \bf{678.92} & 3.43 & 
689.40 & 3.31 & 678.92 & 0.00
 & 1.54\\CON8-7 & \bf{811.96} & 6.55 & 
813.20 & 3.91 & 811.96 & 0.00
 & 0.15\\CON8-8 & \bf{767.53} & 7.16 & 
771.31 & 4.52 & 767.53 & 0.00
 & 0.49\\CON8-9 & 812.35 & 4.10 & 
816.29 & 4.85 & \bf{809.00} & 
0.41 & 0.90\\\bf{PROM.} & 
\bf{759.43} & \bf{4.42} & \bf{765.14} & \bf{3.86} & \bf{758.54} & \bf{0.11} & \bf{0.79}\\[1ex]\hline
\end{tabular}
\label{table:nonlin}
\end{table} \clearpage
\begin{table}[ht]
\caption{Resultados de la ejecución de la metaheurística GTS, utilizando instancias de SalhiNagy con la configuración -mni 4000 -lambda1 0.05 -lambda2 0.05 -tabu 33}
\centering
\small
\begin{tabular}{c c c c c c c c}
\hline\hline
Instancia & Costo mínimo & Tiempo(seg.) & Costo promedio & Tiempo promedio(seg.) & CME & \%G & \%GP \\ [0.5ex]
\hline
CMT1X & \bf{470.48} & 4.44 & 
472.15 & 3.42 & 470.48 & 0.00
 & 0.35\\CMT1Y & \bf{470.48} & 3.74 & 
473.71 & 3.35 & 470.48 & 0.00
 & 0.69\\CMT2X & 686.75 & 5.13 & 
691.74 & 6.11 & \bf{682.39} & 
0.64 & 1.37\\CMT2Y & 684.29 & 4.98 & 
689.66 & 5.98 & \bf{682.39} & 
0.28 & 1.07\\CMT3X & \bf{\underline{718.94}} & 11.30 & 
727.38 & 12.19 & 719.06 & 
\bf{-0.02} & 1.16\\CMT3Y & 727.53 & 16.75 & 
732.01 & 11.62 & \bf{719.06} & 
1.18 & 1.80\\CMT4X & 862.24 & 34.77 & 
872.88 & 33.04 & \bf{854.21} & 
0.94 & 2.19\\CMT4Y & 868.74 & 41.79 & 
883.56 & 32.16 & \bf{852.46} & 
1.91 & 3.65\\CMT5X & 1045.89 & 115.03 & 
1066.31 & 75.94 & \bf{1030.56} & 
1.49 & 3.47\\CMT5Y & 1043.17 & 55.61 & 
1076.20 & 58.83 & \bf{1031.69} & 
1.11 & 4.31\\CMT11X & 881.18 & 16.25 & 
890.05 & 35.73 & \bf{831.09} & 
6.03 & 7.09\\CMT11Y & 877.72 & 21.65 & 
920.29 & 23.57 & \bf{829.85} & 
5.77 & 10.90\\CMT12X & 671.81 & 11.75 & 
677.84 & 8.51 & \bf{658.83} & 
1.97 & 2.88\\CMT12Y & 673.64 & 14.27 & 
679.94 & 9.87 & \bf{660.47} & 
1.99 & 2.95\\\bf{PROM.} & 
\bf{763.06} & \bf{25.53} & \bf{775.27} & \bf{22.88} & \bf{749.50} & \bf{1.66} & \bf{3.13}\\[1ex]\hline
\end{tabular}
\label{table:nonlin}
\end{table} \clearpage
\begin{table}[ht]
\caption{Resultados de la ejecución de la metaheurística GTS, utilizando instancias de Dethloff con la configuración -mni 4000 -lambda1 0.05 -lambda2 0.05 -tabu 37}
\centering
\small
\begin{tabular}{c c c c c c c c}
\hline\hline
Instancia & Costo mínimo & Tiempo(seg.) & Costo promedio & Tiempo promedio(seg.) & CME & \%G & \%GP \\ [0.5ex]
\hline
SCA3-0 & 636.06 & 4.24 & 
638.30 & 5.06 & \bf{635.62} & 
0.07 & 0.42\\SCA3-1 & \bf{697.84} & 2.40 & 
697.84 & 3.65 & 697.84 & 0.00
 & 0.00\\
SCA3-2 & \bf{659.34} & 4.62 & 
659.34 & 3.25 & 659.34 & 0.00
 & 0.00\\
SCA3-3 & \bf{680.04} & 8.20 & 
680.46 & 4.85 & 680.04 & 0.00
 & 0.06\\SCA3-4 & \bf{690.50} & 5.85 & 
702.78 & 3.40 & 690.50 & 0.00
 & 1.78\\SCA3-5 & \bf{659.90} & 6.72 & 
663.16 & 5.43 & 659.90 & 0.00
 & 0.49\\SCA3-6 & \bf{651.09} & 3.16 & 
651.09 & 3.92 & 651.09 & 0.00
 & 0.00\\
SCA3-7 & 666.15 & 7.03 & 
666.43 & 4.33 & \bf{659.17} & 
1.06 & 1.10\\SCA3-8 & \bf{719.47} & 5.57 & 
719.47 & 4.51 & 719.47 & 0.00
 & 0.00\\
SCA3-9 & \bf{681.00} & 2.05 & 
681.00 & 3.52 & 681.00 & 0.00
 & 0.00\\
SCA8-0 & 970.64 & 4.80 & 
988.57 & 3.87 & \bf{961.50} & 
0.95 & 2.81\\SCA8-1 & 1051.40 & 1.46 & 
1063.83 & 2.93 & \bf{1049.65} & 
0.17 & 1.35\\SCA8-2 & \bf{1039.64} & 4.80 & 
1049.10 & 3.13 & 1039.64 & 0.00
 & 0.91\\SCA8-3 & \bf{983.34} & 5.66 & 
995.71 & 4.41 & 983.34 & 0.00
 & 1.26\\SCA8-4 & 1067.55 & 3.31 & 
1069.06 & 4.59 & \bf{1065.49} & 
0.19 & 0.34\\SCA8-5 & 1042.30 & 3.79 & 
1043.89 & 3.25 & \bf{1027.08} & 
1.48 & 1.64\\SCA8-6 & 972.48 & 1.93 & 
982.54 & 2.82 & \bf{971.82} & 
0.07 & 1.10\\SCA8-7 & 1071.64 & 2.69 & 
1078.79 & 2.86 & \bf{1051.28} & 
1.94 & 2.62\\SCA8-8 & \bf{1071.18} & 1.56 & 
1079.96 & 3.13 & 1071.18 & 0.00
 & 0.82\\SCA8-9 & \bf{1060.50} & 8.34 & 
1063.03 & 5.14 & 1060.50 & 0.00
 & 0.24\\CON3-0 & \bf{616.52} & 6.78 & 
616.79 & 5.28 & 616.52 & 0.00
 & 0.04\\CON3-1 & \bf{554.47} & 2.74 & 
556.83 & 3.96 & 554.47 & 0.00
 & 0.42\\CON3-2 & \bf{518.00} & 2.92 & 
522.10 & 3.65 & 518.00 & 0.00
 & 0.79\\CON3-3 & \bf{591.19} & 4.23 & 
591.19 & 3.54 & 591.19 & 0.00
 & 0.00\\
CON3-4 & \bf{588.79} & 4.09 & 
591.33 & 3.26 & 588.79 & 0.00
 & 0.43\\CON3-5 & \bf{563.70} & 4.00 & 
574.99 & 3.78 & 563.70 & 0.00
 & 2.00\\CON3-6 & \bf{499.05} & 3.04 & 
499.46 & 2.90 & 499.05 & 0.00
 & 0.08\\CON3-7 & \bf{576.48} & 9.14 & 
580.34 & 4.80 & 576.48 & 0.00
 & 0.67\\CON3-8 & \bf{523.05} & 4.86 & 
523.21 & 3.75 & 523.05 & 0.00
 & 0.03\\CON3-9 & 578.25 & 3.40 & 
582.76 & 4.70 & \bf{578.24} & 
0.00 & 0.78\\CON8-0 & 858.63 & 5.50 & 
862.75 & 4.29 & \bf{857.17} & 
0.17 & 0.65\\CON8-1 & \bf{740.85} & 3.13 & 
750.62 & 3.68 & 740.85 & 0.00
 & 1.32\\CON8-2 & 713.44 & 2.90 & 
726.76 & 3.06 & \bf{712.89} & 
0.08 & 1.95\\CON8-3 & 825.59 & 3.12 & 
838.00 & 2.73 & \bf{811.07} & 
1.79 & 3.32\\CON8-4 & \bf{772.25} & 2.47 & 
778.01 & 2.65 & 772.25 & 0.00
 & 0.75\\CON8-5 & 755.14 & 5.95 & 
765.61 & 3.62 & \bf{754.88} & 
0.03 & 1.42\\CON8-6 & 683.83 & 1.66 & 
693.21 & 2.40 & \bf{678.92} & 
0.72 & 2.10\\CON8-7 & 813.80 & 4.02 & 
814.61 & 2.77 & \bf{811.96} & 
0.23 & 0.33\\CON8-8 & 776.55 & 1.70 & 
788.48 & 1.96 & \bf{767.53} & 
1.18 & 2.73\\CON8-9 & 810.30 & 11.52 & 
815.22 & 5.88 & \bf{809.00} & 
0.16 & 0.77\\\bf{PROM.} & 
\bf{760.80} & \bf{4.38} & \bf{766.16} & \bf{3.77} & \bf{758.54} & \bf{0.26} & \bf{0.94}\\[1ex]\hline
\end{tabular}
\label{table:nonlin}
\end{table} \clearpage
\begin{table}[ht]
\caption{Resultados de la ejecución de la metaheurística GTS, utilizando instancias de SalhiNagy con la configuración -mni 4000 -lambda1 0.05 -lambda2 0.05 -tabu 37}
\centering
\small
\begin{tabular}{c c c c c c c c}
\hline\hline
Instancia & Costo mínimo & Tiempo(seg.) & Costo promedio & Tiempo promedio(seg.) & CME & \%G & \%GP \\ [0.5ex]
\hline
CMT1X & 470.67 & 8.03 & 
471.94 & 5.64 & \bf{470.48} & 
0.04 & 0.31\\CMT1Y & \bf{470.48} & 5.46 & 
471.00 & 4.18 & 470.48 & 0.00
 & 0.11\\CMT2X & 686.93 & 6.64 & 
690.55 & 5.38 & \bf{682.39} & 
0.67 & 1.20\\CMT2Y & 687.99 & 5.46 & 
690.59 & 4.99 & \bf{682.39} & 
0.82 & 1.20\\CMT3X & 726.69 & 12.09 & 
732.30 & 13.71 & \bf{719.06} & 
1.06 & 1.84\\CMT3Y & 720.18 & 14.42 & 
725.28 & 10.59 & \bf{719.06} & 
0.16 & 0.87\\CMT4X & 863.91 & 30.86 & 
872.14 & 40.36 & \bf{854.21} & 
1.14 & 2.10\\CMT4Y & 873.11 & 48.59 & 
878.62 & 37.42 & \bf{852.46} & 
2.42 & 3.07\\CMT5X & 1065.30 & 50.19 & 
1075.46 & 61.46 & \bf{1030.56} & 
3.37 & 4.36\\CMT5Y & 1043.76 & 78.80 & 
1057.11 & 91.23 & \bf{1031.69} & 
1.17 & 2.46\\CMT11X & 877.15 & 54.11 & 
920.16 & 32.06 & \bf{831.09} & 
5.54 & 10.72\\CMT11Y & 878.21 & 31.21 & 
923.25 & 30.60 & \bf{829.85} & 
5.83 & 11.25\\CMT12X & 664.57 & 13.80 & 
670.66 & 12.14 & \bf{658.83} & 
0.87 & 1.80\\CMT12Y & 675.31 & 7.18 & 
686.51 & 9.45 & \bf{660.47} & 
2.25 & 3.94\\\bf{PROM.} & 
\bf{764.59} & \bf{26.20} & \bf{776.11} & \bf{25.66} & \bf{749.50} & \bf{1.81} & \bf{3.23}\\[1ex]\hline
\end{tabular}
\label{table:nonlin}
\end{table} \clearpage
\begin{table}[ht]
\caption{Resultados de la ejecución de la metaheurística GTS, utilizando instancias de Dethloff con la configuración -mni 4500 -lambda1 0.05 -lambda2 0.05 -tabu 17}
\centering
\small
\begin{tabular}{c c c c c c c c}
\hline\hline
Instancia & Costo mínimo & Tiempo(seg.) & Costo promedio & Tiempo promedio(seg.) & CME & \%G & \%GP \\ [0.5ex]
\hline
SCA3-0 & 636.06 & 1.94 & 
639.43 & 3.91 & \bf{635.62} & 
0.07 & 0.60\\SCA3-1 & \bf{697.84} & 1.83 & 
698.76 & 3.52 & 697.84 & 0.00
 & 0.13\\SCA3-2 & \bf{659.34} & 1.66 & 
659.34 & 2.44 & 659.34 & 0.00
 & 0.00\\
SCA3-3 & \bf{680.04} & 4.51 & 
685.47 & 2.56 & 680.04 & 0.00
 & 0.80\\SCA3-4 & \bf{690.50} & 3.16 & 
690.50 & 2.64 & 690.50 & 0.00
 & 0.00\\
SCA3-5 & \bf{659.90} & 1.90 & 
666.42 & 2.64 & 659.90 & 0.00
 & 0.99\\SCA3-6 & \bf{651.09} & 2.16 & 
651.55 & 4.22 & 651.09 & 0.00
 & 0.07\\SCA3-7 & 666.15 & 2.10 & 
666.15 & 2.56 & \bf{659.17} & 
1.06 & 1.06\\SCA3-8 & \bf{719.47} & 5.09 & 
719.47 & 3.95 & 719.47 & 0.00
 & 0.00\\
SCA3-9 & \bf{681.00} & 2.36 & 
681.00 & 3.22 & 681.00 & 0.00
 & 0.00\\
SCA8-0 & 970.64 & 2.87 & 
977.04 & 3.31 & \bf{961.50} & 
0.95 & 1.62\\SCA8-1 & 1050.38 & 3.23 & 
1065.51 & 3.00 & \bf{1049.65} & 
0.07 & 1.51\\SCA8-2 & 1050.37 & 2.86 & 
1050.37 & 2.92 & \bf{1039.64} & 
1.03 & 1.03\\SCA8-3 & \bf{983.34} & 8.04 & 
1005.41 & 4.51 & 983.34 & 0.00
 & 2.24\\SCA8-4 & 1071.16 & 5.08 & 
1081.26 & 3.95 & \bf{1065.49} & 
0.53 & 1.48\\SCA8-5 & \bf{1027.08} & 3.76 & 
1030.88 & 4.06 & 1027.08 & 0.00
 & 0.37\\SCA8-6 & 972.48 & 3.08 & 
982.12 & 4.17 & \bf{971.82} & 
0.07 & 1.06\\SCA8-7 & 1069.46 & 4.14 & 
1078.40 & 2.71 & \bf{1051.28} & 
1.73 & 2.58\\SCA8-8 & \bf{1071.18} & 2.47 & 
1080.26 & 2.81 & 1071.18 & 0.00
 & 0.85\\SCA8-9 & \bf{1060.50} & 4.07 & 
1062.23 & 3.37 & 1060.50 & 0.00
 & 0.16\\CON3-0 & \bf{616.52} & 4.58 & 
632.55 & 2.92 & 616.52 & 0.00
 & 2.60\\CON3-1 & \bf{554.47} & 2.85 & 
556.79 & 2.37 & 554.47 & 0.00
 & 0.42\\CON3-2 & 521.38 & 4.49 & 
523.54 & 4.46 & \bf{518.00} & 
0.65 & 1.07\\CON3-3 & \bf{591.19} & 3.54 & 
609.92 & 2.50 & 591.19 & 0.00
 & 3.17\\CON3-4 & \bf{588.79} & 2.81 & 
594.41 & 3.50 & 588.79 & 0.00
 & 0.96\\CON3-5 & 572.56 & 2.78 & 
575.36 & 3.00 & \bf{563.70} & 
1.57 & 2.07\\CON3-6 & \bf{499.05} & 2.67 & 
501.91 & 2.19 & 499.05 & 0.00
 & 0.57\\CON3-7 & \bf{576.48} & 4.00 & 
578.46 & 3.05 & 576.48 & 0.00
 & 0.34\\CON3-8 & \bf{523.05} & 2.77 & 
523.07 & 3.78 & 523.05 & 0.00
 & 0.00\\CON3-9 & 578.25 & 1.90 & 
587.54 & 2.49 & \bf{578.24} & 
0.00 & 1.61\\CON8-0 & \bf{857.17} & 5.75 & 
868.86 & 5.02 & 857.17 & 0.00
 & 1.36\\CON8-1 & \bf{740.85} & 5.19 & 
754.00 & 3.32 & 740.85 & 0.00
 & 1.77\\CON8-2 & 713.44 & 2.22 & 
722.34 & 2.55 & \bf{712.89} & 
0.08 & 1.33\\CON8-3 & \bf{811.07} & 3.26 & 
811.07 & 4.58 & 811.07 & 0.00
 & 0.00\\
CON8-4 & 784.83 & 4.17 & 
786.35 & 3.78 & \bf{772.25} & 
1.63 & 1.83\\CON8-5 & 755.67 & 7.40 & 
757.33 & 3.77 & \bf{754.88} & 
0.10 & 0.32\\CON8-6 & \bf{678.92} & 2.87 & 
689.21 & 3.68 & 678.92 & 0.00
 & 1.52\\CON8-7 & 812.89 & 6.39 & 
813.77 & 6.53 & \bf{811.96} & 
0.11 & 0.22\\CON8-8 & \bf{767.53} & 2.62 & 
777.13 & 3.16 & 767.53 & 0.00
 & 1.25\\CON8-9 & 814.14 & 3.77 & 
821.32 & 4.59 & \bf{809.00} & 
0.64 & 1.52\\\bf{PROM.} & 
\bf{760.66} & \bf{3.56} & \bf{766.41} & \bf{3.44} & \bf{758.54} & \bf{0.26} & \bf{1.01}\\[1ex]\hline
\end{tabular}
\label{table:nonlin}
\end{table} \clearpage
\begin{table}[ht]
\caption{Resultados de la ejecución de la metaheurística GTS, utilizando instancias de SalhiNagy con la configuración -mni 4500 -lambda1 0.05 -lambda2 0.05 -tabu 17}
\centering
\small
\begin{tabular}{c c c c c c c c}
\hline\hline
Instancia & Costo mínimo & Tiempo(seg.) & Costo promedio & Tiempo promedio(seg.) & CME & \%G & \%GP \\ [0.5ex]
\hline
CMT1X & 472.37 & 2.79 & 
472.52 & 4.24 & \bf{470.48} & 
0.40 & 0.43\\CMT1Y & \bf{470.48} & 5.16 & 
473.74 & 4.84 & 470.48 & 0.00
 & 0.69\\CMT2X & 684.20 & 4.15 & 
687.88 & 4.00 & \bf{682.39} & 
0.27 & 0.81\\CMT2Y & \bf{682.39} & 7.20 & 
685.49 & 4.96 & 682.39 & 0.00
 & 0.45\\CMT3X & \bf{\underline{718.40}} & 28.40 & 
728.48 & 15.15 & 719.06 & 
\bf{-0.09} & 1.31\\CMT3Y & 725.72 & 13.01 & 
726.45 & 14.46 & \bf{719.06} & 
0.93 & 1.03\\CMT4X & 860.68 & 72.20 & 
882.92 & 48.37 & \bf{854.21} & 
0.76 & 3.36\\CMT4Y & 861.56 & 56.73 & 
869.76 & 44.46 & \bf{852.46} & 
1.07 & 2.03\\CMT5X & 1046.80 & 64.11 & 
1074.88 & 82.49 & \bf{1030.56} & 
1.58 & 4.30\\CMT5Y & 1060.45 & 67.34 & 
1074.24 & 66.02 & \bf{1031.69} & 
2.79 & 4.12\\CMT11X & 862.75 & 43.96 & 
918.41 & 30.32 & \bf{831.09} & 
3.81 & 10.51\\CMT11Y & 875.67 & 60.81 & 
902.01 & 39.19 & \bf{829.85} & 
5.52 & 8.70\\CMT12X & 669.94 & 17.29 & 
672.12 & 12.12 & \bf{658.83} & 
1.69 & 2.02\\CMT12Y & 673.64 & 7.72 & 
676.27 & 11.15 & \bf{660.47} & 
1.99 & 2.39\\\bf{PROM.} & 
\bf{761.79} & \bf{32.21} & \bf{774.66} & \bf{27.27} & \bf{749.50} & \bf{1.48} & \bf{3.01}\\[1ex]\hline
\end{tabular}
\label{table:nonlin}
\end{table} \clearpage
\begin{table}[ht]
\caption{Resultados de la ejecución de la metaheurística GTS, utilizando instancias de Dethloff con la configuración -mni 4500 -lambda1 0.05 -lambda2 0.05 -tabu 21}
\centering
\small
\begin{tabular}{c c c c c c c c}
\hline\hline
Instancia & Costo mínimo & Tiempo(seg.) & Costo promedio & Tiempo promedio(seg.) & CME & \%G & \%GP \\ [0.5ex]
\hline
SCA3-0 & 640.55 & 2.45 & 
640.55 & 2.55 & \bf{635.62} & 
0.78 & 0.78\\SCA3-1 & \bf{697.84} & 3.36 & 
697.84 & 4.83 & 697.84 & 0.00
 & 0.00\\
SCA3-2 & \bf{659.34} & 3.60 & 
659.34 & 3.10 & 659.34 & 0.00
 & 0.00\\
SCA3-3 & \bf{680.04} & 2.46 & 
680.46 & 3.35 & 680.04 & 0.00
 & 0.06\\SCA3-4 & \bf{690.50} & 2.72 & 
690.50 & 3.27 & 690.50 & 0.00
 & 0.00\\
SCA3-5 & \bf{659.90} & 2.44 & 
663.16 & 3.33 & 659.90 & 0.00
 & 0.49\\SCA3-6 & \bf{651.09} & 4.29 & 
651.55 & 3.52 & 651.09 & 0.00
 & 0.07\\SCA3-7 & 666.15 & 4.18 & 
666.15 & 3.79 & \bf{659.17} & 
1.06 & 1.06\\SCA3-8 & \bf{719.47} & 3.19 & 
719.47 & 2.37 & 719.47 & 0.00
 & 0.00\\
SCA3-9 & \bf{681.00} & 2.30 & 
681.00 & 3.11 & 681.00 & 0.00
 & 0.00\\
SCA8-0 & 970.64 & 2.40 & 
975.07 & 3.92 & \bf{961.50} & 
0.95 & 1.41\\SCA8-1 & \bf{1049.65} & 3.79 & 
1055.12 & 3.74 & 1049.65 & 0.00
 & 0.52\\SCA8-2 & 1064.23 & 4.16 & 
1066.36 & 3.16 & \bf{1039.64} & 
2.37 & 2.57\\SCA8-3 & \bf{983.34} & 3.14 & 
1003.98 & 5.17 & 983.34 & 0.00
 & 2.10\\SCA8-4 & \bf{1065.49} & 5.32 & 
1067.87 & 4.86 & 1065.49 & 0.00
 & 0.22\\SCA8-5 & \bf{1027.08} & 3.16 & 
1036.28 & 3.27 & 1027.08 & 0.00
 & 0.90\\SCA8-6 & 972.48 & 3.06 & 
981.16 & 2.86 & \bf{971.82} & 
0.07 & 0.96\\SCA8-7 & 1071.72 & 7.81 & 
1082.59 & 4.24 & \bf{1051.28} & 
1.94 & 2.98\\SCA8-8 & 1082.12 & 3.99 & 
1088.16 & 2.49 & \bf{1071.18} & 
1.02 & 1.59\\SCA8-9 & \bf{1060.50} & 2.57 & 
1062.23 & 2.65 & 1060.50 & 0.00
 & 0.16\\CON3-0 & \bf{616.52} & 1.66 & 
627.37 & 2.69 & 616.52 & 0.00
 & 1.76\\CON3-1 & \bf{554.47} & 2.64 & 
555.05 & 4.67 & 554.47 & 0.00
 & 0.10\\CON3-2 & 519.61 & 3.76 & 
521.82 & 3.83 & \bf{518.00} & 
0.31 & 0.74\\CON3-3 & \bf{591.19} & 7.78 & 
591.19 & 5.80 & 591.19 & 0.00
 & 0.00\\
CON3-4 & \bf{588.79} & 2.34 & 
590.66 & 3.09 & 588.79 & 0.00
 & 0.32\\CON3-5 & \bf{563.70} & 3.21 & 
569.43 & 3.42 & 563.70 & 0.00
 & 1.02\\CON3-6 & \bf{499.05} & 4.50 & 
500.94 & 3.64 & 499.05 & 0.00
 & 0.38\\CON3-7 & 576.84 & 6.52 & 
582.58 & 4.35 & \bf{576.48} & 
0.06 & 1.06\\CON3-8 & \bf{523.05} & 2.22 & 
523.07 & 2.78 & 523.05 & 0.00
 & 0.00\\CON3-9 & 578.25 & 5.75 & 
584.75 & 3.88 & \bf{578.24} & 
0.00 & 1.13\\CON8-0 & 858.03 & 3.97 & 
899.95 & 3.36 & \bf{857.17} & 
0.10 & 4.99\\CON8-1 & 755.11 & 5.10 & 
759.36 & 3.98 & \bf{740.85} & 
1.92 & 2.50\\CON8-2 & 716.03 & 3.49 & 
728.79 & 3.23 & \bf{712.89} & 
0.44 & 2.23\\CON8-3 & 821.26 & 6.74 & 
835.83 & 3.87 & \bf{811.07} & 
1.26 & 3.05\\CON8-4 & \bf{772.25} & 7.37 & 
783.40 & 4.13 & 772.25 & 0.00
 & 1.44\\CON8-5 & 756.91 & 4.34 & 
761.03 & 4.12 & \bf{754.88} & 
0.27 & 0.82\\CON8-6 & 691.20 & 2.26 & 
695.23 & 3.80 & \bf{678.92} & 
1.81 & 2.40\\CON8-7 & \bf{811.96} & 3.32 & 
823.53 & 3.31 & 811.96 & 0.00
 & 1.42\\CON8-8 & \bf{767.53} & 5.79 & 
787.54 & 3.30 & 767.53 & 0.00
 & 2.61\\CON8-9 & \bf{809.00} & 5.08 & 
811.27 & 3.62 & 809.00 & 0.00
 & 0.28\\\bf{PROM.} & 
\bf{761.60} & \bf{3.96} & \bf{767.54} & \bf{3.61} & \bf{758.54} & \bf{0.36} & \bf{1.10}\\[1ex]\hline
\end{tabular}
\label{table:nonlin}
\end{table} \clearpage
\begin{table}[ht]
\caption{Resultados de la ejecución de la metaheurística GTS, utilizando instancias de SalhiNagy con la configuración -mni 4500 -lambda1 0.05 -lambda2 0.05 -tabu 21}
\centering
\small
\begin{tabular}{c c c c c c c c}
\hline\hline
Instancia & Costo mínimo & Tiempo(seg.) & Costo promedio & Tiempo promedio(seg.) & CME & \%G & \%GP \\ [0.5ex]
\hline
CMT1X & \bf{470.48} & 7.36 & 
471.90 & 5.14 & 470.48 & 0.00
 & 0.30\\CMT1Y & \bf{470.48} & 9.27 & 
471.19 & 4.45 & 470.48 & 0.00
 & 0.15\\CMT2X & 683.51 & 8.15 & 
687.71 & 8.04 & \bf{682.39} & 
0.16 & 0.78\\CMT2Y & 685.92 & 9.48 & 
689.32 & 6.79 & \bf{682.39} & 
0.52 & 1.02\\CMT3X & \bf{\underline{718.40}} & 30.61 & 
726.26 & 17.61 & 719.06 & 
\bf{-0.09} & 1.00\\CMT3Y & 726.28 & 9.88 & 
728.32 & 14.34 & \bf{719.06} & 
1.00 & 1.29\\CMT4X & 856.72 & 53.30 & 
869.48 & 39.88 & \bf{854.21} & 
0.29 & 1.79\\CMT4Y & 865.44 & 40.26 & 
872.71 & 37.08 & \bf{852.46} & 
1.52 & 2.38\\CMT5X & 1049.92 & 136.07 & 
1061.92 & 98.72 & \bf{1030.56} & 
1.88 & 3.04\\CMT5Y & 1044.78 & 39.18 & 
1060.40 & 107.94 & \bf{1031.69} & 
1.27 & 2.78\\CMT11X & 900.78 & 47.17 & 
924.06 & 29.43 & \bf{831.09} & 
8.39 & 11.19\\CMT11Y & 875.81 & 55.78 & 
888.33 & 42.24 & \bf{829.85} & 
5.54 & 7.05\\CMT12X & 673.13 & 20.12 & 
681.49 & 13.01 & \bf{658.83} & 
2.17 & 3.44\\CMT12Y & 674.44 & 8.98 & 
676.90 & 13.66 & \bf{660.47} & 
2.12 & 2.49\\\bf{PROM.} & 
\bf{764.01} & \bf{33.97} & \bf{772.14} & \bf{31.31} & \bf{749.50} & \bf{1.77} & \bf{2.76}\\[1ex]\hline
\end{tabular}
\label{table:nonlin}
\end{table} \clearpage
\begin{table}[ht]
\caption{Resultados de la ejecución de la metaheurística GTS, utilizando instancias de Dethloff con la configuración -mni 4500 -lambda1 0.05 -lambda2 0.05 -tabu 25}
\centering
\small
\begin{tabular}{c c c c c c c c}
\hline\hline
Instancia & Costo mínimo & Tiempo(seg.) & Costo promedio & Tiempo promedio(seg.) & CME & \%G & \%GP \\ [0.5ex]
\hline
SCA3-0 & 636.06 & 3.01 & 
638.30 & 3.77 & \bf{635.62} & 
0.07 & 0.42\\SCA3-1 & \bf{697.84} & 4.36 & 
697.84 & 3.88 & 697.84 & 0.00
 & 0.00\\
SCA3-2 & \bf{659.34} & 3.47 & 
659.34 & 3.98 & 659.34 & 0.00
 & 0.00\\
SCA3-3 & 680.60 & 2.16 & 
680.60 & 2.85 & \bf{680.04} & 
0.08 & 0.08\\SCA3-4 & \bf{690.50} & 4.86 & 
694.98 & 3.84 & 690.50 & 0.00
 & 0.65\\SCA3-5 & \bf{659.90} & 3.13 & 
666.42 & 3.12 & 659.90 & 0.00
 & 0.99\\SCA3-6 & \bf{651.09} & 3.92 & 
651.09 & 4.20 & 651.09 & 0.00
 & 0.00\\
SCA3-7 & 666.15 & 2.82 & 
666.15 & 3.56 & \bf{659.17} & 
1.06 & 1.06\\SCA3-8 & \bf{719.47} & 5.20 & 
719.47 & 3.46 & 719.47 & 0.00
 & 0.00\\
SCA3-9 & \bf{681.00} & 3.23 & 
681.00 & 3.67 & 681.00 & 0.00
 & 0.00\\
SCA8-0 & \bf{961.50} & 6.90 & 
987.26 & 4.55 & 961.50 & 0.00
 & 2.68\\SCA8-1 & 1067.45 & 3.83 & 
1071.42 & 3.26 & \bf{1049.65} & 
1.70 & 2.07\\SCA8-2 & \bf{1039.64} & 3.29 & 
1048.61 & 3.07 & 1039.64 & 0.00
 & 0.86\\SCA8-3 & \bf{983.34} & 4.19 & 
1007.37 & 2.56 & 983.34 & 0.00
 & 2.44\\SCA8-4 & 1067.55 & 3.01 & 
1077.24 & 2.47 & \bf{1065.49} & 
0.19 & 1.10\\SCA8-5 & \bf{1027.08} & 2.76 & 
1044.86 & 2.95 & 1027.08 & 0.00
 & 1.73\\SCA8-6 & 972.48 & 2.54 & 
980.75 & 4.09 & \bf{971.82} & 
0.07 & 0.92\\SCA8-7 & \bf{1051.28} & 3.62 & 
1061.09 & 3.46 & 1051.28 & 0.00
 & 0.93\\SCA8-8 & \bf{1071.18} & 2.73 & 
1082.55 & 1.86 & 1071.18 & 0.00
 & 1.06\\SCA8-9 & \bf{1060.50} & 1.82 & 
1065.81 & 2.87 & 1060.50 & 0.00
 & 0.50\\CON3-0 & \bf{616.52} & 4.32 & 
616.79 & 5.12 & 616.52 & 0.00
 & 0.04\\CON3-1 & \bf{554.47} & 2.96 & 
555.25 & 3.73 & 554.47 & 0.00
 & 0.14\\CON3-2 & 519.61 & 3.42 & 
522.61 & 4.15 & \bf{518.00} & 
0.31 & 0.89\\CON3-3 & \bf{591.19} & 4.87 & 
591.19 & 3.79 & 591.19 & 0.00
 & 0.00\\CON3-4 & \bf{588.79} & 4.34 & 
594.23 & 4.17 & 588.79 & 0.00
 & 0.92\\CON3-5 & \bf{563.70} & 3.08 & 
565.92 & 3.06 & 563.70 & 0.00
 & 0.39\\CON3-6 & \bf{499.05} & 4.16 & 
500.61 & 3.58 & 499.05 & 0.00
 & 0.31\\CON3-7 & \bf{576.48} & 3.09 & 
577.93 & 3.48 & 576.48 & 0.00
 & 0.25\\CON3-8 & \bf{523.05} & 3.54 & 
523.05 & 3.26 & 523.05 & 0.00
 & 0.00\\
CON3-9 & 580.78 & 4.29 & 
586.62 & 4.60 & \bf{578.24} & 
0.44 & 1.45\\CON8-0 & 858.63 & 8.96 & 
871.45 & 5.06 & \bf{857.17} & 
0.17 & 1.67\\CON8-1 & 751.76 & 2.50 & 
757.30 & 2.54 & \bf{740.85} & 
1.47 & 2.22\\CON8-2 & 717.97 & 1.92 & 
725.97 & 2.55 & \bf{712.89} & 
0.71 & 1.83\\CON8-3 & \bf{811.07} & 4.56 & 
830.56 & 4.33 & 811.07 & 0.00
 & 2.40\\CON8-4 & \bf{772.25} & 5.67 & 
780.75 & 3.51 & 772.25 & 0.00
 & 1.10\\CON8-5 & \bf{754.88} & 3.22 & 
762.23 & 2.73 & 754.88 & 0.00
 & 0.97\\CON8-6 & 690.51 & 6.25 & 
695.78 & 4.57 & \bf{678.92} & 
1.71 & 2.48\\CON8-7 & 814.50 & 3.54 & 
814.89 & 3.29 & \bf{811.96} & 
0.31 & 0.36\\CON8-8 & \bf{767.53} & 2.39 & 
767.53 & 3.11 & 767.53 & 0.00
 & 0.00\\
CON8-9 & 810.30 & 4.23 & 
812.81 & 3.91 & \bf{809.00} & 
0.16 & 0.47\\\bf{PROM.} & 
\bf{760.17} & \bf{3.80} & \bf{765.89} & \bf{3.55} & \bf{758.54} & \bf{0.21} & \bf{0.89}\\[1ex]\hline
\end{tabular}
\label{table:nonlin}
\end{table} \clearpage
\begin{table}[ht]
\caption{Resultados de la ejecución de la metaheurística GTS, utilizando instancias de SalhiNagy con la configuración -mni 4500 -lambda1 0.05 -lambda2 0.05 -tabu 25}
\centering
\small
\begin{tabular}{c c c c c c c c}
\hline\hline
Instancia & Costo mínimo & Tiempo(seg.) & Costo promedio & Tiempo promedio(seg.) & CME & \%G & \%GP \\ [0.5ex]
\hline
CMT1X & \bf{470.48} & 4.00 & 
470.95 & 3.08 & 470.48 & 0.00
 & 0.10\\CMT1Y & \bf{470.48} & 3.96 & 
471.15 & 4.18 & 470.48 & 0.00
 & 0.14\\CMT2X & \bf{682.39} & 5.08 & 
687.63 & 8.08 & 682.39 & 0.00
 & 0.77\\CMT2Y & \bf{682.39} & 5.12 & 
683.43 & 5.87 & 682.39 & 0.00
 & 0.15\\CMT3X & 726.12 & 10.64 & 
731.18 & 10.41 & \bf{719.06} & 
0.98 & 1.69\\CMT3Y & 724.57 & 10.99 & 
728.15 & 13.04 & \bf{719.06} & 
0.77 & 1.26\\CMT4X & 861.12 & 46.51 & 
871.71 & 43.47 & \bf{854.21} & 
0.81 & 2.05\\CMT4Y & 856.52 & 63.59 & 
875.70 & 49.16 & \bf{852.46} & 
0.48 & 2.73\\CMT5X & 1045.54 & 87.22 & 
1062.17 & 68.57 & \bf{1030.56} & 
1.45 & 3.07\\CMT5Y & 1058.66 & 54.63 & 
1065.39 & 65.19 & \bf{1031.69} & 
2.61 & 3.27\\CMT11X & 866.17 & 12.40 & 
909.68 & 23.97 & \bf{831.09} & 
4.22 & 9.46\\CMT11Y & 878.37 & 60.77 & 
908.99 & 39.07 & \bf{829.85} & 
5.85 & 9.54\\CMT12X & 673.26 & 12.41 & 
674.63 & 9.77 & \bf{658.83} & 
2.19 & 2.40\\CMT12Y & 673.67 & 12.79 & 
674.39 & 13.12 & \bf{660.47} & 
2.00 & 2.11\\\bf{PROM.} & 
\bf{762.12} & \bf{27.87} & \bf{772.51} & \bf{25.50} & \bf{749.50} & \bf{1.53} & \bf{2.77}\\[1ex]\hline
\end{tabular}
\label{table:nonlin}
\end{table} \clearpage
\begin{table}[ht]
\caption{Resultados de la ejecución de la metaheurística GTS, utilizando instancias de Dethloff con la configuración -mni 4500 -lambda1 0.05 -lambda2 0.05 -tabu 29}
\centering
\small
\begin{tabular}{c c c c c c c c}
\hline\hline
Instancia & Costo mínimo & Tiempo(seg.) & Costo promedio & Tiempo promedio(seg.) & CME & \%G & \%GP \\ [0.5ex]
\hline
SCA3-0 & 640.55 & 3.34 & 
640.55 & 4.13 & \bf{635.62} & 
0.78 & 0.78\\SCA3-1 & \bf{697.84} & 1.99 & 
697.84 & 3.20 & 697.84 & 0.00
 & 0.00\\
SCA3-2 & \bf{659.34} & 3.00 & 
659.34 & 3.81 & 659.34 & 0.00
 & 0.00\\
SCA3-3 & \bf{680.04} & 4.86 & 
680.46 & 4.74 & 680.04 & 0.00
 & 0.06\\SCA3-4 & \bf{690.50} & 4.93 & 
699.27 & 3.39 & 690.50 & 0.00
 & 1.27\\SCA3-5 & \bf{659.90} & 2.21 & 
659.90 & 3.83 & 659.90 & 0.00
 & 0.00\\
SCA3-6 & \bf{651.09} & 3.66 & 
653.70 & 3.07 & 651.09 & 0.00
 & 0.40\\SCA3-7 & \bf{659.17} & 2.26 & 
666.27 & 3.36 & 659.17 & 0.00
 & 1.08\\SCA3-8 & \bf{719.47} & 3.24 & 
719.47 & 3.61 & 719.47 & 0.00
 & 0.00\\
SCA3-9 & \bf{681.00} & 2.44 & 
681.00 & 2.71 & 681.00 & 0.00
 & 0.00\\
SCA8-0 & 987.87 & 5.60 & 
1000.39 & 7.81 & \bf{961.50} & 
2.74 & 4.04\\SCA8-1 & \bf{1049.65} & 2.88 & 
1064.68 & 3.67 & 1049.65 & 0.00
 & 1.43\\SCA8-2 & 1049.22 & 3.92 & 
1050.08 & 2.79 & \bf{1039.64} & 
0.92 & 1.00\\SCA8-3 & \bf{983.34} & 3.08 & 
988.92 & 4.05 & 983.34 & 0.00
 & 0.57\\SCA8-4 & 1067.28 & 5.26 & 
1072.80 & 2.98 & \bf{1065.49} & 
0.17 & 0.69\\SCA8-5 & \bf{1027.08} & 2.58 & 
1048.41 & 2.57 & 1027.08 & 0.00
 & 2.08\\SCA8-6 & 972.48 & 2.42 & 
977.01 & 4.29 & \bf{971.82} & 
0.07 & 0.53\\SCA8-7 & 1054.73 & 3.98 & 
1068.46 & 4.48 & \bf{1051.28} & 
0.33 & 1.63\\SCA8-8 & \bf{1071.18} & 3.18 & 
1077.80 & 2.25 & 1071.18 & 0.00
 & 0.62\\SCA8-9 & \bf{1060.50} & 4.20 & 
1069.09 & 3.48 & 1060.50 & 0.00
 & 0.81\\CON3-0 & \bf{616.52} & 5.54 & 
620.86 & 4.16 & 616.52 & 0.00
 & 0.70\\CON3-1 & \bf{554.47} & 3.72 & 
557.01 & 3.26 & 554.47 & 0.00
 & 0.46\\CON3-2 & 519.61 & 2.94 & 
522.23 & 4.80 & \bf{518.00} & 
0.31 & 0.82\\CON3-3 & \bf{591.19} & 7.00 & 
594.58 & 5.71 & 591.19 & 0.00
 & 0.57\\CON3-4 & \bf{588.79} & 2.77 & 
594.41 & 3.73 & 588.79 & 0.00
 & 0.96\\CON3-5 & \bf{563.70} & 4.17 & 
565.91 & 4.30 & 563.70 & 0.00
 & 0.39\\CON3-6 & \bf{499.05} & 3.91 & 
501.46 & 3.90 & 499.05 & 0.00
 & 0.48\\CON3-7 & \bf{576.48} & 3.25 & 
576.96 & 4.13 & 576.48 & 0.00
 & 0.08\\CON3-8 & \bf{523.05} & 3.50 & 
523.05 & 3.38 & 523.05 & 0.00
 & 0.00\\
CON3-9 & 578.25 & 10.06 & 
585.35 & 4.09 & \bf{578.24} & 
0.00 & 1.23\\CON8-0 & \bf{857.17} & 3.25 & 
864.08 & 3.69 & 857.17 & 0.00
 & 0.81\\CON8-1 & 740.93 & 9.40 & 
753.09 & 5.29 & \bf{740.85} & 
0.01 & 1.65\\CON8-2 & 713.44 & 2.34 & 
720.66 & 2.42 & \bf{712.89} & 
0.08 & 1.09\\CON8-3 & \bf{811.07} & 3.04 & 
818.62 & 4.35 & 811.07 & 0.00
 & 0.93\\CON8-4 & \bf{772.25} & 3.44 & 
780.92 & 2.39 & 772.25 & 0.00
 & 1.12\\CON8-5 & \bf{754.88} & 2.67 & 
758.20 & 3.21 & 754.88 & 0.00
 & 0.44\\CON8-6 & 679.98 & 5.10 & 
687.53 & 3.98 & \bf{678.92} & 
0.16 & 1.27\\CON8-7 & \bf{811.96} & 6.24 & 
813.59 & 4.13 & 811.96 & 0.00
 & 0.20\\CON8-8 & \bf{767.53} & 6.64 & 
778.21 & 3.32 & 767.53 & 0.00
 & 1.39\\CON8-9 & 812.44 & 2.60 & 
817.07 & 2.87 & \bf{809.00} & 
0.43 & 1.00\\\bf{PROM.} & 
\bf{759.87} & \bf{4.02} & \bf{765.23} & \bf{3.78} & \bf{758.54} & \bf{0.15} & \bf{0.81}\\[1ex]\hline
\end{tabular}
\label{table:nonlin}
\end{table} \clearpage
\begin{table}[ht]
\caption{Resultados de la ejecución de la metaheurística GTS, utilizando instancias de SalhiNagy con la configuración -mni 4500 -lambda1 0.05 -lambda2 0.05 -tabu 29}
\centering
\small
\begin{tabular}{c c c c c c c c}
\hline\hline
Instancia & Costo mínimo & Tiempo(seg.) & Costo promedio & Tiempo promedio(seg.) & CME & \%G & \%GP \\ [0.5ex]
\hline
CMT1X & \bf{470.48} & 2.96 & 
471.47 & 4.28 & 470.48 & 0.00
 & 0.21\\CMT1Y & \bf{470.48} & 1.88 & 
470.53 & 3.13 & 470.48 & 0.00
 & 0.01\\CMT2X & 685.10 & 9.26 & 
691.03 & 5.59 & \bf{682.39} & 
0.40 & 1.27\\CMT2Y & \bf{682.39} & 7.38 & 
685.92 & 8.10 & 682.39 & 0.00
 & 0.52\\CMT3X & 723.19 & 10.34 & 
727.48 & 11.76 & \bf{719.06} & 
0.57 & 1.17\\CMT3Y & 725.93 & 14.11 & 
730.83 & 12.22 & \bf{719.06} & 
0.96 & 1.64\\CMT4X & 863.21 & 35.00 & 
876.18 & 46.53 & \bf{854.21} & 
1.05 & 2.57\\CMT4Y & 871.49 & 31.14 & 
885.58 & 40.67 & \bf{852.46} & 
2.23 & 3.88\\CMT5X & 1043.65 & 93.79 & 
1047.01 & 90.48 & \bf{1030.56} & 
1.27 & 1.60\\CMT5Y & 1044.02 & 87.24 & 
1061.70 & 80.42 & \bf{1031.69} & 
1.20 & 2.91\\CMT11X & 887.87 & 37.30 & 
924.70 & 26.85 & \bf{831.09} & 
6.83 & 11.26\\CMT11Y & 877.55 & 33.35 & 
925.18 & 30.99 & \bf{829.85} & 
5.75 & 11.49\\CMT12X & 671.64 & 6.94 & 
674.25 & 13.55 & \bf{658.83} & 
1.94 & 2.34\\CMT12Y & 662.60 & 31.73 & 
671.82 & 17.34 & \bf{660.47} & 
0.32 & 1.72\\\bf{PROM.} & 
\bf{762.83} & \bf{28.74} & \bf{774.55} & \bf{27.99} & \bf{749.50} & \bf{1.61} & \bf{3.04}\\[1ex]\hline
\end{tabular}
\label{table:nonlin}
\end{table} \clearpage
\begin{table}[ht]
\caption{Resultados de la ejecución de la metaheurística GTS, utilizando instancias de Dethloff con la configuración -mni 4500 -lambda1 0.05 -lambda2 0.05 -tabu 33}
\centering
\small
\begin{tabular}{c c c c c c c c}
\hline\hline
Instancia & Costo mínimo & Tiempo(seg.) & Costo promedio & Tiempo promedio(seg.) & CME & \%G & \%GP \\ [0.5ex]
\hline
SCA3-0 & 636.06 & 7.41 & 
638.30 & 4.82 & \bf{635.62} & 
0.07 & 0.42\\SCA3-1 & \bf{697.84} & 3.14 & 
697.84 & 4.02 & 697.84 & 0.00
 & 0.00\\
SCA3-2 & \bf{659.34} & 3.94 & 
659.37 & 3.86 & 659.34 & 0.00
 & 0.01\\SCA3-3 & \bf{680.04} & 3.16 & 
680.32 & 3.13 & 680.04 & 0.00
 & 0.04\\SCA3-4 & \bf{690.50} & 3.33 & 
690.50 & 6.75 & 690.50 & 0.00
 & 0.00\\
SCA3-5 & \bf{659.90} & 4.93 & 
659.90 & 6.09 & 659.90 & 0.00
 & 0.00\\
SCA3-6 & \bf{651.09} & 5.54 & 
654.10 & 3.69 & 651.09 & 0.00
 & 0.46\\SCA3-7 & 666.15 & 5.92 & 
666.15 & 4.49 & \bf{659.17} & 
1.06 & 1.06\\SCA3-8 & \bf{719.47} & 6.34 & 
719.47 & 7.33 & 719.47 & 0.00
 & 0.00\\
SCA3-9 & \bf{681.00} & 4.49 & 
681.00 & 4.25 & 681.00 & 0.00
 & 0.00\\
SCA8-0 & \bf{961.50} & 6.06 & 
971.55 & 5.16 & 961.50 & 0.00
 & 1.05\\SCA8-1 & \bf{1049.65} & 9.81 & 
1060.09 & 4.88 & 1049.65 & 0.00
 & 0.99\\SCA8-2 & 1039.71 & 4.01 & 
1055.92 & 2.68 & \bf{1039.64} & 
0.01 & 1.57\\SCA8-3 & \bf{983.34} & 4.39 & 
996.26 & 3.22 & 983.34 & 0.00
 & 1.31\\SCA8-4 & 1067.28 & 4.86 & 
1067.70 & 4.76 & \bf{1065.49} & 
0.17 & 0.21\\SCA8-5 & \bf{1027.08} & 3.26 & 
1045.85 & 3.46 & 1027.08 & 0.00
 & 1.83\\SCA8-6 & 972.48 & 4.55 & 
979.37 & 4.29 & \bf{971.82} & 
0.07 & 0.78\\SCA8-7 & 1063.22 & 2.70 & 
1068.83 & 3.30 & \bf{1051.28} & 
1.14 & 1.67\\SCA8-8 & \bf{1071.18} & 2.28 & 
1079.38 & 3.26 & 1071.18 & 0.00
 & 0.77\\SCA8-9 & \bf{1060.50} & 3.72 & 
1065.70 & 3.52 & 1060.50 & 0.00
 & 0.49\\CON3-0 & \bf{616.52} & 3.78 & 
631.02 & 3.35 & 616.52 & 0.00
 & 2.35\\CON3-1 & \bf{554.47} & 2.93 & 
558.22 & 3.10 & 554.47 & 0.00
 & 0.68\\CON3-2 & 521.38 & 2.78 & 
522.83 & 5.23 & \bf{518.00} & 
0.65 & 0.93\\CON3-3 & \bf{591.19} & 7.05 & 
594.58 & 5.35 & 591.19 & 0.00
 & 0.57\\CON3-4 & 591.43 & 2.65 & 
595.08 & 4.28 & \bf{588.79} & 
0.45 & 1.07\\CON3-5 & \bf{563.70} & 8.90 & 
569.42 & 5.87 & 563.70 & 0.00
 & 1.02\\CON3-6 & \bf{499.05} & 4.38 & 
501.10 & 4.35 & 499.05 & 0.00
 & 0.41\\CON3-7 & \bf{576.48} & 4.12 & 
584.20 & 3.86 & 576.48 & 0.00
 & 1.34\\CON3-8 & \bf{523.05} & 4.22 & 
523.05 & 4.43 & 523.05 & 0.00
 & 0.00\\
CON3-9 & 578.25 & 5.33 & 
583.06 & 3.45 & \bf{578.24} & 
0.00 & 0.83\\CON8-0 & \bf{857.17} & 4.93 & 
870.37 & 4.89 & 857.17 & 0.00
 & 1.54\\CON8-1 & 751.76 & 3.65 & 
755.80 & 3.64 & \bf{740.85} & 
1.47 & 2.02\\CON8-2 & 718.64 & 2.49 & 
722.73 & 3.86 & \bf{712.89} & 
0.81 & 1.38\\CON8-3 & \bf{811.07} & 4.00 & 
826.46 & 5.37 & 811.07 & 0.00
 & 1.90\\CON8-4 & \bf{772.25} & 3.36 & 
779.74 & 4.08 & 772.25 & 0.00
 & 0.97\\CON8-5 & 756.91 & 1.80 & 
758.95 & 4.43 & \bf{754.88} & 
0.27 & 0.54\\CON8-6 & \bf{678.92} & 4.50 & 
687.39 & 4.25 & 678.92 & 0.00
 & 1.25\\CON8-7 & 812.89 & 7.19 & 
816.88 & 3.80 & \bf{811.96} & 
0.11 & 0.61\\CON8-8 & \bf{767.53} & 7.00 & 
772.24 & 4.07 & 767.53 & 0.00
 & 0.61\\CON8-9 & \bf{809.00} & 4.29 & 
812.88 & 3.60 & 809.00 & 0.00
 & 0.48\\\bf{PROM.} & 
\bf{759.72} & \bf{4.58} & \bf{765.09} & \bf{4.31} & \bf{758.54} & \bf{0.16} & \bf{0.83}\\[1ex]\hline
\end{tabular}
\label{table:nonlin}
\end{table} \clearpage
\begin{table}[ht]
\caption{Resultados de la ejecución de la metaheurística GTS, utilizando instancias de SalhiNagy con la configuración -mni 4500 -lambda1 0.05 -lambda2 0.05 -tabu 33}
\centering
\small
\begin{tabular}{c c c c c c c c}
\hline\hline
Instancia & Costo mínimo & Tiempo(seg.) & Costo promedio & Tiempo promedio(seg.) & CME & \%G & \%GP \\ [0.5ex]
\hline
CMT1X & \bf{470.48} & 3.02 & 
471.43 & 3.75 & 470.48 & 0.00
 & 0.20\\CMT1Y & \bf{470.48} & 3.72 & 
471.14 & 4.61 & 470.48 & 0.00
 & 0.14\\CMT2X & 683.52 & 10.60 & 
688.37 & 9.19 & \bf{682.39} & 
0.17 & 0.88\\CMT2Y & 685.12 & 8.25 & 
686.75 & 8.21 & \bf{682.39} & 
0.40 & 0.64\\CMT3X & \bf{719.06} & 29.62 & 
726.64 & 17.77 & 719.06 & 0.00
 & 1.05\\CMT3Y & 719.93 & 17.25 & 
724.41 & 16.45 & \bf{719.06} & 
0.12 & 0.74\\CMT4X & 861.58 & 33.62 & 
864.17 & 37.65 & \bf{854.21} & 
0.86 & 1.17\\CMT4Y & 869.56 & 21.92 & 
877.42 & 38.56 & \bf{852.46} & 
2.01 & 2.93\\CMT5X & 1060.15 & 66.09 & 
1071.01 & 69.36 & \bf{1030.56} & 
2.87 & 3.93\\CMT5Y & 1039.31 & 80.76 & 
1051.07 & 77.98 & \bf{1031.69} & 
0.74 & 1.88\\CMT11X & 866.03 & 62.05 & 
882.49 & 51.05 & \bf{831.09} & 
4.20 & 6.18\\CMT11Y & 878.89 & 43.24 & 
904.29 & 36.77 & \bf{829.85} & 
5.91 & 8.97\\CMT12X & 671.54 & 27.90 & 
675.00 & 17.35 & \bf{658.83} & 
1.93 & 2.46\\CMT12Y & 673.92 & 17.85 & 
676.90 & 24.74 & \bf{660.47} & 
2.04 & 2.49\\\bf{PROM.} & 
\bf{762.11} & \bf{30.42} & \bf{769.36} & \bf{29.53} & \bf{749.50} & \bf{1.52} & \bf{2.40}\\[1ex]\hline
\end{tabular}
\label{table:nonlin}
\end{table} \clearpage
\begin{table}[ht]
\caption{Resultados de la ejecución de la metaheurística GTS, utilizando instancias de Dethloff con la configuración -mni 4500 -lambda1 0.05 -lambda2 0.05 -tabu 37}
\centering
\small
\begin{tabular}{c c c c c c c c}
\hline\hline
Instancia & Costo mínimo & Tiempo(seg.) & Costo promedio & Tiempo promedio(seg.) & CME & \%G & \%GP \\ [0.5ex]
\hline
SCA3-0 & 636.06 & 1.98 & 
639.43 & 3.45 & \bf{635.62} & 
0.07 & 0.60\\SCA3-1 & \bf{697.84} & 3.83 & 
697.84 & 4.51 & 697.84 & 0.00
 & 0.00\\
SCA3-2 & \bf{659.34} & 4.43 & 
659.34 & 4.09 & 659.34 & 0.00
 & 0.00\\
SCA3-3 & \bf{680.04} & 4.85 & 
683.03 & 3.56 & 680.04 & 0.00
 & 0.44\\SCA3-4 & \bf{690.50} & 8.01 & 
690.50 & 6.60 & 690.50 & 0.00
 & 0.00\\
SCA3-5 & \bf{659.90} & 2.73 & 
659.90 & 3.19 & 659.90 & 0.00
 & 0.00\\
SCA3-6 & \bf{651.09} & 5.95 & 
654.16 & 3.32 & 651.09 & 0.00
 & 0.47\\SCA3-7 & 666.15 & 2.97 & 
667.31 & 3.12 & \bf{659.17} & 
1.06 & 1.23\\SCA3-8 & \bf{719.47} & 2.24 & 
719.47 & 3.31 & 719.47 & 0.00
 & 0.00\\
SCA3-9 & \bf{681.00} & 2.51 & 
681.00 & 3.65 & 681.00 & 0.00
 & 0.00\\
SCA8-0 & \bf{961.50} & 3.26 & 
985.49 & 5.79 & 961.50 & 0.00
 & 2.49\\SCA8-1 & \bf{1049.65} & 8.46 & 
1054.42 & 5.58 & 1049.65 & 0.00
 & 0.45\\SCA8-2 & \bf{1039.64} & 11.70 & 
1045.19 & 5.67 & 1039.64 & 0.00
 & 0.53\\SCA8-3 & \bf{983.34} & 10.52 & 
1004.57 & 5.27 & 983.34 & 0.00
 & 2.16\\SCA8-4 & 1067.28 & 3.67 & 
1074.44 & 2.94 & \bf{1065.49} & 
0.17 & 0.84\\SCA8-5 & \bf{1027.08} & 4.57 & 
1033.19 & 3.69 & 1027.08 & 0.00
 & 0.59\\SCA8-6 & \bf{971.82} & 2.24 & 
980.64 & 3.17 & 971.82 & 0.00
 & 0.91\\SCA8-7 & 1052.04 & 4.46 & 
1068.65 & 3.57 & \bf{1051.28} & 
0.07 & 1.65\\SCA8-8 & 1080.58 & 2.67 & 
1082.11 & 2.54 & \bf{1071.18} & 
0.88 & 1.02\\SCA8-9 & \bf{1060.50} & 2.72 & 
1064.72 & 3.29 & 1060.50 & 0.00
 & 0.40\\CON3-0 & \bf{616.52} & 2.05 & 
623.54 & 2.60 & 616.52 & 0.00
 & 1.14\\CON3-1 & \bf{554.47} & 5.64 & 
556.23 & 4.87 & 554.47 & 0.00
 & 0.32\\CON3-2 & 519.06 & 3.16 & 
522.19 & 5.74 & \bf{518.00} & 
0.20 & 0.81\\CON3-3 & \bf{591.19} & 2.09 & 
607.35 & 3.43 & 591.19 & 0.00
 & 2.73\\CON3-4 & \bf{588.79} & 4.22 & 
596.33 & 3.45 & 588.79 & 0.00
 & 1.28\\CON3-5 & \bf{563.70} & 4.03 & 
572.22 & 4.69 & 563.70 & 0.00
 & 1.51\\CON3-6 & \bf{499.05} & 6.22 & 
501.09 & 4.88 & 499.05 & 0.00
 & 0.41\\CON3-7 & \bf{576.48} & 6.62 & 
581.66 & 4.05 & 576.48 & 0.00
 & 0.90\\CON3-8 & \bf{523.05} & 2.87 & 
523.24 & 2.62 & 523.05 & 0.00
 & 0.04\\CON3-9 & 578.25 & 3.88 & 
584.36 & 3.65 & \bf{578.24} & 
0.00 & 1.06\\CON8-0 & 858.03 & 9.27 & 
871.04 & 6.62 & \bf{857.17} & 
0.10 & 1.62\\CON8-1 & 742.47 & 4.81 & 
753.09 & 3.55 & \bf{740.85} & 
0.22 & 1.65\\CON8-2 & 713.44 & 3.31 & 
717.02 & 5.58 & \bf{712.89} & 
0.08 & 0.58\\CON8-3 & 811.58 & 4.68 & 
818.11 & 4.50 & \bf{811.07} & 
0.06 & 0.87\\CON8-4 & \bf{772.25} & 2.24 & 
772.25 & 3.02 & 772.25 & 0.00
 & 0.00\\
CON8-5 & 758.12 & 3.63 & 
760.00 & 3.28 & \bf{754.88} & 
0.43 & 0.68\\CON8-6 & \bf{678.92} & 4.40 & 
691.81 & 2.98 & 678.92 & 0.00
 & 1.90\\CON8-7 & 812.89 & 3.38 & 
813.90 & 3.41 & \bf{811.96} & 
0.11 & 0.24\\CON8-8 & 767.61 & 3.69 & 
781.05 & 3.20 & \bf{767.53} & 
0.01 & 1.76\\CON8-9 & 811.16 & 3.12 & 
820.54 & 3.66 & \bf{809.00} & 
0.27 & 1.43\\\bf{PROM.} & 
\bf{759.30} & \bf{4.43} & \bf{765.31} & \bf{4.00} & \bf{758.54} & \bf{0.09} & \bf{0.87}\\[1ex]\hline
\end{tabular}
\label{table:nonlin}
\end{table} \clearpage
\begin{table}[ht]
\caption{Resultados de la ejecución de la metaheurística GTS, utilizando instancias de SalhiNagy con la configuración -mni 4500 -lambda1 0.05 -lambda2 0.05 -tabu 37}
\centering
\small
\begin{tabular}{c c c c c c c c}
\hline\hline
Instancia & Costo mínimo & Tiempo(seg.) & Costo promedio & Tiempo promedio(seg.) & CME & \%G & \%GP \\ [0.5ex]
\hline
CMT1X & \bf{470.48} & 7.37 & 
470.72 & 4.31 & 470.48 & 0.00
 & 0.05\\CMT1Y & 470.67 & 3.19 & 
474.91 & 4.19 & \bf{470.48} & 
0.04 & 0.94\\CMT2X & 683.64 & 6.47 & 
685.59 & 7.60 & \bf{682.39} & 
0.18 & 0.47\\CMT2Y & 686.70 & 4.62 & 
689.31 & 5.28 & \bf{682.39} & 
0.63 & 1.01\\CMT3X & 719.84 & 8.62 & 
723.61 & 12.46 & \bf{719.06} & 
0.11 & 0.63\\CMT3Y & 723.58 & 23.44 & 
726.15 & 15.69 & \bf{719.06} & 
0.63 & 0.99\\CMT4X & 860.59 & 26.08 & 
875.02 & 30.61 & \bf{854.21} & 
0.75 & 2.44\\CMT4Y & 857.40 & 52.75 & 
873.40 & 38.12 & \bf{852.46} & 
0.58 & 2.46\\CMT5X & 1056.10 & 103.18 & 
1069.88 & 68.71 & \bf{1030.56} & 
2.48 & 3.82\\CMT5Y & 1044.71 & 60.37 & 
1054.17 & 87.89 & \bf{1031.69} & 
1.26 & 2.18\\CMT11X & 875.60 & 61.64 & 
887.62 & 45.88 & \bf{831.09} & 
5.36 & 6.80\\CMT11Y & 877.44 & 33.92 & 
918.83 & 22.59 & \bf{829.85} & 
5.73 & 10.72\\CMT12X & 674.12 & 19.99 & 
683.27 & 16.34 & \bf{658.83} & 
2.32 & 3.71\\CMT12Y & 674.31 & 15.36 & 
680.65 & 13.35 & \bf{660.47} & 
2.10 & 3.06\\\bf{PROM.} & 
\bf{762.51} & \bf{30.50} & \bf{772.37} & \bf{26.64} & \bf{749.50} & \bf{1.58} & \bf{2.81}\\[1ex]\hline
\end{tabular}
\label{table:nonlin}
\end{table} \clearpage
\begin{table}[ht]
\caption{Resultados de la ejecución de la metaheurística GTS, utilizando instancias de Dethloff con la configuración -mni 5000 -lambda1 0.05 -lambda2 0.05 -tabu 17}
\centering
\small
\begin{tabular}{c c c c c c c c}
\hline\hline
Instancia & Costo mínimo & Tiempo(seg.) & Costo promedio & Tiempo promedio(seg.) & CME & \%G & \%GP \\ [0.5ex]
\hline
SCA3-0 & \bf{635.62} & 6.06 & 
638.76 & 4.92 & 635.62 & 0.00
 & 0.49\\SCA3-1 & \bf{697.84} & 2.48 & 
698.50 & 2.53 & 697.84 & 0.00
 & 0.10\\SCA3-2 & \bf{659.34} & 4.07 & 
659.34 & 3.60 & 659.34 & 0.00
 & 0.00\\
SCA3-3 & \bf{680.04} & 6.00 & 
680.04 & 3.62 & 680.04 & 0.00
 & 0.00\\
SCA3-4 & \bf{690.50} & 3.30 & 
690.50 & 4.07 & 690.50 & 0.00
 & 0.00\\
SCA3-5 & \bf{659.90} & 3.08 & 
663.16 & 3.23 & 659.90 & 0.00
 & 0.49\\SCA3-6 & \bf{651.09} & 7.50 & 
652.34 & 4.39 & 651.09 & 0.00
 & 0.19\\SCA3-7 & 666.15 & 2.18 & 
667.09 & 2.52 & \bf{659.17} & 
1.06 & 1.20\\SCA3-8 & \bf{719.47} & 2.75 & 
719.47 & 3.62 & 719.47 & 0.00
 & 0.00\\
SCA3-9 & \bf{681.00} & 2.68 & 
681.00 & 3.39 & 681.00 & 0.00
 & 0.00\\
SCA8-0 & 979.79 & 5.38 & 
983.57 & 4.39 & \bf{961.50} & 
1.90 & 2.29\\SCA8-1 & \bf{1049.65} & 6.08 & 
1059.30 & 3.80 & 1049.65 & 0.00
 & 0.92\\SCA8-2 & 1042.10 & 2.04 & 
1048.61 & 2.19 & \bf{1039.64} & 
0.24 & 0.86\\SCA8-3 & \bf{983.34} & 6.30 & 
1006.00 & 3.85 & 983.34 & 0.00
 & 2.30\\SCA8-4 & \bf{1065.49} & 5.80 & 
1066.90 & 5.39 & 1065.49 & 0.00
 & 0.13\\SCA8-5 & 1042.30 & 3.27 & 
1054.00 & 2.78 & \bf{1027.08} & 
1.48 & 2.62\\SCA8-6 & 972.48 & 2.03 & 
972.48 & 2.49 & \bf{971.82} & 
0.07 & 0.07\\SCA8-7 & 1052.17 & 4.54 & 
1063.90 & 3.63 & \bf{1051.28} & 
0.08 & 1.20\\SCA8-8 & \bf{1071.18} & 3.78 & 
1080.26 & 3.91 & 1071.18 & 0.00
 & 0.85\\SCA8-9 & \bf{1060.50} & 5.92 & 
1068.61 & 4.94 & 1060.50 & 0.00
 & 0.76\\CON3-0 & \bf{616.52} & 4.64 & 
622.21 & 3.90 & 616.52 & 0.00
 & 0.92\\CON3-1 & \bf{554.47} & 8.22 & 
557.96 & 3.81 & 554.47 & 0.00
 & 0.63\\CON3-2 & 519.61 & 5.16 & 
521.66 & 6.03 & \bf{518.00} & 
0.31 & 0.71\\CON3-3 & \bf{591.19} & 4.24 & 
594.58 & 3.53 & 591.19 & 0.00
 & 0.57\\CON3-4 & 596.29 & 2.67 & 
601.18 & 2.49 & \bf{588.79} & 
1.27 & 2.10\\CON3-5 & \bf{563.70} & 3.24 & 
567.42 & 3.65 & 563.70 & 0.00
 & 0.66\\CON3-6 & 500.37 & 3.28 & 
501.71 & 3.89 & \bf{499.05} & 
0.26 & 0.53\\CON3-7 & \bf{576.48} & 3.52 & 
580.84 & 3.29 & 576.48 & 0.00
 & 0.76\\CON3-8 & \bf{523.05} & 1.82 & 
523.05 & 2.75 & 523.05 & 0.00
 & 0.00\\
CON3-9 & 578.25 & 3.96 & 
581.92 & 4.00 & \bf{578.24} & 
0.00 & 0.64\\CON8-0 & 858.63 & 7.16 & 
900.72 & 6.12 & \bf{857.17} & 
0.17 & 5.08\\CON8-1 & \bf{740.85} & 6.73 & 
746.97 & 4.78 & 740.85 & 0.00
 & 0.83\\CON8-2 & 713.44 & 3.35 & 
726.01 & 4.61 & \bf{712.89} & 
0.08 & 1.84\\CON8-3 & 824.69 & 3.52 & 
843.09 & 3.82 & \bf{811.07} & 
1.68 & 3.95\\CON8-4 & \bf{772.25} & 5.89 & 
775.39 & 3.21 & 772.25 & 0.00
 & 0.41\\CON8-5 & \bf{754.88} & 4.00 & 
758.43 & 4.75 & 754.88 & 0.00
 & 0.47\\CON8-6 & 695.24 & 2.61 & 
697.44 & 3.45 & \bf{678.92} & 
2.40 & 2.73\\CON8-7 & 813.20 & 6.29 & 
817.54 & 4.71 & \bf{811.96} & 
0.15 & 0.69\\CON8-8 & \bf{767.53} & 2.72 & 
773.49 & 2.96 & 767.53 & 0.00
 & 0.78\\CON8-9 & 810.18 & 3.26 & 
812.88 & 3.07 & \bf{809.00} & 
0.15 & 0.48\\\bf{PROM.} & 
\bf{760.77} & \bf{4.29} & \bf{766.46} & \bf{3.80} & \bf{758.54} & \bf{0.28} & \bf{0.98}\\[1ex]\hline
\end{tabular}
\label{table:nonlin}
\end{table} \clearpage
\begin{table}[ht]
\caption{Resultados de la ejecución de la metaheurística GTS, utilizando instancias de SalhiNagy con la configuración -mni 5000 -lambda1 0.05 -lambda2 0.05 -tabu 17}
\centering
\small
\begin{tabular}{c c c c c c c c}
\hline\hline
Instancia & Costo mínimo & Tiempo(seg.) & Costo promedio & Tiempo promedio(seg.) & CME & \%G & \%GP \\ [0.5ex]
\hline
CMT1X & \bf{470.48} & 4.68 & 
471.90 & 3.94 & 470.48 & 0.00
 & 0.30\\CMT1Y & \bf{470.48} & 3.05 & 
471.90 & 4.27 & 470.48 & 0.00
 & 0.30\\CMT2X & 684.64 & 9.35 & 
687.25 & 8.01 & \bf{682.39} & 
0.33 & 0.71\\CMT2Y & \bf{682.39} & 11.18 & 
686.10 & 8.12 & 682.39 & 0.00
 & 0.54\\CMT3X & 725.72 & 11.59 & 
726.82 & 13.47 & \bf{719.06} & 
0.93 & 1.08\\CMT3Y & 720.18 & 14.37 & 
727.81 & 10.12 & \bf{719.06} & 
0.16 & 1.22\\CMT4X & 877.56 & 41.77 & 
880.40 & 34.66 & \bf{854.21} & 
2.73 & 3.07\\CMT4Y & 860.95 & 37.43 & 
875.02 & 32.80 & \bf{852.46} & 
1.00 & 2.65\\CMT5X & 1049.86 & 82.45 & 
1073.03 & 84.34 & \bf{1030.56} & 
1.87 & 4.12\\CMT5Y & 1037.34 & 140.26 & 
1061.82 & 82.97 & \bf{1031.69} & 
0.55 & 2.92\\CMT11X & 876.36 & 59.60 & 
879.91 & 52.58 & \bf{831.09} & 
5.45 & 5.87\\CMT11Y & 876.79 & 17.30 & 
934.97 & 25.20 & \bf{829.85} & 
5.66 & 12.67\\CMT12X & 671.33 & 13.50 & 
676.85 & 10.84 & \bf{658.83} & 
1.90 & 2.74\\CMT12Y & 672.04 & 9.40 & 
675.56 & 11.87 & \bf{660.47} & 
1.75 & 2.28\\\bf{PROM.} & 
\bf{762.58} & \bf{32.57} & \bf{773.52} & \bf{27.37} & \bf{749.50} & \bf{1.59} & \bf{2.89}\\[1ex]\hline
\end{tabular}
\label{table:nonlin}
\end{table} \clearpage
\begin{table}[ht]
\caption{Resultados de la ejecución de la metaheurística GTS, utilizando instancias de Dethloff con la configuración -mni 5000 -lambda1 0.05 -lambda2 0.05 -tabu 21}
\centering
\small
\begin{tabular}{c c c c c c c c}
\hline\hline
Instancia & Costo mínimo & Tiempo(seg.) & Costo promedio & Tiempo promedio(seg.) & CME & \%G & \%GP \\ [0.5ex]
\hline
SCA3-0 & 636.06 & 2.05 & 
637.18 & 3.23 & \bf{635.62} & 
0.07 & 0.25\\SCA3-1 & \bf{697.84} & 3.05 & 
697.84 & 3.47 & 697.84 & 0.00
 & 0.00\\
SCA3-2 & \bf{659.34} & 3.27 & 
659.34 & 2.94 & 659.34 & 0.00
 & 0.00\\
SCA3-3 & 680.60 & 7.06 & 
686.16 & 3.48 & \bf{680.04} & 
0.08 & 0.90\\SCA3-4 & \bf{690.50} & 3.41 & 
690.50 & 5.06 & 690.50 & 0.00
 & 0.00\\
SCA3-5 & \bf{659.90} & 4.82 & 
659.90 & 4.80 & 659.90 & 0.00
 & 0.00\\
SCA3-6 & \bf{651.09} & 1.80 & 
656.83 & 3.03 & 651.09 & 0.00
 & 0.88\\SCA3-7 & 666.15 & 5.31 & 
666.15 & 4.95 & \bf{659.17} & 
1.06 & 1.06\\SCA3-8 & \bf{719.47} & 4.07 & 
719.97 & 3.70 & 719.47 & 0.00
 & 0.07\\SCA3-9 & \bf{681.00} & 5.02 & 
681.00 & 3.60 & 681.00 & 0.00
 & 0.00\\
SCA8-0 & 970.64 & 3.12 & 
980.20 & 2.77 & \bf{961.50} & 
0.95 & 1.95\\SCA8-1 & \bf{1049.65} & 6.82 & 
1064.72 & 4.04 & 1049.65 & 0.00
 & 1.44\\SCA8-2 & \bf{1039.64} & 3.92 & 
1048.47 & 3.67 & 1039.64 & 0.00
 & 0.85\\SCA8-3 & \bf{983.34} & 6.40 & 
1006.63 & 8.69 & 983.34 & 0.00
 & 2.37\\SCA8-4 & 1067.55 & 5.30 & 
1074.21 & 3.80 & \bf{1065.49} & 
0.19 & 0.82\\SCA8-5 & 1040.66 & 3.33 & 
1045.53 & 3.98 & \bf{1027.08} & 
1.32 & 1.80\\SCA8-6 & 972.48 & 3.71 & 
980.75 & 3.52 & \bf{971.82} & 
0.07 & 0.92\\SCA8-7 & 1063.22 & 1.82 & 
1072.96 & 2.55 & \bf{1051.28} & 
1.14 & 2.06\\SCA8-8 & \bf{1071.18} & 2.26 & 
1073.92 & 3.17 & 1071.18 & 0.00
 & 0.26\\SCA8-9 & \bf{1060.50} & 3.42 & 
1063.82 & 4.08 & 1060.50 & 0.00
 & 0.31\\CON3-0 & 628.47 & 3.26 & 
632.10 & 3.10 & \bf{616.52} & 
1.94 & 2.53\\CON3-1 & \bf{554.47} & 3.82 & 
556.04 & 4.77 & 554.47 & 0.00
 & 0.28\\CON3-2 & 522.86 & 3.44 & 
523.14 & 4.99 & \bf{518.00} & 
0.94 & 0.99\\CON3-3 & \bf{591.19} & 5.12 & 
591.19 & 4.79 & 591.19 & 0.00
 & 0.00\\
CON3-4 & 589.26 & 2.20 & 
594.53 & 3.46 & \bf{588.79} & 
0.08 & 0.98\\CON3-5 & \bf{563.70} & 5.26 & 
565.91 & 4.42 & 563.70 & 0.00
 & 0.39\\CON3-6 & \bf{499.05} & 3.68 & 
499.05 & 4.86 & 499.05 & 0.00
 & 0.00\\
CON3-7 & \bf{576.48} & 2.87 & 
576.96 & 2.96 & 576.48 & 0.00
 & 0.08\\CON3-8 & \bf{523.05} & 6.66 & 
523.24 & 3.29 & 523.05 & 0.00
 & 0.04\\CON3-9 & 582.79 & 9.90 & 
585.44 & 4.78 & \bf{578.24} & 
0.79 & 1.25\\CON8-0 & 857.40 & 2.91 & 
876.99 & 3.90 & \bf{857.17} & 
0.03 & 2.31\\CON8-1 & \bf{740.85} & 4.98 & 
752.45 & 3.60 & 740.85 & 0.00
 & 1.57\\CON8-2 & 717.67 & 5.68 & 
732.75 & 4.19 & \bf{712.89} & 
0.67 & 2.79\\CON8-3 & \bf{811.07} & 5.44 & 
826.44 & 4.45 & 811.07 & 0.00
 & 1.90\\CON8-4 & \bf{772.25} & 5.56 & 
779.16 & 3.19 & 772.25 & 0.00
 & 0.89\\CON8-5 & \bf{754.88} & 5.00 & 
758.03 & 4.64 & 754.88 & 0.00
 & 0.42\\CON8-6 & \bf{678.92} & 7.02 & 
683.95 & 5.12 & 678.92 & 0.00
 & 0.74\\CON8-7 & 813.80 & 3.17 & 
826.74 & 3.39 & \bf{811.96} & 
0.23 & 1.82\\CON8-8 & \bf{767.53} & 3.63 & 
775.70 & 3.88 & 767.53 & 0.00
 & 1.06\\CON8-9 & \bf{809.00} & 6.36 & 
811.16 & 5.61 & 809.00 & 0.00
 & 0.27\\\bf{PROM.} & 
\bf{760.39} & \bf{4.40} & \bf{765.93} & \bf{4.05} & \bf{758.54} & \bf{0.24} & \bf{0.91}\\[1ex]\hline
\end{tabular}
\label{table:nonlin}
\end{table} \clearpage
\begin{table}[ht]
\caption{Resultados de la ejecución de la metaheurística GTS, utilizando instancias de SalhiNagy con la configuración -mni 5000 -lambda1 0.05 -lambda2 0.05 -tabu 21}
\centering
\small
\begin{tabular}{c c c c c c c c}
\hline\hline
Instancia & Costo mínimo & Tiempo(seg.) & Costo promedio & Tiempo promedio(seg.) & CME & \%G & \%GP \\ [0.5ex]
\hline
CMT1X & \bf{470.48} & 2.78 & 
472.99 & 3.05 & 470.48 & 0.00
 & 0.53\\CMT1Y & \bf{470.48} & 3.38 & 
471.00 & 4.10 & 470.48 & 0.00
 & 0.11\\CMT2X & 684.29 & 9.76 & 
687.93 & 6.70 & \bf{682.39} & 
0.28 & 0.81\\CMT2Y & 686.82 & 9.26 & 
689.54 & 8.01 & \bf{682.39} & 
0.65 & 1.05\\CMT3X & 726.28 & 14.25 & 
728.79 & 13.93 & \bf{719.06} & 
1.00 & 1.35\\CMT3Y & 720.28 & 14.64 & 
727.91 & 10.70 & \bf{719.06} & 
0.17 & 1.23\\CMT4X & 866.54 & 43.37 & 
891.12 & 34.46 & \bf{854.21} & 
1.44 & 4.32\\CMT4Y & 876.45 & 60.04 & 
883.50 & 48.25 & \bf{852.46} & 
2.81 & 3.64\\CMT5X & 1044.33 & 116.22 & 
1076.41 & 111.60 & \bf{1030.56} & 
1.34 & 4.45\\CMT5Y & 1042.68 & 126.53 & 
1064.25 & 92.49 & \bf{1031.69} & 
1.07 & 3.16\\CMT11X & 879.56 & 35.10 & 
928.48 & 37.97 & \bf{831.09} & 
5.83 & 11.72\\CMT11Y & 879.87 & 31.26 & 
901.46 & 37.81 & \bf{829.85} & 
6.03 & 8.63\\CMT12X & 669.79 & 12.44 & 
676.65 & 13.34 & \bf{658.83} & 
1.66 & 2.71\\CMT12Y & 671.46 & 20.84 & 
676.90 & 15.10 & \bf{660.47} & 
1.66 & 2.49\\\bf{PROM.} & 
\bf{763.52} & \bf{35.70} & \bf{776.92} & \bf{31.25} & \bf{749.50} & \bf{1.71} & \bf{3.30}\\[1ex]\hline
\end{tabular}
\label{table:nonlin}
\end{table} \clearpage
\begin{table}[ht]
\caption{Resultados de la ejecución de la metaheurística GTS, utilizando instancias de Dethloff con la configuración -mni 5000 -lambda1 0.05 -lambda2 0.05 -tabu 25}
\centering
\small
\begin{tabular}{c c c c c c c c}
\hline\hline
Instancia & Costo mínimo & Tiempo(seg.) & Costo promedio & Tiempo promedio(seg.) & CME & \%G & \%GP \\ [0.5ex]
\hline
SCA3-0 & 636.06 & 3.16 & 
637.18 & 3.69 & \bf{635.62} & 
0.07 & 0.25\\SCA3-1 & \bf{697.84} & 5.74 & 
698.50 & 4.66 & 697.84 & 0.00
 & 0.10\\SCA3-2 & \bf{659.34} & 5.05 & 
659.34 & 5.19 & 659.34 & 0.00
 & 0.00\\
SCA3-3 & \bf{680.04} & 5.65 & 
682.80 & 4.36 & 680.04 & 0.00
 & 0.41\\SCA3-4 & \bf{690.50} & 6.20 & 
690.50 & 6.31 & 690.50 & 0.00
 & 0.00\\
SCA3-5 & \bf{659.90} & 2.26 & 
659.90 & 4.39 & 659.90 & 0.00
 & 0.00\\
SCA3-6 & \bf{651.09} & 2.92 & 
654.16 & 3.38 & 651.09 & 0.00
 & 0.47\\SCA3-7 & \bf{659.17} & 3.19 & 
665.34 & 4.46 & 659.17 & 0.00
 & 0.94\\SCA3-8 & \bf{719.47} & 4.47 & 
719.47 & 3.33 & 719.47 & 0.00
 & 0.00\\
SCA3-9 & \bf{681.00} & 3.97 & 
681.00 & 2.99 & 681.00 & 0.00
 & 0.00\\
SCA8-0 & 979.79 & 6.00 & 
983.25 & 3.61 & \bf{961.50} & 
1.90 & 2.26\\SCA8-1 & \bf{1049.65} & 3.78 & 
1058.95 & 4.35 & 1049.65 & 0.00
 & 0.89\\SCA8-2 & \bf{1039.64} & 2.97 & 
1052.12 & 3.41 & 1039.64 & 0.00
 & 1.20\\SCA8-3 & \bf{983.34} & 2.68 & 
1007.05 & 3.71 & 983.34 & 0.00
 & 2.41\\SCA8-4 & \bf{1065.49} & 4.22 & 
1069.09 & 3.32 & 1065.49 & 0.00
 & 0.34\\SCA8-5 & 1042.30 & 2.40 & 
1055.42 & 3.54 & \bf{1027.08} & 
1.48 & 2.76\\SCA8-6 & 972.48 & 8.58 & 
985.70 & 5.01 & \bf{971.82} & 
0.07 & 1.43\\SCA8-7 & 1063.22 & 4.53 & 
1066.40 & 3.45 & \bf{1051.28} & 
1.14 & 1.44\\SCA8-8 & 1082.12 & 1.51 & 
1082.12 & 2.94 & \bf{1071.18} & 
1.02 & 1.02\\SCA8-9 & \bf{1060.50} & 3.27 & 
1064.32 & 2.88 & 1060.50 & 0.00
 & 0.36\\CON3-0 & \bf{616.52} & 5.20 & 
619.18 & 3.98 & 616.52 & 0.00
 & 0.43\\CON3-1 & \bf{554.47} & 5.25 & 
557.38 & 4.69 & 554.47 & 0.00
 & 0.53\\CON3-2 & 519.11 & 5.35 & 
521.29 & 5.49 & \bf{518.00} & 
0.21 & 0.64\\CON3-3 & \bf{591.19} & 4.54 & 
591.19 & 5.80 & 591.19 & 0.00
 & 0.00\\
CON3-4 & \bf{588.79} & 2.94 & 
593.25 & 3.63 & 588.79 & 0.00
 & 0.76\\CON3-5 & \bf{563.70} & 7.35 & 
576.27 & 4.62 & 563.70 & 0.00
 & 2.23\\CON3-6 & \bf{499.05} & 5.70 & 
499.83 & 4.71 & 499.05 & 0.00
 & 0.16\\CON3-7 & \bf{576.48} & 3.11 & 
580.34 & 4.88 & 576.48 & 0.00
 & 0.67\\CON3-8 & \bf{523.05} & 2.07 & 
523.05 & 2.58 & 523.05 & 0.00
 & 0.00\\
CON3-9 & 578.25 & 4.12 & 
585.03 & 3.85 & \bf{578.24} & 
0.00 & 1.18\\CON8-0 & \bf{857.17} & 10.60 & 
866.98 & 6.32 & 857.17 & 0.00
 & 1.14\\CON8-1 & 751.76 & 6.09 & 
758.49 & 5.04 & \bf{740.85} & 
1.47 & 2.38\\CON8-2 & 713.05 & 5.01 & 
723.00 & 3.73 & \bf{712.89} & 
0.02 & 1.42\\CON8-3 & 821.26 & 4.39 & 
823.33 & 4.61 & \bf{811.07} & 
1.26 & 1.51\\CON8-4 & \bf{772.25} & 5.85 & 
786.73 & 4.14 & 772.25 & 0.00
 & 1.87\\CON8-5 & \bf{754.88} & 4.10 & 
757.31 & 5.34 & 754.88 & 0.00
 & 0.32\\CON8-6 & 690.63 & 2.50 & 
694.22 & 2.93 & \bf{678.92} & 
1.72 & 2.25\\CON8-7 & 813.00 & 3.78 & 
823.99 & 4.04 & \bf{811.96} & 
0.13 & 1.48\\CON8-8 & \bf{767.53} & 2.49 & 
771.62 & 4.04 & 767.53 & 0.00
 & 0.53\\CON8-9 & \bf{809.00} & 2.12 & 
827.75 & 2.90 & 809.00 & 0.00
 & 2.32\\\bf{PROM.} & 
\bf{760.85} & \bf{4.38} & \bf{766.32} & \bf{4.16} & \bf{758.54} & \bf{0.26} & \bf{0.95}\\[1ex]\hline
\end{tabular}
\label{table:nonlin}
\end{table} \clearpage
\begin{table}[ht]
\caption{Resultados de la ejecución de la metaheurística GTS, utilizando instancias de SalhiNagy con la configuración -mni 5000 -lambda1 0.05 -lambda2 0.05 -tabu 25}
\centering
\small
\begin{tabular}{c c c c c c c c}
\hline\hline
Instancia & Costo mínimo & Tiempo(seg.) & Costo promedio & Tiempo promedio(seg.) & CME & \%G & \%GP \\ [0.5ex]
\hline
CMT1X & \bf{470.48} & 4.24 & 
471.11 & 3.96 & 470.48 & 0.00
 & 0.13\\CMT1Y & 470.67 & 5.98 & 
472.11 & 3.98 & \bf{470.48} & 
0.04 & 0.35\\CMT2X & 687.73 & 7.82 & 
689.29 & 6.98 & \bf{682.39} & 
0.78 & 1.01\\CMT2Y & \bf{682.39} & 6.24 & 
689.96 & 7.01 & 682.39 & 0.00
 & 1.11\\CMT3X & 727.43 & 13.53 & 
729.52 & 16.15 & \bf{719.06} & 
1.16 & 1.45\\CMT3Y & 720.18 & 22.66 & 
726.82 & 20.70 & \bf{719.06} & 
0.16 & 1.08\\CMT4X & 857.64 & 54.81 & 
867.15 & 47.76 & \bf{854.21} & 
0.40 & 1.51\\CMT4Y & 871.17 & 32.98 & 
874.04 & 35.36 & \bf{852.46} & 
2.19 & 2.53\\CMT5X & 1070.79 & 61.77 & 
1083.86 & 88.94 & \bf{1030.56} & 
3.90 & 5.17\\CMT5Y & 1049.99 & 108.48 & 
1062.60 & 108.06 & \bf{1031.69} & 
1.77 & 3.00\\CMT11X & 877.35 & 34.40 & 
914.16 & 32.08 & \bf{831.09} & 
5.57 & 10.00\\CMT11Y & 859.39 & 18.14 & 
922.12 & 32.37 & \bf{829.85} & 
3.56 & 11.12\\CMT12X & 663.81 & 14.95 & 
668.27 & 11.66 & \bf{658.83} & 
0.76 & 1.43\\CMT12Y & 688.06 & 11.13 & 
702.93 & 13.58 & \bf{660.47} & 
4.18 & 6.43\\\bf{PROM.} & 
\bf{764.08} & \bf{28.37} & \bf{776.71} & \bf{30.61} & \bf{749.50} & \bf{1.75} & \bf{3.31}\\[1ex]\hline
\end{tabular}
\label{table:nonlin}
\end{table} \clearpage
\begin{table}[ht]
\caption{Resultados de la ejecución de la metaheurística GTS, utilizando instancias de Dethloff con la configuración -mni 5000 -lambda1 0.05 -lambda2 0.05 -tabu 29}
\centering
\small
\begin{tabular}{c c c c c c c c}
\hline\hline
Instancia & Costo mínimo & Tiempo(seg.) & Costo promedio & Tiempo promedio(seg.) & CME & \%G & \%GP \\ [0.5ex]
\hline
SCA3-0 & 636.06 & 3.06 & 
638.30 & 3.88 & \bf{635.62} & 
0.07 & 0.42\\SCA3-1 & \bf{697.84} & 2.10 & 
698.50 & 3.98 & 697.84 & 0.00
 & 0.10\\SCA3-2 & \bf{659.34} & 2.44 & 
659.34 & 3.67 & 659.34 & 0.00
 & 0.00\\
SCA3-3 & \bf{680.04} & 8.85 & 
680.32 & 5.89 & 680.04 & 0.00
 & 0.04\\SCA3-4 & \bf{690.50} & 10.47 & 
690.50 & 7.42 & 690.50 & 0.00
 & 0.00\\
SCA3-5 & \bf{659.90} & 3.75 & 
663.16 & 4.25 & 659.90 & 0.00
 & 0.49\\SCA3-6 & \bf{651.09} & 4.28 & 
651.09 & 4.32 & 651.09 & 0.00
 & 0.00\\
SCA3-7 & 666.15 & 2.59 & 
667.09 & 2.99 & \bf{659.17} & 
1.06 & 1.20\\SCA3-8 & \bf{719.47} & 5.04 & 
721.90 & 5.95 & 719.47 & 0.00
 & 0.34\\SCA3-9 & \bf{681.00} & 5.14 & 
681.00 & 4.54 & 681.00 & 0.00
 & 0.00\\
SCA8-0 & \bf{961.50} & 5.78 & 
975.16 & 3.68 & 961.50 & 0.00
 & 1.42\\SCA8-1 & 1050.20 & 4.29 & 
1062.09 & 3.39 & \bf{1049.65} & 
0.05 & 1.19\\SCA8-2 & \bf{1039.64} & 4.32 & 
1055.00 & 4.32 & 1039.64 & 0.00
 & 1.48\\SCA8-3 & \bf{983.34} & 6.90 & 
990.89 & 6.25 & 983.34 & 0.00
 & 0.77\\SCA8-4 & 1067.55 & 2.50 & 
1068.70 & 3.99 & \bf{1065.49} & 
0.19 & 0.30\\SCA8-5 & \bf{1027.08} & 4.50 & 
1042.58 & 4.23 & 1027.08 & 0.00
 & 1.51\\SCA8-6 & \bf{971.82} & 2.18 & 
972.32 & 2.72 & 971.82 & 0.00
 & 0.05\\SCA8-7 & 1062.66 & 2.87 & 
1066.24 & 4.37 & \bf{1051.28} & 
1.08 & 1.42\\SCA8-8 & \bf{1071.18} & 5.92 & 
1073.91 & 4.16 & 1071.18 & 0.00
 & 0.25\\SCA8-9 & \bf{1060.50} & 3.65 & 
1063.91 & 3.69 & 1060.50 & 0.00
 & 0.32\\CON3-0 & \bf{616.52} & 7.00 & 
620.84 & 5.05 & 616.52 & 0.00
 & 0.70\\CON3-1 & \bf{554.47} & 6.44 & 
556.43 & 5.21 & 554.47 & 0.00
 & 0.35\\CON3-2 & \bf{518.00} & 3.76 & 
521.13 & 5.85 & 518.00 & 0.00
 & 0.61\\CON3-3 & \bf{591.19} & 6.58 & 
591.19 & 7.26 & 591.19 & 0.00
 & 0.00\\
CON3-4 & \bf{588.79} & 3.25 & 
593.38 & 3.59 & 588.79 & 0.00
 & 0.78\\CON3-5 & \bf{563.70} & 4.65 & 
567.42 & 3.48 & 563.70 & 0.00
 & 0.66\\CON3-6 & \bf{499.05} & 4.15 & 
500.72 & 5.99 & 499.05 & 0.00
 & 0.34\\CON3-7 & \bf{576.48} & 5.39 & 
576.48 & 5.95 & 576.48 & 0.00
 & 0.00\\
CON3-8 & \bf{523.05} & 4.17 & 
523.05 & 4.02 & 523.05 & 0.00
 & 0.00\\
CON3-9 & 578.98 & 3.39 & 
584.86 & 4.11 & \bf{578.24} & 
0.13 & 1.15\\CON8-0 & 857.38 & 1.92 & 
863.91 & 3.57 & \bf{857.17} & 
0.02 & 0.79\\CON8-1 & \bf{740.85} & 2.77 & 
753.43 & 4.88 & 740.85 & 0.00
 & 1.70\\CON8-2 & \bf{712.89} & 9.53 & 
718.66 & 5.56 & 712.89 & 0.00
 & 0.81\\CON8-3 & \bf{811.07} & 7.72 & 
845.44 & 5.38 & 811.07 & 0.00
 & 4.24\\CON8-4 & \bf{772.25} & 3.12 & 
783.80 & 3.67 & 772.25 & 0.00
 & 1.50\\CON8-5 & \bf{754.88} & 3.23 & 
756.04 & 4.84 & 754.88 & 0.00
 & 0.15\\CON8-6 & \bf{678.92} & 4.15 & 
691.58 & 4.79 & 678.92 & 0.00
 & 1.86\\CON8-7 & 812.89 & 7.68 & 
813.84 & 4.25 & \bf{811.96} & 
0.11 & 0.23\\CON8-8 & \bf{767.53} & 6.04 & 
794.22 & 4.30 & 767.53 & 0.00
 & 3.48\\CON8-9 & \bf{809.00} & 4.66 & 
816.45 & 4.35 & 809.00 & 0.00
 & 0.92\\\bf{PROM.} & 
\bf{759.12} & \bf{4.76} & \bf{764.87} & \bf{4.59} & \bf{758.54} & \bf{0.07} & \bf{0.79}\\[1ex]\hline
\end{tabular}
\label{table:nonlin}
\end{table} \clearpage
\begin{table}[ht]
\caption{Resultados de la ejecución de la metaheurística GTS, utilizando instancias de SalhiNagy con la configuración -mni 5000 -lambda1 0.05 -lambda2 0.05 -tabu 29}
\centering
\small
\begin{tabular}{c c c c c c c c}
\hline\hline
Instancia & Costo mínimo & Tiempo(seg.) & Costo promedio & Tiempo promedio(seg.) & CME & \%G & \%GP \\ [0.5ex]
\hline
CMT1X & 470.67 & 4.63 & 
472.06 & 4.64 & \bf{470.48} & 
0.04 & 0.34\\CMT1Y & \bf{470.48} & 4.63 & 
471.62 & 5.22 & 470.48 & 0.00
 & 0.24\\CMT2X & 683.95 & 12.96 & 
688.00 & 8.90 & \bf{682.39} & 
0.23 & 0.82\\CMT2Y & 685.40 & 3.82 & 
689.71 & 6.48 & \bf{682.39} & 
0.44 & 1.07\\CMT3X & 720.85 & 18.09 & 
728.09 & 13.77 & \bf{719.06} & 
0.25 & 1.26\\CMT3Y & \bf{\underline{718.40}} & 17.16 & 
724.92 & 15.59 & 719.06 & 
\bf{-0.09} & 0.81\\CMT4X & 869.88 & 39.87 & 
875.35 & 48.16 & \bf{854.21} & 
1.83 & 2.47\\CMT4Y & 857.21 & 40.89 & 
866.88 & 39.30 & \bf{852.46} & 
0.56 & 1.69\\CMT5X & 1053.03 & 90.68 & 
1066.16 & 87.20 & \bf{1030.56} & 
2.18 & 3.45\\CMT5Y & 1052.09 & 55.50 & 
1069.69 & 73.75 & \bf{1031.69} & 
1.98 & 3.68\\CMT11X & 887.38 & 64.62 & 
927.71 & 33.04 & \bf{831.09} & 
6.77 & 11.63\\CMT11Y & 864.03 & 39.67 & 
881.12 & 39.27 & \bf{829.85} & 
4.12 & 6.18\\CMT12X & 664.46 & 26.26 & 
668.91 & 22.70 & \bf{658.83} & 
0.85 & 1.53\\CMT12Y & 667.08 & 20.05 & 
674.38 & 14.53 & \bf{660.47} & 
1.00 & 2.11\\\bf{PROM.} & 
\bf{761.78} & \bf{31.35} & \bf{771.76} & \bf{29.47} & \bf{749.50} & \bf{1.44} & \bf{2.66}\\[1ex]\hline
\end{tabular}
\label{table:nonlin}
\end{table} \clearpage
\begin{table}[ht]
\caption{Resultados de la ejecución de la metaheurística GTS, utilizando instancias de Dethloff con la configuración -mni 5000 -lambda1 0.05 -lambda2 0.05 -tabu 33}
\centering
\small
\begin{tabular}{c c c c c c c c}
\hline\hline
Instancia & Costo mínimo & Tiempo(seg.) & Costo promedio & Tiempo promedio(seg.) & CME & \%G & \%GP \\ [0.5ex]
\hline
SCA3-0 & 635.67 & 6.40 & 
638.21 & 5.36 & \bf{635.62} & 
0.01 & 0.41\\SCA3-1 & \bf{697.84} & 7.92 & 
697.84 & 5.01 & 697.84 & 0.00
 & 0.00\\
SCA3-2 & \bf{659.34} & 5.59 & 
659.34 & 4.45 & 659.34 & 0.00
 & 0.00\\
SCA3-3 & \bf{680.04} & 2.24 & 
683.23 & 3.10 & 680.04 & 0.00
 & 0.47\\SCA3-4 & \bf{690.50} & 6.16 & 
690.50 & 4.67 & 690.50 & 0.00
 & 0.00\\
SCA3-5 & \bf{659.90} & 4.46 & 
663.16 & 7.12 & 659.90 & 0.00
 & 0.49\\SCA3-6 & \bf{651.09} & 5.11 & 
651.09 & 4.20 & 651.09 & 0.00
 & 0.00\\
SCA3-7 & 666.15 & 5.40 & 
666.15 & 3.75 & \bf{659.17} & 
1.06 & 1.06\\SCA3-8 & \bf{719.47} & 5.48 & 
719.47 & 5.17 & 719.47 & 0.00
 & 0.00\\
SCA3-9 & \bf{681.00} & 10.58 & 
681.00 & 5.53 & 681.00 & 0.00
 & 0.00\\
SCA8-0 & \bf{961.50} & 4.72 & 
968.36 & 4.76 & 961.50 & 0.00
 & 0.71\\SCA8-1 & 1067.45 & 6.33 & 
1067.89 & 3.40 & \bf{1049.65} & 
1.70 & 1.74\\SCA8-2 & 1039.76 & 3.49 & 
1049.11 & 3.53 & \bf{1039.64} & 
0.01 & 0.91\\SCA8-3 & 1002.38 & 6.38 & 
1012.13 & 4.61 & \bf{983.34} & 
1.94 & 2.93\\SCA8-4 & 1067.55 & 3.22 & 
1069.06 & 3.88 & \bf{1065.49} & 
0.19 & 0.34\\SCA8-5 & \bf{1027.08} & 8.00 & 
1049.91 & 6.11 & 1027.08 & 0.00
 & 2.22\\SCA8-6 & \bf{971.82} & 4.83 & 
976.45 & 3.23 & 971.82 & 0.00
 & 0.48\\SCA8-7 & 1052.04 & 6.13 & 
1063.05 & 4.62 & \bf{1051.28} & 
0.07 & 1.12\\SCA8-8 & 1082.12 & 4.57 & 
1082.12 & 5.11 & \bf{1071.18} & 
1.02 & 1.02\\SCA8-9 & \bf{1060.50} & 3.68 & 
1065.90 & 3.50 & 1060.50 & 0.00
 & 0.51\\CON3-0 & \bf{616.52} & 3.20 & 
625.12 & 3.09 & 616.52 & 0.00
 & 1.39\\CON3-1 & \bf{554.47} & 8.56 & 
555.71 & 4.12 & 554.47 & 0.00
 & 0.22\\CON3-2 & 520.67 & 4.56 & 
521.66 & 5.63 & \bf{518.00} & 
0.52 & 0.71\\CON3-3 & \bf{591.19} & 2.02 & 
591.19 & 4.46 & 591.19 & 0.00
 & 0.00\\
CON3-4 & \bf{588.79} & 5.67 & 
593.25 & 6.85 & 588.79 & 0.00
 & 0.76\\CON3-5 & \bf{563.70} & 2.44 & 
565.91 & 4.19 & 563.70 & 0.00
 & 0.39\\CON3-6 & \bf{499.05} & 3.57 & 
499.38 & 5.03 & 499.05 & 0.00
 & 0.07\\CON3-7 & \bf{576.48} & 4.87 & 
585.97 & 5.44 & 576.48 & 0.00
 & 1.65\\CON3-8 & \bf{523.05} & 2.59 & 
523.05 & 4.73 & 523.05 & 0.00
 & 0.00\\
CON3-9 & 578.25 & 4.94 & 
580.72 & 5.25 & \bf{578.24} & 
0.00 & 0.43\\CON8-0 & 865.73 & 3.25 & 
884.43 & 5.20 & \bf{857.17} & 
1.00 & 3.18\\CON8-1 & \bf{740.85} & 2.87 & 
748.86 & 4.12 & 740.85 & 0.00
 & 1.08\\CON8-2 & 718.64 & 1.82 & 
728.22 & 3.52 & \bf{712.89} & 
0.81 & 2.15\\CON8-3 & \bf{811.07} & 2.58 & 
823.25 & 4.88 & 811.07 & 0.00
 & 1.50\\CON8-4 & \bf{772.25} & 3.27 & 
783.93 & 3.11 & 772.25 & 0.00
 & 1.51\\CON8-5 & 756.91 & 2.05 & 
757.97 & 3.10 & \bf{754.88} & 
0.27 & 0.41\\CON8-6 & 684.69 & 4.20 & 
690.63 & 3.73 & \bf{678.92} & 
0.85 & 1.72\\CON8-7 & 812.89 & 5.67 & 
825.82 & 4.17 & \bf{811.96} & 
0.11 & 1.71\\CON8-8 & \bf{767.53} & 4.50 & 
767.53 & 3.63 & 767.53 & 0.00
 & 0.00\\
CON8-9 & \bf{809.00} & 2.63 & 
809.80 & 2.57 & 809.00 & 0.00
 & 0.10\\\bf{PROM.} & 
\bf{760.62} & \bf{4.65} & \bf{765.41} & \bf{4.45} & \bf{758.54} & \bf{0.24} & \bf{0.83}\\[1ex]\hline
\end{tabular}
\label{table:nonlin}
\end{table} \clearpage
\begin{table}[ht]
\caption{Resultados de la ejecución de la metaheurística GTS, utilizando instancias de SalhiNagy con la configuración -mni 5000 -lambda1 0.05 -lambda2 0.05 -tabu 33}
\centering
\small
\begin{tabular}{c c c c c c c c}
\hline\hline
Instancia & Costo mínimo & Tiempo(seg.) & Costo promedio & Tiempo promedio(seg.) & CME & \%G & \%GP \\ [0.5ex]
\hline
CMT1X & \bf{470.48} & 5.40 & 
470.60 & 3.40 & 470.48 & 0.00
 & 0.03\\CMT1Y & \bf{470.48} & 7.22 & 
471.05 & 4.98 & 470.48 & 0.00
 & 0.12\\CMT2X & 683.97 & 5.71 & 
691.15 & 5.19 & \bf{682.39} & 
0.23 & 1.28\\CMT2Y & 684.24 & 13.17 & 
686.82 & 11.35 & \bf{682.39} & 
0.27 & 0.65\\CMT3X & 723.52 & 14.56 & 
730.69 & 13.03 & \bf{719.06} & 
0.62 & 1.62\\CMT3Y & 725.95 & 11.62 & 
731.28 & 12.36 & \bf{719.06} & 
0.96 & 1.70\\CMT4X & 856.82 & 50.21 & 
863.69 & 43.32 & \bf{854.21} & 
0.31 & 1.11\\CMT4Y & 860.97 & 27.21 & 
872.45 & 27.52 & \bf{852.46} & 
1.00 & 2.34\\CMT5X & 1051.33 & 107.73 & 
1068.54 & 98.37 & \bf{1030.56} & 
2.02 & 3.69\\CMT5Y & 1043.44 & 93.78 & 
1058.88 & 96.40 & \bf{1031.69} & 
1.14 & 2.64\\CMT11X & 876.55 & 30.47 & 
923.99 & 41.05 & \bf{831.09} & 
5.47 & 11.18\\CMT11Y & 877.56 & 54.57 & 
881.90 & 44.29 & \bf{829.85} & 
5.75 & 6.27\\CMT12X & 665.43 & 26.04 & 
669.99 & 16.96 & \bf{658.83} & 
1.00 & 1.69\\CMT12Y & 673.64 & 8.54 & 
676.47 & 17.72 & \bf{660.47} & 
1.99 & 2.42\\\bf{PROM.} & 
\bf{761.74} & \bf{32.59} & \bf{771.25} & \bf{31.14} & \bf{749.50} & \bf{1.48} & \bf{2.62}\\[1ex]\hline
\end{tabular}
\label{table:nonlin}
\end{table} \clearpage
\begin{table}[ht]
\caption{Resultados de la ejecución de la metaheurística GTS, utilizando instancias de Dethloff con la configuración -mni 5000 -lambda1 0.05 -lambda2 0.05 -tabu 37}
\centering
\small
\begin{tabular}{c c c c c c c c}
\hline\hline
Instancia & Costo mínimo & Tiempo(seg.) & Costo promedio & Tiempo promedio(seg.) & CME & \%G & \%GP \\ [0.5ex]
\hline
SCA3-0 & 636.06 & 7.16 & 
637.18 & 5.71 & \bf{635.62} & 
0.07 & 0.25\\SCA3-1 & \bf{697.84} & 4.38 & 
697.84 & 4.59 & 697.84 & 0.00
 & 0.00\\
SCA3-2 & \bf{659.34} & 3.73 & 
659.34 & 4.69 & 659.34 & 0.00
 & 0.00\\
SCA3-3 & \bf{680.04} & 6.40 & 
680.46 & 6.03 & 680.04 & 0.00
 & 0.06\\SCA3-4 & \bf{690.50} & 8.30 & 
690.50 & 5.34 & 690.50 & 0.00
 & 0.00\\
SCA3-5 & \bf{659.90} & 3.40 & 
663.16 & 4.40 & 659.90 & 0.00
 & 0.49\\SCA3-6 & \bf{651.09} & 4.08 & 
656.49 & 3.56 & 651.09 & 0.00
 & 0.83\\SCA3-7 & 666.15 & 2.75 & 
666.26 & 5.43 & \bf{659.17} & 
1.06 & 1.08\\SCA3-8 & \bf{719.47} & 8.67 & 
719.47 & 5.46 & 719.47 & 0.00
 & 0.00\\
SCA3-9 & \bf{681.00} & 2.54 & 
681.00 & 3.60 & 681.00 & 0.00
 & 0.00\\
SCA8-0 & \bf{961.50} & 3.82 & 
984.43 & 4.53 & 961.50 & 0.00
 & 2.39\\SCA8-1 & \bf{1049.65} & 6.98 & 
1058.93 & 4.28 & 1049.65 & 0.00
 & 0.88\\SCA8-2 & \bf{1039.64} & 8.42 & 
1051.15 & 5.54 & 1039.64 & 0.00
 & 1.11\\SCA8-3 & \bf{983.34} & 12.80 & 
990.89 & 7.06 & 983.34 & 0.00
 & 0.77\\SCA8-4 & 1068.97 & 6.31 & 
1071.20 & 5.15 & \bf{1065.49} & 
0.33 & 0.54\\SCA8-5 & \bf{1027.08} & 5.18 & 
1043.38 & 4.48 & 1027.08 & 0.00
 & 1.59\\SCA8-6 & 972.48 & 3.74 & 
976.62 & 3.62 & \bf{971.82} & 
0.07 & 0.49\\SCA8-7 & \bf{1051.28} & 3.65 & 
1063.92 & 5.83 & 1051.28 & 0.00
 & 1.20\\SCA8-8 & \bf{1071.18} & 2.40 & 
1073.91 & 5.11 & 1071.18 & 0.00
 & 0.25\\SCA8-9 & \bf{1060.50} & 3.55 & 
1063.70 & 5.04 & 1060.50 & 0.00
 & 0.30\\CON3-0 & \bf{616.52} & 4.25 & 
622.47 & 6.22 & 616.52 & 0.00
 & 0.96\\CON3-1 & \bf{554.47} & 4.86 & 
556.16 & 6.07 & 554.47 & 0.00
 & 0.30\\CON3-2 & 521.33 & 8.78 & 
522.75 & 5.00 & \bf{518.00} & 
0.64 & 0.92\\CON3-3 & \bf{591.19} & 5.06 & 
591.19 & 6.50 & 591.19 & 0.00
 & 0.00\\
CON3-4 & \bf{588.79} & 5.12 & 
597.23 & 4.52 & 588.79 & 0.00
 & 1.43\\CON3-5 & \bf{563.70} & 6.46 & 
563.70 & 5.22 & 563.70 & 0.00
 & 0.00\\
CON3-6 & \bf{499.05} & 3.89 & 
500.64 & 4.64 & 499.05 & 0.00
 & 0.32\\CON3-7 & \bf{576.48} & 7.53 & 
580.84 & 7.29 & 576.48 & 0.00
 & 0.76\\CON3-8 & \bf{523.05} & 7.40 & 
523.05 & 4.88 & 523.05 & 0.00
 & 0.00\\
CON3-9 & 578.25 & 7.86 & 
584.36 & 8.21 & \bf{578.24} & 
0.00 & 1.06\\CON8-0 & 858.16 & 5.70 & 
863.87 & 4.29 & \bf{857.17} & 
0.12 & 0.78\\CON8-1 & \bf{740.85} & 2.74 & 
748.04 & 4.15 & 740.85 & 0.00
 & 0.97\\CON8-2 & 716.03 & 8.42 & 
728.59 & 5.54 & \bf{712.89} & 
0.44 & 2.20\\CON8-3 & 821.26 & 8.54 & 
827.25 & 5.25 & \bf{811.07} & 
1.26 & 1.99\\CON8-4 & \bf{772.25} & 5.22 & 
775.39 & 5.34 & 772.25 & 0.00
 & 0.41\\CON8-5 & 754.95 & 3.66 & 
756.80 & 4.62 & \bf{754.88} & 
0.01 & 0.26\\CON8-6 & 681.33 & 4.39 & 
688.75 & 4.58 & \bf{678.92} & 
0.35 & 1.45\\CON8-7 & 814.50 & 8.41 & 
837.74 & 4.51 & \bf{811.96} & 
0.31 & 3.17\\CON8-8 & \bf{767.53} & 3.67 & 
782.86 & 5.41 & 767.53 & 0.00
 & 2.00\\CON8-9 & 811.18 & 3.52 & 
817.45 & 3.23 & \bf{809.00} & 
0.27 & 1.04\\\bf{PROM.} & 
\bf{759.45} & \bf{5.59} & \bf{764.97} & \bf{5.12} & \bf{758.54} & \bf{0.12} & \bf{0.81}\\[1ex]\hline
\end{tabular}
\label{table:nonlin}
\end{table} \clearpage
\begin{table}[ht]
\caption{Resultados de la ejecución de la metaheurística GTS, utilizando instancias de SalhiNagy con la configuración -mni 5000 -lambda1 0.05 -lambda2 0.05 -tabu 37}
\centering
\small
\begin{tabular}{c c c c c c c c}
\hline\hline
Instancia & Costo mínimo & Tiempo(seg.) & Costo promedio & Tiempo promedio(seg.) & CME & \%G & \%GP \\ [0.5ex]
\hline
CMT1X & \bf{470.48} & 3.23 & 
471.05 & 4.15 & 470.48 & 0.00
 & 0.12\\CMT1Y & 472.37 & 7.38 & 
472.75 & 4.57 & \bf{470.48} & 
0.40 & 0.48\\CMT2X & 684.40 & 8.12 & 
686.93 & 7.93 & \bf{682.39} & 
0.29 & 0.67\\CMT2Y & 684.61 & 9.51 & 
685.83 & 6.67 & \bf{682.39} & 
0.33 & 0.50\\CMT3X & 720.08 & 17.55 & 
724.32 & 13.92 & \bf{719.06} & 
0.14 & 0.73\\CMT3Y & 719.24 & 12.88 & 
722.30 & 21.97 & \bf{719.06} & 
0.03 & 0.45\\CMT4X & 859.91 & 42.04 & 
868.65 & 45.02 & \bf{854.21} & 
0.67 & 1.69\\CMT4Y & 855.50 & 30.56 & 
872.61 & 40.44 & \bf{852.46} & 
0.36 & 2.36\\CMT5X & 1034.47 & 149.62 & 
1074.57 & 100.35 & \bf{1030.56} & 
0.38 & 4.27\\CMT5Y & 1058.07 & 188.37 & 
1072.12 & 114.88 & \bf{1031.69} & 
2.56 & 3.92\\CMT11X & 886.09 & 11.84 & 
927.52 & 26.26 & \bf{831.09} & 
6.62 & 11.60\\CMT11Y & 877.73 & 62.72 & 
934.12 & 42.12 & \bf{829.85} & 
5.77 & 12.56\\CMT12X & 670.50 & 14.44 & 
673.27 & 12.21 & \bf{658.83} & 
1.77 & 2.19\\CMT12Y & 673.77 & 13.84 & 
682.41 & 14.23 & \bf{660.47} & 
2.01 & 3.32\\\bf{PROM.} & 
\bf{761.94} & \bf{40.86} & \bf{776.32} & \bf{32.48} & \bf{749.50} & \bf{1.52} & \bf{3.21}\\[1ex]\hline
\end{tabular}
\label{table:nonlin}
\end{table} \clearpage
\begin{table}[ht]
\caption{Resultados de la ejecución de la metaheurística GTS, utilizando instancias de Dethloff con la configuración -mni 5500 -lambda1 0.05 -lambda2 0.05 -tabu 17}
\centering
\small
\begin{tabular}{c c c c c c c c}
\hline\hline
Instancia & Costo mínimo & Tiempo(seg.) & Costo promedio & Tiempo promedio(seg.) & CME & \%G & \%GP \\ [0.5ex]
\hline
SCA3-0 & 636.06 & 3.12 & 
639.43 & 3.83 & \bf{635.62} & 
0.07 & 0.60\\SCA3-1 & \bf{697.84} & 2.31 & 
698.50 & 4.04 & 697.84 & 0.00
 & 0.10\\SCA3-2 & \bf{659.34} & 8.26 & 
659.34 & 4.64 & 659.34 & 0.00
 & 0.00\\
SCA3-3 & \bf{680.04} & 3.83 & 
680.04 & 4.35 & 680.04 & 0.00
 & 0.00\\
SCA3-4 & \bf{690.50} & 6.95 & 
690.50 & 6.27 & 690.50 & 0.00
 & 0.00\\
SCA3-5 & \bf{659.90} & 4.88 & 
666.42 & 4.41 & 659.90 & 0.00
 & 0.99\\SCA3-6 & \bf{651.09} & 1.93 & 
654.56 & 3.72 & 651.09 & 0.00
 & 0.53\\SCA3-7 & 666.15 & 2.04 & 
666.15 & 5.13 & \bf{659.17} & 
1.06 & 1.06\\SCA3-8 & \bf{719.47} & 3.08 & 
721.90 & 4.26 & 719.47 & 0.00
 & 0.34\\SCA3-9 & \bf{681.00} & 3.73 & 
681.00 & 4.25 & 681.00 & 0.00
 & 0.00\\
SCA8-0 & 978.90 & 4.41 & 
981.53 & 5.27 & \bf{961.50} & 
1.81 & 2.08\\SCA8-1 & 1050.20 & 7.53 & 
1064.53 & 4.60 & \bf{1049.65} & 
0.05 & 1.42\\SCA8-2 & 1039.71 & 5.03 & 
1053.38 & 5.58 & \bf{1039.64} & 
0.01 & 1.32\\SCA8-3 & 1006.05 & 12.84 & 
1010.32 & 6.08 & \bf{983.34} & 
2.31 & 2.74\\SCA8-4 & 1067.55 & 8.66 & 
1070.26 & 4.82 & \bf{1065.49} & 
0.19 & 0.45\\SCA8-5 & \bf{1027.08} & 3.84 & 
1045.60 & 4.51 & 1027.08 & 0.00
 & 1.80\\SCA8-6 & 972.48 & 2.29 & 
981.16 & 2.56 & \bf{971.82} & 
0.07 & 0.96\\SCA8-7 & 1063.22 & 4.50 & 
1065.68 & 5.56 & \bf{1051.28} & 
1.14 & 1.37\\SCA8-8 & \bf{1071.18} & 2.60 & 
1079.00 & 2.87 & 1071.18 & 0.00
 & 0.73\\SCA8-9 & \bf{1060.50} & 5.96 & 
1063.03 & 4.67 & 1060.50 & 0.00
 & 0.24\\CON3-0 & \bf{616.52} & 6.04 & 
624.62 & 3.65 & 616.52 & 0.00
 & 1.31\\CON3-1 & \bf{554.47} & 4.92 & 
554.86 & 3.97 & 554.47 & 0.00
 & 0.07\\CON3-2 & \bf{518.00} & 2.27 & 
521.08 & 3.40 & 518.00 & 0.00
 & 0.59\\CON3-3 & \bf{591.19} & 8.21 & 
591.19 & 6.18 & 591.19 & 0.00
 & 0.00\\
CON3-4 & \bf{588.79} & 7.87 & 
590.66 & 5.21 & 588.79 & 0.00
 & 0.32\\CON3-5 & \bf{563.70} & 3.47 & 
569.64 & 4.13 & 563.70 & 0.00
 & 1.05\\CON3-6 & \bf{499.05} & 3.35 & 
500.98 & 3.11 & 499.05 & 0.00
 & 0.39\\CON3-7 & \bf{576.48} & 3.92 & 
576.96 & 7.47 & 576.48 & 0.00
 & 0.08\\CON3-8 & \bf{523.05} & 2.09 & 
523.05 & 3.62 & 523.05 & 0.00
 & 0.00\\
CON3-9 & 578.25 & 7.27 & 
578.56 & 4.82 & \bf{578.24} & 
0.00 & 0.06\\CON8-0 & 864.50 & 5.14 & 
882.45 & 4.91 & \bf{857.17} & 
0.86 & 2.95\\CON8-1 & 751.76 & 3.50 & 
757.44 & 3.38 & \bf{740.85} & 
1.47 & 2.24\\CON8-2 & 724.44 & 3.54 & 
727.02 & 3.42 & \bf{712.89} & 
1.62 & 1.98\\CON8-3 & \bf{811.07} & 4.82 & 
821.18 & 5.68 & 811.07 & 0.00
 & 1.25\\CON8-4 & \bf{772.25} & 6.28 & 
779.17 & 4.50 & 772.25 & 0.00
 & 0.90\\CON8-5 & 756.91 & 2.19 & 
758.49 & 6.35 & \bf{754.88} & 
0.27 & 0.48\\CON8-6 & 690.63 & 3.20 & 
694.62 & 5.31 & \bf{678.92} & 
1.72 & 2.31\\CON8-7 & \bf{811.96} & 6.29 & 
816.21 & 3.77 & 811.96 & 0.00
 & 0.52\\CON8-8 & \bf{767.53} & 7.15 & 
771.30 & 5.22 & 767.53 & 0.00
 & 0.49\\CON8-9 & \bf{809.00} & 5.04 & 
835.95 & 4.38 & 809.00 & 0.00
 & 3.33\\\bf{PROM.} & 
\bf{761.20} & \bf{4.86} & \bf{766.19} & \bf{4.60} & \bf{758.54} & \bf{0.32} & \bf{0.93}\\[1ex]\hline
\end{tabular}
\label{table:nonlin}
\end{table} \clearpage
\begin{table}[ht]
\caption{Resultados de la ejecución de la metaheurística GTS, utilizando instancias de SalhiNagy con la configuración -mni 5500 -lambda1 0.05 -lambda2 0.05 -tabu 17}
\centering
\small
\begin{tabular}{c c c c c c c c}
\hline\hline
Instancia & Costo mínimo & Tiempo(seg.) & Costo promedio & Tiempo promedio(seg.) & CME & \%G & \%GP \\ [0.5ex]
\hline
CMT1X & \bf{470.48} & 3.57 & 
470.95 & 5.21 & 470.48 & 0.00
 & 0.10\\CMT1Y & \bf{470.48} & 3.01 & 
474.13 & 3.93 & 470.48 & 0.00
 & 0.78\\CMT2X & 685.79 & 14.81 & 
688.73 & 10.72 & \bf{682.39} & 
0.50 & 0.93\\CMT2Y & 684.25 & 4.60 & 
686.00 & 6.58 & \bf{682.39} & 
0.27 & 0.53\\CMT3X & \bf{719.06} & 18.26 & 
724.06 & 17.63 & 719.06 & 0.00
 & 0.70\\CMT3Y & 726.28 & 20.17 & 
729.63 & 16.84 & \bf{719.06} & 
1.00 & 1.47\\CMT4X & 862.75 & 71.57 & 
872.82 & 58.05 & \bf{854.21} & 
1.00 & 2.18\\CMT4Y & 860.20 & 25.56 & 
871.34 & 54.57 & \bf{852.46} & 
0.91 & 2.21\\CMT5X & 1036.15 & 129.40 & 
1051.16 & 111.59 & \bf{1030.56} & 
0.54 & 2.00\\CMT5Y & 1046.92 & 91.52 & 
1068.83 & 98.48 & \bf{1031.69} & 
1.48 & 3.60\\CMT11X & 939.61 & 41.52 & 
959.62 & 43.90 & \bf{831.09} & 
13.06 & 15.47\\CMT11Y & 857.51 & 73.20 & 
894.03 & 56.01 & \bf{829.85} & 
3.33 & 7.73\\CMT12X & 669.04 & 23.14 & 
674.78 & 15.63 & \bf{658.83} & 
1.55 & 2.42\\CMT12Y & 674.44 & 15.46 & 
681.41 & 10.87 & \bf{660.47} & 
2.12 & 3.17\\\bf{PROM.} & 
\bf{764.50} & \bf{38.27} & \bf{774.82} & \bf{36.43} & \bf{749.50} & \bf{1.84} & \bf{3.09}\\[1ex]\hline
\end{tabular}
\label{table:nonlin}
\end{table} \clearpage
\begin{table}[ht]
\caption{Resultados de la ejecución de la metaheurística GTS, utilizando instancias de Dethloff con la configuración -mni 5500 -lambda1 0.05 -lambda2 0.05 -tabu 21}
\centering
\small
\begin{tabular}{c c c c c c c c}
\hline\hline
Instancia & Costo mínimo & Tiempo(seg.) & Costo promedio & Tiempo promedio(seg.) & CME & \%G & \%GP \\ [0.5ex]
\hline
SCA3-0 & 636.06 & 3.82 & 
636.06 & 2.99 & \bf{635.62} & 
0.07 & 0.07\\SCA3-1 & \bf{697.84} & 3.61 & 
699.17 & 4.35 & 697.84 & 0.00
 & 0.19\\SCA3-2 & \bf{659.34} & 4.19 & 
659.34 & 3.48 & 659.34 & 0.00
 & 0.00\\
SCA3-3 & \bf{680.04} & 5.70 & 
686.91 & 3.67 & 680.04 & 0.00
 & 1.01\\SCA3-4 & \bf{690.50} & 7.08 & 
690.50 & 6.09 & 690.50 & 0.00
 & 0.00\\
SCA3-5 & \bf{659.90} & 4.96 & 
666.42 & 3.17 & 659.90 & 0.00
 & 0.99\\SCA3-6 & \bf{651.09} & 3.69 & 
656.25 & 2.97 & 651.09 & 0.00
 & 0.79\\SCA3-7 & 666.15 & 5.82 & 
667.09 & 3.96 & \bf{659.17} & 
1.06 & 1.20\\SCA3-8 & \bf{719.47} & 11.76 & 
719.47 & 7.49 & 719.47 & 0.00
 & 0.00\\
SCA3-9 & \bf{681.00} & 4.23 & 
681.00 & 4.02 & 681.00 & 0.00
 & 0.00\\
SCA8-0 & 970.64 & 2.37 & 
980.55 & 3.76 & \bf{961.50} & 
0.95 & 1.98\\SCA8-1 & \bf{1049.65} & 3.30 & 
1054.47 & 3.87 & 1049.65 & 0.00
 & 0.46\\SCA8-2 & \bf{1039.64} & 3.61 & 
1054.62 & 3.45 & 1039.64 & 0.00
 & 1.44\\SCA8-3 & \bf{983.34} & 6.75 & 
1004.32 & 3.90 & 983.34 & 0.00
 & 2.13\\SCA8-4 & 1067.28 & 7.87 & 
1069.41 & 5.30 & \bf{1065.49} & 
0.17 & 0.37\\SCA8-5 & \bf{1027.08} & 3.69 & 
1034.69 & 3.38 & 1027.08 & 0.00
 & 0.74\\SCA8-6 & 972.48 & 4.24 & 
976.62 & 4.15 & \bf{971.82} & 
0.07 & 0.49\\SCA8-7 & 1060.98 & 3.32 & 
1072.86 & 4.18 & \bf{1051.28} & 
0.92 & 2.05\\SCA8-8 & \bf{1071.18} & 2.34 & 
1076.65 & 2.80 & 1071.18 & 0.00
 & 0.51\\SCA8-9 & \bf{1060.50} & 10.16 & 
1064.40 & 7.45 & 1060.50 & 0.00
 & 0.37\\CON3-0 & \bf{616.52} & 7.33 & 
625.00 & 6.00 & 616.52 & 0.00
 & 1.38\\CON3-1 & 554.59 & 3.39 & 
556.38 & 4.36 & \bf{554.47} & 
0.02 & 0.34\\CON3-2 & 519.26 & 2.15 & 
522.39 & 3.54 & \bf{518.00} & 
0.24 & 0.85\\CON3-3 & \bf{591.19} & 6.27 & 
591.92 & 3.74 & 591.19 & 0.00
 & 0.12\\CON3-4 & \bf{588.79} & 13.48 & 
593.25 & 6.01 & 588.79 & 0.00
 & 0.76\\CON3-5 & \bf{563.70} & 2.81 & 
563.70 & 5.01 & 563.70 & 0.00
 & 0.00\\
CON3-6 & \bf{499.05} & 4.46 & 
501.38 & 6.18 & 499.05 & 0.00
 & 0.47\\CON3-7 & \bf{576.48} & 4.68 & 
580.34 & 6.13 & 576.48 & 0.00
 & 0.67\\CON3-8 & \bf{523.05} & 8.63 & 
523.05 & 4.24 & 523.05 & 0.00
 & 0.00\\
CON3-9 & 582.79 & 4.60 & 
582.79 & 6.38 & \bf{578.24} & 
0.79 & 0.79\\CON8-0 & \bf{857.17} & 3.49 & 
879.00 & 4.19 & 857.17 & 0.00
 & 2.55\\CON8-1 & \bf{740.85} & 4.28 & 
755.67 & 3.84 & 740.85 & 0.00
 & 2.00\\CON8-2 & 718.64 & 4.48 & 
732.52 & 4.91 & \bf{712.89} & 
0.81 & 2.75\\CON8-3 & \bf{811.07} & 3.05 & 
837.04 & 4.72 & 811.07 & 0.00
 & 3.20\\CON8-4 & \bf{772.25} & 2.38 & 
782.92 & 4.00 & 772.25 & 0.00
 & 1.38\\CON8-5 & 754.95 & 2.33 & 
757.46 & 3.92 & \bf{754.88} & 
0.01 & 0.34\\CON8-6 & 688.68 & 6.66 & 
690.83 & 5.34 & \bf{678.92} & 
1.44 & 1.75\\CON8-7 & 813.00 & 6.41 & 
816.83 & 3.75 & \bf{811.96} & 
0.13 & 0.60\\CON8-8 & \bf{767.53} & 9.81 & 
771.62 & 4.46 & 767.53 & 0.00
 & 0.53\\CON8-9 & 810.18 & 2.52 & 
830.88 & 3.30 & \bf{809.00} & 
0.15 & 2.70\\\bf{PROM.} & 
\bf{759.85} & \bf{5.14} & \bf{766.14} & \bf{4.46} & \bf{758.54} & \bf{0.17} & \bf{0.95}\\[1ex]\hline
\end{tabular}
\label{table:nonlin}
\end{table} \clearpage
\begin{table}[ht]
\caption{Resultados de la ejecución de la metaheurística GTS, utilizando instancias de SalhiNagy con la configuración -mni 5500 -lambda1 0.05 -lambda2 0.05 -tabu 21}
\centering
\small
\begin{tabular}{c c c c c c c c}
\hline\hline
Instancia & Costo mínimo & Tiempo(seg.) & Costo promedio & Tiempo promedio(seg.) & CME & \%G & \%GP \\ [0.5ex]
\hline
CMT1X & \bf{470.48} & 3.62 & 
471.00 & 6.30 & 470.48 & 0.00
 & 0.11\\CMT1Y & \bf{470.48} & 7.74 & 
471.47 & 6.68 & 470.48 & 0.00
 & 0.21\\CMT2X & 683.52 & 4.82 & 
685.54 & 8.60 & \bf{682.39} & 
0.17 & 0.46\\CMT2Y & 683.19 & 5.48 & 
688.44 & 6.50 & \bf{682.39} & 
0.12 & 0.89\\CMT3X & 723.68 & 7.32 & 
727.75 & 17.01 & \bf{719.06} & 
0.64 & 1.21\\CMT3Y & 724.57 & 13.04 & 
728.03 & 12.78 & \bf{719.06} & 
0.77 & 1.25\\CMT4X & 861.38 & 44.13 & 
871.78 & 49.55 & \bf{854.21} & 
0.84 & 2.06\\CMT4Y & 864.47 & 45.79 & 
872.30 & 58.51 & \bf{852.46} & 
1.41 & 2.33\\CMT5X & 1049.78 & 55.59 & 
1063.39 & 75.47 & \bf{1030.56} & 
1.87 & 3.19\\CMT5Y & 1058.69 & 75.65 & 
1088.09 & 90.76 & \bf{1031.69} & 
2.62 & 5.47\\CMT11X & 869.17 & 41.14 & 
920.63 & 37.32 & \bf{831.09} & 
4.58 & 10.77\\CMT11Y & 876.68 & 34.19 & 
888.97 & 48.49 & \bf{829.85} & 
5.64 & 7.12\\CMT12X & 673.59 & 18.88 & 
688.85 & 16.20 & \bf{658.83} & 
2.24 & 4.56\\CMT12Y & 673.92 & 12.42 & 
676.71 & 14.24 & \bf{660.47} & 
2.04 & 2.46\\\bf{PROM.} & 
\bf{763.11} & \bf{26.41} & \bf{774.50} & \bf{32.03} & \bf{749.50} & \bf{1.64} & \bf{3.01}\\[1ex]\hline
\end{tabular}
\label{table:nonlin}
\end{table} \clearpage
\begin{table}[ht]
\caption{Resultados de la ejecución de la metaheurística GTS, utilizando instancias de Dethloff con la configuración -mni 5500 -lambda1 0.05 -lambda2 0.05 -tabu 25}
\centering
\small
\begin{tabular}{c c c c c c c c}
\hline\hline
Instancia & Costo mínimo & Tiempo(seg.) & Costo promedio & Tiempo promedio(seg.) & CME & \%G & \%GP \\ [0.5ex]
\hline
SCA3-0 & 636.06 & 7.14 & 
639.43 & 5.61 & \bf{635.62} & 
0.07 & 0.60\\SCA3-1 & \bf{697.84} & 5.14 & 
697.84 & 4.15 & 697.84 & 0.00
 & 0.00\\
SCA3-2 & \bf{659.34} & 3.44 & 
659.34 & 3.84 & 659.34 & 0.00
 & 0.00\\
SCA3-3 & \bf{680.04} & 5.13 & 
682.48 & 5.57 & 680.04 & 0.00
 & 0.36\\SCA3-4 & \bf{690.50} & 4.74 & 
690.50 & 3.72 & 690.50 & 0.00
 & 0.00\\
SCA3-5 & \bf{659.90} & 3.39 & 
663.16 & 3.90 & 659.90 & 0.00
 & 0.49\\SCA3-6 & \bf{651.09} & 6.60 & 
652.15 & 4.09 & 651.09 & 0.00
 & 0.16\\SCA3-7 & 666.15 & 3.46 & 
666.15 & 4.41 & \bf{659.17} & 
1.06 & 1.06\\SCA3-8 & \bf{719.47} & 6.64 & 
719.47 & 3.78 & 719.47 & 0.00
 & 0.00\\
SCA3-9 & \bf{681.00} & 3.53 & 
681.00 & 4.11 & 681.00 & 0.00
 & 0.00\\
SCA8-0 & \bf{961.50} & 5.34 & 
969.26 & 4.39 & 961.50 & 0.00
 & 0.81\\SCA8-1 & \bf{1049.65} & 2.94 & 
1058.95 & 3.76 & 1049.65 & 0.00
 & 0.89\\SCA8-2 & \bf{1039.64} & 4.43 & 
1050.22 & 2.88 & 1039.64 & 0.00
 & 1.02\\SCA8-3 & \bf{983.34} & 6.72 & 
997.02 & 6.25 & 983.34 & 0.00
 & 1.39\\SCA8-4 & \bf{1065.49} & 4.79 & 
1067.32 & 6.11 & 1065.49 & 0.00
 & 0.17\\SCA8-5 & \bf{1027.08} & 7.58 & 
1042.86 & 4.12 & 1027.08 & 0.00
 & 1.54\\SCA8-6 & 972.48 & 6.58 & 
976.62 & 4.64 & \bf{971.82} & 
0.07 & 0.49\\SCA8-7 & 1066.65 & 5.39 & 
1077.87 & 4.89 & \bf{1051.28} & 
1.46 & 2.53\\SCA8-8 & \bf{1071.18} & 1.92 & 
1075.65 & 2.30 & 1071.18 & 0.00
 & 0.42\\SCA8-9 & \bf{1060.50} & 8.92 & 
1060.50 & 6.74 & 1060.50 & 0.00
 & 0.00\\
CON3-0 & \bf{616.52} & 5.13 & 
625.48 & 4.95 & 616.52 & 0.00
 & 1.45\\CON3-1 & \bf{554.47} & 3.55 & 
554.86 & 6.30 & 554.47 & 0.00
 & 0.07\\CON3-2 & 523.23 & 5.86 & 
523.60 & 4.87 & \bf{518.00} & 
1.01 & 1.08\\CON3-3 & \bf{591.19} & 7.54 & 
610.35 & 5.02 & 591.19 & 0.00
 & 3.24\\CON3-4 & \bf{588.79} & 5.38 & 
589.56 & 5.14 & 588.79 & 0.00
 & 0.13\\CON3-5 & \bf{563.70} & 4.43 & 
569.53 & 4.41 & 563.70 & 0.00
 & 1.04\\CON3-6 & \bf{499.05} & 6.72 & 
500.26 & 5.33 & 499.05 & 0.00
 & 0.24\\CON3-7 & \bf{576.48} & 3.28 & 
581.50 & 5.89 & 576.48 & 0.00
 & 0.87\\CON3-8 & \bf{523.05} & 4.25 & 
523.05 & 4.67 & 523.05 & 0.00
 & 0.00\\
CON3-9 & 582.79 & 6.51 & 
587.31 & 4.48 & \bf{578.24} & 
0.79 & 1.57\\CON8-0 & \bf{857.17} & 7.66 & 
858.00 & 4.76 & 857.17 & 0.00
 & 0.10\\CON8-1 & \bf{740.85} & 5.72 & 
752.17 & 5.22 & 740.85 & 0.00
 & 1.53\\CON8-2 & \bf{712.89} & 8.61 & 
723.56 & 5.33 & 712.89 & 0.00
 & 1.50\\CON8-3 & \bf{811.07} & 2.89 & 
827.51 & 3.91 & 811.07 & 0.00
 & 2.03\\CON8-4 & \bf{772.25} & 4.25 & 
784.30 & 3.46 & 772.25 & 0.00
 & 1.56\\CON8-5 & 756.91 & 3.71 & 
758.17 & 4.32 & \bf{754.88} & 
0.27 & 0.44\\CON8-6 & 691.30 & 5.43 & 
691.88 & 4.72 & \bf{678.92} & 
1.82 & 1.91\\CON8-7 & 812.26 & 6.34 & 
812.84 & 4.72 & \bf{811.96} & 
0.04 & 0.11\\CON8-8 & \bf{767.53} & 7.26 & 
772.56 & 6.02 & 767.53 & 0.00
 & 0.66\\CON8-9 & 809.12 & 3.92 & 
813.24 & 4.92 & \bf{809.00} & 
0.01 & 0.52\\\bf{PROM.} & 
\bf{759.74} & \bf{5.31} & \bf{764.69} & \bf{4.69} & \bf{758.54} & \bf{0.16} & \bf{0.80}\\[1ex]\hline
\end{tabular}
\label{table:nonlin}
\end{table} \clearpage
\begin{table}[ht]
\caption{Resultados de la ejecución de la metaheurística ILS, utilizando instancias de Dethloff con la configuración -n 5 -LS 10 -y 0.1}
\centering
\small
\begin{tabular}{c c c c c c c c}
\hline\hline
Instancia & Costo mínimo & Tiempo(seg.) & Costo promedio & Tiempo promedio(seg.) & CME & \%G & \%GP \\ [0.5ex]
\hline
SCA3-0 & 668.10 & 0.10 & 
668.10 & 0.10 & \bf{635.62} & 
5.11 & 5.11\\SCA3-1 & 731.03 & 0.11 & 
731.03 & 0.11 & \bf{697.84} & 
4.76 & 4.76\\SCA3-2 & 696.54 & 0.18 & 
696.54 & 0.18 & \bf{659.34} & 
5.64 & 5.64\\SCA3-3 & 695.69 & 0.08 & 
695.69 & 0.08 & \bf{680.04} & 
2.30 & 2.30\\SCA3-4 & 741.36 & 0.10 & 
741.36 & 0.10 & \bf{690.50} & 
7.37 & 7.37\\SCA3-5 & 704.34 & 0.07 & 
704.34 & 0.07 & \bf{659.90} & 
6.73 & 6.73\\SCA3-6 & 653.93 & 0.10 & 
653.93 & 0.10 & \bf{651.09} & 
0.44 & 0.44\\SCA3-7 & 734.23 & 0.11 & 
734.23 & 0.11 & \bf{659.17} & 
11.39 & 11.39\\SCA3-8 & 731.44 & 0.10 & 
731.44 & 0.10 & \bf{719.47} & 
1.66 & 1.66\\SCA3-9 & 720.04 & 0.12 & 
720.04 & 0.12 & \bf{681.00} & 
5.73 & 5.73\\SCA8-0 & 1093.22 & 0.11 & 
1093.22 & 0.11 & \bf{961.50} & 
13.70 & 13.70\\SCA8-1 & 1100.55 & 0.10 & 
1100.55 & 0.10 & \bf{1049.65} & 
4.85 & 4.85\\SCA8-2 & 1089.76 & 0.15 & 
1089.76 & 0.15 & \bf{1039.64} & 
4.82 & 4.82\\SCA8-3 & 1120.23 & 0.12 & 
1120.23 & 0.12 & \bf{983.34} & 
13.92 & 13.92\\SCA8-4 & 1168.58 & 0.13 & 
1168.58 & 0.13 & \bf{1065.49} & 
9.68 & 9.68\\SCA8-5 & 1093.24 & 0.10 & 
1093.24 & 0.10 & \bf{1027.08} & 
6.44 & 6.44\\SCA8-6 & 1079.25 & 0.13 & 
1079.25 & 0.13 & \bf{971.82} & 
11.05 & 11.05\\SCA8-7 & 1142.04 & 0.15 & 
1142.04 & 0.15 & \bf{1051.28} & 
8.63 & 8.63\\SCA8-8 & 1092.02 & 0.20 & 
1092.02 & 0.20 & \bf{1071.18} & 
1.95 & 1.95\\SCA8-9 & 1163.94 & 0.10 & 
1163.94 & 0.10 & \bf{1060.50} & 
9.75 & 9.75\\CON3-0 & 644.60 & 0.10 & 
644.60 & 0.10 & \bf{616.52} & 
4.55 & 4.55\\CON3-1 & 594.57 & 0.11 & 
594.57 & 0.11 & \bf{554.47} & 
7.23 & 7.23\\CON3-2 & 532.60 & 0.10 & 
532.60 & 0.10 & \bf{518.00} & 
2.82 & 2.82\\CON3-3 & 604.90 & 0.17 & 
604.90 & 0.17 & \bf{591.19} & 
2.32 & 2.32\\CON3-4 & 593.78 & 0.09 & 
593.78 & 0.09 & \bf{588.79} & 
0.85 & 0.85\\CON3-5 & 587.10 & 0.11 & 
587.10 & 0.11 & \bf{563.70} & 
4.15 & 4.15\\CON3-6 & 536.61 & 0.11 & 
536.61 & 0.11 & \bf{499.05} & 
7.53 & 7.53\\CON3-7 & 628.03 & 0.13 & 
628.03 & 0.13 & \bf{576.48} & 
8.94 & 8.94\\CON3-8 & 602.35 & 0.08 & 
602.35 & 0.08 & \bf{523.05} & 
15.16 & 15.16\\CON3-9 & 611.82 & 0.09 & 
611.82 & 0.09 & \bf{578.24} & 
5.81 & 5.81\\CON8-0 & 939.24 & 0.09 & 
939.24 & 0.09 & \bf{857.17} & 
9.57 & 9.57\\CON8-1 & 792.96 & 0.16 & 
792.96 & 0.16 & \bf{740.85} & 
7.03 & 7.03\\CON8-2 & 807.92 & 0.20 & 
807.92 & 0.20 & \bf{712.89} & 
13.33 & 13.33\\CON8-3 & 879.74 & 0.14 & 
879.74 & 0.14 & \bf{811.07} & 
8.47 & 8.47\\CON8-4 & 864.34 & 0.07 & 
864.34 & 0.07 & \bf{772.25} & 
11.92 & 11.92\\CON8-5 & 847.65 & 0.07 & 
847.65 & 0.07 & \bf{754.88} & 
12.29 & 12.29\\CON8-6 & 739.72 & 0.12 & 
739.72 & 0.12 & \bf{678.92} & 
8.96 & 8.96\\CON8-7 & 857.41 & 0.15 & 
857.41 & 0.15 & \bf{811.96} & 
5.60 & 5.60\\CON8-8 & 865.40 & 0.10 & 
865.40 & 0.10 & \bf{767.53} & 
12.75 & 12.75\\CON8-9 & 931.13 & 0.10 & 
931.13 & 0.10 & \bf{809.00} & 
15.10 & 15.10\\\bf{PROM.} & 
\bf{817.03} & \bf{0.12} & \bf{817.03} & \bf{0.12} & \bf{758.54} & \bf{7.51} & \bf{7.51}\\[1ex]\hline
\end{tabular}
\label{table:nonlin}
\end{table} \clearpage
\begin{table}[ht]
\caption{Resultados de la ejecución de la metaheurística SCA, utilizando instancias de Dethloff con la configuración -n 50.0 -b 10 -y 0.1}
\centering
\small
\begin{tabular}{c c c c c c c c}
\hline\hline
Instancia & Costo mínimo & Tiempo(seg.) & Costo promedio & Tiempo promedio(seg.) & CME & \%G & \%GP \\ [0.5ex]
\hline
SCA3-0 & 636.06 & 2.50 & 
638.30 & 2.68 & \bf{635.62} & 
0.07 & 0.42\\SCA3-1 & \bf{697.84} & 1.94 & 
698.50 & 2.69 & 697.84 & 0.00
 & 0.10\\SCA3-2 & 661.13 & 2.27 & 
663.89 & 3.13 & \bf{659.34} & 
0.27 & 0.69\\SCA3-3 & 681.74 & 1.80 & 
681.74 & 2.74 & \bf{680.04} & 
0.25 & 0.25\\SCA3-4 & 692.57 & 2.71 & 
693.42 & 3.00 & \bf{690.50} & 
0.30 & 0.42\\SCA3-5 & 665.04 & 1.76 & 
674.17 & 2.27 & \bf{659.90} & 
0.78 & 2.16\\SCA3-6 & \bf{651.09} & 1.45 & 
651.09 & 1.38 & 651.09 & 0.00
 & 0.00\\
SCA3-7 & 671.67 & 3.22 & 
671.67 & 2.62 & \bf{659.17} & 
1.90 & 1.90\\SCA3-8 & \bf{719.47} & 3.34 & 
719.47 & 2.97 & 719.47 & 0.00
 & 0.00\\
SCA3-9 & 685.14 & 3.62 & 
685.75 & 2.83 & \bf{681.00} & 
0.61 & 0.70\\SCA8-0 & 974.40 & 10.60 & 
989.38 & 10.00 & \bf{961.50} & 
1.34 & 2.90\\SCA8-1 & 1057.41 & 11.32 & 
1067.80 & 11.60 & \bf{1049.65} & 
0.74 & 1.73\\SCA8-2 & 1053.00 & 10.54 & 
1053.44 & 10.48 & \bf{1039.64} & 
1.29 & 1.33\\SCA8-3 & 1019.57 & 6.69 & 
1022.20 & 7.71 & \bf{983.34} & 
3.68 & 3.95\\SCA8-4 & 1072.75 & 6.35 & 
1075.28 & 7.42 & \bf{1065.49} & 
0.68 & 0.92\\SCA8-5 & 1050.44 & 14.44 & 
1053.92 & 10.57 & \bf{1027.08} & 
2.27 & 2.61\\SCA8-6 & 972.48 & 12.97 & 
980.33 & 10.23 & \bf{971.82} & 
0.07 & 0.88\\SCA8-7 & 1063.60 & 9.24 & 
1068.39 & 10.22 & \bf{1051.28} & 
1.17 & 1.63\\SCA8-8 & 1085.98 & 6.58 & 
1090.76 & 6.97 & \bf{1071.18} & 
1.38 & 1.83\\SCA8-9 & 1070.34 & 9.07 & 
1080.14 & 9.61 & \bf{1060.50} & 
0.93 & 1.85\\CON3-0 & \bf{616.52} & 1.09 & 
616.52 & 1.08 & 616.52 & 0.00
 & 0.00\\
CON3-1 & 556.79 & 2.98 & 
559.21 & 3.24 & \bf{554.47} & 
0.42 & 0.86\\CON3-2 & 521.38 & 2.91 & 
521.57 & 2.69 & \bf{518.00} & 
0.65 & 0.69\\CON3-3 & \bf{591.19} & 3.39 & 
591.20 & 3.83 & 591.19 & 0.00
 & 0.00\\CON3-4 & \bf{588.79} & 2.14 & 
589.45 & 2.31 & 588.79 & 0.00
 & 0.11\\CON3-5 & 568.69 & 1.91 & 
568.69 & 1.97 & \bf{563.70} & 
0.89 & 0.89\\CON3-6 & 502.16 & 3.35 & 
502.16 & 4.05 & \bf{499.05} & 
0.62 & 0.62\\CON3-7 & 578.41 & 4.79 & 
582.64 & 3.87 & \bf{576.48} & 
0.33 & 1.07\\CON3-8 & 523.14 & 3.67 & 
523.66 & 2.95 & \bf{523.05} & 
0.02 & 0.12\\CON3-9 & 588.40 & 2.39 & 
588.55 & 2.38 & \bf{578.24} & 
1.76 & 1.78\\CON8-0 & 881.17 & 8.65 & 
885.22 & 8.59 & \bf{857.17} & 
2.80 & 3.27\\CON8-1 & 742.47 & 8.56 & 
747.61 & 11.40 & \bf{740.85} & 
0.22 & 0.91\\CON8-2 & 713.05 & 7.47 & 
716.73 & 9.12 & \bf{712.89} & 
0.02 & 0.54\\CON8-3 & 832.18 & 12.78 & 
835.44 & 11.52 & \bf{811.07} & 
2.60 & 3.00\\CON8-4 & 778.37 & 10.24 & 
786.81 & 8.44 & \bf{772.25} & 
0.79 & 1.89\\CON8-5 & 754.95 & 8.22 & 
758.88 & 10.62 & \bf{754.88} & 
0.01 & 0.53\\CON8-6 & 686.39 & 5.52 & 
695.23 & 6.82 & \bf{678.92} & 
1.10 & 2.40\\CON8-7 & 815.44 & 8.79 & 
817.42 & 10.06 & \bf{811.96} & 
0.43 & 0.67\\CON8-8 & 775.49 & 8.24 & 
781.99 & 7.36 & \bf{767.53} & 
1.04 & 1.88\\CON8-9 & 820.76 & 6.71 & 
829.91 & 7.77 & \bf{809.00} & 
1.45 & 2.58\\\bf{PROM.} & 
\bf{765.44} & \bf{5.91} & \bf{768.96} & \bf{6.03} & \bf{758.54} & \bf{0.82} & \bf{1.25}\\[1ex]\hline
\end{tabular}
\label{table:nonlin}
\end{table} \clearpage
\begin{table}[ht]
\caption{Resultados de la ejecución de la metaheurística SCA, utilizando instancias de Dethloff con la configuración -n 50.0 -b 10 -y .2}
\centering
\small
\begin{tabular}{c c c c c c c c}
\hline\hline
Instancia & Costo mínimo & Tiempo(seg.) & Costo promedio & Tiempo promedio(seg.) & CME & \%G & \%GP \\ [0.5ex]
\hline
SCA3-0 & 640.55 & 3.36 & 
640.55 & 3.36 & \bf{635.62} & 
0.78 & 0.78\\SCA3-1 & \bf{697.84} & 2.05 & 
699.77 & 2.08 & 697.84 & 0.00
 & 0.28\\SCA3-2 & 664.92 & 4.28 & 
668.82 & 3.85 & \bf{659.34} & 
0.85 & 1.44\\SCA3-3 & \bf{680.04} & 2.15 & 
680.46 & 3.28 & 680.04 & 0.00
 & 0.06\\SCA3-4 & \bf{690.50} & 2.56 & 
692.22 & 2.60 & 690.50 & 0.00
 & 0.25\\SCA3-5 & 665.04 & 3.74 & 
668.00 & 2.54 & \bf{659.90} & 
0.78 & 1.23\\SCA3-6 & 653.69 & 4.28 & 
655.68 & 3.50 & \bf{651.09} & 
0.40 & 0.70\\SCA3-7 & 666.15 & 3.20 & 
666.15 & 2.66 & \bf{659.17} & 
1.06 & 1.06\\SCA3-8 & \bf{719.47} & 4.38 & 
719.47 & 3.40 & 719.47 & 0.00
 & 0.00\\
SCA3-9 & \bf{681.00} & 3.36 & 
683.07 & 3.60 & 681.00 & 0.00
 & 0.30\\SCA8-0 & 979.79 & 6.90 & 
989.39 & 8.66 & \bf{961.50} & 
1.90 & 2.90\\SCA8-1 & 1060.62 & 7.82 & 
1068.43 & 8.72 & \bf{1049.65} & 
1.05 & 1.79\\SCA8-2 & 1050.98 & 12.34 & 
1052.92 & 10.94 & \bf{1039.64} & 
1.09 & 1.28\\SCA8-3 & 1022.82 & 10.04 & 
1025.50 & 9.21 & \bf{983.34} & 
4.01 & 4.29\\SCA8-4 & \bf{1065.49} & 8.04 & 
1069.55 & 8.97 & 1065.49 & 0.00
 & 0.38\\SCA8-5 & 1042.64 & 15.44 & 
1047.34 & 24.25 & \bf{1027.08} & 
1.51 & 1.97\\SCA8-6 & 972.48 & 11.12 & 
974.03 & 10.94 & \bf{971.82} & 
0.07 & 0.23\\SCA8-7 & 1067.88 & 9.81 & 
1073.60 & 9.16 & \bf{1051.28} & 
1.58 & 2.12\\SCA8-8 & \bf{1071.18} & 6.28 & 
1087.60 & 8.86 & 1071.18 & 0.00
 & 1.53\\SCA8-9 & 1070.34 & 7.15 & 
1073.62 & 7.56 & \bf{1060.50} & 
0.93 & 1.24\\CON3-0 & 617.59 & 1.76 & 
621.24 & 1.69 & \bf{616.52} & 
0.17 & 0.77\\CON3-1 & 558.83 & 3.22 & 
560.27 & 3.98 & \bf{554.47} & 
0.79 & 1.05\\CON3-2 & 521.38 & 2.22 & 
521.50 & 2.42 & \bf{518.00} & 
0.65 & 0.68\\CON3-3 & \bf{591.19} & 2.46 & 
591.20 & 2.33 & 591.19 & 0.00
 & 0.00\\CON3-4 & 591.43 & 3.76 & 
591.43 & 3.35 & \bf{588.79} & 
0.45 & 0.45\\CON3-5 & \bf{563.70} & 1.61 & 
564.16 & 1.74 & 563.70 & 0.00
 & 0.08\\CON3-6 & 502.16 & 2.16 & 
502.16 & 2.77 & \bf{499.05} & 
0.62 & 0.62\\CON3-7 & 578.41 & 3.26 & 
583.52 & 2.90 & \bf{576.48} & 
0.33 & 1.22\\CON3-8 & 524.59 & 3.58 & 
524.59 & 3.43 & \bf{523.05} & 
0.29 & 0.29\\CON3-9 & 588.40 & 2.36 & 
588.40 & 2.22 & \bf{578.24} & 
1.76 & 1.76\\CON8-0 & 879.99 & 6.52 & 
888.25 & 7.79 & \bf{857.17} & 
2.66 & 3.63\\CON8-1 & 748.85 & 7.87 & 
753.01 & 7.68 & \bf{740.85} & 
1.08 & 1.64\\CON8-2 & 713.60 & 9.36 & 
719.89 & 8.80 & \bf{712.89} & 
0.10 & 0.98\\CON8-3 & 831.87 & 6.25 & 
834.52 & 8.30 & \bf{811.07} & 
2.56 & 2.89\\CON8-4 & 790.99 & 9.35 & 
793.07 & 11.02 & \bf{772.25} & 
2.43 & 2.70\\CON8-5 & \bf{754.88} & 11.90 & 
759.90 & 10.39 & 754.88 & 0.00
 & 0.67\\CON8-6 & 686.81 & 5.86 & 
693.37 & 6.48 & \bf{678.92} & 
1.16 & 2.13\\CON8-7 & 814.79 & 11.74 & 
817.04 & 10.49 & \bf{811.96} & 
0.35 & 0.63\\CON8-8 & 779.18 & 6.95 & 
785.61 & 7.20 & \bf{767.53} & 
1.52 & 2.36\\CON8-9 & 814.61 & 7.25 & 
822.23 & 8.23 & \bf{809.00} & 
0.69 & 1.64\\\bf{PROM.} & 
\bf{765.42} & \bf{5.94} & \bf{768.79} & \bf{6.28} & \bf{758.54} & \bf{0.84} & \bf{1.25}\\[1ex]\hline
\end{tabular}
\label{table:nonlin}
\end{table} \clearpage
\begin{table}[ht]
\caption{Resultados de la ejecución de la metaheurística SCA, utilizando instancias de Dethloff con la configuración -n 50.0 -b 10 -y .3}
\centering
\small
\begin{tabular}{c c c c c c c c}
\hline\hline
Instancia & Costo mínimo & Tiempo(seg.) & Costo promedio & Tiempo promedio(seg.) & CME & \%G & \%GP \\ [0.5ex]
\hline
SCA3-0 & 640.55 & 3.66 & 
640.55 & 3.53 & \bf{635.62} & 
0.78 & 0.78\\SCA3-1 & \bf{697.84} & 3.31 & 
697.84 & 3.36 & 697.84 & 0.00
 & 0.00\\
SCA3-2 & \bf{659.34} & 2.47 & 
660.24 & 3.60 & 659.34 & 0.00
 & 0.14\\SCA3-3 & 680.60 & 2.94 & 
680.60 & 3.48 & \bf{680.04} & 
0.08 & 0.08\\SCA3-4 & \bf{690.50} & 2.92 & 
691.70 & 2.77 & 690.50 & 0.00
 & 0.17\\SCA3-5 & 673.56 & 1.72 & 
674.75 & 2.03 & \bf{659.90} & 
2.07 & 2.25\\SCA3-6 & 653.83 & 0.97 & 
656.10 & 1.94 & \bf{651.09} & 
0.42 & 0.77\\SCA3-7 & 671.67 & 2.90 & 
671.67 & 2.35 & \bf{659.17} & 
1.90 & 1.90\\SCA3-8 & \bf{719.47} & 2.27 & 
719.54 & 3.23 & 719.47 & 0.00
 & 0.01\\SCA3-9 & \bf{681.00} & 3.91 & 
682.03 & 3.74 & 681.00 & 0.00
 & 0.15\\SCA8-0 & 976.24 & 10.00 & 
986.10 & 8.20 & \bf{961.50} & 
1.53 & 2.56\\SCA8-1 & 1057.41 & 8.79 & 
1073.07 & 16.31 & \bf{1049.65} & 
0.74 & 2.23\\SCA8-2 & 1051.21 & 10.12 & 
1052.17 & 12.45 & \bf{1039.64} & 
1.11 & 1.21\\SCA8-3 & 1013.56 & 7.94 & 
1024.58 & 9.99 & \bf{983.34} & 
3.07 & 4.19\\SCA8-4 & 1069.71 & 8.30 & 
1072.76 & 8.06 & \bf{1065.49} & 
0.40 & 0.68\\SCA8-5 & 1043.05 & 10.16 & 
1047.02 & 11.15 & \bf{1027.08} & 
1.55 & 1.94\\SCA8-6 & 973.30 & 10.56 & 
980.62 & 11.16 & \bf{971.82} & 
0.15 & 0.91\\SCA8-7 & 1067.11 & 8.12 & 
1073.94 & 8.96 & \bf{1051.28} & 
1.51 & 2.16\\SCA8-8 & 1075.00 & 10.83 & 
1086.30 & 10.09 & \bf{1071.18} & 
0.36 & 1.41\\SCA8-9 & 1073.62 & 9.90 & 
1079.99 & 9.20 & \bf{1060.50} & 
1.24 & 1.84\\CON3-0 & 624.84 & 1.66 & 
624.90 & 1.71 & \bf{616.52} & 
1.35 & 1.36\\CON3-1 & 557.21 & 2.98 & 
559.87 & 3.64 & \bf{554.47} & 
0.49 & 0.97\\CON3-2 & 521.38 & 2.29 & 
521.38 & 1.36 & \bf{518.00} & 
0.65 & 0.65\\CON3-3 & \bf{591.19} & 4.89 & 
591.20 & 3.46 & 591.19 & 0.00
 & 0.00\\CON3-4 & 591.43 & 2.20 & 
591.43 & 3.08 & \bf{588.79} & 
0.45 & 0.45\\CON3-5 & 564.88 & 4.64 & 
566.01 & 3.09 & \bf{563.70} & 
0.21 & 0.41\\CON3-6 & 502.16 & 3.43 & 
502.56 & 2.94 & \bf{499.05} & 
0.62 & 0.70\\CON3-7 & 586.01 & 3.00 & 
586.01 & 2.67 & \bf{576.48} & 
1.65 & 1.65\\CON3-8 & \bf{523.05} & 1.92 & 
524.51 & 2.65 & 523.05 & 0.00
 & 0.28\\CON3-9 & 588.11 & 2.50 & 
588.34 & 2.40 & \bf{578.24} & 
1.71 & 1.75\\CON8-0 & 878.44 & 6.34 & 
882.91 & 6.75 & \bf{857.17} & 
2.48 & 3.00\\CON8-1 & 750.06 & 8.81 & 
757.36 & 7.45 & \bf{740.85} & 
1.24 & 2.23\\CON8-2 & \bf{712.89} & 14.23 & 
719.96 & 11.09 & 712.89 & 0.00
 & 0.99\\CON8-3 & 831.15 & 7.73 & 
833.27 & 9.16 & \bf{811.07} & 
2.48 & 2.74\\CON8-4 & 790.29 & 12.55 & 
792.25 & 9.91 & \bf{772.25} & 
2.34 & 2.59\\CON8-5 & 763.13 & 8.16 & 
768.50 & 9.39 & \bf{754.88} & 
1.09 & 1.80\\CON8-6 & 695.68 & 6.40 & 
698.03 & 7.96 & \bf{678.92} & 
2.47 & 2.82\\CON8-7 & 815.06 & 9.68 & 
817.17 & 12.09 & \bf{811.96} & 
0.38 & 0.64\\CON8-8 & 784.24 & 6.97 & 
785.79 & 7.90 & \bf{767.53} & 
2.18 & 2.38\\CON8-9 & 811.66 & 7.94 & 
823.83 & 6.79 & \bf{809.00} & 
0.33 & 1.83\\\bf{PROM.} & 
\bf{766.29} & \bf{6.00} & \bf{769.67} & \bf{6.28} & \bf{758.54} & \bf{0.98} & \bf{1.37}\\[1ex]\hline
\end{tabular}
\label{table:nonlin}
\end{table} \clearpage
\begin{table}[ht]
\caption{Resultados de la ejecución de la metaheurística SCA, utilizando instancias de Dethloff con la configuración -n 50.0 -b 10 -y .4}
\centering
\small
\begin{tabular}{c c c c c c c c}
\hline\hline
Instancia & Costo mínimo & Tiempo(seg.) & Costo promedio & Tiempo promedio(seg.) & CME & \%G & \%GP \\ [0.5ex]
\hline
SCA3-0 & 640.55 & 1.76 & 
641.77 & 2.99 & \bf{635.62} & 
0.78 & 0.97\\SCA3-1 & 701.53 & 2.92 & 
701.78 & 3.00 & \bf{697.84} & 
0.53 & 0.56\\SCA3-2 & 661.13 & 3.97 & 
661.13 & 3.62 & \bf{659.34} & 
0.27 & 0.27\\SCA3-3 & 681.16 & 2.97 & 
681.45 & 2.93 & \bf{680.04} & 
0.16 & 0.21\\SCA3-4 & \bf{690.50} & 3.62 & 
691.02 & 4.25 & 690.50 & 0.00
 & 0.08\\SCA3-5 & \bf{659.90} & 2.40 & 
667.73 & 2.40 & 659.90 & 0.00
 & 1.19\\SCA3-6 & \bf{651.09} & 5.41 & 
653.81 & 4.08 & 651.09 & 0.00
 & 0.42\\SCA3-7 & 666.15 & 2.28 & 
669.07 & 2.46 & \bf{659.17} & 
1.06 & 1.50\\SCA3-8 & \bf{719.47} & 3.04 & 
720.34 & 3.38 & 719.47 & 0.00
 & 0.12\\SCA3-9 & \bf{681.00} & 4.40 & 
684.14 & 3.60 & 681.00 & 0.00
 & 0.46\\SCA8-0 & 976.39 & 7.90 & 
984.41 & 7.39 & \bf{961.50} & 
1.55 & 2.38\\SCA8-1 & 1067.02 & 8.24 & 
1071.96 & 8.21 & \bf{1049.65} & 
1.65 & 2.13\\SCA8-2 & 1051.55 & 10.55 & 
1052.74 & 11.89 & \bf{1039.64} & 
1.15 & 1.26\\SCA8-3 & 1008.67 & 10.74 & 
1022.64 & 8.76 & \bf{983.34} & 
2.58 & 4.00\\SCA8-4 & 1067.29 & 11.06 & 
1083.35 & 9.76 & \bf{1065.49} & 
0.17 & 1.68\\SCA8-5 & 1043.05 & 14.15 & 
1051.82 & 11.39 & \bf{1027.08} & 
1.55 & 2.41\\SCA8-6 & 972.48 & 15.19 & 
978.12 & 12.40 & \bf{971.82} & 
0.07 & 0.65\\SCA8-7 & 1071.08 & 7.18 & 
1077.59 & 11.07 & \bf{1051.28} & 
1.88 & 2.50\\SCA8-8 & \bf{1071.18} & 13.43 & 
1078.39 & 10.59 & 1071.18 & 0.00
 & 0.67\\SCA8-9 & 1072.10 & 9.57 & 
1074.66 & 9.39 & \bf{1060.50} & 
1.09 & 1.34\\CON3-0 & 619.09 & 2.06 & 
626.69 & 1.39 & \bf{616.52} & 
0.42 & 1.65\\CON3-1 & 559.72 & 2.19 & 
560.24 & 2.72 & \bf{554.47} & 
0.95 & 1.04\\CON3-2 & 521.38 & 2.76 & 
521.38 & 3.84 & \bf{518.00} & 
0.65 & 0.65\\CON3-3 & \bf{591.19} & 4.23 & 
591.27 & 3.84 & 591.19 & 0.00
 & 0.01\\CON3-4 & 591.43 & 1.66 & 
591.43 & 2.02 & \bf{588.79} & 
0.45 & 0.45\\CON3-5 & \bf{563.70} & 2.09 & 
566.15 & 2.98 & 563.70 & 0.00
 & 0.43\\CON3-6 & 500.80 & 2.44 & 
502.31 & 2.43 & \bf{499.05} & 
0.35 & 0.65\\CON3-7 & 577.54 & 3.37 & 
581.77 & 3.35 & \bf{576.48} & 
0.18 & 0.92\\CON3-8 & \bf{523.05} & 2.43 & 
524.48 & 2.14 & 523.05 & 0.00
 & 0.27\\CON3-9 & 588.40 & 2.10 & 
588.40 & 2.49 & \bf{578.24} & 
1.76 & 1.76\\CON8-0 & 862.49 & 7.77 & 
877.48 & 9.33 & \bf{857.17} & 
0.62 & 2.37\\CON8-1 & 741.70 & 6.87 & 
753.43 & 8.29 & \bf{740.85} & 
0.11 & 1.70\\CON8-2 & 713.60 & 7.82 & 
716.51 & 8.75 & \bf{712.89} & 
0.10 & 0.51\\CON8-3 & 819.73 & 11.56 & 
828.64 & 11.46 & \bf{811.07} & 
1.07 & 2.17\\CON8-4 & 775.73 & 13.14 & 
782.24 & 12.03 & \bf{772.25} & 
0.45 & 1.29\\CON8-5 & 755.86 & 13.32 & 
760.87 & 12.34 & \bf{754.88} & 
0.13 & 0.79\\CON8-6 & 684.90 & 6.78 & 
689.67 & 7.84 & \bf{678.92} & 
0.88 & 1.58\\CON8-7 & 814.50 & 13.27 & 
815.50 & 12.01 & \bf{811.96} & 
0.31 & 0.44\\CON8-8 & 784.89 & 6.95 & 
785.93 & 7.98 & \bf{767.53} & 
2.26 & 2.40\\CON8-9 & 827.05 & 8.90 & 
828.80 & 9.01 & \bf{809.00} & 
2.23 & 2.45\\\bf{PROM.} & 
\bf{764.25} & \bf{6.56} & \bf{768.53} & \bf{6.49} & \bf{758.54} & \bf{0.69} & \bf{1.21}\\[1ex]\hline
\end{tabular}
\label{table:nonlin}
\end{table} \clearpage
\begin{table}[ht]
\caption{Resultados de la ejecución de la metaheurística SCA, utilizando instancias de Dethloff con la configuración -n 50.0 -b 10 -y .5}
\centering
\small
\begin{tabular}{c c c c c c c c}
\hline\hline
Instancia & Costo mínimo & Tiempo(seg.) & Costo promedio & Tiempo promedio(seg.) & CME & \%G & \%GP \\ [0.5ex]
\hline
SCA3-0 & 640.55 & 3.40 & 
640.55 & 3.49 & \bf{635.62} & 
0.78 & 0.78\\SCA3-1 & \bf{697.84} & 2.11 & 
697.84 & 2.14 & 697.84 & 0.00
 & 0.00\\
SCA3-2 & 661.13 & 3.04 & 
664.06 & 3.80 & \bf{659.34} & 
0.27 & 0.72\\SCA3-3 & 680.60 & 3.24 & 
680.60 & 2.78 & \bf{680.04} & 
0.08 & 0.08\\SCA3-4 & \bf{690.50} & 3.09 & 
690.50 & 2.30 & 690.50 & 0.00
 & 0.00\\
SCA3-5 & 674.01 & 2.14 & 
679.03 & 2.36 & \bf{659.90} & 
2.14 & 2.90\\SCA3-6 & 652.94 & 3.93 & 
654.55 & 4.12 & \bf{651.09} & 
0.28 & 0.53\\SCA3-7 & 669.89 & 2.69 & 
670.78 & 2.54 & \bf{659.17} & 
1.63 & 1.76\\SCA3-8 & \bf{719.47} & 3.95 & 
719.47 & 3.44 & 719.47 & 0.00
 & 0.00\\
SCA3-9 & \bf{681.00} & 3.73 & 
682.00 & 3.33 & 681.00 & 0.00
 & 0.15\\SCA8-0 & 982.03 & 8.08 & 
983.96 & 7.83 & \bf{961.50} & 
2.14 & 2.34\\SCA8-1 & 1053.90 & 12.40 & 
1077.41 & 9.87 & \bf{1049.65} & 
0.40 & 2.64\\SCA8-2 & 1051.95 & 8.87 & 
1052.45 & 9.72 & \bf{1039.64} & 
1.18 & 1.23\\SCA8-3 & 1024.97 & 7.12 & 
1032.10 & 8.09 & \bf{983.34} & 
4.23 & 4.96\\SCA8-4 & 1069.30 & 36.13 & 
1088.76 & 16.74 & \bf{1065.49} & 
0.36 & 2.18\\SCA8-5 & 1050.64 & 8.80 & 
1057.90 & 8.89 & \bf{1027.08} & 
2.29 & 3.00\\SCA8-6 & 972.48 & 14.98 & 
975.41 & 13.62 & \bf{971.82} & 
0.07 & 0.37\\SCA8-7 & 1066.82 & 12.52 & 
1069.57 & 10.88 & \bf{1051.28} & 
1.48 & 1.74\\SCA8-8 & 1085.98 & 15.17 & 
1091.49 & 10.56 & \bf{1071.18} & 
1.38 & 1.90\\SCA8-9 & 1072.10 & 9.98 & 
1078.36 & 9.43 & \bf{1060.50} & 
1.09 & 1.68\\CON3-0 & 617.59 & 3.15 & 
621.36 & 2.47 & \bf{616.52} & 
0.17 & 0.78\\CON3-1 & 560.75 & 3.23 & 
560.75 & 3.29 & \bf{554.47} & 
1.13 & 1.13\\CON3-2 & 521.38 & 3.42 & 
521.38 & 2.88 & \bf{518.00} & 
0.65 & 0.65\\CON3-3 & 591.20 & 2.75 & 
591.20 & 4.99 & \bf{591.19} & 
0.00 & 0.00\\CON3-4 & 591.43 & 2.63 & 
591.43 & 3.13 & \bf{588.79} & 
0.45 & 0.45\\CON3-5 & 564.88 & 3.11 & 
565.92 & 2.05 & \bf{563.70} & 
0.21 & 0.39\\CON3-6 & 502.16 & 3.01 & 
504.25 & 2.71 & \bf{499.05} & 
0.62 & 1.04\\CON3-7 & 586.01 & 2.84 & 
586.01 & 3.26 & \bf{576.48} & 
1.65 & 1.65\\CON3-8 & 523.68 & 1.90 & 
525.49 & 2.19 & \bf{523.05} & 
0.12 & 0.47\\CON3-9 & 588.38 & 2.63 & 
588.53 & 2.87 & \bf{578.24} & 
1.75 & 1.78\\CON8-0 & 881.90 & 8.96 & 
886.37 & 8.28 & \bf{857.17} & 
2.89 & 3.41\\CON8-1 & 743.42 & 10.49 & 
747.79 & 11.10 & \bf{740.85} & 
0.35 & 0.94\\CON8-2 & 713.05 & 13.06 & 
716.27 & 11.59 & \bf{712.89} & 
0.02 & 0.47\\CON8-3 & 817.57 & 10.56 & 
830.30 & 8.79 & \bf{811.07} & 
0.80 & 2.37\\CON8-4 & 782.59 & 8.43 & 
787.20 & 9.34 & \bf{772.25} & 
1.34 & 1.94\\CON8-5 & 760.41 & 11.82 & 
765.65 & 9.86 & \bf{754.88} & 
0.73 & 1.43\\CON8-6 & 696.64 & 9.43 & 
698.09 & 9.70 & \bf{678.92} & 
2.61 & 2.82\\CON8-7 & 814.50 & 13.93 & 
816.76 & 20.17 & \bf{811.96} & 
0.31 & 0.59\\CON8-8 & 771.32 & 5.62 & 
783.38 & 6.70 & \bf{767.53} & 
0.49 & 2.07\\CON8-9 & 813.57 & 9.50 & 
822.80 & 8.44 & \bf{809.00} & 
0.56 & 1.71\\\bf{PROM.} & 
\bf{766.01} & \bf{7.40} & \bf{769.94} & \bf{6.74} & \bf{758.54} & \bf{0.92} & \bf{1.38}\\[1ex]\hline
\end{tabular}
\label{table:nonlin}
\end{table} \clearpage
\begin{table}[ht]
\caption{Resultados de la ejecución de la metaheurística SCA, utilizando instancias de Dethloff con la configuración -n 50.0 -b 10 -y .6}
\centering
\small
\begin{tabular}{c c c c c c c c}
\hline\hline
Instancia & Costo mínimo & Tiempo(seg.) & Costo promedio & Tiempo promedio(seg.) & CME & \%G & \%GP \\ [0.5ex]
\hline
SCA3-0 & 640.55 & 3.38 & 
640.55 & 3.95 & \bf{635.62} & 
0.78 & 0.78\\SCA3-1 & \bf{697.84} & 3.26 & 
698.50 & 2.83 & 697.84 & 0.00
 & 0.10\\SCA3-2 & \bf{659.34} & 4.27 & 
664.84 & 3.93 & 659.34 & 0.00
 & 0.83\\SCA3-3 & \bf{680.04} & 2.28 & 
680.46 & 3.19 & 680.04 & 0.00
 & 0.06\\SCA3-4 & \bf{690.50} & 3.58 & 
692.38 & 3.38 & 690.50 & 0.00
 & 0.27\\SCA3-5 & 678.28 & 3.50 & 
679.81 & 3.73 & \bf{659.90} & 
2.79 & 3.02\\SCA3-6 & \bf{651.09} & 3.18 & 
652.70 & 3.72 & 651.09 & 0.00
 & 0.25\\SCA3-7 & 669.89 & 2.96 & 
671.23 & 2.69 & \bf{659.17} & 
1.63 & 1.83\\SCA3-8 & 722.05 & 2.90 & 
721.29 & 3.18 & \bf{719.47} & 
0.36 & 0.25\\SCA3-9 & \bf{681.00} & 5.46 & 
681.00 & 3.21 & 681.00 & 0.00
 & 0.00\\
SCA8-0 & 990.41 & 10.45 & 
994.14 & 8.74 & \bf{961.50} & 
3.01 & 3.39\\SCA8-1 & 1053.99 & 10.86 & 
1066.40 & 10.00 & \bf{1049.65} & 
0.41 & 1.60\\SCA8-2 & 1050.37 & 9.40 & 
1053.19 & 13.04 & \bf{1039.64} & 
1.03 & 1.30\\SCA8-3 & 1023.12 & 9.21 & 
1031.16 & 8.58 & \bf{983.34} & 
4.05 & 4.86\\SCA8-4 & 1067.55 & 10.32 & 
1070.62 & 11.63 & \bf{1065.49} & 
0.19 & 0.48\\SCA8-5 & 1043.05 & 14.48 & 
1046.56 & 12.35 & \bf{1027.08} & 
1.55 & 1.90\\SCA8-6 & 972.48 & 13.98 & 
974.87 & 13.30 & \bf{971.82} & 
0.07 & 0.31\\SCA8-7 & 1063.60 & 9.18 & 
1068.40 & 9.93 & \bf{1051.28} & 
1.17 & 1.63\\SCA8-8 & 1093.38 & 9.44 & 
1094.40 & 9.03 & \bf{1071.18} & 
2.07 & 2.17\\SCA8-9 & 1078.30 & 7.23 & 
1085.01 & 9.62 & \bf{1060.50} & 
1.68 & 2.31\\CON3-0 & 622.84 & 2.25 & 
622.84 & 1.68 & \bf{616.52} & 
1.03 & 1.03\\CON3-1 & 558.34 & 4.43 & 
558.34 & 2.69 & \bf{554.47} & 
0.70 & 0.70\\CON3-2 & 521.38 & 2.23 & 
525.72 & 3.35 & \bf{518.00} & 
0.65 & 1.49\\CON3-3 & \bf{591.19} & 3.68 & 
591.20 & 3.16 & 591.19 & 0.00
 & 0.00\\CON3-4 & 591.43 & 3.80 & 
591.43 & 3.68 & \bf{588.79} & 
0.45 & 0.45\\CON3-5 & 564.89 & 2.58 & 
566.18 & 2.66 & \bf{563.70} & 
0.21 & 0.44\\CON3-6 & 502.16 & 4.15 & 
502.16 & 3.59 & \bf{499.05} & 
0.62 & 0.62\\CON3-7 & 586.01 & 3.56 & 
586.01 & 2.88 & \bf{576.48} & 
1.65 & 1.65\\CON3-8 & 523.68 & 2.31 & 
523.68 & 2.46 & \bf{523.05} & 
0.12 & 0.12\\CON3-9 & 580.58 & 2.15 & 
586.45 & 4.11 & \bf{578.24} & 
0.40 & 1.42\\CON8-0 & 858.63 & 10.02 & 
880.80 & 8.75 & \bf{857.17} & 
0.17 & 2.76\\CON8-1 & 741.70 & 12.29 & 
749.70 & 10.95 & \bf{740.85} & 
0.11 & 1.19\\CON8-2 & 713.60 & 10.70 & 
714.33 & 11.24 & \bf{712.89} & 
0.10 & 0.20\\CON8-3 & 820.31 & 11.04 & 
831.55 & 9.59 & \bf{811.07} & 
1.14 & 2.53\\CON8-4 & \bf{772.25} & 10.22 & 
780.92 & 9.96 & 772.25 & 0.00
 & 1.12\\CON8-5 & 760.41 & 12.20 & 
766.50 & 9.98 & \bf{754.88} & 
0.73 & 1.54\\CON8-6 & 687.83 & 8.00 & 
692.81 & 7.34 & \bf{678.92} & 
1.31 & 2.05\\CON8-7 & 814.79 & 11.63 & 
817.33 & 10.74 & \bf{811.96} & 
0.35 & 0.66\\CON8-8 & 781.51 & 7.88 & 
789.36 & 6.87 & \bf{767.53} & 
1.82 & 2.84\\CON8-9 & 819.55 & 9.74 & 
822.47 & 11.13 & \bf{809.00} & 
1.30 & 1.67\\\bf{PROM.} & 
\bf{765.50} & \bf{6.85} & \bf{769.18} & \bf{6.67} & \bf{758.54} & \bf{0.84} & \bf{1.30}\\[1ex]\hline
\end{tabular}
\label{table:nonlin}
\end{table} \clearpage
\begin{table}[ht]
\caption{Resultados de la ejecución de la metaheurística SCA, utilizando instancias de Dethloff con la configuración -n 50.0 -b 10 -y .7}
\centering
\small
\begin{tabular}{c c c c c c c c}
\hline\hline
Instancia & Costo mínimo & Tiempo(seg.) & Costo promedio & Tiempo promedio(seg.) & CME & \%G & \%GP \\ [0.5ex]
\hline
SCA3-0 & 640.55 & 2.39 & 
640.55 & 3.02 & \bf{635.62} & 
0.78 & 0.78\\SCA3-1 & 701.53 & 2.11 & 
701.53 & 2.85 & \bf{697.84} & 
0.53 & 0.53\\SCA3-2 & 664.18 & 3.06 & 
665.48 & 3.98 & \bf{659.34} & 
0.73 & 0.93\\SCA3-3 & 680.60 & 3.68 & 
680.96 & 3.51 & \bf{680.04} & 
0.08 & 0.13\\SCA3-4 & \bf{690.50} & 3.79 & 
692.67 & 3.29 & 690.50 & 0.00
 & 0.31\\SCA3-5 & 673.61 & 2.13 & 
677.38 & 3.10 & \bf{659.90} & 
2.08 & 2.65\\SCA3-6 & 653.93 & 3.28 & 
655.24 & 3.19 & \bf{651.09} & 
0.44 & 0.64\\SCA3-7 & 666.60 & 3.16 & 
670.40 & 3.66 & \bf{659.17} & 
1.13 & 1.70\\SCA3-8 & \bf{719.47} & 2.42 & 
719.47 & 2.48 & 719.47 & 0.00
 & 0.00\\
SCA3-9 & \bf{681.00} & 2.54 & 
684.11 & 2.49 & 681.00 & 0.00
 & 0.46\\SCA8-0 & 973.22 & 13.54 & 
987.91 & 9.92 & \bf{961.50} & 
1.22 & 2.75\\SCA8-1 & 1056.87 & 10.30 & 
1067.00 & 11.21 & \bf{1049.65} & 
0.69 & 1.65\\SCA8-2 & 1050.37 & 10.35 & 
1052.55 & 10.57 & \bf{1039.64} & 
1.03 & 1.24\\SCA8-3 & 1020.80 & 10.38 & 
1034.09 & 9.27 & \bf{983.34} & 
3.81 & 5.16\\SCA8-4 & 1079.30 & 10.03 & 
1091.22 & 9.09 & \bf{1065.49} & 
1.30 & 2.41\\SCA8-5 & 1050.64 & 12.52 & 
1056.38 & 11.22 & \bf{1027.08} & 
2.29 & 2.85\\SCA8-6 & 972.48 & 13.62 & 
974.03 & 10.57 & \bf{971.82} & 
0.07 & 0.23\\SCA8-7 & 1063.22 & 7.70 & 
1072.56 & 9.01 & \bf{1051.28} & 
1.14 & 2.02\\SCA8-8 & 1084.41 & 13.82 & 
1094.03 & 13.49 & \bf{1071.18} & 
1.24 & 2.13\\SCA8-9 & 1070.36 & 8.27 & 
1082.16 & 9.65 & \bf{1060.50} & 
0.93 & 2.04\\CON3-0 & 630.73 & 2.34 & 
631.36 & 2.39 & \bf{616.52} & 
2.30 & 2.41\\CON3-1 & 560.75 & 3.38 & 
560.75 & 2.90 & \bf{554.47} & 
1.13 & 1.13\\CON3-2 & 524.58 & 3.16 & 
526.19 & 3.57 & \bf{518.00} & 
1.27 & 1.58\\CON3-3 & \bf{591.19} & 2.66 & 
591.19 & 2.60 & 591.19 & 0.00
 & 0.00\\
CON3-4 & \bf{588.79} & 4.21 & 
589.45 & 3.76 & 588.79 & 0.00
 & 0.11\\CON3-5 & 564.88 & 3.92 & 
564.88 & 3.81 & \bf{563.70} & 
0.21 & 0.21\\CON3-6 & 502.16 & 3.43 & 
502.97 & 2.85 & \bf{499.05} & 
0.62 & 0.79\\CON3-7 & 586.01 & 3.97 & 
586.01 & 3.81 & \bf{576.48} & 
1.65 & 1.65\\CON3-8 & \bf{523.05} & 2.34 & 
523.15 & 2.54 & 523.05 & 0.00
 & 0.02\\CON3-9 & 588.40 & 2.65 & 
588.40 & 3.04 & \bf{578.24} & 
1.76 & 1.76\\CON8-0 & 866.43 & 6.22 & 
883.14 & 10.21 & \bf{857.17} & 
1.08 & 3.03\\CON8-1 & 742.47 & 8.60 & 
747.93 & 10.61 & \bf{740.85} & 
0.22 & 0.96\\CON8-2 & 713.60 & 10.28 & 
717.72 & 18.06 & \bf{712.89} & 
0.10 & 0.68\\CON8-3 & 835.86 & 12.42 & 
837.50 & 10.49 & \bf{811.07} & 
3.06 & 3.26\\CON8-4 & 782.41 & 9.42 & 
786.64 & 10.27 & \bf{772.25} & 
1.32 & 1.86\\CON8-5 & 762.24 & 12.03 & 
764.68 & 10.32 & \bf{754.88} & 
0.97 & 1.30\\CON8-6 & 693.53 & 8.59 & 
696.49 & 10.48 & \bf{678.92} & 
2.15 & 2.59\\CON8-7 & 814.77 & 9.68 & 
820.08 & 10.25 & \bf{811.96} & 
0.35 & 1.00\\CON8-8 & 783.08 & 7.10 & 
783.63 & 6.80 & \bf{767.53} & 
2.03 & 2.10\\CON8-9 & 814.50 & 7.24 & 
821.71 & 8.97 & \bf{809.00} & 
0.68 & 1.57\\\bf{PROM.} & 
\bf{766.58} & \bf{6.57} & \bf{770.59} & \bf{6.83} & \bf{758.54} & \bf{1.01} & \bf{1.47}\\[1ex]\hline
\end{tabular}
\label{table:nonlin}
\end{table} \clearpage
\begin{table}[ht]
\caption{Resultados de la ejecución de la metaheurística SCA, utilizando instancias de Dethloff con la configuración -n 50.0 -b 10 -y .8}
\centering
\small
\begin{tabular}{c c c c c c c c}
\hline\hline
Instancia & Costo mínimo & Tiempo(seg.) & Costo promedio & Tiempo promedio(seg.) & CME & \%G & \%GP \\ [0.5ex]
\hline
SCA3-0 & 640.55 & 3.04 & 
640.55 & 3.71 & \bf{635.62} & 
0.78 & 0.78\\SCA3-1 & \bf{697.84} & 2.21 & 
698.82 & 3.24 & 697.84 & 0.00
 & 0.14\\SCA3-2 & 664.92 & 4.03 & 
666.55 & 3.67 & \bf{659.34} & 
0.85 & 1.09\\SCA3-3 & 681.74 & 5.04 & 
681.74 & 3.00 & \bf{680.04} & 
0.25 & 0.25\\SCA3-4 & \bf{690.50} & 3.58 & 
691.02 & 3.92 & 690.50 & 0.00
 & 0.08\\SCA3-5 & 665.04 & 2.72 & 
665.04 & 3.24 & \bf{659.90} & 
0.78 & 0.78\\SCA3-6 & 652.94 & 2.09 & 
653.50 & 2.44 & \bf{651.09} & 
0.28 & 0.37\\SCA3-7 & 666.60 & 3.50 & 
670.43 & 3.11 & \bf{659.17} & 
1.13 & 1.71\\SCA3-8 & \bf{719.47} & 5.42 & 
719.47 & 4.49 & 719.47 & 0.00
 & 0.00\\
SCA3-9 & \bf{681.00} & 3.68 & 
683.44 & 3.76 & 681.00 & 0.00
 & 0.36\\SCA8-0 & 980.64 & 11.99 & 
988.92 & 10.01 & \bf{961.50} & 
1.99 & 2.85\\SCA8-1 & 1050.93 & 10.76 & 
1070.08 & 10.25 & \bf{1049.65} & 
0.12 & 1.95\\SCA8-2 & 1051.42 & 15.30 & 
1053.25 & 13.74 & \bf{1039.64} & 
1.13 & 1.31\\SCA8-3 & 998.53 & 10.08 & 
998.53 & 8.11 & \bf{983.34} & 
1.54 & 1.54\\SCA8-4 & 1067.55 & 9.57 & 
1080.38 & 10.25 & \bf{1065.49} & 
0.19 & 1.40\\SCA8-5 & 1053.09 & 15.98 & 
1061.45 & 10.44 & \bf{1027.08} & 
2.53 & 3.35\\SCA8-6 & 972.48 & 9.11 & 
977.46 & 11.15 & \bf{971.82} & 
0.07 & 0.58\\SCA8-7 & 1063.60 & 10.92 & 
1070.99 & 10.63 & \bf{1051.28} & 
1.17 & 1.87\\SCA8-8 & \bf{1071.18} & 13.66 & 
1087.85 & 11.60 & 1071.18 & 0.00
 & 1.56\\SCA8-9 & 1081.44 & 9.86 & 
1087.95 & 10.79 & \bf{1060.50} & 
1.97 & 2.59\\CON3-0 & 619.09 & 2.08 & 
624.44 & 2.40 & \bf{616.52} & 
0.42 & 1.28\\CON3-1 & 560.75 & 3.68 & 
560.75 & 3.31 & \bf{554.47} & 
1.13 & 1.13\\CON3-2 & 521.38 & 2.83 & 
521.38 & 3.04 & \bf{518.00} & 
0.65 & 0.65\\CON3-3 & 591.20 & 3.74 & 
591.20 & 3.79 & \bf{591.19} & 
0.00 & 0.00\\CON3-4 & 591.43 & 3.68 & 
591.43 & 3.31 & \bf{588.79} & 
0.45 & 0.45\\CON3-5 & \bf{563.70} & 3.54 & 
566.66 & 2.65 & 563.70 & 0.00
 & 0.53\\CON3-6 & 501.33 & 1.75 & 
501.33 & 1.81 & \bf{499.05} & 
0.46 & 0.46\\CON3-7 & 578.41 & 3.23 & 
579.56 & 3.70 & \bf{576.48} & 
0.33 & 0.53\\CON3-8 & \bf{523.05} & 3.31 & 
524.31 & 2.75 & 523.05 & 0.00
 & 0.24\\CON3-9 & 588.40 & 2.98 & 
588.40 & 2.83 & \bf{578.24} & 
1.76 & 1.76\\CON8-0 & 873.77 & 7.52 & 
884.85 & 8.65 & \bf{857.17} & 
1.94 & 3.23\\CON8-1 & 741.20 & 10.78 & 
747.37 & 8.73 & \bf{740.85} & 
0.05 & 0.88\\CON8-2 & 713.91 & 10.24 & 
715.84 & 9.35 & \bf{712.89} & 
0.14 & 0.41\\CON8-3 & 817.57 & 11.70 & 
825.92 & 11.62 & \bf{811.07} & 
0.80 & 1.83\\CON8-4 & 780.46 & 10.19 & 
789.00 & 10.37 & \bf{772.25} & 
1.06 & 2.17\\CON8-5 & 758.12 & 8.54 & 
764.38 & 8.93 & \bf{754.88} & 
0.43 & 1.26\\CON8-6 & 698.32 & 9.34 & 
701.25 & 9.57 & \bf{678.92} & 
2.86 & 3.29\\CON8-7 & 815.43 & 11.74 & 
818.62 & 11.32 & \bf{811.96} & 
0.43 & 0.82\\CON8-8 & 784.28 & 8.26 & 
786.90 & 8.53 & \bf{767.53} & 
2.18 & 2.52\\CON8-9 & 821.42 & 7.90 & 
825.52 & 8.55 & \bf{809.00} & 
1.54 & 2.04\\\bf{PROM.} & 
\bf{764.87} & \bf{6.99} & \bf{768.91} & \bf{6.67} & \bf{758.54} & \bf{0.79} & \bf{1.25}\\[1ex]\hline
\end{tabular}
\label{table:nonlin}
\end{table} \clearpage
\begin{table}[ht]
\caption{Resultados de la ejecución de la metaheurística SCA, utilizando instancias de Dethloff con la configuración -n 50.0 -b 10 -y .9}
\centering
\small
\begin{tabular}{c c c c c c c c}
\hline\hline
Instancia & Costo mínimo & Tiempo(seg.) & Costo promedio & Tiempo promedio(seg.) & CME & \%G & \%GP \\ [0.5ex]
\hline
SCA3-0 & 636.06 & 3.89 & 
639.43 & 4.51 & \bf{635.62} & 
0.07 & 0.60\\SCA3-1 & \bf{697.84} & 4.83 & 
699.51 & 4.16 & 697.84 & 0.00
 & 0.24\\SCA3-2 & 665.71 & 2.79 & 
666.95 & 4.63 & \bf{659.34} & 
0.97 & 1.15\\SCA3-3 & 680.60 & 3.66 & 
680.88 & 3.66 & \bf{680.04} & 
0.08 & 0.12\\SCA3-4 & 692.57 & 2.94 & 
692.57 & 3.27 & \bf{690.50} & 
0.30 & 0.30\\SCA3-5 & \bf{659.90} & 3.06 & 
672.14 & 3.11 & 659.90 & 0.00
 & 1.85\\SCA3-6 & 653.68 & 3.66 & 
653.75 & 3.50 & \bf{651.09} & 
0.40 & 0.41\\SCA3-7 & 666.60 & 4.76 & 
669.96 & 3.48 & \bf{659.17} & 
1.13 & 1.64\\SCA3-8 & \bf{719.47} & 3.10 & 
719.47 & 3.71 & 719.47 & 0.00
 & 0.00\\
SCA3-9 & 681.68 & 2.78 & 
684.29 & 2.93 & \bf{681.00} & 
0.10 & 0.48\\SCA8-0 & 984.01 & 8.18 & 
994.43 & 7.95 & \bf{961.50} & 
2.34 & 3.42\\SCA8-1 & 1066.50 & 10.91 & 
1076.51 & 11.39 & \bf{1049.65} & 
1.61 & 2.56\\SCA8-2 & 1051.21 & 10.23 & 
1052.41 & 12.39 & \bf{1039.64} & 
1.11 & 1.23\\SCA8-3 & 1022.39 & 10.82 & 
1026.68 & 9.12 & \bf{983.34} & 
3.97 & 4.41\\SCA8-4 & 1074.19 & 13.02 & 
1080.93 & 10.36 & \bf{1065.49} & 
0.82 & 1.45\\SCA8-5 & 1050.09 & 8.89 & 
1053.83 & 11.18 & \bf{1027.08} & 
2.24 & 2.60\\SCA8-6 & 972.48 & 14.72 & 
978.42 & 11.66 & \bf{971.82} & 
0.07 & 0.68\\SCA8-7 & 1067.11 & 12.17 & 
1069.01 & 11.99 & \bf{1051.28} & 
1.51 & 1.69\\SCA8-8 & 1085.34 & 10.18 & 
1094.10 & 9.87 & \bf{1071.18} & 
1.32 & 2.14\\SCA8-9 & 1067.42 & 9.34 & 
1074.40 & 10.29 & \bf{1060.50} & 
0.65 & 1.31\\CON3-0 & 628.08 & 1.06 & 
629.28 & 1.22 & \bf{616.52} & 
1.88 & 2.07\\CON3-1 & 560.75 & 2.33 & 
561.45 & 2.75 & \bf{554.47} & 
1.13 & 1.26\\CON3-2 & 521.38 & 3.05 & 
521.38 & 2.71 & \bf{518.00} & 
0.65 & 0.65\\CON3-3 & 591.20 & 2.82 & 
591.20 & 3.73 & \bf{591.19} & 
0.00 & 0.00\\CON3-4 & 591.43 & 3.76 & 
591.43 & 3.27 & \bf{588.79} & 
0.45 & 0.45\\CON3-5 & 564.88 & 3.26 & 
564.89 & 2.80 & \bf{563.70} & 
0.21 & 0.21\\CON3-6 & 502.16 & 5.23 & 
502.16 & 4.27 & \bf{499.05} & 
0.62 & 0.62\\CON3-7 & 582.12 & 4.73 & 
585.04 & 2.50 & \bf{576.48} & 
0.98 & 1.48\\CON3-8 & \bf{523.05} & 4.62 & 
523.12 & 3.43 & 523.05 & 0.00
 & 0.01\\CON3-9 & 578.25 & 1.53 & 
581.65 & 1.71 & \bf{578.24} & 
0.00 & 0.59\\CON8-0 & 868.71 & 6.29 & 
881.58 & 7.16 & \bf{857.17} & 
1.35 & 2.85\\CON8-1 & 742.47 & 9.78 & 
750.79 & 10.45 & \bf{740.85} & 
0.22 & 1.34\\CON8-2 & 713.60 & 9.42 & 
715.20 & 9.82 & \bf{712.89} & 
0.10 & 0.32\\CON8-3 & 829.58 & 9.63 & 
831.45 & 9.35 & \bf{811.07} & 
2.28 & 2.51\\CON8-4 & 781.39 & 6.50 & 
788.00 & 8.96 & \bf{772.25} & 
1.18 & 2.04\\CON8-5 & 760.41 & 9.01 & 
764.13 & 8.49 & \bf{754.88} & 
0.73 & 1.23\\CON8-6 & 694.99 & 7.33 & 
700.25 & 10.45 & \bf{678.92} & 
2.37 & 3.14\\CON8-7 & 815.73 & 12.30 & 
817.29 & 9.62 & \bf{811.96} & 
0.46 & 0.66\\CON8-8 & 788.37 & 7.12 & 
792.11 & 7.47 & \bf{767.53} & 
2.72 & 3.20\\CON8-9 & 819.55 & 8.66 & 
825.00 & 9.27 & \bf{809.00} & 
1.30 & 1.98\\\bf{PROM.} & 
\bf{766.32} & \bf{6.56} & \bf{769.93} & \bf{6.56} & \bf{758.54} & \bf{0.93} & \bf{1.37}\\[1ex]\hline
\end{tabular}
\label{table:nonlin}
\end{table} \clearpage
\begin{table}[ht]
\caption{Resultados de la ejecución de la metaheurística SCA, utilizando instancias de Dethloff con la configuración -n 50.0 -b 10 -y 1.0}
\centering
\small
\begin{tabular}{c c c c c c c c}
\hline\hline
Instancia & Costo mínimo & Tiempo(seg.) & Costo promedio & Tiempo promedio(seg.) & CME & \%G & \%GP \\ [0.5ex]
\hline
SCA3-0 & 640.55 & 3.94 & 
640.55 & 3.22 & \bf{635.62} & 
0.78 & 0.78\\SCA3-1 & \bf{697.84} & 5.00 & 
700.43 & 6.41 & 697.84 & 0.00
 & 0.37\\SCA3-2 & 661.13 & 4.31 & 
663.60 & 3.57 & \bf{659.34} & 
0.27 & 0.65\\SCA3-3 & 680.60 & 2.93 & 
680.96 & 3.35 & \bf{680.04} & 
0.08 & 0.13\\SCA3-4 & \bf{690.50} & 3.75 & 
690.50 & 3.58 & 690.50 & 0.00
 & 0.00\\
SCA3-5 & 661.07 & 3.18 & 
676.62 & 3.24 & \bf{659.90} & 
0.18 & 2.53\\SCA3-6 & 653.68 & 5.16 & 
654.54 & 4.57 & \bf{651.09} & 
0.40 & 0.53\\SCA3-7 & 671.67 & 3.76 & 
671.67 & 3.20 & \bf{659.17} & 
1.90 & 1.90\\SCA3-8 & \bf{719.47} & 4.79 & 
719.47 & 3.50 & 719.47 & 0.00
 & 0.00\\
SCA3-9 & \bf{681.00} & 6.29 & 
683.07 & 4.03 & 681.00 & 0.00
 & 0.30\\SCA8-0 & 988.41 & 11.67 & 
993.34 & 9.93 & \bf{961.50} & 
2.80 & 3.31\\SCA8-1 & 1067.36 & 11.45 & 
1072.48 & 9.96 & \bf{1049.65} & 
1.69 & 2.18\\SCA8-2 & 1051.48 & 11.26 & 
1053.04 & 12.36 & \bf{1039.64} & 
1.14 & 1.29\\SCA8-3 & 1019.54 & 9.98 & 
1031.45 & 9.54 & \bf{983.34} & 
3.68 & 4.89\\SCA8-4 & 1067.28 & 11.20 & 
1074.67 & 10.86 & \bf{1065.49} & 
0.17 & 0.86\\SCA8-5 & 1050.64 & 8.24 & 
1064.45 & 9.86 & \bf{1027.08} & 
2.29 & 3.64\\SCA8-6 & 972.48 & 8.98 & 
977.31 & 11.09 & \bf{971.82} & 
0.07 & 0.56\\SCA8-7 & 1063.22 & 10.48 & 
1069.30 & 10.33 & \bf{1051.28} & 
1.14 & 1.71\\SCA8-8 & 1082.91 & 10.96 & 
1090.99 & 9.52 & \bf{1071.18} & 
1.10 & 1.85\\SCA8-9 & 1067.42 & 13.77 & 
1075.34 & 11.33 & \bf{1060.50} & 
0.65 & 1.40\\CON3-0 & 632.98 & 1.86 & 
633.17 & 1.98 & \bf{616.52} & 
2.67 & 2.70\\CON3-1 & 560.75 & 2.51 & 
560.75 & 2.40 & \bf{554.47} & 
1.13 & 1.13\\CON3-2 & 521.38 & 2.03 & 
521.38 & 2.29 & \bf{518.00} & 
0.65 & 0.65\\CON3-3 & 591.20 & 2.11 & 
591.20 & 2.53 & \bf{591.19} & 
0.00 & 0.00\\CON3-4 & \bf{588.79} & 3.76 & 
590.21 & 2.85 & 588.79 & 0.00
 & 0.24\\CON3-5 & \bf{563.70} & 4.46 & 
566.80 & 3.56 & 563.70 & 0.00
 & 0.55\\CON3-6 & 504.15 & 2.26 & 
504.88 & 2.50 & \bf{499.05} & 
1.02 & 1.17\\CON3-7 & 584.01 & 3.22 & 
585.33 & 3.25 & \bf{576.48} & 
1.31 & 1.54\\CON3-8 & \bf{523.05} & 3.36 & 
523.21 & 3.07 & 523.05 & 0.00
 & 0.03\\CON3-9 & 588.40 & 1.85 & 
588.40 & 2.80 & \bf{578.24} & 
1.76 & 1.76\\CON8-0 & 877.17 & 9.05 & 
880.45 & 8.82 & \bf{857.17} & 
2.33 & 2.72\\CON8-1 & 742.47 & 7.02 & 
744.09 & 9.41 & \bf{740.85} & 
0.22 & 0.44\\CON8-2 & 713.60 & 11.58 & 
716.59 & 11.46 & \bf{712.89} & 
0.10 & 0.52\\CON8-3 & 832.75 & 8.54 & 
835.43 & 9.49 & \bf{811.07} & 
2.67 & 3.00\\CON8-4 & 777.99 & 7.85 & 
786.67 & 9.66 & \bf{772.25} & 
0.74 & 1.87\\CON8-5 & 758.84 & 8.80 & 
760.00 & 10.46 & \bf{754.88} & 
0.52 & 0.68\\CON8-6 & 689.23 & 11.18 & 
693.27 & 10.43 & \bf{678.92} & 
1.52 & 2.11\\CON8-7 & 814.79 & 11.57 & 
816.30 & 12.08 & \bf{811.96} & 
0.35 & 0.54\\CON8-8 & 779.43 & 6.28 & 
785.41 & 7.15 & \bf{767.53} & 
1.55 & 2.33\\CON8-9 & 815.02 & 9.66 & 
823.94 & 8.36 & \bf{809.00} & 
0.74 & 1.85\\\bf{PROM.} & 
\bf{766.20} & \bf{6.75} & \bf{769.78} & \bf{6.70} & \bf{758.54} & \bf{0.94} & \bf{1.37}\\[1ex]\hline
\end{tabular}
\label{table:nonlin}
\end{table} \clearpage
\begin{table}[ht]
\caption{Resultados de la ejecución de la metaheurística SCA, utilizando instancias de Dethloff con la configuración -n 100.0 -b 10 -y 0.1}
\centering
\small
\begin{tabular}{c c c c c c c c}
\hline\hline
Instancia & Costo mínimo & Tiempo(seg.) & Costo promedio & Tiempo promedio(seg.) & CME & \%G & \%GP \\ [0.5ex]
\hline
SCA3-0 & 640.55 & 3.12 & 
640.55 & 3.13 & \bf{635.62} & 
0.78 & 0.78\\SCA3-1 & 701.53 & 2.94 & 
701.61 & 2.92 & \bf{697.84} & 
0.53 & 0.54\\SCA3-2 & \bf{659.34} & 5.23 & 
667.78 & 3.94 & 659.34 & 0.00
 & 1.28\\SCA3-3 & 681.31 & 2.78 & 
681.31 & 2.67 & \bf{680.04} & 
0.19 & 0.19\\SCA3-4 & 692.57 & 2.87 & 
692.57 & 2.79 & \bf{690.50} & 
0.30 & 0.30\\SCA3-5 & 670.10 & 2.22 & 
673.50 & 2.27 & \bf{659.90} & 
1.55 & 2.06\\SCA3-6 & 653.93 & 1.32 & 
653.93 & 1.33 & \bf{651.09} & 
0.44 & 0.44\\SCA3-7 & 666.15 & 2.70 & 
666.15 & 2.69 & \bf{659.17} & 
1.06 & 1.06\\SCA3-8 & \bf{719.47} & 2.85 & 
719.47 & 3.85 & 719.47 & 0.00
 & 0.00\\
SCA3-9 & \bf{681.00} & 3.43 & 
683.07 & 4.03 & 681.00 & 0.00
 & 0.30\\SCA8-0 & 985.12 & 8.62 & 
991.28 & 10.82 & \bf{961.50} & 
2.46 & 3.10\\SCA8-1 & 1053.44 & 13.86 & 
1071.02 & 17.20 & \bf{1049.65} & 
0.36 & 2.04\\SCA8-2 & 1053.78 & 9.63 & 
1054.01 & 11.54 & \bf{1039.64} & 
1.36 & 1.38\\SCA8-3 & 1029.00 & 14.64 & 
1032.12 & 9.43 & \bf{983.34} & 
4.64 & 4.96\\SCA8-4 & 1071.86 & 12.46 & 
1092.77 & 11.52 & \bf{1065.49} & 
0.60 & 2.56\\SCA8-5 & 1029.95 & 9.36 & 
1044.70 & 10.26 & \bf{1027.08} & 
0.28 & 1.72\\SCA8-6 & 972.48 & 11.75 & 
982.84 & 8.95 & \bf{971.82} & 
0.07 & 1.13\\SCA8-7 & 1067.03 & 9.88 & 
1071.47 & 10.66 & \bf{1051.28} & 
1.50 & 1.92\\SCA8-8 & 1092.01 & 4.88 & 
1094.74 & 6.19 & \bf{1071.18} & 
1.94 & 2.20\\SCA8-9 & 1082.94 & 8.41 & 
1089.82 & 7.44 & \bf{1060.50} & 
2.12 & 2.76\\CON3-0 & 620.76 & 2.38 & 
629.38 & 2.40 & \bf{616.52} & 
0.69 & 2.09\\CON3-1 & 556.04 & 2.79 & 
557.22 & 2.92 & \bf{554.47} & 
0.28 & 0.50\\CON3-2 & 521.38 & 2.47 & 
521.38 & 2.60 & \bf{518.00} & 
0.65 & 0.65\\CON3-3 & \bf{591.19} & 3.37 & 
591.20 & 3.75 & 591.19 & 0.00
 & 0.00\\CON3-4 & 589.32 & 2.59 & 
589.32 & 2.58 & \bf{588.79} & 
0.09 & 0.09\\CON3-5 & 569.04 & 3.14 & 
570.69 & 2.80 & \bf{563.70} & 
0.95 & 1.24\\CON3-6 & 502.16 & 2.38 & 
502.16 & 2.22 & \bf{499.05} & 
0.62 & 0.62\\CON3-7 & 582.33 & 3.15 & 
585.09 & 3.43 & \bf{576.48} & 
1.01 & 1.49\\CON3-8 & \bf{523.05} & 3.03 & 
524.25 & 2.77 & 523.05 & 0.00
 & 0.23\\CON3-9 & 581.06 & 1.86 & 
586.93 & 2.17 & \bf{578.24} & 
0.49 & 1.50\\CON8-0 & 872.09 & 8.14 & 
877.49 & 8.04 & \bf{857.17} & 
1.74 & 2.37\\CON8-1 & 741.70 & 10.36 & 
749.20 & 9.27 & \bf{740.85} & 
0.11 & 1.13\\CON8-2 & 714.06 & 49.88 & 
721.03 & 18.67 & \bf{712.89} & 
0.16 & 1.14\\CON8-3 & 832.89 & 9.62 & 
834.91 & 8.89 & \bf{811.07} & 
2.69 & 2.94\\CON8-4 & 786.79 & 9.87 & 
788.03 & 7.66 & \bf{772.25} & 
1.88 & 2.04\\CON8-5 & 757.75 & 11.22 & 
761.28 & 9.40 & \bf{754.88} & 
0.38 & 0.85\\CON8-6 & 687.70 & 9.91 & 
699.46 & 8.05 & \bf{678.92} & 
1.29 & 3.02\\CON8-7 & 814.50 & 10.99 & 
815.41 & 9.77 & \bf{811.96} & 
0.31 & 0.43\\CON8-8 & 789.44 & 10.06 & 
789.44 & 8.23 & \bf{767.53} & 
2.85 & 2.85\\CON8-9 & 814.36 & 10.32 & 
824.84 & 8.38 & \bf{809.00} & 
0.66 & 1.96\\\bf{PROM.} & 
\bf{766.28} & \bf{7.51} & \bf{770.59} & \bf{6.44} & \bf{758.54} & \bf{0.93} & \bf{1.45}\\[1ex]\hline
\end{tabular}
\label{table:nonlin}
\end{table} \clearpage
\begin{table}[ht]
\caption{Resultados de la ejecución de la metaheurística SCA, utilizando instancias de Dethloff con la configuración -n 100.0 -b 10 -y .2}
\centering
\small
\begin{tabular}{c c c c c c c c}
\hline\hline
Instancia & Costo mínimo & Tiempo(seg.) & Costo promedio & Tiempo promedio(seg.) & CME & \%G & \%GP \\ [0.5ex]
\hline
SCA3-0 & 640.55 & 2.21 & 
640.55 & 2.40 & \bf{635.62} & 
0.78 & 0.78\\SCA3-1 & \bf{697.84} & 3.73 & 
700.76 & 2.36 & 697.84 & 0.00
 & 0.42\\SCA3-2 & 664.21 & 3.20 & 
664.21 & 3.23 & \bf{659.34} & 
0.74 & 0.74\\SCA3-3 & 681.31 & 4.71 & 
681.31 & 4.04 & \bf{680.04} & 
0.19 & 0.19\\SCA3-4 & \bf{690.50} & 7.16 & 
693.29 & 4.13 & 690.50 & 0.00
 & 0.40\\SCA3-5 & 664.69 & 1.30 & 
664.69 & 1.34 & \bf{659.90} & 
0.73 & 0.73\\SCA3-6 & 652.47 & 3.01 & 
654.56 & 3.55 & \bf{651.09} & 
0.21 & 0.53\\SCA3-7 & 669.89 & 3.24 & 
670.78 & 3.73 & \bf{659.17} & 
1.63 & 1.76\\SCA3-8 & \bf{719.47} & 3.79 & 
719.54 & 3.27 & 719.47 & 0.00
 & 0.01\\SCA3-9 & \bf{681.00} & 5.10 & 
683.94 & 4.01 & 681.00 & 0.00
 & 0.43\\SCA8-0 & 995.97 & 7.62 & 
997.61 & 8.10 & \bf{961.50} & 
3.59 & 3.76\\SCA8-1 & 1050.93 & 9.30 & 
1061.37 & 8.98 & \bf{1049.65} & 
0.12 & 1.12\\SCA8-2 & 1051.33 & 13.15 & 
1053.01 & 14.88 & \bf{1039.64} & 
1.12 & 1.29\\SCA8-3 & 1014.10 & 7.38 & 
1020.46 & 8.38 & \bf{983.34} & 
3.13 & 3.77\\SCA8-4 & 1073.96 & 5.79 & 
1088.39 & 8.69 & \bf{1065.49} & 
0.79 & 2.15\\SCA8-5 & 1045.89 & 10.09 & 
1051.64 & 10.51 & \bf{1027.08} & 
1.83 & 2.39\\SCA8-6 & 972.48 & 18.57 & 
980.53 & 10.00 & \bf{971.82} & 
0.07 & 0.90\\SCA8-7 & 1067.88 & 12.18 & 
1073.80 & 10.85 & \bf{1051.28} & 
1.58 & 2.14\\SCA8-8 & \bf{1071.18} & 8.39 & 
1090.13 & 10.10 & 1071.18 & 0.00
 & 1.77\\SCA8-9 & 1078.23 & 10.22 & 
1082.46 & 10.11 & \bf{1060.50} & 
1.67 & 2.07\\CON3-0 & 624.84 & 2.89 & 
632.79 & 1.73 & \bf{616.52} & 
1.35 & 2.64\\CON3-1 & 556.04 & 2.75 & 
556.04 & 2.23 & \bf{554.47} & 
0.28 & 0.28\\CON3-2 & 521.38 & 1.85 & 
521.44 & 1.86 & \bf{518.00} & 
0.65 & 0.66\\CON3-3 & 591.20 & 1.77 & 
591.20 & 2.29 & \bf{591.19} & 
0.00 & 0.00\\CON3-4 & 591.43 & 3.57 & 
591.43 & 3.41 & \bf{588.79} & 
0.45 & 0.45\\CON3-5 & 564.88 & 2.55 & 
567.07 & 2.94 & \bf{563.70} & 
0.21 & 0.60\\CON3-6 & 502.16 & 2.35 & 
502.16 & 2.42 & \bf{499.05} & 
0.62 & 0.62\\CON3-7 & 578.41 & 3.22 & 
578.41 & 3.22 & \bf{576.48} & 
0.33 & 0.33\\CON3-8 & 523.19 & 1.56 & 
524.89 & 2.10 & \bf{523.05} & 
0.03 & 0.35\\CON3-9 & 588.40 & 2.35 & 
588.40 & 2.00 & \bf{578.24} & 
1.76 & 1.76\\CON8-0 & 858.88 & 4.61 & 
880.60 & 6.65 & \bf{857.17} & 
0.20 & 2.73\\CON8-1 & 742.47 & 11.16 & 
749.07 & 10.49 & \bf{740.85} & 
0.22 & 1.11\\CON8-2 & \bf{712.89} & 22.66 & 
718.60 & 12.74 & 712.89 & 0.00
 & 0.80\\CON8-3 & 814.50 & 13.52 & 
829.50 & 11.04 & \bf{811.07} & 
0.42 & 2.27\\CON8-4 & \bf{772.25} & 7.99 & 
784.47 & 10.68 & 772.25 & 0.00
 & 1.58\\CON8-5 & 755.67 & 12.56 & 
763.10 & 10.72 & \bf{754.88} & 
0.10 & 1.09\\CON8-6 & 695.45 & 10.32 & 
696.71 & 8.67 & \bf{678.92} & 
2.43 & 2.62\\CON8-7 & 814.86 & 7.36 & 
817.24 & 12.18 & \bf{811.96} & 
0.36 & 0.65\\CON8-8 & 783.01 & 8.52 & 
785.67 & 9.08 & \bf{767.53} & 
2.02 & 2.36\\CON8-9 & 825.75 & 6.23 & 
829.33 & 7.51 & \bf{809.00} & 
2.07 & 2.51\\\bf{PROM.} & 
\bf{765.04} & \bf{6.75} & \bf{769.53} & \bf{6.42} & \bf{758.54} & \bf{0.79} & \bf{1.32}\\[1ex]\hline
\end{tabular}
\label{table:nonlin}
\end{table} \clearpage
\begin{table}[ht]
\caption{Resultados de la ejecución de la metaheurística SCA, utilizando instancias de Dethloff con la configuración -n 100.0 -b 10 -y .3}
\centering
\small
\begin{tabular}{c c c c c c c c}
\hline\hline
Instancia & Costo mínimo & Tiempo(seg.) & Costo promedio & Tiempo promedio(seg.) & CME & \%G & \%GP \\ [0.5ex]
\hline
SCA3-0 & 640.55 & 3.20 & 
640.55 & 3.71 & \bf{635.62} & 
0.78 & 0.78\\SCA3-1 & 700.50 & 2.54 & 
701.07 & 3.33 & \bf{697.84} & 
0.38 & 0.46\\SCA3-2 & 661.13 & 4.74 & 
664.26 & 4.93 & \bf{659.34} & 
0.27 & 0.75\\SCA3-3 & \bf{680.04} & 3.66 & 
680.99 & 2.98 & 680.04 & 0.00
 & 0.14\\SCA3-4 & \bf{690.50} & 1.74 & 
691.18 & 3.05 & 690.50 & 0.00
 & 0.10\\SCA3-5 & 665.04 & 2.40 & 
665.04 & 2.17 & \bf{659.90} & 
0.78 & 0.78\\SCA3-6 & 653.68 & 2.48 & 
653.68 & 2.27 & \bf{651.09} & 
0.40 & 0.40\\SCA3-7 & 671.67 & 2.40 & 
671.67 & 2.81 & \bf{659.17} & 
1.90 & 1.90\\SCA3-8 & \bf{719.47} & 2.23 & 
719.47 & 2.34 & 719.47 & 0.00
 & 0.00\\
SCA3-9 & \bf{681.00} & 2.38 & 
681.00 & 2.75 & 681.00 & 0.00
 & 0.00\\
SCA8-0 & 980.29 & 7.57 & 
988.76 & 8.60 & \bf{961.50} & 
1.95 & 2.84\\SCA8-1 & 1062.88 & 9.68 & 
1070.31 & 8.60 & \bf{1049.65} & 
1.26 & 1.97\\SCA8-2 & 1052.94 & 13.40 & 
1053.15 & 13.49 & \bf{1039.64} & 
1.28 & 1.30\\SCA8-3 & 1002.12 & 8.74 & 
1008.34 & 9.53 & \bf{983.34} & 
1.91 & 2.54\\SCA8-4 & 1076.16 & 10.60 & 
1080.76 & 10.26 & \bf{1065.49} & 
1.00 & 1.43\\SCA8-5 & 1043.05 & 7.30 & 
1048.14 & 11.52 & \bf{1027.08} & 
1.55 & 2.05\\SCA8-6 & 972.48 & 10.20 & 
973.82 & 10.15 & \bf{971.82} & 
0.07 & 0.21\\SCA8-7 & 1067.11 & 9.20 & 
1071.47 & 8.80 & \bf{1051.28} & 
1.51 & 1.92\\SCA8-8 & 1084.41 & 7.86 & 
1091.28 & 9.28 & \bf{1071.18} & 
1.24 & 1.88\\SCA8-9 & 1076.77 & 10.25 & 
1080.44 & 8.85 & \bf{1060.50} & 
1.53 & 1.88\\CON3-0 & 620.29 & 3.08 & 
621.46 & 2.84 & \bf{616.52} & 
0.61 & 0.80\\CON3-1 & 560.75 & 3.23 & 
560.75 & 2.89 & \bf{554.47} & 
1.13 & 1.13\\CON3-2 & 521.38 & 2.60 & 
522.65 & 2.38 & \bf{518.00} & 
0.65 & 0.90\\CON3-3 & \bf{591.19} & 2.08 & 
591.19 & 2.17 & 591.19 & 0.00
 & 0.00\\
CON3-4 & 591.43 & 3.16 & 
591.43 & 3.35 & \bf{588.79} & 
0.45 & 0.45\\CON3-5 & \bf{563.70} & 2.73 & 
563.70 & 2.94 & 563.70 & 0.00
 & 0.00\\
CON3-6 & 501.63 & 1.74 & 
503.20 & 2.60 & \bf{499.05} & 
0.52 & 0.83\\CON3-7 & 586.01 & 2.74 & 
586.01 & 2.88 & \bf{576.48} & 
1.65 & 1.65\\CON3-8 & 523.19 & 1.86 & 
523.70 & 2.23 & \bf{523.05} & 
0.03 & 0.12\\CON3-9 & 588.40 & 2.62 & 
588.40 & 2.77 & \bf{578.24} & 
1.76 & 1.76\\CON8-0 & 881.66 & 9.63 & 
891.20 & 7.40 & \bf{857.17} & 
2.86 & 3.97\\CON8-1 & 740.93 & 11.30 & 
743.98 & 12.30 & \bf{740.85} & 
0.01 & 0.42\\CON8-2 & 713.84 & 11.57 & 
719.16 & 11.34 & \bf{712.89} & 
0.13 & 0.88\\CON8-3 & 823.31 & 9.62 & 
831.93 & 9.09 & \bf{811.07} & 
1.51 & 2.57\\CON8-4 & 775.22 & 7.96 & 
786.67 & 7.75 & \bf{772.25} & 
0.38 & 1.87\\CON8-5 & 759.44 & 10.67 & 
762.83 & 11.57 & \bf{754.88} & 
0.60 & 1.05\\CON8-6 & 695.86 & 13.02 & 
700.10 & 10.64 & \bf{678.92} & 
2.50 & 3.12\\CON8-7 & 815.32 & 9.10 & 
818.80 & 11.79 & \bf{811.96} & 
0.41 & 0.84\\CON8-8 & 779.43 & 6.86 & 
783.18 & 7.13 & \bf{767.53} & 
1.55 & 2.04\\CON8-9 & 827.94 & 7.57 & 
833.80 & 8.15 & \bf{809.00} & 
2.34 & 3.07\\\bf{PROM.} & 
\bf{766.07} & \bf{6.14} & \bf{768.99} & \bf{6.34} & \bf{758.54} & \bf{0.92} & \bf{1.27}\\[1ex]\hline
\end{tabular}
\label{table:nonlin}
\end{table} \clearpage
\begin{table}[ht]
\caption{Resultados de la ejecución de la metaheurística SCA, utilizando instancias de Dethloff con la configuración -n 100.0 -b 10 -y .4}
\centering
\small
\begin{tabular}{c c c c c c c c}
\hline\hline
Instancia & Costo mínimo & Tiempo(seg.) & Costo promedio & Tiempo promedio(seg.) & CME & \%G & \%GP \\ [0.5ex]
\hline
SCA3-0 & 640.55 & 3.87 & 
640.55 & 3.81 & \bf{635.62} & 
0.78 & 0.78\\SCA3-1 & 700.50 & 2.03 & 
700.50 & 2.19 & \bf{697.84} & 
0.38 & 0.38\\SCA3-2 & \bf{659.34} & 3.41 & 
659.34 & 3.37 & 659.34 & 0.00
 & 0.00\\
SCA3-3 & 682.44 & 2.14 & 
682.44 & 2.15 & \bf{680.04} & 
0.35 & 0.35\\SCA3-4 & 692.57 & 3.19 & 
693.07 & 3.15 & \bf{690.50} & 
0.30 & 0.37\\SCA3-5 & \bf{659.90} & 2.38 & 
666.56 & 2.48 & 659.90 & 0.00
 & 1.01\\SCA3-6 & 654.26 & 2.24 & 
654.26 & 2.23 & \bf{651.09} & 
0.49 & 0.49\\SCA3-7 & 666.60 & 4.10 & 
668.03 & 3.89 & \bf{659.17} & 
1.13 & 1.34\\SCA3-8 & \bf{719.47} & 2.86 & 
719.47 & 2.72 & 719.47 & 0.00
 & 0.00\\
SCA3-9 & \bf{681.00} & 3.12 & 
682.05 & 3.52 & 681.00 & 0.00
 & 0.15\\SCA8-0 & 980.51 & 7.42 & 
988.57 & 8.22 & \bf{961.50} & 
1.98 & 2.82\\SCA8-1 & 1057.09 & 8.82 & 
1061.80 & 9.51 & \bf{1049.65} & 
0.71 & 1.16\\SCA8-2 & 1051.48 & 12.39 & 
1053.15 & 12.39 & \bf{1039.64} & 
1.14 & 1.30\\SCA8-3 & 997.17 & 6.62 & 
1010.73 & 6.14 & \bf{983.34} & 
1.41 & 2.79\\SCA8-4 & 1067.55 & 7.22 & 
1081.92 & 8.07 & \bf{1065.49} & 
0.19 & 1.54\\SCA8-5 & 1043.05 & 11.90 & 
1053.83 & 11.14 & \bf{1027.08} & 
1.55 & 2.60\\SCA8-6 & 972.48 & 19.35 & 
980.02 & 12.99 & \bf{971.82} & 
0.07 & 0.84\\SCA8-7 & 1067.11 & 14.98 & 
1068.82 & 11.53 & \bf{1051.28} & 
1.51 & 1.67\\SCA8-8 & 1080.21 & 8.40 & 
1088.03 & 10.00 & \bf{1071.18} & 
0.84 & 1.57\\SCA8-9 & 1082.56 & 7.16 & 
1084.74 & 8.30 & \bf{1060.50} & 
2.08 & 2.29\\CON3-0 & 632.87 & 2.57 & 
632.87 & 2.28 & \bf{616.52} & 
2.65 & 2.65\\CON3-1 & 560.75 & 3.58 & 
560.75 & 4.08 & \bf{554.47} & 
1.13 & 1.13\\CON3-2 & 521.38 & 2.31 & 
521.38 & 2.01 & \bf{518.00} & 
0.65 & 0.65\\CON3-3 & \bf{591.19} & 4.12 & 
591.20 & 3.53 & 591.19 & 0.00
 & 0.00\\CON3-4 & 591.43 & 3.49 & 
591.43 & 3.06 & \bf{588.79} & 
0.45 & 0.45\\CON3-5 & 564.89 & 2.23 & 
567.12 & 2.34 & \bf{563.70} & 
0.21 & 0.61\\CON3-6 & 502.16 & 3.14 & 
502.16 & 2.75 & \bf{499.05} & 
0.62 & 0.62\\CON3-7 & 578.22 & 2.35 & 
578.36 & 3.08 & \bf{576.48} & 
0.30 & 0.33\\CON3-8 & 524.59 & 4.64 & 
524.60 & 3.03 & \bf{523.05} & 
0.29 & 0.30\\CON3-9 & 588.99 & 1.67 & 
588.99 & 1.82 & \bf{578.24} & 
1.86 & 1.86\\CON8-0 & 871.90 & 8.02 & 
875.03 & 9.68 & \bf{857.17} & 
1.72 & 2.08\\CON8-1 & 742.47 & 8.03 & 
746.58 & 9.40 & \bf{740.85} & 
0.22 & 0.77\\CON8-2 & 713.05 & 10.19 & 
715.30 & 11.98 & \bf{712.89} & 
0.02 & 0.34\\CON8-3 & 822.73 & 11.11 & 
827.63 & 12.08 & \bf{811.07} & 
1.44 & 2.04\\CON8-4 & 787.80 & 9.88 & 
793.38 & 9.11 & \bf{772.25} & 
2.01 & 2.74\\CON8-5 & 760.41 & 9.60 & 
763.58 & 9.51 & \bf{754.88} & 
0.73 & 1.15\\CON8-6 & 695.70 & 12.71 & 
699.39 & 9.01 & \bf{678.92} & 
2.47 & 3.02\\CON8-7 & 815.43 & 14.63 & 
817.89 & 12.16 & \bf{811.96} & 
0.43 & 0.73\\CON8-8 & 779.36 & 7.29 & 
782.58 & 8.58 & \bf{767.53} & 
1.54 & 1.96\\CON8-9 & 818.98 & 12.81 & 
829.66 & 19.77 & \bf{809.00} & 
1.23 & 2.55\\\bf{PROM.} & 
\bf{765.50} & \bf{6.70} & \bf{768.69} & \bf{6.68} & \bf{758.54} & \bf{0.87} & \bf{1.24}\\[1ex]\hline
\end{tabular}
\label{table:nonlin}
\end{table} \clearpage
\begin{table}[ht]
\caption{Resultados de la ejecución de la metaheurística SCA, utilizando instancias de Dethloff con la configuración -n 100.0 -b 10 -y .5}
\centering
\small
\begin{tabular}{c c c c c c c c}
\hline\hline
Instancia & Costo mínimo & Tiempo(seg.) & Costo promedio & Tiempo promedio(seg.) & CME & \%G & \%GP \\ [0.5ex]
\hline
SCA3-0 & 640.55 & 2.71 & 
640.55 & 3.15 & \bf{635.62} & 
0.78 & 0.78\\SCA3-1 & 700.50 & 2.26 & 
700.50 & 2.35 & \bf{697.84} & 
0.38 & 0.38\\SCA3-2 & 666.01 & 3.02 & 
666.01 & 3.55 & \bf{659.34} & 
1.01 & 1.01\\SCA3-3 & \bf{680.04} & 1.59 & 
680.99 & 2.88 & 680.04 & 0.00
 & 0.14\\SCA3-4 & 692.57 & 3.88 & 
693.25 & 4.21 & \bf{690.50} & 
0.30 & 0.40\\SCA3-5 & 669.80 & 3.05 & 
672.51 & 3.10 & \bf{659.90} & 
1.50 & 1.91\\SCA3-6 & 652.94 & 4.04 & 
653.16 & 4.11 & \bf{651.09} & 
0.28 & 0.32\\SCA3-7 & 666.60 & 3.54 & 
669.13 & 3.75 & \bf{659.17} & 
1.13 & 1.51\\SCA3-8 & \bf{719.47} & 1.27 & 
719.47 & 1.28 & 719.47 & 0.00
 & 0.00\\
SCA3-9 & \bf{681.00} & 4.52 & 
684.47 & 3.78 & 681.00 & 0.00
 & 0.51\\SCA8-0 & 978.76 & 10.18 & 
983.44 & 8.87 & \bf{961.50} & 
1.80 & 2.28\\SCA8-1 & 1067.02 & 6.95 & 
1070.99 & 7.85 & \bf{1049.65} & 
1.65 & 2.03\\SCA8-2 & 1051.95 & 8.69 & 
1052.96 & 11.53 & \bf{1039.64} & 
1.18 & 1.28\\SCA8-3 & 1018.32 & 7.82 & 
1021.88 & 8.92 & \bf{983.34} & 
3.56 & 3.92\\SCA8-4 & 1086.81 & 5.86 & 
1089.60 & 7.38 & \bf{1065.49} & 
2.00 & 2.26\\SCA8-5 & 1050.64 & 8.94 & 
1057.71 & 10.95 & \bf{1027.08} & 
2.29 & 2.98\\SCA8-6 & 972.48 & 13.18 & 
975.93 & 12.16 & \bf{971.82} & 
0.07 & 0.42\\SCA8-7 & 1063.22 & 10.48 & 
1069.38 & 9.56 & \bf{1051.28} & 
1.14 & 1.72\\SCA8-8 & 1086.54 & 8.88 & 
1091.65 & 10.56 & \bf{1071.18} & 
1.43 & 1.91\\SCA8-9 & 1072.60 & 8.58 & 
1085.00 & 8.87 & \bf{1060.50} & 
1.14 & 2.31\\CON3-0 & 633.22 & 1.74 & 
633.22 & 1.78 & \bf{616.52} & 
2.71 & 2.71\\CON3-1 & 558.34 & 1.67 & 
559.54 & 2.30 & \bf{554.47} & 
0.70 & 0.92\\CON3-2 & 521.38 & 2.98 & 
523.88 & 3.18 & \bf{518.00} & 
0.65 & 1.14\\CON3-3 & 591.20 & 3.29 & 
591.20 & 2.76 & \bf{591.19} & 
0.00 & 0.00\\CON3-4 & 589.32 & 2.76 & 
590.90 & 3.46 & \bf{588.79} & 
0.09 & 0.36\\CON3-5 & \bf{563.70} & 2.65 & 
563.70 & 1.94 & 563.70 & 0.00
 & 0.00\\
CON3-6 & 502.16 & 2.26 & 
502.31 & 2.49 & \bf{499.05} & 
0.62 & 0.65\\CON3-7 & 577.68 & 2.91 & 
577.68 & 3.02 & \bf{576.48} & 
0.21 & 0.21\\CON3-8 & 526.59 & 3.58 & 
533.46 & 2.14 & \bf{523.05} & 
0.68 & 1.99\\CON3-9 & 588.40 & 2.20 & 
588.40 & 2.20 & \bf{578.24} & 
1.76 & 1.76\\CON8-0 & 873.22 & 9.08 & 
877.00 & 8.44 & \bf{857.17} & 
1.87 & 2.31\\CON8-1 & 742.47 & 12.73 & 
752.79 & 8.82 & \bf{740.85} & 
0.22 & 1.61\\CON8-2 & 712.94 & 9.74 & 
719.17 & 10.54 & \bf{712.89} & 
0.01 & 0.88\\CON8-3 & 817.57 & 6.27 & 
827.22 & 14.55 & \bf{811.07} & 
0.80 & 1.99\\CON8-4 & 780.51 & 12.09 & 
784.79 & 11.92 & \bf{772.25} & 
1.07 & 1.62\\CON8-5 & 755.14 & 12.81 & 
759.15 & 10.81 & \bf{754.88} & 
0.03 & 0.57\\CON8-6 & 698.41 & 9.56 & 
701.97 & 9.58 & \bf{678.92} & 
2.87 & 3.39\\CON8-7 & 814.50 & 16.83 & 
816.09 & 12.52 & \bf{811.96} & 
0.31 & 0.51\\CON8-8 & 781.74 & 9.03 & 
786.14 & 9.32 & \bf{767.53} & 
1.85 & 2.42\\CON8-9 & 817.16 & 6.47 & 
826.89 & 9.24 & \bf{809.00} & 
1.01 & 2.21\\\bf{PROM.} & 
\bf{766.59} & \bf{6.25} & \bf{769.85} & \bf{6.50} & \bf{758.54} & \bf{0.98} & \bf{1.38}\\[1ex]\hline
\end{tabular}
\label{table:nonlin}
\end{table} \clearpage
\begin{table}[ht]
\caption{Resultados de la ejecución de la metaheurística SCA, utilizando instancias de Dethloff con la configuración -n 100.0 -b 10 -y .6}
\centering
\small
\begin{tabular}{c c c c c c c c}
\hline\hline
Instancia & Costo mínimo & Tiempo(seg.) & Costo promedio & Tiempo promedio(seg.) & CME & \%G & \%GP \\ [0.5ex]
\hline
SCA3-0 & 640.55 & 4.02 & 
640.55 & 4.08 & \bf{635.62} & 
0.78 & 0.78\\SCA3-1 & \bf{697.84} & 2.88 & 
697.84 & 2.98 & 697.84 & 0.00
 & 0.00\\
SCA3-2 & \bf{659.34} & 5.00 & 
664.34 & 4.17 & 659.34 & 0.00
 & 0.76\\SCA3-3 & 681.31 & 4.31 & 
681.63 & 4.04 & \bf{680.04} & 
0.19 & 0.23\\SCA3-4 & \bf{690.50} & 2.61 & 
691.02 & 3.78 & 690.50 & 0.00
 & 0.08\\SCA3-5 & 668.63 & 3.08 & 
670.21 & 3.47 & \bf{659.90} & 
1.32 & 1.56\\SCA3-6 & \bf{651.09} & 4.15 & 
653.81 & 3.93 & 651.09 & 0.00
 & 0.42\\SCA3-7 & 671.67 & 3.15 & 
671.67 & 3.13 & \bf{659.17} & 
1.90 & 1.90\\SCA3-8 & \bf{719.47} & 3.96 & 
720.76 & 3.13 & 719.47 & 0.00
 & 0.18\\SCA3-9 & 681.23 & 4.78 & 
683.31 & 4.07 & \bf{681.00} & 
0.03 & 0.34\\SCA8-0 & 970.64 & 8.29 & 
981.42 & 9.56 & \bf{961.50} & 
0.95 & 2.07\\SCA8-1 & 1059.35 & 8.01 & 
1063.14 & 10.18 & \bf{1049.65} & 
0.92 & 1.29\\SCA8-2 & 1051.95 & 9.71 & 
1053.55 & 15.95 & \bf{1039.64} & 
1.18 & 1.34\\SCA8-3 & 1020.43 & 7.26 & 
1027.70 & 7.63 & \bf{983.34} & 
3.77 & 4.51\\SCA8-4 & 1067.55 & 8.60 & 
1077.42 & 9.23 & \bf{1065.49} & 
0.19 & 1.12\\SCA8-5 & 1042.64 & 12.83 & 
1045.31 & 12.72 & \bf{1027.08} & 
1.51 & 1.77\\SCA8-6 & 981.21 & 10.25 & 
991.76 & 10.16 & \bf{971.82} & 
0.97 & 2.05\\SCA8-7 & 1063.22 & 10.58 & 
1075.00 & 10.92 & \bf{1051.28} & 
1.14 & 2.26\\SCA8-8 & 1090.39 & 9.30 & 
1096.64 & 11.07 & \bf{1071.18} & 
1.79 & 2.38\\SCA8-9 & 1075.03 & 9.01 & 
1080.80 & 8.12 & \bf{1060.50} & 
1.37 & 1.91\\CON3-0 & 623.31 & 1.35 & 
627.75 & 1.69 & \bf{616.52} & 
1.10 & 1.82\\CON3-1 & 560.75 & 3.15 & 
560.75 & 3.15 & \bf{554.47} & 
1.13 & 1.13\\CON3-2 & 521.38 & 2.23 & 
522.10 & 2.30 & \bf{518.00} & 
0.65 & 0.79\\CON3-3 & 591.20 & 2.16 & 
591.82 & 3.63 & \bf{591.19} & 
0.00 & 0.11\\CON3-4 & 591.43 & 2.32 & 
591.43 & 2.39 & \bf{588.79} & 
0.45 & 0.45\\CON3-5 & 564.89 & 3.08 & 
568.00 & 2.84 & \bf{563.70} & 
0.21 & 0.76\\CON3-6 & 502.16 & 2.74 & 
502.16 & 2.77 & \bf{499.05} & 
0.62 & 0.62\\CON3-7 & 586.01 & 4.70 & 
586.01 & 3.63 & \bf{576.48} & 
1.65 & 1.65\\CON3-8 & \bf{523.05} & 2.37 & 
523.05 & 2.55 & 523.05 & 0.00
 & 0.00\\
CON3-9 & 590.48 & 2.46 & 
590.48 & 2.46 & \bf{578.24} & 
2.12 & 2.12\\CON8-0 & 868.56 & 12.12 & 
873.32 & 8.43 & \bf{857.17} & 
1.33 & 1.88\\CON8-1 & 742.47 & 11.31 & 
748.67 & 10.11 & \bf{740.85} & 
0.22 & 1.06\\CON8-2 & 717.76 & 11.10 & 
721.12 & 10.96 & \bf{712.89} & 
0.68 & 1.15\\CON8-3 & 828.82 & 8.14 & 
834.66 & 15.76 & \bf{811.07} & 
2.19 & 2.91\\CON8-4 & 784.32 & 12.68 & 
787.57 & 9.76 & \bf{772.25} & 
1.56 & 1.98\\CON8-5 & 758.12 & 17.86 & 
765.56 & 13.31 & \bf{754.88} & 
0.43 & 1.41\\CON8-6 & 690.06 & 8.61 & 
698.35 & 6.86 & \bf{678.92} & 
1.64 & 2.86\\CON8-7 & 816.00 & 13.70 & 
818.30 & 11.47 & \bf{811.96} & 
0.50 & 0.78\\CON8-8 & 782.06 & 6.80 & 
782.20 & 7.08 & \bf{767.53} & 
1.89 & 1.91\\CON8-9 & 825.78 & 6.19 & 
830.25 & 7.74 & \bf{809.00} & 
2.07 & 2.63\\\bf{PROM.} & 
\bf{766.32} & \bf{6.67} & \bf{769.79} & \bf{6.78} & \bf{758.54} & \bf{0.96} & \bf{1.37}\\[1ex]\hline
\end{tabular}
\label{table:nonlin}
\end{table} \clearpage
\begin{table}[ht]
\caption{Resultados de la ejecución de la metaheurística SCA, utilizando instancias de Dethloff con la configuración -n 100.0 -b 10 -y .7}
\centering
\small
\begin{tabular}{c c c c c c c c}
\hline\hline
Instancia & Costo mínimo & Tiempo(seg.) & Costo promedio & Tiempo promedio(seg.) & CME & \%G & \%GP \\ [0.5ex]
\hline
SCA3-0 & 640.55 & 3.34 & 
640.55 & 3.52 & \bf{635.62} & 
0.78 & 0.78\\SCA3-1 & \bf{697.84} & 3.05 & 
699.66 & 2.85 & 697.84 & 0.00
 & 0.26\\SCA3-2 & \bf{659.34} & 3.04 & 
660.93 & 3.07 & 659.34 & 0.00
 & 0.24\\SCA3-3 & 680.60 & 3.31 & 
681.06 & 2.77 & \bf{680.04} & 
0.08 & 0.15\\SCA3-4 & \bf{690.50} & 3.09 & 
690.50 & 2.65 & 690.50 & 0.00
 & 0.00\\
SCA3-5 & 665.04 & 4.60 & 
671.43 & 3.29 & \bf{659.90} & 
0.78 & 1.75\\SCA3-6 & 652.94 & 3.26 & 
653.50 & 3.14 & \bf{651.09} & 
0.28 & 0.37\\SCA3-7 & 669.89 & 2.26 & 
669.89 & 2.96 & \bf{659.17} & 
1.63 & 1.63\\SCA3-8 & \bf{719.47} & 4.45 & 
719.47 & 4.43 & 719.47 & 0.00
 & 0.00\\
SCA3-9 & \bf{681.00} & 2.71 & 
681.00 & 5.08 & 681.00 & 0.00
 & 0.00\\
SCA8-0 & 977.93 & 8.91 & 
986.41 & 10.54 & \bf{961.50} & 
1.71 & 2.59\\SCA8-1 & 1077.60 & 8.60 & 
1079.15 & 8.31 & \bf{1049.65} & 
2.66 & 2.81\\SCA8-2 & 1050.37 & 10.95 & 
1051.43 & 11.24 & \bf{1039.64} & 
1.03 & 1.13\\SCA8-3 & 1019.77 & 34.23 & 
1025.17 & 13.97 & \bf{983.34} & 
3.70 & 4.25\\SCA8-4 & 1067.66 & 15.31 & 
1078.80 & 12.87 & \bf{1065.49} & 
0.20 & 1.25\\SCA8-5 & 1043.05 & 10.80 & 
1053.36 & 12.11 & \bf{1027.08} & 
1.55 & 2.56\\SCA8-6 & 972.48 & 11.11 & 
974.36 & 10.03 & \bf{971.82} & 
0.07 & 0.26\\SCA8-7 & 1063.22 & 13.78 & 
1072.07 & 11.54 & \bf{1051.28} & 
1.14 & 1.98\\SCA8-8 & 1082.91 & 11.32 & 
1093.38 & 10.32 & \bf{1071.18} & 
1.10 & 2.07\\SCA8-9 & 1072.62 & 9.11 & 
1082.10 & 8.91 & \bf{1060.50} & 
1.14 & 2.04\\CON3-0 & 633.24 & 1.51 & 
633.24 & 2.73 & \bf{616.52} & 
2.71 & 2.71\\CON3-1 & 560.75 & 2.91 & 
560.75 & 2.95 & \bf{554.47} & 
1.13 & 1.13\\CON3-2 & 521.38 & 2.92 & 
521.38 & 2.22 & \bf{518.00} & 
0.65 & 0.65\\CON3-3 & \bf{591.19} & 2.28 & 
591.19 & 2.46 & 591.19 & 0.00
 & 0.00\\CON3-4 & 591.43 & 3.05 & 
591.43 & 3.25 & \bf{588.79} & 
0.45 & 0.45\\CON3-5 & \bf{563.70} & 4.03 & 
566.84 & 2.88 & 563.70 & 0.00
 & 0.56\\CON3-6 & 502.16 & 2.01 & 
502.65 & 2.30 & \bf{499.05} & 
0.62 & 0.72\\CON3-7 & 585.42 & 3.40 & 
585.57 & 3.66 & \bf{576.48} & 
1.55 & 1.58\\CON3-8 & \bf{523.05} & 2.41 & 
523.09 & 3.08 & 523.05 & 0.00
 & 0.01\\CON3-9 & 588.40 & 5.72 & 
588.70 & 7.47 & \bf{578.24} & 
1.76 & 1.81\\CON8-0 & 878.60 & 7.52 & 
879.27 & 9.36 & \bf{857.17} & 
2.50 & 2.58\\CON8-1 & 742.47 & 10.41 & 
747.55 & 9.59 & \bf{740.85} & 
0.22 & 0.91\\CON8-2 & 713.60 & 8.50 & 
716.35 & 8.82 & \bf{712.89} & 
0.10 & 0.49\\CON8-3 & 822.94 & 13.76 & 
831.19 & 10.57 & \bf{811.07} & 
1.46 & 2.48\\CON8-4 & 780.54 & 12.30 & 
784.25 & 11.96 & \bf{772.25} & 
1.07 & 1.55\\CON8-5 & 759.44 & 9.65 & 
760.85 & 9.64 & \bf{754.88} & 
0.60 & 0.79\\CON8-6 & 691.34 & 7.60 & 
694.12 & 8.86 & \bf{678.92} & 
1.83 & 2.24\\CON8-7 & 814.77 & 12.22 & 
815.34 & 13.28 & \bf{811.96} & 
0.35 & 0.42\\CON8-8 & 788.89 & 6.23 & 
789.65 & 7.14 & \bf{767.53} & 
2.78 & 2.88\\CON8-9 & 814.72 & 9.23 & 
826.00 & 7.99 & \bf{809.00} & 
0.71 & 2.10\\\bf{PROM.} & 
\bf{766.32} & \bf{7.37} & \bf{769.34} & \bf{6.85} & \bf{758.54} & \bf{0.96} & \bf{1.30}\\[1ex]\hline
\end{tabular}
\label{table:nonlin}
\end{table} \clearpage
\begin{table}[ht]
\caption{Resultados de la ejecución de la metaheurística SCA, utilizando instancias de Dethloff con la configuración -n 100.0 -b 10 -y .8}
\centering
\small
\begin{tabular}{c c c c c c c c}
\hline\hline
Instancia & Costo mínimo & Tiempo(seg.) & Costo promedio & Tiempo promedio(seg.) & CME & \%G & \%GP \\ [0.5ex]
\hline
SCA3-0 & 640.55 & 3.71 & 
640.55 & 3.60 & \bf{635.62} & 
0.78 & 0.78\\SCA3-1 & \bf{697.84} & 3.96 & 
699.84 & 3.28 & 697.84 & 0.00
 & 0.29\\SCA3-2 & 665.71 & 4.42 & 
667.18 & 3.99 & \bf{659.34} & 
0.97 & 1.19\\SCA3-3 & \bf{680.04} & 4.59 & 
680.99 & 3.50 & 680.04 & 0.00
 & 0.14\\SCA3-4 & \bf{690.50} & 3.87 & 
690.50 & 3.47 & 690.50 & 0.00
 & 0.00\\
SCA3-5 & 665.64 & 2.01 & 
667.20 & 2.37 & \bf{659.90} & 
0.87 & 1.11\\SCA3-6 & 653.68 & 3.55 & 
653.83 & 3.71 & \bf{651.09} & 
0.40 & 0.42\\SCA3-7 & 666.15 & 2.38 & 
668.91 & 2.67 & \bf{659.17} & 
1.06 & 1.48\\SCA3-8 & \bf{719.47} & 3.54 & 
719.47 & 3.45 & 719.47 & 0.00
 & 0.00\\
SCA3-9 & \bf{681.00} & 4.54 & 
682.03 & 3.86 & 681.00 & 0.00
 & 0.15\\SCA8-0 & 982.18 & 7.86 & 
992.85 & 7.58 & \bf{961.50} & 
2.15 & 3.26\\SCA8-1 & 1063.12 & 9.90 & 
1076.92 & 9.67 & \bf{1049.65} & 
1.28 & 2.60\\SCA8-2 & 1052.94 & 16.01 & 
1053.36 & 11.72 & \bf{1039.64} & 
1.28 & 1.32\\SCA8-3 & 1010.98 & 7.01 & 
1015.08 & 8.17 & \bf{983.34} & 
2.81 & 3.23\\SCA8-4 & 1080.57 & 9.17 & 
1089.12 & 10.80 & \bf{1065.49} & 
1.42 & 2.22\\SCA8-5 & 1043.55 & 17.25 & 
1050.05 & 12.08 & \bf{1027.08} & 
1.60 & 2.24\\SCA8-6 & 972.48 & 10.79 & 
976.05 & 12.94 & \bf{971.82} & 
0.07 & 0.44\\SCA8-7 & 1067.49 & 8.70 & 
1072.45 & 8.87 & \bf{1051.28} & 
1.54 & 2.01\\SCA8-8 & 1082.67 & 8.76 & 
1085.94 & 9.70 & \bf{1071.18} & 
1.07 & 1.38\\SCA8-9 & 1081.28 & 39.98 & 
1085.29 & 17.46 & \bf{1060.50} & 
1.96 & 2.34\\CON3-0 & 621.06 & 1.72 & 
621.06 & 1.43 & \bf{616.52} & 
0.74 & 0.74\\CON3-1 & 560.75 & 2.30 & 
560.75 & 3.01 & \bf{554.47} & 
1.13 & 1.13\\CON3-2 & 521.38 & 2.10 & 
521.38 & 1.99 & \bf{518.00} & 
0.65 & 0.65\\CON3-3 & \bf{591.19} & 2.82 & 
591.20 & 3.33 & 591.19 & 0.00
 & 0.00\\CON3-4 & \bf{588.79} & 2.88 & 
590.77 & 3.23 & 588.79 & 0.00
 & 0.34\\CON3-5 & 564.88 & 2.02 & 
564.88 & 2.03 & \bf{563.70} & 
0.21 & 0.21\\CON3-6 & 502.16 & 3.60 & 
502.16 & 3.71 & \bf{499.05} & 
0.62 & 0.62\\CON3-7 & 582.33 & 3.71 & 
583.25 & 3.71 & \bf{576.48} & 
1.01 & 1.17\\CON3-8 & \bf{523.05} & 2.72 & 
523.05 & 3.21 & 523.05 & 0.00
 & 0.00\\
CON3-9 & 585.13 & 1.80 & 
585.13 & 1.98 & \bf{578.24} & 
1.19 & 1.19\\CON8-0 & 882.60 & 8.09 & 
890.47 & 8.05 & \bf{857.17} & 
2.97 & 3.88\\CON8-1 & 741.70 & 12.05 & 
745.33 & 12.77 & \bf{740.85} & 
0.11 & 0.60\\CON8-2 & 713.60 & 8.36 & 
717.14 & 8.83 & \bf{712.89} & 
0.10 & 0.60\\CON8-3 & 832.40 & 9.03 & 
832.59 & 8.67 & \bf{811.07} & 
2.63 & 2.65\\CON8-4 & 777.24 & 5.94 & 
788.42 & 6.53 & \bf{772.25} & 
0.65 & 2.09\\CON8-5 & 759.44 & 11.24 & 
763.65 & 9.46 & \bf{754.88} & 
0.60 & 1.16\\CON8-6 & 683.83 & 9.90 & 
693.80 & 9.07 & \bf{678.92} & 
0.72 & 2.19\\CON8-7 & 814.50 & 11.52 & 
827.14 & 9.87 & \bf{811.96} & 
0.31 & 1.87\\CON8-8 & 781.96 & 10.84 & 
786.26 & 10.71 & \bf{767.53} & 
1.88 & 2.44\\CON8-9 & 817.56 & 12.12 & 
822.34 & 9.25 & \bf{809.00} & 
1.06 & 1.65\\\bf{PROM.} & 
\bf{766.08} & \bf{7.42} & \bf{769.46} & \bf{6.59} & \bf{758.54} & \bf{0.90} & \bf{1.29}\\[1ex]\hline
\end{tabular}
\label{table:nonlin}
\end{table} \clearpage
\begin{table}[ht]
\caption{Resultados de la ejecución de la metaheurística SCA, utilizando instancias de Dethloff con la configuración -n 100.0 -b 10 -y .9}
\centering
\small
\begin{tabular}{c c c c c c c c}
\hline\hline
Instancia & Costo mínimo & Tiempo(seg.) & Costo promedio & Tiempo promedio(seg.) & CME & \%G & \%GP \\ [0.5ex]
\hline
SCA3-0 & 640.55 & 4.11 & 
640.55 & 4.29 & \bf{635.62} & 
0.78 & 0.78\\SCA3-1 & 701.53 & 4.02 & 
701.59 & 2.90 & \bf{697.84} & 
0.53 & 0.54\\SCA3-2 & 661.13 & 3.40 & 
663.92 & 2.96 & \bf{659.34} & 
0.27 & 0.70\\SCA3-3 & 680.60 & 2.25 & 
680.60 & 2.63 & \bf{680.04} & 
0.08 & 0.08\\SCA3-4 & \bf{690.50} & 4.80 & 
692.38 & 3.73 & 690.50 & 0.00
 & 0.27\\SCA3-5 & 679.84 & 2.70 & 
680.83 & 2.99 & \bf{659.90} & 
3.02 & 3.17\\SCA3-6 & 653.68 & 3.35 & 
653.68 & 2.93 & \bf{651.09} & 
0.40 & 0.40\\SCA3-7 & 671.21 & 2.06 & 
671.55 & 2.57 & \bf{659.17} & 
1.83 & 1.88\\SCA3-8 & \bf{719.47} & 4.35 & 
719.47 & 3.96 & 719.47 & 0.00
 & 0.00\\
SCA3-9 & \bf{681.00} & 3.52 & 
682.03 & 3.79 & 681.00 & 0.00
 & 0.15\\SCA8-0 & 970.64 & 8.30 & 
999.38 & 8.24 & \bf{961.50} & 
0.95 & 3.94\\SCA8-1 & 1065.18 & 8.42 & 
1065.79 & 10.94 & \bf{1049.65} & 
1.48 & 1.54\\SCA8-2 & 1051.33 & 11.72 & 
1053.32 & 12.18 & \bf{1039.64} & 
1.12 & 1.32\\SCA8-3 & 1014.10 & 8.88 & 
1026.87 & 11.46 & \bf{983.34} & 
3.13 & 4.43\\SCA8-4 & 1067.66 & 9.87 & 
1087.78 & 9.92 & \bf{1065.49} & 
0.20 & 2.09\\SCA8-5 & 1043.05 & 7.23 & 
1046.38 & 10.04 & \bf{1027.08} & 
1.55 & 1.88\\SCA8-6 & 972.48 & 21.89 & 
979.97 & 16.22 & \bf{971.82} & 
0.07 & 0.84\\SCA8-7 & 1063.22 & 10.51 & 
1066.17 & 11.13 & \bf{1051.28} & 
1.14 & 1.42\\SCA8-8 & \bf{1071.18} & 9.01 & 
1091.53 & 9.01 & 1071.18 & 0.00
 & 1.90\\SCA8-9 & 1070.34 & 10.68 & 
1076.08 & 11.06 & \bf{1060.50} & 
0.93 & 1.47\\CON3-0 & 623.84 & 2.48 & 
628.71 & 2.29 & \bf{616.52} & 
1.19 & 1.98\\CON3-1 & 560.75 & 3.50 & 
560.75 & 3.27 & \bf{554.47} & 
1.13 & 1.13\\CON3-2 & 521.38 & 2.72 & 
521.38 & 2.96 & \bf{518.00} & 
0.65 & 0.65\\CON3-3 & \bf{591.19} & 2.92 & 
591.19 & 2.75 & 591.19 & 0.00
 & 0.00\\CON3-4 & 591.43 & 2.20 & 
591.72 & 2.27 & \bf{588.79} & 
0.45 & 0.50\\CON3-5 & 564.89 & 3.26 & 
566.71 & 2.95 & \bf{563.70} & 
0.21 & 0.53\\CON3-6 & 502.16 & 2.82 & 
502.65 & 2.59 & \bf{499.05} & 
0.62 & 0.72\\CON3-7 & 586.01 & 4.75 & 
586.01 & 4.42 & \bf{576.48} & 
1.65 & 1.65\\CON3-8 & \bf{523.05} & 1.87 & 
523.28 & 3.31 & 523.05 & 0.00
 & 0.04\\CON3-9 & 588.40 & 4.65 & 
588.40 & 4.24 & \bf{578.24} & 
1.76 & 1.76\\CON8-0 & 874.62 & 8.65 & 
879.64 & 8.46 & \bf{857.17} & 
2.04 & 2.62\\CON8-1 & 748.85 & 16.27 & 
751.96 & 11.29 & \bf{740.85} & 
1.08 & 1.50\\CON8-2 & 713.05 & 14.75 & 
716.45 & 11.21 & \bf{712.89} & 
0.02 & 0.50\\CON8-3 & 813.84 & 11.12 & 
825.19 & 10.16 & \bf{811.07} & 
0.34 & 1.74\\CON8-4 & 775.22 & 10.94 & 
781.57 & 11.95 & \bf{772.25} & 
0.38 & 1.21\\CON8-5 & 755.14 & 10.77 & 
760.15 & 10.21 & \bf{754.88} & 
0.03 & 0.70\\CON8-6 & 680.47 & 8.42 & 
688.69 & 9.37 & \bf{678.92} & 
0.23 & 1.44\\CON8-7 & 815.44 & 11.03 & 
818.27 & 9.89 & \bf{811.96} & 
0.43 & 0.78\\CON8-8 & 784.24 & 9.73 & 
784.24 & 8.21 & \bf{767.53} & 
2.18 & 2.18\\CON8-9 & 821.18 & 9.13 & 
827.56 & 8.56 & \bf{809.00} & 
1.51 & 2.29\\\bf{PROM.} & 
\bf{765.10} & \bf{7.08} & \bf{769.36} & \bf{6.83} & \bf{758.54} & \bf{0.83} & \bf{1.32}\\[1ex]\hline
\end{tabular}
\label{table:nonlin}
\end{table} \clearpage
\begin{table}[ht]
\caption{Resultados de la ejecución de la metaheurística SCA, utilizando instancias de Dethloff con la configuración -n 100.0 -b 10 -y 1.0}
\centering
\small
\begin{tabular}{c c c c c c c c}
\hline\hline
Instancia & Costo mínimo & Tiempo(seg.) & Costo promedio & Tiempo promedio(seg.) & CME & \%G & \%GP \\ [0.5ex]
\hline
SCA3-0 & 640.55 & 4.38 & 
640.55 & 3.97 & \bf{635.62} & 
0.78 & 0.78\\SCA3-1 & \bf{697.84} & 4.82 & 
699.68 & 4.76 & 697.84 & 0.00
 & 0.26\\SCA3-2 & 661.13 & 3.48 & 
661.13 & 3.37 & \bf{659.34} & 
0.27 & 0.27\\SCA3-3 & 680.60 & 3.00 & 
681.06 & 3.58 & \bf{680.04} & 
0.08 & 0.15\\SCA3-4 & \bf{690.50} & 3.46 & 
691.02 & 3.85 & 690.50 & 0.00
 & 0.08\\SCA3-5 & 681.30 & 2.32 & 
681.43 & 2.60 & \bf{659.90} & 
3.24 & 3.26\\SCA3-6 & 652.94 & 4.68 & 
653.68 & 2.70 & \bf{651.09} & 
0.28 & 0.40\\SCA3-7 & 666.60 & 2.92 & 
667.87 & 2.95 & \bf{659.17} & 
1.13 & 1.32\\SCA3-8 & \bf{719.47} & 5.55 & 
719.47 & 5.59 & 719.47 & 0.00
 & 0.00\\
SCA3-9 & \bf{681.00} & 4.24 & 
683.07 & 4.20 & 681.00 & 0.00
 & 0.30\\SCA8-0 & 984.75 & 12.83 & 
990.95 & 12.51 & \bf{961.50} & 
2.42 & 3.06\\SCA8-1 & 1063.76 & 10.70 & 
1069.78 & 10.10 & \bf{1049.65} & 
1.34 & 1.92\\SCA8-2 & 1052.94 & 10.12 & 
1054.06 & 10.69 & \bf{1039.64} & 
1.28 & 1.39\\SCA8-3 & 1018.37 & 7.21 & 
1021.04 & 8.02 & \bf{983.34} & 
3.56 & 3.83\\SCA8-4 & 1069.53 & 12.64 & 
1084.12 & 10.89 & \bf{1065.49} & 
0.38 & 1.75\\SCA8-5 & 1043.05 & 14.46 & 
1049.65 & 12.38 & \bf{1027.08} & 
1.55 & 2.20\\SCA8-6 & 972.48 & 13.68 & 
976.96 & 11.49 & \bf{971.82} & 
0.07 & 0.53\\SCA8-7 & 1064.06 & 11.43 & 
1073.54 & 11.35 & \bf{1051.28} & 
1.22 & 2.12\\SCA8-8 & 1091.20 & 8.20 & 
1092.13 & 9.11 & \bf{1071.18} & 
1.87 & 1.96\\SCA8-9 & 1070.71 & 13.66 & 
1086.18 & 12.24 & \bf{1060.50} & 
0.96 & 2.42\\CON3-0 & 630.73 & 3.44 & 
630.73 & 3.41 & \bf{616.52} & 
2.30 & 2.30\\CON3-1 & 560.55 & 2.61 & 
560.65 & 2.73 & \bf{554.47} & 
1.10 & 1.11\\CON3-2 & 521.38 & 2.58 & 
521.38 & 2.79 & \bf{518.00} & 
0.65 & 0.65\\CON3-3 & 591.20 & 1.94 & 
591.20 & 2.18 & \bf{591.19} & 
0.00 & 0.00\\CON3-4 & 591.43 & 2.44 & 
591.43 & 2.45 & \bf{588.79} & 
0.45 & 0.45\\CON3-5 & 564.89 & 3.73 & 
565.22 & 3.54 & \bf{563.70} & 
0.21 & 0.27\\CON3-6 & 502.16 & 2.52 & 
502.16 & 3.11 & \bf{499.05} & 
0.62 & 0.62\\CON3-7 & 585.42 & 2.91 & 
585.86 & 3.24 & \bf{576.48} & 
1.55 & 1.63\\CON3-8 & \bf{523.05} & 3.30 & 
523.43 & 2.69 & 523.05 & 0.00
 & 0.07\\CON3-9 & 588.40 & 3.88 & 
588.40 & 3.46 & \bf{578.24} & 
1.76 & 1.76\\CON8-0 & 875.11 & 7.67 & 
880.36 & 7.91 & \bf{857.17} & 
2.09 & 2.71\\CON8-1 & 742.47 & 10.80 & 
744.00 & 11.50 & \bf{740.85} & 
0.22 & 0.42\\CON8-2 & 716.32 & 9.76 & 
720.70 & 12.14 & \bf{712.89} & 
0.48 & 1.10\\CON8-3 & 812.54 & 11.42 & 
828.24 & 9.93 & \bf{811.07} & 
0.18 & 2.12\\CON8-4 & 785.55 & 11.60 & 
789.55 & 12.68 & \bf{772.25} & 
1.72 & 2.24\\CON8-5 & 755.14 & 9.94 & 
762.53 & 9.63 & \bf{754.88} & 
0.03 & 1.01\\CON8-6 & 686.39 & 9.17 & 
689.18 & 9.91 & \bf{678.92} & 
1.10 & 1.51\\CON8-7 & 815.04 & 11.18 & 
820.36 & 8.79 & \bf{811.96} & 
0.38 & 1.03\\CON8-8 & 784.36 & 11.68 & 
786.65 & 9.57 & \bf{767.53} & 
2.19 & 2.49\\CON8-9 & 816.55 & 12.39 & 
826.63 & 9.46 & \bf{809.00} & 
0.93 & 2.18\\\bf{PROM.} & 
\bf{766.29} & \bf{7.22} & \bf{769.65} & \bf{6.94} & \bf{758.54} & \bf{0.96} & \bf{1.34}\\[1ex]\hline
\end{tabular}
\label{table:nonlin}
\end{table} \clearpage
\begin{table}[ht]
\caption{Resultados de la ejecución de la metaheurística SCA, utilizando instancias de Dethloff con la configuración -n 150.0 -b 10 -y 0.1}
\centering
\small
\begin{tabular}{c c c c c c c c}
\hline\hline
Instancia & Costo mínimo & Tiempo(seg.) & Costo promedio & Tiempo promedio(seg.) & CME & \%G & \%GP \\ [0.5ex]
\hline
SCA3-0 & 640.55 & 4.72 & 
640.55 & 8.17 & \bf{635.62} & 
0.78 & 0.78\\SCA3-1 & \bf{697.84} & 2.47 & 
697.84 & 2.56 & 697.84 & 0.00
 & 0.00\\
SCA3-2 & 661.13 & 3.63 & 
670.16 & 2.05 & \bf{659.34} & 
0.27 & 1.64\\SCA3-3 & 680.60 & 4.48 & 
680.88 & 4.19 & \bf{680.04} & 
0.08 & 0.12\\SCA3-4 & \bf{690.50} & 5.48 & 
690.50 & 4.64 & 690.50 & 0.00
 & 0.00\\
SCA3-5 & 665.04 & 2.76 & 
677.62 & 2.27 & \bf{659.90} & 
0.78 & 2.68\\SCA3-6 & \bf{651.09} & 4.55 & 
652.48 & 2.46 & 651.09 & 0.00
 & 0.21\\SCA3-7 & 667.34 & 3.46 & 
667.34 & 3.44 & \bf{659.17} & 
1.24 & 1.24\\SCA3-8 & \bf{719.47} & 3.66 & 
719.47 & 3.47 & 719.47 & 0.00
 & 0.00\\
SCA3-9 & \bf{681.00} & 4.80 & 
681.00 & 4.80 & 681.00 & 0.00
 & 0.00\\
SCA8-0 & 987.51 & 7.26 & 
989.46 & 9.85 & \bf{961.50} & 
2.71 & 2.91\\SCA8-1 & 1057.41 & 9.26 & 
1068.66 & 15.75 & \bf{1049.65} & 
0.74 & 1.81\\SCA8-2 & 1050.37 & 15.52 & 
1050.79 & 14.44 & \bf{1039.64} & 
1.03 & 1.07\\SCA8-3 & 1004.25 & 7.76 & 
1019.72 & 8.03 & \bf{983.34} & 
2.13 & 3.70\\SCA8-4 & 1068.97 & 9.93 & 
1078.50 & 9.71 & \bf{1065.49} & 
0.33 & 1.22\\SCA8-5 & 1049.44 & 8.91 & 
1053.40 & 9.96 & \bf{1027.08} & 
2.18 & 2.56\\SCA8-6 & 977.83 & 6.30 & 
977.83 & 6.94 & \bf{971.82} & 
0.62 & 0.62\\SCA8-7 & 1067.88 & 6.82 & 
1069.30 & 6.06 & \bf{1051.28} & 
1.58 & 1.71\\SCA8-8 & \bf{1071.18} & 9.17 & 
1085.23 & 8.62 & 1071.18 & 0.00
 & 1.31\\SCA8-9 & 1072.60 & 8.81 & 
1081.75 & 9.97 & \bf{1060.50} & 
1.14 & 2.00\\CON3-0 & 617.59 & 2.43 & 
618.26 & 2.30 & \bf{616.52} & 
0.17 & 0.28\\CON3-1 & 560.75 & 3.39 & 
560.75 & 2.89 & \bf{554.47} & 
1.13 & 1.13\\CON3-2 & 521.38 & 2.66 & 
521.38 & 1.93 & \bf{518.00} & 
0.65 & 0.65\\CON3-3 & 591.48 & 2.43 & 
591.48 & 2.54 & \bf{591.19} & 
0.05 & 0.05\\CON3-4 & 591.43 & 3.01 & 
591.43 & 4.00 & \bf{588.79} & 
0.45 & 0.45\\CON3-5 & 565.30 & 2.06 & 
565.30 & 2.04 & \bf{563.70} & 
0.28 & 0.28\\CON3-6 & 502.16 & 1.96 & 
502.16 & 1.97 & \bf{499.05} & 
0.62 & 0.62\\CON3-7 & 581.37 & 5.06 & 
581.58 & 4.63 & \bf{576.48} & 
0.85 & 0.89\\CON3-8 & \bf{523.05} & 2.74 & 
523.12 & 2.87 & 523.05 & 0.00
 & 0.01\\CON3-9 & 588.40 & 3.08 & 
588.40 & 2.87 & \bf{578.24} & 
1.76 & 1.76\\CON8-0 & 874.01 & 9.13 & 
874.16 & 8.50 & \bf{857.17} & 
1.96 & 1.98\\CON8-1 & 742.47 & 9.91 & 
745.25 & 9.85 & \bf{740.85} & 
0.22 & 0.59\\CON8-2 & 713.60 & 10.72 & 
716.90 & 11.21 & \bf{712.89} & 
0.10 & 0.56\\CON8-3 & 822.73 & 10.53 & 
831.38 & 9.98 & \bf{811.07} & 
1.44 & 2.50\\CON8-4 & 777.24 & 52.96 & 
779.77 & 20.37 & \bf{772.25} & 
0.65 & 0.97\\CON8-5 & 755.86 & 12.68 & 
760.77 & 12.01 & \bf{754.88} & 
0.13 & 0.78\\CON8-6 & 693.34 & 7.96 & 
698.94 & 7.66 & \bf{678.92} & 
2.12 & 2.95\\CON8-7 & 814.79 & 12.42 & 
814.99 & 13.75 & \bf{811.96} & 
0.35 & 0.37\\CON8-8 & 786.80 & 5.52 & 
789.30 & 6.42 & \bf{767.53} & 
2.51 & 2.84\\CON8-9 & 821.70 & 7.89 & 
821.70 & 7.03 & \bf{809.00} & 
1.57 & 1.57\\\bf{PROM.} & 
\bf{765.19} & \bf{7.46} & \bf{768.24} & \bf{6.80} & \bf{758.54} & \bf{0.82} & \bf{1.17}\\[1ex]\hline
\end{tabular}
\label{table:nonlin}
\end{table} \clearpage
\begin{table}[ht]
\caption{Resultados de la ejecución de la metaheurística SCA, utilizando instancias de Dethloff con la configuración -n 150.0 -b 10 -y .2}
\centering
\small
\begin{tabular}{c c c c c c c c}
\hline\hline
Instancia & Costo mínimo & Tiempo(seg.) & Costo promedio & Tiempo promedio(seg.) & CME & \%G & \%GP \\ [0.5ex]
\hline
SCA3-0 & 640.55 & 2.60 & 
640.55 & 2.73 & \bf{635.62} & 
0.78 & 0.78\\SCA3-1 & 701.74 & 2.32 & 
701.74 & 2.12 & \bf{697.84} & 
0.56 & 0.56\\SCA3-2 & \bf{659.34} & 3.61 & 
661.90 & 3.81 & 659.34 & 0.00
 & 0.39\\SCA3-3 & \bf{680.04} & 4.90 & 
680.46 & 7.37 & 680.04 & 0.00
 & 0.06\\SCA3-4 & \bf{690.50} & 3.33 & 
690.50 & 3.25 & 690.50 & 0.00
 & 0.00\\
SCA3-5 & 681.03 & 2.82 & 
681.23 & 2.81 & \bf{659.90} & 
3.20 & 3.23\\SCA3-6 & \bf{651.09} & 2.13 & 
651.09 & 1.98 & 651.09 & 0.00
 & 0.00\\
SCA3-7 & 671.67 & 2.96 & 
671.67 & 2.97 & \bf{659.17} & 
1.90 & 1.90\\SCA3-8 & \bf{719.47} & 4.26 & 
719.70 & 3.40 & 719.47 & 0.00
 & 0.03\\SCA3-9 & 685.14 & 3.70 & 
685.16 & 3.53 & \bf{681.00} & 
0.61 & 0.61\\SCA8-0 & 989.47 & 8.45 & 
995.33 & 8.10 & \bf{961.50} & 
2.91 & 3.52\\SCA8-1 & 1053.09 & 8.10 & 
1068.91 & 7.42 & \bf{1049.65} & 
0.33 & 1.83\\SCA8-2 & 1050.37 & 9.21 & 
1052.09 & 10.47 & \bf{1039.64} & 
1.03 & 1.20\\SCA8-3 & 1008.67 & 9.98 & 
1018.51 & 9.48 & \bf{983.34} & 
2.58 & 3.58\\SCA8-4 & 1067.55 & 9.09 & 
1073.10 & 10.15 & \bf{1065.49} & 
0.19 & 0.71\\SCA8-5 & 1036.88 & 14.24 & 
1044.87 & 11.61 & \bf{1027.08} & 
0.95 & 1.73\\SCA8-6 & 972.48 & 12.38 & 
976.85 & 11.18 & \bf{971.82} & 
0.07 & 0.52\\SCA8-7 & 1070.92 & 8.98 & 
1074.68 & 9.68 & \bf{1051.28} & 
1.87 & 2.23\\SCA8-8 & \bf{1071.18} & 7.99 & 
1077.28 & 8.66 & 1071.18 & 0.00
 & 0.57\\SCA8-9 & \bf{1060.50} & 7.75 & 
1072.78 & 8.49 & 1060.50 & 0.00
 & 1.16\\CON3-0 & 620.29 & 1.98 & 
620.29 & 2.04 & \bf{616.52} & 
0.61 & 0.61\\CON3-1 & 560.75 & 3.73 & 
560.75 & 3.16 & \bf{554.47} & 
1.13 & 1.13\\CON3-2 & 521.38 & 4.16 & 
521.38 & 3.81 & \bf{518.00} & 
0.65 & 0.65\\CON3-3 & 591.20 & 3.84 & 
591.20 & 4.02 & \bf{591.19} & 
0.00 & 0.00\\CON3-4 & 591.43 & 3.52 & 
591.43 & 3.53 & \bf{588.79} & 
0.45 & 0.45\\CON3-5 & 564.89 & 2.90 & 
565.94 & 2.93 & \bf{563.70} & 
0.21 & 0.40\\CON3-6 & 500.88 & 1.75 & 
500.88 & 1.86 & \bf{499.05} & 
0.37 & 0.37\\CON3-7 & 586.01 & 3.48 & 
586.01 & 3.99 & \bf{576.48} & 
1.65 & 1.65\\CON3-8 & \bf{523.05} & 2.54 & 
523.43 & 3.35 & 523.05 & 0.00
 & 0.07\\CON3-9 & 588.40 & 2.42 & 
588.40 & 2.43 & \bf{578.24} & 
1.76 & 1.76\\CON8-0 & 871.92 & 8.18 & 
886.03 & 8.15 & \bf{857.17} & 
1.72 & 3.37\\CON8-1 & 747.59 & 8.44 & 
748.84 & 13.41 & \bf{740.85} & 
0.91 & 1.08\\CON8-2 & 713.60 & 12.65 & 
717.77 & 9.72 & \bf{712.89} & 
0.10 & 0.68\\CON8-3 & 818.51 & 12.24 & 
828.09 & 11.68 & \bf{811.07} & 
0.92 & 2.10\\CON8-4 & 773.27 & 16.58 & 
781.17 & 12.30 & \bf{772.25} & 
0.13 & 1.16\\CON8-5 & 755.86 & 7.50 & 
760.49 & 10.67 & \bf{754.88} & 
0.13 & 0.74\\CON8-6 & 685.45 & 11.46 & 
692.70 & 9.06 & \bf{678.92} & 
0.96 & 2.03\\CON8-7 & 814.86 & 8.31 & 
817.28 & 8.75 & \bf{811.96} & 
0.36 & 0.66\\CON8-8 & 779.55 & 5.33 & 
780.26 & 6.55 & \bf{767.53} & 
1.57 & 1.66\\CON8-9 & 829.22 & 7.16 & 
829.58 & 7.85 & \bf{809.00} & 
2.50 & 2.54\\\bf{PROM.} & 
\bf{764.99} & \bf{6.42} & \bf{768.26} & \bf{6.46} & \bf{758.54} & \bf{0.83} & \bf{1.19}\\[1ex]\hline
\end{tabular}
\label{table:nonlin}
\end{table} \clearpage
\begin{table}[ht]
\caption{Resultados de la ejecución de la metaheurística SCA, utilizando instancias de Dethloff con la configuración -n 150.0 -b 10 -y .3}
\centering
\small
\begin{tabular}{c c c c c c c c}
\hline\hline
Instancia & Costo mínimo & Tiempo(seg.) & Costo promedio & Tiempo promedio(seg.) & CME & \%G & \%GP \\ [0.5ex]
\hline
SCA3-0 & 640.55 & 3.59 & 
640.55 & 3.36 & \bf{635.62} & 
0.78 & 0.78\\SCA3-1 & 701.53 & 2.90 & 
701.63 & 4.57 & \bf{697.84} & 
0.53 & 0.54\\SCA3-2 & 666.01 & 4.74 & 
666.01 & 4.28 & \bf{659.34} & 
1.01 & 1.01\\SCA3-3 & \bf{680.04} & 2.53 & 
680.47 & 2.71 & 680.04 & 0.00
 & 0.06\\SCA3-4 & 692.57 & 3.61 & 
692.57 & 3.61 & \bf{690.50} & 
0.30 & 0.30\\SCA3-5 & 662.75 & 3.62 & 
663.96 & 3.59 & \bf{659.90} & 
0.43 & 0.61\\SCA3-6 & 652.94 & 4.60 & 
652.94 & 4.73 & \bf{651.09} & 
0.28 & 0.28\\SCA3-7 & 671.67 & 2.98 & 
671.67 & 2.92 & \bf{659.17} & 
1.90 & 1.90\\SCA3-8 & \bf{719.47} & 3.78 & 
719.47 & 3.80 & 719.47 & 0.00
 & 0.00\\
SCA3-9 & \bf{681.00} & 5.13 & 
681.00 & 3.89 & 681.00 & 0.00
 & 0.00\\
SCA8-0 & 994.27 & 9.05 & 
994.27 & 9.02 & \bf{961.50} & 
3.41 & 3.41\\SCA8-1 & 1066.48 & 11.93 & 
1073.75 & 11.46 & \bf{1049.65} & 
1.60 & 2.30\\SCA8-2 & 1053.94 & 16.43 & 
1054.32 & 13.35 & \bf{1039.64} & 
1.38 & 1.41\\SCA8-3 & 1011.26 & 12.05 & 
1020.01 & 8.39 & \bf{983.34} & 
2.84 & 3.73\\SCA8-4 & \bf{1065.49} & 7.94 & 
1069.31 & 8.20 & 1065.49 & 0.00
 & 0.36\\SCA8-5 & 1043.05 & 13.00 & 
1047.91 & 10.48 & \bf{1027.08} & 
1.55 & 2.03\\SCA8-6 & 986.93 & 10.50 & 
991.25 & 8.62 & \bf{971.82} & 
1.55 & 2.00\\SCA8-7 & 1063.22 & 9.11 & 
1067.26 & 8.44 & \bf{1051.28} & 
1.14 & 1.52\\SCA8-8 & 1084.74 & 7.70 & 
1090.23 & 9.96 & \bf{1071.18} & 
1.27 & 1.78\\SCA8-9 & 1077.19 & 8.95 & 
1078.40 & 7.37 & \bf{1060.50} & 
1.57 & 1.69\\CON3-0 & 617.59 & 1.80 & 
619.17 & 1.48 & \bf{616.52} & 
0.17 & 0.43\\CON3-1 & 559.72 & 3.71 & 
559.72 & 3.36 & \bf{554.47} & 
0.95 & 0.95\\CON3-2 & 521.38 & 3.62 & 
521.38 & 3.37 & \bf{518.00} & 
0.65 & 0.65\\CON3-3 & 591.20 & 3.48 & 
591.20 & 3.41 & \bf{591.19} & 
0.00 & 0.00\\CON3-4 & 591.43 & 2.16 & 
591.43 & 2.26 & \bf{588.79} & 
0.45 & 0.45\\CON3-5 & \bf{563.70} & 3.29 & 
564.59 & 2.62 & 563.70 & 0.00
 & 0.16\\CON3-6 & 502.16 & 1.43 & 
502.16 & 2.54 & \bf{499.05} & 
0.62 & 0.62\\CON3-7 & 586.01 & 2.03 & 
586.01 & 2.08 & \bf{576.48} & 
1.65 & 1.65\\CON3-8 & 523.19 & 4.73 & 
523.19 & 4.94 & \bf{523.05} & 
0.03 & 0.03\\CON3-9 & 588.40 & 2.06 & 
588.40 & 2.13 & \bf{578.24} & 
1.76 & 1.76\\CON8-0 & 870.41 & 7.14 & 
878.63 & 7.79 & \bf{857.17} & 
1.54 & 2.50\\CON8-1 & 740.93 & 9.99 & 
750.97 & 23.13 & \bf{740.85} & 
0.01 & 1.37\\CON8-2 & 713.05 & 16.32 & 
714.89 & 12.52 & \bf{712.89} & 
0.02 & 0.28\\CON8-3 & 824.57 & 10.40 & 
824.57 & 9.97 & \bf{811.07} & 
1.66 & 1.66\\CON8-4 & 792.37 & 11.79 & 
794.57 & 9.02 & \bf{772.25} & 
2.61 & 2.89\\CON8-5 & 758.12 & 11.20 & 
761.96 & 9.77 & \bf{754.88} & 
0.43 & 0.94\\CON8-6 & 698.93 & 7.34 & 
702.46 & 7.91 & \bf{678.92} & 
2.95 & 3.47\\CON8-7 & 815.80 & 9.69 & 
819.39 & 10.18 & \bf{811.96} & 
0.47 & 0.92\\CON8-8 & 781.81 & 8.87 & 
788.47 & 7.59 & \bf{767.53} & 
1.86 & 2.73\\CON8-9 & 816.10 & 7.90 & 
818.43 & 8.42 & \bf{809.00} & 
0.88 & 1.17\\\bf{PROM.} & 
\bf{766.80} & \bf{6.83} & \bf{768.96} & \bf{6.68} & \bf{758.54} & \bf{1.01} & \bf{1.26}\\[1ex]\hline
\end{tabular}
\label{table:nonlin}
\end{table} \clearpage
\begin{table}[ht]
\caption{Resultados de la ejecución de la metaheurística SCA, utilizando instancias de Dethloff con la configuración -n 150.0 -b 10 -y .4}
\centering
\small
\begin{tabular}{c c c c c c c c}
\hline\hline
Instancia & Costo mínimo & Tiempo(seg.) & Costo promedio & Tiempo promedio(seg.) & CME & \%G & \%GP \\ [0.5ex]
\hline
SCA3-0 & 640.55 & 4.64 & 
640.55 & 4.23 & \bf{635.62} & 
0.78 & 0.78\\SCA3-1 & 701.53 & 4.41 & 
701.53 & 3.04 & \bf{697.84} & 
0.53 & 0.53\\SCA3-2 & 661.13 & 5.14 & 
663.57 & 4.07 & \bf{659.34} & 
0.27 & 0.64\\SCA3-3 & \bf{680.04} & 2.35 & 
680.04 & 2.67 & 680.04 & 0.00
 & 0.00\\
SCA3-4 & 692.57 & 2.98 & 
692.57 & 2.99 & \bf{690.50} & 
0.30 & 0.30\\SCA3-5 & 673.39 & 2.50 & 
675.48 & 2.65 & \bf{659.90} & 
2.04 & 2.36\\SCA3-6 & \bf{651.09} & 2.49 & 
653.16 & 4.07 & 651.09 & 0.00
 & 0.32\\SCA3-7 & 667.34 & 2.40 & 
667.34 & 2.48 & \bf{659.17} & 
1.24 & 1.24\\SCA3-8 & \bf{719.47} & 3.36 & 
719.47 & 3.27 & 719.47 & 0.00
 & 0.00\\
SCA3-9 & 685.88 & 2.92 & 
685.88 & 3.06 & \bf{681.00} & 
0.72 & 0.72\\SCA8-0 & 984.23 & 7.92 & 
988.63 & 9.93 & \bf{961.50} & 
2.36 & 2.82\\SCA8-1 & 1066.48 & 8.86 & 
1068.93 & 9.49 & \bf{1049.65} & 
1.60 & 1.84\\SCA8-2 & 1051.95 & 12.47 & 
1053.20 & 12.39 & \bf{1039.64} & 
1.18 & 1.30\\SCA8-3 & 1016.70 & 11.31 & 
1024.12 & 11.65 & \bf{983.34} & 
3.39 & 4.15\\SCA8-4 & 1067.82 & 8.68 & 
1082.77 & 9.48 & \bf{1065.49} & 
0.22 & 1.62\\SCA8-5 & 1065.51 & 7.23 & 
1065.51 & 7.01 & \bf{1027.08} & 
3.74 & 3.74\\SCA8-6 & 977.87 & 9.53 & 
977.87 & 10.35 & \bf{971.82} & 
0.62 & 0.62\\SCA8-7 & 1067.49 & 7.55 & 
1074.16 & 8.84 & \bf{1051.28} & 
1.54 & 2.18\\SCA8-8 & 1082.12 & 7.41 & 
1085.81 & 11.03 & \bf{1071.18} & 
1.02 & 1.37\\SCA8-9 & 1070.71 & 10.14 & 
1073.22 & 9.98 & \bf{1060.50} & 
0.96 & 1.20\\CON3-0 & 626.86 & 2.94 & 
628.36 & 2.92 & \bf{616.52} & 
1.68 & 1.92\\CON3-1 & 560.75 & 4.04 & 
560.75 & 4.11 & \bf{554.47} & 
1.13 & 1.13\\CON3-2 & 521.38 & 3.63 & 
521.57 & 2.30 & \bf{518.00} & 
0.65 & 0.69\\CON3-3 & 591.20 & 2.56 & 
591.52 & 3.10 & \bf{591.19} & 
0.00 & 0.06\\CON3-4 & 591.43 & 2.91 & 
591.43 & 2.75 & \bf{588.79} & 
0.45 & 0.45\\CON3-5 & 564.88 & 3.52 & 
565.09 & 3.52 & \bf{563.70} & 
0.21 & 0.25\\CON3-6 & 502.16 & 3.32 & 
502.16 & 3.64 & \bf{499.05} & 
0.62 & 0.62\\CON3-7 & 586.01 & 2.71 & 
586.01 & 2.83 & \bf{576.48} & 
1.65 & 1.65\\CON3-8 & \bf{523.05} & 2.60 & 
523.05 & 2.70 & 523.05 & 0.00
 & 0.00\\
CON3-9 & 588.11 & 1.97 & 
588.62 & 2.44 & \bf{578.24} & 
1.71 & 1.80\\CON8-0 & 864.32 & 8.22 & 
876.51 & 7.10 & \bf{857.17} & 
0.83 & 2.26\\CON8-1 & 744.12 & 8.54 & 
744.12 & 8.01 & \bf{740.85} & 
0.44 & 0.44\\CON8-2 & 713.10 & 17.23 & 
714.41 & 13.28 & \bf{712.89} & 
0.03 & 0.21\\CON8-3 & 820.89 & 7.77 & 
829.11 & 11.63 & \bf{811.07} & 
1.21 & 2.22\\CON8-4 & 786.70 & 12.90 & 
786.70 & 12.91 & \bf{772.25} & 
1.87 & 1.87\\CON8-5 & 759.87 & 10.58 & 
761.17 & 10.15 & \bf{754.88} & 
0.66 & 0.83\\CON8-6 & 691.32 & 10.31 & 
696.64 & 11.80 & \bf{678.92} & 
1.83 & 2.61\\CON8-7 & 814.79 & 12.10 & 
815.23 & 10.85 & \bf{811.96} & 
0.35 & 0.40\\CON8-8 & 785.21 & 12.14 & 
786.80 & 7.99 & \bf{767.53} & 
2.30 & 2.51\\CON8-9 & 822.81 & 11.21 & 
826.46 & 9.57 & \bf{809.00} & 
1.71 & 2.16\\\bf{PROM.} & 
\bf{767.07} & \bf{6.64} & \bf{769.24} & \bf{6.66} & \bf{758.54} & \bf{1.05} & \bf{1.30}\\[1ex]\hline
\end{tabular}
\label{table:nonlin}
\end{table} \clearpage
\begin{table}[ht]
\caption{Resultados de la ejecución de la metaheurística SCA, utilizando instancias de Dethloff con la configuración -n 150.0 -b 10 -y .5}
\centering
\small
\begin{tabular}{c c c c c c c c}
\hline\hline
Instancia & Costo mínimo & Tiempo(seg.) & Costo promedio & Tiempo promedio(seg.) & CME & \%G & \%GP \\ [0.5ex]
\hline
SCA3-0 & 640.55 & 3.86 & 
640.55 & 3.38 & \bf{635.62} & 
0.78 & 0.78\\SCA3-1 & 701.53 & 3.81 & 
701.53 & 3.39 & \bf{697.84} & 
0.53 & 0.53\\SCA3-2 & 665.71 & 3.46 & 
665.71 & 3.73 & \bf{659.34} & 
0.97 & 0.97\\SCA3-3 & 680.60 & 3.63 & 
681.17 & 3.03 & \bf{680.04} & 
0.08 & 0.17\\SCA3-4 & \bf{690.50} & 3.96 & 
690.50 & 4.01 & 690.50 & 0.00
 & 0.00\\
SCA3-5 & 666.67 & 2.46 & 
666.67 & 2.49 & \bf{659.90} & 
1.03 & 1.03\\SCA3-6 & 654.26 & 3.47 & 
654.26 & 3.30 & \bf{651.09} & 
0.49 & 0.49\\SCA3-7 & 666.60 & 2.77 & 
666.60 & 2.87 & \bf{659.17} & 
1.13 & 1.13\\SCA3-8 & \bf{719.47} & 3.44 & 
719.47 & 3.60 & 719.47 & 0.00
 & 0.00\\
SCA3-9 & 685.14 & 3.23 & 
685.14 & 3.45 & \bf{681.00} & 
0.61 & 0.61\\SCA8-0 & 984.96 & 12.25 & 
996.72 & 9.20 & \bf{961.50} & 
2.44 & 3.66\\SCA8-1 & 1065.72 & 10.93 & 
1068.84 & 11.06 & \bf{1049.65} & 
1.53 & 1.83\\SCA8-2 & 1051.95 & 11.50 & 
1052.69 & 11.47 & \bf{1039.64} & 
1.18 & 1.26\\SCA8-3 & 1017.21 & 8.32 & 
1017.76 & 8.99 & \bf{983.34} & 
3.44 & 3.50\\SCA8-4 & 1078.39 & 8.74 & 
1080.53 & 9.91 & \bf{1065.49} & 
1.21 & 1.41\\SCA8-5 & 1042.51 & 12.29 & 
1049.60 & 11.30 & \bf{1027.08} & 
1.50 & 2.19\\SCA8-6 & 972.48 & 9.30 & 
973.83 & 11.41 & \bf{971.82} & 
0.07 & 0.21\\SCA8-7 & 1067.49 & 10.12 & 
1073.93 & 8.94 & \bf{1051.28} & 
1.54 & 2.15\\SCA8-8 & \bf{1071.18} & 10.41 & 
1084.98 & 9.33 & 1071.18 & 0.00
 & 1.29\\SCA8-9 & 1077.03 & 9.42 & 
1080.55 & 9.20 & \bf{1060.50} & 
1.56 & 1.89\\CON3-0 & 633.24 & 2.59 & 
633.24 & 2.55 & \bf{616.52} & 
2.71 & 2.71\\CON3-1 & 563.72 & 2.45 & 
563.72 & 2.56 & \bf{554.47} & 
1.67 & 1.67\\CON3-2 & 521.38 & 2.43 & 
521.38 & 2.49 & \bf{518.00} & 
0.65 & 0.65\\CON3-3 & 591.20 & 2.67 & 
591.20 & 2.87 & \bf{591.19} & 
0.00 & 0.00\\CON3-4 & 591.43 & 3.16 & 
591.43 & 3.15 & \bf{588.79} & 
0.45 & 0.45\\CON3-5 & \bf{563.70} & 2.02 & 
567.71 & 3.19 & 563.70 & 0.00
 & 0.71\\CON3-6 & 502.16 & 3.63 & 
502.16 & 3.77 & \bf{499.05} & 
0.62 & 0.62\\CON3-7 & 586.01 & 2.79 & 
586.01 & 3.07 & \bf{576.48} & 
1.65 & 1.65\\CON3-8 & \bf{523.05} & 3.03 & 
523.12 & 3.15 & 523.05 & 0.00
 & 0.01\\CON3-9 & 588.99 & 2.15 & 
588.99 & 2.15 & \bf{578.24} & 
1.86 & 1.86\\CON8-0 & 873.65 & 9.32 & 
881.06 & 10.06 & \bf{857.17} & 
1.92 & 2.79\\CON8-1 & 748.30 & 8.64 & 
751.35 & 10.39 & \bf{740.85} & 
1.01 & 1.42\\CON8-2 & 713.05 & 12.39 & 
717.61 & 9.85 & \bf{712.89} & 
0.02 & 0.66\\CON8-3 & 826.06 & 7.75 & 
827.04 & 7.00 & \bf{811.07} & 
1.85 & 1.97\\CON8-4 & 776.98 & 11.42 & 
781.70 & 10.89 & \bf{772.25} & 
0.61 & 1.22\\CON8-5 & 760.82 & 11.83 & 
762.19 & 11.90 & \bf{754.88} & 
0.79 & 0.97\\CON8-6 & 698.41 & 10.35 & 
703.11 & 8.33 & \bf{678.92} & 
2.87 & 3.56\\CON8-7 & 814.79 & 10.55 & 
815.59 & 9.32 & \bf{811.96} & 
0.35 & 0.45\\CON8-8 & 782.82 & 8.76 & 
785.17 & 9.07 & \bf{767.53} & 
1.99 & 2.30\\CON8-9 & 818.17 & 12.75 & 
820.37 & 9.89 & \bf{809.00} & 
1.13 & 1.41\\\bf{PROM.} & 
\bf{766.95} & \bf{6.70} & \bf{769.13} & \bf{6.49} & \bf{758.54} & \bf{1.06} & \bf{1.30}\\[1ex]\hline
\end{tabular}
\label{table:nonlin}
\end{table} \clearpage
\begin{table}[ht]
\caption{Resultados de la ejecución de la metaheurística SCA, utilizando instancias de Dethloff con la configuración -n 150.0 -b 10 -y .6}
\centering
\small
\begin{tabular}{c c c c c c c c}
\hline\hline
Instancia & Costo mínimo & Tiempo(seg.) & Costo promedio & Tiempo promedio(seg.) & CME & \%G & \%GP \\ [0.5ex]
\hline
SCA3-0 & 640.55 & 4.02 & 
640.55 & 3.94 & \bf{635.62} & 
0.78 & 0.78\\SCA3-1 & 700.50 & 3.28 & 
700.50 & 3.22 & \bf{697.84} & 
0.38 & 0.38\\SCA3-2 & \bf{659.34} & 3.60 & 
659.34 & 4.16 & 659.34 & 0.00
 & 0.00\\
SCA3-3 & \bf{680.04} & 2.66 & 
681.03 & 4.04 & 680.04 & 0.00
 & 0.15\\SCA3-4 & 692.57 & 3.29 & 
692.57 & 2.96 & \bf{690.50} & 
0.30 & 0.30\\SCA3-5 & 668.63 & 3.85 & 
677.57 & 3.07 & \bf{659.90} & 
1.32 & 2.68\\SCA3-6 & 653.68 & 2.44 & 
654.39 & 2.60 & \bf{651.09} & 
0.40 & 0.51\\SCA3-7 & 666.15 & 2.89 & 
668.91 & 3.30 & \bf{659.17} & 
1.06 & 1.48\\SCA3-8 & \bf{719.47} & 2.78 & 
719.47 & 3.38 & 719.47 & 0.00
 & 0.00\\
SCA3-9 & \bf{681.00} & 3.47 & 
683.08 & 3.57 & 681.00 & 0.00
 & 0.31\\SCA8-0 & 970.64 & 11.38 & 
984.31 & 9.75 & \bf{961.50} & 
0.95 & 2.37\\SCA8-1 & 1068.14 & 9.93 & 
1069.01 & 9.45 & \bf{1049.65} & 
1.76 & 1.84\\SCA8-2 & 1051.21 & 11.84 & 
1052.49 & 11.02 & \bf{1039.64} & 
1.11 & 1.24\\SCA8-3 & 1026.47 & 9.17 & 
1026.47 & 8.69 & \bf{983.34} & 
4.39 & 4.39\\SCA8-4 & 1077.80 & 9.80 & 
1088.32 & 21.95 & \bf{1065.49} & 
1.16 & 2.14\\SCA8-5 & 1049.98 & 14.68 & 
1051.71 & 10.22 & \bf{1027.08} & 
2.23 & 2.40\\SCA8-6 & 972.48 & 15.61 & 
978.68 & 10.49 & \bf{971.82} & 
0.07 & 0.71\\SCA8-7 & 1066.65 & 8.94 & 
1072.90 & 9.57 & \bf{1051.28} & 
1.46 & 2.06\\SCA8-8 & 1090.66 & 9.63 & 
1092.64 & 9.96 & \bf{1071.18} & 
1.82 & 2.00\\SCA8-9 & 1085.11 & 8.25 & 
1092.28 & 7.62 & \bf{1060.50} & 
2.32 & 3.00\\CON3-0 & 627.46 & 2.46 & 
627.46 & 2.09 & \bf{616.52} & 
1.77 & 1.77\\CON3-1 & 557.21 & 3.50 & 
557.21 & 2.51 & \bf{554.47} & 
0.49 & 0.49\\CON3-2 & 521.38 & 2.56 & 
521.38 & 2.59 & \bf{518.00} & 
0.65 & 0.65\\CON3-3 & 591.20 & 3.44 & 
591.88 & 3.06 & \bf{591.19} & 
0.00 & 0.12\\CON3-4 & 597.08 & 1.91 & 
597.08 & 1.97 & \bf{588.79} & 
1.41 & 1.41\\CON3-5 & \bf{563.70} & 2.03 & 
567.66 & 1.69 & 563.70 & 0.00
 & 0.70\\CON3-6 & 503.97 & 9.19 & 
506.25 & 4.44 & \bf{499.05} & 
0.99 & 1.44\\CON3-7 & 578.41 & 3.83 & 
584.11 & 2.94 & \bf{576.48} & 
0.33 & 1.32\\CON3-8 & \bf{523.05} & 2.04 & 
523.05 & 3.62 & 523.05 & 0.00
 & 0.00\\
CON3-9 & 588.40 & 2.55 & 
588.40 & 3.08 & \bf{578.24} & 
1.76 & 1.76\\CON8-0 & 879.55 & 8.23 & 
887.02 & 8.43 & \bf{857.17} & 
2.61 & 3.48\\CON8-1 & 742.28 & 8.97 & 
745.37 & 11.51 & \bf{740.85} & 
0.19 & 0.61\\CON8-2 & 712.94 & 11.39 & 
719.68 & 10.93 & \bf{712.89} & 
0.01 & 0.95\\CON8-3 & 826.80 & 10.31 & 
829.73 & 10.86 & \bf{811.07} & 
1.94 & 2.30\\CON8-4 & 790.77 & 8.48 & 
791.92 & 9.24 & \bf{772.25} & 
2.40 & 2.55\\CON8-5 & 758.12 & 12.22 & 
762.01 & 12.21 & \bf{754.88} & 
0.43 & 0.94\\CON8-6 & 697.89 & 9.64 & 
698.78 & 8.84 & \bf{678.92} & 
2.79 & 2.92\\CON8-7 & 814.79 & 12.18 & 
818.53 & 10.09 & \bf{811.96} & 
0.35 & 0.81\\CON8-8 & 783.48 & 10.95 & 
790.74 & 8.43 & \bf{767.53} & 
2.08 & 3.02\\CON8-9 & 814.87 & 8.58 & 
823.60 & 9.84 & \bf{809.00} & 
0.73 & 1.80\\\bf{PROM.} & 
\bf{767.36} & \bf{6.90} & \bf{770.45} & \bf{6.78} & \bf{758.54} & \bf{1.06} & \bf{1.44}\\[1ex]\hline
\end{tabular}
\label{table:nonlin}
\end{table} \clearpage
\begin{table}[ht]
\caption{Resultados de la ejecución de la metaheurística SCA, utilizando instancias de Dethloff con la configuración -n 150.0 -b 10 -y .7}
\centering
\small
\begin{tabular}{c c c c c c c c}
\hline\hline
Instancia & Costo mínimo & Tiempo(seg.) & Costo promedio & Tiempo promedio(seg.) & CME & \%G & \%GP \\ [0.5ex]
\hline
SCA3-0 & 640.55 & 4.24 & 
640.55 & 4.01 & \bf{635.62} & 
0.78 & 0.78\\SCA3-1 & 701.53 & 2.32 & 
701.53 & 2.33 & \bf{697.84} & 
0.53 & 0.53\\SCA3-2 & 664.18 & 6.25 & 
664.56 & 5.24 & \bf{659.34} & 
0.73 & 0.79\\SCA3-3 & 680.60 & 3.59 & 
680.60 & 3.61 & \bf{680.04} & 
0.08 & 0.08\\SCA3-4 & 692.57 & 3.58 & 
693.80 & 3.32 & \bf{690.50} & 
0.30 & 0.48\\SCA3-5 & 670.10 & 2.85 & 
670.10 & 2.96 & \bf{659.90} & 
1.55 & 1.55\\SCA3-6 & \bf{651.09} & 3.81 & 
654.94 & 2.76 & 651.09 & 0.00
 & 0.59\\SCA3-7 & 666.15 & 3.31 & 
667.75 & 3.68 & \bf{659.17} & 
1.06 & 1.30\\SCA3-8 & \bf{719.47} & 4.10 & 
719.47 & 4.11 & 719.47 & 0.00
 & 0.00\\
SCA3-9 & 685.14 & 3.60 & 
685.14 & 3.75 & \bf{681.00} & 
0.61 & 0.61\\SCA8-0 & 989.76 & 6.49 & 
993.90 & 6.72 & \bf{961.50} & 
2.94 & 3.37\\SCA8-1 & 1053.44 & 9.44 & 
1068.38 & 8.63 & \bf{1049.65} & 
0.36 & 1.78\\SCA8-2 & 1052.74 & 11.09 & 
1053.75 & 13.54 & \bf{1039.64} & 
1.26 & 1.36\\SCA8-3 & 1027.60 & 15.41 & 
1036.85 & 9.95 & \bf{983.34} & 
4.50 & 5.44\\SCA8-4 & 1085.73 & 12.69 & 
1095.72 & 10.54 & \bf{1065.49} & 
1.90 & 2.84\\SCA8-5 & 1050.44 & 18.35 & 
1060.36 & 11.33 & \bf{1027.08} & 
2.27 & 3.24\\SCA8-6 & 972.48 & 12.00 & 
975.17 & 12.30 & \bf{971.82} & 
0.07 & 0.35\\SCA8-7 & 1067.49 & 10.56 & 
1074.74 & 11.26 & \bf{1051.28} & 
1.54 & 2.23\\SCA8-8 & \bf{1071.18} & 9.09 & 
1079.14 & 8.83 & 1071.18 & 0.00
 & 0.74\\SCA8-9 & 1070.71 & 10.26 & 
1078.79 & 8.55 & \bf{1060.50} & 
0.96 & 1.72\\CON3-0 & 631.81 & 2.97 & 
632.74 & 2.48 & \bf{616.52} & 
2.48 & 2.63\\CON3-1 & 556.79 & 2.32 & 
559.76 & 3.66 & \bf{554.47} & 
0.42 & 0.95\\CON3-2 & 521.38 & 2.72 & 
521.38 & 2.99 & \bf{518.00} & 
0.65 & 0.65\\CON3-3 & \bf{591.19} & 3.38 & 
591.41 & 3.34 & 591.19 & 0.00
 & 0.04\\CON3-4 & 591.43 & 3.40 & 
592.00 & 3.52 & \bf{588.79} & 
0.45 & 0.55\\CON3-5 & 564.88 & 2.76 & 
564.88 & 2.43 & \bf{563.70} & 
0.21 & 0.21\\CON3-6 & 502.16 & 3.42 & 
502.16 & 3.44 & \bf{499.05} & 
0.62 & 0.62\\CON3-7 & 586.01 & 2.90 & 
586.01 & 2.45 & \bf{576.48} & 
1.65 & 1.65\\CON3-8 & \bf{523.05} & 2.58 & 
526.40 & 2.39 & 523.05 & 0.00
 & 0.64\\CON3-9 & 588.40 & 2.61 & 
588.40 & 2.67 & \bf{578.24} & 
1.76 & 1.76\\CON8-0 & 888.38 & 9.58 & 
890.67 & 9.87 & \bf{857.17} & 
3.64 & 3.91\\CON8-1 & 748.85 & 15.08 & 
749.15 & 13.44 & \bf{740.85} & 
1.08 & 1.12\\CON8-2 & \bf{712.89} & 11.07 & 
716.25 & 12.81 & 712.89 & 0.00
 & 0.47\\CON8-3 & 823.77 & 13.42 & 
830.84 & 11.03 & \bf{811.07} & 
1.57 & 2.44\\CON8-4 & 788.77 & 11.71 & 
791.41 & 9.32 & \bf{772.25} & 
2.14 & 2.48\\CON8-5 & 763.83 & 7.92 & 
764.80 & 8.91 & \bf{754.88} & 
1.19 & 1.31\\CON8-6 & 693.30 & 15.01 & 
700.19 & 10.95 & \bf{678.92} & 
2.12 & 3.13\\CON8-7 & 814.50 & 15.74 & 
816.75 & 12.00 & \bf{811.96} & 
0.31 & 0.59\\CON8-8 & 780.33 & 9.67 & 
788.48 & 9.66 & \bf{767.53} & 
1.67 & 2.73\\CON8-9 & 815.17 & 11.14 & 
823.52 & 10.04 & \bf{809.00} & 
0.76 & 1.79\\\bf{PROM.} & 
\bf{767.50} & \bf{7.56} & \bf{770.81} & \bf{6.87} & \bf{758.54} & \bf{1.10} & \bf{1.49}\\[1ex]\hline
\end{tabular}
\label{table:nonlin}
\end{table} \clearpage
\begin{table}[ht]
\caption{Resultados de la ejecución de la metaheurística SCA, utilizando instancias de Dethloff con la configuración -n 150.0 -b 10 -y .8}
\centering
\small
\begin{tabular}{c c c c c c c c}
\hline\hline
Instancia & Costo mínimo & Tiempo(seg.) & Costo promedio & Tiempo promedio(seg.) & CME & \%G & \%GP \\ [0.5ex]
\hline
SCA3-0 & 640.55 & 4.34 & 
640.55 & 4.53 & \bf{635.62} & 
0.78 & 0.78\\SCA3-1 & 701.74 & 3.29 & 
701.74 & 3.50 & \bf{697.84} & 
0.56 & 0.56\\SCA3-2 & 666.01 & 5.60 & 
667.55 & 4.73 & \bf{659.34} & 
1.01 & 1.25\\SCA3-3 & 680.60 & 4.34 & 
680.88 & 4.54 & \bf{680.04} & 
0.08 & 0.12\\SCA3-4 & \bf{690.50} & 3.51 & 
691.02 & 3.15 & 690.50 & 0.00
 & 0.08\\SCA3-5 & 665.04 & 4.85 & 
665.04 & 5.09 & \bf{659.90} & 
0.78 & 0.78\\SCA3-6 & 652.94 & 2.45 & 
654.11 & 3.35 & \bf{651.09} & 
0.28 & 0.46\\SCA3-7 & 666.60 & 3.49 & 
666.60 & 3.53 & \bf{659.17} & 
1.13 & 1.13\\SCA3-8 & \bf{719.47} & 3.01 & 
719.47 & 3.00 & 719.47 & 0.00
 & 0.00\\
SCA3-9 & \bf{681.00} & 3.68 & 
682.03 & 7.55 & 681.00 & 0.00
 & 0.15\\SCA8-0 & 975.50 & 12.06 & 
982.85 & 10.49 & \bf{961.50} & 
1.46 & 2.22\\SCA8-1 & 1057.41 & 11.25 & 
1067.15 & 10.30 & \bf{1049.65} & 
0.74 & 1.67\\SCA8-2 & 1051.95 & 16.24 & 
1053.82 & 14.23 & \bf{1039.64} & 
1.18 & 1.36\\SCA8-3 & 1023.70 & 10.42 & 
1025.79 & 15.70 & \bf{983.34} & 
4.10 & 4.32\\SCA8-4 & 1077.80 & 8.38 & 
1077.80 & 8.72 & \bf{1065.49} & 
1.16 & 1.16\\SCA8-5 & 1048.65 & 11.59 & 
1051.88 & 11.34 & \bf{1027.08} & 
2.10 & 2.41\\SCA8-6 & 972.48 & 11.81 & 
974.03 & 11.78 & \bf{971.82} & 
0.07 & 0.23\\SCA8-7 & 1067.49 & 9.60 & 
1067.49 & 10.29 & \bf{1051.28} & 
1.54 & 1.54\\SCA8-8 & 1086.54 & 10.04 & 
1090.86 & 11.71 & \bf{1071.18} & 
1.43 & 1.84\\SCA8-9 & 1070.71 & 8.04 & 
1073.38 & 7.99 & \bf{1060.50} & 
0.96 & 1.21\\CON3-0 & 620.49 & 2.15 & 
620.49 & 2.25 & \bf{616.52} & 
0.64 & 0.64\\CON3-1 & 560.75 & 2.98 & 
560.75 & 3.71 & \bf{554.47} & 
1.13 & 1.13\\CON3-2 & 521.38 & 3.80 & 
521.38 & 4.10 & \bf{518.00} & 
0.65 & 0.65\\CON3-3 & 591.20 & 3.01 & 
591.20 & 3.03 & \bf{591.19} & 
0.00 & 0.00\\CON3-4 & \bf{588.79} & 2.82 & 
590.77 & 5.26 & 588.79 & 0.00
 & 0.34\\CON3-5 & \bf{563.70} & 2.65 & 
563.70 & 2.62 & 563.70 & 0.00
 & 0.00\\
CON3-6 & 502.16 & 3.43 & 
502.83 & 3.38 & \bf{499.05} & 
0.62 & 0.76\\CON3-7 & 582.12 & 2.96 & 
584.07 & 2.72 & \bf{576.48} & 
0.98 & 1.32\\CON3-8 & 524.30 & 3.90 & 
524.76 & 2.96 & \bf{523.05} & 
0.24 & 0.33\\CON3-9 & 588.40 & 3.71 & 
588.40 & 2.93 & \bf{578.24} & 
1.76 & 1.76\\CON8-0 & 883.67 & 6.31 & 
887.77 & 9.30 & \bf{857.17} & 
3.09 & 3.57\\CON8-1 & 740.93 & 13.32 & 
748.61 & 11.79 & \bf{740.85} & 
0.01 & 1.05\\CON8-2 & 713.05 & 13.97 & 
715.04 & 10.11 & \bf{712.89} & 
0.02 & 0.30\\CON8-3 & 828.40 & 11.28 & 
830.79 & 9.94 & \bf{811.07} & 
2.14 & 2.43\\CON8-4 & 784.29 & 13.98 & 
793.98 & 10.29 & \bf{772.25} & 
1.56 & 2.81\\CON8-5 & 762.36 & 8.92 & 
763.10 & 9.86 & \bf{754.88} & 
0.99 & 1.09\\CON8-6 & 692.48 & 8.41 & 
700.29 & 8.39 & \bf{678.92} & 
2.00 & 3.15\\CON8-7 & 815.43 & 12.00 & 
818.68 & 12.45 & \bf{811.96} & 
0.43 & 0.83\\CON8-8 & 779.55 & 10.75 & 
786.42 & 10.36 & \bf{767.53} & 
1.57 & 2.46\\CON8-9 & 822.84 & 10.58 & 
828.76 & 9.81 & \bf{809.00} & 
1.71 & 2.44\\\bf{PROM.} & 
\bf{766.57} & \bf{7.22} & \bf{768.90} & \bf{7.27} & \bf{758.54} & \bf{0.97} & \bf{1.26}\\[1ex]\hline
\end{tabular}
\label{table:nonlin}
\end{table} \clearpage
\begin{table}[ht]
\caption{Resultados de la ejecución de la metaheurística SCA, utilizando instancias de Dethloff con la configuración -n 150.0 -b 10 -y .9}
\centering
\small
\begin{tabular}{c c c c c c c c}
\hline\hline
Instancia & Costo mínimo & Tiempo(seg.) & Costo promedio & Tiempo promedio(seg.) & CME & \%G & \%GP \\ [0.5ex]
\hline
SCA3-0 & 640.55 & 5.05 & 
640.55 & 4.55 & \bf{635.62} & 
0.78 & 0.78\\SCA3-1 & 701.86 & 2.99 & 
701.86 & 3.09 & \bf{697.84} & 
0.58 & 0.58\\SCA3-2 & 661.13 & 4.64 & 
665.14 & 4.19 & \bf{659.34} & 
0.27 & 0.88\\SCA3-3 & 680.60 & 3.25 & 
680.78 & 3.66 & \bf{680.04} & 
0.08 & 0.11\\SCA3-4 & 692.57 & 2.72 & 
692.57 & 2.61 & \bf{690.50} & 
0.30 & 0.30\\SCA3-5 & 681.03 & 3.08 & 
681.23 & 3.04 & \bf{659.90} & 
3.20 & 3.23\\SCA3-6 & \bf{651.09} & 2.99 & 
652.94 & 3.77 & 651.09 & 0.00
 & 0.28\\SCA3-7 & 669.89 & 3.95 & 
671.23 & 3.31 & \bf{659.17} & 
1.63 & 1.83\\SCA3-8 & 719.77 & 2.82 & 
719.77 & 3.36 & \bf{719.47} & 
0.04 & 0.04\\SCA3-9 & \bf{681.00} & 3.38 & 
682.05 & 3.70 & 681.00 & 0.00
 & 0.15\\SCA8-0 & 972.59 & 11.28 & 
989.13 & 9.96 & \bf{961.50} & 
1.15 & 2.87\\SCA8-1 & 1053.09 & 6.12 & 
1062.14 & 7.54 & \bf{1049.65} & 
0.33 & 1.19\\SCA8-2 & 1051.95 & 14.10 & 
1053.38 & 13.27 & \bf{1039.64} & 
1.18 & 1.32\\SCA8-3 & 1016.80 & 8.88 & 
1027.43 & 9.54 & \bf{983.34} & 
3.40 & 4.48\\SCA8-4 & \bf{1065.49} & 15.81 & 
1076.09 & 12.20 & 1065.49 & 0.00
 & 0.99\\SCA8-5 & 1040.18 & 12.83 & 
1049.29 & 11.94 & \bf{1027.08} & 
1.28 & 2.16\\SCA8-6 & 976.74 & 8.17 & 
978.46 & 8.75 & \bf{971.82} & 
0.51 & 0.68\\SCA8-7 & 1066.82 & 10.21 & 
1070.63 & 10.04 & \bf{1051.28} & 
1.48 & 1.84\\SCA8-8 & 1086.54 & 12.33 & 
1092.43 & 12.36 & \bf{1071.18} & 
1.43 & 1.98\\SCA8-9 & 1071.18 & 11.96 & 
1078.78 & 11.20 & \bf{1060.50} & 
1.01 & 1.72\\CON3-0 & 633.55 & 2.24 & 
633.55 & 2.35 & \bf{616.52} & 
2.76 & 2.76\\CON3-1 & 560.75 & 4.28 & 
560.75 & 3.43 & \bf{554.47} & 
1.13 & 1.13\\CON3-2 & 521.38 & 2.37 & 
521.38 & 2.60 & \bf{518.00} & 
0.65 & 0.65\\CON3-3 & 591.20 & 3.44 & 
591.20 & 3.27 & \bf{591.19} & 
0.00 & 0.00\\CON3-4 & 591.43 & 3.59 & 
591.43 & 3.64 & \bf{588.79} & 
0.45 & 0.45\\CON3-5 & 569.88 & 2.42 & 
569.88 & 2.36 & \bf{563.70} & 
1.10 & 1.10\\CON3-6 & 502.16 & 2.35 & 
502.45 & 2.84 & \bf{499.05} & 
0.62 & 0.68\\CON3-7 & 578.41 & 3.25 & 
582.21 & 3.12 & \bf{576.48} & 
0.33 & 0.99\\CON3-8 & \bf{523.05} & 3.92 & 
523.05 & 4.31 & 523.05 & 0.00
 & 0.00\\
CON3-9 & 588.40 & 3.01 & 
588.40 & 3.60 & \bf{578.24} & 
1.76 & 1.76\\CON8-0 & 888.97 & 8.82 & 
889.18 & 6.81 & \bf{857.17} & 
3.71 & 3.73\\CON8-1 & 744.11 & 9.86 & 
745.29 & 12.30 & \bf{740.85} & 
0.44 & 0.60\\CON8-2 & 713.05 & 9.75 & 
715.27 & 10.54 & \bf{712.89} & 
0.02 & 0.33\\CON8-3 & 822.84 & 13.89 & 
829.34 & 10.95 & \bf{811.07} & 
1.45 & 2.25\\CON8-4 & 788.77 & 13.70 & 
791.64 & 10.72 & \bf{772.25} & 
2.14 & 2.51\\CON8-5 & 754.95 & 10.15 & 
758.48 & 10.99 & \bf{754.88} & 
0.01 & 0.48\\CON8-6 & 688.20 & 8.70 & 
696.13 & 8.81 & \bf{678.92} & 
1.37 & 2.53\\CON8-7 & 815.79 & 14.61 & 
817.53 & 12.53 & \bf{811.96} & 
0.47 & 0.69\\CON8-8 & 789.92 & 8.92 & 
791.08 & 9.18 & \bf{767.53} & 
2.92 & 3.07\\CON8-9 & 820.55 & 10.30 & 
823.99 & 9.12 & \bf{809.00} & 
1.43 & 1.85\\\bf{PROM.} & 
\bf{766.71} & \bf{7.15} & \bf{769.70} & \bf{6.89} & \bf{758.54} & \bf{1.04} & \bf{1.38}\\[1ex]\hline
\end{tabular}
\label{table:nonlin}
\end{table} \clearpage
\begin{table}[ht]
\caption{Resultados de la ejecución de la metaheurística SCA, utilizando instancias de Dethloff con la configuración -n 150.0 -b 10 -y 1.0}
\centering
\small
\begin{tabular}{c c c c c c c c}
\hline\hline
Instancia & Costo mínimo & Tiempo(seg.) & Costo promedio & Tiempo promedio(seg.) & CME & \%G & \%GP \\ [0.5ex]
\hline
SCA3-0 & 640.55 & 3.70 & 
640.55 & 3.66 & \bf{635.62} & 
0.78 & 0.78\\SCA3-1 & 700.50 & 3.67 & 
700.76 & 3.89 & \bf{697.84} & 
0.38 & 0.42\\SCA3-2 & \bf{659.34} & 3.85 & 
661.18 & 4.59 & 659.34 & 0.00
 & 0.28\\SCA3-3 & 681.74 & 1.48 & 
681.74 & 3.11 & \bf{680.04} & 
0.25 & 0.25\\SCA3-4 & \bf{690.50} & 4.41 & 
692.55 & 3.50 & 690.50 & 0.00
 & 0.30\\SCA3-5 & 679.54 & 3.07 & 
679.54 & 3.19 & \bf{659.90} & 
2.98 & 2.98\\SCA3-6 & 652.94 & 3.59 & 
653.43 & 3.64 & \bf{651.09} & 
0.28 & 0.36\\SCA3-7 & 666.60 & 3.26 & 
666.60 & 3.52 & \bf{659.17} & 
1.13 & 1.13\\SCA3-8 & \bf{719.47} & 4.73 & 
719.47 & 3.77 & 719.47 & 0.00
 & 0.00\\
SCA3-9 & \bf{681.00} & 3.31 & 
681.00 & 3.73 & 681.00 & 0.00
 & 0.00\\
SCA8-0 & 975.50 & 13.43 & 
986.08 & 10.58 & \bf{961.50} & 
1.46 & 2.56\\SCA8-1 & 1050.38 & 8.27 & 
1055.87 & 10.00 & \bf{1049.65} & 
0.07 & 0.59\\SCA8-2 & 1052.56 & 13.52 & 
1053.19 & 11.59 & \bf{1039.64} & 
1.24 & 1.30\\SCA8-3 & 1013.56 & 10.74 & 
1026.66 & 12.17 & \bf{983.34} & 
3.07 & 4.41\\SCA8-4 & 1078.86 & 16.96 & 
1089.70 & 11.71 & \bf{1065.49} & 
1.25 & 2.27\\SCA8-5 & 1043.17 & 16.91 & 
1050.84 & 13.12 & \bf{1027.08} & 
1.57 & 2.31\\SCA8-6 & 972.48 & 10.70 & 
986.46 & 19.02 & \bf{971.82} & 
0.07 & 1.51\\SCA8-7 & 1070.53 & 9.59 & 
1073.11 & 10.43 & \bf{1051.28} & 
1.83 & 2.08\\SCA8-8 & 1082.11 & 10.09 & 
1089.93 & 9.30 & \bf{1071.18} & 
1.02 & 1.75\\SCA8-9 & 1081.06 & 8.18 & 
1091.49 & 8.29 & \bf{1060.50} & 
1.94 & 2.92\\CON3-0 & 630.73 & 1.56 & 
630.80 & 1.32 & \bf{616.52} & 
2.30 & 2.32\\CON3-1 & 559.25 & 2.53 & 
559.25 & 2.67 & \bf{554.47} & 
0.86 & 0.86\\CON3-2 & 521.38 & 3.72 & 
521.38 & 3.72 & \bf{518.00} & 
0.65 & 0.65\\CON3-3 & 591.20 & 3.50 & 
591.51 & 3.25 & \bf{591.19} & 
0.00 & 0.05\\CON3-4 & 591.43 & 3.47 & 
591.43 & 3.75 & \bf{588.79} & 
0.45 & 0.45\\CON3-5 & \bf{563.70} & 2.43 & 
563.70 & 2.48 & 563.70 & 0.00
 & 0.00\\
CON3-6 & 502.16 & 2.76 & 
502.16 & 2.53 & \bf{499.05} & 
0.62 & 0.62\\CON3-7 & 586.01 & 3.24 & 
586.01 & 2.83 & \bf{576.48} & 
1.65 & 1.65\\CON3-8 & \bf{523.05} & 1.57 & 
523.67 & 1.96 & 523.05 & 0.00
 & 0.12\\CON3-9 & 588.40 & 2.10 & 
588.84 & 2.26 & \bf{578.24} & 
1.76 & 1.83\\CON8-0 & 883.76 & 8.87 & 
893.42 & 8.67 & \bf{857.17} & 
3.10 & 4.23\\CON8-1 & 748.39 & 13.46 & 
752.37 & 9.28 & \bf{740.85} & 
1.02 & 1.56\\CON8-2 & 713.10 & 10.38 & 
713.48 & 7.96 & \bf{712.89} & 
0.03 & 0.08\\CON8-3 & 817.57 & 6.82 & 
824.40 & 7.97 & \bf{811.07} & 
0.80 & 1.64\\CON8-4 & 785.55 & 10.03 & 
789.18 & 9.54 & \bf{772.25} & 
1.72 & 2.19\\CON8-5 & 760.41 & 10.04 & 
762.05 & 11.29 & \bf{754.88} & 
0.73 & 0.95\\CON8-6 & 698.32 & 10.54 & 
700.23 & 8.08 & \bf{678.92} & 
2.86 & 3.14\\CON8-7 & 814.77 & 11.90 & 
815.58 & 11.44 & \bf{811.96} & 
0.35 & 0.45\\CON8-8 & 783.92 & 9.54 & 
788.04 & 9.42 & \bf{767.53} & 
2.14 & 2.67\\CON8-9 & 814.45 & 11.82 & 
825.20 & 11.40 & \bf{809.00} & 
0.67 & 2.00\\\bf{PROM.} & 
\bf{766.75} & \bf{7.09} & \bf{770.07} & \bf{6.87} & \bf{758.54} & \bf{1.03} & \bf{1.39}\\[1ex]\hline
\end{tabular}
\label{table:nonlin}
\end{table} \clearpage
\begin{table}[ht]
\caption{Resultados de la ejecución de la metaheurística SCA, utilizando instancias de Dethloff con la configuración -n 200.0 -b 10 -y 0.1}
\centering
\small
\begin{tabular}{c c c c c c c c}
\hline\hline
Instancia & Costo mínimo & Tiempo(seg.) & Costo promedio & Tiempo promedio(seg.) & CME & \%G & \%GP \\ [0.5ex]
\hline
SCA3-0 & 640.55 & 4.24 & 
640.55 & 4.36 & \bf{635.62} & 
0.78 & 0.78\\SCA3-1 & 701.53 & 2.27 & 
701.53 & 2.25 & \bf{697.84} & 
0.53 & 0.53\\SCA3-2 & 666.01 & 4.21 & 
666.01 & 4.21 & \bf{659.34} & 
1.01 & 1.01\\SCA3-3 & \bf{680.04} & 4.18 & 
680.18 & 4.09 & 680.04 & 0.00
 & 0.02\\SCA3-4 & 692.57 & 2.71 & 
692.90 & 3.06 & \bf{690.50} & 
0.30 & 0.35\\SCA3-5 & 661.07 & 2.44 & 
671.44 & 2.52 & \bf{659.90} & 
0.18 & 1.75\\SCA3-6 & 652.94 & 3.49 & 
653.56 & 3.64 & \bf{651.09} & 
0.28 & 0.38\\SCA3-7 & 671.67 & 3.16 & 
671.67 & 3.15 & \bf{659.17} & 
1.90 & 1.90\\SCA3-8 & \bf{719.47} & 3.54 & 
719.47 & 3.70 & 719.47 & 0.00
 & 0.00\\
SCA3-9 & 685.14 & 2.04 & 
685.14 & 2.04 & \bf{681.00} & 
0.61 & 0.61\\SCA8-0 & \bf{961.50} & 10.30 & 
978.85 & 9.79 & 961.50 & 0.00
 & 1.80\\SCA8-1 & 1054.11 & 8.94 & 
1059.39 & 7.36 & \bf{1049.65} & 
0.42 & 0.93\\SCA8-2 & 1054.05 & 13.28 & 
1054.53 & 11.25 & \bf{1039.64} & 
1.39 & 1.43\\SCA8-3 & 1020.07 & 6.18 & 
1020.07 & 6.40 & \bf{983.34} & 
3.74 & 3.74\\SCA8-4 & 1080.56 & 10.98 & 
1085.17 & 10.30 & \bf{1065.49} & 
1.41 & 1.85\\SCA8-5 & 1050.64 & 9.81 & 
1052.03 & 10.15 & \bf{1027.08} & 
2.29 & 2.43\\SCA8-6 & 972.48 & 6.51 & 
975.81 & 10.44 & \bf{971.82} & 
0.07 & 0.41\\SCA8-7 & 1063.22 & 12.64 & 
1066.42 & 12.82 & \bf{1051.28} & 
1.14 & 1.44\\SCA8-8 & 1082.67 & 8.24 & 
1089.82 & 9.41 & \bf{1071.18} & 
1.07 & 1.74\\SCA8-9 & 1070.71 & 11.35 & 
1072.97 & 11.39 & \bf{1060.50} & 
0.96 & 1.18\\CON3-0 & 626.78 & 4.43 & 
628.23 & 3.07 & \bf{616.52} & 
1.66 & 1.90\\CON3-1 & 560.75 & 3.30 & 
560.75 & 3.11 & \bf{554.47} & 
1.13 & 1.13\\CON3-2 & 521.38 & 2.56 & 
522.49 & 3.06 & \bf{518.00} & 
0.65 & 0.87\\CON3-3 & 591.20 & 2.36 & 
591.20 & 2.03 & \bf{591.19} & 
0.00 & 0.00\\CON3-4 & 591.43 & 3.71 & 
591.43 & 3.35 & \bf{588.79} & 
0.45 & 0.45\\CON3-5 & 564.88 & 2.96 & 
564.88 & 3.19 & \bf{563.70} & 
0.21 & 0.21\\CON3-6 & 502.16 & 3.84 & 
502.16 & 2.81 & \bf{499.05} & 
0.62 & 0.62\\CON3-7 & 586.01 & 3.82 & 
586.01 & 3.87 & \bf{576.48} & 
1.65 & 1.65\\CON3-8 & \bf{523.05} & 1.84 & 
523.10 & 2.90 & 523.05 & 0.00
 & 0.01\\CON3-9 & 578.25 & 2.38 & 
580.93 & 2.71 & \bf{578.24} & 
0.00 & 0.47\\CON8-0 & 886.54 & 7.81 & 
887.72 & 9.01 & \bf{857.17} & 
3.43 & 3.56\\CON8-1 & 748.85 & 16.37 & 
749.15 & 12.65 & \bf{740.85} & 
1.08 & 1.12\\CON8-2 & 716.16 & 11.55 & 
717.87 & 11.56 & \bf{712.89} & 
0.46 & 0.70\\CON8-3 & 827.88 & 15.25 & 
835.83 & 10.00 & \bf{811.07} & 
2.07 & 3.05\\CON8-4 & 777.24 & 8.98 & 
782.09 & 10.47 & \bf{772.25} & 
0.65 & 1.27\\CON8-5 & 764.94 & 10.31 & 
764.94 & 9.48 & \bf{754.88} & 
1.33 & 1.33\\CON8-6 & 693.95 & 6.87 & 
696.87 & 7.18 & \bf{678.92} & 
2.21 & 2.64\\CON8-7 & 815.54 & 8.69 & 
815.54 & 9.02 & \bf{811.96} & 
0.44 & 0.44\\CON8-8 & 785.74 & 9.62 & 
788.77 & 8.96 & \bf{767.53} & 
2.37 & 2.77\\CON8-9 & 815.75 & 6.22 & 
817.78 & 7.15 & \bf{809.00} & 
0.83 & 1.09\\\bf{PROM.} & 
\bf{766.49} & \bf{6.58} & \bf{768.63} & \bf{6.45} & \bf{758.54} & \bf{0.98} & \bf{1.24}\\[1ex]\hline
\end{tabular}
\label{table:nonlin}
\end{table} \clearpage
\begin{table}[ht]
\caption{Resultados de la ejecución de la metaheurística SCA, utilizando instancias de Dethloff con la configuración -n 200.0 -b 10 -y .2}
\centering
\small
\begin{tabular}{c c c c c c c c}
\hline\hline
Instancia & Costo mínimo & Tiempo(seg.) & Costo promedio & Tiempo promedio(seg.) & CME & \%G & \%GP \\ [0.5ex]
\hline
SCA3-0 & 640.55 & 2.77 & 
640.55 & 2.60 & \bf{635.62} & 
0.78 & 0.78\\SCA3-1 & 700.50 & 4.27 & 
700.50 & 4.02 & \bf{697.84} & 
0.38 & 0.38\\SCA3-2 & 661.13 & 4.30 & 
663.02 & 4.34 & \bf{659.34} & 
0.27 & 0.56\\SCA3-3 & \bf{680.04} & 3.28 & 
681.03 & 3.69 & 680.04 & 0.00
 & 0.15\\SCA3-4 & \bf{690.50} & 2.58 & 
690.50 & 2.33 & 690.50 & 0.00
 & 0.00\\
SCA3-5 & 673.71 & 2.91 & 
673.71 & 3.06 & \bf{659.90} & 
2.09 & 2.09\\SCA3-6 & 652.94 & 2.34 & 
656.85 & 3.69 & \bf{651.09} & 
0.28 & 0.88\\SCA3-7 & 666.60 & 2.08 & 
666.60 & 2.09 & \bf{659.17} & 
1.13 & 1.13\\SCA3-8 & \bf{719.47} & 3.51 & 
719.47 & 3.62 & 719.47 & 0.00
 & 0.00\\
SCA3-9 & \bf{681.00} & 2.06 & 
682.22 & 3.38 & 681.00 & 0.00
 & 0.18\\SCA8-0 & 996.78 & 8.98 & 
1003.39 & 10.07 & \bf{961.50} & 
3.67 & 4.36\\SCA8-1 & 1058.75 & 19.88 & 
1067.72 & 12.52 & \bf{1049.65} & 
0.87 & 1.72\\SCA8-2 & 1053.78 & 11.01 & 
1054.24 & 12.76 & \bf{1039.64} & 
1.36 & 1.40\\SCA8-3 & 1021.34 & 9.18 & 
1024.57 & 10.53 & \bf{983.34} & 
3.86 & 4.19\\SCA8-4 & 1095.96 & 7.82 & 
1095.96 & 8.49 & \bf{1065.49} & 
2.86 & 2.86\\SCA8-5 & 1049.44 & 12.80 & 
1054.04 & 10.32 & \bf{1027.08} & 
2.18 & 2.62\\SCA8-6 & 972.48 & 8.78 & 
976.85 & 8.08 & \bf{971.82} & 
0.07 & 0.52\\SCA8-7 & 1066.65 & 8.53 & 
1068.10 & 8.89 & \bf{1051.28} & 
1.46 & 1.60\\SCA8-8 & 1088.65 & 8.66 & 
1089.92 & 8.93 & \bf{1071.18} & 
1.63 & 1.75\\SCA8-9 & 1067.42 & 11.70 & 
1077.93 & 9.72 & \bf{1060.50} & 
0.65 & 1.64\\CON3-0 & \bf{616.52} & 1.60 & 
618.11 & 1.94 & 616.52 & 0.00
 & 0.26\\CON3-1 & 560.75 & 3.68 & 
560.75 & 3.33 & \bf{554.47} & 
1.13 & 1.13\\CON3-2 & 521.38 & 4.25 & 
521.38 & 4.41 & \bf{518.00} & 
0.65 & 0.65\\CON3-3 & \bf{591.19} & 2.10 & 
591.19 & 2.17 & 591.19 & 0.00
 & 0.00\\
CON3-4 & 591.43 & 2.05 & 
591.43 & 2.25 & \bf{588.79} & 
0.45 & 0.45\\CON3-5 & \bf{563.70} & 2.38 & 
566.68 & 2.57 & 563.70 & 0.00
 & 0.53\\CON3-6 & 502.16 & 4.36 & 
502.16 & 2.86 & \bf{499.05} & 
0.62 & 0.62\\CON3-7 & 582.12 & 2.84 & 
582.12 & 2.94 & \bf{576.48} & 
0.98 & 0.98\\CON3-8 & 523.14 & 2.56 & 
523.14 & 2.60 & \bf{523.05} & 
0.02 & 0.02\\CON3-9 & 588.40 & 2.55 & 
588.40 & 2.57 & \bf{578.24} & 
1.76 & 1.76\\CON8-0 & 890.08 & 11.36 & 
892.55 & 10.83 & \bf{857.17} & 
3.84 & 4.13\\CON8-1 & 743.42 & 11.00 & 
752.12 & 10.12 & \bf{740.85} & 
0.35 & 1.52\\CON8-2 & 713.44 & 8.56 & 
714.98 & 9.35 & \bf{712.89} & 
0.08 & 0.29\\CON8-3 & 817.34 & 7.70 & 
826.66 & 8.88 & \bf{811.07} & 
0.77 & 1.92\\CON8-4 & 776.72 & 8.93 & 
787.35 & 9.08 & \bf{772.25} & 
0.58 & 1.96\\CON8-5 & 762.61 & 13.58 & 
764.18 & 10.77 & \bf{754.88} & 
1.02 & 1.23\\CON8-6 & 698.59 & 9.13 & 
699.59 & 8.41 & \bf{678.92} & 
2.90 & 3.04\\CON8-7 & 815.06 & 11.98 & 
815.31 & 12.51 & \bf{811.96} & 
0.38 & 0.41\\CON8-8 & 794.78 & 7.89 & 
794.78 & 8.21 & \bf{767.53} & 
3.55 & 3.55\\CON8-9 & 825.34 & 8.16 & 
826.33 & 7.87 & \bf{809.00} & 
2.02 & 2.14\\\bf{PROM.} & 
\bf{767.90} & \bf{6.60} & \bf{770.16} & \bf{6.42} & \bf{758.54} & \bf{1.12} & \bf{1.39}\\[1ex]\hline
\end{tabular}
\label{table:nonlin}
\end{table} \clearpage
\begin{table}[ht]
\caption{Resultados de la ejecución de la metaheurística SCA, utilizando instancias de Dethloff con la configuración -n 200.0 -b 10 -y .3}
\centering
\small
\begin{tabular}{c c c c c c c c}
\hline\hline
Instancia & Costo mínimo & Tiempo(seg.) & Costo promedio & Tiempo promedio(seg.) & CME & \%G & \%GP \\ [0.5ex]
\hline
SCA3-0 & 640.55 & 3.42 & 
640.55 & 3.40 & \bf{635.62} & 
0.78 & 0.78\\SCA3-1 & \bf{697.84} & 3.37 & 
697.84 & 3.43 & 697.84 & 0.00
 & 0.00\\
SCA3-2 & 666.72 & 4.34 & 
666.72 & 4.48 & \bf{659.34} & 
1.12 & 1.12\\SCA3-3 & 680.60 & 2.12 & 
680.60 & 2.08 & \bf{680.04} & 
0.08 & 0.08\\SCA3-4 & \bf{690.50} & 1.95 & 
690.50 & 2.54 & 690.50 & 0.00
 & 0.00\\
SCA3-5 & 681.81 & 2.85 & 
681.81 & 3.30 & \bf{659.90} & 
3.32 & 3.32\\SCA3-6 & 652.94 & 3.93 & 
653.31 & 3.53 & \bf{651.09} & 
0.28 & 0.34\\SCA3-7 & 666.60 & 5.04 & 
666.60 & 4.11 & \bf{659.17} & 
1.13 & 1.13\\SCA3-8 & \bf{719.47} & 4.26 & 
719.47 & 4.40 & 719.47 & 0.00
 & 0.00\\
SCA3-9 & \bf{681.00} & 3.30 & 
681.00 & 4.45 & 681.00 & 0.00
 & 0.00\\
SCA8-0 & 997.45 & 10.42 & 
1001.93 & 10.03 & \bf{961.50} & 
3.74 & 4.20\\SCA8-1 & 1053.44 & 8.12 & 
1063.63 & 11.53 & \bf{1049.65} & 
0.36 & 1.33\\SCA8-2 & 1050.37 & 10.15 & 
1051.45 & 10.20 & \bf{1039.64} & 
1.03 & 1.14\\SCA8-3 & 1025.40 & 8.72 & 
1026.31 & 9.20 & \bf{983.34} & 
4.28 & 4.37\\SCA8-4 & 1068.27 & 12.45 & 
1069.17 & 11.14 & \bf{1065.49} & 
0.26 & 0.35\\SCA8-5 & 1049.44 & 9.02 & 
1050.53 & 11.13 & \bf{1027.08} & 
2.18 & 2.28\\SCA8-6 & 972.48 & 8.21 & 
979.04 & 8.16 & \bf{971.82} & 
0.07 & 0.74\\SCA8-7 & 1067.49 & 9.78 & 
1078.81 & 10.22 & \bf{1051.28} & 
1.54 & 2.62\\SCA8-8 & 1085.22 & 10.08 & 
1088.08 & 12.60 & \bf{1071.18} & 
1.31 & 1.58\\SCA8-9 & 1081.23 & 7.25 & 
1088.17 & 9.60 & \bf{1060.50} & 
1.95 & 2.61\\CON3-0 & 632.72 & 2.42 & 
632.72 & 2.40 & \bf{616.52} & 
2.63 & 2.63\\CON3-1 & 560.75 & 3.14 & 
560.75 & 3.54 & \bf{554.47} & 
1.13 & 1.13\\CON3-2 & 521.63 & 1.38 & 
524.95 & 2.06 & \bf{518.00} & 
0.70 & 1.34\\CON3-3 & 591.20 & 2.48 & 
591.27 & 2.78 & \bf{591.19} & 
0.00 & 0.01\\CON3-4 & 591.43 & 3.32 & 
591.43 & 3.42 & \bf{588.79} & 
0.45 & 0.45\\CON3-5 & \bf{563.70} & 1.89 & 
566.14 & 3.15 & 563.70 & 0.00
 & 0.43\\CON3-6 & 502.16 & 2.00 & 
502.16 & 2.13 & \bf{499.05} & 
0.62 & 0.62\\CON3-7 & 582.33 & 4.80 & 
585.09 & 3.49 & \bf{576.48} & 
1.01 & 1.49\\CON3-8 & \bf{523.05} & 2.80 & 
523.59 & 3.49 & 523.05 & 0.00
 & 0.10\\CON3-9 & 580.78 & 3.08 & 
580.78 & 2.90 & \bf{578.24} & 
0.44 & 0.44\\CON8-0 & 882.28 & 10.36 & 
882.28 & 10.77 & \bf{857.17} & 
2.93 & 2.93\\CON8-1 & 745.98 & 10.58 & 
748.02 & 9.65 & \bf{740.85} & 
0.69 & 0.97\\CON8-2 & 717.50 & 9.70 & 
717.81 & 8.96 & \bf{712.89} & 
0.65 & 0.69\\CON8-3 & 817.57 & 12.48 & 
825.38 & 11.05 & \bf{811.07} & 
0.80 & 1.76\\CON8-4 & 790.80 & 8.69 & 
791.67 & 9.91 & \bf{772.25} & 
2.40 & 2.51\\CON8-5 & 769.34 & 10.31 & 
771.17 & 9.96 & \bf{754.88} & 
1.92 & 2.16\\CON8-6 & 696.83 & 10.31 & 
700.47 & 9.53 & \bf{678.92} & 
2.64 & 3.17\\CON8-7 & 819.64 & 9.91 & 
821.94 & 9.34 & \bf{811.96} & 
0.95 & 1.23\\CON8-8 & 785.30 & 5.32 & 
785.42 & 6.91 & \bf{767.53} & 
2.32 & 2.33\\CON8-9 & 819.62 & 6.96 & 
820.45 & 7.67 & \bf{809.00} & 
1.31 & 1.41\\\bf{PROM.} & 
\bf{768.09} & \bf{6.27} & \bf{769.98} & \bf{6.57} & \bf{758.54} & \bf{1.18} & \bf{1.40}\\[1ex]\hline
\end{tabular}
\label{table:nonlin}
\end{table} \clearpage
\begin{table}[ht]
\caption{Resultados de la ejecución de la metaheurística SCA, utilizando instancias de Dethloff con la configuración -n 200.0 -b 10 -y .4}
\centering
\small
\begin{tabular}{c c c c c c c c}
\hline\hline
Instancia & Costo mínimo & Tiempo(seg.) & Costo promedio & Tiempo promedio(seg.) & CME & \%G & \%GP \\ [0.5ex]
\hline
SCA3-0 & 640.55 & 3.37 & 
640.55 & 3.46 & \bf{635.62} & 
0.78 & 0.78\\SCA3-1 & \bf{697.84} & 5.16 & 
699.17 & 4.28 & 697.84 & 0.00
 & 0.19\\SCA3-2 & 661.13 & 4.09 & 
663.44 & 4.13 & \bf{659.34} & 
0.27 & 0.62\\SCA3-3 & 683.16 & 3.43 & 
683.16 & 3.10 & \bf{680.04} & 
0.46 & 0.46\\SCA3-4 & 692.57 & 4.76 & 
692.90 & 3.98 & \bf{690.50} & 
0.30 & 0.35\\SCA3-5 & \bf{659.90} & 2.41 & 
659.90 & 2.36 & 659.90 & 0.00
 & 0.00\\
SCA3-6 & 652.94 & 4.76 & 
652.94 & 4.86 & \bf{651.09} & 
0.28 & 0.28\\SCA3-7 & 666.60 & 3.52 & 
668.25 & 3.27 & \bf{659.17} & 
1.13 & 1.38\\SCA3-8 & \bf{719.47} & 4.67 & 
719.47 & 4.51 & 719.47 & 0.00
 & 0.00\\
SCA3-9 & \bf{681.00} & 2.83 & 
682.03 & 3.11 & 681.00 & 0.00
 & 0.15\\SCA8-0 & 984.96 & 8.51 & 
986.76 & 8.55 & \bf{961.50} & 
2.44 & 2.63\\SCA8-1 & 1066.39 & 9.03 & 
1070.52 & 8.77 & \bf{1049.65} & 
1.59 & 1.99\\SCA8-2 & 1051.95 & 14.33 & 
1053.44 & 14.33 & \bf{1039.64} & 
1.18 & 1.33\\SCA8-3 & 1028.27 & 6.59 & 
1029.18 & 7.29 & \bf{983.34} & 
4.57 & 4.66\\SCA8-4 & 1067.28 & 11.26 & 
1071.21 & 8.80 & \bf{1065.49} & 
0.17 & 0.54\\SCA8-5 & 1047.30 & 12.39 & 
1052.20 & 12.35 & \bf{1027.08} & 
1.97 & 2.45\\SCA8-6 & 981.21 & 11.29 & 
981.33 & 10.53 & \bf{971.82} & 
0.97 & 0.98\\SCA8-7 & 1070.53 & 7.95 & 
1070.53 & 8.61 & \bf{1051.28} & 
1.83 & 1.83\\SCA8-8 & 1082.67 & 9.80 & 
1093.81 & 10.96 & \bf{1071.18} & 
1.07 & 2.11\\SCA8-9 & 1072.10 & 8.66 & 
1073.71 & 9.15 & \bf{1060.50} & 
1.09 & 1.25\\CON3-0 & 617.59 & 2.66 & 
619.68 & 2.00 & \bf{616.52} & 
0.17 & 0.51\\CON3-1 & 558.09 & 2.52 & 
558.09 & 2.87 & \bf{554.47} & 
0.65 & 0.65\\CON3-2 & 521.38 & 3.35 & 
521.38 & 3.02 & \bf{518.00} & 
0.65 & 0.65\\CON3-3 & \bf{591.19} & 2.78 & 
591.20 & 3.59 & 591.19 & 0.00
 & 0.00\\CON3-4 & 591.43 & 2.84 & 
591.43 & 2.83 & \bf{588.79} & 
0.45 & 0.45\\CON3-5 & 564.89 & 3.02 & 
568.00 & 2.88 & \bf{563.70} & 
0.21 & 0.76\\CON3-6 & 502.16 & 3.06 & 
502.16 & 3.48 & \bf{499.05} & 
0.62 & 0.62\\CON3-7 & 578.41 & 3.25 & 
578.41 & 3.26 & \bf{576.48} & 
0.33 & 0.33\\CON3-8 & \bf{523.05} & 3.89 & 
523.99 & 4.43 & 523.05 & 0.00
 & 0.18\\CON3-9 & 588.40 & 2.88 & 
588.40 & 2.96 & \bf{578.24} & 
1.76 & 1.76\\CON8-0 & 866.22 & 12.58 & 
869.49 & 11.72 & \bf{857.17} & 
1.06 & 1.44\\CON8-1 & 742.47 & 12.78 & 
750.17 & 11.26 & \bf{740.85} & 
0.22 & 1.26\\CON8-2 & 720.10 & 7.96 & 
721.16 & 8.48 & \bf{712.89} & 
1.01 & 1.16\\CON8-3 & 823.23 & 8.45 & 
826.87 & 9.23 & \bf{811.07} & 
1.50 & 1.95\\CON8-4 & 791.76 & 6.98 & 
792.16 & 10.02 & \bf{772.25} & 
2.53 & 2.58\\CON8-5 & 759.63 & 11.51 & 
761.97 & 12.23 & \bf{754.88} & 
0.63 & 0.94\\CON8-6 & 698.32 & 11.58 & 
699.75 & 9.45 & \bf{678.92} & 
2.86 & 3.07\\CON8-7 & 815.71 & 10.84 & 
817.96 & 9.87 & \bf{811.96} & 
0.46 & 0.74\\CON8-8 & 778.22 & 7.17 & 
785.23 & 7.38 & \bf{767.53} & 
1.39 & 2.31\\CON8-9 & 830.17 & 9.18 & 
831.88 & 11.14 & \bf{809.00} & 
2.62 & 2.83\\\bf{PROM.} & 
\bf{766.76} & \bf{6.70} & \bf{768.60} & \bf{6.71} & \bf{758.54} & \bf{0.98} & \bf{1.20}\\[1ex]\hline
\end{tabular}
\label{table:nonlin}
\end{table} \clearpage
\begin{table}[ht]
\caption{Resultados de la ejecución de la metaheurística SCA, utilizando instancias de Dethloff con la configuración -n 200.0 -b 10 -y .5}
\centering
\small
\begin{tabular}{c c c c c c c c}
\hline\hline
Instancia & Costo mínimo & Tiempo(seg.) & Costo promedio & Tiempo promedio(seg.) & CME & \%G & \%GP \\ [0.5ex]
\hline
SCA3-0 & 640.55 & 4.32 & 
640.55 & 4.38 & \bf{635.62} & 
0.78 & 0.78\\SCA3-1 & 701.53 & 3.35 & 
701.53 & 3.61 & \bf{697.84} & 
0.53 & 0.53\\SCA3-2 & \bf{659.34} & 3.83 & 
660.68 & 4.43 & 659.34 & 0.00
 & 0.20\\SCA3-3 & 681.74 & 4.26 & 
681.74 & 4.47 & \bf{680.04} & 
0.25 & 0.25\\SCA3-4 & \bf{690.50} & 1.98 & 
690.50 & 2.10 & 690.50 & 0.00
 & 0.00\\
SCA3-5 & 665.04 & 2.62 & 
669.04 & 2.75 & \bf{659.90} & 
0.78 & 1.38\\SCA3-6 & 653.83 & 4.38 & 
654.73 & 4.14 & \bf{651.09} & 
0.42 & 0.56\\SCA3-7 & 666.60 & 1.98 & 
666.60 & 2.06 & \bf{659.17} & 
1.13 & 1.13\\SCA3-8 & \bf{719.47} & 2.28 & 
719.47 & 2.22 & 719.47 & 0.00
 & 0.00\\
SCA3-9 & 681.23 & 2.09 & 
681.23 & 2.48 & \bf{681.00} & 
0.03 & 0.03\\SCA8-0 & 968.79 & 7.96 & 
976.25 & 8.29 & \bf{961.50} & 
0.76 & 1.53\\SCA8-1 & 1062.88 & 10.36 & 
1081.08 & 7.48 & \bf{1049.65} & 
1.26 & 2.99\\SCA8-2 & 1050.37 & 10.92 & 
1052.84 & 11.00 & \bf{1039.64} & 
1.03 & 1.27\\SCA8-3 & 1014.02 & 8.84 & 
1014.04 & 8.92 & \bf{983.34} & 
3.12 & 3.12\\SCA8-4 & 1077.42 & 10.78 & 
1091.24 & 8.38 & \bf{1065.49} & 
1.12 & 2.42\\SCA8-5 & 1045.69 & 9.86 & 
1057.39 & 9.11 & \bf{1027.08} & 
1.81 & 2.95\\SCA8-6 & 972.48 & 11.81 & 
988.00 & 11.27 & \bf{971.82} & 
0.07 & 1.66\\SCA8-7 & 1070.92 & 9.48 & 
1073.17 & 11.21 & \bf{1051.28} & 
1.87 & 2.08\\SCA8-8 & \bf{1071.18} & 10.48 & 
1081.44 & 10.81 & 1071.18 & 0.00
 & 0.96\\SCA8-9 & 1070.34 & 9.19 & 
1079.39 & 9.69 & \bf{1060.50} & 
0.93 & 1.78\\CON3-0 & 619.09 & 2.27 & 
629.30 & 1.59 & \bf{616.52} & 
0.42 & 2.07\\CON3-1 & 559.72 & 1.86 & 
560.49 & 2.64 & \bf{554.47} & 
0.95 & 1.09\\CON3-2 & 521.38 & 3.16 & 
521.38 & 3.55 & \bf{518.00} & 
0.65 & 0.65\\CON3-3 & 591.20 & 4.76 & 
591.20 & 3.85 & \bf{591.19} & 
0.00 & 0.00\\CON3-4 & 591.43 & 3.61 & 
591.43 & 3.38 & \bf{588.79} & 
0.45 & 0.45\\CON3-5 & \bf{563.70} & 2.87 & 
564.29 & 2.79 & 563.70 & 0.00
 & 0.10\\CON3-6 & 502.16 & 3.78 & 
502.16 & 6.91 & \bf{499.05} & 
0.62 & 0.62\\CON3-7 & 585.42 & 4.43 & 
585.42 & 4.24 & \bf{576.48} & 
1.55 & 1.55\\CON3-8 & \bf{523.05} & 2.92 & 
523.71 & 2.81 & 523.05 & 0.00
 & 0.13\\CON3-9 & 588.38 & 2.14 & 
588.38 & 2.25 & \bf{578.24} & 
1.75 & 1.75\\CON8-0 & 882.25 & 6.84 & 
885.58 & 7.54 & \bf{857.17} & 
2.93 & 3.31\\CON8-1 & 742.47 & 16.15 & 
750.15 & 12.34 & \bf{740.85} & 
0.22 & 1.26\\CON8-2 & 713.05 & 11.18 & 
715.53 & 13.97 & \bf{712.89} & 
0.02 & 0.37\\CON8-3 & 822.51 & 7.30 & 
822.51 & 7.86 & \bf{811.07} & 
1.41 & 1.41\\CON8-4 & 783.13 & 10.54 & 
796.91 & 10.98 & \bf{772.25} & 
1.41 & 3.19\\CON8-5 & 763.60 & 9.90 & 
769.33 & 8.61 & \bf{754.88} & 
1.16 & 1.91\\CON8-6 & 688.20 & 11.49 & 
693.98 & 11.76 & \bf{678.92} & 
1.37 & 2.22\\CON8-7 & 814.79 & 9.16 & 
815.58 & 10.25 & \bf{811.96} & 
0.35 & 0.45\\CON8-8 & 793.13 & 7.05 & 
793.13 & 7.24 & \bf{767.53} & 
3.34 & 3.34\\CON8-9 & 825.10 & 10.22 & 
831.09 & 11.57 & \bf{809.00} & 
1.99 & 2.73\\\bf{PROM.} & 
\bf{765.94} & \bf{6.56} & \bf{769.81} & \bf{6.62} & \bf{758.54} & \bf{0.91} & \bf{1.36}\\[1ex]\hline
\end{tabular}
\label{table:nonlin}
\end{table} \clearpage
\begin{table}[ht]
\caption{Resultados de la ejecución de la metaheurística SCA, utilizando instancias de Dethloff con la configuración -n 200.0 -b 10 -y .6}
\centering
\small
\begin{tabular}{c c c c c c c c}
\hline\hline
Instancia & Costo mínimo & Tiempo(seg.) & Costo promedio & Tiempo promedio(seg.) & CME & \%G & \%GP \\ [0.5ex]
\hline
SCA3-0 & 640.55 & 3.85 & 
640.55 & 3.70 & \bf{635.62} & 
0.78 & 0.78\\SCA3-1 & \bf{697.84} & 1.80 & 
697.84 & 1.82 & 697.84 & 0.00
 & 0.00\\
SCA3-2 & \bf{659.34} & 3.81 & 
659.34 & 3.46 & 659.34 & 0.00
 & 0.00\\
SCA3-3 & 680.60 & 4.06 & 
680.60 & 3.79 & \bf{680.04} & 
0.08 & 0.08\\SCA3-4 & \bf{690.50} & 2.92 & 
691.53 & 3.18 & 690.50 & 0.00
 & 0.15\\SCA3-5 & 665.04 & 3.12 & 
676.86 & 3.23 & \bf{659.90} & 
0.78 & 2.57\\SCA3-6 & 652.94 & 2.31 & 
654.18 & 2.52 & \bf{651.09} & 
0.28 & 0.47\\SCA3-7 & 666.60 & 3.10 & 
670.01 & 2.39 & \bf{659.17} & 
1.13 & 1.64\\SCA3-8 & \bf{719.47} & 4.61 & 
719.47 & 3.94 & 719.47 & 0.00
 & 0.00\\
SCA3-9 & 684.25 & 2.90 & 
684.49 & 3.24 & \bf{681.00} & 
0.48 & 0.51\\SCA8-0 & 977.93 & 10.86 & 
979.23 & 11.95 & \bf{961.50} & 
1.71 & 1.84\\SCA8-1 & 1086.34 & 11.04 & 
1088.88 & 10.07 & \bf{1049.65} & 
3.50 & 3.74\\SCA8-2 & 1054.69 & 10.92 & 
1054.69 & 9.52 & \bf{1039.64} & 
1.45 & 1.45\\SCA8-3 & 1020.60 & 13.18 & 
1025.66 & 18.28 & \bf{983.34} & 
3.79 & 4.30\\SCA8-4 & 1079.30 & 10.25 & 
1093.55 & 8.51 & \bf{1065.49} & 
1.30 & 2.63\\SCA8-5 & 1061.29 & 10.06 & 
1061.29 & 9.39 & \bf{1027.08} & 
3.33 & 3.33\\SCA8-6 & 972.48 & 13.96 & 
973.62 & 11.49 & \bf{971.82} & 
0.07 & 0.18\\SCA8-7 & 1070.92 & 10.21 & 
1075.99 & 10.88 & \bf{1051.28} & 
1.87 & 2.35\\SCA8-8 & 1092.12 & 14.66 & 
1094.42 & 11.19 & \bf{1071.18} & 
1.95 & 2.17\\SCA8-9 & 1073.62 & 16.15 & 
1081.01 & 12.53 & \bf{1060.50} & 
1.24 & 1.93\\CON3-0 & 617.98 & 2.34 & 
625.85 & 2.63 & \bf{616.52} & 
0.24 & 1.51\\CON3-1 & 560.75 & 4.69 & 
560.75 & 4.23 & \bf{554.47} & 
1.13 & 1.13\\CON3-2 & 521.38 & 2.99 & 
521.75 & 5.24 & \bf{518.00} & 
0.65 & 0.72\\CON3-3 & 591.48 & 3.39 & 
592.61 & 2.54 & \bf{591.19} & 
0.05 & 0.24\\CON3-4 & 591.43 & 2.10 & 
591.43 & 2.14 & \bf{588.79} & 
0.45 & 0.45\\CON3-5 & 567.94 & 3.01 & 
571.39 & 2.98 & \bf{563.70} & 
0.75 & 1.36\\CON3-6 & 502.16 & 2.87 & 
502.16 & 2.99 & \bf{499.05} & 
0.62 & 0.62\\CON3-7 & 577.54 & 4.10 & 
577.54 & 3.73 & \bf{576.48} & 
0.18 & 0.18\\CON3-8 & 524.30 & 2.39 & 
524.30 & 2.48 & \bf{523.05} & 
0.24 & 0.24\\CON3-9 & 588.40 & 2.85 & 
588.40 & 2.75 & \bf{578.24} & 
1.76 & 1.76\\CON8-0 & 866.81 & 10.91 & 
875.75 & 10.84 & \bf{857.17} & 
1.12 & 2.17\\CON8-1 & 742.47 & 10.23 & 
745.54 & 9.72 & \bf{740.85} & 
0.22 & 0.63\\CON8-2 & 717.63 & 10.72 & 
719.43 & 9.70 & \bf{712.89} & 
0.66 & 0.92\\CON8-3 & 835.64 & 11.52 & 
835.64 & 12.01 & \bf{811.07} & 
3.03 & 3.03\\CON8-4 & 779.20 & 10.46 & 
784.83 & 9.38 & \bf{772.25} & 
0.90 & 1.63\\CON8-5 & 754.95 & 14.16 & 
756.18 & 14.22 & \bf{754.88} & 
0.01 & 0.17\\CON8-6 & 685.80 & 8.59 & 
693.56 & 8.28 & \bf{678.92} & 
1.01 & 2.16\\CON8-7 & 815.43 & 11.82 & 
815.64 & 9.57 & \bf{811.96} & 
0.43 & 0.45\\CON8-8 & 785.30 & 7.92 & 
787.11 & 14.71 & \bf{767.53} & 
2.32 & 2.55\\CON8-9 & 818.91 & 10.25 & 
818.91 & 23.77 & \bf{809.00} & 
1.22 & 1.22\\\bf{PROM.} & 
\bf{767.30} & \bf{7.28} & \bf{769.80} & \bf{7.47} & \bf{758.54} & \bf{1.02} & \bf{1.33}\\[1ex]\hline
\end{tabular}
\label{table:nonlin}
\end{table} \clearpage
\begin{table}[ht]
\caption{Resultados de la ejecución de la metaheurística SCA, utilizando instancias de Dethloff con la configuración -n 200.0 -b 10 -y .7}
\centering
\small
\begin{tabular}{c c c c c c c c}
\hline\hline
Instancia & Costo mínimo & Tiempo(seg.) & Costo promedio & Tiempo promedio(seg.) & CME & \%G & \%GP \\ [0.5ex]
\hline
SCA3-0 & 636.06 & 4.53 & 
637.18 & 3.83 & \bf{635.62} & 
0.07 & 0.25\\SCA3-1 & \bf{697.84} & 3.28 & 
697.84 & 3.33 & 697.84 & 0.00
 & 0.00\\
SCA3-2 & \bf{659.34} & 2.61 & 
660.74 & 3.12 & 659.34 & 0.00
 & 0.21\\SCA3-3 & 680.60 & 4.04 & 
680.60 & 3.89 & \bf{680.04} & 
0.08 & 0.08\\SCA3-4 & \bf{690.50} & 2.73 & 
690.50 & 2.73 & 690.50 & 0.00
 & 0.00\\
SCA3-5 & 670.10 & 2.74 & 
672.83 & 2.74 & \bf{659.90} & 
1.55 & 1.96\\SCA3-6 & 652.94 & 4.01 & 
655.54 & 4.49 & \bf{651.09} & 
0.28 & 0.68\\SCA3-7 & 666.60 & 2.75 & 
666.60 & 2.83 & \bf{659.17} & 
1.13 & 1.13\\SCA3-8 & \bf{719.47} & 3.04 & 
719.47 & 2.88 & 719.47 & 0.00
 & 0.00\\
SCA3-9 & 681.23 & 2.21 & 
684.72 & 2.32 & \bf{681.00} & 
0.03 & 0.55\\SCA8-0 & 973.22 & 12.41 & 
988.44 & 11.57 & \bf{961.50} & 
1.22 & 2.80\\SCA8-1 & 1063.55 & 11.94 & 
1073.88 & 17.98 & \bf{1049.65} & 
1.32 & 2.31\\SCA8-2 & 1050.37 & 12.08 & 
1052.76 & 13.03 & \bf{1039.64} & 
1.03 & 1.26\\SCA8-3 & 1021.95 & 15.38 & 
1034.89 & 10.84 & \bf{983.34} & 
3.93 & 5.24\\SCA8-4 & \bf{1065.49} & 9.18 & 
1067.85 & 8.84 & 1065.49 & 0.00
 & 0.22\\SCA8-5 & 1043.05 & 15.87 & 
1048.28 & 12.24 & \bf{1027.08} & 
1.55 & 2.06\\SCA8-6 & 972.48 & 8.38 & 
974.70 & 10.38 & \bf{971.82} & 
0.07 & 0.30\\SCA8-7 & 1072.17 & 12.67 & 
1073.18 & 11.54 & \bf{1051.28} & 
1.99 & 2.08\\SCA8-8 & 1088.65 & 12.20 & 
1096.16 & 9.53 & \bf{1071.18} & 
1.63 & 2.33\\SCA8-9 & 1075.43 & 7.47 & 
1083.83 & 7.63 & \bf{1060.50} & 
1.41 & 2.20\\CON3-0 & \bf{616.52} & 1.54 & 
617.95 & 2.21 & 616.52 & 0.00
 & 0.23\\CON3-1 & 560.75 & 3.39 & 
560.75 & 3.44 & \bf{554.47} & 
1.13 & 1.13\\CON3-2 & 521.38 & 2.58 & 
522.46 & 2.94 & \bf{518.00} & 
0.65 & 0.86\\CON3-3 & \bf{591.19} & 3.24 & 
591.19 & 3.03 & 591.19 & 0.00
 & 0.00\\
CON3-4 & 591.43 & 3.86 & 
591.43 & 3.64 & \bf{588.79} & 
0.45 & 0.45\\CON3-5 & 564.88 & 2.46 & 
564.88 & 2.33 & \bf{563.70} & 
0.21 & 0.21\\CON3-6 & 506.97 & 2.68 & 
506.97 & 2.69 & \bf{499.05} & 
1.59 & 1.59\\CON3-7 & 582.31 & 3.23 & 
582.31 & 3.33 & \bf{576.48} & 
1.01 & 1.01\\CON3-8 & 523.19 & 3.07 & 
524.32 & 3.05 & \bf{523.05} & 
0.03 & 0.24\\CON3-9 & 588.40 & 2.61 & 
588.40 & 3.06 & \bf{578.24} & 
1.76 & 1.76\\CON8-0 & 873.81 & 11.31 & 
878.26 & 6.72 & \bf{857.17} & 
1.94 & 2.46\\CON8-1 & 740.93 & 11.16 & 
740.93 & 10.45 & \bf{740.85} & 
0.01 & 0.01\\CON8-2 & 713.44 & 13.08 & 
722.65 & 9.47 & \bf{712.89} & 
0.08 & 1.37\\CON8-3 & 835.43 & 7.96 & 
836.23 & 9.12 & \bf{811.07} & 
3.00 & 3.10\\CON8-4 & 790.47 & 7.56 & 
791.50 & 9.28 & \bf{772.25} & 
2.36 & 2.49\\CON8-5 & 755.14 & 12.26 & 
755.68 & 10.72 & \bf{754.88} & 
0.03 & 0.11\\CON8-6 & 687.68 & 9.00 & 
692.49 & 9.72 & \bf{678.92} & 
1.29 & 2.00\\CON8-7 & 814.86 & 17.10 & 
817.89 & 13.26 & \bf{811.96} & 
0.36 & 0.73\\CON8-8 & 786.28 & 10.22 & 
792.78 & 8.46 & \bf{767.53} & 
2.44 & 3.29\\CON8-9 & 819.41 & 8.04 & 
832.95 & 8.20 & \bf{809.00} & 
1.29 & 2.96\\\bf{PROM.} & 
\bf{766.14} & \bf{7.15} & \bf{769.30} & \bf{6.77} & \bf{758.54} & \bf{0.92} & \bf{1.29}\\[1ex]\hline
\end{tabular}
\label{table:nonlin}
\end{table} \clearpage
\begin{table}[ht]
\caption{Resultados de la ejecución de la metaheurística SCA, utilizando instancias de Dethloff con la configuración -n 200.0 -b 10 -y .8}
\centering
\small
\begin{tabular}{c c c c c c c c}
\hline\hline
Instancia & Costo mínimo & Tiempo(seg.) & Costo promedio & Tiempo promedio(seg.) & CME & \%G & \%GP \\ [0.5ex]
\hline
SCA3-0 & 640.55 & 2.72 & 
640.55 & 3.33 & \bf{635.62} & 
0.78 & 0.78\\SCA3-1 & 701.86 & 3.15 & 
701.86 & 3.31 & \bf{697.84} & 
0.58 & 0.58\\SCA3-2 & 661.13 & 4.37 & 
661.89 & 4.08 & \bf{659.34} & 
0.27 & 0.39\\SCA3-3 & 680.60 & 3.24 & 
680.60 & 3.35 & \bf{680.04} & 
0.08 & 0.08\\SCA3-4 & \bf{690.50} & 3.47 & 
692.38 & 3.04 & 690.50 & 0.00
 & 0.27\\SCA3-5 & 678.67 & 2.52 & 
678.67 & 2.50 & \bf{659.90} & 
2.84 & 2.84\\SCA3-6 & 654.61 & 3.59 & 
655.37 & 3.40 & \bf{651.09} & 
0.54 & 0.66\\SCA3-7 & 671.67 & 2.82 & 
671.67 & 3.13 & \bf{659.17} & 
1.90 & 1.90\\SCA3-8 & \bf{719.47} & 3.13 & 
724.70 & 2.92 & 719.47 & 0.00
 & 0.73\\SCA3-9 & \bf{681.00} & 3.96 & 
682.05 & 4.43 & 681.00 & 0.00
 & 0.15\\SCA8-0 & 989.02 & 10.95 & 
997.84 & 11.92 & \bf{961.50} & 
2.86 & 3.78\\SCA8-1 & 1069.74 & 10.09 & 
1069.74 & 9.20 & \bf{1049.65} & 
1.91 & 1.91\\SCA8-2 & 1051.42 & 12.43 & 
1053.00 & 15.31 & \bf{1039.64} & 
1.13 & 1.29\\SCA8-3 & 1014.09 & 7.16 & 
1021.14 & 8.03 & \bf{983.34} & 
3.13 & 3.84\\SCA8-4 & 1068.48 & 18.53 & 
1084.14 & 12.21 & \bf{1065.49} & 
0.28 & 1.75\\SCA8-5 & 1047.83 & 9.76 & 
1054.64 & 10.85 & \bf{1027.08} & 
2.02 & 2.68\\SCA8-6 & 972.48 & 12.51 & 
984.75 & 9.45 & \bf{971.82} & 
0.07 & 1.33\\SCA8-7 & 1067.03 & 7.12 & 
1067.05 & 10.41 & \bf{1051.28} & 
1.50 & 1.50\\SCA8-8 & 1082.91 & 10.46 & 
1087.49 & 10.38 & \bf{1071.18} & 
1.10 & 1.52\\SCA8-9 & 1072.10 & 9.12 & 
1077.52 & 7.96 & \bf{1060.50} & 
1.09 & 1.60\\CON3-0 & 624.98 & 2.95 & 
627.49 & 2.97 & \bf{616.52} & 
1.37 & 1.78\\CON3-1 & 557.21 & 2.91 & 
559.87 & 3.86 & \bf{554.47} & 
0.49 & 0.97\\CON3-2 & 521.38 & 2.48 & 
521.38 & 2.96 & \bf{518.00} & 
0.65 & 0.65\\CON3-3 & 591.20 & 3.37 & 
591.20 & 3.42 & \bf{591.19} & 
0.00 & 0.00\\CON3-4 & 591.43 & 3.49 & 
591.43 & 3.40 & \bf{588.79} & 
0.45 & 0.45\\CON3-5 & 564.89 & 2.74 & 
564.89 & 3.08 & \bf{563.70} & 
0.21 & 0.21\\CON3-6 & 502.16 & 3.28 & 
503.69 & 1.99 & \bf{499.05} & 
0.62 & 0.93\\CON3-7 & 582.12 & 3.68 & 
583.09 & 3.64 & \bf{576.48} & 
0.98 & 1.15\\CON3-8 & \bf{523.05} & 2.94 & 
523.13 & 3.32 & 523.05 & 0.00
 & 0.02\\CON3-9 & 588.40 & 3.84 & 
588.70 & 2.90 & \bf{578.24} & 
1.76 & 1.81\\CON8-0 & 874.95 & 12.18 & 
886.65 & 9.04 & \bf{857.17} & 
2.07 & 3.44\\CON8-1 & 748.85 & 12.35 & 
750.88 & 11.99 & \bf{740.85} & 
1.08 & 1.35\\CON8-2 & \bf{712.89} & 17.85 & 
713.42 & 13.88 & 712.89 & 0.00
 & 0.07\\CON8-3 & 821.85 & 10.61 & 
826.46 & 10.33 & \bf{811.07} & 
1.33 & 1.90\\CON8-4 & 780.79 & 10.10 & 
787.29 & 10.54 & \bf{772.25} & 
1.11 & 1.95\\CON8-5 & 760.86 & 11.01 & 
762.79 & 9.25 & \bf{754.88} & 
0.79 & 1.05\\CON8-6 & 692.41 & 6.92 & 
698.66 & 7.86 & \bf{678.92} & 
1.99 & 2.91\\CON8-7 & 814.79 & 13.02 & 
817.87 & 11.89 & \bf{811.96} & 
0.35 & 0.73\\CON8-8 & 785.30 & 10.56 & 
789.84 & 11.18 & \bf{767.53} & 
2.32 & 2.91\\CON8-9 & 811.18 & 12.45 & 
820.62 & 10.84 & \bf{809.00} & 
0.27 & 1.44\\\bf{PROM.} & 
\bf{766.65} & \bf{7.25} & \bf{769.91} & \bf{6.94} & \bf{758.54} & \bf{1.00} & \bf{1.38}\\[1ex]\hline
\end{tabular}
\label{table:nonlin}
\end{table} \clearpage
\begin{table}[ht]
\caption{Resultados de la ejecución de la metaheurística SCA, utilizando instancias de Dethloff con la configuración -n 200.0 -b 10 -y .9}
\centering
\small
\begin{tabular}{c c c c c c c c}
\hline\hline
Instancia & Costo mínimo & Tiempo(seg.) & Costo promedio & Tiempo promedio(seg.) & CME & \%G & \%GP \\ [0.5ex]
\hline
SCA3-0 & 640.55 & 3.16 & 
641.02 & 4.14 & \bf{635.62} & 
0.78 & 0.85\\SCA3-1 & \bf{697.84} & 4.40 & 
699.43 & 4.22 & 697.84 & 0.00
 & 0.23\\SCA3-2 & 661.13 & 4.13 & 
662.08 & 4.05 & \bf{659.34} & 
0.27 & 0.42\\SCA3-3 & \bf{680.04} & 3.22 & 
680.04 & 3.44 & 680.04 & 0.00
 & 0.00\\
SCA3-4 & \bf{690.50} & 4.56 & 
690.50 & 4.49 & 690.50 & 0.00
 & 0.00\\
SCA3-5 & 665.04 & 2.91 & 
676.53 & 3.07 & \bf{659.90} & 
0.78 & 2.52\\SCA3-6 & 652.94 & 4.74 & 
652.94 & 4.59 & \bf{651.09} & 
0.28 & 0.28\\SCA3-7 & 666.60 & 3.78 & 
667.75 & 3.55 & \bf{659.17} & 
1.13 & 1.30\\SCA3-8 & 719.77 & 3.58 & 
719.77 & 3.87 & \bf{719.47} & 
0.04 & 0.04\\SCA3-9 & 681.23 & 3.09 & 
683.37 & 3.31 & \bf{681.00} & 
0.03 & 0.35\\SCA8-0 & 996.47 & 12.08 & 
997.64 & 11.90 & \bf{961.50} & 
3.64 & 3.76\\SCA8-1 & 1069.65 & 10.56 & 
1071.08 & 9.35 & \bf{1049.65} & 
1.91 & 2.04\\SCA8-2 & 1051.95 & 13.48 & 
1054.21 & 12.37 & \bf{1039.64} & 
1.18 & 1.40\\SCA8-3 & 1015.03 & 12.76 & 
1034.56 & 9.57 & \bf{983.34} & 
3.22 & 5.21\\SCA8-4 & 1074.63 & 10.60 & 
1079.08 & 9.34 & \bf{1065.49} & 
0.86 & 1.28\\SCA8-5 & 1056.45 & 10.19 & 
1061.82 & 11.55 & \bf{1027.08} & 
2.86 & 3.38\\SCA8-6 & 972.48 & 10.48 & 
975.25 & 11.79 & \bf{971.82} & 
0.07 & 0.35\\SCA8-7 & 1064.45 & 11.74 & 
1075.12 & 11.42 & \bf{1051.28} & 
1.25 & 2.27\\SCA8-8 & 1085.22 & 10.43 & 
1093.72 & 13.13 & \bf{1071.18} & 
1.31 & 2.10\\SCA8-9 & 1073.62 & 9.06 & 
1084.06 & 9.43 & \bf{1060.50} & 
1.24 & 2.22\\CON3-0 & 620.76 & 2.91 & 
631.03 & 3.23 & \bf{616.52} & 
0.69 & 2.35\\CON3-1 & 560.75 & 2.08 & 
560.75 & 2.44 & \bf{554.47} & 
1.13 & 1.13\\CON3-2 & 521.38 & 2.53 & 
521.38 & 2.52 & \bf{518.00} & 
0.65 & 0.65\\CON3-3 & 591.20 & 3.44 & 
591.20 & 3.71 & \bf{591.19} & 
0.00 & 0.00\\CON3-4 & 591.43 & 4.36 & 
591.43 & 4.21 & \bf{588.79} & 
0.45 & 0.45\\CON3-5 & 564.88 & 3.62 & 
564.88 & 3.38 & \bf{563.70} & 
0.21 & 0.21\\CON3-6 & 502.16 & 2.83 & 
502.16 & 2.42 & \bf{499.05} & 
0.62 & 0.62\\CON3-7 & \bf{576.48} & 3.37 & 
583.63 & 4.24 & 576.48 & 0.00
 & 1.24\\CON3-8 & 524.59 & 2.67 & 
530.71 & 2.12 & \bf{523.05} & 
0.29 & 1.46\\CON3-9 & 590.50 & 2.32 & 
590.50 & 2.40 & \bf{578.24} & 
2.12 & 2.12\\CON8-0 & 879.64 & 6.93 & 
884.00 & 9.84 & \bf{857.17} & 
2.62 & 3.13\\CON8-1 & 748.85 & 9.43 & 
752.26 & 10.01 & \bf{740.85} & 
1.08 & 1.54\\CON8-2 & \bf{712.89} & 14.76 & 
716.15 & 11.91 & 712.89 & 0.00
 & 0.46\\CON8-3 & 813.40 & 11.50 & 
821.21 & 10.86 & \bf{811.07} & 
0.29 & 1.25\\CON8-4 & 786.36 & 9.18 & 
789.74 & 11.56 & \bf{772.25} & 
1.83 & 2.26\\CON8-5 & 758.12 & 10.96 & 
766.66 & 9.96 & \bf{754.88} & 
0.43 & 1.56\\CON8-6 & 701.24 & 6.76 & 
703.48 & 7.21 & \bf{678.92} & 
3.29 & 3.62\\CON8-7 & 816.85 & 11.44 & 
821.41 & 11.31 & \bf{811.96} & 
0.60 & 1.16\\CON8-8 & 783.75 & 8.85 & 
783.75 & 9.60 & \bf{767.53} & 
2.11 & 2.11\\CON8-9 & 816.47 & 8.87 & 
822.80 & 8.73 & \bf{809.00} & 
0.92 & 1.71\\\bf{PROM.} & 
\bf{766.93} & \bf{6.94} & \bf{770.73} & \bf{7.01} & \bf{758.54} & \bf{1.00} & \bf{1.48}\\[1ex]\hline
\end{tabular}
\label{table:nonlin}
\end{table} \clearpage
\begin{table}[ht]
\caption{Resultados de la ejecución de la metaheurística SCA, utilizando instancias de Dethloff con la configuración -n 200.0 -b 10 -y 1.0}
\centering
\small
\begin{tabular}{c c c c c c c c}
\hline\hline
Instancia & Costo mínimo & Tiempo(seg.) & Costo promedio & Tiempo promedio(seg.) & CME & \%G & \%GP \\ [0.5ex]
\hline
SCA3-0 & 640.55 & 4.33 & 
640.55 & 4.54 & \bf{635.62} & 
0.78 & 0.78\\SCA3-1 & 701.53 & 2.60 & 
701.70 & 2.94 & \bf{697.84} & 
0.53 & 0.55\\SCA3-2 & 666.72 & 2.88 & 
666.72 & 2.88 & \bf{659.34} & 
1.12 & 1.12\\SCA3-3 & \bf{680.04} & 3.78 & 
680.36 & 3.83 & 680.04 & 0.00
 & 0.05\\SCA3-4 & \bf{690.50} & 2.78 & 
691.02 & 3.63 & 690.50 & 0.00
 & 0.08\\SCA3-5 & 673.46 & 2.94 & 
673.53 & 3.25 & \bf{659.90} & 
2.05 & 2.07\\SCA3-6 & 653.68 & 3.64 & 
654.38 & 3.66 & \bf{651.09} & 
0.40 & 0.51\\SCA3-7 & 669.89 & 2.93 & 
671.23 & 3.27 & \bf{659.17} & 
1.63 & 1.83\\SCA3-8 & \bf{719.47} & 3.24 & 
719.47 & 3.37 & 719.47 & 0.00
 & 0.00\\
SCA3-9 & \bf{681.00} & 4.36 & 
684.83 & 3.99 & 681.00 & 0.00
 & 0.56\\SCA8-0 & 977.94 & 10.36 & 
981.89 & 10.56 & \bf{961.50} & 
1.71 & 2.12\\SCA8-1 & 1063.76 & 9.95 & 
1072.07 & 9.45 & \bf{1049.65} & 
1.34 & 2.14\\SCA8-2 & 1051.80 & 14.45 & 
1053.06 & 12.43 & \bf{1039.64} & 
1.17 & 1.29\\SCA8-3 & 1010.76 & 8.88 & 
1016.73 & 9.65 & \bf{983.34} & 
2.79 & 3.40\\SCA8-4 & 1067.55 & 9.60 & 
1086.38 & 10.67 & \bf{1065.49} & 
0.19 & 1.96\\SCA8-5 & 1049.98 & 10.86 & 
1055.17 & 10.48 & \bf{1027.08} & 
2.23 & 2.73\\SCA8-6 & 972.48 & 14.78 & 
976.00 & 15.60 & \bf{971.82} & 
0.07 & 0.43\\SCA8-7 & 1070.63 & 11.92 & 
1071.30 & 11.07 & \bf{1051.28} & 
1.84 & 1.90\\SCA8-8 & \bf{1071.18} & 14.15 & 
1088.54 & 10.45 & 1071.18 & 0.00
 & 1.62\\SCA8-9 & 1072.10 & 10.20 & 
1079.87 & 10.04 & \bf{1060.50} & 
1.09 & 1.83\\CON3-0 & 633.24 & 2.75 & 
633.24 & 2.84 & \bf{616.52} & 
2.71 & 2.71\\CON3-1 & 560.75 & 1.96 & 
560.75 & 1.97 & \bf{554.47} & 
1.13 & 1.13\\CON3-2 & 521.38 & 3.20 & 
521.38 & 2.68 & \bf{518.00} & 
0.65 & 0.65\\CON3-3 & 591.20 & 3.08 & 
591.20 & 3.23 & \bf{591.19} & 
0.00 & 0.00\\CON3-4 & 591.43 & 3.05 & 
591.43 & 3.14 & \bf{588.79} & 
0.45 & 0.45\\CON3-5 & 568.09 & 2.12 & 
568.09 & 2.25 & \bf{563.70} & 
0.78 & 0.78\\CON3-6 & 502.16 & 2.67 & 
502.16 & 2.62 & \bf{499.05} & 
0.62 & 0.62\\CON3-7 & 585.42 & 2.96 & 
585.42 & 3.73 & \bf{576.48} & 
1.55 & 1.55\\CON3-8 & 523.19 & 3.42 & 
524.89 & 2.96 & \bf{523.05} & 
0.03 & 0.35\\CON3-9 & 588.40 & 2.74 & 
588.40 & 2.90 & \bf{578.24} & 
1.76 & 1.76\\CON8-0 & 861.11 & 12.57 & 
875.49 & 9.13 & \bf{857.17} & 
0.46 & 2.14\\CON8-1 & 748.19 & 11.94 & 
748.82 & 13.09 & \bf{740.85} & 
0.99 & 1.08\\CON8-2 & 713.05 & 14.53 & 
713.84 & 11.21 & \bf{712.89} & 
0.02 & 0.13\\CON8-3 & 816.27 & 11.46 & 
820.25 & 10.82 & \bf{811.07} & 
0.64 & 1.13\\CON8-4 & 793.57 & 8.33 & 
795.72 & 8.85 & \bf{772.25} & 
2.76 & 3.04\\CON8-5 & 757.99 & 7.96 & 
761.86 & 9.41 & \bf{754.88} & 
0.41 & 0.92\\CON8-6 & 698.05 & 8.45 & 
698.84 & 9.22 & \bf{678.92} & 
2.82 & 2.93\\CON8-7 & 815.79 & 9.51 & 
818.51 & 10.27 & \bf{811.96} & 
0.47 & 0.81\\CON8-8 & 782.44 & 6.88 & 
782.44 & 7.06 & \bf{767.53} & 
1.94 & 1.94\\CON8-9 & 815.91 & 13.94 & 
820.00 & 10.27 & \bf{809.00} & 
0.85 & 1.36\\\bf{PROM.} & 
\bf{766.32} & \bf{7.05} & \bf{769.19} & \bf{6.84} & \bf{758.54} & \bf{1.00} & \bf{1.31}\\[1ex]\hline
\end{tabular}
\label{table:nonlin}
\end{table} \clearpage
\begin{table}[ht]
\caption{Resultados de la ejecución de la metaheurística ACO, utilizando instancias de SalhiNagy con la configuración -n 3.0 -alpha 1.0 -beta 3.0 -q 0.1 -ro 0.015}
\centering
\small
\begin{tabular}{c c c c c c c c}
\hline\hline
Instancia & Costo mínimo & Tiempo(seg.) & Costo promedio & Tiempo promedio(seg.) & CME & \%G & \%GP \\ [0.5ex]
\hline
CMT1X & 100000 & 0 & 
nan & nan & \bf{470.48} & 
21154.89 & nan\\CMT1Y & 100000 & 0 & 
nan & nan & \bf{470.48} & 
21154.89 & nan\\CMT2X & 100000 & 0 & 
nan & nan & \bf{682.39} & 
14554.38 & nan\\CMT2Y & 100000 & 0 & 
nan & nan & \bf{682.39} & 
14554.38 & nan\\CMT3X & 100000 & 0 & 
nan & nan & \bf{719.06} & 
13807.05 & nan\\CMT3Y & 100000 & 0 & 
nan & nan & \bf{719.06} & 
13807.05 & nan\\CMT4X & 100000 & 0 & 
nan & nan & \bf{854.21} & 
11606.72 & nan\\CMT4Y & 100000 & 0 & 
nan & nan & \bf{852.46} & 
11630.76 & nan\\CMT5X & 100000 & 0 & 
nan & nan & \bf{1030.56} & 
9603.46 & nan\\CMT5Y & 100000 & 0 & 
nan & nan & \bf{1031.69} & 
9592.83 & nan\\CMT11X & 100000 & 0 & 
nan & nan & \bf{831.09} & 
11932.39 & nan\\CMT11Y & 100000 & 0 & 
nan & nan & \bf{829.85} & 
11950.37 & nan\\CMT12X & 100000 & 0 & 
nan & nan & \bf{658.83} & 
15078.42 & nan\\CMT12Y & 100000 & 0 & 
nan & nan & \bf{660.47} & 
15040.73 & nan\\\bf{PROM.} & 
\bf{100000.00} & \bf{0.00} & \bf{nan} & \bf{nan} & \bf{749.50} & \bf{13962.02} & \bf{nan}\\[1ex]\hline
\end{tabular}
\label{table:nonlin}
\end{table} \clearpage
\begin{table}[ht]
\caption{Resultados de la ejecución de la metaheurística ACO, utilizando instancias de SalhiNagy con la configuración -n 3.0 -alpha 1.0 -beta 3.0 -q .2 -ro 0.015}
\centering
\small
\begin{tabular}{c c c c c c c c}
\hline\hline
Instancia & Costo mínimo & Tiempo(seg.) & Costo promedio & Tiempo promedio(seg.) & CME & \%G & \%GP \\ [0.5ex]
\hline
CMT1X & 100000 & 0 & 
nan & nan & \bf{470.48} & 
21154.89 & nan\\CMT1Y & 100000 & 0 & 
nan & nan & \bf{470.48} & 
21154.89 & nan\\CMT2X & 100000 & 0 & 
nan & nan & \bf{682.39} & 
14554.38 & nan\\CMT2Y & 100000 & 0 & 
nan & nan & \bf{682.39} & 
14554.38 & nan\\CMT3X & 100000 & 0 & 
nan & nan & \bf{719.06} & 
13807.05 & nan\\CMT3Y & 100000 & 0 & 
nan & nan & \bf{719.06} & 
13807.05 & nan\\CMT4X & 100000 & 0 & 
nan & nan & \bf{854.21} & 
11606.72 & nan\\CMT4Y & 100000 & 0 & 
nan & nan & \bf{852.46} & 
11630.76 & nan\\CMT5X & 100000 & 0 & 
nan & nan & \bf{1030.56} & 
9603.46 & nan\\CMT5Y & 100000 & 0 & 
nan & nan & \bf{1031.69} & 
9592.83 & nan\\CMT11X & 100000 & 0 & 
nan & nan & \bf{831.09} & 
11932.39 & nan\\CMT11Y & 100000 & 0 & 
nan & nan & \bf{829.85} & 
11950.37 & nan\\CMT12X & 100000 & 0 & 
nan & nan & \bf{658.83} & 
15078.42 & nan\\CMT12Y & 100000 & 0 & 
nan & nan & \bf{660.47} & 
15040.73 & nan\\\bf{PROM.} & 
\bf{100000.00} & \bf{0.00} & \bf{nan} & \bf{nan} & \bf{749.50} & \bf{13962.02} & \bf{nan}\\[1ex]\hline
\end{tabular}
\label{table:nonlin}
\end{table} \clearpage
\begin{table}[ht]
\caption{Resultados de la ejecución de la metaheurística ACO, utilizando instancias de SalhiNagy con la configuración -n 3.0 -alpha 1.0 -beta 3.0 -q .3 -ro 0.015}
\centering
\small
\begin{tabular}{c c c c c c c c}
\hline\hline
Instancia & Costo mínimo & Tiempo(seg.) & Costo promedio & Tiempo promedio(seg.) & CME & \%G & \%GP \\ [0.5ex]
\hline
CMT1X & 100000 & 0 & 
nan & nan & \bf{470.48} & 
21154.89 & nan\\CMT1Y & 100000 & 0 & 
nan & nan & \bf{470.48} & 
21154.89 & nan\\CMT2X & 100000 & 0 & 
nan & nan & \bf{682.39} & 
14554.38 & nan\\CMT2Y & 100000 & 0 & 
nan & nan & \bf{682.39} & 
14554.38 & nan\\CMT3X & 100000 & 0 & 
nan & nan & \bf{719.06} & 
13807.05 & nan\\CMT3Y & 100000 & 0 & 
nan & nan & \bf{719.06} & 
13807.05 & nan\\CMT4X & 100000 & 0 & 
nan & nan & \bf{854.21} & 
11606.72 & nan\\CMT4Y & 100000 & 0 & 
nan & nan & \bf{852.46} & 
11630.76 & nan\\CMT5X & 100000 & 0 & 
nan & nan & \bf{1030.56} & 
9603.46 & nan\\CMT5Y & 100000 & 0 & 
nan & nan & \bf{1031.69} & 
9592.83 & nan\\CMT11X & 100000 & 0 & 
nan & nan & \bf{831.09} & 
11932.39 & nan\\CMT11Y & 100000 & 0 & 
nan & nan & \bf{829.85} & 
11950.37 & nan\\CMT12X & 100000 & 0 & 
nan & nan & \bf{658.83} & 
15078.42 & nan\\CMT12Y & 100000 & 0 & 
nan & nan & \bf{660.47} & 
15040.73 & nan\\\bf{PROM.} & 
\bf{100000.00} & \bf{0.00} & \bf{nan} & \bf{nan} & \bf{749.50} & \bf{13962.02} & \bf{nan}\\[1ex]\hline
\end{tabular}
\label{table:nonlin}
\end{table} \clearpage
\begin{table}[ht]
\caption{Resultados de la ejecución de la metaheurística ACO, utilizando instancias de SalhiNagy con la configuración -n 3.0 -alpha 1.0 -beta 3.0 -q .4 -ro 0.015}
\centering
\small
\begin{tabular}{c c c c c c c c}
\hline\hline
Instancia & Costo mínimo & Tiempo(seg.) & Costo promedio & Tiempo promedio(seg.) & CME & \%G & \%GP \\ [0.5ex]
\hline
CMT1X & 100000 & 0 & 
nan & nan & \bf{470.48} & 
21154.89 & nan\\CMT1Y & 100000 & 0 & 
nan & nan & \bf{470.48} & 
21154.89 & nan\\CMT2X & 100000 & 0 & 
nan & nan & \bf{682.39} & 
14554.38 & nan\\CMT2Y & 100000 & 0 & 
nan & nan & \bf{682.39} & 
14554.38 & nan\\CMT3X & 100000 & 0 & 
nan & nan & \bf{719.06} & 
13807.05 & nan\\CMT3Y & 100000 & 0 & 
nan & nan & \bf{719.06} & 
13807.05 & nan\\CMT4X & 100000 & 0 & 
nan & nan & \bf{854.21} & 
11606.72 & nan\\CMT4Y & 100000 & 0 & 
nan & nan & \bf{852.46} & 
11630.76 & nan\\CMT5X & 100000 & 0 & 
nan & nan & \bf{1030.56} & 
9603.46 & nan\\CMT5Y & 100000 & 0 & 
nan & nan & \bf{1031.69} & 
9592.83 & nan\\CMT11X & 100000 & 0 & 
nan & nan & \bf{831.09} & 
11932.39 & nan\\CMT11Y & 100000 & 0 & 
nan & nan & \bf{829.85} & 
11950.37 & nan\\CMT12X & 100000 & 0 & 
nan & nan & \bf{658.83} & 
15078.42 & nan\\CMT12Y & 100000 & 0 & 
nan & nan & \bf{660.47} & 
15040.73 & nan\\\bf{PROM.} & 
\bf{100000.00} & \bf{0.00} & \bf{nan} & \bf{nan} & \bf{749.50} & \bf{13962.02} & \bf{nan}\\[1ex]\hline
\end{tabular}
\label{table:nonlin}
\end{table} \clearpage
\begin{table}[ht]
\caption{Resultados de la ejecución de la metaheurística ACO, utilizando instancias de SalhiNagy con la configuración -n 4.0 -alpha 1.0 -beta 3.0 -q 0.1 -ro 0.015}
\centering
\small
\begin{tabular}{c c c c c c c c}
\hline\hline
Instancia & Costo mínimo & Tiempo(seg.) & Costo promedio & Tiempo promedio(seg.) & CME & \%G & \%GP \\ [0.5ex]
\hline
CMT1X & 100000 & 0 & 
nan & nan & \bf{470.48} & 
21154.89 & nan\\CMT1Y & 100000 & 0 & 
nan & nan & \bf{470.48} & 
21154.89 & nan\\CMT2X & 100000 & 0 & 
nan & nan & \bf{682.39} & 
14554.38 & nan\\CMT2Y & 100000 & 0 & 
nan & nan & \bf{682.39} & 
14554.38 & nan\\CMT3X & 100000 & 0 & 
nan & nan & \bf{719.06} & 
13807.05 & nan\\CMT3Y & 100000 & 0 & 
nan & nan & \bf{719.06} & 
13807.05 & nan\\CMT4X & 100000 & 0 & 
nan & nan & \bf{854.21} & 
11606.72 & nan\\CMT4Y & 100000 & 0 & 
nan & nan & \bf{852.46} & 
11630.76 & nan\\CMT5X & 100000 & 0 & 
nan & nan & \bf{1030.56} & 
9603.46 & nan\\CMT5Y & 100000 & 0 & 
nan & nan & \bf{1031.69} & 
9592.83 & nan\\CMT11X & 100000 & 0 & 
nan & nan & \bf{831.09} & 
11932.39 & nan\\CMT11Y & 100000 & 0 & 
nan & nan & \bf{829.85} & 
11950.37 & nan\\CMT12X & 100000 & 0 & 
nan & nan & \bf{658.83} & 
15078.42 & nan\\CMT12Y & 100000 & 0 & 
nan & nan & \bf{660.47} & 
15040.73 & nan\\\bf{PROM.} & 
\bf{100000.00} & \bf{0.00} & \bf{nan} & \bf{nan} & \bf{749.50} & \bf{13962.02} & \bf{nan}\\[1ex]\hline
\end{tabular}
\label{table:nonlin}
\end{table} \clearpage
\begin{table}[ht]
\caption{Resultados de la ejecución de la metaheurística ACO, utilizando instancias de SalhiNagy con la configuración -n 4.0 -alpha 1.0 -beta 3.0 -q .2 -ro 0.015}
\centering
\small
\begin{tabular}{c c c c c c c c}
\hline\hline
Instancia & Costo mínimo & Tiempo(seg.) & Costo promedio & Tiempo promedio(seg.) & CME & \%G & \%GP \\ [0.5ex]
\hline
CMT1X & 100000 & 0 & 
nan & nan & \bf{470.48} & 
21154.89 & nan\\CMT1Y & 100000 & 0 & 
nan & nan & \bf{470.48} & 
21154.89 & nan\\CMT2X & 100000 & 0 & 
nan & nan & \bf{682.39} & 
14554.38 & nan\\CMT2Y & 100000 & 0 & 
nan & nan & \bf{682.39} & 
14554.38 & nan\\CMT3X & 100000 & 0 & 
nan & nan & \bf{719.06} & 
13807.05 & nan\\CMT3Y & 100000 & 0 & 
nan & nan & \bf{719.06} & 
13807.05 & nan\\CMT4X & 100000 & 0 & 
nan & nan & \bf{854.21} & 
11606.72 & nan\\CMT4Y & 100000 & 0 & 
nan & nan & \bf{852.46} & 
11630.76 & nan\\CMT5X & 100000 & 0 & 
nan & nan & \bf{1030.56} & 
9603.46 & nan\\CMT5Y & 100000 & 0 & 
nan & nan & \bf{1031.69} & 
9592.83 & nan\\CMT11X & 100000 & 0 & 
nan & nan & \bf{831.09} & 
11932.39 & nan\\CMT11Y & 100000 & 0 & 
nan & nan & \bf{829.85} & 
11950.37 & nan\\CMT12X & 100000 & 0 & 
nan & nan & \bf{658.83} & 
15078.42 & nan\\CMT12Y & 100000 & 0 & 
nan & nan & \bf{660.47} & 
15040.73 & nan\\\bf{PROM.} & 
\bf{100000.00} & \bf{0.00} & \bf{nan} & \bf{nan} & \bf{749.50} & \bf{13962.02} & \bf{nan}\\[1ex]\hline
\end{tabular}
\label{table:nonlin}
\end{table} \clearpage
\begin{table}[ht]
\caption{Resultados de la ejecución de la metaheurística ACO, utilizando instancias de SalhiNagy con la configuración -n 4.0 -alpha 1.0 -beta 3.0 -q .3 -ro 0.015}
\centering
\small
\begin{tabular}{c c c c c c c c}
\hline\hline
Instancia & Costo mínimo & Tiempo(seg.) & Costo promedio & Tiempo promedio(seg.) & CME & \%G & \%GP \\ [0.5ex]
\hline
CMT1X & 100000 & 0 & 
nan & nan & \bf{470.48} & 
21154.89 & nan\\CMT1Y & 100000 & 0 & 
nan & nan & \bf{470.48} & 
21154.89 & nan\\CMT2X & 100000 & 0 & 
nan & nan & \bf{682.39} & 
14554.38 & nan\\CMT2Y & 100000 & 0 & 
nan & nan & \bf{682.39} & 
14554.38 & nan\\CMT3X & 100000 & 0 & 
nan & nan & \bf{719.06} & 
13807.05 & nan\\CMT3Y & 100000 & 0 & 
nan & nan & \bf{719.06} & 
13807.05 & nan\\CMT4X & 100000 & 0 & 
nan & nan & \bf{854.21} & 
11606.72 & nan\\CMT4Y & 100000 & 0 & 
nan & nan & \bf{852.46} & 
11630.76 & nan\\CMT5X & 100000 & 0 & 
nan & nan & \bf{1030.56} & 
9603.46 & nan\\CMT5Y & 100000 & 0 & 
nan & nan & \bf{1031.69} & 
9592.83 & nan\\CMT11X & 100000 & 0 & 
nan & nan & \bf{831.09} & 
11932.39 & nan\\CMT11Y & 100000 & 0 & 
nan & nan & \bf{829.85} & 
11950.37 & nan\\CMT12X & 100000 & 0 & 
nan & nan & \bf{658.83} & 
15078.42 & nan\\CMT12Y & 100000 & 0 & 
nan & nan & \bf{660.47} & 
15040.73 & nan\\\bf{PROM.} & 
\bf{100000.00} & \bf{0.00} & \bf{nan} & \bf{nan} & \bf{749.50} & \bf{13962.02} & \bf{nan}\\[1ex]\hline
\end{tabular}
\label{table:nonlin}
\end{table} \clearpage
\begin{table}[ht]
\caption{Resultados de la ejecución de la metaheurística ACO, utilizando instancias de SalhiNagy con la configuración -n 4.0 -alpha 1.0 -beta 3.0 -q .4 -ro 0.015}
\centering
\small
\begin{tabular}{c c c c c c c c}
\hline\hline
Instancia & Costo mínimo & Tiempo(seg.) & Costo promedio & Tiempo promedio(seg.) & CME & \%G & \%GP \\ [0.5ex]
\hline
CMT1X & 100000 & 0 & 
nan & nan & \bf{470.48} & 
21154.89 & nan\\CMT1Y & 100000 & 0 & 
nan & nan & \bf{470.48} & 
21154.89 & nan\\CMT2X & 100000 & 0 & 
nan & nan & \bf{682.39} & 
14554.38 & nan\\CMT2Y & 100000 & 0 & 
nan & nan & \bf{682.39} & 
14554.38 & nan\\CMT3X & \bf{\underline{88.00}} & Command & 
88.00 & 0.00 & 719.06 & 
\bf{-87.76} & \bf{-87.76}\\CMT3Y & 100000 & 0 & 
nan & nan & \bf{719.06} & 
13807.05 & nan\\CMT4X & 100000 & 0 & 
nan & nan & \bf{854.21} & 
11606.72 & nan\\CMT4Y & 100000 & 0 & 
nan & nan & \bf{852.46} & 
11630.76 & nan\\CMT5X & 100000 & 0 & 
nan & nan & \bf{1030.56} & 
9603.46 & nan\\CMT5Y & 100000 & 0 & 
nan & nan & \bf{1031.69} & 
9592.83 & nan\\CMT11X & 100000 & 0 & 
nan & nan & \bf{831.09} & 
11932.39 & nan\\CMT11Y & 100000 & 0 & 
nan & nan & \bf{829.85} & 
11950.37 & nan\\CMT12X & 100000 & 0 & 
nan & nan & \bf{658.83} & 
15078.42 & nan\\CMT12Y & 100000 & 0 & 
nan & nan & \bf{660.47} & 
15040.73 & nan\\\bf{PROM.} & 
\bf{92863.43} & \bf{0.00} & \bf{nan} & \bf{nan} & \bf{749.50} & \bf{12969.54} & \bf{nan}\\[1ex]\hline
\end{tabular}
\label{table:nonlin}
\end{table} \clearpage
\begin{table}[ht]
\caption{Resultados de la ejecución de la metaheurística ACO, utilizando instancias de SalhiNagy con la configuración -n 5.0 -alpha 1.0 -beta 3.0 -q 0.1 -ro 0.015}
\centering
\small
\begin{tabular}{c c c c c c c c}
\hline\hline
Instancia & Costo mínimo & Tiempo(seg.) & Costo promedio & Tiempo promedio(seg.) & CME & \%G & \%GP \\ [0.5ex]
\hline
CMT1X & 100000 & 0 & 
nan & nan & \bf{470.48} & 
21154.89 & nan\\CMT1Y & 100000 & 0 & 
nan & nan & \bf{470.48} & 
21154.89 & nan\\CMT2X & 100000 & 0 & 
nan & nan & \bf{682.39} & 
14554.38 & nan\\CMT2Y & 100000 & 0 & 
nan & nan & \bf{682.39} & 
14554.38 & nan\\CMT3X & 100000 & 0 & 
nan & nan & \bf{719.06} & 
13807.05 & nan\\CMT3Y & 100000 & 0 & 
nan & nan & \bf{719.06} & 
13807.05 & nan\\CMT4X & 100000 & 0 & 
nan & nan & \bf{854.21} & 
11606.72 & nan\\CMT4Y & 100000 & 0 & 
nan & nan & \bf{852.46} & 
11630.76 & nan\\CMT5X & 100000 & 0 & 
nan & nan & \bf{1030.56} & 
9603.46 & nan\\CMT5Y & 100000 & 0 & 
nan & nan & \bf{1031.69} & 
9592.83 & nan\\CMT11X & 100000 & 0 & 
nan & nan & \bf{831.09} & 
11932.39 & nan\\CMT11Y & 100000 & 0 & 
nan & nan & \bf{829.85} & 
11950.37 & nan\\CMT12X & 100000 & 0 & 
nan & nan & \bf{658.83} & 
15078.42 & nan\\CMT12Y & 100000 & 0 & 
nan & nan & \bf{660.47} & 
15040.73 & nan\\\bf{PROM.} & 
\bf{100000.00} & \bf{0.00} & \bf{nan} & \bf{nan} & \bf{749.50} & \bf{13962.02} & \bf{nan}\\[1ex]\hline
\end{tabular}
\label{table:nonlin}
\end{table} \clearpage
\begin{table}[ht]
\caption{Resultados de la ejecución de la metaheurística ACO, utilizando instancias de SalhiNagy con la configuración -n 5.0 -alpha 1.0 -beta 3.0 -q .2 -ro 0.015}
\centering
\small
\begin{tabular}{c c c c c c c c}
\hline\hline
Instancia & Costo mínimo & Tiempo(seg.) & Costo promedio & Tiempo promedio(seg.) & CME & \%G & \%GP \\ [0.5ex]
\hline
CMT1X & 100000 & 0 & 
nan & nan & \bf{470.48} & 
21154.89 & nan\\CMT1Y & 100000 & 0 & 
nan & nan & \bf{470.48} & 
21154.89 & nan\\CMT2X & 100000 & 0 & 
nan & nan & \bf{682.39} & 
14554.38 & nan\\CMT2Y & 100000 & 0 & 
nan & nan & \bf{682.39} & 
14554.38 & nan\\CMT3X & 100000 & 0 & 
nan & nan & \bf{719.06} & 
13807.05 & nan\\CMT3Y & 100000 & 0 & 
nan & nan & \bf{719.06} & 
13807.05 & nan\\CMT4X & 100000 & 0 & 
nan & nan & \bf{854.21} & 
11606.72 & nan\\CMT4Y & 100000 & 0 & 
nan & nan & \bf{852.46} & 
11630.76 & nan\\CMT5X & 100000 & 0 & 
nan & nan & \bf{1030.56} & 
9603.46 & nan\\CMT5Y & 100000 & 0 & 
nan & nan & \bf{1031.69} & 
9592.83 & nan\\CMT11X & 100000 & 0 & 
nan & nan & \bf{831.09} & 
11932.39 & nan\\CMT11Y & 100000 & 0 & 
nan & nan & \bf{829.85} & 
11950.37 & nan\\CMT12X & 100000 & 0 & 
nan & nan & \bf{658.83} & 
15078.42 & nan\\CMT12Y & 100000 & 0 & 
nan & nan & \bf{660.47} & 
15040.73 & nan\\\bf{PROM.} & 
\bf{100000.00} & \bf{0.00} & \bf{nan} & \bf{nan} & \bf{749.50} & \bf{13962.02} & \bf{nan}\\[1ex]\hline
\end{tabular}
\label{table:nonlin}
\end{table} \clearpage
\begin{table}[ht]
\caption{Resultados de la ejecución de la metaheurística ACO, utilizando instancias de SalhiNagy con la configuración -n 5.0 -alpha 1.0 -beta 3.0 -q .3 -ro 0.015}
\centering
\small
\begin{tabular}{c c c c c c c c}
\hline\hline
Instancia & Costo mínimo & Tiempo(seg.) & Costo promedio & Tiempo promedio(seg.) & CME & \%G & \%GP \\ [0.5ex]
\hline
CMT1X & 100000 & 0 & 
nan & nan & \bf{470.48} & 
21154.89 & nan\\CMT1Y & 100000 & 0 & 
nan & nan & \bf{470.48} & 
21154.89 & nan\\CMT2X & 100000 & 0 & 
nan & nan & \bf{682.39} & 
14554.38 & nan\\CMT2Y & 100000 & 0 & 
nan & nan & \bf{682.39} & 
14554.38 & nan\\CMT3X & 100000 & 0 & 
nan & nan & \bf{719.06} & 
13807.05 & nan\\CMT3Y & 100000 & 0 & 
nan & nan & \bf{719.06} & 
13807.05 & nan\\CMT4X & 100000 & 0 & 
nan & nan & \bf{854.21} & 
11606.72 & nan\\CMT4Y & 100000 & 0 & 
nan & nan & \bf{852.46} & 
11630.76 & nan\\CMT5X & 100000 & 0 & 
nan & nan & \bf{1030.56} & 
9603.46 & nan\\CMT5Y & 100000 & 0 & 
nan & nan & \bf{1031.69} & 
9592.83 & nan\\CMT11X & 100000 & 0 & 
nan & nan & \bf{831.09} & 
11932.39 & nan\\CMT11Y & 100000 & 0 & 
nan & nan & \bf{829.85} & 
11950.37 & nan\\CMT12X & 100000 & 0 & 
nan & nan & \bf{658.83} & 
15078.42 & nan\\CMT12Y & 100000 & 0 & 
nan & nan & \bf{660.47} & 
15040.73 & nan\\\bf{PROM.} & 
\bf{100000.00} & \bf{0.00} & \bf{nan} & \bf{nan} & \bf{749.50} & \bf{13962.02} & \bf{nan}\\[1ex]\hline
\end{tabular}
\label{table:nonlin}
\end{table} \clearpage
\begin{table}[ht]
\caption{Resultados de la ejecución de la metaheurística ACO, utilizando instancias de SalhiNagy con la configuración -n 3.0 -alpha 1.0 -beta 3.0 -q 0.1 -ro 0.015}
\centering
\small
\begin{tabular}{c c c c c c c c}
\hline\hline
Instancia & Costo mínimo & Tiempo(seg.) & Costo promedio & Tiempo promedio(seg.) & CME & \%G & \%GP \\ [0.5ex]
\hline
CMT1X & 476.70 & 1.84 & 
477.65 & 1.93 & \bf{470.48} & 
1.32 & 1.53\\CMT1Y & 477.00 & 1.80 & 
477.11 & 1.95 & \bf{470.48} & 
1.39 & 1.41\\CMT2X & 707.56 & 9.02 & 
708.16 & 8.98 & \bf{682.39} & 
3.69 & 3.78\\CMT2Y & 704.36 & 8.15 & 
705.82 & 8.52 & \bf{682.39} & 
3.22 & 3.43\\CMT3X & 731.61 & 28.72 & 
733.67 & 28.33 & \bf{719.06} & 
1.75 & 2.03\\CMT3Y & 728.04 & 29.10 & 
729.72 & 28.79 & \bf{719.06} & 
1.25 & 1.48\\CMT4X & 879.38 & 121.93 & 
884.29 & 122.60 & \bf{854.21} & 
2.95 & 3.52\\CMT4Y & 891.66 & 125.77 & 
893.05 & 122.78 & \bf{852.46} & 
4.60 & 4.76\\CMT5X & 1067.33 & 354.90 & 
1072.17 & 357.15 & \bf{1030.56} & 
3.57 & 4.04\\CMT5Y & 1095.90 & 379.71 & 
1095.99 & 370.67 & \bf{1031.69} & 
6.22 & 6.23\\CMT11X & 849.07 & 51.71 & 
863.19 & 51.98 & \bf{831.09} & 
2.16 & 3.86\\CMT11Y & 893.21 & 47.78 & 
894.27 & 46.76 & \bf{829.85} & 
7.64 & 7.76\\CMT12X & 674.82 & 23.84 & 
675.88 & 24.30 & \bf{658.83} & 
2.43 & 2.59\\CMT12Y & 680.30 & 24.77 & 
680.67 & 24.43 & \bf{660.47} & 
3.00 & 3.06\\\bf{PROM.} & 
\bf{775.50} & \bf{86.36} & \bf{777.97} & \bf{85.65} & \bf{749.50} & \bf{3.23} & \bf{3.53}\\[1ex]\hline
\end{tabular}
\label{table:nonlin}
\end{table} \clearpage
\begin{table}[ht]
\caption{Resultados de la ejecución de la metaheurística ACO, utilizando instancias de SalhiNagy con la configuración -n 3.0 -alpha 1.0 -beta 3.0 -q .2 -ro 0.015}
\centering
\small
\begin{tabular}{c c c c c c c c}
\hline\hline
Instancia & Costo mínimo & Tiempo(seg.) & Costo promedio & Tiempo promedio(seg.) & CME & \%G & \%GP \\ [0.5ex]
\hline
CMT1X & 474.70 & 1.96 & 
475.32 & 2.02 & \bf{470.48} & 
0.90 & 1.03\\CMT1Y & \bf{470.48} & 1.86 & 
472.60 & 1.87 & 470.48 & 0.00
 & 0.45\\CMT2X & 707.37 & 9.26 & 
708.14 & 8.98 & \bf{682.39} & 
3.66 & 3.77\\CMT2Y & 702.74 & 9.24 & 
706.11 & 8.82 & \bf{682.39} & 
2.98 & 3.48\\CMT3X & 728.93 & 27.80 & 
730.09 & 27.61 & \bf{719.06} & 
1.37 & 1.53\\CMT3Y & 734.56 & 30.07 & 
736.53 & 29.07 & \bf{719.06} & 
2.16 & 2.43\\CMT4X & 892.02 & 119.35 & 
895.98 & 118.46 & \bf{854.21} & 
4.43 & 4.89\\CMT4Y & 885.84 & 120.65 & 
887.61 & 123.80 & \bf{852.46} & 
3.92 & 4.12\\CMT5X & 100000 & 0 & 
nan & nan & \bf{1030.56} & 
9603.46 & nan\\CMT5Y & 1089.21 & 356.79 & 
1089.32 & 364.68 & \bf{1031.69} & 
5.58 & 5.59\\CMT11X & 874.72 & 55.52 & 
880.06 & 53.18 & \bf{831.09} & 
5.25 & 5.89\\CMT11Y & 843.04 & 45.21 & 
848.46 & 50.86 & \bf{829.85} & 
1.59 & 2.24\\CMT12X & 665.66 & 22.40 & 
672.15 & 23.30 & \bf{658.83} & 
1.04 & 2.02\\CMT12Y & 682.59 & 23.37 & 
684.42 & 23.16 & \bf{660.47} & 
3.35 & 3.63\\\bf{PROM.} & 
\bf{7839.42} & \bf{58.82} & \bf{nan} & \bf{nan} & \bf{749.50} & \bf{688.55} & \bf{nan}\\[1ex]\hline
\end{tabular}
\label{table:nonlin}
\end{table} \clearpage
\begin{table}[ht]
\caption{Resultados de la ejecución de la metaheurística ACO, utilizando instancias de SalhiNagy con la configuración -n 3.0 -alpha 1.0 -beta 3.0 -q .3 -ro 0.015}
\centering
\small
\begin{tabular}{c c c c c c c c}
\hline\hline
Instancia & Costo mínimo & Tiempo(seg.) & Costo promedio & Tiempo promedio(seg.) & CME & \%G & \%GP \\ [0.5ex]
\hline
CMT1X & 476.71 & 1.84 & 
476.92 & 1.90 & \bf{470.48} & 
1.32 & 1.37\\CMT1Y & 472.37 & 1.74 & 
473.54 & 1.81 & \bf{470.48} & 
0.40 & 0.65\\CMT2X & 696.28 & 8.97 & 
701.77 & 8.91 & \bf{682.39} & 
2.04 & 2.84\\CMT2Y & 707.08 & 8.98 & 
708.40 & 9.02 & \bf{682.39} & 
3.62 & 3.81\\CMT3X & 733.71 & 26.87 & 
734.40 & 27.36 & \bf{719.06} & 
2.04 & 2.13\\CMT3Y & 731.81 & 29.07 & 
733.43 & 28.59 & \bf{719.06} & 
1.77 & 2.00\\CMT4X & 879.06 & 118.74 & 
882.26 & 121.38 & \bf{854.21} & 
2.91 & 3.28\\CMT4Y & 886.46 & 121.27 & 
888.43 & 121.99 & \bf{852.46} & 
3.99 & 4.22\\CMT5X & 1097.03 & 371.51 & 
1099.79 & 360.80 & \bf{1030.56} & 
6.45 & 6.72\\CMT5Y & 1078.60 & 344.92 & 
1079.73 & 354.44 & \bf{1031.69} & 
4.55 & 4.66\\CMT11X & 859.99 & 54.65 & 
880.09 & 54.96 & \bf{831.09} & 
3.48 & 5.90\\CMT11Y & 896.28 & 46.71 & 
903.67 & 47.96 & \bf{829.85} & 
8.01 & 8.90\\CMT12X & 677.28 & 25.64 & 
677.32 & 23.54 & \bf{658.83} & 
2.80 & 2.81\\CMT12Y & 678.07 & 22.76 & 
678.63 & 22.86 & \bf{660.47} & 
2.66 & 2.75\\\bf{PROM.} & 
\bf{776.48} & \bf{84.55} & \bf{779.88} & \bf{84.68} & \bf{749.50} & \bf{3.29} & \bf{3.72}\\[1ex]\hline
\end{tabular}
\label{table:nonlin}
\end{table} \clearpage
\begin{table}[ht]
\caption{Resultados de la ejecución de la metaheurística ACO, utilizando instancias de SalhiNagy con la configuración -n 3.0 -alpha 1.0 -beta 3.0 -q .4 -ro 0.015}
\centering
\small
\begin{tabular}{c c c c c c c c}
\hline\hline
Instancia & Costo mínimo & Tiempo(seg.) & Costo promedio & Tiempo promedio(seg.) & CME & \%G & \%GP \\ [0.5ex]
\hline
CMT1X & 477.20 & 1.85 & 
479.00 & 1.82 & \bf{470.48} & 
1.43 & 1.81\\CMT1Y & 471.25 & 2.02 & 
474.71 & 1.97 & \bf{470.48} & 
0.16 & 0.90\\CMT2X & 689.46 & 8.86 & 
695.48 & 8.94 & \bf{682.39} & 
1.04 & 1.92\\CMT2Y & 701.54 & 8.73 & 
704.24 & 8.89 & \bf{682.39} & 
2.81 & 3.20\\CMT3X & 736.81 & 27.36 & 
737.33 & 26.68 & \bf{719.06} & 
2.47 & 2.54\\CMT3Y & 728.81 & 28.37 & 
731.99 & 29.32 & \bf{719.06} & 
1.36 & 1.80\\CMT4X & 884.71 & 115.10 & 
889.33 & 115.58 & \bf{854.21} & 
3.57 & 4.11\\CMT4Y & 888.67 & 119.54 & 
892.54 & 119.23 & \bf{852.46} & 
4.25 & 4.70\\CMT5X & 1089.32 & 345.02 & 
1091.24 & 352.59 & \bf{1030.56} & 
5.70 & 5.89\\CMT5Y & 1082.96 & 355.85 & 
1089.63 & 358.55 & \bf{1031.69} & 
4.97 & 5.62\\CMT11X & 864.72 & 53.23 & 
871.75 & 52.06 & \bf{831.09} & 
4.05 & 4.89\\CMT11Y & 888.34 & 44.67 & 
891.84 & 47.52 & \bf{829.85} & 
7.05 & 7.47\\CMT12X & 677.81 & 22.46 & 
677.94 & 22.03 & \bf{658.83} & 
2.88 & 2.90\\CMT12Y & 674.03 & 20.52 & 
679.54 & 20.84 & \bf{660.47} & 
2.05 & 2.89\\\bf{PROM.} & 
\bf{775.40} & \bf{82.40} & \bf{779.04} & \bf{83.29} & \bf{749.50} & \bf{3.13} & \bf{3.62}\\[1ex]\hline
\end{tabular}
\label{table:nonlin}
\end{table} \clearpage
\begin{table}[ht]
\caption{Resultados de la ejecución de la metaheurística ACO, utilizando instancias de SalhiNagy con la configuración -n 4.0 -alpha 1.0 -beta 3.0 -q 0.1 -ro 0.015}
\centering
\small
\begin{tabular}{c c c c c c c c}
\hline\hline
Instancia & Costo mínimo & Tiempo(seg.) & Costo promedio & Tiempo promedio(seg.) & CME & \%G & \%GP \\ [0.5ex]
\hline
CMT1X & 474.91 & 2.60 & 
478.94 & 2.52 & \bf{470.48} & 
0.94 & 1.80\\CMT1Y & 474.72 & 2.56 & 
476.78 & 2.60 & \bf{470.48} & 
0.90 & 1.34\\CMT2X & 698.66 & 11.86 & 
702.43 & 12.04 & \bf{682.39} & 
2.38 & 2.94\\CMT2Y & 698.20 & 11.69 & 
702.35 & 11.85 & \bf{682.39} & 
2.32 & 2.93\\CMT3X & 733.61 & 35.87 & 
734.20 & 35.70 & \bf{719.06} & 
2.02 & 2.11\\CMT3Y & 732.39 & 38.05 & 
733.18 & 39.04 & \bf{719.06} & 
1.85 & 1.96\\CMT4X & 876.09 & 158.14 & 
886.90 & 160.46 & \bf{854.21} & 
2.56 & 3.83\\CMT4Y & 893.40 & 162.86 & 
894.73 & 162.59 & \bf{852.46} & 
4.80 & 4.96\\CMT5X & 1076.36 & 463.69 & 
1081.05 & 494.47 & \bf{1030.56} & 
4.44 & 4.90\\CMT5Y & 1081.99 & 499.68 & 
1088.77 & 496.27 & \bf{1031.69} & 
4.88 & 5.53\\CMT11X & 842.10 & 70.23 & 
867.16 & 69.14 & \bf{831.09} & 
1.32 & 4.34\\CMT11Y & 862.98 & 68.71 & 
879.25 & 69.78 & \bf{829.85} & 
3.99 & 5.95\\CMT12X & 674.79 & 30.61 & 
675.29 & 30.77 & \bf{658.83} & 
2.42 & 2.50\\CMT12Y & 678.10 & 30.16 & 
679.05 & 30.00 & \bf{660.47} & 
2.67 & 2.81\\\bf{PROM.} & 
\bf{771.31} & \bf{113.34} & \bf{777.15} & \bf{115.52} & \bf{749.50} & \bf{2.68} & \bf{3.42}\\[1ex]\hline
\end{tabular}
\label{table:nonlin}
\end{table} \clearpage
\begin{table}[ht]
\caption{Resultados de la ejecución de la metaheurística ACO, utilizando instancias de SalhiNagy con la configuración -n 4.0 -alpha 1.0 -beta 3.0 -q .2 -ro 0.015}
\centering
\small
\begin{tabular}{c c c c c c c c}
\hline\hline
Instancia & Costo mínimo & Tiempo(seg.) & Costo promedio & Tiempo promedio(seg.) & CME & \%G & \%GP \\ [0.5ex]
\hline
CMT1X & 476.95 & 2.68 & 
477.81 & 2.79 & \bf{470.48} & 
1.38 & 1.56\\CMT1Y & \bf{470.48} & 2.62 & 
474.73 & 2.58 & 470.48 & 0.00
 & 0.90\\CMT2X & 697.58 & 12.47 & 
697.90 & 12.21 & \bf{682.39} & 
2.23 & 2.27\\CMT2Y & 702.01 & 11.91 & 
703.46 & 11.79 & \bf{682.39} & 
2.88 & 3.09\\CMT3X & 733.36 & 35.59 & 
733.83 & 35.47 & \bf{719.06} & 
1.99 & 2.05\\CMT3Y & 727.83 & 37.77 & 
730.38 & 37.51 & \bf{719.06} & 
1.22 & 1.57\\CMT4X & 872.89 & 162.83 & 
877.74 & 164.30 & \bf{854.21} & 
2.19 & 2.75\\CMT4Y & 881.66 & 169.30 & 
887.10 & 166.22 & \bf{852.46} & 
3.43 & 4.06\\CMT5X & 1085.25 & 472.46 & 
1087.04 & 469.17 & \bf{1030.56} & 
5.31 & 5.48\\CMT5Y & 1088.55 & 504.08 & 
1090.24 & 491.81 & \bf{1031.69} & 
5.51 & 5.68\\CMT11X & 856.76 & 67.78 & 
863.33 & 68.98 & \bf{831.09} & 
3.09 & 3.88\\CMT11Y & 866.49 & 64.77 & 
868.27 & 63.53 & \bf{829.85} & 
4.42 & 4.63\\CMT12X & 675.25 & 29.85 & 
676.45 & 30.74 & \bf{658.83} & 
2.49 & 2.67\\CMT12Y & 677.25 & 31.14 & 
677.93 & 31.45 & \bf{660.47} & 
2.54 & 2.64\\\bf{PROM.} & 
\bf{772.31} & \bf{114.66} & \bf{774.73} & \bf{113.47} & \bf{749.50} & \bf{2.76} & \bf{3.09}\\[1ex]\hline
\end{tabular}
\label{table:nonlin}
\end{table} \clearpage
\begin{table}[ht]
\caption{Resultados de la ejecución de la metaheurística ACO, utilizando instancias de SalhiNagy con la configuración -n 4.0 -alpha 1.0 -beta 3.0 -q .3 -ro 0.015}
\centering
\small
\begin{tabular}{c c c c c c c c}
\hline\hline
Instancia & Costo mínimo & Tiempo(seg.) & Costo promedio & Tiempo promedio(seg.) & CME & \%G & \%GP \\ [0.5ex]
\hline
CMT1X & 472.37 & 2.52 & 
476.04 & 2.63 & \bf{470.48} & 
0.40 & 1.18\\CMT1Y & 474.41 & 2.51 & 
477.21 & 2.50 & \bf{470.48} & 
0.84 & 1.43\\CMT2X & 691.69 & 11.74 & 
698.67 & 12.03 & \bf{682.39} & 
1.36 & 2.39\\CMT2Y & 701.44 & 10.91 & 
701.90 & 11.38 & \bf{682.39} & 
2.79 & 2.86\\CMT3X & 729.13 & 35.54 & 
733.76 & 36.40 & \bf{719.06} & 
1.40 & 2.04\\CMT3Y & 728.91 & 42.67 & 
731.51 & 40.35 & \bf{719.06} & 
1.37 & 1.73\\CMT4X & 886.04 & 164.93 & 
886.47 & 161.44 & \bf{854.21} & 
3.73 & 3.78\\CMT4Y & 889.66 & 161.03 & 
889.74 & 170.28 & \bf{852.46} & 
4.36 & 4.37\\CMT5X & 1080.32 & 474.04 & 
1085.86 & 474.11 & \bf{1030.56} & 
4.83 & 5.37\\CMT5Y & 1078.95 & 475.12 & 
1084.78 & 479.30 & \bf{1031.69} & 
4.58 & 5.15\\CMT11X & 855.77 & 67.37 & 
870.47 & 70.45 & \bf{831.09} & 
2.97 & 4.74\\CMT11Y & 853.00 & 60.41 & 
854.34 & 59.88 & \bf{829.85} & 
2.79 & 2.95\\CMT12X & 675.00 & 30.70 & 
676.03 & 29.64 & \bf{658.83} & 
2.45 & 2.61\\CMT12Y & 672.60 & 27.45 & 
676.11 & 28.50 & \bf{660.47} & 
1.84 & 2.37\\\bf{PROM.} & 
\bf{770.66} & \bf{111.92} & \bf{774.49} & \bf{112.78} & \bf{749.50} & \bf{2.55} & \bf{3.07}\\[1ex]\hline
\end{tabular}
\label{table:nonlin}
\end{table} \clearpage
\begin{table}[ht]
\caption{Resultados de la ejecución de la metaheurística ACO, utilizando instancias de SalhiNagy con la configuración -n 4.0 -alpha 1.0 -beta 3.0 -q .4 -ro 0.015}
\centering
\small
\begin{tabular}{c c c c c c c c}
\hline\hline
Instancia & Costo mínimo & Tiempo(seg.) & Costo promedio & Tiempo promedio(seg.) & CME & \%G & \%GP \\ [0.5ex]
\hline
CMT1X & 472.87 & 2.48 & 
474.89 & 2.52 & \bf{470.48} & 
0.51 & 0.94\\CMT1Y & 472.01 & 2.43 & 
474.69 & 2.51 & \bf{470.48} & 
0.33 & 0.90\\CMT2X & 706.66 & 11.95 & 
708.45 & 11.88 & \bf{682.39} & 
3.56 & 3.82\\CMT2Y & 695.12 & 11.26 & 
701.43 & 11.40 & \bf{682.39} & 
1.87 & 2.79\\CMT3X & 732.71 & 35.86 & 
734.28 & 35.48 & \bf{719.06} & 
1.90 & 2.12\\CMT3Y & 734.62 & 36.93 & 
734.69 & 37.60 & \bf{719.06} & 
2.16 & 2.17\\CMT4X & 882.04 & 154.90 & 
887.68 & 155.34 & \bf{854.21} & 
3.26 & 3.92\\CMT4Y & 880.37 & 155.88 & 
887.30 & 156.56 & \bf{852.46} & 
3.27 & 4.09\\CMT5X & 1090.07 & 496.77 & 
1090.93 & 482.64 & \bf{1030.56} & 
5.77 & 5.86\\CMT5Y & 1072.19 & 469.75 & 
1078.80 & 470.04 & \bf{1031.69} & 
3.93 & 4.57\\CMT11X & 859.99 & 76.33 & 
863.21 & 72.75 & \bf{831.09} & 
3.48 & 3.86\\CMT11Y & 853.57 & 67.32 & 
869.67 & 63.69 & \bf{829.85} & 
2.86 & 4.80\\CMT12X & 675.72 & 31.70 & 
676.74 & 31.72 & \bf{658.83} & 
2.56 & 2.72\\CMT12Y & 668.42 & 27.88 & 
673.22 & 28.41 & \bf{660.47} & 
1.20 & 1.93\\\bf{PROM.} & 
\bf{771.17} & \bf{112.96} & \bf{775.43} & \bf{111.61} & \bf{749.50} & \bf{2.62} & \bf{3.18}\\[1ex]\hline
\end{tabular}
\label{table:nonlin}
\end{table} \clearpage
\begin{table}[ht]
\caption{Resultados de la ejecución de la metaheurística ACO, utilizando instancias de SalhiNagy con la configuración -n 5.0 -alpha 1.0 -beta 3.0 -q 0.1 -ro 0.015}
\centering
\small
\begin{tabular}{c c c c c c c c}
\hline\hline
Instancia & Costo mínimo & Tiempo(seg.) & Costo promedio & Tiempo promedio(seg.) & CME & \%G & \%GP \\ [0.5ex]
\hline
CMT1X & 474.41 & 3.49 & 
476.51 & 3.50 & \bf{470.48} & 
0.84 & 1.28\\CMT1Y & 475.22 & 3.20 & 
475.40 & 3.16 & \bf{470.48} & 
1.01 & 1.05\\CMT2X & 696.29 & 15.01 & 
699.96 & 14.68 & \bf{682.39} & 
2.04 & 2.57\\CMT2Y & 700.56 & 14.88 & 
703.67 & 14.53 & \bf{682.39} & 
2.66 & 3.12\\CMT3X & 724.14 & 44.89 & 
726.48 & 45.26 & \bf{719.06} & 
0.71 & 1.03\\CMT3Y & 721.81 & 47.45 & 
724.25 & 47.20 & \bf{719.06} & 
0.38 & 0.72\\CMT4X & 883.57 & 203.36 & 
886.51 & 204.58 & \bf{854.21} & 
3.44 & 3.78\\CMT4Y & 882.16 & 197.48 & 
886.42 & 199.82 & \bf{852.46} & 
3.48 & 3.98\\CMT5X & 1083.56 & 603.75 & 
1086.90 & 614.36 & \bf{1030.56} & 
5.14 & 5.47\\CMT5Y & 1081.01 & 600.45 & 
1087.33 & 588.44 & \bf{1031.69} & 
4.78 & 5.39\\CMT11X & 882.60 & 89.41 & 
882.66 & 92.17 & \bf{831.09} & 
6.20 & 6.21\\CMT11Y & 838.84 & 75.93 & 
855.70 & 77.39 & \bf{829.85} & 
1.08 & 3.11\\CMT12X & 669.35 & 40.06 & 
671.24 & 39.80 & \bf{658.83} & 
1.60 & 1.88\\CMT12Y & 662.46 & 36.05 & 
668.60 & 37.15 & \bf{660.47} & 
0.30 & 1.23\\\bf{PROM.} & 
\bf{769.71} & \bf{141.10} & \bf{773.69} & \bf{141.57} & \bf{749.50} & \bf{2.40} & \bf{2.92}\\[1ex]\hline
\end{tabular}
\label{table:nonlin}
\end{table} \clearpage
\begin{table}[ht]
\caption{Resultados de la ejecución de la metaheurística ACO, utilizando instancias de SalhiNagy con la configuración -n 5.0 -alpha 1.0 -beta 3.0 -q .2 -ro 0.015}
\centering
\small
\begin{tabular}{c c c c c c c c}
\hline\hline
Instancia & Costo mínimo & Tiempo(seg.) & Costo promedio & Tiempo promedio(seg.) & CME & \%G & \%GP \\ [0.5ex]
\hline
CMT1X & 472.87 & 3.29 & 
474.70 & 3.20 & \bf{470.48} & 
0.51 & 0.90\\CMT1Y & \bf{470.48} & 3.21 & 
473.10 & 3.12 & 470.48 & 0.00
 & 0.56\\CMT2X & 696.08 & 15.01 & 
700.63 & 15.13 & \bf{682.39} & 
2.01 & 2.67\\CMT2Y & 695.32 & 13.91 & 
697.67 & 14.16 & \bf{682.39} & 
1.89 & 2.24\\CMT3X & 728.60 & 45.19 & 
729.36 & 46.02 & \bf{719.06} & 
1.33 & 1.43\\CMT3Y & 728.12 & 46.64 & 
728.52 & 51.11 & \bf{719.06} & 
1.26 & 1.32\\CMT4X & 881.08 & 198.30 & 
882.75 & 199.28 & \bf{854.21} & 
3.15 & 3.34\\CMT4Y & 886.35 & 199.78 & 
891.46 & 204.73 & \bf{852.46} & 
3.98 & 4.57\\CMT5X & 1080.81 & 602.13 & 
1081.40 & 630.07 & \bf{1030.56} & 
4.88 & 4.93\\CMT5Y & 100000 & 0 & 
nan & nan & \bf{1031.69} & 
9592.83 & nan\\CMT11X & 847.52 & 83.91 & 
860.72 & 90.90 & \bf{831.09} & 
1.98 & 3.57\\CMT11Y & 844.12 & 78.27 & 
846.60 & 77.60 & \bf{829.85} & 
1.72 & 2.02\\CMT12X & 666.81 & 37.34 & 
670.82 & 37.83 & \bf{658.83} & 
1.21 & 1.82\\CMT12Y & 675.49 & 41.47 & 
676.02 & 39.85 & \bf{660.47} & 
2.27 & 2.35\\\bf{PROM.} & 
\bf{7833.83} & \bf{97.75} & \bf{nan} & \bf{nan} & \bf{749.50} & \bf{687.07} & \bf{nan}\\[1ex]\hline
\end{tabular}
\label{table:nonlin}
\end{table} \clearpage
\begin{table}[ht]
\caption{Resultados de la ejecución de la metaheurística ACO, utilizando instancias de SalhiNagy con la configuración -n 5.0 -alpha 1.0 -beta 3.0 -q .3 -ro 0.015}
\centering
\small
\begin{tabular}{c c c c c c c c}
\hline\hline
Instancia & Costo mínimo & Tiempo(seg.) & Costo promedio & Tiempo promedio(seg.) & CME & \%G & \%GP \\ [0.5ex]
\hline
CMT1X & 470.67 & 3.19 & 
474.61 & 3.15 & \bf{470.48} & 
0.04 & 0.88\\CMT1Y & 475.37 & 3.20 & 
476.64 & 3.04 & \bf{470.48} & 
1.04 & 1.31\\CMT2X & 697.30 & 14.31 & 
701.66 & 14.39 & \bf{682.39} & 
2.18 & 2.82\\CMT2Y & 694.79 & 14.39 & 
697.64 & 14.38 & \bf{682.39} & 
1.82 & 2.24\\CMT3X & 726.85 & 47.55 & 
727.23 & 46.10 & \bf{719.06} & 
1.08 & 1.14\\CMT3Y & 732.70 & 46.26 & 
733.92 & 47.01 & \bf{719.06} & 
1.90 & 2.07\\CMT4X & 884.79 & 195.24 & 
885.15 & 195.69 & \bf{854.21} & 
3.58 & 3.62\\CMT4Y & 878.35 & 205.48 & 
885.23 & 201.16 & \bf{852.46} & 
3.04 & 3.84\\CMT5X & 1076.69 & 574.24 & 
1080.76 & 582.87 & \bf{1030.56} & 
4.48 & 4.87\\CMT5Y & 1081.86 & 618.64 & 
1087.61 & 618.00 & \bf{1031.69} & 
4.86 & 5.42\\CMT11X & 852.51 & 95.24 & 
865.27 & 90.78 & \bf{831.09} & 
2.58 & 4.11\\CMT11Y & 847.88 & 77.54 & 
847.96 & 76.77 & \bf{829.85} & 
2.17 & 2.18\\CMT12X & 674.88 & 38.02 & 
677.86 & 38.45 & \bf{658.83} & 
2.44 & 2.89\\CMT12Y & 674.33 & 35.29 & 
677.83 & 36.09 & \bf{660.47} & 
2.10 & 2.63\\\bf{PROM.} & 
\bf{769.21} & \bf{140.61} & \bf{772.81} & \bf{140.56} & \bf{749.50} & \bf{2.38} & \bf{2.86}\\[1ex]\hline
\end{tabular}
\label{table:nonlin}
\end{table} \clearpage
\begin{table}[ht]
\caption{Resultados de la ejecución de la metaheurística ACO, utilizando instancias de SalhiNagy con la configuración -n 5.0 -alpha 1.0 -beta 3.0 -q .4 -ro 0.015}
\centering
\small
\begin{tabular}{c c c c c c c c}
\hline\hline
Instancia & Costo mínimo & Tiempo(seg.) & Costo promedio & Tiempo promedio(seg.) & CME & \%G & \%GP \\ [0.5ex]
\hline
CMT1X & 472.87 & 3.46 & 
475.30 & 3.32 & \bf{470.48} & 
0.51 & 1.02\\CMT1Y & 471.09 & 3.15 & 
473.25 & 3.25 & \bf{470.48} & 
0.13 & 0.59\\CMT2X & 700.26 & 15.92 & 
702.00 & 15.52 & \bf{682.39} & 
2.62 & 2.87\\CMT2Y & 696.51 & 14.78 & 
698.41 & 14.43 & \bf{682.39} & 
2.07 & 2.35\\CMT3X & 726.47 & 45.78 & 
729.82 & 45.08 & \bf{719.06} & 
1.03 & 1.50\\CMT3Y & 731.62 & 45.49 & 
733.32 & 47.52 & \bf{719.06} & 
1.75 & 1.98\\CMT4X & 873.05 & 201.00 & 
880.00 & 196.01 & \bf{854.21} & 
2.21 & 3.02\\CMT4Y & 880.95 & 201.32 & 
885.86 & 203.00 & \bf{852.46} & 
3.34 & 3.92\\CMT5X & 1088.15 & 603.70 & 
1090.07 & 620.89 & \bf{1030.56} & 
5.59 & 5.77\\CMT5Y & 1086.61 & 587.27 & 
1089.18 & 583.59 & \bf{1031.69} & 
5.32 & 5.57\\CMT11X & 877.52 & 93.34 & 
877.58 & 92.02 & \bf{831.09} & 
5.59 & 5.59\\CMT11Y & 849.74 & 80.47 & 
869.46 & 81.64 & \bf{829.85} & 
2.40 & 4.77\\CMT12X & 674.29 & 38.61 & 
675.30 & 37.77 & \bf{658.83} & 
2.35 & 2.50\\CMT12Y & 672.55 & 38.69 & 
678.02 & 36.80 & \bf{660.47} & 
1.83 & 2.66\\\bf{PROM.} & 
\bf{771.55} & \bf{140.93} & \bf{775.54} & \bf{141.49} & \bf{749.50} & \bf{2.62} & \bf{3.15}\\[1ex]\hline
\end{tabular}
\label{table:nonlin}
\end{table} \clearpage
\begin{table}[ht]
\caption{Resultados de la ejecución de la metaheurística ACO, utilizando instancias de SalhiNagy con la configuración -n 6.0 -alpha 1.0 -beta 3.0 -q 0.1 -ro 0.015}
\centering
\small
\begin{tabular}{c c c c c c c c}
\hline\hline
Instancia & Costo mínimo & Tiempo(seg.) & Costo promedio & Tiempo promedio(seg.) & CME & \%G & \%GP \\ [0.5ex]
\hline
CMT1X & 472.37 & 3.91 & 
474.38 & 3.73 & \bf{470.48} & 
0.40 & 0.83\\CMT1Y & 472.87 & 3.79 & 
474.79 & 3.89 & \bf{470.48} & 
0.51 & 0.92\\CMT2X & 707.59 & 17.75 & 
707.94 & 18.12 & \bf{682.39} & 
3.69 & 3.74\\CMT2Y & 692.84 & 17.46 & 
695.51 & 17.46 & \bf{682.39} & 
1.53 & 1.92\\CMT3X & 731.48 & 68.94 & 
731.48 & 65.67 & \bf{719.06} & 
1.73 & 1.73\\CMT3Y & 725.12 & 56.05 & 
728.42 & 55.62 & \bf{719.06} & 
0.84 & 1.30\\CMT4X & 890.27 & 237.22 & 
891.14 & 240.45 & \bf{854.21} & 
4.22 & 4.32\\CMT4Y & 883.64 & 236.18 & 
888.63 & 258.64 & \bf{852.46} & 
3.66 & 4.24\\CMT5X & 1077.77 & 711.55 & 
1085.11 & 714.21 & \bf{1030.56} & 
4.58 & 5.29\\CMT5Y & 1085.11 & | & 
0.00 & 0.00 & \bf{1031.69} & 
5.18 & -100.00\\CMT11X & 851.62 & 104.75 & 
855.48 & 107.10 & \bf{831.09} & 
2.47 & 2.93\\CMT11Y & 842.27 & 94.41 & 
844.58 & 97.94 & \bf{829.85} & 
1.50 & 1.78\\CMT12X & 673.32 & 45.98 & 
673.50 & 45.17 & \bf{658.83} & 
2.20 & 2.23\\CMT12Y & 671.73 & 47.80 & 
673.52 & 47.37 & \bf{660.47} & 
1.70 & 1.98\\\bf{PROM.} & 
\bf{769.86} & \bf{117.56} & \bf{694.61} & \bf{119.67} & \bf{749.50} & \bf{2.44} & \bf{-4.77}\\[1ex]\hline
\end{tabular}
\label{table:nonlin}
\end{table} \clearpage
\begin{table}[ht]
\caption{Resultados de la ejecución de la metaheurística ACO, utilizando instancias de SalhiNagy con la configuración -n 6.0 -alpha 1.0 -beta 3.0 -q .2 -ro 0.015}
\centering
\small
\begin{tabular}{c c c c c c c c}
\hline\hline
Instancia & Costo mínimo & Tiempo(seg.) & Costo promedio & Tiempo promedio(seg.) & CME & \%G & \%GP \\ [0.5ex]
\hline
CMT1X & 473.29 & 3.76 & 
476.89 & 3.81 & \bf{470.48} & 
0.60 & 1.36\\CMT1Y & 476.95 & 3.97 & 
478.42 & 3.81 & \bf{470.48} & 
1.38 & 1.69\\CMT2X & 697.46 & 17.42 & 
698.37 & 17.54 & \bf{682.39} & 
2.21 & 2.34\\CMT2Y & 702.13 & 16.63 & 
702.94 & 16.58 & \bf{682.39} & 
2.89 & 3.01\\CMT3X & 727.04 & 52.55 & 
729.83 & 53.02 & \bf{719.06} & 
1.11 & 1.50\\CMT3Y & 726.68 & 56.06 & 
729.75 & 57.70 & \bf{719.06} & 
1.06 & 1.49\\CMT4X & 886.43 & 239.55 & 
888.27 & 240.36 & \bf{854.21} & 
3.77 & 3.99\\CMT4Y & 885.39 & 244.69 & 
888.26 & 244.94 & \bf{852.46} & 
3.86 & 4.20\\CMT5X & 1072.74 & 731.95 & 
1084.05 & 712.10 & \bf{1030.56} & 
4.09 & 5.19\\CMT5Y & 1081.67 & 755.15 & 
1090.54 & 725.95 & \bf{1031.69} & 
4.84 & 5.70\\CMT11X & 860.50 & 104.53 & 
872.68 & 103.09 & \bf{831.09} & 
3.54 & 5.00\\CMT11Y & 862.86 & 102.41 & 
865.51 & 100.66 & \bf{829.85} & 
3.98 & 4.30\\CMT12X & 666.05 & 49.70 & 
672.72 & 48.59 & \bf{658.83} & 
1.10 & 2.11\\CMT12Y & 677.81 & 43.75 & 
681.31 & 45.36 & \bf{660.47} & 
2.63 & 3.16\\\bf{PROM.} & 
\bf{771.21} & \bf{173.01} & \bf{775.68} & \bf{169.54} & \bf{749.50} & \bf{2.65} & \bf{3.22}\\[1ex]\hline
\end{tabular}
\label{table:nonlin}
\end{table} \clearpage
\begin{table}[ht]
\caption{Resultados de la ejecución de la metaheurística ACO, utilizando instancias de SalhiNagy con la configuración -n 6.0 -alpha 1.0 -beta 3.0 -q .3 -ro 0.015}
\centering
\small
\begin{tabular}{c c c c c c c c}
\hline\hline
Instancia & Costo mínimo & Tiempo(seg.) & Costo promedio & Tiempo promedio(seg.) & CME & \%G & \%GP \\ [0.5ex]
\hline
CMT1X & 474.72 & 4.03 & 
477.20 & 3.91 & \bf{470.48} & 
0.90 & 1.43\\CMT1Y & 474.41 & 3.50 & 
474.56 & 3.63 & \bf{470.48} & 
0.84 & 0.87\\CMT2X & 701.00 & 17.42 & 
701.53 & 17.57 & \bf{682.39} & 
2.73 & 2.80\\CMT2Y & 699.88 & 17.50 & 
701.62 & 17.32 & \bf{682.39} & 
2.56 & 2.82\\CMT3X & 732.74 & 55.14 & 
732.92 & 55.13 & \bf{719.06} & 
1.90 & 1.93\\CMT3Y & 727.10 & 60.45 & 
729.36 & 57.66 & \bf{719.06} & 
1.12 & 1.43\\CMT4X & 883.00 & 233.07 & 
889.30 & 233.16 & \bf{854.21} & 
3.37 & 4.11\\CMT4Y & 884.90 & 243.10 & 
888.06 & 242.22 & \bf{852.46} & 
3.81 & 4.18\\CMT5X & 1088.88 & 716.46 & 
1090.60 & 695.30 & \bf{1030.56} & 
5.66 & 5.83\\CMT5Y & 1084.99 & 706.68 & 
1088.22 & 707.80 & \bf{1031.69} & 
5.17 & 5.48\\CMT11X & 863.81 & 104.32 & 
871.64 & 102.93 & \bf{831.09} & 
3.94 & 4.88\\CMT11Y & 861.68 & 100.27 & 
872.79 & 95.83 & \bf{829.85} & 
3.84 & 5.18\\CMT12X & 676.05 & 44.71 & 
676.06 & 45.96 & \bf{658.83} & 
2.61 & 2.62\\CMT12Y & 670.27 & 42.27 & 
673.05 & 44.69 & \bf{660.47} & 
1.48 & 1.91\\\bf{PROM.} & 
\bf{773.10} & \bf{167.78} & \bf{776.21} & \bf{165.94} & \bf{749.50} & \bf{2.85} & \bf{3.25}\\[1ex]\hline
\end{tabular}
\label{table:nonlin}
\end{table} \clearpage
\begin{table}[ht]
\caption{Resultados de la ejecución de la metaheurística ACO, utilizando instancias de SalhiNagy con la configuración -n 6.0 -alpha 1.0 -beta 3.0 -q .4 -ro 0.015}
\centering
\small
\begin{tabular}{c c c c c c c c}
\hline\hline
Instancia & Costo mínimo & Tiempo(seg.) & Costo promedio & Tiempo promedio(seg.) & CME & \%G & \%GP \\ [0.5ex]
\hline
CMT1X & 470.67 & 3.83 & 
473.59 & 3.89 & \bf{470.48} & 
0.04 & 0.66\\CMT1Y & 472.87 & 3.97 & 
474.05 & 3.85 & \bf{470.48} & 
0.51 & 0.76\\CMT2X & 688.97 & 17.13 & 
695.06 & 17.55 & \bf{682.39} & 
0.96 & 1.86\\CMT2Y & 690.17 & 16.36 & 
696.62 & 16.65 & \bf{682.39} & 
1.14 & 2.08\\CMT3X & 724.04 & 54.73 & 
728.50 & 53.91 & \bf{719.06} & 
0.69 & 1.31\\CMT3Y & 727.09 & 55.92 & 
730.07 & 57.35 & \bf{719.06} & 
1.12 & 1.53\\CMT4X & 880.63 & 255.25 & 
882.48 & 241.53 & \bf{854.21} & 
3.09 & 3.31\\CMT4Y & 866.70 & 235.04 & 
876.96 & 234.60 & \bf{852.46} & 
1.67 & 2.87\\CMT5X & 100000 & 0 & 
nan & nan & \bf{1030.56} & 
9603.46 & nan\\CMT5Y & 1083.94 & 705.72 & 
1089.61 & 713.62 & \bf{1031.69} & 
5.06 & 5.61\\CMT11X & 870.94 & 101.28 & 
877.43 & 102.33 & \bf{831.09} & 
4.79 & 5.58\\CMT11Y & 844.32 & 86.83 & 
854.38 & 88.69 & \bf{829.85} & 
1.74 & 2.96\\CMT12X & 676.00 & 43.75 & 
676.18 & 44.04 & \bf{658.83} & 
2.61 & 2.63\\CMT12Y & 671.33 & 43.21 & 
673.64 & 41.91 & \bf{660.47} & 
1.64 & 1.99\\\bf{PROM.} & 
\bf{7833.41} & \bf{115.93} & \bf{nan} & \bf{nan} & \bf{749.50} & \bf{687.75} & \bf{nan}\\[1ex]\hline
\end{tabular}
\label{table:nonlin}
\end{table} \clearpage
\begin{table}[ht]
\caption{Resultados de la ejecución de la metaheurística -n, utilizando instancias de -h con la configuración 3.0 -alpha 1.0 -beta 3.0 -q .2 -ro 0.015}
\centering
\small
\begin{tabular}{c c c c c c c c}
\hline\hline
Instancia & Costo mínimo & Tiempo(seg.) & Costo promedio & Tiempo promedio(seg.) & CME & \%G & \%GP \\ [0.5ex]
\hline
\bf{PROM.} & 
\bf{nan} & \bf{nan} & \bf{nan} & \bf{nan} & \bf{nan} & \bf{nan} & \bf{nan}\\[1ex]\hline
\end{tabular}
\label{table:nonlin}
\end{table} \clearpage
\begin{table}[ht]
\caption{Resultados de la ejecución de la metaheurística ACO, utilizando instancias de SalhiNagy con la configuración -n 3.0 -alpha 1.0 -beta 3.0 -q .2 -ro 0.015}
\centering
\small
\begin{tabular}{c c c c c c c c}
\hline\hline
Instancia & Costo mínimo & Tiempo(seg.) & Costo promedio & Tiempo promedio(seg.) & CME & \%G & \%GP \\ [0.5ex]
\hline
CMT1X & 476.71 & 2.14 & 
476.92 & 2.04 & \bf{470.48} & 
1.32 & 1.37\\CMT1Y & 472.87 & 1.99 & 
477.98 & 1.91 & \bf{470.48} & 
0.51 & 1.59\\CMT2X & 702.24 & 9.02 & 
704.24 & 9.05 & \bf{682.39} & 
2.91 & 3.20\\CMT2Y & 697.58 & 8.68 & 
698.92 & 8.71 & \bf{682.39} & 
2.23 & 2.42\\CMT3X & 727.29 & 27.26 & 
728.07 & 27.34 & \bf{719.06} & 
1.14 & 1.25\\CMT3Y & 731.48 & 29.43 & 
732.23 & 30.05 & \bf{719.06} & 
1.73 & 1.83\\CMT4X & 883.28 & 127.33 & 
891.00 & 124.53 & \bf{854.21} & 
3.40 & 4.31\\CMT4Y & 877.14 & 125.47 & 
881.01 & 128.90 & \bf{852.46} & 
2.90 & 3.35\\CMT5X & 1084.37 & 372.50 & 
1085.16 & 376.74 & \bf{1030.56} & 
5.22 & 5.30\\CMT5Y & 1091.01 & 385.72 & 
1093.49 & 382.95 & \bf{1031.69} & 
5.75 & 5.99\\CMT11X & 848.94 & 51.91 & 
853.50 & 52.27 & \bf{831.09} & 
2.15 & 2.70\\CMT11Y & 854.59 & 46.67 & 
865.20 & 48.85 & \bf{829.85} & 
2.98 & 4.26\\CMT12X & 676.35 & 23.46 & 
677.80 & 24.18 & \bf{658.83} & 
2.66 & 2.88\\CMT12Y & 668.99 & 23.01 & 
677.97 & 22.64 & \bf{660.47} & 
1.29 & 2.65\\\bf{PROM.} & 
\bf{770.92} & \bf{88.19} & \bf{774.53} & \bf{88.58} & \bf{749.50} & \bf{2.58} & \bf{3.08}\\[1ex]\hline
\end{tabular}
\label{table:nonlin}
\end{table} \clearpage
\begin{table}[ht]
\caption{Resultados de la ejecución de la metaheurística ACO, utilizando instancias de SalhiNagy con la configuración -n 5.0 -alpha 1.0 -beta 3.0 -q .2 -ro 0.015}
\centering
\small
\begin{tabular}{c c c c c c c c}
\hline\hline
Instancia & Costo mínimo & Tiempo(seg.) & Costo promedio & Tiempo promedio(seg.) & CME & \%G & \%GP \\ [0.5ex]
\hline
CMT1X & 471.25 & 3.44 & 
474.58 & 3.36 & \bf{470.48} & 
0.16 & 0.87\\CMT1Y & 474.72 & 3.09 & 
475.86 & 3.19 & \bf{470.48} & 
0.90 & 1.14\\CMT2X & 696.13 & 15.20 & 
701.39 & 14.91 & \bf{682.39} & 
2.01 & 2.79\\CMT2Y & 699.32 & 14.28 & 
699.43 & 14.38 & \bf{682.39} & 
2.48 & 2.50\\CMT3X & 727.48 & 45.33 & 
729.67 & 45.27 & \bf{719.06} & 
1.17 & 1.48\\CMT3Y & 724.95 & 47.49 & 
728.29 & 48.15 & \bf{719.06} & 
0.82 & 1.28\\CMT4X & 880.73 & 210.38 & 
886.06 & 206.81 & \bf{854.21} & 
3.10 & 3.73\\CMT4Y & 877.82 & 204.04 & 
879.16 & 203.10 & \bf{852.46} & 
2.97 & 3.13\\CMT5X & 1091.23 & 602.04 & 
1093.83 & 604.97 & \bf{1030.56} & 
5.89 & 6.14\\CMT5Y & 1079.94 & 593.30 & 
1090.51 & 603.59 & \bf{1031.69} & 
4.68 & 5.70\\CMT11X & 857.49 & 91.98 & 
861.01 & 88.90 & \bf{831.09} & 
3.18 & 3.60\\CMT11Y & 854.68 & 87.97 & 
868.82 & 83.85 & \bf{829.85} & 
2.99 & 4.70\\CMT12X & 672.45 & 40.15 & 
673.42 & 39.91 & \bf{658.83} & 
2.07 & 2.22\\CMT12Y & 670.12 & 37.07 & 
674.52 & 36.77 & \bf{660.47} & 
1.46 & 2.13\\\bf{PROM.} & 
\bf{769.88} & \bf{142.55} & \bf{774.04} & \bf{142.65} & \bf{749.50} & \bf{2.42} & \bf{2.96}\\[1ex]\hline
\end{tabular}
\label{table:nonlin}
\end{table} \clearpage
\begin{table}[ht]
\caption{Resultados de la ejecución de la metaheurística ACO, utilizando instancias de SalhiNagy con la configuración -n 6.0 -alpha 1.0 -beta 3.0 -q .4 -ro 0.015}
\centering
\small
\begin{tabular}{c c c c c c c c}
\hline\hline
Instancia & Costo mínimo & Tiempo(seg.) & Costo promedio & Tiempo promedio(seg.) & CME & \%G & \%GP \\ [0.5ex]
\hline
CMT1X & 476.38 & 3.78 & 
476.75 & 3.79 & \bf{470.48} & 
1.25 & 1.33\\CMT1Y & 472.87 & 3.70 & 
473.80 & 3.71 & \bf{470.48} & 
0.51 & 0.70\\CMT2X & 701.09 & 18.37 & 
701.25 & 18.31 & \bf{682.39} & 
2.74 & 2.76\\CMT2Y & 704.13 & 17.56 & 
704.86 & 17.43 & \bf{682.39} & 
3.19 & 3.29\\CMT3X & 729.80 & 55.76 & 
730.01 & 56.73 & \bf{719.06} & 
1.49 & 1.52\\CMT3Y & 728.03 & 54.68 & 
728.79 & 55.38 & \bf{719.06} & 
1.25 & 1.35\\CMT4X & 888.83 & 241.75 & 
891.51 & 242.74 & \bf{854.21} & 
4.05 & 4.37\\CMT4Y & 875.18 & 243.20 & 
881.96 & 242.04 & \bf{852.46} & 
2.67 & 3.46\\CMT5X & 1076.30 & 729.69 & 
1084.36 & 722.86 & \bf{1030.56} & 
4.44 & 5.22\\CMT5Y & 1081.46 & 716.50 & 
1083.43 & 760.62 & \bf{1031.69} & 
4.82 & 5.02\\CMT11X & 865.40 & 104.85 & 
875.25 & 103.21 & \bf{831.09} & 
4.13 & 5.31\\CMT11Y & 863.12 & 92.05 & 
873.37 & 92.59 & \bf{829.85} & 
4.01 & 5.24\\CMT12X & 677.01 & 48.09 & 
678.55 & 46.95 & \bf{658.83} & 
2.76 & 2.99\\CMT12Y & 669.13 & 44.57 & 
669.33 & 44.81 & \bf{660.47} & 
1.31 & 1.34\\\bf{PROM.} & 
\bf{772.05} & \bf{169.61} & \bf{775.23} & \bf{172.23} & \bf{749.50} & \bf{2.76} & \bf{3.14}\\[1ex]\hline
\end{tabular}
\label{table:nonlin}
\end{table} \clearpage
\begin{table}[ht]
\caption{Resultados de la ejecución de la metaheurística IGA, utilizando instancias de Dethloff con la configuración -n 200 -p 40 -cprob 10.0 -mprob 10.0}
\centering
\small
\begin{tabular}{c c c c c c c c}
\hline\hline
Instancia & Costo mínimo & Tiempo(seg.) & Costo promedio & Tiempo promedio(seg.) & CME & \%G & \%GP \\ [0.5ex]
\hline
SCA3-0 & 640.55 & 0.45 & 
641.40 & 0.43 & \bf{635.62} & 
0.78 & 0.91\\SCA3-1 & 701.78 & 0.55 & 
703.34 & 0.55 & \bf{697.84} & 
0.56 & 0.79\\SCA3-2 & \bf{659.34} & 0.40 & 
660.24 & 0.41 & 659.34 & 0.00
 & 0.14\\SCA3-3 & 685.47 & 0.41 & 
685.47 & 0.42 & \bf{680.04} & 
0.80 & 0.80\\SCA3-4 & \bf{690.50} & 0.40 & 
690.50 & 0.43 & 690.50 & 0.00
 & 0.00\\
SCA3-5 & 679.27 & 0.38 & 
681.16 & 0.39 & \bf{659.90} & 
2.94 & 3.22\\SCA3-6 & 652.94 & 0.44 & 
652.94 & 0.47 & \bf{651.09} & 
0.28 & 0.28\\SCA3-7 & 667.24 & 0.71 & 
667.24 & 0.58 & \bf{659.17} & 
1.22 & 1.22\\SCA3-8 & 726.57 & 0.56 & 
726.57 & 0.60 & \bf{719.47} & 
0.99 & 0.99\\SCA3-9 & \bf{681.00} & 0.44 & 
681.00 & 0.42 & 681.00 & 0.00
 & 0.00\\
SCA8-0 & 1007.19 & 0.49 & 
1007.19 & 0.45 & \bf{961.50} & 
4.75 & 4.75\\SCA8-1 & 1074.57 & 0.40 & 
1083.59 & 0.45 & \bf{1049.65} & 
2.37 & 3.23\\SCA8-2 & 1054.47 & 0.57 & 
1054.53 & 0.54 & \bf{1039.64} & 
1.43 & 1.43\\SCA8-3 & 1025.19 & 0.41 & 
1028.83 & 0.41 & \bf{983.34} & 
4.26 & 4.63\\SCA8-4 & 1080.08 & 0.41 & 
1094.22 & 0.39 & \bf{1065.49} & 
1.37 & 2.70\\SCA8-5 & 1057.68 & 0.39 & 
1058.36 & 0.47 & \bf{1027.08} & 
2.98 & 3.05\\SCA8-6 & 976.74 & 0.46 & 
976.74 & 0.45 & \bf{971.82} & 
0.51 & 0.51\\SCA8-7 & 1078.81 & 0.37 & 
1078.81 & 0.41 & \bf{1051.28} & 
2.62 & 2.62\\SCA8-8 & 1087.21 & 0.49 & 
1088.38 & 0.45 & \bf{1071.18} & 
1.50 & 1.61\\SCA8-9 & 1078.49 & 0.65 & 
1078.49 & 0.47 & \bf{1060.50} & 
1.70 & 1.70\\CON3-0 & 620.76 & 0.73 & 
620.76 & 0.51 & \bf{616.52} & 
0.69 & 0.69\\CON3-1 & 556.04 & 0.44 & 
562.04 & 0.46 & \bf{554.47} & 
0.28 & 1.37\\CON3-2 & 521.38 & 0.47 & 
521.38 & 0.48 & \bf{518.00} & 
0.65 & 0.65\\CON3-3 & 592.57 & 0.41 & 
592.76 & 0.41 & \bf{591.19} & 
0.23 & 0.27\\CON3-4 & 591.43 & 0.42 & 
594.82 & 0.41 & \bf{588.79} & 
0.45 & 1.02\\CON3-5 & 568.76 & 0.58 & 
569.60 & 0.48 & \bf{563.70} & 
0.90 & 1.05\\CON3-6 & 504.20 & 0.57 & 
507.52 & 0.57 & \bf{499.05} & 
1.03 & 1.70\\CON3-7 & \bf{576.48} & 0.55 & 
578.92 & 0.50 & 576.48 & 0.00
 & 0.42\\CON3-8 & \bf{523.05} & 0.50 & 
525.82 & 0.47 & 523.05 & 0.00
 & 0.53\\CON3-9 & 590.67 & 0.46 & 
591.01 & 0.46 & \bf{578.24} & 
2.15 & 2.21\\CON8-0 & 857.40 & 0.45 & 
878.45 & 0.46 & \bf{857.17} & 
0.03 & 2.48\\CON8-1 & 760.08 & 0.49 & 
760.08 & 0.46 & \bf{740.85} & 
2.60 & 2.60\\CON8-2 & 725.93 & 0.50 & 
731.18 & 0.50 & \bf{712.89} & 
1.83 & 2.57\\CON8-3 & 833.50 & 0.59 & 
836.37 & 0.47 & \bf{811.07} & 
2.77 & 3.12\\CON8-4 & 780.34 & 0.49 & 
780.34 & 0.48 & \bf{772.25} & 
1.05 & 1.05\\CON8-5 & 763.90 & 0.44 & 
763.90 & 0.42 & \bf{754.88} & 
1.19 & 1.19\\CON8-6 & 684.69 & 0.44 & 
705.55 & 0.46 & \bf{678.92} & 
0.85 & 3.92\\CON8-7 & 821.28 & 0.39 & 
821.86 & 0.48 & \bf{811.96} & 
1.15 & 1.22\\CON8-8 & 793.77 & 0.42 & 
797.03 & 0.41 & \bf{767.53} & 
3.42 & 3.84\\CON8-9 & 819.88 & 0.40 & 
819.88 & 0.46 & \bf{809.00} & 
1.34 & 1.34\\\bf{PROM.} & 
\bf{769.78} & \bf{0.48} & \bf{772.46} & \bf{0.46} & \bf{758.54} & \bf{1.34} & \bf{1.69}\\[1ex]\hline
\end{tabular}
\label{table:nonlin}
\end{table} \clearpage
\begin{table}[ht]
\caption{Resultados de la ejecución de la metaheurística IGA, utilizando instancias de SalhiNagy con la configuración -n 200 -p 40 -cprob 10.0 -mprob 10.0}
\centering
\small
\begin{tabular}{c c c c c c c c}
\hline\hline
Instancia & Costo mínimo & Tiempo(seg.) & Costo promedio & Tiempo promedio(seg.) & CME & \%G & \%GP \\ [0.5ex]
\hline
CMT1X & 480.02 & 0.40 & 
486.13 & 0.40 & \bf{470.48} & 
2.03 & 3.33\\CMT1Y & 477.06 & 0.42 & 
483.85 & 0.37 & \bf{470.48} & 
1.40 & 2.84\\CMT2X & 703.41 & 1.08 & 
708.40 & 1.04 & \bf{682.39} & 
3.08 & 3.81\\CMT2Y & 711.30 & 1.01 & 
712.00 & 1.01 & \bf{682.39} & 
4.24 & 4.34\\CMT3X & 742.27 & 2.66 & 
744.50 & 2.60 & \bf{719.06} & 
3.23 & 3.54\\CMT3Y & 740.27 & 2.36 & 
744.13 & 2.44 & \bf{719.06} & 
2.95 & 3.49\\CMT4X & 898.12 & 7.24 & 
914.44 & 7.42 & \bf{854.21} & 
5.14 & 7.05\\CMT4Y & 892.37 & 7.20 & 
913.15 & 7.29 & \bf{852.46} & 
4.68 & 7.12\\CMT5X & 1093.27 & 15.22 & 
1121.11 & 15.44 & \bf{1030.56} & 
6.09 & 8.79\\CMT5Y & 1116.16 & 16.28 & 
1132.67 & 15.86 & \bf{1031.69} & 
8.19 & 9.79\\CMT11X & 898.36 & 4.38 & 
911.32 & 4.49 & \bf{831.09} & 
8.09 & 9.65\\CMT11Y & 883.51 & 5.05 & 
907.64 & 5.14 & \bf{829.85} & 
6.47 & 9.37\\CMT12X & 674.53 & 2.78 & 
678.89 & 2.76 & \bf{658.83} & 
2.38 & 3.05\\CMT12Y & 680.71 & 2.36 & 
681.07 & 2.50 & \bf{660.47} & 
3.06 & 3.12\\\bf{PROM.} & 
\bf{785.10} & \bf{4.89} & \bf{795.66} & \bf{4.91} & \bf{749.50} & \bf{4.36} & \bf{5.66}\\[1ex]\hline
\end{tabular}
\label{table:nonlin}
\end{table} \clearpage
\begin{table}[ht]
\caption{Resultados de la ejecución de la metaheurística IGA, utilizando instancias de Dethloff con la configuración -n 200 -p 40 -cprob 10.0 -mprob 20.0}
\centering
\small
\begin{tabular}{c c c c c c c c}
\hline\hline
Instancia & Costo mínimo & Tiempo(seg.) & Costo promedio & Tiempo promedio(seg.) & CME & \%G & \%GP \\ [0.5ex]
\hline
SCA3-0 & 640.55 & 0.45 & 
640.55 & 0.53 & \bf{635.62} & 
0.78 & 0.78\\SCA3-1 & 701.53 & 0.41 & 
701.53 & 0.42 & \bf{697.84} & 
0.53 & 0.53\\SCA3-2 & 665.71 & 0.44 & 
669.38 & 0.41 & \bf{659.34} & 
0.97 & 1.52\\SCA3-3 & 685.47 & 0.41 & 
687.14 & 0.44 & \bf{680.04} & 
0.80 & 1.04\\SCA3-4 & \bf{690.50} & 0.46 & 
691.87 & 0.45 & 690.50 & 0.00
 & 0.20\\SCA3-5 & 680.80 & 0.41 & 
683.01 & 0.42 & \bf{659.90} & 
3.17 & 3.50\\SCA3-6 & 652.94 & 0.46 & 
652.94 & 0.53 & \bf{651.09} & 
0.28 & 0.28\\SCA3-7 & 666.15 & 0.57 & 
670.37 & 0.52 & \bf{659.17} & 
1.06 & 1.70\\SCA3-8 & 723.99 & 0.68 & 
724.14 & 0.51 & \bf{719.47} & 
0.63 & 0.65\\SCA3-9 & 684.44 & 0.42 & 
684.90 & 0.42 & \bf{681.00} & 
0.51 & 0.57\\SCA8-0 & 979.79 & 0.43 & 
979.79 & 0.50 & \bf{961.50} & 
1.90 & 1.90\\SCA8-1 & 1056.78 & 0.40 & 
1056.78 & 0.41 & \bf{1049.65} & 
0.68 & 0.68\\SCA8-2 & 1053.78 & 0.42 & 
1053.78 & 0.56 & \bf{1039.64} & 
1.36 & 1.36\\SCA8-3 & 1016.28 & 0.44 & 
1019.42 & 0.51 & \bf{983.34} & 
3.35 & 3.67\\SCA8-4 & 1086.37 & 0.40 & 
1086.37 & 0.50 & \bf{1065.49} & 
1.96 & 1.96\\SCA8-5 & 1065.82 & 0.65 & 
1065.82 & 0.52 & \bf{1027.08} & 
3.77 & 3.77\\SCA8-6 & 993.64 & 0.41 & 
993.64 & 0.43 & \bf{971.82} & 
2.25 & 2.25\\SCA8-7 & 1084.50 & 0.40 & 
1084.50 & 0.41 & \bf{1051.28} & 
3.16 & 3.16\\SCA8-8 & 1082.12 & 0.56 & 
1084.20 & 0.49 & \bf{1071.18} & 
1.02 & 1.22\\SCA8-9 & 1081.23 & 0.35 & 
1081.23 & 0.38 & \bf{1060.50} & 
1.95 & 1.95\\CON3-0 & 624.96 & 0.78 & 
628.43 & 0.63 & \bf{616.52} & 
1.37 & 1.93\\CON3-1 & 560.75 & 0.44 & 
560.97 & 0.53 & \bf{554.47} & 
1.13 & 1.17\\CON3-2 & 523.04 & 0.47 & 
523.56 & 0.47 & \bf{518.00} & 
0.97 & 1.07\\CON3-3 & 591.48 & 0.40 & 
598.20 & 0.42 & \bf{591.19} & 
0.05 & 1.19\\CON3-4 & 596.19 & 0.42 & 
596.26 & 0.43 & \bf{588.79} & 
1.26 & 1.27\\CON3-5 & 567.94 & 0.46 & 
568.35 & 0.45 & \bf{563.70} & 
0.75 & 0.82\\CON3-6 & 504.15 & 0.46 & 
506.93 & 0.45 & \bf{499.05} & 
1.02 & 1.58\\CON3-7 & 582.14 & 0.39 & 
582.14 & 0.39 & \bf{576.48} & 
0.98 & 0.98\\CON3-8 & \bf{523.05} & 0.42 & 
524.05 & 0.45 & 523.05 & 0.00
 & 0.19\\CON3-9 & 588.11 & 0.57 & 
588.11 & 0.52 & \bf{578.24} & 
1.71 & 1.71\\CON8-0 & 879.91 & 0.44 & 
888.54 & 0.59 & \bf{857.17} & 
2.65 & 3.66\\CON8-1 & 748.58 & 0.43 & 
752.99 & 0.46 & \bf{740.85} & 
1.04 & 1.64\\CON8-2 & 726.42 & 0.54 & 
728.78 & 0.47 & \bf{712.89} & 
1.90 & 2.23\\CON8-3 & 840.15 & 0.46 & 
846.06 & 0.45 & \bf{811.07} & 
3.59 & 4.31\\CON8-4 & 781.48 & 0.38 & 
781.48 & 0.36 & \bf{772.25} & 
1.20 & 1.20\\CON8-5 & 758.12 & 0.44 & 
761.22 & 0.51 & \bf{754.88} & 
0.43 & 0.84\\CON8-6 & 700.75 & 0.51 & 
703.27 & 0.56 & \bf{678.92} & 
3.22 & 3.59\\CON8-7 & 821.88 & 0.42 & 
821.88 & 0.40 & \bf{811.96} & 
1.22 & 1.22\\CON8-8 & 780.78 & 0.42 & 
790.49 & 0.51 & \bf{767.53} & 
1.73 & 2.99\\CON8-9 & 826.93 & 0.47 & 
826.93 & 0.56 & \bf{809.00} & 
2.22 & 2.22\\\bf{PROM.} & 
\bf{770.48} & \bf{0.46} & \bf{772.25} & \bf{0.47} & \bf{758.54} & \bf{1.46} & \bf{1.71}\\[1ex]\hline
\end{tabular}
\label{table:nonlin}
\end{table} \clearpage
\begin{table}[ht]
\caption{Resultados de la ejecución de la metaheurística IGA, utilizando instancias de SalhiNagy con la configuración -n 200 -p 40 -cprob 10.0 -mprob 20.0}
\centering
\small
\begin{tabular}{c c c c c c c c}
\hline\hline
Instancia & Costo mínimo & Tiempo(seg.) & Costo promedio & Tiempo promedio(seg.) & CME & \%G & \%GP \\ [0.5ex]
\hline
CMT1X & 487.15 & 0.41 & 
487.27 & 0.39 & \bf{470.48} & 
3.54 & 3.57\\CMT1Y & 480.39 & 0.39 & 
486.48 & 0.36 & \bf{470.48} & 
2.11 & 3.40\\CMT2X & 706.43 & 1.38 & 
709.83 & 1.10 & \bf{682.39} & 
3.52 & 4.02\\CMT2Y & 706.73 & 1.01 & 
714.15 & 0.98 & \bf{682.39} & 
3.57 & 4.65\\CMT3X & 736.53 & 2.54 & 
743.33 & 2.39 & \bf{719.06} & 
2.43 & 3.37\\CMT3Y & 744.89 & 2.47 & 
754.62 & 2.55 & \bf{719.06} & 
3.59 & 4.95\\CMT4X & 893.77 & 7.11 & 
907.22 & 7.22 & \bf{854.21} & 
4.63 & 6.21\\CMT4Y & 906.86 & 7.38 & 
915.10 & 7.31 & \bf{852.46} & 
6.38 & 7.35\\CMT5X & 1092.11 & 15.62 & 
1107.94 & 15.52 & \bf{1030.56} & 
5.97 & 7.51\\CMT5Y & 1113.31 & 15.42 & 
1123.89 & 15.85 & \bf{1031.69} & 
7.91 & 8.94\\CMT11X & 901.66 & 4.42 & 
919.70 & 4.55 & \bf{831.09} & 
8.49 & 10.66\\CMT11Y & 898.93 & 4.96 & 
908.66 & 4.80 & \bf{829.85} & 
8.32 & 9.50\\CMT12X & 677.07 & 2.56 & 
680.54 & 2.58 & \bf{658.83} & 
2.77 & 3.30\\CMT12Y & 675.30 & 2.43 & 
680.22 & 2.54 & \bf{660.47} & 
2.25 & 2.99\\\bf{PROM.} & 
\bf{787.22} & \bf{4.86} & \bf{795.64} & \bf{4.87} & \bf{749.50} & \bf{4.68} & \bf{5.74}\\[1ex]\hline
\end{tabular}
\label{table:nonlin}
\end{table} \clearpage
\begin{table}[ht]
\caption{Resultados de la ejecución de la metaheurística IGA, utilizando instancias de Dethloff con la configuración -n 200 -p 40 -cprob 10.0 -mprob 30.0}
\centering
\small
\begin{tabular}{c c c c c c c c}
\hline\hline
Instancia & Costo mínimo & Tiempo(seg.) & Costo promedio & Tiempo promedio(seg.) & CME & \%G & \%GP \\ [0.5ex]
\hline
SCA3-0 & 640.55 & 0.62 & 
640.55 & 0.49 & \bf{635.62} & 
0.78 & 0.78\\SCA3-1 & \bf{697.84} & 0.44 & 
699.81 & 0.41 & 697.84 & 0.00
 & 0.28\\SCA3-2 & 664.21 & 0.40 & 
670.24 & 0.48 & \bf{659.34} & 
0.74 & 1.65\\SCA3-3 & 681.16 & 0.44 & 
681.16 & 0.45 & \bf{680.04} & 
0.16 & 0.16\\SCA3-4 & \bf{690.50} & 0.43 & 
690.50 & 0.43 & 690.50 & 0.00
 & 0.00\\
SCA3-5 & 665.64 & 0.57 & 
669.55 & 0.54 & \bf{659.90} & 
0.87 & 1.46\\SCA3-6 & 656.17 & 0.46 & 
658.64 & 0.46 & \bf{651.09} & 
0.78 & 1.16\\SCA3-7 & 666.15 & 0.65 & 
666.15 & 0.46 & \bf{659.17} & 
1.06 & 1.06\\SCA3-8 & 722.05 & 0.42 & 
722.76 & 0.58 & \bf{719.47} & 
0.36 & 0.46\\SCA3-9 & \bf{681.00} & 0.43 & 
681.00 & 0.43 & 681.00 & 0.00
 & 0.00\\
SCA8-0 & 1005.18 & 0.42 & 
1005.18 & 0.46 & \bf{961.50} & 
4.54 & 4.54\\SCA8-1 & 1082.21 & 0.57 & 
1091.95 & 0.65 & \bf{1049.65} & 
3.10 & 4.03\\SCA8-2 & 1052.94 & 0.45 & 
1053.24 & 0.45 & \bf{1039.64} & 
1.28 & 1.31\\SCA8-3 & 1015.27 & 0.39 & 
1015.27 & 0.50 & \bf{983.34} & 
3.25 & 3.25\\SCA8-4 & 1083.00 & 0.37 & 
1083.00 & 0.44 & \bf{1065.49} & 
1.64 & 1.64\\SCA8-5 & 1051.28 & 0.44 & 
1051.28 & 0.54 & \bf{1027.08} & 
2.36 & 2.36\\SCA8-6 & 992.96 & 0.44 & 
993.56 & 0.39 & \bf{971.82} & 
2.18 & 2.24\\SCA8-7 & 1071.79 & 0.46 & 
1079.12 & 0.43 & \bf{1051.28} & 
1.95 & 2.65\\SCA8-8 & 1084.41 & 0.42 & 
1084.82 & 0.41 & \bf{1071.18} & 
1.24 & 1.27\\SCA8-9 & 1097.21 & 0.35 & 
1097.91 & 0.40 & \bf{1060.50} & 
3.46 & 3.53\\CON3-0 & 628.47 & 0.44 & 
628.47 & 0.58 & \bf{616.52} & 
1.94 & 1.94\\CON3-1 & 556.04 & 0.77 & 
560.74 & 0.54 & \bf{554.47} & 
0.28 & 1.13\\CON3-2 & 521.38 & 0.55 & 
522.89 & 0.60 & \bf{518.00} & 
0.65 & 0.94\\CON3-3 & 591.48 & 0.44 & 
597.77 & 0.56 & \bf{591.19} & 
0.05 & 1.11\\CON3-4 & 592.58 & 0.43 & 
594.19 & 0.52 & \bf{588.79} & 
0.64 & 0.92\\CON3-5 & 564.89 & 0.42 & 
564.89 & 0.43 & \bf{563.70} & 
0.21 & 0.21\\CON3-6 & 502.88 & 0.78 & 
505.44 & 0.61 & \bf{499.05} & 
0.77 & 1.28\\CON3-7 & 583.65 & 0.39 & 
587.20 & 0.41 & \bf{576.48} & 
1.24 & 1.86\\CON3-8 & \bf{523.05} & 0.47 & 
523.82 & 0.60 & 523.05 & 0.00
 & 0.15\\CON3-9 & 584.46 & 0.46 & 
585.94 & 0.51 & \bf{578.24} & 
1.08 & 1.33\\CON8-0 & 880.14 & 0.48 & 
883.68 & 0.56 & \bf{857.17} & 
2.68 & 3.09\\CON8-1 & 754.34 & 0.45 & 
755.43 & 0.44 & \bf{740.85} & 
1.82 & 1.97\\CON8-2 & 713.60 & 0.48 & 
714.02 & 0.47 & \bf{712.89} & 
0.10 & 0.16\\CON8-3 & 834.85 & 0.48 & 
834.85 & 0.49 & \bf{811.07} & 
2.93 & 2.93\\CON8-4 & \bf{772.25} & 0.66 & 
792.02 & 0.45 & 772.25 & 0.00
 & 2.56\\CON8-5 & 763.31 & 0.56 & 
764.55 & 0.52 & \bf{754.88} & 
1.12 & 1.28\\CON8-6 & 696.11 & 0.45 & 
697.95 & 0.53 & \bf{678.92} & 
2.53 & 2.80\\CON8-7 & 822.69 & 0.48 & 
826.73 & 0.43 & \bf{811.96} & 
1.32 & 1.82\\CON8-8 & 784.31 & 0.60 & 
790.38 & 0.52 & \bf{767.53} & 
2.19 & 2.98\\CON8-9 & 851.95 & 0.42 & 
855.23 & 0.45 & \bf{809.00} & 
5.31 & 5.71\\\bf{PROM.} & 
\bf{770.60} & \bf{0.48} & \bf{773.05} & \bf{0.49} & \bf{758.54} & \bf{1.42} & \bf{1.75}\\[1ex]\hline
\end{tabular}
\label{table:nonlin}
\end{table} \clearpage
\begin{table}[ht]
\caption{Resultados de la ejecución de la metaheurística IGA, utilizando instancias de SalhiNagy con la configuración -n 200 -p 40 -cprob 10.0 -mprob 30.0}
\centering
\small
\begin{tabular}{c c c c c c c c}
\hline\hline
Instancia & Costo mínimo & Tiempo(seg.) & Costo promedio & Tiempo promedio(seg.) & CME & \%G & \%GP \\ [0.5ex]
\hline
CMT1X & 486.42 & 0.62 & 
488.31 & 0.41 & \bf{470.48} & 
3.39 & 3.79\\CMT1Y & 478.97 & 0.48 & 
484.82 & 0.38 & \bf{470.48} & 
1.80 & 3.05\\CMT2X & 710.11 & 0.94 & 
716.47 & 1.04 & \bf{682.39} & 
4.06 & 4.99\\CMT2Y & 707.43 & 1.01 & 
721.32 & 1.07 & \bf{682.39} & 
3.67 & 5.71\\CMT3X & 731.16 & 2.38 & 
739.74 & 2.49 & \bf{719.06} & 
1.68 & 2.88\\CMT3Y & 734.51 & 2.44 & 
746.10 & 2.40 & \bf{719.06} & 
2.15 & 3.76\\CMT4X & 911.32 & 7.46 & 
922.55 & 7.33 & \bf{854.21} & 
6.69 & 8.00\\CMT4Y & 895.31 & 7.32 & 
903.49 & 7.38 & \bf{852.46} & 
5.03 & 5.99\\CMT5X & 1101.37 & 15.29 & 
1114.01 & 15.79 & \bf{1030.56} & 
6.87 & 8.10\\CMT5Y & 1088.49 & 15.78 & 
1121.53 & 15.87 & \bf{1031.69} & 
5.51 & 8.71\\CMT11X & 915.86 & 4.72 & 
925.46 & 4.68 & \bf{831.09} & 
10.20 & 11.36\\CMT11Y & 891.78 & 5.28 & 
915.11 & 5.10 & \bf{829.85} & 
7.46 & 10.27\\CMT12X & 674.84 & 2.52 & 
680.99 & 2.54 & \bf{658.83} & 
2.43 & 3.36\\CMT12Y & 675.27 & 2.33 & 
681.49 & 2.40 & \bf{660.47} & 
2.24 & 3.18\\\bf{PROM.} & 
\bf{785.92} & \bf{4.90} & \bf{797.24} & \bf{4.92} & \bf{749.50} & \bf{4.51} & \bf{5.94}\\[1ex]\hline
\end{tabular}
\label{table:nonlin}
\end{table} \clearpage
\begin{table}[ht]
\caption{Resultados de la ejecución de la metaheurística IGA, utilizando instancias de Dethloff con la configuración -n 200 -p 40 -cprob 10.0 -mprob 40.0}
\centering
\small
\begin{tabular}{c c c c c c c c}
\hline\hline
Instancia & Costo mínimo & Tiempo(seg.) & Costo promedio & Tiempo promedio(seg.) & CME & \%G & \%GP \\ [0.5ex]
\hline
SCA3-0 & 640.55 & 0.41 & 
642.52 & 0.43 & \bf{635.62} & 
0.78 & 1.09\\SCA3-1 & 701.53 & 0.46 & 
702.90 & 0.47 & \bf{697.84} & 
0.53 & 0.72\\SCA3-2 & 666.01 & 0.38 & 
666.77 & 0.39 & \bf{659.34} & 
1.01 & 1.13\\SCA3-3 & 682.47 & 0.43 & 
682.47 & 0.47 & \bf{680.04} & 
0.36 & 0.36\\SCA3-4 & \bf{690.50} & 0.40 & 
691.02 & 0.59 & 690.50 & 0.00
 & 0.08\\SCA3-5 & \bf{659.90} & 0.42 & 
659.90 & 0.45 & 659.90 & 0.00
 & 0.00\\
SCA3-6 & 652.94 & 0.49 & 
655.19 & 0.51 & \bf{651.09} & 
0.28 & 0.63\\SCA3-7 & 667.24 & 0.41 & 
667.24 & 0.45 & \bf{659.17} & 
1.22 & 1.22\\SCA3-8 & 724.29 & 0.46 & 
730.24 & 0.58 & \bf{719.47} & 
0.67 & 1.50\\SCA3-9 & \bf{681.00} & 0.42 & 
681.00 & 0.44 & 681.00 & 0.00
 & 0.00\\
SCA8-0 & 1003.65 & 0.39 & 
1003.65 & 0.53 & \bf{961.50} & 
4.38 & 4.38\\SCA8-1 & 1070.29 & 0.38 & 
1070.29 & 0.40 & \bf{1049.65} & 
1.97 & 1.97\\SCA8-2 & 1054.47 & 0.49 & 
1056.26 & 0.53 & \bf{1039.64} & 
1.43 & 1.60\\SCA8-3 & 1021.38 & 0.57 & 
1022.05 & 0.48 & \bf{983.34} & 
3.87 & 3.94\\SCA8-4 & 1096.04 & 0.38 & 
1109.22 & 0.44 & \bf{1065.49} & 
2.87 & 4.10\\SCA8-5 & 1059.41 & 0.67 & 
1061.05 & 0.48 & \bf{1027.08} & 
3.15 & 3.31\\SCA8-6 & 986.77 & 0.77 & 
986.77 & 0.52 & \bf{971.82} & 
1.54 & 1.54\\SCA8-7 & 1081.14 & 0.38 & 
1081.14 & 0.40 & \bf{1051.28} & 
2.84 & 2.84\\SCA8-8 & 1084.74 & 0.38 & 
1092.39 & 0.40 & \bf{1071.18} & 
1.27 & 1.98\\SCA8-9 & 1071.72 & 0.49 & 
1073.51 & 0.56 & \bf{1060.50} & 
1.06 & 1.23\\CON3-0 & 633.24 & 0.55 & 
633.24 & 0.55 & \bf{616.52} & 
2.71 & 2.71\\CON3-1 & 560.75 & 0.44 & 
561.31 & 0.44 & \bf{554.47} & 
1.13 & 1.23\\CON3-2 & 521.38 & 0.83 & 
523.34 & 0.61 & \bf{518.00} & 
0.65 & 1.03\\CON3-3 & 592.43 & 0.42 & 
602.65 & 0.47 & \bf{591.19} & 
0.21 & 1.94\\CON3-4 & 604.18 & 0.40 & 
604.67 & 0.49 & \bf{588.79} & 
2.61 & 2.70\\CON3-5 & \bf{563.70} & 0.43 & 
567.23 & 0.43 & 563.70 & 0.00
 & 0.63\\CON3-6 & 507.37 & 0.45 & 
508.66 & 0.46 & \bf{499.05} & 
1.67 & 1.93\\CON3-7 & 582.33 & 0.37 & 
584.59 & 0.46 & \bf{576.48} & 
1.01 & 1.41\\CON3-8 & 524.59 & 0.47 & 
529.91 & 0.53 & \bf{523.05} & 
0.29 & 1.31\\CON3-9 & 589.63 & 0.58 & 
590.77 & 0.54 & \bf{578.24} & 
1.97 & 2.17\\CON8-0 & 879.15 & 0.45 & 
879.15 & 0.47 & \bf{857.17} & 
2.56 & 2.56\\CON8-1 & 753.38 & 0.75 & 
761.66 & 0.60 & \bf{740.85} & 
1.69 & 2.81\\CON8-2 & 728.49 & 0.54 & 
731.63 & 0.55 & \bf{712.89} & 
2.19 & 2.63\\CON8-3 & 846.00 & 0.40 & 
846.00 & 0.44 & \bf{811.07} & 
4.31 & 4.31\\CON8-4 & 779.34 & 0.70 & 
782.54 & 0.48 & \bf{772.25} & 
0.92 & 1.33\\CON8-5 & 764.51 & 0.46 & 
765.17 & 0.60 & \bf{754.88} & 
1.28 & 1.36\\CON8-6 & 694.70 & 0.44 & 
694.70 & 0.51 & \bf{678.92} & 
2.32 & 2.32\\CON8-7 & 815.73 & 0.41 & 
823.89 & 0.54 & \bf{811.96} & 
0.46 & 1.47\\CON8-8 & 792.87 & 0.51 & 
792.87 & 0.66 & \bf{767.53} & 
3.30 & 3.30\\CON8-9 & 835.69 & 0.44 & 
842.71 & 0.46 & \bf{809.00} & 
3.30 & 4.17\\\bf{PROM.} & 
\bf{771.64} & \bf{0.48} & \bf{774.06} & \bf{0.50} & \bf{758.54} & \bf{1.60} & \bf{1.92}\\[1ex]\hline
\end{tabular}
\label{table:nonlin}
\end{table} \clearpage
\begin{table}[ht]
\caption{Resultados de la ejecución de la metaheurística IGA, utilizando instancias de SalhiNagy con la configuración -n 200 -p 40 -cprob 10.0 -mprob 40.0}
\centering
\small
\begin{tabular}{c c c c c c c c}
\hline\hline
Instancia & Costo mínimo & Tiempo(seg.) & Costo promedio & Tiempo promedio(seg.) & CME & \%G & \%GP \\ [0.5ex]
\hline
CMT1X & 484.85 & 0.39 & 
486.60 & 0.34 & \bf{470.48} & 
3.05 & 3.43\\CMT1Y & 489.33 & 0.38 & 
489.86 & 0.40 & \bf{470.48} & 
4.01 & 4.12\\CMT2X & 704.72 & 1.00 & 
714.52 & 0.98 & \bf{682.39} & 
3.27 & 4.71\\CMT2Y & 713.26 & 0.99 & 
716.61 & 0.97 & \bf{682.39} & 
4.52 & 5.02\\CMT3X & 731.70 & 2.55 & 
747.02 & 2.54 & \bf{719.06} & 
1.76 & 3.89\\CMT3Y & 739.09 & 3.02 & 
745.09 & 2.60 & \bf{719.06} & 
2.79 & 3.62\\CMT4X & 915.09 & 7.07 & 
921.12 & 7.25 & \bf{854.21} & 
7.13 & 7.83\\CMT4Y & 916.06 & 7.44 & 
921.33 & 7.40 & \bf{852.46} & 
7.46 & 8.08\\CMT5X & 1106.42 & 16.58 & 
1125.36 & 16.07 & \bf{1030.56} & 
7.36 & 9.20\\CMT5Y & 1124.73 & 16.56 & 
1131.04 & 15.63 & \bf{1031.69} & 
9.02 & 9.63\\CMT11X & 905.90 & 4.78 & 
910.31 & 4.46 & \bf{831.09} & 
9.00 & 9.53\\CMT11Y & 898.80 & 4.56 & 
904.42 & 4.87 & \bf{829.85} & 
8.31 & 8.99\\CMT12X & 675.38 & 2.56 & 
679.39 & 2.54 & \bf{658.83} & 
2.51 & 3.12\\CMT12Y & 674.93 & 2.94 & 
681.40 & 2.60 & \bf{660.47} & 
2.19 & 3.17\\\bf{PROM.} & 
\bf{791.45} & \bf{5.06} & \bf{798.15} & \bf{4.90} & \bf{749.50} & \bf{5.17} & \bf{6.02}\\[1ex]\hline
\end{tabular}
\label{table:nonlin}
\end{table} \clearpage
\begin{table}[ht]
\caption{Resultados de la ejecución de la metaheurística IGA, utilizando instancias de Dethloff con la configuración -n 200 -p 40 -cprob 10.0 -mprob 50.0}
\centering
\small
\begin{tabular}{c c c c c c c c}
\hline\hline
Instancia & Costo mínimo & Tiempo(seg.) & Costo promedio & Tiempo promedio(seg.) & CME & \%G & \%GP \\ [0.5ex]
\hline
SCA3-0 & 640.55 & 0.74 & 
640.55 & 0.57 & \bf{635.62} & 
0.78 & 0.78\\SCA3-1 & 701.86 & 0.42 & 
701.86 & 0.41 & \bf{697.84} & 
0.58 & 0.58\\SCA3-2 & 661.13 & 0.50 & 
666.59 & 0.43 & \bf{659.34} & 
0.27 & 1.10\\SCA3-3 & \bf{680.04} & 0.42 & 
680.32 & 0.52 & 680.04 & 0.00
 & 0.04\\SCA3-4 & \bf{690.50} & 0.39 & 
690.50 & 0.46 & 690.50 & 0.00
 & 0.00\\
SCA3-5 & 674.20 & 0.58 & 
676.28 & 0.54 & \bf{659.90} & 
2.17 & 2.48\\SCA3-6 & 652.94 & 0.52 & 
652.94 & 0.50 & \bf{651.09} & 
0.28 & 0.28\\SCA3-7 & 669.89 & 0.40 & 
669.89 & 0.41 & \bf{659.17} & 
1.63 & 1.63\\SCA3-8 & 727.73 & 0.45 & 
729.77 & 0.52 & \bf{719.47} & 
1.15 & 1.43\\SCA3-9 & 684.25 & 0.44 & 
684.25 & 0.41 & \bf{681.00} & 
0.48 & 0.48\\SCA8-0 & 970.64 & 0.44 & 
970.64 & 0.42 & \bf{961.50} & 
0.95 & 0.95\\SCA8-1 & 1083.59 & 0.45 & 
1083.59 & 0.45 & \bf{1049.65} & 
3.23 & 3.23\\SCA8-2 & 1055.32 & 0.36 & 
1055.32 & 0.49 & \bf{1039.64} & 
1.51 & 1.51\\SCA8-3 & 1002.56 & 0.43 & 
1002.56 & 0.56 & \bf{983.34} & 
1.95 & 1.95\\SCA8-4 & 1087.97 & 0.42 & 
1095.58 & 0.39 & \bf{1065.49} & 
2.11 & 2.82\\SCA8-5 & 1029.95 & 0.43 & 
1030.35 & 0.45 & \bf{1027.08} & 
0.28 & 0.32\\SCA8-6 & 998.10 & 0.48 & 
998.10 & 0.44 & \bf{971.82} & 
2.70 & 2.70\\SCA8-7 & 1067.03 & 0.45 & 
1067.03 & 0.48 & \bf{1051.28} & 
1.50 & 1.50\\SCA8-8 & 1082.91 & 0.47 & 
1086.25 & 0.53 & \bf{1071.18} & 
1.10 & 1.41\\SCA8-9 & 1068.10 & 0.36 & 
1068.10 & 0.39 & \bf{1060.50} & 
0.72 & 0.72\\CON3-0 & \bf{616.52} & 0.42 & 
616.52 & 0.45 & 616.52 & 0.00
 & 0.00\\
CON3-1 & 560.75 & 0.39 & 
562.11 & 0.41 & \bf{554.47} & 
1.13 & 1.38\\CON3-2 & 521.38 & 0.55 & 
521.38 & 0.51 & \bf{518.00} & 
0.65 & 0.65\\CON3-3 & 591.20 & 0.41 & 
600.86 & 0.46 & \bf{591.19} & 
0.00 & 1.64\\CON3-4 & 598.21 & 0.50 & 
598.43 & 0.51 & \bf{588.79} & 
1.60 & 1.64\\CON3-5 & 564.89 & 0.51 & 
566.96 & 0.52 & \bf{563.70} & 
0.21 & 0.58\\CON3-6 & 504.44 & 0.67 & 
508.20 & 0.61 & \bf{499.05} & 
1.08 & 1.83\\CON3-7 & 578.41 & 0.44 & 
578.41 & 0.43 & \bf{576.48} & 
0.33 & 0.33\\CON3-8 & 524.59 & 0.43 & 
524.59 & 0.44 & \bf{523.05} & 
0.29 & 0.29\\CON3-9 & 588.55 & 0.45 & 
589.75 & 0.46 & \bf{578.24} & 
1.78 & 1.99\\CON8-0 & 871.29 & 0.46 & 
874.26 & 0.53 & \bf{857.17} & 
1.65 & 1.99\\CON8-1 & 761.13 & 0.54 & 
761.13 & 0.47 & \bf{740.85} & 
2.74 & 2.74\\CON8-2 & 733.90 & 0.53 & 
733.90 & 0.49 & \bf{712.89} & 
2.95 & 2.95\\CON8-3 & 836.92 & 0.48 & 
836.92 & 0.43 & \bf{811.07} & 
3.19 & 3.19\\CON8-4 & 772.83 & 0.41 & 
772.83 & 0.42 & \bf{772.25} & 
0.08 & 0.08\\CON8-5 & 762.01 & 0.41 & 
762.01 & 0.43 & \bf{754.88} & 
0.94 & 0.94\\CON8-6 & 701.47 & 0.47 & 
702.36 & 0.52 & \bf{678.92} & 
3.32 & 3.45\\CON8-7 & 821.21 & 0.40 & 
821.21 & 0.42 & \bf{811.96} & 
1.14 & 1.14\\CON8-8 & 777.75 & 0.43 & 
780.61 & 0.49 & \bf{767.53} & 
1.33 & 1.70\\CON8-9 & 831.40 & 0.48 & 
831.46 & 0.57 & \bf{809.00} & 
2.77 & 2.78\\\bf{PROM.} & 
\bf{768.70} & \bf{0.46} & \bf{769.86} & \bf{0.47} & \bf{758.54} & \bf{1.26} & \bf{1.43}\\[1ex]\hline
\end{tabular}
\label{table:nonlin}
\end{table} \clearpage
\begin{table}[ht]
\caption{Resultados de la ejecución de la metaheurística IGA, utilizando instancias de SalhiNagy con la configuración -n 200 -p 40 -cprob 10.0 -mprob 50.0}
\centering
\small
\begin{tabular}{c c c c c c c c}
\hline\hline
Instancia & Costo mínimo & Tiempo(seg.) & Costo promedio & Tiempo promedio(seg.) & CME & \%G & \%GP \\ [0.5ex]
\hline
CMT1X & 478.84 & 0.39 & 
482.27 & 0.36 & \bf{470.48} & 
1.78 & 2.51\\CMT1Y & 484.09 & 0.29 & 
485.06 & 0.38 & \bf{470.48} & 
2.89 & 3.10\\CMT2X & 701.76 & 0.99 & 
710.70 & 0.98 & \bf{682.39} & 
2.84 & 4.15\\CMT2Y & 703.92 & 1.22 & 
711.13 & 1.07 & \bf{682.39} & 
3.16 & 4.21\\CMT3X & 745.45 & 2.61 & 
749.60 & 2.69 & \bf{719.06} & 
3.67 & 4.25\\CMT3Y & 737.54 & 2.62 & 
744.72 & 2.48 & \bf{719.06} & 
2.57 & 3.57\\CMT4X & 883.31 & 7.39 & 
909.93 & 7.68 & \bf{854.21} & 
3.41 & 6.52\\CMT4Y & 919.71 & 7.12 & 
925.72 & 7.36 & \bf{852.46} & 
7.89 & 8.59\\CMT5X & 1104.80 & 16.08 & 
1118.89 & 15.41 & \bf{1030.56} & 
7.20 & 8.57\\CMT5Y & 1113.68 & 15.62 & 
1128.47 & 15.98 & \bf{1031.69} & 
7.95 & 9.38\\CMT11X & 902.33 & 4.66 & 
915.41 & 4.69 & \bf{831.09} & 
8.57 & 10.15\\CMT11Y & 894.31 & 4.69 & 
903.04 & 4.88 & \bf{829.85} & 
7.77 & 8.82\\CMT12X & 683.25 & 3.43 & 
686.61 & 2.86 & \bf{658.83} & 
3.71 & 4.22\\CMT12Y & 676.86 & 3.28 & 
683.41 & 2.94 & \bf{660.47} & 
2.48 & 3.47\\\bf{PROM.} & 
\bf{787.85} & \bf{5.03} & \bf{796.78} & \bf{4.98} & \bf{749.50} & \bf{4.71} & \bf{5.82}\\[1ex]\hline
\end{tabular}
\label{table:nonlin}
\end{table} \clearpage
\begin{table}[ht]
\caption{Resultados de la ejecución de la metaheurística IGA, utilizando instancias de Dethloff con la configuración -n 200 -p 40 -cprob 10.0 -mprob 60.0}
\centering
\small
\begin{tabular}{c c c c c c c c}
\hline\hline
Instancia & Costo mínimo & Tiempo(seg.) & Costo promedio & Tiempo promedio(seg.) & CME & \%G & \%GP \\ [0.5ex]
\hline
SCA3-0 & 640.55 & 0.59 & 
641.12 & 0.52 & \bf{635.62} & 
0.78 & 0.87\\SCA3-1 & 706.90 & 0.41 & 
707.03 & 0.47 & \bf{697.84} & 
1.30 & 1.32\\SCA3-2 & \bf{659.34} & 0.39 & 
665.66 & 0.40 & 659.34 & 0.00
 & 0.96\\SCA3-3 & 685.47 & 0.64 & 
685.47 & 0.46 & \bf{680.04} & 
0.80 & 0.80\\SCA3-4 & \bf{690.50} & 0.37 & 
690.50 & 0.43 & 690.50 & 0.00
 & 0.00\\
SCA3-5 & 665.64 & 0.40 & 
676.98 & 0.47 & \bf{659.90} & 
0.87 & 2.59\\SCA3-6 & 652.94 & 0.56 & 
654.52 & 0.49 & \bf{651.09} & 
0.28 & 0.53\\SCA3-7 & 666.15 & 0.43 & 
666.26 & 0.45 & \bf{659.17} & 
1.06 & 1.08\\SCA3-8 & 726.44 & 0.48 & 
726.44 & 0.43 & \bf{719.47} & 
0.97 & 0.97\\SCA3-9 & \bf{681.00} & 0.56 & 
681.00 & 0.62 & 681.00 & 0.00
 & 0.00\\
SCA8-0 & 975.50 & 0.43 & 
975.50 & 0.45 & \bf{961.50} & 
1.46 & 1.46\\SCA8-1 & 1064.15 & 0.51 & 
1064.15 & 0.51 & \bf{1049.65} & 
1.38 & 1.38\\SCA8-2 & 1053.78 & 0.40 & 
1053.78 & 0.43 & \bf{1039.64} & 
1.36 & 1.36\\SCA8-3 & 1020.78 & 0.40 & 
1022.87 & 0.50 & \bf{983.34} & 
3.81 & 4.02\\SCA8-4 & 1090.34 & 0.45 & 
1090.34 & 0.45 & \bf{1065.49} & 
2.33 & 2.33\\SCA8-5 & 1050.79 & 0.44 & 
1052.80 & 0.45 & \bf{1027.08} & 
2.31 & 2.50\\SCA8-6 & 976.74 & 0.40 & 
977.83 & 0.57 & \bf{971.82} & 
0.51 & 0.62\\SCA8-7 & 1067.49 & 0.43 & 
1067.49 & 0.47 & \bf{1051.28} & 
1.54 & 1.54\\SCA8-8 & 1100.26 & 0.42 & 
1101.64 & 0.55 & \bf{1071.18} & 
2.71 & 2.84\\SCA8-9 & 1075.22 & 0.38 & 
1075.22 & 0.41 & \bf{1060.50} & 
1.39 & 1.39\\CON3-0 & 623.84 & 0.45 & 
625.63 & 0.49 & \bf{616.52} & 
1.19 & 1.48\\CON3-1 & 560.75 & 0.47 & 
561.63 & 0.50 & \bf{554.47} & 
1.13 & 1.29\\CON3-2 & 522.86 & 0.50 & 
522.86 & 0.55 & \bf{518.00} & 
0.94 & 0.94\\CON3-3 & 594.31 & 0.50 & 
605.68 & 0.45 & \bf{591.19} & 
0.53 & 2.45\\CON3-4 & 592.58 & 0.48 & 
593.91 & 0.56 & \bf{588.79} & 
0.64 & 0.87\\CON3-5 & 567.63 & 0.44 & 
567.63 & 0.47 & \bf{563.70} & 
0.70 & 0.70\\CON3-6 & 505.01 & 0.44 & 
505.67 & 0.46 & \bf{499.05} & 
1.19 & 1.33\\CON3-7 & 577.68 & 0.60 & 
577.68 & 0.50 & \bf{576.48} & 
0.21 & 0.21\\CON3-8 & \bf{523.05} & 0.45 & 
525.32 & 0.45 & 523.05 & 0.00
 & 0.43\\CON3-9 & 589.73 & 0.53 & 
589.73 & 0.48 & \bf{578.24} & 
1.99 & 1.99\\CON8-0 & 870.22 & 0.58 & 
870.22 & 0.54 & \bf{857.17} & 
1.52 & 1.52\\CON8-1 & 745.98 & 0.46 & 
745.98 & 0.54 & \bf{740.85} & 
0.69 & 0.69\\CON8-2 & 724.13 & 0.52 & 
728.04 & 0.63 & \bf{712.89} & 
1.58 & 2.13\\CON8-3 & 831.34 & 0.47 & 
831.34 & 0.53 & \bf{811.07} & 
2.50 & 2.50\\CON8-4 & 803.41 & 0.53 & 
807.68 & 0.46 & \bf{772.25} & 
4.03 & 4.59\\CON8-5 & 770.27 & 0.68 & 
775.93 & 0.63 & \bf{754.88} & 
2.04 & 2.79\\CON8-6 & 689.56 & 0.43 & 
691.77 & 0.58 & \bf{678.92} & 
1.57 & 1.89\\CON8-7 & 815.54 & 0.41 & 
815.54 & 0.40 & \bf{811.96} & 
0.44 & 0.44\\CON8-8 & 778.25 & 0.88 & 
778.25 & 0.71 & \bf{767.53} & 
1.40 & 1.40\\CON8-9 & 819.67 & 0.46 & 
834.02 & 0.52 & \bf{809.00} & 
1.32 & 3.09\\\bf{PROM.} & 
\bf{768.89} & \bf{0.48} & \bf{770.78} & \bf{0.50} & \bf{758.54} & \bf{1.26} & \bf{1.53}\\[1ex]\hline
\end{tabular}
\label{table:nonlin}
\end{table} \clearpage
\begin{table}[ht]
\caption{Resultados de la ejecución de la metaheurística IGA, utilizando instancias de SalhiNagy con la configuración -n 200 -p 40 -cprob 10.0 -mprob 60.0}
\centering
\small
\begin{tabular}{c c c c c c c c}
\hline\hline
Instancia & Costo mínimo & Tiempo(seg.) & Costo promedio & Tiempo promedio(seg.) & CME & \%G & \%GP \\ [0.5ex]
\hline
CMT1X & 481.31 & 0.45 & 
486.95 & 0.40 & \bf{470.48} & 
2.30 & 3.50\\CMT1Y & 481.19 & 0.31 & 
482.93 & 0.31 & \bf{470.48} & 
2.28 & 2.65\\CMT2X & 719.23 & 0.98 & 
719.83 & 0.97 & \bf{682.39} & 
5.40 & 5.49\\CMT2Y & 708.37 & 0.96 & 
714.02 & 1.06 & \bf{682.39} & 
3.81 & 4.63\\CMT3X & 734.94 & 2.57 & 
747.96 & 2.69 & \bf{719.06} & 
2.21 & 4.02\\CMT3Y & 737.31 & 2.47 & 
742.92 & 2.40 & \bf{719.06} & 
2.54 & 3.32\\CMT4X & 897.34 & 6.80 & 
916.75 & 7.24 & \bf{854.21} & 
5.05 & 7.32\\CMT4Y & 919.86 & 7.72 & 
925.17 & 7.44 & \bf{852.46} & 
7.91 & 8.53\\CMT5X & 1098.63 & 15.53 & 
1109.68 & 14.97 & \bf{1030.56} & 
6.61 & 7.68\\CMT5Y & 1116.89 & 15.58 & 
1132.56 & 15.36 & \bf{1031.69} & 
8.26 & 9.78\\CMT11X & 903.80 & 5.04 & 
920.52 & 4.79 & \bf{831.09} & 
8.75 & 10.76\\CMT11Y & 904.29 & 5.13 & 
910.87 & 5.21 & \bf{829.85} & 
8.97 & 9.76\\CMT12X & 680.80 & 2.70 & 
684.20 & 2.64 & \bf{658.83} & 
3.33 & 3.85\\CMT12Y & 678.60 & 2.63 & 
682.38 & 2.75 & \bf{660.47} & 
2.75 & 3.32\\\bf{PROM.} & 
\bf{790.18} & \bf{4.92} & \bf{798.34} & \bf{4.88} & \bf{749.50} & \bf{5.01} & \bf{6.04}\\[1ex]\hline
\end{tabular}
\label{table:nonlin}
\end{table} \clearpage
\begin{table}[ht]
\caption{Resultados de la ejecución de la metaheurística IGA, utilizando instancias de Dethloff con la configuración -n 200 -p 40 -cprob 10.0 -mprob 70.0}
\centering
\small
\begin{tabular}{c c c c c c c c}
\hline\hline
Instancia & Costo mínimo & Tiempo(seg.) & Costo promedio & Tiempo promedio(seg.) & CME & \%G & \%GP \\ [0.5ex]
\hline
SCA3-0 & 640.55 & 0.66 & 
640.84 & 0.51 & \bf{635.62} & 
0.78 & 0.82\\SCA3-1 & 701.53 & 0.52 & 
701.53 & 0.48 & \bf{697.84} & 
0.53 & 0.53\\SCA3-2 & 666.19 & 0.41 & 
669.90 & 0.46 & \bf{659.34} & 
1.04 & 1.60\\SCA3-3 & 681.16 & 0.42 & 
681.21 & 0.55 & \bf{680.04} & 
0.16 & 0.17\\SCA3-4 & \bf{690.50} & 0.39 & 
690.50 & 0.42 & 690.50 & 0.00
 & 0.00\\
SCA3-5 & 677.56 & 0.45 & 
679.27 & 0.45 & \bf{659.90} & 
2.68 & 2.94\\SCA3-6 & 652.94 & 0.44 & 
654.00 & 0.47 & \bf{651.09} & 
0.28 & 0.45\\SCA3-7 & 666.15 & 0.50 & 
666.38 & 0.45 & \bf{659.17} & 
1.06 & 1.09\\SCA3-8 & 724.29 & 0.92 & 
727.71 & 0.53 & \bf{719.47} & 
0.67 & 1.14\\SCA3-9 & \bf{681.00} & 0.41 & 
682.03 & 0.41 & 681.00 & 0.00
 & 0.15\\SCA8-0 & 987.26 & 0.45 & 
987.26 & 0.57 & \bf{961.50} & 
2.68 & 2.68\\SCA8-1 & 1095.15 & 0.41 & 
1095.15 & 0.40 & \bf{1049.65} & 
4.33 & 4.33\\SCA8-2 & 1053.59 & 0.40 & 
1053.59 & 0.40 & \bf{1039.64} & 
1.34 & 1.34\\SCA8-3 & 1014.10 & 0.49 & 
1014.10 & 0.47 & \bf{983.34} & 
3.13 & 3.13\\SCA8-4 & 1074.18 & 0.39 & 
1074.18 & 0.42 & \bf{1065.49} & 
0.82 & 0.82\\SCA8-5 & 1038.26 & 0.66 & 
1053.77 & 0.55 & \bf{1027.08} & 
1.09 & 2.60\\SCA8-6 & 982.65 & 0.38 & 
985.85 & 0.52 & \bf{971.82} & 
1.11 & 1.44\\SCA8-7 & 1084.88 & 0.37 & 
1084.88 & 0.51 & \bf{1051.28} & 
3.20 & 3.20\\SCA8-8 & 1089.91 & 0.44 & 
1089.91 & 0.53 & \bf{1071.18} & 
1.75 & 1.75\\SCA8-9 & 1067.27 & 0.38 & 
1067.27 & 0.44 & \bf{1060.50} & 
0.64 & 0.64\\CON3-0 & 628.47 & 0.46 & 
631.54 & 0.50 & \bf{616.52} & 
1.94 & 2.44\\CON3-1 & 558.67 & 0.46 & 
560.23 & 0.45 & \bf{554.47} & 
0.76 & 1.04\\CON3-2 & 521.38 & 0.63 & 
522.30 & 0.53 & \bf{518.00} & 
0.65 & 0.83\\CON3-3 & 591.20 & 0.41 & 
597.07 & 0.43 & \bf{591.19} & 
0.00 & 0.99\\CON3-4 & 592.58 & 0.40 & 
597.47 & 0.46 & \bf{588.79} & 
0.64 & 1.47\\CON3-5 & \bf{563.70} & 0.44 & 
567.85 & 0.45 & 563.70 & 0.00
 & 0.74\\CON3-6 & 502.16 & 0.48 & 
504.76 & 0.47 & \bf{499.05} & 
0.62 & 1.14\\CON3-7 & 578.41 & 0.41 & 
588.70 & 0.40 & \bf{576.48} & 
0.33 & 2.12\\CON3-8 & 525.30 & 0.57 & 
532.36 & 0.48 & \bf{523.05} & 
0.43 & 1.78\\CON3-9 & 588.40 & 0.49 & 
589.07 & 0.46 & \bf{578.24} & 
1.76 & 1.87\\CON8-0 & 891.49 & 0.58 & 
891.49 & 0.49 & \bf{857.17} & 
4.00 & 4.00\\CON8-1 & 760.30 & 0.46 & 
767.24 & 0.55 & \bf{740.85} & 
2.63 & 3.56\\CON8-2 & 718.80 & 0.56 & 
722.15 & 0.56 & \bf{712.89} & 
0.83 & 1.30\\CON8-3 & 832.93 & 0.46 & 
833.39 & 0.53 & \bf{811.07} & 
2.70 & 2.75\\CON8-4 & 787.50 & 0.38 & 
792.90 & 0.39 & \bf{772.25} & 
1.97 & 2.67\\CON8-5 & 763.13 & 0.47 & 
763.13 & 0.47 & \bf{754.88} & 
1.09 & 1.09\\CON8-6 & 700.80 & 0.62 & 
700.84 & 0.52 & \bf{678.92} & 
3.22 & 3.23\\CON8-7 & 816.18 & 0.77 & 
826.36 & 0.49 & \bf{811.96} & 
0.52 & 1.77\\CON8-8 & 787.58 & 0.49 & 
787.58 & 0.48 & \bf{767.53} & 
2.61 & 2.61\\CON8-9 & 833.49 & 0.42 & 
838.12 & 0.47 & \bf{809.00} & 
3.03 & 3.60\\\bf{PROM.} & 
\bf{770.29} & \bf{0.49} & \bf{772.85} & \bf{0.48} & \bf{758.54} & \bf{1.43} & \bf{1.80}\\[1ex]\hline
\end{tabular}
\label{table:nonlin}
\end{table} \clearpage
\begin{table}[ht]
\caption{Resultados de la ejecución de la metaheurística IGA, utilizando instancias de SalhiNagy con la configuración -n 200 -p 40 -cprob 10.0 -mprob 70.0}
\centering
\small
\begin{tabular}{c c c c c c c c}
\hline\hline
Instancia & Costo mínimo & Tiempo(seg.) & Costo promedio & Tiempo promedio(seg.) & CME & \%G & \%GP \\ [0.5ex]
\hline
CMT1X & 482.53 & 0.39 & 
483.85 & 0.40 & \bf{470.48} & 
2.56 & 2.84\\CMT1Y & 479.65 & 0.34 & 
481.20 & 0.39 & \bf{470.48} & 
1.95 & 2.28\\CMT2X & 712.85 & 0.95 & 
717.94 & 1.03 & \bf{682.39} & 
4.46 & 5.21\\CMT2Y & 720.06 & 1.00 & 
723.66 & 1.17 & \bf{682.39} & 
5.52 & 6.05\\CMT3X & 743.40 & 2.56 & 
744.52 & 2.54 & \bf{719.06} & 
3.38 & 3.54\\CMT3Y & 746.95 & 2.39 & 
751.00 & 2.45 & \bf{719.06} & 
3.88 & 4.44\\CMT4X & 910.89 & 7.59 & 
914.77 & 7.48 & \bf{854.21} & 
6.64 & 7.09\\CMT4Y & 896.62 & 7.79 & 
907.25 & 7.68 & \bf{852.46} & 
5.18 & 6.43\\CMT5X & 1110.94 & 16.15 & 
1117.30 & 15.46 & \bf{1030.56} & 
7.80 & 8.42\\CMT5Y & 1105.72 & 15.40 & 
1121.09 & 15.76 & \bf{1031.69} & 
7.18 & 8.67\\CMT11X & 905.00 & 4.48 & 
912.13 & 4.67 & \bf{831.09} & 
8.89 & 9.75\\CMT11Y & 869.06 & 4.86 & 
887.79 & 4.91 & \bf{829.85} & 
4.72 & 6.98\\CMT12X & 674.30 & 2.63 & 
680.43 & 2.69 & \bf{658.83} & 
2.35 & 3.28\\CMT12Y & 674.94 & 2.44 & 
675.72 & 2.63 & \bf{660.47} & 
2.19 & 2.31\\\bf{PROM.} & 
\bf{788.06} & \bf{4.93} & \bf{794.19} & \bf{4.95} & \bf{749.50} & \bf{4.76} & \bf{5.52}\\[1ex]\hline
\end{tabular}
\label{table:nonlin}
\end{table} \clearpage
\begin{table}[ht]
\caption{Resultados de la ejecución de la metaheurística IGA, utilizando instancias de Dethloff con la configuración -n 200 -p 40 -cprob 10.0 -mprob 80.0}
\centering
\small
\begin{tabular}{c c c c c c c c}
\hline\hline
Instancia & Costo mínimo & Tiempo(seg.) & Costo promedio & Tiempo promedio(seg.) & CME & \%G & \%GP \\ [0.5ex]
\hline
SCA3-0 & 641.69 & 0.42 & 
641.69 & 0.43 & \bf{635.62} & 
0.95 & 0.95\\SCA3-1 & 708.01 & 0.44 & 
708.01 & 0.68 & \bf{697.84} & 
1.46 & 1.46\\SCA3-2 & 669.60 & 0.45 & 
671.63 & 0.42 & \bf{659.34} & 
1.56 & 1.86\\SCA3-3 & 682.47 & 0.41 & 
682.47 & 0.43 & \bf{680.04} & 
0.36 & 0.36\\SCA3-4 & \bf{690.50} & 0.44 & 
690.50 & 0.49 & 690.50 & 0.00
 & 0.00\\
SCA3-5 & 680.66 & 0.53 & 
681.62 & 0.47 & \bf{659.90} & 
3.15 & 3.29\\SCA3-6 & 652.94 & 0.42 & 
654.20 & 0.46 & \bf{651.09} & 
0.28 & 0.48\\SCA3-7 & 669.89 & 0.45 & 
671.30 & 0.45 & \bf{659.17} & 
1.63 & 1.84\\SCA3-8 & 724.29 & 0.39 & 
724.29 & 0.44 & \bf{719.47} & 
0.67 & 0.67\\SCA3-9 & 685.88 & 0.53 & 
685.88 & 0.45 & \bf{681.00} & 
0.72 & 0.72\\SCA8-0 & 1017.29 & 0.68 & 
1017.29 & 0.53 & \bf{961.50} & 
5.80 & 5.80\\SCA8-1 & 1073.98 & 0.68 & 
1076.45 & 0.48 & \bf{1049.65} & 
2.32 & 2.55\\SCA8-2 & 1051.80 & 0.47 & 
1051.80 & 0.59 & \bf{1039.64} & 
1.17 & 1.17\\SCA8-3 & 1025.40 & 0.45 & 
1025.40 & 0.59 & \bf{983.34} & 
4.28 & 4.28\\SCA8-4 & 1074.87 & 0.48 & 
1074.87 & 0.45 & \bf{1065.49} & 
0.88 & 0.88\\SCA8-5 & 1056.83 & 0.38 & 
1058.87 & 0.40 & \bf{1027.08} & 
2.90 & 3.10\\SCA8-6 & 989.31 & 0.58 & 
989.31 & 0.54 & \bf{971.82} & 
1.80 & 1.80\\SCA8-7 & 1074.99 & 0.42 & 
1079.50 & 0.43 & \bf{1051.28} & 
2.26 & 2.68\\SCA8-8 & 1087.21 & 0.41 & 
1087.21 & 0.41 & \bf{1071.18} & 
1.50 & 1.50\\SCA8-9 & 1086.61 & 0.56 & 
1087.04 & 0.59 & \bf{1060.50} & 
2.46 & 2.50\\CON3-0 & 623.97 & 0.50 & 
623.97 & 0.50 & \bf{616.52} & 
1.21 & 1.21\\CON3-1 & 560.75 & 0.46 & 
560.75 & 0.58 & \bf{554.47} & 
1.13 & 1.13\\CON3-2 & 521.38 & 0.54 & 
521.38 & 0.58 & \bf{518.00} & 
0.65 & 0.65\\CON3-3 & 598.27 & 0.47 & 
598.52 & 0.47 & \bf{591.19} & 
1.20 & 1.24\\CON3-4 & 605.10 & 0.56 & 
605.10 & 0.47 & \bf{588.79} & 
2.77 & 2.77\\CON3-5 & \bf{563.70} & 0.44 & 
566.00 & 0.49 & 563.70 & 0.00
 & 0.41\\CON3-6 & 504.15 & 0.44 & 
504.62 & 0.49 & \bf{499.05} & 
1.02 & 1.12\\CON3-7 & 582.14 & 0.39 & 
584.49 & 0.46 & \bf{576.48} & 
0.98 & 1.39\\CON3-8 & 524.59 & 0.56 & 
527.22 & 0.52 & \bf{523.05} & 
0.29 & 0.80\\CON3-9 & 590.67 & 0.43 & 
591.55 & 0.44 & \bf{578.24} & 
2.15 & 2.30\\CON8-0 & 870.73 & 0.45 & 
870.73 & 0.47 & \bf{857.17} & 
1.58 & 1.58\\CON8-1 & 767.28 & 0.60 & 
767.91 & 0.51 & \bf{740.85} & 
3.57 & 3.65\\CON8-2 & 717.31 & 0.50 & 
717.43 & 0.56 & \bf{712.89} & 
0.62 & 0.64\\CON8-3 & 836.25 & 0.53 & 
836.25 & 0.48 & \bf{811.07} & 
3.10 & 3.10\\CON8-4 & 794.28 & 0.54 & 
794.28 & 0.46 & \bf{772.25} & 
2.85 & 2.85\\CON8-5 & 766.29 & 0.44 & 
766.29 & 0.56 & \bf{754.88} & 
1.51 & 1.51\\CON8-6 & 700.05 & 0.56 & 
702.16 & 0.47 & \bf{678.92} & 
3.11 & 3.42\\CON8-7 & 835.73 & 0.37 & 
837.54 & 0.39 & \bf{811.96} & 
2.93 & 3.15\\CON8-8 & 785.45 & 0.48 & 
794.38 & 0.47 & \bf{767.53} & 
2.33 & 3.50\\CON8-9 & 814.77 & 0.44 & 
830.36 & 0.48 & \bf{809.00} & 
0.71 & 2.64\\\bf{PROM.} & 
\bf{772.68} & \bf{0.48} & \bf{774.01} & \bf{0.49} & \bf{758.54} & \bf{1.75} & \bf{1.92}\\[1ex]\hline
\end{tabular}
\label{table:nonlin}
\end{table} \clearpage
\begin{table}[ht]
\caption{Resultados de la ejecución de la metaheurística IGA, utilizando instancias de SalhiNagy con la configuración -n 200 -p 40 -cprob 10.0 -mprob 80.0}
\centering
\small
\begin{tabular}{c c c c c c c c}
\hline\hline
Instancia & Costo mínimo & Tiempo(seg.) & Costo promedio & Tiempo promedio(seg.) & CME & \%G & \%GP \\ [0.5ex]
\hline
CMT1X & 485.35 & 0.35 & 
486.65 & 0.34 & \bf{470.48} & 
3.16 & 3.44\\CMT1Y & 485.15 & 0.38 & 
487.95 & 0.48 & \bf{470.48} & 
3.12 & 3.71\\CMT2X & 700.40 & 0.99 & 
715.30 & 1.15 & \bf{682.39} & 
2.64 & 4.82\\CMT2Y & 707.09 & 1.02 & 
714.00 & 0.99 & \bf{682.39} & 
3.62 & 4.63\\CMT3X & 739.24 & 2.60 & 
744.72 & 2.62 & \bf{719.06} & 
2.81 & 3.57\\CMT3Y & 741.87 & 2.47 & 
745.85 & 2.57 & \bf{719.06} & 
3.17 & 3.73\\CMT4X & 882.59 & 7.84 & 
905.89 & 7.45 & \bf{854.21} & 
3.32 & 6.05\\CMT4Y & 897.58 & 7.92 & 
916.26 & 7.95 & \bf{852.46} & 
5.29 & 7.48\\CMT5X & 1103.02 & 16.62 & 
1122.26 & 16.21 & \bf{1030.56} & 
7.03 & 8.90\\CMT5Y & 1079.10 & 16.51 & 
1114.41 & 15.40 & \bf{1031.69} & 
4.60 & 8.02\\CMT11X & 909.14 & 4.15 & 
922.55 & 4.33 & \bf{831.09} & 
9.39 & 11.00\\CMT11Y & 894.75 & 4.90 & 
910.24 & 5.14 & \bf{829.85} & 
7.82 & 9.69\\CMT12X & 674.65 & 2.84 & 
677.65 & 2.60 & \bf{658.83} & 
2.40 & 2.86\\CMT12Y & 674.98 & 2.82 & 
678.18 & 2.73 & \bf{660.47} & 
2.20 & 2.68\\\bf{PROM.} & 
\bf{783.92} & \bf{5.10} & \bf{795.85} & \bf{5.00} & \bf{749.50} & \bf{4.33} & \bf{5.76}\\[1ex]\hline
\end{tabular}
\label{table:nonlin}
\end{table} \clearpage
\begin{table}[ht]
\caption{Resultados de la ejecución de la metaheurística IGA, utilizando instancias de Dethloff con la configuración -n 200 -p 40 -cprob 10.0 -mprob 90.0}
\centering
\small
\begin{tabular}{c c c c c c c c}
\hline\hline
Instancia & Costo mínimo & Tiempo(seg.) & Costo promedio & Tiempo promedio(seg.) & CME & \%G & \%GP \\ [0.5ex]
\hline
SCA3-0 & 640.55 & 0.54 & 
640.84 & 0.56 & \bf{635.62} & 
0.78 & 0.82\\SCA3-1 & \bf{697.84} & 0.46 & 
697.84 & 0.47 & 697.84 & 0.00
 & 0.00\\
SCA3-2 & 668.65 & 0.45 & 
668.75 & 0.44 & \bf{659.34} & 
1.41 & 1.43\\SCA3-3 & 681.16 & 0.91 & 
681.49 & 0.58 & \bf{680.04} & 
0.16 & 0.21\\SCA3-4 & \bf{690.50} & 0.48 & 
690.50 & 0.44 & 690.50 & 0.00
 & 0.00\\
SCA3-5 & 666.67 & 0.78 & 
666.67 & 0.62 & \bf{659.90} & 
1.03 & 1.03\\SCA3-6 & 652.94 & 0.42 & 
652.94 & 0.42 & \bf{651.09} & 
0.28 & 0.28\\SCA3-7 & 666.15 & 0.46 & 
666.15 & 0.43 & \bf{659.17} & 
1.06 & 1.06\\SCA3-8 & 726.58 & 0.44 & 
726.88 & 0.43 & \bf{719.47} & 
0.99 & 1.03\\SCA3-9 & \bf{681.00} & 0.45 & 
681.00 & 0.53 & 681.00 & 0.00
 & 0.00\\
SCA8-0 & 1002.18 & 0.51 & 
1002.18 & 0.60 & \bf{961.50} & 
4.23 & 4.23\\SCA8-1 & 1072.81 & 0.40 & 
1072.81 & 0.45 & \bf{1049.65} & 
2.21 & 2.21\\SCA8-2 & 1050.37 & 0.46 & 
1050.37 & 0.44 & \bf{1039.64} & 
1.03 & 1.03\\SCA8-3 & 1019.89 & 0.50 & 
1025.42 & 0.53 & \bf{983.34} & 
3.72 & 4.28\\SCA8-4 & 1075.02 & 0.57 & 
1075.02 & 0.53 & \bf{1065.49} & 
0.89 & 0.89\\SCA8-5 & 1059.89 & 0.40 & 
1059.89 & 0.55 & \bf{1027.08} & 
3.19 & 3.19\\SCA8-6 & 972.48 & 0.42 & 
972.48 & 0.46 & \bf{971.82} & 
0.07 & 0.07\\SCA8-7 & 1077.82 & 0.45 & 
1077.82 & 0.54 & \bf{1051.28} & 
2.52 & 2.52\\SCA8-8 & 1085.34 & 0.51 & 
1085.34 & 0.55 & \bf{1071.18} & 
1.32 & 1.32\\SCA8-9 & 1067.27 & 0.59 & 
1081.79 & 0.57 & \bf{1060.50} & 
0.64 & 2.01\\CON3-0 & 631.60 & 0.47 & 
632.86 & 0.47 & \bf{616.52} & 
2.45 & 2.65\\CON3-1 & 560.75 & 0.43 & 
561.31 & 0.50 & \bf{554.47} & 
1.13 & 1.23\\CON3-2 & 521.38 & 0.57 & 
521.38 & 0.56 & \bf{518.00} & 
0.65 & 0.65\\CON3-3 & 591.20 & 0.51 & 
594.81 & 0.51 & \bf{591.19} & 
0.00 & 0.61\\CON3-4 & 592.58 & 0.52 & 
593.18 & 0.50 & \bf{588.79} & 
0.64 & 0.75\\CON3-5 & 567.94 & 0.43 & 
568.35 & 0.48 & \bf{563.70} & 
0.75 & 0.82\\CON3-6 & 504.91 & 0.50 & 
508.58 & 0.56 & \bf{499.05} & 
1.17 & 1.91\\CON3-7 & 577.54 & 0.45 & 
580.22 & 0.53 & \bf{576.48} & 
0.18 & 0.65\\CON3-8 & 524.59 & 0.46 & 
526.09 & 0.50 & \bf{523.05} & 
0.29 & 0.58\\CON3-9 & 588.48 & 0.48 & 
588.48 & 0.48 & \bf{578.24} & 
1.77 & 1.77\\CON8-0 & 881.68 & 0.49 & 
881.68 & 0.49 & \bf{857.17} & 
2.86 & 2.86\\CON8-1 & 763.20 & 0.44 & 
767.04 & 0.47 & \bf{740.85} & 
3.02 & 3.53\\CON8-2 & 725.43 & 0.51 & 
725.43 & 0.52 & \bf{712.89} & 
1.76 & 1.76\\CON8-3 & 828.14 & 0.50 & 
831.85 & 0.46 & \bf{811.07} & 
2.10 & 2.56\\CON8-4 & 792.16 & 0.40 & 
792.16 & 0.49 & \bf{772.25} & 
2.58 & 2.58\\CON8-5 & 769.99 & 0.76 & 
769.99 & 0.71 & \bf{754.88} & 
2.00 & 2.00\\CON8-6 & 700.68 & 0.68 & 
700.73 & 0.50 & \bf{678.92} & 
3.21 & 3.21\\CON8-7 & 822.15 & 0.42 & 
825.64 & 0.46 & \bf{811.96} & 
1.25 & 1.68\\CON8-8 & 791.19 & 0.52 & 
791.91 & 0.49 & \bf{767.53} & 
3.08 & 3.18\\CON8-9 & 821.81 & 0.48 & 
821.81 & 0.67 & \bf{809.00} & 
1.58 & 1.58\\\bf{PROM.} & 
\bf{770.31} & \bf{0.51} & \bf{771.49} & \bf{0.51} & \bf{758.54} & \bf{1.45} & \bf{1.60}\\[1ex]\hline
\end{tabular}
\label{table:nonlin}
\end{table} \clearpage
\begin{table}[ht]
\caption{Resultados de la ejecución de la metaheurística IGA, utilizando instancias de SalhiNagy con la configuración -n 200 -p 40 -cprob 10.0 -mprob 90.0}
\centering
\small
\begin{tabular}{c c c c c c c c}
\hline\hline
Instancia & Costo mínimo & Tiempo(seg.) & Costo promedio & Tiempo promedio(seg.) & CME & \%G & \%GP \\ [0.5ex]
\hline
CMT1X & 482.25 & 0.60 & 
483.91 & 0.47 & \bf{470.48} & 
2.50 & 2.86\\CMT1Y & 479.45 & 0.38 & 
482.04 & 0.38 & \bf{470.48} & 
1.91 & 2.46\\CMT2X & 708.71 & 1.18 & 
719.07 & 1.12 & \bf{682.39} & 
3.86 & 5.37\\CMT2Y & 707.30 & 0.94 & 
716.08 & 1.09 & \bf{682.39} & 
3.65 & 4.94\\CMT3X & 736.42 & 2.56 & 
740.04 & 2.56 & \bf{719.06} & 
2.41 & 2.92\\CMT3Y & 741.84 & 2.45 & 
745.73 & 2.67 & \bf{719.06} & 
3.17 & 3.71\\CMT4X & 894.11 & 7.47 & 
904.24 & 7.50 & \bf{854.21} & 
4.67 & 5.86\\CMT4Y & 907.65 & 7.60 & 
927.33 & 7.30 & \bf{852.46} & 
6.47 & 8.78\\CMT5X & 1087.51 & 15.60 & 
1107.03 & 16.01 & \bf{1030.56} & 
5.53 & 7.42\\CMT5Y & 1105.43 & 16.54 & 
1127.08 & 16.28 & \bf{1031.69} & 
7.15 & 9.25\\CMT11X & 889.74 & 4.80 & 
906.13 & 4.79 & \bf{831.09} & 
7.06 & 9.03\\CMT11Y & 850.48 & 4.68 & 
890.00 & 4.74 & \bf{829.85} & 
2.49 & 7.25\\CMT12X & 675.75 & 2.45 & 
681.71 & 2.63 & \bf{658.83} & 
2.57 & 3.47\\CMT12Y & 676.60 & 3.14 & 
680.66 & 2.64 & \bf{660.47} & 
2.44 & 3.06\\\bf{PROM.} & 
\bf{781.66} & \bf{5.03} & \bf{793.65} & \bf{5.01} & \bf{749.50} & \bf{3.99} & \bf{5.45}\\[1ex]\hline
\end{tabular}
\label{table:nonlin}
\end{table} \clearpage
\begin{table}[ht]
\caption{Resultados de la ejecución de la metaheurística IGA, utilizando instancias de Dethloff con la configuración -n 200 -p 40 -cprob 10.0 -mprob 100.0}
\centering
\small
\begin{tabular}{c c c c c c c c}
\hline\hline
Instancia & Costo mínimo & Tiempo(seg.) & Costo promedio & Tiempo promedio(seg.) & CME & \%G & \%GP \\ [0.5ex]
\hline
SCA3-0 & 640.55 & 0.54 & 
640.55 & 0.48 & \bf{635.62} & 
0.78 & 0.78\\SCA3-1 & 701.53 & 0.93 & 
701.53 & 0.61 & \bf{697.84} & 
0.53 & 0.53\\SCA3-2 & 668.65 & 0.42 & 
668.86 & 0.43 & \bf{659.34} & 
1.41 & 1.44\\SCA3-3 & 681.35 & 0.44 & 
685.11 & 0.43 & \bf{680.04} & 
0.19 & 0.75\\SCA3-4 & \bf{690.50} & 0.46 & 
690.50 & 0.43 & 690.50 & 0.00
 & 0.00\\
SCA3-5 & 668.48 & 0.57 & 
669.70 & 0.53 & \bf{659.90} & 
1.30 & 1.48\\SCA3-6 & 660.26 & 0.90 & 
660.26 & 0.67 & \bf{651.09} & 
1.41 & 1.41\\SCA3-7 & 671.77 & 0.45 & 
672.00 & 0.46 & \bf{659.17} & 
1.91 & 1.95\\SCA3-8 & 719.77 & 0.48 & 
725.70 & 0.50 & \bf{719.47} & 
0.04 & 0.87\\SCA3-9 & \bf{681.00} & 0.46 & 
681.00 & 0.43 & 681.00 & 0.00
 & 0.00\\
SCA8-0 & 976.98 & 0.41 & 
986.80 & 0.52 & \bf{961.50} & 
1.61 & 2.63\\SCA8-1 & 1078.78 & 0.47 & 
1081.80 & 0.44 & \bf{1049.65} & 
2.78 & 3.06\\SCA8-2 & 1054.85 & 0.41 & 
1054.85 & 0.60 & \bf{1039.64} & 
1.46 & 1.46\\SCA8-3 & 1024.73 & 0.44 & 
1024.87 & 0.43 & \bf{983.34} & 
4.21 & 4.22\\SCA8-4 & 1092.53 & 0.38 & 
1092.53 & 0.47 & \bf{1065.49} & 
2.54 & 2.54\\SCA8-5 & 1060.65 & 0.41 & 
1060.65 & 0.45 & \bf{1027.08} & 
3.27 & 3.27\\SCA8-6 & 981.24 & 0.40 & 
983.65 & 0.47 & \bf{971.82} & 
0.97 & 1.22\\SCA8-7 & 1075.21 & 0.47 & 
1075.21 & 0.47 & \bf{1051.28} & 
2.28 & 2.28\\SCA8-8 & 1090.39 & 0.44 & 
1090.39 & 0.54 & \bf{1071.18} & 
1.79 & 1.79\\SCA8-9 & 1081.16 & 0.38 & 
1081.16 & 0.41 & \bf{1060.50} & 
1.95 & 1.95\\CON3-0 & 624.96 & 0.45 & 
631.68 & 0.44 & \bf{616.52} & 
1.37 & 2.46\\CON3-1 & 560.75 & 0.48 & 
562.73 & 0.52 & \bf{554.47} & 
1.13 & 1.49\\CON3-2 & 521.38 & 0.50 & 
521.38 & 0.51 & \bf{518.00} & 
0.65 & 0.65\\CON3-3 & 599.26 & 0.66 & 
599.46 & 0.51 & \bf{591.19} & 
1.37 & 1.40\\CON3-4 & 591.43 & 0.50 & 
593.56 & 0.56 & \bf{588.79} & 
0.45 & 0.81\\CON3-5 & 567.94 & 0.54 & 
567.94 & 0.48 & \bf{563.70} & 
0.75 & 0.75\\CON3-6 & 502.26 & 0.46 & 
504.83 & 0.51 & \bf{499.05} & 
0.64 & 1.16\\CON3-7 & 582.14 & 0.40 & 
586.23 & 0.41 & \bf{576.48} & 
0.98 & 1.69\\CON3-8 & 526.59 & 0.52 & 
532.61 & 0.51 & \bf{523.05} & 
0.68 & 1.83\\CON3-9 & 588.18 & 0.48 & 
588.84 & 0.59 & \bf{578.24} & 
1.72 & 1.83\\CON8-0 & 873.62 & 0.43 & 
884.50 & 0.45 & \bf{857.17} & 
1.92 & 3.19\\CON8-1 & 752.70 & 0.73 & 
768.00 & 0.61 & \bf{740.85} & 
1.60 & 3.66\\CON8-2 & 713.44 & 0.49 & 
715.34 & 0.62 & \bf{712.89} & 
0.08 & 0.34\\CON8-3 & 830.96 & 0.44 & 
830.96 & 0.47 & \bf{811.07} & 
2.45 & 2.45\\CON8-4 & 797.16 & 0.44 & 
797.16 & 0.41 & \bf{772.25} & 
3.23 & 3.23\\CON8-5 & 766.01 & 1.00 & 
768.51 & 0.60 & \bf{754.88} & 
1.47 & 1.81\\CON8-6 & 705.16 & 0.47 & 
705.16 & 0.62 & \bf{678.92} & 
3.86 & 3.86\\CON8-7 & 821.91 & 0.47 & 
822.22 & 0.48 & \bf{811.96} & 
1.23 & 1.26\\CON8-8 & 802.70 & 0.46 & 
803.24 & 0.52 & \bf{767.53} & 
4.58 & 4.65\\CON8-9 & 822.33 & 0.42 & 
823.31 & 0.49 & \bf{809.00} & 
1.65 & 1.77\\\bf{PROM.} & 
\bf{771.28} & \bf{0.51} & \bf{773.37} & \bf{0.50} & \bf{758.54} & \bf{1.56} & \bf{1.85}\\[1ex]\hline
\end{tabular}
\label{table:nonlin}
\end{table} \clearpage
\begin{table}[ht]
\caption{Resultados de la ejecución de la metaheurística IGA, utilizando instancias de SalhiNagy con la configuración -n 200 -p 40 -cprob 10.0 -mprob 100.0}
\centering
\small
\begin{tabular}{c c c c c c c c}
\hline\hline
Instancia & Costo mínimo & Tiempo(seg.) & Costo promedio & Tiempo promedio(seg.) & CME & \%G & \%GP \\ [0.5ex]
\hline
CMT1X & 483.83 & 0.38 & 
486.00 & 0.48 & \bf{470.48} & 
2.84 & 3.30\\CMT1Y & 488.48 & 0.30 & 
490.30 & 0.33 & \bf{470.48} & 
3.83 & 4.21\\CMT2X & 707.74 & 0.90 & 
711.26 & 1.03 & \bf{682.39} & 
3.71 & 4.23\\CMT2Y & 706.50 & 0.92 & 
708.19 & 1.04 & \bf{682.39} & 
3.53 & 3.78\\CMT3X & 737.63 & 2.36 & 
745.83 & 2.46 & \bf{719.06} & 
2.58 & 3.72\\CMT3Y & 733.14 & 2.28 & 
741.17 & 2.40 & \bf{719.06} & 
1.96 & 3.07\\CMT4X & 902.13 & 7.55 & 
910.31 & 7.14 & \bf{854.21} & 
5.61 & 6.57\\CMT4Y & 912.23 & 7.19 & 
918.72 & 7.25 & \bf{852.46} & 
7.01 & 7.77\\CMT5X & 1101.55 & 15.33 & 
1117.89 & 15.26 & \bf{1030.56} & 
6.89 & 8.47\\CMT5Y & 1116.18 & 15.18 & 
1123.25 & 15.72 & \bf{1031.69} & 
8.19 & 8.87\\CMT11X & 885.85 & 4.66 & 
905.14 & 4.60 & \bf{831.09} & 
6.59 & 8.91\\CMT11Y & 849.52 & 5.10 & 
897.51 & 4.98 & \bf{829.85} & 
2.37 & 8.15\\CMT12X & 673.99 & 2.72 & 
679.58 & 2.63 & \bf{658.83} & 
2.30 & 3.15\\CMT12Y & 677.04 & 2.55 & 
679.02 & 2.70 & \bf{660.47} & 
2.51 & 2.81\\\bf{PROM.} & 
\bf{783.99} & \bf{4.82} & \bf{793.87} & \bf{4.86} & \bf{749.50} & \bf{4.28} & \bf{5.50}\\[1ex]\hline
\end{tabular}
\label{table:nonlin}
\end{table} \clearpage
\begin{table}[ht]
\caption{Resultados de la ejecución de la metaheurística IGA, utilizando instancias de Dethloff con la configuración -n 200 -p 40 -cprob 20.0 -mprob 10.0}
\centering
\small
\begin{tabular}{c c c c c c c c}
\hline\hline
Instancia & Costo mínimo & Tiempo(seg.) & Costo promedio & Tiempo promedio(seg.) & CME & \%G & \%GP \\ [0.5ex]
\hline
SCA3-0 & 640.55 & 0.47 & 
640.84 & 0.49 & \bf{635.62} & 
0.78 & 0.82\\SCA3-1 & \bf{697.84} & 0.73 & 
699.49 & 0.59 & 697.84 & 0.00
 & 0.24\\SCA3-2 & 669.06 & 0.53 & 
669.98 & 0.45 & \bf{659.34} & 
1.47 & 1.61\\SCA3-3 & \bf{680.04} & 0.59 & 
682.88 & 0.58 & 680.04 & 0.00
 & 0.42\\SCA3-4 & \bf{690.50} & 0.40 & 
690.50 & 0.43 & 690.50 & 0.00
 & 0.00\\
SCA3-5 & 683.05 & 0.46 & 
683.05 & 0.50 & \bf{659.90} & 
3.51 & 3.51\\SCA3-6 & 653.68 & 0.66 & 
656.15 & 0.53 & \bf{651.09} & 
0.40 & 0.78\\SCA3-7 & 666.15 & 0.50 & 
666.15 & 0.48 & \bf{659.17} & 
1.06 & 1.06\\SCA3-8 & 726.44 & 0.45 & 
728.05 & 0.45 & \bf{719.47} & 
0.97 & 1.19\\SCA3-9 & 681.68 & 0.42 & 
682.95 & 0.47 & \bf{681.00} & 
0.10 & 0.29\\SCA8-0 & 985.83 & 0.57 & 
999.36 & 0.54 & \bf{961.50} & 
2.53 & 3.94\\SCA8-1 & 1081.67 & 0.48 & 
1081.67 & 0.47 & \bf{1049.65} & 
3.05 & 3.05\\SCA8-2 & 1052.56 & 0.39 & 
1053.70 & 0.41 & \bf{1039.64} & 
1.24 & 1.35\\SCA8-3 & 1018.02 & 0.48 & 
1018.02 & 0.44 & \bf{983.34} & 
3.53 & 3.53\\SCA8-4 & 1089.91 & 0.38 & 
1089.91 & 0.40 & \bf{1065.49} & 
2.29 & 2.29\\SCA8-5 & 1038.93 & 0.41 & 
1038.93 & 0.50 & \bf{1027.08} & 
1.15 & 1.15\\SCA8-6 & 987.65 & 0.55 & 
987.65 & 0.61 & \bf{971.82} & 
1.63 & 1.63\\SCA8-7 & 1075.99 & 0.39 & 
1076.08 & 0.47 & \bf{1051.28} & 
2.35 & 2.36\\SCA8-8 & 1086.27 & 0.71 & 
1086.47 & 0.70 & \bf{1071.18} & 
1.41 & 1.43\\SCA8-9 & 1070.71 & 0.42 & 
1078.39 & 0.47 & \bf{1060.50} & 
0.96 & 1.69\\CON3-0 & 621.09 & 0.70 & 
625.33 & 0.49 & \bf{616.52} & 
0.74 & 1.43\\CON3-1 & 559.00 & 0.45 & 
559.44 & 0.51 & \bf{554.47} & 
0.82 & 0.90\\CON3-2 & 521.38 & 0.55 & 
521.38 & 0.59 & \bf{518.00} & 
0.65 & 0.65\\CON3-3 & 594.31 & 0.49 & 
597.38 & 0.49 & \bf{591.19} & 
0.53 & 1.05\\CON3-4 & 595.25 & 0.54 & 
597.47 & 0.64 & \bf{588.79} & 
1.10 & 1.47\\CON3-5 & 566.19 & 0.56 & 
567.50 & 0.57 & \bf{563.70} & 
0.44 & 0.67\\CON3-6 & 506.32 & 0.76 & 
506.85 & 0.56 & \bf{499.05} & 
1.46 & 1.56\\CON3-7 & 581.15 & 0.46 & 
584.17 & 0.45 & \bf{576.48} & 
0.81 & 1.33\\CON3-8 & 528.09 & 0.59 & 
534.25 & 0.55 & \bf{523.05} & 
0.96 & 2.14\\CON3-9 & 588.11 & 0.75 & 
588.11 & 0.64 & \bf{578.24} & 
1.71 & 1.71\\CON8-0 & 874.06 & 0.48 & 
876.72 & 0.68 & \bf{857.17} & 
1.97 & 2.28\\CON8-1 & 765.99 & 0.52 & 
765.99 & 0.58 & \bf{740.85} & 
3.39 & 3.39\\CON8-2 & 719.25 & 0.56 & 
719.56 & 0.70 & \bf{712.89} & 
0.89 & 0.94\\CON8-3 & 822.73 & 0.43 & 
833.77 & 0.50 & \bf{811.07} & 
1.44 & 2.80\\CON8-4 & 797.75 & 0.49 & 
797.75 & 0.47 & \bf{772.25} & 
3.30 & 3.30\\CON8-5 & 766.44 & 0.55 & 
777.12 & 0.53 & \bf{754.88} & 
1.53 & 2.95\\CON8-6 & 696.10 & 0.59 & 
696.10 & 0.51 & \bf{678.92} & 
2.53 & 2.53\\CON8-7 & 814.86 & 0.46 & 
814.86 & 0.54 & \bf{811.96} & 
0.36 & 0.36\\CON8-8 & 791.72 & 0.45 & 
793.50 & 0.53 & \bf{767.53} & 
3.15 & 3.38\\CON8-9 & 820.24 & 0.42 & 
820.24 & 0.49 & \bf{809.00} & 
1.39 & 1.39\\\bf{PROM.} & 
\bf{770.16} & \bf{0.52} & \bf{772.19} & \bf{0.52} & \bf{758.54} & \bf{1.44} & \bf{1.71}\\[1ex]\hline
\end{tabular}
\label{table:nonlin}
\end{table} \clearpage
\begin{table}[ht]
\caption{Resultados de la ejecución de la metaheurística IGA, utilizando instancias de SalhiNagy con la configuración -n 200 -p 40 -cprob 20.0 -mprob 10.0}
\centering
\small
\begin{tabular}{c c c c c c c c}
\hline\hline
Instancia & Costo mínimo & Tiempo(seg.) & Costo promedio & Tiempo promedio(seg.) & CME & \%G & \%GP \\ [0.5ex]
\hline
CMT1X & 481.91 & 0.34 & 
482.96 & 0.41 & \bf{470.48} & 
2.43 & 2.65\\CMT1Y & 479.21 & 0.47 & 
479.21 & 0.40 & \bf{470.48} & 
1.86 & 1.86\\CMT2X & 712.12 & 1.16 & 
716.40 & 1.01 & \bf{682.39} & 
4.36 & 4.98\\CMT2Y & 708.52 & 1.25 & 
722.02 & 1.09 & \bf{682.39} & 
3.83 & 5.81\\CMT3X & 746.77 & 2.60 & 
750.30 & 2.52 & \bf{719.06} & 
3.85 & 4.35\\CMT3Y & 741.57 & 2.46 & 
744.25 & 2.52 & \bf{719.06} & 
3.13 & 3.50\\CMT4X & 899.22 & 7.48 & 
913.15 & 7.20 & \bf{854.21} & 
5.27 & 6.90\\CMT4Y & 913.17 & 7.31 & 
920.05 & 7.76 & \bf{852.46} & 
7.12 & 7.93\\CMT5X & 1126.25 & 15.53 & 
1130.14 & 15.25 & \bf{1030.56} & 
9.29 & 9.66\\CMT5Y & 1125.86 & 19.14 & 
1137.13 & 16.23 & \bf{1031.69} & 
9.13 & 10.22\\CMT11X & 912.23 & 4.82 & 
919.70 & 4.65 & \bf{831.09} & 
9.76 & 10.66\\CMT11Y & 888.14 & 5.55 & 
900.21 & 5.38 & \bf{829.85} & 
7.02 & 8.48\\CMT12X & 675.31 & 2.87 & 
684.72 & 2.69 & \bf{658.83} & 
2.50 & 3.93\\CMT12Y & 676.58 & 2.58 & 
680.57 & 2.66 & \bf{660.47} & 
2.44 & 3.04\\\bf{PROM.} & 
\bf{791.92} & \bf{5.25} & \bf{798.63} & \bf{4.98} & \bf{749.50} & \bf{5.14} & \bf{6.00}\\[1ex]\hline
\end{tabular}
\label{table:nonlin}
\end{table} \clearpage
\begin{table}[ht]
\caption{Resultados de la ejecución de la metaheurística IGA, utilizando instancias de Dethloff con la configuración -n 200 -p 40 -cprob 20.0 -mprob 20.0}
\centering
\small
\begin{tabular}{c c c c c c c c}
\hline\hline
Instancia & Costo mínimo & Tiempo(seg.) & Costo promedio & Tiempo promedio(seg.) & CME & \%G & \%GP \\ [0.5ex]
\hline
SCA3-0 & 640.55 & 0.66 & 
641.20 & 0.51 & \bf{635.62} & 
0.78 & 0.88\\SCA3-1 & \bf{697.84} & 0.47 & 
700.15 & 0.49 & 697.84 & 0.00
 & 0.33\\SCA3-2 & 661.13 & 0.41 & 
661.13 & 0.56 & \bf{659.34} & 
0.27 & 0.27\\SCA3-3 & 681.16 & 0.42 & 
681.49 & 0.57 & \bf{680.04} & 
0.16 & 0.21\\SCA3-4 & \bf{690.50} & 0.70 & 
690.50 & 0.59 & 690.50 & 0.00
 & 0.00\\
SCA3-5 & 665.64 & 0.51 & 
665.64 & 0.52 & \bf{659.90} & 
0.87 & 0.87\\SCA3-6 & 656.87 & 0.43 & 
658.55 & 0.55 & \bf{651.09} & 
0.89 & 1.15\\SCA3-7 & 671.15 & 0.68 & 
671.25 & 0.55 & \bf{659.17} & 
1.82 & 1.83\\SCA3-8 & 726.44 & 0.47 & 
727.64 & 0.53 & \bf{719.47} & 
0.97 & 1.14\\SCA3-9 & \bf{681.00} & 0.40 & 
681.00 & 0.42 & 681.00 & 0.00
 & 0.00\\
SCA8-0 & 970.64 & 0.46 & 
970.64 & 0.74 & \bf{961.50} & 
0.95 & 0.95\\SCA8-1 & 1080.57 & 0.44 & 
1080.57 & 0.55 & \bf{1049.65} & 
2.95 & 2.95\\SCA8-2 & 1055.11 & 0.49 & 
1055.11 & 0.45 & \bf{1039.64} & 
1.49 & 1.49\\SCA8-3 & 1013.56 & 0.69 & 
1027.19 & 0.54 & \bf{983.34} & 
3.07 & 4.46\\SCA8-4 & 1078.92 & 0.98 & 
1078.92 & 0.62 & \bf{1065.49} & 
1.26 & 1.26\\SCA8-5 & 1054.99 & 0.48 & 
1054.99 & 0.57 & \bf{1027.08} & 
2.72 & 2.72\\SCA8-6 & \bf{971.82} & 0.56 & 
971.99 & 0.59 & 971.82 & 0.00
 & 0.02\\SCA8-7 & 1070.92 & 0.56 & 
1070.92 & 0.50 & \bf{1051.28} & 
1.87 & 1.87\\SCA8-8 & 1091.18 & 0.53 & 
1091.18 & 0.44 & \bf{1071.18} & 
1.87 & 1.87\\SCA8-9 & 1070.34 & 0.40 & 
1070.34 & 0.52 & \bf{1060.50} & 
0.93 & 0.93\\CON3-0 & 629.72 & 0.43 & 
633.28 & 0.50 & \bf{616.52} & 
2.14 & 2.72\\CON3-1 & 560.75 & 0.67 & 
564.99 & 0.71 & \bf{554.47} & 
1.13 & 1.90\\CON3-2 & 521.38 & 0.50 & 
523.13 & 0.55 & \bf{518.00} & 
0.65 & 0.99\\CON3-3 & 591.20 & 0.74 & 
592.16 & 0.63 & \bf{591.19} & 
0.00 & 0.16\\CON3-4 & 592.58 & 0.69 & 
593.48 & 0.57 & \bf{588.79} & 
0.64 & 0.80\\CON3-5 & \bf{563.70} & 0.56 & 
567.75 & 0.66 & 563.70 & 0.00
 & 0.72\\CON3-6 & 506.16 & 0.46 & 
507.74 & 0.62 & \bf{499.05} & 
1.42 & 1.74\\CON3-7 & 585.94 & 0.62 & 
587.95 & 0.60 & \bf{576.48} & 
1.64 & 1.99\\CON3-8 & 532.86 & 0.58 & 
533.70 & 0.51 & \bf{523.05} & 
1.88 & 2.04\\CON3-9 & 582.79 & 0.48 & 
585.85 & 0.52 & \bf{578.24} & 
0.79 & 1.32\\CON8-0 & 884.58 & 0.40 & 
888.20 & 0.50 & \bf{857.17} & 
3.20 & 3.62\\CON8-1 & 759.63 & 0.47 & 
765.85 & 0.52 & \bf{740.85} & 
2.53 & 3.37\\CON8-2 & 723.21 & 0.75 & 
729.29 & 0.67 & \bf{712.89} & 
1.45 & 2.30\\CON8-3 & 837.82 & 0.78 & 
837.82 & 0.61 & \bf{811.07} & 
3.30 & 3.30\\CON8-4 & 789.18 & 0.42 & 
789.18 & 0.45 & \bf{772.25} & 
2.19 & 2.19\\CON8-5 & 758.12 & 0.53 & 
758.12 & 0.59 & \bf{754.88} & 
0.43 & 0.43\\CON8-6 & 688.39 & 0.51 & 
688.39 & 0.54 & \bf{678.92} & 
1.39 & 1.39\\CON8-7 & 816.18 & 0.42 & 
817.91 & 0.47 & \bf{811.96} & 
0.52 & 0.73\\CON8-8 & 794.41 & 0.78 & 
798.61 & 0.56 & \bf{767.53} & 
3.50 & 4.05\\CON8-9 & 827.31 & 0.48 & 
834.08 & 0.51 & \bf{809.00} & 
2.26 & 3.10\\\bf{PROM.} & 
\bf{769.41} & \bf{0.55} & \bf{771.20} & \bf{0.55} & \bf{758.54} & \bf{1.35} & \bf{1.60}\\[1ex]\hline
\end{tabular}
\label{table:nonlin}
\end{table} \clearpage
\begin{table}[ht]
\caption{Resultados de la ejecución de la metaheurística IGA, utilizando instancias de SalhiNagy con la configuración -n 200 -p 40 -cprob 20.0 -mprob 20.0}
\centering
\small
\begin{tabular}{c c c c c c c c}
\hline\hline
Instancia & Costo mínimo & Tiempo(seg.) & Costo promedio & Tiempo promedio(seg.) & CME & \%G & \%GP \\ [0.5ex]
\hline
CMT1X & 480.08 & 0.40 & 
485.45 & 0.44 & \bf{470.48} & 
2.04 & 3.18\\CMT1Y & 475.26 & 0.36 & 
478.60 & 0.54 & \bf{470.48} & 
1.02 & 1.73\\CMT2X & 711.67 & 1.20 & 
722.01 & 1.13 & \bf{682.39} & 
4.29 & 5.81\\CMT2Y & 696.62 & 1.42 & 
711.33 & 1.13 & \bf{682.39} & 
2.09 & 4.24\\CMT3X & 732.00 & 2.32 & 
742.80 & 2.53 & \bf{719.06} & 
1.80 & 3.30\\CMT3Y & 743.28 & 2.83 & 
749.42 & 2.76 & \bf{719.06} & 
3.37 & 4.22\\CMT4X & 901.24 & 7.55 & 
908.95 & 7.91 & \bf{854.21} & 
5.51 & 6.41\\CMT4Y & 920.43 & 7.97 & 
925.26 & 7.74 & \bf{852.46} & 
7.97 & 8.54\\CMT5X & 1092.91 & 16.27 & 
1110.82 & 15.56 & \bf{1030.56} & 
6.05 & 7.79\\CMT5Y & 1114.72 & 16.25 & 
1127.70 & 16.24 & \bf{1031.69} & 
8.05 & 9.31\\CMT11X & 903.44 & 4.97 & 
914.06 & 4.56 & \bf{831.09} & 
8.71 & 9.98\\CMT11Y & 892.12 & 5.67 & 
897.42 & 5.11 & \bf{829.85} & 
7.50 & 8.14\\CMT12X & 679.46 & 2.51 & 
683.50 & 2.69 & \bf{658.83} & 
3.13 & 3.75\\CMT12Y & 674.92 & 2.44 & 
678.86 & 2.56 & \bf{660.47} & 
2.19 & 2.78\\\bf{PROM.} & 
\bf{787.01} & \bf{5.15} & \bf{795.44} & \bf{5.06} & \bf{749.50} & \bf{4.55} & \bf{5.66}\\[1ex]\hline
\end{tabular}
\label{table:nonlin}
\end{table} \clearpage
\begin{table}[ht]
\caption{Resultados de la ejecución de la metaheurística IGA, utilizando instancias de Dethloff con la configuración -n 200 -p 40 -cprob 20.0 -mprob 30.0}
\centering
\small
\begin{tabular}{c c c c c c c c}
\hline\hline
Instancia & Costo mínimo & Tiempo(seg.) & Costo promedio & Tiempo promedio(seg.) & CME & \%G & \%GP \\ [0.5ex]
\hline
SCA3-0 & 640.55 & 0.59 & 
641.53 & 0.56 & \bf{635.62} & 
0.78 & 0.93\\SCA3-1 & \bf{697.84} & 0.48 & 
697.84 & 0.48 & 697.84 & 0.00
 & 0.00\\
SCA3-2 & 664.18 & 0.47 & 
667.22 & 0.56 & \bf{659.34} & 
0.73 & 1.20\\SCA3-3 & 685.05 & 0.47 & 
685.48 & 0.51 & \bf{680.04} & 
0.74 & 0.80\\SCA3-4 & \bf{690.50} & 0.54 & 
690.50 & 0.53 & 690.50 & 0.00
 & 0.00\\
SCA3-5 & 670.10 & 0.54 & 
670.10 & 0.55 & \bf{659.90} & 
1.55 & 1.55\\SCA3-6 & 652.94 & 0.48 & 
657.27 & 0.51 & \bf{651.09} & 
0.28 & 0.95\\SCA3-7 & 667.24 & 0.48 & 
667.24 & 0.44 & \bf{659.17} & 
1.22 & 1.22\\SCA3-8 & 726.58 & 0.50 & 
729.12 & 0.66 & \bf{719.47} & 
0.99 & 1.34\\SCA3-9 & \bf{681.00} & 0.48 & 
681.00 & 0.46 & 681.00 & 0.00
 & 0.00\\
SCA8-0 & 997.94 & 0.56 & 
1020.66 & 0.54 & \bf{961.50} & 
3.79 & 6.15\\SCA8-1 & 1081.42 & 0.78 & 
1082.36 & 0.59 & \bf{1049.65} & 
3.03 & 3.12\\SCA8-2 & 1054.47 & 0.46 & 
1054.47 & 0.42 & \bf{1039.64} & 
1.43 & 1.43\\SCA8-3 & 1016.59 & 0.42 & 
1016.59 & 0.64 & \bf{983.34} & 
3.38 & 3.38\\SCA8-4 & 1071.79 & 0.71 & 
1071.79 & 0.51 & \bf{1065.49} & 
0.59 & 0.59\\SCA8-5 & 1043.05 & 0.70 & 
1049.49 & 0.48 & \bf{1027.08} & 
1.55 & 2.18\\SCA8-6 & 990.58 & 0.60 & 
990.58 & 0.52 & \bf{971.82} & 
1.93 & 1.93\\SCA8-7 & 1070.92 & 0.51 & 
1070.92 & 0.57 & \bf{1051.28} & 
1.87 & 1.87\\SCA8-8 & 1095.67 & 0.53 & 
1098.31 & 0.45 & \bf{1071.18} & 
2.29 & 2.53\\SCA8-9 & 1081.50 & 0.72 & 
1081.50 & 0.55 & \bf{1060.50} & 
1.98 & 1.98\\CON3-0 & 622.44 & 0.45 & 
628.56 & 0.44 & \bf{616.52} & 
0.96 & 1.95\\CON3-1 & 557.38 & 0.48 & 
559.91 & 0.54 & \bf{554.47} & 
0.52 & 0.98\\CON3-2 & 521.38 & 0.63 & 
523.32 & 0.69 & \bf{518.00} & 
0.65 & 1.03\\CON3-3 & 591.85 & 0.55 & 
599.41 & 0.49 & \bf{591.19} & 
0.11 & 1.39\\CON3-4 & 591.43 & 0.41 & 
594.90 & 0.46 & \bf{588.79} & 
0.45 & 1.04\\CON3-5 & 564.88 & 0.45 & 
568.90 & 0.51 & \bf{563.70} & 
0.21 & 0.92\\CON3-6 & 502.88 & 0.47 & 
503.44 & 0.55 & \bf{499.05} & 
0.77 & 0.88\\CON3-7 & 581.15 & 0.40 & 
582.37 & 0.47 & \bf{576.48} & 
0.81 & 1.02\\CON3-8 & 523.14 & 0.54 & 
523.14 & 0.49 & \bf{523.05} & 
0.02 & 0.02\\CON3-9 & 589.61 & 0.94 & 
590.11 & 0.62 & \bf{578.24} & 
1.97 & 2.05\\CON8-0 & 882.72 & 0.56 & 
883.17 & 0.45 & \bf{857.17} & 
2.98 & 3.03\\CON8-1 & 763.21 & 0.43 & 
763.21 & 0.54 & \bf{740.85} & 
3.02 & 3.02\\CON8-2 & 717.57 & 0.53 & 
717.57 & 0.66 & \bf{712.89} & 
0.66 & 0.66\\CON8-3 & 831.37 & 0.46 & 
831.37 & 0.61 & \bf{811.07} & 
2.50 & 2.50\\CON8-4 & 781.78 & 0.42 & 
781.78 & 0.39 & \bf{772.25} & 
1.23 & 1.23\\CON8-5 & 768.34 & 1.04 & 
777.22 & 0.69 & \bf{754.88} & 
1.78 & 2.96\\CON8-6 & 701.62 & 0.77 & 
703.15 & 0.60 & \bf{678.92} & 
3.34 & 3.57\\CON8-7 & 814.50 & 0.56 & 
815.20 & 0.64 & \bf{811.96} & 
0.31 & 0.40\\CON8-8 & 796.10 & 0.60 & 
798.56 & 0.62 & \bf{767.53} & 
3.72 & 4.04\\CON8-9 & 816.54 & 1.03 & 
818.07 & 0.68 & \bf{809.00} & 
0.93 & 1.12\\\bf{PROM.} & 
\bf{769.99} & \bf{0.57} & \bf{772.18} & \bf{0.54} & \bf{758.54} & \bf{1.38} & \bf{1.67}\\[1ex]\hline
\end{tabular}
\label{table:nonlin}
\end{table} \clearpage
\begin{table}[ht]
\caption{Resultados de la ejecución de la metaheurística IGA, utilizando instancias de SalhiNagy con la configuración -n 200 -p 40 -cprob 20.0 -mprob 30.0}
\centering
\small
\begin{tabular}{c c c c c c c c}
\hline\hline
Instancia & Costo mínimo & Tiempo(seg.) & Costo promedio & Tiempo promedio(seg.) & CME & \%G & \%GP \\ [0.5ex]
\hline
CMT1X & 478.25 & 0.46 & 
482.89 & 0.55 & \bf{470.48} & 
1.65 & 2.64\\CMT1Y & 485.73 & 0.34 & 
486.00 & 0.39 & \bf{470.48} & 
3.24 & 3.30\\CMT2X & 712.29 & 1.33 & 
714.38 & 1.26 & \bf{682.39} & 
4.38 & 4.69\\CMT2Y & 703.08 & 0.94 & 
712.09 & 1.00 & \bf{682.39} & 
3.03 & 4.35\\CMT3X & 741.07 & 2.68 & 
746.07 & 2.63 & \bf{719.06} & 
3.06 & 3.76\\CMT3Y & 737.03 & 3.34 & 
747.37 & 2.62 & \bf{719.06} & 
2.50 & 3.94\\CMT4X & 900.58 & 7.66 & 
907.13 & 7.66 & \bf{854.21} & 
5.43 & 6.20\\CMT4Y & 902.77 & 7.36 & 
918.74 & 7.57 & \bf{852.46} & 
5.90 & 7.77\\CMT5X & 1100.79 & 15.46 & 
1121.47 & 15.25 & \bf{1030.56} & 
6.81 & 8.82\\CMT5Y & 1108.85 & 16.06 & 
1125.36 & 16.52 & \bf{1031.69} & 
7.48 & 9.08\\CMT11X & 906.53 & 4.92 & 
914.11 & 4.45 & \bf{831.09} & 
9.08 & 9.99\\CMT11Y & 891.53 & 4.72 & 
904.17 & 4.95 & \bf{829.85} & 
7.43 & 8.96\\CMT12X & 676.25 & 2.72 & 
680.40 & 2.83 & \bf{658.83} & 
2.64 & 3.27\\CMT12Y & 673.16 & 3.24 & 
674.60 & 2.70 & \bf{660.47} & 
1.92 & 2.14\\\bf{PROM.} & 
\bf{786.99} & \bf{5.09} & \bf{795.34} & \bf{5.03} & \bf{749.50} & \bf{4.61} & \bf{5.64}\\[1ex]\hline
\end{tabular}
\label{table:nonlin}
\end{table} \clearpage
\begin{table}[ht]
\caption{Resultados de la ejecución de la metaheurística IGA, utilizando instancias de Dethloff con la configuración -n 200 -p 40 -cprob 20.0 -mprob 40.0}
\centering
\small
\begin{tabular}{c c c c c c c c}
\hline\hline
Instancia & Costo mínimo & Tiempo(seg.) & Costo promedio & Tiempo promedio(seg.) & CME & \%G & \%GP \\ [0.5ex]
\hline
SCA3-0 & 641.69 & 0.47 & 
642.43 & 0.54 & \bf{635.62} & 
0.95 & 1.07\\SCA3-1 & 707.56 & 0.66 & 
707.56 & 0.59 & \bf{697.84} & 
1.39 & 1.39\\SCA3-2 & 668.65 & 0.44 & 
668.65 & 0.46 & \bf{659.34} & 
1.41 & 1.41\\SCA3-3 & 681.35 & 0.43 & 
683.76 & 0.44 & \bf{680.04} & 
0.19 & 0.55\\SCA3-4 & 692.57 & 0.47 & 
692.57 & 0.56 & \bf{690.50} & 
0.30 & 0.30\\SCA3-5 & 673.56 & 0.43 & 
673.75 & 0.45 & \bf{659.90} & 
2.07 & 2.10\\SCA3-6 & 653.81 & 0.51 & 
656.50 & 0.46 & \bf{651.09} & 
0.42 & 0.83\\SCA3-7 & 666.15 & 0.47 & 
668.91 & 0.42 & \bf{659.17} & 
1.06 & 1.48\\SCA3-8 & \bf{719.47} & 0.43 & 
719.47 & 0.57 & 719.47 & 0.00
 & 0.00\\
SCA3-9 & \bf{681.00} & 0.45 & 
681.00 & 0.42 & 681.00 & 0.00
 & 0.00\\
SCA8-0 & 984.40 & 0.43 & 
984.40 & 0.45 & \bf{961.50} & 
2.38 & 2.38\\SCA8-1 & 1056.27 & 0.58 & 
1074.10 & 0.49 & \bf{1049.65} & 
0.63 & 2.33\\SCA8-2 & 1055.32 & 0.49 & 
1055.32 & 0.46 & \bf{1039.64} & 
1.51 & 1.51\\SCA8-3 & 1023.96 & 0.40 & 
1023.96 & 0.54 & \bf{983.34} & 
4.13 & 4.13\\SCA8-4 & 1080.08 & 0.70 & 
1095.30 & 0.53 & \bf{1065.49} & 
1.37 & 2.80\\SCA8-5 & 1052.84 & 0.50 & 
1056.92 & 0.43 & \bf{1027.08} & 
2.51 & 2.91\\SCA8-6 & 972.48 & 0.41 & 
991.10 & 0.42 & \bf{971.82} & 
0.07 & 1.98\\SCA8-7 & 1080.53 & 0.48 & 
1080.53 & 0.52 & \bf{1051.28} & 
2.78 & 2.78\\SCA8-8 & 1083.61 & 0.69 & 
1087.15 & 0.50 & \bf{1071.18} & 
1.16 & 1.49\\SCA8-9 & 1081.51 & 0.46 & 
1081.51 & 0.43 & \bf{1060.50} & 
1.98 & 1.98\\CON3-0 & 630.32 & 0.46 & 
630.32 & 0.44 & \bf{616.52} & 
2.24 & 2.24\\CON3-1 & 560.75 & 0.43 & 
562.41 & 0.57 & \bf{554.47} & 
1.13 & 1.43\\CON3-2 & 521.38 & 0.50 & 
523.13 & 0.47 & \bf{518.00} & 
0.65 & 0.99\\CON3-3 & 592.78 & 0.48 & 
592.78 & 0.47 & \bf{591.19} & 
0.27 & 0.27\\CON3-4 & 592.58 & 0.50 & 
593.25 & 0.57 & \bf{588.79} & 
0.64 & 0.76\\CON3-5 & 567.94 & 0.41 & 
567.94 & 0.45 & \bf{563.70} & 
0.75 & 0.75\\CON3-6 & 504.20 & 0.56 & 
504.50 & 0.58 & \bf{499.05} & 
1.03 & 1.09\\CON3-7 & 577.91 & 0.53 & 
582.05 & 0.57 & \bf{576.48} & 
0.25 & 0.97\\CON3-8 & 534.28 & 0.59 & 
534.76 & 0.54 & \bf{523.05} & 
2.15 & 2.24\\CON3-9 & 590.58 & 0.62 & 
591.02 & 0.64 & \bf{578.24} & 
2.13 & 2.21\\CON8-0 & 869.87 & 0.57 & 
874.21 & 0.55 & \bf{857.17} & 
1.48 & 1.99\\CON8-1 & 761.89 & 0.61 & 
761.89 & 0.57 & \bf{740.85} & 
2.84 & 2.84\\CON8-2 & 730.97 & 0.46 & 
730.97 & 0.48 & \bf{712.89} & 
2.54 & 2.54\\CON8-3 & 824.69 & 0.42 & 
824.69 & 0.44 & \bf{811.07} & 
1.68 & 1.68\\CON8-4 & 787.55 & 0.43 & 
787.97 & 0.61 & \bf{772.25} & 
1.98 & 2.03\\CON8-5 & 771.63 & 0.74 & 
772.13 & 0.59 & \bf{754.88} & 
2.22 & 2.29\\CON8-6 & 694.29 & 0.46 & 
695.86 & 0.48 & \bf{678.92} & 
2.26 & 2.50\\CON8-7 & 815.32 & 0.49 & 
819.67 & 0.47 & \bf{811.96} & 
0.41 & 0.95\\CON8-8 & 780.80 & 0.62 & 
780.80 & 0.62 & \bf{767.53} & 
1.73 & 1.73\\CON8-9 & 826.17 & 0.53 & 
833.21 & 0.58 & \bf{809.00} & 
2.12 & 2.99\\\bf{PROM.} & 
\bf{769.82} & \bf{0.51} & \bf{772.21} & \bf{0.51} & \bf{758.54} & \bf{1.42} & \bf{1.70}\\[1ex]\hline
\end{tabular}
\label{table:nonlin}
\end{table} \clearpage
\begin{table}[ht]
\caption{Resultados de la ejecución de la metaheurística IGA, utilizando instancias de SalhiNagy con la configuración -n 200 -p 40 -cprob 20.0 -mprob 40.0}
\centering
\small
\begin{tabular}{c c c c c c c c}
\hline\hline
Instancia & Costo mínimo & Tiempo(seg.) & Costo promedio & Tiempo promedio(seg.) & CME & \%G & \%GP \\ [0.5ex]
\hline
CMT1X & 481.19 & 0.32 & 
481.63 & 0.38 & \bf{470.48} & 
2.28 & 2.37\\CMT1Y & 480.38 & 0.40 & 
485.68 & 0.38 & \bf{470.48} & 
2.10 & 3.23\\CMT2X & 696.61 & 1.02 & 
708.77 & 1.16 & \bf{682.39} & 
2.08 & 3.87\\CMT2Y & 715.36 & 0.97 & 
715.36 & 1.09 & \bf{682.39} & 
4.83 & 4.83\\CMT3X & 742.43 & 2.50 & 
747.88 & 2.59 & \bf{719.06} & 
3.25 & 4.01\\CMT3Y & 736.37 & 2.41 & 
742.00 & 2.43 & \bf{719.06} & 
2.41 & 3.19\\CMT4X & 897.10 & 7.44 & 
911.75 & 7.50 & \bf{854.21} & 
5.02 & 6.74\\CMT4Y & 896.34 & 7.52 & 
907.60 & 7.84 & \bf{852.46} & 
5.15 & 6.47\\CMT5X & 1100.31 & 16.22 & 
1119.14 & 15.66 & \bf{1030.56} & 
6.77 & 8.60\\CMT5Y & 1115.95 & 16.88 & 
1130.35 & 15.86 & \bf{1031.69} & 
8.17 & 9.56\\CMT11X & 889.60 & 4.47 & 
905.86 & 4.41 & \bf{831.09} & 
7.04 & 9.00\\CMT11Y & 898.45 & 5.56 & 
909.93 & 5.24 & \bf{829.85} & 
8.27 & 9.65\\CMT12X & 675.73 & 3.04 & 
678.79 & 2.75 & \bf{658.83} & 
2.57 & 3.03\\CMT12Y & 673.13 & 3.00 & 
676.02 & 2.62 & \bf{660.47} & 
1.92 & 2.35\\\bf{PROM.} & 
\bf{785.64} & \bf{5.12} & \bf{794.34} & \bf{4.99} & \bf{749.50} & \bf{4.42} & \bf{5.49}\\[1ex]\hline
\end{tabular}
\label{table:nonlin}
\end{table} \clearpage
\begin{table}[ht]
\caption{Resultados de la ejecución de la metaheurística IGA, utilizando instancias de Dethloff con la configuración -n 200 -p 40 -cprob 20.0 -mprob 50.0}
\centering
\small
\begin{tabular}{c c c c c c c c}
\hline\hline
Instancia & Costo mínimo & Tiempo(seg.) & Costo promedio & Tiempo promedio(seg.) & CME & \%G & \%GP \\ [0.5ex]
\hline
SCA3-0 & 640.55 & 0.50 & 
641.40 & 0.48 & \bf{635.62} & 
0.78 & 0.91\\SCA3-1 & \bf{697.84} & 0.42 & 
700.76 & 0.46 & 697.84 & 0.00
 & 0.42\\SCA3-2 & 673.67 & 0.46 & 
673.67 & 0.46 & \bf{659.34} & 
2.17 & 2.17\\SCA3-3 & \bf{680.04} & 0.48 & 
680.04 & 0.47 & 680.04 & 0.00
 & 0.00\\
SCA3-4 & \bf{690.50} & 0.50 & 
692.38 & 0.71 & 690.50 & 0.00
 & 0.27\\SCA3-5 & 668.48 & 0.43 & 
671.02 & 0.44 & \bf{659.90} & 
1.30 & 1.69\\SCA3-6 & 652.94 & 0.55 & 
652.94 & 0.49 & \bf{651.09} & 
0.28 & 0.28\\SCA3-7 & 666.15 & 0.70 & 
666.15 & 0.60 & \bf{659.17} & 
1.06 & 1.06\\SCA3-8 & 719.77 & 0.50 & 
723.60 & 0.53 & \bf{719.47} & 
0.04 & 0.57\\SCA3-9 & \bf{681.00} & 0.43 & 
681.00 & 0.43 & 681.00 & 0.00
 & 0.00\\
SCA8-0 & 1005.07 & 0.74 & 
1009.78 & 0.58 & \bf{961.50} & 
4.53 & 5.02\\SCA8-1 & 1079.83 & 0.42 & 
1079.83 & 0.47 & \bf{1049.65} & 
2.88 & 2.88\\SCA8-2 & 1051.21 & 0.69 & 
1051.21 & 0.76 & \bf{1039.64} & 
1.11 & 1.11\\SCA8-3 & 1014.99 & 0.48 & 
1025.48 & 0.49 & \bf{983.34} & 
3.22 & 4.29\\SCA8-4 & 1088.81 & 0.46 & 
1089.88 & 0.48 & \bf{1065.49} & 
2.19 & 2.29\\SCA8-5 & 1043.52 & 0.50 & 
1047.08 & 0.48 & \bf{1027.08} & 
1.60 & 1.95\\SCA8-6 & 985.76 & 0.50 & 
985.76 & 0.56 & \bf{971.82} & 
1.43 & 1.43\\SCA8-7 & 1069.83 & 0.44 & 
1069.83 & 0.51 & \bf{1051.28} & 
1.76 & 1.76\\SCA8-8 & 1087.82 & 0.63 & 
1088.12 & 0.57 & \bf{1071.18} & 
1.55 & 1.58\\SCA8-9 & 1074.21 & 0.99 & 
1079.12 & 0.55 & \bf{1060.50} & 
1.29 & 1.76\\CON3-0 & 625.35 & 0.48 & 
628.70 & 0.49 & \bf{616.52} & 
1.43 & 1.98\\CON3-1 & 557.21 & 0.46 & 
558.98 & 0.47 & \bf{554.47} & 
0.49 & 0.81\\CON3-2 & 521.38 & 0.51 & 
521.38 & 0.59 & \bf{518.00} & 
0.65 & 0.65\\CON3-3 & 591.20 & 0.49 & 
604.86 & 0.57 & \bf{591.19} & 
0.00 & 2.31\\CON3-4 & 593.69 & 0.44 & 
593.74 & 0.44 & \bf{588.79} & 
0.83 & 0.84\\CON3-5 & 564.88 & 0.44 & 
566.82 & 0.56 & \bf{563.70} & 
0.21 & 0.55\\CON3-6 & 504.15 & 0.64 & 
504.19 & 0.67 & \bf{499.05} & 
1.02 & 1.03\\CON3-7 & 581.83 & 0.52 & 
587.07 & 0.62 & \bf{576.48} & 
0.93 & 1.84\\CON3-8 & 524.38 & 0.58 & 
524.54 & 0.56 & \bf{523.05} & 
0.25 & 0.28\\CON3-9 & 578.25 & 0.46 & 
585.46 & 0.56 & \bf{578.24} & 
0.00 & 1.25\\CON8-0 & 877.18 & 0.49 & 
879.28 & 0.61 & \bf{857.17} & 
2.33 & 2.58\\CON8-1 & 764.12 & 0.58 & 
766.76 & 0.53 & \bf{740.85} & 
3.14 & 3.50\\CON8-2 & 731.11 & 0.60 & 
737.13 & 0.61 & \bf{712.89} & 
2.56 & 3.40\\CON8-3 & 831.09 & 0.44 & 
831.09 & 0.47 & \bf{811.07} & 
2.47 & 2.47\\CON8-4 & 795.85 & 0.41 & 
795.88 & 0.62 & \bf{772.25} & 
3.06 & 3.06\\CON8-5 & 778.15 & 0.46 & 
778.87 & 0.48 & \bf{754.88} & 
3.08 & 3.18\\CON8-6 & 696.11 & 0.47 & 
696.11 & 0.53 & \bf{678.92} & 
2.53 & 2.53\\CON8-7 & 821.88 & 0.50 & 
822.09 & 0.56 & \bf{811.96} & 
1.22 & 1.25\\CON8-8 & 795.97 & 0.44 & 
798.21 & 0.46 & \bf{767.53} & 
3.71 & 4.00\\CON8-9 & 816.54 & 0.46 & 
822.36 & 0.53 & \bf{809.00} & 
0.93 & 1.65\\\bf{PROM.} & 
\bf{770.56} & \bf{0.52} & \bf{772.81} & \bf{0.54} & \bf{758.54} & \bf{1.45} & \bf{1.77}\\[1ex]\hline
\end{tabular}
\label{table:nonlin}
\end{table} \clearpage
\begin{table}[ht]
\caption{Resultados de la ejecución de la metaheurística IGA, utilizando instancias de SalhiNagy con la configuración -n 200 -p 40 -cprob 20.0 -mprob 50.0}
\centering
\small
\begin{tabular}{c c c c c c c c}
\hline\hline
Instancia & Costo mínimo & Tiempo(seg.) & Costo promedio & Tiempo promedio(seg.) & CME & \%G & \%GP \\ [0.5ex]
\hline
CMT1X & 481.84 & 0.38 & 
485.49 & 0.38 & \bf{470.48} & 
2.41 & 3.19\\CMT1Y & 478.97 & 0.90 & 
479.73 & 0.55 & \bf{470.48} & 
1.80 & 1.97\\CMT2X & 709.87 & 0.94 & 
710.49 & 1.19 & \bf{682.39} & 
4.03 & 4.12\\CMT2Y & 702.47 & 1.40 & 
708.54 & 1.35 & \bf{682.39} & 
2.94 & 3.83\\CMT3X & 736.87 & 2.88 & 
742.24 & 2.75 & \bf{719.06} & 
2.48 & 3.22\\CMT3Y & 752.30 & 2.95 & 
753.35 & 2.62 & \bf{719.06} & 
4.62 & 4.77\\CMT4X & 903.06 & 7.73 & 
909.92 & 7.58 & \bf{854.21} & 
5.72 & 6.52\\CMT4Y & 908.88 & 7.34 & 
918.50 & 7.39 & \bf{852.46} & 
6.62 & 7.75\\CMT5X & 1107.56 & 15.02 & 
1120.87 & 15.27 & \bf{1030.56} & 
7.47 & 8.76\\CMT5Y & 1111.30 & 15.28 & 
1122.73 & 15.46 & \bf{1031.69} & 
7.72 & 8.82\\CMT11X & 896.31 & 4.94 & 
903.20 & 4.83 & \bf{831.09} & 
7.85 & 8.68\\CMT11Y & 884.87 & 5.46 & 
908.52 & 5.21 & \bf{829.85} & 
6.63 & 9.48\\CMT12X & 687.77 & 2.48 & 
691.16 & 2.61 & \bf{658.83} & 
4.39 & 4.91\\CMT12Y & 679.05 & 2.99 & 
680.60 & 2.74 & \bf{660.47} & 
2.81 & 3.05\\\bf{PROM.} & 
\bf{788.65} & \bf{5.05} & \bf{795.38} & \bf{4.99} & \bf{749.50} & \bf{4.82} & \bf{5.65}\\[1ex]\hline
\end{tabular}
\label{table:nonlin}
\end{table} \clearpage
\begin{table}[ht]
\caption{Resultados de la ejecución de la metaheurística IGA, utilizando instancias de Dethloff con la configuración -n 200 -p 40 -cprob 20.0 -mprob 60.0}
\centering
\small
\begin{tabular}{c c c c c c c c}
\hline\hline
Instancia & Costo mínimo & Tiempo(seg.) & Costo promedio & Tiempo promedio(seg.) & CME & \%G & \%GP \\ [0.5ex]
\hline
SCA3-0 & 640.55 & 0.42 & 
642.38 & 0.48 & \bf{635.62} & 
0.78 & 1.06\\SCA3-1 & \bf{697.84} & 0.56 & 
701.27 & 0.58 & 697.84 & 0.00
 & 0.49\\SCA3-2 & 664.21 & 0.45 & 
664.70 & 0.50 & \bf{659.34} & 
0.74 & 0.81\\SCA3-3 & \bf{680.04} & 0.57 & 
680.04 & 0.51 & 680.04 & 0.00
 & 0.00\\
SCA3-4 & \bf{690.50} & 0.63 & 
690.50 & 0.56 & 690.50 & 0.00
 & 0.00\\
SCA3-5 & 665.64 & 0.60 & 
665.64 & 0.52 & \bf{659.90} & 
0.87 & 0.87\\SCA3-6 & 652.94 & 0.46 & 
652.94 & 0.44 & \bf{651.09} & 
0.28 & 0.28\\SCA3-7 & 666.15 & 0.42 & 
666.15 & 0.43 & \bf{659.17} & 
1.06 & 1.06\\SCA3-8 & 731.95 & 0.83 & 
734.00 & 0.65 & \bf{719.47} & 
1.73 & 2.02\\SCA3-9 & \bf{681.00} & 0.40 & 
681.00 & 0.44 & 681.00 & 0.00
 & 0.00\\
SCA8-0 & 1013.28 & 0.44 & 
1013.31 & 0.48 & \bf{961.50} & 
5.39 & 5.39\\SCA8-1 & 1079.92 & 0.70 & 
1085.47 & 0.57 & \bf{1049.65} & 
2.88 & 3.41\\SCA8-2 & 1047.57 & 0.54 & 
1051.60 & 0.50 & \bf{1039.64} & 
0.76 & 1.15\\SCA8-3 & 1015.19 & 0.55 & 
1019.19 & 0.52 & \bf{983.34} & 
3.24 & 3.65\\SCA8-4 & 1080.08 & 0.56 & 
1082.82 & 0.57 & \bf{1065.49} & 
1.37 & 1.63\\SCA8-5 & 1067.18 & 0.45 & 
1070.83 & 0.43 & \bf{1027.08} & 
3.90 & 4.26\\SCA8-6 & 987.96 & 0.48 & 
987.96 & 0.43 & \bf{971.82} & 
1.66 & 1.66\\SCA8-7 & 1073.05 & 0.54 & 
1081.56 & 0.53 & \bf{1051.28} & 
2.07 & 2.88\\SCA8-8 & 1102.92 & 0.63 & 
1102.92 & 0.60 & \bf{1071.18} & 
2.96 & 2.96\\SCA8-9 & 1072.28 & 0.41 & 
1076.80 & 0.52 & \bf{1060.50} & 
1.11 & 1.54\\CON3-0 & 619.09 & 0.53 & 
622.19 & 0.59 & \bf{616.52} & 
0.42 & 0.92\\CON3-1 & 556.92 & 0.73 & 
557.88 & 0.54 & \bf{554.47} & 
0.44 & 0.61\\CON3-2 & 521.38 & 0.50 & 
521.38 & 0.68 & \bf{518.00} & 
0.65 & 0.65\\CON3-3 & 598.94 & 0.56 & 
598.94 & 0.56 & \bf{591.19} & 
1.31 & 1.31\\CON3-4 & \bf{588.79} & 0.69 & 
597.50 & 0.67 & 588.79 & 0.00
 & 1.48\\CON3-5 & 566.96 & 0.46 & 
566.96 & 0.46 & \bf{563.70} & 
0.58 & 0.58\\CON3-6 & 502.88 & 0.49 & 
502.88 & 0.58 & \bf{499.05} & 
0.77 & 0.77\\CON3-7 & 582.12 & 0.45 & 
584.07 & 0.51 & \bf{576.48} & 
0.98 & 1.32\\CON3-8 & 524.59 & 0.78 & 
526.09 & 0.65 & \bf{523.05} & 
0.29 & 0.58\\CON3-9 & 588.18 & 0.60 & 
590.23 & 0.59 & \bf{578.24} & 
1.72 & 2.07\\CON8-0 & 874.61 & 0.58 & 
874.61 & 0.50 & \bf{857.17} & 
2.03 & 2.03\\CON8-1 & 762.46 & 0.62 & 
764.19 & 0.65 & \bf{740.85} & 
2.92 & 3.15\\CON8-2 & 727.76 & 0.73 & 
728.73 & 0.60 & \bf{712.89} & 
2.09 & 2.22\\CON8-3 & 830.64 & 0.47 & 
842.64 & 0.50 & \bf{811.07} & 
2.41 & 3.89\\CON8-4 & 780.03 & 0.39 & 
785.32 & 0.40 & \bf{772.25} & 
1.01 & 1.69\\CON8-5 & 766.55 & 0.50 & 
766.55 & 0.56 & \bf{754.88} & 
1.55 & 1.55\\CON8-6 & 694.02 & 0.50 & 
697.38 & 0.52 & \bf{678.92} & 
2.22 & 2.72\\CON8-7 & 822.26 & 0.56 & 
835.88 & 0.62 & \bf{811.96} & 
1.27 & 2.95\\CON8-8 & 782.87 & 0.61 & 
792.85 & 0.56 & \bf{767.53} & 
2.00 & 3.30\\CON8-9 & 821.80 & 0.70 & 
824.60 & 0.54 & \bf{809.00} & 
1.58 & 1.93\\\bf{PROM.} & 
\bf{770.58} & \bf{0.55} & \bf{773.30} & \bf{0.54} & \bf{758.54} & \bf{1.43} & \bf{1.77}\\[1ex]\hline
\end{tabular}
\label{table:nonlin}
\end{table} \clearpage
\begin{table}[ht]
\caption{Resultados de la ejecución de la metaheurística IGA, utilizando instancias de SalhiNagy con la configuración -n 200 -p 40 -cprob 20.0 -mprob 60.0}
\centering
\small
\begin{tabular}{c c c c c c c c}
\hline\hline
Instancia & Costo mínimo & Tiempo(seg.) & Costo promedio & Tiempo promedio(seg.) & CME & \%G & \%GP \\ [0.5ex]
\hline
CMT1X & 477.72 & 0.56 & 
481.66 & 0.46 & \bf{470.48} & 
1.54 & 2.38\\CMT1Y & 487.50 & 0.44 & 
492.07 & 0.40 & \bf{470.48} & 
3.62 & 4.59\\CMT2X & 710.74 & 1.38 & 
715.36 & 1.14 & \bf{682.39} & 
4.15 & 4.83\\CMT2Y & 699.06 & 0.99 & 
710.45 & 1.10 & \bf{682.39} & 
2.44 & 4.11\\CMT3X & 737.70 & 2.42 & 
741.82 & 2.69 & \bf{719.06} & 
2.59 & 3.16\\CMT3Y & 745.06 & 2.76 & 
747.46 & 2.54 & \bf{719.06} & 
3.62 & 3.95\\CMT4X & 894.56 & 7.73 & 
909.23 & 7.55 & \bf{854.21} & 
4.72 & 6.44\\CMT4Y & 907.95 & 7.87 & 
923.30 & 7.75 & \bf{852.46} & 
6.51 & 8.31\\CMT5X & 1094.76 & 16.95 & 
1114.83 & 15.34 & \bf{1030.56} & 
6.23 & 8.18\\CMT5Y & 1102.37 & 16.72 & 
1124.20 & 15.78 & \bf{1031.69} & 
6.85 & 8.97\\CMT11X & 897.51 & 4.51 & 
913.83 & 4.50 & \bf{831.09} & 
7.99 & 9.96\\CMT11Y & 887.82 & 4.58 & 
895.93 & 5.02 & \bf{829.85} & 
6.99 & 7.96\\CMT12X & 674.94 & 2.65 & 
679.77 & 2.81 & \bf{658.83} & 
2.45 & 3.18\\CMT12Y & 673.67 & 2.73 & 
675.56 & 2.80 & \bf{660.47} & 
2.00 & 2.28\\\bf{PROM.} & 
\bf{785.10} & \bf{5.16} & \bf{794.68} & \bf{4.99} & \bf{749.50} & \bf{4.41} & \bf{5.59}\\[1ex]\hline
\end{tabular}
\label{table:nonlin}
\end{table} \clearpage
\begin{table}[ht]
\caption{Resultados de la ejecución de la metaheurística IGA, utilizando instancias de Dethloff con la configuración -n 200 -p 40 -cprob 20.0 -mprob 70.0}
\centering
\small
\begin{tabular}{c c c c c c c c}
\hline\hline
Instancia & Costo mínimo & Tiempo(seg.) & Costo promedio & Tiempo promedio(seg.) & CME & \%G & \%GP \\ [0.5ex]
\hline
SCA3-0 & 641.69 & 0.53 & 
641.69 & 0.54 & \bf{635.62} & 
0.95 & 0.95\\SCA3-1 & \bf{697.84} & 0.91 & 
697.84 & 0.60 & 697.84 & 0.00
 & 0.00\\
SCA3-2 & 661.13 & 0.46 & 
666.13 & 0.64 & \bf{659.34} & 
0.27 & 1.03\\SCA3-3 & \bf{680.04} & 0.70 & 
680.97 & 0.53 & 680.04 & 0.00
 & 0.14\\SCA3-4 & \bf{690.50} & 0.43 & 
690.50 & 0.45 & 690.50 & 0.00
 & 0.00\\
SCA3-5 & 668.48 & 0.57 & 
669.29 & 0.65 & \bf{659.90} & 
1.30 & 1.42\\SCA3-6 & 653.68 & 0.50 & 
657.06 & 0.54 & \bf{651.09} & 
0.40 & 0.92\\SCA3-7 & 666.15 & 0.55 & 
666.15 & 0.49 & \bf{659.17} & 
1.06 & 1.06\\SCA3-8 & 726.88 & 0.78 & 
726.88 & 0.61 & \bf{719.47} & 
1.03 & 1.03\\SCA3-9 & 684.72 & 0.65 & 
686.43 & 0.60 & \bf{681.00} & 
0.55 & 0.80\\SCA8-0 & 970.64 & 0.47 & 
992.10 & 0.57 & \bf{961.50} & 
0.95 & 3.18\\SCA8-1 & 1085.05 & 0.49 & 
1086.14 & 0.48 & \bf{1049.65} & 
3.37 & 3.48\\SCA8-2 & 1050.37 & 0.68 & 
1050.37 & 0.68 & \bf{1039.64} & 
1.03 & 1.03\\SCA8-3 & 1013.35 & 0.48 & 
1013.35 & 0.46 & \bf{983.34} & 
3.05 & 3.05\\SCA8-4 & 1075.27 & 0.55 & 
1095.12 & 0.43 & \bf{1065.49} & 
0.92 & 2.78\\SCA8-5 & 1069.47 & 0.46 & 
1073.26 & 0.71 & \bf{1027.08} & 
4.13 & 4.50\\SCA8-6 & 972.48 & 0.46 & 
980.71 & 0.63 & \bf{971.82} & 
0.07 & 0.91\\SCA8-7 & 1073.05 & 0.37 & 
1076.81 & 0.38 & \bf{1051.28} & 
2.07 & 2.43\\SCA8-8 & 1075.00 & 0.44 & 
1082.75 & 0.46 & \bf{1071.18} & 
0.36 & 1.08\\SCA8-9 & 1078.72 & 0.38 & 
1078.72 & 0.44 & \bf{1060.50} & 
1.72 & 1.72\\CON3-0 & 629.51 & 0.48 & 
632.41 & 0.52 & \bf{616.52} & 
2.11 & 2.58\\CON3-1 & 556.04 & 0.73 & 
558.39 & 0.58 & \bf{554.47} & 
0.28 & 0.71\\CON3-2 & 521.38 & 0.52 & 
523.13 & 0.58 & \bf{518.00} & 
0.65 & 0.99\\CON3-3 & 595.06 & 0.94 & 
599.80 & 0.62 & \bf{591.19} & 
0.65 & 1.46\\CON3-4 & 591.43 & 0.52 & 
592.29 & 0.50 & \bf{588.79} & 
0.45 & 0.59\\CON3-5 & 568.76 & 0.48 & 
571.62 & 0.48 & \bf{563.70} & 
0.90 & 1.41\\CON3-6 & 503.97 & 0.69 & 
505.20 & 0.76 & \bf{499.05} & 
0.99 & 1.23\\CON3-7 & 588.25 & 0.70 & 
589.93 & 0.55 & \bf{576.48} & 
2.04 & 2.33\\CON3-8 & 534.89 & 0.44 & 
537.24 & 0.63 & \bf{523.05} & 
2.26 & 2.71\\CON3-9 & 590.16 & 0.55 & 
590.33 & 0.60 & \bf{578.24} & 
2.06 & 2.09\\CON8-0 & 884.49 & 0.56 & 
885.56 & 0.58 & \bf{857.17} & 
3.19 & 3.31\\CON8-1 & 741.70 & 0.48 & 
741.70 & 0.65 & \bf{740.85} & 
0.11 & 0.11\\CON8-2 & 726.90 & 0.53 & 
728.35 & 0.61 & \bf{712.89} & 
1.97 & 2.17\\CON8-3 & \bf{811.07} & 0.51 & 
811.07 & 0.47 & 811.07 & 0.00
 & 0.00\\
CON8-4 & 773.33 & 0.45 & 
773.33 & 0.47 & \bf{772.25} & 
0.14 & 0.14\\CON8-5 & 762.61 & 0.46 & 
762.61 & 0.56 & \bf{754.88} & 
1.02 & 1.02\\CON8-6 & 689.74 & 0.70 & 
689.74 & 0.85 & \bf{678.92} & 
1.59 & 1.59\\CON8-7 & 823.05 & 0.46 & 
823.75 & 0.41 & \bf{811.96} & 
1.37 & 1.45\\CON8-8 & 792.19 & 0.62 & 
804.23 & 0.56 & \bf{767.53} & 
3.21 & 4.78\\CON8-9 & 821.73 & 0.80 & 
821.73 & 0.63 & \bf{809.00} & 
1.57 & 1.57\\\bf{PROM.} & 
\bf{768.52} & \bf{0.56} & \bf{771.37} & \bf{0.56} & \bf{758.54} & \bf{1.24} & \bf{1.59}\\[1ex]\hline
\end{tabular}
\label{table:nonlin}
\end{table} \clearpage
\begin{table}[ht]
\caption{Resultados de la ejecución de la metaheurística IGA, utilizando instancias de SalhiNagy con la configuración -n 200 -p 40 -cprob 20.0 -mprob 70.0}
\centering
\small
\begin{tabular}{c c c c c c c c}
\hline\hline
Instancia & Costo mínimo & Tiempo(seg.) & Costo promedio & Tiempo promedio(seg.) & CME & \%G & \%GP \\ [0.5ex]
\hline
CMT1X & 478.23 & 0.37 & 
480.97 & 0.41 & \bf{470.48} & 
1.65 & 2.23\\CMT1Y & 486.83 & 0.33 & 
487.71 & 0.33 & \bf{470.48} & 
3.48 & 3.66\\CMT2X & 707.97 & 1.15 & 
709.72 & 1.23 & \bf{682.39} & 
3.75 & 4.01\\CMT2Y & 712.33 & 0.96 & 
717.30 & 1.09 & \bf{682.39} & 
4.39 & 5.12\\CMT3X & 737.91 & 2.52 & 
743.21 & 2.83 & \bf{719.06} & 
2.62 & 3.36\\CMT3Y & 750.17 & 3.08 & 
752.28 & 3.00 & \bf{719.06} & 
4.33 & 4.62\\CMT4X & 899.01 & 8.22 & 
907.40 & 7.78 & \bf{854.21} & 
5.24 & 6.23\\CMT4Y & 898.58 & 7.91 & 
905.10 & 7.62 & \bf{852.46} & 
5.41 & 6.17\\CMT5X & 1094.76 & 16.04 & 
1122.49 & 15.57 & \bf{1030.56} & 
6.23 & 8.92\\CMT5Y & 1113.71 & 16.07 & 
1125.16 & 15.62 & \bf{1031.69} & 
7.95 & 9.06\\CMT11X & 895.41 & 5.22 & 
909.16 & 4.67 & \bf{831.09} & 
7.74 & 9.39\\CMT11Y & 853.36 & 4.95 & 
900.89 & 5.04 & \bf{829.85} & 
2.83 & 8.56\\CMT12X & 674.87 & 2.68 & 
683.55 & 2.56 & \bf{658.83} & 
2.43 & 3.75\\CMT12Y & 674.13 & 2.37 & 
680.47 & 2.40 & \bf{660.47} & 
2.07 & 3.03\\\bf{PROM.} & 
\bf{784.09} & \bf{5.13} & \bf{794.67} & \bf{5.01} & \bf{749.50} & \bf{4.29} & \bf{5.58}\\[1ex]\hline
\end{tabular}
\label{table:nonlin}
\end{table} \clearpage
\begin{table}[ht]
\caption{Resultados de la ejecución de la metaheurística IGA, utilizando instancias de Dethloff con la configuración -n 200 -p 40 -cprob 20.0 -mprob 80.0}
\centering
\small
\begin{tabular}{c c c c c c c c}
\hline\hline
Instancia & Costo mínimo & Tiempo(seg.) & Costo promedio & Tiempo promedio(seg.) & CME & \%G & \%GP \\ [0.5ex]
\hline
SCA3-0 & 640.55 & 0.57 & 
640.55 & 0.56 & \bf{635.62} & 
0.78 & 0.78\\SCA3-1 & 706.90 & 0.43 & 
708.29 & 0.48 & \bf{697.84} & 
1.30 & 1.50\\SCA3-2 & \bf{659.34} & 0.42 & 
660.24 & 0.47 & 659.34 & 0.00
 & 0.14\\SCA3-3 & \bf{680.04} & 0.51 & 
681.53 & 0.53 & 680.04 & 0.00
 & 0.22\\SCA3-4 & \bf{690.50} & 0.54 & 
690.50 & 0.55 & 690.50 & 0.00
 & 0.00\\
SCA3-5 & 683.44 & 0.58 & 
683.44 & 0.55 & \bf{659.90} & 
3.57 & 3.57\\SCA3-6 & 652.94 & 0.44 & 
652.94 & 0.54 & \bf{651.09} & 
0.28 & 0.28\\SCA3-7 & 666.15 & 0.72 & 
670.37 & 0.56 & \bf{659.17} & 
1.06 & 1.70\\SCA3-8 & 723.99 & 0.46 & 
723.99 & 0.48 & \bf{719.47} & 
0.63 & 0.63\\SCA3-9 & \bf{681.00} & 0.43 & 
681.00 & 0.64 & 681.00 & 0.00
 & 0.00\\
SCA8-0 & 973.22 & 0.45 & 
982.58 & 0.59 & \bf{961.50} & 
1.22 & 2.19\\SCA8-1 & 1078.75 & 0.56 & 
1085.79 & 0.71 & \bf{1049.65} & 
2.77 & 3.44\\SCA8-2 & 1049.22 & 0.70 & 
1055.07 & 0.52 & \bf{1039.64} & 
0.92 & 1.48\\SCA8-3 & 1018.31 & 1.00 & 
1018.31 & 0.68 & \bf{983.34} & 
3.56 & 3.56\\SCA8-4 & 1102.84 & 0.38 & 
1105.15 & 0.47 & \bf{1065.49} & 
3.51 & 3.72\\SCA8-5 & 1055.85 & 0.48 & 
1056.74 & 0.54 & \bf{1027.08} & 
2.80 & 2.89\\SCA8-6 & 992.85 & 0.51 & 
992.85 & 0.50 & \bf{971.82} & 
2.16 & 2.16\\SCA8-7 & 1084.96 & 0.39 & 
1086.07 & 0.46 & \bf{1051.28} & 
3.20 & 3.31\\SCA8-8 & 1092.02 & 0.56 & 
1092.02 & 0.56 & \bf{1071.18} & 
1.95 & 1.95\\SCA8-9 & 1076.54 & 0.94 & 
1077.25 & 0.74 & \bf{1060.50} & 
1.51 & 1.58\\CON3-0 & 624.84 & 0.48 & 
624.84 & 0.67 & \bf{616.52} & 
1.35 & 1.35\\CON3-1 & 562.52 & 0.57 & 
562.77 & 0.47 & \bf{554.47} & 
1.45 & 1.50\\CON3-2 & 521.38 & 0.72 & 
521.38 & 0.56 & \bf{518.00} & 
0.65 & 0.65\\CON3-3 & 600.02 & 0.48 & 
604.12 & 0.44 & \bf{591.19} & 
1.49 & 2.19\\CON3-4 & 597.75 & 0.47 & 
602.92 & 0.45 & \bf{588.79} & 
1.52 & 2.40\\CON3-5 & 570.84 & 0.47 & 
571.05 & 0.58 & \bf{563.70} & 
1.27 & 1.30\\CON3-6 & 502.26 & 0.73 & 
504.87 & 0.76 & \bf{499.05} & 
0.64 & 1.17\\CON3-7 & 582.12 & 0.47 & 
584.40 & 0.49 & \bf{576.48} & 
0.98 & 1.37\\CON3-8 & 535.94 & 0.50 & 
535.94 & 0.62 & \bf{523.05} & 
2.46 & 2.46\\CON3-9 & 588.11 & 0.67 & 
588.37 & 0.64 & \bf{578.24} & 
1.71 & 1.75\\CON8-0 & 875.31 & 0.86 & 
884.11 & 0.68 & \bf{857.17} & 
2.12 & 3.14\\CON8-1 & 743.27 & 0.60 & 
743.27 & 0.64 & \bf{740.85} & 
0.33 & 0.33\\CON8-2 & 716.69 & 0.62 & 
716.69 & 0.56 & \bf{712.89} & 
0.53 & 0.53\\CON8-3 & 837.04 & 0.61 & 
840.40 & 0.73 & \bf{811.07} & 
3.20 & 3.62\\CON8-4 & 781.64 & 0.41 & 
781.64 & 0.41 & \bf{772.25} & 
1.22 & 1.22\\CON8-5 & 766.86 & 0.44 & 
766.86 & 0.69 & \bf{754.88} & 
1.59 & 1.59\\CON8-6 & 686.39 & 0.53 & 
690.44 & 0.68 & \bf{678.92} & 
1.10 & 1.70\\CON8-7 & 815.79 & 0.38 & 
827.23 & 0.43 & \bf{811.96} & 
0.47 & 1.88\\CON8-8 & 794.63 & 0.88 & 
794.63 & 0.73 & \bf{767.53} & 
3.53 & 3.53\\CON8-9 & 826.11 & 0.72 & 
826.27 & 0.52 & \bf{809.00} & 
2.11 & 2.13\\\bf{PROM.} & 
\bf{770.97} & \bf{0.57} & \bf{772.92} & \bf{0.57} & \bf{758.54} & \bf{1.52} & \bf{1.77}\\[1ex]\hline
\end{tabular}
\label{table:nonlin}
\end{table} \clearpage
\begin{table}[ht]
\caption{Resultados de la ejecución de la metaheurística IGA, utilizando instancias de SalhiNagy con la configuración -n 200 -p 40 -cprob 20.0 -mprob 80.0}
\centering
\small
\begin{tabular}{c c c c c c c c}
\hline\hline
Instancia & Costo mínimo & Tiempo(seg.) & Costo promedio & Tiempo promedio(seg.) & CME & \%G & \%GP \\ [0.5ex]
\hline
CMT1X & 475.69 & 0.51 & 
479.91 & 0.55 & \bf{470.48} & 
1.11 & 2.00\\CMT1Y & 477.41 & 0.74 & 
484.75 & 0.50 & \bf{470.48} & 
1.47 & 3.03\\CMT2X & 714.89 & 1.30 & 
719.20 & 1.19 & \bf{682.39} & 
4.76 & 5.39\\CMT2Y & 714.23 & 1.60 & 
721.50 & 1.36 & \bf{682.39} & 
4.67 & 5.73\\CMT3X & 743.59 & 2.49 & 
746.90 & 2.54 & \bf{719.06} & 
3.41 & 3.87\\CMT3Y & 733.50 & 2.31 & 
742.16 & 2.49 & \bf{719.06} & 
2.01 & 3.21\\CMT4X & 912.69 & 6.83 & 
918.75 & 7.49 & \bf{854.21} & 
6.85 & 7.55\\CMT4Y & 905.56 & 7.65 & 
917.47 & 7.66 & \bf{852.46} & 
6.23 & 7.63\\CMT5X & 1104.50 & 14.84 & 
1110.38 & 15.23 & \bf{1030.56} & 
7.17 & 7.75\\CMT5Y & 1114.98 & 15.62 & 
1122.86 & 16.97 & \bf{1031.69} & 
8.07 & 8.84\\CMT11X & 889.51 & 4.41 & 
904.36 & 4.47 & \bf{831.09} & 
7.03 & 8.82\\CMT11Y & 897.31 & 7.42 & 
911.79 & 6.13 & \bf{829.85} & 
8.13 & 9.87\\CMT12X & 671.01 & 3.21 & 
680.49 & 2.71 & \bf{658.83} & 
1.85 & 3.29\\CMT12Y & 680.08 & 3.01 & 
682.54 & 2.66 & \bf{660.47} & 
2.97 & 3.34\\\bf{PROM.} & 
\bf{788.21} & \bf{5.14} & \bf{795.93} & \bf{5.14} & \bf{749.50} & \bf{4.69} & \bf{5.74}\\[1ex]\hline
\end{tabular}
\label{table:nonlin}
\end{table} \clearpage
\begin{table}[ht]
\caption{Resultados de la ejecución de la metaheurística IGA, utilizando instancias de Dethloff con la configuración -n 200 -p 40 -cprob 20.0 -mprob 90.0}
\centering
\small
\begin{tabular}{c c c c c c c c}
\hline\hline
Instancia & Costo mínimo & Tiempo(seg.) & Costo promedio & Tiempo promedio(seg.) & CME & \%G & \%GP \\ [0.5ex]
\hline
SCA3-0 & 640.55 & 0.65 & 
640.55 & 0.61 & \bf{635.62} & 
0.78 & 0.78\\SCA3-1 & 700.50 & 0.56 & 
700.81 & 0.64 & \bf{697.84} & 
0.38 & 0.43\\SCA3-2 & 671.53 & 0.45 & 
674.43 & 0.56 & \bf{659.34} & 
1.85 & 2.29\\SCA3-3 & 681.16 & 0.48 & 
683.67 & 0.63 & \bf{680.04} & 
0.16 & 0.53\\SCA3-4 & \bf{690.50} & 0.88 & 
690.50 & 0.67 & 690.50 & 0.00
 & 0.00\\
SCA3-5 & \bf{659.90} & 0.49 & 
670.14 & 0.58 & 659.90 & 0.00
 & 1.55\\SCA3-6 & 656.23 & 0.49 & 
657.06 & 0.57 & \bf{651.09} & 
0.79 & 0.92\\SCA3-7 & 666.15 & 0.45 & 
666.15 & 0.56 & \bf{659.17} & 
1.06 & 1.06\\SCA3-8 & 731.44 & 0.54 & 
731.93 & 0.49 & \bf{719.47} & 
1.66 & 1.73\\SCA3-9 & 687.61 & 0.48 & 
687.64 & 0.49 & \bf{681.00} & 
0.97 & 0.98\\SCA8-0 & 1004.88 & 0.50 & 
1008.12 & 0.60 & \bf{961.50} & 
4.51 & 4.85\\SCA8-1 & 1087.21 & 0.59 & 
1088.11 & 0.58 & \bf{1049.65} & 
3.58 & 3.66\\SCA8-2 & 1050.17 & 0.78 & 
1050.32 & 0.68 & \bf{1039.64} & 
1.01 & 1.03\\SCA8-3 & 1024.48 & 0.52 & 
1027.24 & 0.55 & \bf{983.34} & 
4.18 & 4.46\\SCA8-4 & 1078.86 & 0.66 & 
1080.14 & 0.47 & \bf{1065.49} & 
1.25 & 1.37\\SCA8-5 & 1038.59 & 0.42 & 
1050.09 & 0.53 & \bf{1027.08} & 
1.12 & 2.24\\SCA8-6 & 996.46 & 0.48 & 
996.46 & 0.48 & \bf{971.82} & 
2.54 & 2.54\\SCA8-7 & 1083.63 & 0.56 & 
1083.63 & 0.60 & \bf{1051.28} & 
3.08 & 3.08\\SCA8-8 & 1083.61 & 0.42 & 
1089.20 & 0.46 & \bf{1071.18} & 
1.16 & 1.68\\SCA8-9 & 1072.10 & 0.41 & 
1072.10 & 0.46 & \bf{1060.50} & 
1.09 & 1.09\\CON3-0 & 620.76 & 0.60 & 
622.80 & 0.58 & \bf{616.52} & 
0.69 & 1.02\\CON3-1 & 560.75 & 0.46 & 
561.63 & 0.49 & \bf{554.47} & 
1.13 & 1.29\\CON3-2 & 521.38 & 0.64 & 
521.38 & 0.65 & \bf{518.00} & 
0.65 & 0.65\\CON3-3 & 594.31 & 1.06 & 
597.88 & 0.69 & \bf{591.19} & 
0.53 & 1.13\\CON3-4 & 591.43 & 0.67 & 
593.10 & 0.67 & \bf{588.79} & 
0.45 & 0.73\\CON3-5 & \bf{563.70} & 0.50 & 
565.55 & 0.51 & 563.70 & 0.00
 & 0.33\\CON3-6 & 502.16 & 0.70 & 
502.90 & 0.56 & \bf{499.05} & 
0.62 & 0.77\\CON3-7 & 578.22 & 0.92 & 
578.22 & 0.61 & \bf{576.48} & 
0.30 & 0.30\\CON3-8 & 523.14 & 0.48 & 
530.31 & 0.50 & \bf{523.05} & 
0.02 & 1.39\\CON3-9 & 590.16 & 0.79 & 
590.85 & 0.67 & \bf{578.24} & 
2.06 & 2.18\\CON8-0 & 892.92 & 0.93 & 
892.92 & 0.60 & \bf{857.17} & 
4.17 & 4.17\\CON8-1 & 749.07 & 0.52 & 
753.08 & 0.52 & \bf{740.85} & 
1.11 & 1.65\\CON8-2 & 729.59 & 0.46 & 
732.38 & 0.53 & \bf{712.89} & 
2.34 & 2.73\\CON8-3 & 835.15 & 0.46 & 
835.15 & 0.57 & \bf{811.07} & 
2.97 & 2.97\\CON8-4 & 799.29 & 0.67 & 
799.29 & 0.64 & \bf{772.25} & 
3.50 & 3.50\\CON8-5 & 758.12 & 0.42 & 
758.12 & 0.57 & \bf{754.88} & 
0.43 & 0.43\\CON8-6 & 697.10 & 1.00 & 
697.10 & 0.60 & \bf{678.92} & 
2.68 & 2.68\\CON8-7 & 821.80 & 0.39 & 
831.28 & 0.46 & \bf{811.96} & 
1.21 & 2.38\\CON8-8 & 798.93 & 0.61 & 
798.93 & 0.76 & \bf{767.53} & 
4.09 & 4.09\\CON8-9 & 835.53 & 0.70 & 
840.40 & 0.85 & \bf{809.00} & 
3.28 & 3.88\\\bf{PROM.} & 
\bf{771.73} & \bf{0.59} & \bf{773.79} & \bf{0.58} & \bf{758.54} & \bf{1.59} & \bf{1.86}\\[1ex]\hline
\end{tabular}
\label{table:nonlin}
\end{table} \clearpage
\begin{table}[ht]
\caption{Resultados de la ejecución de la metaheurística SCA, utilizando instancias de SalhiNagy con la configuración -n 50.0 -b 10 -y 0.1}
\centering
\small
\begin{tabular}{c c c c c c c c}
\hline\hline
Instancia & Costo mínimo & Tiempo(seg.) & Costo promedio & Tiempo promedio(seg.) & CME & \%G & \%GP \\ [0.5ex]
\hline
CMT1X & 472.37 & 3.50 & 
473.33 & 2.94 & \bf{470.48} & 
0.40 & 0.61\\CMT1Y & 472.87 & 2.58 & 
474.63 & 1.58 & \bf{470.48} & 
0.51 & 0.88\\CMT2X & 704.15 & 15.88 & 
708.62 & 12.10 & \bf{682.39} & 
3.19 & 3.84\\CMT2Y & 704.57 & 16.85 & 
705.94 & 17.72 & \bf{682.39} & 
3.25 & 3.45\\CMT3X & 726.32 & 29.41 & 
737.06 & 31.96 & \bf{719.06} & 
1.01 & 2.50\\CMT3Y & 737.83 & 42.83 & 
741.20 & 37.65 & \bf{719.06} & 
2.61 & 3.08\\CMT4X & 900.80 & 237.52 & 
906.71 & 234.19 & \bf{854.21} & 
5.45 & 6.15\\CMT4Y & 881.86 & 402.81 & 
898.65 & 231.40 & \bf{852.46} & 
3.45 & 5.42\\CMT5X & 1100.80 & 1063.01 & 
1100.80 & 1063.01 & \bf{1030.56} & 
6.82 & 6.82\\CMT5Y & 100000 & 0 & 
nan & nan & \bf{1031.69} & 
9592.83 & \bf{nan}\\CMT11X & 890.64 & 44.83 & 
900.71 & 47.67 & \bf{831.09} & 
7.17 & 8.38\\CMT11Y & 906.29911.74 & 36.56 & 
913.19 & 34.48 & \bf{829.85} & 
9.21 & 10.04\\CMT12X & 685.32 & 38.33 & 
686.50 & 58.66 & \bf{658.83} & 
4.02 & 4.20\\CMT12Y & 682.79 & 82.46 & 
689.40 & 47.61 & \bf{660.47} & 
3.38 & 4.38\\\bf{PROM.} & 
\bf{7847.62} & \bf{144.04} & \bf{nan} & \bf{nan} & \bf{749.50} & \bf{688.81} & \bf{nan}\\[1ex]\hline
\end{tabular}
\label{table:nonlin}
\end{table} \clearpage
\begin{table}[ht]
\caption{Resultados de la ejecución de la metaheurística IGA, utilizando instancias de SalhiNagy con la configuración -n 200 -p 40 -cprob 20.0 -mprob 90.0}
\centering
\small
\begin{tabular}{c c c c c c c c}
\hline\hline
Instancia & Costo mínimo & Tiempo(seg.) & Costo promedio & Tiempo promedio(seg.) & CME & \%G & \%GP \\ [0.5ex]
\hline
CMT1X & 481.31 & 0.41 & 
483.50 & 0.43 & \bf{470.48} & 
2.30 & 2.77\\CMT1Y & 478.97 & 0.42 & 
482.35 & 0.54 & \bf{470.48} & 
1.80 & 2.52\\CMT2X & 707.49 & 1.47 & 
712.14 & 0.91 & \bf{682.39} & 
3.68 & 4.36\\CMT2Y & 715.13 & 1.62 & 
717.91 & 1.24 & \bf{682.39} & 
4.80 & 5.21\\CMT3X & 732.32 & 2.43 & 
742.63 & 2.53 & \bf{719.06} & 
1.84 & 3.28\\CMT3Y & 746.78 & 2.55 & 
751.93 & 2.50 & \bf{719.06} & 
3.86 & 4.57\\CMT4X & 905.34 & 8.70 & 
916.65 & 7.93 & \bf{854.21} & 
5.99 & 7.31\\CMT4Y & 896.37 & 7.45 & 
916.34 & 7.75 & \bf{852.46} & 
5.15 & 7.49\\CMT5X & 1110.18 & 18.76 & 
1122.23 & 16.62 & \bf{1030.56} & 
7.73 & 8.90\\CMT5Y & 1118.26 & 15.70 & 
1126.79 & 16.50 & \bf{1031.69} & 
8.39 & 9.22\\CMT11X & 892.56 & 4.73 & 
917.74 & 4.97 & \bf{831.09} & 
7.40 & 10.43\\CMT11Y & 884.44 & 5.34 & 
899.85 & 5.10 & \bf{829.85} & 
6.58 & 8.44\\CMT12X & 680.21 & 2.59 & 
685.29 & 2.77 & \bf{658.83} & 
3.25 & 4.02\\CMT12Y & 677.26 & 3.40 & 
682.65 & 2.92 & \bf{660.47} & 
2.54 & 3.36\\\bf{PROM.} & 
\bf{787.62} & \bf{5.40} & \bf{797.00} & \bf{5.19} & \bf{749.50} & \bf{4.66} & \bf{5.85}\\[1ex]\hline
\end{tabular}
\label{table:nonlin}
\end{table} \clearpage
\begin{table}[ht]
\caption{Resultados de la ejecución de la metaheurística IGA, utilizando instancias de Dethloff con la configuración -n 200 -p 40 -cprob 20.0 -mprob 100.0}
\centering
\small
\begin{tabular}{c c c c c c c c}
\hline\hline
Instancia & Costo mínimo & Tiempo(seg.) & Costo promedio & Tiempo promedio(seg.) & CME & \%G & \%GP \\ [0.5ex]
\hline
SCA3-0 & 640.55 & 0.54 & 
640.55 & 0.60 & \bf{635.62} & 
0.78 & 0.78\\SCA3-1 & 701.53 & 0.60 & 
701.53 & 0.64 & \bf{697.84} & 
0.53 & 0.53\\SCA3-2 & 661.13 & 0.53 & 
665.24 & 0.65 & \bf{659.34} & 
0.27 & 0.89\\SCA3-3 & 686.62 & 0.44 & 
689.15 & 0.54 & \bf{680.04} & 
0.97 & 1.34\\SCA3-4 & \bf{690.50} & 0.55 & 
690.50 & 0.52 & 690.50 & 0.00
 & 0.00\\
SCA3-5 & 673.56 & 0.58 & 
675.57 & 0.52 & \bf{659.90} & 
2.07 & 2.37\\SCA3-6 & 652.94 & 0.90 & 
652.94 & 0.65 & \bf{651.09} & 
0.28 & 0.28\\SCA3-7 & 666.15 & 0.41 & 
666.15 & 0.44 & \bf{659.17} & 
1.06 & 1.06\\SCA3-8 & 719.77 & 0.51 & 
722.03 & 0.50 & \bf{719.47} & 
0.04 & 0.36\\SCA3-9 & 684.44 & 0.47 & 
684.82 & 0.48 & \bf{681.00} & 
0.51 & 0.56\\SCA8-0 & 988.37 & 0.45 & 
988.37 & 0.61 & \bf{961.50} & 
2.79 & 2.79\\SCA8-1 & 1082.31 & 0.48 & 
1088.02 & 0.58 & \bf{1049.65} & 
3.11 & 3.66\\SCA8-2 & 1051.95 & 0.65 & 
1051.95 & 0.59 & \bf{1039.64} & 
1.18 & 1.18\\SCA8-3 & 1009.99 & 0.68 & 
1009.99 & 0.52 & \bf{983.34} & 
2.71 & 2.71\\SCA8-4 & 1078.23 & 0.62 & 
1085.25 & 0.54 & \bf{1065.49} & 
1.20 & 1.85\\SCA8-5 & 1066.16 & 0.78 & 
1066.16 & 0.91 & \bf{1027.08} & 
3.80 & 3.80\\SCA8-6 & 983.38 & 0.95 & 
988.01 & 0.58 & \bf{971.82} & 
1.19 & 1.67\\SCA8-7 & 1091.26 & 0.58 & 
1092.33 & 0.48 & \bf{1051.28} & 
3.80 & 3.90\\SCA8-8 & 1093.44 & 0.65 & 
1093.44 & 0.72 & \bf{1071.18} & 
2.08 & 2.08\\SCA8-9 & 1078.33 & 0.40 & 
1082.97 & 0.41 & \bf{1060.50} & 
1.68 & 2.12\\CON3-0 & 617.59 & 0.42 & 
618.38 & 0.55 & \bf{616.52} & 
0.17 & 0.30\\CON3-1 & 556.04 & 0.49 & 
558.96 & 0.51 & \bf{554.47} & 
0.28 & 0.81\\CON3-2 & 521.38 & 0.54 & 
521.38 & 0.58 & \bf{518.00} & 
0.65 & 0.65\\CON3-3 & 592.43 & 0.68 & 
592.80 & 0.55 & \bf{591.19} & 
0.21 & 0.27\\CON3-4 & 591.43 & 0.95 & 
597.85 & 0.84 & \bf{588.79} & 
0.45 & 1.54\\CON3-5 & 567.94 & 0.94 & 
568.14 & 0.68 & \bf{563.70} & 
0.75 & 0.79\\CON3-6 & 505.01 & 0.49 & 
505.07 & 0.49 & \bf{499.05} & 
1.19 & 1.21\\CON3-7 & 582.14 & 0.42 & 
583.11 & 0.55 & \bf{576.48} & 
0.98 & 1.15\\CON3-8 & 524.59 & 0.54 & 
524.59 & 0.52 & \bf{523.05} & 
0.29 & 0.29\\CON3-9 & 582.79 & 0.69 & 
586.67 & 0.64 & \bf{578.24} & 
0.79 & 1.46\\CON8-0 & 873.42 & 0.44 & 
873.42 & 0.46 & \bf{857.17} & 
1.90 & 1.90\\CON8-1 & 742.28 & 0.49 & 
742.28 & 0.59 & \bf{740.85} & 
0.19 & 0.19\\CON8-2 & 724.59 & 0.69 & 
729.14 & 0.67 & \bf{712.89} & 
1.64 & 2.28\\CON8-3 & 837.82 & 0.50 & 
848.42 & 0.47 & \bf{811.07} & 
3.30 & 4.61\\CON8-4 & 781.64 & 0.42 & 
781.64 & 0.49 & \bf{772.25} & 
1.22 & 1.22\\CON8-5 & 777.99 & 0.68 & 
777.99 & 0.54 & \bf{754.88} & 
3.06 & 3.06\\CON8-6 & 698.75 & 0.60 & 
698.75 & 0.58 & \bf{678.92} & 
2.92 & 2.92\\CON8-7 & 837.10 & 0.48 & 
837.11 & 0.62 & \bf{811.96} & 
3.10 & 3.10\\CON8-8 & 796.91 & 0.53 & 
796.91 & 0.60 & \bf{767.53} & 
3.83 & 3.83\\CON8-9 & 817.35 & 0.54 & 
830.18 & 0.54 & \bf{809.00} & 
1.03 & 2.62\\\bf{PROM.} & 
\bf{770.74} & \bf{0.58} & \bf{772.69} & \bf{0.57} & \bf{758.54} & \bf{1.45} & \bf{1.70}\\[1ex]\hline
\end{tabular}
\label{table:nonlin}
\end{table} \clearpage
\begin{table}[ht]
\caption{Resultados de la ejecución de la metaheurística IGA, utilizando instancias de SalhiNagy con la configuración -n 200 -p 40 -cprob 20.0 -mprob 100.0}
\centering
\small
\begin{tabular}{c c c c c c c c}
\hline\hline
Instancia & Costo mínimo & Tiempo(seg.) & Costo promedio & Tiempo promedio(seg.) & CME & \%G & \%GP \\ [0.5ex]
\hline
CMT1X & 476.32 & 0.36 & 
479.44 & 0.48 & \bf{470.48} & 
1.24 & 1.91\\CMT1Y & 478.23 & 0.40 & 
486.09 & 0.42 & \bf{470.48} & 
1.65 & 3.32\\CMT2X & 709.60 & 1.14 & 
713.96 & 1.04 & \bf{682.39} & 
3.99 & 4.63\\CMT2Y & 711.15 & 1.08 & 
713.41 & 1.17 & \bf{682.39} & 
4.21 & 4.55\\CMT3X & 738.00 & 2.85 & 
745.59 & 2.83 & \bf{719.06} & 
2.63 & 3.69\\CMT3Y & 737.55 & 2.63 & 
746.13 & 2.73 & \bf{719.06} & 
2.57 & 3.76\\CMT4X & 904.10 & 7.44 & 
918.17 & 7.35 & \bf{854.21} & 
5.84 & 7.49\\CMT4Y & 919.68 & 8.41 & 
930.64 & 8.48 & \bf{852.46} & 
7.89 & 9.17\\CMT5X & 1113.94 & 14.95 & 
1126.31 & 15.29 & \bf{1030.56} & 
8.09 & 9.29\\CMT5Y & 1102.37 & 15.78 & 
1116.16 & 16.40 & \bf{1031.69} & 
6.85 & 8.19\\CMT11X & 885.87 & 4.23 & 
900.50 & 4.68 & \bf{831.09} & 
6.59 & 8.35\\CMT11Y & 894.93 & 5.14 & 
906.53 & 5.29 & \bf{829.85} & 
7.84 & 9.24\\CMT12X & 680.69 & 2.45 & 
682.28 & 2.52 & \bf{658.83} & 
3.32 & 3.56\\CMT12Y & 676.37 & 2.43 & 
680.77 & 2.48 & \bf{660.47} & 
2.41 & 3.07\\\bf{PROM.} & 
\bf{787.77} & \bf{4.95} & \bf{796.14} & \bf{5.08} & \bf{749.50} & \bf{4.65} & \bf{5.73}\\[1ex]\hline
\end{tabular}
\label{table:nonlin}
\end{table} \clearpage
\begin{table}[ht]
\caption{Resultados de la ejecución de la metaheurística IGA, utilizando instancias de Dethloff con la configuración -n 200 -p 40 -cprob 30.0 -mprob 10.0}
\centering
\small
\begin{tabular}{c c c c c c c c}
\hline\hline
Instancia & Costo mínimo & Tiempo(seg.) & Costo promedio & Tiempo promedio(seg.) & CME & \%G & \%GP \\ [0.5ex]
\hline
SCA3-0 & 640.55 & 0.47 & 
641.12 & 0.59 & \bf{635.62} & 
0.78 & 0.87\\SCA3-1 & 700.50 & 0.42 & 
707.70 & 0.49 & \bf{697.84} & 
0.38 & 1.41\\SCA3-2 & 664.92 & 0.52 & 
665.85 & 0.56 & \bf{659.34} & 
0.85 & 0.99\\SCA3-3 & 681.35 & 0.61 & 
684.30 & 0.69 & \bf{680.04} & 
0.19 & 0.63\\SCA3-4 & \bf{690.50} & 0.64 & 
691.02 & 0.48 & 690.50 & 0.00
 & 0.08\\SCA3-5 & 673.56 & 0.43 & 
677.05 & 0.51 & \bf{659.90} & 
2.07 & 2.60\\SCA3-6 & 652.94 & 0.43 & 
655.64 & 0.46 & \bf{651.09} & 
0.28 & 0.70\\SCA3-7 & 666.15 & 0.63 & 
666.15 & 0.53 & \bf{659.17} & 
1.06 & 1.06\\SCA3-8 & 723.99 & 0.48 & 
724.21 & 0.51 & \bf{719.47} & 
0.63 & 0.66\\SCA3-9 & 685.19 & 0.55 & 
685.36 & 0.61 & \bf{681.00} & 
0.62 & 0.64\\SCA8-0 & 1004.90 & 0.48 & 
1004.90 & 0.51 & \bf{961.50} & 
4.51 & 4.51\\SCA8-1 & 1067.92 & 0.57 & 
1077.75 & 0.48 & \bf{1049.65} & 
1.74 & 2.68\\SCA8-2 & 1051.21 & 0.76 & 
1052.60 & 0.60 & \bf{1039.64} & 
1.11 & 1.25\\SCA8-3 & 1023.96 & 0.56 & 
1024.55 & 0.52 & \bf{983.34} & 
4.13 & 4.19\\SCA8-4 & 1104.48 & 0.57 & 
1105.78 & 0.65 & \bf{1065.49} & 
3.66 & 3.78\\SCA8-5 & 1036.97 & 0.56 & 
1049.16 & 0.56 & \bf{1027.08} & 
0.96 & 2.15\\SCA8-6 & 990.40 & 0.46 & 
992.11 & 0.66 & \bf{971.82} & 
1.91 & 2.09\\SCA8-7 & 1071.53 & 0.56 & 
1071.53 & 0.60 & \bf{1051.28} & 
1.93 & 1.93\\SCA8-8 & \bf{1071.18} & 0.68 & 
1071.18 & 0.59 & 1071.18 & 0.00
 & 0.00\\
SCA8-9 & 1081.83 & 0.72 & 
1084.13 & 0.49 & \bf{1060.50} & 
2.01 & 2.23\\CON3-0 & 625.35 & 0.47 & 
628.07 & 0.46 & \bf{616.52} & 
1.43 & 1.87\\CON3-1 & 560.75 & 0.50 & 
562.98 & 0.52 & \bf{554.47} & 
1.13 & 1.53\\CON3-2 & 521.38 & 0.60 & 
521.38 & 0.59 & \bf{518.00} & 
0.65 & 0.65\\CON3-3 & 591.48 & 0.48 & 
598.42 & 0.53 & \bf{591.19} & 
0.05 & 1.22\\CON3-4 & 592.58 & 0.48 & 
603.85 & 0.43 & \bf{588.79} & 
0.64 & 2.56\\CON3-5 & 567.94 & 0.51 & 
567.94 & 0.50 & \bf{563.70} & 
0.75 & 0.75\\CON3-6 & 504.15 & 0.58 & 
507.05 & 0.57 & \bf{499.05} & 
1.02 & 1.60\\CON3-7 & 578.41 & 0.71 & 
579.39 & 0.55 & \bf{576.48} & 
0.33 & 0.50\\CON3-8 & 524.30 & 0.49 & 
531.17 & 0.51 & \bf{523.05} & 
0.24 & 1.55\\CON3-9 & 588.18 & 0.52 & 
589.16 & 0.56 & \bf{578.24} & 
1.72 & 1.89\\CON8-0 & 878.44 & 0.84 & 
888.09 & 0.57 & \bf{857.17} & 
2.48 & 3.61\\CON8-1 & 759.12 & 0.62 & 
759.12 & 0.68 & \bf{740.85} & 
2.47 & 2.47\\CON8-2 & 724.68 & 0.49 & 
725.66 & 0.66 & \bf{712.89} & 
1.65 & 1.79\\CON8-3 & 831.86 & 0.58 & 
836.44 & 0.56 & \bf{811.07} & 
2.56 & 3.13\\CON8-4 & 791.38 & 0.74 & 
804.38 & 0.51 & \bf{772.25} & 
2.48 & 4.16\\CON8-5 & 763.90 & 0.46 & 
769.97 & 0.52 & \bf{754.88} & 
1.19 & 2.00\\CON8-6 & 699.44 & 0.45 & 
699.44 & 0.56 & \bf{678.92} & 
3.02 & 3.02\\CON8-7 & 826.17 & 0.39 & 
830.74 & 0.46 & \bf{811.96} & 
1.75 & 2.31\\CON8-8 & 779.96 & 0.50 & 
782.46 & 0.57 & \bf{767.53} & 
1.62 & 1.94\\CON8-9 & 830.61 & 0.60 & 
830.61 & 0.56 & \bf{809.00} & 
2.67 & 2.67\\\bf{PROM.} & 
\bf{770.60} & \bf{0.55} & \bf{773.71} & \bf{0.55} & \bf{758.54} & \bf{1.47} & \bf{1.89}\\[1ex]\hline
\end{tabular}
\label{table:nonlin}
\end{table} \clearpage
\begin{table}[ht]
\caption{Resultados de la ejecución de la metaheurística IGA, utilizando instancias de SalhiNagy con la configuración -n 200 -p 40 -cprob 30.0 -mprob 10.0}
\centering
\small
\begin{tabular}{c c c c c c c c}
\hline\hline
Instancia & Costo mínimo & Tiempo(seg.) & Costo promedio & Tiempo promedio(seg.) & CME & \%G & \%GP \\ [0.5ex]
\hline
CMT1X & 481.93 & 0.48 & 
481.93 & 0.50 & \bf{470.48} & 
2.43 & 2.43\\CMT1Y & 482.16 & 0.36 & 
482.88 & 0.35 & \bf{470.48} & 
2.48 & 2.63\\CMT2X & 705.94 & 1.28 & 
711.97 & 1.20 & \bf{682.39} & 
3.45 & 4.33\\CMT2Y & 702.50 & 1.04 & 
711.60 & 1.15 & \bf{682.39} & 
2.95 & 4.28\\CMT3X & 732.77 & 2.60 & 
747.14 & 2.76 & \bf{719.06} & 
1.91 & 3.90\\CMT3Y & 753.71 & 3.01 & 
753.96 & 2.75 & \bf{719.06} & 
4.82 & 4.85\\CMT4X & 904.64 & 9.36 & 
917.19 & 8.19 & \bf{854.21} & 
5.90 & 7.37\\CMT4Y & 884.38 & 7.85 & 
904.54 & 7.68 & \bf{852.46} & 
3.74 & 6.11\\CMT5X & 1090.68 & 14.92 & 
1108.87 & 15.34 & \bf{1030.56} & 
5.83 & 7.60\\CMT5Y & 1083.20 & 15.73 & 
1117.58 & 16.20 & \bf{1031.69} & 
4.99 & 8.33\\CMT11X & 908.21 & 5.11 & 
918.48 & 4.65 & \bf{831.09} & 
9.28 & 10.52\\CMT11Y & 891.12 & 5.24 & 
897.35 & 5.20 & \bf{829.85} & 
7.38 & 8.13\\CMT12X & 676.16 & 2.45 & 
680.41 & 2.77 & \bf{658.83} & 
2.63 & 3.28\\CMT12Y & 670.72 & 3.01 & 
680.28 & 2.72 & \bf{660.47} & 
1.55 & 3.00\\\bf{PROM.} & 
\bf{783.44} & \bf{5.17} & \bf{793.87} & \bf{5.10} & \bf{749.50} & \bf{4.24} & \bf{5.48}\\[1ex]\hline
\end{tabular}
\label{table:nonlin}
\end{table} \clearpage
\begin{table}[ht]
\caption{Resultados de la ejecución de la metaheurística IGA, utilizando instancias de Dethloff con la configuración -n 200 -p 40 -cprob 30.0 -mprob 20.0}
\centering
\small
\begin{tabular}{c c c c c c c c}
\hline\hline
Instancia & Costo mínimo & Tiempo(seg.) & Costo promedio & Tiempo promedio(seg.) & CME & \%G & \%GP \\ [0.5ex]
\hline
SCA3-0 & 640.55 & 0.45 & 
641.39 & 0.55 & \bf{635.62} & 
0.78 & 0.91\\SCA3-1 & 700.50 & 0.58 & 
701.27 & 0.60 & \bf{697.84} & 
0.38 & 0.49\\SCA3-2 & 666.85 & 0.40 & 
667.78 & 0.48 & \bf{659.34} & 
1.14 & 1.28\\SCA3-3 & \bf{680.04} & 0.66 & 
681.29 & 0.52 & 680.04 & 0.00
 & 0.18\\SCA3-4 & \bf{690.50} & 0.44 & 
690.50 & 0.55 & 690.50 & 0.00
 & 0.00\\
SCA3-5 & 673.56 & 0.48 & 
673.56 & 0.56 & \bf{659.90} & 
2.07 & 2.07\\SCA3-6 & 652.94 & 0.64 & 
652.94 & 0.55 & \bf{651.09} & 
0.28 & 0.28\\SCA3-7 & 666.15 & 0.43 & 
668.60 & 0.46 & \bf{659.17} & 
1.06 & 1.43\\SCA3-8 & 719.77 & 0.57 & 
719.77 & 0.58 & \bf{719.47} & 
0.04 & 0.04\\SCA3-9 & \bf{681.00} & 0.43 & 
684.00 & 0.42 & 681.00 & 0.00
 & 0.44\\SCA8-0 & 988.26 & 0.40 & 
993.84 & 0.56 & \bf{961.50} & 
2.78 & 3.36\\SCA8-1 & 1063.24 & 0.60 & 
1063.24 & 0.60 & \bf{1049.65} & 
1.29 & 1.29\\SCA8-2 & 1051.21 & 0.50 & 
1051.21 & 0.51 & \bf{1039.64} & 
1.11 & 1.11\\SCA8-3 & 1021.95 & 0.62 & 
1023.50 & 0.68 & \bf{983.34} & 
3.93 & 4.08\\SCA8-4 & 1079.09 & 0.59 & 
1079.09 & 0.66 & \bf{1065.49} & 
1.28 & 1.28\\SCA8-5 & 1062.05 & 0.74 & 
1062.05 & 0.62 & \bf{1027.08} & 
3.40 & 3.40\\SCA8-6 & 987.65 & 0.58 & 
988.62 & 0.60 & \bf{971.82} & 
1.63 & 1.73\\SCA8-7 & 1085.63 & 0.43 & 
1085.63 & 0.43 & \bf{1051.28} & 
3.27 & 3.27\\SCA8-8 & 1082.34 & 0.71 & 
1082.34 & 0.65 & \bf{1071.18} & 
1.04 & 1.04\\SCA8-9 & 1079.78 & 0.57 & 
1099.70 & 0.47 & \bf{1060.50} & 
1.82 & 3.70\\CON3-0 & 623.15 & 0.58 & 
625.67 & 0.62 & \bf{616.52} & 
1.08 & 1.48\\CON3-1 & 560.75 & 0.54 & 
560.75 & 0.60 & \bf{554.47} & 
1.13 & 1.13\\CON3-2 & 521.38 & 0.58 & 
521.38 & 0.66 & \bf{518.00} & 
0.65 & 0.65\\CON3-3 & 602.16 & 0.50 & 
604.40 & 0.48 & \bf{591.19} & 
1.86 & 2.23\\CON3-4 & 593.78 & 0.48 & 
594.51 & 0.51 & \bf{588.79} & 
0.85 & 0.97\\CON3-5 & \bf{563.70} & 0.52 & 
563.70 & 0.51 & 563.70 & 0.00
 & 0.00\\
CON3-6 & 505.14 & 0.47 & 
505.27 & 0.57 & \bf{499.05} & 
1.22 & 1.25\\CON3-7 & \bf{576.48} & 0.74 & 
576.84 & 0.61 & 576.48 & 0.00
 & 0.06\\CON3-8 & 525.47 & 0.61 & 
534.66 & 0.57 & \bf{523.05} & 
0.46 & 2.22\\CON3-9 & 588.28 & 0.92 & 
590.95 & 0.65 & \bf{578.24} & 
1.74 & 2.20\\CON8-0 & 883.41 & 0.42 & 
886.64 & 0.48 & \bf{857.17} & 
3.06 & 3.44\\CON8-1 & 761.55 & 0.59 & 
764.47 & 0.71 & \bf{740.85} & 
2.79 & 3.19\\CON8-2 & 731.70 & 0.50 & 
737.22 & 0.51 & \bf{712.89} & 
2.64 & 3.41\\CON8-3 & 820.43 & 0.50 & 
820.43 & 0.53 & \bf{811.07} & 
1.15 & 1.15\\CON8-4 & 789.10 & 0.45 & 
790.89 & 0.46 & \bf{772.25} & 
2.18 & 2.41\\CON8-5 & 768.19 & 0.73 & 
774.45 & 0.57 & \bf{754.88} & 
1.76 & 2.59\\CON8-6 & 698.86 & 0.60 & 
699.97 & 0.62 & \bf{678.92} & 
2.94 & 3.10\\CON8-7 & 815.54 & 0.47 & 
820.71 & 0.53 & \bf{811.96} & 
0.44 & 1.08\\CON8-8 & 782.67 & 0.50 & 
791.93 & 0.58 & \bf{767.53} & 
1.97 & 3.18\\CON8-9 & 821.74 & 0.65 & 
821.88 & 0.60 & \bf{809.00} & 
1.57 & 1.59\\\bf{PROM.} & 
\bf{770.16} & \bf{0.55} & \bf{772.43} & \bf{0.56} & \bf{758.54} & \bf{1.42} & \bf{1.72}\\[1ex]\hline
\end{tabular}
\label{table:nonlin}
\end{table} \clearpage
\begin{table}[ht]
\caption{Resultados de la ejecución de la metaheurística IGA, utilizando instancias de SalhiNagy con la configuración -n 200 -p 40 -cprob 30.0 -mprob 20.0}
\centering
\small
\begin{tabular}{c c c c c c c c}
\hline\hline
Instancia & Costo mínimo & Tiempo(seg.) & Costo promedio & Tiempo promedio(seg.) & CME & \%G & \%GP \\ [0.5ex]
\hline
CMT1X & 481.77 & 0.61 & 
484.27 & 0.43 & \bf{470.48} & 
2.40 & 2.93\\CMT1Y & 478.97 & 0.30 & 
480.87 & 0.39 & \bf{470.48} & 
1.80 & 2.21\\CMT2X & 704.08 & 1.10 & 
716.37 & 1.17 & \bf{682.39} & 
3.18 & 4.98\\CMT2Y & 697.42 & 1.04 & 
708.75 & 1.19 & \bf{682.39} & 
2.20 & 3.86\\CMT3X & 743.53 & 2.74 & 
746.76 & 2.71 & \bf{719.06} & 
3.40 & 3.85\\CMT3Y & 746.46 & 2.50 & 
751.34 & 2.48 & \bf{719.06} & 
3.81 & 4.49\\CMT4X & 907.27 & 8.02 & 
915.11 & 7.70 & \bf{854.21} & 
6.21 & 7.13\\CMT4Y & 908.25 & 7.21 & 
919.85 & 7.41 & \bf{852.46} & 
6.54 & 7.90\\CMT5X & 1106.95 & 14.86 & 
1123.77 & 15.18 & \bf{1030.56} & 
7.41 & 9.04\\CMT5Y & 1105.58 & 15.72 & 
1118.13 & 16.04 & \bf{1031.69} & 
7.16 & 8.38\\CMT11X & 908.93 & 5.01 & 
923.22 & 4.85 & \bf{831.09} & 
9.37 & 11.09\\CMT11Y & 891.22 & 5.40 & 
907.50 & 4.99 & \bf{829.85} & 
7.40 & 9.36\\CMT12X & 675.42 & 2.52 & 
681.12 & 2.73 & \bf{658.83} & 
2.52 & 3.38\\CMT12Y & 673.67 & 2.48 & 
680.49 & 2.69 & \bf{660.47} & 
2.00 & 3.03\\\bf{PROM.} & 
\bf{787.82} & \bf{4.96} & \bf{796.97} & \bf{5.00} & \bf{749.50} & \bf{4.67} & \bf{5.83}\\[1ex]\hline
\end{tabular}
\label{table:nonlin}
\end{table} \clearpage
\begin{table}[ht]
\caption{Resultados de la ejecución de la metaheurística IGA, utilizando instancias de Dethloff con la configuración -n 200 -p 40 -cprob 30.0 -mprob 30.0}
\centering
\small
\begin{tabular}{c c c c c c c c}
\hline\hline
Instancia & Costo mínimo & Tiempo(seg.) & Costo promedio & Tiempo promedio(seg.) & CME & \%G & \%GP \\ [0.5ex]
\hline
SCA3-0 & 640.55 & 0.68 & 
640.55 & 0.60 & \bf{635.62} & 
0.78 & 0.78\\SCA3-1 & \bf{697.84} & 0.50 & 
699.48 & 0.53 & 697.84 & 0.00
 & 0.24\\SCA3-2 & \bf{659.34} & 0.41 & 
662.40 & 0.49 & 659.34 & 0.00
 & 0.46\\SCA3-3 & 683.37 & 0.70 & 
683.37 & 0.57 & \bf{680.04} & 
0.49 & 0.49\\SCA3-4 & \bf{690.50} & 0.53 & 
690.50 & 0.47 & 690.50 & 0.00
 & 0.00\\
SCA3-5 & \bf{659.90} & 0.48 & 
668.73 & 0.49 & 659.90 & 0.00
 & 1.34\\SCA3-6 & 652.94 & 0.48 & 
656.01 & 0.56 & \bf{651.09} & 
0.28 & 0.76\\SCA3-7 & 671.47 & 0.89 & 
671.47 & 0.58 & \bf{659.17} & 
1.87 & 1.87\\SCA3-8 & \bf{719.47} & 0.73 & 
725.42 & 0.72 & 719.47 & 0.00
 & 0.83\\SCA3-9 & \bf{681.00} & 0.69 & 
681.00 & 0.46 & 681.00 & 0.00
 & 0.00\\
SCA8-0 & 975.50 & 0.59 & 
984.97 & 0.59 & \bf{961.50} & 
1.46 & 2.44\\SCA8-1 & 1065.36 & 0.40 & 
1074.72 & 0.53 & \bf{1049.65} & 
1.50 & 2.39\\SCA8-2 & 1054.69 & 0.97 & 
1055.16 & 0.67 & \bf{1039.64} & 
1.45 & 1.49\\SCA8-3 & 1014.10 & 0.58 & 
1016.03 & 0.57 & \bf{983.34} & 
3.13 & 3.32\\SCA8-4 & 1084.34 & 0.41 & 
1084.69 & 0.59 & \bf{1065.49} & 
1.77 & 1.80\\SCA8-5 & 1062.24 & 0.56 & 
1062.24 & 0.51 & \bf{1027.08} & 
3.42 & 3.42\\SCA8-6 & 992.25 & 0.57 & 
993.66 & 0.54 & \bf{971.82} & 
2.10 & 2.25\\SCA8-7 & 1075.53 & 0.62 & 
1075.53 & 0.56 & \bf{1051.28} & 
2.31 & 2.31\\SCA8-8 & 1089.55 & 0.84 & 
1089.55 & 0.62 & \bf{1071.18} & 
1.71 & 1.71\\SCA8-9 & 1096.59 & 0.55 & 
1098.03 & 0.61 & \bf{1060.50} & 
3.40 & 3.54\\CON3-0 & 620.76 & 0.59 & 
628.99 & 0.59 & \bf{616.52} & 
0.69 & 2.02\\CON3-1 & 560.75 & 0.74 & 
560.75 & 0.62 & \bf{554.47} & 
1.13 & 1.13\\CON3-2 & 521.38 & 0.60 & 
523.50 & 0.63 & \bf{518.00} & 
0.65 & 1.06\\CON3-3 & 592.44 & 0.72 & 
599.31 & 0.65 & \bf{591.19} & 
0.21 & 1.37\\CON3-4 & 593.78 & 0.54 & 
593.78 & 0.67 & \bf{588.79} & 
0.85 & 0.85\\CON3-5 & 564.88 & 0.53 & 
565.92 & 0.50 & \bf{563.70} & 
0.21 & 0.39\\CON3-6 & 504.15 & 0.58 & 
504.37 & 0.57 & \bf{499.05} & 
1.02 & 1.07\\CON3-7 & 578.41 & 0.61 & 
585.09 & 0.64 & \bf{576.48} & 
0.33 & 1.49\\CON3-8 & \bf{523.05} & 0.93 & 
526.17 & 0.76 & 523.05 & 0.00
 & 0.60\\CON3-9 & 590.17 & 0.46 & 
590.93 & 0.65 & \bf{578.24} & 
2.06 & 2.19\\CON8-0 & 865.86 & 0.50 & 
882.51 & 0.57 & \bf{857.17} & 
1.01 & 2.96\\CON8-1 & 749.07 & 0.53 & 
752.00 & 0.54 & \bf{740.85} & 
1.11 & 1.50\\CON8-2 & 722.22 & 0.52 & 
725.48 & 0.54 & \bf{712.89} & 
1.31 & 1.77\\CON8-3 & 831.60 & 0.85 & 
831.60 & 0.66 & \bf{811.07} & 
2.53 & 2.53\\CON8-4 & 802.36 & 0.72 & 
802.36 & 0.64 & \bf{772.25} & 
3.90 & 3.90\\CON8-5 & 769.34 & 0.45 & 
769.34 & 0.62 & \bf{754.88} & 
1.92 & 1.92\\CON8-6 & 698.80 & 0.47 & 
698.80 & 0.59 & \bf{678.92} & 
2.93 & 2.93\\CON8-7 & 825.69 & 0.44 & 
827.23 & 0.51 & \bf{811.96} & 
1.69 & 1.88\\CON8-8 & 787.39 & 0.92 & 
797.16 & 0.66 & \bf{767.53} & 
2.59 & 3.86\\CON8-9 & 826.17 & 0.54 & 
828.13 & 0.61 & \bf{809.00} & 
2.12 & 2.36\\\bf{PROM.} & 
\bf{769.87} & \bf{0.61} & \bf{772.67} & \bf{0.59} & \bf{758.54} & \bf{1.35} & \bf{1.73}\\[1ex]\hline
\end{tabular}
\label{table:nonlin}
\end{table} \clearpage
\begin{table}[ht]
\caption{Resultados de la ejecución de la metaheurística IGA, utilizando instancias de SalhiNagy con la configuración -n 200 -p 40 -cprob 30.0 -mprob 30.0}
\centering
\small
\begin{tabular}{c c c c c c c c}
\hline\hline
Instancia & Costo mínimo & Tiempo(seg.) & Costo promedio & Tiempo promedio(seg.) & CME & \%G & \%GP \\ [0.5ex]
\hline
CMT1X & 481.21 & 0.44 & 
485.92 & 0.48 & \bf{470.48} & 
2.28 & 3.28\\CMT1Y & 479.50 & 0.46 & 
483.35 & 0.43 & \bf{470.48} & 
1.92 & 2.73\\CMT2X & 704.77 & 1.09 & 
712.69 & 1.29 & \bf{682.39} & 
3.28 & 4.44\\CMT2Y & 706.03 & 1.17 & 
709.39 & 1.13 & \bf{682.39} & 
3.46 & 3.96\\CMT3X & 737.30 & 2.56 & 
746.31 & 2.54 & \bf{719.06} & 
2.54 & 3.79\\CMT3Y & 731.47 & 2.52 & 
740.26 & 2.54 & \bf{719.06} & 
1.73 & 2.95\\CMT4X & 914.53 & 7.21 & 
921.63 & 7.27 & \bf{854.21} & 
7.06 & 7.89\\CMT4Y & 903.41 & 8.26 & 
920.01 & 7.85 & \bf{852.46} & 
5.98 & 7.92\\CMT5X & 1108.11 & 15.86 & 
1127.72 & 15.62 & \bf{1030.56} & 
7.53 & 9.43\\CMT5Y & 1107.26 & 16.76 & 
1117.46 & 16.34 & \bf{1031.69} & 
7.32 & 8.31\\CMT11X & 920.29 & 4.81 & 
931.55 & 4.93 & \bf{831.09} & 
10.73 & 12.09\\CMT11Y & 860.99 & 5.18 & 
890.70 & 5.27 & \bf{829.85} & 
3.75 & 7.33\\CMT12X & 679.07 & 2.77 & 
684.77 & 2.91 & \bf{658.83} & 
3.07 & 3.94\\CMT12Y & 676.85 & 2.47 & 
680.25 & 2.65 & \bf{660.47} & 
2.48 & 3.00\\\bf{PROM.} & 
\bf{786.49} & \bf{5.11} & \bf{796.57} & \bf{5.09} & \bf{749.50} & \bf{4.51} & \bf{5.79}\\[1ex]\hline
\end{tabular}
\label{table:nonlin}
\end{table} \clearpage
\begin{table}[ht]
\caption{Resultados de la ejecución de la metaheurística IGA, utilizando instancias de Dethloff con la configuración -n 200 -p 40 -cprob 30.0 -mprob 40.0}
\centering
\small
\begin{tabular}{c c c c c c c c}
\hline\hline
Instancia & Costo mínimo & Tiempo(seg.) & Costo promedio & Tiempo promedio(seg.) & CME & \%G & \%GP \\ [0.5ex]
\hline
SCA3-0 & 640.55 & 0.70 & 
641.40 & 0.62 & \bf{635.62} & 
0.78 & 0.91\\SCA3-1 & \bf{697.84} & 0.64 & 
697.84 & 0.74 & 697.84 & 0.00
 & 0.00\\
SCA3-2 & 661.13 & 0.70 & 
661.13 & 0.55 & \bf{659.34} & 
0.27 & 0.27\\SCA3-3 & \bf{680.04} & 0.72 & 
681.86 & 0.59 & 680.04 & 0.00
 & 0.27\\SCA3-4 & \bf{690.50} & 0.45 & 
690.50 & 0.46 & 690.50 & 0.00
 & 0.00\\
SCA3-5 & 665.64 & 0.57 & 
665.64 & 0.53 & \bf{659.90} & 
0.87 & 0.87\\SCA3-6 & 652.94 & 0.60 & 
654.93 & 0.57 & \bf{651.09} & 
0.28 & 0.59\\SCA3-7 & 666.15 & 0.71 & 
666.15 & 0.68 & \bf{659.17} & 
1.06 & 1.06\\SCA3-8 & 723.99 & 0.58 & 
724.14 & 0.65 & \bf{719.47} & 
0.63 & 0.65\\SCA3-9 & \bf{681.00} & 0.50 & 
682.05 & 0.64 & 681.00 & 0.00
 & 0.15\\SCA8-0 & 1000.17 & 0.57 & 
1004.75 & 0.48 & \bf{961.50} & 
4.02 & 4.50\\SCA8-1 & 1088.42 & 0.79 & 
1091.81 & 0.57 & \bf{1049.65} & 
3.69 & 4.02\\SCA8-2 & 1051.21 & 0.49 & 
1052.58 & 0.68 & \bf{1039.64} & 
1.11 & 1.24\\SCA8-3 & 1011.61 & 0.45 & 
1017.10 & 0.52 & \bf{983.34} & 
2.87 & 3.43\\SCA8-4 & 1093.10 & 0.39 & 
1099.92 & 0.49 & \bf{1065.49} & 
2.59 & 3.23\\SCA8-5 & 1034.74 & 0.44 & 
1034.74 & 0.51 & \bf{1027.08} & 
0.75 & 0.75\\SCA8-6 & 994.09 & 0.43 & 
994.09 & 0.43 & \bf{971.82} & 
2.29 & 2.29\\SCA8-7 & 1093.05 & 0.46 & 
1099.95 & 0.55 & \bf{1051.28} & 
3.97 & 4.63\\SCA8-8 & 1084.41 & 0.55 & 
1084.41 & 0.51 & \bf{1071.18} & 
1.24 & 1.24\\SCA8-9 & 1078.33 & 0.39 & 
1078.33 & 0.51 & \bf{1060.50} & 
1.68 & 1.68\\CON3-0 & 620.76 & 0.63 & 
621.76 & 0.56 & \bf{616.52} & 
0.69 & 0.85\\CON3-1 & 556.04 & 0.66 & 
558.39 & 0.56 & \bf{554.47} & 
0.28 & 0.71\\CON3-2 & 523.46 & 0.51 & 
524.72 & 0.64 & \bf{518.00} & 
1.05 & 1.30\\CON3-3 & 592.98 & 0.65 & 
600.14 & 0.63 & \bf{591.19} & 
0.30 & 1.51\\CON3-4 & 595.25 & 0.54 & 
597.86 & 0.61 & \bf{588.79} & 
1.10 & 1.54\\CON3-5 & 564.88 & 0.52 & 
566.59 & 0.49 & \bf{563.70} & 
0.21 & 0.51\\CON3-6 & 504.20 & 0.48 & 
504.38 & 0.67 & \bf{499.05} & 
1.03 & 1.07\\CON3-7 & 589.93 & 0.52 & 
589.93 & 0.54 & \bf{576.48} & 
2.33 & 2.33\\CON3-8 & 534.28 & 0.58 & 
535.38 & 0.58 & \bf{523.05} & 
2.15 & 2.36\\CON3-9 & 582.79 & 0.54 & 
587.16 & 0.74 & \bf{578.24} & 
0.79 & 1.54\\CON8-0 & 878.92 & 0.52 & 
885.91 & 0.55 & \bf{857.17} & 
2.54 & 3.35\\CON8-1 & 755.11 & 0.59 & 
755.58 & 0.65 & \bf{740.85} & 
1.92 & 1.99\\CON8-2 & 727.14 & 0.47 & 
728.99 & 0.58 & \bf{712.89} & 
2.00 & 2.26\\CON8-3 & 844.13 & 0.44 & 
844.13 & 0.49 & \bf{811.07} & 
4.08 & 4.08\\CON8-4 & 798.88 & 0.74 & 
798.88 & 0.72 & \bf{772.25} & 
3.45 & 3.45\\CON8-5 & 766.88 & 0.48 & 
766.88 & 0.47 & \bf{754.88} & 
1.59 & 1.59\\CON8-6 & 696.12 & 0.76 & 
697.46 & 0.75 & \bf{678.92} & 
2.53 & 2.73\\CON8-7 & 827.93 & 0.82 & 
832.91 & 0.60 & \bf{811.96} & 
1.97 & 2.58\\CON8-8 & 784.67 & 0.46 & 
792.65 & 0.58 & \bf{767.53} & 
2.23 & 3.27\\CON8-9 & 836.84 & 0.52 & 
836.84 & 0.60 & \bf{809.00} & 
3.44 & 3.44\\\bf{PROM.} & 
\bf{771.75} & \bf{0.56} & \bf{773.75} & \bf{0.58} & \bf{758.54} & \bf{1.59} & \bf{1.86}\\[1ex]\hline
\end{tabular}
\label{table:nonlin}
\end{table} \clearpage
\begin{table}[ht]
\caption{Resultados de la ejecución de la metaheurística IGA, utilizando instancias de SalhiNagy con la configuración -n 200 -p 40 -cprob 30.0 -mprob 40.0}
\centering
\small
\begin{tabular}{c c c c c c c c}
\hline\hline
Instancia & Costo mínimo & Tiempo(seg.) & Costo promedio & Tiempo promedio(seg.) & CME & \%G & \%GP \\ [0.5ex]
\hline
CMT1X & 479.21 & 0.60 & 
479.21 & 0.50 & \bf{470.48} & 
1.86 & 1.86\\CMT1Y & 480.88 & 0.33 & 
485.00 & 0.41 & \bf{470.48} & 
2.21 & 3.09\\CMT2X & 690.61 & 1.12 & 
703.65 & 1.17 & \bf{682.39} & 
1.20 & 3.12\\CMT2Y & 717.94 & 1.04 & 
723.60 & 1.14 & \bf{682.39} & 
5.21 & 6.04\\CMT3X & 742.72 & 2.72 & 
744.00 & 2.84 & \bf{719.06} & 
3.29 & 3.47\\CMT3Y & 734.09 & 2.66 & 
746.13 & 2.65 & \bf{719.06} & 
2.09 & 3.77\\CMT4X & 911.05 & 7.73 & 
915.66 & 7.52 & \bf{854.21} & 
6.65 & 7.19\\CMT4Y & 897.09 & 7.38 & 
920.21 & 7.86 & \bf{852.46} & 
5.24 & 7.95\\CMT5X & 1111.75 & 15.54 & 
1124.00 & 15.55 & \bf{1030.56} & 
7.88 & 9.07\\CMT5Y & 1098.25 & 17.46 & 
1122.75 & 16.02 & \bf{1031.69} & 
6.45 & 8.83\\CMT11X & 869.20 & 4.78 & 
911.85 & 4.89 & \bf{831.09} & 
4.59 & 9.72\\CMT11Y & 882.17 & 4.72 & 
919.75 & 5.06 & \bf{829.85} & 
6.30 & 10.83\\CMT12X & 675.27 & 2.63 & 
675.53 & 2.79 & \bf{658.83} & 
2.50 & 2.53\\CMT12Y & 674.11 & 3.12 & 
677.04 & 2.91 & \bf{660.47} & 
2.07 & 2.51\\\bf{PROM.} & 
\bf{783.17} & \bf{5.13} & \bf{796.31} & \bf{5.09} & \bf{749.50} & \bf{4.11} & \bf{5.71}\\[1ex]\hline
\end{tabular}
\label{table:nonlin}
\end{table} \clearpage
\begin{table}[ht]
\caption{Resultados de la ejecución de la metaheurística IGA, utilizando instancias de Dethloff con la configuración -n 200 -p 40 -cprob 30.0 -mprob 50.0}
\centering
\small
\begin{tabular}{c c c c c c c c}
\hline\hline
Instancia & Costo mínimo & Tiempo(seg.) & Costo promedio & Tiempo promedio(seg.) & CME & \%G & \%GP \\ [0.5ex]
\hline
SCA3-0 & 640.55 & 0.70 & 
640.55 & 0.57 & \bf{635.62} & 
0.78 & 0.78\\SCA3-1 & 701.53 & 0.74 & 
701.69 & 0.65 & \bf{697.84} & 
0.53 & 0.55\\SCA3-2 & 664.21 & 0.43 & 
668.79 & 0.67 & \bf{659.34} & 
0.74 & 1.43\\SCA3-3 & 680.60 & 0.44 & 
680.79 & 0.51 & \bf{680.04} & 
0.08 & 0.11\\SCA3-4 & \bf{690.50} & 0.90 & 
690.50 & 0.57 & 690.50 & 0.00
 & 0.00\\
SCA3-5 & 683.30 & 0.49 & 
684.78 & 0.47 & \bf{659.90} & 
3.55 & 3.77\\SCA3-6 & 652.94 & 0.56 & 
655.69 & 0.47 & \bf{651.09} & 
0.28 & 0.71\\SCA3-7 & 666.15 & 0.66 & 
667.09 & 0.59 & \bf{659.17} & 
1.06 & 1.20\\SCA3-8 & 726.86 & 0.44 & 
727.60 & 0.46 & \bf{719.47} & 
1.03 & 1.13\\SCA3-9 & \bf{681.00} & 0.45 & 
682.03 & 0.65 & 681.00 & 0.00
 & 0.15\\SCA8-0 & 987.04 & 0.75 & 
996.17 & 0.66 & \bf{961.50} & 
2.66 & 3.61\\SCA8-1 & 1081.96 & 0.59 & 
1083.44 & 0.70 & \bf{1049.65} & 
3.08 & 3.22\\SCA8-2 & 1054.47 & 0.56 & 
1054.47 & 0.61 & \bf{1039.64} & 
1.43 & 1.43\\SCA8-3 & 1016.11 & 0.45 & 
1022.79 & 0.46 & \bf{983.34} & 
3.33 & 4.01\\SCA8-4 & 1088.73 & 0.43 & 
1098.79 & 0.51 & \bf{1065.49} & 
2.18 & 3.13\\SCA8-5 & 1049.62 & 0.42 & 
1049.89 & 0.45 & \bf{1027.08} & 
2.19 & 2.22\\SCA8-6 & 991.47 & 0.82 & 
994.25 & 0.68 & \bf{971.82} & 
2.02 & 2.31\\SCA8-7 & 1078.87 & 0.72 & 
1086.26 & 0.56 & \bf{1051.28} & 
2.62 & 3.33\\SCA8-8 & 1095.67 & 0.51 & 
1095.67 & 0.54 & \bf{1071.18} & 
2.29 & 2.29\\SCA8-9 & 1073.62 & 0.46 & 
1073.62 & 0.48 & \bf{1060.50} & 
1.24 & 1.24\\CON3-0 & 620.76 & 0.46 & 
624.08 & 0.52 & \bf{616.52} & 
0.69 & 1.23\\CON3-1 & 560.75 & 0.60 & 
560.75 & 0.65 & \bf{554.47} & 
1.13 & 1.13\\CON3-2 & 521.38 & 0.77 & 
522.32 & 0.64 & \bf{518.00} & 
0.65 & 0.83\\CON3-3 & 602.49 & 0.54 & 
605.11 & 0.63 & \bf{591.19} & 
1.91 & 2.35\\CON3-4 & 592.58 & 0.92 & 
595.70 & 0.70 & \bf{588.79} & 
0.64 & 1.17\\CON3-5 & 564.88 & 0.53 & 
564.88 & 0.65 & \bf{563.70} & 
0.21 & 0.21\\CON3-6 & 504.15 & 0.50 & 
504.15 & 0.62 & \bf{499.05} & 
1.02 & 1.02\\CON3-7 & 586.01 & 0.52 & 
586.01 & 0.57 & \bf{576.48} & 
1.65 & 1.65\\CON3-8 & 523.14 & 0.57 & 
525.93 & 0.58 & \bf{523.05} & 
0.02 & 0.55\\CON3-9 & 590.48 & 0.72 & 
593.71 & 0.58 & \bf{578.24} & 
2.12 & 2.68\\CON8-0 & 885.37 & 0.55 & 
885.37 & 0.56 & \bf{857.17} & 
3.29 & 3.29\\CON8-1 & 763.62 & 0.52 & 
763.62 & 0.59 & \bf{740.85} & 
3.07 & 3.07\\CON8-2 & 728.01 & 0.94 & 
728.01 & 0.71 & \bf{712.89} & 
2.12 & 2.12\\CON8-3 & 818.06 & 0.58 & 
818.06 & 0.56 & \bf{811.07} & 
0.86 & 0.86\\CON8-4 & 787.98 & 0.43 & 
789.57 & 0.69 & \bf{772.25} & 
2.04 & 2.24\\CON8-5 & 763.13 & 0.45 & 
768.64 & 0.52 & \bf{754.88} & 
1.09 & 1.82\\CON8-6 & 690.74 & 0.64 & 
690.74 & 0.69 & \bf{678.92} & 
1.74 & 1.74\\CON8-7 & 815.06 & 0.44 & 
815.06 & 0.46 & \bf{811.96} & 
0.38 & 0.38\\CON8-8 & 790.80 & 0.56 & 
790.80 & 0.61 & \bf{767.53} & 
3.03 & 3.03\\CON8-9 & 828.98 & 0.57 & 
835.28 & 0.64 & \bf{809.00} & 
2.47 & 3.25\\\bf{PROM.} & 
\bf{771.09} & \bf{0.58} & \bf{773.07} & \bf{0.59} & \bf{758.54} & \bf{1.53} & \bf{1.78}\\[1ex]\hline
\end{tabular}
\label{table:nonlin}
\end{table} \clearpage
\begin{table}[ht]
\caption{Resultados de la ejecución de la metaheurística IGA, utilizando instancias de SalhiNagy con la configuración -n 200 -p 40 -cprob 30.0 -mprob 50.0}
\centering
\small
\begin{tabular}{c c c c c c c c}
\hline\hline
Instancia & Costo mínimo & Tiempo(seg.) & Costo promedio & Tiempo promedio(seg.) & CME & \%G & \%GP \\ [0.5ex]
\hline
CMT1X & 481.62 & 0.86 & 
485.87 & 0.60 & \bf{470.48} & 
2.37 & 3.27\\CMT1Y & 485.29 & 0.72 & 
486.00 & 0.53 & \bf{470.48} & 
3.15 & 3.30\\CMT2X & 711.38 & 1.36 & 
715.11 & 1.28 & \bf{682.39} & 
4.25 & 4.79\\CMT2Y & 712.80 & 1.24 & 
715.83 & 1.13 & \bf{682.39} & 
4.46 & 4.90\\CMT3X & 744.46 & 2.88 & 
745.66 & 2.88 & \bf{719.06} & 
3.53 & 3.70\\CMT3Y & 746.02 & 2.72 & 
750.96 & 2.66 & \bf{719.06} & 
3.75 & 4.44\\CMT4X & 906.70 & 6.90 & 
913.35 & 7.58 & \bf{854.21} & 
6.14 & 6.92\\CMT4Y & 895.51 & 7.42 & 
909.55 & 7.41 & \bf{852.46} & 
5.05 & 6.70\\CMT5X & 1124.94 & 15.38 & 
1132.53 & 15.72 & \bf{1030.56} & 
9.16 & 9.89\\CMT5Y & 1110.99 & 15.70 & 
1127.67 & 16.27 & \bf{1031.69} & 
7.69 & 9.30\\CMT11X & 898.82 & 4.78 & 
919.89 & 4.95 & \bf{831.09} & 
8.15 & 10.68\\CMT11Y & 889.02 & 4.64 & 
897.90 & 5.16 & \bf{829.85} & 
7.13 & 8.20\\CMT12X & 674.30 & 3.07 & 
679.88 & 2.83 & \bf{658.83} & 
2.35 & 3.20\\CMT12Y & 674.74 & 2.59 & 
678.57 & 2.81 & \bf{660.47} & 
2.16 & 2.74\\\bf{PROM.} & 
\bf{789.76} & \bf{5.02} & \bf{797.06} & \bf{5.13} & \bf{749.50} & \bf{4.95} & \bf{5.86}\\[1ex]\hline
\end{tabular}
\label{table:nonlin}
\end{table} \clearpage
\begin{table}[ht]
\caption{Resultados de la ejecución de la metaheurística IGA, utilizando instancias de Dethloff con la configuración -n 200 -p 40 -cprob 30.0 -mprob 60.0}
\centering
\small
\begin{tabular}{c c c c c c c c}
\hline\hline
Instancia & Costo mínimo & Tiempo(seg.) & Costo promedio & Tiempo promedio(seg.) & CME & \%G & \%GP \\ [0.5ex]
\hline
SCA3-0 & 640.55 & 0.48 & 
640.55 & 0.52 & \bf{635.62} & 
0.78 & 0.78\\SCA3-1 & 701.53 & 0.52 & 
701.70 & 0.56 & \bf{697.84} & 
0.53 & 0.55\\SCA3-2 & 661.13 & 0.44 & 
661.13 & 0.53 & \bf{659.34} & 
0.27 & 0.27\\SCA3-3 & 682.44 & 0.70 & 
683.66 & 0.67 & \bf{680.04} & 
0.35 & 0.53\\SCA3-4 & \bf{690.50} & 0.57 & 
691.02 & 0.53 & 690.50 & 0.00
 & 0.08\\SCA3-5 & 677.99 & 0.56 & 
680.06 & 0.61 & \bf{659.90} & 
2.74 & 3.06\\SCA3-6 & 657.24 & 0.52 & 
657.44 & 0.56 & \bf{651.09} & 
0.94 & 0.98\\SCA3-7 & 666.15 & 0.52 & 
666.15 & 0.62 & \bf{659.17} & 
1.06 & 1.06\\SCA3-8 & 723.99 & 0.50 & 
723.99 & 0.46 & \bf{719.47} & 
0.63 & 0.63\\SCA3-9 & \bf{681.00} & 0.86 & 
681.51 & 0.69 & 681.00 & 0.00
 & 0.07\\SCA8-0 & 992.57 & 0.52 & 
992.57 & 0.58 & \bf{961.50} & 
3.23 & 3.23\\SCA8-1 & 1084.46 & 0.76 & 
1084.58 & 0.72 & \bf{1049.65} & 
3.32 & 3.33\\SCA8-2 & 1057.51 & 0.82 & 
1059.83 & 0.66 & \bf{1039.64} & 
1.72 & 1.94\\SCA8-3 & 1010.13 & 0.44 & 
1023.32 & 0.57 & \bf{983.34} & 
2.72 & 4.07\\SCA8-4 & 1085.41 & 0.40 & 
1085.41 & 0.45 & \bf{1065.49} & 
1.87 & 1.87\\SCA8-5 & 1052.23 & 0.74 & 
1052.23 & 0.71 & \bf{1027.08} & 
2.45 & 2.45\\SCA8-6 & 982.57 & 0.64 & 
985.35 & 0.63 & \bf{971.82} & 
1.11 & 1.39\\SCA8-7 & 1083.05 & 0.46 & 
1083.05 & 0.52 & \bf{1051.28} & 
3.02 & 3.02\\SCA8-8 & 1094.94 & 0.60 & 
1094.94 & 0.65 & \bf{1071.18} & 
2.22 & 2.22\\SCA8-9 & 1079.43 & 0.91 & 
1079.43 & 0.52 & \bf{1060.50} & 
1.79 & 1.79\\CON3-0 & 626.62 & 0.45 & 
629.50 & 0.52 & \bf{616.52} & 
1.64 & 2.11\\CON3-1 & 560.75 & 0.61 & 
560.75 & 0.53 & \bf{554.47} & 
1.13 & 1.13\\CON3-2 & 521.38 & 0.80 & 
522.96 & 0.73 & \bf{518.00} & 
0.65 & 0.96\\CON3-3 & 591.20 & 0.49 & 
600.48 & 0.56 & \bf{591.19} & 
0.00 & 1.57\\CON3-4 & 593.78 & 0.50 & 
598.53 & 0.58 & \bf{588.79} & 
0.85 & 1.66\\CON3-5 & 568.76 & 0.56 & 
568.76 & 0.70 & \bf{563.70} & 
0.90 & 0.90\\CON3-6 & 504.91 & 0.51 & 
506.79 & 0.52 & \bf{499.05} & 
1.17 & 1.55\\CON3-7 & 578.22 & 0.77 & 
584.83 & 0.62 & \bf{576.48} & 
0.30 & 1.45\\CON3-8 & 524.30 & 0.44 & 
531.96 & 0.60 & \bf{523.05} & 
0.24 & 1.70\\CON3-9 & 589.57 & 0.69 & 
590.14 & 0.61 & \bf{578.24} & 
1.96 & 2.06\\CON8-0 & 879.97 & 0.59 & 
879.97 & 0.64 & \bf{857.17} & 
2.66 & 2.66\\CON8-1 & 760.21 & 0.74 & 
764.27 & 0.64 & \bf{740.85} & 
2.61 & 3.16\\CON8-2 & 718.79 & 1.00 & 
719.24 & 0.64 & \bf{712.89} & 
0.83 & 0.89\\CON8-3 & 836.41 & 0.57 & 
836.41 & 0.69 & \bf{811.07} & 
3.12 & 3.12\\CON8-4 & 785.27 & 0.52 & 
786.38 & 0.57 & \bf{772.25} & 
1.69 & 1.83\\CON8-5 & 764.40 & 0.50 & 
771.10 & 0.62 & \bf{754.88} & 
1.26 & 2.15\\CON8-6 & 702.66 & 0.73 & 
703.22 & 0.66 & \bf{678.92} & 
3.50 & 3.58\\CON8-7 & 819.07 & 0.91 & 
821.14 & 0.72 & \bf{811.96} & 
0.88 & 1.13\\CON8-8 & 788.71 & 0.71 & 
791.77 & 0.65 & \bf{767.53} & 
2.76 & 3.16\\CON8-9 & 827.54 & 0.51 & 
834.07 & 0.49 & \bf{809.00} & 
2.29 & 3.10\\\bf{PROM.} & 
\bf{771.18} & \bf{0.61} & \bf{773.25} & \bf{0.60} & \bf{758.54} & \bf{1.53} & \bf{1.83}\\[1ex]\hline
\end{tabular}
\label{table:nonlin}
\end{table} \clearpage
\begin{table}[ht]
\caption{Resultados de la ejecución de la metaheurística IGA, utilizando instancias de SalhiNagy con la configuración -n 200 -p 40 -cprob 30.0 -mprob 60.0}
\centering
\small
\begin{tabular}{c c c c c c c c}
\hline\hline
Instancia & Costo mínimo & Tiempo(seg.) & Costo promedio & Tiempo promedio(seg.) & CME & \%G & \%GP \\ [0.5ex]
\hline
CMT1X & 481.98 & 0.39 & 
485.70 & 0.39 & \bf{470.48} & 
2.44 & 3.23\\CMT1Y & 481.65 & 0.30 & 
483.79 & 0.47 & \bf{470.48} & 
2.37 & 2.83\\CMT2X & 714.63 & 1.39 & 
721.22 & 1.15 & \bf{682.39} & 
4.72 & 5.69\\CMT2Y & 710.73 & 1.22 & 
719.91 & 1.27 & \bf{682.39} & 
4.15 & 5.50\\CMT3X & 725.90 & 2.55 & 
738.80 & 2.90 & \bf{719.06} & 
0.95 & 2.74\\CMT3Y & 741.55 & 2.81 & 
744.03 & 2.85 & \bf{719.06} & 
3.13 & 3.47\\CMT4X & 894.79 & 7.43 & 
899.70 & 7.47 & \bf{854.21} & 
4.75 & 5.33\\CMT4Y & 900.37 & 8.57 & 
918.12 & 8.03 & \bf{852.46} & 
5.62 & 7.70\\CMT5X & 1106.50 & 16.83 & 
1115.95 & 16.01 & \bf{1030.56} & 
7.37 & 8.29\\CMT5Y & 1117.10 & 15.20 & 
1126.54 & 16.00 & \bf{1031.69} & 
8.28 & 9.19\\CMT11X & 895.17 & 5.41 & 
912.09 & 5.01 & \bf{831.09} & 
7.71 & 9.75\\CMT11Y & 906.04 & 4.74 & 
937.99 & 4.91 & \bf{829.85} & 
9.18 & 13.03\\CMT12X & 674.11 & 2.83 & 
675.40 & 2.75 & \bf{658.83} & 
2.32 & 2.52\\CMT12Y & 676.41 & 2.49 & 
679.75 & 2.67 & \bf{660.47} & 
2.41 & 2.92\\\bf{PROM.} & 
\bf{787.64} & \bf{5.15} & \bf{797.07} & \bf{5.13} & \bf{749.50} & \bf{4.67} & \bf{5.87}\\[1ex]\hline
\end{tabular}
\label{table:nonlin}
\end{table} \clearpage
\begin{table}[ht]
\caption{Resultados de la ejecución de la metaheurística IGA, utilizando instancias de Dethloff con la configuración -n 200 -p 40 -cprob 30.0 -mprob 70.0}
\centering
\small
\begin{tabular}{c c c c c c c c}
\hline\hline
Instancia & Costo mínimo & Tiempo(seg.) & Costo promedio & Tiempo promedio(seg.) & CME & \%G & \%GP \\ [0.5ex]
\hline
SCA3-0 & 640.55 & 0.57 & 
642.50 & 0.50 & \bf{635.62} & 
0.78 & 1.08\\SCA3-1 & \bf{697.84} & 0.79 & 
697.84 & 0.64 & 697.84 & 0.00
 & 0.00\\
SCA3-2 & 664.21 & 0.40 & 
666.77 & 0.42 & \bf{659.34} & 
0.74 & 1.13\\SCA3-3 & \bf{680.04} & 0.48 & 
680.36 & 0.49 & 680.04 & 0.00
 & 0.05\\SCA3-4 & \bf{690.50} & 0.43 & 
690.50 & 0.47 & 690.50 & 0.00
 & 0.00\\
SCA3-5 & 673.39 & 0.54 & 
673.42 & 0.63 & \bf{659.90} & 
2.04 & 2.05\\SCA3-6 & 652.94 & 0.79 & 
652.94 & 0.62 & \bf{651.09} & 
0.28 & 0.28\\SCA3-7 & 666.15 & 0.53 & 
666.15 & 0.75 & \bf{659.17} & 
1.06 & 1.06\\SCA3-8 & 727.65 & 0.66 & 
727.65 & 0.55 & \bf{719.47} & 
1.14 & 1.14\\SCA3-9 & \bf{681.00} & 0.69 & 
681.00 & 0.67 & 681.00 & 0.00
 & 0.00\\
SCA8-0 & 995.95 & 0.61 & 
996.85 & 0.65 & \bf{961.50} & 
3.58 & 3.68\\SCA8-1 & 1080.76 & 0.51 & 
1080.76 & 0.59 & \bf{1049.65} & 
2.96 & 2.96\\SCA8-2 & 1050.37 & 0.70 & 
1054.70 & 0.60 & \bf{1039.64} & 
1.03 & 1.45\\SCA8-3 & 1009.27 & 0.42 & 
1021.04 & 0.56 & \bf{983.34} & 
2.64 & 3.83\\SCA8-4 & 1077.80 & 0.53 & 
1077.80 & 0.56 & \bf{1065.49} & 
1.16 & 1.16\\SCA8-5 & 1052.31 & 0.90 & 
1052.31 & 0.90 & \bf{1027.08} & 
2.46 & 2.46\\SCA8-6 & 984.08 & 0.58 & 
988.37 & 0.58 & \bf{971.82} & 
1.26 & 1.70\\SCA8-7 & 1089.56 & 0.40 & 
1089.56 & 0.50 & \bf{1051.28} & 
3.64 & 3.64\\SCA8-8 & 1075.00 & 0.54 & 
1075.00 & 0.62 & \bf{1071.18} & 
0.36 & 0.36\\SCA8-9 & 1092.48 & 0.47 & 
1093.37 & 0.59 & \bf{1060.50} & 
3.02 & 3.10\\CON3-0 & 632.87 & 0.71 & 
633.86 & 0.81 & \bf{616.52} & 
2.65 & 2.81\\CON3-1 & 560.75 & 0.64 & 
563.28 & 0.58 & \bf{554.47} & 
1.13 & 1.59\\CON3-2 & 521.38 & 0.54 & 
521.38 & 0.62 & \bf{518.00} & 
0.65 & 0.65\\CON3-3 & 591.48 & 0.56 & 
596.48 & 0.51 & \bf{591.19} & 
0.05 & 0.89\\CON3-4 & 591.43 & 0.52 & 
600.57 & 0.60 & \bf{588.79} & 
0.45 & 2.00\\CON3-5 & 564.88 & 0.71 & 
567.17 & 0.74 & \bf{563.70} & 
0.21 & 0.62\\CON3-6 & 510.26 & 1.01 & 
510.77 & 0.88 & \bf{499.05} & 
2.25 & 2.35\\CON3-7 & \bf{576.48} & 0.53 & 
586.95 & 0.52 & 576.48 & 0.00
 & 1.82\\CON3-8 & 524.59 & 0.54 & 
527.10 & 0.58 & \bf{523.05} & 
0.29 & 0.77\\CON3-9 & 588.99 & 0.54 & 
590.21 & 0.53 & \bf{578.24} & 
1.86 & 2.07\\CON8-0 & 870.22 & 0.81 & 
884.09 & 0.66 & \bf{857.17} & 
1.52 & 3.14\\CON8-1 & 762.04 & 0.54 & 
762.04 & 0.58 & \bf{740.85} & 
2.86 & 2.86\\CON8-2 & 727.99 & 0.51 & 
729.22 & 0.54 & \bf{712.89} & 
2.12 & 2.29\\CON8-3 & 830.41 & 0.64 & 
832.09 & 0.77 & \bf{811.07} & 
2.38 & 2.59\\CON8-4 & 790.77 & 0.68 & 
801.33 & 0.53 & \bf{772.25} & 
2.40 & 3.77\\CON8-5 & 763.13 & 0.58 & 
768.61 & 0.55 & \bf{754.88} & 
1.09 & 1.82\\CON8-6 & 689.51 & 0.73 & 
696.29 & 0.57 & \bf{678.92} & 
1.56 & 2.56\\CON8-7 & 820.08 & 0.56 & 
823.62 & 0.50 & \bf{811.96} & 
1.00 & 1.44\\CON8-8 & 785.09 & 0.56 & 
789.83 & 0.55 & \bf{767.53} & 
2.29 & 2.90\\CON8-9 & 834.14 & 0.54 & 
839.52 & 0.55 & \bf{809.00} & 
3.11 & 3.77\\\bf{PROM.} & 
\bf{770.46} & \bf{0.60} & \bf{773.33} & \bf{0.60} & \bf{758.54} & \bf{1.45} & \bf{1.85}\\[1ex]\hline
\end{tabular}
\label{table:nonlin}
\end{table} \clearpage
\begin{table}[ht]
\caption{Resultados de la ejecución de la metaheurística IGA, utilizando instancias de SalhiNagy con la configuración -n 200 -p 40 -cprob 30.0 -mprob 70.0}
\centering
\small
\begin{tabular}{c c c c c c c c}
\hline\hline
Instancia & Costo mínimo & Tiempo(seg.) & Costo promedio & Tiempo promedio(seg.) & CME & \%G & \%GP \\ [0.5ex]
\hline
CMT1X & 478.38 & 0.54 & 
479.99 & 0.48 & \bf{470.48} & 
1.68 & 2.02\\CMT1Y & 485.35 & 0.36 & 
487.63 & 0.47 & \bf{470.48} & 
3.16 & 3.65\\CMT2X & 707.62 & 1.16 & 
716.87 & 1.36 & \bf{682.39} & 
3.70 & 5.05\\CMT2Y & 709.08 & 1.06 & 
719.12 & 1.24 & \bf{682.39} & 
3.91 & 5.38\\CMT3X & 733.33 & 3.09 & 
745.32 & 3.06 & \bf{719.06} & 
1.98 & 3.65\\CMT3Y & 742.13 & 2.36 & 
749.27 & 2.47 & \bf{719.06} & 
3.21 & 4.20\\CMT4X & 908.53 & 7.56 & 
913.46 & 8.01 & \bf{854.21} & 
6.36 & 6.94\\CMT4Y & 910.89 & 8.12 & 
918.04 & 7.84 & \bf{852.46} & 
6.85 & 7.69\\CMT5X & 1107.28 & 15.66 & 
1118.16 & 15.59 & \bf{1030.56} & 
7.44 & 8.50\\CMT5Y & 1104.84 & 18.08 & 
1114.19 & 16.45 & \bf{1031.69} & 
7.09 & 8.00\\CMT11X & 895.88 & 4.56 & 
910.58 & 4.66 & \bf{831.09} & 
7.80 & 9.56\\CMT11Y & 908.25 & 5.16 & 
914.45 & 5.10 & \bf{829.85} & 
9.45 & 10.19\\CMT12X & 675.70 & 2.50 & 
680.54 & 2.84 & \bf{658.83} & 
2.56 & 3.30\\CMT12Y & 674.96 & 2.40 & 
677.44 & 2.65 & \bf{660.47} & 
2.19 & 2.57\\\bf{PROM.} & 
\bf{788.73} & \bf{5.19} & \bf{796.07} & \bf{5.16} & \bf{749.50} & \bf{4.81} & \bf{5.76}\\[1ex]\hline
\end{tabular}
\label{table:nonlin}
\end{table} \clearpage
\begin{table}[ht]
\caption{Resultados de la ejecución de la metaheurística IGA, utilizando instancias de Dethloff con la configuración -n 200 -p 40 -cprob 30.0 -mprob 80.0}
\centering
\small
\begin{tabular}{c c c c c c c c}
\hline\hline
Instancia & Costo mínimo & Tiempo(seg.) & Costo promedio & Tiempo promedio(seg.) & CME & \%G & \%GP \\ [0.5ex]
\hline
SCA3-0 & 640.55 & 0.59 & 
641.49 & 0.53 & \bf{635.62} & 
0.78 & 0.92\\SCA3-1 & \bf{697.84} & 0.49 & 
700.78 & 0.51 & 697.84 & 0.00
 & 0.42\\SCA3-2 & 664.92 & 1.14 & 
666.57 & 0.63 & \bf{659.34} & 
0.85 & 1.10\\SCA3-3 & 681.16 & 0.45 & 
681.25 & 0.44 & \bf{680.04} & 
0.16 & 0.18\\SCA3-4 & \bf{690.50} & 0.44 & 
690.50 & 0.48 & 690.50 & 0.00
 & 0.00\\
SCA3-5 & 670.10 & 0.49 & 
679.42 & 0.67 & \bf{659.90} & 
1.55 & 2.96\\SCA3-6 & 657.24 & 0.71 & 
657.24 & 0.81 & \bf{651.09} & 
0.94 & 0.94\\SCA3-7 & 666.15 & 0.60 & 
666.15 & 0.56 & \bf{659.17} & 
1.06 & 1.06\\SCA3-8 & \bf{719.47} & 0.72 & 
725.12 & 0.61 & 719.47 & 0.00
 & 0.79\\SCA3-9 & \bf{681.00} & 0.50 & 
684.13 & 0.62 & 681.00 & 0.00
 & 0.46\\SCA8-0 & 991.47 & 0.57 & 
995.14 & 0.74 & \bf{961.50} & 
3.12 & 3.50\\SCA8-1 & 1097.36 & 0.76 & 
1097.36 & 0.76 & \bf{1049.65} & 
4.55 & 4.55\\SCA8-2 & 1051.42 & 0.52 & 
1051.42 & 0.64 & \bf{1039.64} & 
1.13 & 1.13\\SCA8-3 & 1015.81 & 0.70 & 
1018.61 & 0.68 & \bf{983.34} & 
3.30 & 3.59\\SCA8-4 & 1085.80 & 0.64 & 
1085.80 & 0.49 & \bf{1065.49} & 
1.91 & 1.91\\SCA8-5 & 1052.23 & 0.38 & 
1053.72 & 0.55 & \bf{1027.08} & 
2.45 & 2.59\\SCA8-6 & 982.58 & 0.44 & 
991.43 & 0.47 & \bf{971.82} & 
1.11 & 2.02\\SCA8-7 & 1084.87 & 0.53 & 
1088.22 & 0.62 & \bf{1051.28} & 
3.20 & 3.51\\SCA8-8 & 1089.07 & 0.73 & 
1089.07 & 0.65 & \bf{1071.18} & 
1.67 & 1.67\\SCA8-9 & 1070.39 & 0.61 & 
1072.81 & 0.57 & \bf{1060.50} & 
0.93 & 1.16\\CON3-0 & 626.52 & 0.48 & 
626.52 & 0.66 & \bf{616.52} & 
1.62 & 1.62\\CON3-1 & 560.75 & 0.56 & 
562.84 & 0.64 & \bf{554.47} & 
1.13 & 1.51\\CON3-2 & 520.38 & 0.97 & 
520.38 & 0.69 & \bf{518.00} & 
0.46 & 0.46\\CON3-3 & 591.20 & 0.56 & 
600.62 & 0.55 & \bf{591.19} & 
0.00 & 1.60\\CON3-4 & 589.88 & 0.59 & 
595.02 & 0.67 & \bf{588.79} & 
0.19 & 1.06\\CON3-5 & 564.88 & 0.56 & 
567.38 & 0.55 & \bf{563.70} & 
0.21 & 0.65\\CON3-6 & 502.16 & 0.79 & 
504.77 & 0.72 & \bf{499.05} & 
0.62 & 1.15\\CON3-7 & 582.14 & 0.42 & 
588.48 & 0.60 & \bf{576.48} & 
0.98 & 2.08\\CON3-8 & \bf{523.05} & 0.66 & 
529.40 & 0.79 & 523.05 & 0.00
 & 1.21\\CON3-9 & 590.17 & 0.55 & 
592.17 & 0.73 & \bf{578.24} & 
2.06 & 2.41\\CON8-0 & 871.19 & 0.74 & 
874.09 & 0.65 & \bf{857.17} & 
1.64 & 1.97\\CON8-1 & 763.26 & 0.52 & 
767.60 & 0.57 & \bf{740.85} & 
3.02 & 3.61\\CON8-2 & 727.69 & 0.53 & 
728.59 & 0.53 & \bf{712.89} & 
2.08 & 2.20\\CON8-3 & 837.15 & 0.90 & 
837.15 & 0.89 & \bf{811.07} & 
3.22 & 3.22\\CON8-4 & 781.48 & 0.39 & 
787.80 & 0.61 & \bf{772.25} & 
1.20 & 2.01\\CON8-5 & 766.55 & 0.61 & 
768.29 & 0.53 & \bf{754.88} & 
1.55 & 1.78\\CON8-6 & 699.72 & 0.49 & 
699.72 & 0.60 & \bf{678.92} & 
3.06 & 3.06\\CON8-7 & 814.50 & 0.84 & 
830.45 & 0.59 & \bf{811.96} & 
0.31 & 2.28\\CON8-8 & 794.33 & 0.54 & 
796.31 & 0.57 & \bf{767.53} & 
3.49 & 3.75\\CON8-9 & 820.40 & 0.80 & 
820.40 & 0.61 & \bf{809.00} & 
1.41 & 1.41\\\bf{PROM.} & 
\bf{770.43} & \bf{0.61} & \bf{773.36} & \bf{0.62} & \bf{758.54} & \bf{1.42} & \bf{1.84}\\[1ex]\hline
\end{tabular}
\label{table:nonlin}
\end{table} \clearpage
\begin{table}[ht]
\caption{Resultados de la ejecución de la metaheurística IGA, utilizando instancias de SalhiNagy con la configuración -n 200 -p 40 -cprob 30.0 -mprob 80.0}
\centering
\small
\begin{tabular}{c c c c c c c c}
\hline\hline
Instancia & Costo mínimo & Tiempo(seg.) & Costo promedio & Tiempo promedio(seg.) & CME & \%G & \%GP \\ [0.5ex]
\hline
CMT1X & 477.21 & 0.47 & 
482.27 & 0.49 & \bf{470.48} & 
1.43 & 2.51\\CMT1Y & 478.82 & 0.76 & 
481.94 & 0.61 & \bf{470.48} & 
1.77 & 2.44\\CMT2X & 709.31 & 1.14 & 
712.77 & 1.32 & \bf{682.39} & 
3.94 & 4.45\\CMT2Y & 702.46 & 1.64 & 
709.80 & 1.28 & \bf{682.39} & 
2.94 & 4.02\\CMT3X & 740.25 & 3.11 & 
741.40 & 3.00 & \bf{719.06} & 
2.95 & 3.11\\CMT3Y & 740.90 & 2.45 & 
750.38 & 3.01 & \bf{719.06} & 
3.04 & 4.35\\CMT4X & 899.29 & 7.98 & 
915.32 & 7.70 & \bf{854.21} & 
5.28 & 7.15\\CMT4Y & 913.66 & 7.64 & 
919.41 & 7.63 & \bf{852.46} & 
7.18 & 7.85\\CMT5X & 1101.12 & 16.28 & 
1108.53 & 15.65 & \bf{1030.56} & 
6.85 & 7.57\\CMT5Y & 1106.24 & 16.47 & 
1121.41 & 16.20 & \bf{1031.69} & 
7.23 & 8.70\\CMT11X & 891.95 & 4.30 & 
917.82 & 4.76 & \bf{831.09} & 
7.32 & 10.44\\CMT11Y & 889.20 & 5.29 & 
912.13 & 5.17 & \bf{829.85} & 
7.15 & 9.92\\CMT12X & 673.28 & 2.82 & 
683.17 & 2.83 & \bf{658.83} & 
2.19 & 3.70\\CMT12Y & 673.32 & 2.80 & 
679.38 & 2.70 & \bf{660.47} & 
1.95 & 2.86\\\bf{PROM.} & 
\bf{785.50} & \bf{5.22} & \bf{795.41} & \bf{5.17} & \bf{749.50} & \bf{4.37} & \bf{5.65}\\[1ex]\hline
\end{tabular}
\label{table:nonlin}
\end{table} \clearpage
\begin{table}[ht]
\caption{Resultados de la ejecución de la metaheurística IGA, utilizando instancias de Dethloff con la configuración -n 200 -p 40 -cprob 30.0 -mprob 90.0}
\centering
\small
\begin{tabular}{c c c c c c c c}
\hline\hline
Instancia & Costo mínimo & Tiempo(seg.) & Costo promedio & Tiempo promedio(seg.) & CME & \%G & \%GP \\ [0.5ex]
\hline
SCA3-0 & 640.55 & 0.51 & 
640.55 & 0.58 & \bf{635.62} & 
0.78 & 0.78\\SCA3-1 & 701.78 & 0.59 & 
703.14 & 0.77 & \bf{697.84} & 
0.56 & 0.76\\SCA3-2 & \bf{659.34} & 0.66 & 
663.26 & 0.54 & 659.34 & 0.00
 & 0.59\\SCA3-3 & 685.05 & 0.52 & 
685.48 & 0.86 & \bf{680.04} & 
0.74 & 0.80\\SCA3-4 & \bf{690.50} & 0.66 & 
690.50 & 0.54 & 690.50 & 0.00
 & 0.00\\
SCA3-5 & 668.48 & 0.50 & 
668.48 & 0.61 & \bf{659.90} & 
1.30 & 1.30\\SCA3-6 & 652.94 & 0.60 & 
654.17 & 0.64 & \bf{651.09} & 
0.28 & 0.47\\SCA3-7 & 666.15 & 0.44 & 
666.26 & 0.47 & \bf{659.17} & 
1.06 & 1.08\\SCA3-8 & \bf{719.47} & 0.83 & 
721.39 & 0.59 & 719.47 & 0.00
 & 0.27\\SCA3-9 & \bf{681.00} & 0.64 & 
681.00 & 0.58 & 681.00 & 0.00
 & 0.00\\
SCA8-0 & 970.64 & 1.01 & 
997.80 & 0.83 & \bf{961.50} & 
0.95 & 3.78\\SCA8-1 & 1086.16 & 0.60 & 
1086.97 & 0.67 & \bf{1049.65} & 
3.48 & 3.56\\SCA8-2 & 1051.42 & 0.55 & 
1051.42 & 0.67 & \bf{1039.64} & 
1.13 & 1.13\\SCA8-3 & 1029.05 & 0.46 & 
1032.30 & 0.70 & \bf{983.34} & 
4.65 & 4.98\\SCA8-4 & \bf{1065.49} & 0.61 & 
1065.49 & 0.60 & 1065.49 & 0.00
 & 0.00\\
SCA8-5 & 1067.75 & 0.96 & 
1068.13 & 0.82 & \bf{1027.08} & 
3.96 & 4.00\\SCA8-6 & 991.47 & 0.58 & 
993.99 & 0.75 & \bf{971.82} & 
2.02 & 2.28\\SCA8-7 & 1085.24 & 0.89 & 
1085.24 & 0.64 & \bf{1051.28} & 
3.23 & 3.23\\SCA8-8 & 1088.19 & 0.42 & 
1102.24 & 0.48 & \bf{1071.18} & 
1.59 & 2.90\\SCA8-9 & 1097.48 & 0.65 & 
1104.11 & 0.64 & \bf{1060.50} & 
3.49 & 4.11\\CON3-0 & 633.24 & 0.59 & 
635.29 & 0.66 & \bf{616.52} & 
2.71 & 3.04\\CON3-1 & 560.03 & 0.94 & 
561.93 & 0.69 & \bf{554.47} & 
1.00 & 1.35\\CON3-2 & 521.38 & 0.96 & 
521.38 & 0.71 & \bf{518.00} & 
0.65 & 0.65\\CON3-3 & 592.41 & 0.57 & 
593.59 & 0.70 & \bf{591.19} & 
0.21 & 0.41\\CON3-4 & 589.32 & 0.67 & 
594.24 & 0.74 & \bf{588.79} & 
0.09 & 0.93\\CON3-5 & 564.89 & 0.86 & 
567.37 & 0.61 & \bf{563.70} & 
0.21 & 0.65\\CON3-6 & 504.44 & 0.78 & 
505.56 & 0.76 & \bf{499.05} & 
1.08 & 1.30\\CON3-7 & 582.33 & 0.57 & 
587.68 & 0.73 & \bf{576.48} & 
1.01 & 1.94\\CON3-8 & \bf{523.05} & 0.46 & 
524.31 & 0.60 & 523.05 & 0.00
 & 0.24\\CON3-9 & 582.79 & 0.65 & 
582.79 & 0.65 & \bf{578.24} & 
0.79 & 0.79\\CON8-0 & 868.46 & 0.84 & 
868.46 & 0.70 & \bf{857.17} & 
1.32 & 1.32\\CON8-1 & 744.43 & 0.70 & 
744.43 & 0.64 & \bf{740.85} & 
0.48 & 0.48\\CON8-2 & 726.48 & 1.04 & 
727.37 & 0.76 & \bf{712.89} & 
1.91 & 2.03\\CON8-3 & 832.63 & 0.60 & 
832.63 & 0.59 & \bf{811.07} & 
2.66 & 2.66\\CON8-4 & 772.83 & 0.73 & 
772.83 & 0.68 & \bf{772.25} & 
0.08 & 0.08\\CON8-5 & 769.55 & 0.62 & 
771.37 & 0.77 & \bf{754.88} & 
1.94 & 2.18\\CON8-6 & 701.55 & 0.48 & 
701.66 & 0.55 & \bf{678.92} & 
3.33 & 3.35\\CON8-7 & 825.22 & 0.62 & 
825.22 & 0.67 & \bf{811.96} & 
1.63 & 1.63\\CON8-8 & 771.26 & 0.44 & 
772.48 & 0.65 & \bf{767.53} & 
0.49 & 0.64\\CON8-9 & 811.79 & 0.53 & 
825.00 & 0.55 & \bf{809.00} & 
0.34 & 1.98\\\bf{PROM.} & 
\bf{769.41} & \bf{0.66} & \bf{771.94} & \bf{0.66} & \bf{758.54} & \bf{1.28} & \bf{1.59}\\[1ex]\hline
\end{tabular}
\label{table:nonlin}
\end{table} \clearpage
\begin{table}[ht]
\caption{Resultados de la ejecución de la metaheurística IGA, utilizando instancias de SalhiNagy con la configuración -n 200 -p 40 -cprob 30.0 -mprob 90.0}
\centering
\small
\begin{tabular}{c c c c c c c c}
\hline\hline
Instancia & Costo mínimo & Tiempo(seg.) & Costo promedio & Tiempo promedio(seg.) & CME & \%G & \%GP \\ [0.5ex]
\hline
CMT1X & 479.19 & 0.67 & 
484.88 & 0.76 & \bf{470.48} & 
1.85 & 3.06\\CMT1Y & 479.62 & 0.41 & 
483.27 & 0.52 & \bf{470.48} & 
1.94 & 2.72\\CMT2X & 709.09 & 1.70 & 
713.65 & 1.32 & \bf{682.39} & 
3.91 & 4.58\\CMT2Y & 712.07 & 0.98 & 
715.57 & 1.22 & \bf{682.39} & 
4.35 & 4.86\\CMT3X & 741.77 & 2.58 & 
748.62 & 2.66 & \bf{719.06} & 
3.16 & 4.11\\CMT3Y & 751.14 & 2.64 & 
752.17 & 2.86 & \bf{719.06} & 
4.46 & 4.60\\CMT4X & 887.27 & 7.41 & 
897.93 & 7.62 & \bf{854.21} & 
3.87 & 5.12\\CMT4Y & 906.83 & 7.61 & 
910.73 & 8.43 & \bf{852.46} & 
6.38 & 6.84\\CMT5X & 1129.40 & 15.42 & 
1134.66 & 15.79 & \bf{1030.56} & 
9.59 & 10.10\\CMT5Y & 1075.45 & 15.72 & 
1119.99 & 15.94 & \bf{1031.69} & 
4.24 & 8.56\\CMT11X & 894.63 & 4.37 & 
909.85 & 4.55 & \bf{831.09} & 
7.65 & 9.48\\CMT11Y & 891.22 & 5.36 & 
904.11 & 5.61 & \bf{829.85} & 
7.40 & 8.95\\CMT12X & 674.94 & 2.96 & 
677.69 & 2.81 & \bf{658.83} & 
2.45 & 2.86\\CMT12Y & 675.63 & 3.20 & 
680.91 & 2.87 & \bf{660.47} & 
2.30 & 3.09\\\bf{PROM.} & 
\bf{786.30} & \bf{5.07} & \bf{795.29} & \bf{5.21} & \bf{749.50} & \bf{4.54} & \bf{5.64}\\[1ex]\hline
\end{tabular}
\label{table:nonlin}
\end{table} \clearpage
\begin{table}[ht]
\caption{Resultados de la ejecución de la metaheurística IGA, utilizando instancias de Dethloff con la configuración -n 200 -p 40 -cprob 30.0 -mprob 100.0}
\centering
\small
\begin{tabular}{c c c c c c c c}
\hline\hline
Instancia & Costo mínimo & Tiempo(seg.) & Costo promedio & Tiempo promedio(seg.) & CME & \%G & \%GP \\ [0.5ex]
\hline
SCA3-0 & 640.55 & 0.60 & 
641.12 & 0.57 & \bf{635.62} & 
0.78 & 0.87\\SCA3-1 & \bf{697.84} & 0.74 & 
700.61 & 0.56 & 697.84 & 0.00
 & 0.40\\SCA3-2 & 669.60 & 0.62 & 
669.60 & 0.60 & \bf{659.34} & 
1.56 & 1.56\\SCA3-3 & 681.16 & 0.56 & 
682.86 & 0.69 & \bf{680.04} & 
0.16 & 0.41\\SCA3-4 & \bf{690.50} & 0.91 & 
690.50 & 0.64 & 690.50 & 0.00
 & 0.00\\
SCA3-5 & 670.10 & 0.70 & 
677.18 & 0.54 & \bf{659.90} & 
1.55 & 2.62\\SCA3-6 & 652.94 & 0.51 & 
655.62 & 0.53 & \bf{651.09} & 
0.28 & 0.70\\SCA3-7 & 666.15 & 0.65 & 
666.15 & 0.62 & \bf{659.17} & 
1.06 & 1.06\\SCA3-8 & 726.88 & 0.52 & 
726.88 & 0.52 & \bf{719.47} & 
1.03 & 1.03\\SCA3-9 & \bf{681.00} & 0.44 & 
681.00 & 0.53 & 681.00 & 0.00
 & 0.00\\
SCA8-0 & 985.50 & 0.54 & 
985.50 & 0.58 & \bf{961.50} & 
2.50 & 2.50\\SCA8-1 & 1068.21 & 0.59 & 
1070.36 & 0.64 & \bf{1049.65} & 
1.77 & 1.97\\SCA8-2 & 1054.47 & 0.46 & 
1057.95 & 0.55 & \bf{1039.64} & 
1.43 & 1.76\\SCA8-3 & 1014.71 & 0.42 & 
1027.94 & 0.54 & \bf{983.34} & 
3.19 & 4.54\\SCA8-4 & 1098.96 & 0.74 & 
1098.96 & 0.72 & \bf{1065.49} & 
3.14 & 3.14\\SCA8-5 & 1061.98 & 0.55 & 
1061.98 & 0.66 & \bf{1027.08} & 
3.40 & 3.40\\SCA8-6 & 972.48 & 0.56 & 
972.48 & 0.56 & \bf{971.82} & 
0.07 & 0.07\\SCA8-7 & 1085.84 & 0.43 & 
1091.17 & 0.44 & \bf{1051.28} & 
3.29 & 3.79\\SCA8-8 & \bf{1071.18} & 0.48 & 
1095.86 & 0.62 & 1071.18 & 0.00
 & 2.30\\SCA8-9 & 1074.21 & 0.53 & 
1078.08 & 0.55 & \bf{1060.50} & 
1.29 & 1.66\\CON3-0 & 620.76 & 0.86 & 
628.50 & 0.78 & \bf{616.52} & 
0.69 & 1.94\\CON3-1 & 556.92 & 0.56 & 
557.14 & 0.67 & \bf{554.47} & 
0.44 & 0.48\\CON3-2 & 521.38 & 0.62 & 
522.26 & 0.76 & \bf{518.00} & 
0.65 & 0.82\\CON3-3 & 599.26 & 0.58 & 
603.67 & 0.78 & \bf{591.19} & 
1.37 & 2.11\\CON3-4 & 593.78 & 0.62 & 
596.98 & 0.59 & \bf{588.79} & 
0.85 & 1.39\\CON3-5 & 568.76 & 0.82 & 
569.51 & 0.65 & \bf{563.70} & 
0.90 & 1.03\\CON3-6 & 505.01 & 0.74 & 
506.37 & 0.73 & \bf{499.05} & 
1.19 & 1.47\\CON3-7 & 582.14 & 0.53 & 
582.14 & 0.53 & \bf{576.48} & 
0.98 & 0.98\\CON3-8 & 524.59 & 0.68 & 
527.73 & 0.69 & \bf{523.05} & 
0.29 & 0.89\\CON3-9 & 590.39 & 0.56 & 
590.52 & 0.62 & \bf{578.24} & 
2.10 & 2.12\\CON8-0 & 866.81 & 0.57 & 
879.30 & 0.59 & \bf{857.17} & 
1.12 & 2.58\\CON8-1 & 757.69 & 0.84 & 
765.73 & 0.81 & \bf{740.85} & 
2.27 & 3.36\\CON8-2 & 716.69 & 0.62 & 
716.69 & 0.77 & \bf{712.89} & 
0.53 & 0.53\\CON8-3 & 839.78 & 0.64 & 
839.78 & 0.62 & \bf{811.07} & 
3.54 & 3.54\\CON8-4 & 785.13 & 0.66 & 
786.70 & 0.56 & \bf{772.25} & 
1.67 & 1.87\\CON8-5 & 766.55 & 0.80 & 
766.55 & 0.76 & \bf{754.88} & 
1.55 & 1.55\\CON8-6 & 678.99 & 0.50 & 
678.99 & 0.61 & \bf{678.92} & 
0.01 & 0.01\\CON8-7 & 826.17 & 0.63 & 
826.45 & 0.60 & \bf{811.96} & 
1.75 & 1.78\\CON8-8 & 789.39 & 1.04 & 
789.39 & 0.85 & \bf{767.53} & 
2.85 & 2.85\\CON8-9 & 820.66 & 0.62 & 
820.66 & 0.73 & \bf{809.00} & 
1.44 & 1.44\\\bf{PROM.} & 
\bf{769.38} & \bf{0.63} & \bf{772.17} & \bf{0.63} & \bf{758.54} & \bf{1.32} & \bf{1.66}\\[1ex]\hline
\end{tabular}
\label{table:nonlin}
\end{table} \clearpage
\begin{table}[ht]
\caption{Resultados de la ejecución de la metaheurística IGA, utilizando instancias de SalhiNagy con la configuración -n 200 -p 40 -cprob 30.0 -mprob 100.0}
\centering
\small
\begin{tabular}{c c c c c c c c}
\hline\hline
Instancia & Costo mínimo & Tiempo(seg.) & Costo promedio & Tiempo promedio(seg.) & CME & \%G & \%GP \\ [0.5ex]
\hline
CMT1X & 483.72 & 0.50 & 
484.03 & 0.73 & \bf{470.48} & 
2.81 & 2.88\\CMT1Y & 485.88 & 0.44 & 
487.42 & 0.59 & \bf{470.48} & 
3.27 & 3.60\\CMT2X & 711.60 & 1.06 & 
711.67 & 1.20 & \bf{682.39} & 
4.28 & 4.29\\CMT2Y & 713.68 & 1.02 & 
713.68 & 1.02 & \bf{682.39} & 
4.59 & 4.59\\CMT3X & 734.10 & 3.42 & 
743.49 & 3.06 & \bf{719.06} & 
2.09 & 3.40\\CMT3Y & 732.61 & 2.74 & 
736.88 & 2.97 & \bf{719.06} & 
1.88 & 2.48\\CMT4X & 912.21 & 8.58 & 
915.16 & 7.69 & \bf{854.21} & 
6.79 & 7.14\\CMT4Y & 915.40 & 7.40 & 
926.69 & 7.65 & \bf{852.46} & 
7.38 & 8.71\\CMT5X & 1115.79 & 16.65 & 
1123.73 & 15.68 & \bf{1030.56} & 
8.27 & 9.04\\CMT5Y & 1124.12 & 16.99 & 
1130.13 & 16.14 & \bf{1031.69} & 
8.96 & 9.54\\CMT11X & 912.61 & 4.48 & 
920.15 & 4.67 & \bf{831.09} & 
9.81 & 10.72\\CMT11Y & 917.60 & 5.56 & 
941.28 & 5.30 & \bf{829.85} & 
10.57 & 13.43\\CMT12X & 677.67 & 2.92 & 
683.32 & 2.91 & \bf{658.83} & 
2.86 & 3.72\\CMT12Y & 674.87 & 2.92 & 
675.66 & 2.96 & \bf{660.47} & 
2.18 & 2.30\\\bf{PROM.} & 
\bf{793.70} & \bf{5.33} & \bf{799.52} & \bf{5.18} & \bf{749.50} & \bf{5.41} & \bf{6.13}\\[1ex]\hline
\end{tabular}
\label{table:nonlin}
\end{table} \clearpage
\begin{table}[ht]
\caption{Resultados de la ejecución de la metaheurística IGA, utilizando instancias de Dethloff con la configuración -n 200 -p 40 -cprob 40.0 -mprob 10.0}
\centering
\small
\begin{tabular}{c c c c c c c c}
\hline\hline
Instancia & Costo mínimo & Tiempo(seg.) & Costo promedio & Tiempo promedio(seg.) & CME & \%G & \%GP \\ [0.5ex]
\hline
SCA3-0 & 636.06 & 0.70 & 
640.28 & 0.57 & \bf{635.62} & 
0.07 & 0.73\\SCA3-1 & 700.50 & 0.54 & 
701.01 & 0.66 & \bf{697.84} & 
0.38 & 0.45\\SCA3-2 & 669.06 & 0.52 & 
673.08 & 0.53 & \bf{659.34} & 
1.47 & 2.08\\SCA3-3 & \bf{680.04} & 0.69 & 
680.04 & 0.67 & 680.04 & 0.00
 & 0.00\\
SCA3-4 & \bf{690.50} & 0.67 & 
691.53 & 0.52 & 690.50 & 0.00
 & 0.15\\SCA3-5 & 680.80 & 0.68 & 
680.80 & 0.65 & \bf{659.90} & 
3.17 & 3.17\\SCA3-6 & 652.94 & 0.70 & 
652.94 & 0.65 & \bf{651.09} & 
0.28 & 0.28\\SCA3-7 & 671.77 & 0.47 & 
672.50 & 0.60 & \bf{659.17} & 
1.91 & 2.02\\SCA3-8 & 723.99 & 0.69 & 
724.07 & 0.55 & \bf{719.47} & 
0.63 & 0.64\\SCA3-9 & \bf{681.00} & 0.46 & 
681.00 & 0.53 & 681.00 & 0.00
 & 0.00\\
SCA8-0 & 998.79 & 0.50 & 
1008.02 & 0.80 & \bf{961.50} & 
3.88 & 4.84\\SCA8-1 & 1079.39 & 0.58 & 
1079.39 & 0.59 & \bf{1049.65} & 
2.83 & 2.83\\SCA8-2 & 1054.69 & 0.49 & 
1054.69 & 0.59 & \bf{1039.64} & 
1.45 & 1.45\\SCA8-3 & 1023.67 & 0.39 & 
1023.67 & 0.56 & \bf{983.34} & 
4.10 & 4.10\\SCA8-4 & 1091.52 & 0.57 & 
1091.52 & 0.48 & \bf{1065.49} & 
2.44 & 2.44\\SCA8-5 & 1040.87 & 0.75 & 
1040.87 & 0.67 & \bf{1027.08} & 
1.34 & 1.34\\SCA8-6 & 987.58 & 0.56 & 
989.51 & 0.57 & \bf{971.82} & 
1.62 & 1.82\\SCA8-7 & 1081.07 & 0.40 & 
1081.07 & 0.41 & \bf{1051.28} & 
2.83 & 2.83\\SCA8-8 & 1087.30 & 0.45 & 
1089.65 & 0.53 & \bf{1071.18} & 
1.50 & 1.72\\SCA8-9 & 1075.03 & 0.43 & 
1075.03 & 0.58 & \bf{1060.50} & 
1.37 & 1.37\\CON3-0 & 628.47 & 0.57 & 
628.47 & 0.52 & \bf{616.52} & 
1.94 & 1.94\\CON3-1 & 556.92 & 0.52 & 
560.75 & 0.60 & \bf{554.47} & 
0.44 & 1.13\\CON3-2 & 521.38 & 0.52 & 
524.01 & 0.64 & \bf{518.00} & 
0.65 & 1.16\\CON3-3 & 601.33 & 0.52 & 
601.95 & 0.51 & \bf{591.19} & 
1.72 & 1.82\\CON3-4 & 592.58 & 0.50 & 
597.26 & 0.65 & \bf{588.79} & 
0.64 & 1.44\\CON3-5 & 564.89 & 0.59 & 
566.81 & 0.55 & \bf{563.70} & 
0.21 & 0.55\\CON3-6 & 505.97 & 0.48 & 
508.88 & 0.50 & \bf{499.05} & 
1.39 & 1.97\\CON3-7 & 581.46 & 0.42 & 
583.13 & 0.42 & \bf{576.48} & 
0.86 & 1.15\\CON3-8 & 524.59 & 0.52 & 
524.59 & 0.53 & \bf{523.05} & 
0.29 & 0.29\\CON3-9 & 588.11 & 0.53 & 
588.75 & 0.62 & \bf{578.24} & 
1.71 & 1.82\\CON8-0 & 878.85 & 0.51 & 
878.85 & 0.53 & \bf{857.17} & 
2.53 & 2.53\\CON8-1 & 764.90 & 0.53 & 
764.90 & 0.55 & \bf{740.85} & 
3.25 & 3.25\\CON8-2 & 722.22 & 0.84 & 
722.22 & 0.78 & \bf{712.89} & 
1.31 & 1.31\\CON8-3 & 825.14 & 0.54 & 
833.86 & 0.65 & \bf{811.07} & 
1.73 & 2.81\\CON8-4 & 782.31 & 0.44 & 
794.30 & 0.57 & \bf{772.25} & 
1.30 & 2.86\\CON8-5 & 760.62 & 0.47 & 
760.62 & 0.58 & \bf{754.88} & 
0.76 & 0.76\\CON8-6 & 701.80 & 0.74 & 
702.46 & 0.62 & \bf{678.92} & 
3.37 & 3.47\\CON8-7 & 816.18 & 0.45 & 
820.93 & 0.53 & \bf{811.96} & 
0.52 & 1.11\\CON8-8 & 795.86 & 0.56 & 
796.92 & 0.57 & \bf{767.53} & 
3.69 & 3.83\\CON8-9 & 835.63 & 0.64 & 
839.62 & 0.82 & \bf{809.00} & 
3.29 & 3.78\\\bf{PROM.} & 
\bf{771.39} & \bf{0.55} & \bf{773.25} & \bf{0.59} & \bf{758.54} & \bf{1.57} & \bf{1.83}\\[1ex]\hline
\end{tabular}
\label{table:nonlin}
\end{table} \clearpage
\begin{table}[ht]
\caption{Resultados de la ejecución de la metaheurística IGA, utilizando instancias de SalhiNagy con la configuración -n 200 -p 40 -cprob 40.0 -mprob 10.0}
\centering
\small
\begin{tabular}{c c c c c c c c}
\hline\hline
Instancia & Costo mínimo & Tiempo(seg.) & Costo promedio & Tiempo promedio(seg.) & CME & \%G & \%GP \\ [0.5ex]
\hline
CMT1X & 478.84 & 0.46 & 
486.30 & 0.45 & \bf{470.48} & 
1.78 & 3.36\\CMT1Y & 481.64 & 0.59 & 
484.14 & 0.48 & \bf{470.48} & 
2.37 & 2.90\\CMT2X & 712.09 & 1.12 & 
719.28 & 1.18 & \bf{682.39} & 
4.35 & 5.41\\CMT2Y & 705.06 & 1.36 & 
710.07 & 1.22 & \bf{682.39} & 
3.32 & 4.06\\CMT3X & 729.33 & 2.98 & 
742.21 & 3.09 & \bf{719.06} & 
1.43 & 3.22\\CMT3Y & 738.62 & 2.45 & 
739.85 & 2.57 & \bf{719.06} & 
2.72 & 2.89\\CMT4X & 906.49 & 7.72 & 
913.06 & 8.01 & \bf{854.21} & 
6.12 & 6.89\\CMT4Y & 904.96 & 7.56 & 
915.36 & 7.46 & \bf{852.46} & 
6.16 & 7.38\\CMT5X & 1079.48 & 15.86 & 
1115.01 & 16.16 & \bf{1030.56} & 
4.75 & 8.19\\CMT5Y & 1108.23 & 17.50 & 
1124.05 & 16.66 & \bf{1031.69} & 
7.42 & 8.95\\CMT11X & 881.52 & 4.47 & 
908.50 & 5.00 & \bf{831.09} & 
6.07 & 9.31\\CMT11Y & 855.05 & 5.19 & 
878.80 & 5.28 & \bf{829.85} & 
3.04 & 5.90\\CMT12X & 675.69 & 2.75 & 
680.51 & 2.67 & \bf{658.83} & 
2.56 & 3.29\\CMT12Y & 675.07 & 3.28 & 
678.80 & 2.88 & \bf{660.47} & 
2.21 & 2.78\\\bf{PROM.} & 
\bf{780.86} & \bf{5.23} & \bf{792.57} & \bf{5.22} & \bf{749.50} & \bf{3.88} & \bf{5.32}\\[1ex]\hline
\end{tabular}
\label{table:nonlin}
\end{table} \clearpage
\begin{table}[ht]
\caption{Resultados de la ejecución de la metaheurística IGA, utilizando instancias de Dethloff con la configuración -n 200 -p 40 -cprob 40.0 -mprob 20.0}
\centering
\small
\begin{tabular}{c c c c c c c c}
\hline\hline
Instancia & Costo mínimo & Tiempo(seg.) & Costo promedio & Tiempo promedio(seg.) & CME & \%G & \%GP \\ [0.5ex]
\hline
SCA3-0 & 640.55 & 0.65 & 
640.55 & 0.58 & \bf{635.62} & 
0.78 & 0.78\\SCA3-1 & \bf{697.84} & 0.60 & 
697.84 & 0.50 & 697.84 & 0.00
 & 0.00\\
SCA3-2 & 664.18 & 0.54 & 
667.88 & 0.61 & \bf{659.34} & 
0.73 & 1.29\\SCA3-3 & 681.35 & 0.45 & 
682.19 & 0.50 & \bf{680.04} & 
0.19 & 0.32\\SCA3-4 & \bf{690.50} & 0.51 & 
690.50 & 0.52 & 690.50 & 0.00
 & 0.00\\
SCA3-5 & \bf{659.90} & 0.47 & 
670.55 & 0.56 & 659.90 & 0.00
 & 1.61\\SCA3-6 & 659.74 & 0.49 & 
660.12 & 0.66 & \bf{651.09} & 
1.33 & 1.39\\SCA3-7 & 666.15 & 0.46 & 
666.15 & 0.49 & \bf{659.17} & 
1.06 & 1.06\\SCA3-8 & \bf{719.47} & 0.56 & 
723.52 & 0.50 & 719.47 & 0.00
 & 0.56\\SCA3-9 & 683.37 & 0.43 & 
683.71 & 0.45 & \bf{681.00} & 
0.35 & 0.40\\SCA8-0 & 1008.41 & 0.66 & 
1010.14 & 0.76 & \bf{961.50} & 
4.88 & 5.06\\SCA8-1 & 1053.24 & 0.56 & 
1071.37 & 0.58 & \bf{1049.65} & 
0.34 & 2.07\\SCA8-2 & 1053.94 & 0.56 & 
1053.94 & 0.62 & \bf{1039.64} & 
1.38 & 1.38\\SCA8-3 & 1015.05 & 0.75 & 
1020.66 & 0.78 & \bf{983.34} & 
3.22 & 3.80\\SCA8-4 & 1086.49 & 0.72 & 
1086.49 & 0.70 & \bf{1065.49} & 
1.97 & 1.97\\SCA8-5 & 1052.74 & 0.76 & 
1052.74 & 0.56 & \bf{1027.08} & 
2.50 & 2.50\\SCA8-6 & 972.48 & 0.58 & 
979.28 & 0.64 & \bf{971.82} & 
0.07 & 0.77\\SCA8-7 & 1073.05 & 0.66 & 
1084.13 & 0.51 & \bf{1051.28} & 
2.07 & 3.12\\SCA8-8 & \bf{1071.18} & 0.87 & 
1082.50 & 0.77 & 1071.18 & 0.00
 & 1.06\\SCA8-9 & 1075.94 & 0.74 & 
1080.25 & 0.73 & \bf{1060.50} & 
1.46 & 1.86\\CON3-0 & 620.76 & 0.72 & 
625.54 & 0.56 & \bf{616.52} & 
0.69 & 1.46\\CON3-1 & 560.75 & 0.46 & 
560.75 & 0.47 & \bf{554.47} & 
1.13 & 1.13\\CON3-2 & 523.23 & 0.54 & 
524.48 & 0.55 & \bf{518.00} & 
1.01 & 1.25\\CON3-3 & 599.26 & 0.74 & 
599.83 & 0.56 & \bf{591.19} & 
1.37 & 1.46\\CON3-4 & 593.69 & 0.92 & 
595.50 & 0.64 & \bf{588.79} & 
0.83 & 1.14\\CON3-5 & \bf{563.70} & 0.44 & 
567.30 & 0.53 & 563.70 & 0.00
 & 0.64\\CON3-6 & 504.48 & 0.68 & 
504.81 & 0.61 & \bf{499.05} & 
1.09 & 1.15\\CON3-7 & 588.15 & 0.53 & 
589.45 & 0.65 & \bf{576.48} & 
2.02 & 2.25\\CON3-8 & 532.86 & 0.74 & 
534.67 & 0.72 & \bf{523.05} & 
1.88 & 2.22\\CON3-9 & 590.39 & 0.51 & 
591.02 & 0.60 & \bf{578.24} & 
2.10 & 2.21\\CON8-0 & 877.14 & 0.47 & 
879.69 & 0.59 & \bf{857.17} & 
2.33 & 2.63\\CON8-1 & 766.34 & 0.62 & 
769.36 & 0.60 & \bf{740.85} & 
3.44 & 3.85\\CON8-2 & 716.03 & 0.50 & 
718.23 & 0.64 & \bf{712.89} & 
0.44 & 0.75\\CON8-3 & 834.38 & 0.58 & 
834.38 & 0.63 & \bf{811.07} & 
2.87 & 2.87\\CON8-4 & 789.26 & 0.50 & 
790.30 & 0.56 & \bf{772.25} & 
2.20 & 2.34\\CON8-5 & 769.34 & 0.66 & 
769.44 & 0.57 & \bf{754.88} & 
1.92 & 1.93\\CON8-6 & 695.27 & 0.67 & 
695.27 & 0.75 & \bf{678.92} & 
2.41 & 2.41\\CON8-7 & 819.05 & 0.67 & 
819.05 & 0.63 & \bf{811.96} & 
0.87 & 0.87\\CON8-8 & 801.50 & 0.51 & 
801.50 & 0.48 & \bf{767.53} & 
4.43 & 4.43\\CON8-9 & 819.36 & 0.57 & 
821.44 & 0.59 & \bf{809.00} & 
1.28 & 1.54\\\bf{PROM.} & 
\bf{769.76} & \bf{0.60} & \bf{772.41} & \bf{0.60} & \bf{758.54} & \bf{1.42} & \bf{1.74}\\[1ex]\hline
\end{tabular}
\label{table:nonlin}
\end{table} \clearpage
\begin{table}[ht]
\caption{Resultados de la ejecución de la metaheurística IGA, utilizando instancias de SalhiNagy con la configuración -n 200 -p 40 -cprob 40.0 -mprob 20.0}
\centering
\small
\begin{tabular}{c c c c c c c c}
\hline\hline
Instancia & Costo mínimo & Tiempo(seg.) & Costo promedio & Tiempo promedio(seg.) & CME & \%G & \%GP \\ [0.5ex]
\hline
CMT1X & 477.72 & 0.42 & 
481.29 & 0.63 & \bf{470.48} & 
1.54 & 2.30\\CMT1Y & 484.27 & 0.39 & 
489.42 & 0.36 & \bf{470.48} & 
2.93 & 4.03\\CMT2X & 704.70 & 1.45 & 
718.96 & 1.34 & \bf{682.39} & 
3.27 & 5.36\\CMT2Y & 707.33 & 1.38 & 
714.50 & 1.16 & \bf{682.39} & 
3.65 & 4.71\\CMT3X & 738.16 & 3.34 & 
743.59 & 3.05 & \bf{719.06} & 
2.66 & 3.41\\CMT3Y & 750.48 & 3.16 & 
753.66 & 2.80 & \bf{719.06} & 
4.37 & 4.81\\CMT4X & 892.11 & 7.74 & 
908.14 & 7.91 & \bf{854.21} & 
4.44 & 6.31\\CMT4Y & 901.31 & 7.48 & 
911.23 & 7.62 & \bf{852.46} & 
5.73 & 6.89\\CMT5X & 1102.05 & 15.52 & 
1117.85 & 15.62 & \bf{1030.56} & 
6.94 & 8.47\\CMT5Y & 1119.67 & 16.79 & 
1130.47 & 16.35 & \bf{1031.69} & 
8.53 & 9.57\\CMT11X & 910.12 & 5.49 & 
918.01 & 4.96 & \bf{831.09} & 
9.51 & 10.46\\CMT11Y & 848.26 & 5.67 & 
888.19 & 5.33 & \bf{829.85} & 
2.22 & 7.03\\CMT12X & 676.74 & 3.36 & 
681.81 & 3.25 & \bf{658.83} & 
2.72 & 3.49\\CMT12Y & 675.08 & 2.47 & 
677.33 & 2.58 & \bf{660.47} & 
2.21 & 2.55\\\bf{PROM.} & 
\bf{784.86} & \bf{5.33} & \bf{795.32} & \bf{5.21} & \bf{749.50} & \bf{4.34} & \bf{5.67}\\[1ex]\hline
\end{tabular}
\label{table:nonlin}
\end{table} \clearpage
\begin{table}[ht]
\caption{Resultados de la ejecución de la metaheurística IGA, utilizando instancias de Dethloff con la configuración -n 200 -p 40 -cprob 40.0 -mprob 30.0}
\centering
\small
\begin{tabular}{c c c c c c c c}
\hline\hline
Instancia & Costo mínimo & Tiempo(seg.) & Costo promedio & Tiempo promedio(seg.) & CME & \%G & \%GP \\ [0.5ex]
\hline
SCA3-0 & 640.55 & 0.43 & 
640.55 & 0.57 & \bf{635.62} & 
0.78 & 0.78\\SCA3-1 & 700.50 & 0.49 & 
700.50 & 0.60 & \bf{697.84} & 
0.38 & 0.38\\SCA3-2 & 666.01 & 0.67 & 
672.50 & 0.61 & \bf{659.34} & 
1.01 & 2.00\\SCA3-3 & \bf{680.04} & 0.46 & 
683.15 & 0.62 & 680.04 & 0.00
 & 0.46\\SCA3-4 & \bf{690.50} & 0.69 & 
690.50 & 0.59 & 690.50 & 0.00
 & 0.00\\
SCA3-5 & 666.67 & 0.57 & 
666.67 & 0.67 & \bf{659.90} & 
1.03 & 1.03\\SCA3-6 & 657.24 & 0.48 & 
659.18 & 0.49 & \bf{651.09} & 
0.94 & 1.24\\SCA3-7 & 666.60 & 0.56 & 
667.08 & 0.49 & \bf{659.17} & 
1.13 & 1.20\\SCA3-8 & \bf{719.47} & 0.55 & 
725.14 & 0.62 & 719.47 & 0.00
 & 0.79\\SCA3-9 & \bf{681.00} & 0.42 & 
681.00 & 0.45 & 681.00 & 0.00
 & 0.00\\
SCA8-0 & 997.56 & 1.01 & 
1013.19 & 0.67 & \bf{961.50} & 
3.75 & 5.38\\SCA8-1 & 1074.33 & 0.57 & 
1078.12 & 0.57 & \bf{1049.65} & 
2.35 & 2.71\\SCA8-2 & 1054.47 & 0.79 & 
1054.47 & 0.71 & \bf{1039.64} & 
1.43 & 1.43\\SCA8-3 & 1024.13 & 0.58 & 
1024.13 & 0.68 & \bf{983.34} & 
4.15 & 4.15\\SCA8-4 & 1085.64 & 0.42 & 
1108.93 & 0.47 & \bf{1065.49} & 
1.89 & 4.08\\SCA8-5 & 1068.56 & 0.52 & 
1070.36 & 0.61 & \bf{1027.08} & 
4.04 & 4.21\\SCA8-6 & 979.28 & 0.56 & 
982.03 & 0.96 & \bf{971.82} & 
0.77 & 1.05\\SCA8-7 & 1086.89 & 0.55 & 
1086.89 & 0.52 & \bf{1051.28} & 
3.39 & 3.39\\SCA8-8 & 1086.29 & 0.75 & 
1090.60 & 0.55 & \bf{1071.18} & 
1.41 & 1.81\\SCA8-9 & \bf{1060.50} & 0.48 & 
1079.90 & 0.45 & 1060.50 & 0.00
 & 1.83\\CON3-0 & 620.76 & 0.56 & 
625.60 & 0.54 & \bf{616.52} & 
0.69 & 1.47\\CON3-1 & 560.75 & 0.59 & 
561.03 & 0.59 & \bf{554.47} & 
1.13 & 1.18\\CON3-2 & 521.38 & 0.53 & 
522.18 & 0.66 & \bf{518.00} & 
0.65 & 0.81\\CON3-3 & 601.66 & 0.64 & 
601.66 & 0.62 & \bf{591.19} & 
1.77 & 1.77\\CON3-4 & 593.78 & 0.52 & 
597.21 & 0.53 & \bf{588.79} & 
0.85 & 1.43\\CON3-5 & \bf{563.70} & 0.53 & 
566.88 & 0.65 & 563.70 & 0.00
 & 0.56\\CON3-6 & 505.01 & 0.76 & 
505.78 & 0.67 & \bf{499.05} & 
1.19 & 1.35\\CON3-7 & 586.01 & 0.54 & 
586.01 & 0.66 & \bf{576.48} & 
1.65 & 1.65\\CON3-8 & 526.59 & 0.49 & 
533.38 & 0.51 & \bf{523.05} & 
0.68 & 1.97\\CON3-9 & 582.79 & 0.60 & 
586.95 & 0.66 & \bf{578.24} & 
0.79 & 1.51\\CON8-0 & 875.09 & 0.58 & 
875.09 & 0.57 & \bf{857.17} & 
2.09 & 2.09\\CON8-1 & 753.35 & 0.79 & 
763.69 & 0.65 & \bf{740.85} & 
1.69 & 3.08\\CON8-2 & 720.86 & 0.57 & 
724.92 & 0.63 & \bf{712.89} & 
1.12 & 1.69\\CON8-3 & 832.99 & 0.53 & 
832.99 & 0.52 & \bf{811.07} & 
2.70 & 2.70\\CON8-4 & 780.03 & 0.44 & 
780.03 & 0.58 & \bf{772.25} & 
1.01 & 1.01\\CON8-5 & 766.37 & 0.57 & 
775.36 & 0.54 & \bf{754.88} & 
1.52 & 2.71\\CON8-6 & 696.03 & 0.60 & 
696.03 & 0.65 & \bf{678.92} & 
2.52 & 2.52\\CON8-7 & 815.82 & 0.62 & 
815.82 & 0.66 & \bf{811.96} & 
0.48 & 0.48\\CON8-8 & \bf{767.53} & 0.93 & 
780.54 & 0.75 & 767.53 & 0.00
 & 1.70\\CON8-9 & 833.43 & 0.79 & 
833.43 & 0.79 & \bf{809.00} & 
3.02 & 3.02\\\bf{PROM.} & 
\bf{769.75} & \bf{0.59} & \bf{773.49} & \bf{0.61} & \bf{758.54} & \bf{1.35} & \bf{1.82}\\[1ex]\hline
\end{tabular}
\label{table:nonlin}
\end{table} \clearpage
\begin{table}[ht]
\caption{Resultados de la ejecución de la metaheurística IGA, utilizando instancias de SalhiNagy con la configuración -n 200 -p 40 -cprob 40.0 -mprob 30.0}
\centering
\small
\begin{tabular}{c c c c c c c c}
\hline\hline
Instancia & Costo mínimo & Tiempo(seg.) & Costo promedio & Tiempo promedio(seg.) & CME & \%G & \%GP \\ [0.5ex]
\hline
CMT1X & 478.84 & 0.39 & 
482.89 & 0.45 & \bf{470.48} & 
1.78 & 2.64\\CMT1Y & 484.89 & 0.36 & 
486.87 & 0.35 & \bf{470.48} & 
3.06 & 3.48\\CMT2X & 713.21 & 1.39 & 
715.18 & 1.31 & \bf{682.39} & 
4.52 & 4.81\\CMT2Y & 707.44 & 1.32 & 
709.44 & 1.24 & \bf{682.39} & 
3.67 & 3.96\\CMT3X & 741.47 & 2.59 & 
747.36 & 2.90 & \bf{719.06} & 
3.12 & 3.93\\CMT3Y & 726.15 & 3.15 & 
732.74 & 2.86 & \bf{719.06} & 
0.99 & 1.90\\CMT4X & 908.90 & 8.07 & 
916.45 & 8.16 & \bf{854.21} & 
6.40 & 7.29\\CMT4Y & 907.79 & 7.47 & 
918.40 & 7.77 & \bf{852.46} & 
6.49 & 7.74\\CMT5X & 1119.47 & 16.38 & 
1121.03 & 16.30 & \bf{1030.56} & 
8.63 & 8.78\\CMT5Y & 1106.23 & 16.18 & 
1128.88 & 15.89 & \bf{1031.69} & 
7.23 & 9.42\\CMT11X & 886.21 & 4.37 & 
904.52 & 4.41 & \bf{831.09} & 
6.63 & 8.84\\CMT11Y & 886.89 & 5.49 & 
898.73 & 5.21 & \bf{829.85} & 
6.87 & 8.30\\CMT12X & 675.78 & 2.57 & 
681.76 & 2.65 & \bf{658.83} & 
2.57 & 3.48\\CMT12Y & 675.14 & 3.10 & 
678.27 & 2.81 & \bf{660.47} & 
2.22 & 2.70\\\bf{PROM.} & 
\bf{787.03} & \bf{5.20} & \bf{794.47} & \bf{5.16} & \bf{749.50} & \bf{4.58} & \bf{5.52}\\[1ex]\hline
\end{tabular}
\label{table:nonlin}
\end{table} \clearpage
\begin{table}[ht]
\caption{Resultados de la ejecución de la metaheurística IGA, utilizando instancias de Dethloff con la configuración -n 200 -p 40 -cprob 40.0 -mprob 40.0}
\centering
\small
\begin{tabular}{c c c c c c c c}
\hline\hline
Instancia & Costo mínimo & Tiempo(seg.) & Costo promedio & Tiempo promedio(seg.) & CME & \%G & \%GP \\ [0.5ex]
\hline
SCA3-0 & 640.55 & 0.58 & 
640.55 & 0.61 & \bf{635.62} & 
0.78 & 0.78\\SCA3-1 & \bf{697.84} & 0.58 & 
697.84 & 0.60 & 697.84 & 0.00
 & 0.00\\
SCA3-2 & 666.72 & 0.54 & 
671.40 & 0.59 & \bf{659.34} & 
1.12 & 1.83\\SCA3-3 & 681.35 & 0.54 & 
682.18 & 0.55 & \bf{680.04} & 
0.19 & 0.32\\SCA3-4 & \bf{690.50} & 0.41 & 
691.02 & 0.60 & 690.50 & 0.00
 & 0.08\\SCA3-5 & 662.75 & 0.58 & 
665.43 & 0.60 & \bf{659.90} & 
0.43 & 0.84\\SCA3-6 & 655.05 & 0.53 & 
658.14 & 0.59 & \bf{651.09} & 
0.61 & 1.08\\SCA3-7 & 666.15 & 0.45 & 
666.15 & 0.53 & \bf{659.17} & 
1.06 & 1.06\\SCA3-8 & 724.29 & 0.68 & 
724.29 & 0.64 & \bf{719.47} & 
0.67 & 0.67\\SCA3-9 & 685.88 & 0.48 & 
685.88 & 0.54 & \bf{681.00} & 
0.72 & 0.72\\SCA8-0 & 994.55 & 0.73 & 
1008.34 & 0.69 & \bf{961.50} & 
3.44 & 4.87\\SCA8-1 & 1077.76 & 0.93 & 
1085.88 & 0.78 & \bf{1049.65} & 
2.68 & 3.45\\SCA8-2 & 1052.94 & 0.48 & 
1054.09 & 0.56 & \bf{1039.64} & 
1.28 & 1.39\\SCA8-3 & 1024.98 & 0.52 & 
1031.11 & 0.54 & \bf{983.34} & 
4.23 & 4.86\\SCA8-4 & 1091.52 & 0.56 & 
1092.97 & 0.56 & \bf{1065.49} & 
2.44 & 2.58\\SCA8-5 & 1061.96 & 0.47 & 
1061.96 & 0.49 & \bf{1027.08} & 
3.40 & 3.40\\SCA8-6 & 995.45 & 0.56 & 
998.57 & 0.57 & \bf{971.82} & 
2.43 & 2.75\\SCA8-7 & 1084.34 & 0.70 & 
1091.38 & 0.73 & \bf{1051.28} & 
3.14 & 3.81\\SCA8-8 & \bf{1071.18} & 0.66 & 
1071.18 & 0.58 & 1071.18 & 0.00
 & 0.00\\
SCA8-9 & 1072.28 & 0.74 & 
1076.80 & 0.67 & \bf{1060.50} & 
1.11 & 1.54\\CON3-0 & 620.76 & 0.53 & 
630.28 & 0.54 & \bf{616.52} & 
0.69 & 2.23\\CON3-1 & 560.75 & 0.56 & 
560.75 & 0.75 & \bf{554.47} & 
1.13 & 1.13\\CON3-2 & 521.38 & 0.77 & 
521.38 & 0.71 & \bf{518.00} & 
0.65 & 0.65\\CON3-3 & \bf{591.19} & 0.46 & 
599.58 & 0.59 & 591.19 & 0.00
 & 1.42\\CON3-4 & 592.58 & 0.49 & 
598.05 & 0.64 & \bf{588.79} & 
0.64 & 1.57\\CON3-5 & 564.88 & 0.78 & 
569.78 & 0.66 & \bf{563.70} & 
0.21 & 1.08\\CON3-6 & 502.16 & 0.87 & 
506.00 & 0.69 & \bf{499.05} & 
0.62 & 1.39\\CON3-7 & 582.14 & 0.63 & 
583.68 & 0.62 & \bf{576.48} & 
0.98 & 1.25\\CON3-8 & 535.24 & 0.93 & 
535.24 & 0.72 & \bf{523.05} & 
2.33 & 2.33\\CON3-9 & 582.79 & 0.59 & 
589.64 & 0.58 & \bf{578.24} & 
0.79 & 1.97\\CON8-0 & 874.06 & 0.42 & 
876.65 & 0.48 & \bf{857.17} & 
1.97 & 2.27\\CON8-1 & 763.44 & 0.62 & 
763.44 & 0.77 & \bf{740.85} & 
3.05 & 3.05\\CON8-2 & 730.75 & 0.59 & 
730.75 & 0.72 & \bf{712.89} & 
2.51 & 2.51\\CON8-3 & 838.16 & 0.74 & 
838.16 & 0.76 & \bf{811.07} & 
3.34 & 3.34\\CON8-4 & 794.34 & 0.70 & 
794.34 & 0.62 & \bf{772.25} & 
2.86 & 2.86\\CON8-5 & 769.35 & 0.52 & 
771.84 & 0.54 & \bf{754.88} & 
1.92 & 2.25\\CON8-6 & 678.99 & 0.50 & 
685.99 & 0.56 & \bf{678.92} & 
0.01 & 1.04\\CON8-7 & 832.92 & 0.70 & 
832.92 & 0.65 & \bf{811.96} & 
2.58 & 2.58\\CON8-8 & 787.45 & 0.58 & 
787.45 & 0.70 & \bf{767.53} & 
2.60 & 2.60\\CON8-9 & 824.27 & 0.57 & 
837.46 & 0.67 & \bf{809.00} & 
1.89 & 3.52\\\bf{PROM.} & 
\bf{771.14} & \bf{0.61} & \bf{774.21} & \bf{0.62} & \bf{758.54} & \bf{1.51} & \bf{1.93}\\[1ex]\hline
\end{tabular}
\label{table:nonlin}
\end{table} \clearpage
\begin{table}[ht]
\caption{Resultados de la ejecución de la metaheurística IGA, utilizando instancias de SalhiNagy con la configuración -n 200 -p 40 -cprob 40.0 -mprob 40.0}
\centering
\small
\begin{tabular}{c c c c c c c c}
\hline\hline
Instancia & Costo mínimo & Tiempo(seg.) & Costo promedio & Tiempo promedio(seg.) & CME & \%G & \%GP \\ [0.5ex]
\hline
CMT1X & 479.32 & 0.46 & 
480.46 & 0.53 & \bf{470.48} & 
1.88 & 2.12\\CMT1Y & 475.37 & 0.54 & 
484.61 & 0.65 & \bf{470.48} & 
1.04 & 3.00\\CMT2X & 703.29 & 1.27 & 
711.03 & 1.23 & \bf{682.39} & 
3.06 & 4.20\\CMT2Y & 702.76 & 1.16 & 
706.16 & 1.19 & \bf{682.39} & 
2.99 & 3.48\\CMT3X & 736.85 & 2.72 & 
742.96 & 2.68 & \bf{719.06} & 
2.47 & 3.32\\CMT3Y & 735.17 & 2.76 & 
737.11 & 2.55 & \bf{719.06} & 
2.24 & 2.51\\CMT4X & 896.02 & 7.50 & 
911.99 & 7.64 & \bf{854.21} & 
4.89 & 6.76\\CMT4Y & 896.00 & 7.58 & 
919.11 & 7.84 & \bf{852.46} & 
5.11 & 7.82\\CMT5X & 1105.64 & 14.75 & 
1127.72 & 15.61 & \bf{1030.56} & 
7.29 & 9.43\\CMT5Y & 1107.89 & 17.24 & 
1129.69 & 16.08 & \bf{1031.69} & 
7.39 & 9.50\\CMT11X & 883.35 & 5.32 & 
895.63 & 5.04 & \bf{831.09} & 
6.29 & 7.77\\CMT11Y & 905.65 & 5.90 & 
915.11 & 5.48 & \bf{829.85} & 
9.13 & 10.27\\CMT12X & 677.62 & 3.42 & 
681.92 & 2.97 & \bf{658.83} & 
2.85 & 3.51\\CMT12Y & 674.63 & 3.00 & 
676.57 & 2.87 & \bf{660.47} & 
2.14 & 2.44\\\bf{PROM.} & 
\bf{784.25} & \bf{5.26} & \bf{794.29} & \bf{5.17} & \bf{749.50} & \bf{4.20} & \bf{5.44}\\[1ex]\hline
\end{tabular}
\label{table:nonlin}
\end{table} \clearpage
\begin{table}[ht]
\caption{Resultados de la ejecución de la metaheurística IGA, utilizando instancias de Dethloff con la configuración -n 200 -p 40 -cprob 40.0 -mprob 50.0}
\centering
\small
\begin{tabular}{c c c c c c c c}
\hline\hline
Instancia & Costo mínimo & Tiempo(seg.) & Costo promedio & Tiempo promedio(seg.) & CME & \%G & \%GP \\ [0.5ex]
\hline
SCA3-0 & 640.55 & 0.76 & 
641.12 & 0.68 & \bf{635.62} & 
0.78 & 0.87\\SCA3-1 & 706.90 & 0.74 & 
706.90 & 0.80 & \bf{697.84} & 
1.30 & 1.30\\SCA3-2 & \bf{659.34} & 0.45 & 
661.90 & 0.47 & 659.34 & 0.00
 & 0.39\\SCA3-3 & \bf{680.04} & 0.71 & 
680.79 & 0.61 & 680.04 & 0.00
 & 0.11\\SCA3-4 & \bf{690.50} & 0.91 & 
690.50 & 0.63 & 690.50 & 0.00
 & 0.00\\
SCA3-5 & 665.64 & 0.49 & 
665.64 & 0.58 & \bf{659.90} & 
0.87 & 0.87\\SCA3-6 & 652.94 & 0.52 & 
658.42 & 0.72 & \bf{651.09} & 
0.28 & 1.13\\SCA3-7 & 669.89 & 0.84 & 
670.78 & 0.64 & \bf{659.17} & 
1.63 & 1.76\\SCA3-8 & 722.05 & 0.50 & 
724.22 & 0.52 & \bf{719.47} & 
0.36 & 0.66\\SCA3-9 & \bf{681.00} & 0.73 & 
681.00 & 0.76 & 681.00 & 0.00
 & 0.00\\
SCA8-0 & 989.47 & 0.82 & 
989.47 & 0.74 & \bf{961.50} & 
2.91 & 2.91\\SCA8-1 & 1078.79 & 0.59 & 
1082.01 & 0.62 & \bf{1049.65} & 
2.78 & 3.08\\SCA8-2 & 1054.47 & 0.48 & 
1054.47 & 0.64 & \bf{1039.64} & 
1.43 & 1.43\\SCA8-3 & 1018.67 & 0.69 & 
1018.96 & 0.67 & \bf{983.34} & 
3.59 & 3.62\\SCA8-4 & 1069.71 & 0.61 & 
1069.71 & 0.53 & \bf{1065.49} & 
0.40 & 0.40\\SCA8-5 & 1077.05 & 0.57 & 
1078.55 & 0.59 & \bf{1027.08} & 
4.87 & 5.01\\SCA8-6 & 997.29 & 0.59 & 
997.29 & 0.60 & \bf{971.82} & 
2.62 & 2.62\\SCA8-7 & 1080.57 & 0.74 & 
1080.57 & 0.68 & \bf{1051.28} & 
2.79 & 2.79\\SCA8-8 & 1091.18 & 0.72 & 
1091.85 & 0.63 & \bf{1071.18} & 
1.87 & 1.93\\SCA8-9 & 1081.23 & 0.70 & 
1081.23 & 0.47 & \bf{1060.50} & 
1.95 & 1.95\\CON3-0 & 628.08 & 0.73 & 
633.08 & 0.62 & \bf{616.52} & 
1.88 & 2.69\\CON3-1 & 560.61 & 0.52 & 
561.68 & 0.53 & \bf{554.47} & 
1.11 & 1.30\\CON3-2 & 521.38 & 0.82 & 
524.01 & 0.81 & \bf{518.00} & 
0.65 & 1.16\\CON3-3 & 592.41 & 0.74 & 
592.50 & 0.61 & \bf{591.19} & 
0.21 & 0.22\\CON3-4 & 593.78 & 0.58 & 
593.78 & 0.69 & \bf{588.79} & 
0.85 & 0.85\\CON3-5 & 567.94 & 0.48 & 
568.35 & 0.56 & \bf{563.70} & 
0.75 & 0.82\\CON3-6 & 505.41 & 0.49 & 
508.30 & 0.63 & \bf{499.05} & 
1.27 & 1.85\\CON3-7 & 578.41 & 0.64 & 
583.97 & 0.64 & \bf{576.48} & 
0.33 & 1.30\\CON3-8 & 524.59 & 0.51 & 
524.59 & 0.54 & \bf{523.05} & 
0.29 & 0.29\\CON3-9 & 588.11 & 0.74 & 
589.71 & 0.72 & \bf{578.24} & 
1.71 & 1.98\\CON8-0 & 881.00 & 0.96 & 
889.03 & 0.91 & \bf{857.17} & 
2.78 & 3.72\\CON8-1 & 763.74 & 0.77 & 
763.74 & 0.64 & \bf{740.85} & 
3.09 & 3.09\\CON8-2 & 718.80 & 0.79 & 
718.80 & 0.68 & \bf{712.89} & 
0.83 & 0.83\\CON8-3 & 833.31 & 0.62 & 
833.31 & 0.59 & \bf{811.07} & 
2.74 & 2.74\\CON8-4 & 806.03 & 0.44 & 
810.01 & 0.46 & \bf{772.25} & 
4.37 & 4.89\\CON8-5 & 759.82 & 0.56 & 
759.82 & 0.60 & \bf{754.88} & 
0.65 & 0.65\\CON8-6 & 696.02 & 0.57 & 
701.54 & 0.55 & \bf{678.92} & 
2.52 & 3.33\\CON8-7 & 814.50 & 0.56 & 
823.87 & 0.53 & \bf{811.96} & 
0.31 & 1.47\\CON8-8 & 788.35 & 0.53 & 
788.35 & 0.82 & \bf{767.53} & 
2.71 & 2.71\\CON8-9 & 838.27 & 0.53 & 
840.67 & 0.61 & \bf{809.00} & 
3.62 & 3.92\\\bf{PROM.} & 
\bf{771.70} & \bf{0.64} & \bf{773.36} & \bf{0.63} & \bf{758.54} & \bf{1.58} & \bf{1.82}\\[1ex]\hline
\end{tabular}
\label{table:nonlin}
\end{table} \clearpage
\begin{table}[ht]
\caption{Resultados de la ejecución de la metaheurística IGA, utilizando instancias de SalhiNagy con la configuración -n 200 -p 40 -cprob 40.0 -mprob 50.0}
\centering
\small
\begin{tabular}{c c c c c c c c}
\hline\hline
Instancia & Costo mínimo & Tiempo(seg.) & Costo promedio & Tiempo promedio(seg.) & CME & \%G & \%GP \\ [0.5ex]
\hline
CMT1X & 479.97 & 0.40 & 
481.19 & 0.60 & \bf{470.48} & 
2.02 & 2.28\\CMT1Y & 479.45 & 0.41 & 
479.45 & 0.41 & \bf{470.48} & 
1.91 & 1.91\\CMT2X & 705.01 & 1.63 & 
712.47 & 1.49 & \bf{682.39} & 
3.31 & 4.41\\CMT2Y & 704.71 & 1.38 & 
713.42 & 1.19 & \bf{682.39} & 
3.27 & 4.55\\CMT3X & 740.54 & 3.35 & 
743.72 & 3.08 & \bf{719.06} & 
2.99 & 3.43\\CMT3Y & 743.54 & 2.58 & 
749.02 & 2.53 & \bf{719.06} & 
3.40 & 4.17\\CMT4X & 896.52 & 7.27 & 
906.87 & 7.51 & \bf{854.21} & 
4.95 & 6.16\\CMT4Y & 903.56 & 8.40 & 
917.70 & 8.07 & \bf{852.46} & 
5.99 & 7.65\\CMT5X & 1114.13 & 15.71 & 
1128.88 & 15.76 & \bf{1030.56} & 
8.11 & 9.54\\CMT5Y & 1100.28 & 16.96 & 
1119.45 & 16.68 & \bf{1031.69} & 
6.65 & 8.51\\CMT11X & 907.44 & 4.31 & 
916.42 & 4.54 & \bf{831.09} & 
9.19 & 10.27\\CMT11Y & 911.90 & 4.90 & 
916.90 & 5.10 & \bf{829.85} & 
9.89 & 10.49\\CMT12X & 679.09 & 3.16 & 
687.11 & 2.89 & \bf{658.83} & 
3.08 & 4.29\\CMT12Y & 675.11 & 2.56 & 
676.18 & 2.70 & \bf{660.47} & 
2.22 & 2.38\\\bf{PROM.} & 
\bf{788.66} & \bf{5.22} & \bf{796.34} & \bf{5.18} & \bf{749.50} & \bf{4.78} & \bf{5.72}\\[1ex]\hline
\end{tabular}
\label{table:nonlin}
\end{table} \clearpage
\begin{table}[ht]
\caption{Resultados de la ejecución de la metaheurística IGA, utilizando instancias de Dethloff con la configuración -n 200 -p 40 -cprob 40.0 -mprob 60.0}
\centering
\small
\begin{tabular}{c c c c c c c c}
\hline\hline
Instancia & Costo mínimo & Tiempo(seg.) & Costo promedio & Tiempo promedio(seg.) & CME & \%G & \%GP \\ [0.5ex]
\hline
SCA3-0 & 640.55 & 0.72 & 
640.84 & 0.62 & \bf{635.62} & 
0.78 & 0.82\\SCA3-1 & \bf{697.84} & 0.53 & 
697.84 & 0.63 & 697.84 & 0.00
 & 0.00\\
SCA3-2 & 666.01 & 0.63 & 
672.36 & 0.70 & \bf{659.34} & 
1.01 & 1.97\\SCA3-3 & 681.35 & 0.42 & 
681.63 & 0.56 & \bf{680.04} & 
0.19 & 0.23\\SCA3-4 & \bf{690.50} & 0.90 & 
690.50 & 0.61 & 690.50 & 0.00
 & 0.00\\
SCA3-5 & 662.75 & 0.60 & 
662.75 & 0.74 & \bf{659.90} & 
0.43 & 0.43\\SCA3-6 & 652.94 & 0.64 & 
652.94 & 0.61 & \bf{651.09} & 
0.28 & 0.28\\SCA3-7 & 666.15 & 0.54 & 
666.49 & 0.51 & \bf{659.17} & 
1.06 & 1.11\\SCA3-8 & \bf{719.47} & 0.72 & 
724.84 & 0.85 & 719.47 & 0.00
 & 0.75\\SCA3-9 & \bf{681.00} & 0.60 & 
682.62 & 0.57 & 681.00 & 0.00
 & 0.24\\SCA8-0 & 1013.67 & 0.76 & 
1015.63 & 0.62 & \bf{961.50} & 
5.43 & 5.63\\SCA8-1 & 1080.46 & 0.64 & 
1082.04 & 0.59 & \bf{1049.65} & 
2.94 & 3.09\\SCA8-2 & 1054.47 & 0.43 & 
1056.69 & 0.56 & \bf{1039.64} & 
1.43 & 1.64\\SCA8-3 & 1014.10 & 0.62 & 
1025.37 & 0.61 & \bf{983.34} & 
3.13 & 4.27\\SCA8-4 & 1074.78 & 1.01 & 
1076.23 & 0.81 & \bf{1065.49} & 
0.87 & 1.01\\SCA8-5 & 1050.17 & 0.69 & 
1051.57 & 0.58 & \bf{1027.08} & 
2.25 & 2.38\\SCA8-6 & 988.79 & 0.77 & 
997.14 & 0.70 & \bf{971.82} & 
1.75 & 2.61\\SCA8-7 & 1070.72 & 0.61 & 
1079.97 & 0.57 & \bf{1051.28} & 
1.85 & 2.73\\SCA8-8 & 1085.34 & 0.55 & 
1085.57 & 0.75 & \bf{1071.18} & 
1.32 & 1.34\\SCA8-9 & 1084.09 & 0.42 & 
1084.09 & 0.49 & \bf{1060.50} & 
2.22 & 2.22\\CON3-0 & 621.82 & 0.91 & 
623.38 & 0.84 & \bf{616.52} & 
0.86 & 1.11\\CON3-1 & 557.21 & 0.93 & 
560.61 & 0.76 & \bf{554.47} & 
0.49 & 1.11\\CON3-2 & 521.38 & 0.78 & 
522.91 & 0.75 & \bf{518.00} & 
0.65 & 0.95\\CON3-3 & 601.41 & 0.80 & 
603.62 & 0.74 & \bf{591.19} & 
1.73 & 2.10\\CON3-4 & 593.78 & 0.42 & 
596.13 & 0.54 & \bf{588.79} & 
0.85 & 1.25\\CON3-5 & 567.07 & 0.52 & 
567.07 & 0.53 & \bf{563.70} & 
0.60 & 0.60\\CON3-6 & 503.97 & 0.68 & 
504.01 & 0.74 & \bf{499.05} & 
0.99 & 0.99\\CON3-7 & 582.14 & 0.90 & 
587.00 & 0.71 & \bf{576.48} & 
0.98 & 1.83\\CON3-8 & 523.14 & 0.70 & 
528.36 & 0.59 & \bf{523.05} & 
0.02 & 1.02\\CON3-9 & 588.18 & 0.74 & 
589.31 & 0.69 & \bf{578.24} & 
1.72 & 1.91\\CON8-0 & 879.00 & 0.52 & 
883.87 & 0.56 & \bf{857.17} & 
2.55 & 3.11\\CON8-1 & \bf{740.85} & 0.56 & 
753.21 & 0.61 & 740.85 & 0.00
 & 1.67\\CON8-2 & 720.47 & 0.69 & 
720.47 & 0.72 & \bf{712.89} & 
1.06 & 1.06\\CON8-3 & 827.08 & 0.93 & 
840.03 & 0.71 & \bf{811.07} & 
1.97 & 3.57\\CON8-4 & 785.04 & 0.77 & 
785.04 & 0.61 & \bf{772.25} & 
1.66 & 1.66\\CON8-5 & 758.12 & 0.51 & 
761.25 & 0.58 & \bf{754.88} & 
0.43 & 0.84\\CON8-6 & 696.07 & 0.68 & 
698.23 & 0.86 & \bf{678.92} & 
2.53 & 2.84\\CON8-7 & 830.80 & 0.68 & 
832.08 & 0.63 & \bf{811.96} & 
2.32 & 2.48\\CON8-8 & 795.76 & 0.54 & 
798.33 & 0.66 & \bf{767.53} & 
3.68 & 4.01\\CON8-9 & 820.95 & 0.59 & 
824.69 & 0.67 & \bf{809.00} & 
1.48 & 1.94\\\bf{PROM.} & 
\bf{769.73} & \bf{0.67} & \bf{772.67} & \bf{0.66} & \bf{758.54} & \bf{1.34} & \bf{1.72}\\[1ex]\hline
\end{tabular}
\label{table:nonlin}
\end{table} \clearpage
\begin{table}[ht]
\caption{Resultados de la ejecución de la metaheurística IGA, utilizando instancias de SalhiNagy con la configuración -n 200 -p 40 -cprob 40.0 -mprob 60.0}
\centering
\small
\begin{tabular}{c c c c c c c c}
\hline\hline
Instancia & Costo mínimo & Tiempo(seg.) & Costo promedio & Tiempo promedio(seg.) & CME & \%G & \%GP \\ [0.5ex]
\hline
CMT1X & 479.21 & 0.42 & 
482.52 & 0.63 & \bf{470.48} & 
1.86 & 2.56\\CMT1Y & 476.71 & 0.57 & 
482.58 & 0.59 & \bf{470.48} & 
1.32 & 2.57\\CMT2X & 718.47 & 1.28 & 
722.04 & 1.24 & \bf{682.39} & 
5.29 & 5.81\\CMT2Y & 709.96 & 1.04 & 
714.92 & 1.14 & \bf{682.39} & 
4.04 & 4.77\\CMT3X & 735.98 & 2.87 & 
744.87 & 2.73 & \bf{719.06} & 
2.35 & 3.59\\CMT3Y & 744.97 & 3.47 & 
748.85 & 3.10 & \bf{719.06} & 
3.60 & 4.14\\CMT4X & 890.31 & 7.65 & 
919.28 & 7.63 & \bf{854.21} & 
4.23 & 7.62\\CMT4Y & 897.93 & 7.91 & 
910.43 & 7.81 & \bf{852.46} & 
5.33 & 6.80\\CMT5X & 1118.23 & 16.13 & 
1124.46 & 15.99 & \bf{1030.56} & 
8.51 & 9.11\\CMT5Y & 1109.95 & 15.40 & 
1122.41 & 16.19 & \bf{1031.69} & 
7.59 & 8.79\\CMT11X & 917.50 & 4.89 & 
918.07 & 5.07 & \bf{831.09} & 
10.40 & 10.47\\CMT11Y & 892.00 & 6.20 & 
907.99 & 5.65 & \bf{829.85} & 
7.49 & 9.42\\CMT12X & 679.77 & 3.16 & 
683.86 & 2.93 & \bf{658.83} & 
3.18 & 3.80\\CMT12Y & 680.27 & 2.42 & 
683.75 & 2.95 & \bf{660.47} & 
3.00 & 3.53\\\bf{PROM.} & 
\bf{789.38} & \bf{5.24} & \bf{797.57} & \bf{5.26} & \bf{749.50} & \bf{4.87} & \bf{5.93}\\[1ex]\hline
\end{tabular}
\label{table:nonlin}
\end{table} \clearpage
\begin{table}[ht]
\caption{Resultados de la ejecución de la metaheurística IGA, utilizando instancias de Dethloff con la configuración -n 200 -p 40 -cprob 40.0 -mprob 70.0}
\centering
\small
\begin{tabular}{c c c c c c c c}
\hline\hline
Instancia & Costo mínimo & Tiempo(seg.) & Costo promedio & Tiempo promedio(seg.) & CME & \%G & \%GP \\ [0.5ex]
\hline
SCA3-0 & 641.69 & 0.56 & 
641.88 & 0.52 & \bf{635.62} & 
0.95 & 0.98\\SCA3-1 & \bf{697.84} & 0.56 & 
697.84 & 0.82 & 697.84 & 0.00
 & 0.00\\
SCA3-2 & 666.33 & 0.45 & 
669.67 & 0.57 & \bf{659.34} & 
1.06 & 1.57\\SCA3-3 & 681.16 & 0.58 & 
681.40 & 0.58 & \bf{680.04} & 
0.16 & 0.20\\SCA3-4 & \bf{690.50} & 0.42 & 
690.50 & 0.59 & 690.50 & 0.00
 & 0.00\\
SCA3-5 & 673.56 & 0.72 & 
673.56 & 0.63 & \bf{659.90} & 
2.07 & 2.07\\SCA3-6 & 656.23 & 0.65 & 
659.43 & 0.57 & \bf{651.09} & 
0.79 & 1.28\\SCA3-7 & 664.88 & 0.52 & 
665.83 & 0.66 & \bf{659.17} & 
0.87 & 1.01\\SCA3-8 & 728.24 & 0.76 & 
729.24 & 0.62 & \bf{719.47} & 
1.22 & 1.36\\SCA3-9 & \bf{681.00} & 0.92 & 
681.17 & 0.80 & 681.00 & 0.00
 & 0.02\\SCA8-0 & 973.22 & 0.63 & 
976.04 & 0.56 & \bf{961.50} & 
1.22 & 1.51\\SCA8-1 & 1065.38 & 1.02 & 
1079.83 & 0.67 & \bf{1049.65} & 
1.50 & 2.88\\SCA8-2 & 1051.60 & 0.56 & 
1053.14 & 0.63 & \bf{1039.64} & 
1.15 & 1.30\\SCA8-3 & 1002.86 & 0.66 & 
1002.86 & 0.79 & \bf{983.34} & 
1.99 & 1.99\\SCA8-4 & 1104.93 & 0.95 & 
1109.41 & 0.67 & \bf{1065.49} & 
3.70 & 4.12\\SCA8-5 & 1037.54 & 0.98 & 
1043.76 & 0.75 & \bf{1027.08} & 
1.02 & 1.62\\SCA8-6 & 979.60 & 0.69 & 
983.10 & 0.66 & \bf{971.82} & 
0.80 & 1.16\\SCA8-7 & 1075.42 & 0.44 & 
1083.66 & 0.58 & \bf{1051.28} & 
2.30 & 3.08\\SCA8-8 & 1075.00 & 0.45 & 
1075.00 & 0.59 & \bf{1071.18} & 
0.36 & 0.36\\SCA8-9 & 1076.67 & 0.40 & 
1076.67 & 0.47 & \bf{1060.50} & 
1.52 & 1.52\\CON3-0 & 625.14 & 0.60 & 
627.60 & 0.72 & \bf{616.52} & 
1.40 & 1.80\\CON3-1 & 557.38 & 0.72 & 
559.07 & 0.65 & \bf{554.47} & 
0.52 & 0.83\\CON3-2 & 521.38 & 0.80 & 
521.38 & 0.71 & \bf{518.00} & 
0.65 & 0.65\\CON3-3 & 591.20 & 0.68 & 
597.53 & 0.68 & \bf{591.19} & 
0.00 & 1.07\\CON3-4 & 592.58 & 0.62 & 
596.88 & 0.69 & \bf{588.79} & 
0.64 & 1.37\\CON3-5 & 566.96 & 0.49 & 
568.16 & 0.68 & \bf{563.70} & 
0.58 & 0.79\\CON3-6 & 504.15 & 0.71 & 
504.80 & 0.68 & \bf{499.05} & 
1.02 & 1.15\\CON3-7 & 582.14 & 0.47 & 
582.14 & 0.55 & \bf{576.48} & 
0.98 & 0.98\\CON3-8 & 523.14 & 0.54 & 
531.60 & 0.69 & \bf{523.05} & 
0.02 & 1.63\\CON3-9 & 590.39 & 0.80 & 
591.39 & 0.81 & \bf{578.24} & 
2.10 & 2.27\\CON8-0 & 874.78 & 0.52 & 
875.55 & 0.66 & \bf{857.17} & 
2.05 & 2.14\\CON8-1 & 768.44 & 0.60 & 
772.53 & 0.79 & \bf{740.85} & 
3.72 & 4.28\\CON8-2 & 717.31 & 0.51 & 
717.34 & 0.60 & \bf{712.89} & 
0.62 & 0.62\\CON8-3 & 822.54 & 0.70 & 
822.54 & 0.77 & \bf{811.07} & 
1.41 & 1.41\\CON8-4 & 784.78 & 0.96 & 
784.78 & 0.79 & \bf{772.25} & 
1.62 & 1.62\\CON8-5 & 762.01 & 0.43 & 
762.01 & 0.61 & \bf{754.88} & 
0.94 & 0.94\\CON8-6 & 688.93 & 0.99 & 
697.49 & 0.78 & \bf{678.92} & 
1.47 & 2.73\\CON8-7 & 822.83 & 0.46 & 
825.37 & 0.56 & \bf{811.96} & 
1.34 & 1.65\\CON8-8 & 785.64 & 0.52 & 
791.52 & 0.76 & \bf{767.53} & 
2.36 & 3.13\\CON8-9 & 818.99 & 0.42 & 
822.87 & 0.56 & \bf{809.00} & 
1.23 & 1.71\\\bf{PROM.} & 
\bf{768.11} & \bf{0.64} & \bf{770.66} & \bf{0.66} & \bf{758.54} & \bf{1.18} & \bf{1.52}\\[1ex]\hline
\end{tabular}
\label{table:nonlin}
\end{table} \clearpage
\begin{table}[ht]
\caption{Resultados de la ejecución de la metaheurística IGA, utilizando instancias de SalhiNagy con la configuración -n 200 -p 40 -cprob 40.0 -mprob 70.0}
\centering
\small
\begin{tabular}{c c c c c c c c}
\hline\hline
Instancia & Costo mínimo & Tiempo(seg.) & Costo promedio & Tiempo promedio(seg.) & CME & \%G & \%GP \\ [0.5ex]
\hline
CMT1X & 478.82 & 0.50 & 
480.37 & 0.73 & \bf{470.48} & 
1.77 & 2.10\\CMT1Y & 489.68 & 0.82 & 
489.68 & 0.82 & \bf{470.48} & 
4.08 & 4.08\\CMT2X & 699.53 & 1.11 & 
701.81 & 1.12 & \bf{682.39} & 
2.51 & 2.85\\CMT2Y & 697.50 & 1.18 & 
704.64 & 1.38 & \bf{682.39} & 
2.21 & 3.26\\CMT3X & 737.01 & 3.39 & 
747.27 & 3.01 & \bf{719.06} & 
2.50 & 3.92\\CMT3Y & 730.43 & 2.75 & 
739.94 & 2.90 & \bf{719.06} & 
1.58 & 2.90\\CMT4X & 874.20 & 7.71 & 
909.50 & 8.03 & \bf{854.21} & 
2.34 & 6.47\\CMT4Y & 900.12 & 8.57 & 
916.16 & 8.29 & \bf{852.46} & 
5.59 & 7.47\\CMT5X & 1115.08 & 16.67 & 
1124.91 & 16.09 & \bf{1030.56} & 
8.20 & 9.16\\CMT5Y & 1130.57 & 17.02 & 
1137.87 & 16.56 & \bf{1031.69} & 
9.58 & 10.29\\CMT11X & 902.47 & 5.05 & 
917.15 & 5.07 & \bf{831.09} & 
8.59 & 10.36\\CMT11Y & 898.41 & 5.62 & 
912.28 & 5.81 & \bf{829.85} & 
8.26 & 9.93\\CMT12X & 674.69 & 3.02 & 
676.08 & 3.06 & \bf{658.83} & 
2.41 & 2.62\\CMT12Y & 675.40 & 3.30 & 
678.62 & 2.92 & \bf{660.47} & 
2.26 & 2.75\\\bf{PROM.} & 
\bf{785.99} & \bf{5.48} & \bf{795.45} & \bf{5.41} & \bf{749.50} & \bf{4.42} & \bf{5.58}\\[1ex]\hline
\end{tabular}
\label{table:nonlin}
\end{table} \clearpage
\begin{table}[ht]
\caption{Resultados de la ejecución de la metaheurística IGA, utilizando instancias de Dethloff con la configuración -n 200 -p 40 -cprob 40.0 -mprob 80.0}
\centering
\small
\begin{tabular}{c c c c c c c c}
\hline\hline
Instancia & Costo mínimo & Tiempo(seg.) & Costo promedio & Tiempo promedio(seg.) & CME & \%G & \%GP \\ [0.5ex]
\hline
SCA3-0 & 640.55 & 0.75 & 
640.55 & 0.69 & \bf{635.62} & 
0.78 & 0.78\\SCA3-1 & 701.53 & 0.48 & 
701.53 & 0.76 & \bf{697.84} & 
0.53 & 0.53\\SCA3-2 & 666.19 & 0.54 & 
666.27 & 0.66 & \bf{659.34} & 
1.04 & 1.05\\SCA3-3 & \bf{680.04} & 0.47 & 
683.20 & 0.58 & 680.04 & 0.00
 & 0.46\\SCA3-4 & \bf{690.50} & 0.45 & 
690.50 & 0.48 & 690.50 & 0.00
 & 0.00\\
SCA3-5 & 665.64 & 0.45 & 
665.64 & 0.68 & \bf{659.90} & 
0.87 & 0.87\\SCA3-6 & 656.40 & 0.65 & 
656.63 & 0.70 & \bf{651.09} & 
0.82 & 0.85\\SCA3-7 & 666.15 & 0.63 & 
668.47 & 0.63 & \bf{659.17} & 
1.06 & 1.41\\SCA3-8 & 723.99 & 0.70 & 
723.99 & 0.66 & \bf{719.47} & 
0.63 & 0.63\\SCA3-9 & 683.37 & 0.55 & 
684.67 & 0.73 & \bf{681.00} & 
0.35 & 0.54\\SCA8-0 & 991.02 & 0.90 & 
991.02 & 0.66 & \bf{961.50} & 
3.07 & 3.07\\SCA8-1 & 1091.16 & 0.61 & 
1091.16 & 0.59 & \bf{1049.65} & 
3.95 & 3.95\\SCA8-2 & 1053.94 & 0.52 & 
1053.94 & 0.51 & \bf{1039.64} & 
1.38 & 1.38\\SCA8-3 & 1020.80 & 0.49 & 
1026.28 & 0.48 & \bf{983.34} & 
3.81 & 4.37\\SCA8-4 & 1080.08 & 0.54 & 
1084.97 & 0.51 & \bf{1065.49} & 
1.37 & 1.83\\SCA8-5 & 1070.18 & 0.41 & 
1070.18 & 0.41 & \bf{1027.08} & 
4.20 & 4.20\\SCA8-6 & 982.42 & 0.58 & 
988.01 & 0.50 & \bf{971.82} & 
1.09 & 1.67\\SCA8-7 & 1074.08 & 0.64 & 
1085.77 & 0.65 & \bf{1051.28} & 
2.17 & 3.28\\SCA8-8 & 1088.65 & 0.95 & 
1089.28 & 0.84 & \bf{1071.18} & 
1.63 & 1.69\\SCA8-9 & 1078.30 & 0.49 & 
1090.07 & 0.63 & \bf{1060.50} & 
1.68 & 2.79\\CON3-0 & 624.96 & 0.59 & 
626.67 & 0.79 & \bf{616.52} & 
1.37 & 1.65\\CON3-1 & 560.75 & 0.82 & 
561.59 & 0.64 & \bf{554.47} & 
1.13 & 1.28\\CON3-2 & 521.38 & 1.02 & 
521.89 & 0.84 & \bf{518.00} & 
0.65 & 0.75\\CON3-3 & 604.78 & 0.45 & 
609.64 & 0.63 & \bf{591.19} & 
2.30 & 3.12\\CON3-4 & 591.43 & 0.45 & 
591.43 & 0.61 & \bf{588.79} & 
0.45 & 0.45\\CON3-5 & 566.96 & 0.48 & 
567.90 & 0.61 & \bf{563.70} & 
0.58 & 0.75\\CON3-6 & 505.14 & 0.61 & 
505.14 & 0.60 & \bf{499.05} & 
1.22 & 1.22\\CON3-7 & 586.01 & 0.88 & 
587.39 & 0.80 & \bf{576.48} & 
1.65 & 1.89\\CON3-8 & 524.59 & 0.96 & 
525.47 & 0.89 & \bf{523.05} & 
0.29 & 0.46\\CON3-9 & 582.79 & 0.70 & 
589.31 & 0.71 & \bf{578.24} & 
0.79 & 1.91\\CON8-0 & 883.76 & 0.86 & 
900.09 & 0.81 & \bf{857.17} & 
3.10 & 5.01\\CON8-1 & 757.58 & 0.58 & 
758.09 & 0.79 & \bf{740.85} & 
2.26 & 2.33\\CON8-2 & 725.00 & 0.56 & 
725.00 & 0.62 & \bf{712.89} & 
1.70 & 1.70\\CON8-3 & 834.23 & 0.56 & 
836.82 & 0.71 & \bf{811.07} & 
2.86 & 3.17\\CON8-4 & 780.03 & 0.66 & 
790.32 & 0.77 & \bf{772.25} & 
1.01 & 2.34\\CON8-5 & 767.38 & 0.99 & 
767.38 & 0.76 & \bf{754.88} & 
1.66 & 1.66\\CON8-6 & 706.57 & 1.22 & 
706.57 & 0.89 & \bf{678.92} & 
4.07 & 4.07\\CON8-7 & 835.11 & 0.60 & 
836.28 & 0.52 & \bf{811.96} & 
2.85 & 3.00\\CON8-8 & 788.13 & 0.53 & 
790.74 & 0.65 & \bf{767.53} & 
2.68 & 3.02\\CON8-9 & 833.05 & 0.50 & 
833.05 & 0.57 & \bf{809.00} & 
2.97 & 2.97\\\bf{PROM.} & 
\bf{772.12} & \bf{0.65} & \bf{774.57} & \bf{0.66} & \bf{758.54} & \bf{1.65} & \bf{1.95}\\[1ex]\hline
\end{tabular}
\label{table:nonlin}
\end{table} \clearpage
\begin{table}[ht]
\caption{Resultados de la ejecución de la metaheurística IGA, utilizando instancias de SalhiNagy con la configuración -n 200 -p 40 -cprob 40.0 -mprob 80.0}
\centering
\small
\begin{tabular}{c c c c c c c c}
\hline\hline
Instancia & Costo mínimo & Tiempo(seg.) & Costo promedio & Tiempo promedio(seg.) & CME & \%G & \%GP \\ [0.5ex]
\hline
CMT1X & 476.66 & 0.44 & 
479.38 & 0.49 & \bf{470.48} & 
1.31 & 1.89\\CMT1Y & 483.34 & 0.36 & 
490.60 & 0.38 & \bf{470.48} & 
2.73 & 4.28\\CMT2X & 714.02 & 1.45 & 
715.56 & 1.38 & \bf{682.39} & 
4.64 & 4.86\\CMT2Y & 711.89 & 1.03 & 
719.82 & 1.25 & \bf{682.39} & 
4.32 & 5.48\\CMT3X & 735.81 & 2.71 & 
744.42 & 2.77 & \bf{719.06} & 
2.33 & 3.53\\CMT3Y & 744.41 & 3.16 & 
750.11 & 2.83 & \bf{719.06} & 
3.53 & 4.32\\CMT4X & 901.63 & 8.07 & 
912.63 & 7.94 & \bf{854.21} & 
5.55 & 6.84\\CMT4Y & 916.06 & 8.63 & 
922.17 & 7.90 & \bf{852.46} & 
7.46 & 8.18\\CMT5X & 1091.86 & 16.29 & 
1109.71 & 15.95 & \bf{1030.56} & 
5.95 & 7.68\\CMT5Y & 1124.59 & 15.77 & 
1134.33 & 16.64 & \bf{1031.69} & 
9.00 & 9.95\\CMT11X & 887.13 & 5.19 & 
910.11 & 4.88 & \bf{831.09} & 
6.74 & 9.51\\CMT11Y & 893.66 & 5.40 & 
905.06 & 5.30 & \bf{829.85} & 
7.69 & 9.06\\CMT12X & 679.56 & 2.80 & 
681.99 & 2.95 & \bf{658.83} & 
3.15 & 3.52\\CMT12Y & 675.13 & 3.18 & 
678.96 & 3.26 & \bf{660.47} & 
2.22 & 2.80\\\bf{PROM.} & 
\bf{788.27} & \bf{5.32} & \bf{796.78} & \bf{5.28} & \bf{749.50} & \bf{4.76} & \bf{5.85}\\[1ex]\hline
\end{tabular}
\label{table:nonlin}
\end{table} \clearpage
\begin{table}[ht]
\caption{Resultados de la ejecución de la metaheurística IGA, utilizando instancias de Dethloff con la configuración -n 200 -p 40 -cprob 40.0 -mprob 90.0}
\centering
\small
\begin{tabular}{c c c c c c c c}
\hline\hline
Instancia & Costo mínimo & Tiempo(seg.) & Costo promedio & Tiempo promedio(seg.) & CME & \%G & \%GP \\ [0.5ex]
\hline
SCA3-0 & 636.06 & 0.60 & 
638.88 & 0.58 & \bf{635.62} & 
0.07 & 0.51\\SCA3-1 & 701.53 & 0.52 & 
701.63 & 0.74 & \bf{697.84} & 
0.53 & 0.54\\SCA3-2 & 673.17 & 0.74 & 
676.73 & 0.74 & \bf{659.34} & 
2.10 & 2.64\\SCA3-3 & 681.35 & 0.65 & 
681.72 & 0.58 & \bf{680.04} & 
0.19 & 0.25\\SCA3-4 & \bf{690.50} & 0.59 & 
690.50 & 0.56 & 690.50 & 0.00
 & 0.00\\
SCA3-5 & 661.07 & 0.74 & 
661.07 & 0.67 & \bf{659.90} & 
0.18 & 0.18\\SCA3-6 & 656.87 & 0.84 & 
656.87 & 0.69 & \bf{651.09} & 
0.89 & 0.89\\SCA3-7 & 666.15 & 0.91 & 
666.15 & 0.63 & \bf{659.17} & 
1.06 & 1.06\\SCA3-8 & 723.99 & 0.97 & 
730.75 & 0.75 & \bf{719.47} & 
0.63 & 1.57\\SCA3-9 & \bf{681.00} & 0.48 & 
681.00 & 0.64 & 681.00 & 0.00
 & 0.00\\
SCA8-0 & 1014.71 & 0.69 & 
1014.71 & 0.69 & \bf{961.50} & 
5.53 & 5.53\\SCA8-1 & 1076.19 & 0.74 & 
1076.19 & 0.59 & \bf{1049.65} & 
2.53 & 2.53\\SCA8-2 & 1047.05 & 0.54 & 
1049.37 & 0.65 & \bf{1039.64} & 
0.71 & 0.94\\SCA8-3 & 1030.86 & 0.70 & 
1032.99 & 0.60 & \bf{983.34} & 
4.83 & 5.05\\SCA8-4 & 1085.80 & 0.50 & 
1085.80 & 0.78 & \bf{1065.49} & 
1.91 & 1.91\\SCA8-5 & 1066.56 & 0.53 & 
1066.56 & 0.70 & \bf{1027.08} & 
3.84 & 3.84\\SCA8-6 & 993.45 & 0.80 & 
993.45 & 0.55 & \bf{971.82} & 
2.23 & 2.23\\SCA8-7 & 1087.26 & 0.73 & 
1087.26 & 0.73 & \bf{1051.28} & 
3.42 & 3.42\\SCA8-8 & \bf{1071.18} & 0.54 & 
1071.18 & 0.53 & 1071.18 & 0.00
 & 0.00\\
SCA8-9 & 1067.42 & 0.41 & 
1073.33 & 0.67 & \bf{1060.50} & 
0.65 & 1.21\\CON3-0 & 617.59 & 0.50 & 
625.57 & 0.65 & \bf{616.52} & 
0.17 & 1.47\\CON3-1 & 557.21 & 0.60 & 
557.21 & 0.64 & \bf{554.47} & 
0.49 & 0.49\\CON3-2 & 521.38 & 0.64 & 
521.38 & 0.80 & \bf{518.00} & 
0.65 & 0.65\\CON3-3 & \bf{591.19} & 0.48 & 
600.29 & 0.75 & 591.19 & 0.00
 & 1.54\\CON3-4 & 592.58 & 0.47 & 
592.58 & 0.53 & \bf{588.79} & 
0.64 & 0.64\\CON3-5 & 564.88 & 0.58 & 
567.79 & 0.57 & \bf{563.70} & 
0.21 & 0.73\\CON3-6 & 508.94 & 0.86 & 
510.32 & 0.64 & \bf{499.05} & 
1.98 & 2.26\\CON3-7 & 578.41 & 0.66 & 
580.37 & 0.67 & \bf{576.48} & 
0.33 & 0.67\\CON3-8 & 528.44 & 0.59 & 
535.17 & 0.69 & \bf{523.05} & 
1.03 & 2.32\\CON3-9 & 588.18 & 0.70 & 
588.53 & 0.87 & \bf{578.24} & 
1.72 & 1.78\\CON8-0 & 882.52 & 0.99 & 
883.79 & 0.68 & \bf{857.17} & 
2.96 & 3.11\\CON8-1 & 759.36 & 0.55 & 
759.36 & 0.63 & \bf{740.85} & 
2.50 & 2.50\\CON8-2 & 719.86 & 0.53 & 
719.86 & 0.65 & \bf{712.89} & 
0.98 & 0.98\\CON8-3 & 834.49 & 0.97 & 
834.49 & 0.90 & \bf{811.07} & 
2.89 & 2.89\\CON8-4 & 787.15 & 0.68 & 
787.15 & 0.63 & \bf{772.25} & 
1.93 & 1.93\\CON8-5 & 758.84 & 0.60 & 
767.07 & 0.67 & \bf{754.88} & 
0.52 & 1.61\\CON8-6 & 707.70 & 0.61 & 
707.70 & 0.64 & \bf{678.92} & 
4.24 & 4.24\\CON8-7 & 820.92 & 0.52 & 
836.52 & 0.53 & \bf{811.96} & 
1.10 & 3.03\\CON8-8 & 785.23 & 0.70 & 
796.93 & 0.77 & \bf{767.53} & 
2.31 & 3.83\\CON8-9 & 843.77 & 0.97 & 
843.84 & 0.64 & \bf{809.00} & 
4.30 & 4.31\\\bf{PROM.} & 
\bf{771.52} & \bf{0.66} & \bf{773.80} & \bf{0.67} & \bf{758.54} & \bf{1.56} & \bf{1.88}\\[1ex]\hline
\end{tabular}
\label{table:nonlin}
\end{table} \clearpage
\begin{table}[ht]
\caption{Resultados de la ejecución de la metaheurística IGA, utilizando instancias de SalhiNagy con la configuración -n 200 -p 40 -cprob 40.0 -mprob 90.0}
\centering
\small
\begin{tabular}{c c c c c c c c}
\hline\hline
Instancia & Costo mínimo & Tiempo(seg.) & Costo promedio & Tiempo promedio(seg.) & CME & \%G & \%GP \\ [0.5ex]
\hline
CMT1X & 479.57 & 0.63 & 
483.07 & 0.63 & \bf{470.48} & 
1.93 & 2.68\\CMT1Y & 476.48 & 0.41 & 
479.21 & 0.41 & \bf{470.48} & 
1.28 & 1.86\\CMT2X & 714.42 & 1.32 & 
725.42 & 1.33 & \bf{682.39} & 
4.69 & 6.31\\CMT2Y & 721.44 & 1.05 & 
723.81 & 1.18 & \bf{682.39} & 
5.72 & 6.07\\CMT3X & 733.59 & 2.98 & 
743.85 & 2.87 & \bf{719.06} & 
2.02 & 3.45\\CMT3Y & 741.41 & 3.01 & 
746.09 & 2.79 & \bf{719.06} & 
3.11 & 3.76\\CMT4X & 901.60 & 8.71 & 
912.36 & 8.10 & \bf{854.21} & 
5.55 & 6.81\\CMT4Y & 906.82 & 7.66 & 
913.75 & 8.04 & \bf{852.46} & 
6.38 & 7.19\\CMT5X & 1094.36 & 17.22 & 
1117.35 & 16.34 & \bf{1030.56} & 
6.19 & 8.42\\CMT5Y & 1112.14 & 17.16 & 
1129.51 & 16.99 & \bf{1031.69} & 
7.80 & 9.48\\CMT11X & 897.37 & 4.83 & 
918.36 & 5.05 & \bf{831.09} & 
7.98 & 10.50\\CMT11Y & 875.47 & 5.22 & 
896.27 & 5.52 & \bf{829.85} & 
5.50 & 8.00\\CMT12X & 682.45 & 3.61 & 
685.27 & 3.25 & \bf{658.83} & 
3.59 & 4.01\\CMT12Y & 674.30 & 2.94 & 
676.74 & 2.94 & \bf{660.47} & 
2.09 & 2.46\\\bf{PROM.} & 
\bf{786.53} & \bf{5.48} & \bf{796.50} & \bf{5.39} & \bf{749.50} & \bf{4.56} & \bf{5.79}\\[1ex]\hline
\end{tabular}
\label{table:nonlin}
\end{table} \clearpage
\begin{table}[ht]
\caption{Resultados de la ejecución de la metaheurística IGA, utilizando instancias de Dethloff con la configuración -n 200 -p 40 -cprob 40.0 -mprob 100.0}
\centering
\small
\begin{tabular}{c c c c c c c c}
\hline\hline
Instancia & Costo mínimo & Tiempo(seg.) & Costo promedio & Tiempo promedio(seg.) & CME & \%G & \%GP \\ [0.5ex]
\hline
SCA3-0 & 640.55 & 0.82 & 
640.55 & 0.83 & \bf{635.62} & 
0.78 & 0.78\\SCA3-1 & 707.07 & 0.66 & 
707.32 & 0.64 & \bf{697.84} & 
1.32 & 1.36\\SCA3-2 & 664.18 & 0.54 & 
664.64 & 0.60 & \bf{659.34} & 
0.73 & 0.80\\SCA3-3 & \bf{680.04} & 0.72 & 
681.29 & 0.70 & 680.04 & 0.00
 & 0.18\\SCA3-4 & \bf{690.50} & 0.53 & 
692.55 & 0.64 & 690.50 & 0.00
 & 0.30\\SCA3-5 & \bf{659.90} & 0.59 & 
662.47 & 0.79 & 659.90 & 0.00
 & 0.39\\SCA3-6 & \bf{651.09} & 0.63 & 
655.30 & 0.76 & 651.09 & 0.00
 & 0.65\\SCA3-7 & 666.15 & 0.65 & 
666.15 & 0.64 & \bf{659.17} & 
1.06 & 1.06\\SCA3-8 & 719.77 & 0.57 & 
721.47 & 0.63 & \bf{719.47} & 
0.04 & 0.28\\SCA3-9 & \bf{681.00} & 0.51 & 
681.00 & 0.66 & 681.00 & 0.00
 & 0.00\\
SCA8-0 & 972.59 & 0.56 & 
982.18 & 0.61 & \bf{961.50} & 
1.15 & 2.15\\SCA8-1 & 1074.65 & 0.69 & 
1074.90 & 0.61 & \bf{1049.65} & 
2.38 & 2.41\\SCA8-2 & 1050.37 & 0.50 & 
1050.37 & 0.66 & \bf{1039.64} & 
1.03 & 1.03\\SCA8-3 & 1022.58 & 0.56 & 
1023.37 & 0.54 & \bf{983.34} & 
3.99 & 4.07\\SCA8-4 & 1074.87 & 0.49 & 
1075.76 & 0.79 & \bf{1065.49} & 
0.88 & 0.96\\SCA8-5 & 1045.69 & 0.53 & 
1062.05 & 0.55 & \bf{1027.08} & 
1.81 & 3.40\\SCA8-6 & 976.74 & 0.68 & 
982.43 & 0.69 & \bf{971.82} & 
0.51 & 1.09\\SCA8-7 & 1075.60 & 0.51 & 
1075.60 & 0.51 & \bf{1051.28} & 
2.31 & 2.31\\SCA8-8 & 1084.41 & 0.68 & 
1084.41 & 0.77 & \bf{1071.18} & 
1.24 & 1.24\\SCA8-9 & 1077.86 & 0.95 & 
1091.96 & 0.89 & \bf{1060.50} & 
1.64 & 2.97\\CON3-0 & 620.76 & 0.94 & 
630.75 & 0.64 & \bf{616.52} & 
0.69 & 2.31\\CON3-1 & 557.21 & 0.95 & 
562.96 & 0.67 & \bf{554.47} & 
0.49 & 1.53\\CON3-2 & 521.38 & 1.02 & 
522.26 & 0.90 & \bf{518.00} & 
0.65 & 0.82\\CON3-3 & 598.45 & 0.52 & 
600.79 & 0.72 & \bf{591.19} & 
1.23 & 1.62\\CON3-4 & 595.25 & 0.57 & 
600.20 & 0.70 & \bf{588.79} & 
1.10 & 1.94\\CON3-5 & 564.88 & 0.62 & 
565.64 & 0.65 & \bf{563.70} & 
0.21 & 0.35\\CON3-6 & 503.97 & 1.03 & 
504.55 & 0.73 & \bf{499.05} & 
0.99 & 1.10\\CON3-7 & 578.22 & 0.65 & 
578.27 & 0.58 & \bf{576.48} & 
0.30 & 0.31\\CON3-8 & 524.38 & 0.96 & 
529.30 & 0.77 & \bf{523.05} & 
0.25 & 1.19\\CON3-9 & 590.16 & 0.61 & 
590.25 & 0.69 & \bf{578.24} & 
2.06 & 2.08\\CON8-0 & 879.38 & 1.01 & 
889.84 & 0.94 & \bf{857.17} & 
2.59 & 3.81\\CON8-1 & 771.70 & 1.00 & 
771.70 & 0.90 & \bf{740.85} & 
4.16 & 4.16\\CON8-2 & 716.19 & 0.73 & 
717.46 & 0.87 & \bf{712.89} & 
0.46 & 0.64\\CON8-3 & 831.73 & 0.55 & 
835.04 & 0.69 & \bf{811.07} & 
2.55 & 2.96\\CON8-4 & 790.59 & 1.02 & 
790.73 & 0.78 & \bf{772.25} & 
2.37 & 2.39\\CON8-5 & 762.61 & 1.02 & 
762.61 & 0.77 & \bf{754.88} & 
1.02 & 1.02\\CON8-6 & 698.19 & 0.85 & 
704.55 & 0.85 & \bf{678.92} & 
2.84 & 3.78\\CON8-7 & 817.98 & 0.68 & 
824.23 & 0.59 & \bf{811.96} & 
0.74 & 1.51\\CON8-8 & 784.32 & 0.71 & 
786.37 & 0.62 & \bf{767.53} & 
2.19 & 2.45\\CON8-9 & 830.79 & 0.94 & 
834.54 & 0.72 & \bf{809.00} & 
2.69 & 3.16\\\bf{PROM.} & 
\bf{768.84} & \bf{0.72} & \bf{771.94} & \bf{0.71} & \bf{758.54} & \bf{1.26} & \bf{1.66}\\[1ex]\hline
\end{tabular}
\label{table:nonlin}
\end{table} \clearpage
\begin{table}[ht]
\caption{Resultados de la ejecución de la metaheurística IGA, utilizando instancias de SalhiNagy con la configuración -n 200 -p 40 -cprob 40.0 -mprob 100.0}
\centering
\small
\begin{tabular}{c c c c c c c c}
\hline\hline
Instancia & Costo mínimo & Tiempo(seg.) & Costo promedio & Tiempo promedio(seg.) & CME & \%G & \%GP \\ [0.5ex]
\hline
CMT1X & 476.53 & 0.52 & 
486.06 & 0.76 & \bf{470.48} & 
1.29 & 3.31\\CMT1Y & 483.02 & 0.54 & 
485.81 & 0.54 & \bf{470.48} & 
2.67 & 3.26\\CMT2X & 703.21 & 1.51 & 
712.71 & 1.47 & \bf{682.39} & 
3.05 & 4.44\\CMT2Y & 716.02 & 1.54 & 
719.77 & 1.32 & \bf{682.39} & 
4.93 & 5.48\\CMT3X & 734.38 & 2.65 & 
739.02 & 3.09 & \bf{719.06} & 
2.13 & 2.78\\CMT3Y & 735.08 & 2.55 & 
748.39 & 3.00 & \bf{719.06} & 
2.23 & 4.08\\CMT4X & 900.05 & 8.04 & 
905.88 & 7.66 & \bf{854.21} & 
5.37 & 6.05\\CMT4Y & 908.78 & 8.23 & 
916.92 & 7.93 & \bf{852.46} & 
6.61 & 7.56\\CMT5X & 1108.60 & 16.24 & 
1115.09 & 16.10 & \bf{1030.56} & 
7.57 & 8.20\\CMT5Y & 1106.28 & 17.35 & 
1117.31 & 17.02 & \bf{1031.69} & 
7.23 & 8.30\\CMT11X & 907.01 & 4.92 & 
916.09 & 5.10 & \bf{831.09} & 
9.13 & 10.23\\CMT11Y & 885.25 & 6.17 & 
895.28 & 5.55 & \bf{829.85} & 
6.68 & 7.89\\CMT12X & 675.73 & 3.01 & 
683.49 & 2.94 & \bf{658.83} & 
2.57 & 3.74\\CMT12Y & 681.33 & 3.36 & 
684.09 & 2.98 & \bf{660.47} & 
3.16 & 3.58\\\bf{PROM.} & 
\bf{787.23} & \bf{5.47} & \bf{794.71} & \bf{5.39} & \bf{749.50} & \bf{4.61} & \bf{5.63}\\[1ex]\hline
\end{tabular}
\label{table:nonlin}
\end{table} \clearpage
\begin{table}[ht]
\caption{Resultados de la ejecución de la metaheurística IGA, utilizando instancias de Dethloff con la configuración -n 200 -p 40 -cprob 50.0 -mprob 10.0}
\centering
\small
\begin{tabular}{c c c c c c c c}
\hline\hline
Instancia & Costo mínimo & Tiempo(seg.) & Costo promedio & Tiempo promedio(seg.) & CME & \%G & \%GP \\ [0.5ex]
\hline
SCA3-0 & 640.55 & 0.65 & 
640.55 & 0.53 & \bf{635.62} & 
0.78 & 0.78\\SCA3-1 & \bf{697.84} & 0.70 & 
697.84 & 0.63 & 697.84 & 0.00
 & 0.00\\
SCA3-2 & 664.21 & 0.68 & 
668.06 & 0.64 & \bf{659.34} & 
0.74 & 1.32\\SCA3-3 & \bf{680.04} & 0.74 & 
682.07 & 0.70 & 680.04 & 0.00
 & 0.30\\SCA3-4 & \bf{690.50} & 0.57 & 
691.02 & 0.62 & 690.50 & 0.00
 & 0.08\\SCA3-5 & 670.10 & 0.52 & 
678.13 & 0.60 & \bf{659.90} & 
1.55 & 2.76\\SCA3-6 & 652.94 & 0.55 & 
656.82 & 0.55 & \bf{651.09} & 
0.28 & 0.88\\SCA3-7 & 669.89 & 0.64 & 
669.89 & 0.70 & \bf{659.17} & 
1.63 & 1.63\\SCA3-8 & \bf{719.47} & 0.55 & 
722.93 & 0.55 & 719.47 & 0.00
 & 0.48\\SCA3-9 & 684.44 & 0.54 & 
684.65 & 0.62 & \bf{681.00} & 
0.51 & 0.54\\SCA8-0 & 999.44 & 0.76 & 
1007.33 & 0.79 & \bf{961.50} & 
3.95 & 4.77\\SCA8-1 & 1095.68 & 0.77 & 
1095.68 & 0.70 & \bf{1049.65} & 
4.39 & 4.39\\SCA8-2 & 1054.69 & 0.74 & 
1054.69 & 0.69 & \bf{1039.64} & 
1.45 & 1.45\\SCA8-3 & 1025.88 & 0.75 & 
1025.88 & 0.65 & \bf{983.34} & 
4.33 & 4.33\\SCA8-4 & 1100.71 & 0.72 & 
1104.10 & 0.70 & \bf{1065.49} & 
3.31 & 3.62\\SCA8-5 & 1057.96 & 0.50 & 
1058.65 & 0.67 & \bf{1027.08} & 
3.01 & 3.07\\SCA8-6 & 993.18 & 0.74 & 
993.35 & 0.62 & \bf{971.82} & 
2.20 & 2.22\\SCA8-7 & 1077.28 & 0.52 & 
1077.28 & 0.49 & \bf{1051.28} & 
2.47 & 2.47\\SCA8-8 & 1088.20 & 0.72 & 
1088.20 & 0.60 & \bf{1071.18} & 
1.59 & 1.59\\SCA8-9 & 1081.23 & 0.55 & 
1084.22 & 0.75 & \bf{1060.50} & 
1.95 & 2.24\\CON3-0 & 620.76 & 0.71 & 
620.76 & 0.62 & \bf{616.52} & 
0.69 & 0.69\\CON3-1 & 560.75 & 0.54 & 
561.75 & 0.61 & \bf{554.47} & 
1.13 & 1.31\\CON3-2 & 521.38 & 0.64 & 
521.38 & 0.62 & \bf{518.00} & 
0.65 & 0.65\\CON3-3 & 594.48 & 0.73 & 
603.97 & 0.58 & \bf{591.19} & 
0.56 & 2.16\\CON3-4 & 593.78 & 0.57 & 
594.51 & 0.56 & \bf{588.79} & 
0.85 & 0.97\\CON3-5 & 568.69 & 0.66 & 
568.74 & 0.64 & \bf{563.70} & 
0.89 & 0.89\\CON3-6 & 505.01 & 0.67 & 
508.20 & 0.76 & \bf{499.05} & 
1.19 & 1.83\\CON3-7 & 581.46 & 0.71 & 
586.12 & 0.59 & \bf{576.48} & 
0.86 & 1.67\\CON3-8 & 523.14 & 0.96 & 
528.04 & 0.81 & \bf{523.05} & 
0.02 & 0.95\\CON3-9 & 582.79 & 0.97 & 
588.19 & 0.64 & \bf{578.24} & 
0.79 & 1.72\\CON8-0 & 866.18 & 0.79 & 
866.18 & 0.57 & \bf{857.17} & 
1.05 & 1.05\\CON8-1 & 773.64 & 0.53 & 
775.51 & 0.57 & \bf{740.85} & 
4.43 & 4.68\\CON8-2 & 725.31 & 0.78 & 
725.31 & 0.71 & \bf{712.89} & 
1.74 & 1.74\\CON8-3 & 812.22 & 0.58 & 
831.40 & 0.73 & \bf{811.07} & 
0.14 & 2.51\\CON8-4 & 789.97 & 0.42 & 
791.17 & 0.48 & \bf{772.25} & 
2.29 & 2.45\\CON8-5 & 779.67 & 0.55 & 
780.86 & 0.56 & \bf{754.88} & 
3.28 & 3.44\\CON8-6 & 708.73 & 0.58 & 
708.73 & 0.64 & \bf{678.92} & 
4.39 & 4.39\\CON8-7 & 822.26 & 0.55 & 
822.26 & 0.55 & \bf{811.96} & 
1.27 & 1.27\\CON8-8 & 797.82 & 0.58 & 
797.82 & 0.59 & \bf{767.53} & 
3.95 & 3.95\\CON8-9 & 818.02 & 0.49 & 
830.11 & 0.62 & \bf{809.00} & 
1.11 & 2.61\\\bf{PROM.} & 
\bf{772.26} & \bf{0.65} & \bf{774.81} & \bf{0.63} & \bf{758.54} & \bf{1.63} & \bf{2.00}\\[1ex]\hline
\end{tabular}
\label{table:nonlin}
\end{table} \clearpage
\begin{table}[ht]
\caption{Resultados de la ejecución de la metaheurística IGA, utilizando instancias de SalhiNagy con la configuración -n 200 -p 40 -cprob 50.0 -mprob 10.0}
\centering
\small
\begin{tabular}{c c c c c c c c}
\hline\hline
Instancia & Costo mínimo & Tiempo(seg.) & Costo promedio & Tiempo promedio(seg.) & CME & \%G & \%GP \\ [0.5ex]
\hline
CMT1X & 478.84 & 0.64 & 
482.70 & 0.64 & \bf{470.48} & 
1.78 & 2.60\\CMT1Y & 478.84 & 0.53 & 
483.04 & 0.52 & \bf{470.48} & 
1.78 & 2.67\\CMT2X & 708.42 & 1.05 & 
720.46 & 1.26 & \bf{682.39} & 
3.81 & 5.58\\CMT2Y & 705.61 & 1.02 & 
709.67 & 1.30 & \bf{682.39} & 
3.40 & 4.00\\CMT3X & 743.17 & 2.95 & 
747.98 & 3.07 & \bf{719.06} & 
3.35 & 4.02\\CMT3Y & 743.06 & 2.66 & 
751.96 & 2.84 & \bf{719.06} & 
3.34 & 4.58\\CMT4X & 903.95 & 7.38 & 
917.26 & 7.67 & \bf{854.21} & 
5.82 & 7.38\\CMT4Y & 918.11 & 7.22 & 
928.46 & 7.42 & \bf{852.46} & 
7.70 & 8.92\\CMT5X & 1102.80 & 16.92 & 
1110.72 & 16.69 & \bf{1030.56} & 
7.01 & 7.78\\CMT5Y & 1105.58 & 16.84 & 
1125.73 & 16.88 & \bf{1031.69} & 
7.16 & 9.12\\CMT11X & 901.66 & 5.07 & 
919.39 & 4.99 & \bf{831.09} & 
8.49 & 10.63\\CMT11Y & 897.71 & 5.88 & 
915.92 & 5.63 & \bf{829.85} & 
8.18 & 10.37\\CMT12X & 674.61 & 2.94 & 
679.95 & 2.91 & \bf{658.83} & 
2.40 & 3.21\\CMT12Y & 676.31 & 2.82 & 
682.38 & 2.76 & \bf{660.47} & 
2.40 & 3.32\\\bf{PROM.} & 
\bf{788.48} & \bf{5.28} & \bf{798.26} & \bf{5.33} & \bf{749.50} & \bf{4.76} & \bf{6.01}\\[1ex]\hline
\end{tabular}
\label{table:nonlin}
\end{table} \clearpage
\begin{table}[ht]
\caption{Resultados de la ejecución de la metaheurística IGA, utilizando instancias de Dethloff con la configuración -n 200 -p 40 -cprob 50.0 -mprob 20.0}
\centering
\small
\begin{tabular}{c c c c c c c c}
\hline\hline
Instancia & Costo mínimo & Tiempo(seg.) & Costo promedio & Tiempo promedio(seg.) & CME & \%G & \%GP \\ [0.5ex]
\hline
SCA3-0 & 640.55 & 0.72 & 
640.55 & 0.65 & \bf{635.62} & 
0.78 & 0.78\\SCA3-1 & 701.53 & 0.55 & 
701.53 & 0.74 & \bf{697.84} & 
0.53 & 0.53\\SCA3-2 & 661.13 & 0.87 & 
666.13 & 0.69 & \bf{659.34} & 
0.27 & 1.03\\SCA3-3 & 681.16 & 0.54 & 
681.21 & 0.54 & \bf{680.04} & 
0.16 & 0.17\\SCA3-4 & \bf{690.50} & 0.71 & 
692.17 & 0.57 & 690.50 & 0.00
 & 0.24\\SCA3-5 & 673.56 & 0.73 & 
676.75 & 0.73 & \bf{659.90} & 
2.07 & 2.55\\SCA3-6 & 653.81 & 0.73 & 
654.86 & 0.64 & \bf{651.09} & 
0.42 & 0.58\\SCA3-7 & 666.60 & 0.62 & 
666.76 & 0.65 & \bf{659.17} & 
1.13 & 1.15\\SCA3-8 & 731.10 & 0.53 & 
731.10 & 0.58 & \bf{719.47} & 
1.62 & 1.62\\SCA3-9 & 685.19 & 0.61 & 
686.68 & 0.59 & \bf{681.00} & 
0.62 & 0.83\\SCA8-0 & 990.65 & 0.52 & 
1007.95 & 0.65 & \bf{961.50} & 
3.03 & 4.83\\SCA8-1 & 1086.03 & 0.56 & 
1095.31 & 0.75 & \bf{1049.65} & 
3.47 & 4.35\\SCA8-2 & 1054.69 & 0.74 & 
1054.81 & 0.64 & \bf{1039.64} & 
1.45 & 1.46\\SCA8-3 & 1014.19 & 0.88 & 
1023.36 & 0.74 & \bf{983.34} & 
3.14 & 4.07\\SCA8-4 & 1068.27 & 0.50 & 
1068.27 & 0.57 & \bf{1065.49} & 
0.26 & 0.26\\SCA8-5 & 1060.95 & 0.76 & 
1064.45 & 0.57 & \bf{1027.08} & 
3.30 & 3.64\\SCA8-6 & 991.07 & 0.66 & 
991.82 & 0.74 & \bf{971.82} & 
1.98 & 2.06\\SCA8-7 & 1067.49 & 0.71 & 
1067.49 & 0.71 & \bf{1051.28} & 
1.54 & 1.54\\SCA8-8 & \bf{1071.18} & 0.60 & 
1086.81 & 0.68 & 1071.18 & 0.00
 & 1.46\\SCA8-9 & 1072.10 & 0.76 & 
1074.30 & 0.72 & \bf{1060.50} & 
1.09 & 1.30\\CON3-0 & 621.82 & 0.75 & 
629.49 & 0.61 & \bf{616.52} & 
0.86 & 2.10\\CON3-1 & 560.75 & 0.71 & 
560.75 & 0.73 & \bf{554.47} & 
1.13 & 1.13\\CON3-2 & 521.38 & 0.60 & 
521.89 & 0.62 & \bf{518.00} & 
0.65 & 0.75\\CON3-3 & 591.85 & 0.66 & 
592.65 & 0.60 & \bf{591.19} & 
0.11 & 0.25\\CON3-4 & 594.59 & 0.55 & 
596.96 & 0.60 & \bf{588.79} & 
0.99 & 1.39\\CON3-5 & 567.94 & 0.71 & 
569.36 & 0.65 & \bf{563.70} & 
0.75 & 1.00\\CON3-6 & 504.70 & 0.76 & 
506.56 & 0.76 & \bf{499.05} & 
1.13 & 1.50\\CON3-7 & 577.54 & 0.47 & 
585.32 & 0.71 & \bf{576.48} & 
0.18 & 1.53\\CON3-8 & 523.14 & 0.70 & 
524.23 & 0.64 & \bf{523.05} & 
0.02 & 0.23\\CON3-9 & 587.23 & 0.76 & 
589.19 & 0.64 & \bf{578.24} & 
1.55 & 1.89\\CON8-0 & 874.47 & 0.76 & 
874.47 & 0.75 & \bf{857.17} & 
2.02 & 2.02\\CON8-1 & 741.70 & 0.73 & 
741.70 & 0.76 & \bf{740.85} & 
0.11 & 0.11\\CON8-2 & 722.56 & 0.54 & 
726.09 & 0.60 & \bf{712.89} & 
1.36 & 1.85\\CON8-3 & 832.72 & 0.64 & 
832.77 & 0.77 & \bf{811.07} & 
2.67 & 2.68\\CON8-4 & 805.40 & 0.47 & 
805.40 & 0.49 & \bf{772.25} & 
4.29 & 4.29\\CON8-5 & 760.82 & 0.58 & 
762.28 & 0.62 & \bf{754.88} & 
0.79 & 0.98\\CON8-6 & 698.76 & 0.80 & 
698.76 & 0.76 & \bf{678.92} & 
2.92 & 2.92\\CON8-7 & 831.38 & 0.71 & 
833.12 & 0.71 & \bf{811.96} & 
2.39 & 2.61\\CON8-8 & 791.99 & 0.77 & 
793.68 & 0.78 & \bf{767.53} & 
3.19 & 3.41\\CON8-9 & 825.20 & 0.59 & 
828.81 & 0.72 & \bf{809.00} & 
2.00 & 2.45\\\bf{PROM.} & 
\bf{769.94} & \bf{0.66} & \bf{772.64} & \bf{0.67} & \bf{758.54} & \bf{1.40} & \bf{1.74}\\[1ex]\hline
\end{tabular}
\label{table:nonlin}
\end{table} \clearpage
\begin{table}[ht]
\caption{Resultados de la ejecución de la metaheurística IGA, utilizando instancias de SalhiNagy con la configuración -n 200 -p 40 -cprob 50.0 -mprob 20.0}
\centering
\small
\begin{tabular}{c c c c c c c c}
\hline\hline
Instancia & Costo mínimo & Tiempo(seg.) & Costo promedio & Tiempo promedio(seg.) & CME & \%G & \%GP \\ [0.5ex]
\hline
CMT1X & 478.84 & 0.45 & 
479.43 & 0.47 & \bf{470.48} & 
1.78 & 1.90\\CMT1Y & 478.82 & 0.40 & 
481.52 & 0.54 & \bf{470.48} & 
1.77 & 2.35\\CMT2X & 716.76 & 1.20 & 
721.17 & 1.17 & \bf{682.39} & 
5.04 & 5.68\\CMT2Y & 708.15 & 0.97 & 
710.23 & 1.02 & \bf{682.39} & 
3.77 & 4.08\\CMT3X & 734.06 & 3.01 & 
737.69 & 3.06 & \bf{719.06} & 
2.09 & 2.59\\CMT3Y & 738.51 & 2.97 & 
743.83 & 2.71 & \bf{719.06} & 
2.70 & 3.45\\CMT4X & 915.71 & 7.52 & 
919.49 & 7.46 & \bf{854.21} & 
7.20 & 7.64\\CMT4Y & 917.57 & 7.60 & 
919.29 & 7.72 & \bf{852.46} & 
7.64 & 7.84\\CMT5X & 1121.27 & 16.83 & 
1126.15 & 16.56 & \bf{1030.56} & 
8.80 & 9.28\\CMT5Y & 1093.56 & 16.74 & 
1119.74 & 16.32 & \bf{1031.69} & 
6.00 & 8.53\\CMT11X & 898.13 & 4.61 & 
916.22 & 4.65 & \bf{831.09} & 
8.07 & 10.24\\CMT11Y & 884.98 & 4.72 & 
899.96 & 5.00 & \bf{829.85} & 
6.64 & 8.45\\CMT12X & 676.25 & 3.49 & 
685.65 & 3.03 & \bf{658.83} & 
2.64 & 4.07\\CMT12Y & 673.85 & 2.58 & 
675.60 & 2.65 & \bf{660.47} & 
2.03 & 2.29\\\bf{PROM.} & 
\bf{788.32} & \bf{5.22} & \bf{795.43} & \bf{5.17} & \bf{749.50} & \bf{4.73} & \bf{5.60}\\[1ex]\hline
\end{tabular}
\label{table:nonlin}
\end{table} \clearpage
\begin{table}[ht]
\caption{Resultados de la ejecución de la metaheurística IGA, utilizando instancias de Dethloff con la configuración -n 200 -p 40 -cprob 50.0 -mprob 30.0}
\centering
\small
\begin{tabular}{c c c c c c c c}
\hline\hline
Instancia & Costo mínimo & Tiempo(seg.) & Costo promedio & Tiempo promedio(seg.) & CME & \%G & \%GP \\ [0.5ex]
\hline
SCA3-0 & 641.64 & 0.69 & 
641.68 & 0.63 & \bf{635.62} & 
0.95 & 0.95\\SCA3-1 & 700.50 & 0.70 & 
703.37 & 0.71 & \bf{697.84} & 
0.38 & 0.79\\SCA3-2 & 659.86 & 0.42 & 
663.12 & 0.55 & \bf{659.34} & 
0.08 & 0.57\\SCA3-3 & 682.46 & 0.94 & 
682.46 & 0.65 & \bf{680.04} & 
0.36 & 0.36\\SCA3-4 & \bf{690.50} & 0.54 & 
690.50 & 0.65 & 690.50 & 0.00
 & 0.00\\
SCA3-5 & 670.10 & 0.48 & 
670.10 & 0.60 & \bf{659.90} & 
1.55 & 1.55\\SCA3-6 & 652.94 & 0.74 & 
655.33 & 0.64 & \bf{651.09} & 
0.28 & 0.65\\SCA3-7 & 666.15 & 0.81 & 
666.49 & 0.66 & \bf{659.17} & 
1.06 & 1.11\\SCA3-8 & 726.88 & 0.58 & 
726.88 & 0.57 & \bf{719.47} & 
1.03 & 1.03\\SCA3-9 & \bf{681.00} & 0.52 & 
681.00 & 0.60 & 681.00 & 0.00
 & 0.00\\
SCA8-0 & 973.22 & 0.57 & 
973.22 & 0.70 & \bf{961.50} & 
1.22 & 1.22\\SCA8-1 & 1070.85 & 0.80 & 
1070.85 & 0.67 & \bf{1049.65} & 
2.02 & 2.02\\SCA8-2 & 1053.78 & 0.73 & 
1053.95 & 0.70 & \bf{1039.64} & 
1.36 & 1.38\\SCA8-3 & 1013.56 & 0.43 & 
1017.38 & 0.60 & \bf{983.34} & 
3.07 & 3.46\\SCA8-4 & 1067.55 & 0.74 & 
1072.01 & 0.65 & \bf{1065.49} & 
0.19 & 0.61\\SCA8-5 & 1052.77 & 0.95 & 
1053.02 & 0.64 & \bf{1027.08} & 
2.50 & 2.53\\SCA8-6 & 981.41 & 0.88 & 
985.53 & 0.76 & \bf{971.82} & 
0.99 & 1.41\\SCA8-7 & 1079.01 & 0.44 & 
1085.33 & 0.56 & \bf{1051.28} & 
2.64 & 3.24\\SCA8-8 & 1090.51 & 0.74 & 
1090.51 & 0.70 & \bf{1071.18} & 
1.80 & 1.80\\SCA8-9 & \bf{1060.50} & 0.68 & 
1067.35 & 0.68 & 1060.50 & 0.00
 & 0.65\\CON3-0 & 628.47 & 0.58 & 
632.50 & 0.55 & \bf{616.52} & 
1.94 & 2.59\\CON3-1 & 557.21 & 0.92 & 
558.10 & 0.61 & \bf{554.47} & 
0.49 & 0.65\\CON3-2 & 521.38 & 0.47 & 
521.38 & 0.60 & \bf{518.00} & 
0.65 & 0.65\\CON3-3 & 601.49 & 0.78 & 
604.08 & 0.73 & \bf{591.19} & 
1.74 & 2.18\\CON3-4 & 592.58 & 0.74 & 
595.90 & 0.67 & \bf{588.79} & 
0.64 & 1.21\\CON3-5 & 568.76 & 0.74 & 
569.14 & 0.73 & \bf{563.70} & 
0.90 & 0.96\\CON3-6 & 509.95 & 0.56 & 
510.40 & 0.64 & \bf{499.05} & 
2.18 & 2.27\\CON3-7 & 580.40 & 0.71 & 
583.56 & 0.73 & \bf{576.48} & 
0.68 & 1.23\\CON3-8 & 524.59 & 0.51 & 
527.10 & 0.63 & \bf{523.05} & 
0.29 & 0.77\\CON3-9 & 590.50 & 0.76 & 
591.33 & 0.75 & \bf{578.24} & 
2.12 & 2.26\\CON8-0 & 894.22 & 0.69 & 
897.67 & 0.56 & \bf{857.17} & 
4.32 & 4.72\\CON8-1 & 757.97 & 0.65 & 
758.47 & 0.70 & \bf{740.85} & 
2.31 & 2.38\\CON8-2 & 719.16 & 0.56 & 
722.66 & 0.75 & \bf{712.89} & 
0.88 & 1.37\\CON8-3 & 837.16 & 0.53 & 
837.16 & 0.53 & \bf{811.07} & 
3.22 & 3.22\\CON8-4 & 772.76 & 0.72 & 
775.92 & 0.54 & \bf{772.25} & 
0.07 & 0.48\\CON8-5 & 763.13 & 0.81 & 
765.23 & 0.64 & \bf{754.88} & 
1.09 & 1.37\\CON8-6 & 704.84 & 0.50 & 
704.84 & 0.49 & \bf{678.92} & 
3.82 & 3.82\\CON8-7 & 826.34 & 0.72 & 
826.46 & 0.66 & \bf{811.96} & 
1.77 & 1.79\\CON8-8 & 800.55 & 0.48 & 
800.55 & 0.61 & \bf{767.53} & 
4.30 & 4.30\\CON8-9 & 829.10 & 0.78 & 
829.10 & 0.62 & \bf{809.00} & 
2.48 & 2.48\\\bf{PROM.} & 
\bf{769.89} & \bf{0.66} & \bf{771.54} & \bf{0.64} & \bf{758.54} & \bf{1.43} & \bf{1.65}\\[1ex]\hline
\end{tabular}
\label{table:nonlin}
\end{table} \clearpage
\begin{table}[ht]
\caption{Resultados de la ejecución de la metaheurística IGA, utilizando instancias de SalhiNagy con la configuración -n 200 -p 40 -cprob 50.0 -mprob 30.0}
\centering
\small
\begin{tabular}{c c c c c c c c}
\hline\hline
Instancia & Costo mínimo & Tiempo(seg.) & Costo promedio & Tiempo promedio(seg.) & CME & \%G & \%GP \\ [0.5ex]
\hline
CMT1X & 477.59 & 0.65 & 
478.53 & 0.55 & \bf{470.48} & 
1.51 & 1.71\\CMT1Y & 478.23 & 0.82 & 
485.51 & 0.65 & \bf{470.48} & 
1.65 & 3.20\\CMT2X & 710.45 & 1.69 & 
721.27 & 1.43 & \bf{682.39} & 
4.11 & 5.70\\CMT2Y & 699.73 & 1.27 & 
705.37 & 1.30 & \bf{682.39} & 
2.54 & 3.37\\CMT3X & 732.45 & 2.42 & 
735.93 & 2.92 & \bf{719.06} & 
1.86 & 2.35\\CMT3Y & 739.24 & 2.72 & 
748.56 & 3.00 & \bf{719.06} & 
2.81 & 4.10\\CMT4X & 920.22 & 7.27 & 
924.26 & 7.59 & \bf{854.21} & 
7.73 & 8.20\\CMT4Y & 900.42 & 9.37 & 
913.14 & 8.46 & \bf{852.46} & 
5.63 & 7.12\\CMT5X & 1114.21 & 15.35 & 
1124.68 & 15.98 & \bf{1030.56} & 
8.12 & 9.13\\CMT5Y & 1119.33 & 16.55 & 
1122.88 & 16.04 & \bf{1031.69} & 
8.49 & 8.84\\CMT11X & 908.78 & 4.94 & 
918.02 & 5.33 & \bf{831.09} & 
9.35 & 10.46\\CMT11Y & 905.17 & 5.55 & 
916.38 & 5.47 & \bf{829.85} & 
9.08 & 10.43\\CMT12X & 674.20 & 3.09 & 
682.81 & 2.85 & \bf{658.83} & 
2.33 & 3.64\\CMT12Y & 675.31 & 2.67 & 
678.49 & 2.73 & \bf{660.47} & 
2.25 & 2.73\\\bf{PROM.} & 
\bf{789.67} & \bf{5.31} & \bf{796.85} & \bf{5.31} & \bf{749.50} & \bf{4.82} & \bf{5.78}\\[1ex]\hline
\end{tabular}
\label{table:nonlin}
\end{table} \clearpage
\begin{table}[ht]
\caption{Resultados de la ejecución de la metaheurística IGA, utilizando instancias de Dethloff con la configuración -n 200 -p 40 -cprob 50.0 -mprob 40.0}
\centering
\small
\begin{tabular}{c c c c c c c c}
\hline\hline
Instancia & Costo mínimo & Tiempo(seg.) & Costo promedio & Tiempo promedio(seg.) & CME & \%G & \%GP \\ [0.5ex]
\hline
SCA3-0 & 641.69 & 0.57 & 
642.40 & 0.64 & \bf{635.62} & 
0.95 & 1.07\\SCA3-1 & \bf{697.84} & 0.82 & 
697.84 & 0.72 & 697.84 & 0.00
 & 0.00\\
SCA3-2 & 661.13 & 0.71 & 
670.72 & 0.71 & \bf{659.34} & 
0.27 & 1.73\\SCA3-3 & 682.45 & 0.49 & 
687.04 & 0.54 & \bf{680.04} & 
0.35 & 1.03\\SCA3-4 & \bf{690.50} & 0.62 & 
690.50 & 0.58 & 690.50 & 0.00
 & 0.00\\
SCA3-5 & 661.07 & 0.68 & 
662.33 & 0.70 & \bf{659.90} & 
0.18 & 0.37\\SCA3-6 & 656.40 & 0.52 & 
657.35 & 0.59 & \bf{651.09} & 
0.82 & 0.96\\SCA3-7 & 666.15 & 0.90 & 
666.15 & 0.69 & \bf{659.17} & 
1.06 & 1.06\\SCA3-8 & 727.35 & 0.57 & 
732.63 & 0.57 & \bf{719.47} & 
1.10 & 1.83\\SCA3-9 & \bf{681.00} & 0.67 & 
681.00 & 0.55 & 681.00 & 0.00
 & 0.00\\
SCA8-0 & 973.22 & 0.78 & 
976.67 & 0.65 & \bf{961.50} & 
1.22 & 1.58\\SCA8-1 & 1087.14 & 0.54 & 
1087.48 & 0.61 & \bf{1049.65} & 
3.57 & 3.60\\SCA8-2 & 1057.39 & 0.74 & 
1057.39 & 0.60 & \bf{1039.64} & 
1.71 & 1.71\\SCA8-3 & 1021.34 & 0.77 & 
1021.34 & 0.67 & \bf{983.34} & 
3.86 & 3.86\\SCA8-4 & \bf{1065.49} & 0.74 & 
1076.42 & 0.60 & 1065.49 & 0.00
 & 1.03\\SCA8-5 & 1043.52 & 0.96 & 
1054.57 & 0.63 & \bf{1027.08} & 
1.60 & 2.68\\SCA8-6 & 990.02 & 0.76 & 
993.35 & 0.65 & \bf{971.82} & 
1.87 & 2.22\\SCA8-7 & 1077.82 & 0.48 & 
1083.42 & 0.53 & \bf{1051.28} & 
2.52 & 3.06\\SCA8-8 & 1094.27 & 0.98 & 
1095.06 & 0.60 & \bf{1071.18} & 
2.16 & 2.23\\SCA8-9 & 1079.93 & 0.53 & 
1079.93 & 0.51 & \bf{1060.50} & 
1.83 & 1.83\\CON3-0 & 632.29 & 0.81 & 
632.69 & 0.76 & \bf{616.52} & 
2.56 & 2.62\\CON3-1 & 558.09 & 0.68 & 
560.14 & 0.56 & \bf{554.47} & 
0.65 & 1.02\\CON3-2 & 521.38 & 0.72 & 
521.50 & 0.72 & \bf{518.00} & 
0.65 & 0.68\\CON3-3 & 608.29 & 0.56 & 
608.63 & 0.69 & \bf{591.19} & 
2.89 & 2.95\\CON3-4 & 593.69 & 0.76 & 
593.69 & 0.70 & \bf{588.79} & 
0.83 & 0.83\\CON3-5 & 568.66 & 0.71 & 
574.63 & 0.72 & \bf{563.70} & 
0.88 & 1.94\\CON3-6 & 504.44 & 0.56 & 
504.88 & 0.70 & \bf{499.05} & 
1.08 & 1.17\\CON3-7 & 588.38 & 0.53 & 
590.08 & 0.52 & \bf{576.48} & 
2.06 & 2.36\\CON3-8 & 524.59 & 0.54 & 
533.17 & 0.56 & \bf{523.05} & 
0.29 & 1.93\\CON3-9 & 582.79 & 0.50 & 
587.83 & 0.56 & \bf{578.24} & 
0.79 & 1.66\\CON8-0 & 870.73 & 0.70 & 
870.87 & 0.62 & \bf{857.17} & 
1.58 & 1.60\\CON8-1 & 754.20 & 0.78 & 
757.13 & 0.78 & \bf{740.85} & 
1.80 & 2.20\\CON8-2 & 721.57 & 0.92 & 
722.06 & 0.97 & \bf{712.89} & 
1.22 & 1.29\\CON8-3 & 833.78 & 0.99 & 
839.05 & 0.80 & \bf{811.07} & 
2.80 & 3.45\\CON8-4 & 772.76 & 0.76 & 
781.10 & 0.70 & \bf{772.25} & 
0.07 & 1.15\\CON8-5 & 771.35 & 0.46 & 
777.53 & 0.62 & \bf{754.88} & 
2.18 & 3.00\\CON8-6 & 685.80 & 0.75 & 
685.80 & 0.68 & \bf{678.92} & 
1.01 & 1.01\\CON8-7 & 818.79 & 0.53 & 
818.79 & 0.72 & \bf{811.96} & 
0.84 & 0.84\\CON8-8 & 791.86 & 0.78 & 
791.86 & 0.81 & \bf{767.53} & 
3.17 & 3.17\\CON8-9 & 834.97 & 0.80 & 
834.97 & 0.74 & \bf{809.00} & 
3.21 & 3.21\\\bf{PROM.} & 
\bf{769.85} & \bf{0.69} & \bf{772.50} & \bf{0.66} & \bf{758.54} & \bf{1.39} & \bf{1.75}\\[1ex]\hline
\end{tabular}
\label{table:nonlin}
\end{table} \clearpage
\begin{table}[ht]
\caption{Resultados de la ejecución de la metaheurística IGA, utilizando instancias de SalhiNagy con la configuración -n 200 -p 40 -cprob 50.0 -mprob 40.0}
\centering
\small
\begin{tabular}{c c c c c c c c}
\hline\hline
Instancia & Costo mínimo & Tiempo(seg.) & Costo promedio & Tiempo promedio(seg.) & CME & \%G & \%GP \\ [0.5ex]
\hline
CMT1X & 484.51 & 0.46 & 
487.90 & 0.55 & \bf{470.48} & 
2.98 & 3.70\\CMT1Y & 476.77 & 0.43 & 
489.55 & 0.40 & \bf{470.48} & 
1.34 & 4.05\\CMT2X & 721.13 & 1.72 & 
721.13 & 1.36 & \bf{682.39} & 
5.68 & 5.68\\CMT2Y & 699.79 & 1.66 & 
710.74 & 1.36 & \bf{682.39} & 
2.55 & 4.15\\CMT3X & 733.85 & 3.01 & 
741.97 & 2.90 & \bf{719.06} & 
2.06 & 3.19\\CMT3Y & 724.29 & 2.71 & 
736.67 & 2.80 & \bf{719.06} & 
0.73 & 2.45\\CMT4X & 913.27 & 7.69 & 
917.62 & 7.64 & \bf{854.21} & 
6.91 & 7.42\\CMT4Y & 894.96 & 7.78 & 
910.69 & 7.86 & \bf{852.46} & 
4.99 & 6.83\\CMT5X & 1104.04 & 16.15 & 
1117.54 & 16.77 & \bf{1030.56} & 
7.13 & 8.44\\CMT5Y & 1108.36 & 17.26 & 
1118.00 & 16.83 & \bf{1031.69} & 
7.43 & 8.37\\CMT11X & 893.25 & 6.77 & 
910.98 & 5.46 & \bf{831.09} & 
7.48 & 9.61\\CMT11Y & 867.14 & 5.60 & 
905.07 & 5.47 & \bf{829.85} & 
4.49 & 9.06\\CMT12X & 675.65 & 3.09 & 
684.01 & 3.05 & \bf{658.83} & 
2.55 & 3.82\\CMT12Y & 675.59 & 3.12 & 
680.11 & 2.71 & \bf{660.47} & 
2.29 & 2.97\\\bf{PROM.} & 
\bf{783.76} & \bf{5.53} & \bf{795.14} & \bf{5.37} & \bf{749.50} & \bf{4.19} & \bf{5.70}\\[1ex]\hline
\end{tabular}
\label{table:nonlin}
\end{table} \clearpage
\begin{table}[ht]
\caption{Resultados de la ejecución de la metaheurística IGA, utilizando instancias de Dethloff con la configuración -n 200 -p 40 -cprob 50.0 -mprob 50.0}
\centering
\small
\begin{tabular}{c c c c c c c c}
\hline\hline
Instancia & Costo mínimo & Tiempo(seg.) & Costo promedio & Tiempo promedio(seg.) & CME & \%G & \%GP \\ [0.5ex]
\hline
SCA3-0 & 642.44 & 0.56 & 
642.93 & 0.57 & \bf{635.62} & 
1.07 & 1.15\\SCA3-1 & \bf{697.84} & 0.75 & 
697.84 & 0.80 & 697.84 & 0.00
 & 0.00\\
SCA3-2 & 670.59 & 0.67 & 
673.50 & 0.65 & \bf{659.34} & 
1.71 & 2.15\\SCA3-3 & 681.74 & 0.60 & 
683.33 & 0.50 & \bf{680.04} & 
0.25 & 0.48\\SCA3-4 & \bf{690.50} & 0.62 & 
690.50 & 0.61 & 690.50 & 0.00
 & 0.00\\
SCA3-5 & 665.64 & 0.67 & 
671.47 & 0.86 & \bf{659.90} & 
0.87 & 1.75\\SCA3-6 & 652.94 & 0.65 & 
656.24 & 0.58 & \bf{651.09} & 
0.28 & 0.79\\SCA3-7 & 666.15 & 0.66 & 
666.15 & 0.59 & \bf{659.17} & 
1.06 & 1.06\\SCA3-8 & \bf{719.47} & 0.85 & 
719.47 & 0.69 & 719.47 & 0.00
 & 0.00\\
SCA3-9 & \bf{681.00} & 0.56 & 
681.00 & 0.62 & 681.00 & 0.00
 & 0.00\\
SCA8-0 & 985.87 & 0.63 & 
985.87 & 0.65 & \bf{961.50} & 
2.53 & 2.53\\SCA8-1 & 1082.71 & 0.78 & 
1082.71 & 0.70 & \bf{1049.65} & 
3.15 & 3.15\\SCA8-2 & 1054.69 & 0.77 & 
1057.10 & 0.62 & \bf{1039.64} & 
1.45 & 1.68\\SCA8-3 & 1012.89 & 0.97 & 
1012.89 & 0.86 & \bf{983.34} & 
3.01 & 3.01\\SCA8-4 & 1077.80 & 0.72 & 
1085.87 & 0.77 & \bf{1065.49} & 
1.16 & 1.91\\SCA8-5 & 1050.63 & 0.73 & 
1050.63 & 0.56 & \bf{1027.08} & 
2.29 & 2.29\\SCA8-6 & 993.28 & 0.54 & 
993.49 & 0.56 & \bf{971.82} & 
2.21 & 2.23\\SCA8-7 & 1066.65 & 0.78 & 
1075.09 & 0.63 & \bf{1051.28} & 
1.46 & 2.26\\SCA8-8 & 1094.97 & 0.54 & 
1094.97 & 0.56 & \bf{1071.18} & 
2.22 & 2.22\\SCA8-9 & 1082.02 & 0.71 & 
1082.02 & 0.78 & \bf{1060.50} & 
2.03 & 2.03\\CON3-0 & 619.09 & 0.54 & 
628.61 & 0.64 & \bf{616.52} & 
0.42 & 1.96\\CON3-1 & 560.75 & 0.84 & 
560.75 & 0.81 & \bf{554.47} & 
1.13 & 1.13\\CON3-2 & 521.38 & 0.69 & 
521.38 & 0.72 & \bf{518.00} & 
0.65 & 0.65\\CON3-3 & 604.90 & 0.90 & 
607.89 & 0.72 & \bf{591.19} & 
2.32 & 2.82\\CON3-4 & 605.17 & 0.50 & 
605.80 & 0.57 & \bf{588.79} & 
2.78 & 2.89\\CON3-5 & 564.88 & 0.52 & 
564.88 & 0.66 & \bf{563.70} & 
0.21 & 0.21\\CON3-6 & 503.97 & 0.77 & 
506.42 & 0.70 & \bf{499.05} & 
0.99 & 1.48\\CON3-7 & 582.14 & 0.50 & 
584.08 & 0.64 & \bf{576.48} & 
0.98 & 1.32\\CON3-8 & 523.60 & 0.95 & 
529.80 & 0.65 & \bf{523.05} & 
0.11 & 1.29\\CON3-9 & 588.11 & 0.91 & 
590.91 & 0.79 & \bf{578.24} & 
1.71 & 2.19\\CON8-0 & 885.37 & 0.55 & 
885.37 & 0.56 & \bf{857.17} & 
3.29 & 3.29\\CON8-1 & 755.11 & 0.49 & 
755.11 & 0.51 & \bf{740.85} & 
1.92 & 1.92\\CON8-2 & 722.34 & 0.51 & 
725.41 & 0.80 & \bf{712.89} & 
1.33 & 1.76\\CON8-3 & 817.23 & 0.74 & 
817.23 & 0.70 & \bf{811.07} & 
0.76 & 0.76\\CON8-4 & 802.30 & 0.70 & 
802.30 & 0.71 & \bf{772.25} & 
3.89 & 3.89\\CON8-5 & 768.42 & 0.68 & 
769.61 & 0.60 & \bf{754.88} & 
1.79 & 1.95\\CON8-6 & 696.10 & 0.53 & 
696.10 & 0.70 & \bf{678.92} & 
2.53 & 2.53\\CON8-7 & 827.89 & 0.48 & 
827.89 & 0.49 & \bf{811.96} & 
1.96 & 1.96\\CON8-8 & 788.09 & 1.01 & 
797.00 & 0.87 & \bf{767.53} & 
2.68 & 3.84\\CON8-9 & 827.32 & 0.77 & 
832.47 & 0.70 & \bf{809.00} & 
2.26 & 2.90\\\bf{PROM.} & 
\bf{770.85} & \bf{0.68} & \bf{772.80} & \bf{0.67} & \bf{758.54} & \bf{1.51} & \bf{1.79}\\[1ex]\hline
\end{tabular}
\label{table:nonlin}
\end{table} \clearpage
\begin{table}[ht]
\caption{Resultados de la ejecución de la metaheurística SCA, utilizando instancias de SalhiNagy con la configuración -n 50.0 -b 10 -y .2}
\centering
\small
\begin{tabular}{c c c c c c c c}
\hline\hline
Instancia & Costo mínimo & Tiempo(seg.) & Costo promedio & Tiempo promedio(seg.) & CME & \%G & \%GP \\ [0.5ex]
\hline
CMT1X & 479.10 & 2.01 & 
479.10 & 1.94 & \bf{470.48} & 
1.83 & 1.83\\CMT1Y & 472.87 & 2.88 & 
476.40 & 2.24 & \bf{470.48} & 
0.51 & 1.26\\CMT2X & 706.82 & 26.86 & 
710.29 & 16.96 & \bf{682.39} & 
3.58 & 4.09\\CMT2Y & 706.74 & 14.97 & 
709.04 & 19.32 & \bf{682.39} & 
3.57 & 3.91\\CMT3X & 738.72 & 33.61 & 
740.89 & 30.29 & \bf{719.06} & 
2.73 & 3.04\\CMT3Y & 734.49 & 37.84 & 
738.16 & 29.61 & \bf{719.06} & 
2.15 & 2.66\\CMT4X & 902.98 & | & 
0.00 & 0.00 & \bf{854.21} & 
5.71 & -100.00\\CMT4Y & 887.24905.13889.71 & | & 
0.00 & 0.00 & \bf{852.46} & 
4.08 & -100.00\\CMT5X & 100000 & 0 & 
nan & nan & \bf{1030.56} & 
9603.46 & \bf{nan}\\CMT5Y & 100000 & 0 & 
nan & nan & \bf{1031.69} & 
9592.83 & \bf{nan}\\CMT11X & 907.87 & 22.25 & 
910.17 & 43.40 & \bf{831.09} & 
9.24 & 9.52\\CMT11Y & 890.98901.49 & 27.79 & 
903.55 & 41.80 & \bf{829.85} & 
7.37 & 8.88\\CMT12X & 676.66 & 48.77 & 
680.56 & 37.36 & \bf{658.83} & 
2.71 & 3.30\\CMT12Y & 674.95 & 79.03 & 
681.67 & 76.12 & \bf{660.47} & 
2.19 & 3.21\\\bf{PROM.} & 
\bf{14912.82} & \bf{21.14} & \bf{nan} & \bf{nan} & \bf{749.50} & \bf{1374.43} & \bf{nan}\\[1ex]\hline
\end{tabular}
\label{table:nonlin}
\end{table} \clearpage
\begin{table}[ht]
\caption{Resultados de la ejecución de la metaheurística IGA, utilizando instancias de SalhiNagy con la configuración -n 200 -p 40 -cprob 50.0 -mprob 50.0}
\centering
\small
\begin{tabular}{c c c c c c c c}
\hline\hline
Instancia & Costo mínimo & Tiempo(seg.) & Costo promedio & Tiempo promedio(seg.) & CME & \%G & \%GP \\ [0.5ex]
\hline
CMT1X & 481.27 & 0.62 & 
481.27 & 0.62 & \bf{470.48} & 
2.29 & 2.29\\CMT1Y & 477.41 & 0.80 & 
489.77 & 0.58 & \bf{470.48} & 
1.47 & 4.10\\CMT2X & 711.95 & 1.24 & 
714.51 & 1.39 & \bf{682.39} & 
4.33 & 4.71\\CMT2Y & 710.10 & 1.37 & 
714.06 & 1.35 & \bf{682.39} & 
4.06 & 4.64\\CMT3X & 738.45 & 3.49 & 
742.89 & 3.25 & \bf{719.06} & 
2.70 & 3.31\\CMT3Y & 738.21 & 3.09 & 
745.48 & 2.99 & \bf{719.06} & 
2.66 & 3.67\\CMT4X & 905.29 & 7.68 & 
912.88 & 7.66 & \bf{854.21} & 
5.98 & 6.87\\CMT4Y & 908.06 & 8.33 & 
917.86 & 8.18 & \bf{852.46} & 
6.52 & 7.67\\CMT5X & 1111.78 & 15.69 & 
1123.43 & 16.57 & \bf{1030.56} & 
7.88 & 9.01\\CMT5Y & 1110.84 & 19.56 & 
1118.14 & 18.34 & \bf{1031.69} & 
7.67 & 8.38\\CMT11X & 910.63 & 5.36 & 
924.00 & 4.81 & \bf{831.09} & 
9.57 & 11.18\\CMT11Y & 888.56 & 5.38 & 
901.45 & 5.51 & \bf{829.85} & 
7.07 & 8.63\\CMT12X & 674.63 & 2.81 & 
682.52 & 2.78 & \bf{658.83} & 
2.40 & 3.60\\CMT12Y & 676.88 & 2.94 & 
680.50 & 3.12 & \bf{660.47} & 
2.48 & 3.03\\\bf{PROM.} & 
\bf{788.86} & \bf{5.60} & \bf{796.34} & \bf{5.51} & \bf{749.50} & \bf{4.79} & \bf{5.79}\\[1ex]\hline
\end{tabular}
\label{table:nonlin}
\end{table} \clearpage
\begin{table}[ht]
\caption{Resultados de la ejecución de la metaheurística IGA, utilizando instancias de Dethloff con la configuración -n 200 -p 40 -cprob 50.0 -mprob 60.0}
\centering
\small
\begin{tabular}{c c c c c c c c}
\hline\hline
Instancia & Costo mínimo & Tiempo(seg.) & Costo promedio & Tiempo promedio(seg.) & CME & \%G & \%GP \\ [0.5ex]
\hline
SCA3-0 & 641.69 & 0.72 & 
642.42 & 0.68 & \bf{635.62} & 
0.95 & 1.07\\SCA3-1 & \bf{697.84} & 0.75 & 
703.02 & 0.72 & 697.84 & 0.00
 & 0.74\\SCA3-2 & 664.21 & 0.63 & 
670.41 & 0.71 & \bf{659.34} & 
0.74 & 1.68\\SCA3-3 & 685.76 & 0.52 & 
685.76 & 0.52 & \bf{680.04} & 
0.84 & 0.84\\SCA3-4 & \bf{690.50} & 0.90 & 
692.05 & 0.64 & 690.50 & 0.00
 & 0.22\\SCA3-5 & 680.66 & 0.76 & 
680.66 & 0.76 & \bf{659.90} & 
3.15 & 3.15\\SCA3-6 & 656.79 & 0.49 & 
657.38 & 0.53 & \bf{651.09} & 
0.88 & 0.97\\SCA3-7 & 666.15 & 0.54 & 
666.15 & 0.71 & \bf{659.17} & 
1.06 & 1.06\\SCA3-8 & 724.66 & 0.54 & 
725.55 & 0.70 & \bf{719.47} & 
0.72 & 0.85\\SCA3-9 & \bf{681.00} & 0.80 & 
681.00 & 0.74 & 681.00 & 0.00
 & 0.00\\
SCA8-0 & 973.22 & 0.53 & 
981.09 & 0.52 & \bf{961.50} & 
1.22 & 2.04\\SCA8-1 & 1077.15 & 0.69 & 
1078.27 & 0.82 & \bf{1049.65} & 
2.62 & 2.73\\SCA8-2 & 1054.47 & 0.79 & 
1054.47 & 0.78 & \bf{1039.64} & 
1.43 & 1.43\\SCA8-3 & 1012.89 & 0.72 & 
1012.89 & 0.72 & \bf{983.34} & 
3.01 & 3.01\\SCA8-4 & 1105.95 & 0.46 & 
1105.95 & 0.46 & \bf{1065.49} & 
3.80 & 3.80\\SCA8-5 & 1056.49 & 0.68 & 
1056.49 & 0.60 & \bf{1027.08} & 
2.86 & 2.86\\SCA8-6 & 972.48 & 0.49 & 
972.48 & 0.55 & \bf{971.82} & 
0.07 & 0.07\\SCA8-7 & 1067.03 & 0.49 & 
1067.03 & 0.50 & \bf{1051.28} & 
1.50 & 1.50\\SCA8-8 & 1095.58 & 0.66 & 
1099.13 & 0.77 & \bf{1071.18} & 
2.28 & 2.61\\SCA8-9 & 1078.30 & 0.54 & 
1081.70 & 0.56 & \bf{1060.50} & 
1.68 & 2.00\\CON3-0 & 623.84 & 0.57 & 
625.62 & 0.71 & \bf{616.52} & 
1.19 & 1.48\\CON3-1 & 561.87 & 0.49 & 
561.87 & 0.64 & \bf{554.47} & 
1.33 & 1.33\\CON3-2 & 521.38 & 0.64 & 
521.38 & 0.74 & \bf{518.00} & 
0.65 & 0.65\\CON3-3 & 599.53 & 0.74 & 
601.13 & 0.73 & \bf{591.19} & 
1.41 & 1.68\\CON3-4 & 593.78 & 0.56 & 
602.20 & 0.63 & \bf{588.79} & 
0.85 & 2.28\\CON3-5 & 567.94 & 0.72 & 
569.10 & 0.71 & \bf{563.70} & 
0.75 & 0.96\\CON3-6 & 507.81 & 0.64 & 
507.83 & 0.69 & \bf{499.05} & 
1.76 & 1.76\\CON3-7 & 578.41 & 0.45 & 
587.66 & 0.62 & \bf{576.48} & 
0.33 & 1.94\\CON3-8 & \bf{523.05} & 0.73 & 
524.84 & 0.72 & 523.05 & 0.00
 & 0.34\\CON3-9 & 582.79 & 0.58 & 
585.45 & 0.77 & \bf{578.24} & 
0.79 & 1.25\\CON8-0 & 881.15 & 0.57 & 
881.15 & 0.66 & \bf{857.17} & 
2.80 & 2.80\\CON8-1 & 772.63 & 1.02 & 
772.78 & 0.71 & \bf{740.85} & 
4.29 & 4.31\\CON8-2 & 727.98 & 0.72 & 
729.24 & 0.66 & \bf{712.89} & 
2.12 & 2.29\\CON8-3 & 839.56 & 0.48 & 
839.56 & 0.68 & \bf{811.07} & 
3.51 & 3.51\\CON8-4 & 776.98 & 0.73 & 
782.03 & 0.75 & \bf{772.25} & 
0.61 & 1.27\\CON8-5 & 762.36 & 0.77 & 
762.49 & 0.75 & \bf{754.88} & 
0.99 & 1.01\\CON8-6 & 693.86 & 0.72 & 
697.84 & 0.73 & \bf{678.92} & 
2.20 & 2.79\\CON8-7 & 845.40 & 0.55 & 
845.40 & 0.58 & \bf{811.96} & 
4.12 & 4.12\\CON8-8 & 795.12 & 1.02 & 
795.12 & 0.84 & \bf{767.53} & 
3.59 & 3.59\\CON8-9 & 824.39 & 0.68 & 
833.38 & 0.79 & \bf{809.00} & 
1.90 & 3.01\\\bf{PROM.} & 
\bf{771.57} & \bf{0.65} & \bf{773.50} & \bf{0.68} & \bf{758.54} & \bf{1.60} & \bf{1.87}\\[1ex]\hline
\end{tabular}
\label{table:nonlin}
\end{table} \clearpage
\begin{table}[ht]
\caption{Resultados de la ejecución de la metaheurística IGA, utilizando instancias de SalhiNagy con la configuración -n 200 -p 40 -cprob 50.0 -mprob 60.0}
\centering
\small
\begin{tabular}{c c c c c c c c}
\hline\hline
Instancia & Costo mínimo & Tiempo(seg.) & Costo promedio & Tiempo promedio(seg.) & CME & \%G & \%GP \\ [0.5ex]
\hline
CMT1X & 477.21 & 0.64 & 
482.88 & 0.58 & \bf{470.48} & 
1.43 & 2.64\\CMT1Y & 481.54 & 0.39 & 
488.81 & 0.38 & \bf{470.48} & 
2.35 & 3.90\\CMT2X & 715.40 & 1.32 & 
716.93 & 1.30 & \bf{682.39} & 
4.84 & 5.06\\CMT2Y & 709.31 & 1.16 & 
713.52 & 1.17 & \bf{682.39} & 
3.94 & 4.56\\CMT3X & 737.75 & 2.96 & 
742.50 & 2.96 & \bf{719.06} & 
2.60 & 3.26\\CMT3Y & 734.75 & 2.92 & 
741.90 & 3.08 & \bf{719.06} & 
2.18 & 3.18\\CMT4X & 901.93 & 7.31 & 
915.25 & 7.67 & \bf{854.21} & 
5.59 & 7.15\\CMT4Y & 898.22 & 8.50 & 
916.45 & 8.24 & \bf{852.46} & 
5.37 & 7.51\\CMT5X & 1100.34 & 15.12 & 
1120.09 & 16.10 & \bf{1030.56} & 
6.77 & 8.69\\CMT5Y & 1099.91 & 15.68 & 
1121.54 & 16.63 & \bf{1031.69} & 
6.61 & 8.71\\CMT11X & 902.99 & 5.12 & 
914.87 & 4.94 & \bf{831.09} & 
8.65 & 10.08\\CMT11Y & 902.98 & 6.02 & 
909.23 & 5.64 & \bf{829.85} & 
8.81 & 9.57\\CMT12X & 676.48 & 3.16 & 
682.39 & 3.00 & \bf{658.83} & 
2.68 & 3.58\\CMT12Y & 678.99 & 2.72 & 
683.70 & 2.92 & \bf{660.47} & 
2.80 & 3.52\\\bf{PROM.} & 
\bf{786.99} & \bf{5.22} & \bf{796.43} & \bf{5.33} & \bf{749.50} & \bf{4.62} & \bf{5.81}\\[1ex]\hline
\end{tabular}
\label{table:nonlin}
\end{table} \clearpage
\begin{table}[ht]
\caption{Resultados de la ejecución de la metaheurística IGA, utilizando instancias de Dethloff con la configuración -n 200 -p 40 -cprob 50.0 -mprob 70.0}
\centering
\small
\begin{tabular}{c c c c c c c c}
\hline\hline
Instancia & Costo mínimo & Tiempo(seg.) & Costo promedio & Tiempo promedio(seg.) & CME & \%G & \%GP \\ [0.5ex]
\hline
SCA3-0 & 640.55 & 0.72 & 
640.55 & 0.72 & \bf{635.62} & 
0.78 & 0.78\\SCA3-1 & 701.74 & 0.78 & 
703.04 & 0.86 & \bf{697.84} & 
0.56 & 0.75\\SCA3-2 & 661.13 & 0.89 & 
663.50 & 0.80 & \bf{659.34} & 
0.27 & 0.63\\SCA3-3 & \bf{680.04} & 0.48 & 
681.86 & 0.52 & 680.04 & 0.00
 & 0.27\\SCA3-4 & \bf{690.50} & 0.62 & 
690.50 & 0.51 & 690.50 & 0.00
 & 0.00\\
SCA3-5 & 679.84 & 0.58 & 
684.42 & 0.56 & \bf{659.90} & 
3.02 & 3.72\\SCA3-6 & 652.94 & 0.77 & 
654.86 & 0.88 & \bf{651.09} & 
0.28 & 0.58\\SCA3-7 & 666.15 & 0.82 & 
666.97 & 0.58 & \bf{659.17} & 
1.06 & 1.18\\SCA3-8 & \bf{719.47} & 0.95 & 
719.47 & 0.81 & 719.47 & 0.00
 & 0.00\\
SCA3-9 & 687.61 & 0.57 & 
688.14 & 0.59 & \bf{681.00} & 
0.97 & 1.05\\SCA8-0 & 994.87 & 0.63 & 
1002.59 & 0.70 & \bf{961.50} & 
3.47 & 4.27\\SCA8-1 & 1083.65 & 0.69 & 
1083.65 & 0.69 & \bf{1049.65} & 
3.24 & 3.24\\SCA8-2 & 1053.94 & 1.01 & 
1054.34 & 0.79 & \bf{1039.64} & 
1.38 & 1.41\\SCA8-3 & 1010.01 & 0.94 & 
1026.00 & 0.74 & \bf{983.34} & 
2.71 & 4.34\\SCA8-4 & 1097.89 & 0.56 & 
1099.77 & 0.69 & \bf{1065.49} & 
3.04 & 3.22\\SCA8-5 & 1054.10 & 0.72 & 
1054.10 & 0.86 & \bf{1027.08} & 
2.63 & 2.63\\SCA8-6 & 998.31 & 0.43 & 
998.31 & 0.66 & \bf{971.82} & 
2.73 & 2.73\\SCA8-7 & 1062.19 & 0.98 & 
1062.19 & 0.81 & \bf{1051.28} & 
1.04 & 1.04\\SCA8-8 & 1084.41 & 0.43 & 
1084.41 & 0.70 & \bf{1071.18} & 
1.24 & 1.24\\SCA8-9 & 1072.10 & 0.72 & 
1072.10 & 0.61 & \bf{1060.50} & 
1.09 & 1.09\\CON3-0 & 619.09 & 0.68 & 
625.64 & 0.68 & \bf{616.52} & 
0.42 & 1.48\\CON3-1 & 560.75 & 0.93 & 
561.56 & 0.77 & \bf{554.47} & 
1.13 & 1.28\\CON3-2 & 521.38 & 0.69 & 
521.38 & 0.85 & \bf{518.00} & 
0.65 & 0.65\\CON3-3 & 595.87 & 0.58 & 
601.86 & 0.76 & \bf{591.19} & 
0.79 & 1.80\\CON3-4 & 591.43 & 0.57 & 
598.25 & 0.72 & \bf{588.79} & 
0.45 & 1.61\\CON3-5 & 566.96 & 0.51 & 
566.96 & 0.72 & \bf{563.70} & 
0.58 & 0.58\\CON3-6 & 504.20 & 0.94 & 
504.90 & 0.69 & \bf{499.05} & 
1.03 & 1.17\\CON3-7 & 582.14 & 0.48 & 
582.14 & 0.50 & \bf{576.48} & 
0.98 & 0.98\\CON3-8 & 528.09 & 0.84 & 
531.18 & 0.91 & \bf{523.05} & 
0.96 & 1.56\\CON3-9 & 590.29 & 0.74 & 
590.49 & 0.69 & \bf{578.24} & 
2.08 & 2.12\\CON8-0 & 876.68 & 0.86 & 
876.68 & 0.91 & \bf{857.17} & 
2.28 & 2.28\\CON8-1 & 748.96 & 0.68 & 
748.96 & 0.81 & \bf{740.85} & 
1.09 & 1.09\\CON8-2 & 721.62 & 0.52 & 
729.75 & 0.69 & \bf{712.89} & 
1.22 & 2.37\\CON8-3 & 819.89 & 0.86 & 
827.79 & 0.76 & \bf{811.07} & 
1.09 & 2.06\\CON8-4 & 806.79 & 0.55 & 
807.92 & 0.62 & \bf{772.25} & 
4.47 & 4.62\\CON8-5 & 762.36 & 0.50 & 
764.31 & 0.81 & \bf{754.88} & 
0.99 & 1.25\\CON8-6 & 690.27 & 0.83 & 
690.27 & 0.62 & \bf{678.92} & 
1.67 & 1.67\\CON8-7 & 839.24 & 0.72 & 
843.50 & 0.77 & \bf{811.96} & 
3.36 & 3.88\\CON8-8 & 792.28 & 0.63 & 
792.28 & 0.70 & \bf{767.53} & 
3.22 & 3.22\\CON8-9 & 827.11 & 0.52 & 
836.54 & 0.67 & \bf{809.00} & 
2.24 & 3.40\\\bf{PROM.} & 
\bf{770.92} & \bf{0.70} & \bf{773.33} & \bf{0.72} & \bf{758.54} & \bf{1.51} & \bf{1.83}\\[1ex]\hline
\end{tabular}
\label{table:nonlin}
\end{table} \clearpage
\begin{table}[ht]
\caption{Resultados de la ejecución de la metaheurística IGA, utilizando instancias de SalhiNagy con la configuración -n 200 -p 40 -cprob 50.0 -mprob 70.0}
\centering
\small
\begin{tabular}{c c c c c c c c}
\hline\hline
Instancia & Costo mínimo & Tiempo(seg.) & Costo promedio & Tiempo promedio(seg.) & CME & \%G & \%GP \\ [0.5ex]
\hline
CMT1X & 481.74 & 0.45 & 
482.24 & 0.44 & \bf{470.48} & 
2.39 & 2.50\\CMT1Y & 482.44 & 0.40 & 
487.94 & 0.51 & \bf{470.48} & 
2.54 & 3.71\\CMT2X & 699.81 & 1.60 & 
711.10 & 1.44 & \bf{682.39} & 
2.55 & 4.21\\CMT2Y & 706.87 & 1.51 & 
714.84 & 1.56 & \bf{682.39} & 
3.59 & 4.75\\CMT3X & 736.45 & 2.95 & 
745.72 & 2.83 & \bf{719.06} & 
2.42 & 3.71\\CMT3Y & 733.79 & 3.19 & 
747.44 & 3.13 & \bf{719.06} & 
2.05 & 3.95\\CMT4X & 899.93 & 7.14 & 
912.24 & 7.59 & \bf{854.21} & 
5.35 & 6.79\\CMT4Y & 901.27 & 8.12 & 
912.79 & 8.39 & \bf{852.46} & 
5.73 & 7.08\\CMT5X & 1092.30 & 16.52 & 
1105.78 & 16.70 & \bf{1030.56} & 
5.99 & 7.30\\CMT5Y & 1109.18 & 15.57 & 
1126.65 & 16.50 & \bf{1031.69} & 
7.51 & 9.20\\CMT11X & 900.17 & 5.22 & 
913.62 & 5.08 & \bf{831.09} & 
8.31 & 9.93\\CMT11Y & 895.48 & 5.25 & 
902.16 & 5.36 & \bf{829.85} & 
7.91 & 8.71\\CMT12X & 674.49 & 3.11 & 
680.28 & 2.96 & \bf{658.83} & 
2.38 & 3.26\\CMT12Y & 677.74 & 3.34 & 
678.22 & 3.03 & \bf{660.47} & 
2.61 & 2.69\\\bf{PROM.} & 
\bf{785.12} & \bf{5.31} & \bf{794.36} & \bf{5.39} & \bf{749.50} & \bf{4.38} & \bf{5.56}\\[1ex]\hline
\end{tabular}
\label{table:nonlin}
\end{table} \clearpage
\begin{table}[ht]
\caption{Resultados de la ejecución de la metaheurística IGA, utilizando instancias de Dethloff con la configuración -n 200 -p 40 -cprob 50.0 -mprob 80.0}
\centering
\small
\begin{tabular}{c c c c c c c c}
\hline\hline
Instancia & Costo mínimo & Tiempo(seg.) & Costo promedio & Tiempo promedio(seg.) & CME & \%G & \%GP \\ [0.5ex]
\hline
SCA3-0 & 636.06 & 0.50 & 
640.18 & 0.84 & \bf{635.62} & 
0.07 & 0.72\\SCA3-1 & 701.53 & 0.94 & 
707.87 & 0.68 & \bf{697.84} & 
0.53 & 1.44\\SCA3-2 & 664.21 & 0.70 & 
666.03 & 0.77 & \bf{659.34} & 
0.74 & 1.02\\SCA3-3 & 682.46 & 0.72 & 
683.75 & 0.69 & \bf{680.04} & 
0.36 & 0.55\\SCA3-4 & \bf{690.50} & 0.62 & 
690.50 & 0.75 & 690.50 & 0.00
 & 0.00\\
SCA3-5 & 670.74 & 0.49 & 
674.79 & 0.78 & \bf{659.90} & 
1.64 & 2.26\\SCA3-6 & 652.94 & 0.48 & 
655.93 & 0.65 & \bf{651.09} & 
0.28 & 0.74\\SCA3-7 & 666.15 & 0.86 & 
666.15 & 0.65 & \bf{659.17} & 
1.06 & 1.06\\SCA3-8 & 723.99 & 0.76 & 
724.07 & 0.70 & \bf{719.47} & 
0.63 & 0.64\\SCA3-9 & \bf{681.00} & 0.88 & 
685.68 & 0.78 & 681.00 & 0.00
 & 0.69\\SCA8-0 & 999.98 & 0.56 & 
1000.56 & 0.61 & \bf{961.50} & 
4.00 & 4.06\\SCA8-1 & 1071.11 & 0.74 & 
1084.25 & 0.76 & \bf{1049.65} & 
2.04 & 3.30\\SCA8-2 & 1054.68 & 0.58 & 
1057.84 & 0.59 & \bf{1039.64} & 
1.45 & 1.75\\SCA8-3 & 1026.70 & 1.00 & 
1026.70 & 0.72 & \bf{983.34} & 
4.41 & 4.41\\SCA8-4 & 1080.57 & 0.52 & 
1080.57 & 0.68 & \bf{1065.49} & 
1.42 & 1.42\\SCA8-5 & 1061.97 & 0.69 & 
1062.92 & 0.80 & \bf{1027.08} & 
3.40 & 3.49\\SCA8-6 & 990.34 & 0.88 & 
998.75 & 0.85 & \bf{971.82} & 
1.91 & 2.77\\SCA8-7 & 1074.24 & 0.58 & 
1074.24 & 0.68 & \bf{1051.28} & 
2.18 & 2.18\\SCA8-8 & \bf{1071.18} & 0.96 & 
1076.93 & 0.84 & 1071.18 & 0.00
 & 0.54\\SCA8-9 & 1072.10 & 0.86 & 
1072.10 & 0.73 & \bf{1060.50} & 
1.09 & 1.09\\CON3-0 & 622.30 & 0.58 & 
623.15 & 0.82 & \bf{616.52} & 
0.94 & 1.07\\CON3-1 & 560.75 & 0.59 & 
561.25 & 0.70 & \bf{554.47} & 
1.13 & 1.22\\CON3-2 & 521.38 & 1.01 & 
522.26 & 0.80 & \bf{518.00} & 
0.65 & 0.82\\CON3-3 & 592.43 & 0.67 & 
599.47 & 0.72 & \bf{591.19} & 
0.21 & 1.40\\CON3-4 & 594.59 & 0.52 & 
594.59 & 0.59 & \bf{588.79} & 
0.99 & 0.99\\CON3-5 & 567.07 & 0.81 & 
570.20 & 0.81 & \bf{563.70} & 
0.60 & 1.15\\CON3-6 & 504.15 & 0.77 & 
504.83 & 0.66 & \bf{499.05} & 
1.02 & 1.16\\CON3-7 & 581.82 & 0.56 & 
583.08 & 0.58 & \bf{576.48} & 
0.93 & 1.14\\CON3-8 & 524.59 & 0.60 & 
526.47 & 0.76 & \bf{523.05} & 
0.29 & 0.65\\CON3-9 & 582.79 & 0.69 & 
586.59 & 0.78 & \bf{578.24} & 
0.79 & 1.44\\CON8-0 & 876.58 & 1.02 & 
878.73 & 0.74 & \bf{857.17} & 
2.26 & 2.52\\CON8-1 & 756.81 & 1.02 & 
756.81 & 0.72 & \bf{740.85} & 
2.15 & 2.15\\CON8-2 & 719.09 & 0.72 & 
719.09 & 0.85 & \bf{712.89} & 
0.87 & 0.87\\CON8-3 & 838.94 & 1.05 & 
841.00 & 0.86 & \bf{811.07} & 
3.44 & 3.69\\CON8-4 & 772.32 & 0.95 & 
772.32 & 0.84 & \bf{772.25} & 
0.01 & 0.01\\CON8-5 & 768.21 & 0.69 & 
768.86 & 0.62 & \bf{754.88} & 
1.77 & 1.85\\CON8-6 & 698.57 & 0.60 & 
700.16 & 0.65 & \bf{678.92} & 
2.89 & 3.13\\CON8-7 & 815.44 & 0.86 & 
815.44 & 0.72 & \bf{811.96} & 
0.43 & 0.43\\CON8-8 & 782.35 & 1.00 & 
788.43 & 0.85 & \bf{767.53} & 
1.93 & 2.72\\CON8-9 & 826.57 & 0.76 & 
829.42 & 0.63 & \bf{809.00} & 
2.17 & 2.52\\\bf{PROM.} & 
\bf{769.48} & \bf{0.74} & \bf{771.80} & \bf{0.73} & \bf{758.54} & \bf{1.32} & \bf{1.63}\\[1ex]\hline
\end{tabular}
\label{table:nonlin}
\end{table} \clearpage
\begin{table}[ht]
\caption{Resultados de la ejecución de la metaheurística IGA, utilizando instancias de SalhiNagy con la configuración -n 200 -p 40 -cprob 50.0 -mprob 80.0}
\centering
\small
\begin{tabular}{c c c c c c c c}
\hline\hline
Instancia & Costo mínimo & Tiempo(seg.) & Costo promedio & Tiempo promedio(seg.) & CME & \%G & \%GP \\ [0.5ex]
\hline
CMT1X & 481.03 & 0.56 & 
484.81 & 0.68 & \bf{470.48} & 
2.24 & 3.05\\CMT1Y & 482.59 & 0.82 & 
485.93 & 0.62 & \bf{470.48} & 
2.57 & 3.28\\CMT2X & 712.95 & 1.72 & 
716.87 & 1.40 & \bf{682.39} & 
4.48 & 5.05\\CMT2Y & 706.13 & 1.42 & 
714.19 & 1.55 & \bf{682.39} & 
3.48 & 4.66\\CMT3X & 733.95 & 3.00 & 
740.17 & 2.84 & \bf{719.06} & 
2.07 & 2.94\\CMT3Y & 731.44 & 3.03 & 
740.06 & 2.85 & \bf{719.06} & 
1.72 & 2.92\\CMT4X & 906.05 & 7.49 & 
912.21 & 8.13 & \bf{854.21} & 
6.07 & 6.79\\CMT4Y & 896.57 & 8.07 & 
914.90 & 8.16 & \bf{852.46} & 
5.17 & 7.33\\CMT5X & 1124.98 & 17.18 & 
1130.28 & 16.42 & \bf{1030.56} & 
9.16 & 9.68\\CMT5Y & 1114.59 & 16.18 & 
1127.84 & 16.98 & \bf{1031.69} & 
8.04 & 9.32\\CMT11X & 907.44 & 5.25 & 
914.14 & 5.17 & \bf{831.09} & 
9.19 & 9.99\\CMT11Y & 905.98 & 5.34 & 
918.11 & 5.36 & \bf{829.85} & 
9.17 & 10.64\\CMT12X & 677.10 & 2.93 & 
685.61 & 2.82 & \bf{658.83} & 
2.77 & 4.06\\CMT12Y & 673.67 & 3.06 & 
678.04 & 2.93 & \bf{660.47} & 
2.00 & 2.66\\\bf{PROM.} & 
\bf{789.60} & \bf{5.43} & \bf{797.37} & \bf{5.42} & \bf{749.50} & \bf{4.87} & \bf{5.88}\\[1ex]\hline
\end{tabular}
\label{table:nonlin}
\end{table} \clearpage
\begin{table}[ht]
\caption{Resultados de la ejecución de la metaheurística IGA, utilizando instancias de Dethloff con la configuración -n 200 -p 40 -cprob 50.0 -mprob 90.0}
\centering
\small
\begin{tabular}{c c c c c c c c}
\hline\hline
Instancia & Costo mínimo & Tiempo(seg.) & Costo promedio & Tiempo promedio(seg.) & CME & \%G & \%GP \\ [0.5ex]
\hline
SCA3-0 & 636.34 & 0.54 & 
641.38 & 0.65 & \bf{635.62} & 
0.11 & 0.91\\SCA3-1 & \bf{697.84} & 0.80 & 
704.97 & 0.67 & 697.84 & 0.00
 & 1.02\\SCA3-2 & 666.01 & 0.59 & 
669.00 & 0.58 & \bf{659.34} & 
1.01 & 1.47\\SCA3-3 & 680.60 & 0.55 & 
683.58 & 0.60 & \bf{680.04} & 
0.08 & 0.52\\SCA3-4 & \bf{690.50} & 0.88 & 
691.19 & 0.89 & 690.50 & 0.00
 & 0.10\\SCA3-5 & 670.10 & 0.90 & 
675.67 & 0.64 & \bf{659.90} & 
1.55 & 2.39\\SCA3-6 & \bf{651.09} & 0.69 & 
653.71 & 0.59 & 651.09 & 0.00
 & 0.40\\SCA3-7 & 666.15 & 0.90 & 
666.26 & 0.79 & \bf{659.17} & 
1.06 & 1.08\\SCA3-8 & 719.77 & 0.64 & 
725.45 & 0.57 & \bf{719.47} & 
0.04 & 0.83\\SCA3-9 & \bf{681.00} & 0.93 & 
682.92 & 0.73 & 681.00 & 0.00
 & 0.28\\SCA8-0 & 982.18 & 0.58 & 
988.10 & 0.63 & \bf{961.50} & 
2.15 & 2.77\\SCA8-1 & 1085.62 & 0.50 & 
1085.62 & 0.74 & \bf{1049.65} & 
3.43 & 3.43\\SCA8-2 & 1053.78 & 0.64 & 
1053.78 & 0.55 & \bf{1039.64} & 
1.36 & 1.36\\SCA8-3 & 1000.75 & 0.49 & 
1012.23 & 0.58 & \bf{983.34} & 
1.77 & 2.94\\SCA8-4 & 1079.32 & 0.52 & 
1079.32 & 0.54 & \bf{1065.49} & 
1.30 & 1.30\\SCA8-5 & 1050.46 & 0.70 & 
1050.46 & 0.85 & \bf{1027.08} & 
2.28 & 2.28\\SCA8-6 & 992.25 & 0.76 & 
1001.01 & 0.77 & \bf{971.82} & 
2.10 & 3.00\\SCA8-7 & 1089.29 & 0.98 & 
1089.29 & 0.75 & \bf{1051.28} & 
3.62 & 3.62\\SCA8-8 & 1097.58 & 0.94 & 
1097.58 & 0.64 & \bf{1071.18} & 
2.46 & 2.46\\SCA8-9 & 1074.21 & 0.92 & 
1090.08 & 0.65 & \bf{1060.50} & 
1.29 & 2.79\\CON3-0 & 621.82 & 0.54 & 
626.30 & 0.59 & \bf{616.52} & 
0.86 & 1.59\\CON3-1 & 559.52 & 0.70 & 
560.13 & 0.84 & \bf{554.47} & 
0.91 & 1.02\\CON3-2 & 521.38 & 0.93 & 
521.38 & 0.85 & \bf{518.00} & 
0.65 & 0.65\\CON3-3 & 592.43 & 0.67 & 
598.17 & 0.59 & \bf{591.19} & 
0.21 & 1.18\\CON3-4 & 595.25 & 0.93 & 
595.78 & 0.69 & \bf{588.79} & 
1.10 & 1.19\\CON3-5 & 568.76 & 0.65 & 
568.76 & 0.57 & \bf{563.70} & 
0.90 & 0.90\\CON3-6 & 510.41 & 0.69 & 
510.45 & 0.73 & \bf{499.05} & 
2.28 & 2.29\\CON3-7 & 577.54 & 0.68 & 
581.48 & 0.80 & \bf{576.48} & 
0.18 & 0.87\\CON3-8 & 531.84 & 0.79 & 
532.36 & 0.67 & \bf{523.05} & 
1.68 & 1.78\\CON3-9 & 588.11 & 0.56 & 
588.11 & 0.74 & \bf{578.24} & 
1.71 & 1.71\\CON8-0 & 858.63 & 0.54 & 
858.63 & 0.72 & \bf{857.17} & 
0.17 & 0.17\\CON8-1 & 764.79 & 0.74 & 
767.47 & 0.82 & \bf{740.85} & 
3.23 & 3.59\\CON8-2 & 720.68 & 0.56 & 
720.68 & 0.75 & \bf{712.89} & 
1.09 & 1.09\\CON8-3 & 837.01 & 0.71 & 
837.01 & 0.81 & \bf{811.07} & 
3.20 & 3.20\\CON8-4 & 772.32 & 0.90 & 
782.85 & 0.79 & \bf{772.25} & 
0.01 & 1.37\\CON8-5 & 773.73 & 0.98 & 
773.73 & 0.82 & \bf{754.88} & 
2.50 & 2.50\\CON8-6 & 689.53 & 0.48 & 
690.60 & 0.69 & \bf{678.92} & 
1.56 & 1.72\\CON8-7 & 825.89 & 0.68 & 
825.89 & 0.64 & \bf{811.96} & 
1.72 & 1.72\\CON8-8 & 786.26 & 0.84 & 
790.59 & 0.72 & \bf{767.53} & 
2.44 & 3.00\\CON8-9 & 832.00 & 0.79 & 
832.41 & 0.81 & \bf{809.00} & 
2.84 & 2.89\\\bf{PROM.} & 
\bf{769.82} & \bf{0.72} & \bf{772.61} & \bf{0.70} & \bf{758.54} & \bf{1.37} & \bf{1.73}\\[1ex]\hline
\end{tabular}
\label{table:nonlin}
\end{table} \clearpage
\begin{table}[ht]
\caption{Resultados de la ejecución de la metaheurística IGA, utilizando instancias de SalhiNagy con la configuración -n 200 -p 40 -cprob 50.0 -mprob 90.0}
\centering
\small
\begin{tabular}{c c c c c c c c}
\hline\hline
Instancia & Costo mínimo & Tiempo(seg.) & Costo promedio & Tiempo promedio(seg.) & CME & \%G & \%GP \\ [0.5ex]
\hline
CMT1X & 477.41 & 0.74 & 
480.57 & 0.65 & \bf{470.48} & 
1.47 & 2.14\\CMT1Y & 476.56 & 0.48 & 
482.70 & 0.66 & \bf{470.48} & 
1.29 & 2.60\\CMT2X & 708.09 & 1.77 & 
708.46 & 1.53 & \bf{682.39} & 
3.77 & 3.82\\CMT2Y & 709.42 & 0.99 & 
716.55 & 1.35 & \bf{682.39} & 
3.96 & 5.01\\CMT3X & 741.45 & 3.40 & 
746.67 & 3.23 & \bf{719.06} & 
3.11 & 3.84\\CMT3Y & 742.68 & 2.83 & 
748.12 & 3.15 & \bf{719.06} & 
3.28 & 4.04\\CMT4X & 910.07 & 8.36 & 
923.36 & 8.02 & \bf{854.21} & 
6.54 & 8.09\\CMT4Y & 896.37 & 7.51 & 
915.83 & 7.74 & \bf{852.46} & 
5.15 & 7.43\\CMT5X & 1109.92 & 15.65 & 
1132.10 & 16.01 & \bf{1030.56} & 
7.70 & 9.85\\CMT5Y & 1088.99 & 15.81 & 
1117.89 & 16.03 & \bf{1031.69} & 
5.55 & 8.36\\CMT11X & 898.10 & 4.98 & 
908.40 & 5.03 & \bf{831.09} & 
8.06 & 9.30\\CMT11Y & 863.63 & 5.66 & 
910.12 & 5.48 & \bf{829.85} & 
4.07 & 9.67\\CMT12X & 678.57 & 2.74 & 
684.55 & 2.89 & \bf{658.83} & 
3.00 & 3.90\\CMT12Y & 674.93 & 3.14 & 
681.12 & 2.90 & \bf{660.47} & 
2.19 & 3.13\\\bf{PROM.} & 
\bf{784.01} & \bf{5.29} & \bf{796.89} & \bf{5.33} & \bf{749.50} & \bf{4.23} & \bf{5.80}\\[1ex]\hline
\end{tabular}
\label{table:nonlin}
\end{table} \clearpage
\begin{table}[ht]
\caption{Resultados de la ejecución de la metaheurística IGA, utilizando instancias de Dethloff con la configuración -n 200 -p 40 -cprob 50.0 -mprob 100.0}
\centering
\small
\begin{tabular}{c c c c c c c c}
\hline\hline
Instancia & Costo mínimo & Tiempo(seg.) & Costo promedio & Tiempo promedio(seg.) & CME & \%G & \%GP \\ [0.5ex]
\hline
SCA3-0 & 640.55 & 0.72 & 
640.55 & 0.69 & \bf{635.62} & 
0.78 & 0.78\\SCA3-1 & 700.50 & 0.64 & 
700.76 & 0.76 & \bf{697.84} & 
0.38 & 0.42\\SCA3-2 & 666.19 & 0.88 & 
673.26 & 0.71 & \bf{659.34} & 
1.04 & 2.11\\SCA3-3 & 681.74 & 0.64 & 
682.75 & 0.73 & \bf{680.04} & 
0.25 & 0.40\\SCA3-4 & \bf{690.50} & 0.62 & 
692.22 & 0.65 & 690.50 & 0.00
 & 0.25\\SCA3-5 & 670.02 & 0.81 & 
672.60 & 0.81 & \bf{659.90} & 
1.53 & 1.92\\SCA3-6 & 652.94 & 0.80 & 
655.47 & 0.70 & \bf{651.09} & 
0.28 & 0.67\\SCA3-7 & 666.15 & 0.57 & 
666.15 & 0.66 & \bf{659.17} & 
1.06 & 1.06\\SCA3-8 & 724.29 & 0.48 & 
724.29 & 0.65 & \bf{719.47} & 
0.67 & 0.67\\SCA3-9 & \bf{681.00} & 0.68 & 
681.00 & 0.69 & 681.00 & 0.00
 & 0.00\\
SCA8-0 & 982.58 & 1.03 & 
982.58 & 0.78 & \bf{961.50} & 
2.19 & 2.19\\SCA8-1 & 1074.33 & 0.86 & 
1074.33 & 0.75 & \bf{1049.65} & 
2.35 & 2.35\\SCA8-2 & 1050.37 & 0.94 & 
1050.37 & 0.81 & \bf{1039.64} & 
1.03 & 1.03\\SCA8-3 & 1023.22 & 0.50 & 
1023.41 & 0.75 & \bf{983.34} & 
4.06 & 4.07\\SCA8-4 & 1079.09 & 0.54 & 
1084.84 & 0.67 & \bf{1065.49} & 
1.28 & 1.82\\SCA8-5 & 1062.75 & 0.50 & 
1062.75 & 0.75 & \bf{1027.08} & 
3.47 & 3.47\\SCA8-6 & 993.33 & 0.52 & 
993.33 & 0.61 & \bf{971.82} & 
2.21 & 2.21\\SCA8-7 & 1067.49 & 0.67 & 
1067.49 & 0.84 & \bf{1051.28} & 
1.54 & 1.54\\SCA8-8 & 1088.65 & 0.89 & 
1089.12 & 0.77 & \bf{1071.18} & 
1.63 & 1.67\\SCA8-9 & 1074.88 & 0.54 & 
1077.07 & 0.56 & \bf{1060.50} & 
1.36 & 1.56\\CON3-0 & 623.60 & 0.93 & 
624.93 & 0.77 & \bf{616.52} & 
1.15 & 1.36\\CON3-1 & 560.75 & 0.81 & 
562.54 & 0.82 & \bf{554.47} & 
1.13 & 1.46\\CON3-2 & 521.38 & 0.68 & 
524.01 & 0.70 & \bf{518.00} & 
0.65 & 1.16\\CON3-3 & 594.31 & 0.73 & 
595.55 & 0.64 & \bf{591.19} & 
0.53 & 0.74\\CON3-4 & 592.58 & 0.96 & 
593.75 & 0.85 & \bf{588.79} & 
0.64 & 0.84\\CON3-5 & 564.89 & 0.66 & 
565.58 & 0.74 & \bf{563.70} & 
0.21 & 0.33\\CON3-6 & 508.11 & 0.56 & 
509.21 & 0.71 & \bf{499.05} & 
1.82 & 2.04\\CON3-7 & 577.54 & 0.68 & 
584.31 & 0.76 & \bf{576.48} & 
0.18 & 1.36\\CON3-8 & 524.59 & 0.82 & 
527.01 & 0.84 & \bf{523.05} & 
0.29 & 0.76\\CON3-9 & 588.11 & 0.71 & 
589.92 & 0.74 & \bf{578.24} & 
1.71 & 2.02\\CON8-0 & 878.26 & 0.95 & 
880.78 & 0.83 & \bf{857.17} & 
2.46 & 2.75\\CON8-1 & 776.99 & 0.60 & 
779.25 & 0.84 & \bf{740.85} & 
4.88 & 5.18\\CON8-2 & 722.13 & 0.74 & 
724.48 & 0.74 & \bf{712.89} & 
1.30 & 1.63\\CON8-3 & 833.56 & 0.93 & 
835.82 & 0.83 & \bf{811.07} & 
2.77 & 3.05\\CON8-4 & 784.81 & 0.44 & 
784.81 & 0.72 & \bf{772.25} & 
1.63 & 1.63\\CON8-5 & 758.12 & 0.58 & 
763.79 & 0.60 & \bf{754.88} & 
0.43 & 1.18\\CON8-6 & 682.81 & 0.55 & 
682.81 & 1.01 & \bf{678.92} & 
0.57 & 0.57\\CON8-7 & 816.00 & 0.86 & 
816.00 & 0.77 & \bf{811.96} & 
0.50 & 0.50\\CON8-8 & 795.49 & 0.78 & 
797.29 & 0.70 & \bf{767.53} & 
3.64 & 3.88\\CON8-9 & 818.87 & 0.68 & 
818.87 & 0.79 & \bf{809.00} & 
1.22 & 1.22\\\bf{PROM.} & 
\bf{769.84} & \bf{0.71} & \bf{771.38} & \bf{0.74} & \bf{758.54} & \bf{1.37} & \bf{1.60}\\[1ex]\hline
\end{tabular}
\label{table:nonlin}
\end{table} \clearpage
\begin{table}[ht]
\caption{Resultados de la ejecución de la metaheurística IGA, utilizando instancias de SalhiNagy con la configuración -n 200 -p 40 -cprob 50.0 -mprob 100.0}
\centering
\small
\begin{tabular}{c c c c c c c c}
\hline\hline
Instancia & Costo mínimo & Tiempo(seg.) & Costo promedio & Tiempo promedio(seg.) & CME & \%G & \%GP \\ [0.5ex]
\hline
CMT1X & 480.38 & 0.64 & 
485.00 & 0.60 & \bf{470.48} & 
2.10 & 3.09\\CMT1Y & 481.19 & 0.48 & 
486.05 & 0.57 & \bf{470.48} & 
2.28 & 3.31\\CMT2X & 720.26 & 1.14 & 
723.04 & 1.35 & \bf{682.39} & 
5.55 & 5.96\\CMT2Y & 718.64 & 0.98 & 
721.28 & 1.34 & \bf{682.39} & 
5.31 & 5.70\\CMT3X & 738.25 & 2.96 & 
743.35 & 3.27 & \bf{719.06} & 
2.67 & 3.38\\CMT3Y & 726.20 & 2.89 & 
745.14 & 2.79 & \bf{719.06} & 
0.99 & 3.63\\CMT4X & 911.41 & 7.18 & 
918.36 & 8.09 & \bf{854.21} & 
6.70 & 7.51\\CMT4Y & 909.55 & 7.54 & 
919.18 & 7.94 & \bf{852.46} & 
6.70 & 7.83\\CMT5X & 1108.27 & 17.16 & 
1117.72 & 16.56 & \bf{1030.56} & 
7.54 & 8.46\\CMT5Y & 1124.46 & 17.86 & 
1138.19 & 17.61 & \bf{1031.69} & 
8.99 & 10.32\\CMT11X & 891.63 & 4.57 & 
912.39 & 5.17 & \bf{831.09} & 
7.28 & 9.78\\CMT11Y & 879.64 & 5.43 & 
897.82 & 5.81 & \bf{829.85} & 
6.00 & 8.19\\CMT12X & 676.53 & 3.04 & 
680.32 & 3.12 & \bf{658.83} & 
2.69 & 3.26\\CMT12Y & 684.77 & 3.35 & 
685.45 & 3.32 & \bf{660.47} & 
3.68 & 3.78\\\bf{PROM.} & 
\bf{789.37} & \bf{5.37} & \bf{798.09} & \bf{5.54} & \bf{749.50} & \bf{4.89} & \bf{6.01}\\[1ex]\hline
\end{tabular}
\label{table:nonlin}
\end{table} \clearpage
\begin{table}[ht]
\caption{Resultados de la ejecución de la metaheurística IGA, utilizando instancias de Dethloff con la configuración -n 200 -p 40 -cprob 60.0 -mprob 10.0}
\centering
\small
\begin{tabular}{c c c c c c c c}
\hline\hline
Instancia & Costo mínimo & Tiempo(seg.) & Costo promedio & Tiempo promedio(seg.) & CME & \%G & \%GP \\ [0.5ex]
\hline
SCA3-0 & 640.55 & 0.85 & 
640.55 & 0.68 & \bf{635.62} & 
0.78 & 0.78\\SCA3-1 & \bf{697.84} & 0.74 & 
697.84 & 0.73 & 697.84 & 0.00
 & 0.00\\
SCA3-2 & 661.13 & 0.70 & 
664.81 & 0.72 & \bf{659.34} & 
0.27 & 0.83\\SCA3-3 & 681.35 & 0.78 & 
684.71 & 0.68 & \bf{680.04} & 
0.19 & 0.69\\SCA3-4 & \bf{690.50} & 0.81 & 
691.18 & 0.68 & 690.50 & 0.00
 & 0.10\\SCA3-5 & 668.48 & 0.57 & 
681.19 & 0.58 & \bf{659.90} & 
1.30 & 3.23\\SCA3-6 & 652.94 & 0.72 & 
652.94 & 0.73 & \bf{651.09} & 
0.28 & 0.28\\SCA3-7 & 666.15 & 0.89 & 
666.15 & 0.65 & \bf{659.17} & 
1.06 & 1.06\\SCA3-8 & 724.28 & 0.74 & 
727.03 & 0.62 & \bf{719.47} & 
0.67 & 1.05\\SCA3-9 & \bf{681.00} & 0.42 & 
683.07 & 0.65 & 681.00 & 0.00
 & 0.30\\SCA8-0 & 986.02 & 0.72 & 
1007.79 & 0.74 & \bf{961.50} & 
2.55 & 4.81\\SCA8-1 & 1079.16 & 0.72 & 
1079.16 & 0.64 & \bf{1049.65} & 
2.81 & 2.81\\SCA8-2 & 1063.71 & 0.73 & 
1063.71 & 0.73 & \bf{1039.64} & 
2.32 & 2.32\\SCA8-3 & 1016.70 & 0.54 & 
1023.98 & 0.71 & \bf{983.34} & 
3.39 & 4.13\\SCA8-4 & 1078.42 & 0.58 & 
1078.58 & 0.57 & \bf{1065.49} & 
1.21 & 1.23\\SCA8-5 & 1055.09 & 0.54 & 
1055.09 & 0.69 & \bf{1027.08} & 
2.73 & 2.73\\SCA8-6 & 976.69 & 0.58 & 
985.55 & 0.67 & \bf{971.82} & 
0.50 & 1.41\\SCA8-7 & 1075.06 & 0.69 & 
1075.06 & 0.71 & \bf{1051.28} & 
2.26 & 2.26\\SCA8-8 & 1095.67 & 0.68 & 
1096.41 & 0.65 & \bf{1071.18} & 
2.29 & 2.36\\SCA8-9 & 1074.19 & 0.44 & 
1074.19 & 0.63 & \bf{1060.50} & 
1.29 & 1.29\\CON3-0 & 624.84 & 0.54 & 
624.84 & 0.59 & \bf{616.52} & 
1.35 & 1.35\\CON3-1 & 561.87 & 0.77 & 
561.87 & 0.62 & \bf{554.47} & 
1.33 & 1.33\\CON3-2 & 521.38 & 0.59 & 
522.91 & 0.72 & \bf{518.00} & 
0.65 & 0.95\\CON3-3 & 599.26 & 0.49 & 
605.38 & 0.48 & \bf{591.19} & 
1.37 & 2.40\\CON3-4 & 598.56 & 0.49 & 
600.90 & 0.65 & \bf{588.79} & 
1.66 & 2.06\\CON3-5 & \bf{563.70} & 0.77 & 
566.23 & 0.55 & 563.70 & 0.00
 & 0.45\\CON3-6 & 504.91 & 0.78 & 
508.80 & 0.69 & \bf{499.05} & 
1.17 & 1.95\\CON3-7 & 577.91 & 0.49 & 
579.99 & 0.50 & \bf{576.48} & 
0.25 & 0.61\\CON3-8 & 528.69 & 0.74 & 
530.38 & 0.71 & \bf{523.05} & 
1.08 & 1.40\\CON3-9 & 588.11 & 0.76 & 
589.82 & 0.65 & \bf{578.24} & 
1.71 & 2.00\\CON8-0 & 857.40 & 0.75 & 
867.88 & 0.71 & \bf{857.17} & 
0.03 & 1.25\\CON8-1 & 768.94 & 1.04 & 
775.42 & 0.67 & \bf{740.85} & 
3.79 & 4.67\\CON8-2 & 725.39 & 0.70 & 
725.39 & 0.69 & \bf{712.89} & 
1.75 & 1.75\\CON8-3 & 837.87 & 0.80 & 
837.87 & 0.81 & \bf{811.07} & 
3.30 & 3.30\\CON8-4 & 801.32 & 0.67 & 
802.23 & 0.67 & \bf{772.25} & 
3.76 & 3.88\\CON8-5 & 766.08 & 0.78 & 
770.89 & 0.60 & \bf{754.88} & 
1.48 & 2.12\\CON8-6 & 695.54 & 0.70 & 
696.59 & 0.72 & \bf{678.92} & 
2.45 & 2.60\\CON8-7 & 814.86 & 0.69 & 
816.59 & 0.58 & \bf{811.96} & 
0.36 & 0.57\\CON8-8 & 796.35 & 0.58 & 
796.82 & 0.73 & \bf{767.53} & 
3.75 & 3.82\\CON8-9 & 834.20 & 0.64 & 
839.92 & 0.70 & \bf{809.00} & 
3.11 & 3.82\\\bf{PROM.} & 
\bf{770.80} & \bf{0.68} & \bf{773.74} & \bf{0.66} & \bf{758.54} & \bf{1.51} & \bf{1.90}\\[1ex]\hline
\end{tabular}
\label{table:nonlin}
\end{table} \clearpage
\begin{table}[ht]
\caption{Resultados de la ejecución de la metaheurística IGA, utilizando instancias de SalhiNagy con la configuración -n 200 -p 40 -cprob 60.0 -mprob 10.0}
\centering
\small
\begin{tabular}{c c c c c c c c}
\hline\hline
Instancia & Costo mínimo & Tiempo(seg.) & Costo promedio & Tiempo promedio(seg.) & CME & \%G & \%GP \\ [0.5ex]
\hline
CMT1X & 483.46 & 0.49 & 
483.46 & 0.77 & \bf{470.48} & 
2.76 & 2.76\\CMT1Y & 492.49 & 0.60 & 
492.89 & 0.60 & \bf{470.48} & 
4.68 & 4.76\\CMT2X & 689.65 & 1.14 & 
709.50 & 1.22 & \bf{682.39} & 
1.06 & 3.97\\CMT2Y & 707.31 & 1.75 & 
718.04 & 1.34 & \bf{682.39} & 
3.65 & 5.22\\CMT3X & 736.26 & 4.38 & 
741.49 & 3.18 & \bf{719.06} & 
2.39 & 3.12\\CMT3Y & 739.65 & 2.61 & 
745.21 & 2.60 & \bf{719.06} & 
2.86 & 3.64\\CMT4X & 895.94 & 7.99 & 
906.03 & 7.86 & \bf{854.21} & 
4.89 & 6.07\\CMT4Y & 895.08 & 8.03 & 
915.12 & 8.05 & \bf{852.46} & 
5.00 & 7.35\\CMT5X & 1117.45 & 16.09 & 
1128.83 & 15.82 & \bf{1030.56} & 
8.43 & 9.54\\CMT5Y & 1105.33 & 16.74 & 
1117.96 & 16.47 & \bf{1031.69} & 
7.14 & 8.36\\CMT11X & 905.90 & 4.75 & 
916.70 & 4.66 & \bf{831.09} & 
9.00 & 10.30\\CMT11Y & 912.09 & 5.54 & 
923.27 & 5.50 & \bf{829.85} & 
9.91 & 11.26\\CMT12X & 674.70 & 3.11 & 
679.87 & 2.85 & \bf{658.83} & 
2.41 & 3.19\\CMT12Y & 677.30 & 2.98 & 
678.76 & 2.87 & \bf{660.47} & 
2.55 & 2.77\\\bf{PROM.} & 
\bf{788.04} & \bf{5.44} & \bf{796.94} & \bf{5.27} & \bf{749.50} & \bf{4.77} & \bf{5.88}\\[1ex]\hline
\end{tabular}
\label{table:nonlin}
\end{table} \clearpage
\begin{table}[ht]
\caption{Resultados de la ejecución de la metaheurística IGA, utilizando instancias de Dethloff con la configuración -n 200 -p 40 -cprob 60.0 -mprob 20.0}
\centering
\small
\begin{tabular}{c c c c c c c c}
\hline\hline
Instancia & Costo mínimo & Tiempo(seg.) & Costo promedio & Tiempo promedio(seg.) & CME & \%G & \%GP \\ [0.5ex]
\hline
SCA3-0 & 640.55 & 0.58 & 
640.82 & 0.69 & \bf{635.62} & 
0.78 & 0.82\\SCA3-1 & 700.50 & 0.74 & 
702.61 & 0.69 & \bf{697.84} & 
0.38 & 0.68\\SCA3-2 & 666.85 & 0.70 & 
668.20 & 0.61 & \bf{659.34} & 
1.14 & 1.34\\SCA3-3 & 680.60 & 0.66 & 
681.82 & 0.52 & \bf{680.04} & 
0.08 & 0.26\\SCA3-4 & \bf{690.50} & 0.63 & 
690.50 & 0.68 & 690.50 & 0.00
 & 0.00\\
SCA3-5 & 681.81 & 0.56 & 
681.81 & 0.62 & \bf{659.90} & 
3.32 & 3.32\\SCA3-6 & 652.94 & 0.71 & 
652.94 & 0.60 & \bf{651.09} & 
0.28 & 0.28\\SCA3-7 & 667.24 & 0.49 & 
667.24 & 0.59 & \bf{659.17} & 
1.22 & 1.22\\SCA3-8 & 722.05 & 0.56 & 
723.67 & 0.60 & \bf{719.47} & 
0.36 & 0.58\\SCA3-9 & \bf{681.00} & 0.83 & 
681.00 & 0.67 & 681.00 & 0.00
 & 0.00\\
SCA8-0 & 991.27 & 0.52 & 
1001.57 & 0.53 & \bf{961.50} & 
3.10 & 4.17\\SCA8-1 & 1076.92 & 0.75 & 
1087.86 & 0.76 & \bf{1049.65} & 
2.60 & 3.64\\SCA8-2 & 1052.94 & 0.55 & 
1053.24 & 0.61 & \bf{1039.64} & 
1.28 & 1.31\\SCA8-3 & 1014.10 & 0.52 & 
1014.10 & 0.55 & \bf{983.34} & 
3.13 & 3.13\\SCA8-4 & 1077.80 & 0.46 & 
1081.70 & 0.56 & \bf{1065.49} & 
1.16 & 1.52\\SCA8-5 & 1029.95 & 0.56 & 
1038.68 & 0.58 & \bf{1027.08} & 
0.28 & 1.13\\SCA8-6 & 972.48 & 0.76 & 
972.48 & 0.71 & \bf{971.82} & 
0.07 & 0.07\\SCA8-7 & 1077.35 & 0.49 & 
1077.35 & 0.65 & \bf{1051.28} & 
2.48 & 2.48\\SCA8-8 & 1096.51 & 0.44 & 
1099.72 & 0.49 & \bf{1071.18} & 
2.36 & 2.66\\SCA8-9 & 1088.88 & 0.94 & 
1092.94 & 0.79 & \bf{1060.50} & 
2.68 & 3.06\\CON3-0 & 633.24 & 0.58 & 
634.16 & 0.61 & \bf{616.52} & 
2.71 & 2.86\\CON3-1 & 560.75 & 0.94 & 
562.67 & 0.74 & \bf{554.47} & 
1.13 & 1.48\\CON3-2 & 521.38 & 0.63 & 
521.38 & 0.67 & \bf{518.00} & 
0.65 & 0.65\\CON3-3 & 600.02 & 0.81 & 
606.24 & 0.84 & \bf{591.19} & 
1.49 & 2.55\\CON3-4 & 598.56 & 0.72 & 
602.44 & 0.73 & \bf{588.79} & 
1.66 & 2.32\\CON3-5 & \bf{563.70} & 0.78 & 
565.24 & 0.73 & 563.70 & 0.00
 & 0.27\\CON3-6 & 502.16 & 0.94 & 
506.87 & 0.75 & \bf{499.05} & 
0.62 & 1.57\\CON3-7 & 586.01 & 0.50 & 
586.63 & 0.67 & \bf{576.48} & 
1.65 & 1.76\\CON3-8 & 524.59 & 0.72 & 
527.16 & 0.68 & \bf{523.05} & 
0.29 & 0.79\\CON3-9 & 588.11 & 0.74 & 
589.62 & 0.73 & \bf{578.24} & 
1.71 & 1.97\\CON8-0 & 857.40 & 0.76 & 
857.40 & 0.70 & \bf{857.17} & 
0.03 & 0.03\\CON8-1 & 755.32 & 0.52 & 
759.78 & 0.55 & \bf{740.85} & 
1.95 & 2.56\\CON8-2 & 725.24 & 0.62 & 
726.22 & 0.70 & \bf{712.89} & 
1.73 & 1.87\\CON8-3 & 827.30 & 0.76 & 
829.92 & 0.74 & \bf{811.07} & 
2.00 & 2.32\\CON8-4 & 776.23 & 0.74 & 
786.02 & 0.65 & \bf{772.25} & 
0.52 & 1.78\\CON8-5 & 760.78 & 0.60 & 
766.74 & 0.64 & \bf{754.88} & 
0.78 & 1.57\\CON8-6 & 699.01 & 0.59 & 
700.12 & 0.74 & \bf{678.92} & 
2.96 & 3.12\\CON8-7 & 815.43 & 0.72 & 
815.43 & 0.72 & \bf{811.96} & 
0.43 & 0.43\\CON8-8 & 784.71 & 0.60 & 
790.24 & 0.68 & \bf{767.53} & 
2.24 & 2.96\\CON8-9 & 833.40 & 1.34 & 
834.54 & 0.90 & \bf{809.00} & 
3.02 & 3.16\\\bf{PROM.} & 
\bf{769.39} & \bf{0.68} & \bf{771.98} & \bf{0.67} & \bf{758.54} & \bf{1.36} & \bf{1.69}\\[1ex]\hline
\end{tabular}
\label{table:nonlin}
\end{table} \clearpage
\begin{table}[ht]
\caption{Resultados de la ejecución de la metaheurística IGA, utilizando instancias de SalhiNagy con la configuración -n 200 -p 40 -cprob 60.0 -mprob 20.0}
\centering
\small
\begin{tabular}{c c c c c c c c}
\hline\hline
Instancia & Costo mínimo & Tiempo(seg.) & Costo promedio & Tiempo promedio(seg.) & CME & \%G & \%GP \\ [0.5ex]
\hline
CMT1X & 477.38 & 0.39 & 
481.54 & 0.40 & \bf{470.48} & 
1.47 & 2.35\\CMT1Y & 478.97 & 0.38 & 
478.97 & 0.39 & \bf{470.48} & 
1.80 & 1.80\\CMT2X & 709.03 & 1.12 & 
717.74 & 1.14 & \bf{682.39} & 
3.90 & 5.18\\CMT2Y & 716.73 & 1.22 & 
719.80 & 1.29 & \bf{682.39} & 
5.03 & 5.48\\CMT3X & 738.94 & 3.05 & 
750.75 & 2.92 & \bf{719.06} & 
2.76 & 4.41\\CMT3Y & 747.18 & 2.58 & 
751.52 & 2.80 & \bf{719.06} & 
3.91 & 4.51\\CMT4X & 905.12 & 8.24 & 
912.10 & 8.41 & \bf{854.21} & 
5.96 & 6.78\\CMT4Y & 916.21 & 8.14 & 
922.57 & 7.82 & \bf{852.46} & 
7.48 & 8.22\\CMT5X & 1117.82 & 15.52 & 
1120.54 & 15.92 & \bf{1030.56} & 
8.47 & 8.73\\CMT5Y & 1106.77 & 16.84 & 
1120.42 & 16.95 & \bf{1031.69} & 
7.28 & 8.60\\CMT11X & 896.74 & 4.56 & 
904.75 & 4.91 & \bf{831.09} & 
7.90 & 8.86\\CMT11Y & 888.36 & 5.28 & 
934.49 & 5.54 & \bf{829.85} & 
7.05 & 12.61\\CMT12X & 675.25 & 3.02 & 
679.64 & 3.06 & \bf{658.83} & 
2.49 & 3.16\\CMT12Y & 675.18 & 2.59 & 
682.26 & 2.78 & \bf{660.47} & 
2.23 & 3.30\\\bf{PROM.} & 
\bf{789.26} & \bf{5.21} & \bf{798.36} & \bf{5.31} & \bf{749.50} & \bf{4.84} & \bf{6.00}\\[1ex]\hline
\end{tabular}
\label{table:nonlin}
\end{table} \clearpage
\begin{table}[ht]
\caption{Resultados de la ejecución de la metaheurística IGA, utilizando instancias de Dethloff con la configuración -n 200 -p 40 -cprob 60.0 -mprob 30.0}
\centering
\small
\begin{tabular}{c c c c c c c c}
\hline\hline
Instancia & Costo mínimo & Tiempo(seg.) & Costo promedio & Tiempo promedio(seg.) & CME & \%G & \%GP \\ [0.5ex]
\hline
SCA3-0 & 640.55 & 0.76 & 
641.81 & 0.69 & \bf{635.62} & 
0.78 & 0.97\\SCA3-1 & 700.50 & 0.70 & 
701.27 & 0.61 & \bf{697.84} & 
0.38 & 0.49\\SCA3-2 & 669.06 & 0.89 & 
669.06 & 0.75 & \bf{659.34} & 
1.47 & 1.47\\SCA3-3 & \bf{680.04} & 0.72 & 
681.29 & 0.60 & 680.04 & 0.00
 & 0.18\\SCA3-4 & \bf{690.50} & 0.91 & 
690.50 & 0.80 & 690.50 & 0.00
 & 0.00\\
SCA3-5 & 680.80 & 0.56 & 
681.92 & 0.66 & \bf{659.90} & 
3.17 & 3.34\\SCA3-6 & 652.94 & 0.70 & 
655.71 & 0.71 & \bf{651.09} & 
0.28 & 0.71\\SCA3-7 & 669.89 & 0.86 & 
669.89 & 0.65 & \bf{659.17} & 
1.63 & 1.63\\SCA3-8 & 723.99 & 0.75 & 
727.47 & 0.73 & \bf{719.47} & 
0.63 & 1.11\\SCA3-9 & \bf{681.00} & 0.49 & 
681.00 & 0.54 & 681.00 & 0.00
 & 0.00\\
SCA8-0 & 1020.21 & 0.73 & 
1020.21 & 0.68 & \bf{961.50} & 
6.11 & 6.11\\SCA8-1 & 1078.94 & 0.72 & 
1079.02 & 0.75 & \bf{1049.65} & 
2.79 & 2.80\\SCA8-2 & 1051.55 & 0.73 & 
1052.45 & 0.64 & \bf{1039.64} & 
1.15 & 1.23\\SCA8-3 & 1028.99 & 0.72 & 
1031.87 & 0.60 & \bf{983.34} & 
4.64 & 4.94\\SCA8-4 & 1084.88 & 0.52 & 
1088.33 & 0.59 & \bf{1065.49} & 
1.82 & 2.14\\SCA8-5 & 1045.30 & 0.95 & 
1050.06 & 0.73 & \bf{1027.08} & 
1.77 & 2.24\\SCA8-6 & 972.48 & 0.47 & 
972.48 & 0.67 & \bf{971.82} & 
0.07 & 0.07\\SCA8-7 & 1063.60 & 0.61 & 
1063.60 & 0.63 & \bf{1051.28} & 
1.17 & 1.17\\SCA8-8 & 1088.65 & 0.72 & 
1088.65 & 0.71 & \bf{1071.18} & 
1.63 & 1.63\\SCA8-9 & 1087.19 & 0.54 & 
1088.71 & 0.58 & \bf{1060.50} & 
2.52 & 2.66\\CON3-0 & 620.76 & 0.59 & 
627.34 & 0.68 & \bf{616.52} & 
0.69 & 1.76\\CON3-1 & 556.04 & 0.63 & 
559.64 & 0.64 & \bf{554.47} & 
0.28 & 0.93\\CON3-2 & 521.38 & 0.97 & 
521.38 & 0.79 & \bf{518.00} & 
0.65 & 0.65\\CON3-3 & 594.31 & 0.71 & 
600.95 & 0.56 & \bf{591.19} & 
0.53 & 1.65\\CON3-4 & 598.21 & 0.48 & 
599.95 & 0.57 & \bf{588.79} & 
1.60 & 1.89\\CON3-5 & 566.96 & 0.71 & 
569.82 & 0.69 & \bf{563.70} & 
0.58 & 1.09\\CON3-6 & 505.97 & 0.74 & 
507.59 & 0.78 & \bf{499.05} & 
1.39 & 1.71\\CON3-7 & 577.91 & 0.56 & 
578.07 & 0.71 & \bf{576.48} & 
0.25 & 0.27\\CON3-8 & 524.59 & 0.49 & 
532.35 & 0.70 & \bf{523.05} & 
0.29 & 1.78\\CON3-9 & 590.16 & 0.75 & 
590.27 & 0.67 & \bf{578.24} & 
2.06 & 2.08\\CON8-0 & 878.19 & 0.84 & 
878.69 & 0.80 & \bf{857.17} & 
2.45 & 2.51\\CON8-1 & 748.32 & 0.82 & 
748.32 & 0.84 & \bf{740.85} & 
1.01 & 1.01\\CON8-2 & 725.39 & 0.66 & 
725.39 & 0.80 & \bf{712.89} & 
1.75 & 1.75\\CON8-3 & 842.70 & 0.74 & 
846.71 & 0.74 & \bf{811.07} & 
3.90 & 4.39\\CON8-4 & 776.34 & 0.72 & 
777.92 & 0.64 & \bf{772.25} & 
0.53 & 0.73\\CON8-5 & 762.01 & 0.95 & 
770.11 & 0.71 & \bf{754.88} & 
0.94 & 2.02\\CON8-6 & 700.74 & 0.75 & 
702.69 & 0.80 & \bf{678.92} & 
3.21 & 3.50\\CON8-7 & 815.79 & 0.48 & 
817.84 & 0.65 & \bf{811.96} & 
0.47 & 0.72\\CON8-8 & 786.16 & 0.96 & 
786.16 & 0.78 & \bf{767.53} & 
2.43 & 2.43\\CON8-9 & 832.51 & 0.72 & 
833.34 & 0.68 & \bf{809.00} & 
2.91 & 3.01\\\bf{PROM.} & 
\bf{770.89} & \bf{0.71} & \bf{772.75} & \bf{0.69} & \bf{758.54} & \bf{1.50} & \bf{1.77}\\[1ex]\hline
\end{tabular}
\label{table:nonlin}
\end{table} \clearpage
\begin{table}[ht]
\caption{Resultados de la ejecución de la metaheurística IGA, utilizando instancias de SalhiNagy con la configuración -n 200 -p 40 -cprob 60.0 -mprob 30.0}
\centering
\small
\begin{tabular}{c c c c c c c c}
\hline\hline
Instancia & Costo mínimo & Tiempo(seg.) & Costo promedio & Tiempo promedio(seg.) & CME & \%G & \%GP \\ [0.5ex]
\hline
CMT1X & 478.42 & 0.65 & 
484.26 & 0.60 & \bf{470.48} & 
1.69 & 2.93\\CMT1Y & 480.96 & 0.43 & 
481.68 & 0.59 & \bf{470.48} & 
2.23 & 2.38\\CMT2X & 713.41 & 1.37 & 
714.98 & 1.50 & \bf{682.39} & 
4.55 & 4.78\\CMT2Y & 708.33 & 1.36 & 
713.98 & 1.26 & \bf{682.39} & 
3.80 & 4.63\\CMT3X & 740.74 & 3.09 & 
748.82 & 2.98 & \bf{719.06} & 
3.02 & 4.14\\CMT3Y & 736.77 & 3.12 & 
744.12 & 3.08 & \bf{719.06} & 
2.46 & 3.49\\CMT4X & 901.07 & 7.93 & 
918.04 & 7.67 & \bf{854.21} & 
5.49 & 7.47\\CMT4Y & 887.40 & 8.22 & 
911.54 & 8.16 & \bf{852.46} & 
4.10 & 6.93\\CMT5X & 1106.38 & 15.77 & 
1127.47 & 16.09 & \bf{1030.56} & 
7.36 & 9.40\\CMT5Y & 1096.97 & 17.73 & 
1110.94 & 16.87 & \bf{1031.69} & 
6.33 & 7.68\\CMT11X & 907.59 & 4.72 & 
917.60 & 4.89 & \bf{831.09} & 
9.20 & 10.41\\CMT11Y & 886.53 & 5.35 & 
899.09 & 6.08 & \bf{829.85} & 
6.83 & 8.34\\CMT12X & 674.79 & 2.92 & 
684.32 & 3.17 & \bf{658.83} & 
2.42 & 3.87\\CMT12Y & 676.01 & 2.84 & 
680.65 & 2.85 & \bf{660.47} & 
2.35 & 3.06\\\bf{PROM.} & 
\bf{785.38} & \bf{5.39} & \bf{795.54} & \bf{5.41} & \bf{749.50} & \bf{4.42} & \bf{5.68}\\[1ex]\hline
\end{tabular}
\label{table:nonlin}
\end{table} \clearpage
\begin{table}[ht]
\caption{Resultados de la ejecución de la metaheurística IGA, utilizando instancias de Dethloff con la configuración -n 200 -p 40 -cprob 60.0 -mprob 40.0}
\centering
\small
\begin{tabular}{c c c c c c c c}
\hline\hline
Instancia & Costo mínimo & Tiempo(seg.) & Costo promedio & Tiempo promedio(seg.) & CME & \%G & \%GP \\ [0.5ex]
\hline
SCA3-0 & 640.55 & 0.59 & 
641.11 & 0.57 & \bf{635.62} & 
0.78 & 0.86\\SCA3-1 & \bf{697.84} & 0.98 & 
703.79 & 0.77 & 697.84 & 0.00
 & 0.85\\SCA3-2 & 666.01 & 0.69 & 
666.01 & 0.71 & \bf{659.34} & 
1.01 & 1.01\\SCA3-3 & 682.46 & 0.73 & 
682.46 & 0.67 & \bf{680.04} & 
0.36 & 0.36\\SCA3-4 & \bf{690.50} & 0.66 & 
691.70 & 0.62 & 690.50 & 0.00
 & 0.17\\SCA3-5 & 673.56 & 0.46 & 
673.56 & 0.64 & \bf{659.90} & 
2.07 & 2.07\\SCA3-6 & 652.94 & 0.80 & 
657.11 & 0.71 & \bf{651.09} & 
0.28 & 0.92\\SCA3-7 & 666.15 & 0.91 & 
666.15 & 0.70 & \bf{659.17} & 
1.06 & 1.06\\SCA3-8 & 727.49 & 0.49 & 
732.46 & 0.65 & \bf{719.47} & 
1.11 & 1.80\\SCA3-9 & \bf{681.00} & 0.70 & 
687.06 & 0.76 & 681.00 & 0.00
 & 0.89\\SCA8-0 & 1006.83 & 0.55 & 
1017.91 & 0.68 & \bf{961.50} & 
4.71 & 5.87\\SCA8-1 & 1073.03 & 0.70 & 
1073.03 & 0.77 & \bf{1049.65} & 
2.23 & 2.23\\SCA8-2 & 1052.94 & 0.74 & 
1053.44 & 0.75 & \bf{1039.64} & 
1.28 & 1.33\\SCA8-3 & 999.47 & 1.02 & 
999.47 & 0.71 & \bf{983.34} & 
1.64 & 1.64\\SCA8-4 & 1086.45 & 0.47 & 
1093.25 & 0.57 & \bf{1065.49} & 
1.97 & 2.61\\SCA8-5 & 1052.41 & 0.71 & 
1054.14 & 0.71 & \bf{1027.08} & 
2.47 & 2.63\\SCA8-6 & 993.81 & 0.62 & 
998.25 & 0.58 & \bf{971.82} & 
2.26 & 2.72\\SCA8-7 & 1085.45 & 0.92 & 
1085.45 & 0.77 & \bf{1051.28} & 
3.25 & 3.25\\SCA8-8 & 1092.12 & 0.72 & 
1101.22 & 0.76 & \bf{1071.18} & 
1.95 & 2.80\\SCA8-9 & 1081.23 & 0.75 & 
1081.23 & 0.64 & \bf{1060.50} & 
1.95 & 1.95\\CON3-0 & 619.09 & 0.58 & 
628.72 & 0.63 & \bf{616.52} & 
0.42 & 1.98\\CON3-1 & 560.61 & 0.82 & 
562.57 & 0.61 & \bf{554.47} & 
1.11 & 1.46\\CON3-2 & 521.38 & 1.01 & 
522.63 & 0.85 & \bf{518.00} & 
0.65 & 0.89\\CON3-3 & 591.20 & 0.50 & 
592.57 & 0.52 & \bf{591.19} & 
0.00 & 0.23\\CON3-4 & 591.43 & 0.80 & 
598.85 & 0.70 & \bf{588.79} & 
0.45 & 1.71\\CON3-5 & 567.94 & 0.55 & 
569.71 & 0.70 & \bf{563.70} & 
0.75 & 1.07\\CON3-6 & 504.44 & 0.59 & 
504.44 & 0.77 & \bf{499.05} & 
1.08 & 1.08\\CON3-7 & 578.41 & 0.53 & 
578.41 & 0.57 & \bf{576.48} & 
0.33 & 0.33\\CON3-8 & 524.59 & 0.60 & 
524.59 & 0.59 & \bf{523.05} & 
0.29 & 0.29\\CON3-9 & 588.48 & 0.75 & 
590.12 & 0.84 & \bf{578.24} & 
1.77 & 2.05\\CON8-0 & 875.89 & 0.76 & 
876.03 & 0.67 & \bf{857.17} & 
2.18 & 2.20\\CON8-1 & 757.86 & 0.58 & 
764.41 & 0.62 & \bf{740.85} & 
2.30 & 3.18\\CON8-2 & 726.58 & 0.80 & 
728.42 & 0.73 & \bf{712.89} & 
1.92 & 2.18\\CON8-3 & 845.45 & 0.85 & 
850.49 & 0.76 & \bf{811.07} & 
4.24 & 4.86\\CON8-4 & 802.58 & 0.90 & 
802.58 & 0.72 & \bf{772.25} & 
3.93 & 3.93\\CON8-5 & 769.16 & 0.74 & 
769.16 & 0.65 & \bf{754.88} & 
1.89 & 1.89\\CON8-6 & 692.41 & 0.48 & 
693.83 & 0.66 & \bf{678.92} & 
1.99 & 2.20\\CON8-7 & 816.67 & 0.65 & 
823.07 & 0.64 & \bf{811.96} & 
0.58 & 1.37\\CON8-8 & 784.32 & 0.58 & 
784.32 & 0.81 & \bf{767.53} & 
2.19 & 2.19\\CON8-9 & 832.05 & 0.78 & 
834.36 & 0.77 & \bf{809.00} & 
2.85 & 3.13\\\bf{PROM.} & 
\bf{771.32} & \bf{0.70} & \bf{773.95} & \bf{0.69} & \bf{758.54} & \bf{1.53} & \bf{1.88}\\[1ex]\hline
\end{tabular}
\label{table:nonlin}
\end{table} \clearpage
\begin{table}[ht]
\caption{Resultados de la ejecución de la metaheurística IGA, utilizando instancias de SalhiNagy con la configuración -n 200 -p 40 -cprob 60.0 -mprob 40.0}
\centering
\small
\begin{tabular}{c c c c c c c c}
\hline\hline
Instancia & Costo mínimo & Tiempo(seg.) & Costo promedio & Tiempo promedio(seg.) & CME & \%G & \%GP \\ [0.5ex]
\hline
CMT1X & 477.88 & 0.83 & 
483.83 & 0.75 & \bf{470.48} & 
1.57 & 2.84\\CMT1Y & 482.87 & 0.39 & 
489.50 & 0.40 & \bf{470.48} & 
2.63 & 4.04\\CMT2X & 722.32 & 1.72 & 
722.40 & 1.47 & \bf{682.39} & 
5.85 & 5.86\\CMT2Y & 701.38 & 1.09 & 
706.64 & 1.27 & \bf{682.39} & 
2.78 & 3.55\\CMT3X & 731.17 & 2.94 & 
741.91 & 3.15 & \bf{719.06} & 
1.68 & 3.18\\CMT3Y & 736.57 & 3.04 & 
745.39 & 2.90 & \bf{719.06} & 
2.44 & 3.66\\CMT4X & 913.94 & 8.17 & 
918.68 & 8.00 & \bf{854.21} & 
6.99 & 7.55\\CMT4Y & 911.06 & 8.73 & 
918.98 & 8.63 & \bf{852.46} & 
6.87 & 7.80\\CMT5X & 1113.26 & 15.66 & 
1129.63 & 15.96 & \bf{1030.56} & 
8.02 & 9.61\\CMT5Y & 1108.33 & 16.52 & 
1117.99 & 16.47 & \bf{1031.69} & 
7.43 & 8.36\\CMT11X & 883.29 & 5.04 & 
904.62 & 4.96 & \bf{831.09} & 
6.28 & 8.85\\CMT11Y & 902.69 & 5.00 & 
909.37 & 5.44 & \bf{829.85} & 
8.78 & 9.58\\CMT12X & 683.50 & 2.69 & 
686.29 & 3.14 & \bf{658.83} & 
3.74 & 4.17\\CMT12Y & 674.87 & 2.79 & 
677.99 & 3.05 & \bf{660.47} & 
2.18 & 2.65\\\bf{PROM.} & 
\bf{788.80} & \bf{5.33} & \bf{796.66} & \bf{5.40} & \bf{749.50} & \bf{4.80} & \bf{5.84}\\[1ex]\hline
\end{tabular}
\label{table:nonlin}
\end{table} \clearpage
\begin{table}[ht]
\caption{Resultados de la ejecución de la metaheurística IGA, utilizando instancias de Dethloff con la configuración -n 200 -p 40 -cprob 60.0 -mprob 50.0}
\centering
\small
\begin{tabular}{c c c c c c c c}
\hline\hline
Instancia & Costo mínimo & Tiempo(seg.) & Costo promedio & Tiempo promedio(seg.) & CME & \%G & \%GP \\ [0.5ex]
\hline
SCA3-0 & 641.69 & 0.69 & 
641.99 & 0.69 & \bf{635.62} & 
0.95 & 1.00\\SCA3-1 & 701.78 & 0.74 & 
705.66 & 0.67 & \bf{697.84} & 
0.56 & 1.12\\SCA3-2 & 664.18 & 0.64 & 
664.19 & 0.67 & \bf{659.34} & 
0.73 & 0.74\\SCA3-3 & \bf{680.04} & 0.54 & 
681.86 & 0.60 & 680.04 & 0.00
 & 0.27\\SCA3-4 & \bf{690.50} & 0.72 & 
690.50 & 0.64 & 690.50 & 0.00
 & 0.00\\
SCA3-5 & \bf{659.90} & 0.92 & 
672.68 & 0.69 & 659.90 & 0.00
 & 1.94\\SCA3-6 & 652.94 & 0.71 & 
656.38 & 0.71 & \bf{651.09} & 
0.28 & 0.81\\SCA3-7 & 667.75 & 0.69 & 
667.75 & 0.75 & \bf{659.17} & 
1.30 & 1.30\\SCA3-8 & 724.37 & 0.72 & 
725.69 & 0.59 & \bf{719.47} & 
0.68 & 0.86\\SCA3-9 & \bf{681.00} & 0.88 & 
683.79 & 0.74 & 681.00 & 0.00
 & 0.41\\SCA8-0 & 1003.70 & 0.79 & 
1007.71 & 0.68 & \bf{961.50} & 
4.39 & 4.81\\SCA8-1 & 1083.35 & 0.98 & 
1083.35 & 0.77 & \bf{1049.65} & 
3.21 & 3.21\\SCA8-2 & 1053.78 & 0.69 & 
1053.78 & 0.79 & \bf{1039.64} & 
1.36 & 1.36\\SCA8-3 & 1022.37 & 0.69 & 
1025.66 & 0.63 & \bf{983.34} & 
3.97 & 4.30\\SCA8-4 & 1075.27 & 0.74 & 
1082.31 & 0.82 & \bf{1065.49} & 
0.92 & 1.58\\SCA8-5 & 1057.45 & 0.71 & 
1060.23 & 0.59 & \bf{1027.08} & 
2.96 & 3.23\\SCA8-6 & 987.05 & 0.79 & 
990.90 & 0.74 & \bf{971.82} & 
1.57 & 1.96\\SCA8-7 & 1087.60 & 0.74 & 
1087.60 & 0.79 & \bf{1051.28} & 
3.45 & 3.45\\SCA8-8 & 1093.44 & 0.56 & 
1093.44 & 0.71 & \bf{1071.18} & 
2.08 & 2.08\\SCA8-9 & 1078.30 & 0.74 & 
1079.30 & 0.55 & \bf{1060.50} & 
1.68 & 1.77\\CON3-0 & 630.87 & 0.74 & 
632.13 & 0.58 & \bf{616.52} & 
2.33 & 2.53\\CON3-1 & 557.21 & 0.97 & 
558.98 & 0.83 & \bf{554.47} & 
0.49 & 0.81\\CON3-2 & 521.38 & 0.62 & 
521.38 & 0.79 & \bf{518.00} & 
0.65 & 0.65\\CON3-3 & 599.26 & 0.65 & 
600.93 & 0.76 & \bf{591.19} & 
1.37 & 1.65\\CON3-4 & 595.25 & 0.49 & 
597.22 & 0.73 & \bf{588.79} & 
1.10 & 1.43\\CON3-5 & 564.88 & 0.75 & 
566.41 & 0.62 & \bf{563.70} & 
0.21 & 0.48\\CON3-6 & 503.97 & 0.96 & 
506.79 & 0.76 & \bf{499.05} & 
0.99 & 1.55\\CON3-7 & 586.01 & 0.66 & 
588.35 & 0.69 & \bf{576.48} & 
1.65 & 2.06\\CON3-8 & 524.59 & 0.97 & 
524.59 & 0.75 & \bf{523.05} & 
0.29 & 0.29\\CON3-9 & 588.40 & 0.70 & 
589.59 & 0.71 & \bf{578.24} & 
1.76 & 1.96\\CON8-0 & 880.69 & 0.79 & 
880.69 & 0.67 & \bf{857.17} & 
2.74 & 2.74\\CON8-1 & 773.10 & 0.87 & 
773.10 & 0.90 & \bf{740.85} & 
4.35 & 4.35\\CON8-2 & 719.45 & 0.83 & 
722.38 & 0.81 & \bf{712.89} & 
0.92 & 1.33\\CON8-3 & 819.89 & 0.74 & 
831.27 & 0.81 & \bf{811.07} & 
1.09 & 2.49\\CON8-4 & 794.54 & 0.74 & 
794.54 & 0.66 & \bf{772.25} & 
2.89 & 2.89\\CON8-5 & 758.12 & 0.93 & 
758.12 & 0.82 & \bf{754.88} & 
0.43 & 0.43\\CON8-6 & 698.39 & 0.68 & 
703.40 & 0.78 & \bf{678.92} & 
2.87 & 3.61\\CON8-7 & 815.84 & 0.93 & 
815.84 & 0.73 & \bf{811.96} & 
0.48 & 0.48\\CON8-8 & 777.88 & 0.77 & 
785.61 & 0.84 & \bf{767.53} & 
1.35 & 2.35\\CON8-9 & 820.23 & 0.79 & 
820.23 & 0.72 & \bf{809.00} & 
1.39 & 1.39\\\bf{PROM.} & 
\bf{770.91} & \bf{0.76} & \bf{773.16} & \bf{0.72} & \bf{758.54} & \bf{1.49} & \bf{1.79}\\[1ex]\hline
\end{tabular}
\label{table:nonlin}
\end{table} \clearpage
\begin{table}[ht]
\caption{Resultados de la ejecución de la metaheurística IGA, utilizando instancias de SalhiNagy con la configuración -n 200 -p 40 -cprob 60.0 -mprob 50.0}
\centering
\small
\begin{tabular}{c c c c c c c c}
\hline\hline
Instancia & Costo mínimo & Tiempo(seg.) & Costo promedio & Tiempo promedio(seg.) & CME & \%G & \%GP \\ [0.5ex]
\hline
CMT1X & 479.45 & 0.60 & 
481.34 & 0.52 & \bf{470.48} & 
1.91 & 2.31\\CMT1Y & 478.84 & 0.32 & 
483.37 & 0.36 & \bf{470.48} & 
1.78 & 2.74\\CMT2X & 710.63 & 1.44 & 
720.80 & 1.25 & \bf{682.39} & 
4.14 & 5.63\\CMT2Y & 707.15 & 1.41 & 
711.06 & 1.40 & \bf{682.39} & 
3.63 & 4.20\\CMT3X & 732.90 & 3.49 & 
741.01 & 3.19 & \bf{719.06} & 
1.92 & 3.05\\CMT3Y & 732.72 & 3.02 & 
741.01 & 3.05 & \bf{719.06} & 
1.90 & 3.05\\CMT4X & 910.65 & 8.53 & 
912.13 & 7.78 & \bf{854.21} & 
6.61 & 6.78\\CMT4Y & 892.22 & 7.69 & 
904.63 & 7.89 & \bf{852.46} & 
4.66 & 6.12\\CMT5X & 1107.25 & 16.32 & 
1121.87 & 16.69 & \bf{1030.56} & 
7.44 & 8.86\\CMT5Y & 1119.08 & 17.12 & 
1132.28 & 16.59 & \bf{1031.69} & 
8.47 & 9.75\\CMT11X & 902.36 & 4.83 & 
915.39 & 4.89 & \bf{831.09} & 
8.58 & 10.14\\CMT11Y & 912.68 & 5.56 & 
928.57 & 5.38 & \bf{829.85} & 
9.98 & 11.90\\CMT12X & 679.66 & 2.88 & 
683.70 & 3.19 & \bf{658.83} & 
3.16 & 3.78\\CMT12Y & 672.89 & 2.88 & 
677.03 & 2.94 & \bf{660.47} & 
1.88 & 2.51\\\bf{PROM.} & 
\bf{788.46} & \bf{5.43} & \bf{796.73} & \bf{5.37} & \bf{749.50} & \bf{4.72} & \bf{5.77}\\[1ex]\hline
\end{tabular}
\label{table:nonlin}
\end{table} \clearpage
\begin{table}[ht]
\caption{Resultados de la ejecución de la metaheurística IGA, utilizando instancias de Dethloff con la configuración -n 200 -p 40 -cprob 60.0 -mprob 60.0}
\centering
\small
\begin{tabular}{c c c c c c c c}
\hline\hline
Instancia & Costo mínimo & Tiempo(seg.) & Costo promedio & Tiempo promedio(seg.) & CME & \%G & \%GP \\ [0.5ex]
\hline
SCA3-0 & 640.55 & 0.94 & 
640.55 & 0.73 & \bf{635.62} & 
0.78 & 0.78\\SCA3-1 & \bf{697.84} & 0.70 & 
697.84 & 0.74 & 697.84 & 0.00
 & 0.00\\
SCA3-2 & 664.18 & 0.62 & 
666.09 & 0.69 & \bf{659.34} & 
0.73 & 1.02\\SCA3-3 & 682.47 & 0.52 & 
682.47 & 0.62 & \bf{680.04} & 
0.36 & 0.36\\SCA3-4 & \bf{690.50} & 0.66 & 
691.02 & 0.77 & 690.50 & 0.00
 & 0.08\\SCA3-5 & 679.84 & 0.61 & 
679.84 & 0.69 & \bf{659.90} & 
3.02 & 3.02\\SCA3-6 & 652.94 & 0.70 & 
656.16 & 0.71 & \bf{651.09} & 
0.28 & 0.78\\SCA3-7 & 666.60 & 0.72 & 
669.07 & 0.55 & \bf{659.17} & 
1.13 & 1.50\\SCA3-8 & 722.05 & 0.71 & 
727.11 & 0.77 & \bf{719.47} & 
0.36 & 1.06\\SCA3-9 & \bf{681.00} & 0.54 & 
682.38 & 0.67 & 681.00 & 0.00
 & 0.20\\SCA8-0 & 975.50 & 0.72 & 
975.50 & 0.71 & \bf{961.50} & 
1.46 & 1.46\\SCA8-1 & 1075.50 & 0.79 & 
1075.53 & 0.66 & \bf{1049.65} & 
2.46 & 2.47\\SCA8-2 & 1050.37 & 0.66 & 
1051.16 & 0.69 & \bf{1039.64} & 
1.03 & 1.11\\SCA8-3 & 1017.97 & 0.66 & 
1020.98 & 0.91 & \bf{983.34} & 
3.52 & 3.83\\SCA8-4 & 1093.20 & 0.87 & 
1093.20 & 0.72 & \bf{1065.49} & 
2.60 & 2.60\\SCA8-5 & 1069.30 & 0.41 & 
1071.86 & 0.55 & \bf{1027.08} & 
4.11 & 4.36\\SCA8-6 & 976.69 & 0.70 & 
976.69 & 0.67 & \bf{971.82} & 
0.50 & 0.50\\SCA8-7 & 1072.31 & 0.57 & 
1075.82 & 0.76 & \bf{1051.28} & 
2.00 & 2.33\\SCA8-8 & \bf{1071.18} & 0.43 & 
1071.18 & 0.67 & 1071.18 & 0.00
 & 0.00\\
SCA8-9 & 1081.59 & 0.42 & 
1081.59 & 0.41 & \bf{1060.50} & 
1.99 & 1.99\\CON3-0 & 632.66 & 0.50 & 
632.95 & 0.57 & \bf{616.52} & 
2.62 & 2.66\\CON3-1 & 560.75 & 0.47 & 
560.75 & 0.58 & \bf{554.47} & 
1.13 & 1.13\\CON3-2 & 521.38 & 0.60 & 
521.38 & 0.71 & \bf{518.00} & 
0.65 & 0.65\\CON3-3 & 602.16 & 0.50 & 
602.92 & 0.67 & \bf{591.19} & 
1.86 & 1.98\\CON3-4 & 592.58 & 0.72 & 
597.09 & 0.72 & \bf{588.79} & 
0.64 & 1.41\\CON3-5 & \bf{563.70} & 0.73 & 
563.70 & 0.74 & 563.70 & 0.00
 & 0.00\\
CON3-6 & 506.15 & 0.75 & 
506.54 & 0.70 & \bf{499.05} & 
1.42 & 1.50\\CON3-7 & 581.82 & 0.47 & 
585.88 & 0.57 & \bf{576.48} & 
0.93 & 1.63\\CON3-8 & 524.59 & 0.90 & 
529.88 & 0.72 & \bf{523.05} & 
0.29 & 1.31\\CON3-9 & 590.39 & 0.59 & 
590.50 & 0.52 & \bf{578.24} & 
2.10 & 2.12\\CON8-0 & 878.97 & 1.01 & 
878.97 & 0.69 & \bf{857.17} & 
2.54 & 2.54\\CON8-1 & 762.20 & 0.53 & 
767.04 & 0.59 & \bf{740.85} & 
2.88 & 3.54\\CON8-2 & 716.07 & 0.66 & 
726.05 & 0.67 & \bf{712.89} & 
0.45 & 1.85\\CON8-3 & 836.12 & 0.68 & 
842.42 & 0.61 & \bf{811.07} & 
3.09 & 3.87\\CON8-4 & 785.99 & 0.73 & 
785.99 & 0.80 & \bf{772.25} & 
1.78 & 1.78\\CON8-5 & 766.09 & 0.50 & 
769.97 & 0.72 & \bf{754.88} & 
1.49 & 2.00\\CON8-6 & 704.52 & 0.77 & 
705.30 & 0.78 & \bf{678.92} & 
3.77 & 3.89\\CON8-7 & 823.43 & 0.92 & 
832.23 & 0.66 & \bf{811.96} & 
1.41 & 2.50\\CON8-8 & 783.01 & 0.98 & 
783.01 & 0.83 & \bf{767.53} & 
2.02 & 2.02\\CON8-9 & 827.35 & 0.56 & 
827.35 & 0.59 & \bf{809.00} & 
2.27 & 2.27\\\bf{PROM.} & 
\bf{770.54} & \bf{0.66} & \bf{772.40} & \bf{0.68} & \bf{758.54} & \bf{1.49} & \bf{1.75}\\[1ex]\hline
\end{tabular}
\label{table:nonlin}
\end{table} \clearpage
\begin{table}[ht]
\caption{Resultados de la ejecución de la metaheurística IGA, utilizando instancias de SalhiNagy con la configuración -n 200 -p 40 -cprob 60.0 -mprob 60.0}
\centering
\small
\begin{tabular}{c c c c c c c c}
\hline\hline
Instancia & Costo mínimo & Tiempo(seg.) & Costo promedio & Tiempo promedio(seg.) & CME & \%G & \%GP \\ [0.5ex]
\hline
CMT1X & 476.38 & 0.48 & 
478.23 & 0.69 & \bf{470.48} & 
1.25 & 1.65\\CMT1Y & 489.58 & 0.63 & 
490.73 & 0.64 & \bf{470.48} & 
4.06 & 4.30\\CMT2X & 698.97 & 1.76 & 
704.47 & 1.43 & \bf{682.39} & 
2.43 & 3.24\\CMT2Y & 714.35 & 1.12 & 
717.77 & 1.34 & \bf{682.39} & 
4.68 & 5.18\\CMT3X & 741.00 & 3.08 & 
743.37 & 3.33 & \bf{719.06} & 
3.05 & 3.38\\CMT3Y & 740.81 & 3.00 & 
749.57 & 3.38 & \bf{719.06} & 
3.02 & 4.24\\CMT4X & 897.77 & 7.99 & 
911.57 & 8.13 & \bf{854.21} & 
5.10 & 6.71\\CMT4Y & 911.05 & 8.50 & 
918.38 & 8.46 & \bf{852.46} & 
6.87 & 7.73\\CMT5X & 1098.10 & 15.14 & 
1109.95 & 16.18 & \bf{1030.56} & 
6.55 & 7.70\\CMT5Y & 1122.14 & 16.37 & 
1125.51 & 16.73 & \bf{1031.69} & 
8.77 & 9.09\\CMT11X & 893.86 & 5.52 & 
919.48 & 4.82 & \bf{831.09} & 
7.55 & 10.63\\CMT11Y & 882.60 & 5.71 & 
915.53 & 5.66 & \bf{829.85} & 
6.36 & 10.32\\CMT12X & 672.45 & 2.66 & 
676.16 & 2.89 & \bf{658.83} & 
2.07 & 2.63\\CMT12Y & 675.31 & 2.52 & 
681.46 & 3.02 & \bf{660.47} & 
2.25 & 3.18\\\bf{PROM.} & 
\bf{786.74} & \bf{5.32} & \bf{795.87} & \bf{5.48} & \bf{749.50} & \bf{4.57} & \bf{5.71}\\[1ex]\hline
\end{tabular}
\label{table:nonlin}
\end{table} \clearpage
\begin{table}[ht]
\caption{Resultados de la ejecución de la metaheurística IGA, utilizando instancias de Dethloff con la configuración -n 200 -p 40 -cprob 60.0 -mprob 70.0}
\centering
\small
\begin{tabular}{c c c c c c c c}
\hline\hline
Instancia & Costo mínimo & Tiempo(seg.) & Costo promedio & Tiempo promedio(seg.) & CME & \%G & \%GP \\ [0.5ex]
\hline
SCA3-0 & 640.55 & 0.71 & 
640.84 & 0.63 & \bf{635.62} & 
0.78 & 0.82\\SCA3-1 & \bf{697.84} & 0.70 & 
701.03 & 0.65 & 697.84 & 0.00
 & 0.46\\SCA3-2 & 664.18 & 0.61 & 
666.62 & 0.57 & \bf{659.34} & 
0.73 & 1.10\\SCA3-3 & 681.35 & 0.75 & 
683.48 & 0.81 & \bf{680.04} & 
0.19 & 0.51\\SCA3-4 & \bf{690.50} & 0.69 & 
691.18 & 0.66 & 690.50 & 0.00
 & 0.10\\SCA3-5 & 665.64 & 0.64 & 
669.74 & 0.87 & \bf{659.90} & 
0.87 & 1.49\\SCA3-6 & 652.94 & 0.57 & 
656.34 & 0.82 & \bf{651.09} & 
0.28 & 0.81\\SCA3-7 & 666.60 & 0.44 & 
666.60 & 0.63 & \bf{659.17} & 
1.13 & 1.13\\SCA3-8 & \bf{719.47} & 0.69 & 
720.83 & 0.75 & 719.47 & 0.00
 & 0.19\\SCA3-9 & \bf{681.00} & 0.47 & 
683.94 & 0.67 & 681.00 & 0.00
 & 0.43\\SCA8-0 & 1003.16 & 0.66 & 
1008.51 & 0.78 & \bf{961.50} & 
4.33 & 4.89\\SCA8-1 & 1059.57 & 1.03 & 
1070.65 & 0.69 & \bf{1049.65} & 
0.95 & 2.00\\SCA8-2 & 1054.47 & 1.00 & 
1054.63 & 0.90 & \bf{1039.64} & 
1.43 & 1.44\\SCA8-3 & 1028.78 & 0.74 & 
1028.78 & 0.86 & \bf{983.34} & 
4.62 & 4.62\\SCA8-4 & 1085.48 & 0.75 & 
1085.48 & 0.73 & \bf{1065.49} & 
1.88 & 1.88\\SCA8-5 & 1066.15 & 0.73 & 
1066.15 & 0.72 & \bf{1027.08} & 
3.80 & 3.80\\SCA8-6 & 977.03 & 0.95 & 
984.34 & 0.75 & \bf{971.82} & 
0.54 & 1.29\\SCA8-7 & 1086.88 & 0.90 & 
1086.88 & 0.66 & \bf{1051.28} & 
3.39 & 3.39\\SCA8-8 & \bf{1071.18} & 0.88 & 
1077.89 & 0.74 & 1071.18 & 0.00
 & 0.63\\SCA8-9 & 1070.71 & 0.44 & 
1070.71 & 0.62 & \bf{1060.50} & 
0.96 & 0.96\\CON3-0 & 619.09 & 0.56 & 
622.26 & 0.76 & \bf{616.52} & 
0.42 & 0.93\\CON3-1 & 560.75 & 0.92 & 
562.88 & 0.85 & \bf{554.47} & 
1.13 & 1.52\\CON3-2 & 521.38 & 0.88 & 
524.01 & 0.81 & \bf{518.00} & 
0.65 & 1.16\\CON3-3 & 602.24 & 0.52 & 
603.55 & 0.57 & \bf{591.19} & 
1.87 & 2.09\\CON3-4 & 594.59 & 0.58 & 
594.59 & 0.63 & \bf{588.79} & 
0.99 & 0.99\\CON3-5 & 568.76 & 0.50 & 
568.76 & 0.70 & \bf{563.70} & 
0.90 & 0.90\\CON3-6 & 504.20 & 0.58 & 
506.08 & 0.73 & \bf{499.05} & 
1.03 & 1.41\\CON3-7 & 584.06 & 0.69 & 
589.96 & 0.59 & \bf{576.48} & 
1.31 & 2.34\\CON3-8 & \bf{523.05} & 0.52 & 
526.83 & 0.69 & 523.05 & 0.00
 & 0.72\\CON3-9 & 588.48 & 0.58 & 
589.73 & 0.86 & \bf{578.24} & 
1.77 & 1.99\\CON8-0 & 872.91 & 0.50 & 
872.91 & 0.74 & \bf{857.17} & 
1.84 & 1.84\\CON8-1 & 742.61 & 0.79 & 
757.95 & 0.82 & \bf{740.85} & 
0.24 & 2.31\\CON8-2 & 719.45 & 0.71 & 
719.70 & 0.81 & \bf{712.89} & 
0.92 & 0.95\\CON8-3 & 812.54 & 0.48 & 
813.66 & 0.66 & \bf{811.07} & 
0.18 & 0.32\\CON8-4 & 798.62 & 0.60 & 
802.35 & 0.61 & \bf{772.25} & 
3.41 & 3.90\\CON8-5 & 769.25 & 0.95 & 
769.25 & 0.74 & \bf{754.88} & 
1.90 & 1.90\\CON8-6 & 694.01 & 0.79 & 
694.01 & 0.76 & \bf{678.92} & 
2.22 & 2.22\\CON8-7 & 826.81 & 0.93 & 
830.90 & 0.84 & \bf{811.96} & 
1.83 & 2.33\\CON8-8 & 793.86 & 1.05 & 
795.38 & 0.77 & \bf{767.53} & 
3.43 & 3.63\\CON8-9 & 837.99 & 0.70 & 
838.08 & 0.71 & \bf{809.00} & 
3.58 & 3.59\\\bf{PROM.} & 
\bf{769.95} & \bf{0.70} & \bf{772.44} & \bf{0.73} & \bf{758.54} & \bf{1.39} & \bf{1.72}\\[1ex]\hline
\end{tabular}
\label{table:nonlin}
\end{table} \clearpage
\begin{table}[ht]
\caption{Resultados de la ejecución de la metaheurística IGA, utilizando instancias de SalhiNagy con la configuración -n 200 -p 40 -cprob 60.0 -mprob 70.0}
\centering
\small
\begin{tabular}{c c c c c c c c}
\hline\hline
Instancia & Costo mínimo & Tiempo(seg.) & Costo promedio & Tiempo promedio(seg.) & CME & \%G & \%GP \\ [0.5ex]
\hline
CMT1X & 484.85 & 0.64 & 
486.48 & 0.76 & \bf{470.48} & 
3.05 & 3.40\\CMT1Y & 480.08 & 0.54 & 
480.41 & 0.54 & \bf{470.48} & 
2.04 & 2.11\\CMT2X & 714.12 & 1.16 & 
718.29 & 1.26 & \bf{682.39} & 
4.65 & 5.26\\CMT2Y & 710.41 & 1.52 & 
713.02 & 1.36 & \bf{682.39} & 
4.11 & 4.49\\CMT3X & 742.13 & 2.95 & 
746.97 & 3.03 & \bf{719.06} & 
3.21 & 3.88\\CMT3Y & 731.79 & 3.03 & 
743.15 & 2.96 & \bf{719.06} & 
1.77 & 3.35\\CMT4X & 901.46 & 8.42 & 
910.75 & 8.41 & \bf{854.21} & 
5.53 & 6.62\\CMT4Y & 901.32 & 7.98 & 
907.97 & 8.04 & \bf{852.46} & 
5.73 & 6.51\\CMT5X & 1094.05 & 17.38 & 
1109.07 & 16.73 & \bf{1030.56} & 
6.16 & 7.62\\CMT5Y & 1110.31 & 17.22 & 
1119.53 & 16.92 & \bf{1031.69} & 
7.62 & 8.51\\CMT11X & 912.17 & 5.74 & 
915.59 & 5.36 & \bf{831.09} & 
9.76 & 10.17\\CMT11Y & 890.36 & 5.39 & 
906.26 & 5.40 & \bf{829.85} & 
7.29 & 9.21\\CMT12X & 674.56 & 3.13 & 
681.41 & 2.98 & \bf{658.83} & 
2.39 & 3.43\\CMT12Y & 676.28 & 3.18 & 
677.65 & 3.13 & \bf{660.47} & 
2.39 & 2.60\\\bf{PROM.} & 
\bf{787.42} & \bf{5.59} & \bf{794.04} & \bf{5.49} & \bf{749.50} & \bf{4.69} & \bf{5.51}\\[1ex]\hline
\end{tabular}
\label{table:nonlin}
\end{table} \clearpage
\begin{table}[ht]
\caption{Resultados de la ejecución de la metaheurística IGA, utilizando instancias de Dethloff con la configuración -n 200 -p 40 -cprob 60.0 -mprob 80.0}
\centering
\small
\begin{tabular}{c c c c c c c c}
\hline\hline
Instancia & Costo mínimo & Tiempo(seg.) & Costo promedio & Tiempo promedio(seg.) & CME & \%G & \%GP \\ [0.5ex]
\hline
SCA3-0 & 641.69 & 0.74 & 
641.69 & 0.81 & \bf{635.62} & 
0.95 & 0.95\\SCA3-1 & \bf{697.84} & 0.74 & 
700.09 & 0.85 & 697.84 & 0.00
 & 0.32\\SCA3-2 & 661.13 & 0.75 & 
661.90 & 0.68 & \bf{659.34} & 
0.27 & 0.39\\SCA3-3 & 685.05 & 0.65 & 
685.73 & 0.61 & \bf{680.04} & 
0.74 & 0.84\\SCA3-4 & \bf{690.50} & 0.95 & 
690.50 & 0.70 & 690.50 & 0.00
 & 0.00\\
SCA3-5 & 666.67 & 0.72 & 
675.13 & 0.75 & \bf{659.90} & 
1.03 & 2.31\\SCA3-6 & 652.94 & 0.90 & 
652.94 & 0.90 & \bf{651.09} & 
0.28 & 0.28\\SCA3-7 & 666.60 & 0.63 & 
669.13 & 0.78 & \bf{659.17} & 
1.13 & 1.51\\SCA3-8 & 724.66 & 0.95 & 
724.66 & 0.85 & \bf{719.47} & 
0.72 & 0.72\\SCA3-9 & \bf{681.00} & 0.89 & 
681.00 & 0.78 & 681.00 & 0.00
 & 0.00\\
SCA8-0 & 983.21 & 0.66 & 
992.81 & 0.67 & \bf{961.50} & 
2.26 & 3.26\\SCA8-1 & 1087.01 & 0.75 & 
1091.35 & 0.79 & \bf{1049.65} & 
3.56 & 3.97\\SCA8-2 & 1054.69 & 0.96 & 
1054.69 & 0.78 & \bf{1039.64} & 
1.45 & 1.45\\SCA8-3 & 995.60 & 0.75 & 
995.60 & 0.79 & \bf{983.34} & 
1.25 & 1.25\\SCA8-4 & 1085.57 & 0.77 & 
1085.57 & 0.85 & \bf{1065.49} & 
1.88 & 1.88\\SCA8-5 & 1054.29 & 0.94 & 
1056.19 & 0.68 & \bf{1027.08} & 
2.65 & 2.83\\SCA8-6 & 983.76 & 0.52 & 
983.76 & 0.66 & \bf{971.82} & 
1.23 & 1.23\\SCA8-7 & 1080.27 & 0.72 & 
1090.51 & 0.83 & \bf{1051.28} & 
2.76 & 3.73\\SCA8-8 & 1084.41 & 0.97 & 
1087.20 & 0.78 & \bf{1071.18} & 
1.24 & 1.50\\SCA8-9 & 1078.80 & 0.63 & 
1078.80 & 0.63 & \bf{1060.50} & 
1.73 & 1.73\\CON3-0 & 620.76 & 0.71 & 
622.32 & 0.80 & \bf{616.52} & 
0.69 & 0.94\\CON3-1 & 559.25 & 0.72 & 
561.23 & 0.74 & \bf{554.47} & 
0.86 & 1.22\\CON3-2 & 521.38 & 0.56 & 
521.38 & 0.76 & \bf{518.00} & 
0.65 & 0.65\\CON3-3 & 602.14 & 0.66 & 
602.20 & 0.60 & \bf{591.19} & 
1.85 & 1.86\\CON3-4 & 592.58 & 0.76 & 
595.10 & 0.78 & \bf{588.79} & 
0.64 & 1.07\\CON3-5 & 564.88 & 0.94 & 
568.35 & 0.92 & \bf{563.70} & 
0.21 & 0.82\\CON3-6 & 504.15 & 0.96 & 
507.19 & 0.97 & \bf{499.05} & 
1.02 & 1.63\\CON3-7 & 577.68 & 0.89 & 
580.65 & 0.78 & \bf{576.48} & 
0.21 & 0.72\\CON3-8 & 524.59 & 0.80 & 
528.73 & 0.86 & \bf{523.05} & 
0.29 & 1.08\\CON3-9 & 588.48 & 0.67 & 
592.82 & 0.75 & \bf{578.24} & 
1.77 & 2.52\\CON8-0 & 872.66 & 0.50 & 
879.47 & 0.77 & \bf{857.17} & 
1.81 & 2.60\\CON8-1 & 758.44 & 1.02 & 
759.19 & 0.98 & \bf{740.85} & 
2.37 & 2.48\\CON8-2 & 725.38 & 1.00 & 
726.59 & 0.94 & \bf{712.89} & 
1.75 & 1.92\\CON8-3 & 835.73 & 0.64 & 
835.88 & 0.92 & \bf{811.07} & 
3.04 & 3.06\\CON8-4 & 792.67 & 1.00 & 
798.44 & 0.88 & \bf{772.25} & 
2.64 & 3.39\\CON8-5 & 764.31 & 0.93 & 
764.31 & 0.83 & \bf{754.88} & 
1.25 & 1.25\\CON8-6 & 689.56 & 0.93 & 
690.56 & 0.78 & \bf{678.92} & 
1.57 & 1.71\\CON8-7 & 815.79 & 0.76 & 
823.37 & 0.75 & \bf{811.96} & 
0.47 & 1.40\\CON8-8 & 795.75 & 0.68 & 
795.75 & 0.86 & \bf{767.53} & 
3.68 & 3.68\\CON8-9 & 830.27 & 0.94 & 
839.80 & 0.81 & \bf{809.00} & 
2.63 & 3.81\\\bf{PROM.} & 
\bf{769.80} & \bf{0.79} & \bf{772.31} & \bf{0.79} & \bf{758.54} & \bf{1.36} & \bf{1.70}\\[1ex]\hline
\end{tabular}
\label{table:nonlin}
\end{table} \clearpage
\begin{table}[ht]
\caption{Resultados de la ejecución de la metaheurística IGA, utilizando instancias de SalhiNagy con la configuración -n 200 -p 40 -cprob 60.0 -mprob 80.0}
\centering
\small
\begin{tabular}{c c c c c c c c}
\hline\hline
Instancia & Costo mínimo & Tiempo(seg.) & Costo promedio & Tiempo promedio(seg.) & CME & \%G & \%GP \\ [0.5ex]
\hline
CMT1X & 477.41 & 0.55 & 
478.38 & 0.59 & \bf{470.48} & 
1.47 & 1.68\\CMT1Y & 483.34 & 0.56 & 
484.47 & 0.53 & \bf{470.48} & 
2.73 & 2.97\\CMT2X & 701.24 & 1.69 & 
712.01 & 1.60 & \bf{682.39} & 
2.76 & 4.34\\CMT2Y & 713.28 & 1.76 & 
715.70 & 1.34 & \bf{682.39} & 
4.53 & 4.88\\CMT3X & 745.44 & 3.38 & 
748.75 & 3.18 & \bf{719.06} & 
3.67 & 4.13\\CMT3Y & 737.56 & 3.38 & 
746.18 & 3.06 & \bf{719.06} & 
2.57 & 3.77\\CMT4X & 918.25 & 7.79 & 
920.98 & 8.05 & \bf{854.21} & 
7.50 & 7.82\\CMT4Y & 889.10 & 7.98 & 
914.65 & 8.36 & \bf{852.46} & 
4.30 & 7.30\\CMT5X & 1109.86 & 16.09 & 
1125.13 & 15.91 & \bf{1030.56} & 
7.69 & 9.18\\CMT5Y & 1097.16 & 16.11 & 
1114.89 & 16.17 & \bf{1031.69} & 
6.35 & 8.06\\CMT11X & 910.85 & 5.28 & 
922.85 & 4.83 & \bf{831.09} & 
9.60 & 11.04\\CMT11Y & 891.37 & 5.16 & 
899.40 & 5.58 & \bf{829.85} & 
7.41 & 8.38\\CMT12X & 676.89 & 2.78 & 
683.92 & 2.96 & \bf{658.83} & 
2.74 & 3.81\\CMT12Y & 675.37 & 2.89 & 
678.80 & 3.00 & \bf{660.47} & 
2.26 & 2.77\\\bf{PROM.} & 
\bf{787.65} & \bf{5.39} & \bf{796.15} & \bf{5.37} & \bf{749.50} & \bf{4.68} & \bf{5.72}\\[1ex]\hline
\end{tabular}
\label{table:nonlin}
\end{table} \clearpage
\begin{table}[ht]
\caption{Resultados de la ejecución de la metaheurística IGA, utilizando instancias de Dethloff con la configuración -n 200 -p 40 -cprob 60.0 -mprob 90.0}
\centering
\small
\begin{tabular}{c c c c c c c c}
\hline\hline
Instancia & Costo mínimo & Tiempo(seg.) & Costo promedio & Tiempo promedio(seg.) & CME & \%G & \%GP \\ [0.5ex]
\hline
SCA3-0 & 640.55 & 0.48 & 
640.55 & 0.72 & \bf{635.62} & 
0.78 & 0.78\\SCA3-1 & 700.50 & 0.74 & 
700.50 & 0.82 & \bf{697.84} & 
0.38 & 0.38\\SCA3-2 & 661.13 & 0.72 & 
670.03 & 0.75 & \bf{659.34} & 
0.27 & 1.62\\SCA3-3 & 680.60 & 0.56 & 
680.88 & 0.68 & \bf{680.04} & 
0.08 & 0.12\\SCA3-4 & \bf{690.50} & 0.65 & 
690.50 & 0.74 & 690.50 & 0.00
 & 0.00\\
SCA3-5 & 674.18 & 0.85 & 
680.36 & 0.76 & \bf{659.90} & 
2.16 & 3.10\\SCA3-6 & 652.94 & 0.90 & 
654.67 & 0.78 & \bf{651.09} & 
0.28 & 0.55\\SCA3-7 & 666.15 & 0.80 & 
670.71 & 0.81 & \bf{659.17} & 
1.06 & 1.75\\SCA3-8 & 723.99 & 0.88 & 
725.51 & 0.80 & \bf{719.47} & 
0.63 & 0.84\\SCA3-9 & 681.68 & 0.52 & 
683.16 & 0.72 & \bf{681.00} & 
0.10 & 0.32\\SCA8-0 & 1004.66 & 0.97 & 
1010.42 & 0.87 & \bf{961.50} & 
4.49 & 5.09\\SCA8-1 & 1066.49 & 0.58 & 
1070.16 & 0.76 & \bf{1049.65} & 
1.60 & 1.95\\SCA8-2 & 1051.21 & 0.66 & 
1051.21 & 0.81 & \bf{1039.64} & 
1.11 & 1.11\\SCA8-3 & 1014.71 & 0.72 & 
1023.45 & 0.82 & \bf{983.34} & 
3.19 & 4.08\\SCA8-4 & 1079.12 & 0.91 & 
1079.12 & 0.83 & \bf{1065.49} & 
1.28 & 1.28\\SCA8-5 & 1054.37 & 0.77 & 
1054.37 & 0.81 & \bf{1027.08} & 
2.66 & 2.66\\SCA8-6 & 982.42 & 0.69 & 
984.09 & 0.82 & \bf{971.82} & 
1.09 & 1.26\\SCA8-7 & 1087.21 & 0.56 & 
1094.07 & 0.76 & \bf{1051.28} & 
3.42 & 4.07\\SCA8-8 & 1090.51 & 0.82 & 
1090.51 & 0.65 & \bf{1071.18} & 
1.80 & 1.80\\SCA8-9 & 1081.71 & 0.55 & 
1083.81 & 0.79 & \bf{1060.50} & 
2.00 & 2.20\\CON3-0 & 619.09 & 0.64 & 
619.09 & 0.69 & \bf{616.52} & 
0.42 & 0.42\\CON3-1 & 560.75 & 0.70 & 
561.63 & 0.78 & \bf{554.47} & 
1.13 & 1.29\\CON3-2 & 521.38 & 1.00 & 
522.07 & 0.79 & \bf{518.00} & 
0.65 & 0.79\\CON3-3 & 591.20 & 0.89 & 
598.68 & 0.90 & \bf{591.19} & 
0.00 & 1.27\\CON3-4 & 593.78 & 0.57 & 
596.29 & 0.66 & \bf{588.79} & 
0.85 & 1.27\\CON3-5 & 567.94 & 0.59 & 
570.26 & 0.64 & \bf{563.70} & 
0.75 & 1.16\\CON3-6 & 503.97 & 0.51 & 
505.43 & 0.57 & \bf{499.05} & 
0.99 & 1.28\\CON3-7 & 586.01 & 0.91 & 
586.99 & 0.80 & \bf{576.48} & 
1.65 & 1.82\\CON3-8 & 523.14 & 0.95 & 
526.37 & 0.79 & \bf{523.05} & 
0.02 & 0.63\\CON3-9 & 588.11 & 0.94 & 
590.43 & 0.71 & \bf{578.24} & 
1.71 & 2.11\\CON8-0 & 871.45 & 0.85 & 
871.45 & 0.72 & \bf{857.17} & 
1.67 & 1.67\\CON8-1 & 766.03 & 0.65 & 
766.03 & 0.82 & \bf{740.85} & 
3.40 & 3.40\\CON8-2 & 717.36 & 0.76 & 
723.25 & 0.81 & \bf{712.89} & 
0.63 & 1.45\\CON8-3 & 815.14 & 0.71 & 
827.54 & 0.83 & \bf{811.07} & 
0.50 & 2.03\\CON8-4 & 773.71 & 0.50 & 
799.81 & 0.66 & \bf{772.25} & 
0.19 & 3.57\\CON8-5 & 767.03 & 0.94 & 
767.52 & 0.72 & \bf{754.88} & 
1.61 & 1.67\\CON8-6 & 695.38 & 0.98 & 
695.38 & 0.94 & \bf{678.92} & 
2.42 & 2.42\\CON8-7 & 815.91 & 0.94 & 
820.64 & 0.88 & \bf{811.96} & 
0.49 & 1.07\\CON8-8 & 782.34 & 1.04 & 
794.00 & 0.91 & \bf{767.53} & 
1.93 & 3.45\\CON8-9 & 824.42 & 1.02 & 
828.30 & 0.97 & \bf{809.00} & 
1.91 & 2.39\\\bf{PROM.} & 
\bf{769.22} & \bf{0.76} & \bf{772.73} & \bf{0.78} & \bf{758.54} & \bf{1.28} & \bf{1.75}\\[1ex]\hline
\end{tabular}
\label{table:nonlin}
\end{table} \clearpage
\begin{table}[ht]
\caption{Resultados de la ejecución de la metaheurística IGA, utilizando instancias de SalhiNagy con la configuración -n 200 -p 40 -cprob 60.0 -mprob 90.0}
\centering
\small
\begin{tabular}{c c c c c c c c}
\hline\hline
Instancia & Costo mínimo & Tiempo(seg.) & Costo promedio & Tiempo promedio(seg.) & CME & \%G & \%GP \\ [0.5ex]
\hline
CMT1X & 485.35 & 0.48 & 
487.47 & 0.71 & \bf{470.48} & 
3.16 & 3.61\\CMT1Y & 472.87 & 0.42 & 
479.03 & 0.42 & \bf{470.48} & 
0.51 & 1.82\\CMT2X & 703.97 & 1.71 & 
715.31 & 1.50 & \bf{682.39} & 
3.16 & 4.82\\CMT2Y & 710.04 & 1.58 & 
719.09 & 1.42 & \bf{682.39} & 
4.05 & 5.38\\CMT3X & 737.87 & 2.93 & 
742.65 & 2.87 & \bf{719.06} & 
2.62 & 3.28\\CMT3Y & 744.20 & 2.97 & 
749.26 & 2.94 & \bf{719.06} & 
3.50 & 4.20\\CMT4X & 897.44 & 7.79 & 
907.42 & 8.29 & \bf{854.21} & 
5.06 & 6.23\\CMT4Y & 896.79 & 8.70 & 
905.68 & 8.56 & \bf{852.46} & 
5.20 & 6.24\\CMT5X & 1106.41 & 17.40 & 
1120.04 & 16.88 & \bf{1030.56} & 
7.36 & 8.68\\CMT5Y & 1105.19 & 16.33 & 
1119.42 & 16.87 & \bf{1031.69} & 
7.12 & 8.50\\CMT11X & 893.30 & 5.60 & 
915.90 & 5.30 & \bf{831.09} & 
7.49 & 10.21\\CMT11Y & 913.58 & 5.26 & 
927.21 & 5.48 & \bf{829.85} & 
10.09 & 11.73\\CMT12X & 674.72 & 3.63 & 
678.98 & 3.18 & \bf{658.83} & 
2.41 & 3.06\\CMT12Y & 675.05 & 3.00 & 
678.02 & 3.17 & \bf{660.47} & 
2.21 & 2.66\\\bf{PROM.} & 
\bf{786.91} & \bf{5.56} & \bf{796.11} & \bf{5.54} & \bf{749.50} & \bf{4.57} & \bf{5.74}\\[1ex]\hline
\end{tabular}
\label{table:nonlin}
\end{table} \clearpage
\begin{table}[ht]
\caption{Resultados de la ejecución de la metaheurística IGA, utilizando instancias de Dethloff con la configuración -n 200 -p 40 -cprob 60.0 -mprob 100.0}
\centering
\small
\begin{tabular}{c c c c c c c c}
\hline\hline
Instancia & Costo mínimo & Tiempo(seg.) & Costo promedio & Tiempo promedio(seg.) & CME & \%G & \%GP \\ [0.5ex]
\hline
SCA3-0 & 640.55 & 0.69 & 
640.55 & 0.86 & \bf{635.62} & 
0.78 & 0.78\\SCA3-1 & 701.53 & 0.92 & 
701.76 & 0.96 & \bf{697.84} & 
0.53 & 0.56\\SCA3-2 & 665.71 & 0.66 & 
666.70 & 0.75 & \bf{659.34} & 
0.97 & 1.12\\SCA3-3 & 681.16 & 0.92 & 
683.61 & 0.95 & \bf{680.04} & 
0.16 & 0.53\\SCA3-4 & \bf{690.50} & 0.65 & 
691.53 & 0.75 & 690.50 & 0.00
 & 0.15\\SCA3-5 & 670.10 & 0.68 & 
670.10 & 0.72 & \bf{659.90} & 
1.55 & 1.55\\SCA3-6 & 652.94 & 0.65 & 
657.26 & 0.68 & \bf{651.09} & 
0.28 & 0.95\\SCA3-7 & 666.15 & 0.62 & 
667.99 & 0.71 & \bf{659.17} & 
1.06 & 1.34\\SCA3-8 & 727.49 & 0.99 & 
728.79 & 0.87 & \bf{719.47} & 
1.11 & 1.30\\SCA3-9 & \bf{681.00} & 0.56 & 
683.70 & 0.69 & 681.00 & 0.00
 & 0.40\\SCA8-0 & 988.26 & 0.97 & 
988.26 & 0.81 & \bf{961.50} & 
2.78 & 2.78\\SCA8-1 & 1073.34 & 0.84 & 
1082.32 & 0.89 & \bf{1049.65} & 
2.26 & 3.11\\SCA8-2 & 1053.94 & 0.66 & 
1054.03 & 0.68 & \bf{1039.64} & 
1.38 & 1.38\\SCA8-3 & 1022.75 & 0.67 & 
1022.79 & 0.80 & \bf{983.34} & 
4.01 & 4.01\\SCA8-4 & \bf{1065.49} & 0.79 & 
1069.66 & 0.84 & 1065.49 & 0.00
 & 0.39\\SCA8-5 & 1056.60 & 0.92 & 
1057.76 & 0.85 & \bf{1027.08} & 
2.87 & 2.99\\SCA8-6 & 978.03 & 0.68 & 
978.03 & 0.76 & \bf{971.82} & 
0.64 & 0.64\\SCA8-7 & 1078.87 & 0.60 & 
1078.87 & 0.76 & \bf{1051.28} & 
2.62 & 2.62\\SCA8-8 & \bf{1071.18} & 0.73 & 
1071.18 & 0.72 & 1071.18 & 0.00
 & 0.00\\
SCA8-9 & 1072.28 & 0.84 & 
1081.16 & 0.88 & \bf{1060.50} & 
1.11 & 1.95\\CON3-0 & 619.09 & 0.67 & 
619.09 & 0.81 & \bf{616.52} & 
0.42 & 0.42\\CON3-1 & 560.61 & 0.97 & 
560.72 & 0.84 & \bf{554.47} & 
1.11 & 1.13\\CON3-2 & 521.38 & 0.88 & 
521.38 & 0.77 & \bf{518.00} & 
0.65 & 0.65\\CON3-3 & 594.31 & 0.69 & 
602.55 & 0.80 & \bf{591.19} & 
0.53 & 1.92\\CON3-4 & 593.78 & 0.72 & 
596.99 & 0.70 & \bf{588.79} & 
0.85 & 1.39\\CON3-5 & 564.89 & 0.97 & 
567.37 & 0.91 & \bf{563.70} & 
0.21 & 0.65\\CON3-6 & 504.15 & 1.00 & 
506.62 & 0.86 & \bf{499.05} & 
1.02 & 1.52\\CON3-7 & 578.41 & 0.92 & 
578.41 & 0.92 & \bf{576.48} & 
0.33 & 0.33\\CON3-8 & 531.84 & 0.58 & 
532.28 & 0.72 & \bf{523.05} & 
1.68 & 1.76\\CON3-9 & 588.63 & 0.63 & 
590.49 & 0.83 & \bf{578.24} & 
1.80 & 2.12\\CON8-0 & 868.83 & 0.95 & 
873.87 & 0.84 & \bf{857.17} & 
1.36 & 1.95\\CON8-1 & 754.96 & 0.59 & 
767.47 & 0.65 & \bf{740.85} & 
1.90 & 3.59\\CON8-2 & 724.68 & 0.95 & 
724.68 & 0.88 & \bf{712.89} & 
1.65 & 1.65\\CON8-3 & 829.87 & 0.99 & 
834.45 & 0.91 & \bf{811.07} & 
2.32 & 2.88\\CON8-4 & 796.60 & 0.64 & 
796.60 & 0.79 & \bf{772.25} & 
3.15 & 3.15\\CON8-5 & 779.19 & 0.71 & 
779.19 & 0.82 & \bf{754.88} & 
3.22 & 3.22\\CON8-6 & 696.29 & 0.65 & 
696.29 & 0.75 & \bf{678.92} & 
2.56 & 2.56\\CON8-7 & 822.33 & 0.76 & 
822.33 & 0.77 & \bf{811.96} & 
1.28 & 1.28\\CON8-8 & 786.18 & 1.03 & 
786.18 & 1.03 & \bf{767.53} & 
2.43 & 2.43\\CON8-9 & 822.10 & 0.96 & 
822.10 & 0.80 & \bf{809.00} & 
1.62 & 1.62\\\bf{PROM.} & 
\bf{769.40} & \bf{0.78} & \bf{771.38} & \bf{0.81} & \bf{758.54} & \bf{1.36} & \bf{1.62}\\[1ex]\hline
\end{tabular}
\label{table:nonlin}
\end{table} \clearpage
\begin{table}[ht]
\caption{Resultados de la ejecución de la metaheurística IGA, utilizando instancias de SalhiNagy con la configuración -n 200 -p 40 -cprob 60.0 -mprob 100.0}
\centering
\small
\begin{tabular}{c c c c c c c c}
\hline\hline
Instancia & Costo mínimo & Tiempo(seg.) & Costo promedio & Tiempo promedio(seg.) & CME & \%G & \%GP \\ [0.5ex]
\hline
CMT1X & 482.84 & 0.90 & 
485.39 & 0.80 & \bf{470.48} & 
2.63 & 3.17\\CMT1Y & 475.35 & 0.84 & 
485.03 & 0.83 & \bf{470.48} & 
1.04 & 3.09\\CMT2X & 711.05 & 1.73 & 
718.23 & 1.41 & \bf{682.39} & 
4.20 & 5.25\\CMT2Y & 696.12 & 1.50 & 
706.89 & 1.49 & \bf{682.39} & 
2.01 & 3.59\\CMT3X & 741.18 & 2.88 & 
745.32 & 3.27 & \bf{719.06} & 
3.08 & 3.65\\CMT3Y & 737.62 & 3.14 & 
742.69 & 3.70 & \bf{719.06} & 
2.58 & 3.29\\CMT4X & 913.49 & 8.06 & 
918.34 & 8.17 & \bf{854.21} & 
6.94 & 7.51\\CMT4Y & 924.80 & 8.68 & 
930.45 & 8.26 & \bf{852.46} & 
8.49 & 9.15\\CMT5X & 1099.81 & 16.44 & 
1115.13 & 16.69 & \bf{1030.56} & 
6.72 & 8.21\\CMT5Y & 1081.29 & 16.68 & 
1124.64 & 17.34 & \bf{1031.69} & 
4.81 & 9.01\\CMT11X & 901.34 & 5.36 & 
906.58 & 5.24 & \bf{831.09} & 
8.45 & 9.08\\CMT11Y & 886.17 & 5.54 & 
919.80 & 5.49 & \bf{829.85} & 
6.79 & 10.84\\CMT12X & 674.65 & 2.93 & 
676.32 & 3.21 & \bf{658.83} & 
2.40 & 2.66\\CMT12Y & 674.13 & 2.92 & 
679.09 & 3.25 & \bf{660.47} & 
2.07 & 2.82\\\bf{PROM.} & 
\bf{785.70} & \bf{5.54} & \bf{796.71} & \bf{5.65} & \bf{749.50} & \bf{4.44} & \bf{5.81}\\[1ex]\hline
\end{tabular}
\label{table:nonlin}
\end{table} \clearpage
\begin{table}[ht]
\caption{Resultados de la ejecución de la metaheurística IGA, utilizando instancias de Dethloff con la configuración -n 200 -p 40 -cprob 70.0 -mprob 10.0}
\centering
\small
\begin{tabular}{c c c c c c c c}
\hline\hline
Instancia & Costo mínimo & Tiempo(seg.) & Costo promedio & Tiempo promedio(seg.) & CME & \%G & \%GP \\ [0.5ex]
\hline
SCA3-0 & 640.55 & 0.76 & 
640.84 & 0.75 & \bf{635.62} & 
0.78 & 0.82\\SCA3-1 & \bf{697.84} & 0.70 & 
697.84 & 0.74 & 697.84 & 0.00
 & 0.00\\
SCA3-2 & 664.21 & 0.68 & 
665.56 & 0.69 & \bf{659.34} & 
0.74 & 0.94\\SCA3-3 & \bf{680.04} & 0.57 & 
680.98 & 0.67 & 680.04 & 0.00
 & 0.14\\SCA3-4 & \bf{690.50} & 0.48 & 
690.50 & 0.51 & 690.50 & 0.00
 & 0.00\\
SCA3-5 & 670.10 & 0.57 & 
671.83 & 0.61 & \bf{659.90} & 
1.55 & 1.81\\SCA3-6 & 652.94 & 0.74 & 
654.11 & 0.80 & \bf{651.09} & 
0.28 & 0.46\\SCA3-7 & 666.15 & 0.53 & 
666.42 & 0.57 & \bf{659.17} & 
1.06 & 1.10\\SCA3-8 & 724.29 & 0.52 & 
728.28 & 0.57 & \bf{719.47} & 
0.67 & 1.22\\SCA3-9 & \bf{681.00} & 0.56 & 
681.00 & 0.55 & 681.00 & 0.00
 & 0.00\\
SCA8-0 & 1001.18 & 0.58 & 
1001.18 & 0.73 & \bf{961.50} & 
4.13 & 4.13\\SCA8-1 & 1076.87 & 0.98 & 
1076.87 & 0.64 & \bf{1049.65} & 
2.59 & 2.59\\SCA8-2 & 1051.80 & 0.74 & 
1051.80 & 0.70 & \bf{1039.64} & 
1.17 & 1.17\\SCA8-3 & 1022.17 & 0.60 & 
1025.58 & 0.63 & \bf{983.34} & 
3.95 & 4.30\\SCA8-4 & 1092.65 & 0.52 & 
1093.95 & 0.63 & \bf{1065.49} & 
2.55 & 2.67\\SCA8-5 & 1059.40 & 0.75 & 
1059.40 & 0.77 & \bf{1027.08} & 
3.15 & 3.15\\SCA8-6 & 977.87 & 0.57 & 
979.47 & 0.65 & \bf{971.82} & 
0.62 & 0.79\\SCA8-7 & 1077.67 & 0.60 & 
1077.67 & 0.56 & \bf{1051.28} & 
2.51 & 2.51\\SCA8-8 & 1091.18 & 0.52 & 
1091.18 & 0.66 & \bf{1071.18} & 
1.87 & 1.87\\SCA8-9 & 1084.48 & 0.70 & 
1084.48 & 0.56 & \bf{1060.50} & 
2.26 & 2.26\\CON3-0 & 624.67 & 0.49 & 
624.99 & 0.58 & \bf{616.52} & 
1.32 & 1.37\\CON3-1 & 556.04 & 0.55 & 
560.00 & 0.61 & \bf{554.47} & 
0.28 & 1.00\\CON3-2 & 521.38 & 0.57 & 
521.38 & 0.78 & \bf{518.00} & 
0.65 & 0.65\\CON3-3 & 591.20 & 0.50 & 
591.98 & 0.60 & \bf{591.19} & 
0.00 & 0.13\\CON3-4 & 597.39 & 0.50 & 
598.01 & 0.61 & \bf{588.79} & 
1.46 & 1.57\\CON3-5 & 567.94 & 0.50 & 
567.94 & 0.54 & \bf{563.70} & 
0.75 & 0.75\\CON3-6 & 505.31 & 0.76 & 
507.87 & 0.68 & \bf{499.05} & 
1.25 & 1.77\\CON3-7 & \bf{576.48} & 0.68 & 
587.77 & 0.69 & 576.48 & 0.00
 & 1.96\\CON3-8 & 523.14 & 0.72 & 
527.69 & 0.73 & \bf{523.05} & 
0.02 & 0.89\\CON3-9 & 588.11 & 0.84 & 
588.11 & 0.68 & \bf{578.24} & 
1.71 & 1.71\\CON8-0 & 873.42 & 0.73 & 
873.42 & 0.69 & \bf{857.17} & 
1.90 & 1.90\\CON8-1 & 761.87 & 0.61 & 
767.76 & 0.74 & \bf{740.85} & 
2.84 & 3.63\\CON8-2 & 719.89 & 0.80 & 
721.23 & 0.81 & \bf{712.89} & 
0.98 & 1.17\\CON8-3 & 832.54 & 0.78 & 
835.32 & 0.66 & \bf{811.07} & 
2.65 & 2.99\\CON8-4 & 787.98 & 0.75 & 
791.55 & 0.64 & \bf{772.25} & 
2.04 & 2.50\\CON8-5 & 759.59 & 0.74 & 
759.59 & 0.78 & \bf{754.88} & 
0.62 & 0.62\\CON8-6 & 699.31 & 0.70 & 
699.31 & 0.66 & \bf{678.92} & 
3.00 & 3.00\\CON8-7 & 835.78 & 0.71 & 
839.15 & 0.66 & \bf{811.96} & 
2.93 & 3.35\\CON8-8 & 795.58 & 0.74 & 
797.01 & 0.71 & \bf{767.53} & 
3.65 & 3.84\\CON8-9 & 835.44 & 0.75 & 
835.44 & 0.67 & \bf{809.00} & 
3.27 & 3.27\\\bf{PROM.} & 
\bf{771.40} & \bf{0.65} & \bf{772.86} & \bf{0.66} & \bf{758.54} & \bf{1.53} & \bf{1.75}\\[1ex]\hline
\end{tabular}
\label{table:nonlin}
\end{table} \clearpage
\begin{table}[ht]
\caption{Resultados de la ejecución de la metaheurística IGA, utilizando instancias de SalhiNagy con la configuración -n 200 -p 40 -cprob 70.0 -mprob 10.0}
\centering
\small
\begin{tabular}{c c c c c c c c}
\hline\hline
Instancia & Costo mínimo & Tiempo(seg.) & Costo promedio & Tiempo promedio(seg.) & CME & \%G & \%GP \\ [0.5ex]
\hline
CMT1X & 482.49 & 0.64 & 
482.51 & 0.58 & \bf{470.48} & 
2.55 & 2.56\\CMT1Y & 480.79 & 0.46 & 
485.12 & 0.42 & \bf{470.48} & 
2.19 & 3.11\\CMT2X & 704.27 & 1.32 & 
710.36 & 1.33 & \bf{682.39} & 
3.21 & 4.10\\CMT2Y & 715.09 & 1.05 & 
721.36 & 1.11 & \bf{682.39} & 
4.79 & 5.71\\CMT3X & 743.94 & 3.18 & 
746.77 & 3.13 & \bf{719.06} & 
3.46 & 3.85\\CMT3Y & 736.43 & 3.02 & 
745.54 & 2.93 & \bf{719.06} & 
2.42 & 3.68\\CMT4X & 894.57 & 7.94 & 
912.65 & 7.85 & \bf{854.21} & 
4.72 & 6.84\\CMT4Y & 909.94 & 8.16 & 
919.89 & 7.60 & \bf{852.46} & 
6.74 & 7.91\\CMT5X & 1093.08 & 17.04 & 
1114.86 & 17.01 & \bf{1030.56} & 
6.07 & 8.18\\CMT5Y & 1122.21 & 17.16 & 
1133.15 & 16.79 & \bf{1031.69} & 
8.77 & 9.83\\CMT11X & 897.97 & 5.10 & 
913.36 & 5.16 & \bf{831.09} & 
8.05 & 9.90\\CMT11Y & 885.53 & 5.80 & 
903.63 & 5.64 & \bf{829.85} & 
6.71 & 8.89\\CMT12X & 684.33 & 2.85 & 
686.13 & 3.03 & \bf{658.83} & 
3.87 & 4.14\\CMT12Y & 679.47 & 2.69 & 
681.75 & 2.92 & \bf{660.47} & 
2.88 & 3.22\\\bf{PROM.} & 
\bf{787.87} & \bf{5.46} & \bf{796.93} & \bf{5.39} & \bf{749.50} & \bf{4.75} & \bf{5.85}\\[1ex]\hline
\end{tabular}
\label{table:nonlin}
\end{table} \clearpage
\begin{table}[ht]
\caption{Resultados de la ejecución de la metaheurística IGA, utilizando instancias de Dethloff con la configuración -n 200 -p 40 -cprob 70.0 -mprob 20.0}
\centering
\small
\begin{tabular}{c c c c c c c c}
\hline\hline
Instancia & Costo mínimo & Tiempo(seg.) & Costo promedio & Tiempo promedio(seg.) & CME & \%G & \%GP \\ [0.5ex]
\hline
SCA3-0 & 640.55 & 0.76 & 
641.12 & 0.70 & \bf{635.62} & 
0.78 & 0.87\\SCA3-1 & \bf{697.84} & 0.55 & 
697.84 & 0.56 & 697.84 & 0.00
 & 0.00\\
SCA3-2 & 664.18 & 0.68 & 
668.11 & 0.66 & \bf{659.34} & 
0.73 & 1.33\\SCA3-3 & 682.47 & 0.53 & 
683.12 & 0.66 & \bf{680.04} & 
0.36 & 0.45\\SCA3-4 & \bf{690.50} & 0.56 & 
690.50 & 0.49 & 690.50 & 0.00
 & 0.00\\
SCA3-5 & 666.67 & 0.53 & 
669.84 & 0.55 & \bf{659.90} & 
1.03 & 1.51\\SCA3-6 & 652.94 & 0.75 & 
659.17 & 0.72 & \bf{651.09} & 
0.28 & 1.24\\SCA3-7 & 666.15 & 0.61 & 
666.38 & 0.63 & \bf{659.17} & 
1.06 & 1.09\\SCA3-8 & 724.29 & 0.50 & 
725.80 & 0.65 & \bf{719.47} & 
0.67 & 0.88\\SCA3-9 & \bf{681.00} & 0.72 & 
681.00 & 0.70 & 681.00 & 0.00
 & 0.00\\
SCA8-0 & 998.79 & 0.57 & 
998.79 & 0.73 & \bf{961.50} & 
3.88 & 3.88\\SCA8-1 & 1072.65 & 0.72 & 
1072.65 & 0.74 & \bf{1049.65} & 
2.19 & 2.19\\SCA8-2 & 1050.37 & 0.62 & 
1050.37 & 0.67 & \bf{1039.64} & 
1.03 & 1.03\\SCA8-3 & 1035.44 & 0.47 & 
1037.56 & 0.59 & \bf{983.34} & 
5.30 & 5.51\\SCA8-4 & 1074.02 & 0.91 & 
1075.04 & 0.86 & \bf{1065.49} & 
0.80 & 0.90\\SCA8-5 & 1056.89 & 0.73 & 
1056.89 & 0.69 & \bf{1027.08} & 
2.90 & 2.90\\SCA8-6 & 975.81 & 0.76 & 
975.81 & 0.78 & \bf{971.82} & 
0.41 & 0.41\\SCA8-7 & 1070.92 & 0.55 & 
1070.92 & 0.56 & \bf{1051.28} & 
1.87 & 1.87\\SCA8-8 & 1087.10 & 0.89 & 
1087.10 & 0.81 & \bf{1071.18} & 
1.49 & 1.49\\SCA8-9 & 1073.40 & 0.48 & 
1073.40 & 0.68 & \bf{1060.50} & 
1.22 & 1.22\\CON3-0 & 624.96 & 0.73 & 
630.77 & 0.64 & \bf{616.52} & 
1.37 & 2.31\\CON3-1 & 557.21 & 0.96 & 
558.14 & 0.70 & \bf{554.47} & 
0.49 & 0.66\\CON3-2 & 521.38 & 0.64 & 
521.38 & 0.69 & \bf{518.00} & 
0.65 & 0.65\\CON3-3 & 591.85 & 0.72 & 
600.02 & 0.66 & \bf{591.19} & 
0.11 & 1.49\\CON3-4 & 591.43 & 0.72 & 
593.19 & 0.69 & \bf{588.79} & 
0.45 & 0.75\\CON3-5 & 564.89 & 1.08 & 
564.89 & 0.80 & \bf{563.70} & 
0.21 & 0.21\\CON3-6 & 503.97 & 0.62 & 
507.27 & 0.76 & \bf{499.05} & 
0.99 & 1.65\\CON3-7 & 578.41 & 0.56 & 
584.11 & 0.68 & \bf{576.48} & 
0.33 & 1.32\\CON3-8 & 524.30 & 0.71 & 
526.51 & 0.68 & \bf{523.05} & 
0.24 & 0.66\\CON3-9 & 589.63 & 0.62 & 
592.24 & 0.61 & \bf{578.24} & 
1.97 & 2.42\\CON8-0 & 876.85 & 0.78 & 
885.65 & 0.78 & \bf{857.17} & 
2.30 & 3.32\\CON8-1 & 752.72 & 0.78 & 
754.84 & 0.82 & \bf{740.85} & 
1.60 & 1.89\\CON8-2 & 725.76 & 0.56 & 
725.76 & 0.69 & \bf{712.89} & 
1.81 & 1.81\\CON8-3 & 832.59 & 0.58 & 
832.59 & 0.77 & \bf{811.07} & 
2.65 & 2.65\\CON8-4 & 784.06 & 0.70 & 
793.60 & 0.60 & \bf{772.25} & 
1.53 & 2.76\\CON8-5 & 769.19 & 0.59 & 
769.19 & 0.72 & \bf{754.88} & 
1.90 & 1.90\\CON8-6 & 700.74 & 0.57 & 
706.10 & 0.73 & \bf{678.92} & 
3.21 & 4.00\\CON8-7 & 814.86 & 0.66 & 
815.33 & 0.72 & \bf{811.96} & 
0.36 & 0.41\\CON8-8 & 789.24 & 0.77 & 
789.75 & 0.71 & \bf{767.53} & 
2.83 & 2.90\\CON8-9 & 817.13 & 0.75 & 
827.88 & 0.70 & \bf{809.00} & 
1.00 & 2.33\\\bf{PROM.} & 
\bf{769.33} & \bf{0.67} & \bf{771.51} & \bf{0.69} & \bf{758.54} & \bf{1.30} & \bf{1.62}\\[1ex]\hline
\end{tabular}
\label{table:nonlin}
\end{table} \clearpage
\begin{table}[ht]
\caption{Resultados de la ejecución de la metaheurística IGA, utilizando instancias de SalhiNagy con la configuración -n 200 -p 40 -cprob 70.0 -mprob 20.0}
\centering
\small
\begin{tabular}{c c c c c c c c}
\hline\hline
Instancia & Costo mínimo & Tiempo(seg.) & Costo promedio & Tiempo promedio(seg.) & CME & \%G & \%GP \\ [0.5ex]
\hline
CMT1X & 478.97 & 0.63 & 
480.61 & 0.64 & \bf{470.48} & 
1.80 & 2.15\\CMT1Y & 475.05 & 0.60 & 
475.05 & 0.56 & \bf{470.48} & 
0.97 & 0.97\\CMT2X & 708.86 & 1.47 & 
716.95 & 1.42 & \bf{682.39} & 
3.88 & 5.06\\CMT2Y & 709.89 & 1.44 & 
717.26 & 1.40 & \bf{682.39} & 
4.03 & 5.11\\CMT3X & 745.37 & 2.70 & 
748.63 & 2.93 & \bf{719.06} & 
3.66 & 4.11\\CMT3Y & 749.78 & 2.97 & 
751.50 & 2.98 & \bf{719.06} & 
4.27 & 4.51\\CMT4X & 906.53 & 8.31 & 
916.49 & 7.83 & \bf{854.21} & 
6.12 & 7.29\\CMT4Y & 909.82 & 7.99 & 
921.52 & 8.27 & \bf{852.46} & 
6.73 & 8.10\\CMT5X & 1107.92 & 16.29 & 
1118.33 & 16.05 & \bf{1030.56} & 
7.51 & 8.52\\CMT5Y & 1123.76 & 17.82 & 
1135.09 & 16.75 & \bf{1031.69} & 
8.92 & 10.02\\CMT11X & 904.27 & 4.47 & 
906.41 & 5.11 & \bf{831.09} & 
8.81 & 9.06\\CMT11Y & 901.23 & 5.38 & 
902.92 & 5.33 & \bf{829.85} & 
8.60 & 8.81\\CMT12X & 676.81 & 3.05 & 
681.65 & 3.08 & \bf{658.83} & 
2.73 & 3.46\\CMT12Y & 673.85 & 3.04 & 
679.04 & 3.02 & \bf{660.47} & 
2.03 & 2.81\\\bf{PROM.} & 
\bf{790.86} & \bf{5.44} & \bf{796.53} & \bf{5.38} & \bf{749.50} & \bf{5.00} & \bf{5.71}\\[1ex]\hline
\end{tabular}
\label{table:nonlin}
\end{table} \clearpage
\begin{table}[ht]
\caption{Resultados de la ejecución de la metaheurística IGA, utilizando instancias de Dethloff con la configuración -n 200 -p 40 -cprob 70.0 -mprob 30.0}
\centering
\small
\begin{tabular}{c c c c c c c c}
\hline\hline
Instancia & Costo mínimo & Tiempo(seg.) & Costo promedio & Tiempo promedio(seg.) & CME & \%G & \%GP \\ [0.5ex]
\hline
SCA3-0 & 636.06 & 0.90 & 
639.43 & 0.77 & \bf{635.62} & 
0.07 & 0.60\\SCA3-1 & 701.74 & 0.56 & 
704.40 & 0.74 & \bf{697.84} & 
0.56 & 0.94\\SCA3-2 & 661.13 & 0.50 & 
661.13 & 0.64 & \bf{659.34} & 
0.27 & 0.27\\SCA3-3 & 682.46 & 0.61 & 
683.44 & 0.64 & \bf{680.04} & 
0.36 & 0.50\\SCA3-4 & \bf{690.50} & 0.53 & 
690.50 & 0.58 & 690.50 & 0.00
 & 0.00\\
SCA3-5 & 668.53 & 0.54 & 
669.88 & 0.62 & \bf{659.90} & 
1.31 & 1.51\\SCA3-6 & 652.94 & 0.70 & 
652.94 & 0.63 & \bf{651.09} & 
0.28 & 0.28\\SCA3-7 & 666.15 & 0.52 & 
666.15 & 0.55 & \bf{659.17} & 
1.06 & 1.06\\SCA3-8 & 724.29 & 0.57 & 
726.37 & 0.69 & \bf{719.47} & 
0.67 & 0.96\\SCA3-9 & \bf{681.00} & 0.86 & 
681.00 & 0.76 & 681.00 & 0.00
 & 0.00\\
SCA8-0 & 997.53 & 0.74 & 
1012.11 & 0.66 & \bf{961.50} & 
3.75 & 5.26\\SCA8-1 & 1072.48 & 0.84 & 
1073.52 & 0.72 & \bf{1049.65} & 
2.18 & 2.27\\SCA8-2 & 1051.42 & 0.84 & 
1051.42 & 0.74 & \bf{1039.64} & 
1.13 & 1.13\\SCA8-3 & 999.47 & 0.77 & 
1013.28 & 0.70 & \bf{983.34} & 
1.64 & 3.04\\SCA8-4 & 1074.81 & 0.89 & 
1086.86 & 0.68 & \bf{1065.49} & 
0.87 & 2.01\\SCA8-5 & 1038.93 & 0.76 & 
1039.13 & 0.72 & \bf{1027.08} & 
1.15 & 1.17\\SCA8-6 & 980.91 & 0.68 & 
980.91 & 0.64 & \bf{971.82} & 
0.94 & 0.94\\SCA8-7 & 1075.32 & 0.74 & 
1077.04 & 0.84 & \bf{1051.28} & 
2.29 & 2.45\\SCA8-8 & \bf{1071.18} & 0.71 & 
1071.18 & 0.75 & 1071.18 & 0.00
 & 0.00\\
SCA8-9 & 1067.42 & 0.88 & 
1088.78 & 0.82 & \bf{1060.50} & 
0.65 & 2.67\\CON3-0 & 620.76 & 0.92 & 
626.85 & 0.78 & \bf{616.52} & 
0.69 & 1.68\\CON3-1 & 561.87 & 0.44 & 
564.03 & 0.58 & \bf{554.47} & 
1.33 & 1.73\\CON3-2 & 524.13 & 0.90 & 
524.70 & 0.85 & \bf{518.00} & 
1.18 & 1.29\\CON3-3 & 601.66 & 0.73 & 
604.08 & 0.76 & \bf{591.19} & 
1.77 & 2.18\\CON3-4 & 594.59 & 0.74 & 
597.51 & 0.74 & \bf{588.79} & 
0.99 & 1.48\\CON3-5 & \bf{563.70} & 0.58 & 
563.70 & 0.66 & 563.70 & 0.00
 & 0.00\\
CON3-6 & 503.97 & 0.77 & 
503.97 & 0.73 & \bf{499.05} & 
0.99 & 0.99\\CON3-7 & 578.22 & 0.57 & 
583.10 & 0.68 & \bf{576.48} & 
0.30 & 1.15\\CON3-8 & 524.59 & 0.70 & 
527.15 & 0.71 & \bf{523.05} & 
0.29 & 0.78\\CON3-9 & 589.61 & 0.53 & 
591.70 & 0.63 & \bf{578.24} & 
1.97 & 2.33\\CON8-0 & 882.54 & 0.74 & 
882.54 & 0.72 & \bf{857.17} & 
2.96 & 2.96\\CON8-1 & 769.83 & 0.95 & 
772.26 & 0.87 & \bf{740.85} & 
3.91 & 4.24\\CON8-2 & 726.45 & 0.79 & 
730.34 & 0.80 & \bf{712.89} & 
1.90 & 2.45\\CON8-3 & 827.37 & 0.61 & 
827.37 & 0.61 & \bf{811.07} & 
2.01 & 2.01\\CON8-4 & 777.81 & 0.48 & 
777.81 & 0.59 & \bf{772.25} & 
0.72 & 0.72\\CON8-5 & 775.82 & 0.75 & 
777.60 & 0.74 & \bf{754.88} & 
2.77 & 3.01\\CON8-6 & 691.51 & 0.78 & 
691.51 & 0.81 & \bf{678.92} & 
1.85 & 1.85\\CON8-7 & 814.79 & 0.48 & 
814.81 & 0.59 & \bf{811.96} & 
0.35 & 0.35\\CON8-8 & 800.45 & 0.90 & 
800.45 & 0.90 & \bf{767.53} & 
4.29 & 4.29\\CON8-9 & 838.72 & 0.85 & 
844.53 & 0.88 & \bf{809.00} & 
3.67 & 4.39\\\bf{PROM.} & 
\bf{769.07} & \bf{0.71} & \bf{771.89} & \bf{0.71} & \bf{758.54} & \bf{1.33} & \bf{1.67}\\[1ex]\hline
\end{tabular}
\label{table:nonlin}
\end{table} \clearpage
\begin{table}[ht]
\caption{Resultados de la ejecución de la metaheurística IGA, utilizando instancias de SalhiNagy con la configuración -n 200 -p 40 -cprob 70.0 -mprob 30.0}
\centering
\small
\begin{tabular}{c c c c c c c c}
\hline\hline
Instancia & Costo mínimo & Tiempo(seg.) & Costo promedio & Tiempo promedio(seg.) & CME & \%G & \%GP \\ [0.5ex]
\hline
CMT1X & 479.65 & 0.61 & 
484.12 & 0.70 & \bf{470.48} & 
1.95 & 2.90\\CMT1Y & 485.91 & 0.47 & 
485.91 & 0.57 & \bf{470.48} & 
3.28 & 3.28\\CMT2X & 708.36 & 1.13 & 
714.55 & 1.24 & \bf{682.39} & 
3.81 & 4.71\\CMT2Y & 710.22 & 1.51 & 
713.23 & 1.32 & \bf{682.39} & 
4.08 & 4.52\\CMT3X & 737.54 & 3.09 & 
745.25 & 3.04 & \bf{719.06} & 
2.57 & 3.64\\CMT3Y & 735.19 & 2.83 & 
747.48 & 2.98 & \bf{719.06} & 
2.24 & 3.95\\CMT4X & 910.94 & 7.76 & 
922.61 & 7.99 & \bf{854.21} & 
6.64 & 8.01\\CMT4Y & 915.22 & 8.08 & 
918.74 & 8.10 & \bf{852.46} & 
7.36 & 7.78\\CMT5X & 1099.06 & 15.77 & 
1108.26 & 15.69 & \bf{1030.56} & 
6.65 & 7.54\\CMT5Y & 1117.45 & 17.51 & 
1128.24 & 16.84 & \bf{1031.69} & 
8.31 & 9.36\\CMT11X & 899.83 & 5.53 & 
924.18 & 5.53 & \bf{831.09} & 
8.27 & 11.20\\CMT11Y & 905.90 & 5.94 & 
907.55 & 5.83 & \bf{829.85} & 
9.16 & 9.36\\CMT12X & 674.01 & 3.50 & 
677.05 & 3.04 & \bf{658.83} & 
2.30 & 2.77\\CMT12Y & 674.76 & 2.64 & 
681.12 & 2.98 & \bf{660.47} & 
2.16 & 3.13\\\bf{PROM.} & 
\bf{789.57} & \bf{5.46} & \bf{797.02} & \bf{5.42} & \bf{749.50} & \bf{4.91} & \bf{5.87}\\[1ex]\hline
\end{tabular}
\label{table:nonlin}
\end{table} \clearpage
\begin{table}[ht]
\caption{Resultados de la ejecución de la metaheurística IGA, utilizando instancias de Dethloff con la configuración -n 200 -p 40 -cprob 70.0 -mprob 40.0}
\centering
\small
\begin{tabular}{c c c c c c c c}
\hline\hline
Instancia & Costo mínimo & Tiempo(seg.) & Costo promedio & Tiempo promedio(seg.) & CME & \%G & \%GP \\ [0.5ex]
\hline
SCA3-0 & 636.34 & 0.70 & 
639.01 & 0.70 & \bf{635.62} & 
0.11 & 0.53\\SCA3-1 & 701.53 & 0.78 & 
705.31 & 0.71 & \bf{697.84} & 
0.53 & 1.07\\SCA3-2 & 664.21 & 0.58 & 
665.89 & 0.59 & \bf{659.34} & 
0.74 & 0.99\\SCA3-3 & \bf{680.04} & 0.68 & 
680.46 & 0.69 & 680.04 & 0.00
 & 0.06\\SCA3-4 & \bf{690.50} & 0.45 & 
690.50 & 0.62 & 690.50 & 0.00
 & 0.00\\
SCA3-5 & 673.56 & 0.56 & 
673.56 & 0.69 & \bf{659.90} & 
2.07 & 2.07\\SCA3-6 & 652.94 & 0.94 & 
652.94 & 0.92 & \bf{651.09} & 
0.28 & 0.28\\SCA3-7 & 666.15 & 0.70 & 
666.75 & 0.80 & \bf{659.17} & 
1.06 & 1.15\\SCA3-8 & 726.44 & 0.69 & 
730.88 & 0.80 & \bf{719.47} & 
0.97 & 1.59\\SCA3-9 & 685.19 & 0.93 & 
685.36 & 0.72 & \bf{681.00} & 
0.62 & 0.64\\SCA8-0 & 990.65 & 0.74 & 
1001.71 & 0.76 & \bf{961.50} & 
3.03 & 4.18\\SCA8-1 & 1075.98 & 0.79 & 
1077.81 & 0.89 & \bf{1049.65} & 
2.51 & 2.68\\SCA8-2 & 1053.94 & 1.02 & 
1053.94 & 0.71 & \bf{1039.64} & 
1.38 & 1.38\\SCA8-3 & 1026.36 & 0.49 & 
1027.85 & 0.73 & \bf{983.34} & 
4.37 & 4.53\\SCA8-4 & \bf{\underline{85.00}} & Command & 
592.59 & 0.32 & 1065.49 & 
\bf{-92.02} & \bf{-44.38}\\SCA8-5 & 1045.31 & 0.54 & 
1066.00 & 0.76 & \bf{1027.08} & 
1.77 & 3.79\\SCA8-6 & 981.41 & 0.91 & 
988.96 & 0.70 & \bf{971.82} & 
0.99 & 1.76\\SCA8-7 & 1070.53 & 0.92 & 
1070.72 & 0.75 & \bf{1051.28} & 
1.83 & 1.85\\SCA8-8 & 1080.58 & 0.71 & 
1080.58 & 0.73 & \bf{1071.18} & 
0.88 & 0.88\\SCA8-9 & 1074.27 & 0.53 & 
1074.27 & 0.71 & \bf{1060.50} & 
1.30 & 1.30\\CON3-0 & 624.91 & 0.73 & 
626.32 & 0.68 & \bf{616.52} & 
1.36 & 1.59\\CON3-1 & 560.75 & 0.77 & 
560.75 & 0.75 & \bf{554.47} & 
1.13 & 1.13\\CON3-2 & 521.38 & 0.78 & 
521.38 & 0.77 & \bf{518.00} & 
0.65 & 0.65\\CON3-3 & 592.41 & 0.76 & 
594.22 & 0.80 & \bf{591.19} & 
0.21 & 0.51\\CON3-4 & 591.43 & 0.74 & 
597.25 & 0.62 & \bf{588.79} & 
0.45 & 1.44\\CON3-5 & 564.88 & 0.73 & 
567.48 & 0.71 & \bf{563.70} & 
0.21 & 0.67\\CON3-6 & 503.97 & 0.73 & 
508.43 & 0.80 & \bf{499.05} & 
0.99 & 1.88\\CON3-7 & 577.91 & 0.91 & 
583.48 & 0.85 & \bf{576.48} & 
0.25 & 1.21\\CON3-8 & 524.59 & 0.89 & 
524.59 & 0.89 & \bf{523.05} & 
0.29 & 0.29\\CON3-9 & 588.11 & 0.77 & 
590.46 & 0.84 & \bf{578.24} & 
1.71 & 2.11\\CON8-0 & 870.24 & 0.75 & 
870.24 & 0.66 & \bf{857.17} & 
1.52 & 1.52\\CON8-1 & 761.22 & 0.85 & 
765.68 & 0.77 & \bf{740.85} & 
2.75 & 3.35\\CON8-2 & 727.20 & 0.76 & 
729.91 & 0.77 & \bf{712.89} & 
2.01 & 2.39\\CON8-3 & 833.34 & 0.78 & 
834.66 & 0.69 & \bf{811.07} & 
2.75 & 2.91\\CON8-4 & 778.37 & 0.72 & 
791.71 & 0.70 & \bf{772.25} & 
0.79 & 2.52\\CON8-5 & 764.36 & 0.50 & 
764.36 & 0.53 & \bf{754.88} & 
1.26 & 1.26\\CON8-6 & 696.94 & 0.74 & 
700.89 & 0.76 & \bf{678.92} & 
2.65 & 3.24\\CON8-7 & 814.50 & 0.82 & 
818.23 & 0.73 & \bf{811.96} & 
0.31 & 0.77\\CON8-8 & 787.99 & 0.80 & 
787.99 & 0.79 & \bf{767.53} & 
2.67 & 2.67\\CON8-9 & 835.33 & 0.51 & 
839.29 & 0.79 & \bf{809.00} & 
3.25 & 3.74\\\bf{PROM.} & 
\bf{744.52} & \bf{0.72} & \bf{760.06} & \bf{0.73} & \bf{758.54} & \bf{-1.01} & \bf{0.56}\\[1ex]\hline
\end{tabular}
\label{table:nonlin}
\end{table} \clearpage
\begin{table}[ht]
\caption{Resultados de la ejecución de la metaheurística IGA, utilizando instancias de SalhiNagy con la configuración -n 200 -p 40 -cprob 70.0 -mprob 40.0}
\centering
\small
\begin{tabular}{c c c c c c c c}
\hline\hline
Instancia & Costo mínimo & Tiempo(seg.) & Costo promedio & Tiempo promedio(seg.) & CME & \%G & \%GP \\ [0.5ex]
\hline
CMT1X & 483.34 & 0.54 & 
483.74 & 0.60 & \bf{470.48} & 
2.73 & 2.82\\CMT1Y & 481.88 & 0.66 & 
482.71 & 0.68 & \bf{470.48} & 
2.42 & 2.60\\CMT2X & 690.99 & 1.39 & 
700.73 & 1.50 & \bf{682.39} & 
1.26 & 2.69\\CMT2Y & 710.47 & 1.39 & 
712.46 & 1.32 & \bf{682.39} & 
4.11 & 4.41\\CMT3X & 743.38 & 3.19 & 
747.38 & 2.88 & \bf{719.06} & 
3.38 & 3.94\\CMT3Y & 733.77 & 3.42 & 
743.28 & 2.98 & \bf{719.06} & 
2.05 & 3.37\\CMT4X & 914.35 & 8.41 & 
921.11 & 8.45 & \bf{854.21} & 
7.04 & 7.83\\CMT4Y & 903.84 & 8.98 & 
913.52 & 8.62 & \bf{852.46} & 
6.03 & 7.16\\CMT5X & 1096.58 & 16.15 & 
1121.01 & 16.50 & \bf{1030.56} & 
6.41 & 8.78\\CMT5Y & 1112.02 & 17.92 & 
1126.18 & 17.42 & \bf{1031.69} & 
7.79 & 9.16\\CMT11X & 897.06 & 5.25 & 
912.05 & 5.45 & \bf{831.09} & 
7.94 & 9.74\\CMT11Y & 896.68 & 4.92 & 
905.63 & 5.37 & \bf{829.85} & 
8.05 & 9.13\\CMT12X & 675.61 & 3.06 & 
681.64 & 3.04 & \bf{658.83} & 
2.55 & 3.46\\CMT12Y & 674.68 & 3.33 & 
677.67 & 2.94 & \bf{660.47} & 
2.15 & 2.60\\\bf{PROM.} & 
\bf{786.76} & \bf{5.62} & \bf{794.94} & \bf{5.55} & \bf{749.50} & \bf{4.56} & \bf{5.55}\\[1ex]\hline
\end{tabular}
\label{table:nonlin}
\end{table} \clearpage
\begin{table}[ht]
\caption{Resultados de la ejecución de la metaheurística IGA, utilizando instancias de Dethloff con la configuración -n 200 -p 40 -cprob 70.0 -mprob 50.0}
\centering
\small
\begin{tabular}{c c c c c c c c}
\hline\hline
Instancia & Costo mínimo & Tiempo(seg.) & Costo promedio & Tiempo promedio(seg.) & CME & \%G & \%GP \\ [0.5ex]
\hline
SCA3-0 & 640.55 & 0.58 & 
640.55 & 0.71 & \bf{635.62} & 
0.78 & 0.78\\SCA3-1 & 700.50 & 0.73 & 
700.50 & 0.84 & \bf{697.84} & 
0.38 & 0.38\\SCA3-2 & 666.01 & 0.69 & 
669.20 & 0.79 & \bf{659.34} & 
1.01 & 1.49\\SCA3-3 & \bf{680.04} & 0.90 & 
680.37 & 0.84 & 680.04 & 0.00
 & 0.05\\SCA3-4 & 693.23 & 0.74 & 
693.23 & 0.72 & \bf{690.50} & 
0.40 & 0.40\\SCA3-5 & 673.46 & 0.54 & 
673.51 & 0.68 & \bf{659.90} & 
2.05 & 2.06\\SCA3-6 & 652.94 & 0.56 & 
652.94 & 0.70 & \bf{651.09} & 
0.28 & 0.28\\SCA3-7 & 669.89 & 0.52 & 
670.83 & 0.64 & \bf{659.17} & 
1.63 & 1.77\\SCA3-8 & 723.99 & 0.53 & 
727.14 & 0.52 & \bf{719.47} & 
0.63 & 1.07\\SCA3-9 & 685.00 & 0.70 & 
686.15 & 0.78 & \bf{681.00} & 
0.59 & 0.76\\SCA8-0 & 999.80 & 1.00 & 
1003.63 & 0.83 & \bf{961.50} & 
3.98 & 4.38\\SCA8-1 & 1078.13 & 0.91 & 
1078.13 & 0.92 & \bf{1049.65} & 
2.71 & 2.71\\SCA8-2 & 1054.47 & 0.78 & 
1054.47 & 0.73 & \bf{1039.64} & 
1.43 & 1.43\\SCA8-3 & 1009.27 & 0.56 & 
1009.27 & 0.58 & \bf{983.34} & 
2.64 & 2.64\\SCA8-4 & 1080.08 & 0.74 & 
1082.11 & 0.75 & \bf{1065.49} & 
1.37 & 1.56\\SCA8-5 & 1062.66 & 0.74 & 
1062.66 & 0.72 & \bf{1027.08} & 
3.46 & 3.46\\SCA8-6 & 983.83 & 0.84 & 
994.22 & 0.91 & \bf{971.82} & 
1.24 & 2.31\\SCA8-7 & 1080.10 & 0.71 & 
1081.49 & 0.67 & \bf{1051.28} & 
2.74 & 2.87\\SCA8-8 & 1087.21 & 0.50 & 
1088.20 & 0.53 & \bf{1071.18} & 
1.50 & 1.59\\SCA8-9 & 1072.82 & 0.76 & 
1072.82 & 0.86 & \bf{1060.50} & 
1.16 & 1.16\\CON3-0 & 624.91 & 0.71 & 
631.90 & 0.65 & \bf{616.52} & 
1.36 & 2.49\\CON3-1 & 556.04 & 0.87 & 
562.46 & 0.73 & \bf{554.47} & 
0.28 & 1.44\\CON3-2 & 521.38 & 1.01 & 
521.38 & 0.80 & \bf{518.00} & 
0.65 & 0.65\\CON3-3 & 591.20 & 0.72 & 
601.40 & 0.78 & \bf{591.19} & 
0.00 & 1.73\\CON3-4 & 591.43 & 0.91 & 
591.43 & 0.66 & \bf{588.79} & 
0.45 & 0.45\\CON3-5 & 564.89 & 0.70 & 
566.41 & 0.98 & \bf{563.70} & 
0.21 & 0.48\\CON3-6 & 503.97 & 0.96 & 
505.94 & 0.83 & \bf{499.05} & 
0.99 & 1.38\\CON3-7 & 584.47 & 0.81 & 
587.36 & 0.68 & \bf{576.48} & 
1.39 & 1.89\\CON3-8 & 524.59 & 0.72 & 
529.43 & 0.74 & \bf{523.05} & 
0.29 & 1.22\\CON3-9 & 587.23 & 0.65 & 
589.34 & 0.72 & \bf{578.24} & 
1.55 & 1.92\\CON8-0 & 876.33 & 0.82 & 
876.33 & 0.73 & \bf{857.17} & 
2.24 & 2.24\\CON8-1 & 755.84 & 0.99 & 
768.87 & 0.91 & \bf{740.85} & 
2.02 & 3.78\\CON8-2 & 720.69 & 0.94 & 
730.87 & 0.84 & \bf{712.89} & 
1.09 & 2.52\\CON8-3 & 819.66 & 0.65 & 
833.88 & 0.59 & \bf{811.07} & 
1.06 & 2.81\\CON8-4 & 792.48 & 0.60 & 
802.12 & 0.73 & \bf{772.25} & 
2.62 & 3.87\\CON8-5 & 770.99 & 0.82 & 
776.98 & 0.73 & \bf{754.88} & 
2.13 & 2.93\\CON8-6 & 695.43 & 0.75 & 
695.43 & 0.86 & \bf{678.92} & 
2.43 & 2.43\\CON8-7 & 826.80 & 0.77 & 
827.35 & 0.89 & \bf{811.96} & 
1.83 & 1.89\\CON8-8 & 797.45 & 0.79 & 
799.57 & 0.78 & \bf{767.53} & 
3.90 & 4.17\\CON8-9 & 818.21 & 0.76 & 
829.90 & 0.80 & \bf{809.00} & 
1.14 & 2.58\\\bf{PROM.} & 
\bf{770.45} & \bf{0.75} & \bf{773.74} & \bf{0.75} & \bf{758.54} & \bf{1.44} & \bf{1.90}\\[1ex]\hline
\end{tabular}
\label{table:nonlin}
\end{table} \clearpage
\begin{table}[ht]
\caption{Resultados de la ejecución de la metaheurística IGA, utilizando instancias de SalhiNagy con la configuración -n 200 -p 40 -cprob 70.0 -mprob 50.0}
\centering
\small
\begin{tabular}{c c c c c c c c}
\hline\hline
Instancia & Costo mínimo & Tiempo(seg.) & Costo promedio & Tiempo promedio(seg.) & CME & \%G & \%GP \\ [0.5ex]
\hline
CMT1X & 478.38 & 0.65 & 
482.50 & 0.77 & \bf{470.48} & 
1.68 & 2.55\\CMT1Y & 478.97 & 0.80 & 
486.60 & 0.68 & \bf{470.48} & 
1.80 & 3.43\\CMT2X & 714.14 & 1.43 & 
716.68 & 1.46 & \bf{682.39} & 
4.65 & 5.03\\CMT2Y & 695.12 & 0.99 & 
698.25 & 1.22 & \bf{682.39} & 
1.87 & 2.32\\CMT3X & 738.11 & 3.07 & 
744.18 & 3.14 & \bf{719.06} & 
2.65 & 3.49\\CMT3Y & 751.02 & 2.70 & 
751.48 & 2.93 & \bf{719.06} & 
4.44 & 4.51\\CMT4X & 911.64 & 7.97 & 
917.00 & 8.56 & \bf{854.21} & 
6.72 & 7.35\\CMT4Y & 891.02 & 8.78 & 
902.36 & 8.23 & \bf{852.46} & 
4.52 & 5.85\\CMT5X & 1101.47 & 17.01 & 
1114.20 & 17.01 & \bf{1030.56} & 
6.88 & 8.12\\CMT5Y & 1100.20 & 16.94 & 
1119.03 & 17.10 & \bf{1031.69} & 
6.64 & 8.47\\CMT11X & 906.17 & 4.83 & 
909.38 & 4.55 & \bf{831.09} & 
9.03 & 9.42\\CMT11Y & 860.56 & 5.85 & 
891.00 & 5.84 & \bf{829.85} & 
3.70 & 7.37\\CMT12X & 683.51 & 3.08 & 
685.88 & 3.27 & \bf{658.83} & 
3.75 & 4.11\\CMT12Y & 674.85 & 2.94 & 
679.81 & 2.94 & \bf{660.47} & 
2.18 & 2.93\\\bf{PROM.} & 
\bf{784.65} & \bf{5.50} & \bf{792.74} & \bf{5.55} & \bf{749.50} & \bf{4.32} & \bf{5.35}\\[1ex]\hline
\end{tabular}
\label{table:nonlin}
\end{table} \clearpage
\begin{table}[ht]
\caption{Resultados de la ejecución de la metaheurística IGA, utilizando instancias de Dethloff con la configuración -n 200 -p 40 -cprob 70.0 -mprob 60.0}
\centering
\small
\begin{tabular}{c c c c c c c c}
\hline\hline
Instancia & Costo mínimo & Tiempo(seg.) & Costo promedio & Tiempo promedio(seg.) & CME & \%G & \%GP \\ [0.5ex]
\hline
SCA3-0 & 640.55 & 0.96 & 
640.84 & 0.95 & \bf{635.62} & 
0.78 & 0.82\\SCA3-1 & \bf{697.84} & 0.54 & 
699.68 & 0.84 & 697.84 & 0.00
 & 0.26\\SCA3-2 & 661.13 & 0.90 & 
663.29 & 0.77 & \bf{659.34} & 
0.27 & 0.60\\SCA3-3 & 681.35 & 0.54 & 
685.06 & 0.68 & \bf{680.04} & 
0.19 & 0.74\\SCA3-4 & \bf{690.50} & 0.95 & 
690.50 & 0.74 & 690.50 & 0.00
 & 0.00\\
SCA3-5 & 665.64 & 0.57 & 
666.35 & 0.74 & \bf{659.90} & 
0.87 & 0.98\\SCA3-6 & 657.00 & 0.47 & 
657.71 & 0.68 & \bf{651.09} & 
0.91 & 1.02\\SCA3-7 & 666.15 & 0.71 & 
667.53 & 0.71 & \bf{659.17} & 
1.06 & 1.27\\SCA3-8 & 726.88 & 0.94 & 
729.46 & 0.82 & \bf{719.47} & 
1.03 & 1.39\\SCA3-9 & \bf{681.00} & 0.90 & 
682.48 & 0.72 & 681.00 & 0.00
 & 0.22\\SCA8-0 & 981.73 & 0.58 & 
990.50 & 0.72 & \bf{961.50} & 
2.10 & 3.02\\SCA8-1 & 1087.01 & 0.92 & 
1087.01 & 0.90 & \bf{1049.65} & 
3.56 & 3.56\\SCA8-2 & 1050.17 & 0.67 & 
1057.26 & 0.75 & \bf{1039.64} & 
1.01 & 1.69\\SCA8-3 & 1031.19 & 0.55 & 
1038.47 & 0.82 & \bf{983.34} & 
4.87 & 5.61\\SCA8-4 & 1074.10 & 0.79 & 
1082.70 & 0.74 & \bf{1065.49} & 
0.81 & 1.62\\SCA8-5 & 1059.42 & 0.72 & 
1063.97 & 0.68 & \bf{1027.08} & 
3.15 & 3.59\\SCA8-6 & 996.91 & 0.79 & 
996.91 & 0.77 & \bf{971.82} & 
2.58 & 2.58\\SCA8-7 & 1070.53 & 0.56 & 
1074.73 & 0.70 & \bf{1051.28} & 
1.83 & 2.23\\SCA8-8 & \bf{1071.18} & 0.46 & 
1071.18 & 0.50 & 1071.18 & 0.00
 & 0.00\\
SCA8-9 & 1088.43 & 0.56 & 
1088.43 & 0.70 & \bf{1060.50} & 
2.63 & 2.63\\CON3-0 & 620.76 & 0.75 & 
623.34 & 0.74 & \bf{616.52} & 
0.69 & 1.11\\CON3-1 & 557.38 & 0.67 & 
561.48 & 0.73 & \bf{554.47} & 
0.52 & 1.26\\CON3-2 & 521.38 & 0.72 & 
521.38 & 0.78 & \bf{518.00} & 
0.65 & 0.65\\CON3-3 & 605.30 & 0.86 & 
609.19 & 0.71 & \bf{591.19} & 
2.39 & 3.05\\CON3-4 & 591.43 & 0.92 & 
592.29 & 0.74 & \bf{588.79} & 
0.45 & 0.59\\CON3-5 & 567.94 & 0.75 & 
568.63 & 0.74 & \bf{563.70} & 
0.75 & 0.87\\CON3-6 & 501.33 & 0.72 & 
504.12 & 0.79 & \bf{499.05} & 
0.46 & 1.01\\CON3-7 & 577.91 & 0.93 & 
581.12 & 0.78 & \bf{576.48} & 
0.25 & 0.80\\CON3-8 & 523.14 & 0.81 & 
523.14 & 0.75 & \bf{523.05} & 
0.02 & 0.02\\CON3-9 & 588.28 & 0.48 & 
588.34 & 0.58 & \bf{578.24} & 
1.74 & 1.75\\CON8-0 & 874.68 & 0.59 & 
874.68 & 0.74 & \bf{857.17} & 
2.04 & 2.04\\CON8-1 & 764.11 & 1.02 & 
764.11 & 0.79 & \bf{740.85} & 
3.14 & 3.14\\CON8-2 & 721.91 & 1.01 & 
728.93 & 0.85 & \bf{712.89} & 
1.27 & 2.25\\CON8-3 & 830.45 & 0.82 & 
830.45 & 0.76 & \bf{811.07} & 
2.39 & 2.39\\CON8-4 & 801.55 & 0.54 & 
807.09 & 0.57 & \bf{772.25} & 
3.79 & 4.51\\CON8-5 & 767.81 & 0.86 & 
768.85 & 0.83 & \bf{754.88} & 
1.71 & 1.85\\CON8-6 & 700.73 & 0.75 & 
702.71 & 0.91 & \bf{678.92} & 
3.21 & 3.50\\CON8-7 & 826.17 & 0.79 & 
826.17 & 0.71 & \bf{811.96} & 
1.75 & 1.75\\CON8-8 & 783.47 & 0.87 & 
783.47 & 0.84 & \bf{767.53} & 
2.08 & 2.08\\CON8-9 & 822.74 & 1.05 & 
835.27 & 0.81 & \bf{809.00} & 
1.70 & 3.25\\\bf{PROM.} & 
\bf{770.68} & \bf{0.75} & \bf{773.22} & \bf{0.75} & \bf{758.54} & \bf{1.47} & \bf{1.79}\\[1ex]\hline
\end{tabular}
\label{table:nonlin}
\end{table} \clearpage
\begin{table}[ht]
\caption{Resultados de la ejecución de la metaheurística IGA, utilizando instancias de SalhiNagy con la configuración -n 200 -p 40 -cprob 70.0 -mprob 60.0}
\centering
\small
\begin{tabular}{c c c c c c c c}
\hline\hline
Instancia & Costo mínimo & Tiempo(seg.) & Costo promedio & Tiempo promedio(seg.) & CME & \%G & \%GP \\ [0.5ex]
\hline
CMT1X & 475.71 & 0.56 & 
477.90 & 0.56 & \bf{470.48} & 
1.11 & 1.58\\CMT1Y & 479.45 & 0.46 & 
484.18 & 0.58 & \bf{470.48} & 
1.91 & 2.91\\CMT2X & 702.35 & 1.47 & 
706.60 & 1.53 & \bf{682.39} & 
2.93 & 3.55\\CMT2Y & 701.65 & 1.31 & 
706.51 & 1.50 & \bf{682.39} & 
2.82 & 3.53\\CMT3X & 729.52 & 2.80 & 
743.22 & 3.00 & \bf{719.06} & 
1.45 & 3.36\\CMT3Y & 743.82 & 3.26 & 
746.57 & 3.18 & \bf{719.06} & 
3.44 & 3.83\\CMT4X & 902.94 & 8.30 & 
910.25 & 8.03 & \bf{854.21} & 
5.70 & 6.56\\CMT4Y & 890.60 & 9.16 & 
905.08 & 8.50 & \bf{852.46} & 
4.47 & 6.17\\CMT5X & 1121.08 & 15.66 & 
1129.16 & 15.95 & \bf{1030.56} & 
8.78 & 9.57\\CMT5Y & 1107.62 & 18.48 & 
1130.43 & 17.50 & \bf{1031.69} & 
7.36 & 9.57\\CMT11X & 919.94 & 4.70 & 
923.54 & 5.51 & \bf{831.09} & 
10.69 & 11.12\\CMT11Y & 889.57 & 6.34 & 
904.27 & 5.78 & \bf{829.85} & 
7.20 & 8.97\\CMT12X & 678.11 & 3.39 & 
681.41 & 3.16 & \bf{658.83} & 
2.93 & 3.43\\CMT12Y & 674.92 & 2.74 & 
675.74 & 2.89 & \bf{660.47} & 
2.19 & 2.31\\\bf{PROM.} & 
\bf{786.95} & \bf{5.62} & \bf{794.63} & \bf{5.55} & \bf{749.50} & \bf{4.50} & \bf{5.46}\\[1ex]\hline
\end{tabular}
\label{table:nonlin}
\end{table} \clearpage
\begin{table}[ht]
\caption{Resultados de la ejecución de la metaheurística IGA, utilizando instancias de Dethloff con la configuración -n 200 -p 40 -cprob 70.0 -mprob 70.0}
\centering
\small
\begin{tabular}{c c c c c c c c}
\hline\hline
Instancia & Costo mínimo & Tiempo(seg.) & Costo promedio & Tiempo promedio(seg.) & CME & \%G & \%GP \\ [0.5ex]
\hline
SCA3-0 & 640.55 & 0.91 & 
640.55 & 0.86 & \bf{635.62} & 
0.78 & 0.78\\SCA3-1 & 700.50 & 0.64 & 
701.14 & 0.78 & \bf{697.84} & 
0.38 & 0.47\\SCA3-2 & \bf{659.34} & 0.87 & 
665.75 & 0.80 & 659.34 & 0.00
 & 0.97\\SCA3-3 & 681.35 & 0.78 & 
685.09 & 0.78 & \bf{680.04} & 
0.19 & 0.74\\SCA3-4 & \bf{690.50} & 0.52 & 
690.50 & 0.65 & 690.50 & 0.00
 & 0.00\\
SCA3-5 & 670.10 & 0.66 & 
670.10 & 0.76 & \bf{659.90} & 
1.55 & 1.55\\SCA3-6 & 652.94 & 0.97 & 
652.94 & 0.91 & \bf{651.09} & 
0.28 & 0.28\\SCA3-7 & 666.60 & 0.93 & 
666.60 & 0.80 & \bf{659.17} & 
1.13 & 1.13\\SCA3-8 & 719.77 & 0.65 & 
721.28 & 0.84 & \bf{719.47} & 
0.04 & 0.25\\SCA3-9 & \bf{681.00} & 0.64 & 
681.00 & 0.64 & 681.00 & 0.00
 & 0.00\\
SCA8-0 & 996.48 & 0.65 & 
996.48 & 0.61 & \bf{961.50} & 
3.64 & 3.64\\SCA8-1 & 1076.86 & 0.98 & 
1076.86 & 0.97 & \bf{1049.65} & 
2.59 & 2.59\\SCA8-2 & 1054.47 & 0.86 & 
1054.47 & 0.68 & \bf{1039.64} & 
1.43 & 1.43\\SCA8-3 & 1026.45 & 0.76 & 
1026.45 & 0.75 & \bf{983.34} & 
4.38 & 4.38\\SCA8-4 & 1075.81 & 0.81 & 
1077.49 & 0.76 & \bf{1065.49} & 
0.97 & 1.13\\SCA8-5 & 1064.74 & 0.76 & 
1068.90 & 0.73 & \bf{1027.08} & 
3.67 & 4.07\\SCA8-6 & 972.48 & 0.96 & 
972.48 & 0.90 & \bf{971.82} & 
0.07 & 0.07\\SCA8-7 & 1071.53 & 0.88 & 
1081.18 & 0.78 & \bf{1051.28} & 
1.93 & 2.84\\SCA8-8 & 1096.70 & 0.62 & 
1096.70 & 0.67 & \bf{1071.18} & 
2.38 & 2.38\\SCA8-9 & 1067.42 & 0.94 & 
1067.42 & 0.68 & \bf{1060.50} & 
0.65 & 0.65\\CON3-0 & 620.49 & 0.77 & 
620.62 & 0.85 & \bf{616.52} & 
0.64 & 0.67\\CON3-1 & 556.92 & 0.89 & 
559.79 & 0.84 & \bf{554.47} & 
0.44 & 0.96\\CON3-2 & 521.38 & 0.96 & 
521.75 & 0.97 & \bf{518.00} & 
0.65 & 0.72\\CON3-3 & 591.36 & 0.91 & 
598.29 & 0.77 & \bf{591.19} & 
0.03 & 1.20\\CON3-4 & 594.59 & 0.68 & 
598.17 & 0.67 & \bf{588.79} & 
0.99 & 1.59\\CON3-5 & 567.94 & 0.78 & 
569.19 & 0.84 & \bf{563.70} & 
0.75 & 0.97\\CON3-6 & 508.68 & 0.98 & 
508.81 & 0.95 & \bf{499.05} & 
1.93 & 1.96\\CON3-7 & 578.41 & 0.90 & 
580.37 & 0.83 & \bf{576.48} & 
0.33 & 0.67\\CON3-8 & 523.14 & 0.61 & 
524.30 & 0.68 & \bf{523.05} & 
0.02 & 0.24\\CON3-9 & 590.29 & 0.72 & 
590.48 & 0.76 & \bf{578.24} & 
2.08 & 2.12\\CON8-0 & 873.88 & 0.99 & 
876.58 & 0.93 & \bf{857.17} & 
1.95 & 2.26\\CON8-1 & 754.65 & 0.74 & 
758.35 & 0.85 & \bf{740.85} & 
1.86 & 2.36\\CON8-2 & 713.44 & 0.53 & 
721.24 & 0.77 & \bf{712.89} & 
0.08 & 1.17\\CON8-3 & 825.52 & 1.04 & 
830.35 & 0.86 & \bf{811.07} & 
1.78 & 2.38\\CON8-4 & 787.74 & 0.92 & 
787.74 & 0.82 & \bf{772.25} & 
2.01 & 2.01\\CON8-5 & 761.01 & 1.00 & 
773.63 & 0.81 & \bf{754.88} & 
0.81 & 2.48\\CON8-6 & 704.70 & 0.76 & 
704.70 & 0.87 & \bf{678.92} & 
3.80 & 3.80\\CON8-7 & 815.84 & 0.79 & 
819.87 & 0.79 & \bf{811.96} & 
0.48 & 0.97\\CON8-8 & 797.66 & 0.80 & 
801.37 & 0.92 & \bf{767.53} & 
3.93 & 4.41\\CON8-9 & 837.45 & 0.96 & 
837.45 & 0.90 & \bf{809.00} & 
3.52 & 3.52\\\bf{PROM.} & 
\bf{769.77} & \bf{0.81} & \bf{771.91} & \bf{0.80} & \bf{758.54} & \bf{1.35} & \bf{1.65}\\[1ex]\hline
\end{tabular}
\label{table:nonlin}
\end{table} \clearpage
\begin{table}[ht]
\caption{Resultados de la ejecución de la metaheurística IGA, utilizando instancias de SalhiNagy con la configuración -n 200 -p 40 -cprob 70.0 -mprob 70.0}
\centering
\small
\begin{tabular}{c c c c c c c c}
\hline\hline
Instancia & Costo mínimo & Tiempo(seg.) & Costo promedio & Tiempo promedio(seg.) & CME & \%G & \%GP \\ [0.5ex]
\hline
CMT1X & 480.06 & 0.80 & 
480.06 & 0.83 & \bf{470.48} & 
2.04 & 2.04\\CMT1Y & 475.72 & 0.58 & 
480.87 & 0.54 & \bf{470.48} & 
1.11 & 2.21\\CMT2X & 715.03 & 1.14 & 
717.55 & 1.32 & \bf{682.39} & 
4.78 & 5.15\\CMT2Y & 703.94 & 1.40 & 
713.13 & 1.43 & \bf{682.39} & 
3.16 & 4.51\\CMT3X & 730.97 & 3.44 & 
741.43 & 3.30 & \bf{719.06} & 
1.66 & 3.11\\CMT3Y & 736.84 & 2.68 & 
745.73 & 2.85 & \bf{719.06} & 
2.47 & 3.71\\CMT4X & 903.09 & 7.99 & 
914.35 & 8.19 & \bf{854.21} & 
5.72 & 7.04\\CMT4Y & 906.40 & 9.14 & 
914.88 & 8.70 & \bf{852.46} & 
6.33 & 7.32\\CMT5X & 1091.83 & 15.74 & 
1117.59 & 16.23 & \bf{1030.56} & 
5.95 & 8.44\\CMT5Y & 1101.92 & 16.76 & 
1125.47 & 16.52 & \bf{1031.69} & 
6.81 & 9.09\\CMT11X & 912.75 & 5.15 & 
928.60 & 5.15 & \bf{831.09} & 
9.83 & 11.73\\CMT11Y & 880.77 & 6.03 & 
909.54 & 5.95 & \bf{829.85} & 
6.14 & 9.60\\CMT12X & 675.26 & 2.81 & 
679.35 & 3.30 & \bf{658.83} & 
2.49 & 3.11\\CMT12Y & 674.92 & 3.42 & 
680.18 & 3.33 & \bf{660.47} & 
2.19 & 2.98\\\bf{PROM.} & 
\bf{784.96} & \bf{5.51} & \bf{796.34} & \bf{5.55} & \bf{749.50} & \bf{4.33} & \bf{5.72}\\[1ex]\hline
\end{tabular}
\label{table:nonlin}
\end{table} \clearpage
\begin{table}[ht]
\caption{Resultados de la ejecución de la metaheurística IGA, utilizando instancias de Dethloff con la configuración -n 200 -p 40 -cprob 70.0 -mprob 80.0}
\centering
\small
\begin{tabular}{c c c c c c c c}
\hline\hline
Instancia & Costo mínimo & Tiempo(seg.) & Costo promedio & Tiempo promedio(seg.) & CME & \%G & \%GP \\ [0.5ex]
\hline
SCA3-0 & 640.55 & 0.73 & 
640.55 & 0.83 & \bf{635.62} & 
0.78 & 0.78\\SCA3-1 & \bf{697.84} & 0.92 & 
700.61 & 0.76 & 697.84 & 0.00
 & 0.40\\SCA3-2 & 661.13 & 0.88 & 
667.08 & 0.74 & \bf{659.34} & 
0.27 & 1.17\\SCA3-3 & \bf{680.04} & 0.93 & 
680.93 & 0.92 & 680.04 & 0.00
 & 0.13\\SCA3-4 & \bf{690.50} & 0.57 & 
690.50 & 0.75 & 690.50 & 0.00
 & 0.00\\
SCA3-5 & 666.67 & 0.98 & 
673.09 & 0.89 & \bf{659.90} & 
1.03 & 2.00\\SCA3-6 & 652.94 & 0.92 & 
655.48 & 0.74 & \bf{651.09} & 
0.28 & 0.67\\SCA3-7 & 666.15 & 0.46 & 
666.15 & 0.79 & \bf{659.17} & 
1.06 & 1.06\\SCA3-8 & \bf{719.47} & 0.95 & 
722.70 & 0.88 & 719.47 & 0.00
 & 0.45\\SCA3-9 & \bf{681.00} & 0.78 & 
685.09 & 0.84 & 681.00 & 0.00
 & 0.60\\SCA8-0 & 1009.03 & 0.98 & 
1009.03 & 0.98 & \bf{961.50} & 
4.94 & 4.94\\SCA8-1 & 1071.59 & 0.94 & 
1073.75 & 0.87 & \bf{1049.65} & 
2.09 & 2.30\\SCA8-2 & 1053.78 & 1.00 & 
1054.05 & 0.86 & \bf{1039.64} & 
1.36 & 1.39\\SCA8-3 & 1020.03 & 0.64 & 
1027.75 & 0.87 & \bf{983.34} & 
3.73 & 4.52\\SCA8-4 & 1074.81 & 0.81 & 
1074.81 & 0.72 & \bf{1065.49} & 
0.87 & 0.87\\SCA8-5 & 1042.30 & 0.98 & 
1054.65 & 0.82 & \bf{1027.08} & 
1.48 & 2.68\\SCA8-6 & 982.42 & 0.56 & 
982.42 & 0.63 & \bf{971.82} & 
1.09 & 1.09\\SCA8-7 & 1066.82 & 0.70 & 
1066.82 & 0.78 & \bf{1051.28} & 
1.48 & 1.48\\SCA8-8 & 1088.65 & 1.11 & 
1088.65 & 0.96 & \bf{1071.18} & 
1.63 & 1.63\\SCA8-9 & 1068.10 & 0.87 & 
1082.70 & 0.85 & \bf{1060.50} & 
0.72 & 2.09\\CON3-0 & 633.86 & 0.76 & 
634.65 & 0.77 & \bf{616.52} & 
2.81 & 2.94\\CON3-1 & 556.04 & 0.74 & 
557.73 & 0.81 & \bf{554.47} & 
0.28 & 0.59\\CON3-2 & 521.38 & 0.68 & 
522.68 & 0.76 & \bf{518.00} & 
0.65 & 0.90\\CON3-3 & 594.10 & 0.92 & 
601.66 & 0.78 & \bf{591.19} & 
0.49 & 1.77\\CON3-4 & 599.13 & 0.92 & 
602.10 & 0.64 & \bf{588.79} & 
1.76 & 2.26\\CON3-5 & 564.88 & 0.94 & 
567.38 & 0.87 & \bf{563.70} & 
0.21 & 0.65\\CON3-6 & 502.26 & 0.79 & 
505.44 & 0.81 & \bf{499.05} & 
0.64 & 1.28\\CON3-7 & 578.41 & 0.91 & 
580.37 & 0.85 & \bf{576.48} & 
0.33 & 0.67\\CON3-8 & 526.59 & 0.60 & 
529.73 & 0.69 & \bf{523.05} & 
0.68 & 1.28\\CON3-9 & 582.79 & 0.96 & 
585.59 & 0.80 & \bf{578.24} & 
0.79 & 1.27\\CON8-0 & 873.31 & 0.76 & 
873.31 & 0.86 & \bf{857.17} & 
1.88 & 1.88\\CON8-1 & 758.07 & 1.03 & 
761.14 & 0.90 & \bf{740.85} & 
2.32 & 2.74\\CON8-2 & 719.10 & 0.80 & 
719.10 & 0.85 & \bf{712.89} & 
0.87 & 0.87\\CON8-3 & 835.72 & 0.72 & 
835.72 & 0.74 & \bf{811.07} & 
3.04 & 3.04\\CON8-4 & 782.11 & 0.53 & 
787.04 & 0.63 & \bf{772.25} & 
1.28 & 1.92\\CON8-5 & 756.91 & 0.98 & 
756.91 & 0.79 & \bf{754.88} & 
0.27 & 0.27\\CON8-6 & 695.54 & 0.87 & 
695.54 & 0.88 & \bf{678.92} & 
2.45 & 2.45\\CON8-7 & 815.79 & 0.67 & 
815.88 & 0.77 & \bf{811.96} & 
0.47 & 0.48\\CON8-8 & 792.07 & 0.50 & 
792.55 & 0.71 & \bf{767.53} & 
3.20 & 3.26\\CON8-9 & 811.14 & 0.96 & 
811.14 & 0.80 & \bf{809.00} & 
0.26 & 0.26\\\bf{PROM.} & 
\bf{768.33} & \bf{0.82} & \bf{770.81} & \bf{0.80} & \bf{758.54} & \bf{1.19} & \bf{1.53}\\[1ex]\hline
\end{tabular}
\label{table:nonlin}
\end{table} \clearpage
\begin{table}[ht]
\caption{Resultados de la ejecución de la metaheurística IGA, utilizando instancias de SalhiNagy con la configuración -n 200 -p 40 -cprob 70.0 -mprob 80.0}
\centering
\small
\begin{tabular}{c c c c c c c c}
\hline\hline
Instancia & Costo mínimo & Tiempo(seg.) & Costo promedio & Tiempo promedio(seg.) & CME & \%G & \%GP \\ [0.5ex]
\hline
CMT1X & 481.52 & 0.86 & 
481.84 & 0.59 & \bf{470.48} & 
2.35 & 2.42\\CMT1Y & 483.50 & 0.48 & 
487.81 & 0.55 & \bf{470.48} & 
2.77 & 3.68\\CMT2X & 717.56 & 1.74 & 
721.80 & 1.60 & \bf{682.39} & 
5.15 & 5.78\\CMT2Y & 705.07 & 1.96 & 
714.47 & 1.65 & \bf{682.39} & 
3.32 & 4.70\\CMT3X & 735.06 & 3.40 & 
737.68 & 3.08 & \bf{719.06} & 
2.23 & 2.59\\CMT3Y & 739.64 & 3.04 & 
745.65 & 3.19 & \bf{719.06} & 
2.86 & 3.70\\CMT4X & 880.81 & 8.03 & 
906.94 & 8.05 & \bf{854.21} & 
3.11 & 6.17\\CMT4Y & 909.34 & 9.18 & 
919.27 & 9.02 & \bf{852.46} & 
6.67 & 7.84\\CMT5X & 1097.02 & 17.66 & 
1111.45 & 16.64 & \bf{1030.56} & 
6.45 & 7.85\\CMT5Y & 1098.88 & 16.70 & 
1115.55 & 16.56 & \bf{1031.69} & 
6.51 & 8.13\\CMT11X & 898.10 & 5.41 & 
923.88 & 5.08 & \bf{831.09} & 
8.06 & 11.17\\CMT11Y & 894.65 & 6.43 & 
922.43 & 6.37 & \bf{829.85} & 
7.81 & 11.16\\CMT12X & 674.74 & 3.27 & 
682.17 & 3.40 & \bf{658.83} & 
2.41 & 3.54\\CMT12Y & 674.98 & 2.64 & 
675.12 & 2.94 & \bf{660.47} & 
2.20 & 2.22\\\bf{PROM.} & 
\bf{785.06} & \bf{5.77} & \bf{796.15} & \bf{5.62} & \bf{749.50} & \bf{4.42} & \bf{5.78}\\[1ex]\hline
\end{tabular}
\label{table:nonlin}
\end{table} \clearpage
\begin{table}[ht]
\caption{Resultados de la ejecución de la metaheurística IGA, utilizando instancias de Dethloff con la configuración -n 200 -p 40 -cprob 70.0 -mprob 90.0}
\centering
\small
\begin{tabular}{c c c c c c c c}
\hline\hline
Instancia & Costo mínimo & Tiempo(seg.) & Costo promedio & Tiempo promedio(seg.) & CME & \%G & \%GP \\ [0.5ex]
\hline
SCA3-0 & 640.55 & 0.94 & 
641.12 & 0.79 & \bf{635.62} & 
0.78 & 0.87\\SCA3-1 & 700.50 & 0.89 & 
701.66 & 0.83 & \bf{697.84} & 
0.38 & 0.55\\SCA3-2 & 669.06 & 0.90 & 
669.20 & 0.90 & \bf{659.34} & 
1.47 & 1.49\\SCA3-3 & 681.35 & 0.65 & 
681.35 & 0.72 & \bf{680.04} & 
0.19 & 0.19\\SCA3-4 & 692.57 & 0.88 & 
693.58 & 0.75 & \bf{690.50} & 
0.30 & 0.45\\SCA3-5 & \bf{659.90} & 0.70 & 
670.10 & 0.80 & 659.90 & 0.00
 & 1.54\\SCA3-6 & 652.94 & 0.83 & 
653.80 & 0.91 & \bf{651.09} & 
0.28 & 0.42\\SCA3-7 & 666.60 & 0.68 & 
670.48 & 0.70 & \bf{659.17} & 
1.13 & 1.72\\SCA3-8 & 726.44 & 0.95 & 
732.57 & 0.93 & \bf{719.47} & 
0.97 & 1.82\\SCA3-9 & 685.14 & 0.86 & 
685.16 & 0.77 & \bf{681.00} & 
0.61 & 0.61\\SCA8-0 & 994.37 & 0.82 & 
994.37 & 0.87 & \bf{961.50} & 
3.42 & 3.42\\SCA8-1 & 1073.68 & 0.76 & 
1073.71 & 0.66 & \bf{1049.65} & 
2.29 & 2.29\\SCA8-2 & 1060.15 & 0.94 & 
1060.15 & 0.94 & \bf{1039.64} & 
1.97 & 1.97\\SCA8-3 & 1032.46 & 0.97 & 
1032.46 & 0.87 & \bf{983.34} & 
5.00 & 5.00\\SCA8-4 & 1075.27 & 0.72 & 
1075.27 & 0.73 & \bf{1065.49} & 
0.92 & 0.92\\SCA8-5 & 1054.70 & 0.86 & 
1060.79 & 0.89 & \bf{1027.08} & 
2.69 & 3.28\\SCA8-6 & 989.39 & 0.82 & 
989.39 & 0.74 & \bf{971.82} & 
1.81 & 1.81\\SCA8-7 & 1071.47 & 0.88 & 
1071.47 & 0.81 & \bf{1051.28} & 
1.92 & 1.92\\SCA8-8 & 1084.74 & 0.75 & 
1088.80 & 0.84 & \bf{1071.18} & 
1.27 & 1.64\\SCA8-9 & 1072.10 & 0.96 & 
1072.10 & 0.90 & \bf{1060.50} & 
1.09 & 1.09\\CON3-0 & 620.76 & 0.97 & 
621.81 & 0.91 & \bf{616.52} & 
0.69 & 0.86\\CON3-1 & 556.04 & 0.68 & 
561.46 & 0.78 & \bf{554.47} & 
0.28 & 1.26\\CON3-2 & 521.38 & 0.94 & 
523.93 & 0.92 & \bf{518.00} & 
0.65 & 1.15\\CON3-3 & 603.81 & 0.87 & 
604.66 & 0.71 & \bf{591.19} & 
2.13 & 2.28\\CON3-4 & 594.59 & 0.93 & 
595.09 & 0.80 & \bf{588.79} & 
0.99 & 1.07\\CON3-5 & 568.76 & 0.95 & 
568.76 & 0.83 & \bf{563.70} & 
0.90 & 0.90\\CON3-6 & 502.26 & 0.91 & 
504.42 & 0.90 & \bf{499.05} & 
0.64 & 1.08\\CON3-7 & 578.22 & 0.67 & 
578.32 & 0.80 & \bf{576.48} & 
0.30 & 0.32\\CON3-8 & 523.14 & 0.79 & 
524.23 & 0.86 & \bf{523.05} & 
0.02 & 0.23\\CON3-9 & 588.11 & 0.63 & 
589.20 & 0.75 & \bf{578.24} & 
1.71 & 1.90\\CON8-0 & 863.60 & 1.02 & 
878.44 & 0.93 & \bf{857.17} & 
0.75 & 2.48\\CON8-1 & 758.21 & 1.01 & 
765.45 & 0.92 & \bf{740.85} & 
2.34 & 3.32\\CON8-2 & 716.30 & 0.98 & 
721.68 & 0.95 & \bf{712.89} & 
0.48 & 1.23\\CON8-3 & 837.21 & 0.74 & 
838.47 & 0.82 & \bf{811.07} & 
3.22 & 3.38\\CON8-4 & 789.89 & 0.78 & 
790.45 & 0.88 & \bf{772.25} & 
2.28 & 2.36\\CON8-5 & 758.12 & 0.68 & 
758.12 & 0.86 & \bf{754.88} & 
0.43 & 0.43\\CON8-6 & 702.06 & 0.62 & 
702.73 & 0.76 & \bf{678.92} & 
3.41 & 3.51\\CON8-7 & 825.89 & 0.92 & 
825.89 & 0.85 & \bf{811.96} & 
1.72 & 1.72\\CON8-8 & 785.80 & 0.99 & 
790.12 & 0.79 & \bf{767.53} & 
2.38 & 2.94\\CON8-9 & 823.65 & 1.05 & 
829.85 & 0.95 & \bf{809.00} & 
1.81 & 2.58\\\bf{PROM.} & 
\bf{770.03} & \bf{0.85} & \bf{772.26} & \bf{0.83} & \bf{758.54} & \bf{1.39} & \bf{1.70}\\[1ex]\hline
\end{tabular}
\label{table:nonlin}
\end{table} \clearpage
\begin{table}[ht]
\caption{Resultados de la ejecución de la metaheurística IGA, utilizando instancias de SalhiNagy con la configuración -n 200 -p 40 -cprob 70.0 -mprob 90.0}
\centering
\small
\begin{tabular}{c c c c c c c c}
\hline\hline
Instancia & Costo mínimo & Tiempo(seg.) & Costo promedio & Tiempo promedio(seg.) & CME & \%G & \%GP \\ [0.5ex]
\hline
CMT1X & 477.30 & 0.83 & 
481.53 & 0.75 & \bf{470.48} & 
1.45 & 2.35\\CMT1Y & 482.31 & 0.75 & 
491.16 & 0.73 & \bf{470.48} & 
2.51 & 4.39\\CMT2X & 694.86 & 1.69 & 
708.33 & 1.40 & \bf{682.39} & 
1.83 & 3.80\\CMT2Y & 721.16 & 1.66 & 
722.48 & 1.47 & \bf{682.39} & 
5.68 & 5.87\\CMT3X & 732.10 & 3.40 & 
745.00 & 3.40 & \bf{719.06} & 
1.81 & 3.61\\CMT3Y & 745.04 & 3.50 & 
748.96 & 3.14 & \bf{719.06} & 
3.61 & 4.16\\CMT4X & 896.86 & 8.76 & 
911.84 & 8.29 & \bf{854.21} & 
4.99 & 6.75\\CMT4Y & 916.75 & 8.20 & 
918.93 & 8.28 & \bf{852.46} & 
7.54 & 7.80\\CMT5X & 1109.42 & 18.76 & 
1122.47 & 17.27 & \bf{1030.56} & 
7.65 & 8.92\\CMT5Y & 1110.85 & 17.29 & 
1134.88 & 17.24 & \bf{1031.69} & 
7.67 & 10.00\\CMT11X & 887.91 & 5.24 & 
897.40 & 5.36 & \bf{831.09} & 
6.84 & 7.98\\CMT11Y & 900.30 & 6.08 & 
907.11 & 15.72 & \bf{829.85} & 
8.49 & 9.31\\CMT12X & 672.48 & 2.77 & 
680.89 & 2.95 & \bf{658.83} & 
2.07 & 3.35\\CMT12Y & 674.63 & 3.56 & 
681.64 & 3.22 & \bf{660.47} & 
2.14 & 3.21\\\bf{PROM.} & 
\bf{787.28} & \bf{5.89} & \bf{796.61} & \bf{6.37} & \bf{749.50} & \bf{4.59} & \bf{5.82}\\[1ex]\hline
\end{tabular}
\label{table:nonlin}
\end{table} \clearpage
\begin{table}[ht]
\caption{Resultados de la ejecución de la metaheurística IGA, utilizando instancias de Dethloff con la configuración -n 200 -p 40 -cprob 70.0 -mprob 100.0}
\centering
\small
\begin{tabular}{c c c c c c c c}
\hline\hline
Instancia & Costo mínimo & Tiempo(seg.) & Costo promedio & Tiempo promedio(seg.) & CME & \%G & \%GP \\ [0.5ex]
\hline
SCA3-0 & 640.55 & 0.82 & 
640.55 & 0.91 & \bf{635.62} & 
0.78 & 0.78\\SCA3-1 & 701.53 & 0.60 & 
701.53 & 0.74 & \bf{697.84} & 
0.53 & 0.53\\SCA3-2 & 661.13 & 0.64 & 
666.25 & 0.78 & \bf{659.34} & 
0.27 & 1.05\\SCA3-3 & 681.35 & 0.62 & 
681.35 & 0.79 & \bf{680.04} & 
0.19 & 0.19\\SCA3-4 & \bf{690.50} & 0.90 & 
691.18 & 0.70 & 690.50 & 0.00
 & 0.10\\SCA3-5 & 669.28 & 0.94 & 
673.42 & 0.89 & \bf{659.90} & 
1.42 & 2.05\\SCA3-6 & 652.94 & 0.82 & 
656.12 & 0.90 & \bf{651.09} & 
0.28 & 0.77\\SCA3-7 & 666.15 & 0.92 & 
667.78 & 0.86 & \bf{659.17} & 
1.06 & 1.31\\SCA3-8 & 724.29 & 0.85 & 
724.70 & 0.76 & \bf{719.47} & 
0.67 & 0.73\\SCA3-9 & \bf{681.00} & 0.67 & 
684.30 & 0.84 & 681.00 & 0.00
 & 0.49\\SCA8-0 & 985.94 & 0.98 & 
1004.22 & 0.86 & \bf{961.50} & 
2.54 & 4.44\\SCA8-1 & 1098.03 & 0.54 & 
1098.03 & 0.66 & \bf{1049.65} & 
4.61 & 4.61\\SCA8-2 & 1050.17 & 0.95 & 
1052.32 & 0.90 & \bf{1039.64} & 
1.01 & 1.22\\SCA8-3 & 1036.09 & 1.00 & 
1036.09 & 0.86 & \bf{983.34} & 
5.36 & 5.36\\SCA8-4 & 1081.32 & 1.01 & 
1081.32 & 1.01 & \bf{1065.49} & 
1.49 & 1.49\\SCA8-5 & 1063.22 & 0.98 & 
1069.74 & 0.89 & \bf{1027.08} & 
3.52 & 4.15\\SCA8-6 & 999.27 & 0.66 & 
1000.62 & 0.79 & \bf{971.82} & 
2.82 & 2.96\\SCA8-7 & 1067.88 & 0.66 & 
1071.52 & 0.84 & \bf{1051.28} & 
1.58 & 1.93\\SCA8-8 & 1096.04 & 0.51 & 
1096.04 & 0.67 & \bf{1071.18} & 
2.32 & 2.32\\SCA8-9 & 1080.03 & 0.99 & 
1083.78 & 0.85 & \bf{1060.50} & 
1.84 & 2.20\\CON3-0 & 628.47 & 0.67 & 
628.78 & 0.68 & \bf{616.52} & 
1.94 & 1.99\\CON3-1 & 560.75 & 0.67 & 
561.31 & 0.68 & \bf{554.47} & 
1.13 & 1.23\\CON3-2 & 521.38 & 1.01 & 
521.38 & 0.91 & \bf{518.00} & 
0.65 & 0.65\\CON3-3 & 603.24 & 0.58 & 
603.27 & 0.64 & \bf{591.19} & 
2.04 & 2.04\\CON3-4 & 593.78 & 0.93 & 
602.03 & 0.93 & \bf{588.79} & 
0.85 & 2.25\\CON3-5 & 567.94 & 0.88 & 
569.38 & 0.83 & \bf{563.70} & 
0.75 & 1.01\\CON3-6 & 503.97 & 0.96 & 
506.17 & 0.88 & \bf{499.05} & 
0.99 & 1.43\\CON3-7 & 582.14 & 0.72 & 
582.14 & 0.80 & \bf{576.48} & 
0.98 & 0.98\\CON3-8 & \bf{523.05} & 0.94 & 
532.21 & 0.82 & 523.05 & 0.00
 & 1.75\\CON3-9 & 589.00 & 0.88 & 
590.02 & 0.83 & \bf{578.24} & 
1.86 & 2.04\\CON8-0 & 870.00 & 0.98 & 
870.00 & 0.91 & \bf{857.17} & 
1.50 & 1.50\\CON8-1 & 760.41 & 0.82 & 
760.41 & 0.76 & \bf{740.85} & 
2.64 & 2.64\\CON8-2 & 729.46 & 0.84 & 
730.57 & 0.94 & \bf{712.89} & 
2.32 & 2.48\\CON8-3 & 827.66 & 0.90 & 
831.72 & 0.95 & \bf{811.07} & 
2.05 & 2.55\\CON8-4 & 794.10 & 0.96 & 
795.20 & 0.93 & \bf{772.25} & 
2.83 & 2.97\\CON8-5 & 758.84 & 0.68 & 
758.84 & 0.75 & \bf{754.88} & 
0.52 & 0.52\\CON8-6 & 702.93 & 0.71 & 
703.43 & 0.75 & \bf{678.92} & 
3.54 & 3.61\\CON8-7 & 814.50 & 0.55 & 
817.06 & 0.68 & \bf{811.96} & 
0.31 & 0.63\\CON8-8 & 795.31 & 0.93 & 
796.94 & 0.89 & \bf{767.53} & 
3.62 & 3.83\\CON8-9 & 829.83 & 0.60 & 
832.93 & 0.78 & \bf{809.00} & 
2.57 & 2.96\\\bf{PROM.} & 
\bf{772.09} & \bf{0.81} & \bf{774.37} & \bf{0.82} & \bf{758.54} & \bf{1.63} & \bf{1.94}\\[1ex]\hline
\end{tabular}
\label{table:nonlin}
\end{table} \clearpage
\begin{table}[ht]
\caption{Resultados de la ejecución de la metaheurística SCA, utilizando instancias de SalhiNagy con la configuración -n 50.0 -b 10 -y .3}
\centering
\small
\begin{tabular}{c c c c c c c c}
\hline\hline
Instancia & Costo mínimo & Tiempo(seg.) & Costo promedio & Tiempo promedio(seg.) & CME & \%G & \%GP \\ [0.5ex]
\hline
CMT1X & 472.37 & 3.30 & 
474.92 & 1.71 & \bf{470.48} & 
0.40 & 0.94\\CMT1Y & 472.37 & 1.93 & 
473.78 & 1.93 & \bf{470.48} & 
0.40 & 0.70\\CMT2X & 691.15 & 11.95 & 
705.35 & 15.12 & \bf{682.39} & 
1.28 & 3.36\\CMT2Y & 708.21 & 11.57 & 
711.08 & 12.23 & \bf{682.39} & 
3.78 & 4.20\\CMT3X & 740.55 & 35.61 & 
742.90 & 23.04 & \bf{719.06} & 
2.99 & 3.32\\CMT3Y & 732.22 & 42.41 & 
739.06 & 38.12 & \bf{719.06} & 
1.83 & 2.78\\CMT4X & 100000 & 0 & 
896.15 & 155.46 & \bf{854.21} & 
11606.72 & 4.91\\CMT4Y & 100000 & 0 & 
898.71 & 237.04 & \bf{852.46} & 
11630.76 & 5.43\\CMT5X & 100000 & 0 & 
nan & nan & \bf{1030.56} & 
9603.46 & \bf{nan}\\CMT5Y & 100000 & 0 & 
nan & nan & \bf{1031.69} & 
9592.83 & \bf{nan}\\CMT11X & 886.65 & 55.01 & 
904.29 & 38.92 & \bf{831.09} & 
6.69 & 8.81\\CMT11Y & 917.73 & | & 
0.00 & 0.00 & \bf{829.85} & 
10.59 & -100.00\\CMT12X & 680.10 & 45.42 & 
681.93 & 51.05 & \bf{658.83} & 
3.23 & 3.51\\CMT12Y & 677.58 & | & 
0.00 & 0.00 & \bf{660.47} & 
2.59 & -100.00\\\bf{PROM.} & 
\bf{29069.92} & \bf{14.80} & \bf{nan} & \bf{nan} & \bf{749.50} & \bf{3033.40} & \bf{nan}\\[1ex]\hline
\end{tabular}
\label{table:nonlin}
\end{table} \clearpage
\begin{table}[ht]
\caption{Resultados de la ejecución de la metaheurística IGA, utilizando instancias de SalhiNagy con la configuración -n 200 -p 40 -cprob 70.0 -mprob 100.0}
\centering
\small
\begin{tabular}{c c c c c c c c}
\hline\hline
Instancia & Costo mínimo & Tiempo(seg.) & Costo promedio & Tiempo promedio(seg.) & CME & \%G & \%GP \\ [0.5ex]
\hline
CMT1X & 483.32 & 0.87 & 
485.55 & 0.85 & \bf{470.48} & 
2.73 & 3.20\\CMT1Y & 481.84 & 0.66 & 
481.84 & 0.66 & \bf{470.48} & 
2.41 & 2.41\\CMT2X & 708.82 & 1.40 & 
714.81 & 1.32 & \bf{682.39} & 
3.87 & 4.75\\CMT2Y & 706.37 & 1.66 & 
715.72 & 1.44 & \bf{682.39} & 
3.51 & 4.88\\CMT3X & 733.39 & 3.40 & 
745.28 & 3.37 & \bf{719.06} & 
1.99 & 3.65\\CMT3Y & 737.87 & 3.34 & 
747.20 & 3.40 & \bf{719.06} & 
2.62 & 3.91\\CMT4X & 890.49 & 7.50 & 
903.12 & 8.30 & \bf{854.21} & 
4.25 & 5.73\\CMT4Y & 920.51 & 8.18 & 
924.65 & 8.68 & \bf{852.46} & 
7.98 & 8.47\\CMT5X & 1094.71 & 17.82 & 
1113.04 & 17.33 & \bf{1030.56} & 
6.22 & 8.00\\CMT5Y & 1126.48 & 17.94 & 
1134.37 & 17.28 & \bf{1031.69} & 
9.19 & 9.95\\CMT11X & 868.69 & 5.64 & 
905.85 & 5.50 & \bf{831.09} & 
4.52 & 9.00\\CMT11Y & 864.68 & 5.62 & 
906.35 & 5.86 & \bf{829.85} & 
4.20 & 9.22\\CMT12X & 677.06 & 3.36 & 
685.44 & 3.11 & \bf{658.83} & 
2.77 & 4.04\\CMT12Y & 675.62 & 2.67 & 
678.66 & 2.86 & \bf{660.47} & 
2.29 & 2.75\\\bf{PROM.} & 
\bf{783.56} & \bf{5.72} & \bf{795.85} & \bf{5.71} & \bf{749.50} & \bf{4.18} & \bf{5.71}\\[1ex]\hline
\end{tabular}
\label{table:nonlin}
\end{table} \clearpage
\begin{table}[ht]
\caption{Resultados de la ejecución de la metaheurística IGA, utilizando instancias de Dethloff con la configuración -n 200 -p 40 -cprob 80.0 -mprob 10.0}
\centering
\small
\begin{tabular}{c c c c c c c c}
\hline\hline
Instancia & Costo mínimo & Tiempo(seg.) & Costo promedio & Tiempo promedio(seg.) & CME & \%G & \%GP \\ [0.5ex]
\hline
SCA3-0 & 640.55 & 0.94 & 
641.27 & 0.72 & \bf{635.62} & 
0.78 & 0.89\\SCA3-1 & \bf{697.84} & 0.56 & 
697.84 & 0.68 & 697.84 & 0.00
 & 0.00\\
SCA3-2 & 664.18 & 0.72 & 
664.18 & 0.62 & \bf{659.34} & 
0.73 & 0.73\\SCA3-3 & 681.16 & 0.72 & 
682.13 & 0.73 & \bf{680.04} & 
0.16 & 0.31\\SCA3-4 & \bf{690.50} & 0.73 & 
690.50 & 0.66 & 690.50 & 0.00
 & 0.00\\
SCA3-5 & 665.64 & 0.56 & 
668.99 & 0.63 & \bf{659.90} & 
0.87 & 1.38\\SCA3-6 & 652.94 & 0.50 & 
653.16 & 0.58 & \bf{651.09} & 
0.28 & 0.32\\SCA3-7 & 666.15 & 0.54 & 
666.15 & 0.56 & \bf{659.17} & 
1.06 & 1.06\\SCA3-8 & 719.77 & 0.72 & 
724.88 & 0.68 & \bf{719.47} & 
0.04 & 0.75\\SCA3-9 & \bf{681.00} & 0.81 & 
681.00 & 0.67 & 681.00 & 0.00
 & 0.00\\
SCA8-0 & 981.54 & 0.94 & 
1006.88 & 0.82 & \bf{961.50} & 
2.08 & 4.72\\SCA8-1 & 1077.62 & 0.72 & 
1077.67 & 0.74 & \bf{1049.65} & 
2.66 & 2.67\\SCA8-2 & 1050.37 & 0.93 & 
1053.10 & 0.81 & \bf{1039.64} & 
1.03 & 1.29\\SCA8-3 & 1005.22 & 0.74 & 
1025.27 & 0.76 & \bf{983.34} & 
2.23 & 4.26\\SCA8-4 & 1068.97 & 0.73 & 
1068.97 & 0.65 & \bf{1065.49} & 
0.33 & 0.33\\SCA8-5 & 1060.29 & 0.76 & 
1060.29 & 0.69 & \bf{1027.08} & 
3.23 & 3.23\\SCA8-6 & 983.01 & 0.76 & 
983.01 & 0.78 & \bf{971.82} & 
1.15 & 1.15\\SCA8-7 & 1083.24 & 0.70 & 
1083.24 & 0.72 & \bf{1051.28} & 
3.04 & 3.04\\SCA8-8 & 1091.18 & 0.72 & 
1091.97 & 0.72 & \bf{1071.18} & 
1.87 & 1.94\\SCA8-9 & 1082.68 & 0.70 & 
1086.50 & 0.57 & \bf{1060.50} & 
2.09 & 2.45\\CON3-0 & 619.09 & 0.67 & 
620.27 & 0.62 & \bf{616.52} & 
0.42 & 0.61\\CON3-1 & 558.16 & 0.75 & 
560.10 & 0.69 & \bf{554.47} & 
0.67 & 1.02\\CON3-2 & 521.38 & 0.79 & 
521.38 & 0.81 & \bf{518.00} & 
0.65 & 0.65\\CON3-3 & 606.24 & 0.75 & 
609.29 & 0.73 & \bf{591.19} & 
2.55 & 3.06\\CON3-4 & 591.43 & 0.72 & 
599.88 & 0.72 & \bf{588.79} & 
0.45 & 1.88\\CON3-5 & \bf{563.70} & 0.74 & 
563.70 & 0.74 & 563.70 & 0.00
 & 0.00\\
CON3-6 & 503.97 & 0.81 & 
506.35 & 0.78 & \bf{499.05} & 
0.99 & 1.46\\CON3-7 & 578.41 & 0.94 & 
579.10 & 0.79 & \bf{576.48} & 
0.33 & 0.45\\CON3-8 & 523.14 & 0.96 & 
524.15 & 0.80 & \bf{523.05} & 
0.02 & 0.21\\CON3-9 & 588.40 & 0.88 & 
589.51 & 0.79 & \bf{578.24} & 
1.76 & 1.95\\CON8-0 & 875.18 & 0.72 & 
878.74 & 0.65 & \bf{857.17} & 
2.10 & 2.52\\CON8-1 & 771.49 & 0.96 & 
772.39 & 0.82 & \bf{740.85} & 
4.14 & 4.26\\CON8-2 & 723.08 & 0.80 & 
727.41 & 0.83 & \bf{712.89} & 
1.43 & 2.04\\CON8-3 & 835.92 & 0.75 & 
838.79 & 0.77 & \bf{811.07} & 
3.06 & 3.42\\CON8-4 & \bf{772.25} & 0.70 & 
772.41 & 0.73 & 772.25 & 0.00
 & 0.02\\CON8-5 & 766.01 & 0.73 & 
767.68 & 0.71 & \bf{754.88} & 
1.47 & 1.70\\CON8-6 & 702.71 & 0.71 & 
703.63 & 0.73 & \bf{678.92} & 
3.50 & 3.64\\CON8-7 & 814.79 & 0.72 & 
819.13 & 0.81 & \bf{811.96} & 
0.35 & 0.88\\CON8-8 & 778.34 & 0.90 & 
778.34 & 0.85 & \bf{767.53} & 
1.41 & 1.41\\CON8-9 & 832.80 & 0.75 & 
832.80 & 0.79 & \bf{809.00} & 
2.94 & 2.94\\\bf{PROM.} & 
\bf{769.26} & \bf{0.76} & \bf{771.80} & \bf{0.72} & \bf{758.54} & \bf{1.30} & \bf{1.62}\\[1ex]\hline
\end{tabular}
\label{table:nonlin}
\end{table} \clearpage
\begin{table}[ht]
\caption{Resultados de la ejecución de la metaheurística IGA, utilizando instancias de SalhiNagy con la configuración -n 200 -p 40 -cprob 80.0 -mprob 10.0}
\centering
\small
\begin{tabular}{c c c c c c c c}
\hline\hline
Instancia & Costo mínimo & Tiempo(seg.) & Costo promedio & Tiempo promedio(seg.) & CME & \%G & \%GP \\ [0.5ex]
\hline
CMT1X & 479.21 & 0.64 & 
485.04 & 0.63 & \bf{470.48} & 
1.86 & 3.09\\CMT1Y & 484.23 & 0.58 & 
485.73 & 0.61 & \bf{470.48} & 
2.92 & 3.24\\CMT2X & 706.64 & 1.44 & 
716.50 & 1.43 & \bf{682.39} & 
3.55 & 5.00\\CMT2Y & 705.33 & 1.12 & 
718.11 & 1.22 & \bf{682.39} & 
3.36 & 5.23\\CMT3X & 734.51 & 3.06 & 
742.92 & 3.01 & \bf{719.06} & 
2.15 & 3.32\\CMT3Y & 740.67 & 3.05 & 
747.07 & 3.06 & \bf{719.06} & 
3.01 & 3.89\\CMT4X & 904.13 & 8.34 & 
909.56 & 8.48 & \bf{854.21} & 
5.84 & 6.48\\CMT4Y & 916.01 & 8.18 & 
928.61 & 8.16 & \bf{852.46} & 
7.45 & 8.93\\CMT5X & 1116.11 & 15.62 & 
1123.74 & 16.06 & \bf{1030.56} & 
8.30 & 9.04\\CMT5Y & 1133.58 & 15.94 & 
1141.93 & 16.15 & \bf{1031.69} & 
9.88 & 10.69\\CMT11X & 917.76 & 5.72 & 
927.64 & 5.45 & \bf{831.09} & 
10.43 & 11.62\\CMT11Y & 895.99 & 5.18 & 
909.29 & 5.61 & \bf{829.85} & 
7.97 & 9.57\\CMT12X & 676.94 & 3.14 & 
679.39 & 3.12 & \bf{658.83} & 
2.75 & 3.12\\CMT12Y & 673.51 & 3.40 & 
674.61 & 3.12 & \bf{660.47} & 
1.97 & 2.14\\\bf{PROM.} & 
\bf{791.76} & \bf{5.39} & \bf{799.30} & \bf{5.44} & \bf{749.50} & \bf{5.10} & \bf{6.10}\\[1ex]\hline
\end{tabular}
\label{table:nonlin}
\end{table} \clearpage
\begin{table}[ht]
\caption{Resultados de la ejecución de la metaheurística IGA, utilizando instancias de Dethloff con la configuración -n 200 -p 40 -cprob 80.0 -mprob 20.0}
\centering
\small
\begin{tabular}{c c c c c c c c}
\hline\hline
Instancia & Costo mínimo & Tiempo(seg.) & Costo promedio & Tiempo promedio(seg.) & CME & \%G & \%GP \\ [0.5ex]
\hline
SCA3-0 & 640.55 & 1.28 & 
640.55 & 0.93 & \bf{635.62} & 
0.78 & 0.78\\SCA3-1 & 701.53 & 0.70 & 
701.53 & 0.71 & \bf{697.84} & 
0.53 & 0.53\\SCA3-2 & 664.18 & 0.72 & 
665.49 & 0.73 & \bf{659.34} & 
0.73 & 0.93\\SCA3-3 & \bf{680.04} & 0.75 & 
680.60 & 0.77 & 680.04 & 0.00
 & 0.08\\SCA3-4 & \bf{690.50} & 0.68 & 
690.50 & 0.69 & 690.50 & 0.00
 & 0.00\\
SCA3-5 & 670.10 & 0.78 & 
672.70 & 0.68 & \bf{659.90} & 
1.55 & 1.94\\SCA3-6 & 653.69 & 0.70 & 
653.69 & 0.62 & \bf{651.09} & 
0.40 & 0.40\\SCA3-7 & 666.15 & 0.78 & 
668.21 & 0.71 & \bf{659.17} & 
1.06 & 1.37\\SCA3-8 & 731.12 & 0.74 & 
734.94 & 0.80 & \bf{719.47} & 
1.62 & 2.15\\SCA3-9 & \bf{681.00} & 0.69 & 
682.92 & 0.64 & 681.00 & 0.00
 & 0.28\\SCA8-0 & 997.49 & 0.58 & 
1008.74 & 0.72 & \bf{961.50} & 
3.74 & 4.91\\SCA8-1 & 1074.03 & 0.63 & 
1074.11 & 0.65 & \bf{1049.65} & 
2.32 & 2.33\\SCA8-2 & 1053.94 & 0.71 & 
1053.94 & 0.64 & \bf{1039.64} & 
1.38 & 1.38\\SCA8-3 & 1007.99 & 0.76 & 
1007.99 & 0.71 & \bf{983.34} & 
2.51 & 2.51\\SCA8-4 & 1099.14 & 0.74 & 
1104.88 & 0.65 & \bf{1065.49} & 
3.16 & 3.70\\SCA8-5 & 1052.50 & 0.79 & 
1052.50 & 0.73 & \bf{1027.08} & 
2.47 & 2.47\\SCA8-6 & 984.55 & 0.73 & 
984.55 & 0.70 & \bf{971.82} & 
1.31 & 1.31\\SCA8-7 & 1074.24 & 0.65 & 
1074.24 & 0.68 & \bf{1051.28} & 
2.18 & 2.18\\SCA8-8 & \bf{1071.18} & 0.75 & 
1077.65 & 0.80 & 1071.18 & 0.00
 & 0.60\\SCA8-9 & 1072.10 & 0.76 & 
1079.16 & 0.80 & \bf{1060.50} & 
1.09 & 1.76\\CON3-0 & 620.76 & 0.72 & 
633.58 & 0.84 & \bf{616.52} & 
0.69 & 2.77\\CON3-1 & 560.75 & 0.78 & 
561.84 & 0.69 & \bf{554.47} & 
1.13 & 1.33\\CON3-2 & 521.38 & 0.79 & 
521.38 & 0.74 & \bf{518.00} & 
0.65 & 0.65\\CON3-3 & 602.16 & 0.90 & 
603.88 & 0.79 & \bf{591.19} & 
1.86 & 2.15\\CON3-4 & 592.58 & 0.71 & 
594.08 & 0.76 & \bf{588.79} & 
0.64 & 0.90\\CON3-5 & \bf{563.70} & 0.57 & 
563.70 & 0.76 & 563.70 & 0.00
 & 0.00\\
CON3-6 & 504.70 & 0.79 & 
505.69 & 0.76 & \bf{499.05} & 
1.13 & 1.33\\CON3-7 & 577.91 & 0.71 & 
580.91 & 0.75 & \bf{576.48} & 
0.25 & 0.77\\CON3-8 & 531.84 & 0.68 & 
531.84 & 0.70 & \bf{523.05} & 
1.68 & 1.68\\CON3-9 & 582.79 & 1.00 & 
587.99 & 0.83 & \bf{578.24} & 
0.79 & 1.69\\CON8-0 & 869.44 & 0.76 & 
869.44 & 0.77 & \bf{857.17} & 
1.43 & 1.43\\CON8-1 & 762.20 & 0.61 & 
762.20 & 0.80 & \bf{740.85} & 
2.88 & 2.88\\CON8-2 & 720.43 & 0.78 & 
722.88 & 0.88 & \bf{712.89} & 
1.06 & 1.40\\CON8-3 & 833.63 & 0.75 & 
835.18 & 0.76 & \bf{811.07} & 
2.78 & 2.97\\CON8-4 & 783.45 & 0.55 & 
792.97 & 0.66 & \bf{772.25} & 
1.45 & 2.68\\CON8-5 & 768.27 & 0.68 & 
769.37 & 0.77 & \bf{754.88} & 
1.77 & 1.92\\CON8-6 & 693.15 & 0.76 & 
693.15 & 0.82 & \bf{678.92} & 
2.10 & 2.10\\CON8-7 & 822.12 & 0.74 & 
824.50 & 0.68 & \bf{811.96} & 
1.25 & 1.55\\CON8-8 & 796.31 & 0.50 & 
799.24 & 0.68 & \bf{767.53} & 
3.75 & 4.13\\CON8-9 & 826.47 & 0.84 & 
829.79 & 0.74 & \bf{809.00} & 
2.16 & 2.57\\\bf{PROM.} & 
\bf{770.00} & \bf{0.74} & \bf{772.31} & \bf{0.74} & \bf{758.54} & \bf{1.41} & \bf{1.71}\\[1ex]\hline
\end{tabular}
\label{table:nonlin}
\end{table} \clearpage
\begin{table}[ht]
\caption{Resultados de la ejecución de la metaheurística IGA, utilizando instancias de SalhiNagy con la configuración -n 200 -p 40 -cprob 80.0 -mprob 20.0}
\centering
\small
\begin{tabular}{c c c c c c c c}
\hline\hline
Instancia & Costo mínimo & Tiempo(seg.) & Costo promedio & Tiempo promedio(seg.) & CME & \%G & \%GP \\ [0.5ex]
\hline
CMT1X & 481.81 & 0.83 & 
481.81 & 0.72 & \bf{470.48} & 
2.41 & 2.41\\CMT1Y & 475.22 & 0.61 & 
484.01 & 0.62 & \bf{470.48} & 
1.01 & 2.88\\CMT2X & 714.82 & 1.50 & 
720.96 & 1.33 & \bf{682.39} & 
4.75 & 5.65\\CMT2Y & 692.78 & 1.72 & 
720.16 & 1.46 & \bf{682.39} & 
1.52 & 5.53\\CMT3X & 729.53 & 2.88 & 
742.28 & 3.03 & \bf{719.06} & 
1.46 & 3.23\\CMT3Y & 736.18 & 2.58 & 
743.78 & 2.88 & \bf{719.06} & 
2.38 & 3.44\\CMT4X & 889.17 & 8.20 & 
906.33 & 8.43 & \bf{854.21} & 
4.09 & 6.10\\CMT4Y & 894.66 & 8.48 & 
921.92 & 8.03 & \bf{852.46} & 
4.95 & 8.15\\CMT5X & 1095.13 & 16.84 & 
1108.47 & 17.05 & \bf{1030.56} & 
6.27 & 7.56\\CMT5Y & 1116.23 & 16.36 & 
1123.28 & 16.60 & \bf{1031.69} & 
8.19 & 8.88\\CMT11X & 897.77 & 5.26 & 
905.46 & 5.05 & \bf{831.09} & 
8.02 & 8.95\\CMT11Y & 883.37 & 5.33 & 
900.30 & 5.83 & \bf{829.85} & 
6.45 & 8.49\\CMT12X & 685.02 & 3.25 & 
687.25 & 3.26 & \bf{658.83} & 
3.98 & 4.31\\CMT12Y & 673.85 & 3.06 & 
675.58 & 3.13 & \bf{660.47} & 
2.03 & 2.29\\\bf{PROM.} & 
\bf{783.25} & \bf{5.49} & \bf{794.40} & \bf{5.53} & \bf{749.50} & \bf{4.11} & \bf{5.56}\\[1ex]\hline
\end{tabular}
\label{table:nonlin}
\end{table} \clearpage
\begin{table}[ht]
\caption{Resultados de la ejecución de la metaheurística IGA, utilizando instancias de Dethloff con la configuración -n 200 -p 40 -cprob 80.0 -mprob 30.0}
\centering
\small
\begin{tabular}{c c c c c c c c}
\hline\hline
Instancia & Costo mínimo & Tiempo(seg.) & Costo promedio & Tiempo promedio(seg.) & CME & \%G & \%GP \\ [0.5ex]
\hline
SCA3-0 & 640.55 & 0.72 & 
640.55 & 0.73 & \bf{635.62} & 
0.78 & 0.78\\SCA3-1 & 701.53 & 0.89 & 
708.18 & 0.78 & \bf{697.84} & 
0.53 & 1.48\\SCA3-2 & 664.21 & 0.66 & 
667.14 & 0.63 & \bf{659.34} & 
0.74 & 1.18\\SCA3-3 & 685.47 & 0.87 & 
685.47 & 0.83 & \bf{680.04} & 
0.80 & 0.80\\SCA3-4 & \bf{690.50} & 0.91 & 
690.50 & 0.72 & 690.50 & 0.00
 & 0.00\\
SCA3-5 & 674.18 & 0.63 & 
682.62 & 0.73 & \bf{659.90} & 
2.16 & 3.44\\SCA3-6 & 652.94 & 0.57 & 
652.94 & 0.75 & \bf{651.09} & 
0.28 & 0.28\\SCA3-7 & 666.15 & 0.92 & 
668.02 & 0.76 & \bf{659.17} & 
1.06 & 1.34\\SCA3-8 & \bf{719.47} & 0.89 & 
722.74 & 0.78 & 719.47 & 0.00
 & 0.45\\SCA3-9 & 688.68 & 0.55 & 
689.63 & 0.68 & \bf{681.00} & 
1.13 & 1.27\\SCA8-0 & 1003.31 & 0.74 & 
1003.31 & 0.80 & \bf{961.50} & 
4.35 & 4.35\\SCA8-1 & 1073.63 & 0.52 & 
1082.88 & 0.67 & \bf{1049.65} & 
2.28 & 3.17\\SCA8-2 & 1054.69 & 0.74 & 
1054.69 & 0.78 & \bf{1039.64} & 
1.45 & 1.45\\SCA8-3 & 1026.72 & 0.80 & 
1032.40 & 0.76 & \bf{983.34} & 
4.41 & 4.99\\SCA8-4 & 1077.80 & 0.78 & 
1077.80 & 0.80 & \bf{1065.49} & 
1.16 & 1.16\\SCA8-5 & 1054.69 & 0.72 & 
1064.72 & 0.79 & \bf{1027.08} & 
2.69 & 3.66\\SCA8-6 & 988.58 & 0.74 & 
988.58 & 0.74 & \bf{971.82} & 
1.72 & 1.72\\SCA8-7 & 1073.65 & 0.95 & 
1073.65 & 0.82 & \bf{1051.28} & 
2.13 & 2.13\\SCA8-8 & 1091.18 & 0.70 & 
1091.18 & 0.67 & \bf{1071.18} & 
1.87 & 1.87\\SCA8-9 & 1072.28 & 0.68 & 
1072.28 & 0.72 & \bf{1060.50} & 
1.11 & 1.11\\CON3-0 & 628.79 & 0.70 & 
631.02 & 0.69 & \bf{616.52} & 
1.99 & 2.35\\CON3-1 & 556.92 & 0.74 & 
560.40 & 0.80 & \bf{554.47} & 
0.44 & 1.07\\CON3-2 & 521.63 & 0.62 & 
521.63 & 0.85 & \bf{518.00} & 
0.70 & 0.70\\CON3-3 & 600.02 & 0.70 & 
603.32 & 0.61 & \bf{591.19} & 
1.49 & 2.05\\CON3-4 & 595.25 & 0.72 & 
600.12 & 0.70 & \bf{588.79} & 
1.10 & 1.92\\CON3-5 & 564.88 & 0.72 & 
566.82 & 0.78 & \bf{563.70} & 
0.21 & 0.55\\CON3-6 & 504.48 & 0.78 & 
505.10 & 0.66 & \bf{499.05} & 
1.09 & 1.21\\CON3-7 & 582.14 & 0.58 & 
585.04 & 0.72 & \bf{576.48} & 
0.98 & 1.49\\CON3-8 & 528.09 & 0.91 & 
528.09 & 0.74 & \bf{523.05} & 
0.96 & 0.96\\CON3-9 & 588.48 & 0.76 & 
588.48 & 0.72 & \bf{578.24} & 
1.77 & 1.77\\CON8-0 & 875.98 & 0.77 & 
884.69 & 0.83 & \bf{857.17} & 
2.19 & 3.21\\CON8-1 & 759.91 & 0.73 & 
759.91 & 0.75 & \bf{740.85} & 
2.57 & 2.57\\CON8-2 & 716.26 & 0.82 & 
718.80 & 0.79 & \bf{712.89} & 
0.47 & 0.83\\CON8-3 & 813.51 & 0.66 & 
818.13 & 0.63 & \bf{811.07} & 
0.30 & 0.87\\CON8-4 & 792.52 & 0.73 & 
792.60 & 0.77 & \bf{772.25} & 
2.62 & 2.63\\CON8-5 & 768.05 & 0.76 & 
768.05 & 0.79 & \bf{754.88} & 
1.74 & 1.74\\CON8-6 & 704.25 & 1.03 & 
704.30 & 0.79 & \bf{678.92} & 
3.73 & 3.74\\CON8-7 & 825.84 & 0.92 & 
827.71 & 0.81 & \bf{811.96} & 
1.71 & 1.94\\CON8-8 & 793.40 & 0.80 & 
799.14 & 0.80 & \bf{767.53} & 
3.37 & 4.12\\CON8-9 & 831.23 & 0.49 & 
842.12 & 0.76 & \bf{809.00} & 
2.75 & 4.09\\\bf{PROM.} & 
\bf{771.30} & \bf{0.75} & \bf{773.87} & \bf{0.75} & \bf{758.54} & \bf{1.57} & \bf{1.91}\\[1ex]\hline
\end{tabular}
\label{table:nonlin}
\end{table} \clearpage
\begin{table}[ht]
\caption{Resultados de la ejecución de la metaheurística IGA, utilizando instancias de SalhiNagy con la configuración -n 200 -p 40 -cprob 80.0 -mprob 30.0}
\centering
\small
\begin{tabular}{c c c c c c c c}
\hline\hline
Instancia & Costo mínimo & Tiempo(seg.) & Costo promedio & Tiempo promedio(seg.) & CME & \%G & \%GP \\ [0.5ex]
\hline
CMT1X & 480.82 & 0.62 & 
487.26 & 0.64 & \bf{470.48} & 
2.20 & 3.57\\CMT1Y & 473.29 & 0.61 & 
480.26 & 0.63 & \bf{470.48} & 
0.60 & 2.08\\CMT2X & 700.44 & 1.62 & 
705.13 & 1.51 & \bf{682.39} & 
2.65 & 3.33\\CMT2Y & 713.25 & 1.15 & 
713.68 & 1.19 & \bf{682.39} & 
4.52 & 4.59\\CMT3X & 735.99 & 3.04 & 
746.38 & 3.14 & \bf{719.06} & 
2.35 & 3.80\\CMT3Y & 749.53 & 3.24 & 
752.44 & 3.00 & \bf{719.06} & 
4.24 & 4.64\\CMT4X & 899.11 & 8.32 & 
914.83 & 8.26 & \bf{854.21} & 
5.26 & 7.10\\CMT4Y & 903.28 & 8.55 & 
905.83 & 8.40 & \bf{852.46} & 
5.96 & 6.26\\CMT5X & 1101.27 & 16.52 & 
1119.13 & 16.54 & \bf{1030.56} & 
6.86 & 8.59\\CMT5Y & 1103.72 & 17.14 & 
1126.92 & 16.79 & \bf{1031.69} & 
6.98 & 9.23\\CMT11X & 902.85 & 5.19 & 
918.17 & 5.14 & \bf{831.09} & 
8.63 & 10.48\\CMT11Y & 865.40 & 5.99 & 
895.32 & 5.70 & \bf{829.85} & 
4.28 & 7.89\\CMT12X & 674.56 & 3.33 & 
674.85 & 3.35 & \bf{658.83} & 
2.39 & 2.43\\CMT12Y & 675.46 & 3.03 & 
680.35 & 2.96 & \bf{660.47} & 
2.27 & 3.01\\\bf{PROM.} & 
\bf{784.21} & \bf{5.60} & \bf{794.33} & \bf{5.52} & \bf{749.50} & \bf{4.23} & \bf{5.50}\\[1ex]\hline
\end{tabular}
\label{table:nonlin}
\end{table} \clearpage
\begin{table}[ht]
\caption{Resultados de la ejecución de la metaheurística IGA, utilizando instancias de Dethloff con la configuración -n 200 -p 40 -cprob 80.0 -mprob 40.0}
\centering
\small
\begin{tabular}{c c c c c c c c}
\hline\hline
Instancia & Costo mínimo & Tiempo(seg.) & Costo promedio & Tiempo promedio(seg.) & CME & \%G & \%GP \\ [0.5ex]
\hline
SCA3-0 & 641.69 & 0.72 & 
642.10 & 0.78 & \bf{635.62} & 
0.95 & 1.02\\SCA3-1 & \bf{697.84} & 0.73 & 
701.22 & 0.80 & 697.84 & 0.00
 & 0.48\\SCA3-2 & 664.21 & 0.68 & 
667.85 & 0.70 & \bf{659.34} & 
0.74 & 1.29\\SCA3-3 & 682.47 & 0.50 & 
690.70 & 0.70 & \bf{680.04} & 
0.36 & 1.57\\SCA3-4 & \bf{690.50} & 0.48 & 
690.50 & 0.64 & 690.50 & 0.00
 & 0.00\\
SCA3-5 & 668.48 & 0.71 & 
670.29 & 0.72 & \bf{659.90} & 
1.30 & 1.57\\SCA3-6 & 652.94 & 0.69 & 
652.94 & 0.71 & \bf{651.09} & 
0.28 & 0.28\\SCA3-7 & 671.67 & 0.74 & 
671.67 & 0.63 & \bf{659.17} & 
1.90 & 1.90\\SCA3-8 & 726.44 & 0.88 & 
726.58 & 0.76 & \bf{719.47} & 
0.97 & 0.99\\SCA3-9 & \bf{681.00} & 0.89 & 
683.36 & 0.90 & 681.00 & 0.00
 & 0.35\\SCA8-0 & 988.67 & 0.58 & 
994.57 & 0.69 & \bf{961.50} & 
2.83 & 3.44\\SCA8-1 & 1079.17 & 0.79 & 
1082.30 & 0.82 & \bf{1049.65} & 
2.81 & 3.11\\SCA8-2 & 1054.47 & 0.56 & 
1054.47 & 0.81 & \bf{1039.64} & 
1.43 & 1.43\\SCA8-3 & 1022.89 & 0.77 & 
1026.56 & 0.76 & \bf{983.34} & 
4.02 & 4.40\\SCA8-4 & 1095.17 & 0.96 & 
1095.36 & 0.86 & \bf{1065.49} & 
2.79 & 2.80\\SCA8-5 & 1063.46 & 1.00 & 
1063.46 & 0.84 & \bf{1027.08} & 
3.54 & 3.54\\SCA8-6 & 999.03 & 0.74 & 
999.70 & 0.84 & \bf{971.82} & 
2.80 & 2.87\\SCA8-7 & 1078.29 & 0.71 & 
1081.15 & 0.73 & \bf{1051.28} & 
2.57 & 2.84\\SCA8-8 & 1087.82 & 0.96 & 
1087.82 & 0.80 & \bf{1071.18} & 
1.55 & 1.55\\SCA8-9 & 1068.65 & 0.92 & 
1068.65 & 0.76 & \bf{1060.50} & 
0.77 & 0.77\\CON3-0 & 620.76 & 0.57 & 
623.99 & 0.75 & \bf{616.52} & 
0.69 & 1.21\\CON3-1 & 561.63 & 0.70 & 
561.63 & 0.90 & \bf{554.47} & 
1.29 & 1.29\\CON3-2 & 521.38 & 0.80 & 
522.26 & 0.89 & \bf{518.00} & 
0.65 & 0.82\\CON3-3 & 592.43 & 0.81 & 
596.18 & 0.81 & \bf{591.19} & 
0.21 & 0.84\\CON3-4 & 591.43 & 0.95 & 
593.97 & 0.82 & \bf{588.79} & 
0.45 & 0.88\\CON3-5 & \bf{563.70} & 0.74 & 
563.70 & 0.73 & 563.70 & 0.00
 & 0.00\\
CON3-6 & 508.94 & 0.76 & 
509.83 & 0.80 & \bf{499.05} & 
1.98 & 2.16\\CON3-7 & 578.22 & 0.82 & 
581.15 & 0.71 & \bf{576.48} & 
0.30 & 0.81\\CON3-8 & \bf{523.05} & 0.87 & 
525.44 & 0.83 & 523.05 & 0.00
 & 0.46\\CON3-9 & 589.00 & 0.76 & 
589.74 & 0.82 & \bf{578.24} & 
1.86 & 1.99\\CON8-0 & 870.49 & 0.75 & 
870.49 & 0.64 & \bf{857.17} & 
1.55 & 1.55\\CON8-1 & 752.52 & 0.96 & 
759.21 & 0.85 & \bf{740.85} & 
1.58 & 2.48\\CON8-2 & 726.33 & 0.78 & 
727.94 & 0.72 & \bf{712.89} & 
1.89 & 2.11\\CON8-3 & 837.55 & 0.91 & 
837.55 & 0.85 & \bf{811.07} & 
3.26 & 3.26\\CON8-4 & 784.66 & 0.92 & 
789.36 & 0.84 & \bf{772.25} & 
1.61 & 2.22\\CON8-5 & 769.87 & 0.99 & 
775.75 & 0.88 & \bf{754.88} & 
1.99 & 2.76\\CON8-6 & 686.39 & 0.79 & 
694.08 & 0.78 & \bf{678.92} & 
1.10 & 2.23\\CON8-7 & 814.79 & 0.83 & 
815.02 & 0.83 & \bf{811.96} & 
0.35 & 0.38\\CON8-8 & 791.67 & 0.52 & 
791.67 & 0.68 & \bf{767.53} & 
3.15 & 3.15\\CON8-9 & 814.96 & 0.58 & 
821.64 & 0.81 & \bf{809.00} & 
0.74 & 1.56\\\bf{PROM.} & 
\bf{770.37} & \bf{0.77} & \bf{772.55} & \bf{0.78} & \bf{758.54} & \bf{1.41} & \bf{1.71}\\[1ex]\hline
\end{tabular}
\label{table:nonlin}
\end{table} \clearpage
\begin{table}[ht]
\caption{Resultados de la ejecución de la metaheurística IGA, utilizando instancias de SalhiNagy con la configuración -n 200 -p 40 -cprob 80.0 -mprob 40.0}
\centering
\small
\begin{tabular}{c c c c c c c c}
\hline\hline
Instancia & Costo mínimo & Tiempo(seg.) & Costo promedio & Tiempo promedio(seg.) & CME & \%G & \%GP \\ [0.5ex]
\hline
CMT1X & 485.87 & 0.67 & 
486.20 & 0.65 & \bf{470.48} & 
3.27 & 3.34\\CMT1Y & 483.82 & 0.61 & 
485.43 & 0.67 & \bf{470.48} & 
2.84 & 3.18\\CMT2X & 707.73 & 1.37 & 
716.29 & 1.36 & \bf{682.39} & 
3.71 & 4.97\\CMT2Y & 711.96 & 1.67 & 
720.25 & 1.51 & \bf{682.39} & 
4.33 & 5.55\\CMT3X & 734.93 & 3.13 & 
740.66 & 3.39 & \bf{719.06} & 
2.21 & 3.00\\CMT3Y & 734.41 & 3.02 & 
740.91 & 3.12 & \bf{719.06} & 
2.13 & 3.04\\CMT4X & 904.50 & 8.60 & 
911.76 & 7.82 & \bf{854.21} & 
5.89 & 6.74\\CMT4Y & 917.30 & 6.91 & 
920.20 & 7.67 & \bf{852.46} & 
7.61 & 7.95\\CMT5X & 1109.47 & 16.34 & 
1114.25 & 16.55 & \bf{1030.56} & 
7.66 & 8.12\\CMT5Y & 1103.81 & 17.34 & 
1126.40 & 16.70 & \bf{1031.69} & 
6.99 & 9.18\\CMT11X & 903.22 & 5.11 & 
911.73 & 5.47 & \bf{831.09} & 
8.68 & 9.70\\CMT11Y & 876.54 & 5.87 & 
889.53 & 5.51 & \bf{829.85} & 
5.63 & 7.19\\CMT12X & 674.94 & 3.03 & 
682.16 & 2.98 & \bf{658.83} & 
2.45 & 3.54\\CMT12Y & 673.49 & 3.36 & 
676.99 & 3.02 & \bf{660.47} & 
1.97 & 2.50\\\bf{PROM.} & 
\bf{787.28} & \bf{5.50} & \bf{794.48} & \bf{5.46} & \bf{749.50} & \bf{4.67} & \bf{5.57}\\[1ex]\hline
\end{tabular}
\label{table:nonlin}
\end{table} \clearpage
\begin{table}[ht]
\caption{Resultados de la ejecución de la metaheurística IGA, utilizando instancias de Dethloff con la configuración -n 200 -p 40 -cprob 80.0 -mprob 50.0}
\centering
\small
\begin{tabular}{c c c c c c c c}
\hline\hline
Instancia & Costo mínimo & Tiempo(seg.) & Costo promedio & Tiempo promedio(seg.) & CME & \%G & \%GP \\ [0.5ex]
\hline
SCA3-0 & 640.55 & 0.67 & 
641.12 & 0.68 & \bf{635.62} & 
0.78 & 0.87\\SCA3-1 & 701.74 & 0.65 & 
701.80 & 0.74 & \bf{697.84} & 
0.56 & 0.57\\SCA3-2 & 661.13 & 0.67 & 
667.08 & 0.68 & \bf{659.34} & 
0.27 & 1.17\\SCA3-3 & \bf{680.04} & 0.84 & 
682.25 & 0.90 & 680.04 & 0.00
 & 0.33\\SCA3-4 & \bf{690.50} & 0.72 & 
690.50 & 0.72 & 690.50 & 0.00
 & 0.00\\
SCA3-5 & 673.56 & 0.78 & 
679.93 & 0.74 & \bf{659.90} & 
2.07 & 3.04\\SCA3-6 & 654.79 & 0.72 & 
657.61 & 0.70 & \bf{651.09} & 
0.57 & 1.00\\SCA3-7 & 666.15 & 0.93 & 
666.15 & 0.78 & \bf{659.17} & 
1.06 & 1.06\\SCA3-8 & 724.29 & 0.84 & 
724.29 & 0.84 & \bf{719.47} & 
0.67 & 0.67\\SCA3-9 & \bf{681.00} & 0.70 & 
681.00 & 0.78 & 681.00 & 0.00
 & 0.00\\
SCA8-0 & 1001.08 & 0.84 & 
1001.08 & 0.81 & \bf{961.50} & 
4.12 & 4.12\\SCA8-1 & 1077.62 & 1.00 & 
1077.62 & 0.73 & \bf{1049.65} & 
2.66 & 2.66\\SCA8-2 & 1053.78 & 0.94 & 
1053.78 & 0.85 & \bf{1039.64} & 
1.36 & 1.36\\SCA8-3 & 1013.56 & 0.88 & 
1014.25 & 0.85 & \bf{983.34} & 
3.07 & 3.14\\SCA8-4 & 1074.81 & 0.77 & 
1085.22 & 0.86 & \bf{1065.49} & 
0.87 & 1.85\\SCA8-5 & 1043.05 & 0.60 & 
1068.06 & 0.60 & \bf{1027.08} & 
1.55 & 3.99\\SCA8-6 & 975.81 & 0.93 & 
984.45 & 0.83 & \bf{971.82} & 
0.41 & 1.30\\SCA8-7 & 1070.92 & 0.87 & 
1085.29 & 0.78 & \bf{1051.28} & 
1.87 & 3.24\\SCA8-8 & 1095.42 & 0.92 & 
1103.36 & 0.75 & \bf{1071.18} & 
2.26 & 3.00\\SCA8-9 & 1080.01 & 0.72 & 
1080.12 & 0.72 & \bf{1060.50} & 
1.84 & 1.85\\CON3-0 & 620.76 & 0.75 & 
620.76 & 0.72 & \bf{616.52} & 
0.69 & 0.69\\CON3-1 & 559.52 & 0.58 & 
560.13 & 0.69 & \bf{554.47} & 
0.91 & 1.02\\CON3-2 & 521.38 & 1.03 & 
523.12 & 0.86 & \bf{518.00} & 
0.65 & 0.99\\CON3-3 & 597.34 & 0.90 & 
609.85 & 0.80 & \bf{591.19} & 
1.04 & 3.16\\CON3-4 & 595.00 & 0.83 & 
598.84 & 0.75 & \bf{588.79} & 
1.05 & 1.71\\CON3-5 & \bf{563.70} & 0.95 & 
566.88 & 0.86 & 563.70 & 0.00
 & 0.56\\CON3-6 & 502.26 & 0.60 & 
504.20 & 0.66 & \bf{499.05} & 
0.64 & 1.03\\CON3-7 & 577.54 & 0.91 & 
579.68 & 0.76 & \bf{576.48} & 
0.18 & 0.56\\CON3-8 & 524.59 & 0.92 & 
530.25 & 0.80 & \bf{523.05} & 
0.29 & 1.38\\CON3-9 & 588.11 & 0.76 & 
590.00 & 0.83 & \bf{578.24} & 
1.71 & 2.03\\CON8-0 & 867.79 & 0.78 & 
867.79 & 0.68 & \bf{857.17} & 
1.24 & 1.24\\CON8-1 & 760.77 & 0.62 & 
760.77 & 0.86 & \bf{740.85} & 
2.69 & 2.69\\CON8-2 & 716.07 & 0.82 & 
716.98 & 0.77 & \bf{712.89} & 
0.45 & 0.57\\CON8-3 & 835.14 & 0.77 & 
838.83 & 0.85 & \bf{811.07} & 
2.97 & 3.42\\CON8-4 & 778.28 & 0.73 & 
778.28 & 0.79 & \bf{772.25} & 
0.78 & 0.78\\CON8-5 & 770.91 & 1.00 & 
770.91 & 0.88 & \bf{754.88} & 
2.12 & 2.12\\CON8-6 & 709.51 & 0.93 & 
710.51 & 0.89 & \bf{678.92} & 
4.51 & 4.65\\CON8-7 & 828.22 & 0.71 & 
828.77 & 0.76 & \bf{811.96} & 
2.00 & 2.07\\CON8-8 & 794.68 & 0.81 & 
802.51 & 0.86 & \bf{767.53} & 
3.54 & 4.56\\CON8-9 & 817.95 & 0.61 & 
832.71 & 0.73 & \bf{809.00} & 
1.11 & 2.93\\\bf{PROM.} & 
\bf{769.73} & \bf{0.80} & \bf{773.42} & \bf{0.78} & \bf{758.54} & \bf{1.36} & \bf{1.83}\\[1ex]\hline
\end{tabular}
\label{table:nonlin}
\end{table} \clearpage
\begin{table}[ht]
\caption{Resultados de la ejecución de la metaheurística IGA, utilizando instancias de SalhiNagy con la configuración -n 200 -p 40 -cprob 80.0 -mprob 50.0}
\centering
\small
\begin{tabular}{c c c c c c c c}
\hline\hline
Instancia & Costo mínimo & Tiempo(seg.) & Costo promedio & Tiempo promedio(seg.) & CME & \%G & \%GP \\ [0.5ex]
\hline
CMT1X & 478.97 & 0.84 & 
483.82 & 0.79 & \bf{470.48} & 
1.80 & 2.83\\CMT1Y & 478.40 & 0.80 & 
484.60 & 0.67 & \bf{470.48} & 
1.68 & 3.00\\CMT2X & 715.71 & 1.04 & 
719.55 & 1.39 & \bf{682.39} & 
4.88 & 5.45\\CMT2Y & 704.27 & 1.42 & 
712.33 & 1.27 & \bf{682.39} & 
3.21 & 4.39\\CMT3X & 734.04 & 3.27 & 
745.14 & 3.08 & \bf{719.06} & 
2.08 & 3.63\\CMT3Y & 737.37 & 3.35 & 
742.58 & 2.98 & \bf{719.06} & 
2.55 & 3.27\\CMT4X & 901.26 & 8.50 & 
914.07 & 8.25 & \bf{854.21} & 
5.51 & 7.01\\CMT4Y & 913.38 & 8.14 & 
920.40 & 8.27 & \bf{852.46} & 
7.15 & 7.97\\CMT5X & 1119.46 & 16.82 & 
1128.61 & 16.76 & \bf{1030.56} & 
8.63 & 9.51\\CMT5Y & 1115.86 & 16.68 & 
1126.93 & 18.07 & \bf{1031.69} & 
8.16 & 9.23\\CMT11X & 881.73 & 4.67 & 
890.98 & 5.53 & \bf{831.09} & 
6.09 & 7.21\\CMT11Y & 900.90 & 6.19 & 
904.88 & 5.94 & \bf{829.85} & 
8.56 & 9.04\\CMT12X & 678.62 & 3.22 & 
685.84 & 3.11 & \bf{658.83} & 
3.00 & 4.10\\CMT12Y & 675.72 & 2.86 & 
681.43 & 3.02 & \bf{660.47} & 
2.31 & 3.17\\\bf{PROM.} & 
\bf{788.26} & \bf{5.56} & \bf{795.80} & \bf{5.65} & \bf{749.50} & \bf{4.69} & \bf{5.70}\\[1ex]\hline
\end{tabular}
\label{table:nonlin}
\end{table} \clearpage
\begin{table}[ht]
\caption{Resultados de la ejecución de la metaheurística IGA, utilizando instancias de Dethloff con la configuración -n 200 -p 40 -cprob 80.0 -mprob 60.0}
\centering
\small
\begin{tabular}{c c c c c c c c}
\hline\hline
Instancia & Costo mínimo & Tiempo(seg.) & Costo promedio & Tiempo promedio(seg.) & CME & \%G & \%GP \\ [0.5ex]
\hline
SCA3-0 & 640.55 & 0.56 & 
640.84 & 0.78 & \bf{635.62} & 
0.78 & 0.82\\SCA3-1 & \bf{697.84} & 0.94 & 
703.18 & 0.82 & 697.84 & 0.00
 & 0.77\\SCA3-2 & 661.13 & 1.46 & 
666.18 & 0.96 & \bf{659.34} & 
0.27 & 1.04\\SCA3-3 & \bf{680.04} & 0.48 & 
680.04 & 0.70 & 680.04 & 0.00
 & 0.00\\
SCA3-4 & \bf{690.50} & 0.90 & 
690.50 & 0.77 & 690.50 & 0.00
 & 0.00\\
SCA3-5 & 670.10 & 0.91 & 
670.10 & 0.78 & \bf{659.90} & 
1.55 & 1.55\\SCA3-6 & 652.94 & 0.69 & 
656.43 & 0.78 & \bf{651.09} & 
0.28 & 0.82\\SCA3-7 & 666.15 & 0.94 & 
666.26 & 0.70 & \bf{659.17} & 
1.06 & 1.08\\SCA3-8 & 724.29 & 0.68 & 
726.75 & 0.80 & \bf{719.47} & 
0.67 & 1.01\\SCA3-9 & \bf{681.00} & 0.88 & 
681.00 & 0.88 & 681.00 & 0.00
 & 0.00\\
SCA8-0 & 980.51 & 0.79 & 
982.12 & 0.81 & \bf{961.50} & 
1.98 & 2.15\\SCA8-1 & 1070.29 & 0.76 & 
1076.55 & 0.77 & \bf{1049.65} & 
1.97 & 2.56\\SCA8-2 & 1063.38 & 0.92 & 
1065.19 & 0.79 & \bf{1039.64} & 
2.28 & 2.46\\SCA8-3 & 1009.27 & 0.97 & 
1009.52 & 0.89 & \bf{983.34} & 
2.64 & 2.66\\SCA8-4 & 1074.11 & 0.96 & 
1083.65 & 0.81 & \bf{1065.49} & 
0.81 & 1.70\\SCA8-5 & 1051.33 & 0.94 & 
1051.33 & 0.69 & \bf{1027.08} & 
2.36 & 2.36\\SCA8-6 & 987.84 & 0.94 & 
987.84 & 0.83 & \bf{971.82} & 
1.65 & 1.65\\SCA8-7 & 1070.92 & 0.71 & 
1070.92 & 0.84 & \bf{1051.28} & 
1.87 & 1.87\\SCA8-8 & 1084.41 & 0.80 & 
1084.41 & 0.77 & \bf{1071.18} & 
1.24 & 1.24\\SCA8-9 & 1092.85 & 0.94 & 
1098.51 & 0.93 & \bf{1060.50} & 
3.05 & 3.58\\CON3-0 & 619.09 & 0.95 & 
619.09 & 0.78 & \bf{616.52} & 
0.42 & 0.42\\CON3-1 & 560.75 & 0.70 & 
560.75 & 0.70 & \bf{554.47} & 
1.13 & 1.13\\CON3-2 & 521.38 & 0.99 & 
521.38 & 0.86 & \bf{518.00} & 
0.65 & 0.65\\CON3-3 & 591.20 & 0.94 & 
596.93 & 0.84 & \bf{591.19} & 
0.00 & 0.97\\CON3-4 & 592.58 & 0.93 & 
596.90 & 0.89 & \bf{588.79} & 
0.64 & 1.38\\CON3-5 & 567.94 & 0.97 & 
569.87 & 0.86 & \bf{563.70} & 
0.75 & 1.09\\CON3-6 & 504.20 & 1.01 & 
506.88 & 0.94 & \bf{499.05} & 
1.03 & 1.57\\CON3-7 & 577.91 & 0.91 & 
580.91 & 0.78 & \bf{576.48} & 
0.25 & 0.77\\CON3-8 & \bf{523.05} & 0.92 & 
523.46 & 0.86 & 523.05 & 0.00
 & 0.08\\CON3-9 & 582.79 & 0.76 & 
582.79 & 0.80 & \bf{578.24} & 
0.79 & 0.79\\CON8-0 & 880.24 & 0.86 & 
880.24 & 0.85 & \bf{857.17} & 
2.69 & 2.69\\CON8-1 & 767.66 & 0.72 & 
771.00 & 0.94 & \bf{740.85} & 
3.62 & 4.07\\CON8-2 & 727.09 & 0.61 & 
727.09 & 0.69 & \bf{712.89} & 
1.99 & 1.99\\CON8-3 & 834.72 & 0.79 & 
834.72 & 0.84 & \bf{811.07} & 
2.92 & 2.92\\CON8-4 & 780.54 & 0.74 & 
780.54 & 0.91 & \bf{772.25} & 
1.07 & 1.07\\CON8-5 & 763.13 & 0.57 & 
763.13 & 0.83 & \bf{754.88} & 
1.09 & 1.09\\CON8-6 & 703.00 & 1.00 & 
703.00 & 0.96 & \bf{678.92} & 
3.55 & 3.55\\CON8-7 & 816.00 & 0.66 & 
816.00 & 0.73 & \bf{811.96} & 
0.50 & 0.50\\CON8-8 & 784.31 & 0.74 & 
784.31 & 0.81 & \bf{767.53} & 
2.19 & 2.19\\CON8-9 & 829.23 & 1.05 & 
831.70 & 0.82 & \bf{809.00} & 
2.50 & 2.81\\\bf{PROM.} & 
\bf{769.41} & \bf{0.85} & \bf{771.05} & \bf{0.82} & \bf{758.54} & \bf{1.31} & \bf{1.53}\\[1ex]\hline
\end{tabular}
\label{table:nonlin}
\end{table} \clearpage
\begin{table}[ht]
\caption{Resultados de la ejecución de la metaheurística IGA, utilizando instancias de SalhiNagy con la configuración -n 200 -p 40 -cprob 80.0 -mprob 60.0}
\centering
\small
\begin{tabular}{c c c c c c c c}
\hline\hline
Instancia & Costo mínimo & Tiempo(seg.) & Costo promedio & Tiempo promedio(seg.) & CME & \%G & \%GP \\ [0.5ex]
\hline
CMT1X & 481.52 & 1.20 & 
482.61 & 0.85 & \bf{470.48} & 
2.35 & 2.58\\CMT1Y & 484.37 & 0.68 & 
484.73 & 0.73 & \bf{470.48} & 
2.95 & 3.03\\CMT2X & 712.16 & 1.67 & 
718.23 & 1.53 & \bf{682.39} & 
4.36 & 5.25\\CMT2Y & 715.08 & 1.36 & 
715.08 & 1.42 & \bf{682.39} & 
4.79 & 4.79\\CMT3X & 748.92 & 3.30 & 
753.07 & 3.24 & \bf{719.06} & 
4.15 & 4.73\\CMT3Y & 739.93 & 3.20 & 
748.87 & 3.12 & \bf{719.06} & 
2.90 & 4.15\\CMT4X & 905.10 & 8.60 & 
912.14 & 8.31 & \bf{854.21} & 
5.96 & 6.78\\CMT4Y & 893.53 & 8.31 & 
907.93 & 8.69 & \bf{852.46} & 
4.82 & 6.51\\CMT5X & 1095.20 & 17.04 & 
1113.11 & 17.20 & \bf{1030.56} & 
6.27 & 8.01\\CMT5Y & 1109.34 & 16.42 & 
1124.36 & 16.04 & \bf{1031.69} & 
7.53 & 8.98\\CMT11X & 894.21 & 5.48 & 
907.74 & 5.50 & \bf{831.09} & 
7.59 & 9.22\\CMT11Y & 863.94 & 6.25 & 
897.27 & 5.96 & \bf{829.85} & 
4.11 & 8.12\\CMT12X & 680.25 & 3.24 & 
687.32 & 3.12 & \bf{658.83} & 
3.25 & 4.32\\CMT12Y & 673.64 & 3.29 & 
675.87 & 3.23 & \bf{660.47} & 
1.99 & 2.33\\\bf{PROM.} & 
\bf{785.51} & \bf{5.72} & \bf{794.88} & \bf{5.64} & \bf{749.50} & \bf{4.50} & \bf{5.63}\\[1ex]\hline
\end{tabular}
\label{table:nonlin}
\end{table} \clearpage
\begin{table}[ht]
\caption{Resultados de la ejecución de la metaheurística IGA, utilizando instancias de Dethloff con la configuración -n 200 -p 40 -cprob 80.0 -mprob 70.0}
\centering
\small
\begin{tabular}{c c c c c c c c}
\hline\hline
Instancia & Costo mínimo & Tiempo(seg.) & Costo promedio & Tiempo promedio(seg.) & CME & \%G & \%GP \\ [0.5ex]
\hline
SCA3-0 & 640.55 & 0.50 & 
640.84 & 0.74 & \bf{635.62} & 
0.78 & 0.82\\SCA3-1 & 701.53 & 0.91 & 
701.53 & 0.89 & \bf{697.84} & 
0.53 & 0.53\\SCA3-2 & 661.13 & 0.76 & 
661.13 & 0.71 & \bf{659.34} & 
0.27 & 0.27\\SCA3-3 & 681.74 & 0.54 & 
682.28 & 0.66 & \bf{680.04} & 
0.25 & 0.33\\SCA3-4 & \bf{690.50} & 0.70 & 
690.50 & 0.77 & 690.50 & 0.00
 & 0.00\\
SCA3-5 & 665.64 & 0.87 & 
665.64 & 0.82 & \bf{659.90} & 
0.87 & 0.87\\SCA3-6 & 652.94 & 0.93 & 
656.18 & 0.88 & \bf{651.09} & 
0.28 & 0.78\\SCA3-7 & 666.15 & 0.49 & 
668.96 & 0.62 & \bf{659.17} & 
1.06 & 1.48\\SCA3-8 & 722.35 & 0.96 & 
725.53 & 0.90 & \bf{719.47} & 
0.40 & 0.84\\SCA3-9 & 690.83 & 0.64 & 
690.83 & 0.65 & \bf{681.00} & 
1.44 & 1.44\\SCA8-0 & 983.21 & 0.99 & 
1000.46 & 0.79 & \bf{961.50} & 
2.26 & 4.05\\SCA8-1 & 1083.64 & 0.85 & 
1088.05 & 0.84 & \bf{1049.65} & 
3.24 & 3.66\\SCA8-2 & 1057.07 & 0.72 & 
1057.07 & 0.80 & \bf{1039.64} & 
1.68 & 1.68\\SCA8-3 & 1020.23 & 0.56 & 
1020.23 & 0.68 & \bf{983.34} & 
3.75 & 3.75\\SCA8-4 & \bf{1065.49} & 0.74 & 
1070.07 & 0.78 & 1065.49 & 0.00
 & 0.43\\SCA8-5 & 1052.95 & 0.50 & 
1065.92 & 0.70 & \bf{1027.08} & 
2.52 & 3.78\\SCA8-6 & 1002.65 & 0.88 & 
1002.96 & 0.81 & \bf{971.82} & 
3.17 & 3.20\\SCA8-7 & 1083.05 & 0.79 & 
1083.42 & 0.84 & \bf{1051.28} & 
3.02 & 3.06\\SCA8-8 & 1089.55 & 0.65 & 
1089.55 & 0.81 & \bf{1071.18} & 
1.71 & 1.71\\SCA8-9 & 1073.62 & 0.70 & 
1078.58 & 0.76 & \bf{1060.50} & 
1.24 & 1.70\\CON3-0 & 621.22 & 0.61 & 
625.91 & 0.74 & \bf{616.52} & 
0.76 & 1.52\\CON3-1 & 560.75 & 0.56 & 
560.75 & 0.73 & \bf{554.47} & 
1.13 & 1.13\\CON3-2 & 521.38 & 0.62 & 
521.38 & 0.69 & \bf{518.00} & 
0.65 & 0.65\\CON3-3 & 592.41 & 0.82 & 
596.63 & 0.79 & \bf{591.19} & 
0.21 & 0.92\\CON3-4 & 592.58 & 0.93 & 
594.19 & 0.85 & \bf{588.79} & 
0.64 & 0.92\\CON3-5 & 564.89 & 0.69 & 
566.41 & 0.71 & \bf{563.70} & 
0.21 & 0.48\\CON3-6 & 505.14 & 0.98 & 
505.37 & 0.85 & \bf{499.05} & 
1.22 & 1.27\\CON3-7 & 582.14 & 0.74 & 
584.60 & 0.72 & \bf{576.48} & 
0.98 & 1.41\\CON3-8 & 524.59 & 0.49 & 
529.58 & 0.62 & \bf{523.05} & 
0.29 & 1.25\\CON3-9 & 588.18 & 0.59 & 
589.84 & 0.69 & \bf{578.24} & 
1.72 & 2.01\\CON8-0 & 880.45 & 0.69 & 
881.85 & 0.80 & \bf{857.17} & 
2.72 & 2.88\\CON8-1 & 757.27 & 0.86 & 
758.61 & 0.87 & \bf{740.85} & 
2.22 & 2.40\\CON8-2 & 717.31 & 0.70 & 
717.31 & 0.78 & \bf{712.89} & 
0.62 & 0.62\\CON8-3 & 817.21 & 1.04 & 
817.42 & 0.95 & \bf{811.07} & 
0.76 & 0.78\\CON8-4 & 786.76 & 0.70 & 
786.76 & 0.78 & \bf{772.25} & 
1.88 & 1.88\\CON8-5 & 772.35 & 0.84 & 
772.35 & 0.66 & \bf{754.88} & 
2.31 & 2.31\\CON8-6 & 706.51 & 0.78 & 
706.51 & 0.94 & \bf{678.92} & 
4.06 & 4.06\\CON8-7 & 816.49 & 0.84 & 
816.49 & 0.93 & \bf{811.96} & 
0.56 & 0.56\\CON8-8 & 784.71 & 0.83 & 
789.08 & 0.79 & \bf{767.53} & 
2.24 & 2.81\\CON8-9 & 830.85 & 0.96 & 
830.85 & 0.98 & \bf{809.00} & 
2.70 & 2.70\\\bf{PROM.} & 
\bf{770.20} & \bf{0.75} & \bf{772.29} & \bf{0.78} & \bf{758.54} & \bf{1.41} & \bf{1.67}\\[1ex]\hline
\end{tabular}
\label{table:nonlin}
\end{table} \clearpage
\begin{table}[ht]
\caption{Resultados de la ejecución de la metaheurística IGA, utilizando instancias de SalhiNagy con la configuración -n 200 -p 40 -cprob 80.0 -mprob 70.0}
\centering
\small
\begin{tabular}{c c c c c c c c}
\hline\hline
Instancia & Costo mínimo & Tiempo(seg.) & Costo promedio & Tiempo promedio(seg.) & CME & \%G & \%GP \\ [0.5ex]
\hline
CMT1X & 475.37 & 0.65 & 
477.33 & 0.71 & \bf{470.48} & 
1.04 & 1.46\\CMT1Y & 484.34 & 0.59 & 
490.80 & 0.56 & \bf{470.48} & 
2.95 & 4.32\\CMT2X & 697.82 & 1.38 & 
707.13 & 1.34 & \bf{682.39} & 
2.26 & 3.63\\CMT2Y & 693.94 & 1.70 & 
702.34 & 1.51 & \bf{682.39} & 
1.69 & 2.92\\CMT3X & 737.60 & 3.10 & 
744.92 & 3.29 & \bf{719.06} & 
2.58 & 3.60\\CMT3Y & 741.28 & 3.46 & 
749.38 & 3.36 & \bf{719.06} & 
3.09 & 4.22\\CMT4X & 899.42 & 7.73 & 
913.83 & 8.37 & \bf{854.21} & 
5.29 & 6.98\\CMT4Y & 918.39 & 8.38 & 
921.08 & 7.97 & \bf{852.46} & 
7.73 & 8.05\\CMT5X & 1109.24 & 16.31 & 
1118.49 & 16.70 & \bf{1030.56} & 
7.63 & 8.53\\CMT5Y & 1113.23 & 19.08 & 
1125.11 & 18.23 & \bf{1031.69} & 
7.90 & 9.06\\CMT11X & 900.87 & 4.74 & 
918.16 & 5.13 & \bf{831.09} & 
8.40 & 10.48\\CMT11Y & 892.42 & 5.96 & 
904.04 & 5.72 & \bf{829.85} & 
7.54 & 8.94\\CMT12X & 681.65 & 2.68 & 
683.67 & 3.19 & \bf{658.83} & 
3.46 & 3.77\\CMT12Y & 674.32 & 3.32 & 
679.97 & 3.27 & \bf{660.47} & 
2.10 & 2.95\\\bf{PROM.} & 
\bf{787.14} & \bf{5.65} & \bf{795.45} & \bf{5.67} & \bf{749.50} & \bf{4.55} & \bf{5.64}\\[1ex]\hline
\end{tabular}
\label{table:nonlin}
\end{table} \clearpage
\begin{table}[ht]
\caption{Resultados de la ejecución de la metaheurística IGA, utilizando instancias de Dethloff con la configuración -n 200 -p 40 -cprob 80.0 -mprob 80.0}
\centering
\small
\begin{tabular}{c c c c c c c c}
\hline\hline
Instancia & Costo mínimo & Tiempo(seg.) & Costo promedio & Tiempo promedio(seg.) & CME & \%G & \%GP \\ [0.5ex]
\hline
SCA3-0 & 640.55 & 0.94 & 
640.84 & 0.81 & \bf{635.62} & 
0.78 & 0.82\\SCA3-1 & 700.50 & 0.86 & 
700.50 & 0.79 & \bf{697.84} & 
0.38 & 0.38\\SCA3-2 & 664.18 & 0.87 & 
667.08 & 0.84 & \bf{659.34} & 
0.73 & 1.17\\SCA3-3 & 681.16 & 0.68 & 
681.86 & 0.78 & \bf{680.04} & 
0.16 & 0.27\\SCA3-4 & \bf{690.50} & 0.89 & 
691.02 & 0.81 & 690.50 & 0.00
 & 0.08\\SCA3-5 & 670.10 & 0.94 & 
678.90 & 0.80 & \bf{659.90} & 
1.55 & 2.88\\SCA3-6 & 652.94 & 0.91 & 
656.30 & 0.92 & \bf{651.09} & 
0.28 & 0.80\\SCA3-7 & 666.15 & 0.79 & 
666.15 & 0.67 & \bf{659.17} & 
1.06 & 1.06\\SCA3-8 & 723.99 & 0.74 & 
728.34 & 0.90 & \bf{719.47} & 
0.63 & 1.23\\SCA3-9 & \bf{681.00} & 0.91 & 
683.03 & 0.76 & 681.00 & 0.00
 & 0.30\\SCA8-0 & 1005.16 & 0.90 & 
1005.16 & 0.90 & \bf{961.50} & 
4.54 & 4.54\\SCA8-1 & 1069.65 & 0.70 & 
1069.65 & 0.91 & \bf{1049.65} & 
1.91 & 1.91\\SCA8-2 & 1050.37 & 0.95 & 
1052.42 & 0.87 & \bf{1039.64} & 
1.03 & 1.23\\SCA8-3 & 1015.48 & 0.83 & 
1024.27 & 0.87 & \bf{983.34} & 
3.27 & 4.16\\SCA8-4 & 1082.25 & 0.50 & 
1085.24 & 0.66 & \bf{1065.49} & 
1.57 & 1.85\\SCA8-5 & 1073.66 & 0.76 & 
1078.59 & 0.91 & \bf{1027.08} & 
4.54 & 5.02\\SCA8-6 & 994.21 & 0.76 & 
994.47 & 0.68 & \bf{971.82} & 
2.30 & 2.33\\SCA8-7 & 1070.67 & 0.91 & 
1070.67 & 0.72 & \bf{1051.28} & 
1.84 & 1.84\\SCA8-8 & 1094.14 & 0.79 & 
1094.43 & 0.58 & \bf{1071.18} & 
2.14 & 2.17\\SCA8-9 & 1069.20 & 0.96 & 
1069.20 & 0.83 & \bf{1060.50} & 
0.82 & 0.82\\CON3-0 & 621.82 & 0.62 & 
623.65 & 0.77 & \bf{616.52} & 
0.86 & 1.16\\CON3-1 & 560.75 & 0.91 & 
560.75 & 0.84 & \bf{554.47} & 
1.13 & 1.13\\CON3-2 & 521.38 & 0.78 & 
521.38 & 0.84 & \bf{518.00} & 
0.65 & 0.65\\CON3-3 & 604.09 & 0.71 & 
604.09 & 0.71 & \bf{591.19} & 
2.18 & 2.18\\CON3-4 & 592.58 & 0.72 & 
594.83 & 0.77 & \bf{588.79} & 
0.64 & 1.02\\CON3-5 & 564.89 & 0.91 & 
567.44 & 0.90 & \bf{563.70} & 
0.21 & 0.66\\CON3-6 & 505.01 & 0.98 & 
507.37 & 0.94 & \bf{499.05} & 
1.19 & 1.67\\CON3-7 & 578.41 & 0.70 & 
580.37 & 0.81 & \bf{576.48} & 
0.33 & 0.67\\CON3-8 & 524.38 & 0.62 & 
527.04 & 0.71 & \bf{523.05} & 
0.25 & 0.76\\CON3-9 & 591.55 & 0.96 & 
591.98 & 0.91 & \bf{578.24} & 
2.30 & 2.38\\CON8-0 & 867.51 & 1.03 & 
867.51 & 0.89 & \bf{857.17} & 
1.21 & 1.21\\CON8-1 & 757.48 & 1.01 & 
766.41 & 0.90 & \bf{740.85} & 
2.24 & 3.45\\CON8-2 & 723.99 & 1.04 & 
723.99 & 0.83 & \bf{712.89} & 
1.56 & 1.56\\CON8-3 & 836.88 & 0.94 & 
836.88 & 0.94 & \bf{811.07} & 
3.18 & 3.18\\CON8-4 & 778.61 & 0.58 & 
786.52 & 0.82 & \bf{772.25} & 
0.82 & 1.85\\CON8-5 & 776.69 & 0.88 & 
779.39 & 0.84 & \bf{754.88} & 
2.89 & 3.25\\CON8-6 & 688.46 & 0.99 & 
694.17 & 0.98 & \bf{678.92} & 
1.41 & 2.25\\CON8-7 & 815.06 & 1.01 & 
817.02 & 0.88 & \bf{811.96} & 
0.38 & 0.62\\CON8-8 & 792.14 & 1.00 & 
795.90 & 0.90 & \bf{767.53} & 
3.21 & 3.70\\CON8-9 & 835.09 & 1.00 & 
839.24 & 0.94 & \bf{809.00} & 
3.22 & 3.74\\\bf{PROM.} & 
\bf{770.82} & \bf{0.85} & \bf{773.10} & \bf{0.83} & \bf{758.54} & \bf{1.49} & \bf{1.80}\\[1ex]\hline
\end{tabular}
\label{table:nonlin}
\end{table} \clearpage
\begin{table}[ht]
\caption{Resultados de la ejecución de la metaheurística IGA, utilizando instancias de SalhiNagy con la configuración -n 200 -p 40 -cprob 80.0 -mprob 80.0}
\centering
\small
\begin{tabular}{c c c c c c c c}
\hline\hline
Instancia & Costo mínimo & Tiempo(seg.) & Costo promedio & Tiempo promedio(seg.) & CME & \%G & \%GP \\ [0.5ex]
\hline
CMT1X & 476.49 & 0.86 & 
483.27 & 0.86 & \bf{470.48} & 
1.28 & 2.72\\CMT1Y & 480.26 & 0.82 & 
484.08 & 0.67 & \bf{470.48} & 
2.08 & 2.89\\CMT2X & 711.23 & 1.51 & 
713.40 & 1.60 & \bf{682.39} & 
4.23 & 4.54\\CMT2Y & 712.21 & 1.51 & 
722.05 & 1.51 & \bf{682.39} & 
4.37 & 5.81\\CMT3X & 731.44 & 3.52 & 
740.17 & 3.38 & \bf{719.06} & 
1.72 & 2.94\\CMT3Y & 733.75 & 2.50 & 
743.27 & 3.20 & \bf{719.06} & 
2.04 & 3.37\\CMT4X & 910.09 & 9.19 & 
919.67 & 8.63 & \bf{854.21} & 
6.54 & 7.66\\CMT4Y & 915.88 & 7.98 & 
919.70 & 8.22 & \bf{852.46} & 
7.44 & 7.89\\CMT5X & 1091.49 & 17.08 & 
1125.67 & 17.03 & \bf{1030.56} & 
5.91 & 9.23\\CMT5Y & 1112.22 & 18.14 & 
1116.67 & 17.20 & \bf{1031.69} & 
7.81 & 8.24\\CMT11X & 919.92 & 5.39 & 
930.93 & 5.67 & \bf{831.09} & 
10.69 & 12.01\\CMT11Y & 900.03 & 5.78 & 
918.80 & 5.68 & \bf{829.85} & 
8.46 & 10.72\\CMT12X & 676.87 & 3.08 & 
681.53 & 3.39 & \bf{658.83} & 
2.74 & 3.45\\CMT12Y & 674.53 & 3.40 & 
676.64 & 3.21 & \bf{660.47} & 
2.13 & 2.45\\\bf{PROM.} & 
\bf{789.03} & \bf{5.77} & \bf{798.28} & \bf{5.73} & \bf{749.50} & \bf{4.82} & \bf{5.99}\\[1ex]\hline
\end{tabular}
\label{table:nonlin}
\end{table} \clearpage
\begin{table}[ht]
\caption{Resultados de la ejecución de la metaheurística IGA, utilizando instancias de Dethloff con la configuración -n 200 -p 40 -cprob 80.0 -mprob 90.0}
\centering
\small
\begin{tabular}{c c c c c c c c}
\hline\hline
Instancia & Costo mínimo & Tiempo(seg.) & Costo promedio & Tiempo promedio(seg.) & CME & \%G & \%GP \\ [0.5ex]
\hline
SCA3-0 & 641.64 & 0.83 & 
641.98 & 0.89 & \bf{635.62} & 
0.95 & 1.00\\SCA3-1 & 700.50 & 0.86 & 
701.07 & 0.85 & \bf{697.84} & 
0.38 & 0.46\\SCA3-2 & 664.21 & 0.94 & 
664.21 & 0.91 & \bf{659.34} & 
0.74 & 0.74\\SCA3-3 & \bf{680.04} & 0.92 & 
680.32 & 0.83 & 680.04 & 0.00
 & 0.04\\SCA3-4 & \bf{690.50} & 0.65 & 
690.50 & 0.81 & 690.50 & 0.00
 & 0.00\\
SCA3-5 & 673.13 & 0.94 & 
678.75 & 0.81 & \bf{659.90} & 
2.00 & 2.86\\SCA3-6 & 652.94 & 0.95 & 
656.07 & 0.86 & \bf{651.09} & 
0.28 & 0.77\\SCA3-7 & 669.89 & 0.91 & 
669.89 & 0.78 & \bf{659.17} & 
1.63 & 1.63\\SCA3-8 & 726.88 & 0.84 & 
726.88 & 0.84 & \bf{719.47} & 
1.03 & 1.03\\SCA3-9 & \bf{681.00} & 0.88 & 
681.00 & 0.88 & 681.00 & 0.00
 & 0.00\\
SCA8-0 & 1000.99 & 0.94 & 
1002.80 & 0.87 & \bf{961.50} & 
4.11 & 4.30\\SCA8-1 & 1073.51 & 1.05 & 
1081.40 & 0.95 & \bf{1049.65} & 
2.27 & 3.02\\SCA8-2 & 1054.47 & 0.97 & 
1054.47 & 0.95 & \bf{1039.64} & 
1.43 & 1.43\\SCA8-3 & 1027.54 & 1.00 & 
1029.39 & 0.98 & \bf{983.34} & 
4.49 & 4.68\\SCA8-4 & 1074.11 & 0.74 & 
1074.11 & 0.87 & \bf{1065.49} & 
0.81 & 0.81\\SCA8-5 & 1054.14 & 0.65 & 
1054.14 & 0.75 & \bf{1027.08} & 
2.63 & 2.63\\SCA8-6 & 994.62 & 0.73 & 
995.45 & 0.94 & \bf{971.82} & 
2.35 & 2.43\\SCA8-7 & 1070.67 & 0.96 & 
1070.86 & 0.89 & \bf{1051.28} & 
1.84 & 1.86\\SCA8-8 & 1091.12 & 0.85 & 
1096.61 & 0.82 & \bf{1071.18} & 
1.86 & 2.37\\SCA8-9 & 1072.10 & 0.92 & 
1072.90 & 0.79 & \bf{1060.50} & 
1.09 & 1.17\\CON3-0 & 621.82 & 0.93 & 
630.14 & 0.84 & \bf{616.52} & 
0.86 & 2.21\\CON3-1 & 557.38 & 0.88 & 
560.13 & 0.87 & \bf{554.47} & 
0.52 & 1.02\\CON3-2 & \bf{\underline{87.00}} & Command & 
377.57 & 0.66 & 518.00 & 
\bf{-83.20} & \bf{-27.11}\\CON3-3 & 594.31 & 0.94 & 
597.94 & 0.80 & \bf{591.19} & 
0.53 & 1.14\\CON3-4 & 592.58 & 0.67 & 
593.78 & 1.01 & \bf{588.79} & 
0.64 & 0.85\\CON3-5 & \bf{563.70} & 0.80 & 
566.88 & 0.85 & 563.70 & 0.00
 & 0.56\\CON3-6 & 506.04 & 0.98 & 
506.04 & 0.98 & \bf{499.05} & 
1.40 & 1.40\\CON3-7 & 582.14 & 0.91 & 
584.34 & 0.87 & \bf{576.48} & 
0.98 & 1.36\\CON3-8 & 524.59 & 0.78 & 
525.47 & 0.92 & \bf{523.05} & 
0.29 & 0.46\\CON3-9 & 588.18 & 0.94 & 
588.18 & 0.90 & \bf{578.24} & 
1.72 & 1.72\\CON8-0 & 888.91 & 0.59 & 
888.91 & 0.84 & \bf{857.17} & 
3.70 & 3.70\\CON8-1 & 753.43 & 1.05 & 
754.78 & 1.04 & \bf{740.85} & 
1.70 & 1.88\\CON8-2 & 726.10 & 0.87 & 
728.68 & 0.83 & \bf{712.89} & 
1.85 & 2.22\\CON8-3 & 834.50 & 1.00 & 
835.07 & 0.94 & \bf{811.07} & 
2.89 & 2.96\\CON8-4 & 802.70 & 0.74 & 
802.70 & 0.90 & \bf{772.25} & 
3.94 & 3.94\\CON8-5 & 764.92 & 0.88 & 
768.31 & 0.90 & \bf{754.88} & 
1.33 & 1.78\\CON8-6 & 694.88 & 0.84 & 
694.88 & 0.87 & \bf{678.92} & 
2.35 & 2.35\\CON8-7 & 816.00 & 0.96 & 
816.00 & 0.89 & \bf{811.96} & 
0.50 & 0.50\\CON8-8 & 788.84 & 0.78 & 
791.99 & 0.96 & \bf{767.53} & 
2.78 & 3.19\\CON8-9 & 833.76 & 1.00 & 
840.48 & 0.99 & \bf{809.00} & 
3.06 & 3.89\\\bf{PROM.} & 
\bf{760.39} & \bf{0.85} & \bf{769.38} & \bf{0.88} & \bf{758.54} & \bf{-0.56} & \bf{1.08}\\[1ex]\hline
\end{tabular}
\label{table:nonlin}
\end{table} \clearpage
\begin{table}[ht]
\caption{Resultados de la ejecución de la metaheurística IGA, utilizando instancias de SalhiNagy con la configuración -n 200 -p 40 -cprob 80.0 -mprob 90.0}
\centering
\small
\begin{tabular}{c c c c c c c c}
\hline\hline
Instancia & Costo mínimo & Tiempo(seg.) & Costo promedio & Tiempo promedio(seg.) & CME & \%G & \%GP \\ [0.5ex]
\hline
CMT1X & 476.38 & 0.81 & 
481.00 & 0.82 & \bf{470.48} & 
1.25 & 2.24\\CMT1Y & 475.37 & 0.82 & 
486.16 & 0.76 & \bf{470.48} & 
1.04 & 3.33\\CMT2X & 702.62 & 1.39 & 
708.92 & 1.52 & \bf{682.39} & 
2.96 & 3.89\\CMT2Y & 698.31 & 1.48 & 
713.96 & 1.58 & \bf{682.39} & 
2.33 & 4.63\\CMT3X & 731.30 & 3.42 & 
740.83 & 3.38 & \bf{719.06} & 
1.70 & 3.03\\CMT3Y & 733.75 & 3.41 & 
740.68 & 3.32 & \bf{719.06} & 
2.04 & 3.01\\CMT4X & 898.39 & 9.37 & 
913.27 & 8.75 & \bf{854.21} & 
5.17 & 6.91\\CMT4Y & 902.97 & 8.85 & 
910.48 & 8.21 & \bf{852.46} & 
5.93 & 6.81\\CMT5X & 1104.91 & 17.00 & 
1125.64 & 16.79 & \bf{1030.56} & 
7.21 & 9.23\\CMT5Y & 1104.99 & 18.04 & 
1118.51 & 17.74 & \bf{1031.69} & 
7.10 & 8.42\\CMT11X & 919.33 & 5.60 & 
932.29 & 5.51 & \bf{831.09} & 
10.62 & 12.18\\CMT11Y & 890.95 & 6.06 & 
911.11 & 5.82 & \bf{829.85} & 
7.36 & 9.79\\CMT12X & 675.87 & 3.46 & 
677.73 & 3.16 & \bf{658.83} & 
2.59 & 2.87\\CMT12Y & 678.57 & 3.44 & 
682.95 & 3.19 & \bf{660.47} & 
2.74 & 3.40\\\bf{PROM.} & 
\bf{785.27} & \bf{5.94} & \bf{795.96} & \bf{5.75} & \bf{749.50} & \bf{4.29} & \bf{5.69}\\[1ex]\hline
\end{tabular}
\label{table:nonlin}
\end{table} \clearpage
\begin{table}[ht]
\caption{Resultados de la ejecución de la metaheurística IGA, utilizando instancias de Dethloff con la configuración -n 200 -p 40 -cprob 80.0 -mprob 100.0}
\centering
\small
\begin{tabular}{c c c c c c c c}
\hline\hline
Instancia & Costo mínimo & Tiempo(seg.) & Costo promedio & Tiempo promedio(seg.) & CME & \%G & \%GP \\ [0.5ex]
\hline
SCA3-0 & 640.55 & 0.98 & 
640.84 & 0.91 & \bf{635.62} & 
0.78 & 0.82\\SCA3-1 & \bf{697.84} & 0.71 & 
700.09 & 0.80 & 697.84 & 0.00
 & 0.32\\SCA3-2 & \bf{659.34} & 0.75 & 
659.34 & 0.75 & 659.34 & 0.00
 & 0.00\\
SCA3-3 & \bf{680.04} & 0.56 & 
681.32 & 0.83 & 680.04 & 0.00
 & 0.19\\SCA3-4 & \bf{690.50} & 0.62 & 
691.53 & 0.71 & 690.50 & 0.00
 & 0.15\\SCA3-5 & 670.10 & 0.69 & 
670.10 & 0.83 & \bf{659.90} & 
1.55 & 1.55\\SCA3-6 & 652.94 & 0.92 & 
653.91 & 0.85 & \bf{651.09} & 
0.28 & 0.43\\SCA3-7 & 666.15 & 0.91 & 
666.26 & 0.89 & \bf{659.17} & 
1.06 & 1.08\\SCA3-8 & 724.29 & 0.92 & 
725.48 & 0.91 & \bf{719.47} & 
0.67 & 0.83\\SCA3-9 & \bf{681.00} & 0.83 & 
681.00 & 0.81 & 681.00 & 0.00
 & 0.00\\
SCA8-0 & 996.18 & 1.02 & 
1011.56 & 0.95 & \bf{961.50} & 
3.61 & 5.21\\SCA8-1 & 1063.21 & 0.81 & 
1063.21 & 0.83 & \bf{1049.65} & 
1.29 & 1.29\\SCA8-2 & 1054.47 & 0.98 & 
1054.80 & 0.83 & \bf{1039.64} & 
1.43 & 1.46\\SCA8-3 & 1021.35 & 1.00 & 
1024.78 & 0.93 & \bf{983.34} & 
3.87 & 4.21\\SCA8-4 & 1092.49 & 0.94 & 
1100.29 & 0.92 & \bf{1065.49} & 
2.53 & 3.27\\SCA8-5 & 1052.74 & 0.91 & 
1057.97 & 0.88 & \bf{1027.08} & 
2.50 & 3.01\\SCA8-6 & 982.67 & 0.54 & 
985.11 & 0.77 & \bf{971.82} & 
1.12 & 1.37\\SCA8-7 & 1070.89 & 0.94 & 
1070.89 & 0.91 & \bf{1051.28} & 
1.87 & 1.87\\SCA8-8 & 1094.27 & 0.65 & 
1094.27 & 0.65 & \bf{1071.18} & 
2.16 & 2.16\\SCA8-9 & 1068.70 & 0.95 & 
1068.70 & 0.96 & \bf{1060.50} & 
0.77 & 0.77\\CON3-0 & 620.76 & 0.97 & 
626.67 & 0.93 & \bf{616.52} & 
0.69 & 1.65\\CON3-1 & 560.61 & 0.93 & 
560.64 & 0.92 & \bf{554.47} & 
1.11 & 1.11\\CON3-2 & 523.99 & 0.97 & 
524.57 & 0.95 & \bf{518.00} & 
1.16 & 1.27\\CON3-3 & 591.20 & 0.73 & 
601.30 & 0.78 & \bf{591.19} & 
0.00 & 1.71\\CON3-4 & 598.22 & 0.92 & 
599.98 & 0.91 & \bf{588.79} & 
1.60 & 1.90\\CON3-5 & 564.88 & 0.96 & 
566.41 & 0.94 & \bf{563.70} & 
0.21 & 0.48\\CON3-6 & \bf{499.05} & 0.98 & 
504.63 & 0.98 & 499.05 & 0.00
 & 1.12\\CON3-7 & 582.33 & 0.95 & 
585.09 & 0.94 & \bf{576.48} & 
1.01 & 1.49\\CON3-8 & 532.86 & 0.98 & 
532.86 & 0.92 & \bf{523.05} & 
1.88 & 1.88\\CON3-9 & 588.11 & 0.78 & 
589.94 & 0.79 & \bf{578.24} & 
1.71 & 2.02\\CON8-0 & 880.39 & 0.98 & 
889.89 & 0.85 & \bf{857.17} & 
2.71 & 3.82\\CON8-1 & 758.63 & 0.99 & 
758.63 & 0.93 & \bf{740.85} & 
2.40 & 2.40\\CON8-2 & 723.22 & 1.04 & 
723.22 & 1.00 & \bf{712.89} & 
1.45 & 1.45\\CON8-3 & 840.14 & 0.92 & 
840.14 & 0.95 & \bf{811.07} & 
3.58 & 3.58\\CON8-4 & 773.27 & 0.97 & 
773.27 & 0.94 & \bf{772.25} & 
0.13 & 0.13\\CON8-5 & 764.29 & 0.98 & 
769.99 & 0.89 & \bf{754.88} & 
1.25 & 2.00\\CON8-6 & 696.28 & 0.96 & 
696.28 & 0.92 & \bf{678.92} & 
2.56 & 2.56\\CON8-7 & 815.54 & 0.76 & 
815.54 & 0.82 & \bf{811.96} & 
0.44 & 0.44\\CON8-8 & 783.90 & 0.80 & 
796.00 & 0.95 & \bf{767.53} & 
2.13 & 3.71\\CON8-9 & 823.85 & 0.88 & 
825.23 & 0.90 & \bf{809.00} & 
1.84 & 2.01\\\bf{PROM.} & 
\bf{769.53} & \bf{0.88} & \bf{772.04} & \bf{0.88} & \bf{758.54} & \bf{1.33} & \bf{1.67}\\[1ex]\hline
\end{tabular}
\label{table:nonlin}
\end{table} \clearpage
\begin{table}[ht]
\caption{Resultados de la ejecución de la metaheurística IGA, utilizando instancias de SalhiNagy con la configuración -n 200 -p 40 -cprob 80.0 -mprob 100.0}
\centering
\small
\begin{tabular}{c c c c c c c c}
\hline\hline
Instancia & Costo mínimo & Tiempo(seg.) & Costo promedio & Tiempo promedio(seg.) & CME & \%G & \%GP \\ [0.5ex]
\hline
CMT1X & 488.78 & 0.84 & 
489.15 & 0.80 & \bf{470.48} & 
3.89 & 3.97\\CMT1Y & 482.26 & 0.80 & 
486.22 & 0.81 & \bf{470.48} & 
2.50 & 3.35\\CMT2X & 711.51 & 1.76 & 
715.09 & 1.66 & \bf{682.39} & 
4.27 & 4.79\\CMT2Y & 709.44 & 1.39 & 
712.68 & 1.61 & \bf{682.39} & 
3.96 & 4.44\\CMT3X & 738.70 & 3.32 & 
741.59 & 3.35 & \bf{719.06} & 
2.73 & 3.13\\CMT3Y & 742.76 & 3.12 & 
749.83 & 3.31 & \bf{719.06} & 
3.30 & 4.28\\CMT4X & 898.57 & 8.44 & 
906.09 & 8.15 & \bf{854.21} & 
5.19 & 6.07\\CMT4Y & 906.00 & 8.68 & 
917.58 & 8.50 & \bf{852.46} & 
6.28 & 7.64\\CMT5X & 1102.97 & 17.41 & 
1125.89 & 16.91 & \bf{1030.56} & 
7.03 & 9.25\\CMT5Y & 1090.50 & 17.53 & 
1120.50 & 17.59 & \bf{1031.69} & 
5.70 & 8.61\\CMT11X & 921.22 & 5.69 & 
923.60 & 5.53 & \bf{831.09} & 
10.84 & 11.13\\CMT11Y & 904.17 & 6.26 & 
914.13 & 6.17 & \bf{829.85} & 
8.96 & 10.16\\CMT12X & 674.11 & 3.63 & 
680.51 & 3.46 & \bf{658.83} & 
2.32 & 3.29\\CMT12Y & 673.70 & 2.98 & 
675.00 & 3.25 & \bf{660.47} & 
2.00 & 2.20\\\bf{PROM.} & 
\bf{788.91} & \bf{5.85} & \bf{796.99} & \bf{5.79} & \bf{749.50} & \bf{4.93} & \bf{5.88}\\[1ex]\hline
\end{tabular}
\label{table:nonlin}
\end{table} \clearpage
\begin{table}[ht]
\caption{Resultados de la ejecución de la metaheurística IGA, utilizando instancias de Dethloff con la configuración -n 200 -p 40 -cprob 90.0 -mprob 10.0}
\centering
\small
\begin{tabular}{c c c c c c c c}
\hline\hline
Instancia & Costo mínimo & Tiempo(seg.) & Costo promedio & Tiempo promedio(seg.) & CME & \%G & \%GP \\ [0.5ex]
\hline
SCA3-0 & 640.55 & 0.71 & 
641.31 & 0.71 & \bf{635.62} & 
0.78 & 0.89\\SCA3-1 & 700.50 & 0.85 & 
700.50 & 0.74 & \bf{697.84} & 
0.38 & 0.38\\SCA3-2 & 674.45 & 0.68 & 
674.95 & 0.69 & \bf{659.34} & 
2.29 & 2.37\\SCA3-3 & \bf{680.04} & 0.72 & 
680.65 & 0.72 & 680.04 & 0.00
 & 0.09\\SCA3-4 & \bf{690.50} & 0.69 & 
690.50 & 0.69 & 690.50 & 0.00
 & 0.00\\
SCA3-5 & 662.75 & 0.74 & 
676.29 & 0.73 & \bf{659.90} & 
0.43 & 2.48\\SCA3-6 & 652.94 & 0.69 & 
652.94 & 0.71 & \bf{651.09} & 
0.28 & 0.28\\SCA3-7 & 667.24 & 0.68 & 
670.64 & 0.70 & \bf{659.17} & 
1.22 & 1.74\\SCA3-8 & 731.10 & 0.70 & 
732.23 & 0.71 & \bf{719.47} & 
1.62 & 1.77\\SCA3-9 & \bf{681.00} & 0.70 & 
683.03 & 0.70 & 681.00 & 0.00
 & 0.30\\SCA8-0 & 999.14 & 0.68 & 
1011.17 & 0.73 & \bf{961.50} & 
3.91 & 5.17\\SCA8-1 & 1054.87 & 0.76 & 
1059.72 & 0.75 & \bf{1049.65} & 
0.50 & 0.96\\SCA8-2 & 1053.59 & 0.74 & 
1053.59 & 0.84 & \bf{1039.64} & 
1.34 & 1.34\\SCA8-3 & 1014.58 & 1.03 & 
1014.58 & 0.94 & \bf{983.34} & 
3.18 & 3.18\\SCA8-4 & 1077.80 & 0.74 & 
1077.98 & 0.76 & \bf{1065.49} & 
1.16 & 1.17\\SCA8-5 & 1058.57 & 0.58 & 
1059.61 & 0.73 & \bf{1027.08} & 
3.07 & 3.17\\SCA8-6 & 978.03 & 0.77 & 
978.03 & 0.84 & \bf{971.82} & 
0.64 & 0.64\\SCA8-7 & 1089.40 & 0.74 & 
1101.11 & 0.72 & \bf{1051.28} & 
3.63 & 4.74\\SCA8-8 & 1092.02 & 0.69 & 
1092.02 & 0.66 & \bf{1071.18} & 
1.95 & 1.95\\SCA8-9 & 1074.19 & 0.54 & 
1074.19 & 0.73 & \bf{1060.50} & 
1.29 & 1.29\\CON3-0 & 617.59 & 0.94 & 
622.33 & 0.78 & \bf{616.52} & 
0.17 & 0.94\\CON3-1 & 561.87 & 0.77 & 
562.20 & 0.69 & \bf{554.47} & 
1.33 & 1.39\\CON3-2 & 521.38 & 0.78 & 
522.32 & 0.78 & \bf{518.00} & 
0.65 & 0.83\\CON3-3 & 591.20 & 0.83 & 
596.88 & 0.76 & \bf{591.19} & 
0.00 & 0.96\\CON3-4 & 592.58 & 0.75 & 
593.38 & 0.75 & \bf{588.79} & 
0.64 & 0.78\\CON3-5 & 568.76 & 0.74 & 
570.30 & 0.72 & \bf{563.70} & 
0.90 & 1.17\\CON3-6 & 504.44 & 0.76 & 
504.79 & 0.76 & \bf{499.05} & 
1.08 & 1.15\\CON3-7 & 578.22 & 0.70 & 
581.35 & 0.71 & \bf{576.48} & 
0.30 & 0.85\\CON3-8 & \bf{523.05} & 0.80 & 
527.00 & 0.77 & 523.05 & 0.00
 & 0.75\\CON3-9 & 588.11 & 0.76 & 
588.64 & 0.71 & \bf{578.24} & 
1.71 & 1.80\\CON8-0 & 870.49 & 0.60 & 
870.49 & 0.70 & \bf{857.17} & 
1.55 & 1.55\\CON8-1 & 755.30 & 0.60 & 
757.18 & 0.72 & \bf{740.85} & 
1.95 & 2.20\\CON8-2 & 727.69 & 0.82 & 
728.10 & 0.87 & \bf{712.89} & 
2.08 & 2.13\\CON8-3 & 831.59 & 0.73 & 
833.33 & 0.76 & \bf{811.07} & 
2.53 & 2.74\\CON8-4 & 793.13 & 0.73 & 
801.38 & 0.75 & \bf{772.25} & 
2.70 & 3.77\\CON8-5 & 761.01 & 0.81 & 
761.01 & 0.78 & \bf{754.88} & 
0.81 & 0.81\\CON8-6 & 701.09 & 0.83 & 
701.09 & 0.80 & \bf{678.92} & 
3.27 & 3.27\\CON8-7 & 830.80 & 0.80 & 
832.28 & 0.74 & \bf{811.96} & 
2.32 & 2.50\\CON8-8 & 795.08 & 0.78 & 
795.08 & 0.84 & \bf{767.53} & 
3.59 & 3.59\\CON8-9 & 824.39 & 0.74 & 
834.81 & 0.76 & \bf{809.00} & 
1.90 & 3.19\\\bf{PROM.} & 
\bf{770.28} & \bf{0.74} & \bf{772.72} & \bf{0.75} & \bf{758.54} & \bf{1.43} & \bf{1.76}\\[1ex]\hline
\end{tabular}
\label{table:nonlin}
\end{table} \clearpage
\begin{table}[ht]
\caption{Resultados de la ejecución de la metaheurística IGA, utilizando instancias de SalhiNagy con la configuración -n 200 -p 40 -cprob 90.0 -mprob 10.0}
\centering
\small
\begin{tabular}{c c c c c c c c}
\hline\hline
Instancia & Costo mínimo & Tiempo(seg.) & Costo promedio & Tiempo promedio(seg.) & CME & \%G & \%GP \\ [0.5ex]
\hline
CMT1X & 478.97 & 0.49 & 
479.61 & 0.63 & \bf{470.48} & 
1.80 & 1.94\\CMT1Y & 483.19 & 0.58 & 
488.07 & 0.59 & \bf{470.48} & 
2.70 & 3.74\\CMT2X & 712.79 & 1.34 & 
718.08 & 1.41 & \bf{682.39} & 
4.45 & 5.23\\CMT2Y & 712.84 & 1.35 & 
715.98 & 1.35 & \bf{682.39} & 
4.46 & 4.92\\CMT3X & 743.48 & 3.05 & 
747.03 & 2.97 & \bf{719.06} & 
3.40 & 3.89\\CMT3Y & 746.09 & 3.06 & 
748.59 & 2.99 & \bf{719.06} & 
3.76 & 4.11\\CMT4X & 905.26 & 8.23 & 
915.74 & 8.07 & \bf{854.21} & 
5.98 & 7.20\\CMT4Y & 896.37 & 8.65 & 
912.79 & 8.48 & \bf{852.46} & 
5.15 & 7.08\\CMT5X & 1124.72 & 16.26 & 
1129.62 & 16.46 & \bf{1030.56} & 
9.14 & 9.61\\CMT5Y & 1096.90 & 16.42 & 
1119.86 & 16.89 & \bf{1031.69} & 
6.32 & 8.55\\CMT11X & 900.04 & 4.88 & 
909.27 & 4.97 & \bf{831.09} & 
8.30 & 9.41\\CMT11Y & 856.69 & 5.68 & 
898.90 & 5.54 & \bf{829.85} & 
3.23 & 8.32\\CMT12X & 680.30 & 2.81 & 
686.40 & 3.11 & \bf{658.83} & 
3.26 & 4.18\\CMT12Y & 675.94 & 3.16 & 
678.23 & 3.09 & \bf{660.47} & 
2.34 & 2.69\\\bf{PROM.} & 
\bf{786.68} & \bf{5.43} & \bf{796.30} & \bf{5.47} & \bf{749.50} & \bf{4.59} & \bf{5.78}\\[1ex]\hline
\end{tabular}
\label{table:nonlin}
\end{table} \clearpage
\begin{table}[ht]
\caption{Resultados de la ejecución de la metaheurística IGA, utilizando instancias de Dethloff con la configuración -n 200 -p 40 -cprob 90.0 -mprob 20.0}
\centering
\small
\begin{tabular}{c c c c c c c c}
\hline\hline
Instancia & Costo mínimo & Tiempo(seg.) & Costo promedio & Tiempo promedio(seg.) & CME & \%G & \%GP \\ [0.5ex]
\hline
SCA3-0 & 636.06 & 0.71 & 
639.43 & 0.75 & \bf{635.62} & 
0.07 & 0.60\\SCA3-1 & 701.53 & 0.72 & 
705.93 & 0.73 & \bf{697.84} & 
0.53 & 1.16\\SCA3-2 & 670.59 & 0.69 & 
670.59 & 0.74 & \bf{659.34} & 
1.71 & 1.71\\SCA3-3 & 682.45 & 0.73 & 
682.46 & 0.86 & \bf{680.04} & 
0.35 & 0.36\\SCA3-4 & \bf{690.50} & 0.70 & 
690.50 & 0.69 & 690.50 & 0.00
 & 0.00\\
SCA3-5 & 665.64 & 0.64 & 
665.64 & 0.67 & \bf{659.90} & 
0.87 & 0.87\\SCA3-6 & 652.94 & 0.69 & 
656.72 & 0.72 & \bf{651.09} & 
0.28 & 0.86\\SCA3-7 & 666.15 & 0.68 & 
666.15 & 0.75 & \bf{659.17} & 
1.06 & 1.06\\SCA3-8 & \bf{719.47} & 0.74 & 
719.67 & 0.66 & 719.47 & 0.00
 & 0.03\\SCA3-9 & 685.88 & 0.90 & 
687.91 & 0.81 & \bf{681.00} & 
0.72 & 1.02\\SCA8-0 & 982.79 & 0.72 & 
991.38 & 0.65 & \bf{961.50} & 
2.21 & 3.11\\SCA8-1 & 1077.29 & 0.78 & 
1086.49 & 0.81 & \bf{1049.65} & 
2.63 & 3.51\\SCA8-2 & 1054.47 & 0.76 & 
1054.57 & 0.78 & \bf{1039.64} & 
1.43 & 1.44\\SCA8-3 & 1007.82 & 0.78 & 
1007.82 & 0.75 & \bf{983.34} & 
2.49 & 2.49\\SCA8-4 & 1077.80 & 0.70 & 
1077.80 & 0.73 & \bf{1065.49} & 
1.16 & 1.16\\SCA8-5 & 1051.20 & 0.76 & 
1057.84 & 0.81 & \bf{1027.08} & 
2.35 & 2.99\\SCA8-6 & 989.38 & 0.74 & 
990.50 & 0.76 & \bf{971.82} & 
1.81 & 1.92\\SCA8-7 & 1063.60 & 0.66 & 
1063.60 & 0.64 & \bf{1051.28} & 
1.17 & 1.17\\SCA8-8 & 1092.81 & 0.78 & 
1093.16 & 0.79 & \bf{1071.18} & 
2.02 & 2.05\\SCA8-9 & 1082.16 & 0.75 & 
1082.16 & 0.83 & \bf{1060.50} & 
2.04 & 2.04\\CON3-0 & 620.76 & 0.72 & 
624.03 & 0.74 & \bf{616.52} & 
0.69 & 1.22\\CON3-1 & 556.92 & 0.72 & 
558.91 & 0.68 & \bf{554.47} & 
0.44 & 0.80\\CON3-2 & 521.38 & 1.01 & 
521.38 & 0.84 & \bf{518.00} & 
0.65 & 0.65\\CON3-3 & 591.20 & 0.72 & 
605.14 & 0.77 & \bf{591.19} & 
0.00 & 2.36\\CON3-4 & 589.32 & 0.73 & 
595.21 & 0.72 & \bf{588.79} & 
0.09 & 1.09\\CON3-5 & \bf{563.70} & 0.72 & 
563.70 & 0.71 & 563.70 & 0.00
 & 0.00\\
CON3-6 & 504.30 & 0.76 & 
506.93 & 0.71 & \bf{499.05} & 
1.05 & 1.58\\CON3-7 & 588.25 & 0.70 & 
589.65 & 0.73 & \bf{576.48} & 
2.04 & 2.28\\CON3-8 & 536.80 & 0.57 & 
538.35 & 0.69 & \bf{523.05} & 
2.63 & 2.92\\CON3-9 & 588.48 & 0.78 & 
590.18 & 0.76 & \bf{578.24} & 
1.77 & 2.07\\CON8-0 & 875.27 & 0.76 & 
875.27 & 0.76 & \bf{857.17} & 
2.11 & 2.11\\CON8-1 & 756.80 & 0.76 & 
756.80 & 0.76 & \bf{740.85} & 
2.15 & 2.15\\CON8-2 & 725.39 & 0.85 & 
725.39 & 0.74 & \bf{712.89} & 
1.75 & 1.75\\CON8-3 & 824.69 & 0.78 & 
828.96 & 0.79 & \bf{811.07} & 
1.68 & 2.21\\CON8-4 & 804.23 & 0.62 & 
805.00 & 0.71 & \bf{772.25} & 
4.14 & 4.24\\CON8-5 & 775.83 & 0.74 & 
776.74 & 0.81 & \bf{754.88} & 
2.78 & 2.90\\CON8-6 & 695.69 & 0.98 & 
701.41 & 0.82 & \bf{678.92} & 
2.47 & 3.31\\CON8-7 & 814.77 & 0.73 & 
825.87 & 0.79 & \bf{811.96} & 
0.35 & 1.71\\CON8-8 & 786.44 & 1.03 & 
791.46 & 0.97 & \bf{767.53} & 
2.46 & 3.12\\CON8-9 & 813.95 & 0.80 & 
819.00 & 0.72 & \bf{809.00} & 
0.61 & 1.24\\\bf{PROM.} & 
\bf{769.62} & \bf{0.75} & \bf{772.24} & \bf{0.75} & \bf{758.54} & \bf{1.37} & \bf{1.73}\\[1ex]\hline
\end{tabular}
\label{table:nonlin}
\end{table} \clearpage
\begin{table}[ht]
\caption{Resultados de la ejecución de la metaheurística IGA, utilizando instancias de SalhiNagy con la configuración -n 200 -p 40 -cprob 90.0 -mprob 20.0}
\centering
\small
\begin{tabular}{c c c c c c c c}
\hline\hline
Instancia & Costo mínimo & Tiempo(seg.) & Costo promedio & Tiempo promedio(seg.) & CME & \%G & \%GP \\ [0.5ex]
\hline
CMT1X & 479.97 & 0.48 & 
479.97 & 0.58 & \bf{470.48} & 
2.02 & 2.02\\CMT1Y & 483.29 & 0.63 & 
484.70 & 0.69 & \bf{470.48} & 
2.72 & 3.02\\CMT2X & 707.63 & 1.56 & 
714.15 & 1.57 & \bf{682.39} & 
3.70 & 4.65\\CMT2Y & 717.47 & 1.39 & 
719.63 & 1.41 & \bf{682.39} & 
5.14 & 5.46\\CMT3X & 742.40 & 3.17 & 
744.90 & 3.15 & \bf{719.06} & 
3.25 & 3.59\\CMT3Y & 738.17 & 2.98 & 
744.86 & 2.98 & \bf{719.06} & 
2.66 & 3.59\\CMT4X & 909.45 & 8.67 & 
915.39 & 8.50 & \bf{854.21} & 
6.47 & 7.16\\CMT4Y & 902.58 & 8.26 & 
907.41 & 8.49 & \bf{852.46} & 
5.88 & 6.45\\CMT5X & 1096.60 & 17.35 & 
1127.02 & 16.70 & \bf{1030.56} & 
6.41 & 9.36\\CMT5Y & 1121.42 & 16.94 & 
1129.85 & 17.25 & \bf{1031.69} & 
8.70 & 9.51\\CMT11X & 895.43 & 5.18 & 
900.43 & 4.93 & \bf{831.09} & 
7.74 & 8.34\\CMT11Y & 889.47 & 5.90 & 
907.32 & 6.72 & \bf{829.85} & 
7.18 & 9.33\\CMT12X & 692.12 & 3.33 & 
696.41 & 3.23 & \bf{658.83} & 
5.05 & 5.70\\CMT12Y & 675.08 & 3.08 & 
678.79 & 3.03 & \bf{660.47} & 
2.21 & 2.77\\\bf{PROM.} & 
\bf{789.36} & \bf{5.64} & \bf{796.49} & \bf{5.66} & \bf{749.50} & \bf{4.94} & \bf{5.78}\\[1ex]\hline
\end{tabular}
\label{table:nonlin}
\end{table} \clearpage
\begin{table}[ht]
\caption{Resultados de la ejecución de la metaheurística IGA, utilizando instancias de Dethloff con la configuración -n 200 -p 40 -cprob 90.0 -mprob 30.0}
\centering
\small
\begin{tabular}{c c c c c c c c}
\hline\hline
Instancia & Costo mínimo & Tiempo(seg.) & Costo promedio & Tiempo promedio(seg.) & CME & \%G & \%GP \\ [0.5ex]
\hline
SCA3-0 & 640.55 & 0.72 & 
640.55 & 0.77 & \bf{635.62} & 
0.78 & 0.78\\SCA3-1 & \bf{697.84} & 0.72 & 
699.85 & 0.66 & 697.84 & 0.00
 & 0.29\\SCA3-2 & 664.18 & 0.70 & 
666.30 & 0.66 & \bf{659.34} & 
0.73 & 1.06\\SCA3-3 & 685.05 & 0.71 & 
685.63 & 0.81 & \bf{680.04} & 
0.74 & 0.82\\SCA3-4 & \bf{690.50} & 0.93 & 
690.50 & 0.77 & 690.50 & 0.00
 & 0.00\\
SCA3-5 & 681.81 & 0.75 & 
684.05 & 0.71 & \bf{659.90} & 
3.32 & 3.66\\SCA3-6 & 652.94 & 0.95 & 
652.94 & 0.79 & \bf{651.09} & 
0.28 & 0.28\\SCA3-7 & 666.15 & 0.67 & 
666.15 & 0.68 & \bf{659.17} & 
1.06 & 1.06\\SCA3-8 & 719.77 & 0.74 & 
726.86 & 0.76 & \bf{719.47} & 
0.04 & 1.03\\SCA3-9 & \bf{681.00} & 0.94 & 
681.00 & 0.83 & 681.00 & 0.00
 & 0.00\\
SCA8-0 & 983.48 & 0.93 & 
1004.02 & 0.80 & \bf{961.50} & 
2.29 & 4.42\\SCA8-1 & 1066.39 & 0.74 & 
1070.35 & 0.76 & \bf{1049.65} & 
1.59 & 1.97\\SCA8-2 & 1050.37 & 0.72 & 
1050.37 & 0.79 & \bf{1039.64} & 
1.03 & 1.03\\SCA8-3 & 1011.60 & 0.76 & 
1020.47 & 1.01 & \bf{983.34} & 
2.87 & 3.78\\SCA8-4 & 1091.58 & 0.73 & 
1094.89 & 0.73 & \bf{1065.49} & 
2.45 & 2.76\\SCA8-5 & 1085.58 & 0.88 & 
1086.05 & 0.80 & \bf{1027.08} & 
5.70 & 5.74\\SCA8-6 & 986.73 & 0.76 & 
991.28 & 0.79 & \bf{971.82} & 
1.53 & 2.00\\SCA8-7 & 1070.53 & 0.82 & 
1070.53 & 0.84 & \bf{1051.28} & 
1.83 & 1.83\\SCA8-8 & 1089.55 & 0.74 & 
1090.77 & 0.72 & \bf{1071.18} & 
1.71 & 1.83\\SCA8-9 & 1078.30 & 0.71 & 
1078.30 & 0.83 & \bf{1060.50} & 
1.68 & 1.68\\CON3-0 & 619.09 & 0.58 & 
624.40 & 0.80 & \bf{616.52} & 
0.42 & 1.28\\CON3-1 & 556.04 & 0.93 & 
556.57 & 0.88 & \bf{554.47} & 
0.28 & 0.38\\CON3-2 & 521.38 & 0.79 & 
521.44 & 0.79 & \bf{518.00} & 
0.65 & 0.66\\CON3-3 & 591.20 & 0.90 & 
600.73 & 0.75 & \bf{591.19} & 
0.00 & 1.61\\CON3-4 & 592.58 & 0.93 & 
594.09 & 0.77 & \bf{588.79} & 
0.64 & 0.90\\CON3-5 & 567.94 & 0.70 & 
567.94 & 0.76 & \bf{563.70} & 
0.75 & 0.75\\CON3-6 & 505.14 & 0.96 & 
509.43 & 0.94 & \bf{499.05} & 
1.22 & 2.08\\CON3-7 & 578.22 & 0.52 & 
579.20 & 0.66 & \bf{576.48} & 
0.30 & 0.47\\CON3-8 & \bf{523.05} & 0.77 & 
529.27 & 0.79 & 523.05 & 0.00
 & 1.19\\CON3-9 & 588.11 & 0.93 & 
589.83 & 0.83 & \bf{578.24} & 
1.71 & 2.00\\CON8-0 & 875.32 & 1.02 & 
879.64 & 0.81 & \bf{857.17} & 
2.12 & 2.62\\CON8-1 & 752.64 & 0.78 & 
752.64 & 0.77 & \bf{740.85} & 
1.59 & 1.59\\CON8-2 & 726.36 & 0.80 & 
727.66 & 0.86 & \bf{712.89} & 
1.89 & 2.07\\CON8-3 & 832.81 & 0.78 & 
832.99 & 0.82 & \bf{811.07} & 
2.68 & 2.70\\CON8-4 & 781.48 & 0.76 & 
781.48 & 0.81 & \bf{772.25} & 
1.20 & 1.20\\CON8-5 & 769.55 & 0.77 & 
774.20 & 0.70 & \bf{754.88} & 
1.94 & 2.56\\CON8-6 & 702.24 & 0.57 & 
702.24 & 0.68 & \bf{678.92} & 
3.43 & 3.43\\CON8-7 & 816.74 & 0.73 & 
816.74 & 0.74 & \bf{811.96} & 
0.59 & 0.59\\CON8-8 & 787.09 & 0.61 & 
791.29 & 0.69 & \bf{767.53} & 
2.55 & 3.10\\CON8-9 & 812.60 & 0.57 & 
819.23 & 0.79 & \bf{809.00} & 
0.44 & 1.26\\\bf{PROM.} & 
\bf{769.84} & \bf{0.78} & \bf{772.55} & \bf{0.78} & \bf{758.54} & \bf{1.35} & \bf{1.71}\\[1ex]\hline
\end{tabular}
\label{table:nonlin}
\end{table} \clearpage
\begin{table}[ht]
\caption{Resultados de la ejecución de la metaheurística IGA, utilizando instancias de SalhiNagy con la configuración -n 200 -p 40 -cprob 90.0 -mprob 30.0}
\centering
\small
\begin{tabular}{c c c c c c c c}
\hline\hline
Instancia & Costo mínimo & Tiempo(seg.) & Costo promedio & Tiempo promedio(seg.) & CME & \%G & \%GP \\ [0.5ex]
\hline
CMT1X & 480.08 & 0.63 & 
484.77 & 0.69 & \bf{470.48} & 
2.04 & 3.04\\CMT1Y & 479.19 & 0.46 & 
483.95 & 0.56 & \bf{470.48} & 
1.85 & 2.86\\CMT2X & 716.33 & 1.43 & 
718.11 & 1.46 & \bf{682.39} & 
4.97 & 5.23\\CMT2Y & 717.35 & 1.65 & 
721.59 & 1.42 & \bf{682.39} & 
5.12 & 5.74\\CMT3X & 735.94 & 3.11 & 
739.79 & 3.06 & \bf{719.06} & 
2.35 & 2.88\\CMT3Y & 742.25 & 3.18 & 
749.59 & 3.10 & \bf{719.06} & 
3.23 & 4.25\\CMT4X & 905.20 & 7.82 & 
914.70 & 8.25 & \bf{854.21} & 
5.97 & 7.08\\CMT4Y & 908.99 & 8.23 & 
921.57 & 8.14 & \bf{852.46} & 
6.63 & 8.11\\CMT5X & 1095.47 & 17.33 & 
1114.06 & 16.93 & \bf{1030.56} & 
6.30 & 8.10\\CMT5Y & 1114.29 & 16.38 & 
1122.53 & 16.56 & \bf{1031.69} & 
8.01 & 8.80\\CMT11X & 900.62 & 5.34 & 
919.16 & 5.24 & \bf{831.09} & 
8.37 & 10.60\\CMT11Y & 918.13 & 5.32 & 
932.41 & 5.36 & \bf{829.85} & 
10.64 & 12.36\\CMT12X & 677.47 & 3.14 & 
677.60 & 3.21 & \bf{658.83} & 
2.83 & 2.85\\CMT12Y & 675.85 & 3.08 & 
677.86 & 3.09 & \bf{660.47} & 
2.33 & 2.63\\\bf{PROM.} & 
\bf{790.51} & \bf{5.51} & \bf{798.41} & \bf{5.50} & \bf{749.50} & \bf{5.04} & \bf{6.04}\\[1ex]\hline
\end{tabular}
\label{table:nonlin}
\end{table} \clearpage
\begin{table}[ht]
\caption{Resultados de la ejecución de la metaheurística IGA, utilizando instancias de Dethloff con la configuración -n 200 -p 40 -cprob 90.0 -mprob 40.0}
\centering
\small
\begin{tabular}{c c c c c c c c}
\hline\hline
Instancia & Costo mínimo & Tiempo(seg.) & Costo promedio & Tiempo promedio(seg.) & CME & \%G & \%GP \\ [0.5ex]
\hline
SCA3-0 & 641.69 & 0.74 & 
641.69 & 0.79 & \bf{635.62} & 
0.95 & 0.95\\SCA3-1 & \bf{697.84} & 0.72 & 
699.81 & 0.77 & 697.84 & 0.00
 & 0.28\\SCA3-2 & 661.13 & 0.70 & 
664.91 & 0.69 & \bf{659.34} & 
0.27 & 0.84\\SCA3-3 & 681.35 & 0.96 & 
690.99 & 0.85 & \bf{680.04} & 
0.19 & 1.61\\SCA3-4 & \bf{690.50} & 0.69 & 
690.50 & 0.79 & 690.50 & 0.00
 & 0.00\\
SCA3-5 & 682.43 & 0.98 & 
682.46 & 0.88 & \bf{659.90} & 
3.41 & 3.42\\SCA3-6 & 652.94 & 0.74 & 
656.98 & 0.72 & \bf{651.09} & 
0.28 & 0.90\\SCA3-7 & 666.15 & 0.86 & 
666.15 & 0.81 & \bf{659.17} & 
1.06 & 1.06\\SCA3-8 & 724.29 & 0.71 & 
729.40 & 0.81 & \bf{719.47} & 
0.67 & 1.38\\SCA3-9 & \bf{681.00} & 0.90 & 
681.00 & 0.79 & 681.00 & 0.00
 & 0.00\\
SCA8-0 & 989.15 & 0.78 & 
989.15 & 0.82 & \bf{961.50} & 
2.88 & 2.88\\SCA8-1 & 1079.07 & 0.81 & 
1083.45 & 0.99 & \bf{1049.65} & 
2.80 & 3.22\\SCA8-2 & 1053.94 & 0.75 & 
1054.20 & 0.68 & \bf{1039.64} & 
1.38 & 1.40\\SCA8-3 & 1032.68 & 0.78 & 
1032.68 & 0.78 & \bf{983.34} & 
5.02 & 5.02\\SCA8-4 & 1092.33 & 0.72 & 
1096.11 & 0.79 & \bf{1065.49} & 
2.52 & 2.87\\SCA8-5 & 1051.83 & 0.72 & 
1058.95 & 0.74 & \bf{1027.08} & 
2.41 & 3.10\\SCA8-6 & 993.09 & 0.71 & 
993.41 & 0.79 & \bf{971.82} & 
2.19 & 2.22\\SCA8-7 & 1079.18 & 0.77 & 
1079.18 & 0.89 & \bf{1051.28} & 
2.65 & 2.65\\SCA8-8 & 1098.32 & 0.72 & 
1098.32 & 0.84 & \bf{1071.18} & 
2.53 & 2.53\\SCA8-9 & 1079.93 & 0.75 & 
1079.93 & 0.82 & \bf{1060.50} & 
1.83 & 1.83\\CON3-0 & 624.96 & 0.92 & 
624.96 & 0.89 & \bf{616.52} & 
1.37 & 1.37\\CON3-1 & 560.75 & 0.86 & 
561.03 & 0.84 & \bf{554.47} & 
1.13 & 1.18\\CON3-2 & 521.38 & 0.78 & 
523.45 & 0.82 & \bf{518.00} & 
0.65 & 1.05\\CON3-3 & 591.20 & 0.82 & 
594.70 & 0.79 & \bf{591.19} & 
0.00 & 0.59\\CON3-4 & 593.69 & 0.71 & 
594.12 & 0.79 & \bf{588.79} & 
0.83 & 0.91\\CON3-5 & \bf{563.70} & 0.73 & 
564.76 & 0.72 & 563.70 & 0.00
 & 0.19\\CON3-6 & 509.95 & 0.77 & 
510.55 & 0.85 & \bf{499.05} & 
2.18 & 2.30\\CON3-7 & 582.14 & 0.68 & 
587.03 & 0.80 & \bf{576.48} & 
0.98 & 1.83\\CON3-8 & 528.09 & 0.74 & 
533.27 & 0.81 & \bf{523.05} & 
0.96 & 1.95\\CON3-9 & 582.79 & 0.75 & 
588.87 & 0.79 & \bf{578.24} & 
0.79 & 1.84\\CON8-0 & 882.07 & 0.75 & 
884.12 & 0.68 & \bf{857.17} & 
2.90 & 3.14\\CON8-1 & 759.98 & 0.79 & 
765.67 & 1.06 & \bf{740.85} & 
2.58 & 3.35\\CON8-2 & 713.60 & 0.79 & 
716.83 & 0.85 & \bf{712.89} & 
0.10 & 0.55\\CON8-3 & 830.61 & 0.78 & 
832.48 & 0.82 & \bf{811.07} & 
2.41 & 2.64\\CON8-4 & 780.03 & 0.76 & 
784.79 & 0.79 & \bf{772.25} & 
1.01 & 1.62\\CON8-5 & 763.54 & 0.98 & 
767.42 & 0.96 & \bf{754.88} & 
1.15 & 1.66\\CON8-6 & 705.19 & 0.78 & 
706.57 & 0.77 & \bf{678.92} & 
3.87 & 4.07\\CON8-7 & 827.66 & 0.97 & 
827.66 & 0.85 & \bf{811.96} & 
1.93 & 1.93\\CON8-8 & 780.78 & 1.15 & 
782.68 & 0.91 & \bf{767.53} & 
1.73 & 1.97\\CON8-9 & 834.65 & 0.89 & 
834.65 & 0.78 & \bf{809.00} & 
3.17 & 3.17\\\bf{PROM.} & 
\bf{771.64} & \bf{0.80} & \bf{773.87} & \bf{0.81} & \bf{758.54} & \bf{1.57} & \bf{1.89}\\[1ex]\hline
\end{tabular}
\label{table:nonlin}
\end{table} \clearpage
\begin{table}[ht]
\caption{Resultados de la ejecución de la metaheurística IGA, utilizando instancias de SalhiNagy con la configuración -n 200 -p 40 -cprob 90.0 -mprob 40.0}
\centering
\small
\begin{tabular}{c c c c c c c c}
\hline\hline
Instancia & Costo mínimo & Tiempo(seg.) & Costo promedio & Tiempo promedio(seg.) & CME & \%G & \%GP \\ [0.5ex]
\hline
CMT1X & 485.80 & 0.65 & 
485.93 & 0.66 & \bf{470.48} & 
3.26 & 3.28\\CMT1Y & 479.29 & 0.54 & 
486.02 & 0.68 & \bf{470.48} & 
1.87 & 3.30\\CMT2X & 695.65 & 1.50 & 
710.03 & 1.54 & \bf{682.39} & 
1.94 & 4.05\\CMT2Y & 714.97 & 1.40 & 
719.37 & 1.39 & \bf{682.39} & 
4.77 & 5.42\\CMT3X & 740.62 & 3.39 & 
747.19 & 3.23 & \bf{719.06} & 
3.00 & 3.91\\CMT3Y & 731.42 & 2.87 & 
744.07 & 3.01 & \bf{719.06} & 
1.72 & 3.48\\CMT4X & 916.59 & 7.98 & 
917.20 & 7.78 & \bf{854.21} & 
7.30 & 7.37\\CMT4Y & 908.11 & 8.46 & 
911.76 & 8.63 & \bf{852.46} & 
6.53 & 6.96\\CMT5X & 1094.40 & 17.05 & 
1107.86 & 16.53 & \bf{1030.56} & 
6.19 & 7.50\\CMT5Y & 1092.36 & 18.28 & 
1110.96 & 17.75 & \bf{1031.69} & 
5.88 & 7.68\\CMT11X & 877.73 & 5.70 & 
906.39 & 5.20 & \bf{831.09} & 
5.61 & 9.06\\CMT11Y & 894.57 & 5.67 & 
909.23 & 5.72 & \bf{829.85} & 
7.80 & 9.57\\CMT12X & 675.11 & 3.18 & 
678.25 & 3.19 & \bf{658.83} & 
2.47 & 2.95\\CMT12Y & 673.87 & 3.13 & 
675.74 & 3.24 & \bf{660.47} & 
2.03 & 2.31\\\bf{PROM.} & 
\bf{784.32} & \bf{5.70} & \bf{793.57} & \bf{5.61} & \bf{749.50} & \bf{4.31} & \bf{5.49}\\[1ex]\hline
\end{tabular}
\label{table:nonlin}
\end{table} \clearpage
\begin{table}[ht]
\caption{Resultados de la ejecución de la metaheurística IGA, utilizando instancias de Dethloff con la configuración -n 200 -p 40 -cprob 90.0 -mprob 50.0}
\centering
\small
\begin{tabular}{c c c c c c c c}
\hline\hline
Instancia & Costo mínimo & Tiempo(seg.) & Costo promedio & Tiempo promedio(seg.) & CME & \%G & \%GP \\ [0.5ex]
\hline
SCA3-0 & 636.06 & 0.71 & 
637.18 & 0.72 & \bf{635.62} & 
0.07 & 0.25\\SCA3-1 & 710.76 & 0.55 & 
710.81 & 0.71 & \bf{697.84} & 
1.85 & 1.86\\SCA3-2 & 661.13 & 0.89 & 
661.13 & 0.82 & \bf{659.34} & 
0.27 & 0.27\\SCA3-3 & 681.35 & 0.56 & 
681.91 & 0.64 & \bf{680.04} & 
0.19 & 0.27\\SCA3-4 & \bf{690.50} & 0.70 & 
690.50 & 0.73 & 690.50 & 0.00
 & 0.00\\
SCA3-5 & 670.10 & 0.98 & 
682.69 & 0.88 & \bf{659.90} & 
1.55 & 3.45\\SCA3-6 & 652.94 & 0.94 & 
655.33 & 0.82 & \bf{651.09} & 
0.28 & 0.65\\SCA3-7 & 667.24 & 0.53 & 
667.24 & 0.66 & \bf{659.17} & 
1.22 & 1.22\\SCA3-8 & \bf{719.47} & 0.76 & 
719.70 & 0.70 & 719.47 & 0.00
 & 0.03\\SCA3-9 & \bf{681.00} & 0.68 & 
684.11 & 0.78 & 681.00 & 0.00
 & 0.46\\SCA8-0 & 992.72 & 0.57 & 
993.52 & 0.81 & \bf{961.50} & 
3.25 & 3.33\\SCA8-1 & 1062.93 & 0.54 & 
1072.24 & 0.73 & \bf{1049.65} & 
1.27 & 2.15\\SCA8-2 & 1052.56 & 0.86 & 
1052.56 & 0.85 & \bf{1039.64} & 
1.24 & 1.24\\SCA8-3 & 1007.99 & 0.74 & 
1007.99 & 0.74 & \bf{983.34} & 
2.51 & 2.51\\SCA8-4 & 1094.48 & 0.98 & 
1094.48 & 0.76 & \bf{1065.49} & 
2.72 & 2.72\\SCA8-5 & 1057.31 & 0.77 & 
1057.31 & 0.83 & \bf{1027.08} & 
2.94 & 2.94\\SCA8-6 & 982.29 & 0.83 & 
982.83 & 0.80 & \bf{971.82} & 
1.08 & 1.13\\SCA8-7 & 1070.92 & 0.66 & 
1076.25 & 0.63 & \bf{1051.28} & 
1.87 & 2.38\\SCA8-8 & 1082.91 & 0.72 & 
1082.91 & 0.89 & \bf{1071.18} & 
1.10 & 1.10\\SCA8-9 & 1073.62 & 0.76 & 
1078.03 & 0.73 & \bf{1060.50} & 
1.24 & 1.65\\CON3-0 & 620.76 & 0.74 & 
627.82 & 0.73 & \bf{616.52} & 
0.69 & 1.83\\CON3-1 & 557.38 & 0.96 & 
560.15 & 0.94 & \bf{554.47} & 
0.52 & 1.02\\CON3-2 & 521.38 & 0.80 & 
521.38 & 0.90 & \bf{518.00} & 
0.65 & 0.65\\CON3-3 & 591.48 & 0.71 & 
595.21 & 0.74 & \bf{591.19} & 
0.05 & 0.68\\CON3-4 & 591.43 & 0.60 & 
593.12 & 0.75 & \bf{588.79} & 
0.45 & 0.74\\CON3-5 & 564.89 & 0.75 & 
565.65 & 0.76 & \bf{563.70} & 
0.21 & 0.35\\CON3-6 & 508.94 & 0.99 & 
510.82 & 0.97 & \bf{499.05} & 
1.98 & 2.36\\CON3-7 & 578.41 & 0.88 & 
579.39 & 0.91 & \bf{576.48} & 
0.33 & 0.50\\CON3-8 & 523.14 & 0.99 & 
530.78 & 0.92 & \bf{523.05} & 
0.02 & 1.48\\CON3-9 & 590.39 & 0.73 & 
590.83 & 0.81 & \bf{578.24} & 
2.10 & 2.18\\CON8-0 & 866.50 & 0.75 & 
866.50 & 0.76 & \bf{857.17} & 
1.09 & 1.09\\CON8-1 & 745.20 & 1.04 & 
751.65 & 0.85 & \bf{740.85} & 
0.59 & 1.46\\CON8-2 & 731.49 & 1.01 & 
732.67 & 0.88 & \bf{712.89} & 
2.61 & 2.78\\CON8-3 & 835.33 & 0.85 & 
835.33 & 0.78 & \bf{811.07} & 
2.99 & 2.99\\CON8-4 & 785.58 & 0.71 & 
785.58 & 0.71 & \bf{772.25} & 
1.73 & 1.73\\CON8-5 & 760.62 & 0.78 & 
761.62 & 0.79 & \bf{754.88} & 
0.76 & 0.89\\CON8-6 & 696.02 & 0.76 & 
697.01 & 0.82 & \bf{678.92} & 
2.52 & 2.66\\CON8-7 & 825.69 & 0.95 & 
827.11 & 0.80 & \bf{811.96} & 
1.69 & 1.87\\CON8-8 & 793.96 & 1.07 & 
794.84 & 0.89 & \bf{767.53} & 
3.44 & 3.56\\CON8-9 & 833.58 & 0.76 & 
836.47 & 0.84 & \bf{809.00} & 
3.04 & 3.40\\\bf{PROM.} & 
\bf{769.26} & \bf{0.79} & \bf{771.32} & \bf{0.79} & \bf{758.54} & \bf{1.30} & \bf{1.60}\\[1ex]\hline
\end{tabular}
\label{table:nonlin}
\end{table} \clearpage
\begin{table}[ht]
\caption{Resultados de la ejecución de la metaheurística IGA, utilizando instancias de SalhiNagy con la configuración -n 200 -p 40 -cprob 90.0 -mprob 50.0}
\centering
\small
\begin{tabular}{c c c c c c c c}
\hline\hline
Instancia & Costo mínimo & Tiempo(seg.) & Costo promedio & Tiempo promedio(seg.) & CME & \%G & \%GP \\ [0.5ex]
\hline
CMT1X & 485.04 & 0.63 & 
486.01 & 0.63 & \bf{470.48} & 
3.09 & 3.30\\CMT1Y & 478.23 & 0.65 & 
481.13 & 0.72 & \bf{470.48} & 
1.65 & 2.26\\CMT2X & 712.35 & 1.42 & 
718.99 & 1.57 & \bf{682.39} & 
4.39 & 5.36\\CMT2Y & 715.47 & 1.47 & 
718.07 & 1.44 & \bf{682.39} & 
4.85 & 5.23\\CMT3X & 726.48 & 3.05 & 
743.35 & 3.30 & \bf{719.06} & 
1.03 & 3.38\\CMT3Y & 740.81 & 3.24 & 
745.25 & 3.17 & \bf{719.06} & 
3.02 & 3.64\\CMT4X & 906.48 & 8.36 & 
912.53 & 8.68 & \bf{854.21} & 
6.12 & 6.83\\CMT4Y & 896.95 & 8.80 & 
912.35 & 8.30 & \bf{852.46} & 
5.22 & 7.02\\CMT5X & 1105.63 & 16.98 & 
1116.65 & 16.76 & \bf{1030.56} & 
7.28 & 8.35\\CMT5Y & 1122.72 & 17.64 & 
1127.55 & 17.61 & \bf{1031.69} & 
8.82 & 9.29\\CMT11X & 901.85 & 5.65 & 
921.96 & 5.30 & \bf{831.09} & 
8.51 & 10.93\\CMT11Y & 850.85 & 5.90 & 
897.81 & 5.94 & \bf{829.85} & 
2.53 & 8.19\\CMT12X & 679.06 & 3.02 & 
682.50 & 3.11 & \bf{658.83} & 
3.07 & 3.59\\CMT12Y & 675.70 & 3.44 & 
676.19 & 3.16 & \bf{660.47} & 
2.31 & 2.38\\\bf{PROM.} & 
\bf{785.54} & \bf{5.73} & \bf{795.74} & \bf{5.69} & \bf{749.50} & \bf{4.42} & \bf{5.70}\\[1ex]\hline
\end{tabular}
\label{table:nonlin}
\end{table} \clearpage
\begin{table}[ht]
\caption{Resultados de la ejecución de la metaheurística IGA, utilizando instancias de Dethloff con la configuración -n 200 -p 40 -cprob 90.0 -mprob 60.0}
\centering
\small
\begin{tabular}{c c c c c c c c}
\hline\hline
Instancia & Costo mínimo & Tiempo(seg.) & Costo promedio & Tiempo promedio(seg.) & CME & \%G & \%GP \\ [0.5ex]
\hline
SCA3-0 & 640.55 & 0.73 & 
640.84 & 0.76 & \bf{635.62} & 
0.78 & 0.82\\SCA3-1 & 701.74 & 0.98 & 
705.87 & 0.90 & \bf{697.84} & 
0.56 & 1.15\\SCA3-2 & 664.21 & 0.67 & 
665.56 & 0.74 & \bf{659.34} & 
0.74 & 0.94\\SCA3-3 & 681.35 & 0.76 & 
682.27 & 0.72 & \bf{680.04} & 
0.19 & 0.33\\SCA3-4 & \bf{690.50} & 0.69 & 
691.53 & 0.75 & 690.50 & 0.00
 & 0.15\\SCA3-5 & 667.27 & 0.70 & 
667.27 & 0.84 & \bf{659.90} & 
1.12 & 1.12\\SCA3-6 & 656.66 & 0.76 & 
657.66 & 0.76 & \bf{651.09} & 
0.86 & 1.01\\SCA3-7 & 664.88 & 0.93 & 
664.88 & 0.89 & \bf{659.17} & 
0.87 & 0.87\\SCA3-8 & 719.77 & 0.92 & 
723.16 & 0.88 & \bf{719.47} & 
0.04 & 0.51\\SCA3-9 & \bf{681.00} & 0.65 & 
684.69 & 0.77 & 681.00 & 0.00
 & 0.54\\SCA8-0 & 974.40 & 0.52 & 
974.40 & 0.71 & \bf{961.50} & 
1.34 & 1.34\\SCA8-1 & 1079.00 & 0.97 & 
1087.70 & 0.90 & \bf{1049.65} & 
2.80 & 3.63\\SCA8-2 & 1053.55 & 0.94 & 
1053.86 & 0.87 & \bf{1039.64} & 
1.34 & 1.37\\SCA8-3 & 1022.98 & 1.05 & 
1022.98 & 1.04 & \bf{983.34} & 
4.03 & 4.03\\SCA8-4 & 1085.04 & 0.96 & 
1086.63 & 0.86 & \bf{1065.49} & 
1.83 & 1.98\\SCA8-5 & 1057.56 & 0.86 & 
1057.56 & 0.83 & \bf{1027.08} & 
2.97 & 2.97\\SCA8-6 & 972.48 & 1.06 & 
978.66 & 0.95 & \bf{971.82} & 
0.07 & 0.70\\SCA8-7 & 1076.91 & 0.92 & 
1076.91 & 0.84 & \bf{1051.28} & 
2.44 & 2.44\\SCA8-8 & 1075.00 & 0.98 & 
1075.00 & 0.85 & \bf{1071.18} & 
0.36 & 0.36\\SCA8-9 & 1080.15 & 0.75 & 
1081.10 & 0.73 & \bf{1060.50} & 
1.85 & 1.94\\CON3-0 & \bf{616.52} & 0.71 & 
631.46 & 0.83 & 616.52 & 0.00
 & 2.42\\CON3-1 & 560.75 & 0.76 & 
560.75 & 0.73 & \bf{554.47} & 
1.13 & 1.13\\CON3-2 & 521.38 & 0.77 & 
521.38 & 0.88 & \bf{518.00} & 
0.65 & 0.65\\CON3-3 & 595.87 & 0.88 & 
601.12 & 0.74 & \bf{591.19} & 
0.79 & 1.68\\CON3-4 & 592.58 & 0.69 & 
593.18 & 0.82 & \bf{588.79} & 
0.64 & 0.75\\CON3-5 & 567.07 & 0.74 & 
567.07 & 0.86 & \bf{563.70} & 
0.60 & 0.60\\CON3-6 & 502.16 & 0.99 & 
503.43 & 0.91 & \bf{499.05} & 
0.62 & 0.88\\CON3-7 & 580.55 & 0.84 & 
580.87 & 0.84 & \bf{576.48} & 
0.71 & 0.76\\CON3-8 & 528.09 & 0.81 & 
530.14 & 0.93 & \bf{523.05} & 
0.96 & 1.36\\CON3-9 & 588.48 & 0.75 & 
589.43 & 0.76 & \bf{578.24} & 
1.77 & 1.94\\CON8-0 & 886.05 & 1.02 & 
886.05 & 0.93 & \bf{857.17} & 
3.37 & 3.37\\CON8-1 & 761.62 & 0.84 & 
764.38 & 0.92 & \bf{740.85} & 
2.80 & 3.18\\CON8-2 & 718.79 & 1.04 & 
721.77 & 0.86 & \bf{712.89} & 
0.83 & 1.25\\CON8-3 & 826.41 & 0.88 & 
826.41 & 0.88 & \bf{811.07} & 
1.89 & 1.89\\CON8-4 & 785.82 & 1.02 & 
785.82 & 0.85 & \bf{772.25} & 
1.76 & 1.76\\CON8-5 & 763.13 & 0.99 & 
763.13 & 0.97 & \bf{754.88} & 
1.09 & 1.09\\CON8-6 & 698.69 & 0.70 & 
700.81 & 0.81 & \bf{678.92} & 
2.91 & 3.22\\CON8-7 & 832.81 & 0.55 & 
832.81 & 0.60 & \bf{811.96} & 
2.57 & 2.57\\CON8-8 & 779.96 & 0.80 & 
780.50 & 0.92 & \bf{767.53} & 
1.62 & 1.69\\CON8-9 & 814.57 & 0.77 & 
817.89 & 0.82 & \bf{809.00} & 
0.69 & 1.10\\\bf{PROM.} & 
\bf{769.16} & \bf{0.83} & \bf{770.92} & \bf{0.84} & \bf{758.54} & \bf{1.29} & \bf{1.54}\\[1ex]\hline
\end{tabular}
\label{table:nonlin}
\end{table} \clearpage
\begin{table}[ht]
\caption{Resultados de la ejecución de la metaheurística IGA, utilizando instancias de SalhiNagy con la configuración -n 200 -p 40 -cprob 90.0 -mprob 60.0}
\centering
\small
\begin{tabular}{c c c c c c c c}
\hline\hline
Instancia & Costo mínimo & Tiempo(seg.) & Costo promedio & Tiempo promedio(seg.) & CME & \%G & \%GP \\ [0.5ex]
\hline
CMT1X & 478.97 & 0.48 & 
479.85 & 0.62 & \bf{470.48} & 
1.80 & 1.99\\CMT1Y & 484.66 & 0.62 & 
484.66 & 0.58 & \bf{470.48} & 
3.01 & 3.01\\CMT2X & 699.02 & 1.53 & 
707.02 & 1.54 & \bf{682.39} & 
2.44 & 3.61\\CMT2Y & 710.73 & 1.37 & 
713.46 & 1.48 & \bf{682.39} & 
4.15 & 4.55\\CMT3X & 739.46 & 3.62 & 
745.70 & 3.38 & \bf{719.06} & 
2.84 & 3.71\\CMT3Y & 730.89 & 3.10 & 
735.69 & 3.02 & \bf{719.06} & 
1.65 & 2.31\\CMT4X & 894.85 & 8.94 & 
912.59 & 8.60 & \bf{854.21} & 
4.76 & 6.83\\CMT4Y & 904.04 & 9.02 & 
911.26 & 8.91 & \bf{852.46} & 
6.05 & 6.90\\CMT5X & 1088.62 & 17.48 & 
1117.45 & 17.20 & \bf{1030.56} & 
5.63 & 8.43\\CMT5Y & 1101.42 & 17.25 & 
1112.40 & 17.48 & \bf{1031.69} & 
6.76 & 7.82\\CMT11X & 889.99 & 4.64 & 
909.73 & 5.26 & \bf{831.09} & 
7.09 & 9.46\\CMT11Y & 901.80 & 6.32 & 
921.12 & 6.22 & \bf{829.85} & 
8.67 & 11.00\\CMT12X & 675.02 & 3.06 & 
683.58 & 3.16 & \bf{658.83} & 
2.46 & 3.76\\CMT12Y & 679.14 & 3.06 & 
682.67 & 3.27 & \bf{660.47} & 
2.83 & 3.36\\\bf{PROM.} & 
\bf{784.19} & \bf{5.75} & \bf{794.08} & \bf{5.76} & \bf{749.50} & \bf{4.30} & \bf{5.48}\\[1ex]\hline
\end{tabular}
\label{table:nonlin}
\end{table} \clearpage
\begin{table}[ht]
\caption{Resultados de la ejecución de la metaheurística IGA, utilizando instancias de Dethloff con la configuración -n 200 -p 40 -cprob 90.0 -mprob 70.0}
\centering
\small
\begin{tabular}{c c c c c c c c}
\hline\hline
Instancia & Costo mínimo & Tiempo(seg.) & Costo promedio & Tiempo promedio(seg.) & CME & \%G & \%GP \\ [0.5ex]
\hline
SCA3-0 & 641.64 & 0.96 & 
641.64 & 0.94 & \bf{635.62} & 
0.95 & 0.95\\SCA3-1 & 700.50 & 0.77 & 
700.76 & 0.76 & \bf{697.84} & 
0.38 & 0.42\\SCA3-2 & 664.21 & 0.68 & 
664.21 & 0.79 & \bf{659.34} & 
0.74 & 0.74\\SCA3-3 & \bf{680.04} & 0.70 & 
683.51 & 0.71 & 680.04 & 0.00
 & 0.51\\SCA3-4 & \bf{690.50} & 0.89 & 
691.70 & 0.84 & 690.50 & 0.00
 & 0.17\\SCA3-5 & 673.56 & 0.91 & 
673.56 & 0.81 & \bf{659.90} & 
2.07 & 2.07\\SCA3-6 & 652.94 & 0.70 & 
653.81 & 0.85 & \bf{651.09} & 
0.28 & 0.42\\SCA3-7 & 666.15 & 0.74 & 
668.73 & 0.73 & \bf{659.17} & 
1.06 & 1.45\\SCA3-8 & 724.29 & 0.95 & 
729.85 & 0.90 & \bf{719.47} & 
0.67 & 1.44\\SCA3-9 & \bf{681.00} & 0.84 & 
681.00 & 0.77 & 681.00 & 0.00
 & 0.00\\
SCA8-0 & 982.36 & 0.96 & 
982.36 & 0.92 & \bf{961.50} & 
2.17 & 2.17\\SCA8-1 & 1062.35 & 1.04 & 
1062.35 & 0.96 & \bf{1049.65} & 
1.21 & 1.21\\SCA8-2 & 1069.86 & 0.98 & 
1070.50 & 0.84 & \bf{1039.64} & 
2.91 & 2.97\\SCA8-3 & 1000.75 & 0.60 & 
1005.91 & 0.75 & \bf{983.34} & 
1.77 & 2.30\\SCA8-4 & 1067.66 & 0.47 & 
1067.66 & 0.62 & \bf{1065.49} & 
0.20 & 0.20\\SCA8-5 & 1060.37 & 0.98 & 
1060.37 & 0.93 & \bf{1027.08} & 
3.24 & 3.24\\SCA8-6 & \bf{971.82} & 0.97 & 
976.14 & 0.92 & 971.82 & 0.00
 & 0.44\\SCA8-7 & 1079.05 & 0.67 & 
1079.05 & 0.84 & \bf{1051.28} & 
2.64 & 2.64\\SCA8-8 & 1075.00 & 0.97 & 
1075.00 & 0.91 & \bf{1071.18} & 
0.36 & 0.36\\SCA8-9 & 1074.21 & 0.99 & 
1077.26 & 0.90 & \bf{1060.50} & 
1.29 & 1.58\\CON3-0 & 624.84 & 0.79 & 
629.04 & 0.86 & \bf{616.52} & 
1.35 & 2.03\\CON3-1 & 556.92 & 0.94 & 
559.79 & 0.94 & \bf{554.47} & 
0.44 & 0.96\\CON3-2 & 521.38 & 0.96 & 
522.73 & 0.90 & \bf{518.00} & 
0.65 & 0.91\\CON3-3 & \bf{591.19} & 0.75 & 
599.75 & 0.79 & 591.19 & 0.00
 & 1.45\\CON3-4 & 604.71 & 0.92 & 
604.71 & 0.92 & \bf{588.79} & 
2.70 & 2.70\\CON3-5 & 568.66 & 0.72 & 
571.32 & 0.81 & \bf{563.70} & 
0.88 & 1.35\\CON3-6 & 504.15 & 0.79 & 
505.05 & 0.80 & \bf{499.05} & 
1.02 & 1.20\\CON3-7 & 581.27 & 0.62 & 
588.70 & 0.68 & \bf{576.48} & 
0.83 & 2.12\\CON3-8 & 524.30 & 0.82 & 
524.30 & 0.88 & \bf{523.05} & 
0.24 & 0.24\\CON3-9 & 588.11 & 0.81 & 
588.40 & 0.83 & \bf{578.24} & 
1.71 & 1.76\\CON8-0 & 877.38 & 0.96 & 
877.38 & 0.95 & \bf{857.17} & 
2.36 & 2.36\\CON8-1 & 758.03 & 1.05 & 
761.15 & 0.91 & \bf{740.85} & 
2.32 & 2.74\\CON8-2 & 716.26 & 0.88 & 
716.61 & 0.92 & \bf{712.89} & 
0.47 & 0.52\\CON8-3 & 836.01 & 0.98 & 
836.81 & 1.01 & \bf{811.07} & 
3.07 & 3.17\\CON8-4 & 772.76 & 0.76 & 
772.81 & 0.72 & \bf{772.25} & 
0.07 & 0.07\\CON8-5 & 771.15 & 1.02 & 
771.15 & 0.90 & \bf{754.88} & 
2.16 & 2.16\\CON8-6 & 691.83 & 0.86 & 
691.83 & 0.88 & \bf{678.92} & 
1.90 & 1.90\\CON8-7 & 826.02 & 0.83 & 
826.11 & 0.88 & \bf{811.96} & 
1.73 & 1.74\\CON8-8 & 791.19 & 0.96 & 
802.00 & 0.90 & \bf{767.53} & 
3.08 & 4.49\\CON8-9 & 813.67 & 0.94 & 
825.78 & 0.88 & \bf{809.00} & 
0.58 & 2.07\\\bf{PROM.} & 
\bf{768.45} & \bf{0.85} & \bf{770.52} & \bf{0.85} & \bf{758.54} & \bf{1.24} & \bf{1.53}\\[1ex]\hline
\end{tabular}
\label{table:nonlin}
\end{table} \clearpage
\begin{table}[ht]
\caption{Resultados de la ejecución de la metaheurística IGA, utilizando instancias de SalhiNagy con la configuración -n 200 -p 40 -cprob 90.0 -mprob 70.0}
\centering
\small
\begin{tabular}{c c c c c c c c}
\hline\hline
Instancia & Costo mínimo & Tiempo(seg.) & Costo promedio & Tiempo promedio(seg.) & CME & \%G & \%GP \\ [0.5ex]
\hline
CMT1X & 480.19 & 0.82 & 
483.09 & 0.82 & \bf{470.48} & 
2.06 & 2.68\\CMT1Y & 477.00 & 1.52 & 
483.12 & 0.90 & \bf{470.48} & 
1.39 & 2.69\\CMT2X & 703.44 & 1.68 & 
713.65 & 1.62 & \bf{682.39} & 
3.08 & 4.58\\CMT2Y & 704.88 & 1.73 & 
713.29 & 1.62 & \bf{682.39} & 
3.30 & 4.53\\CMT3X & 733.34 & 3.40 & 
736.12 & 3.12 & \bf{719.06} & 
1.99 & 2.37\\CMT3Y & 729.62 & 3.47 & 
735.80 & 3.31 & \bf{719.06} & 
1.47 & 2.33\\CMT4X & 903.54 & 8.17 & 
912.16 & 8.11 & \bf{854.21} & 
5.77 & 6.78\\CMT4Y & 889.48 & 8.68 & 
905.24 & 8.68 & \bf{852.46} & 
4.34 & 6.19\\CMT5X & 1087.66 & 16.62 & 
1116.59 & 16.49 & \bf{1030.56} & 
5.54 & 8.35\\CMT5Y & 1115.55 & 17.37 & 
1128.86 & 17.46 & \bf{1031.69} & 
8.13 & 9.42\\CMT11X & 896.75 & 5.17 & 
921.97 & 5.14 & \bf{831.09} & 
7.90 & 10.94\\CMT11Y & 852.23 & 6.38 & 
892.54 & 5.95 & \bf{829.85} & 
2.70 & 7.55\\CMT12X & 676.64 & 3.22 & 
678.61 & 3.45 & \bf{658.83} & 
2.70 & 3.00\\CMT12Y & 674.87 & 3.26 & 
676.52 & 3.19 & \bf{660.47} & 
2.18 & 2.43\\\bf{PROM.} & 
\bf{780.37} & \bf{5.82} & \bf{792.68} & \bf{5.71} & \bf{749.50} & \bf{3.75} & \bf{5.27}\\[1ex]\hline
\end{tabular}
\label{table:nonlin}
\end{table} \clearpage
\begin{table}[ht]
\caption{Resultados de la ejecución de la metaheurística IGA, utilizando instancias de Dethloff con la configuración -n 200 -p 40 -cprob 90.0 -mprob 80.0}
\centering
\small
\begin{tabular}{c c c c c c c c}
\hline\hline
Instancia & Costo mínimo & Tiempo(seg.) & Costo promedio & Tiempo promedio(seg.) & CME & \%G & \%GP \\ [0.5ex]
\hline
SCA3-0 & 640.55 & 0.75 & 
640.84 & 0.82 & \bf{635.62} & 
0.78 & 0.82\\SCA3-1 & 701.53 & 0.92 & 
701.71 & 0.93 & \bf{697.84} & 
0.53 & 0.55\\SCA3-2 & \bf{659.34} & 0.87 & 
667.73 & 0.83 & 659.34 & 0.00
 & 1.27\\SCA3-3 & \bf{680.04} & 0.94 & 
683.04 & 0.93 & 680.04 & 0.00
 & 0.44\\SCA3-4 & \bf{690.50} & 0.79 & 
690.50 & 0.82 & 690.50 & 0.00
 & 0.00\\
SCA3-5 & 673.46 & 0.91 & 
673.53 & 0.86 & \bf{659.90} & 
2.05 & 2.07\\SCA3-6 & 652.94 & 0.84 & 
655.59 & 0.77 & \bf{651.09} & 
0.28 & 0.69\\SCA3-7 & 666.60 & 0.90 & 
666.60 & 0.86 & \bf{659.17} & 
1.13 & 1.13\\SCA3-8 & \bf{719.47} & 0.94 & 
725.45 & 0.86 & 719.47 & 0.00
 & 0.83\\SCA3-9 & \bf{681.00} & 0.90 & 
683.46 & 0.81 & 681.00 & 0.00
 & 0.36\\SCA8-0 & 981.91 & 0.86 & 
981.91 & 1.03 & \bf{961.50} & 
2.12 & 2.12\\SCA8-1 & 1077.39 & 0.77 & 
1081.88 & 0.93 & \bf{1049.65} & 
2.64 & 3.07\\SCA8-2 & 1050.37 & 0.66 & 
1050.37 & 0.81 & \bf{1039.64} & 
1.03 & 1.03\\SCA8-3 & 1016.70 & 0.98 & 
1022.41 & 0.93 & \bf{983.34} & 
3.39 & 3.97\\SCA8-4 & 1087.32 & 0.96 & 
1087.32 & 0.91 & \bf{1065.49} & 
2.05 & 2.05\\SCA8-5 & 1045.31 & 0.96 & 
1045.31 & 0.86 & \bf{1027.08} & 
1.77 & 1.77\\SCA8-6 & 980.13 & 0.96 & 
983.40 & 0.85 & \bf{971.82} & 
0.86 & 1.19\\SCA8-7 & 1075.87 & 0.99 & 
1078.12 & 0.92 & \bf{1051.28} & 
2.34 & 2.55\\SCA8-8 & 1088.20 & 0.81 & 
1088.63 & 0.86 & \bf{1071.18} & 
1.59 & 1.63\\SCA8-9 & 1079.39 & 0.98 & 
1079.39 & 0.88 & \bf{1060.50} & 
1.78 & 1.78\\CON3-0 & 623.15 & 0.71 & 
623.15 & 0.89 & \bf{616.52} & 
1.08 & 1.08\\CON3-1 & 560.75 & 0.90 & 
561.11 & 0.86 & \bf{554.47} & 
1.13 & 1.20\\CON3-2 & 521.38 & 0.99 & 
522.26 & 0.84 & \bf{518.00} & 
0.65 & 0.82\\CON3-3 & 591.20 & 0.76 & 
591.88 & 0.84 & \bf{591.19} & 
0.00 & 0.12\\CON3-4 & 605.26 & 0.76 & 
605.26 & 0.85 & \bf{588.79} & 
2.80 & 2.80\\CON3-5 & 567.94 & 0.92 & 
568.32 & 0.94 & \bf{563.70} & 
0.75 & 0.82\\CON3-6 & 503.20 & 0.99 & 
503.38 & 0.90 & \bf{499.05} & 
0.83 & 0.87\\CON3-7 & 582.14 & 0.68 & 
582.14 & 0.68 & \bf{576.48} & 
0.98 & 0.98\\CON3-8 & 523.14 & 0.76 & 
530.23 & 0.86 & \bf{523.05} & 
0.02 & 1.37\\CON3-9 & 589.20 & 0.77 & 
590.80 & 0.81 & \bf{578.24} & 
1.90 & 2.17\\CON8-0 & 859.74 & 1.00 & 
859.74 & 0.89 & \bf{857.17} & 
0.30 & 0.30\\CON8-1 & 762.05 & 0.78 & 
762.57 & 0.95 & \bf{740.85} & 
2.86 & 2.93\\CON8-2 & 721.66 & 0.81 & 
721.66 & 0.69 & \bf{712.89} & 
1.23 & 1.23\\CON8-3 & 832.65 & 0.81 & 
833.75 & 0.92 & \bf{811.07} & 
2.66 & 2.80\\CON8-4 & 780.54 & 0.97 & 
780.54 & 0.95 & \bf{772.25} & 
1.07 & 1.07\\CON8-5 & 766.61 & 0.75 & 
769.75 & 0.89 & \bf{754.88} & 
1.55 & 1.97\\CON8-6 & 699.16 & 0.82 & 
699.97 & 0.91 & \bf{678.92} & 
2.98 & 3.10\\CON8-7 & 816.26 & 0.71 & 
824.74 & 0.83 & \bf{811.96} & 
0.53 & 1.57\\CON8-8 & 790.77 & 1.03 & 
790.94 & 1.03 & \bf{767.53} & 
3.03 & 3.05\\CON8-9 & 818.46 & 1.03 & 
823.87 & 0.90 & \bf{809.00} & 
1.17 & 1.84\\\bf{PROM.} & 
\bf{769.08} & \bf{0.87} & \bf{770.83} & \bf{0.87} & \bf{758.54} & \bf{1.30} & \bf{1.54}\\[1ex]\hline
\end{tabular}
\label{table:nonlin}
\end{table} \clearpage
\begin{table}[ht]
\caption{Resultados de la ejecución de la metaheurística IGA, utilizando instancias de SalhiNagy con la configuración -n 200 -p 40 -cprob 90.0 -mprob 80.0}
\centering
\small
\begin{tabular}{c c c c c c c c}
\hline\hline
Instancia & Costo mínimo & Tiempo(seg.) & Costo promedio & Tiempo promedio(seg.) & CME & \%G & \%GP \\ [0.5ex]
\hline
CMT1X & 484.80 & 0.82 & 
484.80 & 0.76 & \bf{470.48} & 
3.04 & 3.04\\CMT1Y & 488.78 & 0.59 & 
492.06 & 0.76 & \bf{470.48} & 
3.89 & 4.59\\CMT2X & 706.35 & 1.66 & 
712.36 & 1.61 & \bf{682.39} & 
3.51 & 4.39\\CMT2Y & 711.61 & 1.54 & 
713.82 & 1.43 & \bf{682.39} & 
4.28 & 4.61\\CMT3X & 744.22 & 3.40 & 
747.21 & 3.38 & \bf{719.06} & 
3.50 & 3.91\\CMT3Y & 726.68 & 3.42 & 
745.66 & 3.18 & \bf{719.06} & 
1.06 & 3.70\\CMT4X & 906.34 & 8.09 & 
918.83 & 8.53 & \bf{854.21} & 
6.10 & 7.57\\CMT4Y & 917.29 & 8.58 & 
921.88 & 8.35 & \bf{852.46} & 
7.61 & 8.14\\CMT5X & 1103.30 & 17.64 & 
1110.40 & 17.17 & \bf{1030.56} & 
7.06 & 7.75\\CMT5Y & 1099.35 & 17.55 & 
1113.99 & 17.39 & \bf{1031.69} & 
6.56 & 7.98\\CMT11X & 890.03 & 5.76 & 
907.52 & 5.46 & \bf{831.09} & 
7.09 & 9.20\\CMT11Y & 902.28 & 5.94 & 
904.88 & 6.00 & \bf{829.85} & 
8.73 & 9.04\\CMT12X & 674.50 & 3.40 & 
677.02 & 3.74 & \bf{658.83} & 
2.38 & 2.76\\CMT12Y & 684.82 & 3.19 & 
685.23 & 3.39 & \bf{660.47} & 
3.69 & 3.75\\\bf{PROM.} & 
\bf{788.60} & \bf{5.83} & \bf{795.40} & \bf{5.80} & \bf{749.50} & \bf{4.89} & \bf{5.74}\\[1ex]\hline
\end{tabular}
\label{table:nonlin}
\end{table} \clearpage
\begin{table}[ht]
\caption{Resultados de la ejecución de la metaheurística IGA, utilizando instancias de Dethloff con la configuración -n 200 -p 40 -cprob 90.0 -mprob 90.0}
\centering
\small
\begin{tabular}{c c c c c c c c}
\hline\hline
Instancia & Costo mínimo & Tiempo(seg.) & Costo promedio & Tiempo promedio(seg.) & CME & \%G & \%GP \\ [0.5ex]
\hline
SCA3-0 & 640.55 & 0.90 & 
640.55 & 0.79 & \bf{635.62} & 
0.78 & 0.78\\SCA3-1 & 701.53 & 0.95 & 
701.53 & 1.07 & \bf{697.84} & 
0.53 & 0.53\\SCA3-2 & 672.66 & 0.89 & 
674.01 & 0.87 & \bf{659.34} & 
2.02 & 2.23\\SCA3-3 & 681.35 & 0.92 & 
681.63 & 0.93 & \bf{680.04} & 
0.19 & 0.23\\SCA3-4 & \bf{690.50} & 0.88 & 
690.50 & 0.83 & 690.50 & 0.00
 & 0.00\\
SCA3-5 & 665.64 & 0.94 & 
676.34 & 0.84 & \bf{659.90} & 
0.87 & 2.49\\SCA3-6 & 655.41 & 0.96 & 
658.85 & 0.87 & \bf{651.09} & 
0.66 & 1.19\\SCA3-7 & 666.15 & 0.91 & 
666.49 & 0.86 & \bf{659.17} & 
1.06 & 1.11\\SCA3-8 & 731.89 & 0.92 & 
736.41 & 0.82 & \bf{719.47} & 
1.73 & 2.35\\SCA3-9 & 681.68 & 0.91 & 
683.60 & 0.91 & \bf{681.00} & 
0.10 & 0.38\\SCA8-0 & 970.64 & 1.01 & 
979.64 & 0.94 & \bf{961.50} & 
0.95 & 1.89\\SCA8-1 & 1069.58 & 0.93 & 
1074.82 & 0.87 & \bf{1049.65} & 
1.90 & 2.40\\SCA8-2 & 1047.57 & 1.06 & 
1058.08 & 0.95 & \bf{1039.64} & 
0.76 & 1.77\\SCA8-3 & 1014.71 & 0.94 & 
1014.71 & 0.81 & \bf{983.34} & 
3.19 & 3.19\\SCA8-4 & 1081.03 & 0.79 & 
1092.21 & 0.84 & \bf{1065.49} & 
1.46 & 2.51\\SCA8-5 & 1050.46 & 0.74 & 
1056.16 & 0.70 & \bf{1027.08} & 
2.28 & 2.83\\SCA8-6 & 987.54 & 0.97 & 
991.17 & 0.90 & \bf{971.82} & 
1.62 & 1.99\\SCA8-7 & 1083.58 & 1.00 & 
1083.58 & 0.79 & \bf{1051.28} & 
3.07 & 3.07\\SCA8-8 & 1090.51 & 1.00 & 
1090.51 & 0.77 & \bf{1071.18} & 
1.80 & 1.80\\SCA8-9 & 1080.47 & 0.95 & 
1080.47 & 0.88 & \bf{1060.50} & 
1.88 & 1.88\\CON3-0 & 624.84 & 0.92 & 
625.48 & 0.91 & \bf{616.52} & 
1.35 & 1.45\\CON3-1 & 557.38 & 0.97 & 
559.35 & 0.93 & \bf{554.47} & 
0.52 & 0.88\\CON3-2 & 524.89 & 0.96 & 
524.89 & 0.96 & \bf{518.00} & 
1.33 & 1.33\\CON3-3 & 592.43 & 1.00 & 
596.98 & 0.88 & \bf{591.19} & 
0.21 & 0.98\\CON3-4 & 592.58 & 0.91 & 
595.39 & 0.93 & \bf{588.79} & 
0.64 & 1.12\\CON3-5 & 568.66 & 0.76 & 
570.67 & 0.81 & \bf{563.70} & 
0.88 & 1.24\\CON3-6 & 502.16 & 0.82 & 
505.00 & 0.92 & \bf{499.05} & 
0.62 & 1.19\\CON3-7 & 586.01 & 0.70 & 
586.99 & 0.85 & \bf{576.48} & 
1.65 & 1.82\\CON3-8 & 523.68 & 0.74 & 
525.39 & 0.90 & \bf{523.05} & 
0.12 & 0.45\\CON3-9 & 589.00 & 0.96 & 
590.58 & 0.95 & \bf{578.24} & 
1.86 & 2.13\\CON8-0 & 871.27 & 1.01 & 
882.69 & 0.90 & \bf{857.17} & 
1.64 & 2.98\\CON8-1 & 753.20 & 0.84 & 
753.20 & 0.87 & \bf{740.85} & 
1.67 & 1.67\\CON8-2 & 716.07 & 0.78 & 
720.23 & 0.85 & \bf{712.89} & 
0.45 & 1.03\\CON8-3 & 832.72 & 1.02 & 
836.63 & 0.98 & \bf{811.07} & 
2.67 & 3.15\\CON8-4 & 799.03 & 0.98 & 
799.71 & 0.93 & \bf{772.25} & 
3.47 & 3.56\\CON8-5 & 773.01 & 0.76 & 
773.01 & 0.90 & \bf{754.88} & 
2.40 & 2.40\\CON8-6 & 684.95 & 0.94 & 
684.95 & 0.82 & \bf{678.92} & 
0.89 & 0.89\\CON8-7 & 815.72 & 0.67 & 
823.43 & 0.85 & \bf{811.96} & 
0.46 & 1.41\\CON8-8 & 786.88 & 0.80 & 
791.83 & 0.96 & \bf{767.53} & 
2.52 & 3.17\\CON8-9 & 814.54 & 0.82 & 
821.55 & 0.77 & \bf{809.00} & 
0.68 & 1.55\\\bf{PROM.} & 
\bf{769.31} & \bf{0.90} & \bf{772.48} & \bf{0.88} & \bf{758.54} & \bf{1.32} & \bf{1.73}\\[1ex]\hline
\end{tabular}
\label{table:nonlin}
\end{table} \clearpage
\begin{table}[ht]
\caption{Resultados de la ejecución de la metaheurística IGA, utilizando instancias de SalhiNagy con la configuración -n 200 -p 40 -cprob 90.0 -mprob 90.0}
\centering
\small
\begin{tabular}{c c c c c c c c}
\hline\hline
Instancia & Costo mínimo & Tiempo(seg.) & Costo promedio & Tiempo promedio(seg.) & CME & \%G & \%GP \\ [0.5ex]
\hline
CMT1X & 481.65 & 0.84 & 
484.52 & 0.81 & \bf{470.48} & 
2.37 & 2.98\\CMT1Y & 485.29 & 0.84 & 
485.47 & 0.83 & \bf{470.48} & 
3.15 & 3.19\\CMT2X & 704.41 & 1.34 & 
708.18 & 1.50 & \bf{682.39} & 
3.23 & 3.78\\CMT2Y & 702.71 & 1.66 & 
711.68 & 1.57 & \bf{682.39} & 
2.98 & 4.29\\CMT3X & 743.13 & 3.60 & 
745.38 & 3.46 & \bf{719.06} & 
3.35 & 3.66\\CMT3Y & 737.91 & 3.21 & 
743.07 & 3.34 & \bf{719.06} & 
2.62 & 3.34\\CMT4X & 904.80 & 8.74 & 
912.69 & 8.60 & \bf{854.21} & 
5.92 & 6.85\\CMT4Y & 910.58 & 8.78 & 
914.86 & 8.57 & \bf{852.46} & 
6.82 & 7.32\\CMT5X & 1086.63 & 18.10 & 
1107.62 & 17.46 & \bf{1030.56} & 
5.44 & 7.48\\CMT5Y & 1072.73 & 17.87 & 
1104.08 & 17.62 & \bf{1031.69} & 
3.98 & 7.02\\CMT11X & 897.68 & 5.94 & 
923.49 & 5.47 & \bf{831.09} & 
8.01 & 11.12\\CMT11Y & 885.61 & 5.62 & 
909.32 & 6.04 & \bf{829.85} & 
6.72 & 9.58\\CMT12X & 674.11 & 3.49 & 
681.29 & 3.34 & \bf{658.83} & 
2.32 & 3.41\\CMT12Y & 674.21 & 3.20 & 
678.52 & 3.26 & \bf{660.47} & 
2.08 & 2.73\\\bf{PROM.} & 
\bf{782.96} & \bf{5.95} & \bf{793.58} & \bf{5.85} & \bf{749.50} & \bf{4.21} & \bf{5.48}\\[1ex]\hline
\end{tabular}
\label{table:nonlin}
\end{table} \clearpage
\begin{table}[ht]
\caption{Resultados de la ejecución de la metaheurística IGA, utilizando instancias de Dethloff con la configuración -n 200 -p 40 -cprob 90.0 -mprob 100.0}
\centering
\small
\begin{tabular}{c c c c c c c c}
\hline\hline
Instancia & Costo mínimo & Tiempo(seg.) & Costo promedio & Tiempo promedio(seg.) & CME & \%G & \%GP \\ [0.5ex]
\hline
SCA3-0 & 640.55 & 0.92 & 
640.84 & 0.93 & \bf{635.62} & 
0.78 & 0.82\\SCA3-1 & 706.23 & 0.90 & 
706.23 & 0.90 & \bf{697.84} & 
1.20 & 1.20\\SCA3-2 & 661.13 & 0.90 & 
661.89 & 0.77 & \bf{659.34} & 
0.27 & 0.39\\SCA3-3 & 681.35 & 0.92 & 
681.35 & 0.94 & \bf{680.04} & 
0.19 & 0.19\\SCA3-4 & \bf{690.50} & 0.91 & 
690.50 & 0.77 & 690.50 & 0.00
 & 0.00\\
SCA3-5 & 677.56 & 0.94 & 
677.56 & 0.86 & \bf{659.90} & 
2.68 & 2.68\\SCA3-6 & 653.81 & 0.93 & 
654.41 & 0.88 & \bf{651.09} & 
0.42 & 0.51\\SCA3-7 & 672.74 & 0.91 & 
672.74 & 0.86 & \bf{659.17} & 
2.06 & 2.06\\SCA3-8 & 724.29 & 0.98 & 
726.81 & 0.88 & \bf{719.47} & 
0.67 & 1.02\\SCA3-9 & \bf{681.00} & 0.75 & 
683.07 & 0.85 & 681.00 & 0.00
 & 0.30\\SCA8-0 & 1008.50 & 0.70 & 
1009.00 & 0.94 & \bf{961.50} & 
4.89 & 4.94\\SCA8-1 & 1086.98 & 1.01 & 
1086.98 & 0.97 & \bf{1049.65} & 
3.56 & 3.56\\SCA8-2 & 1051.95 & 0.80 & 
1051.95 & 0.96 & \bf{1039.64} & 
1.18 & 1.18\\SCA8-3 & 1019.22 & 0.99 & 
1021.33 & 0.91 & \bf{983.34} & 
3.65 & 3.86\\SCA8-4 & 1098.79 & 0.97 & 
1101.78 & 0.94 & \bf{1065.49} & 
3.13 & 3.41\\SCA8-5 & 1061.90 & 0.91 & 
1061.90 & 0.91 & \bf{1027.08} & 
3.39 & 3.39\\SCA8-6 & 972.48 & 0.98 & 
974.68 & 0.96 & \bf{971.82} & 
0.07 & 0.29\\SCA8-7 & 1073.05 & 0.94 & 
1073.25 & 1.03 & \bf{1051.28} & 
2.07 & 2.09\\SCA8-8 & \bf{1071.18} & 1.00 & 
1076.18 & 0.97 & 1071.18 & 0.00
 & 0.47\\SCA8-9 & 1073.96 & 1.07 & 
1073.96 & 0.81 & \bf{1060.50} & 
1.27 & 1.27\\CON3-0 & 624.96 & 0.94 & 
627.53 & 0.89 & \bf{616.52} & 
1.37 & 1.79\\CON3-1 & 556.92 & 0.93 & 
559.72 & 0.89 & \bf{554.47} & 
0.44 & 0.95\\CON3-2 & 521.38 & 0.79 & 
522.26 & 0.94 & \bf{518.00} & 
0.65 & 0.82\\CON3-3 & 601.26 & 0.75 & 
604.53 & 0.77 & \bf{591.19} & 
1.70 & 2.26\\CON3-4 & 595.25 & 0.96 & 
603.07 & 0.87 & \bf{588.79} & 
1.10 & 2.42\\CON3-5 & \bf{563.70} & 0.95 & 
566.99 & 0.95 & 563.70 & 0.00
 & 0.58\\CON3-6 & 504.44 & 0.95 & 
507.06 & 0.93 & \bf{499.05} & 
1.08 & 1.60\\CON3-7 & 582.12 & 0.94 & 
584.06 & 0.91 & \bf{576.48} & 
0.98 & 1.32\\CON3-8 & 532.86 & 0.98 & 
534.74 & 0.97 & \bf{523.05} & 
1.88 & 2.23\\CON3-9 & 589.73 & 0.94 & 
589.85 & 0.95 & \bf{578.24} & 
1.99 & 2.01\\CON8-0 & 886.27 & 0.96 & 
888.29 & 0.93 & \bf{857.17} & 
3.39 & 3.63\\CON8-1 & 769.90 & 1.02 & 
771.00 & 0.94 & \bf{740.85} & 
3.92 & 4.07\\CON8-2 & 726.07 & 1.02 & 
727.90 & 1.02 & \bf{712.89} & 
1.85 & 2.11\\CON8-3 & 840.01 & 1.03 & 
840.01 & 1.00 & \bf{811.07} & 
3.57 & 3.57\\CON8-4 & 800.50 & 0.95 & 
800.50 & 0.95 & \bf{772.25} & 
3.66 & 3.66\\CON8-5 & 758.12 & 1.01 & 
761.71 & 0.99 & \bf{754.88} & 
0.43 & 0.91\\CON8-6 & 691.83 & 0.97 & 
702.44 & 0.98 & \bf{678.92} & 
1.90 & 3.46\\CON8-7 & 815.72 & 0.85 & 
815.72 & 0.92 & \bf{811.96} & 
0.46 & 0.46\\CON8-8 & 810.83 & 0.87 & 
810.83 & 0.87 & \bf{767.53} & 
5.64 & 5.64\\CON8-9 & 814.45 & 1.00 & 
815.79 & 0.98 & \bf{809.00} & 
0.67 & 0.84\\\bf{PROM.} & 
\bf{772.34} & \bf{0.93} & \bf{774.01} & \bf{0.92} & \bf{758.54} & \bf{1.70} & \bf{1.95}\\[1ex]\hline
\end{tabular}
\label{table:nonlin}
\end{table} \clearpage
\begin{table}[ht]
\caption{Resultados de la ejecución de la metaheurística IGA, utilizando instancias de SalhiNagy con la configuración -n 200 -p 40 -cprob 90.0 -mprob 100.0}
\centering
\small
\begin{tabular}{c c c c c c c c}
\hline\hline
Instancia & Costo mínimo & Tiempo(seg.) & Costo promedio & Tiempo promedio(seg.) & CME & \%G & \%GP \\ [0.5ex]
\hline
CMT1X & 484.51 & 0.83 & 
486.91 & 0.81 & \bf{470.48} & 
2.98 & 3.49\\CMT1Y & 481.85 & 0.82 & 
481.85 & 0.77 & \bf{470.48} & 
2.42 & 2.42\\CMT2X & 710.11 & 1.70 & 
718.44 & 1.68 & \bf{682.39} & 
4.06 & 5.28\\CMT2Y & 709.29 & 1.74 & 
711.18 & 1.65 & \bf{682.39} & 
3.94 & 4.22\\CMT3X & 738.81 & 3.51 & 
745.97 & 3.48 & \bf{719.06} & 
2.75 & 3.74\\CMT3Y & 742.21 & 3.36 & 
747.68 & 3.31 & \bf{719.06} & 
3.22 & 3.98\\CMT4X & 899.99 & 8.68 & 
909.11 & 8.52 & \bf{854.21} & 
5.36 & 6.43\\CMT4Y & 907.36 & 8.61 & 
916.09 & 8.69 & \bf{852.46} & 
6.44 & 7.46\\CMT5X & 1103.51 & 17.46 & 
1118.76 & 17.39 & \bf{1030.56} & 
7.08 & 8.56\\CMT5Y & 1102.73 & 18.39 & 
1125.69 & 17.80 & \bf{1031.69} & 
6.89 & 9.11\\CMT11X & 897.15 & 5.54 & 
907.26 & 5.65 & \bf{831.09} & 
7.95 & 9.17\\CMT11Y & 900.07 & 5.91 & 
915.51 & 6.06 & \bf{829.85} & 
8.46 & 10.32\\CMT12X & 675.28 & 5.21 & 
680.98 & 3.77 & \bf{658.83} & 
2.50 & 3.36\\CMT12Y & 673.49 & 3.40 & 
677.92 & 3.38 & \bf{660.47} & 
1.97 & 2.64\\\bf{PROM.} & 
\bf{787.60} & \bf{6.08} & \bf{795.95} & \bf{5.93} & \bf{749.50} & \bf{4.72} & \bf{5.73}\\[1ex]\hline
\end{tabular}
\label{table:nonlin}
\end{table} \clearpage
\begin{table}[ht]
\caption{Resultados de la ejecución de la metaheurística IGA, utilizando instancias de Dethloff con la configuración -n 200 -p 40 -cprob 100.0 -mprob 10.0}
\centering
\small
\begin{tabular}{c c c c c c c c}
\hline\hline
Instancia & Costo mínimo & Tiempo(seg.) & Costo promedio & Tiempo promedio(seg.) & CME & \%G & \%GP \\ [0.5ex]
\hline
SCA3-0 & 640.55 & 0.74 & 
641.50 & 0.73 & \bf{635.62} & 
0.78 & 0.92\\SCA3-1 & 700.50 & 0.80 & 
706.43 & 0.75 & \bf{697.84} & 
0.38 & 1.23\\SCA3-2 & \bf{659.34} & 0.67 & 
663.56 & 0.75 & 659.34 & 0.00
 & 0.64\\SCA3-3 & \bf{680.04} & 0.73 & 
683.23 & 0.72 & 680.04 & 0.00
 & 0.47\\SCA3-4 & \bf{690.50} & 0.68 & 
690.50 & 0.68 & 690.50 & 0.00
 & 0.00\\
SCA3-5 & 662.75 & 0.71 & 
663.73 & 0.72 & \bf{659.90} & 
0.43 & 0.58\\SCA3-6 & 657.24 & 0.73 & 
660.78 & 0.74 & \bf{651.09} & 
0.94 & 1.49\\SCA3-7 & 666.15 & 0.69 & 
666.15 & 0.70 & \bf{659.17} & 
1.06 & 1.06\\SCA3-8 & \bf{719.47} & 0.70 & 
725.66 & 0.71 & 719.47 & 0.00
 & 0.86\\SCA3-9 & \bf{681.00} & 0.71 & 
684.48 & 0.72 & 681.00 & 0.00
 & 0.51\\SCA8-0 & 999.88 & 0.78 & 
1000.61 & 0.82 & \bf{961.50} & 
3.99 & 4.07\\SCA8-1 & 1077.08 & 1.00 & 
1077.14 & 0.82 & \bf{1049.65} & 
2.61 & 2.62\\SCA8-2 & 1054.12 & 0.78 & 
1054.12 & 0.78 & \bf{1039.64} & 
1.39 & 1.39\\SCA8-3 & 1021.95 & 0.76 & 
1021.95 & 0.74 & \bf{983.34} & 
3.93 & 3.93\\SCA8-4 & 1074.11 & 0.77 & 
1074.11 & 0.79 & \bf{1065.49} & 
0.81 & 0.81\\SCA8-5 & 1064.18 & 0.72 & 
1065.71 & 0.78 & \bf{1027.08} & 
3.61 & 3.76\\SCA8-6 & 987.01 & 0.86 & 
994.25 & 0.78 & \bf{971.82} & 
1.56 & 2.31\\SCA8-7 & 1076.22 & 0.95 & 
1078.58 & 0.83 & \bf{1051.28} & 
2.37 & 2.60\\SCA8-8 & \bf{1071.18} & 0.75 & 
1088.22 & 0.73 & 1071.18 & 0.00
 & 1.59\\SCA8-9 & 1086.61 & 0.69 & 
1087.97 & 0.74 & \bf{1060.50} & 
2.46 & 2.59\\CON3-0 & 624.96 & 0.75 & 
629.30 & 0.74 & \bf{616.52} & 
1.37 & 2.07\\CON3-1 & 556.04 & 0.74 & 
560.98 & 0.75 & \bf{554.47} & 
0.28 & 1.17\\CON3-2 & 521.38 & 0.77 & 
523.93 & 0.76 & \bf{518.00} & 
0.65 & 1.15\\CON3-3 & 595.06 & 0.70 & 
602.02 & 0.76 & \bf{591.19} & 
0.65 & 1.83\\CON3-4 & \bf{588.79} & 0.76 & 
593.10 & 0.73 & 588.79 & 0.00
 & 0.73\\CON3-5 & \bf{563.70} & 0.79 & 
564.76 & 0.77 & 563.70 & 0.00
 & 0.19\\CON3-6 & 506.04 & 0.76 & 
506.69 & 0.77 & \bf{499.05} & 
1.40 & 1.53\\CON3-7 & 582.33 & 0.75 & 
584.17 & 0.73 & \bf{576.48} & 
1.01 & 1.33\\CON3-8 & 526.59 & 0.74 & 
527.72 & 0.75 & \bf{523.05} & 
0.68 & 0.89\\CON3-9 & 582.79 & 0.78 & 
586.23 & 0.77 & \bf{578.24} & 
0.79 & 1.38\\CON8-0 & 875.17 & 0.75 & 
878.46 & 0.77 & \bf{857.17} & 
2.10 & 2.48\\CON8-1 & 768.43 & 0.78 & 
768.43 & 0.78 & \bf{740.85} & 
3.72 & 3.72\\CON8-2 & 717.88 & 0.80 & 
721.82 & 0.79 & \bf{712.89} & 
0.70 & 1.25\\CON8-3 & 833.30 & 0.81 & 
837.22 & 0.78 & \bf{811.07} & 
2.74 & 3.22\\CON8-4 & 792.17 & 0.74 & 
793.17 & 0.72 & \bf{772.25} & 
2.58 & 2.71\\CON8-5 & 762.61 & 0.73 & 
763.00 & 0.74 & \bf{754.88} & 
1.02 & 1.08\\CON8-6 & 696.37 & 0.76 & 
696.43 & 0.76 & \bf{678.92} & 
2.57 & 2.58\\CON8-7 & 828.97 & 0.73 & 
830.80 & 0.93 & \bf{811.96} & 
2.09 & 2.32\\CON8-8 & 798.51 & 0.76 & 
799.73 & 0.76 & \bf{767.53} & 
4.04 & 4.19\\CON8-9 & 824.52 & 0.72 & 
824.52 & 0.81 & \bf{809.00} & 
1.92 & 1.92\\\bf{PROM.} & 
\bf{770.39} & \bf{0.76} & \bf{773.03} & \bf{0.76} & \bf{758.54} & \bf{1.42} & \bf{1.78}\\[1ex]\hline
\end{tabular}
\label{table:nonlin}
\end{table} \clearpage
\begin{table}[ht]
\caption{Resultados de la ejecución de la metaheurística IGA, utilizando instancias de SalhiNagy con la configuración -n 200 -p 40 -cprob 100.0 -mprob 10.0}
\centering
\small
\begin{tabular}{c c c c c c c c}
\hline\hline
Instancia & Costo mínimo & Tiempo(seg.) & Costo promedio & Tiempo promedio(seg.) & CME & \%G & \%GP \\ [0.5ex]
\hline
CMT1X & 474.87 & 0.64 & 
476.06 & 0.64 & \bf{470.48} & 
0.93 & 1.19\\CMT1Y & 483.34 & 0.57 & 
485.13 & 0.58 & \bf{470.48} & 
2.73 & 3.11\\CMT2X & 707.93 & 1.45 & 
708.78 & 1.48 & \bf{682.39} & 
3.74 & 3.87\\CMT2Y & 699.52 & 1.73 & 
708.03 & 1.48 & \bf{682.39} & 
2.51 & 3.76\\CMT3X & 744.59 & 3.18 & 
751.20 & 3.19 & \bf{719.06} & 
3.55 & 4.47\\CMT3Y & 739.64 & 3.00 & 
747.82 & 3.02 & \bf{719.06} & 
2.86 & 4.00\\CMT4X & 915.87 & 8.25 & 
918.86 & 8.28 & \bf{854.21} & 
7.22 & 7.57\\CMT4Y & 899.38 & 8.71 & 
916.08 & 8.40 & \bf{852.46} & 
5.50 & 7.46\\CMT5X & 1121.54 & 16.39 & 
1133.75 & 16.54 & \bf{1030.56} & 
8.83 & 10.01\\CMT5Y & 1108.78 & 16.77 & 
1123.13 & 16.53 & \bf{1031.69} & 
7.47 & 8.86\\CMT11X & 903.75 & 4.99 & 
920.28 & 5.13 & \bf{831.09} & 
8.74 & 10.73\\CMT11Y & 896.88 & 5.49 & 
918.74 & 5.53 & \bf{829.85} & 
8.08 & 10.71\\CMT12X & 677.87 & 3.13 & 
685.05 & 3.13 & \bf{658.83} & 
2.89 & 3.98\\CMT12Y & 673.59 & 3.09 & 
675.42 & 3.09 & \bf{660.47} & 
1.99 & 2.26\\\bf{PROM.} & 
\bf{789.11} & \bf{5.53} & \bf{797.74} & \bf{5.50} & \bf{749.50} & \bf{4.79} & \bf{5.86}\\[1ex]\hline
\end{tabular}
\label{table:nonlin}
\end{table} \clearpage
\begin{table}[ht]
\caption{Resultados de la ejecución de la metaheurística IGA, utilizando instancias de Dethloff con la configuración -n 200 -p 40 -cprob 100.0 -mprob 20.0}
\centering
\small
\begin{tabular}{c c c c c c c c}
\hline\hline
Instancia & Costo mínimo & Tiempo(seg.) & Costo promedio & Tiempo promedio(seg.) & CME & \%G & \%GP \\ [0.5ex]
\hline
SCA3-0 & 640.55 & 0.83 & 
641.40 & 0.75 & \bf{635.62} & 
0.78 & 0.91\\SCA3-1 & \bf{697.84} & 0.72 & 
698.50 & 0.77 & 697.84 & 0.00
 & 0.10\\SCA3-2 & 666.72 & 0.72 & 
671.32 & 0.76 & \bf{659.34} & 
1.12 & 1.82\\SCA3-3 & 681.35 & 0.72 & 
682.19 & 0.76 & \bf{680.04} & 
0.19 & 0.32\\SCA3-4 & \bf{690.50} & 0.69 & 
691.02 & 0.68 & 690.50 & 0.00
 & 0.08\\SCA3-5 & 682.53 & 0.72 & 
684.84 & 0.72 & \bf{659.90} & 
3.43 & 3.78\\SCA3-6 & 652.94 & 0.71 & 
654.00 & 0.75 & \bf{651.09} & 
0.28 & 0.45\\SCA3-7 & 667.24 & 0.71 & 
667.24 & 0.78 & \bf{659.17} & 
1.22 & 1.22\\SCA3-8 & 723.99 & 0.72 & 
724.85 & 0.72 & \bf{719.47} & 
0.63 & 0.75\\SCA3-9 & 685.14 & 0.71 & 
685.14 & 0.69 & \bf{681.00} & 
0.61 & 0.61\\SCA8-0 & 1004.88 & 0.78 & 
1011.58 & 0.76 & \bf{961.50} & 
4.51 & 5.21\\SCA8-1 & 1068.14 & 0.73 & 
1070.31 & 0.77 & \bf{1049.65} & 
1.76 & 1.97\\SCA8-2 & 1053.94 & 0.80 & 
1053.94 & 0.81 & \bf{1039.64} & 
1.38 & 1.38\\SCA8-3 & 1000.75 & 0.77 & 
1011.88 & 0.78 & \bf{983.34} & 
1.77 & 2.90\\SCA8-4 & 1081.94 & 0.80 & 
1081.94 & 0.81 & \bf{1065.49} & 
1.54 & 1.54\\SCA8-5 & 1053.73 & 0.75 & 
1055.90 & 0.80 & \bf{1027.08} & 
2.59 & 2.81\\SCA8-6 & 993.77 & 0.76 & 
996.58 & 0.75 & \bf{971.82} & 
2.26 & 2.55\\SCA8-7 & 1082.71 & 0.70 & 
1097.61 & 0.75 & \bf{1051.28} & 
2.99 & 4.41\\SCA8-8 & 1075.00 & 0.73 & 
1090.55 & 0.79 & \bf{1071.18} & 
0.36 & 1.81\\SCA8-9 & 1077.42 & 0.69 & 
1077.42 & 0.80 & \bf{1060.50} & 
1.60 & 1.60\\CON3-0 & 635.60 & 0.71 & 
636.83 & 0.72 & \bf{616.52} & 
3.09 & 3.29\\CON3-1 & 560.75 & 0.75 & 
563.70 & 0.75 & \bf{554.47} & 
1.13 & 1.67\\CON3-2 & 521.38 & 0.79 & 
522.26 & 0.83 & \bf{518.00} & 
0.65 & 0.82\\CON3-3 & 594.11 & 0.76 & 
600.26 & 0.75 & \bf{591.19} & 
0.49 & 1.54\\CON3-4 & 593.78 & 0.72 & 
595.16 & 0.72 & \bf{588.79} & 
0.85 & 1.08\\CON3-5 & 572.75 & 0.72 & 
576.15 & 0.73 & \bf{563.70} & 
1.61 & 2.21\\CON3-6 & 503.97 & 0.75 & 
506.50 & 0.80 & \bf{499.05} & 
0.99 & 1.49\\CON3-7 & 586.84 & 0.94 & 
589.38 & 0.76 & \bf{576.48} & 
1.80 & 2.24\\CON3-8 & 524.59 & 0.78 & 
530.13 & 0.80 & \bf{523.05} & 
0.29 & 1.35\\CON3-9 & 588.48 & 1.01 & 
588.48 & 0.81 & \bf{578.24} & 
1.77 & 1.77\\CON8-0 & 871.40 & 0.77 & 
880.98 & 0.80 & \bf{857.17} & 
1.66 & 2.78\\CON8-1 & 766.47 & 1.03 & 
766.47 & 0.90 & \bf{740.85} & 
3.46 & 3.46\\CON8-2 & 729.06 & 0.80 & 
731.58 & 0.80 & \bf{712.89} & 
2.27 & 2.62\\CON8-3 & 820.82 & 0.78 & 
823.79 & 0.81 & \bf{811.07} & 
1.20 & 1.57\\CON8-4 & 781.78 & 0.87 & 
781.78 & 0.90 & \bf{772.25} & 
1.23 & 1.23\\CON8-5 & 758.12 & 0.76 & 
758.12 & 0.76 & \bf{754.88} & 
0.43 & 0.43\\CON8-6 & 689.23 & 0.76 & 
689.23 & 0.77 & \bf{678.92} & 
1.52 & 1.52\\CON8-7 & 814.79 & 0.71 & 
818.00 & 0.71 & \bf{811.96} & 
0.35 & 0.74\\CON8-8 & 795.16 & 0.80 & 
795.39 & 0.83 & \bf{767.53} & 
3.60 & 3.63\\CON8-9 & 818.11 & 0.73 & 
826.73 & 0.76 & \bf{809.00} & 
1.13 & 2.19\\\bf{PROM.} & 
\bf{770.21} & \bf{0.77} & \bf{773.23} & \bf{0.77} & \bf{758.54} & \bf{1.46} & \bf{1.85}\\[1ex]\hline
\end{tabular}
\label{table:nonlin}
\end{table} \clearpage
\begin{table}[ht]
\caption{Resultados de la ejecución de la metaheurística IGA, utilizando instancias de SalhiNagy con la configuración -n 200 -p 40 -cprob 100.0 -mprob 20.0}
\centering
\small
\begin{tabular}{c c c c c c c c}
\hline\hline
Instancia & Costo mínimo & Tiempo(seg.) & Costo promedio & Tiempo promedio(seg.) & CME & \%G & \%GP \\ [0.5ex]
\hline
CMT1X & 479.29 & 0.73 & 
483.01 & 0.68 & \bf{470.48} & 
1.87 & 2.66\\CMT1Y & 480.65 & 0.59 & 
481.60 & 0.67 & \bf{470.48} & 
2.16 & 2.36\\CMT2X & 701.84 & 1.42 & 
708.77 & 1.49 & \bf{682.39} & 
2.85 & 3.87\\CMT2Y & 701.70 & 1.42 & 
707.50 & 1.41 & \bf{682.39} & 
2.83 & 3.68\\CMT3X & 744.40 & 3.14 & 
745.55 & 3.12 & \bf{719.06} & 
3.52 & 3.68\\CMT3Y & 738.81 & 3.16 & 
744.01 & 3.20 & \bf{719.06} & 
2.75 & 3.47\\CMT4X & 894.84 & 8.39 & 
909.16 & 8.48 & \bf{854.21} & 
4.76 & 6.43\\CMT4Y & 904.38 & 8.22 & 
920.34 & 8.15 & \bf{852.46} & 
6.09 & 7.96\\CMT5X & 1115.75 & 18.32 & 
1124.13 & 17.97 & \bf{1030.56} & 
8.27 & 9.08\\CMT5Y & 1109.93 & 17.31 & 
1119.95 & 17.12 & \bf{1031.69} & 
7.58 & 8.55\\CMT11X & 890.56 & 6.39 & 
904.56 & 5.49 & \bf{831.09} & 
7.16 & 8.84\\CMT11Y & 900.16 & 5.88 & 
904.88 & 5.95 & \bf{829.85} & 
8.47 & 9.04\\CMT12X & 675.25 & 3.20 & 
685.55 & 3.20 & \bf{658.83} & 
2.49 & 4.06\\CMT12Y & 673.70 & 3.17 & 
678.69 & 3.23 & \bf{660.47} & 
2.00 & 2.76\\\bf{PROM.} & 
\bf{786.52} & \bf{5.81} & \bf{794.12} & \bf{5.72} & \bf{749.50} & \bf{4.49} & \bf{5.46}\\[1ex]\hline
\end{tabular}
\label{table:nonlin}
\end{table} \clearpage
\begin{table}[ht]
\caption{Resultados de la ejecución de la metaheurística IGA, utilizando instancias de Dethloff con la configuración -n 200 -p 40 -cprob 100.0 -mprob 30.0}
\centering
\small
\begin{tabular}{c c c c c c c c}
\hline\hline
Instancia & Costo mínimo & Tiempo(seg.) & Costo promedio & Tiempo promedio(seg.) & CME & \%G & \%GP \\ [0.5ex]
\hline
SCA3-0 & 641.69 & 0.73 & 
641.69 & 0.78 & \bf{635.62} & 
0.95 & 0.95\\SCA3-1 & 707.07 & 0.74 & 
709.35 & 0.72 & \bf{697.84} & 
1.32 & 1.65\\SCA3-2 & 661.13 & 0.70 & 
664.99 & 0.77 & \bf{659.34} & 
0.27 & 0.86\\SCA3-3 & \bf{680.04} & 0.71 & 
680.79 & 0.73 & 680.04 & 0.00
 & 0.11\\SCA3-4 & \bf{690.50} & 0.72 & 
690.50 & 0.70 & 690.50 & 0.00
 & 0.00\\
SCA3-5 & 665.04 & 0.71 & 
669.77 & 0.76 & \bf{659.90} & 
0.78 & 1.50\\SCA3-6 & 652.94 & 0.95 & 
655.09 & 0.79 & \bf{651.09} & 
0.28 & 0.61\\SCA3-7 & 666.15 & 0.82 & 
667.80 & 0.73 & \bf{659.17} & 
1.06 & 1.31\\SCA3-8 & 719.77 & 0.71 & 
722.57 & 0.74 & \bf{719.47} & 
0.04 & 0.43\\SCA3-9 & 685.14 & 0.72 & 
687.73 & 0.81 & \bf{681.00} & 
0.61 & 0.99\\SCA8-0 & 1006.80 & 0.78 & 
1009.53 & 0.75 & \bf{961.50} & 
4.71 & 5.00\\SCA8-1 & 1073.23 & 0.73 & 
1073.23 & 0.80 & \bf{1049.65} & 
2.25 & 2.25\\SCA8-2 & 1054.47 & 0.84 & 
1059.53 & 0.79 & \bf{1039.64} & 
1.43 & 1.91\\SCA8-3 & 1002.86 & 0.88 & 
1002.86 & 0.84 & \bf{983.34} & 
1.99 & 1.99\\SCA8-4 & 1067.55 & 0.75 & 
1069.58 & 0.73 & \bf{1065.49} & 
0.19 & 0.38\\SCA8-5 & 1052.08 & 0.76 & 
1052.08 & 0.83 & \bf{1027.08} & 
2.43 & 2.43\\SCA8-6 & 977.87 & 0.80 & 
977.87 & 0.76 & \bf{971.82} & 
0.62 & 0.62\\SCA8-7 & 1077.67 & 0.72 & 
1081.12 & 0.80 & \bf{1051.28} & 
2.51 & 2.84\\SCA8-8 & 1089.07 & 0.92 & 
1089.07 & 0.78 & \bf{1071.18} & 
1.67 & 1.67\\SCA8-9 & 1084.04 & 0.72 & 
1086.31 & 0.73 & \bf{1060.50} & 
2.22 & 2.43\\CON3-0 & 623.15 & 0.70 & 
629.00 & 0.72 & \bf{616.52} & 
1.08 & 2.03\\CON3-1 & 559.25 & 0.68 & 
563.24 & 0.71 & \bf{554.47} & 
0.86 & 1.58\\CON3-2 & 521.38 & 0.83 & 
521.38 & 0.84 & \bf{518.00} & 
0.65 & 0.65\\CON3-3 & 601.41 & 0.74 & 
607.44 & 0.77 & \bf{591.19} & 
1.73 & 2.75\\CON3-4 & 595.25 & 0.72 & 
599.52 & 0.73 & \bf{588.79} & 
1.10 & 1.82\\CON3-5 & 567.94 & 0.74 & 
568.54 & 0.73 & \bf{563.70} & 
0.75 & 0.86\\CON3-6 & 502.34 & 0.80 & 
506.01 & 0.89 & \bf{499.05} & 
0.66 & 1.40\\CON3-7 & 586.01 & 0.92 & 
587.12 & 0.82 & \bf{576.48} & 
1.65 & 1.85\\CON3-8 & 524.59 & 0.75 & 
524.59 & 0.81 & \bf{523.05} & 
0.29 & 0.29\\CON3-9 & 589.61 & 0.73 & 
589.75 & 0.80 & \bf{578.24} & 
1.97 & 1.99\\CON8-0 & 860.92 & 0.85 & 
860.92 & 0.79 & \bf{857.17} & 
0.44 & 0.44\\CON8-1 & 764.97 & 0.85 & 
765.49 & 0.78 & \bf{740.85} & 
3.26 & 3.33\\CON8-2 & 725.13 & 0.77 & 
726.13 & 0.88 & \bf{712.89} & 
1.72 & 1.86\\CON8-3 & 830.45 & 0.90 & 
830.45 & 0.85 & \bf{811.07} & 
2.39 & 2.39\\CON8-4 & 780.48 & 1.06 & 
782.07 & 0.84 & \bf{772.25} & 
1.07 & 1.27\\CON8-5 & 766.65 & 0.75 & 
768.10 & 0.86 & \bf{754.88} & 
1.56 & 1.75\\CON8-6 & 695.97 & 0.78 & 
695.97 & 0.78 & \bf{678.92} & 
2.51 & 2.51\\CON8-7 & 816.85 & 0.74 & 
817.96 & 0.81 & \bf{811.96} & 
0.60 & 0.74\\CON8-8 & 780.71 & 0.86 & 
781.73 & 0.94 & \bf{767.53} & 
1.72 & 1.85\\CON8-9 & 831.15 & 0.74 & 
831.15 & 0.88 & \bf{809.00} & 
2.74 & 2.74\\\bf{PROM.} & 
\bf{769.48} & \bf{0.78} & \bf{771.20} & \bf{0.79} & \bf{758.54} & \bf{1.35} & \bf{1.60}\\[1ex]\hline
\end{tabular}
\label{table:nonlin}
\end{table} \clearpage
\begin{table}[ht]
\caption{Resultados de la ejecución de la metaheurística IGA, utilizando instancias de SalhiNagy con la configuración -n 200 -p 40 -cprob 100.0 -mprob 30.0}
\centering
\small
\begin{tabular}{c c c c c c c c}
\hline\hline
Instancia & Costo mínimo & Tiempo(seg.) & Costo promedio & Tiempo promedio(seg.) & CME & \%G & \%GP \\ [0.5ex]
\hline
CMT1X & 480.39 & 0.73 & 
485.07 & 0.70 & \bf{470.48} & 
2.11 & 3.10\\CMT1Y & 474.87 & 0.62 & 
479.36 & 0.68 & \bf{470.48} & 
0.93 & 1.89\\CMT2X & 704.52 & 1.45 & 
712.53 & 1.52 & \bf{682.39} & 
3.24 & 4.42\\CMT2Y & 705.85 & 1.37 & 
710.55 & 1.50 & \bf{682.39} & 
3.44 & 4.13\\CMT3X & 733.11 & 3.40 & 
739.76 & 3.33 & \bf{719.06} & 
1.95 & 2.88\\CMT3Y & 748.35 & 3.29 & 
752.96 & 3.15 & \bf{719.06} & 
4.07 & 4.71\\CMT4X & 897.25 & 8.67 & 
914.24 & 8.47 & \bf{854.21} & 
5.04 & 7.03\\CMT4Y & 901.19 & 8.76 & 
912.57 & 8.64 & \bf{852.46} & 
5.72 & 7.05\\CMT5X & 1119.50 & 16.68 & 
1127.05 & 16.58 & \bf{1030.56} & 
8.63 & 9.36\\CMT5Y & 1107.54 & 16.81 & 
1116.34 & 17.23 & \bf{1031.69} & 
7.35 & 8.20\\CMT11X & 911.77 & 5.48 & 
924.06 & 5.26 & \bf{831.09} & 
9.71 & 11.19\\CMT11Y & 884.48 & 5.93 & 
897.73 & 5.93 & \bf{829.85} & 
6.58 & 8.18\\CMT12X & 678.73 & 3.15 & 
684.12 & 3.27 & \bf{658.83} & 
3.02 & 3.84\\CMT12Y & 674.63 & 3.33 & 
679.97 & 3.15 & \bf{660.47} & 
2.14 & 2.95\\\bf{PROM.} & 
\bf{787.30} & \bf{5.69} & \bf{795.45} & \bf{5.67} & \bf{749.50} & \bf{4.57} & \bf{5.64}\\[1ex]\hline
\end{tabular}
\label{table:nonlin}
\end{table} \clearpage
\begin{table}[ht]
\caption{Resultados de la ejecución de la metaheurística IGA, utilizando instancias de Dethloff con la configuración -n 200 -p 40 -cprob 100.0 -mprob 40.0}
\centering
\small
\begin{tabular}{c c c c c c c c}
\hline\hline
Instancia & Costo mínimo & Tiempo(seg.) & Costo promedio & Tiempo promedio(seg.) & CME & \%G & \%GP \\ [0.5ex]
\hline
SCA3-0 & 636.06 & 0.74 & 
638.30 & 0.72 & \bf{635.62} & 
0.07 & 0.42\\SCA3-1 & \bf{697.84} & 0.68 & 
699.17 & 0.79 & 697.84 & 0.00
 & 0.19\\SCA3-2 & 666.01 & 0.67 & 
666.95 & 0.78 & \bf{659.34} & 
1.01 & 1.15\\SCA3-3 & \bf{680.04} & 0.91 & 
682.12 & 0.80 & 680.04 & 0.00
 & 0.31\\SCA3-4 & \bf{690.50} & 0.68 & 
692.05 & 0.80 & 690.50 & 0.00
 & 0.22\\SCA3-5 & 662.75 & 0.97 & 
665.62 & 0.93 & \bf{659.90} & 
0.43 & 0.87\\SCA3-6 & \bf{651.09} & 0.70 & 
654.93 & 0.76 & 651.09 & 0.00
 & 0.59\\SCA3-7 & 667.24 & 0.93 & 
668.37 & 0.80 & \bf{659.17} & 
1.22 & 1.40\\SCA3-8 & 728.24 & 1.09 & 
728.24 & 0.82 & \bf{719.47} & 
1.22 & 1.22\\SCA3-9 & 685.14 & 0.72 & 
685.18 & 0.88 & \bf{681.00} & 
0.61 & 0.61\\SCA8-0 & 1011.78 & 0.75 & 
1011.78 & 0.75 & \bf{961.50} & 
5.23 & 5.23\\SCA8-1 & 1070.34 & 1.01 & 
1070.34 & 0.86 & \bf{1049.65} & 
1.97 & 1.97\\SCA8-2 & 1052.94 & 0.74 & 
1052.94 & 0.79 & \bf{1039.64} & 
1.28 & 1.28\\SCA8-3 & 1014.71 & 0.72 & 
1014.71 & 0.74 & \bf{983.34} & 
3.19 & 3.19\\SCA8-4 & 1071.86 & 0.99 & 
1075.41 & 0.83 & \bf{1065.49} & 
0.60 & 0.93\\SCA8-5 & 1050.47 & 0.81 & 
1050.47 & 0.82 & \bf{1027.08} & 
2.28 & 2.28\\SCA8-6 & 981.41 & 0.96 & 
984.87 & 0.80 & \bf{971.82} & 
0.99 & 1.34\\SCA8-7 & 1070.92 & 0.94 & 
1072.39 & 0.84 & \bf{1051.28} & 
1.87 & 2.01\\SCA8-8 & 1091.22 & 0.74 & 
1092.93 & 0.82 & \bf{1071.18} & 
1.87 & 2.03\\SCA8-9 & 1074.19 & 0.71 & 
1084.97 & 0.72 & \bf{1060.50} & 
1.29 & 2.31\\CON3-0 & 632.66 & 0.71 & 
633.82 & 0.82 & \bf{616.52} & 
2.62 & 2.81\\CON3-1 & 557.21 & 0.93 & 
561.23 & 0.89 & \bf{554.47} & 
0.49 & 1.22\\CON3-2 & 524.89 & 0.94 & 
524.89 & 0.82 & \bf{518.00} & 
1.33 & 1.33\\CON3-3 & 595.06 & 0.71 & 
607.09 & 0.71 & \bf{591.19} & 
0.65 & 2.69\\CON3-4 & 591.43 & 0.74 & 
593.55 & 0.83 & \bf{588.79} & 
0.45 & 0.81\\CON3-5 & 564.88 & 0.78 & 
565.65 & 0.84 & \bf{563.70} & 
0.21 & 0.35\\CON3-6 & 502.16 & 0.88 & 
507.25 & 1.02 & \bf{499.05} & 
0.62 & 1.64\\CON3-7 & 586.01 & 0.70 & 
588.14 & 0.71 & \bf{576.48} & 
1.65 & 2.02\\CON3-8 & 524.30 & 0.72 & 
526.39 & 0.78 & \bf{523.05} & 
0.24 & 0.64\\CON3-9 & 582.79 & 0.92 & 
586.83 & 0.81 & \bf{578.24} & 
0.79 & 1.49\\CON8-0 & 870.80 & 0.78 & 
876.90 & 0.85 & \bf{857.17} & 
1.59 & 2.30\\CON8-1 & 757.38 & 0.82 & 
761.98 & 0.83 & \bf{740.85} & 
2.23 & 2.85\\CON8-2 & 725.55 & 0.77 & 
726.02 & 0.85 & \bf{712.89} & 
1.78 & 1.84\\CON8-3 & 838.17 & 0.75 & 
838.17 & 0.88 & \bf{811.07} & 
3.34 & 3.34\\CON8-4 & 795.23 & 0.84 & 
805.65 & 0.66 & \bf{772.25} & 
2.98 & 4.33\\CON8-5 & 770.31 & 0.98 & 
772.51 & 0.81 & \bf{754.88} & 
2.04 & 2.34\\CON8-6 & 688.17 & 0.79 & 
696.64 & 0.84 & \bf{678.92} & 
1.36 & 2.61\\CON8-7 & 828.65 & 0.72 & 
828.65 & 0.79 & \bf{811.96} & 
2.06 & 2.06\\CON8-8 & 782.34 & 1.00 & 
787.55 & 0.84 & \bf{767.53} & 
1.93 & 2.61\\CON8-9 & 834.03 & 0.79 & 
834.03 & 0.82 & \bf{809.00} & 
3.09 & 3.09\\\bf{PROM.} & 
\bf{770.17} & \bf{0.82} & \bf{772.87} & \bf{0.81} & \bf{758.54} & \bf{1.41} & \bf{1.80}\\[1ex]\hline
\end{tabular}
\label{table:nonlin}
\end{table} \clearpage
\begin{table}[ht]
\caption{Resultados de la ejecución de la metaheurística SCA, utilizando instancias de SalhiNagy con la configuración -n 100.0 -b 10 -y 0.1}
\centering
\small
\begin{tabular}{c c c c c c c c}
\hline\hline
Instancia & Costo mínimo & Tiempo(seg.) & Costo promedio & Tiempo promedio(seg.) & CME & \%G & \%GP \\ [0.5ex]
\hline
CMT1X & 472.37 & 2.80 & 
477.26 & 2.26 & \bf{470.48} & 
0.40 & 1.44\\CMT1Y & 472.37 & 0.95 & 
473.92 & 1.29 & \bf{470.48} & 
0.40 & 0.73\\CMT2X & 698.45 & 16.53 & 
706.10 & 20.65 & \bf{682.39} & 
2.35 & 3.47\\CMT2Y & 706.66 & 18.27 & 
709.14 & 16.89 & \bf{682.39} & 
3.56 & 3.92\\CMT3X & 736.30 & 23.68 & 
737.29 & 32.04 & \bf{719.06} & 
2.40 & 2.53\\CMT3Y & 728.61 & 47.81 & 
734.18 & 29.14 & \bf{719.06} & 
1.33 & 2.10\\CMT4X & 897.70884.57903.98 & | & 
0.00 & 0.00 & \bf{854.21} & 
5.09 & -100.00\\CMT4Y & 898.58 & 149.36 & 
912.00 & 101.67 & \bf{852.46} & 
5.41 & 6.98\\CMT5X & 100000 & 0 & 
nan & nan & \bf{1030.56} & 
9603.46 & \bf{nan}\\CMT5Y & 100000 & 0 & 
nan & nan & \bf{1031.69} & 
9592.83 & \bf{nan}\\CMT11X & 890.64 & 0.98 & 
891.64 & 17.47 & \bf{831.09} & 
7.17 & 7.29\\CMT11Y & 892.70 & 58.80 & 
903.41 & 46.51 & \bf{829.85} & 
7.57 & 8.86\\CMT12X & 673.83695.56 & 34.02 & 
679.60 & 51.03 & \bf{658.83} & 
2.28 & 3.15\\CMT12Y & 673.80 & 92.24 & 
678.46 & 66.73 & \bf{660.47} & 
2.02 & 2.72\\\bf{PROM.} & 
\bf{14910.14} & \bf{31.82} & \bf{nan} & \bf{nan} & \bf{749.50} & \bf{1374.02} & \bf{nan}\\[1ex]\hline
\end{tabular}
\label{table:nonlin}
\end{table} \clearpage
\begin{table}[ht]
\caption{Resultados de la ejecución de la metaheurística IGA, utilizando instancias de SalhiNagy con la configuración -n 200 -p 40 -cprob 100.0 -mprob 40.0}
\centering
\small
\begin{tabular}{c c c c c c c c}
\hline\hline
Instancia & Costo mínimo & Tiempo(seg.) & Costo promedio & Tiempo promedio(seg.) & CME & \%G & \%GP \\ [0.5ex]
\hline
CMT1X & 478.97 & 0.64 & 
481.73 & 0.69 & \bf{470.48} & 
1.80 & 2.39\\CMT1Y & 485.84 & 0.84 & 
490.00 & 0.76 & \bf{470.48} & 
3.26 & 4.15\\CMT2X & 707.32 & 1.35 & 
717.14 & 1.45 & \bf{682.39} & 
3.65 & 5.09\\CMT2Y & 703.92 & 1.64 & 
716.49 & 1.51 & \bf{682.39} & 
3.16 & 5.00\\CMT3X & 739.77 & 3.06 & 
742.79 & 3.14 & \bf{719.06} & 
2.88 & 3.30\\CMT3Y & 741.29 & 2.98 & 
745.85 & 2.98 & \bf{719.06} & 
3.09 & 3.73\\CMT4X & 903.96 & 7.89 & 
913.37 & 8.20 & \bf{854.21} & 
5.82 & 6.93\\CMT4Y & 906.63 & 7.87 & 
917.10 & 8.11 & \bf{852.46} & 
6.35 & 7.58\\CMT5X & 1095.81 & 16.07 & 
1115.03 & 16.91 & \bf{1030.56} & 
6.33 & 8.20\\CMT5Y & 1113.35 & 16.40 & 
1119.57 & 12.79 & \bf{1031.69} & 
7.92 & 8.52\\CMT11X & 908.95 & 5.78 & 
918.63 & 5.53 & \bf{831.09} & 
9.37 & 10.53\\CMT11Y & 903.73 & 5.89 & 
913.65 & 5.77 & \bf{829.85} & 
8.90 & 10.10\\CMT12X & 677.68 & 3.39 & 
684.71 & 3.10 & \bf{658.83} & 
2.86 & 3.93\\CMT12Y & 679.44 & 3.23 & 
684.27 & 3.06 & \bf{660.47} & 
2.87 & 3.60\\\bf{PROM.} & 
\bf{789.05} & \bf{5.50} & \bf{797.17} & \bf{5.29} & \bf{749.50} & \bf{4.88} & \bf{5.93}\\[1ex]\hline
\end{tabular}
\label{table:nonlin}
\end{table} \clearpage
\begin{table}[ht]
\caption{Resultados de la ejecución de la metaheurística IGA, utilizando instancias de Dethloff con la configuración -n 200 -p 40 -cprob 100.0 -mprob 50.0}
\centering
\small
\begin{tabular}{c c c c c c c c}
\hline\hline
Instancia & Costo mínimo & Tiempo(seg.) & Costo promedio & Tiempo promedio(seg.) & CME & \%G & \%GP \\ [0.5ex]
\hline
SCA3-0 & 640.55 & 0.71 & 
640.84 & 0.82 & \bf{635.62} & 
0.78 & 0.82\\SCA3-1 & \bf{697.84} & 0.92 & 
698.82 & 0.86 & 697.84 & 0.00
 & 0.14\\SCA3-2 & 664.21 & 0.77 & 
672.39 & 0.75 & \bf{659.34} & 
0.74 & 1.98\\SCA3-3 & 681.16 & 0.92 & 
682.69 & 0.85 & \bf{680.04} & 
0.16 & 0.39\\SCA3-4 & \bf{690.50} & 0.67 & 
691.02 & 0.78 & 690.50 & 0.00
 & 0.08\\SCA3-5 & 665.04 & 0.78 & 
676.22 & 0.85 & \bf{659.90} & 
0.78 & 2.47\\SCA3-6 & 652.94 & 0.75 & 
656.57 & 0.90 & \bf{651.09} & 
0.28 & 0.84\\SCA3-7 & 666.15 & 0.68 & 
667.09 & 0.77 & \bf{659.17} & 
1.06 & 1.20\\SCA3-8 & 719.77 & 0.94 & 
726.15 & 0.87 & \bf{719.47} & 
0.04 & 0.93\\SCA3-9 & \bf{681.00} & 0.90 & 
682.62 & 0.85 & 681.00 & 0.00
 & 0.24\\SCA8-0 & 973.22 & 0.76 & 
998.47 & 0.73 & \bf{961.50} & 
1.22 & 3.85\\SCA8-1 & 1081.37 & 0.79 & 
1094.47 & 0.88 & \bf{1049.65} & 
3.02 & 4.27\\SCA8-2 & 1054.85 & 0.96 & 
1055.36 & 0.95 & \bf{1039.64} & 
1.46 & 1.51\\SCA8-3 & 1026.74 & 0.79 & 
1028.92 & 0.83 & \bf{983.34} & 
4.41 & 4.64\\SCA8-4 & 1073.96 & 0.91 & 
1082.52 & 0.80 & \bf{1065.49} & 
0.79 & 1.60\\SCA8-5 & 1054.86 & 0.70 & 
1054.86 & 0.91 & \bf{1027.08} & 
2.70 & 2.70\\SCA8-6 & 983.46 & 0.78 & 
983.46 & 0.86 & \bf{971.82} & 
1.20 & 1.20\\SCA8-7 & 1071.89 & 0.84 & 
1071.89 & 0.81 & \bf{1051.28} & 
1.96 & 1.96\\SCA8-8 & 1090.51 & 0.73 & 
1091.05 & 0.81 & \bf{1071.18} & 
1.80 & 1.85\\SCA8-9 & 1074.95 & 0.94 & 
1080.97 & 0.82 & \bf{1060.50} & 
1.36 & 1.93\\CON3-0 & 624.91 & 0.72 & 
624.95 & 0.87 & \bf{616.52} & 
1.36 & 1.37\\CON3-1 & 560.75 & 0.93 & 
562.67 & 0.84 & \bf{554.47} & 
1.13 & 1.48\\CON3-2 & 521.38 & 0.80 & 
523.82 & 0.85 & \bf{518.00} & 
0.65 & 1.12\\CON3-3 & 594.11 & 0.71 & 
602.78 & 0.88 & \bf{591.19} & 
0.49 & 1.96\\CON3-4 & 592.58 & 0.92 & 
595.45 & 0.82 & \bf{588.79} & 
0.64 & 1.13\\CON3-5 & 564.89 & 0.79 & 
564.89 & 0.83 & \bf{563.70} & 
0.21 & 0.21\\CON3-6 & 506.15 & 0.75 & 
507.67 & 1.06 & \bf{499.05} & 
1.42 & 1.73\\CON3-7 & 578.22 & 0.74 & 
579.20 & 0.82 & \bf{576.48} & 
0.30 & 0.47\\CON3-8 & 523.14 & 0.94 & 
527.87 & 0.91 & \bf{523.05} & 
0.02 & 0.92\\CON3-9 & 588.11 & 0.75 & 
588.16 & 0.85 & \bf{578.24} & 
1.71 & 1.72\\CON8-0 & 874.20 & 0.98 & 
887.94 & 0.79 & \bf{857.17} & 
1.99 & 3.59\\CON8-1 & 768.17 & 0.96 & 
770.97 & 1.02 & \bf{740.85} & 
3.69 & 4.06\\CON8-2 & 731.26 & 0.83 & 
732.23 & 0.86 & \bf{712.89} & 
2.58 & 2.71\\CON8-3 & 824.71 & 1.02 & 
834.68 & 0.87 & \bf{811.07} & 
1.68 & 2.91\\CON8-4 & 772.76 & 0.81 & 
772.76 & 0.80 & \bf{772.25} & 
0.07 & 0.07\\CON8-5 & 762.01 & 0.72 & 
762.01 & 0.80 & \bf{754.88} & 
0.94 & 0.94\\CON8-6 & 701.80 & 0.78 & 
701.80 & 0.89 & \bf{678.92} & 
3.37 & 3.37\\CON8-7 & 814.50 & 0.75 & 
814.50 & 0.81 & \bf{811.96} & 
0.31 & 0.31\\CON8-8 & 794.05 & 0.90 & 
794.05 & 0.86 & \bf{767.53} & 
3.46 & 3.46\\CON8-9 & 844.84 & 0.98 & 
844.84 & 0.92 & \bf{809.00} & 
4.43 & 4.43\\\bf{PROM.} & 
\bf{769.69} & \bf{0.83} & \bf{773.24} & \bf{0.85} & \bf{758.54} & \bf{1.36} & \bf{1.81}\\[1ex]\hline
\end{tabular}
\label{table:nonlin}
\end{table} \clearpage
\begin{table}[ht]
\caption{Resultados de la ejecución de la metaheurística IGA, utilizando instancias de SalhiNagy con la configuración -n 200 -p 40 -cprob 100.0 -mprob 50.0}
\centering
\small
\begin{tabular}{c c c c c c c c}
\hline\hline
Instancia & Costo mínimo & Tiempo(seg.) & Costo promedio & Tiempo promedio(seg.) & CME & \%G & \%GP \\ [0.5ex]
\hline
CMT1X & 480.02 & 0.62 & 
481.45 & 0.65 & \bf{470.48} & 
2.03 & 2.33\\CMT1Y & \bf{470.48} & 0.80 & 
482.69 & 0.80 & 470.48 & 0.00
 & 2.60\\CMT2X & 711.52 & 1.44 & 
714.77 & 1.54 & \bf{682.39} & 
4.27 & 4.75\\CMT2Y & 709.18 & 1.67 & 
716.77 & 1.56 & \bf{682.39} & 
3.93 & 5.04\\CMT3X & 746.53 & 3.45 & 
748.81 & 3.36 & \bf{719.06} & 
3.82 & 4.14\\CMT3Y & 738.27 & 3.12 & 
743.09 & 3.25 & \bf{719.06} & 
2.67 & 3.34\\CMT4X & 904.17 & 8.77 & 
910.11 & 8.45 & \bf{854.21} & 
5.85 & 6.54\\CMT4Y & 891.74 & 8.91 & 
907.75 & 8.69 & \bf{852.46} & 
4.61 & 6.49\\CMT5X & 1088.05 & 17.10 & 
1112.71 & 16.77 & \bf{1030.56} & 
5.58 & 7.97\\CMT5Y & 1097.46 & 17.74 & 
1110.38 & 17.23 & \bf{1031.69} & 
6.37 & 7.63\\CMT11X & 909.40 & 5.04 & 
914.28 & 5.32 & \bf{831.09} & 
9.42 & 10.01\\CMT11Y & 884.04 & 5.43 & 
899.93 & 5.81 & \bf{829.85} & 
6.53 & 8.44\\CMT12X & 678.91 & 3.42 & 
686.64 & 3.28 & \bf{658.83} & 
3.05 & 4.22\\CMT12Y & 676.88 & 3.39 & 
682.61 & 3.06 & \bf{660.47} & 
2.48 & 3.35\\\bf{PROM.} & 
\bf{784.76} & \bf{5.78} & \bf{793.71} & \bf{5.70} & \bf{749.50} & \bf{4.33} & \bf{5.49}\\[1ex]\hline
\end{tabular}
\label{table:nonlin}
\end{table} \clearpage
\begin{table}[ht]
\caption{Resultados de la ejecución de la metaheurística IGA, utilizando instancias de Dethloff con la configuración -n 200 -p 40 -cprob 100.0 -mprob 60.0}
\centering
\small
\begin{tabular}{c c c c c c c c}
\hline\hline
Instancia & Costo mínimo & Tiempo(seg.) & Costo promedio & Tiempo promedio(seg.) & CME & \%G & \%GP \\ [0.5ex]
\hline
SCA3-0 & 640.55 & 0.76 & 
641.71 & 0.79 & \bf{635.62} & 
0.78 & 0.96\\SCA3-1 & 700.50 & 0.89 & 
700.50 & 0.80 & \bf{697.84} & 
0.38 & 0.38\\SCA3-2 & 661.13 & 0.88 & 
665.62 & 0.83 & \bf{659.34} & 
0.27 & 0.95\\SCA3-3 & 680.60 & 0.90 & 
680.60 & 0.81 & \bf{680.04} & 
0.08 & 0.08\\SCA3-4 & \bf{690.50} & 0.82 & 
690.50 & 0.73 & 690.50 & 0.00
 & 0.00\\
SCA3-5 & 665.64 & 0.76 & 
665.64 & 0.73 & \bf{659.90} & 
0.87 & 0.87\\SCA3-6 & 652.94 & 0.91 & 
654.01 & 0.82 & \bf{651.09} & 
0.28 & 0.45\\SCA3-7 & 666.15 & 0.90 & 
666.15 & 0.79 & \bf{659.17} & 
1.06 & 1.06\\SCA3-8 & \bf{719.47} & 0.88 & 
721.21 & 0.88 & 719.47 & 0.00
 & 0.24\\SCA3-9 & 681.68 & 0.91 & 
682.54 & 0.90 & \bf{681.00} & 
0.10 & 0.23\\SCA8-0 & 1005.05 & 0.72 & 
1016.00 & 0.86 & \bf{961.50} & 
4.53 & 5.67\\SCA8-1 & 1073.37 & 0.74 & 
1079.66 & 0.80 & \bf{1049.65} & 
2.26 & 2.86\\SCA8-2 & 1050.37 & 0.99 & 
1050.37 & 0.94 & \bf{1039.64} & 
1.03 & 1.03\\SCA8-3 & 1025.85 & 0.74 & 
1027.49 & 0.89 & \bf{983.34} & 
4.32 & 4.49\\SCA8-4 & 1098.85 & 0.69 & 
1098.85 & 0.83 & \bf{1065.49} & 
3.13 & 3.13\\SCA8-5 & 1029.95 & 0.77 & 
1038.52 & 0.81 & \bf{1027.08} & 
0.28 & 1.11\\SCA8-6 & 989.44 & 0.70 & 
990.00 & 0.78 & \bf{971.82} & 
1.81 & 1.87\\SCA8-7 & 1074.24 & 0.72 & 
1074.24 & 0.74 & \bf{1051.28} & 
2.18 & 2.18\\SCA8-8 & 1093.62 & 0.93 & 
1093.62 & 0.91 & \bf{1071.18} & 
2.09 & 2.09\\SCA8-9 & 1067.27 & 0.92 & 
1075.49 & 0.88 & \bf{1060.50} & 
0.64 & 1.41\\CON3-0 & 620.76 & 0.76 & 
630.79 & 0.84 & \bf{616.52} & 
0.69 & 2.31\\CON3-1 & 560.75 & 0.72 & 
562.26 & 0.83 & \bf{554.47} & 
1.13 & 1.40\\CON3-2 & 521.38 & 0.86 & 
523.79 & 0.89 & \bf{518.00} & 
0.65 & 1.12\\CON3-3 & 592.43 & 0.92 & 
592.43 & 0.92 & \bf{591.19} & 
0.21 & 0.21\\CON3-4 & 599.13 & 0.92 & 
602.87 & 0.88 & \bf{588.79} & 
1.76 & 2.39\\CON3-5 & 564.88 & 0.96 & 
564.88 & 0.90 & \bf{563.70} & 
0.21 & 0.21\\CON3-6 & 502.88 & 0.84 & 
505.48 & 0.85 & \bf{499.05} & 
0.77 & 1.29\\CON3-7 & 581.83 & 0.78 & 
584.34 & 0.73 & \bf{576.48} & 
0.93 & 1.36\\CON3-8 & 523.60 & 0.72 & 
529.78 & 0.84 & \bf{523.05} & 
0.11 & 1.29\\CON3-9 & 582.79 & 0.96 & 
587.19 & 0.93 & \bf{578.24} & 
0.79 & 1.55\\CON8-0 & 866.81 & 0.94 & 
866.81 & 0.84 & \bf{857.17} & 
1.12 & 1.12\\CON8-1 & 764.88 & 1.00 & 
764.88 & 0.91 & \bf{740.85} & 
3.24 & 3.24\\CON8-2 & 726.99 & 0.79 & 
727.49 & 0.91 & \bf{712.89} & 
1.98 & 2.05\\CON8-3 & 834.42 & 0.86 & 
834.42 & 0.84 & \bf{811.07} & 
2.88 & 2.88\\CON8-4 & 789.34 & 0.92 & 
790.40 & 0.88 & \bf{772.25} & 
2.21 & 2.35\\CON8-5 & 758.12 & 0.95 & 
767.62 & 0.95 & \bf{754.88} & 
0.43 & 1.69\\CON8-6 & 706.74 & 0.76 & 
707.00 & 0.95 & \bf{678.92} & 
4.10 & 4.14\\CON8-7 & 826.80 & 0.73 & 
827.27 & 0.82 & \bf{811.96} & 
1.83 & 1.89\\CON8-8 & 796.21 & 1.00 & 
796.59 & 0.97 & \bf{767.53} & 
3.74 & 3.79\\CON8-9 & 833.92 & 0.74 & 
835.63 & 0.94 & \bf{809.00} & 
3.08 & 3.29\\\bf{PROM.} & 
\bf{770.55} & \bf{0.84} & \bf{772.87} & \bf{0.85} & \bf{758.54} & \bf{1.45} & \bf{1.77}\\[1ex]\hline
\end{tabular}
\label{table:nonlin}
\end{table} \clearpage
\begin{table}[ht]
\caption{Resultados de la ejecución de la metaheurística IGA, utilizando instancias de SalhiNagy con la configuración -n 200 -p 40 -cprob 100.0 -mprob 60.0}
\centering
\small
\begin{tabular}{c c c c c c c c}
\hline\hline
Instancia & Costo mínimo & Tiempo(seg.) & Costo promedio & Tiempo promedio(seg.) & CME & \%G & \%GP \\ [0.5ex]
\hline
CMT1X & 483.89 & 0.78 & 
488.05 & 0.66 & \bf{470.48} & 
2.85 & 3.73\\CMT1Y & 478.97 & 0.61 & 
479.43 & 0.66 & \bf{470.48} & 
1.80 & 1.90\\CMT2X & 703.97 & 1.75 & 
716.07 & 1.90 & \bf{682.39} & 
3.16 & 4.94\\CMT2Y & 712.22 & 1.66 & 
721.57 & 1.62 & \bf{682.39} & 
4.37 & 5.74\\CMT3X & 745.55 & 3.34 & 
749.12 & 3.37 & \bf{719.06} & 
3.68 & 4.18\\CMT3Y & 741.75 & 2.97 & 
753.00 & 3.14 & \bf{719.06} & 
3.16 & 4.72\\CMT4X & 878.29 & 9.12 & 
905.72 & 8.76 & \bf{854.21} & 
2.82 & 6.03\\CMT4Y & 912.52 & 8.78 & 
921.28 & 8.54 & \bf{852.46} & 
7.05 & 8.07\\CMT5X & 1103.13 & 16.90 & 
1121.42 & 16.68 & \bf{1030.56} & 
7.04 & 8.82\\CMT5Y & 1099.14 & 17.94 & 
1113.01 & 17.74 & \bf{1031.69} & 
6.54 & 7.88\\CMT11X & 904.27 & 5.43 & 
915.39 & 6.44 & \bf{831.09} & 
8.81 & 10.14\\CMT11Y & 888.49 & 5.46 & 
909.87 & 5.68 & \bf{829.85} & 
7.07 & 9.64\\CMT12X & 673.17 & 3.44 & 
682.41 & 3.52 & \bf{658.83} & 
2.18 & 3.58\\CMT12Y & 675.07 & 3.52 & 
677.95 & 3.29 & \bf{660.47} & 
2.21 & 2.65\\\bf{PROM.} & 
\bf{785.75} & \bf{5.84} & \bf{796.73} & \bf{5.86} & \bf{749.50} & \bf{4.48} & \bf{5.86}\\[1ex]\hline
\end{tabular}
\label{table:nonlin}
\end{table} \clearpage
\begin{table}[ht]
\caption{Resultados de la ejecución de la metaheurística IGA, utilizando instancias de Dethloff con la configuración -n 200 -p 40 -cprob 100.0 -mprob 70.0}
\centering
\small
\begin{tabular}{c c c c c c c c}
\hline\hline
Instancia & Costo mínimo & Tiempo(seg.) & Costo promedio & Tiempo promedio(seg.) & CME & \%G & \%GP \\ [0.5ex]
\hline
SCA3-0 & 640.55 & 0.89 & 
641.12 & 0.86 & \bf{635.62} & 
0.78 & 0.87\\SCA3-1 & 700.50 & 0.78 & 
705.34 & 0.89 & \bf{697.84} & 
0.38 & 1.07\\SCA3-2 & 664.39 & 0.88 & 
667.80 & 0.83 & \bf{659.34} & 
0.77 & 1.28\\SCA3-3 & \bf{680.04} & 0.92 & 
680.04 & 0.87 & 680.04 & 0.00
 & 0.00\\
SCA3-4 & \bf{690.50} & 0.91 & 
691.02 & 0.82 & 690.50 & 0.00
 & 0.08\\SCA3-5 & 666.67 & 0.89 & 
668.26 & 0.82 & \bf{659.90} & 
1.03 & 1.27\\SCA3-6 & 652.94 & 0.74 & 
652.94 & 0.85 & \bf{651.09} & 
0.28 & 0.28\\SCA3-7 & \bf{659.17} & 0.86 & 
667.21 & 0.83 & 659.17 & 0.00
 & 1.22\\SCA3-8 & \bf{719.47} & 0.92 & 
723.09 & 0.94 & 719.47 & 0.00
 & 0.50\\SCA3-9 & \bf{681.00} & 0.73 & 
686.03 & 0.82 & 681.00 & 0.00
 & 0.74\\SCA8-0 & 982.18 & 1.00 & 
991.24 & 0.94 & \bf{961.50} & 
2.15 & 3.09\\SCA8-1 & 1063.12 & 0.96 & 
1076.63 & 0.90 & \bf{1049.65} & 
1.28 & 2.57\\SCA8-2 & 1050.37 & 0.97 & 
1050.37 & 0.97 & \bf{1039.64} & 
1.03 & 1.03\\SCA8-3 & 1022.53 & 0.98 & 
1022.54 & 0.94 & \bf{983.34} & 
3.99 & 3.99\\SCA8-4 & 1077.80 & 0.97 & 
1081.98 & 0.88 & \bf{1065.49} & 
1.16 & 1.55\\SCA8-5 & 1038.93 & 0.91 & 
1039.13 & 0.89 & \bf{1027.08} & 
1.15 & 1.17\\SCA8-6 & 979.29 & 0.96 & 
993.20 & 0.96 & \bf{971.82} & 
0.77 & 2.20\\SCA8-7 & 1067.11 & 0.95 & 
1069.58 & 0.93 & \bf{1051.28} & 
1.51 & 1.74\\SCA8-8 & 1084.41 & 0.74 & 
1084.41 & 0.85 & \bf{1071.18} & 
1.24 & 1.24\\SCA8-9 & 1072.10 & 0.94 & 
1085.30 & 0.91 & \bf{1060.50} & 
1.09 & 2.34\\CON3-0 & 620.76 & 0.77 & 
625.39 & 0.89 & \bf{616.52} & 
0.69 & 1.44\\CON3-1 & 559.25 & 0.93 & 
560.47 & 0.93 & \bf{554.47} & 
0.86 & 1.08\\CON3-2 & 521.63 & 0.87 & 
522.45 & 0.88 & \bf{518.00} & 
0.70 & 0.86\\CON3-3 & 594.48 & 0.93 & 
603.33 & 0.86 & \bf{591.19} & 
0.56 & 2.05\\CON3-4 & 592.58 & 0.74 & 
592.58 & 0.83 & \bf{588.79} & 
0.64 & 0.64\\CON3-5 & 567.94 & 0.72 & 
569.39 & 0.84 & \bf{563.70} & 
0.75 & 1.01\\CON3-6 & 504.44 & 0.98 & 
506.37 & 0.97 & \bf{499.05} & 
1.08 & 1.47\\CON3-7 & 582.14 & 0.94 & 
582.14 & 0.89 & \bf{576.48} & 
0.98 & 0.98\\CON3-8 & 523.14 & 0.81 & 
527.14 & 0.91 & \bf{523.05} & 
0.02 & 0.78\\CON3-9 & 588.11 & 0.96 & 
589.27 & 0.95 & \bf{578.24} & 
1.71 & 1.91\\CON8-0 & 878.42 & 1.04 & 
879.79 & 1.02 & \bf{857.17} & 
2.48 & 2.64\\CON8-1 & 757.04 & 1.00 & 
765.59 & 0.94 & \bf{740.85} & 
2.19 & 3.34\\CON8-2 & 713.60 & 0.95 & 
713.60 & 0.93 & \bf{712.89} & 
0.10 & 0.10\\CON8-3 & 843.44 & 1.03 & 
843.78 & 0.99 & \bf{811.07} & 
3.99 & 4.03\\CON8-4 & 778.26 & 0.95 & 
779.89 & 0.88 & \bf{772.25} & 
0.78 & 0.99\\CON8-5 & 764.31 & 0.80 & 
771.39 & 0.91 & \bf{754.88} & 
1.25 & 2.19\\CON8-6 & 706.66 & 1.01 & 
708.64 & 0.93 & \bf{678.92} & 
4.09 & 4.38\\CON8-7 & 827.46 & 0.80 & 
827.46 & 0.93 & \bf{811.96} & 
1.91 & 1.91\\CON8-8 & 799.26 & 0.92 & 
801.23 & 0.86 & \bf{767.53} & 
4.13 & 4.39\\CON8-9 & 827.20 & 1.03 & 
832.12 & 0.98 & \bf{809.00} & 
2.25 & 2.86\\\bf{PROM.} & 
\bf{768.58} & \bf{0.90} & \bf{771.98} & \bf{0.90} & \bf{758.54} & \bf{1.24} & \bf{1.68}\\[1ex]\hline
\end{tabular}
\label{table:nonlin}
\end{table} \clearpage
\begin{table}[ht]
\caption{Resultados de la ejecución de la metaheurística IGA, utilizando instancias de SalhiNagy con la configuración -n 200 -p 40 -cprob 100.0 -mprob 70.0}
\centering
\small
\begin{tabular}{c c c c c c c c}
\hline\hline
Instancia & Costo mínimo & Tiempo(seg.) & Costo promedio & Tiempo promedio(seg.) & CME & \%G & \%GP \\ [0.5ex]
\hline
CMT1X & 480.72 & 0.64 & 
481.65 & 0.76 & \bf{470.48} & 
2.18 & 2.37\\CMT1Y & 477.39 & 0.85 & 
481.70 & 0.75 & \bf{470.48} & 
1.47 & 2.38\\CMT2X & 713.34 & 1.51 & 
720.79 & 1.63 & \bf{682.39} & 
4.54 & 5.63\\CMT2Y & 700.40 & 1.61 & 
712.84 & 1.55 & \bf{682.39} & 
2.64 & 4.46\\CMT3X & 734.74 & 3.42 & 
745.89 & 3.21 & \bf{719.06} & 
2.18 & 3.73\\CMT3Y & 750.63 & 3.02 & 
755.97 & 3.02 & \bf{719.06} & 
4.39 & 5.13\\CMT4X & 910.72 & 8.67 & 
915.71 & 8.82 & \bf{854.21} & 
6.62 & 7.20\\CMT4Y & 894.00 & 8.89 & 
908.82 & 8.82 & \bf{852.46} & 
4.87 & 6.61\\CMT5X & 1103.38 & 17.35 & 
1114.19 & 17.39 & \bf{1030.56} & 
7.07 & 8.12\\CMT5Y & 1078.43 & 17.16 & 
1119.18 & 17.05 & \bf{1031.69} & 
4.53 & 8.48\\CMT11X & 908.20 & 5.53 & 
917.28 & 5.39 & \bf{831.09} & 
9.28 & 10.37\\CMT11Y & 898.82 & 5.88 & 
916.05 & 5.89 & \bf{829.85} & 
8.31 & 10.39\\CMT12X & 675.44 & 3.50 & 
680.84 & 3.50 & \bf{658.83} & 
2.52 & 3.34\\CMT12Y & 673.48 & 3.48 & 
675.77 & 3.43 & \bf{660.47} & 
1.97 & 2.32\\\bf{PROM.} & 
\bf{785.69} & \bf{5.82} & \bf{796.19} & \bf{5.80} & \bf{749.50} & \bf{4.47} & \bf{5.75}\\[1ex]\hline
\end{tabular}
\label{table:nonlin}
\end{table} \clearpage
\begin{table}[ht]
\caption{Resultados de la ejecución de la metaheurística IGA, utilizando instancias de Dethloff con la configuración -n 200 -p 40 -cprob 100.0 -mprob 80.0}
\centering
\small
\begin{tabular}{c c c c c c c c}
\hline\hline
Instancia & Costo mínimo & Tiempo(seg.) & Costo promedio & Tiempo promedio(seg.) & CME & \%G & \%GP \\ [0.5ex]
\hline
SCA3-0 & 640.55 & 0.78 & 
640.55 & 0.88 & \bf{635.62} & 
0.78 & 0.78\\SCA3-1 & \bf{697.84} & 0.93 & 
699.17 & 0.94 & 697.84 & 0.00
 & 0.19\\SCA3-2 & 661.13 & 0.72 & 
669.78 & 0.78 & \bf{659.34} & 
0.27 & 1.58\\SCA3-3 & 681.16 & 0.93 & 
681.16 & 0.86 & \bf{680.04} & 
0.16 & 0.16\\SCA3-4 & \bf{690.50} & 0.90 & 
690.50 & 0.91 & 690.50 & 0.00
 & 0.00\\
SCA3-5 & 668.48 & 0.72 & 
675.22 & 0.77 & \bf{659.90} & 
1.30 & 2.32\\SCA3-6 & 652.94 & 0.71 & 
652.94 & 0.82 & \bf{651.09} & 
0.28 & 0.28\\SCA3-7 & 666.15 & 0.69 & 
666.15 & 0.79 & \bf{659.17} & 
1.06 & 1.06\\SCA3-8 & 719.77 & 0.93 & 
723.15 & 0.86 & \bf{719.47} & 
0.04 & 0.51\\SCA3-9 & \bf{681.00} & 0.94 & 
683.69 & 0.94 & 681.00 & 0.00
 & 0.39\\SCA8-0 & 994.56 & 0.72 & 
1001.23 & 0.85 & \bf{961.50} & 
3.44 & 4.13\\SCA8-1 & 1073.28 & 1.13 & 
1076.63 & 0.95 & \bf{1049.65} & 
2.25 & 2.57\\SCA8-2 & 1053.99 & 0.97 & 
1053.99 & 0.97 & \bf{1039.64} & 
1.38 & 1.38\\SCA8-3 & 1009.82 & 0.98 & 
1009.82 & 0.91 & \bf{983.34} & 
2.69 & 2.69\\SCA8-4 & 1074.11 & 0.83 & 
1083.30 & 0.88 & \bf{1065.49} & 
0.81 & 1.67\\SCA8-5 & 1053.42 & 1.01 & 
1053.42 & 0.94 & \bf{1027.08} & 
2.56 & 2.56\\SCA8-6 & 977.87 & 0.98 & 
977.87 & 0.98 & \bf{971.82} & 
0.62 & 0.62\\SCA8-7 & 1084.50 & 0.96 & 
1084.50 & 0.95 & \bf{1051.28} & 
3.16 & 3.16\\SCA8-8 & 1089.91 & 0.76 & 
1089.91 & 0.88 & \bf{1071.18} & 
1.75 & 1.75\\SCA8-9 & 1075.22 & 0.87 & 
1075.22 & 0.92 & \bf{1060.50} & 
1.39 & 1.39\\CON3-0 & 620.76 & 0.94 & 
623.08 & 0.83 & \bf{616.52} & 
0.69 & 1.06\\CON3-1 & 558.16 & 0.90 & 
561.51 & 0.85 & \bf{554.47} & 
0.67 & 1.27\\CON3-2 & 521.38 & 0.97 & 
521.38 & 1.15 & \bf{518.00} & 
0.65 & 0.65\\CON3-3 & 609.68 & 0.89 & 
609.68 & 0.84 & \bf{591.19} & 
3.13 & 3.13\\CON3-4 & 593.78 & 0.91 & 
595.95 & 0.92 & \bf{588.79} & 
0.85 & 1.22\\CON3-5 & 564.88 & 0.73 & 
567.13 & 0.81 & \bf{563.70} & 
0.21 & 0.61\\CON3-6 & 505.97 & 0.80 & 
507.35 & 0.84 & \bf{499.05} & 
1.39 & 1.66\\CON3-7 & 586.01 & 0.90 & 
589.21 & 0.80 & \bf{576.48} & 
1.65 & 2.21\\CON3-8 & 523.14 & 0.93 & 
533.02 & 0.87 & \bf{523.05} & 
0.02 & 1.91\\CON3-9 & 590.48 & 0.98 & 
590.48 & 0.98 & \bf{578.24} & 
2.12 & 2.12\\CON8-0 & 886.45 & 1.04 & 
886.45 & 1.02 & \bf{857.17} & 
3.42 & 3.42\\CON8-1 & 768.15 & 0.98 & 
769.75 & 1.00 & \bf{740.85} & 
3.68 & 3.90\\CON8-2 & 729.46 & 0.94 & 
729.73 & 0.94 & \bf{712.89} & 
2.32 & 2.36\\CON8-3 & 828.04 & 1.02 & 
829.91 & 0.91 & \bf{811.07} & 
2.09 & 2.32\\CON8-4 & 783.52 & 0.97 & 
783.52 & 0.95 & \bf{772.25} & 
1.46 & 1.46\\CON8-5 & 758.12 & 0.95 & 
759.84 & 0.97 & \bf{754.88} & 
0.43 & 0.66\\CON8-6 & 699.21 & 0.98 & 
709.28 & 0.92 & \bf{678.92} & 
2.99 & 4.47\\CON8-7 & 825.23 & 0.97 & 
825.70 & 0.95 & \bf{811.96} & 
1.63 & 1.69\\CON8-8 & 784.66 & 0.75 & 
789.26 & 0.90 & \bf{767.53} & 
2.23 & 2.83\\CON8-9 & 813.25 & 0.98 & 
814.24 & 0.91 & \bf{809.00} & 
0.53 & 0.65\\\bf{PROM.} & 
\bf{769.91} & \bf{0.90} & \bf{772.12} & \bf{0.90} & \bf{758.54} & \bf{1.40} & \bf{1.72}\\[1ex]\hline
\end{tabular}
\label{table:nonlin}
\end{table} \clearpage
\begin{table}[ht]
\caption{Resultados de la ejecución de la metaheurística IGA, utilizando instancias de SalhiNagy con la configuración -n 200 -p 40 -cprob 100.0 -mprob 80.0}
\centering
\small
\begin{tabular}{c c c c c c c c}
\hline\hline
Instancia & Costo mínimo & Tiempo(seg.) & Costo promedio & Tiempo promedio(seg.) & CME & \%G & \%GP \\ [0.5ex]
\hline
CMT1X & 477.02 & 0.83 & 
481.05 & 0.80 & \bf{470.48} & 
1.39 & 2.25\\CMT1Y & 484.22 & 0.83 & 
486.41 & 0.84 & \bf{470.48} & 
2.92 & 3.39\\CMT2X & 709.86 & 1.45 & 
713.01 & 1.55 & \bf{682.39} & 
4.03 & 4.49\\CMT2Y & 705.22 & 1.55 & 
717.52 & 1.59 & \bf{682.39} & 
3.35 & 5.15\\CMT3X & 737.89 & 3.44 & 
744.93 & 3.48 & \bf{719.06} & 
2.62 & 3.60\\CMT3Y & 741.35 & 3.00 & 
747.50 & 3.32 & \bf{719.06} & 
3.10 & 3.96\\CMT4X & 892.62 & 8.94 & 
911.53 & 8.75 & \bf{854.21} & 
4.50 & 6.71\\CMT4Y & 914.56 & 8.96 & 
924.36 & 8.59 & \bf{852.46} & 
7.28 & 8.43\\CMT5X & 1101.67 & 17.71 & 
1112.20 & 17.44 & \bf{1030.56} & 
6.90 & 7.92\\CMT5Y & 1106.53 & 18.90 & 
1124.33 & 18.07 & \bf{1031.69} & 
7.25 & 8.98\\CMT11X & 902.52 & 5.47 & 
925.32 & 5.45 & \bf{831.09} & 
8.59 & 11.34\\CMT11Y & 893.21 & 6.09 & 
907.72 & 6.17 & \bf{829.85} & 
7.64 & 9.38\\CMT12X & 678.03 & 3.34 & 
682.27 & 3.48 & \bf{658.83} & 
2.91 & 3.56\\CMT12Y & 674.39 & 3.38 & 
680.00 & 3.38 & \bf{660.47} & 
2.11 & 2.96\\\bf{PROM.} & 
\bf{787.08} & \bf{5.99} & \bf{797.01} & \bf{5.92} & \bf{749.50} & \bf{4.61} & \bf{5.86}\\[1ex]\hline
\end{tabular}
\label{table:nonlin}
\end{table} \clearpage
\begin{table}[ht]
\caption{Resultados de la ejecución de la metaheurística IGA, utilizando instancias de Dethloff con la configuración -n 200 -p 40 -cprob 100.0 -mprob 90.0}
\centering
\small
\begin{tabular}{c c c c c c c c}
\hline\hline
Instancia & Costo mínimo & Tiempo(seg.) & Costo promedio & Tiempo promedio(seg.) & CME & \%G & \%GP \\ [0.5ex]
\hline
SCA3-0 & 640.55 & 0.94 & 
640.55 & 0.94 & \bf{635.62} & 
0.78 & 0.78\\SCA3-1 & 701.53 & 0.92 & 
703.85 & 0.93 & \bf{697.84} & 
0.53 & 0.86\\SCA3-2 & 676.40 & 0.88 & 
677.22 & 0.89 & \bf{659.34} & 
2.59 & 2.71\\SCA3-3 & 681.35 & 0.92 & 
682.00 & 0.92 & \bf{680.04} & 
0.19 & 0.29\\SCA3-4 & \bf{690.50} & 0.90 & 
692.55 & 0.90 & 690.50 & 0.00
 & 0.30\\SCA3-5 & \bf{659.90} & 0.97 & 
659.90 & 0.93 & 659.90 & 0.00
 & 0.00\\
SCA3-6 & 652.94 & 0.94 & 
655.92 & 0.93 & \bf{651.09} & 
0.28 & 0.74\\SCA3-7 & 666.15 & 0.92 & 
666.15 & 0.85 & \bf{659.17} & 
1.06 & 1.06\\SCA3-8 & \bf{719.47} & 0.91 & 
727.38 & 0.92 & 719.47 & 0.00
 & 1.10\\SCA3-9 & 685.88 & 0.74 & 
686.31 & 0.83 & \bf{681.00} & 
0.72 & 0.78\\SCA8-0 & 982.79 & 0.96 & 
982.79 & 0.97 & \bf{961.50} & 
2.21 & 2.21\\SCA8-1 & 1089.72 & 1.00 & 
1092.95 & 0.97 & \bf{1049.65} & 
3.82 & 4.13\\SCA8-2 & 1050.37 & 0.95 & 
1050.37 & 0.97 & \bf{1039.64} & 
1.03 & 1.03\\SCA8-3 & 1019.13 & 1.04 & 
1019.13 & 0.93 & \bf{983.34} & 
3.64 & 3.64\\SCA8-4 & 1071.86 & 0.93 & 
1073.58 & 0.83 & \bf{1065.49} & 
0.60 & 0.76\\SCA8-5 & 1067.39 & 0.86 & 
1067.39 & 0.95 & \bf{1027.08} & 
3.92 & 3.92\\SCA8-6 & 992.25 & 0.98 & 
992.25 & 0.97 & \bf{971.82} & 
2.10 & 2.10\\SCA8-7 & 1063.22 & 0.96 & 
1068.25 & 0.95 & \bf{1051.28} & 
1.14 & 1.61\\SCA8-8 & 1092.41 & 0.98 & 
1092.61 & 0.95 & \bf{1071.18} & 
1.98 & 2.00\\SCA8-9 & 1078.30 & 0.96 & 
1078.30 & 0.94 & \bf{1060.50} & 
1.68 & 1.68\\CON3-0 & 619.09 & 0.92 & 
622.63 & 0.96 & \bf{616.52} & 
0.42 & 0.99\\CON3-1 & 560.75 & 0.98 & 
561.03 & 0.96 & \bf{554.47} & 
1.13 & 1.18\\CON3-2 & 521.38 & 1.01 & 
521.38 & 0.99 & \bf{518.00} & 
0.65 & 0.65\\CON3-3 & 592.41 & 1.29 & 
597.11 & 1.02 & \bf{591.19} & 
0.21 & 1.00\\CON3-4 & 595.00 & 0.71 & 
595.00 & 0.86 & \bf{588.79} & 
1.05 & 1.05\\CON3-5 & 564.88 & 0.92 & 
565.92 & 0.86 & \bf{563.70} & 
0.21 & 0.39\\CON3-6 & 505.14 & 0.77 & 
506.23 & 0.86 & \bf{499.05} & 
1.22 & 1.44\\CON3-7 & 588.44 & 0.91 & 
589.66 & 0.92 & \bf{576.48} & 
2.07 & 2.29\\CON3-8 & 524.59 & 0.95 & 
524.59 & 0.95 & \bf{523.05} & 
0.29 & 0.29\\CON3-9 & 589.00 & 0.97 & 
590.46 & 0.97 & \bf{578.24} & 
1.86 & 2.11\\CON8-0 & 876.08 & 1.03 & 
876.08 & 0.92 & \bf{857.17} & 
2.21 & 2.21\\CON8-1 & 765.70 & 0.97 & 
765.70 & 0.97 & \bf{740.85} & 
3.35 & 3.35\\CON8-2 & 716.07 & 1.04 & 
720.46 & 1.06 & \bf{712.89} & 
0.45 & 1.06\\CON8-3 & 827.20 & 0.97 & 
829.88 & 0.96 & \bf{811.07} & 
1.99 & 2.32\\CON8-4 & 775.70 & 0.94 & 
775.72 & 0.94 & \bf{772.25} & 
0.45 & 0.45\\CON8-5 & 763.31 & 1.02 & 
765.12 & 1.01 & \bf{754.88} & 
1.12 & 1.36\\CON8-6 & 696.32 & 1.00 & 
696.83 & 0.94 & \bf{678.92} & 
2.56 & 2.64\\CON8-7 & 816.00 & 0.96 & 
816.00 & 0.89 & \bf{811.96} & 
0.50 & 0.50\\CON8-8 & 797.99 & 0.96 & 
803.71 & 0.97 & \bf{767.53} & 
3.97 & 4.71\\CON8-9 & 831.15 & 1.02 & 
840.75 & 0.96 & \bf{809.00} & 
2.74 & 3.92\\\bf{PROM.} & 
\bf{770.21} & \bf{0.95} & \bf{771.84} & \bf{0.94} & \bf{758.54} & \bf{1.42} & \bf{1.64}\\[1ex]\hline
\end{tabular}
\label{table:nonlin}
\end{table} \clearpage
\begin{table}[ht]
\caption{Resultados de la ejecución de la metaheurística IGA, utilizando instancias de SalhiNagy con la configuración -n 200 -p 40 -cprob 100.0 -mprob 90.0}
\centering
\small
\begin{tabular}{c c c c c c c c}
\hline\hline
Instancia & Costo mínimo & Tiempo(seg.) & Costo promedio & Tiempo promedio(seg.) & CME & \%G & \%GP \\ [0.5ex]
\hline
CMT1X & 478.97 & 0.88 & 
483.94 & 0.86 & \bf{470.48} & 
1.80 & 2.86\\CMT1Y & 479.72 & 0.83 & 
485.64 & 0.79 & \bf{470.48} & 
1.96 & 3.22\\CMT2X & 706.65 & 1.76 & 
710.32 & 1.65 & \bf{682.39} & 
3.56 & 4.09\\CMT2Y & 708.99 & 1.64 & 
712.75 & 1.65 & \bf{682.39} & 
3.90 & 4.45\\CMT3X & 732.94 & 3.47 & 
739.71 & 3.49 & \bf{719.06} & 
1.93 & 2.87\\CMT3Y & 739.38 & 3.38 & 
742.35 & 3.41 & \bf{719.06} & 
2.83 & 3.24\\CMT4X & 892.58 & 8.84 & 
899.88 & 8.46 & \bf{854.21} & 
4.49 & 5.35\\CMT4Y & 893.87 & 8.57 & 
912.30 & 8.92 & \bf{852.46} & 
4.86 & 7.02\\CMT5X & 1106.87 & 17.66 & 
1113.98 & 17.61 & \bf{1030.56} & 
7.40 & 8.09\\CMT5Y & 1115.08 & 16.95 & 
1132.18 & 17.15 & \bf{1031.69} & 
8.08 & 9.74\\CMT11X & 904.22 & 5.63 & 
920.05 & 5.53 & \bf{831.09} & 
8.80 & 10.70\\CMT11Y & 892.64 & 6.03 & 
899.86 & 6.00 & \bf{829.85} & 
7.57 & 8.44\\CMT12X & 675.91 & 3.47 & 
677.65 & 3.46 & \bf{658.83} & 
2.59 & 2.86\\CMT12Y & 676.50 & 3.44 & 
680.51 & 3.31 & \bf{660.47} & 
2.43 & 3.03\\\bf{PROM.} & 
\bf{786.02} & \bf{5.90} & \bf{793.65} & \bf{5.88} & \bf{749.50} & \bf{4.44} & \bf{5.43}\\[1ex]\hline
\end{tabular}
\label{table:nonlin}
\end{table} \clearpage
\begin{table}[ht]
\caption{Resultados de la ejecución de la metaheurística IGA, utilizando instancias de Dethloff con la configuración -n 200 -p 40 -cprob 100.0 -mprob 100.0}
\centering
\small
\begin{tabular}{c c c c c c c c}
\hline\hline
Instancia & Costo mínimo & Tiempo(seg.) & Costo promedio & Tiempo promedio(seg.) & CME & \%G & \%GP \\ [0.5ex]
\hline
SCA3-0 & 641.69 & 0.95 & 
641.69 & 0.94 & \bf{635.62} & 
0.95 & 0.95\\SCA3-1 & 700.50 & 0.90 & 
700.50 & 0.90 & \bf{697.84} & 
0.38 & 0.38\\SCA3-2 & 666.01 & 0.86 & 
674.55 & 0.87 & \bf{659.34} & 
1.01 & 2.31\\SCA3-3 & 681.35 & 0.93 & 
681.54 & 0.91 & \bf{680.04} & 
0.19 & 0.22\\SCA3-4 & \bf{690.50} & 0.89 & 
690.50 & 0.90 & 690.50 & 0.00
 & 0.00\\
SCA3-5 & 665.04 & 1.05 & 
673.11 & 0.98 & \bf{659.90} & 
0.78 & 2.00\\SCA3-6 & 652.94 & 0.92 & 
657.67 & 0.92 & \bf{651.09} & 
0.28 & 1.01\\SCA3-7 & 666.60 & 0.91 & 
667.08 & 0.90 & \bf{659.17} & 
1.13 & 1.20\\SCA3-8 & 719.77 & 0.91 & 
730.31 & 0.92 & \bf{719.47} & 
0.04 & 1.51\\SCA3-9 & \bf{681.00} & 0.88 & 
682.92 & 0.89 & 681.00 & 0.00
 & 0.28\\SCA8-0 & 974.40 & 0.99 & 
992.74 & 1.00 & \bf{961.50} & 
1.34 & 3.25\\SCA8-1 & 1077.87 & 1.00 & 
1081.07 & 0.97 & \bf{1049.65} & 
2.69 & 2.99\\SCA8-2 & 1050.17 & 0.98 & 
1050.17 & 0.99 & \bf{1039.64} & 
1.01 & 1.01\\SCA8-3 & 1011.56 & 0.99 & 
1011.56 & 0.99 & \bf{983.34} & 
2.87 & 2.87\\SCA8-4 & 1110.65 & 0.97 & 
1111.69 & 0.95 & \bf{1065.49} & 
4.24 & 4.34\\SCA8-5 & 1067.90 & 0.99 & 
1067.90 & 0.99 & \bf{1027.08} & 
3.97 & 3.97\\SCA8-6 & 972.48 & 1.00 & 
985.83 & 0.98 & \bf{971.82} & 
0.07 & 1.44\\SCA8-7 & 1077.67 & 0.96 & 
1086.22 & 0.96 & \bf{1051.28} & 
2.51 & 3.32\\SCA8-8 & 1093.54 & 0.98 & 
1095.14 & 0.96 & \bf{1071.18} & 
2.09 & 2.24\\SCA8-9 & 1078.30 & 0.96 & 
1078.30 & 0.95 & \bf{1060.50} & 
1.68 & 1.68\\CON3-0 & 624.96 & 0.94 & 
633.14 & 0.93 & \bf{616.52} & 
1.37 & 2.70\\CON3-1 & 560.75 & 0.95 & 
560.75 & 0.94 & \bf{554.47} & 
1.13 & 1.13\\CON3-2 & 521.38 & 0.98 & 
522.26 & 0.98 & \bf{518.00} & 
0.65 & 0.82\\CON3-3 & \bf{591.19} & 0.92 & 
595.75 & 0.93 & 591.19 & 0.00
 & 0.77\\CON3-4 & 593.78 & 0.91 & 
604.80 & 0.94 & \bf{588.79} & 
0.85 & 2.72\\CON3-5 & 564.89 & 0.93 & 
567.39 & 0.95 & \bf{563.70} & 
0.21 & 0.65\\CON3-6 & 504.91 & 0.99 & 
507.21 & 0.97 & \bf{499.05} & 
1.17 & 1.64\\CON3-7 & 578.41 & 0.97 & 
586.24 & 0.98 & \bf{576.48} & 
0.33 & 1.69\\CON3-8 & 525.30 & 0.93 & 
525.30 & 0.93 & \bf{523.05} & 
0.43 & 0.43\\CON3-9 & 588.48 & 0.96 & 
589.74 & 0.94 & \bf{578.24} & 
1.77 & 1.99\\CON8-0 & 887.16 & 1.02 & 
887.16 & 1.02 & \bf{857.17} & 
3.50 & 3.50\\CON8-1 & 764.63 & 0.93 & 
764.63 & 0.93 & \bf{740.85} & 
3.21 & 3.21\\CON8-2 & 725.53 & 1.06 & 
725.53 & 1.03 & \bf{712.89} & 
1.77 & 1.77\\CON8-3 & 839.13 & 1.01 & 
844.61 & 1.00 & \bf{811.07} & 
3.46 & 4.13\\CON8-4 & 786.25 & 0.96 & 
786.25 & 0.95 & \bf{772.25} & 
1.81 & 1.81\\CON8-5 & 767.86 & 0.96 & 
767.86 & 0.97 & \bf{754.88} & 
1.72 & 1.72\\CON8-6 & 702.68 & 0.99 & 
702.68 & 0.97 & \bf{678.92} & 
3.50 & 3.50\\CON8-7 & 814.79 & 0.92 & 
817.26 & 0.93 & \bf{811.96} & 
0.35 & 0.65\\CON8-8 & 778.22 & 1.05 & 
778.75 & 1.03 & \bf{767.53} & 
1.39 & 1.46\\CON8-9 & 821.76 & 1.07 & 
821.76 & 1.05 & \bf{809.00} & 
1.58 & 1.58\\\bf{PROM.} & 
\bf{770.55} & \bf{0.96} & \bf{773.74} & \bf{0.96} & \bf{758.54} & \bf{1.44} & \bf{1.87}\\[1ex]\hline
\end{tabular}
\label{table:nonlin}
\end{table} \clearpage
\begin{table}[ht]
\caption{Resultados de la ejecución de la metaheurística IGA, utilizando instancias de SalhiNagy con la configuración -n 200 -p 40 -cprob 100.0 -mprob 100.0}
\centering
\small
\begin{tabular}{c c c c c c c c}
\hline\hline
Instancia & Costo mínimo & Tiempo(seg.) & Costo promedio & Tiempo promedio(seg.) & CME & \%G & \%GP \\ [0.5ex]
\hline
CMT1X & 484.49 & 0.84 & 
489.65 & 0.84 & \bf{470.48} & 
2.98 & 4.08\\CMT1Y & 474.41 & 0.80 & 
477.50 & 0.81 & \bf{470.48} & 
0.84 & 1.49\\CMT2X & 717.88 & 1.70 & 
719.81 & 1.66 & \bf{682.39} & 
5.20 & 5.48\\CMT2Y & 703.16 & 1.70 & 
709.45 & 1.70 & \bf{682.39} & 
3.04 & 3.97\\CMT3X & 744.53 & 3.41 & 
750.27 & 3.44 & \bf{719.06} & 
3.54 & 4.34\\CMT3Y & 739.99 & 3.50 & 
745.43 & 3.43 & \bf{719.06} & 
2.91 & 3.67\\CMT4X & 902.65 & 9.04 & 
915.14 & 8.63 & \bf{854.21} & 
5.67 & 7.13\\CMT4Y & 903.84 & 9.04 & 
911.58 & 8.99 & \bf{852.46} & 
6.03 & 6.94\\CMT5X & 1086.63 & 18.44 & 
1116.22 & 17.66 & \bf{1030.56} & 
5.44 & 8.31\\CMT5Y & 1100.88 & 17.65 & 
1106.13 & 17.06 & \bf{1031.69} & 
6.71 & 7.22\\CMT11X & 886.25 & 5.96 & 
909.77 & 5.87 & \bf{831.09} & 
6.64 & 9.47\\CMT11Y & 893.00 & 6.15 & 
922.41 & 6.06 & \bf{829.85} & 
7.61 & 11.15\\CMT12X & 673.48 & 3.49 & 
676.65 & 3.67 & \bf{658.83} & 
2.22 & 2.70\\CMT12Y & 680.23 & 3.43 & 
682.49 & 3.41 & \bf{660.47} & 
2.99 & 3.33\\\bf{PROM.} & 
\bf{785.10} & \bf{6.08} & \bf{795.18} & \bf{5.95} & \bf{749.50} & \bf{4.42} & \bf{5.66}\\[1ex]\hline
\end{tabular}
\label{table:nonlin}
\end{table} \clearpage
\begin{table}[ht]
\caption{Resultados de la ejecución de la metaheurística SCA, utilizando instancias de SalhiNagy con la configuración -n 100.0 -b 10 -y .2}
\centering
\small
\begin{tabular}{c c c c c c c c}
\hline\hline
Instancia & Costo mínimo & Tiempo(seg.) & Costo promedio & Tiempo promedio(seg.) & CME & \%G & \%GP \\ [0.5ex]
\hline
CMT1X & 479.62 & 1.84 & 
479.62 & 1.99 & \bf{470.48} & 
1.94 & 1.94\\CMT1Y & 472.37 & 1.55 & 
472.50 & 1.65 & \bf{470.48} & 
0.40 & 0.43\\CMT2X & 703.02 & 27.99 & 
710.69 & 15.57 & \bf{682.39} & 
3.02 & 4.15\\CMT2Y & 694.46 & 18.62 & 
706.70 & 34.28 & \bf{682.39} & 
1.77 & 3.56\\CMT3X & 728.81 & 40.94 & 
736.51 & 29.89 & \bf{719.06} & 
1.36 & 2.43\\CMT3Y & 737.87 & 41.83 & 
741.43 & 32.77 & \bf{719.06} & 
2.62 & 3.11\\CMT4X & 100000 & 0 & 
904.37 & 0.00 & \bf{854.21} & 
11606.72 & 5.87\\CMT4Y & 900.13 & | & 
0.00 & 0.00 & \bf{852.46} & 
5.59 & -100.00\\CMT5X & 1101.46 & 1431.32 & 
1105.81 & 1107.71 & \bf{1030.56} & 
6.88 & 7.30\\CMT5Y & 1095.77 & 1087.09 & 
1102.08 & 1022.50 & \bf{1031.69} & 
6.21 & 6.82\\CMT11X & 897.22 & 50.06 & 
913.70 & 37.51 & \bf{831.09} & 
7.96 & 9.94\\CMT11Y & 885.78 & 57.51 & 
901.72 & 50.05 & \bf{829.85} & 
6.74 & 8.66\\CMT12X & 678.06 & 74.48 & 
682.73 & 65.76 & \bf{658.83} & 
2.92 & 3.63\\CMT12Y & 674.63 & 76.38 & 
677.29 & 70.56 & \bf{660.47} & 
2.14 & 2.55\\\bf{PROM.} & 
\bf{7860.66} & \bf{207.83} & \bf{723.94} & \bf{176.45} & \bf{749.50} & \bf{832.59} & \bf{-2.83}\\[1ex]\hline
\end{tabular}
\label{table:nonlin}
\end{table} \clearpage
\begin{table}[ht]
\caption{Resultados de la ejecución de la metaheurística SCA, utilizando instancias de SalhiNagy con la configuración -n 100.0 -b 10 -y .3}
\centering
\small
\begin{tabular}{c c c c c c c c}
\hline\hline
Instancia & Costo mínimo & Tiempo(seg.) & Costo promedio & Tiempo promedio(seg.) & CME & \%G & \%GP \\ [0.5ex]
\hline
CMT1X & 482.56 & 1.12 & 
482.68 & 1.16 & \bf{470.48} & 
2.57 & 2.59\\CMT1Y & 472.87 & 2.18 & 
475.04 & 1.94 & \bf{470.48} & 
0.51 & 0.97\\CMT2X & 698.73 & 11.70 & 
706.71 & 18.52 & \bf{682.39} & 
2.39 & 3.56\\CMT2Y & 705.84 & 17.50 & 
709.29 & 19.27 & \bf{682.39} & 
3.44 & 3.94\\CMT3X & 730.85 & 57.23 & 
736.80 & 37.32 & \bf{719.06} & 
1.64 & 2.47\\CMT3Y & 733.11 & 29.64 & 
734.91 & 39.62 & \bf{719.06} & 
1.95 & 2.20\\CMT4X & 896.96 & 331.63 & 
903.90 & 281.04 & \bf{854.21} & 
5.00 & 5.82\\CMT4Y & 901.06 & 286.15 & 
908.24 & 529.61 & \bf{852.46} & 
5.70 & 6.54\\CMT5X & 1104.34 & 1466.47 & 
1120.77 & 1086.28 & \bf{1030.56} & 
7.16 & 8.75\\CMT5Y & 1082.90 & 568.50 & 
1111.64 & 635.47 & \bf{1031.69} & 
4.96 & 7.75\\CMT11X & 886.41 & 48.99 & 
896.59 & 42.59 & \bf{831.09} & 
6.66 & 7.88\\CMT11Y & 887.17 & 27.01 & 
900.69 & 43.23 & \bf{829.85} & 
6.91 & 8.54\\CMT12X & 676.55 & 41.84 & 
679.88 & 64.08 & \bf{658.83} & 
2.69 & 3.19\\CMT12Y & 677.27 & 58.11 & 
682.24 & 74.11 & \bf{660.47} & 
2.54 & 3.30\\\bf{PROM.} & 
\bf{781.19} & \bf{210.58} & \bf{789.24} & \bf{205.30} & \bf{749.50} & \bf{3.87} & \bf{4.82}\\[1ex]\hline
\end{tabular}
\label{table:nonlin}
\end{table} \clearpage
\begin{table}[ht]
\caption{Resultados de la ejecución de la metaheurística PSO, utilizando instancias de SalhiNagy con la configuración -n 10.0 -L 10.0 -cp 1 -cg 0 -cl 1 -cn 2 -w1 0.9 -wt 0.1 -K 5}
\centering
\small
\begin{tabular}{c c c c c c c c}
\hline\hline
Instancia & Costo mínimo & Tiempo(seg.) & Costo promedio & Tiempo promedio(seg.) & CME & \%G & \%GP \\ [0.5ex]
\hline
CMT1X & 470.67 & 0.91 & 
477.25 & 0.71 & \bf{470.48} & 
0.04 & 1.44\\CMT1Y & 472.37 & 0.75 & 
483.43 & 0.77 & \bf{470.48} & 
0.40 & 2.75\\CMT2X & 713.87 & 1.03 & 
738.82 & 1.09 & \bf{682.39} & 
4.61 & 8.27\\CMT2Y & 727.42 & 1.27 & 
732.87 & 0.98 & \bf{682.39} & 
6.60 & 7.40\\CMT3X & 737.49 & 7.92 & 
755.71 & 7.56 & \bf{719.06} & 
2.56 & 5.10\\CMT3Y & 732.15 & 7.33 & 
758.34 & 8.93 & \bf{719.06} & 
1.82 & 5.46\\CMT4X & 915.24 & 16.56 & 
922.43 & 14.17 & \bf{854.21} & 
7.14 & 7.99\\CMT4Y & 883.09 & 16.36 & 
925.50 & 14.74 & \bf{852.46} & 
3.59 & 8.57\\CMT5X & 1119.14 & 17.12 & 
1151.53 & 27.19 & \bf{1030.56} & 
8.60 & 11.74\\CMT5Y & 1116.48 & 8.09 & 
1159.22 & 9.97 & \bf{1031.69} & 
8.22 & 12.36\\CMT11X & 874.08 & 1.74 & 
924.86 & 1.69 & \bf{831.09} & 
5.17 & 11.28\\CMT11Y & 906.12 & 4.28 & 
947.95 & 2.40 & \bf{829.85} & 
9.19 & 14.23\\CMT12X & 679.04 & 0.59 & 
744.26 & 0.64 & \bf{658.83} & 
3.07 & 12.97\\CMT12Y & 690.99 & 0.58 & 
788.32 & 0.75 & \bf{660.47} & 
4.62 & 19.36\\\bf{PROM.} & 
\bf{788.44} & \bf{6.04} & \bf{822.18} & \bf{6.54} & \bf{749.50} & \bf{4.69} & \bf{9.21}\\[1ex]\hline
\end{tabular}
\label{table:nonlin}
\end{table} \clearpage
\begin{table}[ht]
\caption{Resultados de la ejecución de la metaheurística PSO, utilizando instancias de SalhiNagy con la configuración -n 10.0 -L 30.0 -cp 1 -cg 0 -cl 1 -cn 2 -w1 0.9 -wt 0.1 -K 5}
\centering
\small
\begin{tabular}{c c c c c c c c}
\hline\hline
Instancia & Costo mínimo & Tiempo(seg.) & Costo promedio & Tiempo promedio(seg.) & CME & \%G & \%GP \\ [0.5ex]
\hline
CMT1X & 475.22 & 1.98 & 
477.46 & 1.95 & \bf{470.48} & 
1.01 & 1.48\\CMT1Y & 473.01 & 1.75 & 
500.65 & 2.16 & \bf{470.48} & 
0.54 & 6.41\\CMT2X & 710.82 & 2.90 & 
737.62 & 3.61 & \bf{682.39} & 
4.17 & 8.09\\CMT2Y & 709.36 & 4.16 & 
728.88 & 3.18 & \bf{682.39} & 
3.95 & 6.81\\CMT3X & 734.71 & 30.81 & 
743.37 & 27.32 & \bf{719.06} & 
2.18 & 3.38\\CMT3Y & 725.34 & 27.74 & 
742.21 & 26.55 & \bf{719.06} & 
0.87 & 3.22\\CMT4X & 926.48 & 41.02 & 
939.27 & 39.20 & \bf{854.21} & 
8.46 & 9.96\\CMT4Y & 900.82 & 26.38 & 
930.63 & 29.66 & \bf{852.46} & 
5.67 & 9.17\\CMT5X & 1104.05 & 36.66 & 
1165.38 & 35.12 & \bf{1030.56} & 
7.13 & 13.08\\CMT5Y & 1089.15 & 56.17 & 
1123.82 & 40.24 & \bf{1031.69} & 
5.57 & 8.93\\CMT11X & 900.93 & 4.40 & 
967.41 & 7.12 & \bf{831.09} & 
8.40 & 16.40\\CMT11Y & 901.48 & 5.51 & 
906.68 & 5.78 & \bf{829.85} & 
8.63 & 9.26\\CMT12X & 665.83 & 2.70 & 
721.48 & 2.50 & \bf{658.83} & 
1.06 & 9.51\\CMT12Y & \bf{\underline{607.85}} & 1.63 & 
708.76 & 1.78 & 660.47 & 
\bf{-7.97} & 7.31\\\bf{PROM.} & 
\bf{780.36} & \bf{17.41} & \bf{813.83} & \bf{16.15} & \bf{749.50} & \bf{3.55} & \bf{8.07}\\[1ex]\hline
\end{tabular}
\label{table:nonlin}
\end{table} \clearpage
\begin{table}[ht]
\caption{Resultados de la ejecución de la metaheurística PSO, utilizando instancias de SalhiNagy con la configuración -n 10.0 -L 50.0 -cp 1 -cg 0 -cl 1 -cn 2 -w1 0.9 -wt 0.1 -K 5}
\centering
\small
\begin{tabular}{c c c c c c c c}
\hline\hline
Instancia & Costo mínimo & Tiempo(seg.) & Costo promedio & Tiempo promedio(seg.) & CME & \%G & \%GP \\ [0.5ex]
\hline
CMT1X & 470.67 & 3.55 & 
475.35 & 3.73 & \bf{470.48} & 
0.04 & 1.04\\CMT1Y & 472.37 & 3.48 & 
473.29 & 3.81 & \bf{470.48} & 
0.40 & 0.60\\CMT2X & 708.58 & 4.52 & 
736.26 & 4.57 & \bf{682.39} & 
3.84 & 7.89\\CMT2Y & 703.08 & 4.47 & 
721.25 & 4.36 & \bf{682.39} & 
3.03 & 5.69\\CMT3X & 734.02 & 42.05 & 
746.31 & 42.56 & \bf{719.06} & 
2.08 & 3.79\\CMT3Y & 727.56 & 45.43 & 
746.20 & 43.38 & \bf{719.06} & 
1.18 & 3.77\\CMT4X & 910.49 & 37.79 & 
922.74 & 49.64 & \bf{854.21} & 
6.59 & 8.02\\CMT4Y & 883.56 & 46.27 & 
913.63 & 50.67 & \bf{852.46} & 
3.65 & 7.18\\CMT5X & 1126.21 & 58.55 & 
1177.30 & 69.31 & \bf{1030.56} & 
9.28 & 14.24\\CMT5Y & 1131.48 & 59.97 & 
1148.82 & 57.90 & \bf{1031.69} & 
9.67 & 11.35\\CMT11X & 901.82 & 7.68 & 
907.16 & 8.91 & \bf{831.09} & 
8.51 & 9.15\\CMT11Y & 890.27 & 9.10 & 
907.90 & 9.97 & \bf{829.85} & 
7.28 & 9.41\\CMT12X & 697.51 & 2.87 & 
770.68 & 3.15 & \bf{658.83} & 
5.87 & 16.98\\CMT12Y & 766.38 & 3.45 & 
783.76 & 4.34 & \bf{660.47} & 
16.04 & 18.67\\\bf{PROM.} & 
\bf{794.57} & \bf{23.51} & \bf{816.48} & \bf{25.45} & \bf{749.50} & \bf{5.53} & \bf{8.41}\\[1ex]\hline
\end{tabular}
\label{table:nonlin}
\end{table} \clearpage
\begin{table}[ht]
\caption{Resultados de la ejecución de la metaheurística SCA, utilizando instancias de SalhiNagy con la configuración -n 150.0 -b 10 -y 0.1}
\centering
\small
\begin{tabular}{c c c c c c c c}
\hline\hline
Instancia & Costo mínimo & Tiempo(seg.) & Costo promedio & Tiempo promedio(seg.) & CME & \%G & \%GP \\ [0.5ex]
\hline
CMT1X & 472.37 & 3.21 & 
473.08 & 2.84 & \bf{470.48} & 
0.40 & 0.55\\CMT1Y & 480.15 & 1.15 & 
480.15 & 1.17 & \bf{470.48} & 
2.06 & 2.06\\CMT2X & 711.65 & 16.82 & 
713.04 & 13.55 & \bf{682.39} & 
4.29 & 4.49\\CMT2Y & 703.00 & 22.52 & 
710.14 & 17.81 & \bf{682.39} & 
3.02 & 4.07\\CMT3X & 729.86 & 51.99 & 
732.82 & 45.16 & \bf{719.06} & 
1.50 & 1.91\\CMT3Y & 727.10 & 34.85 & 
737.27 & 30.00 & \bf{719.06} & 
1.12 & 2.53\\CMT4X & 897.25 & 173.62 & 
906.74 & 202.63 & \bf{854.21} & 
5.04 & 6.15\\CMT4Y & 892.03 & 173.72 & 
901.90 & 183.08 & \bf{852.46} & 
4.64 & 5.80\\CMT5X & 1099.51 & Command & 
1123.62 & 404.51 & \bf{1030.56} & 
6.69 & 9.03\\CMT5Y & 1110.02 & 1055.47 & 
1117.78 & 758.73 & \bf{1031.69} & 
7.59 & 8.34\\CMT11X & 908.56 & 40.80 & 
914.40 & 38.80 & \bf{831.09} & 
9.32 & 10.02\\CMT11Y & 893.14 & 63.46 & 
904.82 & 58.56 & \bf{829.85} & 
7.63 & 9.03\\CMT12X & 682.69 & 30.59 & 
686.61 & 36.78 & \bf{658.83} & 
3.62 & 4.22\\CMT12Y & 678.93 & 37.66 & 
680.04 & 43.23 & \bf{660.47} & 
2.79 & 2.96\\\bf{PROM.} & 
\bf{784.73} & \bf{121.85} & \bf{791.60} & \bf{131.20} & \bf{749.50} & \bf{4.27} & \bf{5.08}\\[1ex]\hline
\end{tabular}
\label{table:nonlin}
\end{table} \clearpage
\begin{table}[ht]
\caption{Resultados de la ejecución de la metaheurística PSO, utilizando instancias de SalhiNagy con la configuración -n 10.0 -L 70.0 -cp 1 -cg 0 -cl 1 -cn 2 -w1 0.9 -wt 0.1 -K 5}
\centering
\small
\begin{tabular}{c c c c c c c c}
\hline\hline
Instancia & Costo mínimo & Tiempo(seg.) & Costo promedio & Tiempo promedio(seg.) & CME & \%G & \%GP \\ [0.5ex]
\hline
CMT1X & 477.63 & 5.23 & 
494.63 & 5.09 & \bf{470.48} & 
1.52 & 5.13\\CMT1Y & 472.37 & 4.28 & 
473.71 & 4.68 & \bf{470.48} & 
0.40 & 0.69\\CMT2X & 702.80 & 9.21 & 
709.74 & 7.82 & \bf{682.39} & 
2.99 & 4.01\\CMT2Y & 697.17 & 7.67 & 
729.26 & 7.12 & \bf{682.39} & 
2.17 & 6.87\\CMT3X & 728.29 & 56.66 & 
734.34 & 59.92 & \bf{719.06} & 
1.28 & 2.13\\CMT3Y & 733.26 & 70.18 & 
743.99 & 63.35 & \bf{719.06} & 
1.97 & 3.47\\CMT4X & 918.09 & 73.10 & 
932.97 & 75.18 & \bf{854.21} & 
7.48 & 9.22\\CMT4Y & 900.06 & 80.02 & 
925.08 & 81.64 & \bf{852.46} & 
5.58 & 8.52\\CMT5X & 1140.88 & 79.84 & 
1174.26 & 101.81 & \bf{1030.56} & 
10.70 & 13.94\\CMT5Y & 1132.39 & 88.23 & 
1203.37 & 86.16 & \bf{1031.69} & 
9.76 & 16.64\\CMT11X & 888.37 & 14.19 & 
903.81 & 15.83 & \bf{831.09} & 
6.89 & 8.75\\CMT11Y & 881.55 & 16.55 & 
922.42 & 16.47 & \bf{829.85} & 
6.23 & 11.16\\CMT12X & 733.89 & 5.05 & 
762.95 & 5.13 & \bf{658.83} & 
11.39 & 15.80\\CMT12Y & 678.23 & 4.34 & 
792.76 & 6.20 & \bf{660.47} & 
2.69 & 20.03\\\bf{PROM.} & 
\bf{791.78} & \bf{36.75} & \bf{821.66} & \bf{38.31} & \bf{749.50} & \bf{5.08} & \bf{9.03}\\[1ex]\hline
\end{tabular}
\label{table:nonlin}
\end{table} \clearpage
\begin{table}[ht]
\caption{Resultados de la ejecución de la metaheurística PSO, utilizando instancias de SalhiNagy con la configuración -n 10.0 -L 90.0 -cp 1 -cg 0 -cl 1 -cn 2 -w1 0.9 -wt 0.1 -K 5}
\centering
\small
\begin{tabular}{c c c c c c c c}
\hline\hline
Instancia & Costo mínimo & Tiempo(seg.) & Costo promedio & Tiempo promedio(seg.) & CME & \%G & \%GP \\ [0.5ex]
\hline
CMT1X & 470.67 & 7.03 & 
474.74 & 7.20 & \bf{470.48} & 
0.04 & 0.91\\CMT1Y & 470.67 & 6.63 & 
476.88 & 6.55 & \bf{470.48} & 
0.04 & 1.36\\CMT2X & 703.06 & 9.36 & 
739.96 & 9.95 & \bf{682.39} & 
3.03 & 8.44\\CMT2Y & 708.97 & 8.68 & 
734.66 & 8.77 & \bf{682.39} & 
3.90 & 7.66\\CMT3X & 735.47 & 76.70 & 
749.79 & 78.25 & \bf{719.06} & 
2.28 & 4.27\\CMT3Y & 727.12 & 76.64 & 
731.68 & 77.50 & \bf{719.06} & 
1.12 & 1.76\\CMT4X & 892.77 & 107.86 & 
920.92 & 96.75 & \bf{854.21} & 
4.51 & 7.81\\CMT4Y & 895.54 & 115.49 & 
908.78 & 108.76 & \bf{852.46} & 
5.05 & 6.61\\CMT5X & 1125.55 & 113.55 & 
1151.82 & 127.66 & \bf{1030.56} & 
9.22 & 11.77\\CMT5Y & 1102.43 & 122.47 & 
1140.06 & 122.49 & \bf{1031.69} & 
6.86 & 10.50\\CMT11X & 900.57 & 17.90 & 
920.39 & 18.61 & \bf{831.09} & 
8.36 & 10.75\\CMT11Y & 891.76 & 15.61 & 
905.48 & 16.68 & \bf{829.85} & 
7.46 & 9.11\\CMT12X & 749.14 & 9.51 & 
824.90 & 7.99 & \bf{658.83} & 
13.71 & 25.21\\CMT12Y & 742.56 & 9.32 & 
756.42 & 6.88 & \bf{660.47} & 
12.43 & 14.53\\\bf{PROM.} & 
\bf{794.02} & \bf{49.77} & \bf{816.89} & \bf{49.57} & \bf{749.50} & \bf{5.57} & \bf{8.62}\\[1ex]\hline
\end{tabular}
\label{table:nonlin}
\end{table} \clearpage
\begin{table}[ht]
\caption{Resultados de la ejecución de la metaheurística PSO, utilizando instancias de SalhiNagy con la configuración -n 10.0 -L 110.0 -cp 1 -cg 0 -cl 1 -cn 2 -w1 0.9 -wt 0.1 -K 5}
\centering
\small
\begin{tabular}{c c c c c c c c}
\hline\hline
Instancia & Costo mínimo & Tiempo(seg.) & Costo promedio & Tiempo promedio(seg.) & CME & \%G & \%GP \\ [0.5ex]
\hline
CMT1X & 472.37 & 9.34 & 
478.00 & 9.29 & \bf{470.48} & 
0.40 & 1.60\\CMT1Y & 471.09 & 8.58 & 
476.19 & 9.11 & \bf{470.48} & 
0.13 & 1.21\\CMT2X & 730.66 & 8.43 & 
745.91 & 10.20 & \bf{682.39} & 
7.07 & 9.31\\CMT2Y & 711.37 & 9.72 & 
715.58 & 10.02 & \bf{682.39} & 
4.25 & 4.86\\CMT3X & 721.74 & 97.92 & 
732.62 & 99.00 & \bf{719.06} & 
0.37 & 1.89\\CMT3Y & 727.84 & 96.91 & 
738.30 & 96.16 & \bf{719.06} & 
1.22 & 2.68\\CMT4X & 894.80 & 132.64 & 
921.74 & 129.38 & \bf{854.21} & 
4.75 & 7.91\\CMT4Y & 883.17 & 120.99 & 
917.53 & 123.47 & \bf{852.46} & 
3.60 & 7.63\\CMT5X & 1115.64 & 117.92 & 
1142.74 & 137.14 & \bf{1030.56} & 
8.26 & 10.89\\CMT5Y & 1090.92 & 188.11 & 
1128.46 & 143.85 & \bf{1031.69} & 
5.74 & 9.38\\CMT11X & 890.47 & 32.01 & 
898.68 & 23.18 & \bf{831.09} & 
7.14 & 8.13\\CMT11Y & 894.31 & 28.72 & 
903.03 & 24.06 & \bf{829.85} & 
7.77 & 8.82\\CMT12X & 695.29 & 7.62 & 
761.70 & 8.47 & \bf{658.83} & 
5.53 & 15.61\\CMT12Y & 733.56 & 8.20 & 
775.12 & 8.97 & \bf{660.47} & 
11.07 & 17.36\\\bf{PROM.} & 
\bf{788.09} & \bf{61.94} & \bf{809.69} & \bf{59.45} & \bf{749.50} & \bf{4.81} & \bf{7.66}\\[1ex]\hline
\end{tabular}
\label{table:nonlin}
\end{table} \clearpage
\begin{table}[ht]
\caption{Resultados de la ejecución de la metaheurística PSO, utilizando instancias de SalhiNagy con la configuración -n 30.0 -L 10.0 -cp 1 -cg 0 -cl 1 -cn 2 -w1 0.9 -wt 0.1 -K 5}
\centering
\small
\begin{tabular}{c c c c c c c c}
\hline\hline
Instancia & Costo mínimo & Tiempo(seg.) & Costo promedio & Tiempo promedio(seg.) & CME & \%G & \%GP \\ [0.5ex]
\hline
CMT1X & 471.25 & 2.05 & 
484.12 & 2.06 & \bf{470.48} & 
0.16 & 2.90\\CMT1Y & 472.37 & 1.56 & 
475.85 & 2.27 & \bf{470.48} & 
0.40 & 1.14\\CMT2X & 704.07 & 2.09 & 
729.54 & 3.19 & \bf{682.39} & 
3.18 & 6.91\\CMT2Y & 707.56 & 4.28 & 
723.16 & 3.23 & \bf{682.39} & 
3.69 & 5.97\\CMT3X & 730.85 & 18.98 & 
746.56 & 19.20 & \bf{719.06} & 
1.64 & 3.82\\CMT3Y & 727.31 & 24.88 & 
741.67 & 23.12 & \bf{719.06} & 
1.15 & 3.14\\CMT4X & 902.96 & 29.28 & 
919.42 & 30.18 & \bf{854.21} & 
5.71 & 7.63\\CMT4Y & 937.43 & 33.70 & 
943.55 & 36.23 & \bf{852.46} & 
9.97 & 10.69\\CMT5X & 1141.04 & 20.23 & 
1147.45 & 16.70 & \bf{1030.56} & 
10.72 & 11.34\\CMT5Y & 1100.68 & 94.64 & 
1156.20 & 48.37 & \bf{1031.69} & 
6.69 & 12.07\\CMT11X & 899.82 & 9.40 & 
928.94 & 5.95 & \bf{831.09} & 
8.27 & 11.77\\CMT11Y & 898.84 & 12.73 & 
932.60 & 15.65 & \bf{829.85} & 
8.31 & 12.38\\CMT12X & \bf{\underline{646.11}} & 2.18 & 
682.71 & 1.89 & 658.83 & 
\bf{-1.93} & 3.62\\CMT12Y & \bf{\underline{621.71}} & 1.38 & 
687.40 & 1.48 & 660.47 & 
\bf{-5.87} & 4.08\\\bf{PROM.} & 
\bf{783.00} & \bf{18.38} & \bf{807.08} & \bf{14.97} & \bf{749.50} & \bf{3.72} & \bf{6.96}\\[1ex]\hline
\end{tabular}
\label{table:nonlin}
\end{table} \clearpage
\begin{table}[ht]
\caption{Resultados de la ejecución de la metaheurística SCA, utilizando instancias de SalhiNagy con la configuración -n 150.0 -b 10 -y .2}
\centering
\small
\begin{tabular}{c c c c c c c c}
\hline\hline
Instancia & Costo mínimo & Tiempo(seg.) & Costo promedio & Tiempo promedio(seg.) & CME & \%G & \%GP \\ [0.5ex]
\hline
CMT1X & 475.22 & 1.28 & 
480.87 & 1.85 & \bf{470.48} & 
1.01 & 2.21\\CMT1Y & 473.01 & 1.52 & 
473.87 & 1.44 & \bf{470.48} & 
0.54 & 0.72\\CMT2X & 704.90 & 17.29 & 
707.20 & 17.41 & \bf{682.39} & 
3.30 & 3.64\\CMT2Y & 695.04 & 15.84 & 
708.48 & 15.51 & \bf{682.39} & 
1.85 & 3.82\\CMT3X & 729.83 & 35.87 & 
739.69 & 38.77 & \bf{719.06} & 
1.50 & 2.87\\CMT3Y & 734.07 & 35.46 & 
737.18 & 39.56 & \bf{719.06} & 
2.09 & 2.52\\CMT4X & 897.78 & 201.25 & 
911.53 & 211.32 & \bf{854.21} & 
5.10 & 6.71\\CMT4Y & 905.39 & 197.35 & 
908.59 & 208.71 & \bf{852.46} & 
6.21 & 6.58\\CMT5X & 1102.97 & 1031.15 & 
1111.08 & 873.47 & \bf{1030.56} & 
7.03 & 7.81\\CMT5Y & 1091.65 & 694.26 & 
1107.65 & 823.40 & \bf{1031.69} & 
5.81 & 7.36\\CMT11X & 881.09 & 43.05 & 
898.58 & 36.48 & \bf{831.09} & 
6.02 & 8.12\\CMT11Y & 884.31 & 47.38 & 
909.57 & 42.63 & \bf{829.85} & 
6.56 & 9.61\\CMT12X & 680.80 & 35.51 & 
684.72 & 49.56 & \bf{658.83} & 
3.33 & 3.93\\CMT12Y & 675.76 & 95.29 & 
683.65 & 63.20 & \bf{660.47} & 
2.32 & 3.51\\\bf{PROM.} & 
\bf{780.84} & \bf{175.18} & \bf{790.19} & \bf{173.09} & \bf{749.50} & \bf{3.76} & \bf{4.96}\\[1ex]\hline
\end{tabular}
\label{table:nonlin}
\end{table} \clearpage
\begin{table}[ht]
\caption{Resultados de la ejecución de la metaheurística PSO, utilizando instancias de SalhiNagy con la configuración -n 30.0 -L 30.0 -cp 1 -cg 0 -cl 1 -cn 2 -w1 0.9 -wt 0.1 -K 5}
\centering
\small
\begin{tabular}{c c c c c c c c}
\hline\hline
Instancia & Costo mínimo & Tiempo(seg.) & Costo promedio & Tiempo promedio(seg.) & CME & \%G & \%GP \\ [0.5ex]
\hline
CMT1X & 472.87 & 6.67 & 
485.43 & 6.44 & \bf{470.48} & 
0.51 & 3.18\\CMT1Y & 470.67 & 5.90 & 
471.40 & 6.33 & \bf{470.48} & 
0.04 & 0.20\\CMT2X & 704.36 & 6.20 & 
715.05 & 6.78 & \bf{682.39} & 
3.22 & 4.79\\CMT2Y & 712.37 & 8.06 & 
735.93 & 8.47 & \bf{682.39} & 
4.39 & 7.85\\CMT3X & 731.50 & 71.38 & 
737.88 & 67.61 & \bf{719.06} & 
1.73 & 2.62\\CMT3Y & 730.86 & 71.26 & 
743.46 & 74.17 & \bf{719.06} & 
1.64 & 3.39\\CMT4X & 889.30 & 127.00 & 
910.14 & 107.54 & \bf{854.21} & 
4.11 & 6.55\\CMT4Y & 879.58 & 111.24 & 
907.74 & 128.17 & \bf{852.46} & 
3.18 & 6.49\\CMT5X & 1131.09 & 140.74 & 
1150.40 & 129.79 & \bf{1030.56} & 
9.75 & 11.63\\CMT5Y & 1102.67 & 62.22 & 
1145.40 & 80.43 & \bf{1031.69} & 
6.88 & 11.02\\CMT11X & 888.29 & 20.76 & 
896.48 & 26.65 & \bf{831.09} & 
6.88 & 7.87\\CMT11Y & 888.23 & 34.19 & 
923.14 & 36.09 & \bf{829.85} & 
7.04 & 11.24\\CMT12X & \bf{\underline{643.58}} & 4.16 & 
691.58 & 6.58 & 658.83 & 
\bf{-2.31} & 4.97\\CMT12Y & \bf{\underline{536.95}} & 4.54 & 
705.78 & 5.23 & 660.47 & 
\bf{-18.70} & 6.86\\\bf{PROM.} & 
\bf{770.17} & \bf{48.17} & \bf{801.42} & \bf{49.30} & \bf{749.50} & \bf{2.03} & \bf{6.33}\\[1ex]\hline
\end{tabular}
\label{table:nonlin}
\end{table} \clearpage
\begin{table}[ht]
\caption{Resultados de la ejecución de la metaheurística PSO, utilizando instancias de SalhiNagy con la configuración -n 30.0 -L 50.0 -cp 1 -cg 0 -cl 1 -cn 2 -w1 0.9 -wt 0.1 -K 5}
\centering
\small
\begin{tabular}{c c c c c c c c}
\hline\hline
Instancia & Costo mínimo & Tiempo(seg.) & Costo promedio & Tiempo promedio(seg.) & CME & \%G & \%GP \\ [0.5ex]
\hline
CMT1X & 470.67 & 9.61 & 
475.30 & 9.46 & \bf{470.48} & 
0.04 & 1.02\\CMT1Y & 470.67 & 11.46 & 
489.56 & 10.95 & \bf{470.48} & 
0.04 & 4.06\\CMT2X & 702.85 & 16.32 & 
715.59 & 11.78 & \bf{682.39} & 
3.00 & 4.86\\CMT2Y & 701.08 & 16.02 & 
718.13 & 15.47 & \bf{682.39} & 
2.74 & 5.24\\CMT3X & 727.88 & 107.20 & 
745.83 & 112.96 & \bf{719.06} & 
1.23 & 3.72\\CMT3Y & 726.26 & 117.12 & 
729.15 & 118.16 & \bf{719.06} & 
1.00 & 1.40\\CMT4X & 878.21 & 190.71 & 
905.79 & 176.65 & \bf{854.21} & 
2.81 & 6.04\\CMT4Y & 908.96 & 134.29 & 
922.73 & 205.28 & \bf{852.46} & 
6.63 & 8.24\\CMT5X & 1091.23 & 116.43 & 
1130.22 & 169.00 & \bf{1030.56} & 
5.89 & 9.67\\CMT5Y & 1126.00 & 170.72 & 
1139.34 & 188.82 & \bf{1031.69} & 
9.14 & 10.43\\CMT11X & 908.97 & 16.90 & 
941.49 & 30.30 & \bf{831.09} & 
9.37 & 13.28\\CMT11Y & 893.80 & 23.32 & 
904.07 & 57.91 & \bf{829.85} & 
7.71 & 8.94\\CMT12X & 678.62 & 7.32 & 
709.52 & 8.89 & \bf{658.83} & 
3.00 & 7.69\\CMT12Y & 708.33 & 7.66 & 
802.17 & 8.12 & \bf{660.47} & 
7.25 & 21.46\\\bf{PROM.} & 
\bf{785.25} & \bf{67.51} & \bf{809.21} & \bf{80.27} & \bf{749.50} & \bf{4.27} & \bf{7.58}\\[1ex]\hline
\end{tabular}
\label{table:nonlin}
\end{table} \clearpage
\begin{table}[ht]
\caption{Resultados de la ejecución de la metaheurística IGA, utilizando instancias de Dethloff con la configuración -n 200 -p 40 -cprob 70 -mprob 40}
\centering
\small
\begin{tabular}{c c c c c c c c}
\hline\hline
Instancia & Costo mínimo & Tiempo(seg.) & Costo promedio & Tiempo promedio(seg.) & CME & \%G & \%GP \\ [0.5ex]
\hline
SCA3-0 & 640.55 & 0.73 & 
640.55 & 0.68 & \bf{635.62} & 
0.78 & 0.78\\SCA3-1 & 701.53 & 0.90 & 
702.71 & 0.82 & \bf{697.84} & 
0.53 & 0.70\\SCA3-2 & 666.01 & 0.61 & 
667.21 & 0.67 & \bf{659.34} & 
1.01 & 1.19\\SCA3-3 & 682.47 & 0.79 & 
685.51 & 0.78 & \bf{680.04} & 
0.36 & 0.80\\SCA3-4 & \bf{690.50} & 0.64 & 
690.50 & 0.68 & 690.50 & 0.00
 & 0.00\\
SCA3-5 & 673.46 & 0.67 & 
677.92 & 0.77 & \bf{659.90} & 
2.05 & 2.73\\SCA3-6 & 652.94 & 0.94 & 
653.92 & 0.69 & \bf{651.09} & 
0.28 & 0.44\\SCA3-7 & 666.15 & 0.80 & 
666.66 & 0.73 & \bf{659.17} & 
1.06 & 1.14\\SCA3-8 & 726.86 & 0.91 & 
730.17 & 0.85 & \bf{719.47} & 
1.03 & 1.49\\SCA3-9 & 685.88 & 0.80 & 
686.67 & 0.80 & \bf{681.00} & 
0.72 & 0.83\\SCA8-0 & 984.75 & 1.00 & 
984.75 & 0.75 & \bf{961.50} & 
2.42 & 2.42\\SCA8-1 & 1070.49 & 0.75 & 
1072.72 & 0.88 & \bf{1049.65} & 
1.99 & 2.20\\SCA8-2 & 1053.85 & 0.68 & 
1054.32 & 0.65 & \bf{1039.64} & 
1.37 & 1.41\\SCA8-3 & 1013.77 & 0.58 & 
1013.77 & 0.68 & \bf{983.34} & 
3.09 & 3.09\\SCA8-4 & 1094.22 & 0.79 & 
1100.04 & 0.67 & \bf{1065.49} & 
2.70 & 3.24\\SCA8-5 & 1061.98 & 0.67 & 
1068.82 & 0.82 & \bf{1027.08} & 
3.40 & 4.06\\SCA8-6 & 981.28 & 0.51 & 
983.47 & 0.66 & \bf{971.82} & 
0.97 & 1.20\\SCA8-7 & 1075.87 & 0.70 & 
1075.87 & 0.61 & \bf{1051.28} & 
2.34 & 2.34\\SCA8-8 & 1086.27 & 0.46 & 
1086.27 & 0.67 & \bf{1071.18} & 
1.41 & 1.41\\SCA8-9 & 1069.83 & 0.99 & 
1069.83 & 0.86 & \bf{1060.50} & 
0.88 & 0.88\\CON3-0 & 620.76 & 0.73 & 
627.93 & 0.74 & \bf{616.52} & 
0.69 & 1.85\\CON3-1 & 556.04 & 0.54 & 
556.04 & 0.55 & \bf{554.47} & 
0.28 & 0.28\\CON3-2 & 521.38 & 0.75 & 
521.75 & 0.89 & \bf{518.00} & 
0.65 & 0.72\\CON3-3 & 592.78 & 0.86 & 
603.05 & 0.71 & \bf{591.19} & 
0.27 & 2.01\\CON3-4 & 598.21 & 0.50 & 
598.21 & 0.67 & \bf{588.79} & 
1.60 & 1.60\\CON3-5 & 567.94 & 1.16 & 
568.32 & 0.85 & \bf{563.70} & 
0.75 & 0.82\\CON3-6 & 503.97 & 0.70 & 
503.97 & 0.68 & \bf{499.05} & 
0.99 & 0.99\\CON3-7 & 578.22 & 0.69 & 
580.16 & 0.75 & \bf{576.48} & 
0.30 & 0.64\\CON3-8 & 524.59 & 0.68 & 
526.18 & 0.70 & \bf{523.05} & 
0.29 & 0.60\\CON3-9 & 582.79 & 0.77 & 
589.99 & 0.73 & \bf{578.24} & 
0.79 & 2.03\\CON8-0 & 881.76 & 0.73 & 
887.08 & 0.67 & \bf{857.17} & 
2.87 & 3.49\\CON8-1 & 758.92 & 0.78 & 
765.22 & 0.85 & \bf{740.85} & 
2.44 & 3.29\\CON8-2 & 723.99 & 0.87 & 
723.99 & 0.67 & \bf{712.89} & 
1.56 & 1.56\\CON8-3 & 830.53 & 0.92 & 
833.20 & 0.71 & \bf{811.07} & 
2.40 & 2.73\\CON8-4 & 778.31 & 0.94 & 
778.31 & 0.87 & \bf{772.25} & 
0.78 & 0.78\\CON8-5 & 764.36 & 0.87 & 
770.55 & 0.73 & \bf{754.88} & 
1.26 & 2.08\\CON8-6 & 692.16 & 0.77 & 
692.16 & 0.84 & \bf{678.92} & 
1.95 & 1.95\\CON8-7 & 822.26 & 0.56 & 
822.81 & 0.79 & \bf{811.96} & 
1.27 & 1.34\\CON8-8 & 782.60 & 0.82 & 
791.61 & 0.88 & \bf{767.53} & 
1.96 & 3.14\\CON8-9 & 816.22 & 1.06 & 
829.73 & 0.80 & \bf{809.00} & 
0.89 & 2.56\\\bf{PROM.} & 
\bf{769.41} & \bf{0.77} & \bf{772.05} & \bf{0.75} & \bf{758.54} & \bf{1.31} & \bf{1.67}\\[1ex]\hline
\end{tabular}
\label{table:nonlin}
\end{table} \clearpage
\begin{table}[ht]
\caption{Resultados de la ejecución de la metaheurística IGA, utilizando instancias de Dethloff con la configuración -n 200 -p 40 -cprob 80 -mprob 90}
\centering
\small
\begin{tabular}{c c c c c c c c}
\hline\hline
Instancia & Costo mínimo & Tiempo(seg.) & Costo promedio & Tiempo promedio(seg.) & CME & \%G & \%GP \\ [0.5ex]
\hline
SCA3-0 & 640.55 & 0.90 & 
641.89 & 0.84 & \bf{635.62} & 
0.78 & 0.99\\SCA3-1 & \bf{697.84} & 0.93 & 
701.78 & 0.83 & 697.84 & 0.00
 & 0.56\\SCA3-2 & 664.21 & 0.62 & 
674.52 & 0.80 & \bf{659.34} & 
0.74 & 2.30\\SCA3-3 & \bf{680.04} & 0.91 & 
681.25 & 0.79 & 680.04 & 0.00
 & 0.18\\SCA3-4 & \bf{690.50} & 0.68 & 
690.50 & 0.66 & 690.50 & 0.00
 & 0.00\\
SCA3-5 & 666.67 & 0.96 & 
666.67 & 0.90 & \bf{659.90} & 
1.03 & 1.03\\SCA3-6 & 652.94 & 0.93 & 
655.25 & 0.86 & \bf{651.09} & 
0.28 & 0.64\\SCA3-7 & 666.15 & 0.91 & 
666.42 & 0.83 & \bf{659.17} & 
1.06 & 1.10\\SCA3-8 & 723.99 & 0.96 & 
725.06 & 0.88 & \bf{719.47} & 
0.63 & 0.78\\SCA3-9 & \bf{681.00} & 0.90 & 
681.00 & 0.81 & 681.00 & 0.00
 & 0.00\\
SCA8-0 & 1007.63 & 1.03 & 
1012.56 & 1.01 & \bf{961.50} & 
4.80 & 5.31\\SCA8-1 & 1066.42 & 1.03 & 
1074.73 & 1.02 & \bf{1049.65} & 
1.60 & 2.39\\SCA8-2 & 1052.74 & 0.98 & 
1053.31 & 0.95 & \bf{1039.64} & 
1.26 & 1.31\\SCA8-3 & 1022.71 & 0.97 & 
1025.87 & 0.81 & \bf{983.34} & 
4.00 & 4.33\\SCA8-4 & 1083.30 & 0.90 & 
1093.17 & 0.88 & \bf{1065.49} & 
1.67 & 2.60\\SCA8-5 & 1067.09 & 1.01 & 
1067.84 & 0.94 & \bf{1027.08} & 
3.90 & 3.97\\SCA8-6 & 972.48 & 0.82 & 
972.48 & 0.88 & \bf{971.82} & 
0.07 & 0.07\\SCA8-7 & 1070.92 & 0.71 & 
1077.40 & 0.71 & \bf{1051.28} & 
1.87 & 2.48\\SCA8-8 & 1086.27 & 0.93 & 
1089.54 & 0.81 & \bf{1071.18} & 
1.41 & 1.71\\SCA8-9 & 1073.62 & 0.80 & 
1073.62 & 0.84 & \bf{1060.50} & 
1.24 & 1.24\\CON3-0 & 633.24 & 0.90 & 
634.32 & 0.87 & \bf{616.52} & 
2.71 & 2.89\\CON3-1 & 560.75 & 0.66 & 
563.21 & 0.80 & \bf{554.47} & 
1.13 & 1.58\\CON3-2 & 521.38 & 0.99 & 
524.01 & 0.88 & \bf{518.00} & 
0.65 & 1.16\\CON3-3 & 591.20 & 0.94 & 
599.41 & 0.84 & \bf{591.19} & 
0.00 & 1.39\\CON3-4 & 596.86 & 0.71 & 
600.99 & 0.75 & \bf{588.79} & 
1.37 & 2.07\\CON3-5 & 567.94 & 0.97 & 
568.75 & 0.87 & \bf{563.70} & 
0.75 & 0.90\\CON3-6 & 504.20 & 0.82 & 
504.64 & 0.84 & \bf{499.05} & 
1.03 & 1.12\\CON3-7 & 577.54 & 0.92 & 
579.66 & 0.89 & \bf{576.48} & 
0.18 & 0.55\\CON3-8 & 527.82 & 0.96 & 
534.74 & 0.97 & \bf{523.05} & 
0.91 & 2.23\\CON3-9 & 582.79 & 0.96 & 
582.79 & 0.90 & \bf{578.24} & 
0.79 & 0.79\\CON8-0 & 880.19 & 0.76 & 
880.33 & 0.87 & \bf{857.17} & 
2.69 & 2.70\\CON8-1 & 742.44 & 0.94 & 
744.04 & 0.94 & \bf{740.85} & 
0.21 & 0.43\\CON8-2 & 725.05 & 1.18 & 
725.05 & 1.00 & \bf{712.89} & 
1.71 & 1.71\\CON8-3 & 819.89 & 0.83 & 
819.89 & 0.96 & \bf{811.07} & 
1.09 & 1.09\\CON8-4 & 796.62 & 0.92 & 
796.62 & 0.89 & \bf{772.25} & 
3.16 & 3.16\\CON8-5 & 772.54 & 0.80 & 
773.18 & 0.91 & \bf{754.88} & 
2.34 & 2.42\\CON8-6 & 692.75 & 0.94 & 
692.75 & 0.94 & \bf{678.92} & 
2.04 & 2.04\\CON8-7 & 817.29 & 0.67 & 
817.54 & 0.77 & \bf{811.96} & 
0.66 & 0.69\\CON8-8 & 788.35 & 0.79 & 
789.92 & 0.82 & \bf{767.53} & 
2.71 & 2.92\\CON8-9 & 817.48 & 1.07 & 
836.42 & 0.90 & \bf{809.00} & 
1.05 & 3.39\\\bf{PROM.} & 
\bf{769.58} & \bf{0.89} & \bf{772.33} & \bf{0.87} & \bf{758.54} & \bf{1.34} & \bf{1.70}\\[1ex]\hline
\end{tabular}
\label{table:nonlin}
\end{table} \clearpage
\begin{table}[ht]
\caption{Resultados de la ejecución de la metaheurística PSO, utilizando instancias de SalhiNagy con la configuración -n 30.0 -L 70.0 -cp 1 -cg 0 -cl 1 -cn 2 -w1 0.9 -wt 0.1 -K 5}
\centering
\small
\begin{tabular}{c c c c c c c c}
\hline\hline
Instancia & Costo mínimo & Tiempo(seg.) & Costo promedio & Tiempo promedio(seg.) & CME & \%G & \%GP \\ [0.5ex]
\hline
CMT1X & 470.67 & 13.87 & 
490.40 & 15.33 & \bf{470.48} & 
0.04 & 4.24\\CMT1Y & 470.67 & 14.06 & 
489.48 & 13.50 & \bf{470.48} & 
0.04 & 4.04\\CMT2X & 704.81 & 14.92 & 
727.48 & 17.70 & \bf{682.39} & 
3.29 & 6.61\\CMT2Y & 701.63 & 22.08 & 
730.39 & 16.88 & \bf{682.39} & 
2.82 & 7.03\\CMT3X & 723.23 & 168.11 & 
741.32 & 155.20 & \bf{719.06} & 
0.58 & 3.10\\CMT3Y & 726.87 & 161.85 & 
732.41 & 161.93 & \bf{719.06} & 
1.09 & 1.86\\CMT4X & 875.30 & 244.69 & 
916.91 & 246.68 & \bf{854.21} & 
2.47 & 7.34\\CMT4Y & 882.80 & 221.94 & 
901.93 & 242.89 & \bf{852.46} & 
3.56 & 5.80\\CMT5X & 1138.05 & 144.22 & 
1139.49 & 171.90 & \bf{1030.56} & 
10.43 & 10.57\\CMT5Y & 1072.95 & 222.09 & 
1134.12 & 275.02 & \bf{1031.69} & 
4.00 & 9.93\\CMT11X & 892.13 & 55.83 & 
906.90 & 53.35 & \bf{831.09} & 
7.34 & 9.12\\CMT11Y & 893.30 & 59.63 & 
925.71 & 57.25 & \bf{829.85} & 
7.65 & 11.55\\CMT12X & 692.10 & 16.94 & 
712.92 & 14.79 & \bf{658.83} & 
5.05 & 8.21\\CMT12Y & 683.03 & 17.40 & 
779.01 & 14.79 & \bf{660.47} & 
3.42 & 17.95\\\bf{PROM.} & 
\bf{780.54} & \bf{98.40} & \bf{809.18} & \bf{104.08} & \bf{749.50} & \bf{3.70} & \bf{7.67}\\[1ex]\hline
\end{tabular}
\label{table:nonlin}
\end{table} \clearpage
\begin{table}[ht]
\caption{Resultados de la ejecución de la metaheurística IGA, utilizando instancias de Dethloff con la configuración -n 200 -p 40 -cprob 80 -mprob 90}
\centering
\small
\begin{tabular}{c c c c c c c c}
\hline\hline
Instancia & Costo mínimo & Tiempo(seg.) & Costo promedio & Tiempo promedio(seg.) & CME & \%G & \%GP \\ [0.5ex]
\hline
SCA3-0 & 640.55 & 0.94 & 
640.55 & 0.91 & \bf{635.62} & 
0.78 & 0.78\\SCA3-1 & \bf{697.84} & 0.84 & 
697.84 & 0.86 & 697.84 & 0.00
 & 0.00\\
SCA3-2 & 664.21 & 0.64 & 
664.79 & 0.83 & \bf{659.34} & 
0.74 & 0.83\\SCA3-3 & 681.16 & 0.75 & 
681.82 & 0.87 & \bf{680.04} & 
0.16 & 0.26\\SCA3-4 & \bf{690.50} & 0.53 & 
690.50 & 0.72 & 690.50 & 0.00
 & 0.00\\
SCA3-5 & 666.67 & 0.90 & 
671.03 & 0.91 & \bf{659.90} & 
1.03 & 1.69\\SCA3-6 & 652.94 & 0.54 & 
654.90 & 0.75 & \bf{651.09} & 
0.28 & 0.59\\SCA3-7 & 666.15 & 0.94 & 
666.15 & 0.93 & \bf{659.17} & 
1.06 & 1.06\\SCA3-8 & \bf{719.47} & 0.97 & 
720.26 & 0.85 & 719.47 & 0.00
 & 0.11\\SCA3-9 & \bf{681.00} & 0.70 & 
685.82 & 0.79 & 681.00 & 0.00
 & 0.71\\SCA8-0 & 1002.06 & 0.76 & 
1006.55 & 0.79 & \bf{961.50} & 
4.22 & 4.69\\SCA8-1 & 1084.34 & 0.90 & 
1084.87 & 0.83 & \bf{1049.65} & 
3.30 & 3.36\\SCA8-2 & 1050.37 & 0.66 & 
1050.37 & 0.78 & \bf{1039.64} & 
1.03 & 1.03\\SCA8-3 & 1015.72 & 0.96 & 
1016.28 & 0.97 & \bf{983.34} & 
3.29 & 3.35\\SCA8-4 & 1078.07 & 0.98 & 
1079.56 & 0.95 & \bf{1065.49} & 
1.18 & 1.32\\SCA8-5 & 1057.68 & 0.97 & 
1057.68 & 0.95 & \bf{1027.08} & 
2.98 & 2.98\\SCA8-6 & 999.47 & 0.96 & 
999.47 & 0.93 & \bf{971.82} & 
2.85 & 2.85\\SCA8-7 & 1070.92 & 0.97 & 
1075.48 & 0.94 & \bf{1051.28} & 
1.87 & 2.30\\SCA8-8 & 1097.48 & 0.73 & 
1097.48 & 0.90 & \bf{1071.18} & 
2.46 & 2.46\\SCA8-9 & 1088.08 & 0.58 & 
1090.15 & 0.75 & \bf{1060.50} & 
2.60 & 2.80\\CON3-0 & 633.24 & 0.70 & 
633.71 & 0.83 & \bf{616.52} & 
2.71 & 2.79\\CON3-1 & 556.92 & 0.94 & 
557.88 & 0.94 & \bf{554.47} & 
0.44 & 0.61\\CON3-2 & 521.38 & 0.84 & 
521.38 & 0.90 & \bf{518.00} & 
0.65 & 0.65\\CON3-3 & 599.26 & 0.72 & 
606.74 & 0.81 & \bf{591.19} & 
1.37 & 2.63\\CON3-4 & 591.43 & 0.78 & 
591.43 & 0.82 & \bf{588.79} & 
0.45 & 0.45\\CON3-5 & 567.94 & 0.74 & 
567.94 & 0.83 & \bf{563.70} & 
0.75 & 0.75\\CON3-6 & 504.44 & 0.97 & 
506.11 & 0.99 & \bf{499.05} & 
1.08 & 1.41\\CON3-7 & 588.32 & 0.93 & 
588.34 & 0.92 & \bf{576.48} & 
2.05 & 2.06\\CON3-8 & 523.14 & 0.70 & 
524.45 & 0.83 & \bf{523.05} & 
0.02 & 0.27\\CON3-9 & 582.79 & 0.94 & 
586.54 & 0.94 & \bf{578.24} & 
0.79 & 1.44\\CON8-0 & 876.41 & 1.02 & 
882.33 & 0.94 & \bf{857.17} & 
2.24 & 2.94\\CON8-1 & 768.33 & 1.04 & 
774.83 & 0.93 & \bf{740.85} & 
3.71 & 4.59\\CON8-2 & 724.35 & 1.02 & 
724.35 & 0.98 & \bf{712.89} & 
1.61 & 1.61\\CON8-3 & 840.75 & 0.51 & 
844.73 & 0.73 & \bf{811.07} & 
3.66 & 4.15\\CON8-4 & 777.59 & 0.91 & 
789.87 & 0.90 & \bf{772.25} & 
0.69 & 2.28\\CON8-5 & 769.50 & 1.00 & 
774.01 & 0.88 & \bf{754.88} & 
1.94 & 2.53\\CON8-6 & 695.78 & 0.79 & 
699.57 & 0.92 & \bf{678.92} & 
2.48 & 3.04\\CON8-7 & 821.21 & 0.98 & 
824.25 & 0.86 & \bf{811.96} & 
1.14 & 1.51\\CON8-8 & 791.60 & 0.90 & 
791.60 & 0.94 & \bf{767.53} & 
3.14 & 3.14\\CON8-9 & 815.11 & 1.00 & 
815.11 & 0.84 & \bf{809.00} & 
0.76 & 0.76\\\bf{PROM.} & 
\bf{771.35} & \bf{0.84} & \bf{773.42} & \bf{0.87} & \bf{758.54} & \bf{1.54} & \bf{1.82}\\[1ex]\hline
\end{tabular}
\label{table:nonlin}
\end{table} \clearpage
\begin{table}[ht]
\caption{Resultados de la ejecución de la metaheurística SCA, utilizando instancias de SalhiNagy con la configuración -n 150.0 -b 10 -y .3}
\centering
\small
\begin{tabular}{c c c c c c c c}
\hline\hline
Instancia & Costo mínimo & Tiempo(seg.) & Costo promedio & Tiempo promedio(seg.) & CME & \%G & \%GP \\ [0.5ex]
\hline
CMT1X & 475.22 & 2.00 & 
476.89 & 2.40 & \bf{470.48} & 
1.01 & 1.36\\CMT1Y & 472.37 & 1.81 & 
472.75 & 1.76 & \bf{470.48} & 
0.40 & 0.48\\CMT2X & 705.17 & 15.97 & 
711.23 & 19.19 & \bf{682.39} & 
3.34 & 4.23\\CMT2Y & 707.35 & 20.19 & 
713.62 & 19.01 & \bf{682.39} & 
3.66 & 4.58\\CMT3X & 729.13 & 223.02 & 
732.96 & 87.68 & \bf{719.06} & 
1.40 & 1.93\\CMT3Y & 734.12 & 23.13 & 
738.11 & 28.32 & \bf{719.06} & 
2.09 & 2.65\\CMT4X & 877.36 & 285.08 & 
897.71 & 243.47 & \bf{854.21} & 
2.71 & 5.09\\CMT4Y & 893.32 & 247.77 & 
901.48 & 197.15 & \bf{852.46} & 
4.79 & 5.75\\CMT5X & 1106.07 & 1021.50 & 
1117.99 & 668.98 & \bf{1030.56} & 
7.33 & 8.48\\CMT5Y & 1096.96 & 649.61 & 
1112.41 & 749.78 & \bf{1031.69} & 
6.33 & 7.82\\CMT11X & 879.98 & 45.55 & 
888.96 & 51.03 & \bf{831.09} & 
5.88 & 6.96\\CMT11Y & 898.92 & 40.38 & 
903.43 & 37.47 & \bf{829.85} & 
8.32 & 8.87\\CMT12X & 685.45 & 107.20 & 
688.53 & 59.28 & \bf{658.83} & 
4.04 & 4.51\\CMT12Y & 680.38 & 43.95 & 
684.75 & 44.25 & \bf{660.47} & 
3.01 & 3.68\\\bf{PROM.} & 
\bf{781.56} & \bf{194.80} & \bf{788.63} & \bf{157.84} & \bf{749.50} & \bf{3.88} & \bf{4.74}\\[1ex]\hline
\end{tabular}
\label{table:nonlin}
\end{table} \clearpage
\begin{table}[ht]
\caption{Resultados de la ejecución de la metaheurística IGA, utilizando instancias de Dethloff con la configuración -n 100.0 -p 30.0 -cprob 40 -mprob 70}
\centering
\small
\begin{tabular}{c c c c c c c c}
\hline\hline
Instancia & Costo mínimo & Tiempo(seg.) & Costo promedio & Tiempo promedio(seg.) & CME & \%G & \%GP \\ [0.5ex]
\hline
SCA3-0 & 640.55 & 0.38 & 
640.55 & 0.37 & \bf{635.62} & 
0.78 & 0.78\\SCA3-1 & \bf{697.84} & 0.37 & 
700.69 & 0.39 & 697.84 & 0.00
 & 0.41\\SCA3-2 & 664.18 & 0.33 & 
664.18 & 0.39 & \bf{659.34} & 
0.73 & 0.73\\SCA3-3 & 681.35 & 0.35 & 
681.35 & 0.35 & \bf{680.04} & 
0.19 & 0.19\\SCA3-4 & \bf{690.50} & 0.37 & 
690.50 & 0.35 & 690.50 & 0.00
 & 0.00\\
SCA3-5 & 662.75 & 0.32 & 
667.97 & 0.37 & \bf{659.90} & 
0.43 & 1.22\\SCA3-6 & 658.00 & 0.33 & 
661.63 & 0.39 & \bf{651.09} & 
1.06 & 1.62\\SCA3-7 & 666.15 & 0.38 & 
669.40 & 0.37 & \bf{659.17} & 
1.06 & 1.55\\SCA3-8 & 724.29 & 0.36 & 
726.65 & 0.40 & \bf{719.47} & 
0.67 & 1.00\\SCA3-9 & 685.00 & 0.36 & 
685.29 & 0.34 & \bf{681.00} & 
0.59 & 0.63\\SCA8-0 & 1016.73 & 0.47 & 
1021.13 & 0.46 & \bf{961.50} & 
5.74 & 6.20\\SCA8-1 & 1073.28 & 0.32 & 
1073.28 & 0.42 & \bf{1049.65} & 
2.25 & 2.25\\SCA8-2 & 1054.85 & 0.52 & 
1054.85 & 0.40 & \bf{1039.64} & 
1.46 & 1.46\\SCA8-3 & 1020.81 & 0.32 & 
1020.81 & 0.44 & \bf{983.34} & 
3.81 & 3.81\\SCA8-4 & 1079.12 & 0.36 & 
1104.66 & 0.43 & \bf{1065.49} & 
1.28 & 3.68\\SCA8-5 & 1065.47 & 0.38 & 
1065.47 & 0.40 & \bf{1027.08} & 
3.74 & 3.74\\SCA8-6 & 981.37 & 0.38 & 
981.37 & 0.40 & \bf{971.82} & 
0.98 & 0.98\\SCA8-7 & 1083.29 & 0.30 & 
1083.29 & 0.34 & \bf{1051.28} & 
3.04 & 3.04\\SCA8-8 & 1095.58 & 0.43 & 
1095.58 & 0.37 & \bf{1071.18} & 
2.28 & 2.28\\SCA8-9 & 1073.73 & 0.35 & 
1075.97 & 0.37 & \bf{1060.50} & 
1.25 & 1.46\\CON3-0 & 621.82 & 0.33 & 
629.85 & 0.39 & \bf{616.52} & 
0.86 & 2.16\\CON3-1 & 560.75 & 0.33 & 
560.75 & 0.38 & \bf{554.47} & 
1.13 & 1.13\\CON3-2 & 521.38 & 0.42 & 
524.01 & 0.44 & \bf{518.00} & 
0.65 & 1.16\\CON3-3 & 609.27 & 0.35 & 
609.40 & 0.43 & \bf{591.19} & 
3.06 & 3.08\\CON3-4 & 599.90 & 0.50 & 
602.33 & 0.41 & \bf{588.79} & 
1.89 & 2.30\\CON3-5 & 568.69 & 0.36 & 
569.58 & 0.38 & \bf{563.70} & 
0.89 & 1.04\\CON3-6 & 504.44 & 0.46 & 
504.88 & 0.41 & \bf{499.05} & 
1.08 & 1.17\\CON3-7 & 580.14 & 0.37 & 
581.74 & 0.39 & \bf{576.48} & 
0.63 & 0.91\\CON3-8 & 524.59 & 0.37 & 
530.81 & 0.43 & \bf{523.05} & 
0.29 & 1.48\\CON3-9 & 589.00 & 0.40 & 
589.87 & 0.38 & \bf{578.24} & 
1.86 & 2.01\\CON8-0 & 884.18 & 0.41 & 
886.29 & 0.38 & \bf{857.17} & 
3.15 & 3.40\\CON8-1 & 754.25 & 0.43 & 
754.25 & 0.38 & \bf{740.85} & 
1.81 & 1.81\\CON8-2 & 716.73 & 0.52 & 
716.73 & 0.46 & \bf{712.89} & 
0.54 & 0.54\\CON8-3 & 839.75 & 0.53 & 
840.67 & 0.40 & \bf{811.07} & 
3.54 & 3.65\\CON8-4 & 791.64 & 0.33 & 
791.64 & 0.39 & \bf{772.25} & 
2.51 & 2.51\\CON8-5 & 774.08 & 0.50 & 
774.08 & 0.40 & \bf{754.88} & 
2.54 & 2.54\\CON8-6 & 699.45 & 0.34 & 
700.40 & 0.34 & \bf{678.92} & 
3.02 & 3.16\\CON8-7 & 827.28 & 0.42 & 
833.26 & 0.40 & \bf{811.96} & 
1.89 & 2.62\\CON8-8 & 794.54 & 0.41 & 
794.54 & 0.42 & \bf{767.53} & 
3.52 & 3.52\\CON8-9 & 831.84 & 0.34 & 
835.86 & 0.38 & \bf{809.00} & 
2.82 & 3.32\\\bf{PROM.} & 
\bf{772.71} & \bf{0.39} & \bf{774.89} & \bf{0.39} & \bf{758.54} & \bf{1.73} & \bf{2.01}\\[1ex]\hline
\end{tabular}
\label{table:nonlin}
\end{table} \clearpage
\begin{table}[ht]
\caption{Resultados de la ejecución de la metaheurística IGA, utilizando instancias de SalhiNagy con la configuración -n 100.0 -p 30.0 -cprob 90 -mprob 70}
\centering
\small
\begin{tabular}{c c c c c c c c}
\hline\hline
Instancia & Costo mínimo & Tiempo(seg.) & Costo promedio & Tiempo promedio(seg.) & CME & \%G & \%GP \\ [0.5ex]
\hline
CMT1X & 477.72 & 0.45 & 
481.66 & 0.39 & \bf{470.48} & 
1.54 & 2.38\\CMT1Y & 492.47 & 0.38 & 
496.62 & 0.34 & \bf{470.48} & 
4.67 & 5.56\\CMT2X & 699.07 & 0.77 & 
705.41 & 0.84 & \bf{682.39} & 
2.44 & 3.37\\CMT2Y & 715.74 & 0.91 & 
718.10 & 1.00 & \bf{682.39} & 
4.89 & 5.23\\CMT3X & 738.08 & 2.08 & 
746.20 & 2.24 & \bf{719.06} & 
2.65 & 3.77\\CMT3Y & 744.99 & 2.08 & 
751.76 & 2.02 & \bf{719.06} & 
3.61 & 4.55\\CMT4X & 904.06 & 5.68 & 
913.55 & 5.73 & \bf{854.21} & 
5.84 & 6.95\\CMT4Y & 902.59 & 5.88 & 
914.92 & 6.14 & \bf{852.46} & 
5.88 & 7.33\\CMT5X & 1108.59 & 12.76 & 
1117.84 & 12.64 & \bf{1030.56} & 
7.57 & 8.47\\CMT5Y & 1110.14 & 12.37 & 
1125.06 & 12.36 & \bf{1031.69} & 
7.60 & 9.05\\CMT11X & 901.99 & 3.98 & 
918.04 & 3.75 & \bf{831.09} & 
8.53 & 10.46\\CMT11Y & 907.49 & 4.12 & 
915.77 & 4.16 & \bf{829.85} & 
9.36 & 10.35\\CMT12X & 675.52 & 2.17 & 
679.40 & 2.16 & \bf{658.83} & 
2.53 & 3.12\\CMT12Y & 675.65 & 1.99 & 
679.63 & 2.09 & \bf{660.47} & 
2.30 & 2.90\\\bf{PROM.} & 
\bf{789.58} & \bf{3.97} & \bf{797.43} & \bf{3.99} & \bf{749.50} & \bf{4.96} & \bf{5.96}\\[1ex]\hline
\end{tabular}
\label{table:nonlin}
\end{table} \clearpage
\begin{table}[ht]
\caption{Resultados de la ejecución de la metaheurística IGA, utilizando instancias de Dethloff con la configuración -n 100.0 -p 40.0 -cprob 40 -mprob 70}
\centering
\small
\begin{tabular}{c c c c c c c c}
\hline\hline
Instancia & Costo mínimo & Tiempo(seg.) & Costo promedio & Tiempo promedio(seg.) & CME & \%G & \%GP \\ [0.5ex]
\hline
SCA3-0 & 641.64 & 0.61 & 
641.68 & 0.57 & \bf{635.62} & 
0.95 & 0.95\\SCA3-1 & 701.53 & 0.60 & 
701.78 & 0.58 & \bf{697.84} & 
0.53 & 0.56\\SCA3-2 & 664.18 & 0.43 & 
664.18 & 0.49 & \bf{659.34} & 
0.73 & 0.73\\SCA3-3 & 681.74 & 0.48 & 
682.92 & 0.47 & \bf{680.04} & 
0.25 & 0.42\\SCA3-4 & \bf{690.50} & 0.44 & 
690.50 & 0.49 & 690.50 & 0.00
 & 0.00\\
SCA3-5 & 668.48 & 0.56 & 
669.29 & 0.51 & \bf{659.90} & 
1.30 & 1.42\\SCA3-6 & 652.94 & 0.48 & 
654.27 & 0.50 & \bf{651.09} & 
0.28 & 0.49\\SCA3-7 & 666.15 & 0.52 & 
670.29 & 0.53 & \bf{659.17} & 
1.06 & 1.69\\SCA3-8 & 724.28 & 0.50 & 
729.61 & 0.51 & \bf{719.47} & 
0.67 & 1.41\\SCA3-9 & \bf{681.00} & 0.43 & 
686.76 & 0.50 & 681.00 & 0.00
 & 0.85\\SCA8-0 & 985.87 & 0.53 & 
988.76 & 0.50 & \bf{961.50} & 
2.53 & 2.84\\SCA8-1 & 1079.25 & 0.66 & 
1079.30 & 0.56 & \bf{1049.65} & 
2.82 & 2.82\\SCA8-2 & 1050.14 & 0.54 & 
1050.14 & 0.52 & \bf{1039.64} & 
1.01 & 1.01\\SCA8-3 & 1025.40 & 0.59 & 
1029.54 & 0.55 & \bf{983.34} & 
4.28 & 4.70\\SCA8-4 & 1105.32 & 0.40 & 
1105.32 & 0.51 & \bf{1065.49} & 
3.74 & 3.74\\SCA8-5 & 1054.99 & 0.49 & 
1054.99 & 0.50 & \bf{1027.08} & 
2.72 & 2.72\\SCA8-6 & 979.99 & 0.55 & 
979.99 & 0.54 & \bf{971.82} & 
0.84 & 0.84\\SCA8-7 & 1079.57 & 0.37 & 
1079.57 & 0.43 & \bf{1051.28} & 
2.69 & 2.69\\SCA8-8 & 1091.18 & 0.38 & 
1091.18 & 0.42 & \bf{1071.18} & 
1.87 & 1.87\\SCA8-9 & 1080.15 & 0.43 & 
1080.15 & 0.46 & \bf{1060.50} & 
1.85 & 1.85\\CON3-0 & 621.82 & 0.45 & 
622.18 & 0.51 & \bf{616.52} & 
0.86 & 0.92\\CON3-1 & 560.75 & 0.57 & 
561.03 & 0.55 & \bf{554.47} & 
1.13 & 1.18\\CON3-2 & 519.11 & 0.52 & 
520.25 & 0.60 & \bf{518.00} & 
0.21 & 0.43\\CON3-3 & 592.57 & 0.65 & 
597.25 & 0.62 & \bf{591.19} & 
0.23 & 1.03\\CON3-4 & 605.17 & 0.54 & 
605.71 & 0.49 & \bf{588.79} & 
2.78 & 2.87\\CON3-5 & 564.88 & 0.50 & 
566.96 & 0.47 & \bf{563.70} & 
0.21 & 0.58\\CON3-6 & 504.15 & 0.54 & 
506.20 & 0.61 & \bf{499.05} & 
1.02 & 1.43\\CON3-7 & 577.91 & 0.63 & 
581.89 & 0.60 & \bf{576.48} & 
0.25 & 0.94\\CON3-8 & 523.14 & 0.68 & 
525.60 & 0.64 & \bf{523.05} & 
0.02 & 0.49\\CON3-9 & 590.16 & 0.56 & 
591.14 & 0.56 & \bf{578.24} & 
2.06 & 2.23\\CON8-0 & 874.92 & 0.52 & 
874.92 & 0.49 & \bf{857.17} & 
2.07 & 2.07\\CON8-1 & 751.00 & 0.62 & 
755.59 & 0.57 & \bf{740.85} & 
1.37 & 1.99\\CON8-2 & 728.70 & 0.45 & 
729.52 & 0.47 & \bf{712.89} & 
2.22 & 2.33\\CON8-3 & 819.99 & 0.51 & 
822.97 & 0.51 & \bf{811.07} & 
1.10 & 1.47\\CON8-4 & 796.42 & 0.66 & 
806.76 & 0.58 & \bf{772.25} & 
3.13 & 4.47\\CON8-5 & 769.87 & 0.52 & 
782.12 & 0.51 & \bf{754.88} & 
1.99 & 3.61\\CON8-6 & 694.92 & 0.45 & 
697.37 & 0.47 & \bf{678.92} & 
2.36 & 2.72\\CON8-7 & 830.48 & 0.46 & 
833.33 & 0.47 & \bf{811.96} & 
2.28 & 2.63\\CON8-8 & 781.50 & 0.48 & 
781.50 & 0.50 & \bf{767.53} & 
1.82 & 1.82\\CON8-9 & 817.09 & 0.48 & 
827.43 & 0.50 & \bf{809.00} & 
1.00 & 2.28\\\bf{PROM.} & 
\bf{770.72} & \bf{0.52} & \bf{773.00} & \bf{0.52} & \bf{758.54} & \bf{1.46} & \bf{1.78}\\[1ex]\hline
\end{tabular}
\label{table:nonlin}
\end{table} \clearpage
\begin{table}[ht]
\caption{Resultados de la ejecución de la metaheurística IGA, utilizando instancias de SalhiNagy con la configuración -n 100.0 -p 40.0 -cprob 90 -mprob 70}
\centering
\small
\begin{tabular}{c c c c c c c c}
\hline\hline
Instancia & Costo mínimo & Tiempo(seg.) & Costo promedio & Tiempo promedio(seg.) & CME & \%G & \%GP \\ [0.5ex]
\hline
CMT1X & 480.44 & 0.58 & 
483.57 & 0.52 & \bf{470.48} & 
2.12 & 2.78\\CMT1Y & 479.45 & 0.53 & 
485.08 & 0.53 & \bf{470.48} & 
1.91 & 3.10\\CMT2X & 713.26 & 1.28 & 
717.18 & 1.35 & \bf{682.39} & 
4.52 & 5.10\\CMT2Y & 713.64 & 1.30 & 
719.26 & 1.24 & \bf{682.39} & 
4.58 & 5.40\\CMT3X & 745.71 & 2.95 & 
749.91 & 2.82 & \bf{719.06} & 
3.71 & 4.29\\CMT3Y & 744.59 & 2.90 & 
746.61 & 2.77 & \bf{719.06} & 
3.55 & 3.83\\CMT4X & 886.86 & 8.09 & 
910.75 & 7.99 & \bf{854.21} & 
3.82 & 6.62\\CMT4Y & 904.31 & 7.77 & 
915.14 & 7.56 & \bf{852.46} & 
6.08 & 7.35\\CMT5X & 1105.30 & 16.58 & 
1121.03 & 16.09 & \bf{1030.56} & 
7.25 & 8.78\\CMT5Y & 1110.85 & 16.10 & 
1123.17 & 16.18 & \bf{1031.69} & 
7.67 & 8.87\\CMT11X & 915.49 & 4.84 & 
921.68 & 4.84 & \bf{831.09} & 
10.16 & 10.90\\CMT11Y & 880.18 & 5.49 & 
888.38 & 5.34 & \bf{829.85} & 
6.06 & 7.05\\CMT12X & 673.83 & 2.82 & 
686.50 & 2.74 & \bf{658.83} & 
2.28 & 4.20\\CMT12Y & 683.86 & 2.73 & 
687.03 & 2.71 & \bf{660.47} & 
3.54 & 4.02\\\bf{PROM.} & 
\bf{788.41} & \bf{5.28} & \bf{796.81} & \bf{5.19} & \bf{749.50} & \bf{4.80} & \bf{5.88}\\[1ex]\hline
\end{tabular}
\label{table:nonlin}
\end{table} \clearpage
\begin{table}[ht]
\caption{Resultados de la ejecución de la metaheurística IGA, utilizando instancias de Dethloff con la configuración -n 100.0 -p 50.0 -cprob 40 -mprob 70}
\centering
\small
\begin{tabular}{c c c c c c c c}
\hline\hline
Instancia & Costo mínimo & Tiempo(seg.) & Costo promedio & Tiempo promedio(seg.) & CME & \%G & \%GP \\ [0.5ex]
\hline
SCA3-0 & 640.55 & 0.82 & 
640.55 & 0.67 & \bf{635.62} & 
0.78 & 0.78\\SCA3-1 & \bf{697.84} & 0.81 & 
700.30 & 0.76 & 697.84 & 0.00
 & 0.35\\SCA3-2 & 664.21 & 0.54 & 
664.21 & 0.56 & \bf{659.34} & 
0.74 & 0.74\\SCA3-3 & \bf{680.04} & 0.57 & 
680.60 & 0.66 & 680.04 & 0.00
 & 0.08\\SCA3-4 & \bf{690.50} & 0.53 & 
690.50 & 0.59 & 690.50 & 0.00
 & 0.00\\
SCA3-5 & 665.64 & 0.66 & 
665.64 & 0.69 & \bf{659.90} & 
0.87 & 0.87\\SCA3-6 & 653.93 & 0.57 & 
657.84 & 0.68 & \bf{651.09} & 
0.44 & 1.04\\SCA3-7 & 666.15 & 0.53 & 
666.15 & 0.59 & \bf{659.17} & 
1.06 & 1.06\\SCA3-8 & 723.99 & 0.64 & 
724.71 & 0.72 & \bf{719.47} & 
0.63 & 0.73\\SCA3-9 & 684.44 & 0.52 & 
685.00 & 0.61 & \bf{681.00} & 
0.51 & 0.59\\SCA8-0 & 1011.05 & 0.52 & 
1011.05 & 0.60 & \bf{961.50} & 
5.15 & 5.15\\SCA8-1 & 1065.36 & 0.54 & 
1072.97 & 0.61 & \bf{1049.65} & 
1.50 & 2.22\\SCA8-2 & 1050.37 & 0.47 & 
1050.37 & 0.54 & \bf{1039.64} & 
1.03 & 1.03\\SCA8-3 & 1031.15 & 0.50 & 
1031.15 & 0.54 & \bf{983.34} & 
4.86 & 4.86\\SCA8-4 & 1091.41 & 0.64 & 
1091.41 & 0.62 & \bf{1065.49} & 
2.43 & 2.43\\SCA8-5 & 1053.99 & 0.66 & 
1054.16 & 0.63 & \bf{1027.08} & 
2.62 & 2.64\\SCA8-6 & 981.41 & 0.71 & 
990.60 & 0.71 & \bf{971.82} & 
0.99 & 1.93\\SCA8-7 & 1067.49 & 0.46 & 
1067.59 & 0.56 & \bf{1051.28} & 
1.54 & 1.55\\SCA8-8 & 1090.51 & 0.55 & 
1090.51 & 0.67 & \bf{1071.18} & 
1.80 & 1.80\\SCA8-9 & 1075.30 & 0.72 & 
1079.24 & 0.66 & \bf{1060.50} & 
1.40 & 1.77\\CON3-0 & 623.97 & 0.60 & 
628.54 & 0.59 & \bf{616.52} & 
1.21 & 1.95\\CON3-1 & 561.87 & 0.60 & 
564.86 & 0.66 & \bf{554.47} & 
1.33 & 1.87\\CON3-2 & 521.38 & 0.76 & 
522.55 & 0.77 & \bf{518.00} & 
0.65 & 0.88\\CON3-3 & 594.11 & 0.80 & 
603.33 & 0.69 & \bf{591.19} & 
0.49 & 2.05\\CON3-4 & 592.58 & 0.73 & 
592.58 & 0.59 & \bf{588.79} & 
0.64 & 0.64\\CON3-5 & 567.94 & 0.83 & 
568.87 & 0.74 & \bf{563.70} & 
0.75 & 0.92\\CON3-6 & 501.33 & 0.90 & 
505.06 & 0.72 & \bf{499.05} & 
0.46 & 1.20\\CON3-7 & 586.01 & 0.50 & 
587.98 & 0.53 & \bf{576.48} & 
1.65 & 1.99\\CON3-8 & 524.38 & 0.58 & 
527.71 & 0.64 & \bf{523.05} & 
0.25 & 0.89\\CON3-9 & 582.79 & 0.89 & 
586.89 & 0.74 & \bf{578.24} & 
0.79 & 1.50\\CON8-0 & 874.24 & 0.63 & 
881.22 & 0.74 & \bf{857.17} & 
1.99 & 2.81\\CON8-1 & 750.26 & 0.92 & 
757.84 & 0.76 & \bf{740.85} & 
1.27 & 2.29\\CON8-2 & 723.69 & 0.76 & 
727.69 & 0.72 & \bf{712.89} & 
1.51 & 2.08\\CON8-3 & 832.11 & 0.74 & 
832.11 & 0.66 & \bf{811.07} & 
2.59 & 2.59\\CON8-4 & 781.62 & 0.57 & 
781.62 & 0.56 & \bf{772.25} & 
1.21 & 1.21\\CON8-5 & 765.54 & 0.56 & 
766.86 & 0.71 & \bf{754.88} & 
1.41 & 1.59\\CON8-6 & 701.80 & 0.56 & 
705.27 & 0.69 & \bf{678.92} & 
3.37 & 3.88\\CON8-7 & 814.79 & 0.56 & 
820.99 & 0.62 & \bf{811.96} & 
0.35 & 1.11\\CON8-8 & 787.90 & 0.58 & 
787.90 & 0.63 & \bf{767.53} & 
2.65 & 2.65\\CON8-9 & 815.91 & 0.59 & 
830.43 & 0.59 & \bf{809.00} & 
0.85 & 2.65\\\bf{PROM.} & 
\bf{769.74} & \bf{0.64} & \bf{772.37} & \bf{0.65} & \bf{758.54} & \bf{1.34} & \bf{1.71}\\[1ex]\hline
\end{tabular}
\label{table:nonlin}
\end{table} \clearpage
\begin{table}[ht]
\caption{Resultados de la ejecución de la metaheurística IGA, utilizando instancias de SalhiNagy con la configuración -n 100.0 -p 50.0 -cprob 90 -mprob 70}
\centering
\small
\begin{tabular}{c c c c c c c c}
\hline\hline
Instancia & Costo mínimo & Tiempo(seg.) & Costo promedio & Tiempo promedio(seg.) & CME & \%G & \%GP \\ [0.5ex]
\hline
CMT1X & 479.41 & 0.70 & 
483.97 & 0.67 & \bf{470.48} & 
1.90 & 2.87\\CMT1Y & 481.00 & 0.68 & 
481.00 & 0.59 & \bf{470.48} & 
2.24 & 2.24\\CMT2X & 708.58 & 1.57 & 
717.29 & 1.58 & \bf{682.39} & 
3.84 & 5.11\\CMT2Y & 710.91 & 1.59 & 
714.05 & 1.57 & \bf{682.39} & 
4.18 & 4.64\\CMT3X & 734.87 & 3.60 & 
742.31 & 3.50 & \bf{719.06} & 
2.20 & 3.23\\CMT3Y & 737.62 & 3.22 & 
745.72 & 3.46 & \bf{719.06} & 
2.58 & 3.71\\CMT4X & 891.64 & 9.47 & 
909.48 & 9.64 & \bf{854.21} & 
4.38 & 6.47\\CMT4Y & 903.92 & 9.77 & 
910.55 & 9.88 & \bf{852.46} & 
6.04 & 6.81\\CMT5X & 1117.27 & 20.27 & 
1127.15 & 20.31 & \bf{1030.56} & 
8.41 & 9.37\\CMT5Y & 1091.49 & 19.00 & 
1117.63 & 20.03 & \bf{1031.69} & 
5.80 & 8.33\\CMT11X & 892.95 & 6.06 & 
905.52 & 5.80 & \bf{831.09} & 
7.44 & 8.96\\CMT11Y & 886.37 & 6.37 & 
916.95 & 6.44 & \bf{829.85} & 
6.81 & 10.50\\CMT12X & 674.61 & 3.46 & 
675.90 & 3.59 & \bf{658.83} & 
2.40 & 2.59\\CMT12Y & 675.22 & 3.48 & 
676.33 & 3.37 & \bf{660.47} & 
2.23 & 2.40\\\bf{PROM.} & 
\bf{784.70} & \bf{6.37} & \bf{794.56} & \bf{6.46} & \bf{749.50} & \bf{4.32} & \bf{5.52}\\[1ex]\hline
\end{tabular}
\label{table:nonlin}
\end{table} \clearpage
\begin{table}[ht]
\caption{Resultados de la ejecución de la metaheurística IGA, utilizando instancias de Dethloff con la configuración -n 100.0 -p 60.0 -cprob 40 -mprob 70}
\centering
\small
\begin{tabular}{c c c c c c c c}
\hline\hline
Instancia & Costo mínimo & Tiempo(seg.) & Costo promedio & Tiempo promedio(seg.) & CME & \%G & \%GP \\ [0.5ex]
\hline
SCA3-0 & 640.55 & 0.88 & 
641.89 & 0.91 & \bf{635.62} & 
0.78 & 0.99\\SCA3-1 & \bf{697.84} & 0.92 & 
697.84 & 0.88 & 697.84 & 0.00
 & 0.00\\
SCA3-2 & 664.18 & 0.67 & 
664.20 & 0.73 & \bf{659.34} & 
0.73 & 0.74\\SCA3-3 & 681.16 & 0.69 & 
681.25 & 0.76 & \bf{680.04} & 
0.16 & 0.18\\SCA3-4 & \bf{690.50} & 0.81 & 
690.50 & 0.80 & 690.50 & 0.00
 & 0.00\\
SCA3-5 & 668.48 & 0.67 & 
669.75 & 0.84 & \bf{659.90} & 
1.30 & 1.49\\SCA3-6 & 652.94 & 0.70 & 
655.09 & 0.85 & \bf{651.09} & 
0.28 & 0.61\\SCA3-7 & 666.60 & 0.74 & 
667.08 & 0.76 & \bf{659.17} & 
1.13 & 1.20\\SCA3-8 & \bf{719.47} & 0.68 & 
727.15 & 0.80 & 719.47 & 0.00
 & 1.07\\SCA3-9 & \bf{681.00} & 0.81 & 
681.00 & 0.86 & 681.00 & 0.00
 & 0.00\\
SCA8-0 & 991.60 & 0.62 & 
997.44 & 0.72 & \bf{961.50} & 
3.13 & 3.74\\SCA8-1 & 1079.76 & 0.78 & 
1083.70 & 0.81 & \bf{1049.65} & 
2.87 & 3.24\\SCA8-2 & 1051.42 & 0.84 & 
1053.63 & 0.74 & \bf{1039.64} & 
1.13 & 1.35\\SCA8-3 & 1022.16 & 0.82 & 
1022.16 & 0.73 & \bf{983.34} & 
3.95 & 3.95\\SCA8-4 & 1075.27 & 0.59 & 
1075.27 & 0.76 & \bf{1065.49} & 
0.92 & 0.92\\SCA8-5 & 1049.22 & 0.80 & 
1049.81 & 0.68 & \bf{1027.08} & 
2.16 & 2.21\\SCA8-6 & 981.37 & 0.64 & 
981.37 & 0.64 & \bf{971.82} & 
0.98 & 0.98\\SCA8-7 & 1081.71 & 0.83 & 
1081.91 & 0.68 & \bf{1051.28} & 
2.89 & 2.91\\SCA8-8 & \bf{1071.18} & 0.86 & 
1071.18 & 0.82 & 1071.18 & 0.00
 & 0.00\\
SCA8-9 & 1069.83 & 0.85 & 
1072.03 & 0.77 & \bf{1060.50} & 
0.88 & 1.09\\CON3-0 & 619.09 & 0.72 & 
622.90 & 0.79 & \bf{616.52} & 
0.42 & 1.04\\CON3-1 & 560.32 & 0.86 & 
560.32 & 0.82 & \bf{554.47} & 
1.06 & 1.06\\CON3-2 & 521.38 & 0.94 & 
521.75 & 0.90 & \bf{518.00} & 
0.65 & 0.72\\CON3-3 & 599.26 & 0.80 & 
600.53 & 0.79 & \bf{591.19} & 
1.37 & 1.58\\CON3-4 & 592.58 & 0.74 & 
600.59 & 0.71 & \bf{588.79} & 
0.64 & 2.00\\CON3-5 & \bf{563.70} & 0.73 & 
567.45 & 0.75 & 563.70 & 0.00
 & 0.67\\CON3-6 & 506.70 & 0.76 & 
507.10 & 0.81 & \bf{499.05} & 
1.53 & 1.61\\CON3-7 & 586.01 & 0.98 & 
586.01 & 0.73 & \bf{576.48} & 
1.65 & 1.65\\CON3-8 & 532.86 & 1.10 & 
533.47 & 0.87 & \bf{523.05} & 
1.88 & 1.99\\CON3-9 & 582.79 & 0.78 & 
584.12 & 0.82 & \bf{578.24} & 
0.79 & 1.02\\CON8-0 & 869.15 & 0.72 & 
869.15 & 0.77 & \bf{857.17} & 
1.40 & 1.40\\CON8-1 & 754.95 & 0.76 & 
764.63 & 0.77 & \bf{740.85} & 
1.90 & 3.21\\CON8-2 & 722.22 & 0.89 & 
722.22 & 0.83 & \bf{712.89} & 
1.31 & 1.31\\CON8-3 & 834.08 & 0.72 & 
834.08 & 0.78 & \bf{811.07} & 
2.84 & 2.84\\CON8-4 & 777.48 & 0.67 & 
789.07 & 0.78 & \bf{772.25} & 
0.68 & 2.18\\CON8-5 & 766.55 & 0.82 & 
769.25 & 0.77 & \bf{754.88} & 
1.55 & 1.90\\CON8-6 & 693.17 & 0.74 & 
701.43 & 0.72 & \bf{678.92} & 
2.10 & 3.32\\CON8-7 & 815.14 & 0.63 & 
815.14 & 0.73 & \bf{811.96} & 
0.39 & 0.39\\CON8-8 & 799.91 & 0.86 & 
801.54 & 0.76 & \bf{767.53} & 
4.22 & 4.43\\CON8-9 & 835.68 & 0.70 & 
837.35 & 0.83 & \bf{809.00} & 
3.30 & 3.50\\\bf{PROM.} & 
\bf{769.23} & \bf{0.78} & \bf{771.31} & \bf{0.78} & \bf{758.54} & \bf{1.32} & \bf{1.61}\\[1ex]\hline
\end{tabular}
\label{table:nonlin}
\end{table} \clearpage
\begin{table}[ht]
\caption{Resultados de la ejecución de la metaheurística IGA, utilizando instancias de SalhiNagy con la configuración -n 100.0 -p 60.0 -cprob 90 -mprob 70}
\centering
\small
\begin{tabular}{c c c c c c c c}
\hline\hline
Instancia & Costo mínimo & Tiempo(seg.) & Costo promedio & Tiempo promedio(seg.) & CME & \%G & \%GP \\ [0.5ex]
\hline
CMT1X & 481.32 & 0.90 & 
483.95 & 0.77 & \bf{470.48} & 
2.30 & 2.86\\CMT1Y & 473.58 & 0.49 & 
480.12 & 0.77 & \bf{470.48} & 
0.66 & 2.05\\CMT2X & 719.69 & 1.95 & 
722.87 & 1.88 & \bf{682.39} & 
5.47 & 5.93\\CMT2Y & 692.73 & 1.94 & 
701.48 & 1.82 & \bf{682.39} & 
1.52 & 2.80\\CMT3X & 729.88 & 4.04 & 
742.71 & 4.35 & \bf{719.06} & 
1.50 & 3.29\\CMT3Y & 739.44 & 4.26 & 
747.12 & 4.36 & \bf{719.06} & 
2.83 & 3.90\\CMT4X & 901.95 & 11.37 & 
909.82 & 12.12 & \bf{854.21} & 
5.59 & 6.51\\CMT4Y & 902.09 & 12.10 & 
913.49 & 11.65 & \bf{852.46} & 
5.82 & 7.16\\CMT5X & 1097.16 & 24.20 & 
1105.74 & 23.91 & \bf{1030.56} & 
6.46 & 7.30\\CMT5Y & 1094.09 & 24.79 & 
1114.39 & 24.34 & \bf{1031.69} & 
6.05 & 8.02\\CMT11X & 898.11 & 7.06 & 
904.74 & 7.04 & \bf{831.09} & 
8.06 & 8.86\\CMT11Y & 858.12 & 8.02 & 
887.07 & 7.96 & \bf{829.85} & 
3.41 & 6.89\\CMT12X & 676.79 & 4.27 & 
681.61 & 4.12 & \bf{658.83} & 
2.73 & 3.46\\CMT12Y & 675.05 & 4.00 & 
679.22 & 4.29 & \bf{660.47} & 
2.21 & 2.84\\\bf{PROM.} & 
\bf{781.43} & \bf{7.81} & \bf{791.02} & \bf{7.81} & \bf{749.50} & \bf{3.90} & \bf{5.13}\\[1ex]\hline
\end{tabular}
\label{table:nonlin}
\end{table} \clearpage
\begin{table}[ht]
\caption{Resultados de la ejecución de la metaheurística IGA, utilizando instancias de Dethloff con la configuración -n 100.0 -p 70.0 -cprob 40 -mprob 70}
\centering
\small
\begin{tabular}{c c c c c c c c}
\hline\hline
Instancia & Costo mínimo & Tiempo(seg.) & Costo promedio & Tiempo promedio(seg.) & CME & \%G & \%GP \\ [0.5ex]
\hline
SCA3-0 & 640.55 & 0.94 & 
640.55 & 1.00 & \bf{635.62} & 
0.78 & 0.78\\SCA3-1 & \bf{697.84} & 0.77 & 
699.17 & 0.93 & 697.84 & 0.00
 & 0.19\\SCA3-2 & 659.86 & 0.80 & 
660.81 & 0.78 & \bf{659.34} & 
0.08 & 0.22\\SCA3-3 & 681.16 & 1.12 & 
682.46 & 0.98 & \bf{680.04} & 
0.16 & 0.36\\SCA3-4 & \bf{690.50} & 0.74 & 
690.50 & 0.86 & 690.50 & 0.00
 & 0.00\\
SCA3-5 & 665.64 & 1.04 & 
668.99 & 0.91 & \bf{659.90} & 
0.87 & 1.38\\SCA3-6 & 652.94 & 1.15 & 
652.94 & 0.90 & \bf{651.09} & 
0.28 & 0.28\\SCA3-7 & \bf{659.17} & 0.80 & 
664.53 & 0.94 & 659.17 & 0.00
 & 0.81\\SCA3-8 & 723.99 & 1.13 & 
723.99 & 0.85 & \bf{719.47} & 
0.63 & 0.63\\SCA3-9 & \bf{681.00} & 0.91 & 
684.14 & 0.86 & 681.00 & 0.00
 & 0.46\\SCA8-0 & 973.22 & 0.79 & 
987.59 & 0.85 & \bf{961.50} & 
1.22 & 2.71\\SCA8-1 & 1079.85 & 0.86 & 
1079.85 & 0.95 & \bf{1049.65} & 
2.88 & 2.88\\SCA8-2 & 1050.37 & 1.20 & 
1050.37 & 0.97 & \bf{1039.64} & 
1.03 & 1.03\\SCA8-3 & 1023.11 & 0.73 & 
1025.86 & 0.81 & \bf{983.34} & 
4.04 & 4.32\\SCA8-4 & 1077.80 & 0.80 & 
1077.80 & 0.75 & \bf{1065.49} & 
1.16 & 1.16\\SCA8-5 & 1050.97 & 0.93 & 
1056.82 & 0.85 & \bf{1027.08} & 
2.33 & 2.90\\SCA8-6 & 972.48 & 0.81 & 
975.25 & 0.84 & \bf{971.82} & 
0.07 & 0.35\\SCA8-7 & 1077.89 & 0.66 & 
1080.95 & 0.91 & \bf{1051.28} & 
2.53 & 2.82\\SCA8-8 & 1091.18 & 0.80 & 
1091.18 & 0.98 & \bf{1071.18} & 
1.87 & 1.87\\SCA8-9 & 1068.10 & 0.87 & 
1068.10 & 0.89 & \bf{1060.50} & 
0.72 & 0.72\\CON3-0 & 620.76 & 1.18 & 
621.81 & 0.97 & \bf{616.52} & 
0.69 & 0.86\\CON3-1 & 559.72 & 0.94 & 
560.26 & 0.91 & \bf{554.47} & 
0.95 & 1.04\\CON3-2 & 521.38 & 0.86 & 
523.13 & 0.96 & \bf{518.00} & 
0.65 & 0.99\\CON3-3 & 591.36 & 0.98 & 
591.36 & 0.86 & \bf{591.19} & 
0.03 & 0.03\\CON3-4 & 592.58 & 0.70 & 
592.88 & 0.82 & \bf{588.79} & 
0.64 & 0.69\\CON3-5 & 566.96 & 0.97 & 
568.34 & 1.03 & \bf{563.70} & 
0.58 & 0.82\\CON3-6 & 502.26 & 0.88 & 
503.49 & 0.98 & \bf{499.05} & 
0.64 & 0.89\\CON3-7 & 586.01 & 0.85 & 
586.01 & 0.92 & \bf{576.48} & 
1.65 & 1.65\\CON3-8 & 524.59 & 1.16 & 
530.74 & 1.10 & \bf{523.05} & 
0.29 & 1.47\\CON3-9 & 588.11 & 0.80 & 
589.25 & 0.91 & \bf{578.24} & 
1.71 & 1.90\\CON8-0 & 885.62 & 0.81 & 
887.68 & 0.99 & \bf{857.17} & 
3.32 & 3.56\\CON8-1 & 760.13 & 0.95 & 
760.13 & 0.91 & \bf{740.85} & 
2.60 & 2.60\\CON8-2 & 720.12 & 1.00 & 
723.36 & 1.03 & \bf{712.89} & 
1.01 & 1.47\\CON8-3 & 838.58 & 1.06 & 
840.62 & 0.98 & \bf{811.07} & 
3.39 & 3.64\\CON8-4 & 782.46 & 0.70 & 
782.46 & 0.77 & \bf{772.25} & 
1.32 & 1.32\\CON8-5 & 773.05 & 1.22 & 
773.38 & 1.13 & \bf{754.88} & 
2.41 & 2.45\\CON8-6 & 693.25 & 1.00 & 
700.26 & 1.01 & \bf{678.92} & 
2.11 & 3.14\\CON8-7 & 814.79 & 0.81 & 
814.92 & 0.79 & \bf{811.96} & 
0.35 & 0.37\\CON8-8 & 784.82 & 1.22 & 
796.81 & 1.11 & \bf{767.53} & 
2.25 & 3.82\\CON8-9 & 818.54 & 0.86 & 
818.54 & 1.05 & \bf{809.00} & 
1.18 & 1.18\\\bf{PROM.} & 
\bf{768.57} & \bf{0.92} & \bf{770.68} & \bf{0.93} & \bf{758.54} & \bf{1.21} & \bf{1.49}\\[1ex]\hline
\end{tabular}
\label{table:nonlin}
\end{table} \clearpage
\begin{table}[ht]
\caption{Resultados de la ejecución de la metaheurística IGA, utilizando instancias de SalhiNagy con la configuración -n 100.0 -p 70.0 -cprob 90 -mprob 70}
\centering
\small
\begin{tabular}{c c c c c c c c}
\hline\hline
Instancia & Costo mínimo & Tiempo(seg.) & Costo promedio & Tiempo promedio(seg.) & CME & \%G & \%GP \\ [0.5ex]
\hline
CMT1X & 476.66 & 1.00 & 
480.24 & 0.97 & \bf{470.48} & 
1.31 & 2.07\\CMT1Y & 476.99 & 0.80 & 
482.23 & 0.84 & \bf{470.48} & 
1.38 & 2.50\\CMT2X & 704.55 & 1.96 & 
706.83 & 2.09 & \bf{682.39} & 
3.25 & 3.58\\CMT2Y & 696.84 & 1.98 & 
709.01 & 1.88 & \bf{682.39} & 
2.12 & 3.90\\CMT3X & 731.12 & 4.42 & 
740.97 & 4.87 & \bf{719.06} & 
1.68 & 3.05\\CMT3Y & 726.86 & 4.54 & 
735.48 & 4.80 & \bf{719.06} & 
1.08 & 2.28\\CMT4X & 894.00 & 13.68 & 
905.27 & 13.55 & \bf{854.21} & 
4.66 & 5.98\\CMT4Y & 896.89 & 14.30 & 
909.63 & 13.77 & \bf{852.46} & 
5.21 & 6.71\\CMT5X & 1115.74 & 26.72 & 
1121.79 & 26.36 & \bf{1030.56} & 
8.27 & 8.85\\CMT5Y & 1101.01 & 28.64 & 
1113.11 & 28.18 & \bf{1031.69} & 
6.72 & 7.89\\CMT11X & 900.45 & 8.39 & 
911.40 & 8.26 & \bf{831.09} & 
8.35 & 9.66\\CMT11Y & 843.80 & 9.15 & 
881.36 & 9.18 & \bf{829.85} & 
1.68 & 6.21\\CMT12X & 674.11 & 5.29 & 
680.35 & 5.04 & \bf{658.83} & 
2.32 & 3.27\\CMT12Y & 675.75 & 5.05 & 
676.17 & 5.05 & \bf{660.47} & 
2.31 & 2.38\\\bf{PROM.} & 
\bf{779.63} & \bf{8.99} & \bf{789.56} & \bf{8.92} & \bf{749.50} & \bf{3.60} & \bf{4.88}\\[1ex]\hline
\end{tabular}
\label{table:nonlin}
\end{table} \clearpage
\begin{table}[ht]
\caption{Resultados de la ejecución de la metaheurística IGA, utilizando instancias de Dethloff con la configuración -n 100.0 -p 80.0 -cprob 40 -mprob 70}
\centering
\small
\begin{tabular}{c c c c c c c c}
\hline\hline
Instancia & Costo mínimo & Tiempo(seg.) & Costo promedio & Tiempo promedio(seg.) & CME & \%G & \%GP \\ [0.5ex]
\hline
SCA3-0 & 640.55 & 1.02 & 
641.71 & 1.01 & \bf{635.62} & 
0.78 & 0.96\\SCA3-1 & \bf{697.84} & 0.89 & 
702.22 & 0.91 & 697.84 & 0.00
 & 0.63\\SCA3-2 & 664.21 & 0.80 & 
665.73 & 0.88 & \bf{659.34} & 
0.74 & 0.97\\SCA3-3 & \bf{680.04} & 0.89 & 
680.60 & 1.01 & 680.04 & 0.00
 & 0.08\\SCA3-4 & \bf{690.50} & 1.28 & 
690.50 & 1.05 & 690.50 & 0.00
 & 0.00\\
SCA3-5 & 662.75 & 1.11 & 
664.92 & 1.06 & \bf{659.90} & 
0.43 & 0.76\\SCA3-6 & 652.94 & 0.96 & 
654.01 & 0.98 & \bf{651.09} & 
0.28 & 0.45\\SCA3-7 & 666.15 & 1.28 & 
666.15 & 1.16 & \bf{659.17} & 
1.06 & 1.06\\SCA3-8 & 723.99 & 1.36 & 
724.75 & 1.13 & \bf{719.47} & 
0.63 & 0.73\\SCA3-9 & \bf{681.00} & 1.03 & 
681.17 & 0.97 & 681.00 & 0.00
 & 0.02\\SCA8-0 & 995.96 & 1.34 & 
999.93 & 1.13 & \bf{961.50} & 
3.58 & 4.00\\SCA8-1 & 1080.38 & 0.83 & 
1080.38 & 0.98 & \bf{1049.65} & 
2.93 & 2.93\\SCA8-2 & 1053.78 & 0.81 & 
1054.16 & 1.03 & \bf{1039.64} & 
1.36 & 1.40\\SCA8-3 & 1013.77 & 0.85 & 
1014.24 & 1.03 & \bf{983.34} & 
3.09 & 3.14\\SCA8-4 & 1074.78 & 0.99 & 
1074.78 & 0.91 & \bf{1065.49} & 
0.87 & 0.87\\SCA8-5 & 1053.36 & 1.10 & 
1053.36 & 1.05 & \bf{1027.08} & 
2.56 & 2.56\\SCA8-6 & 979.29 & 1.04 & 
982.62 & 0.95 & \bf{971.82} & 
0.77 & 1.11\\SCA8-7 & 1071.47 & 0.86 & 
1072.25 & 0.95 & \bf{1051.28} & 
1.92 & 1.99\\SCA8-8 & \bf{1071.18} & 0.98 & 
1085.77 & 1.10 & 1071.18 & 0.00
 & 1.36\\SCA8-9 & 1075.50 & 0.98 & 
1079.91 & 1.06 & \bf{1060.50} & 
1.41 & 1.83\\CON3-0 & 624.96 & 1.06 & 
629.50 & 1.04 & \bf{616.52} & 
1.37 & 2.11\\CON3-1 & 558.16 & 1.16 & 
560.10 & 1.00 & \bf{554.47} & 
0.67 & 1.02\\CON3-2 & 521.38 & 1.05 & 
521.38 & 1.17 & \bf{518.00} & 
0.65 & 0.65\\CON3-3 & 591.36 & 0.94 & 
597.12 & 1.05 & \bf{591.19} & 
0.03 & 1.00\\CON3-4 & 592.58 & 1.04 & 
593.08 & 1.13 & \bf{588.79} & 
0.64 & 0.73\\CON3-5 & \bf{563.70} & 0.96 & 
566.02 & 1.04 & 563.70 & 0.00
 & 0.41\\CON3-6 & 504.07 & 1.31 & 
504.43 & 1.17 & \bf{499.05} & 
1.01 & 1.08\\CON3-7 & 578.41 & 0.86 & 
581.04 & 1.11 & \bf{576.48} & 
0.33 & 0.79\\CON3-8 & 524.38 & 1.26 & 
525.04 & 1.16 & \bf{523.05} & 
0.25 & 0.38\\CON3-9 & 588.11 & 1.19 & 
588.11 & 1.17 & \bf{578.24} & 
1.71 & 1.71\\CON8-0 & 874.34 & 0.98 & 
874.34 & 1.03 & \bf{857.17} & 
2.00 & 2.00\\CON8-1 & 759.66 & 1.02 & 
762.57 & 1.15 & \bf{740.85} & 
2.54 & 2.93\\CON8-2 & 722.22 & 1.15 & 
723.11 & 1.23 & \bf{712.89} & 
1.31 & 1.43\\CON8-3 & 830.61 & 0.99 & 
833.62 & 0.96 & \bf{811.07} & 
2.41 & 2.78\\CON8-4 & 780.03 & 1.34 & 
780.03 & 1.13 & \bf{772.25} & 
1.01 & 1.01\\CON8-5 & 769.52 & 1.12 & 
769.88 & 1.07 & \bf{754.88} & 
1.94 & 1.99\\CON8-6 & 686.26 & 0.85 & 
691.02 & 1.10 & \bf{678.92} & 
1.08 & 1.78\\CON8-7 & 825.23 & 0.99 & 
825.70 & 0.97 & \bf{811.96} & 
1.63 & 1.69\\CON8-8 & 780.80 & 1.05 & 
780.80 & 1.09 & \bf{767.53} & 
1.73 & 1.73\\CON8-9 & 821.80 & 1.17 & 
821.80 & 1.17 & \bf{809.00} & 
1.58 & 1.58\\\bf{PROM.} & 
\bf{768.18} & \bf{1.05} & \bf{769.95} & \bf{1.06} & \bf{758.54} & \bf{1.16} & \bf{1.39}\\[1ex]\hline
\end{tabular}
\label{table:nonlin}
\end{table} \clearpage
\begin{table}[ht]
\caption{Resultados de la ejecución de la metaheurística IGA, utilizando instancias de SalhiNagy con la configuración -n 100.0 -p 80.0 -cprob 90 -mprob 70}
\centering
\small
\begin{tabular}{c c c c c c c c}
\hline\hline
Instancia & Costo mínimo & Tiempo(seg.) & Costo promedio & Tiempo promedio(seg.) & CME & \%G & \%GP \\ [0.5ex]
\hline
CMT1X & 478.84 & 1.14 & 
482.42 & 1.10 & \bf{470.48} & 
1.78 & 2.54\\CMT1Y & 478.82 & 0.93 & 
480.30 & 1.00 & \bf{470.48} & 
1.77 & 2.09\\CMT2X & 717.44 & 2.46 & 
717.77 & 2.52 & \bf{682.39} & 
5.14 & 5.18\\CMT2Y & 705.90 & 2.41 & 
707.66 & 2.25 & \bf{682.39} & 
3.45 & 3.70\\CMT3X & 736.42 & 5.35 & 
741.60 & 5.54 & \bf{719.06} & 
2.41 & 3.13\\CMT3Y & 737.57 & 5.62 & 
740.30 & 5.81 & \bf{719.06} & 
2.57 & 2.95\\CMT4X & 899.84 & 14.83 & 
904.97 & 14.80 & \bf{854.21} & 
5.34 & 5.94\\CMT4Y & 901.00 & 15.65 & 
907.55 & 15.56 & \bf{852.46} & 
5.69 & 6.46\\CMT5X & 1084.29 & 31.54 & 
1111.27 & 32.27 & \bf{1030.56} & 
5.21 & 7.83\\CMT5Y & 1103.43 & 32.54 & 
1122.35 & 31.84 & \bf{1031.69} & 
6.95 & 8.79\\CMT11X & 901.75 & 9.74 & 
912.80 & 9.53 & \bf{831.09} & 
8.50 & 9.83\\CMT11Y & 850.11 & 10.62 & 
894.21 & 10.99 & \bf{829.85} & 
2.44 & 7.76\\CMT12X & 674.52 & 5.62 & 
677.68 & 6.18 & \bf{658.83} & 
2.38 & 2.86\\CMT12Y & 675.72 & 5.40 & 
679.08 & 5.43 & \bf{660.47} & 
2.31 & 2.82\\\bf{PROM.} & 
\bf{781.83} & \bf{10.28} & \bf{791.43} & \bf{10.35} & \bf{749.50} & \bf{4.00} & \bf{5.14}\\[1ex]\hline
\end{tabular}
\label{table:nonlin}
\end{table} \clearpage
\begin{table}[ht]
\caption{Resultados de la ejecución de la metaheurística IGA, utilizando instancias de Dethloff con la configuración -n 100.0 -p 90.0 -cprob 40 -mprob 70}
\centering
\small
\begin{tabular}{c c c c c c c c}
\hline\hline
Instancia & Costo mínimo & Tiempo(seg.) & Costo promedio & Tiempo promedio(seg.) & CME & \%G & \%GP \\ [0.5ex]
\hline
SCA3-0 & 640.55 & 1.02 & 
640.55 & 1.08 & \bf{635.62} & 
0.78 & 0.78\\SCA3-1 & 700.50 & 1.10 & 
700.50 & 1.12 & \bf{697.84} & 
0.38 & 0.38\\SCA3-2 & 661.13 & 1.20 & 
663.42 & 1.14 & \bf{659.34} & 
0.27 & 0.62\\SCA3-3 & 681.35 & 1.16 & 
681.35 & 1.15 & \bf{680.04} & 
0.19 & 0.19\\SCA3-4 & \bf{690.50} & 1.40 & 
690.50 & 1.16 & 690.50 & 0.00
 & 0.00\\
SCA3-5 & 665.64 & 1.02 & 
665.90 & 1.18 & \bf{659.90} & 
0.87 & 0.91\\SCA3-6 & 652.94 & 1.20 & 
652.94 & 1.21 & \bf{651.09} & 
0.28 & 0.28\\SCA3-7 & 664.88 & 1.02 & 
665.51 & 1.07 & \bf{659.17} & 
0.87 & 0.96\\SCA3-8 & 719.77 & 1.26 & 
719.77 & 1.18 & \bf{719.47} & 
0.04 & 0.04\\SCA3-9 & \bf{681.00} & 1.02 & 
681.00 & 1.03 & 681.00 & 0.00
 & 0.00\\
SCA8-0 & 996.46 & 1.02 & 
996.46 & 1.17 & \bf{961.50} & 
3.64 & 3.64\\SCA8-1 & 1062.44 & 0.99 & 
1065.74 & 1.15 & \bf{1049.65} & 
1.22 & 1.53\\SCA8-2 & 1050.37 & 0.92 & 
1052.30 & 1.18 & \bf{1039.64} & 
1.03 & 1.22\\SCA8-3 & 1002.24 & 1.26 & 
1003.68 & 1.15 & \bf{983.34} & 
1.92 & 2.07\\SCA8-4 & 1069.87 & 1.14 & 
1078.76 & 1.12 & \bf{1065.49} & 
0.41 & 1.25\\SCA8-5 & 1039.64 & 1.03 & 
1039.64 & 1.05 & \bf{1027.08} & 
1.22 & 1.22\\SCA8-6 & 991.23 & 1.35 & 
991.23 & 1.20 & \bf{971.82} & 
2.00 & 2.00\\SCA8-7 & 1074.24 & 0.95 & 
1074.24 & 1.02 & \bf{1051.28} & 
2.18 & 2.18\\SCA8-8 & 1081.77 & 1.07 & 
1081.77 & 1.24 & \bf{1071.18} & 
0.99 & 0.99\\SCA8-9 & 1079.39 & 1.06 & 
1079.39 & 1.24 & \bf{1060.50} & 
1.78 & 1.78\\CON3-0 & 620.76 & 1.11 & 
626.64 & 1.07 & \bf{616.52} & 
0.69 & 1.64\\CON3-1 & 560.41 & 1.25 & 
560.51 & 1.17 & \bf{554.47} & 
1.07 & 1.09\\CON3-2 & 521.38 & 1.42 & 
521.38 & 1.56 & \bf{518.00} & 
0.65 & 0.65\\CON3-3 & \bf{591.19} & 1.50 & 
591.92 & 1.19 & 591.19 & 0.00
 & 0.12\\CON3-4 & 592.58 & 1.27 & 
592.88 & 1.12 & \bf{588.79} & 
0.64 & 0.69\\CON3-5 & 567.94 & 1.02 & 
568.18 & 1.14 & \bf{563.70} & 
0.75 & 0.80\\CON3-6 & 502.16 & 1.22 & 
504.75 & 1.22 & \bf{499.05} & 
0.62 & 1.14\\CON3-7 & 581.83 & 1.14 & 
581.99 & 1.22 & \bf{576.48} & 
0.93 & 0.95\\CON3-8 & 523.14 & 1.26 & 
523.14 & 1.13 & \bf{523.05} & 
0.02 & 0.02\\CON3-9 & 582.79 & 1.09 & 
582.79 & 1.20 & \bf{578.24} & 
0.79 & 0.79\\CON8-0 & 876.08 & 1.10 & 
876.08 & 1.08 & \bf{857.17} & 
2.21 & 2.21\\CON8-1 & 770.00 & 1.20 & 
770.00 & 1.29 & \bf{740.85} & 
3.93 & 3.93\\CON8-2 & 713.44 & 1.02 & 
713.44 & 1.23 & \bf{712.89} & 
0.08 & 0.08\\CON8-3 & 828.16 & 1.28 & 
828.16 & 1.30 & \bf{811.07} & 
2.11 & 2.11\\CON8-4 & 772.76 & 1.50 & 
782.04 & 1.37 & \bf{772.25} & 
0.07 & 1.27\\CON8-5 & 767.17 & 1.12 & 
767.17 & 1.19 & \bf{754.88} & 
1.63 & 1.63\\CON8-6 & 685.80 & 1.05 & 
689.17 & 1.14 & \bf{678.92} & 
1.01 & 1.51\\CON8-7 & 821.28 & 1.02 & 
823.67 & 1.08 & \bf{811.96} & 
1.15 & 1.44\\CON8-8 & 784.28 & 1.36 & 
784.33 & 1.36 & \bf{767.53} & 
2.18 & 2.19\\CON8-9 & 813.04 & 0.98 & 
822.08 & 1.17 & \bf{809.00} & 
0.50 & 1.62\\\bf{PROM.} & 
\bf{767.05} & \bf{1.15} & \bf{768.37} & \bf{1.18} & \bf{758.54} & \bf{1.03} & \bf{1.20}\\[1ex]\hline
\end{tabular}
\label{table:nonlin}
\end{table} \clearpage
\begin{table}[ht]
\caption{Resultados de la ejecución de la metaheurística IGA, utilizando instancias de SalhiNagy con la configuración -n 100.0 -p 90.0 -cprob 90 -mprob 70}
\centering
\small
\begin{tabular}{c c c c c c c c}
\hline\hline
Instancia & Costo mínimo & Tiempo(seg.) & Costo promedio & Tiempo promedio(seg.) & CME & \%G & \%GP \\ [0.5ex]
\hline
CMT1X & 479.21 & 1.06 & 
480.97 & 1.19 & \bf{470.48} & 
1.86 & 2.23\\CMT1Y & 481.31 & 1.26 & 
483.93 & 1.22 & \bf{470.48} & 
2.30 & 2.86\\CMT2X & 707.41 & 2.60 & 
709.46 & 2.63 & \bf{682.39} & 
3.67 & 3.97\\CMT2Y & 707.21 & 2.55 & 
712.59 & 2.60 & \bf{682.39} & 
3.64 & 4.42\\CMT3X & 729.63 & 6.47 & 
736.03 & 6.12 & \bf{719.06} & 
1.47 & 2.36\\CMT3Y & 733.66 & 5.53 & 
739.41 & 5.95 & \bf{719.06} & 
2.03 & 2.83\\CMT4X & 897.97 & 17.72 & 
905.41 & 17.64 & \bf{854.21} & 
5.12 & 5.99\\CMT4Y & 885.78 & 17.20 & 
904.36 & 17.16 & \bf{852.46} & 
3.91 & 6.09\\CMT5X & 1106.78 & 35.96 & 
1120.05 & 34.84 & \bf{1030.56} & 
7.40 & 8.68\\CMT5Y & 1112.02 & 35.06 & 
1122.70 & 34.79 & \bf{1031.69} & 
7.79 & 8.82\\CMT11X & 884.51 & 10.88 & 
901.21 & 10.46 & \bf{831.09} & 
6.43 & 8.44\\CMT11Y & 898.74 & 11.14 & 
910.80 & 11.84 & \bf{829.85} & 
8.30 & 9.76\\CMT12X & 675.72 & 6.41 & 
677.19 & 6.32 & \bf{658.83} & 
2.56 & 2.79\\CMT12Y & 675.13 & 6.75 & 
676.21 & 6.37 & \bf{660.47} & 
2.22 & 2.38\\\bf{PROM.} & 
\bf{783.93} & \bf{11.47} & \bf{791.45} & \bf{11.37} & \bf{749.50} & \bf{4.19} & \bf{5.12}\\[1ex]\hline
\end{tabular}
\label{table:nonlin}
\end{table} \clearpage
\begin{table}[ht]
\caption{Resultados de la ejecución de la metaheurística IGA, utilizando instancias de Dethloff con la configuración -n 100.0 -p 100.0 -cprob 40 -mprob 70}
\centering
\small
\begin{tabular}{c c c c c c c c}
\hline\hline
Instancia & Costo mínimo & Tiempo(seg.) & Costo promedio & Tiempo promedio(seg.) & CME & \%G & \%GP \\ [0.5ex]
\hline
SCA3-0 & 640.55 & 1.10 & 
640.55 & 1.23 & \bf{635.62} & 
0.78 & 0.78\\SCA3-1 & \bf{697.84} & 1.06 & 
697.84 & 1.32 & 697.84 & 0.00
 & 0.00\\
SCA3-2 & 664.18 & 1.19 & 
666.36 & 1.19 & \bf{659.34} & 
0.73 & 1.06\\SCA3-3 & \bf{680.04} & 1.13 & 
681.11 & 1.34 & 680.04 & 0.00
 & 0.16\\SCA3-4 & \bf{690.50} & 1.05 & 
690.50 & 1.17 & 690.50 & 0.00
 & 0.00\\
SCA3-5 & 665.64 & 1.18 & 
665.64 & 1.32 & \bf{659.90} & 
0.87 & 0.87\\SCA3-6 & 652.94 & 1.30 & 
654.02 & 1.32 & \bf{651.09} & 
0.28 & 0.45\\SCA3-7 & \bf{659.17} & 1.14 & 
664.40 & 1.33 & 659.17 & 0.00
 & 0.79\\SCA3-8 & \bf{719.47} & 1.34 & 
724.76 & 1.39 & 719.47 & 0.00
 & 0.74\\SCA3-9 & \bf{681.00} & 1.58 & 
681.86 & 1.22 & 681.00 & 0.00
 & 0.13\\SCA8-0 & 985.54 & 1.41 & 
986.04 & 1.34 & \bf{961.50} & 
2.50 & 2.55\\SCA8-1 & 1070.49 & 1.55 & 
1070.49 & 1.35 & \bf{1049.65} & 
1.99 & 1.99\\SCA8-2 & 1050.37 & 1.05 & 
1050.37 & 1.29 & \bf{1039.64} & 
1.03 & 1.03\\SCA8-3 & 1018.31 & 1.49 & 
1018.74 & 1.46 & \bf{983.34} & 
3.56 & 3.60\\SCA8-4 & 1068.97 & 1.35 & 
1077.53 & 1.24 & \bf{1065.49} & 
0.33 & 1.13\\SCA8-5 & 1048.96 & 1.21 & 
1048.96 & 1.31 & \bf{1027.08} & 
2.13 & 2.13\\SCA8-6 & 972.48 & 1.75 & 
972.48 & 1.39 & \bf{971.82} & 
0.07 & 0.07\\SCA8-7 & 1070.67 & 1.25 & 
1070.67 & 1.17 & \bf{1051.28} & 
1.84 & 1.84\\SCA8-8 & \bf{1071.18} & 1.24 & 
1083.20 & 1.27 & 1071.18 & 0.00
 & 1.12\\SCA8-9 & 1081.64 & 1.14 & 
1083.55 & 1.35 & \bf{1060.50} & 
1.99 & 2.17\\CON3-0 & 619.09 & 1.27 & 
619.09 & 1.42 & \bf{616.52} & 
0.42 & 0.42\\CON3-1 & \bf{554.47} & 1.36 & 
558.22 & 1.42 & 554.47 & 0.00
 & 0.68\\CON3-2 & 521.38 & 1.71 & 
521.90 & 1.57 & \bf{518.00} & 
0.65 & 0.75\\CON3-3 & 591.48 & 1.38 & 
598.35 & 1.38 & \bf{591.19} & 
0.05 & 1.21\\CON3-4 & 591.43 & 1.06 & 
593.19 & 1.21 & \bf{588.79} & 
0.45 & 0.75\\CON3-5 & \bf{563.70} & 1.17 & 
565.78 & 1.28 & 563.70 & 0.00
 & 0.37\\CON3-6 & 502.16 & 1.45 & 
502.73 & 1.34 & \bf{499.05} & 
0.62 & 0.74\\CON3-7 & 577.54 & 1.11 & 
579.67 & 1.17 & \bf{576.48} & 
0.18 & 0.55\\CON3-8 & 524.38 & 1.52 & 
526.12 & 1.52 & \bf{523.05} & 
0.25 & 0.59\\CON3-9 & 582.79 & 1.10 & 
585.61 & 1.26 & \bf{578.24} & 
0.79 & 1.27\\CON8-0 & 871.29 & 1.13 & 
871.29 & 1.27 & \bf{857.17} & 
1.65 & 1.65\\CON8-1 & 755.89 & 1.72 & 
757.90 & 1.39 & \bf{740.85} & 
2.03 & 2.30\\CON8-2 & 720.40 & 1.66 & 
721.70 & 1.50 & \bf{712.89} & 
1.05 & 1.24\\CON8-3 & 816.03 & 1.15 & 
825.60 & 1.39 & \bf{811.07} & 
0.61 & 1.79\\CON8-4 & 792.67 & 1.39 & 
792.67 & 1.24 & \bf{772.25} & 
2.64 & 2.64\\CON8-5 & 760.62 & 1.46 & 
760.62 & 1.41 & \bf{754.88} & 
0.76 & 0.76\\CON8-6 & 700.06 & 1.32 & 
700.44 & 1.30 & \bf{678.92} & 
3.11 & 3.17\\CON8-7 & 815.72 & 1.14 & 
815.95 & 1.26 & \bf{811.96} & 
0.46 & 0.49\\CON8-8 & 787.19 & 1.60 & 
788.98 & 1.37 & \bf{767.53} & 
2.56 & 2.79\\CON8-9 & 827.47 & 1.42 & 
831.12 & 1.46 & \bf{809.00} & 
2.28 & 2.73\\\bf{PROM.} & 
\bf{766.64} & \bf{1.32} & \bf{768.65} & \bf{1.33} & \bf{758.54} & \bf{0.97} & \bf{1.24}\\[1ex]\hline
\end{tabular}
\label{table:nonlin}
\end{table} \clearpage
\begin{table}[ht]
\caption{Resultados de la ejecución de la metaheurística IGA, utilizando instancias de SalhiNagy con la configuración -n 100.0 -p 100.0 -cprob 90 -mprob 70}
\centering
\small
\begin{tabular}{c c c c c c c c}
\hline\hline
Instancia & Costo mínimo & Tiempo(seg.) & Costo promedio & Tiempo promedio(seg.) & CME & \%G & \%GP \\ [0.5ex]
\hline
CMT1X & 473.42 & 1.47 & 
475.59 & 1.38 & \bf{470.48} & 
0.62 & 1.09\\CMT1Y & 472.37 & 1.36 & 
480.77 & 1.21 & \bf{470.48} & 
0.40 & 2.19\\CMT2X & 700.48 & 3.50 & 
710.97 & 3.06 & \bf{682.39} & 
2.65 & 4.19\\CMT2Y & 699.75 & 2.58 & 
706.40 & 3.02 & \bf{682.39} & 
2.54 & 3.52\\CMT3X & 732.95 & 7.33 & 
740.04 & 7.13 & \bf{719.06} & 
1.93 & 2.92\\CMT3Y & 732.53 & 7.20 & 
734.79 & 6.84 & \bf{719.06} & 
1.87 & 2.19\\CMT4X & 897.22 & 19.16 & 
902.28 & 19.27 & \bf{854.21} & 
5.04 & 5.63\\CMT4Y & 895.52 & 18.59 & 
905.69 & 18.98 & \bf{852.46} & 
5.05 & 6.24\\CMT5X & 1111.14 & 42.68 & 
1112.61 & 39.86 & \bf{1030.56} & 
7.82 & 7.96\\CMT5Y & 1114.63 & 40.20 & 
1117.45 & 40.10 & \bf{1031.69} & 
8.04 & 8.31\\CMT11X & 904.61 & 11.47 & 
906.03 & 11.87 & \bf{831.09} & 
8.85 & 9.02\\CMT11Y & 866.64 & 13.13 & 
892.79 & 13.31 & \bf{829.85} & 
4.43 & 7.58\\CMT12X & 674.91 & 6.92 & 
676.15 & 7.11 & \bf{658.83} & 
2.44 & 2.63\\CMT12Y & 675.41 & 7.16 & 
676.54 & 7.01 & \bf{660.47} & 
2.26 & 2.43\\\bf{PROM.} & 
\bf{782.26} & \bf{13.05} & \bf{788.44} & \bf{12.87} & \bf{749.50} & \bf{3.85} & \bf{4.71}\\[1ex]\hline
\end{tabular}
\label{table:nonlin}
\end{table} \clearpage
\begin{table}[ht]
\caption{Resultados de la ejecución de la metaheurística IGA, utilizando instancias de Dethloff con la configuración -n 150.0 -p 30.0 -cprob 40 -mprob 70}
\centering
\small
\begin{tabular}{c c c c c c c c}
\hline\hline
Instancia & Costo mínimo & Tiempo(seg.) & Costo promedio & Tiempo promedio(seg.) & CME & \%G & \%GP \\ [0.5ex]
\hline
SCA3-0 & 640.55 & 0.56 & 
641.78 & 0.50 & \bf{635.62} & 
0.78 & 0.97\\SCA3-1 & \bf{697.84} & 0.36 & 
697.84 & 0.49 & 697.84 & 0.00
 & 0.00\\
SCA3-2 & 664.18 & 0.43 & 
671.59 & 0.41 & \bf{659.34} & 
0.73 & 1.86\\SCA3-3 & 683.37 & 0.45 & 
684.42 & 0.41 & \bf{680.04} & 
0.49 & 0.64\\SCA3-4 & \bf{690.50} & 0.43 & 
690.50 & 0.47 & 690.50 & 0.00
 & 0.00\\
SCA3-5 & 673.56 & 0.47 & 
680.68 & 0.50 & \bf{659.90} & 
2.07 & 3.15\\SCA3-6 & 655.05 & 0.61 & 
655.81 & 0.46 & \bf{651.09} & 
0.61 & 0.73\\SCA3-7 & 667.24 & 0.48 & 
671.40 & 0.47 & \bf{659.17} & 
1.22 & 1.86\\SCA3-8 & \bf{719.47} & 0.39 & 
721.88 & 0.36 & 719.47 & 0.00
 & 0.33\\SCA3-9 & 685.00 & 0.58 & 
688.45 & 0.42 & \bf{681.00} & 
0.59 & 1.09\\SCA8-0 & 989.02 & 0.41 & 
1003.71 & 0.39 & \bf{961.50} & 
2.86 & 4.39\\SCA8-1 & 1067.44 & 0.40 & 
1069.05 & 0.37 & \bf{1049.65} & 
1.69 & 1.85\\SCA8-2 & 1052.70 & 0.35 & 
1052.70 & 0.39 & \bf{1039.64} & 
1.26 & 1.26\\SCA8-3 & 1025.34 & 0.50 & 
1025.34 & 0.42 & \bf{983.34} & 
4.27 & 4.27\\SCA8-4 & 1087.97 & 0.64 & 
1106.45 & 0.52 & \bf{1065.49} & 
2.11 & 3.84\\SCA8-5 & 1053.17 & 0.32 & 
1053.17 & 0.34 & \bf{1027.08} & 
2.54 & 2.54\\SCA8-6 & 994.62 & 0.32 & 
994.62 & 0.49 & \bf{971.82} & 
2.35 & 2.35\\SCA8-7 & 1090.68 & 0.30 & 
1094.78 & 0.35 & \bf{1051.28} & 
3.75 & 4.14\\SCA8-8 & 1086.27 & 0.47 & 
1086.27 & 0.38 & \bf{1071.18} & 
1.41 & 1.41\\SCA8-9 & 1098.91 & 0.32 & 
1098.91 & 0.40 & \bf{1060.50} & 
3.62 & 3.62\\CON3-0 & 625.94 & 0.48 & 
629.34 & 0.47 & \bf{616.52} & 
1.53 & 2.08\\CON3-1 & 560.75 & 0.47 & 
560.75 & 0.53 & \bf{554.47} & 
1.13 & 1.13\\CON3-2 & 524.89 & 0.56 & 
524.89 & 0.52 & \bf{518.00} & 
1.33 & 1.33\\CON3-3 & 592.43 & 0.46 & 
595.65 & 0.56 & \bf{591.19} & 
0.21 & 0.75\\CON3-4 & 591.43 & 0.40 & 
591.43 & 0.44 & \bf{588.79} & 
0.45 & 0.45\\CON3-5 & 564.89 & 0.46 & 
565.65 & 0.40 & \bf{563.70} & 
0.21 & 0.35\\CON3-6 & 505.14 & 0.37 & 
507.55 & 0.37 & \bf{499.05} & 
1.22 & 1.70\\CON3-7 & 586.01 & 0.37 & 
586.01 & 0.41 & \bf{576.48} & 
1.65 & 1.65\\CON3-8 & 523.14 & 0.36 & 
525.91 & 0.46 & \bf{523.05} & 
0.02 & 0.55\\CON3-9 & 589.57 & 0.61 & 
590.29 & 0.54 & \bf{578.24} & 
1.96 & 2.08\\CON8-0 & 869.87 & 0.34 & 
889.54 & 0.34 & \bf{857.17} & 
1.48 & 3.78\\CON8-1 & 763.31 & 0.59 & 
763.50 & 0.51 & \bf{740.85} & 
3.03 & 3.06\\CON8-2 & 731.37 & 0.36 & 
732.74 & 0.41 & \bf{712.89} & 
2.59 & 2.78\\CON8-3 & 835.73 & 0.38 & 
847.91 & 0.58 & \bf{811.07} & 
3.04 & 4.54\\CON8-4 & 786.93 & 0.33 & 
790.43 & 0.46 & \bf{772.25} & 
1.90 & 2.35\\CON8-5 & 762.01 & 0.53 & 
764.95 & 0.53 & \bf{754.88} & 
0.94 & 1.33\\CON8-6 & 694.87 & 0.39 & 
699.66 & 0.52 & \bf{678.92} & 
2.35 & 3.06\\CON8-7 & 815.44 & 0.37 & 
822.24 & 0.42 & \bf{811.96} & 
0.43 & 1.27\\CON8-8 & 807.50 & 0.48 & 
808.84 & 0.50 & \bf{767.53} & 
5.21 & 5.38\\CON8-9 & 844.45 & 0.45 & 
849.25 & 0.45 & \bf{809.00} & 
4.38 & 4.98\\\bf{PROM.} & 
\bf{772.46} & \bf{0.44} & \bf{775.90} & \bf{0.45} & \bf{758.54} & \bf{1.69} & \bf{2.12}\\[1ex]\hline
\end{tabular}
\label{table:nonlin}
\end{table} \clearpage
\begin{table}[ht]
\caption{Resultados de la ejecución de la metaheurística IGA, utilizando instancias de SalhiNagy con la configuración -n 150.0 -p 30.0 -cprob 90 -mprob 70}
\centering
\small
\begin{tabular}{c c c c c c c c}
\hline\hline
Instancia & Costo mínimo & Tiempo(seg.) & Costo promedio & Tiempo promedio(seg.) & CME & \%G & \%GP \\ [0.5ex]
\hline
CMT1X & 478.84 & 0.44 & 
478.84 & 0.46 & \bf{470.48} & 
1.78 & 1.78\\CMT1Y & 489.31 & 0.42 & 
490.60 & 0.47 & \bf{470.48} & 
4.00 & 4.28\\CMT2X & 712.14 & 1.10 & 
718.36 & 1.09 & \bf{682.39} & 
4.36 & 5.27\\CMT2Y & 717.04 & 1.09 & 
718.35 & 0.97 & \bf{682.39} & 
5.08 & 5.27\\CMT3X & 745.26 & 2.16 & 
751.49 & 2.30 & \bf{719.06} & 
3.64 & 4.51\\CMT3Y & 734.85 & 2.32 & 
744.64 & 2.35 & \bf{719.06} & 
2.20 & 3.56\\CMT4X & 899.52 & 6.37 & 
908.09 & 6.36 & \bf{854.21} & 
5.30 & 6.31\\CMT4Y & 900.44 & 6.07 & 
915.39 & 6.11 & \bf{852.46} & 
5.63 & 7.38\\CMT5X & 1116.21 & 12.76 & 
1135.67 & 12.71 & \bf{1030.56} & 
8.31 & 10.20\\CMT5Y & 1121.02 & 12.90 & 
1131.26 & 12.27 & \bf{1031.69} & 
8.66 & 9.65\\CMT11X & 889.67 & 4.10 & 
917.93 & 3.74 & \bf{831.09} & 
7.05 & 10.45\\CMT11Y & 905.33 & 4.17 & 
909.30 & 4.13 & \bf{829.85} & 
9.10 & 9.57\\CMT12X & 676.00 & 2.48 & 
683.82 & 2.35 & \bf{658.83} & 
2.61 & 3.79\\CMT12Y & 674.63 & 2.32 & 
679.37 & 2.22 & \bf{660.47} & 
2.14 & 2.86\\\bf{PROM.} & 
\bf{790.02} & \bf{4.19} & \bf{798.79} & \bf{4.11} & \bf{749.50} & \bf{4.99} & \bf{6.06}\\[1ex]\hline
\end{tabular}
\label{table:nonlin}
\end{table} \clearpage
\begin{table}[ht]
\caption{Resultados de la ejecución de la metaheurística IGA, utilizando instancias de Dethloff con la configuración -n 150.0 -p 40.0 -cprob 40 -mprob 70}
\centering
\small
\begin{tabular}{c c c c c c c c}
\hline\hline
Instancia & Costo mínimo & Tiempo(seg.) & Costo promedio & Tiempo promedio(seg.) & CME & \%G & \%GP \\ [0.5ex]
\hline
SCA3-0 & 641.69 & 0.47 & 
641.99 & 0.53 & \bf{635.62} & 
0.95 & 1.00\\SCA3-1 & 700.50 & 0.78 & 
705.51 & 0.59 & \bf{697.84} & 
0.38 & 1.10\\SCA3-2 & \bf{659.34} & 0.74 & 
664.71 & 0.58 & 659.34 & 0.00
 & 0.81\\SCA3-3 & 681.35 & 0.49 & 
681.35 & 0.49 & \bf{680.04} & 
0.19 & 0.19\\SCA3-4 & \bf{690.50} & 0.62 & 
690.50 & 0.58 & 690.50 & 0.00
 & 0.00\\
SCA3-5 & 670.02 & 0.47 & 
674.03 & 0.55 & \bf{659.90} & 
1.53 & 2.14\\SCA3-6 & 652.94 & 0.50 & 
652.94 & 0.54 & \bf{651.09} & 
0.28 & 0.28\\SCA3-7 & 666.15 & 0.77 & 
666.15 & 0.60 & \bf{659.17} & 
1.06 & 1.06\\SCA3-8 & 723.99 & 0.61 & 
725.91 & 0.51 & \bf{719.47} & 
0.63 & 0.90\\SCA3-9 & \bf{681.00} & 0.59 & 
682.03 & 0.49 & 681.00 & 0.00
 & 0.15\\SCA8-0 & 970.64 & 0.43 & 
970.64 & 0.52 & \bf{961.50} & 
0.95 & 0.95\\SCA8-1 & 1082.21 & 0.78 & 
1084.93 & 0.65 & \bf{1049.65} & 
3.10 & 3.36\\SCA8-2 & 1055.32 & 0.80 & 
1055.32 & 0.56 & \bf{1039.64} & 
1.51 & 1.51\\SCA8-3 & 1042.79 & 0.51 & 
1042.79 & 0.51 & \bf{983.34} & 
6.05 & 6.05\\SCA8-4 & 1068.97 & 0.56 & 
1083.37 & 0.53 & \bf{1065.49} & 
0.33 & 1.68\\SCA8-5 & 1058.14 & 0.45 & 
1058.14 & 0.56 & \bf{1027.08} & 
3.02 & 3.02\\SCA8-6 & 977.87 & 0.40 & 
988.09 & 0.52 & \bf{971.82} & 
0.62 & 1.67\\SCA8-7 & 1073.05 & 0.61 & 
1073.05 & 0.59 & \bf{1051.28} & 
2.07 & 2.07\\SCA8-8 & 1095.32 & 0.64 & 
1096.04 & 0.64 & \bf{1071.18} & 
2.25 & 2.32\\SCA8-9 & 1078.21 & 0.48 & 
1082.84 & 0.52 & \bf{1060.50} & 
1.67 & 2.11\\CON3-0 & 624.96 & 0.49 & 
629.06 & 0.66 & \bf{616.52} & 
1.37 & 2.03\\CON3-1 & 560.75 & 0.61 & 
560.75 & 0.71 & \bf{554.47} & 
1.13 & 1.13\\CON3-2 & 520.13 & 0.86 & 
521.25 & 0.72 & \bf{518.00} & 
0.41 & 0.63\\CON3-3 & 601.66 & 0.52 & 
603.28 & 0.53 & \bf{591.19} & 
1.77 & 2.05\\CON3-4 & 601.64 & 0.43 & 
603.34 & 0.47 & \bf{588.79} & 
2.18 & 2.47\\CON3-5 & 564.89 & 0.79 & 
564.89 & 0.65 & \bf{563.70} & 
0.21 & 0.21\\CON3-6 & 507.23 & 0.47 & 
508.12 & 0.57 & \bf{499.05} & 
1.64 & 1.82\\CON3-7 & 586.01 & 0.51 & 
586.01 & 0.60 & \bf{576.48} & 
1.65 & 1.65\\CON3-8 & 523.14 & 0.66 & 
527.04 & 0.60 & \bf{523.05} & 
0.02 & 0.76\\CON3-9 & 589.57 & 0.63 & 
590.38 & 0.89 & \bf{578.24} & 
1.96 & 2.10\\CON8-0 & 876.27 & 0.66 & 
876.27 & 0.57 & \bf{857.17} & 
2.23 & 2.23\\CON8-1 & 761.87 & 0.70 & 
766.37 & 0.66 & \bf{740.85} & 
2.84 & 3.44\\CON8-2 & 716.07 & 0.64 & 
716.07 & 0.91 & \bf{712.89} & 
0.45 & 0.45\\CON8-3 & 828.56 & 0.72 & 
833.74 & 0.68 & \bf{811.07} & 
2.16 & 2.80\\CON8-4 & 784.78 & 0.58 & 
787.86 & 0.57 & \bf{772.25} & 
1.62 & 2.02\\CON8-5 & 771.00 & 0.84 & 
771.00 & 0.59 & \bf{754.88} & 
2.14 & 2.14\\CON8-6 & 686.23 & 0.54 & 
686.23 & 0.57 & \bf{678.92} & 
1.08 & 1.08\\CON8-7 & 815.14 & 0.42 & 
815.14 & 0.52 & \bf{811.96} & 
0.39 & 0.39\\CON8-8 & 793.88 & 0.49 & 
794.35 & 0.74 & \bf{767.53} & 
3.43 & 3.49\\CON8-9 & 813.57 & 0.83 & 
813.57 & 0.68 & \bf{809.00} & 
0.56 & 0.56\\\bf{PROM.} & 
\bf{769.93} & \bf{0.60} & \bf{771.88} & \bf{0.60} & \bf{758.54} & \bf{1.40} & \bf{1.65}\\[1ex]\hline
\end{tabular}
\label{table:nonlin}
\end{table} \clearpage
\begin{table}[ht]
\caption{Resultados de la ejecución de la metaheurística IGA, utilizando instancias de SalhiNagy con la configuración -n 150.0 -p 40.0 -cprob 90 -mprob 70}
\centering
\small
\begin{tabular}{c c c c c c c c}
\hline\hline
Instancia & Costo mínimo & Tiempo(seg.) & Costo promedio & Tiempo promedio(seg.) & CME & \%G & \%GP \\ [0.5ex]
\hline
CMT1X & 477.73 & 0.72 & 
478.01 & 0.66 & \bf{470.48} & 
1.54 & 1.60\\CMT1Y & 476.43 & 0.53 & 
486.54 & 0.62 & \bf{470.48} & 
1.26 & 3.41\\CMT2X & 704.20 & 1.39 & 
712.93 & 1.49 & \bf{682.39} & 
3.20 & 4.48\\CMT2Y & 711.79 & 1.46 & 
714.75 & 1.40 & \bf{682.39} & 
4.31 & 4.74\\CMT3X & 741.18 & 2.80 & 
747.67 & 3.01 & \bf{719.06} & 
3.08 & 3.98\\CMT3Y & 746.09 & 2.86 & 
750.16 & 2.90 & \bf{719.06} & 
3.76 & 4.32\\CMT4X & 918.43 & 7.78 & 
919.69 & 8.07 & \bf{854.21} & 
7.52 & 7.67\\CMT4Y & 905.45 & 8.37 & 
911.21 & 8.09 & \bf{852.46} & 
6.22 & 6.89\\CMT5X & 1106.24 & 17.21 & 
1114.57 & 16.79 & \bf{1030.56} & 
7.34 & 8.15\\CMT5Y & 1118.24 & 16.95 & 
1127.27 & 16.83 & \bf{1031.69} & 
8.39 & 9.26\\CMT11X & 903.17 & 5.14 & 
913.13 & 5.31 & \bf{831.09} & 
8.67 & 9.87\\CMT11Y & 876.04 & 5.45 & 
902.42 & 5.62 & \bf{829.85} & 
5.57 & 8.74\\CMT12X & 680.56 & 3.12 & 
686.57 & 3.11 & \bf{658.83} & 
3.30 & 4.21\\CMT12Y & 673.90 & 3.09 & 
676.27 & 3.00 & \bf{660.47} & 
2.03 & 2.39\\\bf{PROM.} & 
\bf{788.53} & \bf{5.49} & \bf{795.80} & \bf{5.49} & \bf{749.50} & \bf{4.73} & \bf{5.69}\\[1ex]\hline
\end{tabular}
\label{table:nonlin}
\end{table} \clearpage
\begin{table}[ht]
\caption{Resultados de la ejecución de la metaheurística IGA, utilizando instancias de Dethloff con la configuración -n 150.0 -p 50.0 -cprob 40 -mprob 70}
\centering
\small
\begin{tabular}{c c c c c c c c}
\hline\hline
Instancia & Costo mínimo & Tiempo(seg.) & Costo promedio & Tiempo promedio(seg.) & CME & \%G & \%GP \\ [0.5ex]
\hline
SCA3-0 & 640.55 & 0.84 & 
640.55 & 0.83 & \bf{635.62} & 
0.78 & 0.78\\SCA3-1 & \bf{697.84} & 0.72 & 
700.35 & 0.79 & 697.84 & 0.00
 & 0.36\\SCA3-2 & \bf{659.34} & 0.70 & 
663.88 & 0.75 & 659.34 & 0.00
 & 0.69\\SCA3-3 & 681.35 & 0.98 & 
684.12 & 0.75 & \bf{680.04} & 
0.19 & 0.60\\SCA3-4 & 692.57 & 0.58 & 
692.57 & 0.68 & \bf{690.50} & 
0.30 & 0.30\\SCA3-5 & 666.67 & 0.68 & 
672.13 & 0.67 & \bf{659.90} & 
1.03 & 1.85\\SCA3-6 & 657.24 & 0.78 & 
657.24 & 0.68 & \bf{651.09} & 
0.94 & 0.94\\SCA3-7 & 666.15 & 0.79 & 
666.15 & 0.86 & \bf{659.17} & 
1.06 & 1.06\\SCA3-8 & \bf{719.47} & 0.74 & 
721.97 & 0.72 & 719.47 & 0.00
 & 0.35\\SCA3-9 & \bf{681.00} & 0.55 & 
681.51 & 0.60 & 681.00 & 0.00
 & 0.07\\SCA8-0 & 994.55 & 0.55 & 
994.55 & 0.67 & \bf{961.50} & 
3.44 & 3.44\\SCA8-1 & 1062.88 & 0.77 & 
1062.88 & 0.81 & \bf{1049.65} & 
1.26 & 1.26\\SCA8-2 & 1057.93 & 0.87 & 
1057.93 & 0.79 & \bf{1039.64} & 
1.76 & 1.76\\SCA8-3 & 1033.51 & 0.68 & 
1033.51 & 0.81 & \bf{983.34} & 
5.10 & 5.10\\SCA8-4 & 1069.71 & 0.65 & 
1069.71 & 0.66 & \bf{1065.49} & 
0.40 & 0.40\\SCA8-5 & \bf{1027.08} & 0.68 & 
1027.48 & 0.57 & 1027.08 & 0.00
 & 0.04\\SCA8-6 & 972.48 & 1.04 & 
979.86 & 0.74 & \bf{971.82} & 
0.07 & 0.83\\SCA8-7 & 1094.73 & 0.62 & 
1094.73 & 0.65 & \bf{1051.28} & 
4.13 & 4.13\\SCA8-8 & \bf{1071.18} & 0.51 & 
1071.18 & 0.60 & 1071.18 & 0.00
 & 0.00\\
SCA8-9 & 1083.40 & 0.57 & 
1085.89 & 0.75 & \bf{1060.50} & 
2.16 & 2.39\\CON3-0 & 629.03 & 0.67 & 
630.34 & 0.63 & \bf{616.52} & 
2.03 & 2.24\\CON3-1 & 556.92 & 0.66 & 
557.50 & 0.67 & \bf{554.47} & 
0.44 & 0.55\\CON3-2 & 521.38 & 0.67 & 
522.03 & 0.80 & \bf{518.00} & 
0.65 & 0.78\\CON3-3 & 592.81 & 0.79 & 
598.43 & 0.84 & \bf{591.19} & 
0.27 & 1.23\\CON3-4 & 592.58 & 0.64 & 
597.67 & 0.72 & \bf{588.79} & 
0.64 & 1.51\\CON3-5 & 566.45 & 0.80 & 
567.57 & 0.73 & \bf{563.70} & 
0.49 & 0.69\\CON3-6 & 502.88 & 0.58 & 
504.90 & 0.69 & \bf{499.05} & 
0.77 & 1.17\\CON3-7 & 582.14 & 0.86 & 
582.14 & 0.75 & \bf{576.48} & 
0.98 & 0.98\\CON3-8 & 528.09 & 0.83 & 
530.48 & 0.76 & \bf{523.05} & 
0.96 & 1.42\\CON3-9 & 582.79 & 0.61 & 
584.50 & 0.68 & \bf{578.24} & 
0.79 & 1.08\\CON8-0 & 880.82 & 1.06 & 
888.09 & 1.04 & \bf{857.17} & 
2.76 & 3.61\\CON8-1 & 759.89 & 0.87 & 
760.85 & 0.84 & \bf{740.85} & 
2.57 & 2.70\\CON8-2 & 719.10 & 1.11 & 
719.10 & 0.92 & \bf{712.89} & 
0.87 & 0.87\\CON8-3 & 837.02 & 0.86 & 
838.25 & 0.89 & \bf{811.07} & 
3.20 & 3.35\\CON8-4 & 776.34 & 0.95 & 
776.34 & 0.74 & \bf{772.25} & 
0.53 & 0.53\\CON8-5 & 770.91 & 0.81 & 
771.79 & 0.76 & \bf{754.88} & 
2.12 & 2.24\\CON8-6 & 691.59 & 0.78 & 
691.59 & 0.65 & \bf{678.92} & 
1.87 & 1.87\\CON8-7 & 824.96 & 1.04 & 
824.96 & 0.71 & \bf{811.96} & 
1.60 & 1.60\\CON8-8 & 783.01 & 0.60 & 
784.74 & 0.77 & \bf{767.53} & 
2.02 & 2.24\\CON8-9 & 839.47 & 0.62 & 
843.62 & 0.68 & \bf{809.00} & 
3.77 & 4.28\\\bf{PROM.} & 
\bf{769.20} & \bf{0.75} & \bf{770.83} & \bf{0.74} & \bf{758.54} & \bf{1.30} & \bf{1.53}\\[1ex]\hline
\end{tabular}
\label{table:nonlin}
\end{table} \clearpage
\begin{table}[ht]
\caption{Resultados de la ejecución de la metaheurística IGA, utilizando instancias de SalhiNagy con la configuración -n 150.0 -p 50.0 -cprob 90 -mprob 70}
\centering
\small
\begin{tabular}{c c c c c c c c}
\hline\hline
Instancia & Costo mínimo & Tiempo(seg.) & Costo promedio & Tiempo promedio(seg.) & CME & \%G & \%GP \\ [0.5ex]
\hline
CMT1X & 472.37 & 0.92 & 
479.04 & 0.92 & \bf{470.48} & 
0.40 & 1.82\\CMT1Y & 478.71 & 0.72 & 
479.49 & 0.61 & \bf{470.48} & 
1.75 & 1.92\\CMT2X & 700.27 & 1.68 & 
711.90 & 1.71 & \bf{682.39} & 
2.62 & 4.32\\CMT2Y & 709.15 & 1.70 & 
713.62 & 1.73 & \bf{682.39} & 
3.92 & 4.58\\CMT3X & 733.04 & 3.68 & 
740.39 & 3.87 & \bf{719.06} & 
1.94 & 2.97\\CMT3Y & 731.60 & 3.66 & 
741.28 & 3.79 & \bf{719.06} & 
1.74 & 3.09\\CMT4X & 899.96 & 10.44 & 
912.35 & 10.18 & \bf{854.21} & 
5.36 & 6.81\\CMT4Y & 909.89 & 9.59 & 
915.11 & 10.04 & \bf{852.46} & 
6.74 & 7.35\\CMT5X & 1077.95 & 21.23 & 
1111.37 & 20.80 & \bf{1030.56} & 
4.60 & 7.84\\CMT5Y & 1125.16 & 20.75 & 
1134.15 & 20.82 & \bf{1031.69} & 
9.06 & 9.93\\CMT11X & 895.23 & 6.50 & 
914.20 & 6.25 & \bf{831.09} & 
7.72 & 10.00\\CMT11Y & 842.33 & 7.36 & 
898.48 & 6.99 & \bf{829.85} & 
1.50 & 8.27\\CMT12X & 673.30 & 3.72 & 
675.35 & 3.70 & \bf{658.83} & 
2.20 & 2.51\\CMT12Y & 674.71 & 3.66 & 
679.51 & 3.81 & \bf{660.47} & 
2.16 & 2.88\\\bf{PROM.} & 
\bf{780.26} & \bf{6.83} & \bf{793.30} & \bf{6.80} & \bf{749.50} & \bf{3.69} & \bf{5.31}\\[1ex]\hline
\end{tabular}
\label{table:nonlin}
\end{table} \clearpage
\begin{table}[ht]
\caption{Resultados de la ejecución de la metaheurística IGA, utilizando instancias de Dethloff con la configuración -n 150.0 -p 60.0 -cprob 40 -mprob 70}
\centering
\small
\begin{tabular}{c c c c c c c c}
\hline\hline
Instancia & Costo mínimo & Tiempo(seg.) & Costo promedio & Tiempo promedio(seg.) & CME & \%G & \%GP \\ [0.5ex]
\hline
SCA3-0 & 640.55 & 0.92 & 
640.55 & 0.98 & \bf{635.62} & 
0.78 & 0.78\\SCA3-1 & \bf{697.84} & 0.97 & 
697.84 & 0.86 & 697.84 & 0.00
 & 0.00\\
SCA3-2 & 664.21 & 0.64 & 
669.76 & 0.95 & \bf{659.34} & 
0.74 & 1.58\\SCA3-3 & 682.46 & 1.08 & 
683.28 & 0.81 & \bf{680.04} & 
0.36 & 0.48\\SCA3-4 & \bf{690.50} & 0.73 & 
690.50 & 0.92 & 690.50 & 0.00
 & 0.00\\
SCA3-5 & \bf{659.90} & 0.78 & 
664.21 & 0.97 & 659.90 & 0.00
 & 0.65\\SCA3-6 & 652.94 & 0.69 & 
654.46 & 0.85 & \bf{651.09} & 
0.28 & 0.52\\SCA3-7 & 666.15 & 0.67 & 
666.15 & 0.85 & \bf{659.17} & 
1.06 & 1.06\\SCA3-8 & 723.99 & 0.70 & 
724.71 & 0.84 & \bf{719.47} & 
0.63 & 0.73\\SCA3-9 & \bf{681.00} & 0.87 & 
683.77 & 0.77 & 681.00 & 0.00
 & 0.41\\SCA8-0 & 992.51 & 0.73 & 
992.51 & 0.87 & \bf{961.50} & 
3.23 & 3.23\\SCA8-1 & 1057.99 & 0.76 & 
1058.51 & 0.89 & \bf{1049.65} & 
0.79 & 0.84\\SCA8-2 & 1050.17 & 1.00 & 
1051.37 & 0.98 & \bf{1039.64} & 
1.01 & 1.13\\SCA8-3 & 1014.58 & 0.83 & 
1020.13 & 1.01 & \bf{983.34} & 
3.18 & 3.74\\SCA8-4 & 1067.55 & 0.86 & 
1068.67 & 0.82 & \bf{1065.49} & 
0.19 & 0.30\\SCA8-5 & 1060.81 & 0.83 & 
1060.81 & 0.79 & \bf{1027.08} & 
3.28 & 3.28\\SCA8-6 & 984.15 & 0.68 & 
989.30 & 0.73 & \bf{971.82} & 
1.27 & 1.80\\SCA8-7 & 1067.49 & 1.26 & 
1067.49 & 1.02 & \bf{1051.28} & 
1.54 & 1.54\\SCA8-8 & 1089.91 & 0.68 & 
1089.91 & 0.92 & \bf{1071.18} & 
1.75 & 1.75\\SCA8-9 & 1067.42 & 0.91 & 
1074.26 & 0.82 & \bf{1060.50} & 
0.65 & 1.30\\CON3-0 & 620.29 & 0.84 & 
628.88 & 0.91 & \bf{616.52} & 
0.61 & 2.00\\CON3-1 & 556.92 & 0.74 & 
558.84 & 0.83 & \bf{554.47} & 
0.44 & 0.79\\CON3-2 & 520.38 & 1.30 & 
520.38 & 1.06 & \bf{518.00} & 
0.46 & 0.46\\CON3-3 & 591.48 & 0.92 & 
597.40 & 0.92 & \bf{591.19} & 
0.05 & 1.05\\CON3-4 & 593.78 & 0.81 & 
597.49 & 0.80 & \bf{588.79} & 
0.85 & 1.48\\CON3-5 & 567.63 & 0.80 & 
567.78 & 1.01 & \bf{563.70} & 
0.70 & 0.72\\CON3-6 & 504.20 & 0.82 & 
505.39 & 0.81 & \bf{499.05} & 
1.03 & 1.27\\CON3-7 & 577.91 & 0.63 & 
579.95 & 0.73 & \bf{576.48} & 
0.25 & 0.60\\CON3-8 & 524.59 & 0.66 & 
530.41 & 0.79 & \bf{523.05} & 
0.29 & 1.41\\CON3-9 & 582.98 & 0.71 & 
586.91 & 0.82 & \bf{578.24} & 
0.82 & 1.50\\CON8-0 & 883.76 & 0.87 & 
888.34 & 0.79 & \bf{857.17} & 
3.10 & 3.64\\CON8-1 & 761.06 & 0.97 & 
761.06 & 0.88 & \bf{740.85} & 
2.73 & 2.73\\CON8-2 & 713.60 & 1.12 & 
713.60 & 0.91 & \bf{712.89} & 
0.10 & 0.10\\CON8-3 & 812.22 & 1.15 & 
815.98 & 0.91 & \bf{811.07} & 
0.14 & 0.60\\CON8-4 & 794.18 & 1.01 & 
795.66 & 0.89 & \bf{772.25} & 
2.84 & 3.03\\CON8-5 & 758.12 & 0.96 & 
758.12 & 1.02 & \bf{754.88} & 
0.43 & 0.43\\CON8-6 & 688.51 & 0.68 & 
692.82 & 0.94 & \bf{678.92} & 
1.41 & 2.05\\CON8-7 & 821.21 & 0.75 & 
821.68 & 0.80 & \bf{811.96} & 
1.14 & 1.20\\CON8-8 & 797.99 & 1.03 & 
800.00 & 0.91 & \bf{767.53} & 
3.97 & 4.23\\CON8-9 & 834.63 & 0.82 & 
836.62 & 0.99 & \bf{809.00} & 
3.17 & 3.41\\\bf{PROM.} & 
\bf{767.94} & \bf{0.85} & \bf{770.14} & \bf{0.88} & \bf{758.54} & \bf{1.13} & \bf{1.45}\\[1ex]\hline
\end{tabular}
\label{table:nonlin}
\end{table} \clearpage
\begin{table}[ht]
\caption{Resultados de la ejecución de la metaheurística IGA, utilizando instancias de SalhiNagy con la configuración -n 150.0 -p 60.0 -cprob 90 -mprob 70}
\centering
\small
\begin{tabular}{c c c c c c c c}
\hline\hline
Instancia & Costo mínimo & Tiempo(seg.) & Costo promedio & Tiempo promedio(seg.) & CME & \%G & \%GP \\ [0.5ex]
\hline
CMT1X & 480.02 & 0.85 & 
480.67 & 0.95 & \bf{470.48} & 
2.03 & 2.17\\CMT1Y & 474.87 & 0.80 & 
482.78 & 0.90 & \bf{470.48} & 
0.93 & 2.61\\CMT2X & 711.59 & 1.94 & 
717.79 & 2.08 & \bf{682.39} & 
4.28 & 5.19\\CMT2Y & 710.29 & 1.98 & 
711.03 & 2.07 & \bf{682.39} & 
4.09 & 4.20\\CMT3X & 734.77 & 4.74 & 
740.91 & 4.59 & \bf{719.06} & 
2.18 & 3.04\\CMT3Y & 732.60 & 4.40 & 
735.52 & 4.48 & \bf{719.06} & 
1.88 & 2.29\\CMT4X & 904.21 & 11.11 & 
909.90 & 11.75 & \bf{854.21} & 
5.85 & 6.52\\CMT4Y & 908.57 & 12.04 & 
918.47 & 11.87 & \bf{852.46} & 
6.58 & 7.74\\CMT5X & 1100.73 & 23.90 & 
1113.65 & 24.05 & \bf{1030.56} & 
6.81 & 8.06\\CMT5Y & 1080.80 & 25.44 & 
1110.45 & 24.98 & \bf{1031.69} & 
4.76 & 7.63\\CMT11X & 902.70 & 7.64 & 
916.68 & 7.54 & \bf{831.09} & 
8.62 & 10.30\\CMT11Y & 884.42 & 8.10 & 
909.68 & 8.02 & \bf{829.85} & 
6.58 & 9.62\\CMT12X & 681.31 & 4.80 & 
684.72 & 4.72 & \bf{658.83} & 
3.41 & 3.93\\CMT12Y & 673.13 & 4.62 & 
676.84 & 4.54 & \bf{660.47} & 
1.92 & 2.48\\\bf{PROM.} & 
\bf{784.29} & \bf{8.03} & \bf{793.51} & \bf{8.04} & \bf{749.50} & \bf{4.28} & \bf{5.41}\\[1ex]\hline
\end{tabular}
\label{table:nonlin}
\end{table} \clearpage
\begin{table}[ht]
\caption{Resultados de la ejecución de la metaheurística IGA, utilizando instancias de Dethloff con la configuración -n 150.0 -p 70.0 -cprob 40 -mprob 70}
\centering
\small
\begin{tabular}{c c c c c c c c}
\hline\hline
Instancia & Costo mínimo & Tiempo(seg.) & Costo promedio & Tiempo promedio(seg.) & CME & \%G & \%GP \\ [0.5ex]
\hline
SCA3-0 & 640.55 & 1.13 & 
640.55 & 0.93 & \bf{635.62} & 
0.78 & 0.78\\SCA3-1 & \bf{697.84} & 1.04 & 
697.84 & 1.03 & 697.84 & 0.00
 & 0.00\\
SCA3-2 & \bf{659.34} & 1.00 & 
662.97 & 1.04 & 659.34 & 0.00
 & 0.55\\SCA3-3 & 680.60 & 0.94 & 
683.29 & 0.89 & \bf{680.04} & 
0.08 & 0.48\\SCA3-4 & \bf{690.50} & 0.91 & 
690.50 & 1.03 & 690.50 & 0.00
 & 0.00\\
SCA3-5 & 665.64 & 1.26 & 
671.50 & 0.99 & \bf{659.90} & 
0.87 & 1.76\\SCA3-6 & 652.94 & 0.85 & 
654.73 & 0.89 & \bf{651.09} & 
0.28 & 0.56\\SCA3-7 & 666.15 & 0.81 & 
666.15 & 1.12 & \bf{659.17} & 
1.06 & 1.06\\SCA3-8 & 722.35 & 0.86 & 
726.98 & 1.04 & \bf{719.47} & 
0.40 & 1.04\\SCA3-9 & \bf{681.00} & 1.02 & 
681.00 & 0.90 & 681.00 & 0.00
 & 0.00\\
SCA8-0 & 974.40 & 1.05 & 
979.90 & 0.88 & \bf{961.50} & 
1.34 & 1.91\\SCA8-1 & 1059.35 & 1.25 & 
1065.82 & 1.25 & \bf{1049.65} & 
0.92 & 1.54\\SCA8-2 & 1047.57 & 0.77 & 
1049.29 & 0.96 & \bf{1039.64} & 
0.76 & 0.93\\SCA8-3 & 1007.99 & 1.38 & 
1010.14 & 1.17 & \bf{983.34} & 
2.51 & 2.73\\SCA8-4 & 1071.86 & 1.08 & 
1074.83 & 0.98 & \bf{1065.49} & 
0.60 & 0.88\\SCA8-5 & 1055.74 & 1.16 & 
1057.01 & 0.90 & \bf{1027.08} & 
2.79 & 2.91\\SCA8-6 & 977.03 & 1.15 & 
977.03 & 1.01 & \bf{971.82} & 
0.54 & 0.54\\SCA8-7 & 1077.67 & 0.89 & 
1077.67 & 0.89 & \bf{1051.28} & 
2.51 & 2.51\\SCA8-8 & 1086.27 & 0.78 & 
1086.27 & 0.92 & \bf{1071.18} & 
1.41 & 1.41\\SCA8-9 & 1061.23 & 1.01 & 
1069.49 & 1.05 & \bf{1060.50} & 
0.07 & 0.85\\CON3-0 & 624.96 & 1.16 & 
627.34 & 1.13 & \bf{616.52} & 
1.37 & 1.76\\CON3-1 & 556.04 & 1.28 & 
556.04 & 1.13 & \bf{554.47} & 
0.28 & 0.28\\CON3-2 & 521.38 & 0.94 & 
521.38 & 1.07 & \bf{518.00} & 
0.65 & 0.65\\CON3-3 & 591.20 & 0.99 & 
591.41 & 1.04 & \bf{591.19} & 
0.00 & 0.04\\CON3-4 & 592.58 & 1.39 & 
592.58 & 1.14 & \bf{588.79} & 
0.64 & 0.64\\CON3-5 & \bf{563.70} & 1.18 & 
563.70 & 1.06 & 563.70 & 0.00
 & 0.00\\
CON3-6 & 505.14 & 1.12 & 
505.43 & 1.11 & \bf{499.05} & 
1.22 & 1.28\\CON3-7 & 577.54 & 0.76 & 
581.78 & 0.93 & \bf{576.48} & 
0.18 & 0.92\\CON3-8 & 524.59 & 1.19 & 
525.09 & 0.99 & \bf{523.05} & 
0.29 & 0.39\\CON3-9 & 588.11 & 0.94 & 
589.34 & 1.11 & \bf{578.24} & 
1.71 & 1.92\\CON8-0 & 868.83 & 0.84 & 
868.83 & 1.08 & \bf{857.17} & 
1.36 & 1.36\\CON8-1 & 753.06 & 0.92 & 
756.07 & 1.12 & \bf{740.85} & 
1.65 & 2.05\\CON8-2 & 727.29 & 1.13 & 
728.07 & 1.14 & \bf{712.89} & 
2.02 & 2.13\\CON8-3 & 824.10 & 1.36 & 
831.97 & 1.10 & \bf{811.07} & 
1.61 & 2.58\\CON8-4 & 781.05 & 0.75 & 
783.91 & 0.76 & \bf{772.25} & 
1.14 & 1.51\\CON8-5 & 767.16 & 1.06 & 
768.14 & 0.99 & \bf{754.88} & 
1.63 & 1.76\\CON8-6 & 681.11 & 1.46 & 
681.11 & 1.12 & \bf{678.92} & 
0.32 & 0.32\\CON8-7 & 821.21 & 1.12 & 
821.21 & 0.94 & \bf{811.96} & 
1.14 & 1.14\\CON8-8 & 780.71 & 1.08 & 
781.31 & 1.09 & \bf{767.53} & 
1.72 & 1.80\\CON8-9 & 821.09 & 1.46 & 
829.29 & 1.14 & \bf{809.00} & 
1.49 & 2.51\\\bf{PROM.} & 
\bf{766.17} & \bf{1.06} & \bf{768.17} & \bf{1.03} & \bf{758.54} & \bf{0.93} & \bf{1.19}\\[1ex]\hline
\end{tabular}
\label{table:nonlin}
\end{table} \clearpage
\begin{table}[ht]
\caption{Resultados de la ejecución de la metaheurística IGA, utilizando instancias de SalhiNagy con la configuración -n 150.0 -p 70.0 -cprob 90 -mprob 70}
\centering
\small
\begin{tabular}{c c c c c c c c}
\hline\hline
Instancia & Costo mínimo & Tiempo(seg.) & Costo promedio & Tiempo promedio(seg.) & CME & \%G & \%GP \\ [0.5ex]
\hline
CMT1X & 474.72 & 1.31 & 
481.37 & 1.10 & \bf{470.48} & 
0.90 & 2.31\\CMT1Y & 480.19 & 0.94 & 
483.41 & 1.04 & \bf{470.48} & 
2.06 & 2.75\\CMT2X & 696.23 & 2.66 & 
697.61 & 2.37 & \bf{682.39} & 
2.03 & 2.23\\CMT2Y & 715.05 & 2.26 & 
718.94 & 2.38 & \bf{682.39} & 
4.79 & 5.36\\CMT3X & 739.57 & 5.53 & 
743.26 & 5.51 & \bf{719.06} & 
2.85 & 3.37\\CMT3Y & 735.83 & 5.05 & 
744.98 & 5.17 & \bf{719.06} & 
2.33 & 3.60\\CMT4X & 882.24 & 12.81 & 
906.58 & 13.46 & \bf{854.21} & 
3.28 & 6.13\\CMT4Y & 911.09 & 14.04 & 
919.06 & 14.29 & \bf{852.46} & 
6.88 & 7.81\\CMT5X & 1100.79 & 28.61 & 
1115.86 & 28.20 & \bf{1030.56} & 
6.81 & 8.28\\CMT5Y & 1099.82 & 28.48 & 
1120.33 & 28.27 & \bf{1031.69} & 
6.60 & 8.59\\CMT11X & 900.21 & 8.79 & 
905.55 & 8.96 & \bf{831.09} & 
8.32 & 8.96\\CMT11Y & 895.62 & 9.36 & 
899.15 & 9.71 & \bf{829.85} & 
7.93 & 8.35\\CMT12X & 675.26 & 4.74 & 
677.91 & 5.35 & \bf{658.83} & 
2.49 & 2.90\\CMT12Y & 675.52 & 5.16 & 
677.99 & 5.37 & \bf{660.47} & 
2.28 & 2.65\\\bf{PROM.} & 
\bf{784.44} & \bf{9.27} & \bf{792.29} & \bf{9.37} & \bf{749.50} & \bf{4.25} & \bf{5.24}\\[1ex]\hline
\end{tabular}
\label{table:nonlin}
\end{table} \clearpage
\begin{table}[ht]
\caption{Resultados de la ejecución de la metaheurística IGA, utilizando instancias de Dethloff con la configuración -n 150.0 -p 80.0 -cprob 40 -mprob 70}
\centering
\small
\begin{tabular}{c c c c c c c c}
\hline\hline
Instancia & Costo mínimo & Tiempo(seg.) & Costo promedio & Tiempo promedio(seg.) & CME & \%G & \%GP \\ [0.5ex]
\hline
SCA3-0 & 640.55 & 1.28 & 
640.84 & 1.14 & \bf{635.62} & 
0.78 & 0.82\\SCA3-1 & \bf{697.84} & 0.97 & 
699.84 & 1.01 & 697.84 & 0.00
 & 0.29\\SCA3-2 & \bf{659.34} & 1.50 & 
662.99 & 1.15 & 659.34 & 0.00
 & 0.55\\SCA3-3 & \bf{680.04} & 1.00 & 
681.95 & 1.05 & 680.04 & 0.00
 & 0.28\\SCA3-4 & \bf{690.50} & 1.40 & 
690.50 & 1.25 & 690.50 & 0.00
 & 0.00\\
SCA3-5 & 666.67 & 1.34 & 
673.70 & 1.19 & \bf{659.90} & 
1.03 & 2.09\\SCA3-6 & 652.94 & 1.29 & 
653.27 & 1.15 & \bf{651.09} & 
0.28 & 0.33\\SCA3-7 & 666.15 & 1.02 & 
666.15 & 1.13 & \bf{659.17} & 
1.06 & 1.06\\SCA3-8 & 719.77 & 1.03 & 
720.83 & 0.95 & \bf{719.47} & 
0.04 & 0.19\\SCA3-9 & \bf{681.00} & 1.26 & 
681.51 & 1.08 & 681.00 & 0.00
 & 0.07\\SCA8-0 & 990.51 & 1.07 & 
998.56 & 1.28 & \bf{961.50} & 
3.02 & 3.85\\SCA8-1 & 1054.11 & 1.37 & 
1076.21 & 1.22 & \bf{1049.65} & 
0.42 & 2.53\\SCA8-2 & 1052.94 & 0.95 & 
1053.09 & 1.05 & \bf{1039.64} & 
1.28 & 1.29\\SCA8-3 & 1015.72 & 1.41 & 
1015.72 & 1.11 & \bf{983.34} & 
3.29 & 3.29\\SCA8-4 & 1068.97 & 0.85 & 
1075.20 & 1.01 & \bf{1065.49} & 
0.33 & 0.91\\SCA8-5 & 1056.53 & 1.05 & 
1060.50 & 1.14 & \bf{1027.08} & 
2.87 & 3.25\\SCA8-6 & 982.05 & 0.84 & 
982.05 & 1.09 & \bf{971.82} & 
1.05 & 1.05\\SCA8-7 & 1080.35 & 0.84 & 
1080.35 & 1.20 & \bf{1051.28} & 
2.77 & 2.77\\SCA8-8 & \bf{1071.18} & 0.86 & 
1071.18 & 1.05 & 1071.18 & 0.00
 & 0.00\\
SCA8-9 & 1071.72 & 0.97 & 
1076.33 & 0.96 & \bf{1060.50} & 
1.06 & 1.49\\CON3-0 & 619.09 & 1.31 & 
623.67 & 1.21 & \bf{616.52} & 
0.42 & 1.16\\CON3-1 & 560.75 & 0.94 & 
560.75 & 1.04 & \bf{554.47} & 
1.13 & 1.13\\CON3-2 & 521.38 & 1.30 & 
521.38 & 1.43 & \bf{518.00} & 
0.65 & 0.65\\CON3-3 & 591.20 & 1.15 & 
596.37 & 1.19 & \bf{591.19} & 
0.00 & 0.88\\CON3-4 & 592.58 & 0.97 & 
595.62 & 1.27 & \bf{588.79} & 
0.64 & 1.16\\CON3-5 & 564.88 & 1.00 & 
567.38 & 1.10 & \bf{563.70} & 
0.21 & 0.65\\CON3-6 & 504.20 & 1.10 & 
504.44 & 1.19 & \bf{499.05} & 
1.03 & 1.08\\CON3-7 & 577.54 & 1.22 & 
579.67 & 1.20 & \bf{576.48} & 
0.18 & 0.55\\CON3-8 & \bf{523.05} & 1.27 & 
524.63 & 1.16 & 523.05 & 0.00
 & 0.30\\CON3-9 & 588.11 & 1.37 & 
588.93 & 1.24 & \bf{578.24} & 
1.71 & 1.85\\CON8-0 & 866.22 & 1.09 & 
867.74 & 1.31 & \bf{857.17} & 
1.06 & 1.23\\CON8-1 & 754.20 & 1.07 & 
756.67 & 1.08 & \bf{740.85} & 
1.80 & 2.14\\CON8-2 & 716.07 & 1.21 & 
718.35 & 1.19 & \bf{712.89} & 
0.45 & 0.77\\CON8-3 & 813.84 & 0.99 & 
813.84 & 1.10 & \bf{811.07} & 
0.34 & 0.34\\CON8-4 & \bf{772.25} & 1.15 & 
772.25 & 1.19 & 772.25 & 0.00
 & 0.00\\
CON8-5 & 754.95 & 1.21 & 
765.61 & 1.16 & \bf{754.88} & 
0.01 & 1.42\\CON8-6 & 695.86 & 1.47 & 
695.86 & 1.17 & \bf{678.92} & 
2.50 & 2.50\\CON8-7 & 815.80 & 0.95 & 
815.80 & 1.12 & \bf{811.96} & 
0.47 & 0.47\\CON8-8 & 784.56 & 1.11 & 
785.33 & 1.24 & \bf{767.53} & 
2.22 & 2.32\\CON8-9 & 840.91 & 1.77 & 
842.70 & 1.15 & \bf{809.00} & 
3.94 & 4.17\\\bf{PROM.} & 
\bf{766.41} & \bf{1.15} & \bf{768.94} & \bf{1.15} & \bf{758.54} & \bf{0.95} & \bf{1.27}\\[1ex]\hline
\end{tabular}
\label{table:nonlin}
\end{table} \clearpage
\begin{table}[ht]
\caption{Resultados de la ejecución de la metaheurística IGA, utilizando instancias de SalhiNagy con la configuración -n 150.0 -p 80.0 -cprob 90 -mprob 70}
\centering
\small
\begin{tabular}{c c c c c c c c}
\hline\hline
Instancia & Costo mínimo & Tiempo(seg.) & Costo promedio & Tiempo promedio(seg.) & CME & \%G & \%GP \\ [0.5ex]
\hline
CMT1X & 484.85 & 1.13 & 
486.73 & 1.37 & \bf{470.48} & 
3.05 & 3.45\\CMT1Y & 473.91 & 1.34 & 
481.94 & 1.15 & \bf{470.48} & 
0.73 & 2.43\\CMT2X & 707.82 & 3.09 & 
714.65 & 2.86 & \bf{682.39} & 
3.73 & 4.73\\CMT2Y & 698.35 & 2.87 & 
708.77 & 2.74 & \bf{682.39} & 
2.34 & 3.87\\CMT3X & 736.52 & 5.98 & 
740.29 & 5.97 & \bf{719.06} & 
2.43 & 2.95\\CMT3Y & 733.56 & 5.75 & 
742.05 & 6.01 & \bf{719.06} & 
2.02 & 3.20\\CMT4X & 893.61 & 16.33 & 
899.96 & 15.80 & \bf{854.21} & 
4.61 & 5.36\\CMT4Y & 912.56 & 16.79 & 
913.17 & 16.59 & \bf{852.46} & 
7.05 & 7.12\\CMT5X & 1094.98 & 31.60 & 
1107.18 & 32.28 & \bf{1030.56} & 
6.25 & 7.43\\CMT5Y & 1103.78 & 32.68 & 
1114.21 & 33.06 & \bf{1031.69} & 
6.99 & 8.00\\CMT11X & 897.72 & 9.80 & 
910.02 & 10.23 & \bf{831.09} & 
8.02 & 9.50\\CMT11Y & 882.86 & 11.28 & 
889.09 & 11.32 & \bf{829.85} & 
6.39 & 7.14\\CMT12X & 674.79 & 5.90 & 
675.09 & 6.04 & \bf{658.83} & 
2.42 & 2.47\\CMT12Y & 675.94 & 6.01 & 
677.51 & 6.01 & \bf{660.47} & 
2.34 & 2.58\\\bf{PROM.} & 
\bf{783.66} & \bf{10.75} & \bf{790.05} & \bf{10.82} & \bf{749.50} & \bf{4.17} & \bf{5.02}\\[1ex]\hline
\end{tabular}
\label{table:nonlin}
\end{table} \clearpage
\begin{table}[ht]
\caption{Resultados de la ejecución de la metaheurística IGA, utilizando instancias de Dethloff con la configuración -n 150.0 -p 90.0 -cprob 40 -mprob 70}
\centering
\small
\begin{tabular}{c c c c c c c c}
\hline\hline
Instancia & Costo mínimo & Tiempo(seg.) & Costo promedio & Tiempo promedio(seg.) & CME & \%G & \%GP \\ [0.5ex]
\hline
SCA3-0 & 640.55 & 1.67 & 
640.82 & 1.30 & \bf{635.62} & 
0.78 & 0.82\\SCA3-1 & 700.50 & 1.14 & 
700.76 & 1.23 & \bf{697.84} & 
0.38 & 0.42\\SCA3-2 & 661.13 & 1.54 & 
664.64 & 1.24 & \bf{659.34} & 
0.27 & 0.80\\SCA3-3 & \bf{680.04} & 1.14 & 
680.04 & 1.25 & 680.04 & 0.00
 & 0.00\\
SCA3-4 & \bf{690.50} & 1.00 & 
690.50 & 1.07 & 690.50 & 0.00
 & 0.00\\
SCA3-5 & 666.67 & 1.35 & 
666.67 & 1.36 & \bf{659.90} & 
1.03 & 1.03\\SCA3-6 & \bf{651.09} & 1.40 & 
652.48 & 1.31 & 651.09 & 0.00
 & 0.21\\SCA3-7 & 666.15 & 1.14 & 
666.49 & 1.29 & \bf{659.17} & 
1.06 & 1.11\\SCA3-8 & \bf{719.47} & 0.95 & 
719.54 & 1.21 & 719.47 & 0.00
 & 0.01\\SCA3-9 & \bf{681.00} & 1.46 & 
681.00 & 1.25 & 681.00 & 0.00
 & 0.00\\
SCA8-0 & 973.22 & 1.18 & 
976.10 & 1.45 & \bf{961.50} & 
1.22 & 1.52\\SCA8-1 & 1073.51 & 1.11 & 
1075.99 & 1.36 & \bf{1049.65} & 
2.27 & 2.51\\SCA8-2 & 1053.94 & 1.38 & 
1053.94 & 1.32 & \bf{1039.64} & 
1.38 & 1.38\\SCA8-3 & 1023.87 & 1.45 & 
1023.91 & 1.46 & \bf{983.34} & 
4.12 & 4.13\\SCA8-4 & 1070.32 & 1.38 & 
1074.28 & 1.32 & \bf{1065.49} & 
0.45 & 0.82\\SCA8-5 & 1047.83 & 1.00 & 
1051.77 & 1.08 & \bf{1027.08} & 
2.02 & 2.40\\SCA8-6 & 972.48 & 1.52 & 
974.59 & 1.41 & \bf{971.82} & 
0.07 & 0.28\\SCA8-7 & 1075.25 & 1.22 & 
1075.25 & 1.23 & \bf{1051.28} & 
2.28 & 2.28\\SCA8-8 & 1088.65 & 1.02 & 
1092.11 & 1.35 & \bf{1071.18} & 
1.63 & 1.95\\SCA8-9 & 1072.76 & 1.11 & 
1080.73 & 1.30 & \bf{1060.50} & 
1.16 & 1.91\\CON3-0 & 619.09 & 1.62 & 
628.23 & 1.22 & \bf{616.52} & 
0.42 & 1.90\\CON3-1 & 557.21 & 1.30 & 
560.42 & 1.53 & \bf{554.47} & 
0.49 & 1.07\\CON3-2 & 521.38 & 1.32 & 
522.09 & 1.48 & \bf{518.00} & 
0.65 & 0.79\\CON3-3 & 591.36 & 1.24 & 
592.81 & 1.38 & \bf{591.19} & 
0.03 & 0.27\\CON3-4 & 591.43 & 1.32 & 
593.56 & 1.34 & \bf{588.79} & 
0.45 & 0.81\\CON3-5 & \bf{563.70} & 1.54 & 
563.70 & 1.34 & 563.70 & 0.00
 & 0.00\\
CON3-6 & 504.20 & 1.10 & 
505.25 & 1.51 & \bf{499.05} & 
1.03 & 1.24\\CON3-7 & 581.83 & 1.58 & 
582.95 & 1.27 & \bf{576.48} & 
0.93 & 1.12\\CON3-8 & 523.14 & 1.37 & 
526.82 & 1.40 & \bf{523.05} & 
0.02 & 0.72\\CON3-9 & 582.79 & 1.27 & 
584.34 & 1.34 & \bf{578.24} & 
0.79 & 1.06\\CON8-0 & 872.74 & 1.80 & 
876.99 & 1.50 & \bf{857.17} & 
1.82 & 2.31\\CON8-1 & 756.42 & 1.32 & 
756.42 & 1.27 & \bf{740.85} & 
2.10 & 2.10\\CON8-2 & 720.09 & 1.43 & 
722.28 & 1.56 & \bf{712.89} & 
1.01 & 1.32\\CON8-3 & 831.12 & 1.40 & 
831.12 & 1.50 & \bf{811.07} & 
2.47 & 2.47\\CON8-4 & 789.85 & 1.13 & 
789.85 & 1.29 & \bf{772.25} & 
2.28 & 2.28\\CON8-5 & 765.76 & 1.35 & 
770.03 & 1.55 & \bf{754.88} & 
1.44 & 2.01\\CON8-6 & 705.08 & 1.77 & 
705.57 & 1.35 & \bf{678.92} & 
3.85 & 3.92\\CON8-7 & 816.00 & 1.15 & 
820.08 & 1.12 & \bf{811.96} & 
0.50 & 1.00\\CON8-8 & 797.69 & 1.55 & 
797.69 & 1.30 & \bf{767.53} & 
3.93 & 3.93\\CON8-9 & 811.79 & 1.17 & 
817.50 & 1.24 & \bf{809.00} & 
0.34 & 1.05\\\bf{PROM.} & 
\bf{767.79} & \bf{1.32} & \bf{769.73} & \bf{1.33} & \bf{758.54} & \bf{1.12} & \bf{1.37}\\[1ex]\hline
\end{tabular}
\label{table:nonlin}
\end{table} \clearpage
\begin{table}[ht]
\caption{Resultados de la ejecución de la metaheurística PSO, utilizando instancias de SalhiNagy con la configuración -n 30.0 -L 90.0 -cp 1 -cg 0 -cl 1 -cn 2 -w1 0.9 -wt 0.1 -K 5}
\centering
\small
\begin{tabular}{c c c c c c c c}
\hline\hline
Instancia & Costo mínimo & Tiempo(seg.) & Costo promedio & Tiempo promedio(seg.) & CME & \%G & \%GP \\ [0.5ex]
\hline
CMT1X & \bf{470.48} & 16.33 & 
490.64 & 17.06 & 470.48 & 0.00
 & 4.28\\CMT1Y & 470.67 & 16.37 & 
474.83 & 18.51 & \bf{470.48} & 
0.04 & 0.92\\CMT2X & 702.55 & 30.13 & 
743.18 & 27.91 & \bf{682.39} & 
2.95 & 8.91\\CMT2Y & 695.31 & 23.71 & 
718.22 & 21.77 & \bf{682.39} & 
1.89 & 5.25\\CMT3X & 725.97 & 216.73 & 
740.18 & 160.99 & \bf{719.06} & 
0.96 & 2.94\\CMT3Y & 728.14 & 214.60 & 
733.82 & 208.38 & \bf{719.06} & 
1.26 & 2.05\\CMT4X & 884.91 & 289.95 & 
907.23 & 334.20 & \bf{854.21} & 
3.59 & 6.21\\CMT4Y & 893.67 & 352.51 & 
904.29 & 269.57 & \bf{852.46} & 
4.83 & 6.08\\CMT5X & 1133.27 & | & 
0.00 & 0.00 & \bf{1030.56} & 
9.97 & -100.00\\CMT5Y & 1093.10 & 797.87 & 
1140.65 & 471.40 & \bf{1031.69} & 
5.95 & 10.56\\CMT11X & 886.35 & 102.57 & 
951.33 & 87.20 & \bf{831.09} & 
6.65 & 14.47\\CMT11Y & 884.91 & 86.31 & 
899.13 & 100.23 & \bf{829.85} & 
6.63 & 8.35\\CMT12X & 735.49 & 24.06 & 
744.01 & 17.33 & \bf{658.83} & 
11.64 & 12.93\\CMT12Y & 686.18 & 16.51 & 
751.10 & 18.28 & \bf{660.47} & 
3.89 & 13.72\\\bf{PROM.} & 
\bf{785.07} & \bf{156.26} & \bf{728.47} & \bf{125.20} & \bf{749.50} & \bf{4.31} & \bf{-0.24}\\[1ex]\hline
\end{tabular}
\label{table:nonlin}
\end{table} \clearpage
\begin{table}[ht]
\caption{Resultados de la ejecución de la metaheurística SCA, utilizando instancias de SalhiNagy con la configuración -n 200.0 -b 10 -y 0.1}
\centering
\small
\begin{tabular}{c c c c c c c c}
\hline\hline
Instancia & Costo mínimo & Tiempo(seg.) & Costo promedio & Tiempo promedio(seg.) & CME & \%G & \%GP \\ [0.5ex]
\hline
CMT1X & 483.52 & 1.82 & 
483.52 & 1.80 & \bf{470.48} & 
2.77 & 2.77\\CMT1Y & 472.37 & 1.24 & 
472.37 & 1.28 & \bf{470.48} & 
0.40 & 0.40\\CMT2X & 709.35 & 18.29 & 
711.50 & 16.68 & \bf{682.39} & 
3.95 & 4.27\\CMT2Y & 705.78 & 20.25 & 
708.34 & 20.43 & \bf{682.39} & 
3.43 & 3.80\\CMT3X & 740.64 & 24.23 & 
743.11 & 28.98 & \bf{719.06} & 
3.00 & 3.34\\CMT3Y & 728.89 & 18.50 & 
736.64 & 34.58 & \bf{719.06} & 
1.37 & 2.44\\CMT4X & 906.92914.06896.56 & | & 
0.00 & 0.00 & \bf{854.21} & 
6.17 & -100.00\\CMT4Y & 100000 & 0 & 
912.29 & 171.64 & \bf{852.46} & 
11630.76 & 7.02\\CMT5X & 100000 & 0 & 
nan & nan & \bf{1030.56} & 
9603.46 & \bf{nan}\\CMT5Y & 100000 & 0 & 
nan & nan & \bf{1031.69} & 
9592.83 & \bf{nan}\\CMT11X & 889.25 & 67.71 & 
905.69 & 36.89 & \bf{831.09} & 
7.00 & 8.98\\CMT11Y & 902.53 & 18.00 & 
909.23 & 45.40 & \bf{829.85} & 
8.76 & 9.57\\CMT12X & 685.58 & 94.41 & 
687.65 & 59.12 & \bf{658.83} & 
4.06 & 4.37\\CMT12Y & 679.89 & 111.11 & 
688.17 & 73.58 & \bf{660.47} & 
2.94 & 4.19\\\bf{PROM.} & 
\bf{21993.19} & \bf{26.83} & \bf{nan} & \bf{nan} & \bf{749.50} & \bf{2205.06} & \bf{nan}\\[1ex]\hline
\end{tabular}
\label{table:nonlin}
\end{table} \clearpage
\begin{table}[ht]
\caption{Resultados de la ejecución de la metaheurística IGA, utilizando instancias de SalhiNagy con la configuración -n 150.0 -p 90.0 -cprob 90 -mprob 70}
\centering
\small
\begin{tabular}{c c c c c c c c}
\hline\hline
Instancia & Costo mínimo & Tiempo(seg.) & Costo promedio & Tiempo promedio(seg.) & CME & \%G & \%GP \\ [0.5ex]
\hline
CMT1X & 478.84 & 1.29 & 
480.56 & 1.51 & \bf{470.48} & 
1.78 & 2.14\\CMT1Y & 472.37 & 1.48 & 
474.89 & 1.27 & \bf{470.48} & 
0.40 & 0.94\\CMT2X & 704.85 & 3.18 & 
710.67 & 3.10 & \bf{682.39} & 
3.29 & 4.14\\CMT2Y & 699.36 & 3.36 & 
700.38 & 2.88 & \bf{682.39} & 
2.49 & 2.64\\CMT3X & 735.88 & 6.59 & 
742.46 & 6.58 & \bf{719.06} & 
2.34 & 3.25\\CMT3Y & 738.14 & 6.53 & 
739.68 & 6.75 & \bf{719.06} & 
2.65 & 2.87\\CMT4X & 893.94 & 17.29 & 
902.25 & 17.95 & \bf{854.21} & 
4.65 & 5.62\\CMT4Y & 884.91 & 18.82 & 
907.66 & 17.92 & \bf{852.46} & 
3.81 & 6.48\\CMT5X & 1109.64 & 35.16 & 
1119.74 & 36.10 & \bf{1030.56} & 
7.67 & 8.65\\CMT5Y & 1090.15 & 36.15 & 
1109.64 & 35.71 & \bf{1031.69} & 
5.67 & 7.56\\CMT11X & 896.42 & 11.31 & 
904.59 & 11.52 & \bf{831.09} & 
7.86 & 8.84\\CMT11Y & 895.33 & 12.93 & 
904.28 & 12.54 & \bf{829.85} & 
7.89 & 8.97\\CMT12X & 674.50 & 6.66 & 
676.76 & 6.78 & \bf{658.83} & 
2.38 & 2.72\\CMT12Y & 673.67 & 6.58 & 
678.20 & 6.71 & \bf{660.47} & 
2.00 & 2.68\\\bf{PROM.} & 
\bf{782.00} & \bf{11.95} & \bf{789.41} & \bf{11.95} & \bf{749.50} & \bf{3.92} & \bf{4.82}\\[1ex]\hline
\end{tabular}
\label{table:nonlin}
\end{table} \clearpage
\begin{table}[ht]
\caption{Resultados de la ejecución de la metaheurística IGA, utilizando instancias de Dethloff con la configuración -n 150.0 -p 100.0 -cprob 40 -mprob 70}
\centering
\small
\begin{tabular}{c c c c c c c c}
\hline\hline
Instancia & Costo mínimo & Tiempo(seg.) & Costo promedio & Tiempo promedio(seg.) & CME & \%G & \%GP \\ [0.5ex]
\hline
SCA3-0 & 636.06 & 1.29 & 
639.43 & 1.58 & \bf{635.62} & 
0.07 & 0.60\\SCA3-1 & 700.50 & 1.10 & 
700.50 & 1.52 & \bf{697.84} & 
0.38 & 0.38\\SCA3-2 & 664.18 & 1.49 & 
664.18 & 1.66 & \bf{659.34} & 
0.73 & 0.73\\SCA3-3 & 681.16 & 1.47 & 
682.13 & 1.48 & \bf{680.04} & 
0.16 & 0.31\\SCA3-4 & \bf{690.50} & 1.07 & 
690.50 & 1.38 & 690.50 & 0.00
 & 0.00\\
SCA3-5 & 673.46 & 1.72 & 
673.49 & 1.50 & \bf{659.90} & 
2.05 & 2.06\\SCA3-6 & 652.94 & 1.51 & 
653.27 & 1.36 & \bf{651.09} & 
0.28 & 0.33\\SCA3-7 & 666.15 & 1.20 & 
666.15 & 1.39 & \bf{659.17} & 
1.06 & 1.06\\SCA3-8 & \bf{719.47} & 1.62 & 
723.09 & 1.61 & 719.47 & 0.00
 & 0.50\\SCA3-9 & \bf{681.00} & 1.21 & 
681.00 & 1.34 & 681.00 & 0.00
 & 0.00\\
SCA8-0 & 987.04 & 1.28 & 
989.42 & 1.38 & \bf{961.50} & 
2.66 & 2.90\\SCA8-1 & 1073.68 & 1.20 & 
1073.68 & 1.26 & \bf{1049.65} & 
2.29 & 2.29\\SCA8-2 & 1049.22 & 1.06 & 
1049.22 & 1.28 & \bf{1039.64} & 
0.92 & 0.92\\SCA8-3 & 1015.72 & 1.62 & 
1017.66 & 1.61 & \bf{983.34} & 
3.29 & 3.49\\SCA8-4 & 1077.80 & 2.01 & 
1080.52 & 1.60 & \bf{1065.49} & 
1.16 & 1.41\\SCA8-5 & 1050.46 & 1.10 & 
1050.46 & 1.30 & \bf{1027.08} & 
2.28 & 2.28\\SCA8-6 & 984.42 & 1.49 & 
985.96 & 1.46 & \bf{971.82} & 
1.30 & 1.46\\SCA8-7 & 1075.76 & 1.02 & 
1075.76 & 1.45 & \bf{1051.28} & 
2.33 & 2.33\\SCA8-8 & 1084.41 & 1.22 & 
1084.41 & 1.29 & \bf{1071.18} & 
1.24 & 1.24\\SCA8-9 & 1074.27 & 1.20 & 
1074.27 & 1.54 & \bf{1060.50} & 
1.30 & 1.30\\CON3-0 & 619.09 & 1.33 & 
625.39 & 1.48 & \bf{616.52} & 
0.42 & 1.44\\CON3-1 & 559.25 & 2.03 & 
559.72 & 7.80 & \bf{554.47} & 
0.86 & 0.95\\CON3-2 & 521.38 & 1.50 & 
521.38 & 1.59 & \bf{518.00} & 
0.65 & 0.65\\CON3-3 & 591.20 & 1.50 & 
593.00 & 1.66 & \bf{591.19} & 
0.00 & 0.31\\CON3-4 & 593.78 & 1.36 & 
593.78 & 1.51 & \bf{588.79} & 
0.85 & 0.85\\CON3-5 & \bf{563.70} & 1.72 & 
564.97 & 1.60 & 563.70 & 0.00
 & 0.22\\CON3-6 & \bf{499.05} & 1.63 & 
501.12 & 1.59 & 499.05 & 0.00
 & 0.41\\CON3-7 & 577.91 & 1.32 & 
579.26 & 1.38 & \bf{576.48} & 
0.25 & 0.48\\CON3-8 & \bf{523.05} & 1.37 & 
524.21 & 1.66 & 523.05 & 0.00
 & 0.22\\CON3-9 & 588.48 & 1.42 & 
589.25 & 1.48 & \bf{578.24} & 
1.77 & 1.90\\CON8-0 & 876.41 & 1.35 & 
876.41 & 1.38 & \bf{857.17} & 
2.24 & 2.24\\CON8-1 & 746.14 & 1.78 & 
746.14 & 1.63 & \bf{740.85} & 
0.71 & 0.71\\CON8-2 & 716.07 & 2.29 & 
717.43 & 1.62 & \bf{712.89} & 
0.45 & 0.64\\CON8-3 & 828.60 & 1.48 & 
832.59 & 1.50 & \bf{811.07} & 
2.16 & 2.65\\CON8-4 & 781.42 & 1.13 & 
781.42 & 1.40 & \bf{772.25} & 
1.19 & 1.19\\CON8-5 & 761.01 & 2.03 & 
761.01 & 1.85 & \bf{754.88} & 
0.81 & 0.81\\CON8-6 & 683.66 & 1.74 & 
683.66 & 1.58 & \bf{678.92} & 
0.70 & 0.70\\CON8-7 & 817.41 & 1.54 & 
817.41 & 1.54 & \bf{811.96} & 
0.67 & 0.67\\CON8-8 & 783.88 & 1.40 & 
784.39 & 1.55 & \bf{767.53} & 
2.13 & 2.20\\CON8-9 & 819.48 & 1.64 & 
819.48 & 1.56 & \bf{809.00} & 
1.30 & 1.30\\\bf{PROM.} & 
\bf{767.23} & \bf{1.46} & \bf{768.18} & \bf{1.66} & \bf{758.54} & \bf{1.02} & \bf{1.15}\\[1ex]\hline
\end{tabular}
\label{table:nonlin}
\end{table} \clearpage
\begin{table}[ht]
\caption{Resultados de la ejecución de la metaheurística IGA, utilizando instancias de SalhiNagy con la configuración -n 150.0 -p 100.0 -cprob 90 -mprob 70}
\centering
\small
\begin{tabular}{c c c c c c c c}
\hline\hline
Instancia & Costo mínimo & Tiempo(seg.) & Costo promedio & Tiempo promedio(seg.) & CME & \%G & \%GP \\ [0.5ex]
\hline
CMT1X & 479.19 & 1.80 & 
479.19 & 1.68 & \bf{470.48} & 
1.85 & 1.85\\CMT1Y & 479.29 & 1.32 & 
480.53 & 1.39 & \bf{470.48} & 
1.87 & 2.14\\CMT2X & 696.06 & 3.30 & 
711.07 & 3.48 & \bf{682.39} & 
2.00 & 4.20\\CMT2Y & 705.18 & 3.24 & 
711.30 & 3.18 & \bf{682.39} & 
3.34 & 4.24\\CMT3X & 741.36 & 7.17 & 
743.90 & 7.57 & \bf{719.06} & 
3.10 & 3.46\\CMT3Y & 734.03 & 7.46 & 
740.83 & 7.24 & \bf{719.06} & 
2.08 & 3.03\\CMT4X & 883.54 & 20.09 & 
903.91 & 19.71 & \bf{854.21} & 
3.43 & 5.82\\CMT4Y & 873.63 & 20.92 & 
892.97 & 20.49 & \bf{852.46} & 
2.48 & 4.75\\CMT5X & 1107.80 & 40.68 & 
1116.40 & 40.80 & \bf{1030.56} & 
7.49 & 8.33\\CMT5Y & 1103.29 & 39.57 & 
1117.01 & 40.84 & \bf{1031.69} & 
6.94 & 8.27\\CMT11X & 899.89 & 12.09 & 
903.72 & 11.86 & \bf{831.09} & 
8.28 & 8.74\\CMT11Y & 862.67 & 14.10 & 
884.38 & 13.74 & \bf{829.85} & 
3.95 & 6.57\\CMT12X & 675.53 & 7.90 & 
676.34 & 7.77 & \bf{658.83} & 
2.53 & 2.66\\CMT12Y & 674.85 & 7.32 & 
675.21 & 7.20 & \bf{660.47} & 
2.18 & 2.23\\\bf{PROM.} & 
\bf{779.74} & \bf{13.35} & \bf{788.34} & \bf{13.35} & \bf{749.50} & \bf{3.68} & \bf{4.73}\\[1ex]\hline
\end{tabular}
\label{table:nonlin}
\end{table} \clearpage
\begin{table}[ht]
\caption{Resultados de la ejecución de la metaheurística IGA, utilizando instancias de Dethloff con la configuración -n 200.0 -p 30.0 -cprob 40 -mprob 70}
\centering
\small
\begin{tabular}{c c c c c c c c}
\hline\hline
Instancia & Costo mínimo & Tiempo(seg.) & Costo promedio & Tiempo promedio(seg.) & CME & \%G & \%GP \\ [0.5ex]
\hline
SCA3-0 & 641.69 & 0.40 & 
641.69 & 0.45 & \bf{635.62} & 
0.95 & 0.95\\SCA3-1 & \bf{697.84} & 0.50 & 
702.24 & 0.41 & 697.84 & 0.00
 & 0.63\\SCA3-2 & \bf{659.34} & 0.47 & 
661.77 & 0.44 & 659.34 & 0.00
 & 0.37\\SCA3-3 & 685.05 & 0.42 & 
685.33 & 0.40 & \bf{680.04} & 
0.74 & 0.78\\SCA3-4 & \bf{690.50} & 0.33 & 
691.18 & 0.35 & 690.50 & 0.00
 & 0.10\\SCA3-5 & 662.75 & 0.70 & 
672.18 & 0.53 & \bf{659.90} & 
0.43 & 1.86\\SCA3-6 & 652.94 & 0.40 & 
652.94 & 0.45 & \bf{651.09} & 
0.28 & 0.28\\SCA3-7 & 671.77 & 0.50 & 
672.29 & 0.54 & \bf{659.17} & 
1.91 & 1.99\\SCA3-8 & \bf{719.47} & 0.58 & 
719.47 & 0.53 & 719.47 & 0.00
 & 0.00\\
SCA3-9 & 685.19 & 0.44 & 
685.71 & 0.47 & \bf{681.00} & 
0.62 & 0.69\\SCA8-0 & 977.93 & 0.56 & 
983.61 & 0.49 & \bf{961.50} & 
1.71 & 2.30\\SCA8-1 & 1103.14 & 0.50 & 
1103.14 & 0.54 & \bf{1049.65} & 
5.10 & 5.10\\SCA8-2 & 1050.37 & 0.32 & 
1050.37 & 0.42 & \bf{1039.64} & 
1.03 & 1.03\\SCA8-3 & 1014.90 & 0.47 & 
1014.90 & 0.61 & \bf{983.34} & 
3.21 & 3.21\\SCA8-4 & 1091.94 & 0.30 & 
1093.96 & 0.49 & \bf{1065.49} & 
2.48 & 2.67\\SCA8-5 & 1056.20 & 0.47 & 
1056.20 & 0.49 & \bf{1027.08} & 
2.84 & 2.84\\SCA8-6 & 993.84 & 0.71 & 
993.84 & 0.60 & \bf{971.82} & 
2.27 & 2.27\\SCA8-7 & 1071.93 & 0.32 & 
1072.12 & 0.36 & \bf{1051.28} & 
1.96 & 1.98\\SCA8-8 & 1091.18 & 0.38 & 
1091.18 & 0.52 & \bf{1071.18} & 
1.87 & 1.87\\SCA8-9 & 1077.83 & 0.47 & 
1080.38 & 0.48 & \bf{1060.50} & 
1.63 & 1.87\\CON3-0 & 625.94 & 0.38 & 
631.63 & 0.48 & \bf{616.52} & 
1.53 & 2.45\\CON3-1 & 560.75 & 0.61 & 
561.78 & 0.55 & \bf{554.47} & 
1.13 & 1.32\\CON3-2 & 521.38 & 0.56 & 
522.03 & 0.59 & \bf{518.00} & 
0.65 & 0.78\\CON3-3 & 599.66 & 0.64 & 
599.66 & 0.54 & \bf{591.19} & 
1.43 & 1.43\\CON3-4 & 592.58 & 0.48 & 
599.75 & 0.48 & \bf{588.79} & 
0.64 & 1.86\\CON3-5 & 567.94 & 0.57 & 
568.55 & 0.54 & \bf{563.70} & 
0.75 & 0.86\\CON3-6 & 502.26 & 0.56 & 
504.49 & 0.56 & \bf{499.05} & 
0.64 & 1.09\\CON3-7 & 588.32 & 0.55 & 
590.14 & 0.51 & \bf{576.48} & 
2.05 & 2.37\\CON3-8 & 525.47 & 0.67 & 
534.12 & 0.56 & \bf{523.05} & 
0.46 & 2.12\\CON3-9 & 582.79 & 0.62 & 
587.06 & 0.62 & \bf{578.24} & 
0.79 & 1.52\\CON8-0 & 877.74 & 0.58 & 
878.88 & 0.56 & \bf{857.17} & 
2.40 & 2.53\\CON8-1 & 768.27 & 0.36 & 
771.89 & 0.36 & \bf{740.85} & 
3.70 & 4.19\\CON8-2 & 740.76 & 0.43 & 
741.60 & 0.43 & \bf{712.89} & 
3.91 & 4.03\\CON8-3 & 831.15 & 0.37 & 
838.24 & 0.42 & \bf{811.07} & 
2.48 & 3.35\\CON8-4 & 772.76 & 0.74 & 
777.52 & 0.61 & \bf{772.25} & 
0.07 & 0.68\\CON8-5 & 764.40 & 0.51 & 
764.40 & 0.51 & \bf{754.88} & 
1.26 & 1.26\\CON8-6 & 701.67 & 0.34 & 
701.67 & 0.41 & \bf{678.92} & 
3.35 & 3.35\\CON8-7 & 820.67 & 0.41 & 
820.67 & 0.43 & \bf{811.96} & 
1.07 & 1.07\\CON8-8 & 790.83 & 0.52 & 
799.62 & 0.54 & \bf{767.53} & 
3.04 & 4.18\\CON8-9 & 823.13 & 0.36 & 
823.70 & 0.45 & \bf{809.00} & 
1.75 & 1.82\\\bf{PROM.} & 
\bf{771.36} & \bf{0.49} & \bf{773.55} & \bf{0.49} & \bf{758.54} & \bf{1.55} & \bf{1.88}\\[1ex]\hline
\end{tabular}
\label{table:nonlin}
\end{table} \clearpage
\begin{table}[ht]
\caption{Resultados de la ejecución de la metaheurística IGA, utilizando instancias de SalhiNagy con la configuración -n 200.0 -p 30.0 -cprob 90 -mprob 70}
\centering
\small
\begin{tabular}{c c c c c c c c}
\hline\hline
Instancia & Costo mínimo & Tiempo(seg.) & Costo promedio & Tiempo promedio(seg.) & CME & \%G & \%GP \\ [0.5ex]
\hline
CMT1X & 486.44 & 0.47 & 
487.13 & 0.49 & \bf{470.48} & 
3.39 & 3.54\\CMT1Y & 480.79 & 0.54 & 
487.19 & 0.56 & \bf{470.48} & 
2.19 & 3.55\\CMT2X & 720.58 & 0.96 & 
721.17 & 1.18 & \bf{682.39} & 
5.60 & 5.68\\CMT2Y & 709.69 & 1.26 & 
713.37 & 1.17 & \bf{682.39} & 
4.00 & 4.54\\CMT3X & 734.32 & 2.59 & 
744.13 & 2.46 & \bf{719.06} & 
2.12 & 3.49\\CMT3Y & 749.10 & 1.98 & 
750.29 & 2.36 & \bf{719.06} & 
4.18 & 4.34\\CMT4X & 896.01 & 6.79 & 
909.28 & 6.57 & \bf{854.21} & 
4.89 & 6.45\\CMT4Y & 902.20 & 5.90 & 
918.43 & 6.16 & \bf{852.46} & 
5.83 & 7.74\\CMT5X & 1106.76 & 13.62 & 
1111.90 & 13.14 & \bf{1030.56} & 
7.39 & 7.89\\CMT5Y & 1117.16 & 12.92 & 
1128.58 & 13.32 & \bf{1031.69} & 
8.28 & 9.39\\CMT11X & 894.44 & 4.42 & 
906.36 & 4.38 & \bf{831.09} & 
7.62 & 9.06\\CMT11Y & 885.26 & 4.71 & 
889.03 & 4.34 & \bf{829.85} & 
6.68 & 7.13\\CMT12X & 683.98 & 2.70 & 
688.07 & 2.58 & \bf{658.83} & 
3.82 & 4.44\\CMT12Y & 675.06 & 2.29 & 
680.62 & 2.40 & \bf{660.47} & 
2.21 & 3.05\\\bf{PROM.} & 
\bf{788.70} & \bf{4.37} & \bf{795.40} & \bf{4.37} & \bf{749.50} & \bf{4.87} & \bf{5.73}\\[1ex]\hline
\end{tabular}
\label{table:nonlin}
\end{table} \clearpage
\begin{table}[ht]
\caption{Resultados de la ejecución de la metaheurística IGA, utilizando instancias de Dethloff con la configuración -n 200.0 -p 40.0 -cprob 40 -mprob 70}
\centering
\small
\begin{tabular}{c c c c c c c c}
\hline\hline
Instancia & Costo mínimo & Tiempo(seg.) & Costo promedio & Tiempo promedio(seg.) & CME & \%G & \%GP \\ [0.5ex]
\hline
SCA3-0 & 640.55 & 0.54 & 
640.55 & 0.61 & \bf{635.62} & 
0.78 & 0.78\\SCA3-1 & \bf{697.84} & 0.83 & 
702.10 & 0.62 & 697.84 & 0.00
 & 0.61\\SCA3-2 & 664.18 & 0.40 & 
666.55 & 0.48 & \bf{659.34} & 
0.73 & 1.09\\SCA3-3 & 681.16 & 0.73 & 
682.12 & 0.77 & \bf{680.04} & 
0.16 & 0.31\\SCA3-4 & \bf{690.50} & 1.43 & 
690.50 & 0.83 & 690.50 & 0.00
 & 0.00\\
SCA3-5 & 682.53 & 0.52 & 
682.53 & 0.67 & \bf{659.90} & 
3.43 & 3.43\\SCA3-6 & 657.24 & 0.48 & 
658.90 & 0.48 & \bf{651.09} & 
0.94 & 1.20\\SCA3-7 & 666.15 & 0.49 & 
666.15 & 0.54 & \bf{659.17} & 
1.06 & 1.06\\SCA3-8 & 727.35 & 0.92 & 
730.06 & 0.72 & \bf{719.47} & 
1.10 & 1.47\\SCA3-9 & 684.72 & 0.48 & 
684.72 & 0.58 & \bf{681.00} & 
0.55 & 0.55\\SCA8-0 & 1002.59 & 0.50 & 
1002.94 & 0.49 & \bf{961.50} & 
4.27 & 4.31\\SCA8-1 & 1080.67 & 1.00 & 
1083.93 & 0.75 & \bf{1049.65} & 
2.96 & 3.27\\SCA8-2 & 1052.74 & 0.54 & 
1052.74 & 0.67 & \bf{1039.64} & 
1.26 & 1.26\\SCA8-3 & 1015.21 & 0.69 & 
1020.08 & 0.79 & \bf{983.34} & 
3.24 & 3.74\\SCA8-4 & 1105.93 & 0.55 & 
1110.20 & 0.67 & \bf{1065.49} & 
3.80 & 4.20\\SCA8-5 & 1051.83 & 0.47 & 
1063.01 & 0.64 & \bf{1027.08} & 
2.41 & 3.50\\SCA8-6 & 991.79 & 0.51 & 
996.67 & 0.68 & \bf{971.82} & 
2.05 & 2.56\\SCA8-7 & 1077.29 & 0.47 & 
1079.22 & 0.66 & \bf{1051.28} & 
2.47 & 2.66\\SCA8-8 & 1089.91 & 0.48 & 
1089.91 & 0.48 & \bf{1071.18} & 
1.75 & 1.75\\SCA8-9 & 1077.43 & 0.42 & 
1087.05 & 0.62 & \bf{1060.50} & 
1.60 & 2.50\\CON3-0 & 624.84 & 0.60 & 
624.87 & 0.77 & \bf{616.52} & 
1.35 & 1.35\\CON3-1 & 556.92 & 0.67 & 
558.17 & 0.63 & \bf{554.47} & 
0.44 & 0.67\\CON3-2 & 524.89 & 0.85 & 
524.89 & 0.68 & \bf{518.00} & 
1.33 & 1.33\\CON3-3 & 591.48 & 0.46 & 
592.89 & 0.57 & \bf{591.19} & 
0.05 & 0.29\\CON3-4 & 593.78 & 0.50 & 
596.85 & 0.61 & \bf{588.79} & 
0.85 & 1.37\\CON3-5 & 568.69 & 0.54 & 
573.36 & 0.61 & \bf{563.70} & 
0.89 & 1.71\\CON3-6 & 505.31 & 0.72 & 
507.66 & 0.71 & \bf{499.05} & 
1.25 & 1.72\\CON3-7 & 578.22 & 0.54 & 
578.22 & 0.67 & \bf{576.48} & 
0.30 & 0.30\\CON3-8 & \bf{523.05} & 0.53 & 
525.07 & 0.57 & 523.05 & 0.00
 & 0.39\\CON3-9 & 582.79 & 0.52 & 
587.35 & 0.68 & \bf{578.24} & 
0.79 & 1.58\\CON8-0 & 861.11 & 0.60 & 
861.11 & 0.54 & \bf{857.17} & 
0.46 & 0.46\\CON8-1 & 767.90 & 1.08 & 
768.18 & 0.80 & \bf{740.85} & 
3.65 & 3.69\\CON8-2 & 716.07 & 0.68 & 
728.27 & 0.79 & \bf{712.89} & 
0.45 & 2.16\\CON8-3 & 831.87 & 0.62 & 
831.87 & 0.79 & \bf{811.07} & 
2.56 & 2.56\\CON8-4 & 793.91 & 0.54 & 
802.61 & 0.69 & \bf{772.25} & 
2.80 & 3.93\\CON8-5 & 774.31 & 0.54 & 
775.36 & 0.51 & \bf{754.88} & 
2.57 & 2.71\\CON8-6 & 700.32 & 0.48 & 
700.32 & 0.59 & \bf{678.92} & 
3.15 & 3.15\\CON8-7 & 832.32 & 0.67 & 
832.32 & 0.69 & \bf{811.96} & 
2.51 & 2.51\\CON8-8 & 791.79 & 0.52 & 
794.50 & 0.67 & \bf{767.53} & 
3.16 & 3.51\\CON8-9 & 819.67 & 0.70 & 
838.01 & 0.61 & \bf{809.00} & 
1.32 & 3.59\\\bf{PROM.} & 
\bf{771.92} & \bf{0.62} & \bf{774.80} & \bf{0.65} & \bf{758.54} & \bf{1.61} & \bf{1.98}\\[1ex]\hline
\end{tabular}
\label{table:nonlin}
\end{table} \clearpage
\begin{table}[ht]
\caption{Resultados de la ejecución de la metaheurística IGA, utilizando instancias de SalhiNagy con la configuración -n 200.0 -p 40.0 -cprob 90 -mprob 70}
\centering
\small
\begin{tabular}{c c c c c c c c}
\hline\hline
Instancia & Costo mínimo & Tiempo(seg.) & Costo promedio & Tiempo promedio(seg.) & CME & \%G & \%GP \\ [0.5ex]
\hline
CMT1X & 478.71 & 0.60 & 
478.71 & 0.67 & \bf{470.48} & 
1.75 & 1.75\\CMT1Y & 478.97 & 0.81 & 
481.12 & 0.71 & \bf{470.48} & 
1.80 & 2.26\\CMT2X & 695.69 & 1.52 & 
704.94 & 1.54 & \bf{682.39} & 
1.95 & 3.30\\CMT2Y & 710.00 & 1.37 & 
715.07 & 1.46 & \bf{682.39} & 
4.05 & 4.79\\CMT3X & 741.54 & 3.50 & 
745.46 & 3.40 & \bf{719.06} & 
3.13 & 3.67\\CMT3Y & 745.94 & 2.94 & 
746.84 & 3.10 & \bf{719.06} & 
3.74 & 3.86\\CMT4X & 896.44 & 8.59 & 
911.96 & 8.63 & \bf{854.21} & 
4.94 & 6.76\\CMT4Y & 896.76 & 8.23 & 
914.07 & 8.57 & \bf{852.46} & 
5.20 & 7.23\\CMT5X & 1098.29 & 16.24 & 
1117.53 & 16.84 & \bf{1030.56} & 
6.57 & 8.44\\CMT5Y & 1105.25 & 17.76 & 
1123.69 & 17.53 & \bf{1031.69} & 
7.13 & 8.92\\CMT11X & 897.50 & 5.57 & 
915.12 & 5.31 & \bf{831.09} & 
7.99 & 10.11\\CMT11Y & 864.37 & 5.89 & 
903.39 & 6.02 & \bf{829.85} & 
4.16 & 8.86\\CMT12X & 671.22 & 3.15 & 
679.12 & 3.27 & \bf{658.83} & 
1.88 & 3.08\\CMT12Y & 675.25 & 3.12 & 
679.62 & 2.97 & \bf{660.47} & 
2.24 & 2.90\\\bf{PROM.} & 
\bf{782.57} & \bf{5.66} & \bf{794.05} & \bf{5.72} & \bf{749.50} & \bf{4.04} & \bf{5.42}\\[1ex]\hline
\end{tabular}
\label{table:nonlin}
\end{table} \clearpage
\begin{table}[ht]
\caption{Resultados de la ejecución de la metaheurística IGA, utilizando instancias de Dethloff con la configuración -n 200.0 -p 50.0 -cprob 40 -mprob 70}
\centering
\small
\begin{tabular}{c c c c c c c c}
\hline\hline
Instancia & Costo mínimo & Tiempo(seg.) & Costo promedio & Tiempo promedio(seg.) & CME & \%G & \%GP \\ [0.5ex]
\hline
SCA3-0 & 640.55 & 0.72 & 
641.12 & 0.79 & \bf{635.62} & 
0.78 & 0.87\\SCA3-1 & \bf{697.84} & 0.74 & 
705.91 & 0.89 & 697.84 & 0.00
 & 1.16\\SCA3-2 & 661.13 & 0.63 & 
661.13 & 0.82 & \bf{659.34} & 
0.27 & 0.27\\SCA3-3 & \bf{680.04} & 0.72 & 
680.74 & 0.84 & 680.04 & 0.00
 & 0.10\\SCA3-4 & \bf{690.50} & 0.83 & 
690.50 & 0.78 & 690.50 & 0.00
 & 0.00\\
SCA3-5 & 665.04 & 1.20 & 
665.19 & 0.98 & \bf{659.90} & 
0.78 & 0.80\\SCA3-6 & 652.94 & 1.13 & 
654.53 & 0.91 & \bf{651.09} & 
0.28 & 0.53\\SCA3-7 & 666.15 & 0.87 & 
666.26 & 0.93 & \bf{659.17} & 
1.06 & 1.08\\SCA3-8 & 721.45 & 0.82 & 
724.01 & 0.81 & \bf{719.47} & 
0.28 & 0.63\\SCA3-9 & \bf{681.00} & 0.70 & 
682.78 & 0.67 & 681.00 & 0.00
 & 0.26\\SCA8-0 & 996.13 & 1.04 & 
996.70 & 0.82 & \bf{961.50} & 
3.60 & 3.66\\SCA8-1 & 1074.89 & 0.62 & 
1078.43 & 0.77 & \bf{1049.65} & 
2.40 & 2.74\\SCA8-2 & 1054.47 & 0.65 & 
1054.47 & 0.78 & \bf{1039.64} & 
1.43 & 1.43\\SCA8-3 & 1015.48 & 0.97 & 
1023.47 & 0.91 & \bf{983.34} & 
3.27 & 4.08\\SCA8-4 & 1069.33 & 0.63 & 
1071.06 & 0.71 & \bf{1065.49} & 
0.36 & 0.52\\SCA8-5 & 1050.41 & 0.70 & 
1050.41 & 0.68 & \bf{1027.08} & 
2.27 & 2.27\\SCA8-6 & 994.55 & 0.72 & 
996.03 & 0.87 & \bf{971.82} & 
2.34 & 2.49\\SCA8-7 & 1066.82 & 1.18 & 
1082.94 & 0.82 & \bf{1051.28} & 
1.48 & 3.01\\SCA8-8 & \bf{1071.18} & 0.88 & 
1071.18 & 0.81 & 1071.18 & 0.00
 & 0.00\\
SCA8-9 & 1064.16 & 0.93 & 
1080.81 & 0.72 & \bf{1060.50} & 
0.35 & 1.92\\CON3-0 & 619.09 & 0.79 & 
619.39 & 0.91 & \bf{616.52} & 
0.42 & 0.47\\CON3-1 & 560.75 & 0.72 & 
560.75 & 0.79 & \bf{554.47} & 
1.13 & 1.13\\CON3-2 & 521.38 & 0.90 & 
523.48 & 0.77 & \bf{518.00} & 
0.65 & 1.06\\CON3-3 & 591.20 & 0.96 & 
598.14 & 0.74 & \bf{591.19} & 
0.00 & 1.18\\CON3-4 & 592.58 & 0.83 & 
593.88 & 0.75 & \bf{588.79} & 
0.64 & 0.87\\CON3-5 & \bf{563.70} & 0.77 & 
564.76 & 0.78 & 563.70 & 0.00
 & 0.19\\CON3-6 & 501.05 & 1.09 & 
503.38 & 1.01 & \bf{499.05} & 
0.40 & 0.87\\CON3-7 & 577.91 & 0.61 & 
578.89 & 0.69 & \bf{576.48} & 
0.25 & 0.42\\CON3-8 & 524.59 & 1.22 & 
524.59 & 0.84 & \bf{523.05} & 
0.29 & 0.29\\CON3-9 & 590.39 & 1.11 & 
590.90 & 0.89 & \bf{578.24} & 
2.10 & 2.19\\CON8-0 & 879.55 & 0.85 & 
879.55 & 0.79 & \bf{857.17} & 
2.61 & 2.61\\CON8-1 & 748.49 & 0.64 & 
752.13 & 0.90 & \bf{740.85} & 
1.03 & 1.52\\CON8-2 & 716.07 & 0.72 & 
716.52 & 0.89 & \bf{712.89} & 
0.45 & 0.51\\CON8-3 & 812.22 & 0.78 & 
818.54 & 0.81 & \bf{811.07} & 
0.14 & 0.92\\CON8-4 & 780.03 & 0.61 & 
784.04 & 0.79 & \bf{772.25} & 
1.01 & 1.53\\CON8-5 & 773.35 & 0.78 & 
776.87 & 0.72 & \bf{754.88} & 
2.45 & 2.91\\CON8-6 & 687.57 & 0.65 & 
687.57 & 0.78 & \bf{678.92} & 
1.27 & 1.27\\CON8-7 & 816.18 & 1.04 & 
816.18 & 0.76 & \bf{811.96} & 
0.52 & 0.52\\CON8-8 & 793.86 & 0.94 & 
793.86 & 0.84 & \bf{767.53} & 
3.43 & 3.43\\CON8-9 & 842.31 & 1.27 & 
842.31 & 1.20 & \bf{809.00} & 
4.12 & 4.12\\\bf{PROM.} & 
\bf{767.66} & \bf{0.85} & \bf{770.09} & \bf{0.82} & \bf{758.54} & \bf{1.10} & \bf{1.40}\\[1ex]\hline
\end{tabular}
\label{table:nonlin}
\end{table} \clearpage
\begin{table}[ht]
\caption{Resultados de la ejecución de la metaheurística IGA, utilizando instancias de SalhiNagy con la configuración -n 200.0 -p 50.0 -cprob 90 -mprob 70}
\centering
\small
\begin{tabular}{c c c c c c c c}
\hline\hline
Instancia & Costo mínimo & Tiempo(seg.) & Costo promedio & Tiempo promedio(seg.) & CME & \%G & \%GP \\ [0.5ex]
\hline
CMT1X & 478.97 & 1.06 & 
481.16 & 1.02 & \bf{470.48} & 
1.80 & 2.27\\CMT1Y & 482.08 & 0.78 & 
482.62 & 0.82 & \bf{470.48} & 
2.47 & 2.58\\CMT2X & 703.62 & 1.78 & 
708.15 & 1.97 & \bf{682.39} & 
3.11 & 3.77\\CMT2Y & 710.32 & 2.14 & 
716.19 & 1.98 & \bf{682.39} & 
4.09 & 4.95\\CMT3X & 748.00 & 3.75 & 
748.40 & 3.96 & \bf{719.06} & 
4.02 & 4.08\\CMT3Y & 745.96 & 4.12 & 
750.20 & 3.94 & \bf{719.06} & 
3.74 & 4.33\\CMT4X & 892.90 & 10.89 & 
904.78 & 10.16 & \bf{854.21} & 
4.53 & 5.92\\CMT4Y & 908.97 & 10.19 & 
920.26 & 10.20 & \bf{852.46} & 
6.63 & 7.95\\CMT5X & 1110.56 & 20.66 & 
1118.40 & 20.50 & \bf{1030.56} & 
7.76 & 8.52\\CMT5Y & 1110.19 & 22.24 & 
1119.87 & 22.08 & \bf{1031.69} & 
7.61 & 8.55\\CMT11X & 879.25 & 6.84 & 
904.74 & 6.54 & \bf{831.09} & 
5.79 & 8.86\\CMT11Y & 896.75 & 7.61 & 
902.65 & 7.59 & \bf{829.85} & 
8.06 & 8.77\\CMT12X & 676.16 & 4.12 & 
679.70 & 4.27 & \bf{658.83} & 
2.63 & 3.17\\CMT12Y & 673.67 & 3.64 & 
674.34 & 4.14 & \bf{660.47} & 
2.00 & 2.10\\\bf{PROM.} & 
\bf{786.96} & \bf{7.13} & \bf{793.68} & \bf{7.08} & \bf{749.50} & \bf{4.59} & \bf{5.42}\\[1ex]\hline
\end{tabular}
\label{table:nonlin}
\end{table} \clearpage
\begin{table}[ht]
\caption{Resultados de la ejecución de la metaheurística IGA, utilizando instancias de Dethloff con la configuración -n 200.0 -p 60.0 -cprob 40 -mprob 70}
\centering
\small
\begin{tabular}{c c c c c c c c}
\hline\hline
Instancia & Costo mínimo & Tiempo(seg.) & Costo promedio & Tiempo promedio(seg.) & CME & \%G & \%GP \\ [0.5ex]
\hline
SCA3-0 & 636.06 & 1.03 & 
640.30 & 0.98 & \bf{635.62} & 
0.07 & 0.74\\SCA3-1 & 700.50 & 0.84 & 
701.01 & 0.86 & \bf{697.84} & 
0.38 & 0.45\\SCA3-2 & 664.18 & 1.26 & 
664.18 & 0.83 & \bf{659.34} & 
0.73 & 0.73\\SCA3-3 & 681.35 & 0.81 & 
681.35 & 0.81 & \bf{680.04} & 
0.19 & 0.19\\SCA3-4 & \bf{690.50} & 1.03 & 
690.50 & 0.86 & 690.50 & 0.00
 & 0.00\\
SCA3-5 & 665.64 & 1.04 & 
675.80 & 1.09 & \bf{659.90} & 
0.87 & 2.41\\SCA3-6 & 652.94 & 0.96 & 
652.94 & 1.00 & \bf{651.09} & 
0.28 & 0.28\\SCA3-7 & 666.15 & 0.78 & 
667.09 & 0.84 & \bf{659.17} & 
1.06 & 1.20\\SCA3-8 & \bf{719.47} & 1.00 & 
719.70 & 0.96 & 719.47 & 0.00
 & 0.03\\SCA3-9 & \bf{681.00} & 0.94 & 
681.00 & 0.90 & 681.00 & 0.00
 & 0.00\\
SCA8-0 & 994.18 & 0.92 & 
994.18 & 0.93 & \bf{961.50} & 
3.40 & 3.40\\SCA8-1 & 1066.48 & 1.11 & 
1070.43 & 1.28 & \bf{1049.65} & 
1.60 & 1.98\\SCA8-2 & 1053.78 & 0.87 & 
1053.78 & 0.86 & \bf{1039.64} & 
1.36 & 1.36\\SCA8-3 & 1012.34 & 0.70 & 
1012.34 & 1.02 & \bf{983.34} & 
2.95 & 2.95\\SCA8-4 & 1072.75 & 0.81 & 
1072.75 & 1.01 & \bf{1065.49} & 
0.68 & 0.68\\SCA8-5 & 1038.93 & 1.51 & 
1038.93 & 1.14 & \bf{1027.08} & 
1.15 & 1.15\\SCA8-6 & 991.23 & 1.17 & 
991.23 & 1.17 & \bf{971.82} & 
2.00 & 2.00\\SCA8-7 & 1071.89 & 1.14 & 
1071.89 & 0.93 & \bf{1051.28} & 
1.96 & 1.96\\SCA8-8 & 1085.22 & 0.83 & 
1088.51 & 0.78 & \bf{1071.18} & 
1.31 & 1.62\\SCA8-9 & 1070.34 & 1.05 & 
1070.83 & 0.93 & \bf{1060.50} & 
0.93 & 0.97\\CON3-0 & 625.94 & 0.88 & 
631.77 & 0.92 & \bf{616.52} & 
1.53 & 2.47\\CON3-1 & 560.75 & 0.90 & 
560.75 & 1.01 & \bf{554.47} & 
1.13 & 1.13\\CON3-2 & 521.38 & 0.90 & 
521.38 & 1.20 & \bf{518.00} & 
0.65 & 0.65\\CON3-3 & \bf{591.19} & 1.02 & 
591.50 & 0.97 & 591.19 & 0.00
 & 0.05\\CON3-4 & 591.43 & 0.97 & 
592.00 & 0.87 & \bf{588.79} & 
0.45 & 0.55\\CON3-5 & 564.88 & 1.08 & 
566.41 & 1.03 & \bf{563.70} & 
0.21 & 0.48\\CON3-6 & 503.97 & 0.90 & 
506.29 & 0.94 & \bf{499.05} & 
0.99 & 1.45\\CON3-7 & 582.14 & 0.98 & 
584.08 & 0.94 & \bf{576.48} & 
0.98 & 1.32\\CON3-8 & 524.59 & 0.96 & 
524.59 & 0.93 & \bf{523.05} & 
0.29 & 0.29\\CON3-9 & 580.05 & 0.82 & 
586.10 & 0.95 & \bf{578.24} & 
0.31 & 1.36\\CON8-0 & 866.43 & 0.69 & 
868.60 & 0.83 & \bf{857.17} & 
1.08 & 1.33\\CON8-1 & 752.74 & 1.06 & 
757.10 & 0.91 & \bf{740.85} & 
1.60 & 2.19\\CON8-2 & 718.75 & 1.24 & 
718.75 & 1.13 & \bf{712.89} & 
0.82 & 0.82\\CON8-3 & 818.41 & 0.89 & 
820.13 & 1.09 & \bf{811.07} & 
0.90 & 1.12\\CON8-4 & 798.37 & 0.81 & 
798.37 & 0.73 & \bf{772.25} & 
3.38 & 3.38\\CON8-5 & 762.36 & 0.88 & 
762.94 & 0.85 & \bf{754.88} & 
0.99 & 1.07\\CON8-6 & 698.50 & 0.81 & 
698.50 & 0.89 & \bf{678.92} & 
2.88 & 2.88\\CON8-7 & 816.07 & 1.00 & 
823.72 & 1.03 & \bf{811.96} & 
0.51 & 1.45\\CON8-8 & 771.26 & 0.98 & 
771.26 & 0.90 & \bf{767.53} & 
0.49 & 0.49\\CON8-9 & 831.21 & 1.13 & 
831.21 & 0.95 & \bf{809.00} & 
2.75 & 2.75\\\bf{PROM.} & 
\bf{767.38} & \bf{0.97} & \bf{768.85} & \bf{0.96} & \bf{758.54} & \bf{1.07} & \bf{1.28}\\[1ex]\hline
\end{tabular}
\label{table:nonlin}
\end{table} \clearpage
\begin{table}[ht]
\caption{Resultados de la ejecución de la metaheurística IGA, utilizando instancias de SalhiNagy con la configuración -n 200.0 -p 60.0 -cprob 90 -mprob 70}
\centering
\small
\begin{tabular}{c c c c c c c c}
\hline\hline
Instancia & Costo mínimo & Tiempo(seg.) & Costo promedio & Tiempo promedio(seg.) & CME & \%G & \%GP \\ [0.5ex]
\hline
CMT1X & 478.82 & 1.18 & 
478.82 & 1.19 & \bf{470.48} & 
1.77 & 1.77\\CMT1Y & 481.77 & 1.20 & 
482.18 & 1.09 & \bf{470.48} & 
2.40 & 2.49\\CMT2X & 697.62 & 2.14 & 
700.78 & 2.43 & \bf{682.39} & 
2.23 & 2.70\\CMT2Y & 705.36 & 2.48 & 
709.97 & 2.17 & \bf{682.39} & 
3.37 & 4.04\\CMT3X & 742.11 & 4.84 & 
745.48 & 4.89 & \bf{719.06} & 
3.21 & 3.67\\CMT3Y & 733.02 & 4.38 & 
741.87 & 4.82 & \bf{719.06} & 
1.94 & 3.17\\CMT4X & 896.02 & 14.33 & 
908.02 & 12.81 & \bf{854.21} & 
4.89 & 6.30\\CMT4Y & 918.65 & 12.08 & 
922.47 & 12.61 & \bf{852.46} & 
7.76 & 8.21\\CMT5X & 1111.97 & 25.52 & 
1116.30 & 25.21 & \bf{1030.56} & 
7.90 & 8.32\\CMT5Y & 1131.51 & 25.81 & 
1134.73 & 24.95 & \bf{1031.69} & 
9.68 & 9.99\\CMT11X & 900.79 & 7.17 & 
912.90 & 7.87 & \bf{831.09} & 
8.39 & 9.84\\CMT11Y & 856.38 & 8.71 & 
880.58 & 8.86 & \bf{829.85} & 
3.20 & 6.11\\CMT12X & 674.30 & 5.41 & 
676.98 & 5.08 & \bf{658.83} & 
2.35 & 2.75\\CMT12Y & 675.56 & 4.85 & 
680.98 & 4.80 & \bf{660.47} & 
2.28 & 3.11\\\bf{PROM.} & 
\bf{785.99} & \bf{8.58} & \bf{792.29} & \bf{8.48} & \bf{749.50} & \bf{4.38} & \bf{5.18}\\[1ex]\hline
\end{tabular}
\label{table:nonlin}
\end{table} \clearpage
\begin{table}[ht]
\caption{Resultados de la ejecución de la metaheurística IGA, utilizando instancias de Dethloff con la configuración -n 200.0 -p 70.0 -cprob 40 -mprob 70}
\centering
\small
\begin{tabular}{c c c c c c c c}
\hline\hline
Instancia & Costo mínimo & Tiempo(seg.) & Costo promedio & Tiempo promedio(seg.) & CME & \%G & \%GP \\ [0.5ex]
\hline
SCA3-0 & 640.55 & 1.22 & 
640.55 & 1.39 & \bf{635.62} & 
0.78 & 0.78\\SCA3-1 & 701.86 & 1.46 & 
701.86 & 1.28 & \bf{697.84} & 
0.58 & 0.58\\SCA3-2 & 661.13 & 1.50 & 
664.39 & 1.09 & \bf{659.34} & 
0.27 & 0.77\\SCA3-3 & \bf{680.04} & 1.02 & 
682.35 & 1.01 & 680.04 & 0.00
 & 0.34\\SCA3-4 & \bf{690.50} & 1.01 & 
690.50 & 1.12 & 690.50 & 0.00
 & 0.00\\
SCA3-5 & 662.75 & 0.98 & 
668.26 & 0.98 & \bf{659.90} & 
0.43 & 1.27\\SCA3-6 & 652.94 & 1.06 & 
654.07 & 1.21 & \bf{651.09} & 
0.28 & 0.46\\SCA3-7 & 666.15 & 1.12 & 
666.15 & 1.12 & \bf{659.17} & 
1.06 & 1.06\\SCA3-8 & 723.99 & 0.80 & 
724.24 & 0.89 & \bf{719.47} & 
0.63 & 0.66\\SCA3-9 & \bf{681.00} & 1.34 & 
681.34 & 1.21 & 681.00 & 0.00
 & 0.05\\SCA8-0 & 1007.65 & 0.72 & 
1015.05 & 1.02 & \bf{961.50} & 
4.80 & 5.57\\SCA8-1 & 1068.50 & 0.81 & 
1070.28 & 0.95 & \bf{1049.65} & 
1.80 & 1.97\\SCA8-2 & 1050.37 & 1.27 & 
1050.58 & 1.14 & \bf{1039.64} & 
1.03 & 1.05\\SCA8-3 & 1024.42 & 1.36 & 
1025.52 & 1.07 & \bf{983.34} & 
4.18 & 4.29\\SCA8-4 & 1081.05 & 0.80 & 
1082.55 & 0.97 & \bf{1065.49} & 
1.46 & 1.60\\SCA8-5 & 1052.08 & 1.00 & 
1052.21 & 1.06 & \bf{1027.08} & 
2.43 & 2.45\\SCA8-6 & 982.57 & 1.15 & 
985.12 & 1.06 & \bf{971.82} & 
1.11 & 1.37\\SCA8-7 & 1079.96 & 1.07 & 
1079.96 & 1.00 & \bf{1051.28} & 
2.73 & 2.73\\SCA8-8 & \bf{1071.18} & 1.06 & 
1071.18 & 1.24 & 1071.18 & 0.00
 & 0.00\\
SCA8-9 & 1067.42 & 1.08 & 
1067.42 & 1.02 & \bf{1060.50} & 
0.65 & 0.65\\CON3-0 & 625.62 & 1.41 & 
633.12 & 1.22 & \bf{616.52} & 
1.48 & 2.69\\CON3-1 & 556.92 & 0.91 & 
557.88 & 1.15 & \bf{554.47} & 
0.44 & 0.61\\CON3-2 & 521.38 & 0.93 & 
521.38 & 1.10 & \bf{518.00} & 
0.65 & 0.65\\CON3-3 & 592.41 & 1.26 & 
599.73 & 1.22 & \bf{591.19} & 
0.21 & 1.44\\CON3-4 & 596.86 & 1.24 & 
599.87 & 1.14 & \bf{588.79} & 
1.37 & 1.88\\CON3-5 & \bf{563.70} & 1.12 & 
564.76 & 1.10 & 563.70 & 0.00
 & 0.19\\CON3-6 & 505.01 & 0.86 & 
505.25 & 1.04 & \bf{499.05} & 
1.19 & 1.24\\CON3-7 & 586.01 & 1.45 & 
587.20 & 1.08 & \bf{576.48} & 
1.65 & 1.86\\CON3-8 & 523.14 & 0.96 & 
527.97 & 1.29 & \bf{523.05} & 
0.02 & 0.94\\CON3-9 & 582.79 & 0.89 & 
587.33 & 1.13 & \bf{578.24} & 
0.79 & 1.57\\CON8-0 & 866.66 & 0.87 & 
873.15 & 1.00 & \bf{857.17} & 
1.11 & 1.86\\CON8-1 & 772.41 & 1.50 & 
773.67 & 1.20 & \bf{740.85} & 
4.26 & 4.43\\CON8-2 & 726.32 & 1.33 & 
728.14 & 1.27 & \bf{712.89} & 
1.88 & 2.14\\CON8-3 & 832.93 & 0.84 & 
837.52 & 1.06 & \bf{811.07} & 
2.70 & 3.26\\CON8-4 & 785.18 & 0.83 & 
785.18 & 0.97 & \bf{772.25} & 
1.67 & 1.67\\CON8-5 & 762.61 & 1.03 & 
763.00 & 1.03 & \bf{754.88} & 
1.02 & 1.08\\CON8-6 & 684.69 & 0.80 & 
684.69 & 1.22 & \bf{678.92} & 
0.85 & 0.85\\CON8-7 & 815.43 & 1.50 & 
819.10 & 1.25 & \bf{811.96} & 
0.43 & 0.88\\CON8-8 & 786.80 & 1.11 & 
788.93 & 1.38 & \bf{767.53} & 
2.51 & 2.79\\CON8-9 & 829.51 & 1.31 & 
832.71 & 1.13 & \bf{809.00} & 
2.54 & 2.93\\\bf{PROM.} & 
\bf{769.06} & \bf{1.10} & \bf{771.10} & \bf{1.12} & \bf{758.54} & \bf{1.27} & \bf{1.57}\\[1ex]\hline
\end{tabular}
\label{table:nonlin}
\end{table} \clearpage
\begin{table}[ht]
\caption{Resultados de la ejecución de la metaheurística IGA, utilizando instancias de SalhiNagy con la configuración -n 200.0 -p 70.0 -cprob 90 -mprob 70}
\centering
\small
\begin{tabular}{c c c c c c c c}
\hline\hline
Instancia & Costo mínimo & Tiempo(seg.) & Costo promedio & Tiempo promedio(seg.) & CME & \%G & \%GP \\ [0.5ex]
\hline
CMT1X & 482.49 & 1.31 & 
484.51 & 1.10 & \bf{470.48} & 
2.55 & 2.98\\CMT1Y & 478.23 & 1.25 & 
482.69 & 1.32 & \bf{470.48} & 
1.65 & 2.60\\CMT2X & 711.90 & 2.88 & 
714.96 & 2.75 & \bf{682.39} & 
4.32 & 4.77\\CMT2Y & 707.84 & 2.44 & 
710.66 & 2.58 & \bf{682.39} & 
3.73 & 4.14\\CMT3X & 731.11 & 6.30 & 
738.58 & 5.70 & \bf{719.06} & 
1.68 & 2.71\\CMT3Y & 738.22 & 5.90 & 
746.05 & 5.46 & \bf{719.06} & 
2.66 & 3.75\\CMT4X & 914.95 & 13.80 & 
919.00 & 14.34 & \bf{854.21} & 
7.11 & 7.58\\CMT4Y & 882.45 & 14.66 & 
909.04 & 14.82 & \bf{852.46} & 
3.52 & 6.64\\CMT5X & 1107.39 & 29.28 & 
1113.74 & 28.90 & \bf{1030.56} & 
7.46 & 8.07\\CMT5Y & 1116.68 & 29.30 & 
1124.17 & 29.70 & \bf{1031.69} & 
8.24 & 8.96\\CMT11X & 897.29 & 8.82 & 
907.72 & 9.00 & \bf{831.09} & 
7.97 & 9.22\\CMT11Y & 904.43 & 10.67 & 
912.13 & 10.66 & \bf{829.85} & 
8.99 & 9.92\\CMT12X & 674.11 & 6.04 & 
674.96 & 5.65 & \bf{658.83} & 
2.32 & 2.45\\CMT12Y & 673.71 & 5.62 & 
676.61 & 5.41 & \bf{660.47} & 
2.00 & 2.44\\\bf{PROM.} & 
\bf{787.20} & \bf{9.88} & \bf{793.92} & \bf{9.81} & \bf{749.50} & \bf{4.59} & \bf{5.45}\\[1ex]\hline
\end{tabular}
\label{table:nonlin}
\end{table} \clearpage
\begin{table}[ht]
\caption{Resultados de la ejecución de la metaheurística IGA, utilizando instancias de Dethloff con la configuración -n 200.0 -p 80.0 -cprob 40 -mprob 70}
\centering
\small
\begin{tabular}{c c c c c c c c}
\hline\hline
Instancia & Costo mínimo & Tiempo(seg.) & Costo promedio & Tiempo promedio(seg.) & CME & \%G & \%GP \\ [0.5ex]
\hline
SCA3-0 & 640.55 & 1.22 & 
640.55 & 1.37 & \bf{635.62} & 
0.78 & 0.78\\SCA3-1 & \bf{697.84} & 1.44 & 
698.50 & 1.30 & 697.84 & 0.00
 & 0.10\\SCA3-2 & 664.21 & 1.32 & 
665.53 & 1.23 & \bf{659.34} & 
0.74 & 0.94\\SCA3-3 & \bf{680.04} & 1.28 & 
680.93 & 1.16 & 680.04 & 0.00
 & 0.13\\SCA3-4 & \bf{690.50} & 0.96 & 
690.50 & 1.24 & 690.50 & 0.00
 & 0.00\\
SCA3-5 & 665.64 & 1.60 & 
665.64 & 1.37 & \bf{659.90} & 
0.87 & 0.87\\SCA3-6 & 652.94 & 1.24 & 
653.40 & 1.24 & \bf{651.09} & 
0.28 & 0.36\\SCA3-7 & 666.15 & 1.06 & 
666.15 & 1.30 & \bf{659.17} & 
1.06 & 1.06\\SCA3-8 & \bf{719.47} & 1.58 & 
724.70 & 1.42 & 719.47 & 0.00
 & 0.73\\SCA3-9 & \bf{681.00} & 0.98 & 
681.00 & 1.30 & 681.00 & 0.00
 & 0.00\\
SCA8-0 & 970.64 & 1.63 & 
970.64 & 1.28 & \bf{961.50} & 
0.95 & 0.95\\SCA8-1 & 1063.24 & 0.84 & 
1068.56 & 1.38 & \bf{1049.65} & 
1.29 & 1.80\\SCA8-2 & 1053.19 & 1.00 & 
1053.89 & 1.14 & \bf{1039.64} & 
1.30 & 1.37\\SCA8-3 & 1014.13 & 1.38 & 
1014.13 & 1.25 & \bf{983.34} & 
3.13 & 3.13\\SCA8-4 & 1078.42 & 1.32 & 
1078.42 & 1.53 & \bf{1065.49} & 
1.21 & 1.21\\SCA8-5 & 1051.83 & 1.72 & 
1057.94 & 1.22 & \bf{1027.08} & 
2.41 & 3.00\\SCA8-6 & 976.69 & 1.08 & 
978.16 & 1.30 & \bf{971.82} & 
0.50 & 0.65\\SCA8-7 & 1077.43 & 1.46 & 
1077.43 & 1.34 & \bf{1051.28} & 
2.49 & 2.49\\SCA8-8 & 1091.18 & 1.30 & 
1091.18 & 1.19 & \bf{1071.18} & 
1.87 & 1.87\\SCA8-9 & 1072.10 & 0.99 & 
1072.83 & 0.98 & \bf{1060.50} & 
1.09 & 1.16\\CON3-0 & 619.09 & 1.60 & 
626.00 & 1.22 & \bf{616.52} & 
0.42 & 1.54\\CON3-1 & 556.92 & 1.21 & 
560.00 & 1.15 & \bf{554.47} & 
0.44 & 1.00\\CON3-2 & 521.38 & 1.51 & 
522.26 & 1.51 & \bf{518.00} & 
0.65 & 0.82\\CON3-3 & 591.20 & 1.77 & 
595.37 & 1.43 & \bf{591.19} & 
0.00 & 0.71\\CON3-4 & 591.43 & 0.95 & 
592.81 & 1.31 & \bf{588.79} & 
0.45 & 0.68\\CON3-5 & 564.88 & 1.03 & 
566.40 & 1.19 & \bf{563.70} & 
0.21 & 0.48\\CON3-6 & 502.47 & 1.07 & 
503.07 & 1.21 & \bf{499.05} & 
0.69 & 0.81\\CON3-7 & 582.33 & 1.10 & 
584.48 & 1.19 & \bf{576.48} & 
1.01 & 1.39\\CON3-8 & 525.30 & 1.80 & 
533.08 & 1.48 & \bf{523.05} & 
0.43 & 1.92\\CON3-9 & 582.79 & 1.27 & 
582.79 & 1.24 & \bf{578.24} & 
0.79 & 0.79\\CON8-0 & 869.08 & 1.26 & 
869.08 & 1.49 & \bf{857.17} & 
1.39 & 1.39\\CON8-1 & 757.04 & 1.18 & 
758.53 & 1.49 & \bf{740.85} & 
2.19 & 2.39\\CON8-2 & 723.98 & 1.06 & 
723.98 & 1.16 & \bf{712.89} & 
1.56 & 1.56\\CON8-3 & 818.51 & 1.09 & 
818.51 & 1.28 & \bf{811.07} & 
0.92 & 0.92\\CON8-4 & 788.45 & 1.47 & 
788.45 & 1.34 & \bf{772.25} & 
2.10 & 2.10\\CON8-5 & 758.84 & 1.27 & 
762.50 & 1.42 & \bf{754.88} & 
0.52 & 1.01\\CON8-6 & 687.57 & 1.26 & 
690.52 & 1.35 & \bf{678.92} & 
1.27 & 1.71\\CON8-7 & 815.54 & 0.97 & 
816.29 & 1.01 & \bf{811.96} & 
0.44 & 0.53\\CON8-8 & 780.69 & 1.44 & 
789.31 & 1.49 & \bf{767.53} & 
1.71 & 2.84\\CON8-9 & 835.44 & 1.36 & 
836.27 & 1.34 & \bf{809.00} & 
3.27 & 3.37\\\bf{PROM.} & 
\bf{767.00} & \bf{1.28} & \bf{768.74} & \bf{1.30} & \bf{758.54} & \bf{1.01} & \bf{1.26}\\[1ex]\hline
\end{tabular}
\label{table:nonlin}
\end{table} \clearpage
\begin{table}[ht]
\caption{Resultados de la ejecución de la metaheurística IGA, utilizando instancias de SalhiNagy con la configuración -n 200.0 -p 80.0 -cprob 90 -mprob 70}
\centering
\small
\begin{tabular}{c c c c c c c c}
\hline\hline
Instancia & Costo mínimo & Tiempo(seg.) & Costo promedio & Tiempo promedio(seg.) & CME & \%G & \%GP \\ [0.5ex]
\hline
CMT1X & 478.36 & 1.69 & 
480.19 & 1.67 & \bf{470.48} & 
1.67 & 2.06\\CMT1Y & 478.57 & 0.95 & 
482.49 & 1.36 & \bf{470.48} & 
1.72 & 2.55\\CMT2X & 700.74 & 3.14 & 
705.43 & 2.87 & \bf{682.39} & 
2.69 & 3.38\\CMT2Y & 703.23 & 3.28 & 
710.30 & 2.91 & \bf{682.39} & 
3.05 & 4.09\\CMT3X & 736.98 & 5.54 & 
742.50 & 6.38 & \bf{719.06} & 
2.49 & 3.26\\CMT3Y & 732.19 & 6.57 & 
740.73 & 6.38 & \bf{719.06} & 
1.83 & 3.01\\CMT4X & 901.32 & 15.88 & 
906.12 & 16.55 & \bf{854.21} & 
5.52 & 6.08\\CMT4Y & 898.71 & 17.44 & 
910.34 & 17.19 & \bf{852.46} & 
5.43 & 6.79\\CMT5X & 1094.88 & 32.74 & 
1106.73 & 33.00 & \bf{1030.56} & 
6.24 & 7.39\\CMT5Y & 1096.86 & 33.56 & 
1112.76 & 33.37 & \bf{1031.69} & 
6.32 & 7.86\\CMT11X & 882.10 & 10.91 & 
894.60 & 10.56 & \bf{831.09} & 
6.14 & 7.64\\CMT11Y & 886.96 & 11.60 & 
901.57 & 11.47 & \bf{829.85} & 
6.88 & 8.64\\CMT12X & 675.05 & 6.78 & 
678.79 & 6.84 & \bf{658.83} & 
2.46 & 3.03\\CMT12Y & 675.24 & 6.61 & 
679.37 & 6.14 & \bf{660.47} & 
2.24 & 2.86\\\bf{PROM.} & 
\bf{781.51} & \bf{11.19} & \bf{789.42} & \bf{11.19} & \bf{749.50} & \bf{3.91} & \bf{4.90}\\[1ex]\hline
\end{tabular}
\label{table:nonlin}
\end{table} \clearpage
\begin{table}[ht]
\caption{Resultados de la ejecución de la metaheurística IGA, utilizando instancias de Dethloff con la configuración -n 200.0 -p 90.0 -cprob 40 -mprob 70}
\centering
\small
\begin{tabular}{c c c c c c c c}
\hline\hline
Instancia & Costo mínimo & Tiempo(seg.) & Costo promedio & Tiempo promedio(seg.) & CME & \%G & \%GP \\ [0.5ex]
\hline
SCA3-0 & 640.55 & 1.16 & 
640.55 & 1.53 & \bf{635.62} & 
0.78 & 0.78\\SCA3-1 & 700.50 & 1.63 & 
702.67 & 1.49 & \bf{697.84} & 
0.38 & 0.69\\SCA3-2 & 664.21 & 1.54 & 
664.66 & 1.40 & \bf{659.34} & 
0.74 & 0.81\\SCA3-3 & 681.16 & 1.29 & 
681.25 & 1.52 & \bf{680.04} & 
0.16 & 0.18\\SCA3-4 & \bf{690.50} & 1.25 & 
690.50 & 1.30 & 690.50 & 0.00
 & 0.00\\
SCA3-5 & 665.64 & 1.84 & 
671.68 & 1.44 & \bf{659.90} & 
0.87 & 1.78\\SCA3-6 & 652.94 & 1.42 & 
653.40 & 1.50 & \bf{651.09} & 
0.28 & 0.36\\SCA3-7 & 664.88 & 1.56 & 
665.83 & 1.31 & \bf{659.17} & 
0.87 & 1.01\\SCA3-8 & \bf{719.47} & 1.70 & 
724.88 & 1.55 & 719.47 & 0.00
 & 0.75\\SCA3-9 & \bf{681.00} & 2.10 & 
681.00 & 1.77 & 681.00 & 0.00
 & 0.00\\
SCA8-0 & 979.79 & 1.31 & 
989.29 & 1.43 & \bf{961.50} & 
1.90 & 2.89\\SCA8-1 & 1054.54 & 1.27 & 
1061.69 & 1.31 & \bf{1049.65} & 
0.47 & 1.15\\SCA8-2 & 1044.48 & 1.03 & 
1049.21 & 1.31 & \bf{1039.64} & 
0.47 & 0.92\\SCA8-3 & 991.84 & 1.14 & 
996.70 & 1.34 & \bf{983.34} & 
0.86 & 1.36\\SCA8-4 & 1078.42 & 0.99 & 
1078.42 & 1.30 & \bf{1065.49} & 
1.21 & 1.21\\SCA8-5 & 1053.09 & 1.58 & 
1054.00 & 1.58 & \bf{1027.08} & 
2.53 & 2.62\\SCA8-6 & 986.34 & 2.00 & 
993.93 & 1.75 & \bf{971.82} & 
1.49 & 2.28\\SCA8-7 & 1075.27 & 2.00 & 
1078.01 & 1.33 & \bf{1051.28} & 
2.28 & 2.54\\SCA8-8 & \bf{1071.18} & 1.29 & 
1081.46 & 1.65 & 1071.18 & 0.00
 & 0.96\\SCA8-9 & 1078.30 & 1.09 & 
1080.64 & 1.37 & \bf{1060.50} & 
1.68 & 1.90\\CON3-0 & 623.84 & 2.05 & 
632.51 & 1.48 & \bf{616.52} & 
1.19 & 2.59\\CON3-1 & 560.75 & 1.10 & 
560.75 & 1.58 & \bf{554.47} & 
1.13 & 1.13\\CON3-2 & 521.38 & 1.32 & 
521.38 & 1.35 & \bf{518.00} & 
0.65 & 0.65\\CON3-3 & 591.20 & 1.12 & 
597.30 & 1.32 & \bf{591.19} & 
0.00 & 1.03\\CON3-4 & 592.58 & 1.14 & 
593.55 & 1.24 & \bf{588.79} & 
0.64 & 0.81\\CON3-5 & \bf{563.70} & 1.01 & 
566.12 & 1.28 & 563.70 & 0.00
 & 0.43\\CON3-6 & 502.49 & 1.35 & 
502.78 & 1.47 & \bf{499.05} & 
0.69 & 0.75\\CON3-7 & 577.91 & 1.72 & 
580.02 & 1.58 & \bf{576.48} & 
0.25 & 0.61\\CON3-8 & 523.14 & 1.97 & 
527.16 & 1.61 & \bf{523.05} & 
0.02 & 0.79\\CON3-9 & 588.11 & 1.65 & 
588.94 & 1.33 & \bf{578.24} & 
1.71 & 1.85\\CON8-0 & 870.94 & 2.76 & 
872.97 & 1.79 & \bf{857.17} & 
1.61 & 1.84\\CON8-1 & 754.50 & 1.15 & 
754.50 & 1.23 & \bf{740.85} & 
1.84 & 1.84\\CON8-2 & 716.72 & 1.40 & 
722.92 & 1.68 & \bf{712.89} & 
0.54 & 1.41\\CON8-3 & 813.17 & 1.77 & 
820.29 & 1.50 & \bf{811.07} & 
0.26 & 1.14\\CON8-4 & 793.35 & 1.50 & 
798.21 & 1.26 & \bf{772.25} & 
2.73 & 3.36\\CON8-5 & 763.13 & 1.38 & 
763.13 & 1.58 & \bf{754.88} & 
1.09 & 1.09\\CON8-6 & \bf{678.92} & 2.16 & 
682.14 & 1.52 & 678.92 & 0.00
 & 0.48\\CON8-7 & 815.60 & 1.84 & 
815.96 & 1.52 & \bf{811.96} & 
0.45 & 0.49\\CON8-8 & 790.77 & 1.38 & 
793.32 & 1.58 & \bf{767.53} & 
3.03 & 3.36\\CON8-9 & 822.30 & 1.50 & 
825.77 & 1.75 & \bf{809.00} & 
1.64 & 2.07\\\bf{PROM.} & 
\bf{765.97} & \bf{1.51} & \bf{768.99} & \bf{1.47} & \bf{758.54} & \bf{0.91} & \bf{1.30}\\[1ex]\hline
\end{tabular}
\label{table:nonlin}
\end{table} \clearpage
\begin{table}[ht]
\caption{Resultados de la ejecución de la metaheurística IGA, utilizando instancias de SalhiNagy con la configuración -n 200.0 -p 90.0 -cprob 90 -mprob 70}
\centering
\small
\begin{tabular}{c c c c c c c c}
\hline\hline
Instancia & Costo mínimo & Tiempo(seg.) & Costo promedio & Tiempo promedio(seg.) & CME & \%G & \%GP \\ [0.5ex]
\hline
CMT1X & 481.31 & 1.52 & 
481.56 & 1.74 & \bf{470.48} & 
2.30 & 2.36\\CMT1Y & 481.95 & 1.87 & 
485.52 & 1.70 & \bf{470.48} & 
2.44 & 3.20\\CMT2X & 715.77 & 3.57 & 
721.16 & 3.40 & \bf{682.39} & 
4.89 & 5.68\\CMT2Y & 716.36 & 3.44 & 
718.58 & 3.38 & \bf{682.39} & 
4.98 & 5.30\\CMT3X & 734.83 & 8.80 & 
739.42 & 7.64 & \bf{719.06} & 
2.19 & 2.83\\CMT3Y & 731.52 & 6.87 & 
735.55 & 7.21 & \bf{719.06} & 
1.73 & 2.29\\CMT4X & 902.80 & 17.38 & 
910.71 & 17.64 & \bf{854.21} & 
5.69 & 6.61\\CMT4Y & 896.92 & 18.97 & 
906.81 & 18.99 & \bf{852.46} & 
5.22 & 6.38\\CMT5X & 1109.25 & 36.73 & 
1116.83 & 37.74 & \bf{1030.56} & 
7.64 & 8.37\\CMT5Y & 1087.85 & 37.36 & 
1099.78 & 36.95 & \bf{1031.69} & 
5.44 & 6.60\\CMT11X & 883.41 & 11.76 & 
906.31 & 11.26 & \bf{831.09} & 
6.30 & 9.05\\CMT11Y & 887.31 & 13.27 & 
896.03 & 13.03 & \bf{829.85} & 
6.92 & 7.98\\CMT12X & 675.38 & 7.02 & 
681.38 & 7.18 & \bf{658.83} & 
2.51 & 3.42\\CMT12Y & 674.29 & 6.47 & 
677.33 & 7.03 & \bf{660.47} & 
2.09 & 2.55\\\bf{PROM.} & 
\bf{784.21} & \bf{12.50} & \bf{791.21} & \bf{12.49} & \bf{749.50} & \bf{4.31} & \bf{5.19}\\[1ex]\hline
\end{tabular}
\label{table:nonlin}
\end{table} \clearpage
\begin{table}[ht]
\caption{Resultados de la ejecución de la metaheurística IGA, utilizando instancias de Dethloff con la configuración -n 200.0 -p 100.0 -cprob 40 -mprob 70}
\centering
\small
\begin{tabular}{c c c c c c c c}
\hline\hline
Instancia & Costo mínimo & Tiempo(seg.) & Costo promedio & Tiempo promedio(seg.) & CME & \%G & \%GP \\ [0.5ex]
\hline
SCA3-0 & 636.06 & 1.53 & 
639.43 & 1.93 & \bf{635.62} & 
0.07 & 0.60\\SCA3-1 & \bf{697.84} & 1.47 & 
698.50 & 1.64 & 697.84 & 0.00
 & 0.10\\SCA3-2 & \bf{659.34} & 1.76 & 
662.22 & 1.55 & 659.34 & 0.00
 & 0.44\\SCA3-3 & \bf{680.04} & 2.09 & 
680.04 & 1.53 & 680.04 & 0.00
 & 0.00\\
SCA3-4 & \bf{690.50} & 1.74 & 
690.50 & 1.91 & 690.50 & 0.00
 & 0.00\\
SCA3-5 & \bf{659.90} & 1.68 & 
663.87 & 1.77 & 659.90 & 0.00
 & 0.60\\SCA3-6 & 652.94 & 1.72 & 
652.94 & 1.62 & \bf{651.09} & 
0.28 & 0.28\\SCA3-7 & 666.15 & 1.13 & 
666.15 & 1.64 & \bf{659.17} & 
1.06 & 1.06\\SCA3-8 & \bf{719.47} & 1.64 & 
721.96 & 1.77 & 719.47 & 0.00
 & 0.35\\SCA3-9 & \bf{681.00} & 1.63 & 
681.59 & 1.64 & 681.00 & 0.00
 & 0.09\\SCA8-0 & 988.90 & 2.04 & 
995.51 & 1.61 & \bf{961.50} & 
2.85 & 3.54\\SCA8-1 & 1066.06 & 2.18 & 
1074.58 & 1.69 & \bf{1049.65} & 
1.56 & 2.38\\SCA8-2 & 1046.22 & 1.74 & 
1052.01 & 1.80 & \bf{1039.64} & 
0.63 & 1.19\\SCA8-3 & 1003.78 & 1.15 & 
1004.84 & 1.67 & \bf{983.34} & 
2.08 & 2.19\\SCA8-4 & 1070.75 & 1.03 & 
1076.35 & 1.41 & \bf{1065.49} & 
0.49 & 1.02\\SCA8-5 & 1050.50 & 1.45 & 
1051.82 & 1.48 & \bf{1027.08} & 
2.28 & 2.41\\SCA8-6 & 982.05 & 1.72 & 
982.05 & 1.59 & \bf{971.82} & 
1.05 & 1.05\\SCA8-7 & 1064.06 & 1.63 & 
1064.06 & 1.43 & \bf{1051.28} & 
1.22 & 1.22\\SCA8-8 & \bf{1071.18} & 1.22 & 
1075.65 & 1.44 & 1071.18 & 0.00
 & 0.42\\SCA8-9 & 1072.10 & 1.21 & 
1074.08 & 1.44 & \bf{1060.50} & 
1.09 & 1.28\\CON3-0 & 620.76 & 2.44 & 
623.57 & 2.05 & \bf{616.52} & 
0.69 & 1.14\\CON3-1 & 560.75 & 1.62 & 
560.75 & 1.56 & \bf{554.47} & 
1.13 & 1.13\\CON3-2 & 521.38 & 1.74 & 
521.38 & 1.68 & \bf{518.00} & 
0.65 & 0.65\\CON3-3 & 594.11 & 1.26 & 
599.08 & 1.62 & \bf{591.19} & 
0.49 & 1.33\\CON3-4 & 592.58 & 1.45 & 
595.03 & 1.62 & \bf{588.79} & 
0.64 & 1.06\\CON3-5 & \bf{563.70} & 2.33 & 
563.70 & 1.74 & 563.70 & 0.00
 & 0.00\\
CON3-6 & 504.91 & 1.54 & 
505.90 & 1.74 & \bf{499.05} & 
1.17 & 1.37\\CON3-7 & \bf{576.48} & 1.22 & 
576.48 & 1.50 & 576.48 & 0.00
 & 0.00\\
CON3-8 & 524.59 & 1.39 & 
525.47 & 1.57 & \bf{523.05} & 
0.29 & 0.46\\CON3-9 & 584.46 & 1.32 & 
587.27 & 1.57 & \bf{578.24} & 
1.08 & 1.56\\CON8-0 & 870.28 & 1.64 & 
876.28 & 1.61 & \bf{857.17} & 
1.53 & 2.23\\CON8-1 & 748.39 & 1.57 & 
750.99 & 1.75 & \bf{740.85} & 
1.02 & 1.37\\CON8-2 & 719.16 & 2.40 & 
719.16 & 1.85 & \bf{712.89} & 
0.88 & 0.88\\CON8-3 & 834.84 & 2.18 & 
834.84 & 1.98 & \bf{811.07} & 
2.93 & 2.93\\CON8-4 & \bf{772.25} & 1.43 & 
780.07 & 1.48 & 772.25 & 0.00
 & 1.01\\CON8-5 & 758.12 & 1.37 & 
760.07 & 1.46 & \bf{754.88} & 
0.43 & 0.69\\CON8-6 & 693.48 & 1.58 & 
693.73 & 1.51 & \bf{678.92} & 
2.14 & 2.18\\CON8-7 & 815.72 & 1.46 & 
820.58 & 1.41 & \bf{811.96} & 
0.46 & 1.06\\CON8-8 & 786.86 & 1.56 & 
790.58 & 1.82 & \bf{767.53} & 
2.52 & 3.00\\CON8-9 & 815.91 & 1.44 & 
815.91 & 1.42 & \bf{809.00} & 
0.85 & 0.85\\\bf{PROM.} & 
\bf{765.44} & \bf{1.62} & \bf{767.72} & \bf{1.64} & \bf{758.54} & \bf{0.84} & \bf{1.13}\\[1ex]\hline
\end{tabular}
\label{table:nonlin}
\end{table} \clearpage
\begin{table}[ht]
\caption{Resultados de la ejecución de la metaheurística IGA, utilizando instancias de SalhiNagy con la configuración -n 200.0 -p 100.0 -cprob 90 -mprob 70}
\centering
\small
\begin{tabular}{c c c c c c c c}
\hline\hline
Instancia & Costo mínimo & Tiempo(seg.) & Costo promedio & Tiempo promedio(seg.) & CME & \%G & \%GP \\ [0.5ex]
\hline
CMT1X & 478.90 & 2.10 & 
480.88 & 1.89 & \bf{470.48} & 
1.79 & 2.21\\CMT1Y & 480.16 & 1.55 & 
481.58 & 1.74 & \bf{470.48} & 
2.06 & 2.36\\CMT2X & 702.59 & 3.76 & 
710.27 & 3.50 & \bf{682.39} & 
2.96 & 4.09\\CMT2Y & 702.60 & 3.64 & 
712.60 & 3.59 & \bf{682.39} & 
2.96 & 4.43\\CMT3X & 732.70 & 8.17 & 
739.31 & 8.27 & \bf{719.06} & 
1.90 & 2.82\\CMT3Y & 739.63 & 7.50 & 
742.38 & 7.87 & \bf{719.06} & 
2.86 & 3.24\\CMT4X & 900.62 & 20.09 & 
904.90 & 20.55 & \bf{854.21} & 
5.43 & 5.93\\CMT4Y & 898.16 & 21.30 & 
901.42 & 20.98 & \bf{852.46} & 
5.36 & 5.74\\CMT5X & 1088.32 & 42.33 & 
1110.79 & 41.50 & \bf{1030.56} & 
5.60 & 7.79\\CMT5Y & 1088.80 & 42.49 & 
1115.46 & 42.12 & \bf{1031.69} & 
5.54 & 8.12\\CMT11X & 892.39 & 13.58 & 
901.31 & 12.87 & \bf{831.09} & 
7.38 & 8.45\\CMT11Y & 887.76 & 14.56 & 
897.77 & 14.56 & \bf{829.85} & 
6.98 & 8.19\\CMT12X & 675.93 & 8.59 & 
683.86 & 8.08 & \bf{658.83} & 
2.60 & 3.80\\CMT12Y & 673.89 & 8.60 & 
674.97 & 8.09 & \bf{660.47} & 
2.03 & 2.20\\\bf{PROM.} & 
\bf{781.60} & \bf{14.16} & \bf{789.82} & \bf{13.97} & \bf{749.50} & \bf{3.96} & \bf{4.95}\\[1ex]\hline
\end{tabular}
\label{table:nonlin}
\end{table} \clearpage
\begin{table}[ht]
\caption{Resultados de la ejecución de la metaheurística IGA, utilizando instancias de Dethloff con la configuración -n 250.0 -p 30.0 -cprob 40 -mprob 70}
\centering
\small
\begin{tabular}{c c c c c c c c}
\hline\hline
Instancia & Costo mínimo & Tiempo(seg.) & Costo promedio & Tiempo promedio(seg.) & CME & \%G & \%GP \\ [0.5ex]
\hline
SCA3-0 & 640.55 & 0.39 & 
640.84 & 0.64 & \bf{635.62} & 
0.78 & 0.82\\SCA3-1 & 700.50 & 0.60 & 
702.79 & 0.69 & \bf{697.84} & 
0.38 & 0.71\\SCA3-2 & \bf{659.34} & 0.45 & 
667.04 & 0.58 & 659.34 & 0.00
 & 1.17\\SCA3-3 & 681.74 & 0.53 & 
682.89 & 0.59 & \bf{680.04} & 
0.25 & 0.42\\SCA3-4 & \bf{690.50} & 0.42 & 
690.50 & 0.52 & 690.50 & 0.00
 & 0.00\\
SCA3-5 & 679.04 & 0.50 & 
681.07 & 0.46 & \bf{659.90} & 
2.90 & 3.21\\SCA3-6 & 652.94 & 0.56 & 
652.94 & 0.62 & \bf{651.09} & 
0.28 & 0.28\\SCA3-7 & 666.15 & 0.39 & 
669.95 & 0.44 & \bf{659.17} & 
1.06 & 1.64\\SCA3-8 & 729.02 & 0.59 & 
730.90 & 0.62 & \bf{719.47} & 
1.33 & 1.59\\SCA3-9 & \bf{681.00} & 0.53 & 
681.00 & 0.44 & 681.00 & 0.00
 & 0.00\\
SCA8-0 & 1019.67 & 0.63 & 
1022.73 & 0.58 & \bf{961.50} & 
6.05 & 6.37\\SCA8-1 & 1080.32 & 0.41 & 
1080.32 & 0.36 & \bf{1049.65} & 
2.92 & 2.92\\SCA8-2 & 1053.55 & 0.41 & 
1054.24 & 0.57 & \bf{1039.64} & 
1.34 & 1.40\\SCA8-3 & 1022.76 & 0.81 & 
1022.76 & 0.53 & \bf{983.34} & 
4.01 & 4.01\\SCA8-4 & 1070.32 & 0.38 & 
1070.32 & 0.44 & \bf{1065.49} & 
0.45 & 0.45\\SCA8-5 & 1053.36 & 0.38 & 
1053.36 & 0.39 & \bf{1027.08} & 
2.56 & 2.56\\SCA8-6 & 972.48 & 0.56 & 
972.48 & 0.41 & \bf{971.82} & 
0.07 & 0.07\\SCA8-7 & 1083.84 & 0.35 & 
1088.84 & 0.47 & \bf{1051.28} & 
3.10 & 3.57\\SCA8-8 & 1094.63 & 0.36 & 
1094.63 & 0.43 & \bf{1071.18} & 
2.19 & 2.19\\SCA8-9 & 1079.34 & 0.34 & 
1079.58 & 0.35 & \bf{1060.50} & 
1.78 & 1.80\\CON3-0 & 632.87 & 0.46 & 
632.87 & 0.66 & \bf{616.52} & 
2.65 & 2.65\\CON3-1 & 560.75 & 0.49 & 
560.75 & 0.53 & \bf{554.47} & 
1.13 & 1.13\\CON3-2 & 524.89 & 0.68 & 
524.89 & 0.61 & \bf{518.00} & 
1.33 & 1.33\\CON3-3 & 592.43 & 0.60 & 
592.43 & 0.48 & \bf{591.19} & 
0.21 & 0.21\\CON3-4 & 591.43 & 0.38 & 
598.67 & 0.46 & \bf{588.79} & 
0.45 & 1.68\\CON3-5 & 568.79 & 0.44 & 
569.91 & 0.67 & \bf{563.70} & 
0.90 & 1.10\\CON3-6 & 502.16 & 0.51 & 
505.96 & 0.47 & \bf{499.05} & 
0.62 & 1.38\\CON3-7 & 577.54 & 0.32 & 
583.87 & 0.40 & \bf{576.48} & 
0.18 & 1.28\\CON3-8 & 527.52 & 0.44 & 
533.35 & 0.47 & \bf{523.05} & 
0.85 & 1.97\\CON3-9 & 588.99 & 0.48 & 
589.89 & 0.56 & \bf{578.24} & 
1.86 & 2.01\\CON8-0 & 875.52 & 0.48 & 
875.52 & 0.56 & \bf{857.17} & 
2.14 & 2.14\\CON8-1 & 757.59 & 0.58 & 
767.84 & 0.64 & \bf{740.85} & 
2.26 & 3.64\\CON8-2 & 728.69 & 0.46 & 
732.03 & 0.57 & \bf{712.89} & 
2.22 & 2.68\\CON8-3 & 830.13 & 0.50 & 
830.55 & 0.44 & \bf{811.07} & 
2.35 & 2.40\\CON8-4 & 794.66 & 0.56 & 
797.87 & 0.46 & \bf{772.25} & 
2.90 & 3.32\\CON8-5 & 768.53 & 0.64 & 
775.46 & 0.50 & \bf{754.88} & 
1.81 & 2.73\\CON8-6 & 698.90 & 0.59 & 
698.90 & 0.52 & \bf{678.92} & 
2.94 & 2.94\\CON8-7 & 816.67 & 0.36 & 
825.28 & 0.53 & \bf{811.96} & 
0.58 & 1.64\\CON8-8 & 785.02 & 0.72 & 
790.42 & 0.55 & \bf{767.53} & 
2.28 & 2.98\\CON8-9 & 843.52 & 0.66 & 
846.58 & 0.47 & \bf{809.00} & 
4.27 & 4.64\\\bf{PROM.} & 
\bf{771.94} & \bf{0.50} & \bf{774.31} & \bf{0.52} & \bf{758.54} & \bf{1.63} & \bf{1.98}\\[1ex]\hline
\end{tabular}
\label{table:nonlin}
\end{table} \clearpage
\begin{table}[ht]
\caption{Resultados de la ejecución de la metaheurística IGA, utilizando instancias de SalhiNagy con la configuración -n 250.0 -p 30.0 -cprob 90 -mprob 70}
\centering
\small
\begin{tabular}{c c c c c c c c}
\hline\hline
Instancia & Costo mínimo & Tiempo(seg.) & Costo promedio & Tiempo promedio(seg.) & CME & \%G & \%GP \\ [0.5ex]
\hline
CMT1X & 476.71 & 0.75 & 
483.38 & 0.70 & \bf{470.48} & 
1.32 & 2.74\\CMT1Y & 479.97 & 0.52 & 
484.64 & 0.52 & \bf{470.48} & 
2.02 & 3.01\\CMT2X & 718.92 & 1.44 & 
724.34 & 1.30 & \bf{682.39} & 
5.35 & 6.15\\CMT2Y & 718.24 & 1.03 & 
719.45 & 1.10 & \bf{682.39} & 
5.25 & 5.43\\CMT3X & 744.04 & 2.83 & 
750.32 & 2.72 & \bf{719.06} & 
3.47 & 4.35\\CMT3Y & 733.08 & 2.67 & 
735.34 & 2.60 & \bf{719.06} & 
1.95 & 2.26\\CMT4X & 907.15 & 6.88 & 
917.53 & 6.64 & \bf{854.21} & 
6.20 & 7.41\\CMT4Y & 905.90 & 6.91 & 
912.52 & 6.92 & \bf{852.46} & 
6.27 & 7.05\\CMT5X & 1104.09 & 12.13 & 
1120.73 & 12.95 & \bf{1030.56} & 
7.13 & 8.75\\CMT5Y & 1123.50 & 13.30 & 
1132.32 & 13.50 & \bf{1031.69} & 
8.90 & 9.75\\CMT11X & 909.99 & 4.38 & 
925.16 & 4.35 & \bf{831.09} & 
9.49 & 11.32\\CMT11Y & 881.42 & 4.94 & 
903.44 & 4.87 & \bf{829.85} & 
6.21 & 8.87\\CMT12X & 674.59 & 2.37 & 
677.58 & 2.44 & \bf{658.83} & 
2.39 & 2.85\\CMT12Y & 676.12 & 2.83 & 
679.63 & 2.59 & \bf{660.47} & 
2.37 & 2.90\\\bf{PROM.} & 
\bf{789.55} & \bf{4.50} & \bf{797.60} & \bf{4.52} & \bf{749.50} & \bf{4.88} & \bf{5.92}\\[1ex]\hline
\end{tabular}
\label{table:nonlin}
\end{table} \clearpage
\begin{table}[ht]
\caption{Resultados de la ejecución de la metaheurística IGA, utilizando instancias de Dethloff con la configuración -n 250.0 -p 40.0 -cprob 40 -mprob 70}
\centering
\small
\begin{tabular}{c c c c c c c c}
\hline\hline
Instancia & Costo mínimo & Tiempo(seg.) & Costo promedio & Tiempo promedio(seg.) & CME & \%G & \%GP \\ [0.5ex]
\hline
SCA3-0 & 640.55 & 0.71 & 
640.55 & 0.86 & \bf{635.62} & 
0.78 & 0.78\\SCA3-1 & 701.53 & 0.55 & 
701.53 & 0.70 & \bf{697.84} & 
0.53 & 0.53\\SCA3-2 & 669.06 & 0.68 & 
669.06 & 0.60 & \bf{659.34} & 
1.47 & 1.47\\SCA3-3 & 685.47 & 0.59 & 
685.99 & 0.60 & \bf{680.04} & 
0.80 & 0.87\\SCA3-4 & \bf{690.50} & 1.06 & 
690.50 & 0.85 & 690.50 & 0.00
 & 0.00\\
SCA3-5 & 666.67 & 1.02 & 
666.67 & 0.83 & \bf{659.90} & 
1.03 & 1.03\\SCA3-6 & 652.94 & 0.44 & 
653.16 & 0.71 & \bf{651.09} & 
0.28 & 0.32\\SCA3-7 & 666.15 & 0.51 & 
669.76 & 0.77 & \bf{659.17} & 
1.06 & 1.61\\SCA3-8 & 726.58 & 0.58 & 
726.73 & 0.77 & \bf{719.47} & 
0.99 & 1.01\\SCA3-9 & \bf{681.00} & 0.60 & 
681.00 & 0.58 & 681.00 & 0.00
 & 0.00\\
SCA8-0 & 970.64 & 0.60 & 
996.18 & 0.78 & \bf{961.50} & 
0.95 & 3.61\\SCA8-1 & 1063.38 & 0.61 & 
1067.27 & 0.84 & \bf{1049.65} & 
1.31 & 1.68\\SCA8-2 & 1050.37 & 0.75 & 
1052.49 & 0.88 & \bf{1039.64} & 
1.03 & 1.24\\SCA8-3 & 1027.37 & 0.70 & 
1031.05 & 0.76 & \bf{983.34} & 
4.48 & 4.85\\SCA8-4 & 1101.84 & 1.05 & 
1101.84 & 0.64 & \bf{1065.49} & 
3.41 & 3.41\\SCA8-5 & 1058.45 & 0.55 & 
1058.45 & 0.76 & \bf{1027.08} & 
3.05 & 3.05\\SCA8-6 & 998.73 & 0.52 & 
999.02 & 0.66 & \bf{971.82} & 
2.77 & 2.80\\SCA8-7 & 1083.36 & 0.62 & 
1087.13 & 0.81 & \bf{1051.28} & 
3.05 & 3.41\\SCA8-8 & 1095.58 & 0.55 & 
1096.46 & 0.51 & \bf{1071.18} & 
2.28 & 2.36\\SCA8-9 & 1078.30 & 1.10 & 
1087.41 & 0.83 & \bf{1060.50} & 
1.68 & 2.54\\CON3-0 & 630.30 & 0.72 & 
630.92 & 0.77 & \bf{616.52} & 
2.24 & 2.34\\CON3-1 & 560.75 & 0.56 & 
560.75 & 0.58 & \bf{554.47} & 
1.13 & 1.13\\CON3-2 & 521.38 & 0.59 & 
521.38 & 0.66 & \bf{518.00} & 
0.65 & 0.65\\CON3-3 & 592.41 & 0.58 & 
603.87 & 0.64 & \bf{591.19} & 
0.21 & 2.14\\CON3-4 & 593.78 & 0.46 & 
598.86 & 0.56 & \bf{588.79} & 
0.85 & 1.71\\CON3-5 & \bf{563.70} & 0.98 & 
568.01 & 0.77 & 563.70 & 0.00
 & 0.77\\CON3-6 & 505.31 & 0.76 & 
505.81 & 0.71 & \bf{499.05} & 
1.25 & 1.35\\CON3-7 & 582.33 & 0.68 & 
588.86 & 0.51 & \bf{576.48} & 
1.01 & 2.15\\CON3-8 & 525.47 & 0.65 & 
526.68 & 0.64 & \bf{523.05} & 
0.46 & 0.69\\CON3-9 & 588.48 & 0.71 & 
588.75 & 0.74 & \bf{578.24} & 
1.77 & 1.82\\CON8-0 & 859.74 & 0.91 & 
859.74 & 0.87 & \bf{857.17} & 
0.30 & 0.30\\CON8-1 & \bf{740.85} & 0.52 & 
751.49 & 0.64 & 740.85 & 0.00
 & 1.44\\CON8-2 & 713.60 & 0.84 & 
713.60 & 0.73 & \bf{712.89} & 
0.10 & 0.10\\CON8-3 & 814.50 & 0.64 & 
814.50 & 0.67 & \bf{811.07} & 
0.42 & 0.42\\CON8-4 & 804.76 & 0.50 & 
805.77 & 0.63 & \bf{772.25} & 
4.21 & 4.34\\CON8-5 & 758.12 & 0.82 & 
761.00 & 0.85 & \bf{754.88} & 
0.43 & 0.81\\CON8-6 & 692.53 & 0.64 & 
695.24 & 0.70 & \bf{678.92} & 
2.00 & 2.40\\CON8-7 & 815.71 & 0.57 & 
821.68 & 0.72 & \bf{811.96} & 
0.46 & 1.20\\CON8-8 & 783.15 & 0.88 & 
795.76 & 0.60 & \bf{767.53} & 
2.04 & 3.68\\CON8-9 & 834.78 & 0.54 & 
836.97 & 0.61 & \bf{809.00} & 
3.19 & 3.46\\\bf{PROM.} & 
\bf{769.75} & \bf{0.68} & \bf{772.80} & \bf{0.71} & \bf{758.54} & \bf{1.34} & \bf{1.74}\\[1ex]\hline
\end{tabular}
\label{table:nonlin}
\end{table} \clearpage
\begin{table}[ht]
\caption{Resultados de la ejecución de la metaheurística IGA, utilizando instancias de SalhiNagy con la configuración -n 250.0 -p 40.0 -cprob 90 -mprob 70}
\centering
\small
\begin{tabular}{c c c c c c c c}
\hline\hline
Instancia & Costo mínimo & Tiempo(seg.) & Costo promedio & Tiempo promedio(seg.) & CME & \%G & \%GP \\ [0.5ex]
\hline
CMT1X & 478.84 & 0.63 & 
482.25 & 0.81 & \bf{470.48} & 
1.78 & 2.50\\CMT1Y & 482.03 & 0.95 & 
487.35 & 0.84 & \bf{470.48} & 
2.45 & 3.59\\CMT2X & 714.35 & 1.89 & 
719.93 & 1.70 & \bf{682.39} & 
4.68 & 5.50\\CMT2Y & 703.13 & 1.80 & 
707.70 & 1.69 & \bf{682.39} & 
3.04 & 3.71\\CMT3X & 733.69 & 3.23 & 
747.32 & 3.36 & \bf{719.06} & 
2.03 & 3.93\\CMT3Y & 741.41 & 3.38 & 
757.29 & 3.24 & \bf{719.06} & 
3.11 & 5.32\\CMT4X & 899.87 & 8.74 & 
910.57 & 8.61 & \bf{854.21} & 
5.35 & 6.60\\CMT4Y & 910.75 & 8.66 & 
916.74 & 8.66 & \bf{852.46} & 
6.84 & 7.54\\CMT5X & 1085.67 & 16.82 & 
1114.82 & 17.25 & \bf{1030.56} & 
5.35 & 8.18\\CMT5Y & 1098.31 & 17.83 & 
1113.61 & 17.84 & \bf{1031.69} & 
6.46 & 7.94\\CMT11X & 872.86 & 5.76 & 
894.60 & 5.65 & \bf{831.09} & 
5.03 & 7.64\\CMT11Y & 886.96 & 6.64 & 
893.99 & 6.43 & \bf{829.85} & 
6.88 & 7.73\\CMT12X & 675.57 & 3.48 & 
681.49 & 3.62 & \bf{658.83} & 
2.54 & 3.44\\CMT12Y & 674.20 & 3.27 & 
675.61 & 3.36 & \bf{660.47} & 
2.08 & 2.29\\\bf{PROM.} & 
\bf{782.69} & \bf{5.93} & \bf{793.09} & \bf{5.93} & \bf{749.50} & \bf{4.12} & \bf{5.42}\\[1ex]\hline
\end{tabular}
\label{table:nonlin}
\end{table} \clearpage
\begin{table}[ht]
\caption{Resultados de la ejecución de la metaheurística IGA, utilizando instancias de Dethloff con la configuración -n 250.0 -p 50.0 -cprob 40 -mprob 70}
\centering
\small
\begin{tabular}{c c c c c c c c}
\hline\hline
Instancia & Costo mínimo & Tiempo(seg.) & Costo promedio & Tiempo promedio(seg.) & CME & \%G & \%GP \\ [0.5ex]
\hline
SCA3-0 & 640.55 & 0.98 & 
640.55 & 0.84 & \bf{635.62} & 
0.78 & 0.78\\SCA3-1 & 706.90 & 0.62 & 
706.90 & 0.91 & \bf{697.84} & 
1.30 & 1.30\\SCA3-2 & 661.13 & 0.97 & 
665.49 & 0.81 & \bf{659.34} & 
0.27 & 0.93\\SCA3-3 & \bf{680.04} & 0.84 & 
680.37 & 0.87 & 680.04 & 0.00
 & 0.05\\SCA3-4 & \bf{690.50} & 0.73 & 
691.53 & 0.79 & 690.50 & 0.00
 & 0.15\\SCA3-5 & 673.56 & 0.77 & 
673.56 & 0.89 & \bf{659.90} & 
2.07 & 2.07\\SCA3-6 & 652.94 & 0.70 & 
654.00 & 0.73 & \bf{651.09} & 
0.28 & 0.45\\SCA3-7 & 664.88 & 0.84 & 
665.20 & 0.72 & \bf{659.17} & 
0.87 & 0.91\\SCA3-8 & 721.45 & 0.71 & 
722.87 & 0.91 & \bf{719.47} & 
0.28 & 0.47\\SCA3-9 & \bf{681.00} & 0.96 & 
681.59 & 0.97 & 681.00 & 0.00
 & 0.09\\SCA8-0 & 998.35 & 0.75 & 
1000.49 & 0.97 & \bf{961.50} & 
3.83 & 4.06\\SCA8-1 & 1077.62 & 0.77 & 
1077.62 & 0.83 & \bf{1049.65} & 
2.66 & 2.66\\SCA8-2 & 1051.33 & 1.30 & 
1052.54 & 0.96 & \bf{1039.64} & 
1.12 & 1.24\\SCA8-3 & 1015.81 & 0.79 & 
1015.81 & 0.72 & \bf{983.34} & 
3.30 & 3.30\\SCA8-4 & 1093.89 & 0.66 & 
1093.89 & 0.96 & \bf{1065.49} & 
2.67 & 2.67\\SCA8-5 & 1048.35 & 1.07 & 
1054.44 & 0.98 & \bf{1027.08} & 
2.07 & 2.66\\SCA8-6 & 981.37 & 0.75 & 
981.37 & 0.81 & \bf{971.82} & 
0.98 & 0.98\\SCA8-7 & 1076.37 & 1.18 & 
1077.31 & 1.04 & \bf{1051.28} & 
2.39 & 2.48\\SCA8-8 & 1091.18 & 0.67 & 
1091.18 & 0.76 & \bf{1071.18} & 
1.87 & 1.87\\SCA8-9 & 1071.72 & 0.89 & 
1071.72 & 0.86 & \bf{1060.50} & 
1.06 & 1.06\\CON3-0 & 620.76 & 1.04 & 
625.61 & 0.80 & \bf{616.52} & 
0.69 & 1.47\\CON3-1 & 556.92 & 1.18 & 
562.13 & 1.03 & \bf{554.47} & 
0.44 & 1.38\\CON3-2 & 521.38 & 0.92 & 
521.38 & 0.92 & \bf{518.00} & 
0.65 & 0.65\\CON3-3 & 591.20 & 0.89 & 
591.82 & 0.85 & \bf{591.19} & 
0.00 & 0.11\\CON3-4 & 595.25 & 1.04 & 
597.56 & 0.88 & \bf{588.79} & 
1.10 & 1.49\\CON3-5 & 568.76 & 0.71 & 
568.76 & 0.85 & \bf{563.70} & 
0.90 & 0.90\\CON3-6 & 502.64 & 0.66 & 
507.87 & 0.84 & \bf{499.05} & 
0.72 & 1.77\\CON3-7 & 578.41 & 0.80 & 
582.21 & 0.87 & \bf{576.48} & 
0.33 & 0.99\\CON3-8 & \bf{523.05} & 1.00 & 
525.43 & 0.91 & 523.05 & 0.00
 & 0.46\\CON3-9 & 588.18 & 1.02 & 
588.38 & 0.92 & \bf{578.24} & 
1.72 & 1.75\\CON8-0 & 865.86 & 1.31 & 
865.86 & 0.93 & \bf{857.17} & 
1.01 & 1.01\\CON8-1 & 759.08 & 1.14 & 
759.78 & 0.89 & \bf{740.85} & 
2.46 & 2.55\\CON8-2 & 717.31 & 0.76 & 
720.74 & 0.98 & \bf{712.89} & 
0.62 & 1.10\\CON8-3 & 822.72 & 0.66 & 
822.72 & 0.83 & \bf{811.07} & 
1.44 & 1.44\\CON8-4 & 773.64 & 1.38 & 
784.49 & 1.22 & \bf{772.25} & 
0.18 & 1.59\\CON8-5 & 759.82 & 1.04 & 
759.82 & 0.85 & \bf{754.88} & 
0.65 & 0.65\\CON8-6 & 685.80 & 1.01 & 
692.45 & 0.84 & \bf{678.92} & 
1.01 & 1.99\\CON8-7 & 831.04 & 0.69 & 
831.04 & 0.75 & \bf{811.96} & 
2.35 & 2.35\\CON8-8 & 777.98 & 0.72 & 
784.40 & 0.81 & \bf{767.53} & 
1.36 & 2.20\\CON8-9 & 817.21 & 1.47 & 
826.71 & 1.00 & \bf{809.00} & 
1.01 & 2.19\\\bf{PROM.} & 
\bf{768.40} & \bf{0.91} & \bf{770.44} & \bf{0.88} & \bf{758.54} & \bf{1.16} & \bf{1.46}\\[1ex]\hline
\end{tabular}
\label{table:nonlin}
\end{table} \clearpage
\begin{table}[ht]
\caption{Resultados de la ejecución de la metaheurística IGA, utilizando instancias de SalhiNagy con la configuración -n 250.0 -p 50.0 -cprob 90 -mprob 70}
\centering
\small
\begin{tabular}{c c c c c c c c}
\hline\hline
Instancia & Costo mínimo & Tiempo(seg.) & Costo promedio & Tiempo promedio(seg.) & CME & \%G & \%GP \\ [0.5ex]
\hline
CMT1X & 481.52 & 0.95 & 
482.64 & 1.11 & \bf{470.48} & 
2.35 & 2.59\\CMT1Y & 487.95 & 0.84 & 
488.51 & 0.99 & \bf{470.48} & 
3.71 & 3.83\\CMT2X & 703.65 & 2.40 & 
719.01 & 2.27 & \bf{682.39} & 
3.12 & 5.37\\CMT2Y & 707.67 & 1.86 & 
714.99 & 2.11 & \bf{682.39} & 
3.70 & 4.78\\CMT3X & 732.41 & 4.67 & 
739.79 & 4.51 & \bf{719.06} & 
1.86 & 2.88\\CMT3Y & 733.98 & 4.60 & 
742.79 & 4.50 & \bf{719.06} & 
2.07 & 3.30\\CMT4X & 898.97 & 11.24 & 
905.85 & 11.21 & \bf{854.21} & 
5.24 & 6.04\\CMT4Y & 887.83 & 10.87 & 
905.21 & 11.41 & \bf{852.46} & 
4.15 & 6.19\\CMT5X & 1090.97 & 21.89 & 
1110.08 & 22.26 & \bf{1030.56} & 
5.86 & 7.72\\CMT5Y & 1110.79 & 21.12 & 
1123.21 & 21.83 & \bf{1031.69} & 
7.67 & 8.87\\CMT11X & 913.44 & 7.49 & 
922.57 & 7.18 & \bf{831.09} & 
9.91 & 11.01\\CMT11Y & 892.45 & 7.28 & 
898.44 & 7.44 & \bf{829.85} & 
7.54 & 8.27\\CMT12X & 674.21 & 4.37 & 
683.75 & 4.45 & \bf{658.83} & 
2.33 & 3.78\\CMT12Y & 673.67 & 4.36 & 
679.35 & 4.93 & \bf{660.47} & 
2.00 & 2.86\\\bf{PROM.} & 
\bf{784.97} & \bf{7.42} & \bf{794.01} & \bf{7.59} & \bf{749.50} & \bf{4.39} & \bf{5.53}\\[1ex]\hline
\end{tabular}
\label{table:nonlin}
\end{table} \clearpage
\begin{table}[ht]
\caption{Resultados de la ejecución de la metaheurística IGA, utilizando instancias de Dethloff con la configuración -n 250.0 -p 60.0 -cprob 40 -mprob 70}
\centering
\small
\begin{tabular}{c c c c c c c c}
\hline\hline
Instancia & Costo mínimo & Tiempo(seg.) & Costo promedio & Tiempo promedio(seg.) & CME & \%G & \%GP \\ [0.5ex]
\hline
SCA3-0 & 640.55 & 0.95 & 
641.12 & 1.14 & \bf{635.62} & 
0.78 & 0.87\\SCA3-1 & \bf{697.84} & 1.21 & 
698.50 & 1.13 & 697.84 & 0.00
 & 0.10\\SCA3-2 & 664.21 & 0.86 & 
664.21 & 0.78 & \bf{659.34} & 
0.74 & 0.74\\SCA3-3 & \bf{680.04} & 0.74 & 
680.46 & 0.87 & 680.04 & 0.00
 & 0.06\\SCA3-4 & \bf{690.50} & 1.06 & 
690.50 & 0.85 & 690.50 & 0.00
 & 0.00\\
SCA3-5 & 662.75 & 1.12 & 
664.59 & 0.94 & \bf{659.90} & 
0.43 & 0.71\\SCA3-6 & 652.94 & 1.06 & 
655.13 & 1.36 & \bf{651.09} & 
0.28 & 0.62\\SCA3-7 & 666.15 & 1.18 & 
668.49 & 1.19 & \bf{659.17} & 
1.06 & 1.41\\SCA3-8 & 721.45 & 0.99 & 
722.09 & 0.85 & \bf{719.47} & 
0.28 & 0.36\\SCA3-9 & \bf{681.00} & 1.28 & 
684.54 & 1.12 & 681.00 & 0.00
 & 0.52\\SCA8-0 & 987.90 & 0.81 & 
1003.13 & 0.88 & \bf{961.50} & 
2.75 & 4.33\\SCA8-1 & 1059.16 & 0.90 & 
1060.17 & 0.99 & \bf{1049.65} & 
0.91 & 1.00\\SCA8-2 & 1054.47 & 1.04 & 
1054.58 & 1.34 & \bf{1039.64} & 
1.43 & 1.44\\SCA8-3 & 1018.37 & 0.90 & 
1018.37 & 0.89 & \bf{983.34} & 
3.56 & 3.56\\SCA8-4 & 1075.27 & 0.94 & 
1085.69 & 0.84 & \bf{1065.49} & 
0.92 & 1.90\\SCA8-5 & 1029.95 & 0.92 & 
1029.95 & 0.96 & \bf{1027.08} & 
0.28 & 0.28\\SCA8-6 & 981.41 & 1.30 & 
985.53 & 1.35 & \bf{971.82} & 
0.99 & 1.41\\SCA8-7 & 1078.63 & 1.36 & 
1078.63 & 1.05 & \bf{1051.28} & 
2.60 & 2.60\\SCA8-8 & 1084.41 & 0.84 & 
1092.79 & 0.90 & \bf{1071.18} & 
1.24 & 2.02\\SCA8-9 & 1068.10 & 1.39 & 
1071.84 & 1.07 & \bf{1060.50} & 
0.72 & 1.07\\CON3-0 & 620.49 & 1.29 & 
623.99 & 1.11 & \bf{616.52} & 
0.64 & 1.21\\CON3-1 & 556.92 & 1.16 & 
557.95 & 1.12 & \bf{554.47} & 
0.44 & 0.63\\CON3-2 & 521.38 & 1.04 & 
521.75 & 1.09 & \bf{518.00} & 
0.65 & 0.72\\CON3-3 & 598.45 & 1.16 & 
601.25 & 1.11 & \bf{591.19} & 
1.23 & 1.70\\CON3-4 & 592.58 & 1.06 & 
600.80 & 0.95 & \bf{588.79} & 
0.64 & 2.04\\CON3-5 & \bf{563.70} & 1.12 & 
566.79 & 1.03 & 563.70 & 0.00
 & 0.55\\CON3-6 & 503.97 & 0.79 & 
504.78 & 0.99 & \bf{499.05} & 
0.99 & 1.15\\CON3-7 & 583.65 & 1.09 & 
585.63 & 0.88 & \bf{576.48} & 
1.24 & 1.59\\CON3-8 & 523.14 & 1.01 & 
527.91 & 1.20 & \bf{523.05} & 
0.02 & 0.93\\CON3-9 & 587.56 & 1.14 & 
588.49 & 1.08 & \bf{578.24} & 
1.61 & 1.77\\CON8-0 & 879.99 & 0.95 & 
884.84 & 0.98 & \bf{857.17} & 
2.66 & 3.23\\CON8-1 & 765.05 & 0.90 & 
768.40 & 1.09 & \bf{740.85} & 
3.27 & 3.72\\CON8-2 & 720.40 & 0.97 & 
725.29 & 1.18 & \bf{712.89} & 
1.05 & 1.74\\CON8-3 & 821.26 & 1.47 & 
821.26 & 1.34 & \bf{811.07} & 
1.26 & 1.26\\CON8-4 & 787.26 & 0.75 & 
795.98 & 1.33 & \bf{772.25} & 
1.94 & 3.07\\CON8-5 & 754.95 & 0.88 & 
756.89 & 1.33 & \bf{754.88} & 
0.01 & 0.27\\CON8-6 & 698.34 & 1.21 & 
698.34 & 1.24 & \bf{678.92} & 
2.86 & 2.86\\CON8-7 & 816.00 & 0.82 & 
816.50 & 1.14 & \bf{811.96} & 
0.50 & 0.56\\CON8-8 & 780.80 & 0.88 & 
780.80 & 1.30 & \bf{767.53} & 
1.73 & 1.73\\CON8-9 & 840.38 & 1.46 & 
841.85 & 1.32 & \bf{809.00} & 
3.88 & 4.06\\\bf{PROM.} & 
\bf{767.78} & \bf{1.05} & \bf{770.50} & \bf{1.08} & \bf{758.54} & \bf{1.14} & \bf{1.49}\\[1ex]\hline
\end{tabular}
\label{table:nonlin}
\end{table} \clearpage
\begin{table}[ht]
\caption{Resultados de la ejecución de la metaheurística IGA, utilizando instancias de SalhiNagy con la configuración -n 250.0 -p 60.0 -cprob 90 -mprob 70}
\centering
\small
\begin{tabular}{c c c c c c c c}
\hline\hline
Instancia & Costo mínimo & Tiempo(seg.) & Costo promedio & Tiempo promedio(seg.) & CME & \%G & \%GP \\ [0.5ex]
\hline
CMT1X & 480.02 & 1.25 & 
482.65 & 1.32 & \bf{470.48} & 
2.03 & 2.59\\CMT1Y & 484.01 & 1.21 & 
486.70 & 1.32 & \bf{470.48} & 
2.88 & 3.45\\CMT2X & 709.98 & 2.35 & 
714.50 & 2.61 & \bf{682.39} & 
4.04 & 4.71\\CMT2Y & 714.64 & 2.20 & 
717.68 & 2.39 & \bf{682.39} & 
4.73 & 5.17\\CMT3X & 737.94 & 4.87 & 
738.75 & 5.13 & \bf{719.06} & 
2.63 & 2.74\\CMT3Y & 734.38 & 5.50 & 
743.46 & 5.31 & \bf{719.06} & 
2.13 & 3.39\\CMT4X & 905.20 & 19.34 & 
912.90 & 14.61 & \bf{854.21} & 
5.97 & 6.87\\CMT4Y & 905.00 & 12.81 & 
917.82 & 13.12 & \bf{852.46} & 
6.16 & 7.67\\CMT5X & 1096.18 & 25.32 & 
1106.69 & 25.39 & \bf{1030.56} & 
6.37 & 7.39\\CMT5Y & 1115.66 & 26.01 & 
1126.25 & 25.88 & \bf{1031.69} & 
8.14 & 9.17\\CMT11X & 923.73 & 10.57 & 
925.20 & 9.07 & \bf{831.09} & 
11.15 & 11.32\\CMT11Y & 897.59 & 9.89 & 
913.02 & 9.06 & \bf{829.85} & 
8.16 & 10.02\\CMT12X & 676.16 & 5.65 & 
677.66 & 5.49 & \bf{658.83} & 
2.63 & 2.86\\CMT12Y & 682.29 & 5.22 & 
683.71 & 5.25 & \bf{660.47} & 
3.30 & 3.52\\\bf{PROM.} & 
\bf{790.20} & \bf{9.44} & \bf{796.21} & \bf{9.00} & \bf{749.50} & \bf{5.02} & \bf{5.78}\\[1ex]\hline
\end{tabular}
\label{table:nonlin}
\end{table} \clearpage
\begin{table}[ht]
\caption{Resultados de la ejecución de la metaheurística IGA, utilizando instancias de Dethloff con la configuración -n 250.0 -p 70.0 -cprob 40 -mprob 70}
\centering
\small
\begin{tabular}{c c c c c c c c}
\hline\hline
Instancia & Costo mínimo & Tiempo(seg.) & Costo promedio & Tiempo promedio(seg.) & CME & \%G & \%GP \\ [0.5ex]
\hline
SCA3-0 & 636.34 & 0.94 & 
639.50 & 1.18 & \bf{635.62} & 
0.11 & 0.61\\SCA3-1 & 700.50 & 1.04 & 
700.76 & 1.09 & \bf{697.84} & 
0.38 & 0.42\\SCA3-2 & 664.18 & 1.11 & 
666.67 & 1.25 & \bf{659.34} & 
0.73 & 1.11\\SCA3-3 & 681.16 & 1.25 & 
683.43 & 1.11 & \bf{680.04} & 
0.16 & 0.50\\SCA3-4 & \bf{690.50} & 1.06 & 
690.50 & 1.10 & 690.50 & 0.00
 & 0.00\\
SCA3-5 & 666.67 & 1.04 & 
669.58 & 1.20 & \bf{659.90} & 
1.03 & 1.47\\SCA3-6 & 652.94 & 1.13 & 
653.16 & 1.00 & \bf{651.09} & 
0.28 & 0.32\\SCA3-7 & 669.89 & 1.28 & 
670.78 & 1.13 & \bf{659.17} & 
1.63 & 1.76\\SCA3-8 & 723.99 & 0.85 & 
725.03 & 1.06 & \bf{719.47} & 
0.63 & 0.77\\SCA3-9 & \bf{681.00} & 1.02 & 
684.11 & 1.08 & 681.00 & 0.00
 & 0.46\\SCA8-0 & 976.24 & 2.06 & 
976.24 & 1.47 & \bf{961.50} & 
1.53 & 1.53\\SCA8-1 & 1052.36 & 0.96 & 
1052.36 & 1.12 & \bf{1049.65} & 
0.26 & 0.26\\SCA8-2 & 1052.56 & 1.07 & 
1052.56 & 1.04 & \bf{1039.64} & 
1.24 & 1.24\\SCA8-3 & 1019.06 & 1.52 & 
1024.39 & 1.18 & \bf{983.34} & 
3.63 & 4.17\\SCA8-4 & 1092.33 & 0.80 & 
1092.33 & 1.35 & \bf{1065.49} & 
2.52 & 2.52\\SCA8-5 & 1029.95 & 1.42 & 
1036.13 & 1.16 & \bf{1027.08} & 
0.28 & 0.88\\SCA8-6 & 978.03 & 1.15 & 
980.53 & 1.43 & \bf{971.82} & 
0.64 & 0.90\\SCA8-7 & 1070.92 & 1.04 & 
1070.92 & 1.32 & \bf{1051.28} & 
1.87 & 1.87\\SCA8-8 & 1096.04 & 1.02 & 
1096.24 & 1.33 & \bf{1071.18} & 
2.32 & 2.34\\SCA8-9 & 1073.62 & 1.26 & 
1073.62 & 1.18 & \bf{1060.50} & 
1.24 & 1.24\\CON3-0 & 619.09 & 1.13 & 
624.26 & 1.27 & \bf{616.52} & 
0.42 & 1.26\\CON3-1 & 560.75 & 1.22 & 
560.75 & 1.19 & \bf{554.47} & 
1.13 & 1.13\\CON3-2 & 521.38 & 1.40 & 
521.38 & 1.51 & \bf{518.00} & 
0.65 & 0.65\\CON3-3 & 591.20 & 1.31 & 
592.12 & 1.12 & \bf{591.19} & 
0.00 & 0.16\\CON3-4 & 593.78 & 0.93 & 
594.55 & 1.15 & \bf{588.79} & 
0.85 & 0.98\\CON3-5 & \bf{563.70} & 1.34 & 
563.70 & 1.39 & 563.70 & 0.00
 & 0.00\\
CON3-6 & 504.44 & 1.69 & 
505.26 & 1.60 & \bf{499.05} & 
1.08 & 1.24\\CON3-7 & 580.55 & 0.80 & 
584.64 & 1.10 & \bf{576.48} & 
0.71 & 1.42\\CON3-8 & 524.38 & 1.35 & 
526.85 & 1.23 & \bf{523.05} & 
0.25 & 0.73\\CON3-9 & 588.40 & 1.06 & 
589.95 & 1.07 & \bf{578.24} & 
1.76 & 2.02\\CON8-0 & 876.61 & 1.30 & 
876.70 & 1.33 & \bf{857.17} & 
2.27 & 2.28\\CON8-1 & 754.93 & 1.02 & 
758.91 & 1.25 & \bf{740.85} & 
1.90 & 2.44\\CON8-2 & 725.13 & 1.54 & 
725.13 & 1.35 & \bf{712.89} & 
1.72 & 1.72\\CON8-3 & 829.87 & 1.58 & 
835.80 & 1.35 & \bf{811.07} & 
2.32 & 3.05\\CON8-4 & 780.51 & 1.36 & 
780.51 & 1.12 & \bf{772.25} & 
1.07 & 1.07\\CON8-5 & 758.12 & 0.85 & 
758.12 & 1.17 & \bf{754.88} & 
0.43 & 0.43\\CON8-6 & 702.74 & 1.56 & 
703.10 & 1.34 & \bf{678.92} & 
3.51 & 3.56\\CON8-7 & 825.69 & 0.82 & 
832.53 & 1.38 & \bf{811.96} & 
1.69 & 2.53\\CON8-8 & 781.69 & 1.04 & 
791.95 & 1.21 & \bf{767.53} & 
1.84 & 3.18\\CON8-9 & 824.56 & 0.79 & 
828.68 & 1.29 & \bf{809.00} & 
1.92 & 2.43\\\bf{PROM.} & 
\bf{767.89} & \bf{1.18} & \bf{769.84} & \bf{1.23} & \bf{758.54} & \bf{1.15} & \bf{1.42}\\[1ex]\hline
\end{tabular}
\label{table:nonlin}
\end{table} \clearpage
\begin{table}[ht]
\caption{Resultados de la ejecución de la metaheurística IGA, utilizando instancias de SalhiNagy con la configuración -n 250.0 -p 70.0 -cprob 90 -mprob 70}
\centering
\small
\begin{tabular}{c c c c c c c c}
\hline\hline
Instancia & Costo mínimo & Tiempo(seg.) & Costo promedio & Tiempo promedio(seg.) & CME & \%G & \%GP \\ [0.5ex]
\hline
CMT1X & 478.97 & 1.26 & 
478.97 & 1.43 & \bf{470.48} & 
1.80 & 1.80\\CMT1Y & 482.05 & 1.40 & 
483.77 & 1.50 & \bf{470.48} & 
2.46 & 2.82\\CMT2X & 705.57 & 3.26 & 
709.67 & 2.88 & \bf{682.39} & 
3.40 & 4.00\\CMT2Y & 707.94 & 2.99 & 
713.05 & 2.90 & \bf{682.39} & 
3.74 & 4.49\\CMT3X & 742.44 & 6.10 & 
744.48 & 6.28 & \bf{719.06} & 
3.25 & 3.53\\CMT3Y & 742.51 & 6.98 & 
746.54 & 6.38 & \bf{719.06} & 
3.26 & 3.82\\CMT4X & 892.68 & 15.60 & 
908.01 & 14.25 & \bf{854.21} & 
4.50 & 6.30\\CMT4Y & 904.05 & 15.28 & 
910.07 & 15.30 & \bf{852.46} & 
6.05 & 6.76\\CMT5X & 1093.12 & 29.47 & 
1111.93 & 30.02 & \bf{1030.56} & 
6.07 & 7.90\\CMT5Y & 1099.22 & 29.55 & 
1112.35 & 29.45 & \bf{1031.69} & 
6.55 & 7.82\\CMT11X & 914.34 & 9.77 & 
920.02 & 9.96 & \bf{831.09} & 
10.02 & 10.70\\CMT11Y & 884.12 & 12.24 & 
886.65 & 10.97 & \bf{829.85} & 
6.54 & 6.84\\CMT12X & 674.11 & 7.15 & 
677.38 & 6.59 & \bf{658.83} & 
2.32 & 2.82\\CMT12Y & 673.95 & 6.15 & 
674.53 & 5.89 & \bf{660.47} & 
2.04 & 2.13\\\bf{PROM.} & 
\bf{785.36} & \bf{10.51} & \bf{791.24} & \bf{10.27} & \bf{749.50} & \bf{4.43} & \bf{5.12}\\[1ex]\hline
\end{tabular}
\label{table:nonlin}
\end{table} \clearpage
\begin{table}[ht]
\caption{Resultados de la ejecución de la metaheurística IGA, utilizando instancias de Dethloff con la configuración -n 250.0 -p 80.0 -cprob 40 -mprob 70}
\centering
\small
\begin{tabular}{c c c c c c c c}
\hline\hline
Instancia & Costo mínimo & Tiempo(seg.) & Costo promedio & Tiempo promedio(seg.) & CME & \%G & \%GP \\ [0.5ex]
\hline
SCA3-0 & 636.06 & 1.94 & 
639.71 & 1.63 & \bf{635.62} & 
0.07 & 0.64\\SCA3-1 & 701.53 & 1.10 & 
701.53 & 1.34 & \bf{697.84} & 
0.53 & 0.53\\SCA3-2 & 661.13 & 1.13 & 
662.65 & 1.27 & \bf{659.34} & 
0.27 & 0.50\\SCA3-3 & \bf{680.04} & 0.91 & 
681.22 & 1.44 & 680.04 & 0.00
 & 0.17\\SCA3-4 & \bf{690.50} & 1.38 & 
690.50 & 1.23 & 690.50 & 0.00
 & 0.00\\
SCA3-5 & 666.67 & 1.22 & 
670.32 & 1.18 & \bf{659.90} & 
1.03 & 1.58\\SCA3-6 & 653.69 & 1.66 & 
655.51 & 1.60 & \bf{651.09} & 
0.40 & 0.68\\SCA3-7 & 666.15 & 1.60 & 
666.15 & 1.39 & \bf{659.17} & 
1.06 & 1.06\\SCA3-8 & 719.77 & 1.26 & 
724.38 & 1.63 & \bf{719.47} & 
0.04 & 0.68\\SCA3-9 & \bf{681.00} & 1.48 & 
681.00 & 1.61 & 681.00 & 0.00
 & 0.00\\
SCA8-0 & 977.93 & 1.73 & 
987.85 & 1.38 & \bf{961.50} & 
1.71 & 2.74\\SCA8-1 & 1082.93 & 2.09 & 
1082.93 & 1.80 & \bf{1049.65} & 
3.17 & 3.17\\SCA8-2 & 1050.37 & 1.57 & 
1053.44 & 1.23 & \bf{1039.64} & 
1.03 & 1.33\\SCA8-3 & 1017.95 & 1.11 & 
1018.63 & 1.43 & \bf{983.34} & 
3.52 & 3.59\\SCA8-4 & 1071.16 & 0.92 & 
1071.16 & 1.23 & \bf{1065.49} & 
0.53 & 0.53\\SCA8-5 & 1036.88 & 1.25 & 
1036.88 & 1.33 & \bf{1027.08} & 
0.95 & 0.95\\SCA8-6 & 977.03 & 1.23 & 
984.50 & 1.48 & \bf{971.82} & 
0.54 & 1.30\\SCA8-7 & 1063.60 & 1.78 & 
1063.60 & 1.31 & \bf{1051.28} & 
1.17 & 1.17\\SCA8-8 & 1084.74 & 1.08 & 
1084.74 & 1.30 & \bf{1071.18} & 
1.27 & 1.27\\SCA8-9 & 1068.10 & 1.32 & 
1072.88 & 2.17 & \bf{1060.50} & 
0.72 & 1.17\\CON3-0 & 623.97 & 1.48 & 
629.07 & 1.48 & \bf{616.52} & 
1.21 & 2.03\\CON3-1 & 556.92 & 1.48 & 
560.01 & 1.39 & \bf{554.47} & 
0.44 & 1.00\\CON3-2 & 521.38 & 1.62 & 
521.38 & 1.60 & \bf{518.00} & 
0.65 & 0.65\\CON3-3 & 601.26 & 1.10 & 
601.34 & 1.36 & \bf{591.19} & 
1.70 & 1.72\\CON3-4 & 592.58 & 1.21 & 
593.48 & 1.47 & \bf{588.79} & 
0.64 & 0.80\\CON3-5 & 564.88 & 1.52 & 
564.88 & 1.66 & \bf{563.70} & 
0.21 & 0.21\\CON3-6 & 502.16 & 1.57 & 
504.28 & 1.31 & \bf{499.05} & 
0.62 & 1.05\\CON3-7 & 581.46 & 1.38 & 
583.83 & 1.41 & \bf{576.48} & 
0.86 & 1.27\\CON3-8 & \bf{523.05} & 1.53 & 
525.59 & 1.50 & 523.05 & 0.00
 & 0.48\\CON3-9 & 588.11 & 1.21 & 
588.11 & 1.25 & \bf{578.24} & 
1.71 & 1.71\\CON8-0 & 869.40 & 1.60 & 
869.40 & 1.34 & \bf{857.17} & 
1.43 & 1.43\\CON8-1 & 756.14 & 1.86 & 
756.14 & 1.74 & \bf{740.85} & 
2.06 & 2.06\\CON8-2 & 716.07 & 1.33 & 
717.43 & 1.27 & \bf{712.89} & 
0.45 & 0.64\\CON8-3 & 831.99 & 1.36 & 
833.17 & 1.41 & \bf{811.07} & 
2.58 & 2.73\\CON8-4 & 782.67 & 1.32 & 
787.75 & 1.21 & \bf{772.25} & 
1.35 & 2.01\\CON8-5 & 763.13 & 1.59 & 
763.13 & 1.86 & \bf{754.88} & 
1.09 & 1.09\\CON8-6 & 696.30 & 1.25 & 
700.28 & 1.35 & \bf{678.92} & 
2.56 & 3.15\\CON8-7 & 815.72 & 1.04 & 
819.96 & 1.21 & \bf{811.96} & 
0.46 & 0.98\\CON8-8 & 788.75 & 2.16 & 
791.38 & 1.81 & \bf{767.53} & 
2.76 & 3.11\\CON8-9 & 810.18 & 1.54 & 
813.81 & 1.40 & \bf{809.00} & 
0.15 & 0.59\\\bf{PROM.} & 
\bf{766.83} & \bf{1.42} & \bf{768.85} & \bf{1.45} & \bf{758.54} & \bf{1.02} & \bf{1.29}\\[1ex]\hline
\end{tabular}
\label{table:nonlin}
\end{table} \clearpage
\begin{table}[ht]
\caption{Resultados de la ejecución de la metaheurística PSO, utilizando instancias de SalhiNagy con la configuración -n 30.0 -L 110.0 -cp 1 -cg 0 -cl 1 -cn 2 -w1 0.9 -wt 0.1 -K 5}
\centering
\small
\begin{tabular}{c c c c c c c c}
\hline\hline
Instancia & Costo mínimo & Tiempo(seg.) & Costo promedio & Tiempo promedio(seg.) & CME & \%G & \%GP \\ [0.5ex]
\hline
CMT1X & 470.67 & 24.73 & 
472.75 & 22.89 & \bf{470.48} & 
0.04 & 0.48\\CMT1Y & 470.67 & 24.84 & 
472.32 & 23.99 & \bf{470.48} & 
0.04 & 0.39\\CMT2X & 707.72 & 31.82 & 
721.26 & 31.87 & \bf{682.39} & 
3.71 & 5.70\\CMT2Y & 693.66 & 34.07 & 
734.38 & 33.48 & \bf{682.39} & 
1.65 & 7.62\\CMT3X & 724.22 & 245.23 & 
749.45 & 254.35 & \bf{719.06} & 
0.72 & 4.23\\CMT3Y & 725.31 & 256.12 & 
743.96 & 242.43 & \bf{719.06} & 
0.87 & 3.46\\CMT4X & 880.59 & 387.67 & 
906.76 & 349.19 & \bf{854.21} & 
3.09 & 6.15\\CMT4Y & 883.87 & 478.52 & 
884.57 & 457.52 & \bf{852.46} & 
3.68 & 3.77\\CMT5X & 1130.41 & 370.68 & 
1162.70 & 393.31 & \bf{1030.56} & 
9.69 & 12.82\\CMT5Y & 1119.71 & 451.66 & 
1145.63 & 423.10 & \bf{1031.69} & 
8.53 & 11.04\\CMT11X & 887.48 & 91.03 & 
901.37 & 95.39 & \bf{831.09} & 
6.79 & 8.46\\CMT11Y & 886.10 & 56.11 & 
891.66 & 81.72 & \bf{829.85} & 
6.78 & 7.45\\CMT12X & 697.89 & 19.31 & 
725.79 & 22.66 & \bf{658.83} & 
5.93 & 10.16\\CMT12Y & 743.75 & 18.62 & 
771.57 & 21.83 & \bf{660.47} & 
12.61 & 16.82\\\bf{PROM.} & 
\bf{787.29} & \bf{177.89} & \bf{806.01} & \bf{175.27} & \bf{749.50} & \bf{4.58} & \bf{7.04}\\[1ex]\hline
\end{tabular}
\label{table:nonlin}
\end{table} \clearpage
\begin{table}[ht]
\caption{Resultados de la ejecución de la metaheurística IGA, utilizando instancias de SalhiNagy con la configuración -n 250.0 -p 80.0 -cprob 90 -mprob 70}
\centering
\small
\begin{tabular}{c c c c c c c c}
\hline\hline
Instancia & Costo mínimo & Tiempo(seg.) & Costo promedio & Tiempo promedio(seg.) & CME & \%G & \%GP \\ [0.5ex]
\hline
CMT1X & 478.84 & 1.81 & 
480.50 & 1.64 & \bf{470.48} & 
1.78 & 2.13\\CMT1Y & 479.70 & 1.63 & 
481.96 & 1.57 & \bf{470.48} & 
1.96 & 2.44\\CMT2X & 713.07 & 3.82 & 
714.33 & 3.42 & \bf{682.39} & 
4.50 & 4.68\\CMT2Y & 698.16 & 3.65 & 
708.21 & 3.47 & \bf{682.39} & 
2.31 & 3.78\\CMT3X & 735.96 & 7.07 & 
738.33 & 6.88 & \bf{719.06} & 
2.35 & 2.68\\CMT3Y & 744.02 & 7.39 & 
748.58 & 6.85 & \bf{719.06} & 
3.47 & 4.10\\CMT4X & 890.21 & 17.34 & 
908.92 & 17.22 & \bf{854.21} & 
4.21 & 6.41\\CMT4Y & 905.34 & 18.19 & 
913.14 & 17.36 & \bf{852.46} & 
6.20 & 7.12\\CMT5X & 1107.95 & 38.59 & 
1115.26 & 35.02 & \bf{1030.56} & 
7.51 & 8.22\\CMT5Y & 1101.44 & 34.54 & 
1106.97 & 34.56 & \bf{1031.69} & 
6.76 & 7.30\\CMT11X & 883.86 & 10.65 & 
903.26 & 11.24 & \bf{831.09} & 
6.35 & 8.68\\CMT11Y & 890.19 & 12.37 & 
897.48 & 11.47 & \bf{829.85} & 
7.27 & 8.15\\CMT12X & 674.95 & 6.61 & 
675.37 & 6.58 & \bf{658.83} & 
2.45 & 2.51\\CMT12Y & 675.28 & 6.52 & 
675.50 & 6.75 & \bf{660.47} & 
2.24 & 2.28\\\bf{PROM.} & 
\bf{784.21} & \bf{12.16} & \bf{790.56} & \bf{11.72} & \bf{749.50} & \bf{4.24} & \bf{5.03}\\[1ex]\hline
\end{tabular}
\label{table:nonlin}
\end{table} \clearpage
\begin{table}[ht]
\caption{Resultados de la ejecución de la metaheurística IGA, utilizando instancias de Dethloff con la configuración -n 250.0 -p 90.0 -cprob 40 -mprob 70}
\centering
\small
\begin{tabular}{c c c c c c c c}
\hline\hline
Instancia & Costo mínimo & Tiempo(seg.) & Costo promedio & Tiempo promedio(seg.) & CME & \%G & \%GP \\ [0.5ex]
\hline
SCA3-0 & 640.55 & 2.24 & 
640.82 & 1.89 & \bf{635.62} & 
0.78 & 0.82\\SCA3-1 & \bf{697.84} & 1.84 & 
699.43 & 1.57 & 697.84 & 0.00
 & 0.23\\SCA3-2 & 661.13 & 1.35 & 
663.42 & 1.55 & \bf{659.34} & 
0.27 & 0.62\\SCA3-3 & 680.60 & 2.23 & 
681.02 & 1.66 & \bf{680.04} & 
0.08 & 0.14\\SCA3-4 & \bf{690.50} & 1.28 & 
690.50 & 1.51 & 690.50 & 0.00
 & 0.00\\
SCA3-5 & 666.67 & 1.31 & 
666.67 & 1.39 & \bf{659.90} & 
1.03 & 1.03\\SCA3-6 & 652.94 & 1.14 & 
655.42 & 1.42 & \bf{651.09} & 
0.28 & 0.67\\SCA3-7 & 666.15 & 2.37 & 
666.15 & 1.48 & \bf{659.17} & 
1.06 & 1.06\\SCA3-8 & 724.29 & 1.25 & 
724.29 & 1.61 & \bf{719.47} & 
0.67 & 0.67\\SCA3-9 & \bf{681.00} & 1.67 & 
681.00 & 1.51 & 681.00 & 0.00
 & 0.00\\
SCA8-0 & 975.50 & 1.61 & 
990.04 & 1.68 & \bf{961.50} & 
1.46 & 2.97\\SCA8-1 & 1062.88 & 1.06 & 
1069.57 & 1.49 & \bf{1049.65} & 
1.26 & 1.90\\SCA8-2 & 1050.37 & 2.19 & 
1050.37 & 1.85 & \bf{1039.64} & 
1.03 & 1.03\\SCA8-3 & 997.31 & 1.78 & 
1000.48 & 1.60 & \bf{983.34} & 
1.42 & 1.74\\SCA8-4 & 1079.70 & 1.34 & 
1079.70 & 1.22 & \bf{1065.49} & 
1.33 & 1.33\\SCA8-5 & 1061.29 & 1.02 & 
1061.29 & 1.30 & \bf{1027.08} & 
3.33 & 3.33\\SCA8-6 & 983.01 & 1.87 & 
984.49 & 1.67 & \bf{971.82} & 
1.15 & 1.30\\SCA8-7 & 1070.92 & 1.31 & 
1070.92 & 1.69 & \bf{1051.28} & 
1.87 & 1.87\\SCA8-8 & \bf{1071.18} & 0.98 & 
1077.80 & 1.57 & 1071.18 & 0.00
 & 0.62\\SCA8-9 & 1069.83 & 1.76 & 
1069.83 & 1.66 & \bf{1060.50} & 
0.88 & 0.88\\CON3-0 & 628.68 & 1.95 & 
630.51 & 1.85 & \bf{616.52} & 
1.97 & 2.27\\CON3-1 & 559.72 & 1.72 & 
561.34 & 1.60 & \bf{554.47} & 
0.95 & 1.24\\CON3-2 & 520.38 & 2.44 & 
520.88 & 2.02 & \bf{518.00} & 
0.46 & 0.56\\CON3-3 & 591.20 & 1.62 & 
597.65 & 1.77 & \bf{591.19} & 
0.00 & 1.09\\CON3-4 & 593.78 & 1.92 & 
596.07 & 1.69 & \bf{588.79} & 
0.85 & 1.24\\CON3-5 & \bf{563.70} & 1.50 & 
563.70 & 1.69 & 563.70 & 0.00
 & 0.00\\
CON3-6 & 503.21 & 1.54 & 
503.78 & 1.31 & \bf{499.05} & 
0.83 & 0.95\\CON3-7 & \bf{576.48} & 1.99 & 
576.48 & 1.92 & 576.48 & 0.00
 & 0.00\\
CON3-8 & 525.47 & 1.35 & 
530.55 & 1.50 & \bf{523.05} & 
0.46 & 1.43\\CON3-9 & 582.79 & 1.89 & 
584.21 & 1.69 & \bf{578.24} & 
0.79 & 1.03\\CON8-0 & 869.21 & 1.36 & 
869.48 & 1.52 & \bf{857.17} & 
1.40 & 1.44\\CON8-1 & 756.81 & 1.12 & 
758.77 & 1.66 & \bf{740.85} & 
2.15 & 2.42\\CON8-2 & 719.88 & 2.39 & 
719.88 & 1.80 & \bf{712.89} & 
0.98 & 0.98\\CON8-3 & 828.49 & 1.40 & 
828.49 & 1.69 & \bf{811.07} & 
2.15 & 2.15\\CON8-4 & 781.48 & 1.11 & 
784.32 & 1.20 & \bf{772.25} & 
1.20 & 1.56\\CON8-5 & 760.86 & 1.27 & 
760.86 & 1.69 & \bf{754.88} & 
0.79 & 0.79\\CON8-6 & 690.16 & 2.50 & 
690.16 & 2.08 & \bf{678.92} & 
1.66 & 1.66\\CON8-7 & 814.50 & 1.08 & 
815.82 & 1.54 & \bf{811.96} & 
0.31 & 0.48\\CON8-8 & 790.64 & 1.74 & 
793.69 & 1.60 & \bf{767.53} & 
3.01 & 3.41\\CON8-9 & 829.88 & 1.48 & 
831.97 & 1.73 & \bf{809.00} & 
2.58 & 2.84\\\bf{PROM.} & 
\bf{766.77} & \bf{1.62} & \bf{768.55} & \bf{1.62} & \bf{758.54} & \bf{1.01} & \bf{1.24}\\[1ex]\hline
\end{tabular}
\label{table:nonlin}
\end{table} \clearpage
\begin{table}[ht]
\caption{Resultados de la ejecución de la metaheurística SCA, utilizando instancias de SalhiNagy con la configuración -n 200.0 -b 10 -y .2}
\centering
\small
\begin{tabular}{c c c c c c c c}
\hline\hline
Instancia & Costo mínimo & Tiempo(seg.) & Costo promedio & Tiempo promedio(seg.) & CME & \%G & \%GP \\ [0.5ex]
\hline
CMT1X & 471.25 & 2.65 & 
471.25 & 2.99 & \bf{470.48} & 
0.16 & 0.16\\CMT1Y & 472.37 & 2.42 & 
472.88 & 2.83 & \bf{470.48} & 
0.40 & 0.51\\CMT2X & 705.52 & 21.35 & 
708.62 & 16.13 & \bf{682.39} & 
3.39 & 3.84\\CMT2Y & 704.30 & 23.75 & 
707.40 & 16.84 & \bf{682.39} & 
3.21 & 3.67\\CMT3X & 728.79 & 41.82 & 
733.53 & 39.67 & \bf{719.06} & 
1.35 & 2.01\\CMT3Y & 730.67 & 35.06 & 
734.38 & 31.76 & \bf{719.06} & 
1.61 & 2.13\\CMT4X & 901.92 & 182.11 & 
909.67 & 134.63 & \bf{854.21} & 
5.59 & 6.49\\CMT4Y & 100000 & 0 & 
873.80 & 210.91 & \bf{852.46} & 
11630.76 & 2.50\\CMT5X & 100000 & 0 & 
nan & nan & \bf{1030.56} & 
9603.46 & \bf{nan}\\CMT5Y & 100000 & 0 & 
nan & nan & \bf{1031.69} & 
9592.83 & \bf{nan}\\CMT11X & 883.85 & 75.22 & 
901.52 & 45.93 & \bf{831.09} & 
6.35 & 8.47\\CMT11Y & 897.29 & 38.33 & 
902.22 & 60.32 & \bf{829.85} & 
8.13 & 8.72\\CMT12X & 674.74 & 44.55 & 
679.65 & 56.73 & \bf{658.83} & 
2.41 & 3.16\\CMT12Y & 675.00 & 72.82 & 
682.78 & 65.24 & \bf{660.47} & 
2.20 & 3.38\\\bf{PROM.} & 
\bf{21988.98} & \bf{38.58} & \bf{nan} & \bf{nan} & \bf{749.50} & \bf{2204.42} & \bf{nan}\\[1ex]\hline
\end{tabular}
\label{table:nonlin}
\end{table} \clearpage
\begin{table}[ht]
\caption{Resultados de la ejecución de la metaheurística IGA, utilizando instancias de SalhiNagy con la configuración -n 250.0 -p 90.0 -cprob 90 -mprob 70}
\centering
\small
\begin{tabular}{c c c c c c c c}
\hline\hline
Instancia & Costo mínimo & Tiempo(seg.) & Costo promedio & Tiempo promedio(seg.) & CME & \%G & \%GP \\ [0.5ex]
\hline
CMT1X & 477.72 & 1.79 & 
480.64 & 1.99 & \bf{470.48} & 
1.54 & 2.16\\CMT1Y & 477.06 & 1.38 & 
482.55 & 1.81 & \bf{470.48} & 
1.40 & 2.57\\CMT2X & 704.90 & 3.88 & 
708.20 & 3.63 & \bf{682.39} & 
3.30 & 3.78\\CMT2Y & 699.81 & 4.19 & 
708.91 & 3.92 & \bf{682.39} & 
2.55 & 3.89\\CMT3X & 731.43 & 7.78 & 
739.74 & 7.85 & \bf{719.06} & 
1.72 & 2.88\\CMT3Y & 736.84 & 7.11 & 
746.31 & 7.33 & \bf{719.06} & 
2.47 & 3.79\\CMT4X & 898.24 & 48.90 & 
901.81 & 29.55 & \bf{854.21} & 
5.15 & 5.57\\CMT4Y & 904.06 & 19.88 & 
910.62 & 19.23 & \bf{852.46} & 
6.05 & 6.82\\CMT5X & 1094.13 & 38.27 & 
1108.00 & 38.15 & \bf{1030.56} & 
6.17 & 7.51\\CMT5Y & 1120.02 & 38.82 & 
1123.95 & 38.79 & \bf{1031.69} & 
8.56 & 8.94\\CMT11X & 884.94 & 13.14 & 
904.61 & 12.77 & \bf{831.09} & 
6.48 & 8.85\\CMT11Y & 855.55 & 13.56 & 
891.03 & 13.43 & \bf{829.85} & 
3.10 & 7.37\\CMT12X & 674.28 & 8.12 & 
679.20 & 7.83 & \bf{658.83} & 
2.35 & 3.09\\CMT12Y & 674.25 & 8.09 & 
675.58 & 7.66 & \bf{660.47} & 
2.09 & 2.29\\\bf{PROM.} & 
\bf{780.94} & \bf{15.35} & \bf{790.08} & \bf{13.85} & \bf{749.50} & \bf{3.78} & \bf{4.96}\\[1ex]\hline
\end{tabular}
\label{table:nonlin}
\end{table} \clearpage
\begin{table}[ht]
\caption{Resultados de la ejecución de la metaheurística IGA, utilizando instancias de Dethloff con la configuración -n 250.0 -p 100.0 -cprob 40 -mprob 70}
\centering
\small
\begin{tabular}{c c c c c c c c}
\hline\hline
Instancia & Costo mínimo & Tiempo(seg.) & Costo promedio & Tiempo promedio(seg.) & CME & \%G & \%GP \\ [0.5ex]
\hline
SCA3-0 & 640.55 & 1.78 & 
640.55 & 1.63 & \bf{635.62} & 
0.78 & 0.78\\SCA3-1 & \bf{697.84} & 1.54 & 
698.50 & 1.75 & 697.84 & 0.00
 & 0.10\\SCA3-2 & 661.13 & 1.35 & 
663.70 & 1.65 & \bf{659.34} & 
0.27 & 0.66\\SCA3-3 & 680.60 & 1.44 & 
681.16 & 1.77 & \bf{680.04} & 
0.08 & 0.17\\SCA3-4 & \bf{690.50} & 1.04 & 
690.50 & 1.50 & 690.50 & 0.00
 & 0.00\\
SCA3-5 & 665.04 & 2.32 & 
670.19 & 1.84 & \bf{659.90} & 
0.78 & 1.56\\SCA3-6 & \bf{651.09} & 1.66 & 
652.01 & 1.83 & 651.09 & 0.00
 & 0.14\\SCA3-7 & 664.88 & 1.84 & 
664.88 & 1.65 & \bf{659.17} & 
0.87 & 0.87\\SCA3-8 & \bf{719.47} & 1.52 & 
724.73 & 1.67 & 719.47 & 0.00
 & 0.73\\SCA3-9 & \bf{681.00} & 2.20 & 
681.00 & 1.75 & 681.00 & 0.00
 & 0.00\\
SCA8-0 & 973.22 & 1.88 & 
982.83 & 1.56 & \bf{961.50} & 
1.22 & 2.22\\SCA8-1 & 1073.21 & 2.34 & 
1078.95 & 2.05 & \bf{1049.65} & 
2.24 & 2.79\\SCA8-2 & 1049.22 & 2.89 & 
1050.59 & 1.90 & \bf{1039.64} & 
0.92 & 1.05\\SCA8-3 & 1013.02 & 2.10 & 
1021.75 & 1.69 & \bf{983.34} & 
3.02 & 3.91\\SCA8-4 & 1069.71 & 1.38 & 
1069.71 & 1.43 & \bf{1065.49} & 
0.40 & 0.40\\SCA8-5 & 1053.18 & 2.34 & 
1053.18 & 1.92 & \bf{1027.08} & 
2.54 & 2.54\\SCA8-6 & 990.69 & 1.63 & 
990.69 & 1.72 & \bf{971.82} & 
1.94 & 1.94\\SCA8-7 & 1083.00 & 2.30 & 
1083.00 & 1.86 & \bf{1051.28} & 
3.02 & 3.02\\SCA8-8 & \bf{1071.18} & 2.15 & 
1081.10 & 2.07 & 1071.18 & 0.00
 & 0.93\\SCA8-9 & 1072.10 & 1.52 & 
1072.23 & 1.47 & \bf{1060.50} & 
1.09 & 1.11\\CON3-0 & 624.96 & 2.46 & 
625.21 & 2.03 & \bf{616.52} & 
1.37 & 1.41\\CON3-1 & 560.61 & 1.98 & 
560.72 & 2.04 & \bf{554.47} & 
1.11 & 1.13\\CON3-2 & 521.38 & 2.08 & 
523.20 & 1.95 & \bf{518.00} & 
0.65 & 1.00\\CON3-3 & 594.31 & 2.32 & 
598.29 & 1.85 & \bf{591.19} & 
0.53 & 1.20\\CON3-4 & 593.69 & 1.24 & 
594.67 & 1.69 & \bf{588.79} & 
0.83 & 1.00\\CON3-5 & \bf{563.70} & 2.54 & 
563.70 & 2.04 & 563.70 & 0.00
 & 0.00\\
CON3-6 & 502.88 & 1.70 & 
503.98 & 1.76 & \bf{499.05} & 
0.77 & 0.99\\CON3-7 & 578.41 & 1.19 & 
581.45 & 2.02 & \bf{576.48} & 
0.33 & 0.86\\CON3-8 & 524.59 & 1.47 & 
525.82 & 1.47 & \bf{523.05} & 
0.29 & 0.53\\CON3-9 & 588.11 & 2.66 & 
589.54 & 2.08 & \bf{578.24} & 
1.71 & 1.96\\CON8-0 & 871.80 & 1.55 & 
874.02 & 1.97 & \bf{857.17} & 
1.71 & 1.97\\CON8-1 & 740.93 & 1.80 & 
740.93 & 1.87 & \bf{740.85} & 
0.01 & 0.01\\CON8-2 & 717.31 & 2.35 & 
718.64 & 1.93 & \bf{712.89} & 
0.62 & 0.81\\CON8-3 & 821.86 & 1.68 & 
823.63 & 1.67 & \bf{811.07} & 
1.33 & 1.55\\CON8-4 & 792.12 & 1.82 & 
792.12 & 1.75 & \bf{772.25} & 
2.57 & 2.57\\CON8-5 & 768.42 & 1.53 & 
768.42 & 1.76 & \bf{754.88} & 
1.79 & 1.79\\CON8-6 & 686.85 & 1.96 & 
686.85 & 1.56 & \bf{678.92} & 
1.17 & 1.17\\CON8-7 & 816.07 & 1.83 & 
818.59 & 1.62 & \bf{811.96} & 
0.51 & 0.82\\CON8-8 & 784.08 & 1.88 & 
788.81 & 1.72 & \bf{767.53} & 
2.16 & 2.77\\CON8-9 & 820.69 & 2.11 & 
823.50 & 1.73 & \bf{809.00} & 
1.44 & 1.79\\\bf{PROM.} & 
\bf{766.84} & \bf{1.88} & \bf{768.83} & \bf{1.78} & \bf{758.54} & \bf{1.00} & \bf{1.26}\\[1ex]\hline
\end{tabular}
\label{table:nonlin}
\end{table} \clearpage
\begin{table}[ht]
\caption{Resultados de la ejecución de la metaheurística PSO, utilizando instancias de SalhiNagy con la configuración -n 50.0 -L 10.0 -cp 1 -cg 0 -cl 1 -cn 2 -w1 0.9 -wt 0.1 -K 5}
\centering
\small
\begin{tabular}{c c c c c c c c}
\hline\hline
Instancia & Costo mínimo & Tiempo(seg.) & Costo promedio & Tiempo promedio(seg.) & CME & \%G & \%GP \\ [0.5ex]
\hline
CMT1X & 471.25 & 3.12 & 
484.28 & 3.09 & \bf{470.48} & 
0.16 & 2.93\\CMT1Y & 472.87 & 3.89 & 
475.36 & 3.62 & \bf{470.48} & 
0.51 & 1.04\\CMT2X & 727.83 & 8.84 & 
745.40 & 4.94 & \bf{682.39} & 
6.66 & 9.23\\CMT2Y & 693.24 & 4.84 & 
713.01 & 4.86 & \bf{682.39} & 
1.59 & 4.49\\CMT3X & 730.28 & 40.91 & 
731.66 & 38.08 & \bf{719.06} & 
1.56 & 1.75\\CMT3Y & 727.44 & 43.15 & 
745.79 & 39.90 & \bf{719.06} & 
1.17 & 3.72\\CMT4X & 903.44 & 34.19 & 
921.82 & 80.82 & \bf{854.21} & 
5.76 & 7.91\\CMT4Y & 884.46 & 95.04 & 
916.99 & 75.88 & \bf{852.46} & 
3.75 & 7.57\\CMT5X & 1119.82 & 63.83 & 
1145.94 & 97.50 & \bf{1030.56} & 
8.66 & 11.20\\CMT5Y & 1113.71 & 9.27 & 
1132.95 & 94.41 & \bf{1031.69} & 
7.95 & 9.81\\CMT11X & 892.21 & 19.39 & 
935.68 & 18.65 & \bf{831.09} & 
7.35 & 12.58\\CMT11Y & 889.02 & 6.98 & 
955.73 & 20.98 & \bf{829.85} & 
7.13 & 15.17\\CMT12X & \bf{\underline{645.09}} & 2.29 & 
692.86 & 2.19 & 658.83 & 
\bf{-2.09} & 5.16\\CMT12Y & 669.31 & 2.00 & 
733.11 & 3.05 & \bf{660.47} & 
1.34 & 11.00\\\bf{PROM.} & 
\bf{781.43} & \bf{24.12} & \bf{809.33} & \bf{34.86} & \bf{749.50} & \bf{3.68} & \bf{7.40}\\[1ex]\hline
\end{tabular}
\label{table:nonlin}
\end{table} \clearpage
\begin{table}[ht]
\caption{Resultados de la ejecución de la metaheurística IGA, utilizando instancias de SalhiNagy con la configuración -n 250.0 -p 100.0 -cprob 90 -mprob 70}
\centering
\small
\begin{tabular}{c c c c c c c c}
\hline\hline
Instancia & Costo mínimo & Tiempo(seg.) & Costo promedio & Tiempo promedio(seg.) & CME & \%G & \%GP \\ [0.5ex]
\hline
CMT1X & 479.45 & 1.95 & 
480.94 & 2.07 & \bf{470.48} & 
1.91 & 2.22\\CMT1Y & 476.95 & 2.10 & 
480.36 & 2.05 & \bf{470.48} & 
1.38 & 2.10\\CMT2X & 700.73 & 3.72 & 
708.94 & 4.04 & \bf{682.39} & 
2.69 & 3.89\\CMT2Y & 712.94 & 4.59 & 
716.31 & 4.58 & \bf{682.39} & 
4.48 & 4.97\\CMT3X & 734.60 & 8.55 & 
740.57 & 8.22 & \bf{719.06} & 
2.16 & 2.99\\CMT3Y & 730.87 & 8.17 & 
743.05 & 8.54 & \bf{719.06} & 
1.64 & 3.34\\CMT4X & 885.18 & 22.24 & 
904.09 & 21.87 & \bf{854.21} & 
3.63 & 5.84\\CMT4Y & 906.01 & 22.66 & 
910.50 & 22.17 & \bf{852.46} & 
6.28 & 6.81\\CMT5X & 1094.91 & 39.87 & 
1106.33 & 42.02 & \bf{1030.56} & 
6.24 & 7.35\\CMT5Y & 1100.51 & 43.50 & 
1115.91 & 43.92 & \bf{1031.69} & 
6.67 & 8.16\\CMT11X & 891.41 & 13.64 & 
909.21 & 14.52 & \bf{831.09} & 
7.26 & 9.40\\CMT11Y & 858.05 & 15.77 & 
891.76 & 15.23 & \bf{829.85} & 
3.40 & 7.46\\CMT12X & 675.48 & 8.81 & 
678.99 & 9.40 & \bf{658.83} & 
2.53 & 3.06\\CMT12Y & 673.67 & 7.76 & 
676.09 & 7.98 & \bf{660.47} & 
2.00 & 2.36\\\bf{PROM.} & 
\bf{780.05} & \bf{14.52} & \bf{790.22} & \bf{14.76} & \bf{749.50} & \bf{3.73} & \bf{5.00}\\[1ex]\hline
\end{tabular}
\label{table:nonlin}
\end{table} \clearpage
\begin{table}[ht]
\caption{Resultados de la ejecución de la metaheurística IGA, utilizando instancias de Dethloff con la configuración -n 300.0 -p 30.0 -cprob 40 -mprob 70}
\centering
\small
\begin{tabular}{c c c c c c c c}
\hline\hline
Instancia & Costo mínimo & Tiempo(seg.) & Costo promedio & Tiempo promedio(seg.) & CME & \%G & \%GP \\ [0.5ex]
\hline
SCA3-0 & 640.55 & 0.58 & 
640.55 & 0.64 & \bf{635.62} & 
0.78 & 0.78\\SCA3-1 & 701.74 & 0.49 & 
701.74 & 0.61 & \bf{697.84} & 
0.56 & 0.56\\SCA3-2 & 666.01 & 0.89 & 
672.71 & 0.60 & \bf{659.34} & 
1.01 & 2.03\\SCA3-3 & 686.17 & 0.58 & 
687.89 & 0.56 & \bf{680.04} & 
0.90 & 1.16\\SCA3-4 & \bf{690.50} & 0.78 & 
690.50 & 0.58 & 690.50 & 0.00
 & 0.00\\
SCA3-5 & 662.75 & 0.68 & 
662.95 & 0.63 & \bf{659.90} & 
0.43 & 0.46\\SCA3-6 & 657.24 & 0.64 & 
658.93 & 0.58 & \bf{651.09} & 
0.94 & 1.20\\SCA3-7 & 666.15 & 0.61 & 
666.15 & 0.70 & \bf{659.17} & 
1.06 & 1.06\\SCA3-8 & 719.77 & 0.52 & 
725.86 & 0.49 & \bf{719.47} & 
0.04 & 0.89\\SCA3-9 & 685.19 & 0.34 & 
685.19 & 0.34 & \bf{681.00} & 
0.62 & 0.62\\SCA8-0 & 997.25 & 0.70 & 
1010.69 & 0.61 & \bf{961.50} & 
3.72 & 5.12\\SCA8-1 & 1080.63 & 0.48 & 
1088.91 & 0.58 & \bf{1049.65} & 
2.95 & 3.74\\SCA8-2 & 1054.47 & 0.44 & 
1056.14 & 0.49 & \bf{1039.64} & 
1.43 & 1.59\\SCA8-3 & 1033.52 & 0.39 & 
1036.30 & 0.39 & \bf{983.34} & 
5.10 & 5.39\\SCA8-4 & 1068.97 & 0.94 & 
1074.89 & 0.75 & \bf{1065.49} & 
0.33 & 0.88\\SCA8-5 & 1058.65 & 0.78 & 
1058.65 & 0.52 & \bf{1027.08} & 
3.07 & 3.07\\SCA8-6 & 972.48 & 0.74 & 
974.63 & 0.65 & \bf{971.82} & 
0.07 & 0.29\\SCA8-7 & 1076.37 & 0.47 & 
1078.99 & 0.47 & \bf{1051.28} & 
2.39 & 2.64\\SCA8-8 & 1086.27 & 0.36 & 
1089.55 & 0.46 & \bf{1071.18} & 
1.41 & 1.71\\SCA8-9 & 1073.62 & 0.40 & 
1076.24 & 0.56 & \bf{1060.50} & 
1.24 & 1.48\\CON3-0 & 634.46 & 0.90 & 
638.21 & 0.64 & \bf{616.52} & 
2.91 & 3.52\\CON3-1 & 560.75 & 0.43 & 
560.75 & 0.59 & \bf{554.47} & 
1.13 & 1.13\\CON3-2 & 521.38 & 0.50 & 
522.68 & 0.50 & \bf{518.00} & 
0.65 & 0.90\\CON3-3 & 601.66 & 0.45 & 
602.75 & 0.54 & \bf{591.19} & 
1.77 & 1.96\\CON3-4 & 595.25 & 0.68 & 
599.21 & 0.60 & \bf{588.79} & 
1.10 & 1.77\\CON3-5 & 564.88 & 0.61 & 
567.17 & 0.65 & \bf{563.70} & 
0.21 & 0.62\\CON3-6 & 506.53 & 0.45 & 
507.96 & 0.48 & \bf{499.05} & 
1.50 & 1.79\\CON3-7 & 586.63 & 0.42 & 
586.63 & 0.62 & \bf{576.48} & 
1.76 & 1.76\\CON3-8 & 532.86 & 0.40 & 
534.25 & 0.51 & \bf{523.05} & 
1.88 & 2.14\\CON3-9 & 588.11 & 0.43 & 
588.37 & 0.62 & \bf{578.24} & 
1.71 & 1.75\\CON8-0 & 875.94 & 0.54 & 
887.73 & 0.54 & \bf{857.17} & 
2.19 & 3.57\\CON8-1 & 763.56 & 0.56 & 
764.76 & 0.46 & \bf{740.85} & 
3.07 & 3.23\\CON8-2 & 725.57 & 0.48 & 
725.57 & 0.56 & \bf{712.89} & 
1.78 & 1.78\\CON8-3 & 822.51 & 0.53 & 
830.46 & 0.64 & \bf{811.07} & 
1.41 & 2.39\\CON8-4 & 781.80 & 0.49 & 
787.54 & 0.48 & \bf{772.25} & 
1.24 & 1.98\\CON8-5 & 760.91 & 0.77 & 
760.91 & 0.66 & \bf{754.88} & 
0.80 & 0.80\\CON8-6 & 689.56 & 0.56 & 
691.09 & 0.48 & \bf{678.92} & 
1.57 & 1.79\\CON8-7 & 820.01 & 0.69 & 
820.01 & 0.84 & \bf{811.96} & 
0.99 & 0.99\\CON8-8 & 783.02 & 0.44 & 
786.59 & 0.57 & \bf{767.53} & 
2.02 & 2.48\\CON8-9 & 832.96 & 0.44 & 
835.48 & 0.41 & \bf{809.00} & 
2.96 & 3.27\\\bf{PROM.} & 
\bf{770.67} & \bf{0.56} & \bf{773.39} & \bf{0.57} & \bf{758.54} & \bf{1.52} & \bf{1.86}\\[1ex]\hline
\end{tabular}
\label{table:nonlin}
\end{table} \clearpage
\begin{table}[ht]
\caption{Resultados de la ejecución de la metaheurística IGA, utilizando instancias de SalhiNagy con la configuración -n 300.0 -p 30.0 -cprob 90 -mprob 70}
\centering
\small
\begin{tabular}{c c c c c c c c}
\hline\hline
Instancia & Costo mínimo & Tiempo(seg.) & Costo promedio & Tiempo promedio(seg.) & CME & \%G & \%GP \\ [0.5ex]
\hline
CMT1X & 482.16 & 0.81 & 
484.85 & 0.67 & \bf{470.48} & 
2.48 & 3.05\\CMT1Y & 480.50 & 0.42 & 
488.89 & 0.51 & \bf{470.48} & 
2.13 & 3.91\\CMT2X & 709.59 & 1.34 & 
718.44 & 1.46 & \bf{682.39} & 
3.99 & 5.28\\CMT2Y & 706.48 & 1.34 & 
715.67 & 1.46 & \bf{682.39} & 
3.53 & 4.88\\CMT3X & 741.17 & 2.72 & 
744.63 & 2.71 & \bf{719.06} & 
3.07 & 3.56\\CMT3Y & 738.38 & 2.46 & 
743.73 & 2.55 & \bf{719.06} & 
2.69 & 3.43\\CMT4X & 909.16 & 6.90 & 
921.90 & 7.01 & \bf{854.21} & 
6.43 & 7.92\\CMT4Y & 900.52 & 7.43 & 
912.67 & 6.92 & \bf{852.46} & 
5.64 & 7.06\\CMT5X & 1114.72 & 13.18 & 
1141.32 & 13.35 & \bf{1030.56} & 
8.17 & 10.75\\CMT5Y & 1108.44 & 13.22 & 
1133.50 & 13.57 & \bf{1031.69} & 
7.44 & 9.87\\CMT11X & 911.52 & 4.22 & 
936.20 & 4.51 & \bf{831.09} & 
9.68 & 12.65\\CMT11Y & 901.72 & 6.28 & 
907.88 & 5.11 & \bf{829.85} & 
8.66 & 9.40\\CMT12X & 673.82 & 2.84 & 
676.40 & 2.93 & \bf{658.83} & 
2.28 & 2.67\\CMT12Y & 674.65 & 2.70 & 
682.47 & 2.77 & \bf{660.47} & 
2.15 & 3.33\\\bf{PROM.} & 
\bf{789.49} & \bf{4.70} & \bf{800.61} & \bf{4.68} & \bf{749.50} & \bf{4.88} & \bf{6.27}\\[1ex]\hline
\end{tabular}
\label{table:nonlin}
\end{table} \clearpage
\begin{table}[ht]
\caption{Resultados de la ejecución de la metaheurística IGA, utilizando instancias de Dethloff con la configuración -n 300.0 -p 40.0 -cprob 40 -mprob 70}
\centering
\small
\begin{tabular}{c c c c c c c c}
\hline\hline
Instancia & Costo mínimo & Tiempo(seg.) & Costo promedio & Tiempo promedio(seg.) & CME & \%G & \%GP \\ [0.5ex]
\hline
SCA3-0 & 640.55 & 1.10 & 
640.55 & 0.81 & \bf{635.62} & 
0.78 & 0.78\\SCA3-1 & 701.86 & 0.55 & 
701.86 & 0.70 & \bf{697.84} & 
0.58 & 0.58\\SCA3-2 & 664.21 & 0.72 & 
668.85 & 0.65 & \bf{659.34} & 
0.74 & 1.44\\SCA3-3 & 680.60 & 0.60 & 
681.44 & 0.64 & \bf{680.04} & 
0.08 & 0.21\\SCA3-4 & \bf{690.50} & 0.91 & 
690.50 & 0.77 & 690.50 & 0.00
 & 0.00\\
SCA3-5 & 670.10 & 1.18 & 
672.43 & 0.81 & \bf{659.90} & 
1.55 & 1.90\\SCA3-6 & 652.94 & 0.79 & 
652.94 & 0.77 & \bf{651.09} & 
0.28 & 0.28\\SCA3-7 & 666.15 & 0.88 & 
666.15 & 0.80 & \bf{659.17} & 
1.06 & 1.06\\SCA3-8 & 727.49 & 1.13 & 
730.78 & 0.97 & \bf{719.47} & 
1.11 & 1.57\\SCA3-9 & \bf{681.00} & 0.56 & 
682.86 & 0.79 & 681.00 & 0.00
 & 0.27\\SCA8-0 & 992.89 & 1.28 & 
998.67 & 0.73 & \bf{961.50} & 
3.26 & 3.87\\SCA8-1 & 1069.55 & 0.67 & 
1069.55 & 0.73 & \bf{1049.65} & 
1.90 & 1.90\\SCA8-2 & 1053.94 & 0.66 & 
1054.34 & 0.73 & \bf{1039.64} & 
1.38 & 1.41\\SCA8-3 & 1017.63 & 0.93 & 
1021.23 & 0.76 & \bf{983.34} & 
3.49 & 3.85\\SCA8-4 & 1093.29 & 0.45 & 
1093.29 & 0.70 & \bf{1065.49} & 
2.61 & 2.61\\SCA8-5 & 1056.52 & 0.78 & 
1058.99 & 0.77 & \bf{1027.08} & 
2.87 & 3.11\\SCA8-6 & 994.05 & 0.66 & 
997.19 & 0.69 & \bf{971.82} & 
2.29 & 2.61\\SCA8-7 & 1093.90 & 0.83 & 
1096.94 & 0.82 & \bf{1051.28} & 
4.05 & 4.34\\SCA8-8 & 1093.54 & 0.59 & 
1094.57 & 0.68 & \bf{1071.18} & 
2.09 & 2.18\\SCA8-9 & 1078.30 & 0.45 & 
1078.30 & 0.71 & \bf{1060.50} & 
1.68 & 1.68\\CON3-0 & 619.09 & 0.66 & 
631.77 & 0.76 & \bf{616.52} & 
0.42 & 2.47\\CON3-1 & 556.92 & 0.94 & 
558.02 & 0.92 & \bf{554.47} & 
0.44 & 0.64\\CON3-2 & 521.38 & 1.09 & 
521.38 & 1.01 & \bf{518.00} & 
0.65 & 0.65\\CON3-3 & 591.20 & 0.95 & 
593.39 & 0.70 & \bf{591.19} & 
0.00 & 0.37\\CON3-4 & 594.59 & 0.76 & 
600.09 & 0.67 & \bf{588.79} & 
0.99 & 1.92\\CON3-5 & 564.88 & 0.66 & 
569.80 & 0.77 & \bf{563.70} & 
0.21 & 1.08\\CON3-6 & 504.44 & 1.11 & 
504.96 & 0.93 & \bf{499.05} & 
1.08 & 1.18\\CON3-7 & 582.33 & 1.00 & 
586.66 & 0.74 & \bf{576.48} & 
1.01 & 1.77\\CON3-8 & \bf{523.05} & 0.56 & 
523.10 & 0.69 & 523.05 & 0.00
 & 0.01\\CON3-9 & 588.11 & 0.58 & 
589.30 & 0.67 & \bf{578.24} & 
1.71 & 1.91\\CON8-0 & 864.56 & 0.65 & 
890.85 & 0.67 & \bf{857.17} & 
0.86 & 3.93\\CON8-1 & 755.71 & 0.87 & 
764.66 & 0.92 & \bf{740.85} & 
2.01 & 3.21\\CON8-2 & 727.77 & 0.88 & 
727.77 & 0.96 & \bf{712.89} & 
2.09 & 2.09\\CON8-3 & 841.27 & 0.79 & 
842.39 & 0.94 & \bf{811.07} & 
3.72 & 3.86\\CON8-4 & 804.57 & 1.11 & 
804.94 & 1.07 & \bf{772.25} & 
4.19 & 4.23\\CON8-5 & 758.12 & 0.76 & 
758.12 & 0.79 & \bf{754.88} & 
0.43 & 0.43\\CON8-6 & 686.83 & 0.67 & 
686.83 & 0.80 & \bf{678.92} & 
1.17 & 1.17\\CON8-7 & 828.65 & 0.61 & 
834.92 & 0.84 & \bf{811.96} & 
2.06 & 2.83\\CON8-8 & 790.90 & 1.31 & 
794.22 & 0.98 & \bf{767.53} & 
3.04 & 3.48\\CON8-9 & 841.94 & 1.03 & 
849.57 & 0.94 & \bf{809.00} & 
4.07 & 5.01\\\bf{PROM.} & 
\bf{771.63} & \bf{0.82} & \bf{774.60} & \bf{0.79} & \bf{758.54} & \bf{1.55} & \bf{1.95}\\[1ex]\hline
\end{tabular}
\label{table:nonlin}
\end{table} \clearpage
\begin{table}[ht]
\caption{Resultados de la ejecución de la metaheurística IGA, utilizando instancias de SalhiNagy con la configuración -n 300.0 -p 40.0 -cprob 90 -mprob 70}
\centering
\small
\begin{tabular}{c c c c c c c c}
\hline\hline
Instancia & Costo mínimo & Tiempo(seg.) & Costo promedio & Tiempo promedio(seg.) & CME & \%G & \%GP \\ [0.5ex]
\hline
CMT1X & 479.61 & 1.10 & 
481.67 & 0.95 & \bf{470.48} & 
1.94 & 2.38\\CMT1Y & 480.06 & 0.80 & 
483.15 & 0.94 & \bf{470.48} & 
2.04 & 2.69\\CMT2X & 709.85 & 1.89 & 
714.51 & 2.07 & \bf{682.39} & 
4.02 & 4.71\\CMT2Y & 710.18 & 1.55 & 
713.12 & 1.84 & \bf{682.39} & 
4.07 & 4.50\\CMT3X & 739.84 & 3.39 & 
745.27 & 3.74 & \bf{719.06} & 
2.89 & 3.65\\CMT3Y & 737.04 & 3.84 & 
744.27 & 3.88 & \bf{719.06} & 
2.50 & 3.51\\CMT4X & 902.30 & 9.85 & 
906.86 & 9.40 & \bf{854.21} & 
5.63 & 6.16\\CMT4Y & 907.86 & 9.55 & 
925.84 & 9.65 & \bf{852.46} & 
6.50 & 8.61\\CMT5X & 1097.95 & 18.57 & 
1110.91 & 17.89 & \bf{1030.56} & 
6.54 & 7.80\\CMT5Y & 1111.51 & 19.16 & 
1122.43 & 19.04 & \bf{1031.69} & 
7.74 & 8.80\\CMT11X & 902.10 & 6.43 & 
906.94 & 6.11 & \bf{831.09} & 
8.54 & 9.13\\CMT11Y & 900.55 & 6.50 & 
921.03 & 6.70 & \bf{829.85} & 
8.52 & 10.99\\CMT12X & 675.93 & 4.19 & 
679.59 & 3.93 & \bf{658.83} & 
2.60 & 3.15\\CMT12Y & 676.70 & 3.82 & 
679.93 & 3.46 & \bf{660.47} & 
2.46 & 2.95\\\bf{PROM.} & 
\bf{787.96} & \bf{6.47} & \bf{795.39} & \bf{6.40} & \bf{749.50} & \bf{4.71} & \bf{5.64}\\[1ex]\hline
\end{tabular}
\label{table:nonlin}
\end{table} \clearpage
\begin{table}[ht]
\caption{Resultados de la ejecución de la metaheurística IGA, utilizando instancias de Dethloff con la configuración -n 300.0 -p 50.0 -cprob 40 -mprob 70}
\centering
\small
\begin{tabular}{c c c c c c c c}
\hline\hline
Instancia & Costo mínimo & Tiempo(seg.) & Costo promedio & Tiempo promedio(seg.) & CME & \%G & \%GP \\ [0.5ex]
\hline
SCA3-0 & 641.69 & 0.90 & 
642.25 & 1.08 & \bf{635.62} & 
0.95 & 1.04\\SCA3-1 & \bf{697.84} & 1.33 & 
701.43 & 1.22 & 697.84 & 0.00
 & 0.52\\SCA3-2 & 661.13 & 0.93 & 
669.05 & 1.14 & \bf{659.34} & 
0.27 & 1.47\\SCA3-3 & 681.16 & 0.92 & 
681.16 & 0.94 & \bf{680.04} & 
0.16 & 0.16\\SCA3-4 & \bf{690.50} & 0.72 & 
690.50 & 1.08 & 690.50 & 0.00
 & 0.00\\
SCA3-5 & 665.64 & 0.85 & 
665.64 & 0.92 & \bf{659.90} & 
0.87 & 0.87\\SCA3-6 & 652.94 & 0.72 & 
653.94 & 1.03 & \bf{651.09} & 
0.28 & 0.44\\SCA3-7 & 666.15 & 1.11 & 
667.09 & 0.79 & \bf{659.17} & 
1.06 & 1.20\\SCA3-8 & 719.77 & 0.67 & 
725.27 & 0.71 & \bf{719.47} & 
0.04 & 0.81\\SCA3-9 & \bf{681.00} & 0.64 & 
681.00 & 0.64 & 681.00 & 0.00
 & 0.00\\
SCA8-0 & 982.18 & 0.86 & 
985.21 & 0.94 & \bf{961.50} & 
2.15 & 2.47\\SCA8-1 & 1070.65 & 0.71 & 
1077.70 & 0.99 & \bf{1049.65} & 
2.00 & 2.67\\SCA8-2 & 1054.47 & 0.77 & 
1054.47 & 0.93 & \bf{1039.64} & 
1.43 & 1.43\\SCA8-3 & 1010.75 & 1.20 & 
1010.75 & 0.94 & \bf{983.34} & 
2.79 & 2.79\\SCA8-4 & 1088.12 & 1.55 & 
1089.69 & 0.99 & \bf{1065.49} & 
2.12 & 2.27\\SCA8-5 & 1058.10 & 0.72 & 
1059.83 & 1.23 & \bf{1027.08} & 
3.02 & 3.19\\SCA8-6 & 987.70 & 1.04 & 
987.70 & 0.98 & \bf{971.82} & 
1.63 & 1.63\\SCA8-7 & 1071.79 & 0.56 & 
1087.29 & 0.96 & \bf{1051.28} & 
1.95 & 3.43\\SCA8-8 & 1084.41 & 1.16 & 
1084.41 & 1.03 & \bf{1071.18} & 
1.24 & 1.24\\SCA8-9 & 1075.83 & 1.06 & 
1075.83 & 0.85 & \bf{1060.50} & 
1.45 & 1.45\\CON3-0 & 624.96 & 1.00 & 
632.82 & 0.98 & \bf{616.52} & 
1.37 & 2.64\\CON3-1 & 557.21 & 1.24 & 
560.01 & 0.92 & \bf{554.47} & 
0.49 & 1.00\\CON3-2 & 521.38 & 0.98 & 
521.38 & 0.88 & \bf{518.00} & 
0.65 & 0.65\\CON3-3 & 603.31 & 1.47 & 
607.47 & 0.98 & \bf{591.19} & 
2.05 & 2.75\\CON3-4 & 594.59 & 0.83 & 
594.59 & 0.74 & \bf{588.79} & 
0.99 & 0.99\\CON3-5 & 567.94 & 0.66 & 
569.55 & 0.79 & \bf{563.70} & 
0.75 & 1.04\\CON3-6 & 502.26 & 1.37 & 
504.37 & 1.00 & \bf{499.05} & 
0.64 & 1.07\\CON3-7 & 586.01 & 0.70 & 
586.01 & 1.06 & \bf{576.48} & 
1.65 & 1.65\\CON3-8 & 524.59 & 1.46 & 
527.27 & 0.93 & \bf{523.05} & 
0.29 & 0.81\\CON3-9 & 588.40 & 0.95 & 
588.72 & 1.12 & \bf{578.24} & 
1.76 & 1.81\\CON8-0 & 859.74 & 0.92 & 
868.26 & 0.82 & \bf{857.17} & 
0.30 & 1.29\\CON8-1 & 768.70 & 1.40 & 
768.85 & 1.18 & \bf{740.85} & 
3.76 & 3.78\\CON8-2 & 713.60 & 1.20 & 
724.87 & 1.07 & \bf{712.89} & 
0.10 & 1.68\\CON8-3 & 817.57 & 1.08 & 
817.57 & 1.15 & \bf{811.07} & 
0.80 & 0.80\\CON8-4 & 780.54 & 1.01 & 
783.08 & 1.05 & \bf{772.25} & 
1.07 & 1.40\\CON8-5 & 762.61 & 0.92 & 
762.61 & 0.85 & \bf{754.88} & 
1.02 & 1.02\\CON8-6 & 696.77 & 1.58 & 
697.75 & 0.99 & \bf{678.92} & 
2.63 & 2.77\\CON8-7 & 815.60 & 1.24 & 
815.60 & 0.80 & \bf{811.96} & 
0.45 & 0.45\\CON8-8 & 792.94 & 1.06 & 
792.94 & 0.93 & \bf{767.53} & 
3.31 & 3.31\\CON8-9 & 820.40 & 0.72 & 
820.40 & 1.20 & \bf{809.00} & 
1.41 & 1.41\\\bf{PROM.} & 
\bf{768.52} & \bf{1.01} & \bf{770.86} & \bf{0.97} & \bf{758.54} & \bf{1.22} & \bf{1.53}\\[1ex]\hline
\end{tabular}
\label{table:nonlin}
\end{table} \clearpage
\begin{table}[ht]
\caption{Resultados de la ejecución de la metaheurística IGA, utilizando instancias de SalhiNagy con la configuración -n 300.0 -p 50.0 -cprob 90 -mprob 70}
\centering
\small
\begin{tabular}{c c c c c c c c}
\hline\hline
Instancia & Costo mínimo & Tiempo(seg.) & Costo promedio & Tiempo promedio(seg.) & CME & \%G & \%GP \\ [0.5ex]
\hline
CMT1X & 478.84 & 1.26 & 
478.84 & 1.23 & \bf{470.48} & 
1.78 & 1.78\\CMT1Y & 477.78 & 0.94 & 
486.30 & 1.30 & \bf{470.48} & 
1.55 & 3.36\\CMT2X & 703.46 & 2.64 & 
710.45 & 2.47 & \bf{682.39} & 
3.09 & 4.11\\CMT2Y & 701.52 & 2.60 & 
706.74 & 2.46 & \bf{682.39} & 
2.80 & 3.57\\CMT3X & 735.48 & 5.10 & 
743.60 & 4.55 & \bf{719.06} & 
2.28 & 3.41\\CMT3Y & 743.54 & 4.45 & 
748.54 & 4.19 & \bf{719.06} & 
3.40 & 4.10\\CMT4X & 905.05 & 11.38 & 
909.65 & 11.71 & \bf{854.21} & 
5.95 & 6.49\\CMT4Y & 904.87 & 11.66 & 
916.42 & 11.85 & \bf{852.46} & 
6.15 & 7.50\\CMT5X & 1107.79 & 23.02 & 
1118.65 & 22.48 & \bf{1030.56} & 
7.49 & 8.55\\CMT5Y & 1100.43 & 22.17 & 
1122.14 & 23.18 & \bf{1031.69} & 
6.66 & 8.77\\CMT11X & 902.72 & 7.10 & 
908.51 & 7.78 & \bf{831.09} & 
8.62 & 9.32\\CMT11Y & 891.51 & 8.79 & 
918.62 & 8.66 & \bf{829.85} & 
7.43 & 10.70\\CMT12X & 674.30 & 5.05 & 
684.17 & 4.60 & \bf{658.83} & 
2.35 & 3.85\\CMT12Y & 676.92 & 4.67 & 
680.98 & 4.43 & \bf{660.47} & 
2.49 & 3.11\\\bf{PROM.} & 
\bf{786.01} & \bf{7.92} & \bf{795.26} & \bf{7.92} & \bf{749.50} & \bf{4.43} & \bf{5.61}\\[1ex]\hline
\end{tabular}
\label{table:nonlin}
\end{table} \clearpage
\begin{table}[ht]
\caption{Resultados de la ejecución de la metaheurística IGA, utilizando instancias de Dethloff con la configuración -n 300.0 -p 60.0 -cprob 40 -mprob 70}
\centering
\small
\begin{tabular}{c c c c c c c c}
\hline\hline
Instancia & Costo mínimo & Tiempo(seg.) & Costo promedio & Tiempo promedio(seg.) & CME & \%G & \%GP \\ [0.5ex]
\hline
SCA3-0 & 641.64 & 1.31 & 
641.68 & 1.22 & \bf{635.62} & 
0.95 & 0.95\\SCA3-1 & \bf{697.84} & 1.23 & 
697.84 & 1.19 & 697.84 & 0.00
 & 0.00\\
SCA3-2 & 661.13 & 1.23 & 
666.98 & 1.05 & \bf{659.34} & 
0.27 & 1.16\\SCA3-3 & 680.60 & 0.84 & 
681.16 & 0.95 & \bf{680.04} & 
0.08 & 0.17\\SCA3-4 & \bf{690.50} & 0.95 & 
690.50 & 1.14 & 690.50 & 0.00
 & 0.00\\
SCA3-5 & 665.64 & 0.82 & 
666.75 & 1.28 & \bf{659.90} & 
0.87 & 1.04\\SCA3-6 & 652.94 & 1.22 & 
654.54 & 1.06 & \bf{651.09} & 
0.28 & 0.53\\SCA3-7 & 666.15 & 1.52 & 
667.76 & 0.98 & \bf{659.17} & 
1.06 & 1.30\\SCA3-8 & 723.99 & 1.41 & 
724.14 & 1.29 & \bf{719.47} & 
0.63 & 0.65\\SCA3-9 & \bf{681.00} & 1.03 & 
681.00 & 1.03 & 681.00 & 0.00
 & 0.00\\
SCA8-0 & 997.57 & 1.40 & 
997.57 & 1.19 & \bf{961.50} & 
3.75 & 3.75\\SCA8-1 & 1054.12 & 1.20 & 
1056.72 & 1.18 & \bf{1049.65} & 
0.43 & 0.67\\SCA8-2 & 1050.37 & 0.89 & 
1050.37 & 0.95 & \bf{1039.64} & 
1.03 & 1.03\\SCA8-3 & 1002.24 & 0.88 & 
1009.78 & 1.02 & \bf{983.34} & 
1.92 & 2.69\\SCA8-4 & 1069.87 & 0.79 & 
1069.87 & 0.78 & \bf{1065.49} & 
0.41 & 0.41\\SCA8-5 & 1059.94 & 1.24 & 
1059.94 & 1.20 & \bf{1027.08} & 
3.20 & 3.20\\SCA8-6 & 977.87 & 1.39 & 
977.87 & 1.23 & \bf{971.82} & 
0.62 & 0.62\\SCA8-7 & 1063.60 & 0.70 & 
1067.85 & 1.01 & \bf{1051.28} & 
1.17 & 1.58\\SCA8-8 & \bf{1071.18} & 1.21 & 
1071.18 & 0.88 & 1071.18 & 0.00
 & 0.00\\
SCA8-9 & 1067.42 & 0.77 & 
1067.42 & 1.01 & \bf{1060.50} & 
0.65 & 0.65\\CON3-0 & 619.09 & 1.21 & 
629.25 & 1.35 & \bf{616.52} & 
0.42 & 2.07\\CON3-1 & 560.02 & 0.88 & 
560.79 & 1.10 & \bf{554.47} & 
1.00 & 1.14\\CON3-2 & 521.38 & 1.28 & 
522.16 & 1.26 & \bf{518.00} & 
0.65 & 0.80\\CON3-3 & 594.11 & 1.83 & 
599.53 & 1.50 & \bf{591.19} & 
0.49 & 1.41\\CON3-4 & 594.59 & 1.47 & 
600.02 & 1.12 & \bf{588.79} & 
0.99 & 1.91\\CON3-5 & 564.88 & 0.75 & 
565.83 & 0.94 & \bf{563.70} & 
0.21 & 0.38\\CON3-6 & 500.88 & 1.12 & 
503.50 & 1.28 & \bf{499.05} & 
0.37 & 0.89\\CON3-7 & 578.22 & 0.68 & 
579.20 & 1.06 & \bf{576.48} & 
0.30 & 0.47\\CON3-8 & 530.33 & 1.16 & 
532.37 & 1.24 & \bf{523.05} & 
1.39 & 1.78\\CON3-9 & 582.79 & 1.60 & 
582.79 & 1.16 & \bf{578.24} & 
0.79 & 0.79\\CON8-0 & 875.52 & 1.07 & 
878.16 & 1.30 & \bf{857.17} & 
2.14 & 2.45\\CON8-1 & 761.33 & 0.81 & 
762.78 & 1.21 & \bf{740.85} & 
2.76 & 2.96\\CON8-2 & 721.42 & 0.96 & 
724.04 & 1.22 & \bf{712.89} & 
1.20 & 1.56\\CON8-3 & 828.56 & 1.68 & 
834.30 & 1.30 & \bf{811.07} & 
2.16 & 2.86\\CON8-4 & 780.54 & 1.25 & 
785.91 & 1.16 & \bf{772.25} & 
1.07 & 1.77\\CON8-5 & 776.30 & 1.74 & 
776.99 & 1.30 & \bf{754.88} & 
2.84 & 2.93\\CON8-6 & 688.99 & 1.11 & 
691.27 & 1.18 & \bf{678.92} & 
1.48 & 1.82\\CON8-7 & 824.75 & 1.07 & 
824.83 & 0.99 & \bf{811.96} & 
1.58 & 1.59\\CON8-8 & 782.95 & 0.94 & 
786.42 & 1.02 & \bf{767.53} & 
2.01 & 2.46\\CON8-9 & 826.81 & 1.06 & 
827.11 & 1.01 & \bf{809.00} & 
2.20 & 2.24\\\bf{PROM.} & 
\bf{767.23} & \bf{1.14} & \bf{769.20} & \bf{1.13} & \bf{758.54} & \bf{1.08} & \bf{1.37}\\[1ex]\hline
\end{tabular}
\label{table:nonlin}
\end{table} \clearpage
\begin{table}[ht]
\caption{Resultados de la ejecución de la metaheurística IGA, utilizando instancias de SalhiNagy con la configuración -n 300.0 -p 60.0 -cprob 90 -mprob 70}
\centering
\small
\begin{tabular}{c c c c c c c c}
\hline\hline
Instancia & Costo mínimo & Tiempo(seg.) & Costo promedio & Tiempo promedio(seg.) & CME & \%G & \%GP \\ [0.5ex]
\hline
CMT1X & 478.44 & 1.58 & 
481.53 & 1.41 & \bf{470.48} & 
1.69 & 2.35\\CMT1Y & 474.87 & 1.62 & 
480.67 & 1.50 & \bf{470.48} & 
0.93 & 2.17\\CMT2X & 709.86 & 3.10 & 
710.60 & 2.79 & \bf{682.39} & 
4.03 & 4.13\\CMT2Y & 713.64 & 2.67 & 
718.47 & 2.84 & \bf{682.39} & 
4.58 & 5.29\\CMT3X & 741.08 & 5.12 & 
745.92 & 5.43 & \bf{719.06} & 
3.06 & 3.74\\CMT3Y & 740.34 & 5.27 & 
747.12 & 5.36 & \bf{719.06} & 
2.96 & 3.90\\CMT4X & 902.77 & 13.29 & 
909.43 & 13.55 & \bf{854.21} & 
5.68 & 6.46\\CMT4Y & 900.23 & 14.48 & 
912.31 & 13.92 & \bf{852.46} & 
5.60 & 7.02\\CMT5X & 1111.55 & 27.36 & 
1124.21 & 26.67 & \bf{1030.56} & 
7.86 & 9.09\\CMT5Y & 1112.92 & 26.98 & 
1117.28 & 26.50 & \bf{1031.69} & 
7.87 & 8.30\\CMT11X & 913.83 & 8.67 & 
929.13 & 8.95 & \bf{831.09} & 
9.96 & 11.80\\CMT11Y & 851.47 & 8.76 & 
890.25 & 9.22 & \bf{829.85} & 
2.61 & 7.28\\CMT12X & 675.28 & 5.89 & 
678.82 & 5.67 & \bf{658.83} & 
2.50 & 3.03\\CMT12Y & 675.76 & 5.55 & 
677.54 & 5.45 & \bf{660.47} & 
2.32 & 2.58\\\bf{PROM.} & 
\bf{785.86} & \bf{9.31} & \bf{794.52} & \bf{9.23} & \bf{749.50} & \bf{4.40} & \bf{5.51}\\[1ex]\hline
\end{tabular}
\label{table:nonlin}
\end{table} \clearpage
\begin{table}[ht]
\caption{Resultados de la ejecución de la metaheurística IGA, utilizando instancias de Dethloff con la configuración -n 300.0 -p 70.0 -cprob 40 -mprob 70}
\centering
\small
\begin{tabular}{c c c c c c c c}
\hline\hline
Instancia & Costo mínimo & Tiempo(seg.) & Costo promedio & Tiempo promedio(seg.) & CME & \%G & \%GP \\ [0.5ex]
\hline
SCA3-0 & 640.55 & 1.40 & 
641.40 & 1.62 & \bf{635.62} & 
0.78 & 0.91\\SCA3-1 & 700.50 & 1.18 & 
701.10 & 1.33 & \bf{697.84} & 
0.38 & 0.47\\SCA3-2 & 664.18 & 1.27 & 
664.86 & 1.37 & \bf{659.34} & 
0.73 & 0.84\\SCA3-3 & 681.16 & 1.15 & 
681.16 & 1.42 & \bf{680.04} & 
0.16 & 0.16\\SCA3-4 & \bf{690.50} & 1.16 & 
690.50 & 1.07 & 690.50 & 0.00
 & 0.00\\
SCA3-5 & 665.64 & 1.43 & 
671.04 & 1.23 & \bf{659.90} & 
0.87 & 1.69\\SCA3-6 & 652.94 & 1.54 & 
656.04 & 1.29 & \bf{651.09} & 
0.28 & 0.76\\SCA3-7 & 666.15 & 1.39 & 
666.70 & 1.37 & \bf{659.17} & 
1.06 & 1.14\\SCA3-8 & 724.28 & 1.47 & 
724.47 & 1.31 & \bf{719.47} & 
0.67 & 0.69\\SCA3-9 & \bf{681.00} & 1.31 & 
681.00 & 1.37 & 681.00 & 0.00
 & 0.00\\
SCA8-0 & 990.80 & 1.12 & 
1000.11 & 1.64 & \bf{961.50} & 
3.05 & 4.02\\SCA8-1 & 1072.18 & 0.92 & 
1075.19 & 1.24 & \bf{1049.65} & 
2.15 & 2.43\\SCA8-2 & 1054.47 & 2.02 & 
1054.47 & 1.51 & \bf{1039.64} & 
1.43 & 1.43\\SCA8-3 & 1007.99 & 0.95 & 
1007.99 & 1.32 & \bf{983.34} & 
2.51 & 2.51\\SCA8-4 & 1079.41 & 1.58 & 
1084.15 & 1.53 & \bf{1065.49} & 
1.31 & 1.75\\SCA8-5 & 1038.93 & 1.08 & 
1046.27 & 1.10 & \bf{1027.08} & 
1.15 & 1.87\\SCA8-6 & 972.48 & 2.01 & 
987.31 & 1.60 & \bf{971.82} & 
0.07 & 1.59\\SCA8-7 & 1077.00 & 0.88 & 
1077.74 & 1.17 & \bf{1051.28} & 
2.45 & 2.52\\SCA8-8 & \bf{1071.18} & 1.32 & 
1072.82 & 1.32 & 1071.18 & 0.00
 & 0.15\\SCA8-9 & 1066.15 & 1.27 & 
1067.64 & 1.46 & \bf{1060.50} & 
0.53 & 0.67\\CON3-0 & 624.96 & 1.26 & 
627.18 & 1.31 & \bf{616.52} & 
1.37 & 1.73\\CON3-1 & 560.75 & 0.99 & 
560.75 & 1.39 & \bf{554.47} & 
1.13 & 1.13\\CON3-2 & 521.38 & 1.25 & 
521.38 & 1.21 & \bf{518.00} & 
0.65 & 0.65\\CON3-3 & 597.51 & 1.76 & 
603.06 & 1.36 & \bf{591.19} & 
1.07 & 2.01\\CON3-4 & 594.59 & 1.43 & 
599.65 & 1.37 & \bf{588.79} & 
0.99 & 1.84\\CON3-5 & \bf{563.70} & 1.71 & 
563.70 & 1.34 & 563.70 & 0.00
 & 0.00\\
CON3-6 & 505.14 & 2.08 & 
506.02 & 1.32 & \bf{499.05} & 
1.22 & 1.40\\CON3-7 & 578.22 & 1.54 & 
578.22 & 1.37 & \bf{576.48} & 
0.30 & 0.30\\CON3-8 & 524.38 & 1.44 & 
526.12 & 1.17 & \bf{523.05} & 
0.25 & 0.59\\CON3-9 & 588.18 & 1.73 & 
588.33 & 1.53 & \bf{578.24} & 
1.72 & 1.74\\CON8-0 & 865.86 & 2.32 & 
865.92 & 1.49 & \bf{857.17} & 
1.01 & 1.02\\CON8-1 & 762.62 & 0.97 & 
763.76 & 1.39 & \bf{740.85} & 
2.94 & 3.09\\CON8-2 & 718.85 & 1.52 & 
722.33 & 1.49 & \bf{712.89} & 
0.84 & 1.32\\CON8-3 & 821.87 & 1.87 & 
831.29 & 1.88 & \bf{811.07} & 
1.33 & 2.49\\CON8-4 & 794.52 & 1.10 & 
794.52 & 1.69 & \bf{772.25} & 
2.88 & 2.88\\CON8-5 & 771.20 & 1.98 & 
771.20 & 1.20 & \bf{754.88} & 
2.16 & 2.16\\CON8-6 & 689.52 & 1.47 & 
692.34 & 1.40 & \bf{678.92} & 
1.56 & 1.98\\CON8-7 & 816.00 & 1.16 & 
819.49 & 1.75 & \bf{811.96} & 
0.50 & 0.93\\CON8-8 & 787.24 & 1.65 & 
787.24 & 1.39 & \bf{767.53} & 
2.57 & 2.57\\CON8-9 & 824.93 & 0.97 & 
829.68 & 1.33 & \bf{809.00} & 
1.97 & 2.56\\\bf{PROM.} & 
\bf{767.72} & \bf{1.42} & \bf{770.10} & \bf{1.39} & \bf{758.54} & \bf{1.15} & \bf{1.45}\\[1ex]\hline
\end{tabular}
\label{table:nonlin}
\end{table} \clearpage
\begin{table}[ht]
\caption{Resultados de la ejecución de la metaheurística IGA, utilizando instancias de SalhiNagy con la configuración -n 300.0 -p 70.0 -cprob 90 -mprob 70}
\centering
\small
\begin{tabular}{c c c c c c c c}
\hline\hline
Instancia & Costo mínimo & Tiempo(seg.) & Costo promedio & Tiempo promedio(seg.) & CME & \%G & \%GP \\ [0.5ex]
\hline
CMT1X & 479.63 & 1.66 & 
481.02 & 1.31 & \bf{470.48} & 
1.94 & 2.24\\CMT1Y & 472.87 & 1.40 & 
476.23 & 1.47 & \bf{470.48} & 
0.51 & 1.22\\CMT2X & 700.40 & 3.18 & 
709.68 & 3.08 & \bf{682.39} & 
2.64 & 4.00\\CMT2Y & 709.31 & 2.96 & 
715.37 & 3.08 & \bf{682.39} & 
3.94 & 4.83\\CMT3X & 737.01 & 6.61 & 
745.13 & 6.31 & \bf{719.06} & 
2.50 & 3.63\\CMT3Y & 733.60 & 6.00 & 
736.36 & 6.12 & \bf{719.06} & 
2.02 & 2.41\\CMT4X & 907.97 & 16.15 & 
912.63 & 15.60 & \bf{854.21} & 
6.29 & 6.84\\CMT4Y & 909.09 & 15.80 & 
915.35 & 16.10 & \bf{852.46} & 
6.64 & 7.38\\CMT5X & 1116.61 & 30.41 & 
1124.08 & 31.48 & \bf{1030.56} & 
8.35 & 9.07\\CMT5Y & 1087.63 & 30.88 & 
1111.12 & 31.66 & \bf{1031.69} & 
5.42 & 7.70\\CMT11X & 890.21 & 10.58 & 
895.72 & 10.75 & \bf{831.09} & 
7.11 & 7.78\\CMT11Y & 858.09 & 11.25 & 
869.18 & 10.78 & \bf{829.85} & 
3.40 & 4.74\\CMT12X & 673.67 & 6.29 & 
676.25 & 6.36 & \bf{658.83} & 
2.25 & 2.64\\CMT12Y & 674.74 & 6.11 & 
676.50 & 6.35 & \bf{660.47} & 
2.16 & 2.43\\\bf{PROM.} & 
\bf{782.20} & \bf{10.66} & \bf{788.90} & \bf{10.75} & \bf{749.50} & \bf{3.94} & \bf{4.78}\\[1ex]\hline
\end{tabular}
\label{table:nonlin}
\end{table} \clearpage
\begin{table}[ht]
\caption{Resultados de la ejecución de la metaheurística PSO, utilizando instancias de SalhiNagy con la configuración -n 50.0 -L 30.0 -cp 1 -cg 0 -cl 1 -cn 2 -w1 0.9 -wt 0.1 -K 5}
\centering
\small
\begin{tabular}{c c c c c c c c}
\hline\hline
Instancia & Costo mínimo & Tiempo(seg.) & Costo promedio & Tiempo promedio(seg.) & CME & \%G & \%GP \\ [0.5ex]
\hline
CMT1X & 472.00 & 8.31 & 
473.59 & 10.38 & \bf{470.48} & 
0.32 & 0.66\\CMT1Y & \bf{470.48} & 9.46 & 
473.62 & 9.96 & 470.48 & 0.00
 & 0.67\\CMT2X & 692.08 & 10.28 & 
722.43 & 13.97 & \bf{682.39} & 
1.42 & 5.87\\CMT2Y & 699.62 & 20.42 & 
724.05 & 18.03 & \bf{682.39} & 
2.52 & 6.11\\CMT3X & 727.29 & 102.34 & 
739.85 & 113.30 & \bf{719.06} & 
1.14 & 2.89\\CMT3Y & 727.20 & 120.75 & 
745.43 & 116.22 & \bf{719.06} & 
1.13 & 3.67\\CMT4X & 876.33 & 284.29 & 
888.66 & 191.27 & \bf{854.21} & 
2.59 & 4.03\\CMT4Y & 931.03 & 191.28 & 
962.24 & 161.38 & \bf{852.46} & 
9.22 & 12.88\\CMT5X & 1108.60 & 200.89 & 
1125.96 & 202.88 & \bf{1030.56} & 
7.57 & 9.26\\CMT5Y & 1165.64 & 71.88 & 
1175.20 & 161.19 & \bf{1031.69} & 
12.98 & 13.91\\CMT11X & 883.80 & 72.28 & 
908.38 & 39.54 & \bf{831.09} & 
6.34 & 9.30\\CMT11Y & 886.21 & 63.42 & 
911.77 & 39.52 & \bf{829.85} & 
6.79 & 9.87\\CMT12X & \bf{\underline{621.91}} & 6.08 & 
731.07 & 7.75 & 658.83 & 
\bf{-5.60} & 10.97\\CMT12Y & 698.33 & 6.92 & 
783.83 & 7.28 & \bf{660.47} & 
5.73 & 18.68\\\bf{PROM.} & 
\bf{782.89} & \bf{83.47} & \bf{811.86} & \bf{78.05} & \bf{749.50} & \bf{3.73} & \bf{7.77}\\[1ex]\hline
\end{tabular}
\label{table:nonlin}
\end{table} \clearpage
\begin{table}[ht]
\caption{Resultados de la ejecución de la metaheurística PSO, utilizando instancias de SalhiNagy con la configuración -n 50.0 -L 50.0 -cp 1 -cg 0 -cl 1 -cn 2 -w1 0.9 -wt 0.1 -K 5}
\centering
\small
\begin{tabular}{c c c c c c c c}
\hline\hline
Instancia & Costo mínimo & Tiempo(seg.) & Costo promedio & Tiempo promedio(seg.) & CME & \%G & \%GP \\ [0.5ex]
\hline
CMT1X & 472.37 & 21.12 & 
484.52 & 17.05 & \bf{470.48} & 
0.40 & 2.98\\CMT1Y & 470.67 & 16.30 & 
470.77 & 16.73 & \bf{470.48} & 
0.04 & 0.06\\CMT2X & 701.89 & 16.74 & 
716.11 & 24.46 & \bf{682.39} & 
2.86 & 4.94\\CMT2Y & 701.58 & 29.87 & 
718.01 & 23.63 & \bf{682.39} & 
2.81 & 5.22\\CMT3X & 728.25 & 191.40 & 
734.84 & 189.03 & \bf{719.06} & 
1.28 & 2.19\\CMT3Y & 724.41 & 168.99 & 
738.40 & 175.56 & \bf{719.06} & 
0.74 & 2.69\\CMT4X & 898.20 & 332.90 & 
903.16 & 356.10 & \bf{854.21} & 
5.15 & 5.73\\CMT4Y & 869.03 & 314.93 & 
894.32 & 373.69 & \bf{852.46} & 
1.94 & 4.91\\CMT5X & 1090.76 & 283.06 & 
1107.34 & 361.85 & \bf{1030.56} & 
5.84 & 7.45\\CMT5Y & 1128.36 & 714.57 & 
1172.53 & 391.55 & \bf{1031.69} & 
9.37 & 13.65\\CMT11X & 883.64 & 82.71 & 
894.26 & 75.32 & \bf{831.09} & 
6.32 & 7.60\\CMT11Y & 889.98 & 105.52 & 
903.02 & 95.60 & \bf{829.85} & 
7.25 & 8.82\\CMT12X & 728.57 & 12.34 & 
800.89 & 12.22 & \bf{658.83} & 
10.59 & 21.56\\CMT12Y & 740.50 & 11.56 & 
766.11 & 15.29 & \bf{660.47} & 
12.12 & 15.99\\\bf{PROM.} & 
\bf{787.73} & \bf{164.43} & \bf{807.45} & \bf{152.01} & \bf{749.50} & \bf{4.77} & \bf{7.41}\\[1ex]\hline
\end{tabular}
\label{table:nonlin}
\end{table} \clearpage
\begin{table}[ht]
\caption{Resultados de la ejecución de la metaheurística PSO, utilizando instancias de SalhiNagy con la configuración -n 50.0 -L 70.0 -cp 1 -cg 0 -cl 1 -cn 2 -w1 0.9 -wt 0.1 -K 5}
\centering
\small
\begin{tabular}{c c c c c c c c}
\hline\hline
Instancia & Costo mínimo & Tiempo(seg.) & Costo promedio & Tiempo promedio(seg.) & CME & \%G & \%GP \\ [0.5ex]
\hline
CMT1X & \bf{470.48} & 25.08 & 
480.46 & 21.53 & 470.48 & 0.00
 & 2.12\\CMT1Y & \bf{470.48} & 25.56 & 
474.43 & 23.50 & 470.48 & 0.00
 & 0.84\\CMT2X & 697.66 & 36.95 & 
738.80 & 30.49 & \bf{682.39} & 
2.24 & 8.27\\CMT2Y & 693.09 & 33.55 & 
722.19 & 34.14 & \bf{682.39} & 
1.57 & 5.83\\CMT3X & 726.69 & 265.51 & 
734.28 & 248.63 & \bf{719.06} & 
1.06 & 2.12\\CMT3Y & 723.67 & 276.74 & 
725.42 & 254.86 & \bf{719.06} & 
0.64 & 0.88\\CMT4X & 887.94 & 499.27 & 
923.70 & 582.74 & \bf{854.21} & 
3.95 & 8.13\\CMT4Y & 890.80 & 446.18 & 
916.01 & 434.59 & \bf{852.46} & 
4.50 & 7.46\\CMT5X & 1110.31 & 409.61 & 
1137.31 & 639.87 & \bf{1030.56} & 
7.74 & 10.36\\CMT5Y & 1113.44 & 436.73 & 
1154.46 & 595.27 & \bf{1031.69} & 
7.92 & 11.90\\CMT11X & 892.08 & 121.51 & 
902.98 & 109.05 & \bf{831.09} & 
7.34 & 8.65\\CMT11Y & 892.84 & 206.20 & 
918.06 & 137.08 & \bf{829.85} & 
7.59 & 10.63\\CMT12X & 724.99 & 17.65 & 
760.93 & 20.41 & \bf{658.83} & 
10.04 & 15.50\\CMT12Y & 685.55 & 17.41 & 
711.46 & 20.79 & \bf{660.47} & 
3.80 & 7.72\\\bf{PROM.} & 
\bf{784.29} & \bf{201.28} & \bf{807.18} & \bf{225.21} & \bf{749.50} & \bf{4.17} & \bf{7.17}\\[1ex]\hline
\end{tabular}
\label{table:nonlin}
\end{table} \clearpage
\begin{table}[ht]
\caption{Resultados de la ejecución de la metaheurística PSO, utilizando instancias de SalhiNagy con la configuración -n 50.0 -L 90.0 -cp 1 -cg 0 -cl 1 -cn 2 -w1 0.9 -wt 0.1 -K 5}
\centering
\small
\begin{tabular}{c c c c c c c c}
\hline\hline
Instancia & Costo mínimo & Tiempo(seg.) & Costo promedio & Tiempo promedio(seg.) & CME & \%G & \%GP \\ [0.5ex]
\hline
CMT1X & 470.67 & 30.64 & 
470.81 & 29.41 & \bf{470.48} & 
0.04 & 0.07\\CMT1Y & 470.67 & 32.52 & 
472.43 & 30.61 & \bf{470.48} & 
0.04 & 0.42\\CMT2X & 705.76 & 40.29 & 
725.66 & 45.37 & \bf{682.39} & 
3.42 & 6.34\\CMT2Y & 696.81 & 57.52 & 
709.27 & 47.53 & \bf{682.39} & 
2.11 & 3.94\\CMT3X & 723.86 & 307.33 & 
733.88 & 324.24 & \bf{719.06} & 
0.67 & 2.06\\CMT3Y & 724.65 & 324.88 & 
733.63 & 332.69 & \bf{719.06} & 
0.78 & 2.03\\CMT4X & 895.68 & 845.11 & 
908.20 & 655.39 & \bf{854.21} & 
4.85 & 6.32\\CMT4Y & 889.21 & 593.83 & 
915.71 & 556.02 & \bf{852.46} & 
4.31 & 7.42\\CMT5X & 1112.62 & 473.48 & 
1142.06 & 510.50 & \bf{1030.56} & 
7.96 & 10.82\\CMT5Y & 1103.38 & 709.64 & 
1141.14 & 713.62 & \bf{1031.69} & 
6.95 & 10.61\\CMT11X & 886.73 & 127.69 & 
907.00 & 152.12 & \bf{831.09} & 
6.69 & 9.13\\CMT11Y & 900.70 & 107.03 & 
908.28 & 150.45 & \bf{829.85} & 
8.54 & 9.45\\CMT12X & 667.54 & 22.20 & 
722.81 & 26.94 & \bf{658.83} & 
1.32 & 9.71\\CMT12Y & 708.70 & 23.64 & 
768.78 & 23.86 & \bf{660.47} & 
7.30 & 16.40\\\bf{PROM.} & 
\bf{782.64} & \bf{263.99} & \bf{804.26} & \bf{257.05} & \bf{749.50} & \bf{3.93} & \bf{6.77}\\[1ex]\hline
\end{tabular}
\label{table:nonlin}
\end{table} \clearpage
\begin{table}[ht]
\caption{Resultados de la ejecución de la metaheurística PSO, utilizando instancias de SalhiNagy con la configuración -n 10 -L 30 -cp 1 -cg 0 -cl 1 -cn 2 -w1 0.9 -wt 0.1 -K 5}
\centering
\small
\begin{tabular}{c c c c c c c c}
\hline\hline
Instancia & Costo mínimo & Tiempo(seg.) & Costo promedio & Tiempo promedio(seg.) & CME & \%G & \%GP \\ [0.5ex]
\hline
CMT1X & 472.87 & 2.14 & 
473.80 & 2.02 & \bf{470.48} & 
0.51 & 0.70\\CMT1Y & 475.58 & 2.11 & 
477.61 & 2.00 & \bf{470.48} & 
1.08 & 1.52\\CMT2X & 714.41 & 3.97 & 
715.41 & 3.31 & \bf{682.39} & 
4.69 & 4.84\\CMT2Y & 699.31 & 3.27 & 
731.61 & 2.88 & \bf{682.39} & 
2.48 & 7.21\\CMT3X & 740.61 & 24.60 & 
744.02 & 25.36 & \bf{719.06} & 
3.00 & 3.47\\CMT3Y & 735.37 & 25.88 & 
736.10 & 25.82 & \bf{719.06} & 
2.27 & 2.37\\CMT4X & 916.81 & 29.12 & 
944.75 & 24.30 & \bf{854.21} & 
7.33 & 10.60\\CMT4Y & 924.82 & 35.46 & 
930.25 & 31.14 & \bf{852.46} & 
8.49 & 9.12\\CMT5X & 1124.93 & 14.32 & 
1136.32 & 47.59 & \bf{1030.56} & 
9.16 & 10.26\\CMT5Y & 1134.93 & 31.87 & 
1156.65 & 27.30 & \bf{1031.69} & 
10.01 & 12.11\\CMT11X & 903.37 & 11.25 & 
919.01 & 7.62 & \bf{831.09} & 
8.70 & 10.58\\CMT11Y & 915.88 & 5.34 & 
921.73 & 6.26 & \bf{829.85} & 
10.37 & 11.07\\CMT12X & \bf{\underline{654.31}} & 1.58 & 
687.47 & 1.63 & 658.83 & 
\bf{-0.69} & 4.35\\CMT12Y & 751.70 & 1.76 & 
764.23 & 2.25 & \bf{660.47} & 
13.81 & 15.71\\\bf{PROM.} & 
\bf{797.49} & \bf{13.76} & \bf{809.92} & \bf{14.96} & \bf{749.50} & \bf{5.80} & \bf{7.42}\\[1ex]\hline
\end{tabular}
\label{table:nonlin}
\end{table} \clearpage
\begin{table}[ht]
\caption{Resultados de la ejecución de la metaheurística PSO, utilizando instancias de SalhiNagy con la configuración -n 30 -L 10 -cp 1 -cg 0 -cl 1 -cn 2 -w1 0.9 -wt 0.1 -K 5}
\centering
\small
\begin{tabular}{c c c c c c c c}
\hline\hline
Instancia & Costo mínimo & Tiempo(seg.) & Costo promedio & Tiempo promedio(seg.) & CME & \%G & \%GP \\ [0.5ex]
\hline
CMT1X & 478.38 & 1.84 & 
478.46 & 1.75 & \bf{470.48} & 
1.68 & 1.70\\CMT1Y & 470.67 & 1.05 & 
472.96 & 1.64 & \bf{470.48} & 
0.04 & 0.53\\CMT2X & 772.99 & 3.40 & 
776.18 & 2.48 & \bf{682.39} & 
13.28 & 13.74\\CMT2Y & 735.44 & 1.34 & 
743.54 & 2.25 & \bf{682.39} & 
7.77 & 8.96\\CMT3X & 752.58 & 25.06 & 
756.50 & 22.62 & \bf{719.06} & 
4.66 & 5.21\\CMT3Y & 723.23 & 15.43 & 
736.86 & 18.38 & \bf{719.06} & 
0.58 & 2.48\\CMT4X & 887.84 & 47.43 & 
911.34 & 27.93 & \bf{854.21} & 
3.94 & 6.69\\CMT4Y & 930.42 & 34.81 & 
935.70 & 33.02 & \bf{852.46} & 
9.15 & 9.76\\CMT5X & 1145.34 & 33.75 & 
1161.42 & 46.63 & \bf{1030.56} & 
11.14 & 12.70\\CMT5Y & 1138.30 & 13.16 & 
1173.05 & 18.71 & \bf{1031.69} & 
10.33 & 13.70\\CMT11X & 888.53 & 9.04 & 
959.11 & 5.96 & \bf{831.09} & 
6.91 & 15.40\\CMT11Y & 935.63 & 11.82 & 
951.13 & 7.83 & \bf{829.85} & 
12.75 & 14.61\\CMT12X & 784.40 & 2.17 & 
826.93 & 1.92 & \bf{658.83} & 
19.06 & 25.52\\CMT12Y & 747.34 & 2.22 & 
779.37 & 1.89 & \bf{660.47} & 
13.15 & 18.00\\\bf{PROM.} & 
\bf{813.65} & \bf{14.47} & \bf{833.04} & \bf{13.79} & \bf{749.50} & \bf{8.17} & \bf{10.64}\\[1ex]\hline
\end{tabular}
\label{table:nonlin}
\end{table} \clearpage
\begin{table}[ht]
\caption{Resultados de la ejecución de la metaheurística PSO, utilizando instancias de SalhiNagy con la configuración -n 10.0 -L 10.0 -cp 1 -cg 0 -cl 1 -cn 2 -w1 0.9 -wt 0.1 -K 5}
\centering
\small
\begin{tabular}{c c c c c c c c}
\hline\hline
Instancia & Costo mínimo & Tiempo(seg.) & Costo promedio & Tiempo promedio(seg.) & CME & \%G & \%GP \\ [0.5ex]
\hline
CMT1X & 481.04 & 0.76 & 
517.33 & 0.85 & \bf{470.48} & 
2.24 & 9.96\\CMT1Y & 477.88 & 0.81 & 
485.00 & 0.83 & \bf{470.48} & 
1.57 & 3.09\\CMT2X & 724.06 & 1.05 & 
727.00 & 0.86 & \bf{682.39} & 
6.11 & 6.54\\CMT2Y & 709.09 & 0.84 & 
722.23 & 0.81 & \bf{682.39} & 
3.91 & 5.84\\CMT3X & 737.59 & 6.88 & 
754.25 & 7.72 & \bf{719.06} & 
2.58 & 4.89\\CMT3Y & 740.63 & 9.05 & 
742.17 & 7.86 & \bf{719.06} & 
3.00 & 3.21\\CMT4X & 902.33 & 10.26 & 
909.72 & 8.26 & \bf{854.21} & 
5.63 & 6.50\\CMT4Y & 910.05 & 16.72 & 
936.00 & 12.19 & \bf{852.46} & 
6.76 & 9.80\\CMT5X & 1151.82 & 21.43 & 
1159.60 & 14.20 & \bf{1030.56} & 
11.77 & 12.52\\CMT5Y & 1175.20 & 8.04 & 
1203.55 & 8.44 & \bf{1031.69} & 
13.91 & 16.66\\CMT11X & 919.63 & 3.31 & 
962.99 & 2.92 & \bf{831.09} & 
10.65 & 15.87\\CMT11Y & 951.14 & 1.91 & 
961.04 & 1.91 & \bf{829.85} & 
14.62 & 15.81\\CMT12X & 752.36 & 1.05 & 
941.35 & 0.78 & \bf{658.83} & 
14.20 & 42.88\\CMT12Y & 1127.37 & 0.59 & 
1127.37 & 0.56 & \bf{660.47} & 
70.69 & 70.69\\\bf{PROM.} & 
\bf{840.01} & \bf{5.91} & \bf{867.83} & \bf{4.87} & \bf{749.50} & \bf{11.97} & \bf{16.02}\\[1ex]\hline
\end{tabular}
\label{table:nonlin}
\end{table} \clearpage
\begin{table}[ht]
\caption{Resultados de la ejecución de la metaheurística PSO, utilizando instancias de SalhiNagy con la configuración -n 10.0 -L 10.0 -cp 1 -cg 0 -cl 1 -cn 2 -w1 0.9 -wt 0.1 -K 5}
\centering
\small
\begin{tabular}{c c c c c c c c}
\hline\hline
Instancia & Costo mínimo & Tiempo(seg.) & Costo promedio & Tiempo promedio(seg.) & CME & \%G & \%GP \\ [0.5ex]
\hline
CMT1X & 475.35 & 0.55 & 
478.83 & 0.55 & \bf{470.48} & 
1.04 & 1.77\\CMT1Y & 474.70 & 0.84 & 
475.73 & 0.83 & \bf{470.48} & 
0.90 & 1.11\\CMT2X & 722.41 & 0.51 & 
728.73 & 0.85 & \bf{682.39} & 
5.86 & 6.79\\CMT2Y & 719.83 & 1.06 & 
720.38 & 1.23 & \bf{682.39} & 
5.49 & 5.57\\CMT3X & 738.73 & 6.16 & 
738.86 & 6.78 & \bf{719.06} & 
2.74 & 2.75\\CMT3Y & 743.96 & 8.98 & 
757.29 & 8.11 & \bf{719.06} & 
3.46 & 5.32\\CMT4X & 908.59 & 10.83 & 
909.48 & 13.15 & \bf{854.21} & 
6.37 & 6.47\\CMT4Y & 900.50 & 10.93 & 
906.30 & 12.97 & \bf{852.46} & 
5.64 & 6.32\\CMT5X & 1092.93 & 32.14 & 
1138.10 & 38.85 & \bf{1030.56} & 
6.05 & 10.44\\CMT5Y & 1158.65 & 13.44 & 
1160.84 & 14.86 & \bf{1031.69} & 
12.31 & 12.52\\CMT11X & 943.91 & 1.82 & 
981.16 & 1.49 & \bf{831.09} & 
13.57 & 18.06\\CMT11Y & 932.84 & 2.06 & 
1006.66 & 1.54 & \bf{829.85} & 
12.41 & 21.31\\CMT12X & 732.20 & 0.92 & 
746.05 & 0.76 & \bf{658.83} & 
11.14 & 13.24\\CMT12Y & 798.11 & 0.57 & 
798.11 & 0.55 & \bf{660.47} & 
20.84 & 20.84\\\bf{PROM.} & 
\bf{810.19} & \bf{6.49} & \bf{824.75} & \bf{7.32} & \bf{749.50} & \bf{7.70} & \bf{9.46}\\[1ex]\hline
\end{tabular}
\label{table:nonlin}
\end{table} \clearpage
\begin{table}[ht]
\caption{Resultados de la ejecución de la metaheurística PSO, utilizando instancias de SalhiNagy con la configuración -n 10.0 -L 30.0 -cp 1 -cg 0 -cl 1 -cn 2 -w1 0.9 -wt 0.1 -K 5}
\centering
\small
\begin{tabular}{c c c c c c c c}
\hline\hline
Instancia & Costo mínimo & Tiempo(seg.) & Costo promedio & Tiempo promedio(seg.) & CME & \%G & \%GP \\ [0.5ex]
\hline
CMT1X & 474.41 & 1.80 & 
476.42 & 1.89 & \bf{470.48} & 
0.84 & 1.26\\CMT1Y & 477.21 & 1.98 & 
482.05 & 2.18 & \bf{470.48} & 
1.43 & 2.46\\CMT2X & 706.14 & 3.98 & 
734.48 & 3.13 & \bf{682.39} & 
3.48 & 7.63\\CMT2Y & 694.50 & 2.42 & 
707.60 & 2.58 & \bf{682.39} & 
1.77 & 3.69\\CMT3X & 733.20 & 25.64 & 
737.93 & 25.25 & \bf{719.06} & 
1.97 & 2.62\\CMT3Y & 728.71 & 26.41 & 
736.84 & 26.38 & \bf{719.06} & 
1.34 & 2.47\\CMT4X & 935.39 & 27.35 & 
951.11 & 31.12 & \bf{854.21} & 
9.50 & 11.34\\CMT4Y & 955.59 & 27.20 & 
958.10 & 39.60 & \bf{852.46} & 
12.10 & 12.39\\CMT5X & 1130.51 & 43.06 & 
1140.12 & 36.24 & \bf{1030.56} & 
9.70 & 10.63\\CMT5Y & 1168.08 & 19.64 & 
1180.84 & 32.60 & \bf{1031.69} & 
13.22 & 14.46\\CMT11X & 910.23 & 4.01 & 
940.16 & 5.29 & \bf{831.09} & 
9.52 & 13.12\\CMT11Y & 916.08 & 4.31 & 
920.65 & 5.57 & \bf{829.85} & 
10.39 & 10.94\\CMT12X & 775.11 & 1.99 & 
782.45 & 2.64 & \bf{658.83} & 
17.65 & 18.76\\CMT12Y & 773.76 & 1.92 & 
831.21 & 2.56 & \bf{660.47} & 
17.15 & 25.85\\\bf{PROM.} & 
\bf{812.78} & \bf{13.69} & \bf{827.14} & \bf{15.50} & \bf{749.50} & \bf{7.86} & \bf{9.83}\\[1ex]\hline
\end{tabular}
\label{table:nonlin}
\end{table} \clearpage
\begin{table}[ht]
\caption{Resultados de la ejecución de la metaheurística PSO, utilizando instancias de SalhiNagy con la configuración -n 10.0 -L 50.0 -cp 1 -cg 0 -cl 1 -cn 2 -w1 0.9 -wt 0.1 -K 5}
\centering
\small
\begin{tabular}{c c c c c c c c}
\hline\hline
Instancia & Costo mínimo & Tiempo(seg.) & Costo promedio & Tiempo promedio(seg.) & CME & \%G & \%GP \\ [0.5ex]
\hline
CMT1X & 470.67 & 3.31 & 
470.96 & 3.21 & \bf{470.48} & 
0.04 & 0.10\\CMT1Y & 476.51 & 3.80 & 
478.17 & 3.60 & \bf{470.48} & 
1.28 & 1.63\\CMT2X & 709.90 & 4.63 & 
751.30 & 5.25 & \bf{682.39} & 
4.03 & 10.10\\CMT2Y & 707.96 & 5.06 & 
749.18 & 4.53 & \bf{682.39} & 
3.75 & 9.79\\CMT3X & 744.85 & 41.54 & 
748.92 & 40.95 & \bf{719.06} & 
3.59 & 4.15\\CMT3Y & 735.62 & 46.49 & 
736.10 & 42.53 & \bf{719.06} & 
2.30 & 2.37\\CMT4X & 894.65 & 56.73 & 
914.04 & 62.83 & \bf{854.21} & 
4.73 & 7.00\\CMT4Y & 903.49 & 69.63 & 
931.51 & 75.05 & \bf{852.46} & 
5.99 & 9.27\\CMT5X & 1095.96 & 69.39 & 
1118.35 & 34.70 & \bf{1030.56} & 
6.35 & 8.52\\CMT5Y & 1138.83 & 80.81 & 
1142.50 & 78.66 & \bf{1031.69} & 
10.38 & 10.74\\CMT11X & 914.63 & 12.26 & 
931.06 & 13.52 & \bf{831.09} & 
10.05 & 12.03\\CMT11Y & 908.78 & 13.28 & 
914.17 & 11.28 & \bf{829.85} & 
9.51 & 10.16\\CMT12X & 701.26 & 3.60 & 
737.68 & 3.48 & \bf{658.83} & 
6.44 & 11.97\\CMT12Y & 769.18 & 3.36 & 
794.11 & 3.29 & \bf{660.47} & 
16.46 & 20.23\\\bf{PROM.} & 
\bf{798.02} & \bf{29.56} & \bf{815.58} & \bf{27.35} & \bf{749.50} & \bf{6.06} & \bf{8.43}\\[1ex]\hline
\end{tabular}
\label{table:nonlin}
\end{table} \clearpage
\begin{table}[ht]
\caption{Resultados de la ejecución de la metaheurística PSO, utilizando instancias de SalhiNagy con la configuración -n 10.0 -L 70.0 -cp 1 -cg 0 -cl 1 -cn 2 -w1 0.9 -wt 0.1 -K 5}
\centering
\small
\begin{tabular}{c c c c c c c c}
\hline\hline
Instancia & Costo mínimo & Tiempo(seg.) & Costo promedio & Tiempo promedio(seg.) & CME & \%G & \%GP \\ [0.5ex]
\hline
CMT1X & 470.67 & 5.39 & 
472.05 & 5.34 & \bf{470.48} & 
0.04 & 0.33\\CMT1Y & 472.37 & 5.50 & 
472.62 & 5.70 & \bf{470.48} & 
0.40 & 0.45\\CMT2X & 709.80 & 6.60 & 
710.29 & 6.55 & \bf{682.39} & 
4.02 & 4.09\\CMT2Y & 721.62 & 5.44 & 
747.51 & 6.20 & \bf{682.39} & 
5.75 & 9.54\\CMT3X & 730.41 & 63.17 & 
732.65 & 60.74 & \bf{719.06} & 
1.58 & 1.89\\CMT3Y & 735.06 & 59.16 & 
739.25 & 59.55 & \bf{719.06} & 
2.23 & 2.81\\CMT4X & 884.08 & 84.86 & 
905.39 & 77.67 & \bf{854.21} & 
3.50 & 5.99\\CMT4Y & 100000 & 0 & 
888.73 & 92.65 & \bf{852.46} & 
11630.76 & 4.25\\CMT5X & 1123.18 & 69.03 & 
1124.80 & 53.33 & \bf{1030.56} & 
8.99 & 9.14\\CMT5Y & 1153.22 & 89.56 & 
1155.42 & 81.66 & \bf{1031.69} & 
11.78 & 11.99\\CMT11X & 918.67 & 11.93 & 
923.12 & 17.33 & \bf{831.09} & 
10.54 & 11.07\\CMT11Y & 913.68 & 16.32 & 
967.60 & 13.67 & \bf{829.85} & 
10.10 & 16.60\\CMT12X & 784.39 & 4.40 & 
792.32 & 4.41 & \bf{658.83} & 
19.06 & 20.26\\CMT12Y & 816.05 & 4.71 & 
835.10 & 5.95 & \bf{660.47} & 
23.56 & 26.44\\\bf{PROM.} & 
\bf{7888.09} & \bf{30.43} & \bf{819.06} & \bf{35.05} & \bf{749.50} & \bf{838.02} & \bf{8.92}\\[1ex]\hline
\end{tabular}
\label{table:nonlin}
\end{table} \clearpage
\begin{table}[ht]
\caption{Resultados de la ejecución de la metaheurística ACO, utilizando instancias de Dethloff con la configuración -n 5.0 -alpha 1.0 -beta 3.0 -q 0.1 -ro 0.015}
\centering
\small
\begin{tabular}{c c c c c c c c}
\hline\hline
Instancia & Costo mínimo & Tiempo(seg.) & Costo promedio & Tiempo promedio(seg.) & CME & \%G & \%GP \\ [0.5ex]
\hline
SCA3-0 & 636.06 & 1.21 & 
636.23 & 1.26 & \bf{635.62} & 
0.07 & 0.10\\SCA3-1 & \bf{697.84} & 1.35 & 
697.84 & 1.40 & 697.84 & 0.00
 & 0.00\\
SCA3-2 & \bf{659.34} & 1.33 & 
661.75 & 1.25 & 659.34 & 0.00
 & 0.37\\SCA3-3 & \bf{680.04} & 1.26 & 
680.08 & 1.24 & 680.04 & 0.00
 & 0.01\\SCA3-4 & \bf{690.50} & 1.34 & 
690.50 & 1.34 & 690.50 & 0.00
 & 0.00\\
SCA3-5 & \bf{659.90} & 1.30 & 
663.62 & 1.34 & 659.90 & 0.00
 & 0.56\\SCA3-6 & \bf{651.09} & 1.32 & 
653.03 & 1.30 & 651.09 & 0.00
 & 0.30\\SCA3-7 & \bf{659.17} & 1.26 & 
666.01 & 1.18 & 659.17 & 0.00
 & 1.04\\SCA3-8 & \bf{719.47} & 1.29 & 
720.34 & 1.30 & 719.47 & 0.00
 & 0.12\\SCA3-9 & \bf{681.00} & 1.15 & 
681.07 & 1.11 & 681.00 & 0.00
 & 0.01\\SCA8-0 & \bf{961.50} & 1.32 & 
979.44 & 1.36 & 961.50 & 0.00
 & 1.87\\SCA8-1 & 1052.71 & 1.20 & 
1060.59 & 1.19 & \bf{1049.65} & 
0.29 & 1.04\\SCA8-2 & 1044.24 & 1.13 & 
1049.75 & 1.07 & \bf{1039.64} & 
0.44 & 0.97\\SCA8-3 & 985.47 & 1.28 & 
1004.04 & 1.26 & \bf{983.34} & 
0.22 & 2.11\\SCA8-4 & \bf{1065.49} & 1.36 & 
1069.27 & 1.32 & 1065.49 & 0.00
 & 0.35\\SCA8-5 & 1034.74 & 1.44 & 
1049.81 & 1.46 & \bf{1027.08} & 
0.75 & 2.21\\SCA8-6 & 972.48 & 1.44 & 
978.84 & 1.43 & \bf{971.82} & 
0.07 & 0.72\\SCA8-7 & 1066.65 & 1.36 & 
1069.62 & 1.38 & \bf{1051.28} & 
1.46 & 1.74\\SCA8-8 & \bf{1071.18} & 1.48 & 
1075.89 & 1.45 & 1071.18 & 0.00
 & 0.44\\SCA8-9 & 1065.60 & 1.20 & 
1067.40 & 1.16 & \bf{1060.50} & 
0.48 & 0.65\\CON3-0 & \bf{616.52} & 1.46 & 
622.14 & 1.42 & 616.52 & 0.00
 & 0.91\\CON3-1 & \bf{554.47} & 1.38 & 
557.09 & 1.38 & 554.47 & 0.00
 & 0.47\\CON3-2 & 519.11 & 1.31 & 
521.03 & 1.29 & \bf{518.00} & 
0.21 & 0.59\\CON3-3 & \bf{591.19} & 1.40 & 
591.24 & 1.44 & 591.19 & 0.00
 & 0.01\\CON3-4 & \bf{588.79} & 1.27 & 
590.28 & 1.22 & 588.79 & 0.00
 & 0.25\\CON3-5 & \bf{563.70} & 1.27 & 
566.64 & 1.32 & 563.70 & 0.00
 & 0.52\\CON3-6 & 499.07 & 1.56 & 
502.31 & 1.51 & \bf{499.05} & 
0.00 & 0.65\\CON3-7 & \bf{576.48} & 1.24 & 
578.59 & 1.21 & 576.48 & 0.00
 & 0.37\\CON3-8 & \bf{523.05} & 1.28 & 
523.61 & 1.30 & 523.05 & 0.00
 & 0.11\\CON3-9 & 578.25 & 1.32 & 
585.38 & 1.29 & \bf{578.24} & 
0.00 & 1.24\\CON8-0 & 859.74 & 1.28 & 
872.98 & 1.32 & \bf{857.17} & 
0.30 & 1.84\\CON8-1 & \bf{740.85} & 1.41 & 
744.43 & 1.38 & 740.85 & 0.00
 & 0.48\\CON8-2 & \bf{712.89} & 1.63 & 
715.34 & 1.57 & 712.89 & 0.00
 & 0.34\\CON8-3 & \bf{811.07} & 1.38 & 
816.57 & 1.37 & 811.07 & 0.00
 & 0.68\\CON8-4 & 772.76 & 1.30 & 
780.95 & 1.25 & \bf{772.25} & 
0.07 & 1.13\\CON8-5 & 754.95 & 1.31 & 
760.71 & 1.33 & \bf{754.88} & 
0.01 & 0.77\\CON8-6 & 684.05 & 1.46 & 
694.02 & 1.54 & \bf{678.92} & 
0.76 & 2.22\\CON8-7 & 814.50 & 1.22 & 
816.40 & 1.23 & \bf{811.96} & 
0.31 & 0.55\\CON8-8 & 778.39 & 1.46 & 
785.59 & 1.53 & \bf{767.53} & 
1.41 & 2.35\\CON8-9 & 810.18 & 1.41 & 
813.93 & 1.45 & \bf{809.00} & 
0.15 & 0.61\\\bf{PROM.} & 
\bf{760.11} & \bf{1.33} & \bf{764.86} & \bf{1.33} & \bf{758.54} & \bf{0.18} & \bf{0.77}\\[1ex]\hline
\end{tabular}
\label{table:nonlin}
\end{table} \clearpage
\begin{table}[ht]
\caption{Resultados de la ejecución de la metaheurística PSO, utilizando instancias de SalhiNagy con la configuración -n 10.0 -L 90.0 -cp 1 -cg 0 -cl 1 -cn 2 -w1 0.9 -wt 0.1 -K 5}
\centering
\small
\begin{tabular}{c c c c c c c c}
\hline\hline
Instancia & Costo mínimo & Tiempo(seg.) & Costo promedio & Tiempo promedio(seg.) & CME & \%G & \%GP \\ [0.5ex]
\hline
CMT1X & \bf{470.48} & 6.22 & 
479.21 & 6.51 & 470.48 & 0.00
 & 1.86\\CMT1Y & 472.37 & 6.30 & 
484.45 & 6.55 & \bf{470.48} & 
0.40 & 2.97\\CMT2X & 715.54 & 8.80 & 
729.17 & 8.81 & \bf{682.39} & 
4.86 & 6.86\\CMT2Y & 726.07 & 10.33 & 
734.93 & 8.86 & \bf{682.39} & 
6.40 & 7.70\\CMT3X & 723.86 & 76.96 & 
726.55 & 76.88 & \bf{719.06} & 
0.67 & 1.04\\CMT3Y & 733.71 & 74.21 & 
742.61 & 77.59 & \bf{719.06} & 
2.04 & 3.28\\CMT4X & 904.10 & 103.19 & 
917.59 & 94.97 & \bf{854.21} & 
5.84 & 7.42\\CMT4Y & 898.59 & 93.94 & 
937.05 & 97.83 & \bf{852.46} & 
5.41 & 9.92\\CMT5X & 1158.73 & 69.98 & 
1164.42 & 109.78 & \bf{1030.56} & 
12.44 & 12.99\\CMT5Y & 1147.76 & 111.20 & 
1153.11 & 96.53 & \bf{1031.69} & 
11.25 & 11.77\\CMT11X & 907.53 & 14.79 & 
911.03 & 14.99 & \bf{831.09} & 
9.20 & 9.62\\CMT11Y & 903.36 & 15.34 & 
920.98 & 19.52 & \bf{829.85} & 
8.86 & 10.98\\CMT12X & 711.75 & 6.24 & 
730.71 & 6.32 & \bf{658.83} & 
8.03 & 10.91\\CMT12Y & 854.42 & 8.96 & 
888.57 & 7.82 & \bf{660.47} & 
29.37 & 34.54\\\bf{PROM.} & 
\bf{809.16} & \bf{43.32} & \bf{822.88} & \bf{45.21} & \bf{749.50} & \bf{7.48} & \bf{9.42}\\[1ex]\hline
\end{tabular}
\label{table:nonlin}
\end{table} \clearpage
\begin{table}[ht]
\caption{Resultados de la ejecución de la metaheurística PSO, utilizando instancias de SalhiNagy con la configuración -n 30.0 -L 10.0 -cp 1 -cg 0 -cl 1 -cn 2 -w1 0.9 -wt 0.1 -K 5}
\centering
\small
\begin{tabular}{c c c c c c c c}
\hline\hline
Instancia & Costo mínimo & Tiempo(seg.) & Costo promedio & Tiempo promedio(seg.) & CME & \%G & \%GP \\ [0.5ex]
\hline
CMT1X & 472.87 & 1.82 & 
475.80 & 1.86 & \bf{470.48} & 
0.51 & 1.13\\CMT1Y & 495.40 & 1.31 & 
513.36 & 1.93 & \bf{470.48} & 
5.30 & 9.11\\CMT2X & 719.51 & 1.84 & 
728.50 & 3.12 & \bf{682.39} & 
5.44 & 6.76\\CMT2Y & 773.70 & 3.80 & 
787.73 & 3.23 & \bf{682.39} & 
13.38 & 15.44\\CMT3X & 737.86 & 23.18 & 
743.43 & 26.66 & \bf{719.06} & 
2.61 & 3.39\\CMT3Y & 731.93 & 30.18 & 
744.86 & 26.04 & \bf{719.06} & 
1.79 & 3.59\\CMT4X & 917.13 & 28.78 & 
934.46 & 29.30 & \bf{854.21} & 
7.37 & 9.39\\CMT4Y & 908.64 & 20.27 & 
932.59 & 16.29 & \bf{852.46} & 
6.59 & 9.40\\CMT5X & 1165.70 & 160.05 & 
1168.04 & 87.94 & \bf{1030.56} & 
13.11 & 13.34\\CMT5Y & 1131.63 & 15.73 & 
1159.22 & 38.98 & \bf{1031.69} & 
9.69 & 12.36\\CMT11X & 931.33 & 5.15 & 
941.22 & 4.49 & \bf{831.09} & 
12.06 & 13.25\\CMT11Y & 898.85 & 7.12 & 
899.40 & 8.98 & \bf{829.85} & 
8.31 & 8.38\\CMT12X & 722.55 & 2.16 & 
751.77 & 2.17 & \bf{658.83} & 
9.67 & 14.11\\CMT12Y & 798.30 & 1.41 & 
799.28 & 1.40 & \bf{660.47} & 
20.87 & 21.02\\\bf{PROM.} & 
\bf{814.67} & \bf{21.63} & \bf{827.12} & \bf{18.03} & \bf{749.50} & \bf{8.34} & \bf{10.05}\\[1ex]\hline
\end{tabular}
\label{table:nonlin}
\end{table} \clearpage
\begin{table}[ht]
\caption{Resultados de la ejecución de la metaheurística PSO, utilizando instancias de SalhiNagy con la configuración -n 30.0 -L 30.0 -cp 1 -cg 0 -cl 1 -cn 2 -w1 0.9 -wt 0.1 -K 5}
\centering
\small
\begin{tabular}{c c c c c c c c}
\hline\hline
Instancia & Costo mínimo & Tiempo(seg.) & Costo promedio & Tiempo promedio(seg.) & CME & \%G & \%GP \\ [0.5ex]
\hline
CMT1X & 472.37 & 4.96 & 
472.37 & 5.83 & \bf{470.48} & 
0.40 & 0.40\\CMT1Y & 471.25 & 5.93 & 
475.52 & 5.58 & \bf{470.48} & 
0.16 & 1.07\\CMT2X & 701.79 & 10.36 & 
737.60 & 10.65 & \bf{682.39} & 
2.84 & 8.09\\CMT2Y & 741.54 & 12.35 & 
757.29 & 11.29 & \bf{682.39} & 
8.67 & 10.98\\CMT3X & 725.16 & 65.53 & 
731.46 & 66.11 & \bf{719.06} & 
0.85 & 1.72\\CMT3Y & 722.97 & 62.61 & 
746.34 & 69.03 & \bf{719.06} & 
0.54 & 3.79\\CMT4X & 914.28 & 85.78 & 
937.26 & 80.34 & \bf{854.21} & 
7.03 & 9.72\\CMT4Y & 871.22 & 66.15 & 
885.53 & 107.36 & \bf{852.46} & 
2.20 & 3.88\\CMT5X & 1111.68 & 46.21 & 
1122.70 & 131.13 & \bf{1030.56} & 
7.87 & 8.94\\CMT5Y & 1159.74 & 248.55 & 
1175.49 & 247.18 & \bf{1031.69} & 
12.41 & 13.94\\CMT11X & 901.31 & 19.30 & 
925.38 & 14.54 & \bf{831.09} & 
8.45 & 11.35\\CMT11Y & 901.51 & 15.44 & 
903.36 & 23.37 & \bf{829.85} & 
8.64 & 8.86\\CMT12X & 705.34 & 4.92 & 
755.54 & 5.99 & \bf{658.83} & 
7.06 & 14.68\\CMT12Y & 683.58 & 4.31 & 
741.89 & 4.36 & \bf{660.47} & 
3.50 & 12.33\\\bf{PROM.} & 
\bf{791.70} & \bf{46.60} & \bf{811.98} & \bf{55.91} & \bf{749.50} & \bf{5.04} & \bf{7.84}\\[1ex]\hline
\end{tabular}
\label{table:nonlin}
\end{table} \clearpage
\begin{table}[ht]
\caption{Resultados de la ejecución de la metaheurística PSO, utilizando instancias de SalhiNagy con la configuración -n 30.0 -L 50.0 -cp 1 -cg 0 -cl 1 -cn 2 -w1 0.9 -wt 0.1 -K 5}
\centering
\small
\begin{tabular}{c c c c c c c c}
\hline\hline
Instancia & Costo mínimo & Tiempo(seg.) & Costo promedio & Tiempo promedio(seg.) & CME & \%G & \%GP \\ [0.5ex]
\hline
CMT1X & 472.87 & 8.87 & 
473.89 & 9.49 & \bf{470.48} & 
0.51 & 0.72\\CMT1Y & \bf{470.48} & 9.64 & 
470.87 & 10.31 & 470.48 & 0.00
 & 0.08\\CMT2X & 711.98 & 18.66 & 
723.34 & 16.23 & \bf{682.39} & 
4.34 & 6.00\\CMT2Y & 699.23 & 18.49 & 
702.34 & 17.45 & \bf{682.39} & 
2.47 & 2.92\\CMT3X & 732.75 & 117.75 & 
741.67 & 114.59 & \bf{719.06} & 
1.90 & 3.14\\CMT3Y & 731.01 & 113.47 & 
733.44 & 115.78 & \bf{719.06} & 
1.66 & 2.00\\CMT4X & 884.04 & 228.28 & 
888.49 & 195.35 & \bf{854.21} & 
3.49 & 4.01\\CMT4Y & 892.82 & 174.11 & 
898.60 & 167.23 & \bf{852.46} & 
4.73 & 5.41\\CMT5X & 1190.01 & 223.55 & 
1202.96 & 209.81 & \bf{1030.56} & 
15.47 & 16.73\\CMT5Y & 1113.32 & 469.13 & 
1130.37 & 284.38 & \bf{1031.69} & 
7.91 & 9.56\\CMT11X & 915.19 & 49.25 & 
918.00 & 47.55 & \bf{831.09} & 
10.12 & 10.46\\CMT11Y & 897.07 & 33.22 & 
909.38 & 36.27 & \bf{829.85} & 
8.10 & 9.58\\CMT12X & 738.91 & 7.41 & 
742.55 & 7.32 & \bf{658.83} & 
12.15 & 12.71\\CMT12Y & 733.79 & 7.64 & 
772.86 & 10.35 & \bf{660.47} & 
11.10 & 17.02\\\bf{PROM.} & 
\bf{798.82} & \bf{105.68} & \bf{807.77} & \bf{88.72} & \bf{749.50} & \bf{6.00} & \bf{7.17}\\[1ex]\hline
\end{tabular}
\label{table:nonlin}
\end{table} \clearpage
\begin{table}[ht]
\caption{Resultados de la ejecución de la metaheurística PSO, utilizando instancias de SalhiNagy con la configuración -n 30.0 -L 70.0 -cp 1 -cg 0 -cl 1 -cn 2 -w1 0.9 -wt 0.1 -K 5}
\centering
\small
\begin{tabular}{c c c c c c c c}
\hline\hline
Instancia & Costo mínimo & Tiempo(seg.) & Costo promedio & Tiempo promedio(seg.) & CME & \%G & \%GP \\ [0.5ex]
\hline
CMT1X & 470.67 & 14.91 & 
471.52 & 15.95 & \bf{470.48} & 
0.04 & 0.22\\CMT1Y & 470.67 & 11.86 & 
482.54 & 12.18 & \bf{470.48} & 
0.04 & 2.56\\CMT2X & 699.28 & 22.03 & 
715.43 & 19.05 & \bf{682.39} & 
2.48 & 4.84\\CMT2Y & 699.09 & 16.63 & 
725.11 & 21.24 & \bf{682.39} & 
2.45 & 6.26\\CMT3X & 730.30 & 160.07 & 
731.71 & 161.41 & \bf{719.06} & 
1.56 & 1.76\\CMT3Y & 725.30 & 159.14 & 
729.00 & 168.12 & \bf{719.06} & 
0.87 & 1.38\\CMT4X & 892.29 & 216.43 & 
903.11 & 205.19 & \bf{854.21} & 
4.46 & 5.72\\CMT4Y & 887.34 & 209.72 & 
908.81 & 232.06 & \bf{852.46} & 
4.09 & 6.61\\CMT5X & 1135.42 & 342.10 & 
1166.31 & 338.54 & \bf{1030.56} & 
10.18 & 13.17\\CMT5Y & 1103.07 & 162.21 & 
1130.86 & 195.93 & \bf{1031.69} & 
6.92 & 9.61\\CMT11X & 883.11 & 112.29 & 
884.25 & 78.28 & \bf{831.09} & 
6.26 & 6.40\\CMT11Y & 892.43 & 44.67 & 
896.29 & 59.37 & \bf{829.85} & 
7.54 & 8.01\\CMT12X & 701.22 & 11.26 & 
753.39 & 10.94 & \bf{658.83} & 
6.43 & 14.35\\CMT12Y & 694.27 & 10.96 & 
759.38 & 11.74 & \bf{660.47} & 
5.12 & 14.98\\\bf{PROM.} & 
\bf{784.60} & \bf{106.73} & \bf{804.12} & \bf{109.28} & \bf{749.50} & \bf{4.17} & \bf{6.85}\\[1ex]\hline
\end{tabular}
\label{table:nonlin}
\end{table} \clearpage
\begin{table}[ht]
\caption{Resultados de la ejecución de la metaheurística PSO, utilizando instancias de SalhiNagy con la configuración -n 30.0 -L 90.0 -cp 1 -cg 0 -cl 1 -cn 2 -w1 0.9 -wt 0.1 -K 5}
\centering
\small
\begin{tabular}{c c c c c c c c}
\hline\hline
Instancia & Costo mínimo & Tiempo(seg.) & Costo promedio & Tiempo promedio(seg.) & CME & \%G & \%GP \\ [0.5ex]
\hline
CMT1X & 472.37 & 19.09 & 
489.41 & 17.64 & \bf{470.48} & 
0.40 & 4.02\\CMT1Y & 471.25 & 17.19 & 
471.81 & 17.86 & \bf{470.48} & 
0.16 & 0.28\\CMT2X & 703.80 & 24.19 & 
707.75 & 22.39 & \bf{682.39} & 
3.14 & 3.72\\CMT2Y & 706.85 & 20.71 & 
745.30 & 23.94 & \bf{682.39} & 
3.58 & 9.22\\CMT3X & 725.93 & 205.58 & 
726.74 & 209.85 & \bf{719.06} & 
0.96 & 1.07\\CMT3Y & 731.17 & 194.34 & 
734.29 & 200.57 & \bf{719.06} & 
1.68 & 2.12\\CMT4X & 930.31 & 277.54 & 
941.03 & 338.06 & \bf{854.21} & 
8.91 & 10.16\\CMT4Y & 884.77 & 263.89 & 
922.38 & 301.71 & \bf{852.46} & 
3.79 & 8.20\\CMT5X & 1134.02 & 413.46 & 
1148.45 & 340.15 & \bf{1030.56} & 
10.04 & 11.44\\CMT5Y & 1093.02 & 476.60 & 
1101.95 & 435.88 & \bf{1031.69} & 
5.94 & 6.81\\CMT11X & 892.48 & 83.22 & 
910.72 & 91.33 & \bf{831.09} & 
7.39 & 9.58\\CMT11Y & 910.24 & 92.67 & 
912.60 & 74.20 & \bf{829.85} & 
9.69 & 9.97\\CMT12X & 676.18 & 15.97 & 
769.18 & 15.03 & \bf{658.83} & 
2.63 & 16.75\\CMT12Y & 725.35 & 15.14 & 
794.06 & 15.09 & \bf{660.47} & 
9.82 & 20.23\\\bf{PROM.} & 
\bf{789.84} & \bf{151.40} & \bf{812.55} & \bf{150.26} & \bf{749.50} & \bf{4.87} & \bf{8.11}\\[1ex]\hline
\end{tabular}
\label{table:nonlin}
\end{table} \clearpage
\begin{table}[ht]
\caption{Resultados de la ejecución de la metaheurística PSO, utilizando instancias de SalhiNagy con la configuración -n 50.0 -L 10.0 -cp 1 -cg 0 -cl 1 -cn 2 -w1 0.9 -wt 0.1 -K 5}
\centering
\small
\begin{tabular}{c c c c c c c c}
\hline\hline
Instancia & Costo mínimo & Tiempo(seg.) & Costo promedio & Tiempo promedio(seg.) & CME & \%G & \%GP \\ [0.5ex]
\hline
CMT1X & 471.25 & 3.84 & 
471.81 & 4.24 & \bf{470.48} & 
0.16 & 0.28\\CMT1Y & 476.53 & 3.81 & 
482.31 & 3.48 & \bf{470.48} & 
1.29 & 2.51\\CMT2X & 704.31 & 6.51 & 
719.52 & 4.74 & \bf{682.39} & 
3.21 & 5.44\\CMT2Y & 717.09 & 6.59 & 
717.41 & 4.12 & \bf{682.39} & 
5.09 & 5.13\\CMT3X & 735.43 & 35.01 & 
740.28 & 34.91 & \bf{719.06} & 
2.28 & 2.95\\CMT3Y & 733.04 & 42.21 & 
744.48 & 41.38 & \bf{719.06} & 
1.94 & 3.54\\CMT4X & 955.07 & 84.12 & 
1035.63 & 95.45 & \bf{854.21} & 
11.81 & 21.24\\CMT4Y & 887.68 & 16.17 & 
907.20 & 55.52 & \bf{852.46} & 
4.13 & 6.42\\CMT5X & 1146.53 & 22.53 & 
1148.85 & 50.94 & \bf{1030.56} & 
11.25 & 11.48\\CMT5Y & 1208.80 & 12.85 & 
1217.37 & 127.27 & \bf{1031.69} & 
17.17 & 18.00\\CMT11X & 907.94 & 5.53 & 
916.14 & 16.17 & \bf{831.09} & 
9.25 & 10.23\\CMT11Y & 894.19 & 10.12 & 
909.83 & 10.61 & \bf{829.85} & 
7.75 & 9.64\\CMT12X & 737.78 & 2.49 & 
759.64 & 2.90 & \bf{658.83} & 
11.98 & 15.30\\CMT12Y & 675.61 & 3.92 & 
701.05 & 4.59 & \bf{660.47} & 
2.29 & 6.14\\\bf{PROM.} & 
\bf{803.66} & \bf{18.26} & \bf{819.39} & \bf{32.59} & \bf{749.50} & \bf{6.40} & \bf{8.45}\\[1ex]\hline
\end{tabular}
\label{table:nonlin}
\end{table} \clearpage
\begin{table}[ht]
\caption{Resultados de la ejecución de la metaheurística PSO, utilizando instancias de SalhiNagy con la configuración -n 50.0 -L 30.0 -cp 1 -cg 0 -cl 1 -cn 2 -w1 0.9 -wt 0.1 -K 5}
\centering
\small
\begin{tabular}{c c c c c c c c}
\hline\hline
Instancia & Costo mínimo & Tiempo(seg.) & Costo promedio & Tiempo promedio(seg.) & CME & \%G & \%GP \\ [0.5ex]
\hline
CMT1X & 472.37 & 9.00 & 
472.37 & 9.34 & \bf{470.48} & 
0.40 & 0.40\\CMT1Y & 472.37 & 7.91 & 
474.39 & 8.60 & \bf{470.48} & 
0.40 & 0.83\\CMT2X & 705.76 & 8.17 & 
706.54 & 10.19 & \bf{682.39} & 
3.42 & 3.54\\CMT2Y & 697.44 & 17.94 & 
763.76 & 19.52 & \bf{682.39} & 
2.21 & 11.92\\CMT3X & 752.79 & 101.67 & 
762.11 & 94.98 & \bf{719.06} & 
4.69 & 5.99\\CMT3Y & 723.67 & 108.12 & 
730.59 & 115.36 & \bf{719.06} & 
0.64 & 1.60\\CMT4X & 882.12 & 175.82 & 
908.82 & 158.59 & \bf{854.21} & 
3.27 & 6.39\\CMT4Y & 903.63 & 188.86 & 
939.82 & 215.71 & \bf{852.46} & 
6.00 & 10.25\\CMT5X & 1147.12 & 524.72 & 
1154.86 & 437.89 & \bf{1030.56} & 
11.31 & 12.06\\CMT5Y & 1132.69 & 69.30 & 
1177.20 & 125.66 & \bf{1031.69} & 
9.79 & 14.10\\CMT11X & 885.38 & 74.00 & 
942.62 & 56.35 & \bf{831.09} & 
6.53 & 13.42\\CMT11Y & 903.44 & 36.59 & 
905.38 & 40.98 & \bf{829.85} & 
8.87 & 9.10\\CMT12X & 750.03 & 6.97 & 
810.80 & 7.13 & \bf{658.83} & 
13.84 & 23.07\\CMT12Y & 683.58 & 7.69 & 
728.83 & 7.12 & \bf{660.47} & 
3.50 & 10.35\\\bf{PROM.} & 
\bf{793.74} & \bf{95.48} & \bf{819.86} & \bf{93.39} & \bf{749.50} & \bf{5.35} & \bf{8.79}\\[1ex]\hline
\end{tabular}
\label{table:nonlin}
\end{table} \clearpage
\begin{table}[ht]
\caption{Resultados de la ejecución de la metaheurística PSO, utilizando instancias de SalhiNagy con la configuración -n 50.0 -L 50.0 -cp 1 -cg 0 -cl 1 -cn 2 -w1 0.9 -wt 0.1 -K 5}
\centering
\small
\begin{tabular}{c c c c c c c c}
\hline\hline
Instancia & Costo mínimo & Tiempo(seg.) & Costo promedio & Tiempo promedio(seg.) & CME & \%G & \%GP \\ [0.5ex]
\hline
CMT1X & 470.67 & 13.02 & 
472.79 & 14.22 & \bf{470.48} & 
0.04 & 0.49\\CMT1Y & 470.67 & 15.64 & 
474.30 & 14.98 & \bf{470.48} & 
0.04 & 0.81\\CMT2X & 701.09 & 26.70 & 
715.48 & 22.65 & \bf{682.39} & 
2.74 & 4.85\\CMT2Y & 700.57 & 35.45 & 
743.18 & 34.73 & \bf{682.39} & 
2.66 & 8.91\\CMT3X & 723.67 & 169.42 & 
749.00 & 173.84 & \bf{719.06} & 
0.64 & 4.16\\CMT3Y & 725.72 & 184.05 & 
729.92 & 215.19 & \bf{719.06} & 
0.93 & 1.51\\CMT4X & 889.19 & 330.33 & 
894.43 & 327.23 & \bf{854.21} & 
4.10 & 4.71\\CMT4Y & 884.99 & 329.40 & 
918.78 & 335.67 & \bf{852.46} & 
3.82 & 7.78\\CMT5X & 1209.12 & 212.25 & 
1211.50 & 653.50 & \bf{1030.56} & 
17.33 & 17.56\\CMT5Y & 1105.97 & 399.16 & 
1138.25 & 420.59 & \bf{1031.69} & 
7.20 & 10.33\\CMT11X & 888.31 & 52.25 & 
937.61 & 61.49 & \bf{831.09} & 
6.88 & 12.82\\CMT11Y & 912.89 & 99.40 & 
957.08 & 98.62 & \bf{829.85} & 
10.01 & 15.33\\CMT12X & 789.52 & 11.95 & 
837.99 & 12.06 & \bf{658.83} & 
19.84 & 27.19\\CMT12Y & 703.08 & 14.82 & 
721.81 & 17.68 & \bf{660.47} & 
6.45 & 9.29\\\bf{PROM.} & 
\bf{798.25} & \bf{135.27} & \bf{821.58} & \bf{171.60} & \bf{749.50} & \bf{5.90} & \bf{8.98}\\[1ex]\hline
\end{tabular}
\label{table:nonlin}
\end{table} \clearpage
\begin{table}[ht]
\caption{Resultados de la ejecución de la metaheurística PSO, utilizando instancias de SalhiNagy con la configuración -n 50.0 -L 70.0 -cp 1 -cg 0 -cl 1 -cn 2 -w1 0.9 -wt 0.1 -K 5}
\centering
\small
\begin{tabular}{c c c c c c c c}
\hline\hline
Instancia & Costo mínimo & Tiempo(seg.) & Costo promedio & Tiempo promedio(seg.) & CME & \%G & \%GP \\ [0.5ex]
\hline
CMT1X & 471.25 & 22.10 & 
471.25 & 21.90 & \bf{470.48} & 
0.16 & 0.16\\CMT1Y & \bf{470.48} & 21.01 & 
470.87 & 22.02 & 470.48 & 0.00
 & 0.08\\CMT2X & 705.28 & 38.62 & 
721.83 & 36.45 & \bf{682.39} & 
3.35 & 5.78\\CMT2Y & 702.94 & 22.65 & 
707.25 & 27.29 & \bf{682.39} & 
3.01 & 3.64\\CMT3X & 747.16 & 274.61 & 
752.57 & 269.39 & \bf{719.06} & 
3.91 & 4.66\\CMT3Y & 725.57 & 223.59 & 
726.18 & 230.81 & \bf{719.06} & 
0.91 & 0.99\\CMT4X & 914.03 & 388.71 & 
945.83 & 425.42 & \bf{854.21} & 
7.00 & 10.73\\CMT4Y & 925.07 & 328.78 & 
929.32 & 387.35 & \bf{852.46} & 
8.52 & 9.02\\CMT5X & 1127.06 & 469.50 & 
1148.85 & 445.80 & \bf{1030.56} & 
9.36 & 11.48\\CMT5Y & 1155.25 & 500.64 & 
1187.38 & 592.57 & \bf{1031.69} & 
11.98 & 15.09\\CMT11X & 893.06 & 101.30 & 
940.77 & 50.65 & \bf{831.09} & 
7.46 & 13.20\\CMT11Y & 889.74 & 87.47 & 
896.95 & 93.68 & \bf{829.85} & 
7.22 & 8.09\\CMT12X & 705.55 & 18.66 & 
727.97 & 19.20 & \bf{658.83} & 
7.09 & 10.49\\CMT12Y & 708.37 & 18.82 & 
748.34 & 18.71 & \bf{660.47} & 
7.25 & 13.30\\\bf{PROM.} & 
\bf{795.77} & \bf{179.75} & \bf{812.52} & \bf{188.66} & \bf{749.50} & \bf{5.52} & \bf{7.62}\\[1ex]\hline
\end{tabular}
\label{table:nonlin}
\end{table} \clearpage
\begin{table}[ht]
\caption{Resultados de la ejecución de la metaheurística ACO, utilizando instancias de SalhiNagy con la configuración -n 6.0 -alpha 1.0 -beta 3.0 -q 0.1 -ro 0.015}
\centering
\small
\begin{tabular}{c c c c c c c c}
\hline\hline
Instancia & Costo mínimo & Tiempo(seg.) & Costo promedio & Tiempo promedio(seg.) & CME & \%G & \%GP \\ [0.5ex]
\hline
CMT1X & \bf{470.48} & 1.74 & 
474.70 & 1.42 & 470.48 & 0.00
 & 0.90\\CMT1Y & 471.25 & 1.35 & 
474.94 & 1.41 & \bf{470.48} & 
0.16 & 0.95\\CMT2X & 691.13 & 6.64 & 
700.84 & 6.86 & \bf{682.39} & 
1.28 & 2.70\\CMT2Y & 686.76 & 6.76 & 
700.44 & 6.59 & \bf{682.39} & 
0.64 & 2.65\\CMT3X & 724.21 & 21.06 & 
729.24 & 21.57 & \bf{719.06} & 
0.72 & 1.42\\CMT3Y & 723.52 & 22.69 & 
731.28 & 22.67 & \bf{719.06} & 
0.62 & 1.70\\CMT4X & 865.47 & 94.77 & 
883.96 & 96.93 & \bf{854.21} & 
1.32 & 3.48\\CMT4Y & 870.98 & 97.12 & 
886.17 & 97.64 & \bf{852.46} & 
2.17 & 3.95\\CMT5X & 1077.34 & 273.74 & 
1085.90 & 274.51 & \bf{1030.56} & 
4.54 & 5.37\\CMT5Y & 1070.53 & 274.44 & 
1085.51 & 275.57 & \bf{1031.69} & 
3.76 & 5.22\\CMT11X & 847.74 & 43.28 & 
865.98 & 42.96 & \bf{831.09} & 
2.00 & 4.20\\CMT11Y & 845.47 & 37.33 & 
869.98 & 37.90 & \bf{829.85} & 
1.88 & 4.84\\CMT12X & 663.01 & 18.74 & 
675.11 & 18.70 & \bf{658.83} & 
0.63 & 2.47\\CMT12Y & 663.19 & 18.18 & 
674.76 & 17.89 & \bf{660.47} & 
0.41 & 2.16\\\bf{PROM.} & 
\bf{762.22} & \bf{65.56} & \bf{774.20} & \bf{65.90} & \bf{749.50} & \bf{1.44} & \bf{3.00}\\[1ex]\hline
\end{tabular}
\label{table:nonlin}
\end{table} \clearpage
\begin{table}[ht]
\caption{Resultados de la ejecución de la metaheurística PSO, utilizando instancias de SalhiNagy con la configuración -n 50.0 -L 90.0 -cp 1 -cg 0 -cl 1 -cn 2 -w1 0.9 -wt 0.1 -K 5}
\centering
\small
\begin{tabular}{c c c c c c c c}
\hline\hline
Instancia & Costo mínimo & Tiempo(seg.) & Costo promedio & Tiempo promedio(seg.) & CME & \%G & \%GP \\ [0.5ex]
\hline
CMT1X & 470.67 & 26.67 & 
470.88 & 30.02 & \bf{470.48} & 
0.04 & 0.09\\CMT1Y & \bf{470.48} & 29.14 & 
470.57 & 29.32 & 470.48 & 0.00
 & 0.02\\CMT2X & 703.231224 & 34.11 & 
721.18 & 34.11 & \bf{682.39} & 
3.05 & 5.68\\CMT2Y & 692.15 & 35.81 & 
727.80 & 44.34 & \bf{682.39} & 
1.43 & 6.65\\CMT3X & 723.52 & 392.04 & 
763.20 & 358.35 & \bf{719.06} & 
0.62 & 6.14\\CMT3Y & 725.90 & 329.92 & 
727.47 & 327.12 & \bf{719.06} & 
0.95 & 1.17\\CMT4X & 878.88 & 561.14 & 
901.75 & 503.18 & \bf{854.21} & 
2.89 & 5.56\\CMT4Y & 896.19 & 552.75 & 
898.96 & 579.33 & \bf{852.46} & 
5.13 & 5.45\\CMT5X & 1135.60 & 886.99 & 
1136.95 & 828.08 & \bf{1030.56} & 
10.19 & 10.32\\CMT5Y & 1122.67 & 591.57 & 
1145.68 & 799.82 & \bf{1031.69} & 
8.82 & 11.05\\CMT11X & 903.66 & 177.27 & 
959.13 & 169.28 & \bf{831.09} & 
8.73 & 15.41\\CMT11Y & 882.72 & 156.44 & 
888.63 & 176.46 & \bf{829.85} & 
6.37 & 7.08\\CMT12X & 802.01 & 37.15 & 
856.42 & 30.32 & \bf{658.83} & 
21.73 & 29.99\\CMT12Y & 751.72 & 23.86 & 
755.09 & 24.62 & \bf{660.47} & 
13.82 & 14.33\\\bf{PROM.} & 
\bf{797.10} & \bf{273.92} & \bf{815.98} & \bf{281.02} & \bf{749.50} & \bf{5.98} & \bf{8.50}\\[1ex]\hline
\end{tabular}
\label{table:nonlin}
\end{table} \clearpage
\begin{table}[ht]
\caption{Resultados de la ejecución de la metaheurística PSO, utilizando instancias de SalhiNagy con la configuración -n 30.0 -L 70.0 -cp 1 -cg 0 -cl 1 -cn 2 -w1 0.9 -wt 0.1 -K 5}
\centering
\small
\begin{tabular}{c c c c c c c c}
\hline\hline
Instancia & Costo mínimo & Tiempo(seg.) & Costo promedio & Tiempo promedio(seg.) & CME & \%G & \%GP \\ [0.5ex]
\hline
CMT1X & \bf{470.48} & 5.06 & 
477.81 & 4.98 & 470.48 & 0.00
 & 1.56\\CMT1Y & \bf{470.48} & 4.89 & 
476.11 & 5.13 & 470.48 & 0.00
 & 1.20\\CMT2X & 693.96 & 8.25 & 
735.46 & 6.99 & \bf{682.39} & 
1.70 & 7.78\\CMT2Y & 693.74 & 5.95 & 
721.04 & 6.77 & \bf{682.39} & 
1.66 & 5.66\\CMT3X & 723.67 & 60.99 & 
735.31 & 62.71 & \bf{719.06} & 
0.64 & 2.26\\CMT3Y & 723.23 & 63.47 & 
737.00 & 63.39 & \bf{719.06} & 
0.58 & 2.50\\CMT4X & 878.06 & 103.76 & 
916.43 & 101.35 & \bf{854.21} & 
2.79 & 7.28\\CMT4Y & 872.63 & 79.17 & 
906.71 & 101.04 & \bf{852.46} & 
2.37 & 6.36\\CMT5X & 1090.08 & 115.89 & 
1151.52 & 103.82 & \bf{1030.56} & 
5.78 & 11.74\\CMT5Y & 1089.93 & 102.17 & 
1141.93 & 117.49 & \bf{1031.69} & 
5.65 & 10.69\\CMT11X & 883.28 & 14.86 & 
913.41 & 20.12 & \bf{831.09} & 
6.28 & 9.91\\CMT11Y & 887.01 & 16.95 & 
910.41 & 21.66 & \bf{829.85} & 
6.89 & 9.71\\CMT12X & 691.19 & 3.83 & 
749.96 & 3.93 & \bf{658.83} & 
4.91 & 13.83\\CMT12Y & 679.79 & 3.81 & 
755.03 & 3.95 & \bf{660.47} & 
2.93 & 14.32\\\bf{PROM.} & 
\bf{774.82} & \bf{42.08} & \bf{809.15} & \bf{44.52} & \bf{749.50} & \bf{3.01} & \bf{7.48}\\[1ex]\hline
\end{tabular}
\label{table:nonlin}
\end{table} \clearpage
\begin{table}[ht]
\caption{Resultados de la ejecución de la metaheurística IGA, utilizando instancias de Dethloff con la configuración -n 100.0 -p 100.0 -cprob 40 -mprob 70}
\centering
\small
\begin{tabular}{c c c c c c c c}
\hline\hline
Instancia & Costo mínimo & Tiempo(seg.) & Costo promedio & Tiempo promedio(seg.) & CME & \%G & \%GP \\ [0.5ex]
\hline
SCA3-0 & 640.55 & 1.24 & 
640.55 & 1.22 & \bf{635.62} & 
0.78 & 0.78\\SCA3-1 & \bf{697.84} & 1.22 & 
700.61 & 1.36 & 697.84 & 0.00
 & 0.40\\SCA3-2 & \bf{659.34} & 0.99 & 
663.43 & 1.21 & 659.34 & 0.00
 & 0.62\\SCA3-3 & \bf{680.04} & 1.31 & 
681.29 & 1.27 & 680.04 & 0.00
 & 0.18\\SCA3-4 & \bf{690.50} & 1.60 & 
690.50 & 1.33 & 690.50 & 0.00
 & 0.00\\
SCA3-5 & \bf{659.90} & 1.71 & 
664.21 & 1.42 & 659.90 & 0.00
 & 0.65\\SCA3-6 & 652.94 & 1.40 & 
654.02 & 1.26 & \bf{651.09} & 
0.28 & 0.45\\SCA3-7 & 667.45 & 1.34 & 
668.67 & 1.20 & \bf{659.17} & 
1.26 & 1.44\\SCA3-8 & 723.99 & 1.70 & 
725.05 & 1.31 & \bf{719.47} & 
0.63 & 0.78\\SCA3-9 & \bf{681.00} & 1.16 & 
681.00 & 1.32 & 681.00 & 0.00
 & 0.00\\
SCA8-0 & 1002.73 & 1.13 & 
1002.73 & 1.15 & \bf{961.50} & 
4.29 & 4.29\\SCA8-1 & 1067.36 & 1.06 & 
1067.36 & 1.30 & \bf{1049.65} & 
1.69 & 1.69\\SCA8-2 & 1050.17 & 1.08 & 
1050.17 & 1.18 & \bf{1039.64} & 
1.01 & 1.01\\SCA8-3 & 1002.00 & 1.01 & 
1002.00 & 1.28 & \bf{983.34} & 
1.90 & 1.90\\SCA8-4 & 1075.27 & 1.28 & 
1088.55 & 1.18 & \bf{1065.49} & 
0.92 & 2.16\\SCA8-5 & 1059.00 & 1.33 & 
1059.87 & 1.26 & \bf{1027.08} & 
3.11 & 3.19\\SCA8-6 & 972.48 & 1.32 & 
981.86 & 1.40 & \bf{971.82} & 
0.07 & 1.03\\SCA8-7 & 1069.83 & 1.31 & 
1072.50 & 1.28 & \bf{1051.28} & 
1.76 & 2.02\\SCA8-8 & \bf{1071.18} & 1.12 & 
1071.18 & 1.31 & 1071.18 & 0.00
 & 0.00\\
SCA8-9 & 1070.34 & 1.43 & 
1070.34 & 1.26 & \bf{1060.50} & 
0.93 & 0.93\\CON3-0 & 620.76 & 1.18 & 
622.50 & 1.32 & \bf{616.52} & 
0.69 & 0.97\\CON3-1 & \bf{554.47} & 1.25 & 
556.83 & 1.24 & 554.47 & 0.00
 & 0.42\\CON3-2 & 521.38 & 1.34 & 
521.38 & 1.53 & \bf{518.00} & 
0.65 & 0.65\\CON3-3 & 591.48 & 1.32 & 
591.96 & 1.35 & \bf{591.19} & 
0.05 & 0.13\\CON3-4 & 592.58 & 1.22 & 
593.38 & 1.28 & \bf{588.79} & 
0.64 & 0.78\\CON3-5 & \bf{563.70} & 1.10 & 
566.32 & 1.27 & 563.70 & 0.00
 & 0.46\\CON3-6 & 502.16 & 1.41 & 
503.12 & 1.33 & \bf{499.05} & 
0.62 & 0.82\\CON3-7 & 578.41 & 1.34 & 
582.30 & 1.35 & \bf{576.48} & 
0.33 & 1.01\\CON3-8 & 524.59 & 1.38 & 
527.49 & 1.41 & \bf{523.05} & 
0.29 & 0.85\\CON3-9 & 588.18 & 1.30 & 
588.25 & 1.31 & \bf{578.24} & 
1.72 & 1.73\\CON8-0 & 864.93 & 1.21 & 
866.69 & 1.30 & \bf{857.17} & 
0.91 & 1.11\\CON8-1 & 742.44 & 1.36 & 
742.46 & 1.41 & \bf{740.85} & 
0.21 & 0.22\\CON8-2 & 716.07 & 1.20 & 
717.43 & 1.27 & \bf{712.89} & 
0.45 & 0.64\\CON8-3 & 832.40 & 1.54 & 
833.90 & 1.46 & \bf{811.07} & 
2.63 & 2.81\\CON8-4 & 789.18 & 1.21 & 
789.46 & 1.31 & \bf{772.25} & 
2.19 & 2.23\\CON8-5 & 762.01 & 1.41 & 
764.08 & 1.48 & \bf{754.88} & 
0.94 & 1.22\\CON8-6 & 689.23 & 1.78 & 
689.23 & 1.38 & \bf{678.92} & 
1.52 & 1.52\\CON8-7 & 814.79 & 1.12 & 
814.79 & 1.35 & \bf{811.96} & 
0.35 & 0.35\\CON8-8 & 791.04 & 1.30 & 
791.07 & 1.35 & \bf{767.53} & 
3.06 & 3.07\\CON8-9 & 826.82 & 1.34 & 
826.82 & 1.24 & \bf{809.00} & 
2.20 & 2.20\\\bf{PROM.} & 
\bf{766.51} & \bf{1.30} & \bf{768.13} & \bf{1.31} & \bf{758.54} & \bf{0.95} & \bf{1.17}\\[1ex]\hline
\end{tabular}
\label{table:nonlin}
\end{table} \clearpage
\begin{table}[ht]
\caption{Resultados de la ejecución de la metaheurística IGA, utilizando instancias de SalhiNagy con la configuración -n 100.0 -p 100.0 -cprob 90 -mprob 70}
\centering
\small
\begin{tabular}{c c c c c c c c}
\hline\hline
Instancia & Costo mínimo & Tiempo(seg.) & Costo promedio & Tiempo promedio(seg.) & CME & \%G & \%GP \\ [0.5ex]
\hline
CMT1X & 475.42 & 1.44 & 
478.15 & 1.38 & \bf{470.48} & 
1.05 & 1.63\\CMT1Y & 474.72 & 1.26 & 
479.76 & 1.25 & \bf{470.48} & 
0.90 & 1.97\\CMT2X & 703.31 & 2.84 & 
707.90 & 3.09 & \bf{682.39} & 
3.07 & 3.74\\CMT2Y & 695.68 & 3.00 & 
699.08 & 2.93 & \bf{682.39} & 
1.95 & 2.45\\CMT3X & 735.68 & 7.09 & 
744.32 & 6.92 & \bf{719.06} & 
2.31 & 3.51\\CMT3Y & 736.60 & 7.04 & 
743.50 & 6.74 & \bf{719.06} & 
2.44 & 3.40\\CMT4X & 891.75 & 19.48 & 
904.69 & 18.96 & \bf{854.21} & 
4.39 & 5.91\\CMT4Y & 881.95 & 19.59 & 
904.60 & 19.25 & \bf{852.46} & 
3.46 & 6.12\\CMT5X & 1105.54 & 39.49 & 
1114.33 & 38.87 & \bf{1030.56} & 
7.28 & 8.13\\CMT5Y & 1084.07 & 40.92 & 
1111.08 & 39.93 & \bf{1031.69} & 
5.08 & 7.70\\CMT11X & 886.82 & 11.52 & 
905.88 & 11.62 & \bf{831.09} & 
6.71 & 9.00\\CMT11Y & 889.90 & 13.65 & 
901.06 & 13.54 & \bf{829.85} & 
7.24 & 8.58\\CMT12X & 674.00 & 7.06 & 
677.23 & 6.94 & \bf{658.83} & 
2.30 & 2.79\\CMT12Y & 673.16 & 6.81 & 
674.22 & 7.09 & \bf{660.47} & 
1.92 & 2.08\\\bf{PROM.} & 
\bf{779.19} & \bf{12.94} & \bf{788.99} & \bf{12.75} & \bf{749.50} & \bf{3.58} & \bf{4.79}\\[1ex]\hline
\end{tabular}
\label{table:nonlin}
\end{table} \clearpage
\begin{table}[ht]
\caption{Resultados de la ejecución de la metaheurística IGA, utilizando instancias de Dethloff con la configuración -n 100.0 -p 150.0 -cprob 40 -mprob 70}
\centering
\small
\begin{tabular}{c c c c c c c c}
\hline\hline
Instancia & Costo mínimo & Tiempo(seg.) & Costo promedio & Tiempo promedio(seg.) & CME & \%G & \%GP \\ [0.5ex]
\hline
SCA3-0 & 640.55 & 2.34 & 
640.55 & 2.08 & \bf{635.62} & 
0.78 & 0.78\\SCA3-1 & \bf{697.84} & 3.55 & 
699.17 & 2.47 & 697.84 & 0.00
 & 0.19\\SCA3-2 & 664.18 & 1.88 & 
664.29 & 1.70 & \bf{659.34} & 
0.73 & 0.75\\SCA3-3 & 680.60 & 1.69 & 
681.96 & 2.06 & \bf{680.04} & 
0.08 & 0.28\\SCA3-4 & \bf{690.50} & 1.81 & 
690.50 & 2.00 & 690.50 & 0.00
 & 0.00\\
SCA3-5 & 665.64 & 2.40 & 
665.64 & 1.97 & \bf{659.90} & 
0.87 & 0.87\\SCA3-6 & 652.94 & 1.54 & 
652.94 & 1.99 & \bf{651.09} & 
0.28 & 0.28\\SCA3-7 & 664.88 & 1.97 & 
665.20 & 2.09 & \bf{659.17} & 
0.87 & 0.91\\SCA3-8 & \bf{719.47} & 1.78 & 
720.75 & 1.82 & 719.47 & 0.00
 & 0.18\\SCA3-9 & \bf{681.00} & 1.68 & 
681.00 & 1.81 & 681.00 & 0.00
 & 0.00\\
SCA8-0 & 982.18 & 1.89 & 
982.18 & 2.02 & \bf{961.50} & 
2.15 & 2.15\\SCA8-1 & 1053.09 & 1.54 & 
1053.09 & 1.82 & \bf{1049.65} & 
0.33 & 0.33\\SCA8-2 & 1053.78 & 1.96 & 
1053.78 & 1.93 & \bf{1039.64} & 
1.36 & 1.36\\SCA8-3 & 1010.01 & 2.19 & 
1012.10 & 1.87 & \bf{983.34} & 
2.71 & 2.92\\SCA8-4 & 1067.66 & 2.24 & 
1067.66 & 2.08 & \bf{1065.49} & 
0.20 & 0.20\\SCA8-5 & 1053.01 & 1.74 & 
1054.06 & 1.94 & \bf{1027.08} & 
2.52 & 2.63\\SCA8-6 & 976.69 & 1.94 & 
976.69 & 2.08 & \bf{971.82} & 
0.50 & 0.50\\SCA8-7 & 1070.92 & 1.70 & 
1070.92 & 1.56 & \bf{1051.28} & 
1.87 & 1.87\\SCA8-8 & \bf{1071.18} & 1.54 & 
1082.28 & 1.66 & 1071.18 & 0.00
 & 1.04\\SCA8-9 & 1068.65 & 1.86 & 
1068.65 & 1.89 & \bf{1060.50} & 
0.77 & 0.77\\CON3-0 & 619.09 & 2.04 & 
620.28 & 1.86 & \bf{616.52} & 
0.42 & 0.61\\CON3-1 & 557.21 & 1.63 & 
559.87 & 1.93 & \bf{554.47} & 
0.49 & 0.97\\CON3-2 & 521.38 & 2.31 & 
521.38 & 2.07 & \bf{518.00} & 
0.65 & 0.65\\CON3-3 & 592.43 & 1.81 & 
594.89 & 2.02 & \bf{591.19} & 
0.21 & 0.63\\CON3-4 & 591.43 & 2.51 & 
593.19 & 2.07 & \bf{588.79} & 
0.45 & 0.75\\CON3-5 & 564.88 & 2.17 & 
566.41 & 2.10 & \bf{563.70} & 
0.21 & 0.48\\CON3-6 & 502.26 & 1.86 & 
503.86 & 2.17 & \bf{499.05} & 
0.64 & 0.96\\CON3-7 & 578.41 & 1.98 & 
580.37 & 1.99 & \bf{576.48} & 
0.33 & 0.67\\CON3-8 & 524.30 & 2.02 & 
524.45 & 2.04 & \bf{523.05} & 
0.24 & 0.27\\CON3-9 & 588.11 & 1.92 & 
588.11 & 1.98 & \bf{578.24} & 
1.71 & 1.71\\CON8-0 & 860.48 & 1.86 & 
865.74 & 1.98 & \bf{857.17} & 
0.39 & 1.00\\CON8-1 & 753.57 & 2.29 & 
754.04 & 2.12 & \bf{740.85} & 
1.72 & 1.78\\CON8-2 & 717.84 & 1.92 & 
717.84 & 2.19 & \bf{712.89} & 
0.69 & 0.69\\CON8-3 & 822.51 & 1.97 & 
823.78 & 1.85 & \bf{811.07} & 
1.41 & 1.57\\CON8-4 & 780.51 & 2.43 & 
780.51 & 2.05 & \bf{772.25} & 
1.07 & 1.07\\CON8-5 & 765.69 & 1.96 & 
766.05 & 2.17 & \bf{754.88} & 
1.43 & 1.48\\CON8-6 & 688.24 & 2.07 & 
688.24 & 1.98 & \bf{678.92} & 
1.37 & 1.37\\CON8-7 & 815.43 & 1.66 & 
818.14 & 1.83 & \bf{811.96} & 
0.43 & 0.76\\CON8-8 & 782.68 & 2.13 & 
782.68 & 1.99 & \bf{767.53} & 
1.97 & 1.97\\CON8-9 & 818.50 & 1.92 & 
818.54 & 2.01 & \bf{809.00} & 
1.17 & 1.18\\\bf{PROM.} & 
\bf{765.24} & \bf{1.99} & \bf{766.29} & \bf{1.98} & \bf{758.54} & \bf{0.83} & \bf{0.96}\\[1ex]\hline
\end{tabular}
\label{table:nonlin}
\end{table} \clearpage
\begin{table}[ht]
\caption{Resultados de la ejecución de la metaheurística IGA, utilizando instancias de SalhiNagy con la configuración -n 100.0 -p 150.0 -cprob 90 -mprob 70}
\centering
\small
\begin{tabular}{c c c c c c c c}
\hline\hline
Instancia & Costo mínimo & Tiempo(seg.) & Costo promedio & Tiempo promedio(seg.) & CME & \%G & \%GP \\ [0.5ex]
\hline
CMT1X & 476.66 & 2.10 & 
477.61 & 2.06 & \bf{470.48} & 
1.31 & 1.52\\CMT1Y & 474.91 & 1.72 & 
477.88 & 1.89 & \bf{470.48} & 
0.94 & 1.57\\CMT2X & 704.33 & 4.71 & 
706.86 & 4.59 & \bf{682.39} & 
3.22 & 3.59\\CMT2Y & 697.61 & 4.58 & 
708.14 & 4.29 & \bf{682.39} & 
2.23 & 3.77\\CMT3X & 739.58 & 10.05 & 
742.50 & 10.17 & \bf{719.06} & 
2.85 & 3.26\\CMT3Y & 737.23 & 10.64 & 
740.26 & 10.04 & \bf{719.06} & 
2.53 & 2.95\\CMT4X & 897.35 & 27.72 & 
901.54 & 28.54 & \bf{854.21} & 
5.05 & 5.54\\CMT4Y & 887.82 & 28.65 & 
901.00 & 29.27 & \bf{852.46} & 
4.15 & 5.69\\CMT5X & 1093.53 & 57.13 & 
1104.10 & 57.73 & \bf{1030.56} & 
6.11 & 7.14\\CMT5Y & 1097.56 & 59.94 & 
1104.60 & 59.17 & \bf{1031.69} & 
6.38 & 7.07\\CMT11X & 895.94 & 18.10 & 
903.25 & 17.58 & \bf{831.09} & 
7.80 & 8.68\\CMT11Y & 863.65 & 19.81 & 
881.75 & 21.39 & \bf{829.85} & 
4.07 & 6.25\\CMT12X & 674.72 & 10.69 & 
676.28 & 10.66 & \bf{658.83} & 
2.41 & 2.65\\CMT12Y & 674.87 & 10.60 & 
675.93 & 10.65 & \bf{660.47} & 
2.18 & 2.34\\\bf{PROM.} & 
\bf{779.70} & \bf{19.03} & \bf{785.84} & \bf{19.15} & \bf{749.50} & \bf{3.66} & \bf{4.43}\\[1ex]\hline
\end{tabular}
\label{table:nonlin}
\end{table} \clearpage
\begin{table}[ht]
\caption{Resultados de la ejecución de la metaheurística IGA, utilizando instancias de Dethloff con la configuración -n 100.0 -p 200.0 -cprob 40 -mprob 70}
\centering
\small
\begin{tabular}{c c c c c c c c}
\hline\hline
Instancia & Costo mínimo & Tiempo(seg.) & Costo promedio & Tiempo promedio(seg.) & CME & \%G & \%GP \\ [0.5ex]
\hline
SCA3-0 & 636.06 & 2.30 & 
637.52 & 2.53 & \bf{635.62} & 
0.07 & 0.30\\SCA3-1 & \bf{697.84} & 2.44 & 
697.84 & 2.43 & 697.84 & 0.00
 & 0.00\\
SCA3-2 & 661.13 & 2.40 & 
661.89 & 2.40 & \bf{659.34} & 
0.27 & 0.39\\SCA3-3 & \bf{680.04} & 2.40 & 
680.65 & 2.36 & 680.04 & 0.00
 & 0.09\\SCA3-4 & \bf{690.50} & 2.74 & 
690.50 & 2.79 & 690.50 & 0.00
 & 0.00\\
SCA3-5 & 665.04 & 2.37 & 
665.49 & 2.55 & \bf{659.90} & 
0.78 & 0.85\\SCA3-6 & 652.94 & 2.74 & 
653.35 & 2.46 & \bf{651.09} & 
0.28 & 0.35\\SCA3-7 & 666.15 & 2.32 & 
666.15 & 2.24 & \bf{659.17} & 
1.06 & 1.06\\SCA3-8 & \bf{719.47} & 3.36 & 
719.62 & 2.69 & 719.47 & 0.00
 & 0.02\\SCA3-9 & \bf{681.00} & 2.67 & 
681.00 & 2.51 & 681.00 & 0.00
 & 0.00\\
SCA8-0 & 979.79 & 2.99 & 
980.73 & 2.79 & \bf{961.50} & 
1.90 & 2.00\\SCA8-1 & 1063.21 & 2.83 & 
1063.21 & 2.71 & \bf{1049.65} & 
1.29 & 1.29\\SCA8-2 & 1050.17 & 3.20 & 
1050.32 & 2.59 & \bf{1039.64} & 
1.01 & 1.03\\SCA8-3 & 1008.65 & 2.99 & 
1012.33 & 2.61 & \bf{983.34} & 
2.57 & 2.95\\SCA8-4 & 1068.27 & 2.53 & 
1068.27 & 2.27 & \bf{1065.49} & 
0.26 & 0.26\\SCA8-5 & 1029.95 & 2.70 & 
1035.21 & 2.51 & \bf{1027.08} & 
0.28 & 0.79\\SCA8-6 & 976.74 & 2.95 & 
986.35 & 2.88 & \bf{971.82} & 
0.51 & 1.50\\SCA8-7 & 1070.92 & 2.89 & 
1070.92 & 3.01 & \bf{1051.28} & 
1.87 & 1.87\\SCA8-8 & 1084.41 & 2.53 & 
1084.41 & 2.36 & \bf{1071.18} & 
1.24 & 1.24\\SCA8-9 & 1070.54 & 2.56 & 
1071.60 & 2.52 & \bf{1060.50} & 
0.95 & 1.05\\CON3-0 & 617.59 & 3.56 & 
617.97 & 3.07 & \bf{616.52} & 
0.17 & 0.23\\CON3-1 & 556.92 & 2.27 & 
558.91 & 2.77 & \bf{554.47} & 
0.44 & 0.80\\CON3-2 & 521.38 & 2.74 & 
521.38 & 2.78 & \bf{518.00} & 
0.65 & 0.65\\CON3-3 & 591.20 & 2.58 & 
593.63 & 2.86 & \bf{591.19} & 
0.00 & 0.41\\CON3-4 & 591.43 & 2.93 & 
592.59 & 2.61 & \bf{588.79} & 
0.45 & 0.65\\CON3-5 & \bf{563.70} & 2.93 & 
564.29 & 2.92 & 563.70 & 0.00
 & 0.10\\CON3-6 & 502.16 & 3.26 & 
503.97 & 3.00 & \bf{499.05} & 
0.62 & 0.99\\CON3-7 & 576.87 & 2.92 & 
577.47 & 2.78 & \bf{576.48} & 
0.07 & 0.17\\CON3-8 & 523.14 & 2.54 & 
524.00 & 2.75 & \bf{523.05} & 
0.02 & 0.18\\CON3-9 & 582.79 & 2.88 & 
585.47 & 2.68 & \bf{578.24} & 
0.79 & 1.25\\CON8-0 & 860.92 & 2.37 & 
865.51 & 2.42 & \bf{857.17} & 
0.44 & 0.97\\CON8-1 & 743.27 & 2.43 & 
750.79 & 2.65 & \bf{740.85} & 
0.33 & 1.34\\CON8-2 & 713.60 & 2.45 & 
713.60 & 2.68 & \bf{712.89} & 
0.10 & 0.10\\CON8-3 & 831.52 & 2.65 & 
833.47 & 2.77 & \bf{811.07} & 
2.52 & 2.76\\CON8-4 & 773.85 & 2.29 & 
775.55 & 2.79 & \bf{772.25} & 
0.21 & 0.43\\CON8-5 & 760.86 & 2.57 & 
765.43 & 2.54 & \bf{754.88} & 
0.79 & 1.40\\CON8-6 & 691.74 & 2.79 & 
691.74 & 2.86 & \bf{678.92} & 
1.89 & 1.89\\CON8-7 & 814.79 & 2.90 & 
815.54 & 2.81 & \bf{811.96} & 
0.35 & 0.44\\CON8-8 & 784.47 & 2.88 & 
785.39 & 2.60 & \bf{767.53} & 
2.21 & 2.33\\CON8-9 & 816.88 & 2.95 & 
816.88 & 2.48 & \bf{809.00} & 
0.97 & 0.97\\\bf{PROM.} & 
\bf{764.30} & \bf{2.72} & \bf{765.77} & \bf{2.65} & \bf{758.54} & \bf{0.68} & \bf{0.88}\\[1ex]\hline
\end{tabular}
\label{table:nonlin}
\end{table} \clearpage
\begin{table}[ht]
\caption{Resultados de la ejecución de la metaheurística IGA, utilizando instancias de SalhiNagy con la configuración -n 100.0 -p 200.0 -cprob 90 -mprob 70}
\centering
\small
\begin{tabular}{c c c c c c c c}
\hline\hline
Instancia & Costo mínimo & Tiempo(seg.) & Costo promedio & Tiempo promedio(seg.) & CME & \%G & \%GP \\ [0.5ex]
\hline
CMT1X & 477.88 & 2.74 & 
479.06 & 2.44 & \bf{470.48} & 
1.57 & 1.82\\CMT1Y & 472.87 & 2.37 & 
475.12 & 2.34 & \bf{470.48} & 
0.51 & 0.99\\CMT2X & 706.02 & 6.08 & 
708.93 & 6.21 & \bf{682.39} & 
3.46 & 3.89\\CMT2Y & 704.88 & 6.42 & 
705.88 & 6.18 & \bf{682.39} & 
3.30 & 3.44\\CMT3X & 731.87 & 14.40 & 
736.75 & 14.13 & \bf{719.06} & 
1.78 & 2.46\\CMT3Y & 735.29 & 13.83 & 
741.35 & 13.60 & \bf{719.06} & 
2.26 & 3.10\\CMT4X & 891.99 & 38.28 & 
898.33 & 37.66 & \bf{854.21} & 
4.42 & 5.17\\CMT4Y & 896.39 & 37.53 & 
902.99 & 38.37 & \bf{852.46} & 
5.15 & 5.93\\CMT5X & 1072.66 & 78.01 & 
1088.53 & 77.01 & \bf{1030.56} & 
4.09 & 5.63\\CMT5Y & 1095.99 & 77.79 & 
1107.56 & 79.62 & \bf{1031.69} & 
6.23 & 7.35\\CMT11X & 878.87 & 23.47 & 
896.88 & 23.70 & \bf{831.09} & 
5.75 & 7.92\\CMT11Y & 860.10 & 25.24 & 
869.51 & 25.78 & \bf{829.85} & 
3.65 & 4.78\\CMT12X & 674.72 & 14.00 & 
676.13 & 14.18 & \bf{658.83} & 
2.41 & 2.63\\CMT12Y & 673.63 & 14.04 & 
674.41 & 13.92 & \bf{660.47} & 
1.99 & 2.11\\\bf{PROM.} & 
\bf{776.65} & \bf{25.30} & \bf{782.96} & \bf{25.37} & \bf{749.50} & \bf{3.33} & \bf{4.09}\\[1ex]\hline
\end{tabular}
\label{table:nonlin}
\end{table} \clearpage
\begin{table}[ht]
\caption{Resultados de la ejecución de la metaheurística IGA, utilizando instancias de Dethloff con la configuración -n 100.0 -p 250.0 -cprob 40 -mprob 70}
\centering
\small
\begin{tabular}{c c c c c c c c}
\hline\hline
Instancia & Costo mínimo & Tiempo(seg.) & Costo promedio & Tiempo promedio(seg.) & CME & \%G & \%GP \\ [0.5ex]
\hline
SCA3-0 & 640.55 & 3.07 & 
640.55 & 3.17 & \bf{635.62} & 
0.78 & 0.78\\SCA3-1 & \bf{697.84} & 3.49 & 
697.84 & 3.12 & 697.84 & 0.00
 & 0.00\\
SCA3-2 & 661.13 & 3.23 & 
662.67 & 2.98 & \bf{659.34} & 
0.27 & 0.51\\SCA3-3 & 681.16 & 2.98 & 
681.40 & 3.29 & \bf{680.04} & 
0.16 & 0.20\\SCA3-4 & \bf{690.50} & 3.59 & 
690.50 & 3.42 & 690.50 & 0.00
 & 0.00\\
SCA3-5 & 665.64 & 2.76 & 
665.64 & 3.05 & \bf{659.90} & 
0.87 & 0.87\\SCA3-6 & 652.94 & 3.51 & 
652.94 & 3.45 & \bf{651.09} & 
0.28 & 0.28\\SCA3-7 & 666.15 & 2.78 & 
666.15 & 3.25 & \bf{659.17} & 
1.06 & 1.06\\SCA3-8 & \bf{719.47} & 2.88 & 
719.54 & 3.15 & 719.47 & 0.00
 & 0.01\\SCA3-9 & \bf{681.00} & 3.58 & 
681.00 & 3.46 & 681.00 & 0.00
 & 0.00\\
SCA8-0 & 970.64 & 3.41 & 
976.86 & 3.43 & \bf{961.50} & 
0.95 & 1.60\\SCA8-1 & 1058.43 & 4.08 & 
1060.93 & 3.69 & \bf{1049.65} & 
0.84 & 1.07\\SCA8-2 & 1050.37 & 3.29 & 
1051.93 & 3.21 & \bf{1039.64} & 
1.03 & 1.18\\SCA8-3 & 1011.49 & 3.60 & 
1011.49 & 3.35 & \bf{983.34} & 
2.86 & 2.86\\SCA8-4 & 1067.55 & 3.06 & 
1067.55 & 3.28 & \bf{1065.49} & 
0.19 & 0.19\\SCA8-5 & 1048.59 & 2.86 & 
1048.59 & 3.29 & \bf{1027.08} & 
2.09 & 2.09\\SCA8-6 & 976.69 & 3.49 & 
976.77 & 3.33 & \bf{971.82} & 
0.50 & 0.51\\SCA8-7 & 1070.53 & 3.56 & 
1070.53 & 3.14 & \bf{1051.28} & 
1.83 & 1.83\\SCA8-8 & \bf{1071.18} & 2.92 & 
1076.56 & 3.10 & 1071.18 & 0.00
 & 0.50\\SCA8-9 & 1072.10 & 3.33 & 
1073.64 & 2.88 & \bf{1060.50} & 
1.09 & 1.24\\CON3-0 & 617.59 & 3.39 & 
619.73 & 3.35 & \bf{616.52} & 
0.17 & 0.52\\CON3-1 & 557.21 & 3.41 & 
557.21 & 3.35 & \bf{554.47} & 
0.49 & 0.49\\CON3-2 & 521.38 & 3.50 & 
521.38 & 3.36 & \bf{518.00} & 
0.65 & 0.65\\CON3-3 & 591.20 & 3.20 & 
591.50 & 3.35 & \bf{591.19} & 
0.00 & 0.05\\CON3-4 & 593.78 & 3.48 & 
594.51 & 3.27 & \bf{588.79} & 
0.85 & 0.97\\CON3-5 & \bf{563.70} & 2.77 & 
563.70 & 3.24 & 563.70 & 0.00
 & 0.00\\
CON3-6 & \bf{499.05} & 3.27 & 
503.15 & 3.98 & 499.05 & 0.00
 & 0.82\\CON3-7 & 577.91 & 2.82 & 
579.01 & 3.15 & \bf{576.48} & 
0.25 & 0.44\\CON3-8 & \bf{523.05} & 3.99 & 
524.08 & 3.63 & 523.05 & 0.00
 & 0.20\\CON3-9 & 587.79 & 4.14 & 
587.95 & 3.66 & \bf{578.24} & 
1.65 & 1.68\\CON8-0 & 858.88 & 3.56 & 
870.36 & 3.12 & \bf{857.17} & 
0.20 & 1.54\\CON8-1 & \bf{740.85} & 3.82 & 
743.94 & 3.44 & 740.85 & 0.00
 & 0.42\\CON8-2 & 716.03 & 3.44 & 
716.03 & 3.74 & \bf{712.89} & 
0.44 & 0.44\\CON8-3 & 821.26 & 2.97 & 
822.27 & 3.21 & \bf{811.07} & 
1.26 & 1.38\\CON8-4 & 777.62 & 3.01 & 
780.64 & 2.91 & \bf{772.25} & 
0.70 & 1.09\\CON8-5 & 758.12 & 3.75 & 
758.12 & 3.37 & \bf{754.88} & 
0.43 & 0.43\\CON8-6 & 687.81 & 3.21 & 
687.81 & 3.31 & \bf{678.92} & 
1.31 & 1.31\\CON8-7 & 814.77 & 3.13 & 
814.91 & 3.20 & \bf{811.96} & 
0.35 & 0.36\\CON8-8 & 783.47 & 3.17 & 
785.46 & 3.29 & \bf{767.53} & 
2.08 & 2.34\\CON8-9 & 813.04 & 3.88 & 
816.45 & 3.59 & \bf{809.00} & 
0.50 & 0.92\\\bf{PROM.} & 
\bf{763.96} & \bf{3.33} & \bf{765.28} & \bf{3.31} & \bf{758.54} & \bf{0.65} & \bf{0.82}\\[1ex]\hline
\end{tabular}
\label{table:nonlin}
\end{table} \clearpage
\begin{table}[ht]
\caption{Resultados de la ejecución de la metaheurística IGA, utilizando instancias de SalhiNagy con la configuración -n 100.0 -p 250.0 -cprob 90 -mprob 70}
\centering
\small
\begin{tabular}{c c c c c c c c}
\hline\hline
Instancia & Costo mínimo & Tiempo(seg.) & Costo promedio & Tiempo promedio(seg.) & CME & \%G & \%GP \\ [0.5ex]
\hline
CMT1X & 478.84 & 3.21 & 
479.55 & 3.40 & \bf{470.48} & 
1.78 & 1.93\\CMT1Y & 475.53 & 2.84 & 
478.90 & 3.37 & \bf{470.48} & 
1.07 & 1.79\\CMT2X & 705.12 & 8.20 & 
706.38 & 8.03 & \bf{682.39} & 
3.33 & 3.51\\CMT2Y & 702.79 & 7.52 & 
707.14 & 7.46 & \bf{682.39} & 
2.99 & 3.63\\CMT3X & 735.58 & 17.30 & 
738.93 & 17.23 & \bf{719.06} & 
2.30 & 2.76\\CMT3Y & 725.20 & 16.16 & 
730.15 & 16.89 & \bf{719.06} & 
0.85 & 1.54\\CMT4X & 884.54 & 47.17 & 
892.93 & 46.43 & \bf{854.21} & 
3.55 & 4.53\\CMT4Y & 897.49 & 47.75 & 
906.99 & 47.59 & \bf{852.46} & 
5.28 & 6.40\\CMT5X & 1091.98 & 96.68 & 
1098.78 & 95.91 & \bf{1030.56} & 
5.96 & 6.62\\CMT5Y & 1091.82 & 98.94 & 
1105.65 & 98.36 & \bf{1031.69} & 
5.83 & 7.17\\CMT11X & 861.70 & 29.52 & 
889.88 & 28.83 & \bf{831.09} & 
3.68 & 7.07\\CMT11Y & 877.16 & 31.56 & 
888.71 & 32.42 & \bf{829.85} & 
5.70 & 7.09\\CMT12X & 671.23 & 18.42 & 
672.35 & 17.67 & \bf{658.83} & 
1.88 & 2.05\\CMT12Y & 668.95 & 17.41 & 
673.56 & 17.10 & \bf{660.47} & 
1.28 & 1.98\\\bf{PROM.} & 
\bf{776.28} & \bf{31.62} & \bf{783.56} & \bf{31.48} & \bf{749.50} & \bf{3.25} & \bf{4.15}\\[1ex]\hline
\end{tabular}
\label{table:nonlin}
\end{table} \clearpage
\begin{table}[ht]
\caption{Resultados de la ejecución de la metaheurística IGA, utilizando instancias de Dethloff con la configuración -n 100.0 -p 300.0 -cprob 40 -mprob 70}
\centering
\small
\begin{tabular}{c c c c c c c c}
\hline\hline
Instancia & Costo mínimo & Tiempo(seg.) & Costo promedio & Tiempo promedio(seg.) & CME & \%G & \%GP \\ [0.5ex]
\hline
SCA3-0 & 636.06 & 3.51 & 
638.38 & 3.81 & \bf{635.62} & 
0.07 & 0.43\\SCA3-1 & \bf{697.84} & 3.48 & 
698.50 & 3.67 & 697.84 & 0.00
 & 0.10\\SCA3-2 & 661.13 & 3.56 & 
661.13 & 3.81 & \bf{659.34} & 
0.27 & 0.27\\SCA3-3 & \bf{680.04} & 4.15 & 
680.04 & 4.04 & 680.04 & 0.00
 & 0.00\\
SCA3-4 & \bf{690.50} & 3.50 & 
690.50 & 3.75 & 690.50 & 0.00
 & 0.00\\
SCA3-5 & \bf{659.90} & 3.80 & 
663.75 & 4.48 & 659.90 & 0.00
 & 0.58\\SCA3-6 & 652.94 & 4.28 & 
652.94 & 4.20 & \bf{651.09} & 
0.28 & 0.28\\SCA3-7 & \bf{659.17} & 4.56 & 
664.40 & 3.92 & 659.17 & 0.00
 & 0.79\\SCA3-8 & 719.77 & 3.78 & 
719.77 & 4.42 & \bf{719.47} & 
0.04 & 0.04\\SCA3-9 & \bf{681.00} & 3.83 & 
681.00 & 3.90 & 681.00 & 0.00
 & 0.00\\
SCA8-0 & 975.50 & 3.18 & 
979.36 & 3.58 & \bf{961.50} & 
1.46 & 1.86\\SCA8-1 & 1054.11 & 4.02 & 
1063.61 & 4.09 & \bf{1049.65} & 
0.42 & 1.33\\SCA8-2 & 1050.17 & 3.82 & 
1050.22 & 3.72 & \bf{1039.64} & 
1.01 & 1.02\\SCA8-3 & \bf{983.34} & 4.20 & 
983.34 & 4.32 & 983.34 & 0.00
 & 0.00\\
SCA8-4 & 1068.97 & 3.70 & 
1068.97 & 3.87 & \bf{1065.49} & 
0.33 & 0.33\\SCA8-5 & 1037.06 & 4.43 & 
1037.06 & 3.86 & \bf{1027.08} & 
0.97 & 0.97\\SCA8-6 & 972.48 & 3.66 & 
972.48 & 3.83 & \bf{971.82} & 
0.07 & 0.07\\SCA8-7 & 1070.92 & 4.55 & 
1073.30 & 3.88 & \bf{1051.28} & 
1.87 & 2.09\\SCA8-8 & \bf{1071.18} & 3.40 & 
1071.18 & 3.63 & 1071.18 & 0.00
 & 0.00\\
SCA8-9 & 1073.62 & 3.04 & 
1073.91 & 3.38 & \bf{1060.50} & 
1.24 & 1.26\\CON3-0 & 619.09 & 4.52 & 
619.09 & 4.20 & \bf{616.52} & 
0.42 & 0.42\\CON3-1 & 556.04 & 4.63 & 
557.49 & 4.09 & \bf{554.47} & 
0.28 & 0.54\\CON3-2 & 521.38 & 5.07 & 
521.38 & 4.66 & \bf{518.00} & 
0.65 & 0.65\\CON3-3 & \bf{591.19} & 3.90 & 
591.80 & 4.30 & 591.19 & 0.00
 & 0.10\\CON3-4 & 592.58 & 3.73 & 
592.88 & 3.95 & \bf{588.79} & 
0.64 & 0.69\\CON3-5 & \bf{563.70} & 4.14 & 
564.29 & 4.27 & 563.70 & 0.00
 & 0.10\\CON3-6 & 502.16 & 4.89 & 
502.34 & 4.44 & \bf{499.05} & 
0.62 & 0.66\\CON3-7 & 577.54 & 4.50 & 
579.03 & 3.85 & \bf{576.48} & 
0.18 & 0.44\\CON3-8 & 523.14 & 5.22 & 
524.96 & 4.78 & \bf{523.05} & 
0.02 & 0.37\\CON3-9 & 582.79 & 4.44 & 
582.79 & 4.58 & \bf{578.24} & 
0.79 & 0.79\\CON8-0 & 869.08 & 3.66 & 
869.08 & 4.02 & \bf{857.17} & 
1.39 & 1.39\\CON8-1 & 750.36 & 4.24 & 
750.36 & 4.16 & \bf{740.85} & 
1.28 & 1.28\\CON8-2 & 717.20 & 4.71 & 
717.92 & 4.89 & \bf{712.89} & 
0.60 & 0.71\\CON8-3 & 821.85 & 4.02 & 
821.85 & 4.28 & \bf{811.07} & 
1.33 & 1.33\\CON8-4 & 781.39 & 3.94 & 
784.48 & 4.22 & \bf{772.25} & 
1.18 & 1.58\\CON8-5 & 758.84 & 3.59 & 
758.84 & 3.91 & \bf{754.88} & 
0.52 & 0.52\\CON8-6 & 687.17 & 4.35 & 
687.17 & 4.51 & \bf{678.92} & 
1.22 & 1.22\\CON8-7 & 814.50 & 3.21 & 
814.50 & 3.72 & \bf{811.96} & 
0.31 & 0.31\\CON8-8 & 780.80 & 3.97 & 
782.06 & 3.93 & \bf{767.53} & 
1.73 & 1.89\\CON8-9 & 816.54 & 4.14 & 
818.60 & 4.41 & \bf{809.00} & 
0.93 & 1.19\\\bf{PROM.} & 
\bf{763.08} & \bf{4.03} & \bf{764.12} & \bf{4.08} & \bf{758.54} & \bf{0.55} & \bf{0.69}\\[1ex]\hline
\end{tabular}
\label{table:nonlin}
\end{table} \clearpage
\begin{table}[ht]
\caption{Resultados de la ejecución de la metaheurística IGA, utilizando instancias de SalhiNagy con la configuración -n 100.0 -p 300.0 -cprob 90 -mprob 70}
\centering
\small
\begin{tabular}{c c c c c c c c}
\hline\hline
Instancia & Costo mínimo & Tiempo(seg.) & Costo promedio & Tiempo promedio(seg.) & CME & \%G & \%GP \\ [0.5ex]
\hline
CMT1X & 475.37 & 3.73 & 
475.75 & 4.04 & \bf{470.48} & 
1.04 & 1.12\\CMT1Y & 474.87 & 4.32 & 
476.92 & 4.21 & \bf{470.48} & 
0.93 & 1.37\\CMT2X & 700.82 & 9.56 & 
702.70 & 9.73 & \bf{682.39} & 
2.70 & 2.98\\CMT2Y & 689.83 & 9.22 & 
689.83 & 8.99 & \bf{682.39} & 
1.09 & 1.09\\CMT3X & 732.71 & 21.52 & 
735.34 & 21.20 & \bf{719.06} & 
1.90 & 2.26\\CMT3Y & 730.22 & 19.78 & 
736.26 & 20.09 & \bf{719.06} & 
1.55 & 2.39\\CMT4X & 901.06 & 58.53 & 
903.96 & 57.57 & \bf{854.21} & 
5.48 & 5.82\\CMT4Y & 898.18 & 55.85 & 
905.96 & 57.09 & \bf{852.46} & 
5.36 & 6.28\\CMT5X & 1076.19 & 116.82 & 
1096.82 & 116.66 & \bf{1030.56} & 
4.43 & 6.43\\CMT5Y & 1101.15 & 118.27 & 
1109.70 & 118.89 & \bf{1031.69} & 
6.73 & 7.56\\CMT11X & 880.80 & 35.05 & 
893.72 & 35.56 & \bf{831.09} & 
5.98 & 7.54\\CMT11Y & 849.85 & 39.94 & 
875.97 & 39.20 & \bf{829.85} & 
2.41 & 5.56\\CMT12X & 674.29 & 22.12 & 
675.35 & 21.57 & \bf{658.83} & 
2.35 & 2.51\\CMT12Y & 673.67 & 20.61 & 
673.99 & 20.64 & \bf{660.47} & 
2.00 & 2.05\\\bf{PROM.} & 
\bf{775.64} & \bf{38.24} & \bf{782.31} & \bf{38.25} & \bf{749.50} & \bf{3.14} & \bf{3.93}\\[1ex]\hline
\end{tabular}
\label{table:nonlin}
\end{table} \clearpage
\begin{table}[ht]
\caption{Resultados de la ejecución de la metaheurística IGA, utilizando instancias de Dethloff con la configuración -n 100.0 -p 350.0 -cprob 40 -mprob 70}
\centering
\small
\begin{tabular}{c c c c c c c c}
\hline\hline
Instancia & Costo mínimo & Tiempo(seg.) & Costo promedio & Tiempo promedio(seg.) & CME & \%G & \%GP \\ [0.5ex]
\hline
SCA3-0 & 640.55 & 4.18 & 
640.55 & 4.57 & \bf{635.62} & 
0.78 & 0.78\\SCA3-1 & \bf{697.84} & 4.12 & 
697.84 & 4.67 & 697.84 & 0.00
 & 0.00\\
SCA3-2 & \bf{659.34} & 4.56 & 
659.92 & 4.38 & 659.34 & 0.00
 & 0.09\\SCA3-3 & \bf{680.04} & 4.43 & 
680.60 & 4.34 & 680.04 & 0.00
 & 0.08\\SCA3-4 & \bf{690.50} & 3.68 & 
690.50 & 4.04 & 690.50 & 0.00
 & 0.00\\
SCA3-5 & \bf{659.90} & 4.52 & 
661.33 & 4.69 & 659.90 & 0.00
 & 0.22\\SCA3-6 & 652.94 & 5.34 & 
652.94 & 5.06 & \bf{651.09} & 
0.28 & 0.28\\SCA3-7 & 666.15 & 4.72 & 
666.15 & 4.59 & \bf{659.17} & 
1.06 & 1.06\\SCA3-8 & 719.77 & 4.53 & 
720.99 & 4.59 & \bf{719.47} & 
0.04 & 0.21\\SCA3-9 & \bf{681.00} & 4.70 & 
681.00 & 4.79 & 681.00 & 0.00
 & 0.00\\
SCA8-0 & 973.22 & 4.72 & 
973.22 & 4.60 & \bf{961.50} & 
1.22 & 1.22\\SCA8-1 & 1062.84 & 4.11 & 
1066.85 & 4.40 & \bf{1049.65} & 
1.26 & 1.64\\SCA8-2 & 1048.87 & 4.90 & 
1050.21 & 5.18 & \bf{1039.64} & 
0.89 & 1.02\\SCA8-3 & 1005.65 & 3.93 & 
1005.73 & 4.88 & \bf{983.34} & 
2.27 & 2.28\\SCA8-4 & 1071.44 & 3.86 & 
1071.44 & 4.03 & \bf{1065.49} & 
0.56 & 0.56\\SCA8-5 & 1050.09 & 4.35 & 
1050.09 & 4.29 & \bf{1027.08} & 
2.24 & 2.24\\SCA8-6 & 973.30 & 4.40 & 
974.44 & 4.67 & \bf{971.82} & 
0.15 & 0.27\\SCA8-7 & 1072.17 & 3.80 & 
1072.17 & 3.98 & \bf{1051.28} & 
1.99 & 1.99\\SCA8-8 & 1075.00 & 4.60 & 
1077.35 & 4.17 & \bf{1071.18} & 
0.36 & 0.58\\SCA8-9 & 1067.26 & 3.72 & 
1069.57 & 4.71 & \bf{1060.50} & 
0.64 & 0.86\\CON3-0 & 619.09 & 4.73 & 
619.64 & 5.19 & \bf{616.52} & 
0.42 & 0.51\\CON3-1 & 556.04 & 5.18 & 
557.33 & 5.08 & \bf{554.47} & 
0.28 & 0.52\\CON3-2 & 521.38 & 5.23 & 
521.38 & 5.46 & \bf{518.00} & 
0.65 & 0.65\\CON3-3 & 591.20 & 5.70 & 
591.27 & 5.08 & \bf{591.19} & 
0.00 & 0.01\\CON3-4 & 591.43 & 4.63 & 
592.00 & 4.59 & \bf{588.79} & 
0.45 & 0.55\\CON3-5 & \bf{563.70} & 4.76 & 
563.70 & 4.66 & 563.70 & 0.00
 & 0.00\\
CON3-6 & 502.16 & 5.42 & 
503.18 & 5.12 & \bf{499.05} & 
0.62 & 0.83\\CON3-7 & 577.54 & 5.04 & 
578.88 & 4.51 & \bf{576.48} & 
0.18 & 0.42\\CON3-8 & \bf{523.05} & 5.40 & 
523.48 & 5.36 & 523.05 & 0.00
 & 0.08\\CON3-9 & 582.79 & 4.52 & 
586.78 & 5.03 & \bf{578.24} & 
0.79 & 1.48\\CON8-0 & 865.86 & 4.82 & 
869.45 & 4.84 & \bf{857.17} & 
1.01 & 1.43\\CON8-1 & 749.07 & 4.41 & 
753.45 & 4.71 & \bf{740.85} & 
1.11 & 1.70\\CON8-2 & 713.44 & 4.80 & 
713.44 & 5.34 & \bf{712.89} & 
0.08 & 0.08\\CON8-3 & \bf{811.07} & 4.26 & 
811.07 & 4.14 & 811.07 & 0.00
 & 0.00\\
CON8-4 & 772.76 & 4.38 & 
772.76 & 4.61 & \bf{772.25} & 
0.07 & 0.07\\CON8-5 & 754.95 & 4.52 & 
754.95 & 5.08 & \bf{754.88} & 
0.01 & 0.01\\CON8-6 & 693.25 & 5.88 & 
694.16 & 4.89 & \bf{678.92} & 
2.11 & 2.25\\CON8-7 & 814.50 & 3.88 & 
814.78 & 4.72 & \bf{811.96} & 
0.31 & 0.35\\CON8-8 & 771.26 & 4.15 & 
771.26 & 4.84 & \bf{767.53} & 
0.49 & 0.49\\CON8-9 & 819.79 & 4.33 & 
819.79 & 4.50 & \bf{809.00} & 
1.33 & 1.33\\\bf{PROM.} & 
\bf{763.56} & \bf{4.58} & \bf{764.39} & \bf{4.71} & \bf{758.54} & \bf{0.59} & \bf{0.70}\\[1ex]\hline
\end{tabular}
\label{table:nonlin}
\end{table} \clearpage
\begin{table}[ht]
\caption{Resultados de la ejecución de la metaheurística IGA, utilizando instancias de SalhiNagy con la configuración -n 100.0 -p 350.0 -cprob 90 -mprob 70}
\centering
\small
\begin{tabular}{c c c c c c c c}
\hline\hline
Instancia & Costo mínimo & Tiempo(seg.) & Costo promedio & Tiempo promedio(seg.) & CME & \%G & \%GP \\ [0.5ex]
\hline
CMT1X & 475.37 & 4.94 & 
476.89 & 4.91 & \bf{470.48} & 
1.04 & 1.36\\CMT1Y & 477.21 & 3.96 & 
478.35 & 4.15 & \bf{470.48} & 
1.43 & 1.67\\CMT2X & 695.58 & 10.70 & 
701.59 & 11.20 & \bf{682.39} & 
1.93 & 2.81\\CMT2Y & 692.39 & 11.43 & 
703.65 & 11.01 & \bf{682.39} & 
1.47 & 3.12\\CMT3X & 728.88 & 24.93 & 
732.16 & 24.99 & \bf{719.06} & 
1.37 & 1.82\\CMT3Y & 733.44 & 23.79 & 
736.68 & 23.88 & \bf{719.06} & 
2.00 & 2.45\\CMT4X & 896.29 & 64.68 & 
899.29 & 64.94 & \bf{854.21} & 
4.93 & 5.28\\CMT4Y & 872.44 & 66.17 & 
895.17 & 66.64 & \bf{852.46} & 
2.34 & 5.01\\CMT5X & 1091.97 & 135.16 & 
1097.95 & 134.75 & \bf{1030.56} & 
5.96 & 6.54\\CMT5Y & 1074.86 & 135.76 & 
1099.76 & 135.59 & \bf{1031.69} & 
4.18 & 6.60\\CMT11X & 881.94 & 41.26 & 
889.67 & 41.73 & \bf{831.09} & 
6.12 & 7.05\\CMT11Y & 863.57 & 45.09 & 
874.57 & 46.31 & \bf{829.85} & 
4.06 & 5.39\\CMT12X & 673.00 & 23.98 & 
674.30 & 24.90 & \bf{658.83} & 
2.15 & 2.35\\CMT12Y & 670.65 & 24.33 & 
673.51 & 23.98 & \bf{660.47} & 
1.54 & 1.97\\\bf{PROM.} & 
\bf{773.40} & \bf{44.01} & \bf{780.97} & \bf{44.21} & \bf{749.50} & \bf{2.89} & \bf{3.82}\\[1ex]\hline
\end{tabular}
\label{table:nonlin}
\end{table} \clearpage
\begin{table}[ht]
\caption{Resultados de la ejecución de la metaheurística IGA, utilizando instancias de Dethloff con la configuración -n 100.0 -p 400.0 -cprob 40 -mprob 70}
\centering
\small
\begin{tabular}{c c c c c c c c}
\hline\hline
Instancia & Costo mínimo & Tiempo(seg.) & Costo promedio & Tiempo promedio(seg.) & CME & \%G & \%GP \\ [0.5ex]
\hline
SCA3-0 & 636.06 & 5.45 & 
636.06 & 5.45 & \bf{635.62} & 
0.07 & 0.07\\SCA3-1 & \bf{697.84} & 4.68 & 
697.84 & 5.72 & 697.84 & 0.00
 & 0.00\\
SCA3-2 & \bf{659.34} & 5.29 & 
660.24 & 5.37 & 659.34 & 0.00
 & 0.14\\SCA3-3 & \bf{680.04} & 5.51 & 
680.04 & 5.29 & 680.04 & 0.00
 & 0.00\\
SCA3-4 & \bf{690.50} & 4.83 & 
690.50 & 4.75 & 690.50 & 0.00
 & 0.00\\
SCA3-5 & 662.75 & 4.79 & 
662.75 & 5.43 & \bf{659.90} & 
0.43 & 0.43\\SCA3-6 & 652.94 & 5.28 & 
652.94 & 5.26 & \bf{651.09} & 
0.28 & 0.28\\SCA3-7 & 666.15 & 5.40 & 
666.15 & 5.28 & \bf{659.17} & 
1.06 & 1.06\\SCA3-8 & \bf{719.47} & 5.30 & 
720.60 & 5.04 & 719.47 & 0.00
 & 0.16\\SCA3-9 & \bf{681.00} & 5.57 & 
681.00 & 5.74 & 681.00 & 0.00
 & 0.00\\
SCA8-0 & 968.79 & 5.00 & 
973.09 & 5.10 & \bf{961.50} & 
0.76 & 1.21\\SCA8-1 & 1056.27 & 5.51 & 
1064.74 & 5.47 & \bf{1049.65} & 
0.63 & 1.44\\SCA8-2 & 1050.37 & 6.31 & 
1050.79 & 5.83 & \bf{1039.64} & 
1.03 & 1.07\\SCA8-3 & 1000.05 & 5.59 & 
1004.35 & 5.78 & \bf{983.34} & 
1.70 & 2.14\\SCA8-4 & 1069.87 & 4.73 & 
1069.87 & 4.81 & \bf{1065.49} & 
0.41 & 0.41\\SCA8-5 & 1042.28 & 5.07 & 
1042.41 & 5.28 & \bf{1027.08} & 
1.48 & 1.49\\SCA8-6 & 972.48 & 5.30 & 
972.48 & 5.42 & \bf{971.82} & 
0.07 & 0.07\\SCA8-7 & 1070.71 & 5.20 & 
1073.24 & 5.13 & \bf{1051.28} & 
1.85 & 2.09\\SCA8-8 & \bf{1071.18} & 5.57 & 
1071.18 & 5.19 & 1071.18 & 0.00
 & 0.00\\
SCA8-9 & 1065.60 & 4.29 & 
1065.60 & 4.84 & \bf{1060.50} & 
0.48 & 0.48\\CON3-0 & 619.09 & 5.64 & 
620.34 & 5.57 & \bf{616.52} & 
0.42 & 0.62\\CON3-1 & \bf{554.47} & 6.76 & 
557.45 & 5.93 & 554.47 & 0.00
 & 0.54\\CON3-2 & 521.38 & 6.03 & 
521.38 & 6.14 & \bf{518.00} & 
0.65 & 0.65\\CON3-3 & \bf{591.19} & 5.52 & 
591.20 & 5.53 & 591.19 & 0.00
 & 0.00\\CON3-4 & 589.32 & 5.67 & 
591.76 & 5.72 & \bf{588.79} & 
0.09 & 0.51\\CON3-5 & \bf{563.70} & 6.22 & 
563.70 & 5.86 & 563.70 & 0.00
 & 0.00\\
CON3-6 & 502.16 & 5.41 & 
503.15 & 6.17 & \bf{499.05} & 
0.62 & 0.82\\CON3-7 & 578.22 & 5.48 & 
578.27 & 4.89 & \bf{576.48} & 
0.30 & 0.31\\CON3-8 & 523.14 & 5.89 & 
524.23 & 5.51 & \bf{523.05} & 
0.02 & 0.23\\CON3-9 & 587.23 & 6.10 & 
587.67 & 5.79 & \bf{578.24} & 
1.55 & 1.63\\CON8-0 & 872.74 & 5.83 & 
872.79 & 5.53 & \bf{857.17} & 
1.82 & 1.82\\CON8-1 & 744.43 & 5.17 & 
746.71 & 5.78 & \bf{740.85} & 
0.48 & 0.79\\CON8-2 & 713.60 & 5.39 & 
713.60 & 5.69 & \bf{712.89} & 
0.10 & 0.10\\CON8-3 & 814.50 & 5.01 & 
819.94 & 5.53 & \bf{811.07} & 
0.42 & 1.09\\CON8-4 & 780.46 & 6.14 & 
780.46 & 5.16 & \bf{772.25} & 
1.06 & 1.06\\CON8-5 & 758.12 & 5.33 & 
758.12 & 5.56 & \bf{754.88} & 
0.43 & 0.43\\CON8-6 & 690.85 & 4.89 & 
694.57 & 5.36 & \bf{678.92} & 
1.76 & 2.30\\CON8-7 & 814.77 & 5.47 & 
814.98 & 5.53 & \bf{811.96} & 
0.35 & 0.37\\CON8-8 & 784.28 & 5.24 & 
785.02 & 5.46 & \bf{767.53} & 
2.18 & 2.28\\CON8-9 & 818.03 & 5.75 & 
819.85 & 5.65 & \bf{809.00} & 
1.12 & 1.34\\\bf{PROM.} & 
\bf{763.38} & \bf{5.44} & \bf{764.53} & \bf{5.46} & \bf{758.54} & \bf{0.59} & \bf{0.74}\\[1ex]\hline
\end{tabular}
\label{table:nonlin}
\end{table} \clearpage
\begin{table}[ht]
\caption{Resultados de la ejecución de la metaheurística IGA, utilizando instancias de SalhiNagy con la configuración -n 100.0 -p 400.0 -cprob 90 -mprob 70}
\centering
\small
\begin{tabular}{c c c c c c c c}
\hline\hline
Instancia & Costo mínimo & Tiempo(seg.) & Costo promedio & Tiempo promedio(seg.) & CME & \%G & \%GP \\ [0.5ex]
\hline
CMT1X & 474.87 & 5.72 & 
477.08 & 5.70 & \bf{470.48} & 
0.93 & 1.40\\CMT1Y & 474.72 & 5.17 & 
475.24 & 5.79 & \bf{470.48} & 
0.90 & 1.01\\CMT2X & 695.24 & 13.06 & 
698.76 & 12.24 & \bf{682.39} & 
1.88 & 2.40\\CMT2Y & 702.89 & 12.00 & 
704.39 & 12.24 & \bf{682.39} & 
3.00 & 3.22\\CMT3X & 728.89 & 27.91 & 
733.88 & 27.73 & \bf{719.06} & 
1.37 & 2.06\\CMT3Y & 728.89 & 26.94 & 
735.28 & 27.20 & \bf{719.06} & 
1.37 & 2.26\\CMT4X & 898.65 & 73.52 & 
901.32 & 75.55 & \bf{854.21} & 
5.20 & 5.52\\CMT4Y & 895.33 & 76.81 & 
898.49 & 77.09 & \bf{852.46} & 
5.03 & 5.40\\CMT5X & 1100.85 & 155.68 & 
1104.20 & 155.24 & \bf{1030.56} & 
6.82 & 7.15\\CMT5Y & 1094.66 & 155.38 & 
1100.48 & 156.43 & \bf{1031.69} & 
6.10 & 6.67\\CMT11X & 886.33 & 46.99 & 
888.35 & 50.01 & \bf{831.09} & 
6.65 & 6.89\\CMT11Y & 886.53 & 51.75 & 
893.40 & 52.06 & \bf{829.85} & 
6.83 & 7.66\\CMT12X & 673.28 & 28.62 & 
674.99 & 28.86 & \bf{658.83} & 
2.19 & 2.45\\CMT12Y & 673.13 & 27.23 & 
674.17 & 27.53 & \bf{660.47} & 
1.92 & 2.07\\\bf{PROM.} & 
\bf{779.59} & \bf{50.48} & \bf{782.86} & \bf{50.98} & \bf{749.50} & \bf{3.59} & \bf{4.01}\\[1ex]\hline
\end{tabular}
\label{table:nonlin}
\end{table} \clearpage
\begin{table}[ht]
\caption{Resultados de la ejecución de la metaheurística IGA, utilizando instancias de Dethloff con la configuración -n 100 -p 300 -cprob 40 -mprob 70}
\centering
\small
\begin{tabular}{c c c c c c c c}
\hline\hline
Instancia & Costo mínimo & Tiempo(seg.) & Costo promedio & Tiempo promedio(seg.) & CME & \%G & \%GP \\ [0.5ex]
\hline
SCA3-0 & 636.06 & 4.43 & 
639.50 & 4.24 & \bf{635.62} & 
0.07 & 0.61\\SCA3-1 & \bf{697.84} & 4.67 & 
698.20 & 4.14 & 697.84 & 0.00
 & 0.05\\SCA3-2 & 661.13 & 3.77 & 
662.05 & 3.93 & \bf{659.34} & 
0.27 & 0.41\\SCA3-3 & \bf{680.04} & 3.81 & 
680.37 & 4.08 & 680.04 & 0.00
 & 0.05\\SCA3-4 & \bf{690.50} & 4.08 & 
690.50 & 3.91 & 690.50 & 0.00
 & 0.00\\
SCA3-5 & \bf{659.90} & 3.92 & 
663.87 & 4.25 & 659.90 & 0.00
 & 0.60\\SCA3-6 & 652.47 & 4.24 & 
652.92 & 4.07 & \bf{651.09} & 
0.21 & 0.28\\SCA3-7 & \bf{659.17} & 4.27 & 
664.29 & 3.93 & 659.17 & 0.00
 & 0.78\\SCA3-8 & \bf{719.47} & 4.15 & 
720.10 & 4.11 & 719.47 & 0.00
 & 0.09\\SCA3-9 & \bf{681.00} & 4.51 & 
681.00 & 3.93 & 681.00 & 0.00
 & 0.00\\
SCA8-0 & 973.03 & 4.16 & 
974.19 & 4.15 & \bf{961.50} & 
1.20 & 1.32\\SCA8-1 & 1059.16 & 3.77 & 
1061.80 & 3.89 & \bf{1049.65} & 
0.91 & 1.16\\SCA8-2 & 1050.37 & 4.16 & 
1051.48 & 4.12 & \bf{1039.64} & 
1.03 & 1.14\\SCA8-3 & 985.60 & 3.77 & 
1008.15 & 4.25 & \bf{983.34} & 
0.23 & 2.52\\SCA8-4 & 1069.71 & 4.27 & 
1072.14 & 3.83 & \bf{1065.49} & 
0.40 & 0.62\\SCA8-5 & 1043.52 & 3.46 & 
1045.61 & 3.98 & \bf{1027.08} & 
1.60 & 1.80\\SCA8-6 & 976.69 & 3.98 & 
976.69 & 3.99 & \bf{971.82} & 
0.50 & 0.50\\SCA8-7 & 1067.03 & 3.36 & 
1068.46 & 3.92 & \bf{1051.28} & 
1.50 & 1.63\\SCA8-8 & \bf{1071.18} & 4.20 & 
1071.31 & 3.94 & 1071.18 & 0.00
 & 0.01\\SCA8-9 & 1068.10 & 3.66 & 
1071.91 & 3.82 & \bf{1060.50} & 
0.72 & 1.08\\CON3-0 & 617.59 & 4.40 & 
619.16 & 4.26 & \bf{616.52} & 
0.17 & 0.43\\CON3-1 & 556.04 & 7.15 & 
557.63 & 4.34 & \bf{554.47} & 
0.28 & 0.57\\CON3-2 & 521.38 & 4.56 & 
521.38 & 4.57 & \bf{518.00} & 
0.65 & 0.65\\CON3-3 & 591.20 & 5.13 & 
591.31 & 4.10 & \bf{591.19} & 
0.00 & 0.02\\CON3-4 & 589.32 & 5.40 & 
592.13 & 4.23 & \bf{588.79} & 
0.09 & 0.57\\CON3-5 & \bf{563.70} & 3.74 & 
565.23 & 4.29 & 563.70 & 0.00
 & 0.27\\CON3-6 & 500.88 & 4.04 & 
502.67 & 4.29 & \bf{499.05} & 
0.37 & 0.73\\CON3-7 & 577.54 & 3.67 & 
578.91 & 3.93 & \bf{576.48} & 
0.18 & 0.42\\CON3-8 & \bf{523.05} & 4.02 & 
524.11 & 4.36 & 523.05 & 0.00
 & 0.20\\CON3-9 & 582.79 & 4.02 & 
583.33 & 4.43 & \bf{578.24} & 
0.79 & 0.88\\CON8-0 & 869.15 & 4.99 & 
870.06 & 4.17 & \bf{857.17} & 
1.40 & 1.50\\CON8-1 & 741.70 & 4.60 & 
751.80 & 4.31 & \bf{740.85} & 
0.11 & 1.48\\CON8-2 & 713.44 & 4.64 & 
713.49 & 4.40 & \bf{712.89} & 
0.08 & 0.08\\CON8-3 & \bf{811.07} & 4.72 & 
811.07 & 4.23 & 811.07 & 0.00
 & 0.00\\
CON8-4 & \bf{772.25} & 4.74 & 
777.48 & 3.97 & 772.25 & 0.00
 & 0.68\\CON8-5 & 759.44 & 4.56 & 
759.60 & 4.04 & \bf{754.88} & 
0.60 & 0.63\\CON8-6 & 685.80 & 4.83 & 
689.52 & 4.20 & \bf{678.92} & 
1.01 & 1.56\\CON8-7 & 814.79 & 3.67 & 
814.79 & 3.84 & \bf{811.96} & 
0.35 & 0.35\\CON8-8 & 780.80 & 4.10 & 
784.11 & 4.13 & \bf{767.53} & 
1.73 & 2.16\\CON8-9 & 811.66 & 4.06 & 
811.81 & 4.36 & \bf{809.00} & 
0.33 & 0.35\\\bf{PROM.} & 
\bf{762.14} & \bf{4.29} & \bf{764.35} & \bf{4.12} & \bf{758.54} & \bf{0.42} & \bf{0.70}\\[1ex]\hline
\end{tabular}
\label{table:nonlin}
\end{table} \clearpage
\begin{table}[ht]
\caption{Resultados de la ejecución de la metaheurística IGA, utilizando instancias de SalhiNagy con la configuración -n 100 -p 350 -cprob 90 -mprob 70}
\centering
\small
\begin{tabular}{c c c c c c c c}
\hline\hline
Instancia & Costo mínimo & Tiempo(seg.) & Costo promedio & Tiempo promedio(seg.) & CME & \%G & \%GP \\ [0.5ex]
\hline
CMT1X & 476.38 & 5.06 & 
476.71 & 5.04 & \bf{470.48} & 
1.25 & 1.32\\CMT1Y & 472.87 & 4.56 & 
477.66 & 4.64 & \bf{470.48} & 
0.51 & 1.53\\CMT2X & 697.28 & 10.72 & 
705.32 & 11.00 & \bf{682.39} & 
2.18 & 3.36\\CMT2Y & 697.76 & 10.15 & 
706.51 & 10.63 & \bf{682.39} & 
2.25 & 3.54\\CMT3X & 728.50 & 24.66 & 
736.12 & 24.48 & \bf{719.06} & 
1.31 & 2.37\\CMT3Y & 725.30 & 24.70 & 
734.65 & 24.09 & \bf{719.06} & 
0.87 & 2.17\\CMT4X & 880.66 & 63.16 & 
896.91 & 65.36 & \bf{854.21} & 
3.10 & 5.00\\CMT4Y & 892.05 & 65.54 & 
902.53 & 66.57 & \bf{852.46} & 
4.64 & 5.87\\CMT5X & 1068.63 & 134.22 & 
1098.41 & 134.47 & \bf{1030.56} & 
3.69 & 6.58\\CMT5Y & 1087.40 & 137.00 & 
1104.05 & 137.40 & \bf{1031.69} & 
5.40 & 7.01\\CMT11X & 858.94 & 40.64 & 
891.06 & 41.90 & \bf{831.09} & 
3.35 & 7.22\\CMT11Y & 848.68 & 45.25 & 
877.11 & 45.61 & \bf{829.85} & 
2.27 & 5.69\\CMT12X & 668.48 & 24.83 & 
673.47 & 24.73 & \bf{658.83} & 
1.46 & 2.22\\CMT12Y & 672.60 & 24.47 & 
674.22 & 24.43 & \bf{660.47} & 
1.84 & 2.08\\\bf{PROM.} & 
\bf{769.68} & \bf{43.93} & \bf{782.48} & \bf{44.31} & \bf{749.50} & \bf{2.44} & \bf{4.00}\\[1ex]\hline
\end{tabular}
\label{table:nonlin}
\end{table} \clearpage
\begin{table}[ht]
\caption{Resultados de la ejecución de la metaheurística SCA, utilizando instancias de Dethloff con la configuración -n 150 -b 10 -y 0.1}
\centering
\small
\begin{tabular}{c c c c c c c c}
\hline\hline
Instancia & Costo mínimo & Tiempo(seg.) & Costo promedio & Tiempo promedio(seg.) & CME & \%G & \%GP \\ [0.5ex]
\hline
SCA3-0 & 640.55 & 2.45 & 
640.55 & 2.95 & \bf{635.62} & 
0.78 & 0.78\\SCA3-1 & \bf{697.84} & 1.91 & 
700.30 & 3.70 & 697.84 & 0.00
 & 0.35\\SCA3-2 & \bf{659.34} & 3.01 & 
664.13 & 3.74 & 659.34 & 0.00
 & 0.73\\SCA3-3 & \bf{680.04} & 3.26 & 
681.08 & 3.54 & 680.04 & 0.00
 & 0.15\\SCA3-4 & \bf{690.50} & 4.81 & 
690.98 & 3.14 & 690.50 & 0.00
 & 0.07\\SCA3-5 & 665.04 & 2.29 & 
673.31 & 2.57 & \bf{659.90} & 
0.78 & 2.03\\SCA3-6 & 652.94 & 2.97 & 
653.52 & 3.03 & \bf{651.09} & 
0.28 & 0.37\\SCA3-7 & 666.15 & 3.51 & 
670.58 & 3.39 & \bf{659.17} & 
1.06 & 1.73\\SCA3-8 & \bf{719.47} & 3.88 & 
719.70 & 3.31 & 719.47 & 0.00
 & 0.03\\SCA3-9 & \bf{681.00} & 3.80 & 
682.57 & 4.01 & 681.00 & 0.00
 & 0.23\\SCA8-0 & 965.26 & 9.36 & 
985.82 & 8.54 & \bf{961.50} & 
0.39 & 2.53\\SCA8-1 & 1053.57 & 17.13 & 
1069.26 & 9.85 & \bf{1049.65} & 
0.37 & 1.87\\SCA8-2 & 1050.37 & 13.90 & 
1052.30 & 9.72 & \bf{1039.64} & 
1.03 & 1.22\\SCA8-3 & 1014.10 & 8.64 & 
1027.24 & 7.63 & \bf{983.34} & 
3.13 & 4.46\\SCA8-4 & \bf{1065.49} & 9.08 & 
1080.71 & 8.69 & 1065.49 & 0.00
 & 1.43\\SCA8-5 & 1040.18 & 13.45 & 
1057.39 & 10.34 & \bf{1027.08} & 
1.28 & 2.95\\SCA8-6 & 972.48 & 10.50 & 
980.72 & 9.08 & \bf{971.82} & 
0.07 & 0.92\\SCA8-7 & 1067.49 & 9.42 & 
1073.92 & 9.24 & \bf{1051.28} & 
1.54 & 2.15\\SCA8-8 & \bf{1071.18} & 8.71 & 
1090.33 & 8.87 & 1071.18 & 0.00
 & 1.79\\SCA8-9 & 1073.62 & 11.62 & 
1082.97 & 13.35 & \bf{1060.50} & 
1.24 & 2.12\\CON3-0 & 617.59 & 1.85 & 
621.83 & 2.32 & \bf{616.52} & 
0.17 & 0.86\\CON3-1 & \bf{554.47} & 3.18 & 
559.74 & 2.97 & 554.47 & 0.00
 & 0.95\\CON3-2 & 521.38 & 3.99 & 
521.60 & 2.64 & \bf{518.00} & 
0.65 & 0.69\\CON3-3 & \bf{591.19} & 1.59 & 
591.40 & 2.89 & 591.19 & 0.00
 & 0.03\\CON3-4 & \bf{588.79} & 3.17 & 
591.10 & 2.88 & 588.79 & 0.00
 & 0.39\\CON3-5 & 564.88 & 3.59 & 
565.86 & 2.46 & \bf{563.70} & 
0.21 & 0.38\\CON3-6 & 502.16 & 2.07 & 
502.95 & 1.97 & \bf{499.05} & 
0.62 & 0.78\\CON3-7 & \bf{576.48} & 3.92 & 
581.42 & 3.52 & 576.48 & 0.00
 & 0.86\\CON3-8 & \bf{523.05} & 3.21 & 
523.70 & 3.46 & 523.05 & 0.00
 & 0.12\\CON3-9 & 578.98 & 2.21 & 
587.54 & 2.65 & \bf{578.24} & 
0.13 & 1.61\\CON8-0 & 869.08 & 5.89 & 
884.50 & 8.02 & \bf{857.17} & 
1.39 & 3.19\\CON8-1 & \bf{740.85} & 9.85 & 
748.62 & 9.70 & 740.85 & 0.00
 & 1.05\\CON8-2 & 713.05 & 12.08 & 
717.21 & 8.79 & \bf{712.89} & 
0.02 & 0.61\\CON8-3 & 815.71 & 15.30 & 
830.53 & 10.34 & \bf{811.07} & 
0.57 & 2.40\\CON8-4 & 777.24 & 10.68 & 
787.88 & 10.06 & \bf{772.25} & 
0.65 & 2.02\\CON8-5 & 754.95 & 9.39 & 
763.88 & 10.53 & \bf{754.88} & 
0.01 & 1.19\\CON8-6 & 686.34 & 8.60 & 
694.54 & 8.24 & \bf{678.92} & 
1.09 & 2.30\\CON8-7 & 814.50 & 11.56 & 
816.88 & 10.20 & \bf{811.96} & 
0.31 & 0.61\\CON8-8 & 785.30 & 8.20 & 
790.54 & 7.04 & \bf{767.53} & 
2.32 & 3.00\\CON8-9 & 813.10 & 8.80 & 
822.63 & 9.10 & \bf{809.00} & 
0.51 & 1.69\\\bf{PROM.} & 
\bf{762.89} & \bf{6.82} & \bf{769.54} & \bf{6.21} & \bf{758.54} & \bf{0.51} & \bf{1.32}\\[1ex]\hline
\end{tabular}
\label{table:nonlin}
\end{table} \clearpage
\begin{table}[ht]
\caption{Resultados de la ejecución de la metaheurística SCA, utilizando instancias de SalhiNagy con la configuración -n 250 -b 10 -y 0.2}
\centering
\small
\begin{tabular}{c c c c c c c c}
\hline\hline
Instancia & Costo mínimo & Tiempo(seg.) & Costo promedio & Tiempo promedio(seg.) & CME & \%G & \%GP \\ [0.5ex]
\hline
CMT1X & 470.67 & 1.80 & 
473.84 & 1.73 & \bf{470.48} & 
0.04 & 0.71\\CMT1Y & 472.37 & 2.60 & 
473.23 & 2.43 & \bf{470.48} & 
0.40 & 0.59\\CMT2X & 699.80 & 14.75 & 
708.96 & 17.39 & \bf{682.39} & 
2.55 & 3.89\\CMT2Y & 700.28 & 18.89 & 
711.13 & 14.05 & \bf{682.39} & 
2.62 & 4.21\\CMT3X & 726.98 & 30.34 & 
738.05 & 35.62 & \bf{719.06} & 
1.10 & 2.64\\CMT3Y & 727.77 & 27.80 & 
737.32 & 36.29 & \bf{719.06} & 
1.21 & 2.54\\CMT4X & 866.65 & 421.30 & 
905.41 & 231.34 & \bf{854.21} & 
1.46 & 5.99\\CMT4Y & 872.44 & 321.56 & 
905.47 & 232.94 & \bf{852.46} & 
2.34 & 6.22\\CMT5X & 1072.03 & 773.75 & 
1110.77 & 1071.87 & \bf{1030.56} & 
4.02 & 7.78\\CMT5Y & 1077.14 & 1142.85 & 
1110.33 & 949.78 & \bf{1031.69} & 
4.41 & 7.62\\CMT11X & 874.80 & 68.83 & 
903.88 & 40.83 & \bf{831.09} & 
5.26 & 8.76\\CMT11Y & 880.42 & 92.37 & 
906.81 & 44.55 & \bf{829.85} & 
6.09 & 9.27\\CMT12X & 673.49 & 232.82 & 
682.30 & 59.59 & \bf{658.83} & 
2.23 & 3.56\\CMT12Y & 675.05 & 74.83 & 
684.71 & 63.02 & \bf{660.47} & 
2.21 & 3.67\\\bf{PROM.} & 
\bf{770.71} & \bf{230.32} & \bf{789.44} & \bf{200.10} & \bf{749.50} & \bf{2.57} & \bf{4.82}\\[1ex]\hline
\end{tabular}
\label{table:nonlin}
\end{table} \clearpage
\begin{table}[ht]
\caption{Resultados de la ejecución de la metaheurística , utilizando instancias de con la configuración }
\centering
\small
\begin{tabular}{c c c c c c c c}
\hline\hline
Instancia & Costo mínimo & Tiempo(seg.) & Costo promedio & Tiempo promedio(seg.) & CME & \%G & \%GP \\ [0.5ex]
\hline
[1ex]\hline
\end{tabular}
\label{table:nonlin}
\end{table} \clearpage
\begin{table}[ht]
\caption{Resultados de la ejecución de la metaheurística -n, utilizando instancias de -h con la configuración 100 -p 300 -cprob 40 -mprob 70}
\centering
\small
\begin{tabular}{c c c c c c c c}
\hline\hline
Instancia & Costo mínimo & Tiempo(seg.) & Costo promedio & Tiempo promedio(seg.) & CME & \%G & \%GP \\ [0.5ex]
\hline
[1ex]\hline
\end{tabular}
\label{table:nonlin}
\end{table} \clearpage
\begin{table}[ht]
\caption{Resultados de la ejecución de la metaheurística IGA, utilizando instancias de Dethloff con la configuración -n 100 -p 300 -cprob 40 -mprob 70}
\centering
\small
\begin{tabular}{c c c c c c c c}
\hline\hline
Instancia & Costo mínimo & Tiempo(seg.) & Costo promedio & Tiempo promedio(seg.) & CME & \%G & \%GP \\ [0.5ex]
\hline
SCA3-0 & 640.55 & 5.48 & 
640.55 & 5.48 & \bf{635.62} & 
0.78 & 0.78\\SCA3-1 & \bf{697.84} & 5.63 & 
697.84 & 5.63 & 697.84 & 0.00
 & 0.00\\
SCA3-2 & \bf{659.34} & 5.94 & 
659.34 & 5.94 & 659.34 & 0.00
 & 0.00\\
SCA3-3 & 681.16 & 5.62 & 
681.16 & 5.62 & \bf{680.04} & 
0.16 & 0.16\\SCA3-4 & \bf{690.50} & 5.58 & 
690.50 & 5.58 & 690.50 & 0.00
 & 0.00\\
SCA3-5 & \bf{659.90} & 5.53 & 
659.90 & 5.53 & 659.90 & 0.00
 & 0.00\\
SCA3-6 & 652.94 & 4.66 & 
652.94 & 4.66 & \bf{651.09} & 
0.28 & 0.28\\SCA3-7 & 666.15 & 5.40 & 
666.15 & 5.40 & \bf{659.17} & 
1.06 & 1.06\\SCA3-8 & \bf{719.47} & 6.14 & 
719.47 & 6.14 & 719.47 & 0.00
 & 0.00\\
SCA3-9 & \bf{681.00} & 6.04 & 
681.00 & 6.04 & 681.00 & 0.00
 & 0.00\\
SCA8-0 & 986.21 & 5.10 & 
986.21 & 5.10 & \bf{961.50} & 
2.57 & 2.57\\SCA8-1 & 1055.89 & 6.12 & 
1055.89 & 6.12 & \bf{1049.65} & 
0.59 & 0.59\\SCA8-2 & 1051.95 & 5.40 & 
1051.95 & 5.40 & \bf{1039.64} & 
1.18 & 1.18\\SCA8-3 & 1006.94 & 5.70 & 
1006.94 & 5.70 & \bf{983.34} & 
2.40 & 2.40\\SCA8-4 & 1070.53 & 4.93 & 
1070.53 & 4.93 & \bf{1065.49} & 
0.47 & 0.47\\SCA8-5 & 1048.31 & 5.27 & 
1048.31 & 5.27 & \bf{1027.08} & 
2.07 & 2.07\\SCA8-6 & 975.81 & 4.81 & 
975.81 & 4.81 & \bf{971.82} & 
0.41 & 0.41\\SCA8-7 & 1070.67 & 4.70 & 
1070.67 & 4.70 & \bf{1051.28} & 
1.84 & 1.84\\SCA8-8 & \bf{1071.18} & 6.05 & 
1071.18 & 6.05 & 1071.18 & 0.00
 & 0.00\\
SCA8-9 & 1074.27 & 4.28 & 
1074.27 & 4.28 & \bf{1060.50} & 
1.30 & 1.30\\CON3-0 & 619.09 & 6.13 & 
619.09 & 6.13 & \bf{616.52} & 
0.42 & 0.42\\CON3-1 & 557.21 & 5.53 & 
557.21 & 5.53 & \bf{554.47} & 
0.49 & 0.49\\CON3-2 & 521.38 & 7.14 & 
521.38 & 7.14 & \bf{518.00} & 
0.65 & 0.65\\CON3-3 & 591.20 & 6.20 & 
591.20 & 6.20 & \bf{591.19} & 
0.00 & 0.00\\CON3-4 & 591.43 & 5.58 & 
591.43 & 5.58 & \bf{588.79} & 
0.45 & 0.45\\CON3-5 & \bf{563.70} & 6.21 & 
563.70 & 6.21 & 563.70 & 0.00
 & 0.00\\
CON3-6 & 502.16 & 5.36 & 
502.16 & 5.36 & \bf{499.05} & 
0.62 & 0.62\\CON3-7 & 577.54 & 5.95 & 
577.54 & 5.95 & \bf{576.48} & 
0.18 & 0.18\\CON3-8 & 524.59 & 6.14 & 
524.59 & 6.14 & \bf{523.05} & 
0.29 & 0.29\\CON3-9 & 588.11 & 6.54 & 
588.11 & 6.54 & \bf{578.24} & 
1.71 & 1.71\\CON8-0 & 860.72 & 6.04 & 
860.72 & 6.04 & \bf{857.17} & 
0.41 & 0.41\\CON8-1 & 740.93 & 6.17 & 
740.93 & 6.17 & \bf{740.85} & 
0.01 & 0.01\\CON8-2 & 717.13 & 5.20 & 
717.13 & 5.20 & \bf{712.89} & 
0.59 & 0.59\\CON8-3 & 812.54 & 5.33 & 
812.54 & 5.33 & \bf{811.07} & 
0.18 & 0.18\\CON8-4 & \bf{772.25} & 5.55 & 
772.25 & 5.55 & 772.25 & 0.00
 & 0.00\\
CON8-5 & 758.84 & 4.76 & 
758.84 & 4.76 & \bf{754.88} & 
0.52 & 0.52\\CON8-6 & 686.85 & 5.81 & 
686.85 & 5.81 & \bf{678.92} & 
1.17 & 1.17\\CON8-7 & 814.80 & 5.09 & 
814.80 & 5.09 & \bf{811.96} & 
0.35 & 0.35\\CON8-8 & 783.75 & 6.69 & 
783.75 & 6.69 & \bf{767.53} & 
2.11 & 2.11\\CON8-9 & 813.68 & 5.82 & 
813.68 & 5.82 & \bf{809.00} & 
0.58 & 0.58\\\bf{PROM.} & 
\bf{763.96} & \bf{5.64} & \bf{763.96} & \bf{5.64} & \bf{758.54} & \bf{0.65} & \bf{0.65}\\[1ex]\hline
\end{tabular}
\label{table:nonlin}
\end{table} \clearpage
\begin{table}[ht]
\caption{Resultados de la ejecución de la metaheurística IGA, utilizando instancias de Dethloff con la configuración -n 100 -p 300 -cprob 40 -mprob 70}
\centering
\small
\begin{tabular}{c c c c c c c c}
\hline\hline
Instancia & Costo mínimo & Tiempo(seg.) & Costo promedio & Tiempo promedio(seg.) & CME & \%G & \%GP \\ [0.5ex]
\hline
SCA3-0 & 640.55 & 6.33 & 
640.55 & 6.11 & \bf{635.62} & 
0.78 & 0.78\\SCA3-1 & \bf{697.84} & 5.23 & 
697.84 & 5.53 & 697.84 & 0.00
 & 0.00\\
SCA3-2 & \bf{659.34} & 4.82 & 
659.34 & 4.83 & 659.34 & 0.00
 & 0.00\\
SCA3-3 & \bf{680.04} & 4.91 & 
680.60 & 4.75 & 680.04 & 0.00
 & 0.08\\SCA3-4 & \bf{690.50} & 5.78 & 
690.50 & 5.59 & 690.50 & 0.00
 & 0.00\\
SCA3-5 & \bf{659.90} & 6.51 & 
662.77 & 5.76 & 659.90 & 0.00
 & 0.43\\SCA3-6 & 652.94 & 4.60 & 
653.38 & 5.12 & \bf{651.09} & 
0.28 & 0.35\\SCA3-7 & 666.15 & 4.69 & 
666.15 & 4.70 & \bf{659.17} & 
1.06 & 1.06\\SCA3-8 & \bf{719.47} & 5.00 & 
719.47 & 4.99 & 719.47 & 0.00
 & 0.00\\
SCA3-9 & \bf{681.00} & 4.33 & 
681.00 & 4.81 & 681.00 & 0.00
 & 0.00\\
SCA8-0 & 984.75 & 5.38 & 
984.75 & 5.71 & \bf{961.50} & 
2.42 & 2.42\\SCA8-1 & 1063.21 & 5.04 & 
1063.21 & 5.92 & \bf{1049.65} & 
1.29 & 1.29\\SCA8-2 & 1050.37 & 5.08 & 
1050.37 & 5.18 & \bf{1039.64} & 
1.03 & 1.03\\SCA8-3 & 1009.50 & 4.67 & 
1012.25 & 4.99 & \bf{983.34} & 
2.66 & 2.94\\SCA8-4 & 1068.97 & 4.94 & 
1068.97 & 5.16 & \bf{1065.49} & 
0.33 & 0.33\\SCA8-5 & 1046.90 & 4.98 & 
1046.90 & 4.66 & \bf{1027.08} & 
1.93 & 1.93\\SCA8-6 & 972.48 & 4.73 & 
972.48 & 4.99 & \bf{971.82} & 
0.07 & 0.07\\SCA8-7 & 1070.92 & 5.53 & 
1070.92 & 5.65 & \bf{1051.28} & 
1.87 & 1.87\\SCA8-8 & 1075.00 & 4.91 & 
1078.97 & 5.21 & \bf{1071.18} & 
0.36 & 0.73\\SCA8-9 & 1075.27 & 4.70 & 
1075.27 & 5.37 & \bf{1060.50} & 
1.39 & 1.39\\CON3-0 & 617.59 & 5.75 & 
619.17 & 5.89 & \bf{616.52} & 
0.17 & 0.43\\CON3-1 & \bf{554.47} & 6.11 & 
554.47 & 5.55 & 554.47 & 0.00
 & 0.00\\
CON3-2 & 521.38 & 6.55 & 
521.38 & 6.03 & \bf{518.00} & 
0.65 & 0.65\\CON3-3 & \bf{591.19} & 5.71 & 
595.23 & 5.43 & 591.19 & 0.00
 & 0.68\\CON3-4 & 592.58 & 4.73 & 
593.13 & 4.93 & \bf{588.79} & 
0.64 & 0.74\\CON3-5 & 564.88 & 5.17 & 
564.88 & 5.17 & \bf{563.70} & 
0.21 & 0.21\\CON3-6 & 502.16 & 5.79 & 
502.21 & 5.83 & \bf{499.05} & 
0.62 & 0.63\\CON3-7 & 577.54 & 5.53 & 
577.54 & 5.11 & \bf{576.48} & 
0.18 & 0.18\\CON3-8 & 524.59 & 5.34 & 
526.34 & 5.20 & \bf{523.05} & 
0.29 & 0.63\\CON3-9 & 582.79 & 5.65 & 
585.45 & 5.35 & \bf{578.24} & 
0.79 & 1.25\\CON8-0 & 861.85 & 6.13 & 
861.85 & 5.43 & \bf{857.17} & 
0.55 & 0.55\\CON8-1 & 742.38 & 6.08 & 
749.93 & 5.84 & \bf{740.85} & 
0.21 & 1.23\\CON8-2 & 716.07 & 6.13 & 
716.07 & 6.43 & \bf{712.89} & 
0.45 & 0.45\\CON8-3 & 834.78 & 5.90 & 
835.16 & 5.46 & \bf{811.07} & 
2.92 & 2.97\\CON8-4 & 777.59 & 5.17 & 
777.59 & 4.72 & \bf{772.25} & 
0.69 & 0.69\\CON8-5 & 758.12 & 5.48 & 
758.12 & 5.71 & \bf{754.88} & 
0.43 & 0.43\\CON8-6 & 685.80 & 5.76 & 
688.12 & 6.15 & \bf{678.92} & 
1.01 & 1.36\\CON8-7 & 815.79 & 5.45 & 
818.53 & 5.24 & \bf{811.96} & 
0.47 & 0.81\\CON8-8 & 784.28 & 6.41 & 
786.65 & 6.17 & \bf{767.53} & 
2.18 & 2.49\\CON8-9 & 819.91 & 5.43 & 
819.91 & 5.09 & \bf{809.00} & 
1.35 & 1.35\\\bf{PROM.} & 
\bf{764.77} & \bf{5.41} & \bf{765.69} & \bf{5.39} & \bf{758.54} & \bf{0.73} & \bf{0.86}\\[1ex]\hline
\end{tabular}
\label{table:nonlin}
\end{table} \clearpage
\begin{table}[ht]
\caption{Resultados de la ejecución de la metaheurística IGA, utilizando instancias de Dethloff con la configuración -n 100 -p 300 -cprob 40 -mprob 70}
\centering
\small
\begin{tabular}{c c c c c c c c c}
\hline\hline
Instancia & Costo mínimo & Tiempo(seg.) & Costo promedio & Tiempo promedio(seg.) & Por & CME & \%G & \%GP \\ [0.5ex]
\hline
[1ex]\hline
\end{tabular}
\label{table:nonlin}
\end{table} \clearpage
\begin{table}[ht]
\caption{Resultados de la ejecución de la metaheurística IGA, utilizando instancias de Dethloff con la configuración -n 100 -p 300 -cprob 40 -mprob 70}
\centering
\small
\begin{tabular}{c c c c c c c c c}
\hline\hline
Instancia & Costo mínimo & Tiempo(seg.) & Costo promedio & Tiempo promedio(seg.) & Por & CME & \%G & \%GP \\ [0.5ex]
\hline
SCA3-0 & 640.55 & 6.16 & 
640.55 & 6.07 & 0\% & \bf{635.62} & 
0.78 & 0.78\\SCA3-1 & \bf{697.84} & 5.37 & 
697.84 & 5.29 & 100\% & 697.84 & 0.00
 & 0.00\\
SCA3-2 & \bf{659.34} & 5.42 & 
659.34 & 5.77 & 100\% & 659.34 & 0.00
 & 0.00\\
SCA3-3 & \bf{680.04} & 6.22 & 
680.04 & 5.67 & 50\% & 680.04 & 0.00
 & 0.00\\
SCA3-4 & \bf{690.50} & 4.35 & 
690.50 & 5.03 & 0\% & 690.50 & 0.00
 & 0.00\\
SCA3-5 & 665.64 & 5.59 & 
666.46 & 5.34 & 100\% & \bf{659.90} & 
0.87 & 0.99\\SCA3-6 & 652.94 & 5.25 & 
652.94 & 5.24 & 0\% & \bf{651.09} & 
0.28 & 0.28\\SCA3-7 & 666.15 & 5.81 & 
666.15 & 5.39 & 0\% & \bf{659.17} & 
1.06 & 1.06\\SCA3-8 & \bf{719.47} & 6.32 & 
719.47 & 5.66 & 100\% & 719.47 & 0.00
 & 0.00\\
SCA3-9 & \bf{681.00} & 4.84 & 
681.00 & 5.38 & 0\% & 681.00 & 0.00
 & 0.00\\
SCA8-0 & 970.64 & 5.67 & 
976.41 & 6.08 & 100\% & \bf{961.50} & 
0.95 & 1.55\\SCA8-1 & 1063.21 & 6.34 & 
1063.21 & 6.52 & 100\% & \bf{1049.65} & 
1.29 & 1.29\\SCA8-2 & 1051.60 & 4.96 & 
1051.60 & 5.32 & 100\% & \bf{1039.64} & 
1.15 & 1.15\\SCA8-3 & \bf{983.34} & 5.69 & 
996.00 & 5.81 & 100\% & 983.34 & 0.00
 & 1.29\\SCA8-4 & 1069.71 & 5.81 & 
1069.71 & 5.38 & 100\% & \bf{1065.49} & 
0.40 & 0.40\\SCA8-5 & 1045.15 & 5.53 & 
1049.03 & 5.11 & 100\% & \bf{1027.08} & 
1.76 & 2.14\\SCA8-6 & 976.74 & 5.16 & 
976.74 & 5.11 & 100\% & \bf{971.82} & 
0.51 & 0.51\\SCA8-7 & 1067.11 & 5.99 & 
1067.11 & 5.47 & 100\% & \bf{1051.28} & 
1.51 & 1.51\\SCA8-8 & \bf{1071.18} & 5.60 & 
1071.18 & 5.42 & 0\% & 1071.18 & 0.00
 & 0.00\\
SCA8-9 & 1065.60 & 4.93 & 
1065.60 & 4.92 & 100\% & \bf{1060.50} & 
0.48 & 0.48\\CON3-0 & 620.76 & 6.45 & 
620.99 & 5.70 & 100\% & \bf{616.52} & 
0.69 & 0.73\\CON3-1 & 556.04 & 5.55 & 
556.04 & 5.57 & 100\% & \bf{554.47} & 
0.28 & 0.28\\CON3-2 & 521.38 & 5.29 & 
521.38 & 5.47 & 0\% & \bf{518.00} & 
0.65 & 0.65\\CON3-3 & 591.20 & 6.72 & 
591.34 & 6.18 & 50\% & \bf{591.19} & 
0.00 & 0.03\\CON3-4 & 592.58 & 5.13 & 
592.58 & 4.88 & 0\% & \bf{588.79} & 
0.64 & 0.64\\CON3-5 & \bf{563.70} & 5.97 & 
563.70 & 5.47 & 100\% & 563.70 & 0.00
 & 0.00\\
CON3-6 & 502.88 & 6.20 & 
503.51 & 6.00 & 50\% & \bf{499.05} & 
0.77 & 0.89\\CON3-7 & \bf{576.48} & 5.04 & 
577.20 & 5.51 & 50\% & 576.48 & 0.00
 & 0.12\\CON3-8 & \bf{523.05} & 6.42 & 
523.05 & 6.37 & 100\% & 523.05 & 0.00
 & 0.00\\
CON3-9 & 588.11 & 6.48 & 
588.14 & 5.91 & 50\% & \bf{578.24} & 
1.71 & 1.71\\CON8-0 & 858.63 & 6.28 & 
858.63 & 5.54 & 100\% & \bf{857.17} & 
0.17 & 0.17\\CON8-1 & 751.31 & 6.18 & 
755.76 & 6.20 & 100\% & \bf{740.85} & 
1.41 & 2.01\\CON8-2 & 722.58 & 5.36 & 
722.58 & 5.47 & 100\% & \bf{712.89} & 
1.36 & 1.36\\CON8-3 & 819.75 & 4.82 & 
819.75 & 5.35 & 100\% & \bf{811.07} & 
1.07 & 1.07\\CON8-4 & 781.29 & 5.20 & 
784.45 & 5.05 & 100\% & \bf{772.25} & 
1.17 & 1.58\\CON8-5 & 757.25 & 5.63 & 
757.25 & 6.20 & 100\% & \bf{754.88} & 
0.31 & 0.31\\CON8-6 & 694.15 & 6.36 & 
694.30 & 5.94 & 100\% & \bf{678.92} & 
2.24 & 2.27\\CON8-7 & 821.18 & 6.40 & 
821.33 & 5.96 & 100\% & \bf{811.96} & 
1.14 & 1.15\\CON8-8 & 769.65 & 6.14 & 
769.65 & 5.85 & 100\% & \bf{767.53} & 
0.28 & 0.28\\CON8-9 & 817.26 & 5.72 & 
821.73 & 6.02 & 50\% & \bf{809.00} & 
1.02 & 1.57\\\bf{PROM.} & 
\bf{763.67} & \bf{5.71} & \bf{764.61} & \bf{5.62} & & \bf{758.54} & \bf{0.65} & \bf{0.76}\\[1ex]\hline
\end{tabular}
\label{table:nonlin}
\end{table} \clearpage
\begin{table}[ht]
\caption{Resultados de la ejecución de la metaheurística IGA, utilizando instancias de Dethloff con la configuración -n 100 -p 300 -cprob 40 -mprob 70}
\centering
\small
\begin{tabular}{c c c c c c c c c}
\hline\hline
Instancia & Costo mínimo & Tiempo(seg.) & Costo promedio & Tiempo promedio(seg.) & Por & CME & \%G & \%GP \\ [0.5ex]
\hline
SCA3-0 & 636.06 & 5.52 & 
640.40 & 5.70 & 13.3333\% & \bf{635.62} & 
0.07 & 0.75\\SCA3-1 & \bf{697.84} & 5.64 & 
697.84 & 5.44 & 30\% & 697.84 & 0.00
 & 0.00\\
SCA3-2 & \bf{659.34} & 5.71 & 
660.48 & 5.11 & 66.6667\% & 659.34 & 0.00
 & 0.17\\SCA3-3 & \bf{680.04} & 4.79 & 
680.59 & 5.31 & 86.6667\% & 680.04 & 0.00
 & 0.08\\SCA3-4 & \bf{690.50} & 5.47 & 
690.50 & 5.34 & 10\% & 690.50 & 0.00
 & 0.00\\
SCA3-5 & 661.07 & 5.12 & 
662.96 & 5.36 & 90\% & \bf{659.90} & 
0.18 & 0.46\\SCA3-6 & 652.94 & 4.70 & 
652.94 & 5.24 & 16.6667\% & \bf{651.09} & 
0.28 & 0.28\\SCA3-7 & 666.15 & 5.02 & 
666.15 & 5.38 & 0\% & \bf{659.17} & 
1.06 & 1.06\\SCA3-8 & \bf{719.47} & 5.07 & 
720.24 & 5.43 & 86.6667\% & 719.47 & 0.00
 & 0.11\\SCA3-9 & \bf{681.00} & 5.69 & 
681.00 & 5.48 & 3.33333\% & 681.00 & 0.00
 & 0.00\\
SCA8-0 & 970.64 & 5.71 & 
975.20 & 5.22 & 100\% & \bf{961.50} & 
0.95 & 1.43\\SCA8-1 & 1061.29 & 6.57 & 
1061.77 & 5.50 & 100\% & \bf{1049.65} & 
1.11 & 1.15\\SCA8-2 & 1050.37 & 4.41 & 
1050.37 & 5.48 & 40\% & \bf{1039.64} & 
1.03 & 1.03\\SCA8-3 & 995.60 & 6.15 & 
999.51 & 5.50 & 100\% & \bf{983.34} & 
1.25 & 1.64\\SCA8-4 & 1068.97 & 4.84 & 
1071.81 & 5.13 & 93.3333\% & \bf{1065.49} & 
0.33 & 0.59\\SCA8-5 & 1036.88 & 4.29 & 
1042.12 & 5.32 & 100\% & \bf{1027.08} & 
0.95 & 1.46\\SCA8-6 & 976.37 & 6.60 & 
977.26 & 5.70 & 96.6667\% & \bf{971.82} & 
0.47 & 0.56\\SCA8-7 & 1067.49 & 5.30 & 
1069.31 & 5.05 & 100\% & \bf{1051.28} & 
1.54 & 1.72\\SCA8-8 & \bf{1071.18} & 5.24 & 
1079.20 & 5.28 & 100\% & 1071.18 & 0.00
 & 0.75\\SCA8-9 & 1067.42 & 5.06 & 
1070.22 & 5.08 & 96.6667\% & \bf{1060.50} & 
0.65 & 0.92\\CON3-0 & 617.59 & 6.58 & 
620.44 & 5.74 & 93.3333\% & \bf{616.52} & 
0.17 & 0.64\\CON3-1 & 556.04 & 5.47 & 
557.31 & 5.72 & 86.6667\% & \bf{554.47} & 
0.28 & 0.51\\CON3-2 & 519.26 & 5.53 & 
520.96 & 6.15 & 23.3333\% & \bf{518.00} & 
0.24 & 0.57\\CON3-3 & \bf{591.19} & 6.25 & 
591.89 & 5.60 & 93.3333\% & 591.19 & 0.00
 & 0.12\\CON3-4 & 589.32 & 5.29 & 
591.49 & 5.52 & 80\% & \bf{588.79} & 
0.09 & 0.46\\CON3-5 & \bf{563.70} & 4.91 & 
564.46 & 5.83 & 90\% & 563.70 & 0.00
 & 0.14\\CON3-6 & 502.16 & 5.61 & 
503.32 & 5.94 & 86.6667\% & \bf{499.05} & 
0.62 & 0.86\\CON3-7 & \bf{576.48} & 5.17 & 
578.43 & 5.45 & 96.6667\% & 576.48 & 0.00
 & 0.34\\CON3-8 & \bf{523.05} & 6.59 & 
525.13 & 5.79 & 83.3333\% & 523.05 & 0.00
 & 0.40\\CON3-9 & 581.06 & 6.20 & 
583.41 & 5.92 & 83.3333\% & \bf{578.24} & 
0.49 & 0.89\\CON8-0 & 860.48 & 5.06 & 
865.41 & 5.67 & 93.3333\% & \bf{857.17} & 
0.39 & 0.96\\CON8-1 & 741.70 & 6.46 & 
750.95 & 5.61 & 90\% & \bf{740.85} & 
0.11 & 1.36\\CON8-2 & 716.22 & 4.97 & 
716.70 & 5.86 & 100\% & \bf{712.89} & 
0.47 & 0.53\\CON8-3 & 814.50 & 5.35 & 
816.32 & 5.79 & 100\% & \bf{811.07} & 
0.42 & 0.65\\CON8-4 & 772.76 & 5.75 & 
778.10 & 5.22 & 100\% & \bf{772.25} & 
0.07 & 0.76\\CON8-5 & 761.01 & 4.85 & 
762.18 & 5.71 & 63.3333\% & \bf{754.88} & 
0.81 & 0.97\\CON8-6 & 689.56 & 5.60 & 
691.51 & 5.62 & 100\% & \bf{678.92} & 
1.57 & 1.85\\CON8-7 & 814.79 & 5.00 & 
815.74 & 5.43 & 100\% & \bf{811.96} & 
0.35 & 0.47\\CON8-8 & 775.36 & 5.50 & 
780.40 & 5.59 & 96.6667\% & \bf{767.53} & 
1.02 & 1.68\\CON8-9 & 815.52 & 5.33 & 
820.70 & 5.76 & 100\% & \bf{809.00} & 
0.81 & 1.45\\\bf{PROM.} & 
\bf{762.31} & \bf{5.46} & \bf{764.62} & \bf{5.52} & & \bf{758.54} & \bf{0.44} & \bf{0.74}\\[1ex]\hline
\end{tabular}
\label{table:nonlin}
\end{table} \clearpage
\begin{table}[ht]
\caption{Resultados de la ejecución de la metaheurística IGA, utilizando instancias de SalhiNagy con la configuración -n 100 -p 350 -cprob 90 -mprob 70}
\centering
\small
\begin{tabular}{c c c c c c c c c}
\hline\hline
Instancia & Costo mínimo & Tiempo(seg.) & Costo promedio & Tiempo promedio(seg.) & Por & CME & \%G & \%GP \\ [0.5ex]
\hline
CMT1X & 474.91 & 7.26 & 
476.30 & 6.70 & 86.6667\% & \bf{470.48} & 
0.94 & 1.24\\CMT1Y & 472.37 & 7.08 & 
476.58 & 6.39 & 96.6667\% & \bf{470.48} & 
0.40 & 1.30\\CMT2X & 694.31 & 13.61 & 
702.97 & 14.80 & 100\% & \bf{682.39} & 
1.75 & 3.02\\CMT2Y & 693.46 & 14.56 & 
702.54 & 14.72 & 100\% & \bf{682.39} & 
1.62 & 2.95\\CMT3X & 723.46 & 33.76 & 
734.23 & 32.97 & 100\% & \bf{719.06} & 
0.61 & 2.11\\CMT3Y & 725.24 & 32.33 & 
734.62 & 32.25 & 100\% & \bf{719.06} & 
0.86 & 2.16\\CMT4X & 865.53 & 88.79 & 
895.27 & 88.27 & 100\% & \bf{854.21} & 
1.33 & 4.81\\CMT4Y & 886.82 & 89.15 & 
901.01 & 88.80 & 96.6667\% & \bf{852.46} & 
4.03 & 5.70\\CMT5X & 1087.15 & 179.20 & 
1100.22 & 180.73 & 93.3333\% & \bf{1030.56} & 
5.49 & 6.76\\CMT5Y & 1070.68 & 184.15 & 
1100.07 & 183.53 & 86.6667\% & \bf{1031.69} & 
3.78 & 6.63\\CMT11X & 875.48 & 55.28 & 
893.37 & 55.07 & 100\% & \bf{831.09} & 
5.34 & 7.49\\CMT11Y & 846.38 & 61.17 & 
871.31 & 61.62 & 100\% & \bf{829.85} & 
1.99 & 5.00\\CMT12X & 670.29 & 33.64 & 
675.03 & 33.70 & 100\% & \bf{658.83} & 
1.74 & 2.46\\CMT12Y & 669.49 & 34.14 & 
674.08 & 33.06 & 96.6667\% & \bf{660.47} & 
1.37 & 2.06\\\bf{PROM.} & 
\bf{768.25} & \bf{59.58} & \bf{781.26} & \bf{59.47} & & \bf{749.50} & \bf{2.23} & \bf{3.83}\\[1ex]\hline
\end{tabular}
\label{table:nonlin}
\end{table} \clearpage

%\begin{table}[ht]
\caption{Resultados de la ejecución de la metaheurística ILS, utilizando instancias de Dethloff con la configuración -n 15.0 -LS 10.0}
\centering
\small
\begin{tabular}{c c c c c c c}
\hline\hline
Instancia & Costo mínimo & Tiempo(seg.) & Costo promedio & Tiempo promedio(seg.) & Costo ILS & \%Gap \\ [0.5ex]
\hline
SCA3-0 & 642.80 & 1.06 & 
652.30 & 0.96 & \bf{635.62} & 
1.13\\SCA3-1 & 708.40 & 1.12 & 
728.92 & 1.05 & \bf{697.84} & 
1.51\\SCA3-2 & 670.08 & 1.14 & 
682.52 & 0.93 & \bf{659.34} & 
1.63\\SCA3-3 & 682.46 & 0.95 & 
692.43 & 1.06 & \bf{680.04} & 
0.36\\SCA3-4 & 720.66 & 0.97 & 
722.18 & 0.86 & \bf{690.50} & 
4.37\\SCA3-5 & 673.39 & 1.03 & 
683.40 & 0.97 & \bf{659.90} & 
2.04\\SCA3-6 & 656.23 & 1.12 & 
664.69 & 1.00 & \bf{651.09} & 
0.79\\SCA3-7 & 671.77 & 1.06 & 
674.22 & 0.99 & \bf{659.17} & 
1.91\\SCA3-8 & 726.86 & 0.99 & 
736.63 & 0.93 & \bf{719.47} & 
1.03\\SCA3-9 & 690.83 & 1.00 & 
697.85 & 0.85 & \bf{681.00} & 
1.44\\SCA8-0 & 1023.08 & 0.80 & 
1041.97 & 0.96 & \bf{961.50} & 
6.40\\SCA8-1 & 1096.04 & 1.34 & 
1106.20 & 0.93 & \bf{1049.65} & 
4.42\\SCA8-2 & 1082.63 & 1.30 & 
1156.26 & 0.96 & \bf{1039.64} & 
4.14\\SCA8-3 & 1015.27 & 0.86 & 
1042.83 & 0.85 & \bf{983.34} & 
3.25\\SCA8-4 & 1084.34 & 1.06 & 
1108.72 & 0.83 & \bf{1065.49} & 
1.77\\SCA8-5 & 1065.32 & 0.81 & 
1130.69 & 0.77 & \bf{1027.08} & 
3.72\\SCA8-6 & 982.49 & 0.98 & 
1011.27 & 0.89 & \bf{971.82} & 
1.10\\SCA8-7 & 1094.11 & 0.75 & 
1110.32 & 0.84 & \bf{1051.28} & 
4.07\\SCA8-8 & 1123.51 & 0.86 & 
1129.21 & 0.96 & \bf{1071.18} & 
4.89\\SCA8-9 & 1117.91 & 0.90 & 
1120.12 & 0.85 & \bf{1060.50} & 
5.41\\CON3-0 & 619.09 & 0.91 & 
646.35 & 0.85 & \bf{616.52} & 
0.42\\CON3-1 & 556.92 & 0.95 & 
570.77 & 0.83 & \bf{554.47} & 
0.44\\CON3-2 & 527.17 & 0.84 & 
532.64 & 0.85 & \bf{518.00} & 
1.77\\CON3-3 & 591.48 & 0.85 & 
596.91 & 0.97 & \bf{591.19} & 
0.05\\CON3-4 & 593.78 & 1.31 & 
612.62 & 0.93 & \bf{588.79} & 
0.85\\CON3-5 & 581.19 & 1.09 & 
587.28 & 1.04 & \bf{563.70} & 
3.10\\CON3-6 & 503.98 & 1.10 & 
524.96 & 0.99 & \bf{499.05} & 
0.99\\CON3-7 & 604.49 & 1.25 & 
613.06 & 0.88 & \bf{576.48} & 
4.86\\CON3-8 & 527.82 & 0.84 & 
536.82 & 1.02 & \bf{523.05} & 
0.91\\CON3-9 & 598.10 & 0.79 & 
601.45 & 0.85 & \bf{578.24} & 
3.43\\CON8-0 & 885.46 & 0.88 & 
905.12 & 0.78 & \bf{857.17} & 
3.30\\CON8-1 & 763.53 & 0.90 & 
785.00 & 0.95 & \bf{740.85} & 
3.06\\CON8-2 & 729.55 & 1.11 & 
740.91 & 1.02 & \bf{712.89} & 
2.34\\CON8-3 & 812.32 & 0.85 & 
832.98 & 0.81 & \bf{811.07} & 
0.15\\CON8-4 & 814.59 & 0.99 & 
834.41 & 0.92 & \bf{772.25} & 
5.48\\CON8-5 & 770.21 & 0.88 & 
808.48 & 0.88 & \bf{754.88} & 
2.03\\CON8-6 & 734.55 & 1.00 & 
742.34 & 0.86 & \bf{678.92} & 
8.19\\CON8-7 & 853.82 & 0.62 & 
856.22 & 0.76 & \bf{811.96} & 
5.16\\CON8-8 & 797.75 & 0.92 & 
806.02 & 0.89 & \bf{767.53} & 
3.94\\CON8-9 & 849.01 & 0.87 & 
875.06 & 0.94 & \bf{809.00} & 
4.95\\\bf{PROM.} & 
\bf{781.07} & \bf{0.98} & \bf{797.55} & \bf{0.91} & \bf{758.54} & \bf{2.77}\\[1ex]\hline
\end{tabular}
\label{table:nonlin}
\end{table} \clearpage
\begin{table}[ht]
\caption{Resultados de la ejecución de la metaheurística ILS, utilizando instancias de Dethloff con la configuración -n 15.0 -LS 20.0}
\centering
\small
\begin{tabular}{c c c c c c c}
\hline\hline
Instancia & Costo mínimo & Tiempo(seg.) & Costo promedio & Tiempo promedio(seg.) & Costo ILS & \%Gap \\ [0.5ex]
\hline
SCA3-0 & 643.15 & 1.81 & 
651.04 & 1.59 & \bf{635.62} & 
1.18\\SCA3-1 & 701.53 & 1.44 & 
715.73 & 1.44 & \bf{697.84} & 
0.53\\SCA3-2 & 661.13 & 1.46 & 
671.13 & 1.41 & \bf{659.34} & 
0.27\\SCA3-3 & 682.44 & 1.75 & 
693.31 & 1.43 & \bf{680.04} & 
0.35\\SCA3-4 & \bf{690.50} & 2.00 & 
711.76 & 1.52 & 690.50 & 0.00\\
SCA3-5 & 678.28 & 1.94 & 
683.99 & 1.57 & \bf{659.90} & 
2.79\\SCA3-6 & 659.46 & 1.52 & 
681.51 & 1.07 & \bf{651.09} & 
1.29\\SCA3-7 & 671.77 & 1.57 & 
675.61 & 1.28 & \bf{659.17} & 
1.91\\SCA3-8 & \bf{719.47} & 1.26 & 
741.01 & 1.34 & 719.47 & 0.00\\
SCA3-9 & \bf{681.00} & 1.46 & 
687.77 & 1.56 & 681.00 & 0.00\\
SCA8-0 & 985.12 & 1.33 & 
1005.04 & 1.26 & \bf{961.50} & 
2.46\\SCA8-1 & 1079.51 & 1.34 & 
1102.48 & 1.12 & \bf{1049.65} & 
2.84\\SCA8-2 & 1056.87 & 1.03 & 
1093.94 & 1.08 & \bf{1039.64} & 
1.66\\SCA8-3 & 1030.20 & 1.19 & 
1044.39 & 1.41 & \bf{983.34} & 
4.77\\SCA8-4 & 1109.44 & 1.21 & 
1120.95 & 1.06 & \bf{1065.49} & 
4.12\\SCA8-5 & 1086.95 & 1.20 & 
1103.93 & 1.27 & \bf{1027.08} & 
5.83\\SCA8-6 & 1008.43 & 1.29 & 
1035.31 & 1.20 & \bf{971.82} & 
3.77\\SCA8-7 & 1070.92 & 1.31 & 
1116.42 & 1.27 & \bf{1051.28} & 
1.87\\SCA8-8 & 1103.44 & 1.16 & 
1131.56 & 1.18 & \bf{1071.18} & 
3.01\\SCA8-9 & 1085.11 & 1.29 & 
1120.71 & 1.10 & \bf{1060.50} & 
2.32\\CON3-0 & 634.04 & 1.36 & 
642.17 & 1.26 & \bf{616.52} & 
2.84\\CON3-1 & 560.61 & 1.54 & 
575.77 & 1.61 & \bf{554.47} & 
1.11\\CON3-2 & 521.38 & 1.20 & 
534.84 & 1.35 & \bf{518.00} & 
0.65\\CON3-3 & 594.31 & 1.44 & 
601.72 & 1.47 & \bf{591.19} & 
0.53\\CON3-4 & 603.94 & 1.54 & 
611.71 & 1.29 & \bf{588.79} & 
2.57\\CON3-5 & 569.74 & 1.39 & 
575.89 & 1.39 & \bf{563.70} & 
1.07\\CON3-6 & 507.65 & 1.41 & 
512.57 & 1.38 & \bf{499.05} & 
1.72\\CON3-7 & 578.41 & 1.21 & 
587.84 & 1.34 & \bf{576.48} & 
0.33\\CON3-8 & 534.71 & 1.58 & 
541.00 & 1.66 & \bf{523.05} & 
2.23\\CON3-9 & 589.72 & 1.65 & 
595.14 & 1.50 & \bf{578.24} & 
1.99\\CON8-0 & 899.01 & 1.17 & 
918.96 & 1.48 & \bf{857.17} & 
4.88\\CON8-1 & 756.80 & 1.08 & 
798.51 & 1.39 & \bf{740.85} & 
2.15\\CON8-2 & 716.58 & 1.19 & 
756.58 & 1.30 & \bf{712.89} & 
0.52\\CON8-3 & 842.72 & 1.34 & 
852.20 & 1.25 & \bf{811.07} & 
3.90\\CON8-4 & 797.67 & 1.05 & 
812.21 & 1.26 & \bf{772.25} & 
3.29\\CON8-5 & 781.27 & 1.16 & 
792.86 & 1.16 & \bf{754.88} & 
3.50\\CON8-6 & 706.57 & 1.37 & 
709.86 & 1.21 & \bf{678.92} & 
4.07\\CON8-7 & 822.26 & 1.60 & 
834.09 & 1.32 & \bf{811.96} & 
1.27\\CON8-8 & 773.63 & 1.44 & 
799.88 & 1.54 & \bf{767.53} & 
0.79\\CON8-9 & 835.01 & 2.26 & 
853.31 & 1.62 & \bf{809.00} & 
3.22\\\bf{PROM.} & 
\bf{775.77} & \bf{1.41} & \bf{792.37} & \bf{1.35} & \bf{758.54} & \bf{2.09}\\[1ex]\hline
\end{tabular}
\label{table:nonlin}
\end{table} \clearpage
\begin{table}[ht]
\caption{Resultados de la ejecución de la metaheurística ILS, utilizando instancias de Dethloff con la configuración -n 15.0 -LS 30.0}
\centering
\small
\begin{tabular}{c c c c c c c}
\hline\hline
Instancia & Costo mínimo & Tiempo(seg.) & Costo promedio & Tiempo promedio(seg.) & Costo ILS & \%Gap \\ [0.5ex]
\hline
SCA3-0 & 640.55 & 2.17 & 
641.83 & 2.03 & \bf{635.62} & 
0.78\\SCA3-1 & 710.89 & 1.73 & 
719.42 & 1.80 & \bf{697.84} & 
1.87\\SCA3-2 & \bf{659.34} & 1.86 & 
669.22 & 1.90 & 659.34 & 0.00\\
SCA3-3 & 681.74 & 1.96 & 
689.78 & 2.04 & \bf{680.04} & 
0.25\\SCA3-4 & \bf{690.50} & 2.40 & 
714.41 & 2.08 & 690.50 & 0.00\\
SCA3-5 & 668.48 & 1.73 & 
678.48 & 1.89 & \bf{659.90} & 
1.30\\SCA3-6 & 653.69 & 2.32 & 
659.50 & 1.91 & \bf{651.09} & 
0.40\\SCA3-7 & 671.67 & 1.59 & 
672.82 & 1.80 & \bf{659.17} & 
1.90\\SCA3-8 & 731.95 & 2.47 & 
742.70 & 2.10 & \bf{719.47} & 
1.73\\SCA3-9 & \bf{681.00} & 2.16 & 
694.93 & 1.59 & 681.00 & 0.00\\
SCA8-0 & 1032.17 & 1.76 & 
1059.15 & 1.60 & \bf{961.50} & 
7.35\\SCA8-1 & 1097.27 & 1.80 & 
1107.28 & 1.57 & \bf{1049.65} & 
4.54\\SCA8-2 & 1053.94 & 1.84 & 
1092.38 & 1.52 & \bf{1039.64} & 
1.38\\SCA8-3 & 1037.15 & 1.48 & 
1048.16 & 1.52 & \bf{983.34} & 
5.47\\SCA8-4 & 1087.31 & 2.46 & 
1119.05 & 1.88 & \bf{1065.49} & 
2.05\\SCA8-5 & 1074.74 & 1.45 & 
1086.12 & 1.54 & \bf{1027.08} & 
4.64\\SCA8-6 & 1000.59 & 1.48 & 
1007.75 & 1.61 & \bf{971.82} & 
2.96\\SCA8-7 & 1091.44 & 1.73 & 
1104.60 & 1.79 & \bf{1051.28} & 
3.82\\SCA8-8 & 1098.24 & 1.99 & 
1113.80 & 1.85 & \bf{1071.18} & 
2.53\\SCA8-9 & 1075.30 & 1.73 & 
1134.80 & 1.40 & \bf{1060.50} & 
1.40\\CON3-0 & 633.24 & 1.52 & 
642.18 & 1.96 & \bf{616.52} & 
2.71\\CON3-1 & 560.75 & 2.36 & 
566.86 & 2.17 & \bf{554.47} & 
1.13\\CON3-2 & 528.47 & 2.12 & 
532.38 & 2.02 & \bf{518.00} & 
2.02\\CON3-3 & 594.31 & 1.92 & 
606.31 & 2.09 & \bf{591.19} & 
0.53\\CON3-4 & 597.75 & 2.02 & 
606.50 & 2.12 & \bf{588.79} & 
1.52\\CON3-5 & 566.96 & 2.36 & 
575.10 & 2.00 & \bf{563.70} & 
0.58\\CON3-6 & 510.38 & 1.67 & 
513.62 & 1.95 & \bf{499.05} & 
2.27\\CON3-7 & 595.90 & 1.90 & 
601.26 & 1.90 & \bf{576.48} & 
3.37\\CON3-8 & 534.28 & 2.10 & 
549.48 & 1.85 & \bf{523.05} & 
2.15\\CON3-9 & 590.64 & 1.84 & 
594.27 & 1.86 & \bf{578.24} & 
2.14\\CON8-0 & 870.35 & 2.13 & 
919.22 & 1.64 & \bf{857.17} & 
1.54\\CON8-1 & 761.69 & 2.10 & 
772.52 & 1.79 & \bf{740.85} & 
2.81\\CON8-2 & 722.68 & 1.78 & 
746.69 & 1.68 & \bf{712.89} & 
1.37\\CON8-3 & 849.44 & 1.95 & 
868.57 & 1.59 & \bf{811.07} & 
4.73\\CON8-4 & 815.15 & 1.97 & 
827.37 & 1.71 & \bf{772.25} & 
5.56\\CON8-5 & 776.17 & 1.89 & 
788.01 & 1.65 & \bf{754.88} & 
2.82\\CON8-6 & 708.92 & 1.45 & 
709.71 & 1.55 & \bf{678.92} & 
4.42\\CON8-7 & 827.08 & 1.35 & 
875.61 & 1.54 & \bf{811.96} & 
1.86\\CON8-8 & 787.95 & 1.99 & 
808.22 & 1.52 & \bf{767.53} & 
2.66\\CON8-9 & 829.64 & 2.13 & 
851.37 & 1.83 & \bf{809.00} & 
2.55\\\bf{PROM.} & 
\bf{777.49} & \bf{1.92} & \bf{792.79} & \bf{1.80} & \bf{758.54} & \bf{2.33}\\[1ex]\hline
\end{tabular}
\label{table:nonlin}
\end{table} \clearpage
\begin{table}[ht]
\caption{Resultados de la ejecución de la metaheurística ILS, utilizando instancias de Dethloff con la configuración -n 15.0 -LS 40.0}
\centering
\small
\begin{tabular}{c c c c c c c}
\hline\hline
Instancia & Costo mínimo & Tiempo(seg.) & Costo promedio & Tiempo promedio(seg.) & Costo ILS & \%Gap \\ [0.5ex]
\hline
SCA3-0 & 641.69 & 2.64 & 
643.52 & 2.49 & \bf{635.62} & 
0.95\\SCA3-1 & 707.07 & 2.25 & 
711.97 & 2.40 & \bf{697.84} & 
1.32\\SCA3-2 & 661.13 & 2.69 & 
662.36 & 2.43 & \bf{659.34} & 
0.27\\SCA3-3 & 681.74 & 2.48 & 
687.22 & 2.57 & \bf{680.04} & 
0.25\\SCA3-4 & \bf{690.50} & 2.15 & 
699.55 & 2.29 & 690.50 & 0.00\\
SCA3-5 & 665.04 & 2.86 & 
675.96 & 2.64 & \bf{659.90} & 
0.78\\SCA3-6 & 652.94 & 2.00 & 
653.53 & 2.24 & \bf{651.09} & 
0.28\\SCA3-7 & 671.67 & 2.66 & 
676.13 & 2.36 & \bf{659.17} & 
1.90\\SCA3-8 & \bf{719.47} & 2.65 & 
724.19 & 2.48 & 719.47 & 0.00\\
SCA3-9 & \bf{681.00} & 2.62 & 
692.51 & 2.39 & 681.00 & 0.00\\
SCA8-0 & 992.95 & 2.65 & 
1018.17 & 2.04 & \bf{961.50} & 
3.27\\SCA8-1 & 1102.95 & 1.88 & 
1111.22 & 2.09 & \bf{1049.65} & 
5.08\\SCA8-2 & 1070.14 & 2.45 & 
1081.03 & 1.97 & \bf{1039.64} & 
2.93\\SCA8-3 & 1015.63 & 2.52 & 
1037.24 & 2.00 & \bf{983.34} & 
3.28\\SCA8-4 & 1076.95 & 1.71 & 
1118.78 & 1.80 & \bf{1065.49} & 
1.08\\SCA8-5 & 1092.28 & 1.99 & 
1094.41 & 1.99 & \bf{1027.08} & 
6.35\\SCA8-6 & 1004.39 & 1.54 & 
1020.80 & 1.79 & \bf{971.82} & 
3.35\\SCA8-7 & 1076.29 & 2.36 & 
1098.82 & 2.04 & \bf{1051.28} & 
2.38\\SCA8-8 & 1093.40 & 2.67 & 
1117.32 & 1.92 & \bf{1071.18} & 
2.07\\SCA8-9 & 1068.65 & 1.78 & 
1094.78 & 1.87 & \bf{1060.50} & 
0.77\\CON3-0 & 632.57 & 1.88 & 
642.15 & 2.48 & \bf{616.52} & 
2.60\\CON3-1 & 560.75 & 2.24 & 
563.04 & 2.69 & \bf{554.47} & 
1.13\\CON3-2 & 521.38 & 2.46 & 
527.60 & 2.68 & \bf{518.00} & 
0.65\\CON3-3 & 591.48 & 2.21 & 
606.53 & 2.68 & \bf{591.19} & 
0.05\\CON3-4 & 599.13 & 2.76 & 
621.49 & 2.40 & \bf{588.79} & 
1.76\\CON3-5 & 575.00 & 2.62 & 
578.44 & 2.77 & \bf{563.70} & 
2.00\\CON3-6 & 502.16 & 2.29 & 
512.80 & 2.47 & \bf{499.05} & 
0.62\\CON3-7 & 578.41 & 1.70 & 
592.69 & 2.10 & \bf{576.48} & 
0.33\\CON3-8 & 526.59 & 2.20 & 
534.42 & 2.85 & \bf{523.05} & 
0.68\\CON3-9 & 590.17 & 2.01 & 
599.41 & 2.18 & \bf{578.24} & 
2.06\\CON8-0 & 913.38 & 1.79 & 
939.51 & 1.88 & \bf{857.17} & 
6.56\\CON8-1 & 754.51 & 1.97 & 
776.48 & 2.13 & \bf{740.85} & 
1.84\\CON8-2 & 727.20 & 2.87 & 
737.40 & 2.13 & \bf{712.89} & 
2.01\\CON8-3 & 827.30 & 2.37 & 
842.71 & 2.21 & \bf{811.07} & 
2.00\\CON8-4 & 809.28 & 2.08 & 
827.58 & 2.17 & \bf{772.25} & 
4.80\\CON8-5 & 775.02 & 2.38 & 
790.22 & 2.05 & \bf{754.88} & 
2.67\\CON8-6 & 698.19 & 2.26 & 
702.74 & 2.20 & \bf{678.92} & 
2.84\\CON8-7 & 814.86 & 2.38 & 
853.53 & 1.81 & \bf{811.96} & 
0.36\\CON8-8 & 793.75 & 2.15 & 
803.37 & 2.22 & \bf{767.53} & 
3.42\\CON8-9 & 827.65 & 2.42 & 
857.12 & 1.96 & \bf{809.00} & 
2.31\\\bf{PROM.} & 
\bf{774.62} & \bf{2.29} & \bf{788.22} & \bf{2.25} & \bf{758.54} & \bf{1.93}\\[1ex]\hline
\end{tabular}
\label{table:nonlin}
\end{table} \clearpage
\begin{table}[ht]
\caption{Resultados de la ejecución de la metaheurística ILS, utilizando instancias de Dethloff con la configuración -n 15.0 -LS 50.0}
\centering
\small
\begin{tabular}{c c c c c c c}
\hline\hline
Instancia & Costo mínimo & Tiempo(seg.) & Costo promedio & Tiempo promedio(seg.) & Costo ILS & \%Gap \\ [0.5ex]
\hline
SCA3-0 & 640.55 & 3.14 & 
643.89 & 3.06 & \bf{635.62} & 
0.78\\SCA3-1 & 701.53 & 3.91 & 
704.65 & 3.14 & \bf{697.84} & 
0.53\\SCA3-2 & 661.13 & 2.89 & 
665.61 & 2.52 & \bf{659.34} & 
0.27\\SCA3-3 & 680.60 & 2.88 & 
684.20 & 2.99 & \bf{680.04} & 
0.08\\SCA3-4 & 693.23 & 3.05 & 
698.42 & 2.97 & \bf{690.50} & 
0.40\\SCA3-5 & 680.58 & 2.34 & 
685.33 & 2.69 & \bf{659.90} & 
3.13\\SCA3-6 & \bf{651.09} & 3.25 & 
659.08 & 2.77 & 651.09 & 0.00\\
SCA3-7 & 671.77 & 2.62 & 
673.53 & 2.67 & \bf{659.17} & 
1.91\\SCA3-8 & 724.28 & 2.66 & 
736.83 & 2.40 & \bf{719.47} & 
0.67\\SCA3-9 & 685.00 & 1.78 & 
699.89 & 2.37 & \bf{681.00} & 
0.59\\SCA8-0 & 994.55 & 3.31 & 
1025.73 & 2.88 & \bf{961.50} & 
3.44\\SCA8-1 & 1099.37 & 2.85 & 
1109.90 & 2.50 & \bf{1049.65} & 
4.74\\SCA8-2 & 1058.07 & 2.44 & 
1070.44 & 2.57 & \bf{1039.64} & 
1.77\\SCA8-3 & 1022.89 & 1.78 & 
1038.36 & 2.29 & \bf{983.34} & 
4.02\\SCA8-4 & 1073.64 & 2.55 & 
1115.76 & 2.23 & \bf{1065.49} & 
0.76\\SCA8-5 & 1052.57 & 2.26 & 
1074.50 & 2.18 & \bf{1027.08} & 
2.48\\SCA8-6 & 1023.10 & 1.70 & 
1032.89 & 2.16 & \bf{971.82} & 
5.28\\SCA8-7 & 1076.01 & 1.69 & 
1107.86 & 2.21 & \bf{1051.28} & 
2.35\\SCA8-8 & 1082.11 & 2.45 & 
1087.91 & 2.56 & \bf{1071.18} & 
1.02\\SCA8-9 & 1101.27 & 2.40 & 
1114.80 & 2.37 & \bf{1060.50} & 
3.84\\CON3-0 & 637.07 & 3.18 & 
654.38 & 3.33 & \bf{616.52} & 
3.33\\CON3-1 & 560.75 & 2.82 & 
574.83 & 2.75 & \bf{554.47} & 
1.13\\CON3-2 & 526.07 & 2.64 & 
531.14 & 2.78 & \bf{518.00} & 
1.56\\CON3-3 & 594.31 & 2.61 & 
599.31 & 2.73 & \bf{591.19} & 
0.53\\CON3-4 & 605.10 & 2.88 & 
606.88 & 2.79 & \bf{588.79} & 
2.77\\CON3-5 & 575.98 & 2.87 & 
578.77 & 2.92 & \bf{563.70} & 
2.18\\CON3-6 & 504.00 & 2.71 & 
508.92 & 2.79 & \bf{499.05} & 
0.99\\CON3-7 & 586.01 & 2.56 & 
595.49 & 3.02 & \bf{576.48} & 
1.65\\CON3-8 & 526.59 & 3.52 & 
535.67 & 3.09 & \bf{523.05} & 
0.68\\CON3-9 & 588.55 & 3.59 & 
589.32 & 3.17 & \bf{578.24} & 
1.78\\CON8-0 & 875.52 & 1.86 & 
904.12 & 2.25 & \bf{857.17} & 
2.14\\CON8-1 & 764.40 & 2.16 & 
780.44 & 2.31 & \bf{740.85} & 
3.18\\CON8-2 & 727.69 & 2.84 & 
733.58 & 2.85 & \bf{712.89} & 
2.08\\CON8-3 & 835.64 & 1.78 & 
844.65 & 2.65 & \bf{811.07} & 
3.03\\CON8-4 & 813.23 & 2.45 & 
825.06 & 2.29 & \bf{772.25} & 
5.31\\CON8-5 & 778.59 & 2.31 & 
785.41 & 2.54 & \bf{754.88} & 
3.14\\CON8-6 & 699.80 & 3.13 & 
712.44 & 2.65 & \bf{678.92} & 
3.08\\CON8-7 & 863.21 & 2.11 & 
869.54 & 2.40 & \bf{811.96} & 
6.31\\CON8-8 & 793.52 & 2.39 & 
798.35 & 2.39 & \bf{767.53} & 
3.39\\CON8-9 & 827.60 & 3.14 & 
839.03 & 2.85 & \bf{809.00} & 
2.30\\\bf{PROM.} & 
\bf{776.42} & \bf{2.64} & \bf{787.42} & \bf{2.65} & \bf{758.54} & \bf{2.22}\\[1ex]\hline
\end{tabular}
\label{table:nonlin}
\end{table} \clearpage
\begin{table}[ht]
\caption{Resultados de la ejecución de la metaheurística ILS, utilizando instancias de Dethloff con la configuración -n 15.0 -LS 60.0}
\centering
\small
\begin{tabular}{c c c c c c c}
\hline\hline
Instancia & Costo mínimo & Tiempo(seg.) & Costo promedio & Tiempo promedio(seg.) & Costo ILS & \%Gap \\ [0.5ex]
\hline
SCA3-0 & 640.55 & 3.60 & 
642.47 & 3.68 & \bf{635.62} & 
0.78\\SCA3-1 & \bf{697.84} & 3.80 & 
709.67 & 3.16 & 697.84 & 0.00\\
SCA3-2 & 664.21 & 3.61 & 
670.49 & 3.50 & \bf{659.34} & 
0.74\\SCA3-3 & 688.71 & 3.08 & 
691.89 & 3.21 & \bf{680.04} & 
1.27\\SCA3-4 & \bf{690.50} & 2.66 & 
696.44 & 3.25 & 690.50 & 0.00\\
SCA3-5 & 681.81 & 3.40 & 
682.83 & 3.28 & \bf{659.90} & 
3.32\\SCA3-6 & 652.94 & 2.89 & 
657.08 & 3.04 & \bf{651.09} & 
0.28\\SCA3-7 & 671.67 & 3.12 & 
673.98 & 2.99 & \bf{659.17} & 
1.90\\SCA3-8 & 722.05 & 2.24 & 
729.83 & 3.03 & \bf{719.47} & 
0.36\\SCA3-9 & 684.44 & 3.36 & 
688.17 & 2.78 & \bf{681.00} & 
0.51\\SCA8-0 & 1001.24 & 2.78 & 
1024.56 & 3.29 & \bf{961.50} & 
4.13\\SCA8-1 & 1080.34 & 3.33 & 
1098.20 & 3.09 & \bf{1049.65} & 
2.92\\SCA8-2 & 1054.69 & 3.70 & 
1076.45 & 2.96 & \bf{1039.64} & 
1.45\\SCA8-3 & 1013.56 & 3.68 & 
1034.87 & 2.83 & \bf{983.34} & 
3.07\\SCA8-4 & 1116.49 & 2.71 & 
1129.05 & 2.79 & \bf{1065.49} & 
4.79\\SCA8-5 & 1051.68 & 2.83 & 
1084.48 & 2.64 & \bf{1027.08} & 
2.40\\SCA8-6 & 986.87 & 2.86 & 
1008.29 & 2.65 & \bf{971.82} & 
1.55\\SCA8-7 & 1070.53 & 3.13 & 
1097.44 & 3.09 & \bf{1051.28} & 
1.83\\SCA8-8 & 1090.13 & 4.48 & 
1097.67 & 3.42 & \bf{1071.18} & 
1.77\\SCA8-9 & 1119.39 & 2.52 & 
1126.38 & 2.48 & \bf{1060.50} & 
5.55\\CON3-0 & 633.24 & 3.09 & 
641.35 & 3.44 & \bf{616.52} & 
2.71\\CON3-1 & 569.07 & 3.35 & 
569.72 & 3.54 & \bf{554.47} & 
2.63\\CON3-2 & 521.63 & 3.55 & 
528.02 & 3.47 & \bf{518.00} & 
0.70\\CON3-3 & \bf{591.19} & 3.68 & 
611.30 & 3.35 & 591.19 & 0.00\\
CON3-4 & 607.56 & 3.76 & 
613.12 & 3.20 & \bf{588.79} & 
3.19\\CON3-5 & 573.93 & 3.87 & 
584.76 & 4.33 & \bf{563.70} & 
1.81\\CON3-6 & 502.16 & 4.00 & 
509.28 & 3.17 & \bf{499.05} & 
0.62\\CON3-7 & 578.41 & 3.43 & 
590.63 & 3.54 & \bf{576.48} & 
0.33\\CON3-8 & 527.82 & 2.88 & 
529.06 & 3.41 & \bf{523.05} & 
0.91\\CON3-9 & 578.25 & 2.86 & 
588.07 & 3.39 & \bf{578.24} & 
0.00\\CON8-0 & 906.70 & 2.13 & 
924.12 & 2.36 & \bf{857.17} & 
5.78\\CON8-1 & 767.78 & 3.38 & 
782.16 & 2.98 & \bf{740.85} & 
3.64\\CON8-2 & 716.42 & 2.76 & 
730.41 & 3.08 & \bf{712.89} & 
0.50\\CON8-3 & 838.05 & 3.07 & 
847.90 & 2.93 & \bf{811.07} & 
3.33\\CON8-4 & 816.10 & 3.74 & 
829.37 & 2.74 & \bf{772.25} & 
5.68\\CON8-5 & 786.26 & 2.38 & 
799.15 & 2.63 & \bf{754.88} & 
4.16\\CON8-6 & 710.95 & 4.43 & 
716.80 & 3.35 & \bf{678.92} & 
4.72\\CON8-7 & 837.10 & 2.59 & 
848.37 & 2.63 & \bf{811.96} & 
3.10\\CON8-8 & 790.42 & 2.03 & 
818.38 & 2.36 & \bf{767.53} & 
2.98\\CON8-9 & 814.10 & 3.80 & 
851.04 & 3.00 & \bf{809.00} & 
0.63\\\bf{PROM.} & 
\bf{776.17} & \bf{3.21} & \bf{788.33} & \bf{3.10} & \bf{758.54} & \bf{2.15}\\[1ex]\hline
\end{tabular}
\label{table:nonlin}
\end{table} \clearpage
\begin{table}[ht]
\caption{Resultados de la ejecución de la metaheurística ILS, utilizando instancias de Dethloff con la configuración -n 15.0 -LS 70.0}
\centering
\small
\begin{tabular}{c c c c c c c}
\hline\hline
Instancia & Costo mínimo & Tiempo(seg.) & Costo promedio & Tiempo promedio(seg.) & Costo ILS & \%Gap \\ [0.5ex]
\hline
SCA3-0 & 640.55 & 3.85 & 
640.55 & 4.34 & \bf{635.62} & 
0.78\\SCA3-1 & 701.74 & 3.85 & 
710.73 & 3.87 & \bf{697.84} & 
0.56\\SCA3-2 & 664.18 & 3.39 & 
668.87 & 3.85 & \bf{659.34} & 
0.73\\SCA3-3 & 680.60 & 3.79 & 
686.12 & 3.76 & \bf{680.04} & 
0.08\\SCA3-4 & \bf{690.50} & 4.08 & 
699.39 & 3.66 & 690.50 & 0.00\\
SCA3-5 & 670.02 & 3.58 & 
678.28 & 3.53 & \bf{659.90} & 
1.53\\SCA3-6 & 658.26 & 4.14 & 
661.70 & 3.60 & \bf{651.09} & 
1.10\\SCA3-7 & 666.15 & 3.54 & 
670.10 & 3.96 & \bf{659.17} & 
1.06\\SCA3-8 & 724.29 & 5.16 & 
733.03 & 4.09 & \bf{719.47} & 
0.67\\SCA3-9 & 684.44 & 4.22 & 
690.09 & 3.44 & \bf{681.00} & 
0.51\\SCA8-0 & 970.64 & 3.47 & 
992.63 & 3.31 & \bf{961.50} & 
0.95\\SCA8-1 & 1094.34 & 2.84 & 
1107.14 & 2.92 & \bf{1049.65} & 
4.26\\SCA8-2 & 1053.50 & 4.38 & 
1067.80 & 3.66 & \bf{1039.64} & 
1.33\\SCA8-3 & 1019.06 & 2.73 & 
1032.78 & 2.92 & \bf{983.34} & 
3.63\\SCA8-4 & 1067.55 & 3.26 & 
1103.88 & 2.86 & \bf{1065.49} & 
0.19\\SCA8-5 & 1094.60 & 3.55 & 
1097.64 & 3.17 & \bf{1027.08} & 
6.57\\SCA8-6 & 993.28 & 3.48 & 
1006.06 & 3.24 & \bf{971.82} & 
2.21\\SCA8-7 & 1081.46 & 2.83 & 
1092.44 & 2.88 & \bf{1051.28} & 
2.87\\SCA8-8 & \bf{1071.18} & 3.58 & 
1095.32 & 3.32 & 1071.18 & 0.00\\
SCA8-9 & 1078.21 & 2.87 & 
1098.01 & 3.41 & \bf{1060.50} & 
1.67\\CON3-0 & 633.24 & 3.47 & 
634.76 & 3.80 & \bf{616.52} & 
2.71\\CON3-1 & 562.52 & 3.99 & 
569.05 & 3.90 & \bf{554.47} & 
1.45\\CON3-2 & 519.11 & 3.83 & 
523.37 & 4.08 & \bf{518.00} & 
0.21\\CON3-3 & 599.26 & 4.78 & 
612.29 & 4.08 & \bf{591.19} & 
1.37\\CON3-4 & 593.78 & 4.34 & 
607.30 & 4.34 & \bf{588.79} & 
0.85\\CON3-5 & 569.04 & 4.46 & 
572.65 & 4.08 & \bf{563.70} & 
0.95\\CON3-6 & 505.99 & 3.60 & 
510.04 & 3.57 & \bf{499.05} & 
1.39\\CON3-7 & 583.65 & 3.78 & 
592.64 & 3.78 & \bf{576.48} & 
1.24\\CON3-8 & 524.59 & 3.70 & 
536.00 & 3.83 & \bf{523.05} & 
0.29\\CON3-9 & 578.25 & 2.48 & 
587.77 & 4.05 & \bf{578.24} & 
0.00\\CON8-0 & 869.87 & 2.94 & 
897.65 & 3.00 & \bf{857.17} & 
1.48\\CON8-1 & 777.27 & 6.56 & 
791.99 & 4.10 & \bf{740.85} & 
4.92\\CON8-2 & 735.75 & 3.06 & 
743.03 & 3.06 & \bf{712.89} & 
3.21\\CON8-3 & 826.64 & 3.68 & 
843.89 & 3.88 & \bf{811.07} & 
1.92\\CON8-4 & 823.60 & 3.17 & 
829.53 & 2.93 & \bf{772.25} & 
6.65\\CON8-5 & 772.95 & 3.80 & 
775.68 & 3.24 & \bf{754.88} & 
2.39\\CON8-6 & 700.72 & 3.51 & 
709.08 & 3.40 & \bf{678.92} & 
3.21\\CON8-7 & 814.50 & 3.10 & 
830.28 & 3.31 & \bf{811.96} & 
0.31\\CON8-8 & 781.53 & 4.82 & 
795.13 & 3.52 & \bf{767.53} & 
1.82\\CON8-9 & 820.41 & 4.06 & 
836.63 & 4.04 & \bf{809.00} & 
1.41\\\bf{PROM.} & 
\bf{772.43} & \bf{3.74} & \bf{783.28} & \bf{3.60} & \bf{758.54} & \bf{1.71}\\[1ex]\hline
\end{tabular}
\label{table:nonlin}
\end{table} \clearpage
\begin{table}[ht]
\caption{Resultados de la ejecución de la metaheurística ILS, utilizando instancias de Dethloff con la configuración -n 15.0 -LS 80.0}
\centering
\small
\begin{tabular}{c c c c c c c}
\hline\hline
Instancia & Costo mínimo & Tiempo(seg.) & Costo promedio & Tiempo promedio(seg.) & Costo ILS & \%Gap \\ [0.5ex]
\hline
SCA3-0 & 641.69 & 4.44 & 
643.36 & 4.50 & \bf{635.62} & 
0.95\\SCA3-1 & \bf{697.84} & 4.31 & 
708.70 & 4.13 & 697.84 & 0.00\\
SCA3-2 & 664.21 & 4.50 & 
665.51 & 4.58 & \bf{659.34} & 
0.74\\SCA3-3 & \bf{680.04} & 4.42 & 
684.29 & 4.67 & 680.04 & 0.00\\
SCA3-4 & \bf{690.50} & 4.22 & 
697.00 & 4.45 & 690.50 & 0.00\\
SCA3-5 & 670.10 & 4.12 & 
676.84 & 4.44 & \bf{659.90} & 
1.55\\SCA3-6 & \bf{651.09} & 4.23 & 
653.80 & 4.57 & 651.09 & 0.00\\
SCA3-7 & 671.67 & 4.01 & 
673.34 & 4.09 & \bf{659.17} & 
1.90\\SCA3-8 & \bf{719.47} & 5.60 & 
729.26 & 4.71 & 719.47 & 0.00\\
SCA3-9 & \bf{681.00} & 4.16 & 
694.21 & 3.97 & 681.00 & 0.00\\
SCA8-0 & 1016.38 & 3.14 & 
1039.87 & 3.50 & \bf{961.50} & 
5.71\\SCA8-1 & 1081.69 & 2.83 & 
1098.72 & 2.83 & \bf{1049.65} & 
3.05\\SCA8-2 & 1066.96 & 2.91 & 
1072.10 & 3.45 & \bf{1039.64} & 
2.63\\SCA8-3 & 1023.04 & 3.25 & 
1036.71 & 3.63 & \bf{983.34} & 
4.04\\SCA8-4 & 1071.44 & 3.49 & 
1109.73 & 3.29 & \bf{1065.49} & 
0.56\\SCA8-5 & 1054.62 & 4.70 & 
1078.71 & 3.82 & \bf{1027.08} & 
2.68\\SCA8-6 & 992.79 & 4.07 & 
1004.05 & 4.06 & \bf{971.82} & 
2.16\\SCA8-7 & 1091.63 & 5.52 & 
1101.47 & 4.51 & \bf{1051.28} & 
3.84\\SCA8-8 & 1105.20 & 3.46 & 
1112.16 & 3.36 & \bf{1071.18} & 
3.18\\SCA8-9 & 1090.14 & 3.42 & 
1092.73 & 3.16 & \bf{1060.50} & 
2.79\\CON3-0 & 632.57 & 3.90 & 
641.70 & 4.06 & \bf{616.52} & 
2.60\\CON3-1 & 561.87 & 3.76 & 
567.50 & 4.65 & \bf{554.47} & 
1.33\\CON3-2 & 524.89 & 4.16 & 
535.70 & 3.97 & \bf{518.00} & 
1.33\\CON3-3 & 591.20 & 4.02 & 
601.83 & 4.33 & \bf{591.19} & 
0.00\\CON3-4 & 589.32 & 4.44 & 
591.16 & 4.05 & \bf{588.79} & 
0.09\\CON3-5 & 568.85 & 3.71 & 
574.24 & 4.29 & \bf{563.70} & 
0.91\\CON3-6 & 508.14 & 4.36 & 
512.75 & 3.81 & \bf{499.05} & 
1.82\\CON3-7 & 578.41 & 4.57 & 
588.59 & 4.34 & \bf{576.48} & 
0.33\\CON3-8 & 524.59 & 5.34 & 
528.55 & 4.99 & \bf{523.05} & 
0.29\\CON3-9 & 588.99 & 4.10 & 
591.18 & 4.39 & \bf{578.24} & 
1.86\\CON8-0 & 895.71 & 3.50 & 
908.88 & 3.33 & \bf{857.17} & 
4.50\\CON8-1 & 785.75 & 4.37 & 
792.75 & 3.68 & \bf{740.85} & 
6.06\\CON8-2 & 732.65 & 3.75 & 
736.25 & 3.37 & \bf{712.89} & 
2.77\\CON8-3 & 844.61 & 5.44 & 
856.23 & 4.09 & \bf{811.07} & 
4.14\\CON8-4 & 796.27 & 2.85 & 
809.30 & 3.82 & \bf{772.25} & 
3.11\\CON8-5 & 767.91 & 3.94 & 
776.24 & 3.81 & \bf{754.88} & 
1.73\\CON8-6 & 705.14 & 5.48 & 
714.47 & 4.24 & \bf{678.92} & 
3.86\\CON8-7 & 815.32 & 4.17 & 
835.56 & 3.72 & \bf{811.96} & 
0.41\\CON8-8 & 802.01 & 3.02 & 
809.11 & 3.44 & \bf{767.53} & 
4.49\\CON8-9 & 821.98 & 4.36 & 
835.27 & 3.97 & \bf{809.00} & 
1.60\\\bf{PROM.} & 
\bf{774.94} & \bf{4.10} & \bf{784.50} & \bf{4.00} & \bf{758.54} & \bf{1.98}\\[1ex]\hline
\end{tabular}
\label{table:nonlin}
\end{table} \clearpage
\begin{table}[ht]
\caption{Resultados de la ejecución de la metaheurística ILS, utilizando instancias de Dethloff con la configuración -n 25.0 -LS 10.0}
\centering
\small
\begin{tabular}{c c c c c c c}
\hline\hline
Instancia & Costo mínimo & Tiempo(seg.) & Costo promedio & Tiempo promedio(seg.) & Costo ILS & \%Gap \\ [0.5ex]
\hline
SCA3-0 & 640.55 & 1.60 & 
643.19 & 1.53 & \bf{635.62} & 
0.78\\SCA3-1 & 700.50 & 2.12 & 
726.25 & 1.63 & \bf{697.84} & 
0.38\\SCA3-2 & 675.12 & 1.42 & 
683.29 & 1.53 & \bf{659.34} & 
2.39\\SCA3-3 & 682.46 & 1.54 & 
688.54 & 1.49 & \bf{680.04} & 
0.36\\SCA3-4 & \bf{690.50} & 1.38 & 
701.73 & 1.44 & 690.50 & 0.00\\
SCA3-5 & \bf{659.90} & 1.78 & 
673.82 & 1.62 & 659.90 & 0.00\\
SCA3-6 & 652.94 & 1.67 & 
655.67 & 1.55 & \bf{651.09} & 
0.28\\SCA3-7 & 672.85 & 1.42 & 
673.85 & 1.54 & \bf{659.17} & 
2.08\\SCA3-8 & \bf{719.47} & 1.52 & 
733.67 & 1.65 & 719.47 & 0.00\\
SCA3-9 & 690.07 & 1.58 & 
700.76 & 1.62 & \bf{681.00} & 
1.33\\SCA8-0 & 1015.61 & 1.98 & 
1027.39 & 1.87 & \bf{961.50} & 
5.63\\SCA8-1 & 1097.18 & 1.29 & 
1117.42 & 1.26 & \bf{1049.65} & 
4.53\\SCA8-2 & 1092.94 & 1.69 & 
1102.00 & 1.41 & \bf{1039.64} & 
5.13\\SCA8-3 & 1025.40 & 1.03 & 
1044.06 & 1.36 & \bf{983.34} & 
4.28\\SCA8-4 & 1121.03 & 1.63 & 
1156.94 & 1.47 & \bf{1065.49} & 
5.21\\SCA8-5 & 1071.46 & 1.36 & 
1091.91 & 1.50 & \bf{1027.08} & 
4.32\\SCA8-6 & 1013.70 & 1.62 & 
1025.38 & 1.25 & \bf{971.82} & 
4.31\\SCA8-7 & 1083.63 & 1.21 & 
1101.53 & 1.32 & \bf{1051.28} & 
3.08\\SCA8-8 & 1114.83 & 1.36 & 
1120.43 & 1.32 & \bf{1071.18} & 
4.07\\SCA8-9 & 1102.71 & 1.18 & 
1118.88 & 1.22 & \bf{1060.50} & 
3.98\\CON3-0 & 636.77 & 1.72 & 
639.11 & 1.68 & \bf{616.52} & 
3.28\\CON3-1 & 564.81 & 1.46 & 
566.50 & 1.46 & \bf{554.47} & 
1.86\\CON3-2 & 523.23 & 1.45 & 
529.21 & 1.70 & \bf{518.00} & 
1.01\\CON3-3 & 594.31 & 1.34 & 
613.26 & 1.34 & \bf{591.19} & 
0.53\\CON3-4 & 603.13 & 1.98 & 
609.41 & 1.61 & \bf{588.79} & 
2.44\\CON3-5 & 569.04 & 1.52 & 
586.28 & 1.57 & \bf{563.70} & 
0.95\\CON3-6 & 502.95 & 2.05 & 
507.77 & 1.80 & \bf{499.05} & 
0.78\\CON3-7 & 600.35 & 1.54 & 
607.71 & 1.48 & \bf{576.48} & 
4.14\\CON3-8 & 528.59 & 1.38 & 
533.48 & 1.61 & \bf{523.05} & 
1.06\\CON3-9 & 588.99 & 1.71 & 
596.03 & 1.38 & \bf{578.24} & 
1.86\\CON8-0 & 898.15 & 1.76 & 
928.87 & 1.43 & \bf{857.17} & 
4.78\\CON8-1 & 763.10 & 2.39 & 
772.95 & 1.82 & \bf{740.85} & 
3.00\\CON8-2 & 731.70 & 1.88 & 
737.10 & 1.66 & \bf{712.89} & 
2.64\\CON8-3 & 836.03 & 1.34 & 
850.09 & 1.31 & \bf{811.07} & 
3.08\\CON8-4 & 802.74 & 1.38 & 
824.89 & 1.43 & \bf{772.25} & 
3.95\\CON8-5 & 776.35 & 1.65 & 
802.90 & 1.55 & \bf{754.88} & 
2.84\\CON8-6 & 710.16 & 1.58 & 
718.62 & 1.39 & \bf{678.92} & 
4.60\\CON8-7 & 841.49 & 1.34 & 
844.21 & 1.36 & \bf{811.96} & 
3.64\\CON8-8 & 778.39 & 1.60 & 
798.16 & 1.52 & \bf{767.53} & 
1.41\\CON8-9 & 840.92 & 1.47 & 
865.62 & 1.57 & \bf{809.00} & 
3.95\\\bf{PROM.} & 
\bf{780.35} & \bf{1.57} & \bf{792.97} & \bf{1.51} & \bf{758.54} & \bf{2.60}\\[1ex]\hline
\end{tabular}
\label{table:nonlin}
\end{table} \clearpage
\begin{table}[ht]
\caption{Resultados de la ejecución de la metaheurística ILS, utilizando instancias de Dethloff con la configuración -n 25.0 -LS 20.0}
\centering
\small
\begin{tabular}{c c c c c c c}
\hline\hline
Instancia & Costo mínimo & Tiempo(seg.) & Costo promedio & Tiempo promedio(seg.) & Costo ILS & \%Gap \\ [0.5ex]
\hline
SCA3-0 & 640.55 & 2.41 & 
642.67 & 2.39 & \bf{635.62} & 
0.78\\SCA3-1 & 712.59 & 2.59 & 
730.60 & 2.56 & \bf{697.84} & 
2.11\\SCA3-2 & 669.06 & 2.37 & 
674.86 & 2.09 & \bf{659.34} & 
1.47\\SCA3-3 & 680.60 & 2.16 & 
689.32 & 2.32 & \bf{680.04} & 
0.08\\SCA3-4 & \bf{690.50} & 2.26 & 
692.62 & 2.44 & 690.50 & 0.00\\
SCA3-5 & 670.02 & 2.22 & 
682.74 & 2.29 & \bf{659.90} & 
1.53\\SCA3-6 & 660.55 & 2.85 & 
666.62 & 2.71 & \bf{651.09} & 
1.45\\SCA3-7 & 671.77 & 2.36 & 
672.04 & 2.24 & \bf{659.17} & 
1.91\\SCA3-8 & 727.81 & 2.71 & 
735.28 & 2.27 & \bf{719.47} & 
1.16\\SCA3-9 & 689.95 & 2.20 & 
694.33 & 2.31 & \bf{681.00} & 
1.31\\SCA8-0 & 985.12 & 2.01 & 
1024.09 & 2.34 & \bf{961.50} & 
2.46\\SCA8-1 & 1081.37 & 2.43 & 
1090.61 & 2.08 & \bf{1049.65} & 
3.02\\SCA8-2 & 1070.14 & 2.04 & 
1082.03 & 1.80 & \bf{1039.64} & 
2.93\\SCA8-3 & 1010.82 & 1.96 & 
1026.39 & 1.88 & \bf{983.34} & 
2.79\\SCA8-4 & 1074.18 & 1.97 & 
1103.57 & 2.01 & \bf{1065.49} & 
0.82\\SCA8-5 & 1066.92 & 2.56 & 
1081.87 & 2.16 & \bf{1027.08} & 
3.88\\SCA8-6 & 1000.62 & 2.12 & 
1014.21 & 1.91 & \bf{971.82} & 
2.96\\SCA8-7 & 1121.88 & 1.84 & 
1132.35 & 2.06 & \bf{1051.28} & 
6.72\\SCA8-8 & \bf{1071.18} & 1.97 & 
1096.62 & 2.22 & 1071.18 & 0.00\\
SCA8-9 & 1103.55 & 1.97 & 
1109.80 & 1.96 & \bf{1060.50} & 
4.06\\CON3-0 & 630.73 & 1.99 & 
633.23 & 2.22 & \bf{616.52} & 
2.30\\CON3-1 & 562.52 & 2.66 & 
569.99 & 2.38 & \bf{554.47} & 
1.45\\CON3-2 & 528.77 & 2.18 & 
536.15 & 2.19 & \bf{518.00} & 
2.08\\CON3-3 & 594.10 & 2.06 & 
601.51 & 2.20 & \bf{591.19} & 
0.49\\CON3-4 & 603.60 & 2.55 & 
610.82 & 2.51 & \bf{588.79} & 
2.52\\CON3-5 & 575.81 & 2.30 & 
581.30 & 2.30 & \bf{563.70} & 
2.15\\CON3-6 & 510.81 & 2.63 & 
513.96 & 2.48 & \bf{499.05} & 
2.36\\CON3-7 & 592.77 & 2.26 & 
599.76 & 2.30 & \bf{576.48} & 
2.83\\CON3-8 & 532.86 & 2.29 & 
541.09 & 2.49 & \bf{523.05} & 
1.88\\CON3-9 & 588.48 & 2.42 & 
588.96 & 2.19 & \bf{578.24} & 
1.77\\CON8-0 & 890.52 & 2.21 & 
904.29 & 1.91 & \bf{857.17} & 
3.89\\CON8-1 & 767.37 & 2.63 & 
777.86 & 2.08 & \bf{740.85} & 
3.58\\CON8-2 & 727.95 & 2.30 & 
730.27 & 2.29 & \bf{712.89} & 
2.11\\CON8-3 & 845.95 & 2.03 & 
851.84 & 1.89 & \bf{811.07} & 
4.30\\CON8-4 & 772.73 & 2.24 & 
795.42 & 2.13 & \bf{772.25} & 
0.06\\CON8-5 & 778.93 & 1.76 & 
814.13 & 1.90 & \bf{754.88} & 
3.19\\CON8-6 & 701.44 & 1.87 & 
709.59 & 2.07 & \bf{678.92} & 
3.32\\CON8-7 & 816.67 & 2.19 & 
830.79 & 2.16 & \bf{811.96} & 
0.58\\CON8-8 & 793.89 & 2.22 & 
800.81 & 1.90 & \bf{767.53} & 
3.43\\CON8-9 & 835.56 & 2.34 & 
841.42 & 1.95 & \bf{809.00} & 
3.28\\\bf{PROM.} & 
\bf{776.27} & \bf{2.25} & \bf{786.90} & \bf{2.19} & \bf{758.54} & \bf{2.23}\\[1ex]\hline
\end{tabular}
\label{table:nonlin}
\end{table} \clearpage
\begin{table}[ht]
\caption{Resultados de la ejecución de la metaheurística ILS, utilizando instancias de Dethloff con la configuración -n 25.0 -LS 30.0}
\centering
\small
\begin{tabular}{c c c c c c c}
\hline\hline
Instancia & Costo mínimo & Tiempo(seg.) & Costo promedio & Tiempo promedio(seg.) & Costo ILS & \%Gap \\ [0.5ex]
\hline
SCA3-0 & 641.69 & 3.38 & 
643.50 & 3.12 & \bf{635.62} & 
0.95\\SCA3-1 & 701.53 & 2.99 & 
706.08 & 3.09 & \bf{697.84} & 
0.53\\SCA3-2 & 666.05 & 3.72 & 
667.62 & 3.29 & \bf{659.34} & 
1.02\\SCA3-3 & 681.74 & 3.59 & 
686.59 & 3.43 & \bf{680.04} & 
0.25\\SCA3-4 & \bf{690.50} & 2.91 & 
691.18 & 3.05 & 690.50 & 0.00\\
SCA3-5 & 678.64 & 3.15 & 
682.84 & 3.47 & \bf{659.90} & 
2.84\\SCA3-6 & \bf{651.09} & 2.58 & 
653.41 & 3.33 & 651.09 & 0.00\\
SCA3-7 & 671.67 & 3.03 & 
672.78 & 2.92 & \bf{659.17} & 
1.90\\SCA3-8 & 721.45 & 3.58 & 
728.50 & 3.28 & \bf{719.47} & 
0.28\\SCA3-9 & \bf{681.00} & 2.78 & 
686.26 & 2.73 & 681.00 & 0.00\\
SCA8-0 & 1011.47 & 2.72 & 
1028.62 & 2.59 & \bf{961.50} & 
5.20\\SCA8-1 & 1068.14 & 2.45 & 
1088.68 & 2.79 & \bf{1049.65} & 
1.76\\SCA8-2 & 1065.44 & 2.63 & 
1076.08 & 2.85 & \bf{1039.64} & 
2.48\\SCA8-3 & 1024.48 & 2.21 & 
1033.79 & 2.46 & \bf{983.34} & 
4.18\\SCA8-4 & 1124.66 & 2.58 & 
1138.17 & 2.59 & \bf{1065.49} & 
5.55\\SCA8-5 & 1066.92 & 2.46 & 
1081.50 & 2.73 & \bf{1027.08} & 
3.88\\SCA8-6 & 996.13 & 3.22 & 
1004.44 & 2.75 & \bf{971.82} & 
2.50\\SCA8-7 & 1079.57 & 2.16 & 
1098.16 & 2.69 & \bf{1051.28} & 
2.69\\SCA8-8 & 1084.57 & 2.97 & 
1103.40 & 3.04 & \bf{1071.18} & 
1.25\\SCA8-9 & 1080.03 & 3.46 & 
1092.77 & 2.70 & \bf{1060.50} & 
1.84\\CON3-0 & 636.88 & 3.32 & 
639.89 & 3.37 & \bf{616.52} & 
3.30\\CON3-1 & 564.64 & 3.62 & 
568.80 & 3.52 & \bf{554.47} & 
1.83\\CON3-2 & 521.63 & 3.63 & 
522.88 & 3.28 & \bf{518.00} & 
0.70\\CON3-3 & 591.48 & 3.50 & 
601.76 & 3.58 & \bf{591.19} & 
0.05\\CON3-4 & 597.27 & 3.09 & 
606.82 & 3.07 & \bf{588.79} & 
1.44\\CON3-5 & 569.04 & 3.74 & 
570.70 & 3.23 & \bf{563.70} & 
0.95\\CON3-6 & 510.49 & 3.09 & 
515.14 & 3.06 & \bf{499.05} & 
2.29\\CON3-7 & 578.41 & 3.15 & 
584.40 & 2.96 & \bf{576.48} & 
0.33\\CON3-8 & 524.30 & 4.15 & 
526.02 & 3.53 & \bf{523.05} & 
0.24\\CON3-9 & 578.98 & 2.87 & 
587.66 & 3.41 & \bf{578.24} & 
0.13\\CON8-0 & 883.76 & 2.94 & 
902.27 & 2.90 & \bf{857.17} & 
3.10\\CON8-1 & 779.35 & 2.16 & 
788.96 & 2.80 & \bf{740.85} & 
5.20\\CON8-2 & 731.50 & 3.13 & 
736.60 & 2.78 & \bf{712.89} & 
2.61\\CON8-3 & 844.07 & 2.49 & 
865.07 & 2.58 & \bf{811.07} & 
4.07\\CON8-4 & 805.62 & 1.99 & 
818.17 & 2.90 & \bf{772.25} & 
4.32\\CON8-5 & 764.15 & 2.88 & 
782.40 & 2.78 & \bf{754.88} & 
1.23\\CON8-6 & 718.10 & 4.23 & 
726.65 & 3.31 & \bf{678.92} & 
5.77\\CON8-7 & 825.53 & 2.28 & 
836.62 & 2.60 & \bf{811.96} & 
1.67\\CON8-8 & 784.18 & 2.30 & 
791.27 & 2.54 & \bf{767.53} & 
2.17\\CON8-9 & 819.34 & 2.59 & 
842.11 & 2.79 & \bf{809.00} & 
1.28\\\bf{PROM.} & 
\bf{775.39} & \bf{2.99} & \bf{784.46} & \bf{3.00} & \bf{758.54} & \bf{2.04}\\[1ex]\hline
\end{tabular}
\label{table:nonlin}
\end{table} \clearpage
\begin{table}[ht]
\caption{Resultados de la ejecución de la metaheurística ILS, utilizando instancias de Dethloff con la configuración -n 25.0 -LS 40.0}
\centering
\small
\begin{tabular}{c c c c c c c}
\hline\hline
Instancia & Costo mínimo & Tiempo(seg.) & Costo promedio & Tiempo promedio(seg.) & Costo ILS & \%Gap \\ [0.5ex]
\hline
SCA3-0 & 641.69 & 4.63 & 
642.63 & 4.25 & \bf{635.62} & 
0.95\\SCA3-1 & 700.50 & 4.10 & 
709.55 & 4.01 & \bf{697.84} & 
0.38\\SCA3-2 & 664.21 & 4.28 & 
672.85 & 3.95 & \bf{659.34} & 
0.74\\SCA3-3 & 681.74 & 3.90 & 
685.25 & 3.65 & \bf{680.04} & 
0.25\\SCA3-4 & \bf{690.50} & 3.94 & 
692.22 & 4.13 & 690.50 & 0.00\\
SCA3-5 & 680.80 & 4.25 & 
686.26 & 4.26 & \bf{659.90} & 
3.17\\SCA3-6 & 652.47 & 3.46 & 
654.19 & 3.58 & \bf{651.09} & 
0.21\\SCA3-7 & 671.77 & 3.50 & 
676.11 & 3.55 & \bf{659.17} & 
1.91\\SCA3-8 & \bf{719.47} & 3.61 & 
730.11 & 3.52 & 719.47 & 0.00\\
SCA3-9 & \bf{681.00} & 4.48 & 
690.88 & 4.03 & 681.00 & 0.00\\
SCA8-0 & 981.73 & 3.74 & 
1005.63 & 3.87 & \bf{961.50} & 
2.10\\SCA8-1 & 1085.54 & 3.06 & 
1086.60 & 3.27 & \bf{1049.65} & 
3.42\\SCA8-2 & 1062.62 & 5.10 & 
1067.91 & 3.54 & \bf{1039.64} & 
2.21\\SCA8-3 & 1032.69 & 3.24 & 
1040.32 & 3.33 & \bf{983.34} & 
5.02\\SCA8-4 & 1074.28 & 3.17 & 
1112.38 & 3.25 & \bf{1065.49} & 
0.82\\SCA8-5 & 1049.98 & 3.55 & 
1072.99 & 3.35 & \bf{1027.08} & 
2.23\\SCA8-6 & 991.49 & 3.53 & 
998.24 & 2.87 & \bf{971.82} & 
2.02\\SCA8-7 & 1066.65 & 4.56 & 
1093.26 & 3.54 & \bf{1051.28} & 
1.46\\SCA8-8 & 1088.20 & 3.50 & 
1104.36 & 3.13 & \bf{1071.18} & 
1.59\\SCA8-9 & 1119.56 & 3.74 & 
1130.42 & 3.09 & \bf{1060.50} & 
5.57\\CON3-0 & 617.59 & 4.37 & 
631.20 & 4.28 & \bf{616.52} & 
0.17\\CON3-1 & 560.61 & 3.27 & 
564.99 & 3.62 & \bf{554.47} & 
1.11\\CON3-2 & 524.13 & 4.51 & 
528.31 & 4.25 & \bf{518.00} & 
1.18\\CON3-3 & 591.48 & 4.85 & 
600.22 & 4.17 & \bf{591.19} & 
0.05\\CON3-4 & 599.13 & 4.19 & 
601.95 & 4.05 & \bf{588.79} & 
1.76\\CON3-5 & 569.04 & 4.48 & 
574.53 & 4.47 & \bf{563.70} & 
0.95\\CON3-6 & 505.14 & 4.36 & 
511.65 & 4.10 & \bf{499.05} & 
1.22\\CON3-7 & 586.53 & 3.84 & 
598.45 & 3.92 & \bf{576.48} & 
1.74\\CON3-8 & 524.59 & 4.96 & 
532.31 & 4.20 & \bf{523.05} & 
0.29\\CON3-9 & 589.57 & 3.64 & 
591.51 & 4.06 & \bf{578.24} & 
1.96\\CON8-0 & 872.79 & 3.52 & 
894.27 & 3.54 & \bf{857.17} & 
1.82\\CON8-1 & 773.48 & 3.21 & 
777.26 & 3.63 & \bf{740.85} & 
4.40\\CON8-2 & 722.56 & 3.74 & 
732.73 & 3.62 & \bf{712.89} & 
1.36\\CON8-3 & 845.27 & 3.21 & 
848.23 & 3.18 & \bf{811.07} & 
4.22\\CON8-4 & 813.27 & 2.85 & 
825.95 & 2.63 & \bf{772.25} & 
5.31\\CON8-5 & 767.45 & 4.07 & 
775.88 & 3.61 & \bf{754.88} & 
1.67\\CON8-6 & 703.00 & 3.05 & 
710.01 & 3.77 & \bf{678.92} & 
3.55\\CON8-7 & 816.07 & 4.33 & 
829.87 & 3.35 & \bf{811.96} & 
0.51\\CON8-8 & 787.24 & 4.46 & 
800.91 & 3.83 & \bf{767.53} & 
2.57\\CON8-9 & 823.94 & 4.56 & 
849.08 & 4.19 & \bf{809.00} & 
1.85\\\bf{PROM.} & 
\bf{773.24} & \bf{3.92} & \bf{783.29} & \bf{3.72} & \bf{758.54} & \bf{1.79}\\[1ex]\hline
\end{tabular}
\label{table:nonlin}
\end{table} \clearpage
\begin{table}[ht]
\caption{Resultados de la ejecución de la metaheurística ILS, utilizando instancias de Dethloff con la configuración -n 25.0 -LS 50.0}
\centering
\small
\begin{tabular}{c c c c c c c}
\hline\hline
Instancia & Costo mínimo & Tiempo(seg.) & Costo promedio & Tiempo promedio(seg.) & Costo ILS & \%Gap \\ [0.5ex]
\hline
SCA3-0 & 640.55 & 5.43 & 
643.35 & 5.23 & \bf{635.62} & 
0.78\\SCA3-1 & 707.07 & 5.56 & 
709.41 & 4.61 & \bf{697.84} & 
1.32\\SCA3-2 & 661.13 & 5.54 & 
663.98 & 4.70 & \bf{659.34} & 
0.27\\SCA3-3 & 680.60 & 4.60 & 
686.99 & 4.58 & \bf{680.04} & 
0.08\\SCA3-4 & \bf{690.50} & 4.88 & 
710.73 & 4.95 & 690.50 & 0.00\\
SCA3-5 & 665.04 & 5.46 & 
681.04 & 4.74 & \bf{659.90} & 
0.78\\SCA3-6 & 652.47 & 3.88 & 
654.36 & 4.50 & \bf{651.09} & 
0.21\\SCA3-7 & 671.77 & 4.16 & 
674.56 & 4.67 & \bf{659.17} & 
1.91\\SCA3-8 & 719.77 & 4.04 & 
726.91 & 4.68 & \bf{719.47} & 
0.04\\SCA3-9 & 684.44 & 4.60 & 
689.21 & 4.41 & \bf{681.00} & 
0.51\\SCA8-0 & 980.30 & 4.69 & 
1010.69 & 4.41 & \bf{961.50} & 
1.96\\SCA8-1 & 1073.16 & 4.22 & 
1084.70 & 4.16 & \bf{1049.65} & 
2.24\\SCA8-2 & 1067.85 & 3.36 & 
1082.73 & 3.70 & \bf{1039.64} & 
2.71\\SCA8-3 & 1016.06 & 4.00 & 
1030.65 & 3.73 & \bf{983.34} & 
3.33\\SCA8-4 & 1069.87 & 5.32 & 
1092.81 & 4.15 & \bf{1065.49} & 
0.41\\SCA8-5 & 1063.11 & 3.66 & 
1091.71 & 3.60 & \bf{1027.08} & 
3.51\\SCA8-6 & 993.53 & 3.98 & 
1005.38 & 4.04 & \bf{971.82} & 
2.23\\SCA8-7 & 1102.85 & 3.62 & 
1118.36 & 3.78 & \bf{1051.28} & 
4.91\\SCA8-8 & 1087.01 & 4.80 & 
1094.19 & 3.77 & \bf{1071.18} & 
1.48\\SCA8-9 & 1105.18 & 3.10 & 
1120.10 & 3.15 & \bf{1060.50} & 
4.21\\CON3-0 & 630.73 & 4.58 & 
640.84 & 5.03 & \bf{616.52} & 
2.30\\CON3-1 & 562.52 & 4.99 & 
565.22 & 5.41 & \bf{554.47} & 
1.45\\CON3-2 & 523.23 & 5.80 & 
525.95 & 4.71 & \bf{518.00} & 
1.01\\CON3-3 & 610.74 & 4.13 & 
618.97 & 4.52 & \bf{591.19} & 
3.31\\CON3-4 & 593.78 & 4.98 & 
595.92 & 4.80 & \bf{588.79} & 
0.85\\CON3-5 & 570.22 & 4.79 & 
572.23 & 4.56 & \bf{563.70} & 
1.16\\CON3-6 & 504.15 & 5.34 & 
510.82 & 5.10 & \bf{499.05} & 
1.02\\CON3-7 & 585.94 & 4.68 & 
593.67 & 4.45 & \bf{576.48} & 
1.64\\CON3-8 & 523.68 & 6.61 & 
530.25 & 5.36 & \bf{523.05} & 
0.12\\CON3-9 & 588.99 & 4.44 & 
595.07 & 4.49 & \bf{578.24} & 
1.86\\CON8-0 & 874.47 & 4.94 & 
900.56 & 4.11 & \bf{857.17} & 
2.02\\CON8-1 & 761.33 & 3.48 & 
774.79 & 3.70 & \bf{740.85} & 
2.76\\CON8-2 & 733.10 & 4.47 & 
738.47 & 3.75 & \bf{712.89} & 
2.83\\CON8-3 & 842.69 & 4.79 & 
847.35 & 4.43 & \bf{811.07} & 
3.90\\CON8-4 & 808.09 & 3.86 & 
818.74 & 3.77 & \bf{772.25} & 
4.64\\CON8-5 & 770.56 & 4.09 & 
789.18 & 3.92 & \bf{754.88} & 
2.08\\CON8-6 & 708.69 & 4.20 & 
711.39 & 4.29 & \bf{678.92} & 
4.38\\CON8-7 & 835.96 & 3.40 & 
846.83 & 4.03 & \bf{811.96} & 
2.96\\CON8-8 & 784.43 & 4.49 & 
806.84 & 3.68 & \bf{767.53} & 
2.20\\CON8-9 & 815.11 & 4.61 & 
829.53 & 4.19 & \bf{809.00} & 
0.76\\\bf{PROM.} & 
\bf{774.02} & \bf{4.54} & \bf{784.61} & \bf{4.35} & \bf{758.54} & \bf{1.90}\\[1ex]\hline
\end{tabular}
\label{table:nonlin}
\end{table} \clearpage
\begin{table}[ht]
\caption{Resultados de la ejecución de la metaheurística ILS, utilizando instancias de Dethloff con la configuración -n 25.0 -LS 60.0}
\centering
\small
\begin{tabular}{c c c c c c c}
\hline\hline
Instancia & Costo mínimo & Tiempo(seg.) & Costo promedio & Tiempo promedio(seg.) & Costo ILS & \%Gap \\ [0.5ex]
\hline
SCA3-0 & 640.55 & 5.06 & 
642.42 & 5.00 & \bf{635.62} & 
0.78\\SCA3-1 & \bf{697.84} & 6.75 & 
703.99 & 6.09 & 697.84 & 0.00\\
SCA3-2 & 664.18 & 4.83 & 
667.45 & 5.53 & \bf{659.34} & 
0.73\\SCA3-3 & 681.74 & 6.49 & 
688.12 & 6.00 & \bf{680.04} & 
0.25\\SCA3-4 & 692.57 & 5.18 & 
697.76 & 5.15 & \bf{690.50} & 
0.30\\SCA3-5 & 678.22 & 5.63 & 
679.89 & 5.78 & \bf{659.90} & 
2.78\\SCA3-6 & \bf{651.09} & 6.51 & 
657.82 & 5.66 & 651.09 & 0.00\\
SCA3-7 & 671.67 & 5.38 & 
673.50 & 5.26 & \bf{659.17} & 
1.90\\SCA3-8 & 724.29 & 5.23 & 
737.58 & 5.65 & \bf{719.47} & 
0.67\\SCA3-9 & \bf{681.00} & 5.22 & 
684.58 & 5.00 & 681.00 & 0.00\\
SCA8-0 & 1020.44 & 4.86 & 
1038.36 & 5.37 & \bf{961.50} & 
6.13\\SCA8-1 & 1071.23 & 4.20 & 
1086.69 & 4.45 & \bf{1049.65} & 
2.06\\SCA8-2 & 1062.04 & 3.49 & 
1065.39 & 3.88 & \bf{1039.64} & 
2.15\\SCA8-3 & 1010.76 & 5.34 & 
1030.97 & 5.03 & \bf{983.34} & 
2.79\\SCA8-4 & 1074.18 & 4.91 & 
1088.14 & 4.21 & \bf{1065.49} & 
0.82\\SCA8-5 & 1068.16 & 6.72 & 
1087.05 & 4.97 & \bf{1027.08} & 
4.00\\SCA8-6 & 992.65 & 5.05 & 
1006.57 & 5.25 & \bf{971.82} & 
2.14\\SCA8-7 & 1085.42 & 4.96 & 
1091.39 & 4.97 & \bf{1051.28} & 
3.25\\SCA8-8 & 1092.02 & 4.13 & 
1101.21 & 4.68 & \bf{1071.18} & 
1.95\\SCA8-9 & 1084.08 & 4.48 & 
1092.67 & 4.34 & \bf{1060.50} & 
2.22\\CON3-0 & 633.24 & 5.88 & 
635.48 & 6.09 & \bf{616.52} & 
2.71\\CON3-1 & 556.92 & 5.60 & 
562.29 & 5.99 & \bf{554.47} & 
0.44\\CON3-2 & 521.38 & 5.42 & 
527.18 & 5.72 & \bf{518.00} & 
0.65\\CON3-3 & 591.20 & 5.92 & 
592.36 & 5.86 & \bf{591.19} & 
0.00\\CON3-4 & 591.43 & 6.45 & 
596.05 & 5.59 & \bf{588.79} & 
0.45\\CON3-5 & 564.88 & 4.84 & 
572.09 & 5.50 & \bf{563.70} & 
0.21\\CON3-6 & 503.95 & 6.27 & 
515.74 & 5.24 & \bf{499.05} & 
0.98\\CON3-7 & 578.41 & 6.68 & 
589.28 & 6.76 & \bf{576.48} & 
0.33\\CON3-8 & 524.59 & 6.28 & 
530.68 & 5.95 & \bf{523.05} & 
0.29\\CON3-9 & 588.63 & 5.98 & 
590.36 & 5.66 & \bf{578.24} & 
1.80\\CON8-0 & 885.11 & 4.65 & 
906.38 & 5.03 & \bf{857.17} & 
3.26\\CON8-1 & 751.84 & 5.40 & 
761.27 & 6.37 & \bf{740.85} & 
1.48\\CON8-2 & 720.12 & 6.70 & 
728.41 & 5.30 & \bf{712.89} & 
1.01\\CON8-3 & 831.87 & 5.92 & 
845.35 & 4.86 & \bf{811.07} & 
2.56\\CON8-4 & 801.31 & 4.19 & 
816.39 & 4.47 & \bf{772.25} & 
3.76\\CON8-5 & 758.12 & 4.99 & 
773.88 & 4.99 & \bf{754.88} & 
0.43\\CON8-6 & 702.57 & 5.55 & 
711.82 & 5.56 & \bf{678.92} & 
3.48\\CON8-7 & 827.28 & 5.66 & 
844.78 & 5.17 & \bf{811.96} & 
1.89\\CON8-8 & 798.06 & 5.42 & 
803.16 & 4.81 & \bf{767.53} & 
3.98\\CON8-9 & 817.87 & 3.46 & 
832.90 & 4.33 & \bf{809.00} & 
1.10\\\bf{PROM.} & 
\bf{772.32} & \bf{5.39} & \bf{781.44} & \bf{5.29} & \bf{758.54} & \bf{1.64}\\[1ex]\hline
\end{tabular}
\label{table:nonlin}
\end{table} \clearpage
\begin{table}[ht]
\caption{Resultados de la ejecución de la metaheurística ILS, utilizando instancias de Dethloff con la configuración -n 25.0 -LS 70.0}
\centering
\small
\begin{tabular}{c c c c c c c}
\hline\hline
Instancia & Costo mínimo & Tiempo(seg.) & Costo promedio & Tiempo promedio(seg.) & Costo ILS & \%Gap \\ [0.5ex]
\hline
SCA3-0 & 640.55 & 6.50 & 
642.00 & 6.48 & \bf{635.62} & 
0.78\\SCA3-1 & 700.50 & 5.67 & 
701.34 & 6.39 & \bf{697.84} & 
0.38\\SCA3-2 & 668.28 & 6.17 & 
676.70 & 6.05 & \bf{659.34} & 
1.36\\SCA3-3 & 681.74 & 6.30 & 
684.20 & 6.52 & \bf{680.04} & 
0.25\\SCA3-4 & \bf{690.50} & 6.95 & 
691.53 & 6.39 & 690.50 & 0.00\\
SCA3-5 & 661.07 & 6.88 & 
676.80 & 6.29 & \bf{659.90} & 
0.18\\SCA3-6 & \bf{651.09} & 5.87 & 
652.81 & 6.37 & 651.09 & 0.00\\
SCA3-7 & 671.77 & 5.99 & 
672.83 & 6.12 & \bf{659.17} & 
1.91\\SCA3-8 & 727.64 & 6.96 & 
731.02 & 6.23 & \bf{719.47} & 
1.14\\SCA3-9 & \bf{681.00} & 6.69 & 
686.16 & 6.22 & 681.00 & 0.00\\
SCA8-0 & 1004.83 & 6.18 & 
1015.99 & 5.32 & \bf{961.50} & 
4.51\\SCA8-1 & 1066.73 & 5.42 & 
1079.99 & 5.66 & \bf{1049.65} & 
1.63\\SCA8-2 & 1062.47 & 6.14 & 
1072.18 & 5.41 & \bf{1039.64} & 
2.20\\SCA8-3 & 1011.09 & 5.51 & 
1022.09 & 4.94 & \bf{983.34} & 
2.82\\SCA8-4 & 1096.60 & 6.17 & 
1102.88 & 5.68 & \bf{1065.49} & 
2.92\\SCA8-5 & 1065.60 & 5.54 & 
1073.88 & 5.49 & \bf{1027.08} & 
3.75\\SCA8-6 & 993.19 & 5.53 & 
1001.61 & 5.12 & \bf{971.82} & 
2.20\\SCA8-7 & 1089.45 & 4.63 & 
1102.95 & 5.19 & \bf{1051.28} & 
3.63\\SCA8-8 & 1092.82 & 6.61 & 
1107.96 & 5.33 & \bf{1071.18} & 
2.02\\SCA8-9 & 1081.51 & 5.34 & 
1093.93 & 5.40 & \bf{1060.50} & 
1.98\\CON3-0 & 620.76 & 5.70 & 
631.13 & 5.95 & \bf{616.52} & 
0.69\\CON3-1 & 556.04 & 6.20 & 
557.55 & 6.08 & \bf{554.47} & 
0.28\\CON3-2 & 521.63 & 6.66 & 
528.12 & 6.38 & \bf{518.00} & 
0.70\\CON3-3 & 599.26 & 6.28 & 
604.55 & 6.48 & \bf{591.19} & 
1.37\\CON3-4 & 591.43 & 6.60 & 
597.55 & 6.65 & \bf{588.79} & 
0.45\\CON3-5 & 573.23 & 7.94 & 
576.64 & 6.24 & \bf{563.70} & 
1.69\\CON3-6 & 504.20 & 7.02 & 
506.30 & 6.50 & \bf{499.05} & 
1.03\\CON3-7 & 591.91 & 5.98 & 
599.38 & 6.80 & \bf{576.48} & 
2.68\\CON3-8 & 524.59 & 5.91 & 
534.28 & 6.33 & \bf{523.05} & 
0.29\\CON3-9 & 590.50 & 7.40 & 
592.07 & 6.83 & \bf{578.24} & 
2.12\\CON8-0 & 888.34 & 5.03 & 
900.55 & 5.09 & \bf{857.17} & 
3.64\\CON8-1 & 761.69 & 4.48 & 
774.60 & 4.75 & \bf{740.85} & 
2.81\\CON8-2 & 725.72 & 6.17 & 
734.64 & 5.49 & \bf{712.89} & 
1.80\\CON8-3 & 822.12 & 5.98 & 
837.72 & 6.23 & \bf{811.07} & 
1.36\\CON8-4 & 781.78 & 6.84 & 
811.40 & 5.36 & \bf{772.25} & 
1.23\\CON8-5 & 769.55 & 4.99 & 
781.85 & 5.52 & \bf{754.88} & 
1.94\\CON8-6 & 699.04 & 5.97 & 
704.47 & 5.50 & \bf{678.92} & 
2.96\\CON8-7 & 817.70 & 5.43 & 
827.51 & 6.01 & \bf{811.96} & 
0.71\\CON8-8 & 783.97 & 5.09 & 
799.11 & 5.12 & \bf{767.53} & 
2.14\\CON8-9 & 815.52 & 7.19 & 
835.05 & 5.58 & \bf{809.00} & 
0.81\\\bf{PROM.} & 
\bf{771.94} & \bf{6.10} & \bf{780.58} & \bf{5.89} & \bf{758.54} & \bf{1.61}\\[1ex]\hline
\end{tabular}
\label{table:nonlin}
\end{table} \clearpage
\begin{table}[ht]
\caption{Resultados de la ejecución de la metaheurística ILS, utilizando instancias de Dethloff con la configuración -n 25.0 -LS 80.0}
\centering
\small
\begin{tabular}{c c c c c c c}
\hline\hline
Instancia & Costo mínimo & Tiempo(seg.) & Costo promedio & Tiempo promedio(seg.) & Costo ILS & \%Gap \\ [0.5ex]
\hline
SCA3-0 & 640.55 & 8.06 & 
642.13 & 7.49 & \bf{635.62} & 
0.78\\SCA3-1 & \bf{697.84} & 6.63 & 
700.09 & 7.19 & 697.84 & 0.00\\
SCA3-2 & \bf{659.34} & 7.44 & 
665.46 & 7.26 & 659.34 & 0.00\\
SCA3-3 & \bf{680.04} & 7.46 & 
683.14 & 7.11 & 680.04 & 0.00\\
SCA3-4 & \bf{690.50} & 6.86 & 
691.87 & 7.46 & 690.50 & 0.00\\
SCA3-5 & 672.94 & 7.23 & 
678.00 & 7.11 & \bf{659.90} & 
1.98\\SCA3-6 & \bf{651.09} & 7.68 & 
655.40 & 7.56 & 651.09 & 0.00\\
SCA3-7 & 671.77 & 7.85 & 
673.17 & 6.93 & \bf{659.17} & 
1.91\\SCA3-8 & \bf{719.47} & 8.28 & 
725.71 & 6.75 & 719.47 & 0.00\\
SCA3-9 & \bf{681.00} & 6.98 & 
686.59 & 6.90 & 681.00 & 0.00\\
SCA8-0 & 998.31 & 5.97 & 
1020.21 & 5.70 & \bf{961.50} & 
3.83\\SCA8-1 & 1082.44 & 5.93 & 
1093.85 & 5.76 & \bf{1049.65} & 
3.12\\SCA8-2 & 1056.87 & 5.06 & 
1069.57 & 5.42 & \bf{1039.64} & 
1.66\\SCA8-3 & 1017.17 & 6.68 & 
1031.26 & 5.86 & \bf{983.34} & 
3.44\\SCA8-4 & 1069.87 & 5.14 & 
1078.14 & 5.88 & \bf{1065.49} & 
0.41\\SCA8-5 & 1073.57 & 7.71 & 
1081.99 & 6.50 & \bf{1027.08} & 
4.53\\SCA8-6 & 992.79 & 5.59 & 
1011.42 & 5.62 & \bf{971.82} & 
2.16\\SCA8-7 & 1067.11 & 5.17 & 
1090.26 & 5.34 & \bf{1051.28} & 
1.51\\SCA8-8 & 1085.93 & 6.23 & 
1095.58 & 6.00 & \bf{1071.18} & 
1.38\\SCA8-9 & 1083.62 & 6.98 & 
1103.38 & 5.88 & \bf{1060.50} & 
2.18\\CON3-0 & 619.09 & 6.09 & 
639.31 & 6.84 & \bf{616.52} & 
0.42\\CON3-1 & 560.75 & 7.82 & 
562.85 & 7.86 & \bf{554.47} & 
1.13\\CON3-2 & 521.38 & 8.26 & 
528.06 & 6.97 & \bf{518.00} & 
0.65\\CON3-3 & 591.48 & 7.66 & 
598.43 & 7.32 & \bf{591.19} & 
0.05\\CON3-4 & 591.43 & 7.80 & 
605.40 & 7.91 & \bf{588.79} & 
0.45\\CON3-5 & 567.94 & 6.57 & 
571.52 & 7.08 & \bf{563.70} & 
0.75\\CON3-6 & 502.16 & 6.99 & 
511.00 & 7.09 & \bf{499.05} & 
0.62\\CON3-7 & 586.01 & 6.44 & 
595.35 & 6.42 & \bf{576.48} & 
1.65\\CON3-8 & 523.14 & 7.63 & 
530.43 & 7.45 & \bf{523.05} & 
0.02\\CON3-9 & 588.99 & 6.81 & 
594.95 & 7.06 & \bf{578.24} & 
1.86\\CON8-0 & 871.92 & 7.28 & 
900.01 & 6.95 & \bf{857.17} & 
1.72\\CON8-1 & 754.51 & 5.98 & 
772.42 & 6.70 & \bf{740.85} & 
1.84\\CON8-2 & 716.89 & 6.10 & 
726.13 & 6.23 & \bf{712.89} & 
0.56\\CON8-3 & 823.68 & 5.92 & 
830.87 & 6.78 & \bf{811.07} & 
1.55\\CON8-4 & 799.14 & 6.19 & 
813.64 & 6.03 & \bf{772.25} & 
3.48\\CON8-5 & 763.13 & 6.07 & 
767.78 & 5.70 & \bf{754.88} & 
1.09\\CON8-6 & 696.73 & 6.39 & 
710.24 & 6.91 & \bf{678.92} & 
2.62\\CON8-7 & 828.76 & 5.33 & 
843.12 & 5.37 & \bf{811.96} & 
2.07\\CON8-8 & 783.75 & 5.43 & 
785.48 & 5.92 & \bf{767.53} & 
2.11\\CON8-9 & 817.79 & 8.02 & 
824.96 & 6.12 & \bf{809.00} & 
1.09\\\bf{PROM.} & 
\bf{770.02} & \bf{6.74} & \bf{779.73} & \bf{6.61} & \bf{758.54} & \bf{1.37}\\[1ex]\hline
\end{tabular}
\label{table:nonlin}
\end{table} \clearpage
\begin{table}[ht]
\caption{Resultados de la ejecución de la metaheurística ILS, utilizando instancias de Dethloff con la configuración -n 35.0 -LS 10.0}
\centering
\small
\begin{tabular}{c c c c c c c}
\hline\hline
Instancia & Costo mínimo & Tiempo(seg.) & Costo promedio & Tiempo promedio(seg.) & Costo ILS & \%Gap \\ [0.5ex]
\hline
SCA3-0 & 636.06 & 2.12 & 
642.69 & 2.43 & \bf{635.62} & 
0.07\\SCA3-1 & 701.53 & 2.16 & 
709.75 & 2.36 & \bf{697.84} & 
0.53\\SCA3-2 & 673.67 & 2.55 & 
675.62 & 2.13 & \bf{659.34} & 
2.17\\SCA3-3 & 688.71 & 2.21 & 
690.93 & 1.96 & \bf{680.04} & 
1.27\\SCA3-4 & \bf{690.50} & 1.83 & 
703.32 & 2.03 & 690.50 & 0.00\\
SCA3-5 & 673.39 & 2.16 & 
677.60 & 2.30 & \bf{659.90} & 
2.04\\SCA3-6 & 654.79 & 2.53 & 
657.17 & 2.22 & \bf{651.09} & 
0.57\\SCA3-7 & 671.77 & 2.46 & 
679.53 & 2.15 & \bf{659.17} & 
1.91\\SCA3-8 & 719.77 & 2.04 & 
737.71 & 2.09 & \bf{719.47} & 
0.04\\SCA3-9 & 689.95 & 2.14 & 
697.43 & 2.04 & \bf{681.00} & 
1.31\\SCA8-0 & 1012.98 & 2.30 & 
1026.73 & 2.10 & \bf{961.50} & 
5.35\\SCA8-1 & 1094.39 & 1.83 & 
1104.61 & 1.93 & \bf{1049.65} & 
4.26\\SCA8-2 & 1053.94 & 2.16 & 
1071.04 & 2.07 & \bf{1039.64} & 
1.38\\SCA8-3 & 1033.38 & 1.92 & 
1055.79 & 1.97 & \bf{983.34} & 
5.09\\SCA8-4 & 1103.30 & 2.11 & 
1128.46 & 2.08 & \bf{1065.49} & 
3.55\\SCA8-5 & 1078.87 & 1.82 & 
1090.97 & 2.09 & \bf{1027.08} & 
5.04\\SCA8-6 & 998.05 & 1.70 & 
1010.96 & 1.84 & \bf{971.82} & 
2.70\\SCA8-7 & 1096.98 & 1.99 & 
1131.58 & 1.94 & \bf{1051.28} & 
4.35\\SCA8-8 & 1094.90 & 2.57 & 
1101.80 & 2.15 & \bf{1071.18} & 
2.21\\SCA8-9 & 1067.42 & 1.86 & 
1103.45 & 1.83 & \bf{1060.50} & 
0.65\\CON3-0 & 643.80 & 2.44 & 
652.65 & 2.19 & \bf{616.52} & 
4.42\\CON3-1 & 560.75 & 2.33 & 
564.78 & 2.28 & \bf{554.47} & 
1.13\\CON3-2 & 521.38 & 2.25 & 
529.52 & 2.15 & \bf{518.00} & 
0.65\\CON3-3 & 592.41 & 2.02 & 
598.13 & 2.19 & \bf{591.19} & 
0.21\\CON3-4 & 595.00 & 1.97 & 
600.81 & 2.33 & \bf{588.79} & 
1.05\\CON3-5 & 572.58 & 2.42 & 
581.09 & 2.13 & \bf{563.70} & 
1.58\\CON3-6 & 502.16 & 2.27 & 
506.08 & 2.28 & \bf{499.05} & 
0.62\\CON3-7 & 593.17 & 2.18 & 
598.88 & 2.14 & \bf{576.48} & 
2.90\\CON3-8 & 523.14 & 2.35 & 
530.05 & 2.17 & \bf{523.05} & 
0.02\\CON3-9 & 591.79 & 1.65 & 
594.72 & 1.85 & \bf{578.24} & 
2.34\\CON8-0 & 886.10 & 2.23 & 
929.66 & 2.00 & \bf{857.17} & 
3.38\\CON8-1 & 764.60 & 2.10 & 
778.48 & 2.35 & \bf{740.85} & 
3.21\\CON8-2 & 731.83 & 2.22 & 
734.05 & 2.12 & \bf{712.89} & 
2.66\\CON8-3 & 830.02 & 2.12 & 
856.57 & 2.06 & \bf{811.07} & 
2.34\\CON8-4 & 827.79 & 1.96 & 
833.80 & 1.81 & \bf{772.25} & 
7.19\\CON8-5 & 797.48 & 2.57 & 
810.74 & 2.12 & \bf{754.88} & 
5.64\\CON8-6 & 689.56 & 1.63 & 
701.23 & 2.02 & \bf{678.92} & 
1.57\\CON8-7 & 824.73 & 2.08 & 
843.89 & 2.06 & \bf{811.96} & 
1.57\\CON8-8 & 779.68 & 2.34 & 
789.87 & 2.19 & \bf{767.53} & 
1.58\\CON8-9 & 835.03 & 2.16 & 
848.48 & 2.29 & \bf{809.00} & 
3.22\\\bf{PROM.} & 
\bf{777.43} & \bf{2.14} & \bf{789.52} & \bf{2.11} & \bf{758.54} & \bf{2.29}\\[1ex]\hline
\end{tabular}
\label{table:nonlin}
\end{table} \clearpage
\begin{table}[ht]
\caption{Resultados de la ejecución de la metaheurística ILS, utilizando instancias de Dethloff con la configuración -n 35.0 -LS 20.0}
\centering
\small
\begin{tabular}{c c c c c c c}
\hline\hline
Instancia & Costo mínimo & Tiempo(seg.) & Costo promedio & Tiempo promedio(seg.) & Costo ILS & \%Gap \\ [0.5ex]
\hline
SCA3-0 & 640.55 & 4.28 & 
641.85 & 3.72 & \bf{635.62} & 
0.78\\SCA3-1 & 710.84 & 2.78 & 
712.47 & 3.05 & \bf{697.84} & 
1.86\\SCA3-2 & 664.18 & 2.60 & 
666.53 & 3.10 & \bf{659.34} & 
0.73\\SCA3-3 & 680.60 & 3.80 & 
683.75 & 3.41 & \bf{680.04} & 
0.08\\SCA3-4 & \bf{690.50} & 3.53 & 
696.21 & 3.09 & 690.50 & 0.00\\
SCA3-5 & 672.94 & 3.30 & 
677.48 & 3.33 & \bf{659.90} & 
1.98\\SCA3-6 & 652.94 & 3.36 & 
661.55 & 2.94 & \bf{651.09} & 
0.28\\SCA3-7 & 669.89 & 2.96 & 
671.49 & 3.26 & \bf{659.17} & 
1.63\\SCA3-8 & 719.77 & 3.61 & 
728.55 & 3.31 & \bf{719.47} & 
0.04\\SCA3-9 & \bf{681.00} & 3.18 & 
689.09 & 3.25 & 681.00 & 0.00\\
SCA8-0 & 982.79 & 3.17 & 
1008.20 & 3.19 & \bf{961.50} & 
2.21\\SCA8-1 & 1075.20 & 3.80 & 
1087.64 & 2.74 & \bf{1049.65} & 
2.43\\SCA8-2 & 1064.72 & 2.62 & 
1076.79 & 3.04 & \bf{1039.64} & 
2.41\\SCA8-3 & 1014.10 & 2.98 & 
1047.93 & 2.99 & \bf{983.34} & 
3.13\\SCA8-4 & 1081.88 & 2.45 & 
1122.23 & 2.89 & \bf{1065.49} & 
1.54\\SCA8-5 & 1062.04 & 2.89 & 
1089.87 & 3.07 & \bf{1027.08} & 
3.40\\SCA8-6 & 1000.41 & 2.82 & 
1009.59 & 2.62 & \bf{971.82} & 
2.94\\SCA8-7 & 1094.76 & 2.69 & 
1110.06 & 2.72 & \bf{1051.28} & 
4.14\\SCA8-8 & 1096.75 & 3.06 & 
1114.24 & 2.98 & \bf{1071.18} & 
2.39\\SCA8-9 & 1072.10 & 2.35 & 
1080.63 & 2.39 & \bf{1060.50} & 
1.09\\CON3-0 & 633.24 & 3.82 & 
641.73 & 3.55 & \bf{616.52} & 
2.71\\CON3-1 & 564.15 & 3.57 & 
569.21 & 3.35 & \bf{554.47} & 
1.75\\CON3-2 & 524.89 & 3.18 & 
529.54 & 3.00 & \bf{518.00} & 
1.33\\CON3-3 & 591.48 & 3.48 & 
604.11 & 3.44 & \bf{591.19} & 
0.05\\CON3-4 & 591.43 & 4.11 & 
609.73 & 3.46 & \bf{588.79} & 
0.45\\CON3-5 & 568.76 & 4.01 & 
576.59 & 3.31 & \bf{563.70} & 
0.90\\CON3-6 & 503.97 & 3.59 & 
509.85 & 3.23 & \bf{499.05} & 
0.99\\CON3-7 & 578.41 & 3.52 & 
594.52 & 3.64 & \bf{576.48} & 
0.33\\CON3-8 & 524.59 & 4.45 & 
531.83 & 4.03 & \bf{523.05} & 
0.29\\CON3-9 & 588.99 & 3.30 & 
590.12 & 3.05 & \bf{578.24} & 
1.86\\CON8-0 & 884.71 & 2.53 & 
906.20 & 2.90 & \bf{857.17} & 
3.21\\CON8-1 & 764.29 & 2.82 & 
776.08 & 2.77 & \bf{740.85} & 
3.16\\CON8-2 & 717.31 & 3.02 & 
735.01 & 3.01 & \bf{712.89} & 
0.62\\CON8-3 & 846.12 & 3.34 & 
851.99 & 3.23 & \bf{811.07} & 
4.32\\CON8-4 & 810.87 & 3.30 & 
825.42 & 2.79 & \bf{772.25} & 
5.00\\CON8-5 & 777.01 & 2.96 & 
780.32 & 2.98 & \bf{754.88} & 
2.93\\CON8-6 & 700.00 & 3.34 & 
713.91 & 3.15 & \bf{678.92} & 
3.10\\CON8-7 & 829.57 & 3.16 & 
851.14 & 3.01 & \bf{811.96} & 
2.17\\CON8-8 & 802.55 & 2.73 & 
817.89 & 2.68 & \bf{767.53} & 
4.56\\CON8-9 & 836.08 & 3.66 & 
842.86 & 3.12 & \bf{809.00} & 
3.35\\\bf{PROM.} & 
\bf{774.16} & \bf{3.25} & \bf{785.85} & \bf{3.12} & \bf{758.54} & \bf{1.90}\\[1ex]\hline
\end{tabular}
\label{table:nonlin}
\end{table} \clearpage
\begin{table}[ht]
\caption{Resultados de la ejecución de la metaheurística ILS, utilizando instancias de Dethloff con la configuración -n 35.0 -LS 30.0}
\centering
\small
\begin{tabular}{c c c c c c c}
\hline\hline
Instancia & Costo mínimo & Tiempo(seg.) & Costo promedio & Tiempo promedio(seg.) & Costo ILS & \%Gap \\ [0.5ex]
\hline
SCA3-0 & 641.69 & 4.48 & 
642.84 & 4.71 & \bf{635.62} & 
0.95\\SCA3-1 & 701.53 & 4.31 & 
708.07 & 4.03 & \bf{697.84} & 
0.53\\SCA3-2 & 664.21 & 3.67 & 
677.52 & 3.87 & \bf{659.34} & 
0.74\\SCA3-3 & 680.60 & 4.96 & 
682.08 & 4.88 & \bf{680.04} & 
0.08\\SCA3-4 & \bf{690.50} & 4.37 & 
691.53 & 4.44 & 690.50 & 0.00\\
SCA3-5 & 665.04 & 4.31 & 
674.81 & 4.67 & \bf{659.90} & 
0.78\\SCA3-6 & 654.26 & 4.02 & 
661.83 & 3.99 & \bf{651.09} & 
0.49\\SCA3-7 & 669.89 & 4.04 & 
672.48 & 4.09 & \bf{659.17} & 
1.63\\SCA3-8 & 724.29 & 3.50 & 
725.83 & 4.54 & \bf{719.47} & 
0.67\\SCA3-9 & \bf{681.00} & 4.30 & 
687.38 & 4.63 & 681.00 & 0.00\\
SCA8-0 & 1014.79 & 4.10 & 
1020.77 & 4.06 & \bf{961.50} & 
5.54\\SCA8-1 & 1070.87 & 4.00 & 
1084.72 & 4.15 & \bf{1049.65} & 
2.02\\SCA8-2 & 1069.37 & 3.86 & 
1075.78 & 3.07 & \bf{1039.64} & 
2.86\\SCA8-3 & 1030.39 & 3.78 & 
1038.74 & 3.49 & \bf{983.34} & 
4.78\\SCA8-4 & 1074.81 & 3.32 & 
1107.37 & 3.52 & \bf{1065.49} & 
0.87\\SCA8-5 & 1066.75 & 3.45 & 
1079.35 & 3.29 & \bf{1027.08} & 
3.86\\SCA8-6 & 1007.01 & 3.60 & 
1023.67 & 3.75 & \bf{971.82} & 
3.62\\SCA8-7 & 1070.92 & 4.41 & 
1087.35 & 3.95 & \bf{1051.28} & 
1.87\\SCA8-8 & 1087.01 & 3.76 & 
1101.49 & 3.94 & \bf{1071.18} & 
1.48\\SCA8-9 & 1111.43 & 4.04 & 
1115.54 & 3.19 & \bf{1060.50} & 
4.80\\CON3-0 & 633.24 & 4.23 & 
640.71 & 4.35 & \bf{616.52} & 
2.71\\CON3-1 & 564.15 & 4.10 & 
566.60 & 4.53 & \bf{554.47} & 
1.75\\CON3-2 & 521.38 & 5.06 & 
527.30 & 4.25 & \bf{518.00} & 
0.65\\CON3-3 & \bf{591.19} & 4.66 & 
608.76 & 4.43 & 591.19 & 0.00\\
CON3-4 & 593.78 & 3.72 & 
599.93 & 4.40 & \bf{588.79} & 
0.85\\CON3-5 & 566.45 & 5.22 & 
572.87 & 4.89 & \bf{563.70} & 
0.49\\CON3-6 & 502.16 & 4.16 & 
507.56 & 4.36 & \bf{499.05} & 
0.62\\CON3-7 & 589.64 & 3.74 & 
600.75 & 4.08 & \bf{576.48} & 
2.28\\CON3-8 & 524.30 & 4.61 & 
530.03 & 4.69 & \bf{523.05} & 
0.24\\CON3-9 & 588.99 & 3.89 & 
588.99 & 4.11 & \bf{578.24} & 
1.86\\CON8-0 & 887.50 & 3.04 & 
917.39 & 3.68 & \bf{857.17} & 
3.54\\CON8-1 & 756.05 & 3.95 & 
762.53 & 4.47 & \bf{740.85} & 
2.05\\CON8-2 & 723.13 & 3.96 & 
732.35 & 3.70 & \bf{712.89} & 
1.44\\CON8-3 & 835.19 & 3.61 & 
852.04 & 3.99 & \bf{811.07} & 
2.97\\CON8-4 & 796.77 & 4.55 & 
806.62 & 4.16 & \bf{772.25} & 
3.18\\CON8-5 & 771.14 & 4.10 & 
791.63 & 3.56 & \bf{754.88} & 
2.15\\CON8-6 & 700.85 & 4.89 & 
713.37 & 3.85 & \bf{678.92} & 
3.23\\CON8-7 & 826.97 & 3.64 & 
837.82 & 3.82 & \bf{811.96} & 
1.85\\CON8-8 & 784.28 & 3.53 & 
793.77 & 3.53 & \bf{767.53} & 
2.18\\CON8-9 & 820.15 & 3.48 & 
831.49 & 4.00 & \bf{809.00} & 
1.38\\\bf{PROM.} & 
\bf{773.84} & \bf{4.06} & \bf{783.54} & \bf{4.08} & \bf{758.54} & \bf{1.83}\\[1ex]\hline
\end{tabular}
\label{table:nonlin}
\end{table} \clearpage
\begin{table}[ht]
\caption{Resultados de la ejecución de la metaheurística ILS, utilizando instancias de Dethloff con la configuración -n 35.0 -LS 40.0}
\centering
\small
\begin{tabular}{c c c c c c c}
\hline\hline
Instancia & Costo mínimo & Tiempo(seg.) & Costo promedio & Tiempo promedio(seg.) & Costo ILS & \%Gap \\ [0.5ex]
\hline
SCA3-0 & 641.69 & 5.98 & 
642.97 & 5.66 & \bf{635.62} & 
0.95\\SCA3-1 & \bf{697.84} & 5.78 & 
707.11 & 5.64 & 697.84 & 0.00\\
SCA3-2 & 665.71 & 6.28 & 
668.85 & 5.82 & \bf{659.34} & 
0.97\\SCA3-3 & 680.60 & 6.13 & 
685.34 & 5.57 & \bf{680.04} & 
0.08\\SCA3-4 & 693.23 & 5.25 & 
698.26 & 5.36 & \bf{690.50} & 
0.40\\SCA3-5 & 673.56 & 4.72 & 
678.71 & 5.08 & \bf{659.90} & 
2.07\\SCA3-6 & 652.47 & 5.10 & 
662.56 & 5.05 & \bf{651.09} & 
0.21\\SCA3-7 & 671.67 & 4.86 & 
672.09 & 5.29 & \bf{659.17} & 
1.90\\SCA3-8 & 719.77 & 5.29 & 
728.96 & 5.16 & \bf{719.47} & 
0.04\\SCA3-9 & \bf{681.00} & 5.30 & 
693.65 & 4.93 & 681.00 & 0.00\\
SCA8-0 & 997.74 & 4.40 & 
1009.72 & 4.98 & \bf{961.50} & 
3.77\\SCA8-1 & 1087.28 & 5.21 & 
1093.03 & 4.67 & \bf{1049.65} & 
3.59\\SCA8-2 & 1058.01 & 4.92 & 
1064.41 & 4.66 & \bf{1039.64} & 
1.77\\SCA8-3 & 1023.10 & 5.03 & 
1035.03 & 4.36 & \bf{983.34} & 
4.04\\SCA8-4 & 1069.71 & 4.36 & 
1099.14 & 4.53 & \bf{1065.49} & 
0.40\\SCA8-5 & 1067.85 & 3.85 & 
1083.93 & 4.76 & \bf{1027.08} & 
3.97\\SCA8-6 & 980.91 & 4.36 & 
1019.32 & 4.32 & \bf{971.82} & 
0.94\\SCA8-7 & 1076.01 & 4.93 & 
1084.00 & 4.94 & \bf{1051.28} & 
2.35\\SCA8-8 & \bf{1071.18} & 5.45 & 
1095.18 & 4.41 & 1071.18 & 0.00\\
SCA8-9 & 1091.14 & 4.25 & 
1101.87 & 4.50 & \bf{1060.50} & 
2.89\\CON3-0 & 624.91 & 5.81 & 
635.34 & 5.96 & \bf{616.52} & 
1.36\\CON3-1 & 557.21 & 6.18 & 
562.16 & 5.94 & \bf{554.47} & 
0.49\\CON3-2 & 521.38 & 6.20 & 
521.84 & 5.76 & \bf{518.00} & 
0.65\\CON3-3 & 591.48 & 6.41 & 
598.62 & 5.74 & \bf{591.19} & 
0.05\\CON3-4 & 593.78 & 5.24 & 
605.18 & 6.01 & \bf{588.79} & 
0.85\\CON3-5 & 568.76 & 5.74 & 
577.39 & 5.67 & \bf{563.70} & 
0.90\\CON3-6 & 502.16 & 6.74 & 
509.95 & 5.93 & \bf{499.05} & 
0.62\\CON3-7 & 578.41 & 4.85 & 
585.90 & 5.22 & \bf{576.48} & 
0.33\\CON3-8 & 530.94 & 5.85 & 
536.99 & 5.64 & \bf{523.05} & 
1.51\\CON3-9 & 589.57 & 5.25 & 
590.10 & 5.71 & \bf{578.24} & 
1.96\\CON8-0 & 878.47 & 5.70 & 
911.68 & 4.83 & \bf{857.17} & 
2.48\\CON8-1 & 760.41 & 5.80 & 
775.98 & 4.86 & \bf{740.85} & 
2.64\\CON8-2 & 713.44 & 4.34 & 
723.46 & 4.65 & \bf{712.89} & 
0.08\\CON8-3 & 827.92 & 5.18 & 
846.09 & 4.53 & \bf{811.07} & 
2.08\\CON8-4 & 792.33 & 4.02 & 
813.07 & 4.49 & \bf{772.25} & 
2.60\\CON8-5 & 771.35 & 5.21 & 
780.96 & 5.18 & \bf{754.88} & 
2.18\\CON8-6 & 705.38 & 4.99 & 
708.33 & 5.41 & \bf{678.92} & 
3.90\\CON8-7 & 825.09 & 4.58 & 
843.46 & 4.70 & \bf{811.96} & 
1.62\\CON8-8 & 786.38 & 4.24 & 
796.77 & 4.78 & \bf{767.53} & 
2.46\\CON8-9 & 817.60 & 3.96 & 
831.19 & 4.73 & \bf{809.00} & 
1.06\\\bf{PROM.} & 
\bf{770.94} & \bf{5.19} & \bf{781.97} & \bf{5.14} & \bf{758.54} & \bf{1.50}\\[1ex]\hline
\end{tabular}
\label{table:nonlin}
\end{table} \clearpage
\begin{table}[ht]
\caption{Resultados de la ejecución de la metaheurística ILS, utilizando instancias de Dethloff con la configuración -n 35.0 -LS 50.0}
\centering
\small
\begin{tabular}{c c c c c c c}
\hline\hline
Instancia & Costo mínimo & Tiempo(seg.) & Costo promedio & Tiempo promedio(seg.) & Costo ILS & \%Gap \\ [0.5ex]
\hline
SCA3-0 & 636.06 & 7.92 & 
641.33 & 7.13 & \bf{635.62} & 
0.07\\SCA3-1 & 701.53 & 8.16 & 
708.67 & 6.96 & \bf{697.84} & 
0.53\\SCA3-2 & 664.18 & 6.87 & 
667.08 & 6.64 & \bf{659.34} & 
0.73\\SCA3-3 & 680.60 & 7.87 & 
682.97 & 7.40 & \bf{680.04} & 
0.08\\SCA3-4 & \bf{690.50} & 7.79 & 
696.44 & 7.20 & 690.50 & 0.00\\
SCA3-5 & 661.07 & 5.58 & 
672.96 & 6.05 & \bf{659.90} & 
0.18\\SCA3-6 & 652.47 & 6.73 & 
655.45 & 6.35 & \bf{651.09} & 
0.21\\SCA3-7 & 669.89 & 6.71 & 
674.64 & 6.47 & \bf{659.17} & 
1.63\\SCA3-8 & 723.99 & 6.31 & 
727.33 & 6.72 & \bf{719.47} & 
0.63\\SCA3-9 & \bf{681.00} & 7.48 & 
689.67 & 6.56 & 681.00 & 0.00\\
SCA8-0 & 990.48 & 5.54 & 
1002.84 & 5.67 & \bf{961.50} & 
3.01\\SCA8-1 & 1080.63 & 5.81 & 
1092.88 & 5.24 & \bf{1049.65} & 
2.95\\SCA8-2 & 1054.69 & 5.03 & 
1060.31 & 5.05 & \bf{1039.64} & 
1.45\\SCA8-3 & 1026.65 & 4.62 & 
1037.14 & 5.28 & \bf{983.34} & 
4.40\\SCA8-4 & 1077.80 & 5.34 & 
1102.56 & 5.79 & \bf{1065.49} & 
1.16\\SCA8-5 & 1085.33 & 6.14 & 
1087.41 & 5.96 & \bf{1027.08} & 
5.67\\SCA8-6 & 981.41 & 5.69 & 
1000.86 & 5.50 & \bf{971.82} & 
0.99\\SCA8-7 & 1093.27 & 5.04 & 
1111.16 & 5.25 & \bf{1051.28} & 
3.99\\SCA8-8 & 1097.82 & 6.17 & 
1106.07 & 5.42 & \bf{1071.18} & 
2.49\\SCA8-9 & 1084.38 & 5.55 & 
1091.41 & 6.14 & \bf{1060.50} & 
2.25\\CON3-0 & 633.24 & 7.22 & 
637.83 & 7.03 & \bf{616.52} & 
2.71\\CON3-1 & 560.75 & 5.50 & 
563.78 & 6.21 & \bf{554.47} & 
1.13\\CON3-2 & 521.38 & 5.66 & 
522.10 & 6.38 & \bf{518.00} & 
0.65\\CON3-3 & \bf{591.19} & 7.54 & 
594.55 & 7.00 & 591.19 & 0.00\\
CON3-4 & 593.78 & 6.96 & 
601.63 & 6.59 & \bf{588.79} & 
0.85\\CON3-5 & 574.44 & 6.09 & 
575.74 & 6.31 & \bf{563.70} & 
1.91\\CON3-6 & 505.82 & 5.68 & 
509.75 & 6.77 & \bf{499.05} & 
1.36\\CON3-7 & 577.54 & 6.31 & 
587.37 & 6.93 & \bf{576.48} & 
0.18\\CON3-8 & 529.65 & 6.95 & 
536.16 & 6.80 & \bf{523.05} & 
1.26\\CON3-9 & 588.99 & 6.56 & 
589.94 & 6.05 & \bf{578.24} & 
1.86\\CON8-0 & 872.97 & 6.40 & 
887.64 & 6.11 & \bf{857.17} & 
1.84\\CON8-1 & 752.61 & 5.08 & 
773.90 & 5.66 & \bf{740.85} & 
1.59\\CON8-2 & 717.68 & 7.47 & 
733.24 & 6.25 & \bf{712.89} & 
0.67\\CON8-3 & 824.75 & 5.73 & 
838.30 & 5.91 & \bf{811.07} & 
1.69\\CON8-4 & 794.34 & 5.84 & 
804.34 & 5.47 & \bf{772.25} & 
2.86\\CON8-5 & 776.74 & 5.22 & 
780.04 & 5.83 & \bf{754.88} & 
2.90\\CON8-6 & 706.28 & 6.46 & 
708.75 & 6.05 & \bf{678.92} & 
4.03\\CON8-7 & 815.32 & 6.24 & 
843.17 & 6.18 & \bf{811.96} & 
0.41\\CON8-8 & 783.44 & 5.82 & 
793.49 & 5.91 & \bf{767.53} & 
2.07\\CON8-9 & 820.69 & 6.66 & 
839.34 & 5.95 & \bf{809.00} & 
1.44\\\bf{PROM.} & 
\bf{771.88} & \bf{6.29} & \bf{780.76} & \bf{6.21} & \bf{758.54} & \bf{1.60}\\[1ex]\hline
\end{tabular}
\label{table:nonlin}
\end{table} \clearpage
\begin{table}[ht]
\caption{Resultados de la ejecución de la metaheurística ILS, utilizando instancias de Dethloff con la configuración -n 35.0 -LS 60.0}
\centering
\small
\begin{tabular}{c c c c c c c}
\hline\hline
Instancia & Costo mínimo & Tiempo(seg.) & Costo promedio & Tiempo promedio(seg.) & Costo ILS & \%Gap \\ [0.5ex]
\hline
SCA3-0 & 640.55 & 7.48 & 
640.84 & 7.98 & \bf{635.62} & 
0.78\\SCA3-1 & \bf{697.84} & 6.96 & 
699.68 & 7.23 & 697.84 & 0.00\\
SCA3-2 & \bf{659.34} & 8.20 & 
664.19 & 8.00 & 659.34 & 0.00\\
SCA3-3 & 681.74 & 9.01 & 
685.80 & 8.38 & \bf{680.04} & 
0.25\\SCA3-4 & \bf{690.50} & 7.96 & 
691.02 & 7.92 & 690.50 & 0.00\\
SCA3-5 & 679.90 & 6.91 & 
681.14 & 7.42 & \bf{659.90} & 
3.03\\SCA3-6 & \bf{651.09} & 9.50 & 
653.48 & 8.16 & 651.09 & 0.00\\
SCA3-7 & 669.89 & 7.72 & 
671.63 & 7.68 & \bf{659.17} & 
1.63\\SCA3-8 & 719.77 & 7.54 & 
728.64 & 7.63 & \bf{719.47} & 
0.04\\SCA3-9 & \bf{681.00} & 6.84 & 
684.26 & 7.25 & 681.00 & 0.00\\
SCA8-0 & 984.73 & 5.78 & 
1000.13 & 6.35 & \bf{961.50} & 
2.42\\SCA8-1 & 1067.20 & 5.24 & 
1088.59 & 5.96 & \bf{1049.65} & 
1.67\\SCA8-2 & 1066.86 & 6.08 & 
1073.86 & 6.26 & \bf{1039.64} & 
2.62\\SCA8-3 & 1011.26 & 6.61 & 
1030.15 & 6.62 & \bf{983.34} & 
2.84\\SCA8-4 & 1069.71 & 5.34 & 
1089.06 & 6.34 & \bf{1065.49} & 
0.40\\SCA8-5 & 1064.19 & 6.95 & 
1069.62 & 7.34 & \bf{1027.08} & 
3.61\\SCA8-6 & 1004.17 & 6.34 & 
1013.17 & 6.41 & \bf{971.82} & 
3.33\\SCA8-7 & 1070.67 & 6.42 & 
1085.00 & 6.20 & \bf{1051.28} & 
1.84\\SCA8-8 & 1095.32 & 5.43 & 
1095.47 & 5.62 & \bf{1071.18} & 
2.25\\SCA8-9 & 1081.45 & 5.40 & 
1090.67 & 6.00 & \bf{1060.50} & 
1.98\\CON3-0 & 630.73 & 8.50 & 
637.38 & 8.49 & \bf{616.52} & 
2.30\\CON3-1 & 558.09 & 6.29 & 
563.53 & 7.41 & \bf{554.47} & 
0.65\\CON3-2 & 522.51 & 10.01 & 
526.19 & 8.31 & \bf{518.00} & 
0.87\\CON3-3 & \bf{591.19} & 8.47 & 
591.50 & 7.89 & 591.19 & 0.00\\
CON3-4 & 591.43 & 8.71 & 
598.84 & 8.02 & \bf{588.79} & 
0.45\\CON3-5 & 564.88 & 6.64 & 
571.35 & 7.69 & \bf{563.70} & 
0.21\\CON3-6 & 502.85 & 7.96 & 
506.50 & 7.64 & \bf{499.05} & 
0.76\\CON3-7 & 578.41 & 7.07 & 
578.41 & 7.59 & \bf{576.48} & 
0.33\\CON3-8 & 527.82 & 7.84 & 
534.90 & 8.44 & \bf{523.05} & 
0.91\\CON3-9 & 588.99 & 8.69 & 
589.81 & 8.13 & \bf{578.24} & 
1.86\\CON8-0 & 884.18 & 6.34 & 
906.64 & 6.37 & \bf{857.17} & 
3.15\\CON8-1 & 756.16 & 6.87 & 
764.92 & 6.82 & \bf{740.85} & 
2.07\\CON8-2 & 720.09 & 6.78 & 
728.82 & 6.54 & \bf{712.89} & 
1.01\\CON8-3 & 827.30 & 7.82 & 
844.69 & 7.37 & \bf{811.07} & 
2.00\\CON8-4 & 791.89 & 7.14 & 
800.74 & 6.33 & \bf{772.25} & 
2.54\\CON8-5 & 771.58 & 8.37 & 
783.58 & 7.07 & \bf{754.88} & 
2.21\\CON8-6 & 702.65 & 7.04 & 
705.13 & 6.58 & \bf{678.92} & 
3.50\\CON8-7 & 821.28 & 6.44 & 
831.36 & 6.55 & \bf{811.96} & 
1.15\\CON8-8 & 786.71 & 6.22 & 
792.47 & 6.44 & \bf{767.53} & 
2.50\\CON8-9 & 818.44 & 5.83 & 
827.92 & 6.39 & \bf{809.00} & 
1.17\\\bf{PROM.} & 
\bf{770.61} & \bf{7.17} & \bf{778.03} & \bf{7.17} & \bf{758.54} & \bf{1.46}\\[1ex]\hline
\end{tabular}
\label{table:nonlin}
\end{table} \clearpage
\begin{table}[ht]
\caption{Resultados de la ejecución de la metaheurística ILS, utilizando instancias de Dethloff con la configuración -n 35.0 -LS 70.0}
\centering
\small
\begin{tabular}{c c c c c c c}
\hline\hline
Instancia & Costo mínimo & Tiempo(seg.) & Costo promedio & Tiempo promedio(seg.) & Costo ILS & \%Gap \\ [0.5ex]
\hline
SCA3-0 & 636.06 & 9.52 & 
640.69 & 9.36 & \bf{635.62} & 
0.07\\SCA3-1 & \bf{697.84} & 8.94 & 
702.05 & 8.96 & 697.84 & 0.00\\
SCA3-2 & 664.21 & 8.04 & 
668.45 & 8.73 & \bf{659.34} & 
0.74\\SCA3-3 & 681.31 & 8.90 & 
683.98 & 8.99 & \bf{680.04} & 
0.19\\SCA3-4 & \bf{690.50} & 8.64 & 
691.18 & 9.16 & 690.50 & 0.00\\
SCA3-5 & 665.04 & 7.95 & 
672.56 & 8.63 & \bf{659.90} & 
0.78\\SCA3-6 & 652.94 & 7.33 & 
655.47 & 8.88 & \bf{651.09} & 
0.28\\SCA3-7 & 671.67 & 8.16 & 
671.72 & 8.26 & \bf{659.17} & 
1.90\\SCA3-8 & 726.44 & 9.24 & 
729.02 & 8.84 & \bf{719.47} & 
0.97\\SCA3-9 & \bf{681.00} & 9.12 & 
685.71 & 8.53 & 681.00 & 0.00\\
SCA8-0 & 1001.55 & 8.00 & 
1012.47 & 7.54 & \bf{961.50} & 
4.17\\SCA8-1 & 1069.27 & 6.08 & 
1076.14 & 7.01 & \bf{1049.65} & 
1.87\\SCA8-2 & 1054.69 & 7.90 & 
1065.75 & 7.39 & \bf{1039.64} & 
1.45\\SCA8-3 & 1014.10 & 8.23 & 
1025.58 & 7.03 & \bf{983.34} & 
3.13\\SCA8-4 & 1067.55 & 6.56 & 
1088.81 & 7.19 & \bf{1065.49} & 
0.19\\SCA8-5 & 1072.42 & 9.39 & 
1085.94 & 8.71 & \bf{1027.08} & 
4.41\\SCA8-6 & 987.84 & 7.97 & 
998.42 & 7.18 & \bf{971.82} & 
1.65\\SCA8-7 & 1071.93 & 7.35 & 
1078.16 & 7.79 & \bf{1051.28} & 
1.96\\SCA8-8 & 1092.87 & 7.69 & 
1094.95 & 7.18 & \bf{1071.18} & 
2.02\\SCA8-9 & 1077.37 & 6.74 & 
1089.12 & 7.86 & \bf{1060.50} & 
1.59\\CON3-0 & 628.47 & 11.73 & 
630.59 & 9.75 & \bf{616.52} & 
1.94\\CON3-1 & 560.75 & 8.91 & 
564.36 & 9.40 & \bf{554.47} & 
1.13\\CON3-2 & 521.38 & 9.79 & 
523.16 & 8.91 & \bf{518.00} & 
0.65\\CON3-3 & 591.48 & 8.56 & 
600.26 & 8.99 & \bf{591.19} & 
0.05\\CON3-4 & 593.78 & 8.69 & 
594.86 & 9.04 & \bf{588.79} & 
0.85\\CON3-5 & 568.66 & 9.83 & 
571.82 & 8.91 & \bf{563.70} & 
0.88\\CON3-6 & 502.16 & 10.13 & 
508.56 & 9.45 & \bf{499.05} & 
0.62\\CON3-7 & 578.41 & 12.92 & 
586.58 & 9.84 & \bf{576.48} & 
0.33\\CON3-8 & 524.59 & 9.20 & 
529.36 & 9.89 & \bf{523.05} & 
0.29\\CON3-9 & 588.11 & 9.01 & 
589.51 & 8.58 & \bf{578.24} & 
1.71\\CON8-0 & 889.16 & 8.75 & 
904.27 & 7.38 & \bf{857.17} & 
3.73\\CON8-1 & 762.19 & 10.53 & 
764.17 & 9.01 & \bf{740.85} & 
2.88\\CON8-2 & 722.68 & 7.66 & 
732.87 & 8.17 & \bf{712.89} & 
1.37\\CON8-3 & 816.80 & 8.63 & 
834.08 & 7.41 & \bf{811.07} & 
0.71\\CON8-4 & 784.36 & 6.43 & 
806.51 & 7.43 & \bf{772.25} & 
1.57\\CON8-5 & 762.71 & 7.36 & 
777.91 & 7.04 & \bf{754.88} & 
1.04\\CON8-6 & 687.57 & 8.74 & 
698.66 & 8.15 & \bf{678.92} & 
1.27\\CON8-7 & 814.79 & 7.04 & 
830.91 & 7.38 & \bf{811.96} & 
0.35\\CON8-8 & 785.28 & 6.84 & 
794.44 & 6.79 & \bf{767.53} & 
2.31\\CON8-9 & 819.07 & 6.90 & 
822.63 & 7.78 & \bf{809.00} & 
1.24\\\bf{PROM.} & 
\bf{769.47} & \bf{8.48} & \bf{777.04} & \bf{8.31} & \bf{758.54} & \bf{1.31}\\[1ex]\hline
\end{tabular}
\label{table:nonlin}
\end{table} \clearpage
\begin{table}[ht]
\caption{Resultados de la ejecución de la metaheurística ILS, utilizando instancias de Dethloff con la configuración -n 35.0 -LS 80.0}
\centering
\small
\begin{tabular}{c c c c c c c}
\hline\hline
Instancia & Costo mínimo & Tiempo(seg.) & Costo promedio & Tiempo promedio(seg.) & Costo ILS & \%Gap \\ [0.5ex]
\hline
SCA3-0 & 636.06 & 9.54 & 
640.65 & 9.99 & \bf{635.62} & 
0.07\\SCA3-1 & 700.50 & 10.51 & 
705.15 & 9.73 & \bf{697.84} & 
0.38\\SCA3-2 & 664.21 & 9.97 & 
670.65 & 9.84 & \bf{659.34} & 
0.74\\SCA3-3 & 680.60 & 11.54 & 
681.70 & 11.56 & \bf{680.04} & 
0.08\\SCA3-4 & \bf{690.50} & 10.81 & 
690.50 & 10.26 & 690.50 & 0.00\\
SCA3-5 & 670.02 & 10.48 & 
679.16 & 10.13 & \bf{659.90} & 
1.53\\SCA3-6 & 653.81 & 10.09 & 
655.17 & 9.95 & \bf{651.09} & 
0.42\\SCA3-7 & 669.89 & 8.61 & 
671.27 & 9.35 & \bf{659.17} & 
1.63\\SCA3-8 & \bf{719.47} & 9.20 & 
724.37 & 9.64 & 719.47 & 0.00\\
SCA3-9 & \bf{681.00} & 9.06 & 
687.71 & 9.43 & 681.00 & 0.00\\
SCA8-0 & 977.93 & 9.21 & 
1000.08 & 8.83 & \bf{961.50} & 
1.71\\SCA8-1 & 1088.59 & 7.57 & 
1092.20 & 8.29 & \bf{1049.65} & 
3.71\\SCA8-2 & 1056.87 & 8.16 & 
1070.47 & 7.99 & \bf{1039.64} & 
1.66\\SCA8-3 & 1016.01 & 6.82 & 
1027.95 & 7.46 & \bf{983.34} & 
3.32\\SCA8-4 & 1075.71 & 8.92 & 
1094.47 & 8.79 & \bf{1065.49} & 
0.96\\SCA8-5 & 1060.32 & 7.98 & 
1074.32 & 8.25 & \bf{1027.08} & 
3.24\\SCA8-6 & 986.90 & 9.13 & 
998.02 & 8.68 & \bf{971.82} & 
1.55\\SCA8-7 & 1070.92 & 9.91 & 
1080.79 & 8.29 & \bf{1051.28} & 
1.87\\SCA8-8 & 1089.07 & 8.62 & 
1096.24 & 7.50 & \bf{1071.18} & 
1.67\\SCA8-9 & 1091.16 & 9.76 & 
1102.44 & 8.06 & \bf{1060.50} & 
2.89\\CON3-0 & 623.15 & 10.86 & 
632.64 & 10.83 & \bf{616.52} & 
1.08\\CON3-1 & 560.75 & 11.67 & 
562.40 & 10.80 & \bf{554.47} & 
1.13\\CON3-2 & 521.63 & 9.99 & 
524.77 & 9.33 & \bf{518.00} & 
0.70\\CON3-3 & \bf{591.19} & 10.43 & 
593.82 & 9.55 & 591.19 & 0.00\\
CON3-4 & 591.43 & 9.88 & 
598.93 & 9.91 & \bf{588.79} & 
0.45\\CON3-5 & 564.88 & 10.15 & 
569.86 & 10.39 & \bf{563.70} & 
0.21\\CON3-6 & 502.16 & 10.78 & 
505.96 & 10.27 & \bf{499.05} & 
0.62\\CON3-7 & 577.68 & 9.94 & 
591.69 & 9.97 & \bf{576.48} & 
0.21\\CON3-8 & 524.59 & 10.16 & 
529.58 & 10.39 & \bf{523.05} & 
0.29\\CON3-9 & 588.38 & 10.60 & 
589.99 & 9.87 & \bf{578.24} & 
1.75\\CON8-0 & 863.53 & 8.25 & 
887.43 & 8.83 & \bf{857.17} & 
0.74\\CON8-1 & 769.67 & 6.81 & 
780.66 & 7.75 & \bf{740.85} & 
3.89\\CON8-2 & 725.92 & 8.92 & 
730.16 & 8.84 & \bf{712.89} & 
1.83\\CON8-3 & 827.60 & 8.65 & 
833.90 & 8.89 & \bf{811.07} & 
2.04\\CON8-4 & 794.38 & 9.82 & 
812.92 & 8.46 & \bf{772.25} & 
2.87\\CON8-5 & 758.84 & 7.50 & 
768.60 & 7.58 & \bf{754.88} & 
0.52\\CON8-6 & 696.89 & 8.10 & 
706.53 & 8.64 & \bf{678.92} & 
2.65\\CON8-7 & 816.00 & 7.76 & 
823.25 & 8.09 & \bf{811.96} & 
0.50\\CON8-8 & 785.63 & 8.78 & 
793.87 & 8.43 & \bf{767.53} & 
2.36\\CON8-9 & 816.00 & 11.15 & 
831.96 & 9.44 & \bf{809.00} & 
0.87\\\bf{PROM.} & 
\bf{769.50} & \bf{9.40} & \bf{777.81} & \bf{9.21} & \bf{758.54} & \bf{1.30}\\[1ex]\hline
\end{tabular}
\label{table:nonlin}
\end{table} \clearpage
\begin{table}[ht]
\caption{Resultados de la ejecución de la metaheurística ILS, utilizando instancias de Dethloff con la configuración -n 45.0 -LS 10.0}
\centering
\small
\begin{tabular}{c c c c c c c}
\hline\hline
Instancia & Costo mínimo & Tiempo(seg.) & Costo promedio & Tiempo promedio(seg.) & Costo ILS & \%Gap \\ [0.5ex]
\hline
SCA3-0 & 640.55 & 2.74 & 
641.49 & 2.85 & \bf{635.62} & 
0.78\\SCA3-1 & \bf{697.84} & 2.92 & 
703.38 & 3.02 & 697.84 & 0.00\\
SCA3-2 & 661.13 & 2.53 & 
667.91 & 2.42 & \bf{659.34} & 
0.27\\SCA3-3 & 681.74 & 2.63 & 
685.85 & 2.93 & \bf{680.04} & 
0.25\\SCA3-4 & 693.23 & 3.16 & 
707.32 & 2.87 & \bf{690.50} & 
0.40\\SCA3-5 & 681.81 & 3.05 & 
683.70 & 3.06 & \bf{659.90} & 
3.32\\SCA3-6 & 653.83 & 2.81 & 
664.35 & 2.66 & \bf{651.09} & 
0.42\\SCA3-7 & 671.67 & 2.77 & 
673.86 & 2.67 & \bf{659.17} & 
1.90\\SCA3-8 & 726.44 & 2.80 & 
731.76 & 2.88 & \bf{719.47} & 
0.97\\SCA3-9 & \bf{681.00} & 2.61 & 
689.94 & 2.45 & 681.00 & 0.00\\
SCA8-0 & 998.24 & 3.96 & 
1032.58 & 2.91 & \bf{961.50} & 
3.82\\SCA8-1 & 1080.99 & 2.37 & 
1097.00 & 2.27 & \bf{1049.65} & 
2.99\\SCA8-2 & 1054.78 & 2.13 & 
1075.96 & 2.40 & \bf{1039.64} & 
1.46\\SCA8-3 & 1015.31 & 3.03 & 
1025.31 & 2.82 & \bf{983.34} & 
3.25\\SCA8-4 & 1076.42 & 2.58 & 
1093.86 & 2.60 & \bf{1065.49} & 
1.03\\SCA8-5 & 1061.75 & 2.03 & 
1087.66 & 2.69 & \bf{1027.08} & 
3.38\\SCA8-6 & 1003.41 & 2.18 & 
1014.68 & 2.63 & \bf{971.82} & 
3.25\\SCA8-7 & 1085.45 & 2.09 & 
1103.55 & 2.40 & \bf{1051.28} & 
3.25\\SCA8-8 & 1088.19 & 2.79 & 
1099.10 & 2.61 & \bf{1071.18} & 
1.59\\SCA8-9 & 1106.11 & 1.96 & 
1115.38 & 2.50 & \bf{1060.50} & 
4.30\\CON3-0 & 633.24 & 3.59 & 
642.84 & 2.83 & \bf{616.52} & 
2.71\\CON3-1 & 561.87 & 2.86 & 
564.63 & 2.76 & \bf{554.47} & 
1.33\\CON3-2 & 526.95 & 2.42 & 
531.67 & 2.58 & \bf{518.00} & 
1.73\\CON3-3 & 592.43 & 2.59 & 
599.60 & 2.90 & \bf{591.19} & 
0.21\\CON3-4 & 596.43 & 2.74 & 
611.27 & 2.68 & \bf{588.79} & 
1.30\\CON3-5 & 569.88 & 2.08 & 
578.08 & 2.79 & \bf{563.70} & 
1.10\\CON3-6 & 502.64 & 4.04 & 
509.89 & 3.26 & \bf{499.05} & 
0.72\\CON3-7 & 591.04 & 2.78 & 
594.70 & 2.50 & \bf{576.48} & 
2.53\\CON3-8 & 526.59 & 3.10 & 
530.47 & 2.94 & \bf{523.05} & 
0.68\\CON3-9 & 588.40 & 2.69 & 
590.47 & 2.77 & \bf{578.24} & 
1.76\\CON8-0 & 881.36 & 3.05 & 
891.89 & 2.65 & \bf{857.17} & 
2.82\\CON8-1 & 745.98 & 2.66 & 
767.87 & 2.58 & \bf{740.85} & 
0.69\\CON8-2 & 723.29 & 3.24 & 
735.17 & 2.94 & \bf{712.89} & 
1.46\\CON8-3 & 824.87 & 3.06 & 
847.92 & 2.80 & \bf{811.07} & 
1.70\\CON8-4 & 807.10 & 2.73 & 
815.39 & 2.87 & \bf{772.25} & 
4.51\\CON8-5 & 779.59 & 3.24 & 
788.66 & 3.15 & \bf{754.88} & 
3.27\\CON8-6 & 706.94 & 2.98 & 
709.70 & 2.83 & \bf{678.92} & 
4.13\\CON8-7 & 820.92 & 2.92 & 
842.38 & 2.60 & \bf{811.96} & 
1.10\\CON8-8 & 799.89 & 2.56 & 
804.71 & 2.38 & \bf{767.53} & 
4.22\\CON8-9 & 830.43 & 3.35 & 
852.90 & 2.69 & \bf{809.00} & 
2.65\\\bf{PROM.} & 
\bf{774.24} & \bf{2.80} & \bf{785.12} & \bf{2.73} & \bf{758.54} & \bf{1.93}\\[1ex]\hline
\end{tabular}
\label{table:nonlin}
\end{table} \clearpage
\begin{table}[ht]
\caption{Resultados de la ejecución de la metaheurística ILS, utilizando instancias de Dethloff con la configuración -n 45.0 -LS 20.0}
\centering
\small
\begin{tabular}{c c c c c c c}
\hline\hline
Instancia & Costo mínimo & Tiempo(seg.) & Costo promedio & Tiempo promedio(seg.) & Costo ILS & \%Gap \\ [0.5ex]
\hline
SCA3-0 & 640.55 & 4.63 & 
641.12 & 4.44 & \bf{635.62} & 
0.78\\SCA3-1 & \bf{697.84} & 4.27 & 
704.32 & 4.33 & 697.84 & 0.00\\
SCA3-2 & 664.92 & 4.30 & 
669.83 & 4.22 & \bf{659.34} & 
0.85\\SCA3-3 & 681.74 & 4.04 & 
682.10 & 4.32 & \bf{680.04} & 
0.25\\SCA3-4 & 692.57 & 4.48 & 
700.50 & 4.32 & \bf{690.50} & 
0.30\\SCA3-5 & 675.81 & 4.89 & 
681.57 & 4.40 & \bf{659.90} & 
2.41\\SCA3-6 & 655.41 & 4.23 & 
658.48 & 3.89 & \bf{651.09} & 
0.66\\SCA3-7 & 669.89 & 4.10 & 
670.80 & 4.21 & \bf{659.17} & 
1.63\\SCA3-8 & \bf{719.47} & 4.47 & 
727.65 & 4.35 & 719.47 & 0.00\\
SCA3-9 & 685.19 & 3.66 & 
691.86 & 3.56 & \bf{681.00} & 
0.62\\SCA8-0 & 1004.86 & 4.76 & 
1018.33 & 4.57 & \bf{961.50} & 
4.51\\SCA8-1 & 1083.79 & 3.74 & 
1091.69 & 3.99 & \bf{1049.65} & 
3.25\\SCA8-2 & 1054.47 & 4.10 & 
1072.46 & 3.54 & \bf{1039.64} & 
1.43\\SCA8-3 & 1035.16 & 4.19 & 
1042.10 & 3.87 & \bf{983.34} & 
5.27\\SCA8-4 & 1095.53 & 3.31 & 
1111.81 & 3.31 & \bf{1065.49} & 
2.82\\SCA8-5 & 1070.69 & 4.37 & 
1076.72 & 3.62 & \bf{1027.08} & 
4.25\\SCA8-6 & 997.87 & 5.06 & 
1011.21 & 3.69 & \bf{971.82} & 
2.68\\SCA8-7 & 1076.22 & 3.93 & 
1092.18 & 3.65 & \bf{1051.28} & 
2.37\\SCA8-8 & 1090.67 & 3.17 & 
1094.86 & 3.21 & \bf{1071.18} & 
1.82\\SCA8-9 & 1105.14 & 3.29 & 
1111.71 & 3.50 & \bf{1060.50} & 
4.21\\CON3-0 & 619.09 & 3.60 & 
629.91 & 4.17 & \bf{616.52} & 
0.42\\CON3-1 & 560.75 & 3.86 & 
565.05 & 4.30 & \bf{554.47} & 
1.13\\CON3-2 & 521.38 & 4.45 & 
523.49 & 4.25 & \bf{518.00} & 
0.65\\CON3-3 & 591.20 & 5.19 & 
599.75 & 5.05 & \bf{591.19} & 
0.00\\CON3-4 & 593.78 & 4.64 & 
602.95 & 4.52 & \bf{588.79} & 
0.85\\CON3-5 & 572.42 & 3.24 & 
575.76 & 3.89 & \bf{563.70} & 
1.55\\CON3-6 & 502.75 & 4.29 & 
507.19 & 4.38 & \bf{499.05} & 
0.74\\CON3-7 & 586.84 & 4.18 & 
591.97 & 3.94 & \bf{576.48} & 
1.80\\CON3-8 & 527.82 & 4.41 & 
532.93 & 4.67 & \bf{523.05} & 
0.91\\CON3-9 & 587.97 & 3.64 & 
592.50 & 4.20 & \bf{578.24} & 
1.68\\CON8-0 & 900.47 & 4.35 & 
916.84 & 3.71 & \bf{857.17} & 
5.05\\CON8-1 & 755.21 & 4.94 & 
769.86 & 3.89 & \bf{740.85} & 
1.94\\CON8-2 & 727.11 & 4.56 & 
734.75 & 4.27 & \bf{712.89} & 
1.99\\CON8-3 & 836.92 & 3.62 & 
845.15 & 3.93 & \bf{811.07} & 
3.19\\CON8-4 & 817.14 & 3.59 & 
820.09 & 3.78 & \bf{772.25} & 
5.81\\CON8-5 & 768.42 & 4.40 & 
776.64 & 4.45 & \bf{754.88} & 
1.79\\CON8-6 & 699.47 & 3.41 & 
709.08 & 4.07 & \bf{678.92} & 
3.03\\CON8-7 & 815.43 & 4.05 & 
836.11 & 3.90 & \bf{811.96} & 
0.43\\CON8-8 & 798.02 & 4.59 & 
805.63 & 4.29 & \bf{767.53} & 
3.97\\CON8-9 & 836.38 & 3.91 & 
842.14 & 3.84 & \bf{809.00} & 
3.38\\\bf{PROM.} & 
\bf{775.41} & \bf{4.15} & \bf{783.23} & \bf{4.06} & \bf{758.54} & \bf{2.01}\\[1ex]\hline
\end{tabular}
\label{table:nonlin}
\end{table} \clearpage
\begin{table}[ht]
\caption{Resultados de la ejecución de la metaheurística ILS, utilizando instancias de Dethloff con la configuración -n 45.0 -LS 30.0}
\centering
\small
\begin{tabular}{c c c c c c c}
\hline\hline
Instancia & Costo mínimo & Tiempo(seg.) & Costo promedio & Tiempo promedio(seg.) & Costo ILS & \%Gap \\ [0.5ex]
\hline
SCA3-0 & 640.55 & 6.32 & 
642.00 & 5.85 & \bf{635.62} & 
0.78\\SCA3-1 & 706.90 & 6.04 & 
709.27 & 6.24 & \bf{697.84} & 
1.30\\SCA3-2 & \bf{659.34} & 4.90 & 
668.26 & 5.23 & 659.34 & 0.00\\
SCA3-3 & 680.60 & 5.86 & 
683.00 & 5.25 & \bf{680.04} & 
0.08\\SCA3-4 & \bf{690.50} & 6.91 & 
692.22 & 5.92 & 690.50 & 0.00\\
SCA3-5 & \bf{659.90} & 5.84 & 
674.07 & 6.00 & 659.90 & 0.00\\
SCA3-6 & \bf{651.09} & 6.22 & 
655.02 & 5.96 & 651.09 & 0.00\\
SCA3-7 & 669.89 & 6.12 & 
672.57 & 5.55 & \bf{659.17} & 
1.63\\SCA3-8 & 719.77 & 7.00 & 
728.07 & 5.63 & \bf{719.47} & 
0.04\\SCA3-9 & 683.57 & 5.40 & 
689.65 & 5.19 & \bf{681.00} & 
0.38\\SCA8-0 & 996.34 & 5.07 & 
1012.96 & 5.33 & \bf{961.50} & 
3.62\\SCA8-1 & 1067.83 & 3.85 & 
1083.89 & 4.51 & \bf{1049.65} & 
1.73\\SCA8-2 & 1071.06 & 4.77 & 
1078.68 & 4.65 & \bf{1039.64} & 
3.02\\SCA8-3 & 1015.61 & 4.16 & 
1023.27 & 5.03 & \bf{983.34} & 
3.28\\SCA8-4 & 1067.55 & 5.41 & 
1097.38 & 5.37 & \bf{1065.49} & 
0.19\\SCA8-5 & 1070.99 & 4.90 & 
1081.51 & 5.18 & \bf{1027.08} & 
4.28\\SCA8-6 & 1003.87 & 4.96 & 
1011.25 & 4.58 & \bf{971.82} & 
3.30\\SCA8-7 & 1078.87 & 5.39 & 
1093.10 & 5.17 & \bf{1051.28} & 
2.62\\SCA8-8 & 1088.65 & 5.13 & 
1109.46 & 4.54 & \bf{1071.18} & 
1.63\\SCA8-9 & 1069.22 & 4.32 & 
1097.94 & 4.60 & \bf{1060.50} & 
0.82\\CON3-0 & 632.14 & 5.82 & 
642.20 & 5.19 & \bf{616.52} & 
2.53\\CON3-1 & 560.75 & 5.94 & 
566.82 & 5.85 & \bf{554.47} & 
1.13\\CON3-2 & 521.38 & 6.35 & 
527.11 & 5.92 & \bf{518.00} & 
0.65\\CON3-3 & 592.41 & 5.30 & 
612.41 & 5.22 & \bf{591.19} & 
0.21\\CON3-4 & 593.78 & 6.13 & 
597.65 & 6.06 & \bf{588.79} & 
0.85\\CON3-5 & 568.76 & 5.32 & 
577.52 & 5.37 & \bf{563.70} & 
0.90\\CON3-6 & 504.28 & 5.62 & 
507.04 & 5.55 & \bf{499.05} & 
1.05\\CON3-7 & 590.75 & 6.03 & 
598.84 & 5.72 & \bf{576.48} & 
2.48\\CON3-8 & \bf{523.05} & 6.33 & 
523.85 & 6.25 & 523.05 & 0.00\\
CON3-9 & 578.25 & 6.95 & 
587.65 & 6.08 & \bf{578.24} & 
0.00\\CON8-0 & 905.34 & 4.61 & 
913.34 & 4.46 & \bf{857.17} & 
5.62\\CON8-1 & 754.22 & 5.31 & 
766.27 & 5.58 & \bf{740.85} & 
1.80\\CON8-2 & 721.13 & 4.89 & 
729.39 & 5.52 & \bf{712.89} & 
1.16\\CON8-3 & 824.90 & 5.68 & 
836.35 & 4.86 & \bf{811.07} & 
1.71\\CON8-4 & 797.82 & 5.20 & 
816.64 & 5.07 & \bf{772.25} & 
3.31\\CON8-5 & 768.50 & 5.80 & 
775.54 & 5.33 & \bf{754.88} & 
1.80\\CON8-6 & 693.57 & 4.43 & 
705.75 & 4.86 & \bf{678.92} & 
2.16\\CON8-7 & 816.07 & 4.45 & 
820.64 & 4.73 & \bf{811.96} & 
0.51\\CON8-8 & 792.59 & 5.40 & 
798.29 & 5.64 & \bf{767.53} & 
3.27\\CON8-9 & 816.02 & 4.59 & 
836.36 & 4.65 & \bf{809.00} & 
0.87\\\bf{PROM.} & 
\bf{771.20} & \bf{5.47} & \bf{781.08} & \bf{5.34} & \bf{758.54} & \bf{1.52}\\[1ex]\hline
\end{tabular}
\label{table:nonlin}
\end{table} \clearpage
\begin{table}[ht]
\caption{Resultados de la ejecución de la metaheurística ILS, utilizando instancias de Dethloff con la configuración -n 45.0 -LS 40.0}
\centering
\small
\begin{tabular}{c c c c c c c}
\hline\hline
Instancia & Costo mínimo & Tiempo(seg.) & Costo promedio & Tiempo promedio(seg.) & Costo ILS & \%Gap \\ [0.5ex]
\hline
SCA3-0 & 640.55 & 7.36 & 
641.12 & 7.80 & \bf{635.62} & 
0.78\\SCA3-1 & 701.53 & 7.38 & 
705.91 & 7.37 & \bf{697.84} & 
0.53\\SCA3-2 & 666.01 & 6.82 & 
667.16 & 6.91 & \bf{659.34} & 
1.01\\SCA3-3 & \bf{680.04} & 7.46 & 
683.00 & 7.24 & 680.04 & 0.00\\
SCA3-4 & \bf{690.50} & 7.84 & 
691.02 & 7.13 & 690.50 & 0.00\\
SCA3-5 & 665.04 & 7.35 & 
674.42 & 8.64 & \bf{659.90} & 
0.78\\SCA3-6 & \bf{651.09} & 6.97 & 
652.88 & 7.00 & 651.09 & 0.00\\
SCA3-7 & 671.67 & 6.13 & 
674.83 & 6.84 & \bf{659.17} & 
1.90\\SCA3-8 & \bf{719.47} & 7.30 & 
721.55 & 6.93 & 719.47 & 0.00\\
SCA3-9 & \bf{681.00} & 6.10 & 
684.50 & 6.55 & 681.00 & 0.00\\
SCA8-0 & 995.57 & 6.04 & 
1015.42 & 6.37 & \bf{961.50} & 
3.54\\SCA8-1 & 1062.35 & 5.55 & 
1069.39 & 5.63 & \bf{1049.65} & 
1.21\\SCA8-2 & 1053.51 & 5.58 & 
1063.59 & 5.92 & \bf{1039.64} & 
1.33\\SCA8-3 & 1024.57 & 6.36 & 
1030.34 & 6.07 & \bf{983.34} & 
4.19\\SCA8-4 & 1081.35 & 6.17 & 
1087.51 & 5.91 & \bf{1065.49} & 
1.49\\SCA8-5 & 1062.34 & 5.61 & 
1083.94 & 5.94 & \bf{1027.08} & 
3.43\\SCA8-6 & 991.57 & 6.08 & 
1001.83 & 5.75 & \bf{971.82} & 
2.03\\SCA8-7 & 1067.11 & 7.13 & 
1077.29 & 6.25 & \bf{1051.28} & 
1.51\\SCA8-8 & 1095.14 & 5.60 & 
1105.64 & 5.35 & \bf{1071.18} & 
2.24\\SCA8-9 & 1078.23 & 6.22 & 
1089.07 & 6.20 & \bf{1060.50} & 
1.67\\CON3-0 & 629.19 & 8.11 & 
632.49 & 7.00 & \bf{616.52} & 
2.06\\CON3-1 & 562.83 & 8.01 & 
565.63 & 7.43 & \bf{554.47} & 
1.51\\CON3-2 & 521.38 & 5.61 & 
524.26 & 6.82 & \bf{518.00} & 
0.65\\CON3-3 & 591.20 & 10.47 & 
595.37 & 7.87 & \bf{591.19} & 
0.00\\CON3-4 & 593.78 & 7.67 & 
595.46 & 7.00 & \bf{588.79} & 
0.85\\CON3-5 & 569.88 & 6.62 & 
571.58 & 6.80 & \bf{563.70} & 
1.10\\CON3-6 & 502.16 & 6.72 & 
504.56 & 6.78 & \bf{499.05} & 
0.62\\CON3-7 & \bf{576.48} & 6.60 & 
588.74 & 6.64 & 576.48 & 0.00\\
CON3-8 & 524.59 & 7.54 & 
530.33 & 7.45 & \bf{523.05} & 
0.29\\CON3-9 & 588.18 & 6.47 & 
590.09 & 6.42 & \bf{578.24} & 
1.72\\CON8-0 & 873.62 & 6.84 & 
908.50 & 6.25 & \bf{857.17} & 
1.92\\CON8-1 & 759.18 & 7.29 & 
770.33 & 6.72 & \bf{740.85} & 
2.47\\CON8-2 & 727.20 & 7.57 & 
733.46 & 6.78 & \bf{712.89} & 
2.01\\CON8-3 & 836.46 & 6.38 & 
842.99 & 6.24 & \bf{811.07} & 
3.13\\CON8-4 & 792.76 & 5.14 & 
810.67 & 5.96 & \bf{772.25} & 
2.66\\CON8-5 & 769.55 & 5.44 & 
776.99 & 6.70 & \bf{754.88} & 
1.94\\CON8-6 & 699.30 & 8.09 & 
703.11 & 6.44 & \bf{678.92} & 
3.00\\CON8-7 & 816.07 & 6.59 & 
838.78 & 5.82 & \bf{811.96} & 
0.51\\CON8-8 & 791.06 & 5.32 & 
798.75 & 6.02 & \bf{767.53} & 
3.07\\CON8-9 & 827.71 & 8.96 & 
837.66 & 6.88 & \bf{809.00} & 
2.31\\\bf{PROM.} & 
\bf{770.78} & \bf{6.81} & \bf{778.50} & \bf{6.65} & \bf{758.54} & \bf{1.49}\\[1ex]\hline
\end{tabular}
\label{table:nonlin}
\end{table} \clearpage
\begin{table}[ht]
\caption{Resultados de la ejecución de la metaheurística ILS, utilizando instancias de Dethloff con la configuración -n 45.0 -LS 50.0}
\centering
\small
\begin{tabular}{c c c c c c c}
\hline\hline
Instancia & Costo mínimo & Tiempo(seg.) & Costo promedio & Tiempo promedio(seg.) & Costo ILS & \%Gap \\ [0.5ex]
\hline
SCA3-0 & 640.55 & 8.54 & 
641.96 & 8.81 & \bf{635.62} & 
0.78\\SCA3-1 & \bf{697.84} & 9.50 & 
709.19 & 9.06 & 697.84 & 0.00\\
SCA3-2 & 664.18 & 7.73 & 
665.41 & 8.44 & \bf{659.34} & 
0.73\\SCA3-3 & \bf{680.04} & 9.95 & 
681.38 & 8.96 & 680.04 & 0.00\\
SCA3-4 & \bf{690.50} & 9.59 & 
693.82 & 9.09 & 690.50 & 0.00\\
SCA3-5 & 662.75 & 7.54 & 
674.17 & 7.89 & \bf{659.90} & 
0.43\\SCA3-6 & 652.94 & 9.84 & 
654.28 & 9.11 & \bf{651.09} & 
0.28\\SCA3-7 & 671.77 & 8.36 & 
671.77 & 8.31 & \bf{659.17} & 
1.91\\SCA3-8 & \bf{719.47} & 7.66 & 
735.78 & 7.79 & 719.47 & 0.00\\
SCA3-9 & \bf{681.00} & 7.20 & 
683.25 & 8.52 & 681.00 & 0.00\\
SCA8-0 & 974.62 & 7.69 & 
1010.10 & 7.31 & \bf{961.50} & 
1.36\\SCA8-1 & 1069.65 & 8.84 & 
1077.53 & 7.50 & \bf{1049.65} & 
1.91\\SCA8-2 & 1054.47 & 6.75 & 
1068.50 & 7.25 & \bf{1039.64} & 
1.43\\SCA8-3 & 1013.02 & 7.59 & 
1018.40 & 7.07 & \bf{983.34} & 
3.02\\SCA8-4 & 1082.87 & 7.60 & 
1092.04 & 7.38 & \bf{1065.49} & 
1.63\\SCA8-5 & 1058.56 & 10.55 & 
1071.11 & 8.28 & \bf{1027.08} & 
3.06\\SCA8-6 & 1001.38 & 6.92 & 
1011.31 & 6.67 & \bf{971.82} & 
3.04\\SCA8-7 & 1085.83 & 7.84 & 
1098.00 & 6.71 & \bf{1051.28} & 
3.29\\SCA8-8 & 1082.12 & 7.34 & 
1091.25 & 6.71 & \bf{1071.18} & 
1.02\\SCA8-9 & 1077.44 & 6.25 & 
1092.76 & 7.18 & \bf{1060.50} & 
1.60\\CON3-0 & 632.57 & 7.13 & 
634.73 & 8.11 & \bf{616.52} & 
2.60\\CON3-1 & 560.75 & 8.74 & 
564.83 & 8.99 & \bf{554.47} & 
1.13\\CON3-2 & 521.63 & 8.02 & 
524.42 & 8.13 & \bf{518.00} & 
0.70\\CON3-3 & 592.41 & 8.96 & 
601.88 & 8.70 & \bf{591.19} & 
0.21\\CON3-4 & 594.59 & 8.78 & 
599.31 & 8.64 & \bf{588.79} & 
0.99\\CON3-5 & 564.88 & 8.28 & 
571.11 & 8.37 & \bf{563.70} & 
0.21\\CON3-6 & 502.16 & 9.38 & 
506.80 & 9.40 & \bf{499.05} & 
0.62\\CON3-7 & 578.41 & 8.28 & 
585.87 & 8.53 & \bf{576.48} & 
0.33\\CON3-8 & 524.59 & 10.55 & 
528.86 & 9.23 & \bf{523.05} & 
0.29\\CON3-9 & 584.52 & 7.57 & 
588.16 & 8.20 & \bf{578.24} & 
1.09\\CON8-0 & 870.49 & 6.72 & 
897.61 & 6.55 & \bf{857.17} & 
1.55\\CON8-1 & 759.65 & 7.41 & 
771.95 & 7.11 & \bf{740.85} & 
2.54\\CON8-2 & 719.90 & 6.67 & 
725.53 & 8.22 & \bf{712.89} & 
0.98\\CON8-3 & 827.08 & 6.46 & 
843.69 & 7.59 & \bf{811.07} & 
1.97\\CON8-4 & 799.56 & 6.64 & 
810.19 & 6.67 & \bf{772.25} & 
3.54\\CON8-5 & 758.12 & 7.17 & 
786.46 & 8.65 & \bf{754.88} & 
0.43\\CON8-6 & 704.00 & 7.20 & 
713.08 & 6.87 & \bf{678.92} & 
3.69\\CON8-7 & 814.79 & 7.36 & 
824.26 & 6.87 & \bf{811.96} & 
0.35\\CON8-8 & 783.63 & 6.83 & 
792.56 & 7.20 & \bf{767.53} & 
2.10\\CON8-9 & 820.45 & 8.60 & 
831.34 & 8.03 & \bf{809.00} & 
1.42\\\bf{PROM.} & 
\bf{769.38} & \bf{8.00} & \bf{778.62} & \bf{7.95} & \bf{758.54} & \bf{1.31}\\[1ex]\hline
\end{tabular}
\label{table:nonlin}
\end{table} \clearpage
\begin{table}[ht]
\caption{Resultados de la ejecución de la metaheurística ILS, utilizando instancias de Dethloff con la configuración -n 45.0 -LS 60.0}
\centering
\small
\begin{tabular}{c c c c c c c}
\hline\hline
Instancia & Costo mínimo & Tiempo(seg.) & Costo promedio & Tiempo promedio(seg.) & Costo ILS & \%Gap \\ [0.5ex]
\hline
SCA3-0 & 640.55 & 9.08 & 
641.49 & 10.60 & \bf{635.62} & 
0.78\\SCA3-1 & \bf{697.84} & 11.24 & 
702.33 & 10.77 & 697.84 & 0.00\\
SCA3-2 & 664.21 & 9.75 & 
666.45 & 9.97 & \bf{659.34} & 
0.74\\SCA3-3 & 680.60 & 10.93 & 
681.35 & 10.82 & \bf{680.04} & 
0.08\\SCA3-4 & \bf{690.50} & 10.88 & 
691.02 & 10.46 & 690.50 & 0.00\\
SCA3-5 & 661.07 & 10.86 & 
666.70 & 10.28 & \bf{659.90} & 
0.18\\SCA3-6 & \bf{651.09} & 11.29 & 
653.74 & 10.03 & 651.09 & 0.00\\
SCA3-7 & 667.24 & 9.31 & 
673.57 & 9.36 & \bf{659.17} & 
1.22\\SCA3-8 & \bf{719.47} & 9.00 & 
728.76 & 9.78 & 719.47 & 0.00\\
SCA3-9 & \bf{681.00} & 8.84 & 
685.32 & 9.53 & 681.00 & 0.00\\
SCA8-0 & 984.48 & 8.78 & 
1010.62 & 8.56 & \bf{961.50} & 
2.39\\SCA8-1 & 1073.93 & 6.70 & 
1083.73 & 7.60 & \bf{1049.65} & 
2.31\\SCA8-2 & 1064.91 & 8.14 & 
1068.18 & 8.55 & \bf{1039.64} & 
2.43\\SCA8-3 & 1004.78 & 8.02 & 
1030.04 & 7.91 & \bf{983.34} & 
2.18\\SCA8-4 & 1080.34 & 7.56 & 
1089.94 & 8.13 & \bf{1065.49} & 
1.39\\SCA8-5 & 1049.44 & 8.34 & 
1072.51 & 8.43 & \bf{1027.08} & 
2.18\\SCA8-6 & 991.27 & 7.73 & 
1003.27 & 8.10 & \bf{971.82} & 
2.00\\SCA8-7 & 1071.53 & 8.00 & 
1084.18 & 9.24 & \bf{1051.28} & 
1.93\\SCA8-8 & 1085.91 & 7.16 & 
1094.09 & 7.25 & \bf{1071.18} & 
1.38\\SCA8-9 & 1079.22 & 8.24 & 
1097.08 & 8.04 & \bf{1060.50} & 
1.77\\CON3-0 & 619.09 & 9.68 & 
631.68 & 9.90 & \bf{616.52} & 
0.42\\CON3-1 & 560.75 & 10.59 & 
562.06 & 10.38 & \bf{554.47} & 
1.13\\CON3-2 & 521.38 & 11.33 & 
525.88 & 10.57 & \bf{518.00} & 
0.65\\CON3-3 & 591.20 & 11.45 & 
599.15 & 10.84 & \bf{591.19} & 
0.00\\CON3-4 & 591.43 & 11.38 & 
594.84 & 10.16 & \bf{588.79} & 
0.45\\CON3-5 & 567.94 & 9.20 & 
572.42 & 10.10 & \bf{563.70} & 
0.75\\CON3-6 & 502.16 & 10.08 & 
504.49 & 10.86 & \bf{499.05} & 
0.62\\CON3-7 & 586.01 & 11.76 & 
591.83 & 9.80 & \bf{576.48} & 
1.65\\CON3-8 & 524.59 & 10.78 & 
525.64 & 10.50 & \bf{523.05} & 
0.29\\CON3-9 & 590.17 & 11.08 & 
590.57 & 10.04 & \bf{578.24} & 
2.06\\CON8-0 & 873.31 & 8.30 & 
894.91 & 8.10 & \bf{857.17} & 
1.88\\CON8-1 & 763.29 & 8.84 & 
769.31 & 8.16 & \bf{740.85} & 
3.03\\CON8-2 & 720.09 & 8.70 & 
731.88 & 9.08 & \bf{712.89} & 
1.01\\CON8-3 & 830.53 & 8.76 & 
837.94 & 8.81 & \bf{811.07} & 
2.40\\CON8-4 & 779.97 & 7.62 & 
786.59 & 7.79 & \bf{772.25} & 
1.00\\CON8-5 & 758.84 & 8.70 & 
772.86 & 8.34 & \bf{754.88} & 
0.52\\CON8-6 & 699.30 & 9.81 & 
706.80 & 9.28 & \bf{678.92} & 
3.00\\CON8-7 & 824.75 & 8.12 & 
842.00 & 8.61 & \bf{811.96} & 
1.58\\CON8-8 & 791.03 & 8.46 & 
793.64 & 8.64 & \bf{767.53} & 
3.06\\CON8-9 & 816.55 & 9.31 & 
833.28 & 8.93 & \bf{809.00} & 
0.93\\\bf{PROM.} & 
\bf{768.79} & \bf{9.34} & \bf{777.30} & \bf{9.31} & \bf{758.54} & \bf{1.24}\\[1ex]\hline
\end{tabular}
\label{table:nonlin}
\end{table} \clearpage
\begin{table}[ht]
\caption{Resultados de la ejecución de la metaheurística ILS, utilizando instancias de Dethloff con la configuración -n 45.0 -LS 70.0}
\centering
\small
\begin{tabular}{c c c c c c c}
\hline\hline
Instancia & Costo mínimo & Tiempo(seg.) & Costo promedio & Tiempo promedio(seg.) & Costo ILS & \%Gap \\ [0.5ex]
\hline
SCA3-0 & 636.06 & 12.22 & 
639.43 & 11.75 & \bf{635.62} & 
0.07\\SCA3-1 & 701.53 & 11.64 & 
705.30 & 11.28 & \bf{697.84} & 
0.53\\SCA3-2 & 664.21 & 10.81 & 
665.50 & 11.51 & \bf{659.34} & 
0.74\\SCA3-3 & \bf{680.04} & 12.15 & 
680.92 & 12.59 & 680.04 & 0.00\\
SCA3-4 & \bf{690.50} & 12.02 & 
692.55 & 11.01 & 690.50 & 0.00\\
SCA3-5 & \bf{659.90} & 13.17 & 
669.14 & 11.34 & 659.90 & 0.00\\
SCA3-6 & 653.93 & 11.13 & 
654.87 & 10.96 & \bf{651.09} & 
0.44\\SCA3-7 & 669.89 & 10.90 & 
670.94 & 10.91 & \bf{659.17} & 
1.63\\SCA3-8 & \bf{719.47} & 13.42 & 
723.54 & 11.63 & 719.47 & 0.00\\
SCA3-9 & 684.25 & 10.12 & 
688.47 & 11.47 & \bf{681.00} & 
0.48\\SCA8-0 & 982.58 & 8.32 & 
1003.51 & 9.16 & \bf{961.50} & 
2.19\\SCA8-1 & 1067.92 & 8.54 & 
1085.79 & 9.95 & \bf{1049.65} & 
1.74\\SCA8-2 & 1065.76 & 9.68 & 
1075.82 & 8.63 & \bf{1039.64} & 
2.51\\SCA8-3 & 1004.78 & 11.08 & 
1021.59 & 9.99 & \bf{983.34} & 
2.18\\SCA8-4 & 1077.80 & 8.01 & 
1085.46 & 8.90 & \bf{1065.49} & 
1.16\\SCA8-5 & 1054.03 & 9.29 & 
1069.97 & 10.47 & \bf{1027.08} & 
2.62\\SCA8-6 & 979.28 & 10.73 & 
997.70 & 9.91 & \bf{971.82} & 
0.77\\SCA8-7 & 1064.45 & 9.72 & 
1078.18 & 9.61 & \bf{1051.28} & 
1.25\\SCA8-8 & 1086.77 & 9.04 & 
1088.86 & 9.27 & \bf{1071.18} & 
1.46\\SCA8-9 & 1094.93 & 7.82 & 
1103.21 & 9.10 & \bf{1060.50} & 
3.25\\CON3-0 & 617.59 & 11.05 & 
630.25 & 11.19 & \bf{616.52} & 
0.17\\CON3-1 & 560.75 & 12.77 & 
562.47 & 12.56 & \bf{554.47} & 
1.13\\CON3-2 & 521.38 & 11.78 & 
521.96 & 12.14 & \bf{518.00} & 
0.65\\CON3-3 & \bf{591.19} & 12.46 & 
593.78 & 11.81 & 591.19 & 0.00\\
CON3-4 & 591.43 & 12.70 & 
595.25 & 10.74 & \bf{588.79} & 
0.45\\CON3-5 & 569.57 & 11.28 & 
574.96 & 11.78 & \bf{563.70} & 
1.04\\CON3-6 & 502.16 & 9.98 & 
505.14 & 11.28 & \bf{499.05} & 
0.62\\CON3-7 & 578.41 & 12.78 & 
584.40 & 11.55 & \bf{576.48} & 
0.33\\CON3-8 & 524.59 & 16.88 & 
526.36 & 13.27 & \bf{523.05} & 
0.29\\CON3-9 & 589.72 & 12.00 & 
591.80 & 11.75 & \bf{578.24} & 
1.99\\CON8-0 & 878.94 & 9.56 & 
887.88 & 9.95 & \bf{857.17} & 
2.54\\CON8-1 & 754.51 & 10.05 & 
765.19 & 10.44 & \bf{740.85} & 
1.84\\CON8-2 & 718.78 & 8.96 & 
726.68 & 9.86 & \bf{712.89} & 
0.83\\CON8-3 & 825.62 & 9.84 & 
836.72 & 9.65 & \bf{811.07} & 
1.79\\CON8-4 & 795.72 & 10.24 & 
806.65 & 9.16 & \bf{772.25} & 
3.04\\CON8-5 & 762.36 & 8.72 & 
778.55 & 9.29 & \bf{754.88} & 
0.99\\CON8-6 & 693.19 & 8.08 & 
702.67 & 9.52 & \bf{678.92} & 
2.10\\CON8-7 & 815.14 & 10.62 & 
828.24 & 10.05 & \bf{811.96} & 
0.39\\CON8-8 & 782.09 & 8.70 & 
786.27 & 9.80 & \bf{767.53} & 
1.90\\CON8-9 & 814.57 & 11.46 & 
822.56 & 9.77 & \bf{809.00} & 
0.69\\\bf{PROM.} & 
\bf{768.14} & \bf{10.74} & \bf{775.71} & \bf{10.62} & \bf{758.54} & \bf{1.15}\\[1ex]\hline
\end{tabular}
\label{table:nonlin}
\end{table} \clearpage
\begin{table}[ht]
\caption{Resultados de la ejecución de la metaheurística ILS, utilizando instancias de Dethloff con la configuración -n 45.0 -LS 80.0}
\centering
\small
\begin{tabular}{c c c c c c c}
\hline\hline
Instancia & Costo mínimo & Tiempo(seg.) & Costo promedio & Tiempo promedio(seg.) & Costo ILS & \%Gap \\ [0.5ex]
\hline
SCA3-0 & 636.06 & 13.90 & 
639.43 & 13.64 & \bf{635.62} & 
0.07\\SCA3-1 & \bf{697.84} & 14.45 & 
699.68 & 13.27 & 697.84 & 0.00\\
SCA3-2 & 661.13 & 12.22 & 
663.43 & 14.00 & \bf{659.34} & 
0.27\\SCA3-3 & 680.60 & 14.63 & 
681.24 & 14.40 & \bf{680.04} & 
0.08\\SCA3-4 & \bf{690.50} & 14.24 & 
690.50 & 13.84 & 690.50 & 0.00\\
SCA3-5 & \bf{659.90} & 13.53 & 
673.98 & 12.82 & 659.90 & 0.00\\
SCA3-6 & \bf{651.09} & 13.59 & 
653.56 & 13.28 & 651.09 & 0.00\\
SCA3-7 & 669.89 & 13.18 & 
671.49 & 13.48 & \bf{659.17} & 
1.63\\SCA3-8 & \bf{719.47} & 11.30 & 
720.84 & 12.36 & 719.47 & 0.00\\
SCA3-9 & \bf{681.00} & 15.84 & 
687.78 & 13.09 & 681.00 & 0.00\\
SCA8-0 & 970.64 & 11.96 & 
990.79 & 11.40 & \bf{961.50} & 
0.95\\SCA8-1 & 1072.23 & 11.15 & 
1088.81 & 10.21 & \bf{1049.65} & 
2.15\\SCA8-2 & 1042.69 & 11.85 & 
1052.74 & 10.86 & \bf{1039.64} & 
0.29\\SCA8-3 & 1013.07 & 10.60 & 
1033.68 & 10.48 & \bf{983.34} & 
3.02\\SCA8-4 & 1098.11 & 11.72 & 
1103.43 & 11.13 & \bf{1065.49} & 
3.06\\SCA8-5 & 1063.83 & 10.88 & 
1078.22 & 10.75 & \bf{1027.08} & 
3.58\\SCA8-6 & 989.31 & 11.56 & 
998.29 & 10.10 & \bf{971.82} & 
1.80\\SCA8-7 & 1089.02 & 11.89 & 
1096.99 & 11.48 & \bf{1051.28} & 
3.59\\SCA8-8 & 1084.41 & 13.34 & 
1090.98 & 11.50 & \bf{1071.18} & 
1.24\\SCA8-9 & 1067.42 & 11.77 & 
1081.58 & 10.45 & \bf{1060.50} & 
0.65\\CON3-0 & 628.47 & 14.26 & 
634.22 & 13.52 & \bf{616.52} & 
1.94\\CON3-1 & 560.75 & 13.66 & 
561.19 & 13.48 & \bf{554.47} & 
1.13\\CON3-2 & 521.38 & 13.78 & 
521.50 & 13.43 & \bf{518.00} & 
0.65\\CON3-3 & 591.20 & 13.93 & 
593.90 & 13.35 & \bf{591.19} & 
0.00\\CON3-4 & 593.78 & 12.10 & 
599.46 & 13.41 & \bf{588.79} & 
0.85\\CON3-5 & 569.88 & 14.42 & 
572.54 & 12.90 & \bf{563.70} & 
1.10\\CON3-6 & 502.16 & 14.34 & 
505.21 & 13.36 & \bf{499.05} & 
0.62\\CON3-7 & 578.41 & 11.72 & 
580.52 & 13.94 & \bf{576.48} & 
0.33\\CON3-8 & 524.59 & 13.58 & 
525.90 & 13.04 & \bf{523.05} & 
0.29\\CON3-9 & 588.11 & 12.90 & 
590.05 & 13.66 & \bf{578.24} & 
1.71\\CON8-0 & 865.86 & 9.88 & 
882.43 & 10.67 & \bf{857.17} & 
1.01\\CON8-1 & 754.51 & 10.07 & 
763.86 & 10.86 & \bf{740.85} & 
1.84\\CON8-2 & 719.56 & 11.57 & 
724.98 & 11.09 & \bf{712.89} & 
0.94\\CON8-3 & 835.95 & 10.66 & 
839.78 & 11.17 & \bf{811.07} & 
3.07\\CON8-4 & 798.09 & 10.61 & 
807.96 & 10.22 & \bf{772.25} & 
3.35\\CON8-5 & 758.99 & 12.20 & 
768.40 & 11.55 & \bf{754.88} & 
0.54\\CON8-6 & 685.45 & 11.93 & 
700.51 & 11.02 & \bf{678.92} & 
0.96\\CON8-7 & 820.92 & 10.09 & 
829.16 & 10.92 & \bf{811.96} & 
1.10\\CON8-8 & 785.14 & 12.57 & 
793.96 & 10.43 & \bf{767.53} & 
2.29\\CON8-9 & 838.04 & 10.08 & 
844.40 & 11.49 & \bf{809.00} & 
3.59\\\bf{PROM.} & 
\bf{768.99} & \bf{12.45} & \bf{775.93} & \bf{12.15} & \bf{758.54} & \bf{1.24}\\[1ex]\hline
\end{tabular}
\label{table:nonlin}
\end{table} \clearpage
\begin{table}[ht]
\caption{Resultados de la ejecución de la metaheurística ILS, utilizando instancias de Dethloff con la configuración -n 5.0 -LS 10.0}
\centering
\small
\begin{tabular}{c c c c c c c}
\hline\hline
Instancia & Costo mínimo & Tiempo(seg.) & Costo promedio & Tiempo promedio(seg.) & Costo ILS & \%Gap \\ [0.5ex]
\hline
SCA3-0 & 640.55 & 0.30 & 
641.97 & 0.29 & \bf{635.62} & 
0.78\\SCA3-1 & 708.40 & 0.38 & 
722.56 & 0.33 & \bf{697.84} & 
1.51\\SCA3-2 & 674.29 & 0.29 & 
697.58 & 0.27 & \bf{659.34} & 
2.27\\SCA3-3 & 712.47 & 0.36 & 
717.69 & 0.32 & \bf{680.04} & 
4.77\\SCA3-4 & \bf{690.50} & 0.30 & 
697.63 & 0.29 & 690.50 & 0.00\\
SCA3-5 & 689.59 & 0.37 & 
701.64 & 0.30 & \bf{659.90} & 
4.50\\SCA3-6 & 696.72 & 0.38 & 
709.08 & 0.26 & \bf{651.09} & 
7.01\\SCA3-7 & 669.89 & 0.39 & 
676.31 & 0.31 & \bf{659.17} & 
1.63\\SCA3-8 & 745.33 & 0.28 & 
754.95 & 0.31 & \bf{719.47} & 
3.59\\SCA3-9 & 711.93 & 0.27 & 
719.45 & 0.32 & \bf{681.00} & 
4.54\\SCA8-0 & 1001.33 & 0.28 & 
1015.17 & 0.28 & \bf{961.50} & 
4.14\\SCA8-1 & 1080.67 & 0.34 & 
1112.55 & 0.27 & \bf{1049.65} & 
2.96\\SCA8-2 & 1100.07 & 0.24 & 
1103.89 & 0.26 & \bf{1039.64} & 
5.81\\SCA8-3 & 1044.10 & 0.36 & 
1053.97 & 0.32 & \bf{983.34} & 
6.18\\SCA8-4 & 1138.23 & 0.40 & 
1153.97 & 0.37 & \bf{1065.49} & 
6.83\\SCA8-5 & 1088.81 & 0.31 & 
1106.23 & 0.39 & \bf{1027.08} & 
6.01\\SCA8-6 & 1021.99 & 0.27 & 
1036.04 & 0.30 & \bf{971.82} & 
5.16\\SCA8-7 & 1114.68 & 0.31 & 
1127.69 & 0.27 & \bf{1051.28} & 
6.03\\SCA8-8 & \bf{1071.18} & 0.24 & 
1130.81 & 0.26 & 1071.18 & 0.00\\
SCA8-9 & 1114.34 & 0.26 & 
1133.72 & 0.25 & \bf{1060.50} & 
5.08\\CON3-0 & 650.16 & 0.31 & 
669.65 & 0.33 & \bf{616.52} & 
5.46\\CON3-1 & 568.45 & 0.38 & 
573.28 & 0.28 & \bf{554.47} & 
2.52\\CON3-2 & 521.63 & 0.39 & 
540.90 & 0.38 & \bf{518.00} & 
0.70\\CON3-3 & 611.33 & 0.30 & 
635.93 & 0.27 & \bf{591.19} & 
3.41\\CON3-4 & 611.21 & 0.33 & 
628.71 & 0.33 & \bf{588.79} & 
3.81\\CON3-5 & 569.04 & 0.31 & 
572.26 & 0.27 & \bf{563.70} & 
0.95\\CON3-6 & 511.00 & 0.30 & 
529.03 & 0.39 & \bf{499.05} & 
2.39\\CON3-7 & 618.48 & 0.36 & 
630.72 & 0.30 & \bf{576.48} & 
7.29\\CON3-8 & 546.26 & 0.46 & 
566.07 & 0.39 & \bf{523.05} & 
4.44\\CON3-9 & 599.04 & 0.31 & 
599.84 & 0.30 & \bf{578.24} & 
3.60\\CON8-0 & 919.96 & 0.23 & 
934.16 & 0.23 & \bf{857.17} & 
7.33\\CON8-1 & 777.08 & 0.47 & 
815.24 & 0.37 & \bf{740.85} & 
4.89\\CON8-2 & 776.70 & 0.26 & 
787.67 & 0.26 & \bf{712.89} & 
8.95\\CON8-3 & 857.98 & 0.24 & 
858.49 & 0.27 & \bf{811.07} & 
5.78\\CON8-4 & 825.32 & 0.44 & 
848.55 & 0.36 & \bf{772.25} & 
6.87\\CON8-5 & 825.97 & 0.26 & 
832.06 & 0.33 & \bf{754.88} & 
9.42\\CON8-6 & 724.97 & 0.29 & 
741.21 & 0.27 & \bf{678.92} & 
6.78\\CON8-7 & 839.18 & 0.24 & 
868.98 & 0.34 & \bf{811.96} & 
3.35\\CON8-8 & 794.22 & 0.28 & 
809.92 & 0.29 & \bf{767.53} & 
3.48\\CON8-9 & 825.74 & 0.28 & 
851.28 & 0.40 & \bf{809.00} & 
2.07\\\bf{PROM.} & 
\bf{792.22} & \bf{0.32} & \bf{807.67} & \bf{0.31} & \bf{758.54} & \bf{4.31}\\[1ex]\hline
\end{tabular}
\label{table:nonlin}
\end{table} \clearpage
\begin{table}[ht]
\caption{Resultados de la ejecución de la metaheurística ILS, utilizando instancias de Dethloff con la configuración -n 5.0 -LS 20.0}
\centering
\small
\begin{tabular}{c c c c c c c}
\hline\hline
Instancia & Costo mínimo & Tiempo(seg.) & Costo promedio & Tiempo promedio(seg.) & Costo ILS & \%Gap \\ [0.5ex]
\hline
SCA3-0 & 642.44 & 0.50 & 
642.88 & 0.49 & \bf{635.62} & 
1.07\\SCA3-1 & 714.92 & 0.53 & 
725.23 & 0.40 & \bf{697.84} & 
2.45\\SCA3-2 & 682.38 & 0.57 & 
697.93 & 0.55 & \bf{659.34} & 
3.49\\SCA3-3 & 688.71 & 0.43 & 
702.78 & 0.47 & \bf{680.04} & 
1.27\\SCA3-4 & 698.98 & 0.39 & 
725.90 & 0.38 & \bf{690.50} & 
1.23\\SCA3-5 & 686.47 & 0.58 & 
691.48 & 0.52 & \bf{659.90} & 
4.03\\SCA3-6 & 662.50 & 0.68 & 
694.59 & 0.43 & \bf{651.09} & 
1.75\\SCA3-7 & 671.77 & 0.33 & 
673.92 & 0.36 & \bf{659.17} & 
1.91\\SCA3-8 & \bf{719.47} & 0.50 & 
747.48 & 0.52 & 719.47 & 0.00\\
SCA3-9 & 700.27 & 0.39 & 
702.97 & 0.51 & \bf{681.00} & 
2.83\\SCA8-0 & 977.93 & 0.60 & 
1001.69 & 0.48 & \bf{961.50} & 
1.71\\SCA8-1 & 1071.03 & 0.56 & 
1103.83 & 0.45 & \bf{1049.65} & 
2.04\\SCA8-2 & 1064.69 & 0.42 & 
1064.93 & 0.41 & \bf{1039.64} & 
2.41\\SCA8-3 & 1043.01 & 0.46 & 
1061.13 & 0.46 & \bf{983.34} & 
6.07\\SCA8-4 & 1126.94 & 0.40 & 
1132.64 & 0.37 & \bf{1065.49} & 
5.77\\SCA8-5 & 1087.33 & 0.37 & 
1114.02 & 0.39 & \bf{1027.08} & 
5.87\\SCA8-6 & 1013.77 & 0.38 & 
1022.63 & 0.34 & \bf{971.82} & 
4.32\\SCA8-7 & 1077.46 & 0.38 & 
1141.27 & 0.41 & \bf{1051.28} & 
2.49\\SCA8-8 & 1095.65 & 0.57 & 
1101.53 & 0.41 & \bf{1071.18} & 
2.28\\SCA8-9 & 1114.21 & 0.38 & 
1120.16 & 0.35 & \bf{1060.50} & 
5.06\\CON3-0 & 632.57 & 0.62 & 
641.52 & 0.60 & \bf{616.52} & 
2.60\\CON3-1 & 580.49 & 0.44 & 
584.53 & 0.51 & \bf{554.47} & 
4.69\\CON3-2 & 541.17 & 0.38 & 
556.51 & 0.39 & \bf{518.00} & 
4.47\\CON3-3 & 603.99 & 0.49 & 
617.10 & 0.48 & \bf{591.19} & 
2.17\\CON3-4 & 605.94 & 0.42 & 
618.10 & 0.40 & \bf{588.79} & 
2.91\\CON3-5 & 600.88 & 0.46 & 
603.70 & 0.38 & \bf{563.70} & 
6.60\\CON3-6 & 504.15 & 0.54 & 
508.24 & 0.53 & \bf{499.05} & 
1.02\\CON3-7 & 606.84 & 0.56 & 
612.35 & 0.59 & \bf{576.48} & 
5.27\\CON3-8 & 535.32 & 0.39 & 
542.68 & 0.43 & \bf{523.05} & 
2.35\\CON3-9 & 589.85 & 0.65 & 
601.67 & 0.52 & \bf{578.24} & 
2.01\\CON8-0 & 968.67 & 0.34 & 
983.30 & 0.40 & \bf{857.17} & 
13.01\\CON8-1 & 773.56 & 1.08 & 
795.03 & 0.66 & \bf{740.85} & 
4.42\\CON8-2 & 768.27 & 0.50 & 
787.46 & 0.47 & \bf{712.89} & 
7.77\\CON8-3 & 824.90 & 0.40 & 
851.70 & 0.45 & \bf{811.07} & 
1.71\\CON8-4 & 808.96 & 0.33 & 
832.77 & 0.34 & \bf{772.25} & 
4.75\\CON8-5 & 788.25 & 0.46 & 
858.33 & 0.38 & \bf{754.88} & 
4.42\\CON8-6 & 723.09 & 0.52 & 
726.62 & 0.49 & \bf{678.92} & 
6.51\\CON8-7 & 857.41 & 0.32 & 
860.32 & 0.43 & \bf{811.96} & 
5.60\\CON8-8 & 799.04 & 0.35 & 
799.36 & 0.44 & \bf{767.53} & 
4.11\\CON8-9 & 851.34 & 0.67 & 
886.78 & 0.44 & \bf{809.00} & 
5.23\\\bf{PROM.} & 
\bf{787.62} & \bf{0.48} & \bf{803.43} & \bf{0.45} & \bf{758.54} & \bf{3.74}\\[1ex]\hline
\end{tabular}
\label{table:nonlin}
\end{table} \clearpage
\begin{table}[ht]
\caption{Resultados de la ejecución de la metaheurística ILS, utilizando instancias de Dethloff con la configuración -n 5.0 -LS 30.0}
\centering
\small
\begin{tabular}{c c c c c c c}
\hline\hline
Instancia & Costo mínimo & Tiempo(seg.) & Costo promedio & Tiempo promedio(seg.) & Costo ILS & \%Gap \\ [0.5ex]
\hline
SCA3-0 & 640.55 & 0.72 & 
652.82 & 0.67 & \bf{635.62} & 
0.78\\SCA3-1 & 710.89 & 0.78 & 
719.29 & 0.64 & \bf{697.84} & 
1.87\\SCA3-2 & 666.72 & 0.59 & 
680.29 & 0.65 & \bf{659.34} & 
1.12\\SCA3-3 & 681.74 & 0.69 & 
693.16 & 0.61 & \bf{680.04} & 
0.25\\SCA3-4 & \bf{690.50} & 0.54 & 
701.02 & 0.62 & 690.50 & 0.00\\
SCA3-5 & 687.15 & 0.82 & 
701.75 & 0.66 & \bf{659.90} & 
4.13\\SCA3-6 & 654.26 & 0.72 & 
664.74 & 0.61 & \bf{651.09} & 
0.49\\SCA3-7 & 671.77 & 0.60 & 
677.45 & 0.59 & \bf{659.17} & 
1.91\\SCA3-8 & 719.77 & 0.70 & 
742.35 & 0.70 & \bf{719.47} & 
0.04\\SCA3-9 & 684.44 & 0.40 & 
684.72 & 0.48 & \bf{681.00} & 
0.51\\SCA8-0 & 1023.20 & 0.43 & 
1042.98 & 0.54 & \bf{961.50} & 
6.42\\SCA8-1 & 1090.41 & 0.47 & 
1114.91 & 0.55 & \bf{1049.65} & 
3.88\\SCA8-2 & 1072.45 & 0.69 & 
1119.59 & 0.55 & \bf{1039.64} & 
3.16\\SCA8-3 & 1062.82 & 0.38 & 
1073.51 & 0.46 & \bf{983.34} & 
8.08\\SCA8-4 & 1127.21 & 0.58 & 
1142.55 & 0.55 & \bf{1065.49} & 
5.79\\SCA8-5 & 1083.64 & 0.53 & 
1093.15 & 0.56 & \bf{1027.08} & 
5.51\\SCA8-6 & 1010.96 & 0.66 & 
1040.77 & 0.56 & \bf{971.82} & 
4.03\\SCA8-7 & 1098.54 & 0.43 & 
1116.15 & 0.56 & \bf{1051.28} & 
4.50\\SCA8-8 & 1094.03 & 0.72 & 
1133.61 & 0.68 & \bf{1071.18} & 
2.13\\SCA8-9 & 1105.19 & 0.39 & 
1119.37 & 0.42 & \bf{1060.50} & 
4.21\\CON3-0 & 642.11 & 0.61 & 
652.86 & 0.56 & \bf{616.52} & 
4.15\\CON3-1 & 567.39 & 0.65 & 
575.55 & 0.70 & \bf{554.47} & 
2.33\\CON3-2 & 521.63 & 0.65 & 
537.22 & 0.62 & \bf{518.00} & 
0.70\\CON3-3 & 602.16 & 0.80 & 
605.35 & 0.70 & \bf{591.19} & 
1.86\\CON3-4 & 598.21 & 0.61 & 
623.41 & 0.62 & \bf{588.79} & 
1.60\\CON3-5 & 568.69 & 0.98 & 
573.68 & 0.76 & \bf{563.70} & 
0.89\\CON3-6 & 512.74 & 0.60 & 
519.69 & 0.68 & \bf{499.05} & 
2.74\\CON3-7 & 586.01 & 0.74 & 
593.28 & 0.62 & \bf{576.48} & 
1.65\\CON3-8 & 530.94 & 1.01 & 
538.07 & 0.87 & \bf{523.05} & 
1.51\\CON3-9 & 592.60 & 0.62 & 
594.89 & 0.58 & \bf{578.24} & 
2.48\\CON8-0 & 913.70 & 0.84 & 
920.63 & 0.60 & \bf{857.17} & 
6.59\\CON8-1 & 770.15 & 0.45 & 
804.15 & 0.51 & \bf{740.85} & 
3.95\\CON8-2 & 746.90 & 0.51 & 
752.59 & 0.63 & \bf{712.89} & 
4.77\\CON8-3 & 846.19 & 1.09 & 
859.86 & 0.80 & \bf{811.07} & 
4.33\\CON8-4 & 807.88 & 0.67 & 
824.64 & 0.61 & \bf{772.25} & 
4.61\\CON8-5 & 777.34 & 0.84 & 
803.30 & 0.71 & \bf{754.88} & 
2.98\\CON8-6 & 727.18 & 0.43 & 
739.30 & 0.46 & \bf{678.92} & 
7.11\\CON8-7 & 860.80 & 0.63 & 
879.27 & 0.68 & \bf{811.96} & 
6.02\\CON8-8 & 779.43 & 0.47 & 
847.53 & 0.49 & \bf{767.53} & 
1.55\\CON8-9 & 833.37 & 1.11 & 
848.06 & 0.61 & \bf{809.00} & 
3.01\\\bf{PROM.} & 
\bf{784.04} & \bf{0.65} & \bf{800.19} & \bf{0.61} & \bf{758.54} & \bf{3.09}\\[1ex]\hline
\end{tabular}
\label{table:nonlin}
\end{table} \clearpage
\begin{table}[ht]
\caption{Resultados de la ejecución de la metaheurística ILS, utilizando instancias de Dethloff con la configuración -n 5.0 -LS 40.0}
\centering
\small
\begin{tabular}{c c c c c c c}
\hline\hline
Instancia & Costo mínimo & Tiempo(seg.) & Costo promedio & Tiempo promedio(seg.) & Costo ILS & \%Gap \\ [0.5ex]
\hline
SCA3-0 & 640.55 & 1.25 & 
643.75 & 1.08 & \bf{635.62} & 
0.78\\SCA3-1 & 712.59 & 1.06 & 
729.38 & 0.87 & \bf{697.84} & 
2.11\\SCA3-2 & 680.00 & 0.84 & 
700.79 & 0.72 & \bf{659.34} & 
3.13\\SCA3-3 & 685.47 & 1.06 & 
687.22 & 0.94 & \bf{680.04} & 
0.80\\SCA3-4 & \bf{690.50} & 0.93 & 
706.32 & 0.84 & 690.50 & 0.00\\
SCA3-5 & 686.44 & 0.52 & 
686.53 & 0.78 & \bf{659.90} & 
4.02\\SCA3-6 & 654.79 & 0.91 & 
666.88 & 0.73 & \bf{651.09} & 
0.57\\SCA3-7 & 669.89 & 0.68 & 
672.23 & 0.76 & \bf{659.17} & 
1.63\\SCA3-8 & 726.44 & 0.78 & 
750.00 & 0.79 & \bf{719.47} & 
0.97\\SCA3-9 & 683.57 & 1.22 & 
699.63 & 0.96 & \bf{681.00} & 
0.38\\SCA8-0 & 1018.78 & 0.68 & 
1023.67 & 0.63 & \bf{961.50} & 
5.96\\SCA8-1 & 1104.59 & 0.61 & 
1141.33 & 0.62 & \bf{1049.65} & 
5.23\\SCA8-2 & 1066.96 & 0.79 & 
1105.54 & 0.57 & \bf{1039.64} & 
2.63\\SCA8-3 & 1049.88 & 0.80 & 
1087.59 & 0.66 & \bf{983.34} & 
6.77\\SCA8-4 & 1109.30 & 0.73 & 
1119.46 & 0.79 & \bf{1065.49} & 
4.11\\SCA8-5 & 1064.55 & 0.72 & 
1068.26 & 0.89 & \bf{1027.08} & 
3.65\\SCA8-6 & 1026.04 & 0.64 & 
1027.85 & 0.65 & \bf{971.82} & 
5.58\\SCA8-7 & 1110.98 & 1.02 & 
1138.58 & 0.74 & \bf{1051.28} & 
5.68\\SCA8-8 & 1103.10 & 0.53 & 
1139.92 & 0.53 & \bf{1071.18} & 
2.98\\SCA8-9 & 1084.48 & 0.74 & 
1137.56 & 0.60 & \bf{1060.50} & 
2.26\\CON3-0 & 661.33 & 0.72 & 
665.72 & 0.81 & \bf{616.52} & 
7.27\\CON3-1 & 561.63 & 1.04 & 
575.74 & 0.86 & \bf{554.47} & 
1.29\\CON3-2 & 523.99 & 0.87 & 
526.53 & 0.76 & \bf{518.00} & 
1.16\\CON3-3 & 591.20 & 1.18 & 
614.17 & 0.94 & \bf{591.19} & 
0.00\\CON3-4 & 620.15 & 0.79 & 
636.84 & 0.77 & \bf{588.79} & 
5.33\\CON3-5 & 571.63 & 0.83 & 
589.09 & 0.86 & \bf{563.70} & 
1.41\\CON3-6 & 513.13 & 0.85 & 
519.79 & 0.94 & \bf{499.05} & 
2.82\\CON3-7 & 604.95 & 1.23 & 
616.76 & 0.89 & \bf{576.48} & 
4.94\\CON3-8 & 529.65 & 0.88 & 
541.35 & 0.90 & \bf{523.05} & 
1.26\\CON3-9 & 590.64 & 1.01 & 
596.36 & 0.93 & \bf{578.24} & 
2.14\\CON8-0 & 923.06 & 0.68 & 
956.84 & 0.64 & \bf{857.17} & 
7.69\\CON8-1 & 777.60 & 1.00 & 
789.87 & 0.88 & \bf{740.85} & 
4.96\\CON8-2 & 729.26 & 1.09 & 
746.58 & 0.81 & \bf{712.89} & 
2.30\\CON8-3 & 851.97 & 0.99 & 
863.30 & 0.69 & \bf{811.07} & 
5.04\\CON8-4 & 824.36 & 0.71 & 
868.05 & 0.61 & \bf{772.25} & 
6.75\\CON8-5 & 768.63 & 0.55 & 
788.72 & 0.61 & \bf{754.88} & 
1.82\\CON8-6 & 699.30 & 0.63 & 
714.86 & 0.63 & \bf{678.92} & 
3.00\\CON8-7 & 857.87 & 0.61 & 
882.75 & 0.57 & \bf{811.96} & 
5.65\\CON8-8 & 800.37 & 0.64 & 
826.65 & 0.63 & \bf{767.53} & 
4.28\\CON8-9 & 851.07 & 0.72 & 
865.30 & 0.67 & \bf{809.00} & 
5.20\\\bf{PROM.} & 
\bf{785.52} & \bf{0.84} & \bf{802.94} & \bf{0.76} & \bf{758.54} & \bf{3.34}\\[1ex]\hline
\end{tabular}
\label{table:nonlin}
\end{table} \clearpage
\begin{table}[ht]
\caption{Resultados de la ejecución de la metaheurística ILS, utilizando instancias de Dethloff con la configuración -n 5.0 -LS 50.0}
\centering
\small
\begin{tabular}{c c c c c c c}
\hline\hline
Instancia & Costo mínimo & Tiempo(seg.) & Costo promedio & Tiempo promedio(seg.) & Costo ILS & \%Gap \\ [0.5ex]
\hline
SCA3-0 & 640.55 & 1.17 & 
652.70 & 0.99 & \bf{635.62} & 
0.78\\SCA3-1 & 707.56 & 1.04 & 
710.52 & 0.96 & \bf{697.84} & 
1.39\\SCA3-2 & 668.65 & 1.20 & 
672.40 & 1.17 & \bf{659.34} & 
1.41\\SCA3-3 & 680.60 & 1.00 & 
689.88 & 0.92 & \bf{680.04} & 
0.08\\SCA3-4 & \bf{690.50} & 1.04 & 
697.92 & 1.15 & 690.50 & 0.00\\
SCA3-5 & 667.27 & 1.09 & 
694.59 & 1.05 & \bf{659.90} & 
1.12\\SCA3-6 & 652.94 & 0.89 & 
664.62 & 0.97 & \bf{651.09} & 
0.28\\SCA3-7 & 676.07 & 0.77 & 
685.58 & 0.81 & \bf{659.17} & 
2.56\\SCA3-8 & 729.48 & 0.73 & 
741.45 & 1.01 & \bf{719.47} & 
1.39\\SCA3-9 & 697.46 & 0.96 & 
703.26 & 0.96 & \bf{681.00} & 
2.42\\SCA8-0 & 1030.81 & 0.92 & 
1037.78 & 0.80 & \bf{961.50} & 
7.21\\SCA8-1 & 1082.41 & 0.90 & 
1106.43 & 0.82 & \bf{1049.65} & 
3.12\\SCA8-2 & 1071.89 & 1.00 & 
1093.26 & 0.84 & \bf{1039.64} & 
3.10\\SCA8-3 & 1037.53 & 0.64 & 
1046.89 & 0.79 & \bf{983.34} & 
5.51\\SCA8-4 & 1140.61 & 0.74 & 
1152.72 & 0.81 & \bf{1065.49} & 
7.05\\SCA8-5 & 1066.35 & 0.54 & 
1069.58 & 1.02 & \bf{1027.08} & 
3.82\\SCA8-6 & 1018.39 & 0.66 & 
1022.25 & 0.67 & \bf{971.82} & 
4.79\\SCA8-7 & 1096.59 & 0.85 & 
1123.87 & 0.76 & \bf{1051.28} & 
4.31\\SCA8-8 & 1089.91 & 0.56 & 
1104.27 & 0.70 & \bf{1071.18} & 
1.75\\SCA8-9 & 1099.00 & 0.76 & 
1118.58 & 0.80 & \bf{1060.50} & 
3.63\\CON3-0 & 632.57 & 1.22 & 
649.50 & 0.92 & \bf{616.52} & 
2.60\\CON3-1 & 560.75 & 0.97 & 
575.08 & 1.01 & \bf{554.47} & 
1.13\\CON3-2 & 523.93 & 0.80 & 
527.13 & 0.88 & \bf{518.00} & 
1.14\\CON3-3 & 616.13 & 1.02 & 
622.58 & 0.92 & \bf{591.19} & 
4.22\\CON3-4 & 621.06 & 0.96 & 
626.60 & 0.84 & \bf{588.79} & 
5.48\\CON3-5 & 579.41 & 0.72 & 
604.19 & 0.82 & \bf{563.70} & 
2.79\\CON3-6 & 505.41 & 1.17 & 
510.24 & 1.11 & \bf{499.05} & 
1.27\\CON3-7 & 602.08 & 1.08 & 
606.41 & 0.96 & \bf{576.48} & 
4.44\\CON3-8 & 539.17 & 1.02 & 
556.06 & 1.22 & \bf{523.05} & 
3.08\\CON3-9 & 588.99 & 1.08 & 
591.81 & 1.04 & \bf{578.24} & 
1.86\\CON8-0 & 927.09 & 1.17 & 
944.22 & 1.09 & \bf{857.17} & 
8.16\\CON8-1 & 767.10 & 1.03 & 
792.93 & 0.82 & \bf{740.85} & 
3.54\\CON8-2 & 737.95 & 0.82 & 
764.72 & 0.83 & \bf{712.89} & 
3.52\\CON8-3 & 850.89 & 0.73 & 
854.42 & 0.71 & \bf{811.07} & 
4.91\\CON8-4 & 812.88 & 0.78 & 
834.50 & 0.86 & \bf{772.25} & 
5.26\\CON8-5 & 767.03 & 0.78 & 
812.05 & 0.81 & \bf{754.88} & 
1.61\\CON8-6 & 707.65 & 0.82 & 
720.34 & 0.84 & \bf{678.92} & 
4.23\\CON8-7 & 816.26 & 0.62 & 
832.19 & 0.85 & \bf{811.96} & 
0.53\\CON8-8 & 791.03 & 0.74 & 
810.57 & 0.95 & \bf{767.53} & 
3.06\\CON8-9 & 853.51 & 1.04 & 
880.72 & 0.76 & \bf{809.00} & 
5.50\\\bf{PROM.} & 
\bf{783.64} & \bf{0.90} & \bf{797.62} & \bf{0.91} & \bf{758.54} & \bf{3.10}\\[1ex]\hline
\end{tabular}
\label{table:nonlin}
\end{table} \clearpage
\begin{table}[ht]
\caption{Resultados de la ejecución de la metaheurística ILS, utilizando instancias de Dethloff con la configuración -n 5.0 -LS 60.0}
\centering
\small
\begin{tabular}{c c c c c c c}
\hline\hline
Instancia & Costo mínimo & Tiempo(seg.) & Costo promedio & Tiempo promedio(seg.) & Costo ILS & \%Gap \\ [0.5ex]
\hline
SCA3-0 & 641.69 & 1.13 & 
647.58 & 1.20 & \bf{635.62} & 
0.95\\SCA3-1 & 706.90 & 1.49 & 
716.36 & 1.23 & \bf{697.84} & 
1.30\\SCA3-2 & 669.06 & 1.23 & 
683.73 & 1.26 & \bf{659.34} & 
1.47\\SCA3-3 & 687.61 & 1.05 & 
695.33 & 1.01 & \bf{680.04} & 
1.11\\SCA3-4 & \bf{690.50} & 0.98 & 
713.85 & 1.06 & 690.50 & 0.00\\
SCA3-5 & 678.91 & 1.25 & 
688.99 & 1.22 & \bf{659.90} & 
2.88\\SCA3-6 & 653.69 & 1.11 & 
662.59 & 1.00 & \bf{651.09} & 
0.40\\SCA3-7 & 671.77 & 1.59 & 
685.15 & 1.32 & \bf{659.17} & 
1.91\\SCA3-8 & 724.28 & 0.98 & 
740.72 & 1.14 & \bf{719.47} & 
0.67\\SCA3-9 & 687.61 & 1.12 & 
692.96 & 1.07 & \bf{681.00} & 
0.97\\SCA8-0 & 1029.93 & 1.20 & 
1048.09 & 1.02 & \bf{961.50} & 
7.12\\SCA8-1 & 1090.49 & 0.92 & 
1110.47 & 0.91 & \bf{1049.65} & 
3.89\\SCA8-2 & 1054.95 & 1.59 & 
1078.49 & 1.10 & \bf{1039.64} & 
1.47\\SCA8-3 & 1014.71 & 1.21 & 
1043.08 & 1.04 & \bf{983.34} & 
3.19\\SCA8-4 & 1074.81 & 0.65 & 
1087.52 & 0.94 & \bf{1065.49} & 
0.87\\SCA8-5 & 1077.53 & 1.02 & 
1094.76 & 0.85 & \bf{1027.08} & 
4.91\\SCA8-6 & 1030.44 & 0.72 & 
1042.79 & 0.75 & \bf{971.82} & 
6.03\\SCA8-7 & 1090.61 & 1.11 & 
1113.31 & 1.20 & \bf{1051.28} & 
3.74\\SCA8-8 & 1097.05 & 0.94 & 
1105.87 & 1.17 & \bf{1071.18} & 
2.42\\SCA8-9 & 1095.02 & 1.00 & 
1111.54 & 1.16 & \bf{1060.50} & 
3.26\\CON3-0 & 643.28 & 1.00 & 
653.21 & 1.13 & \bf{616.52} & 
4.34\\CON3-1 & 567.40 & 0.94 & 
572.92 & 1.21 & \bf{554.47} & 
2.33\\CON3-2 & 526.55 & 1.52 & 
531.26 & 1.17 & \bf{518.00} & 
1.65\\CON3-3 & 591.20 & 1.24 & 
616.55 & 1.25 & \bf{591.19} & 
0.00\\CON3-4 & 591.43 & 1.04 & 
606.15 & 1.12 & \bf{588.79} & 
0.45\\CON3-5 & 589.19 & 0.98 & 
595.09 & 0.90 & \bf{563.70} & 
4.52\\CON3-6 & 512.50 & 1.06 & 
518.44 & 1.06 & \bf{499.05} & 
2.70\\CON3-7 & 589.93 & 1.03 & 
603.20 & 1.22 & \bf{576.48} & 
2.33\\CON3-8 & 524.59 & 0.89 & 
537.95 & 0.90 & \bf{523.05} & 
0.29\\CON3-9 & 590.17 & 1.29 & 
607.70 & 1.07 & \bf{578.24} & 
2.06\\CON8-0 & 908.39 & 0.67 & 
929.49 & 0.76 & \bf{857.17} & 
5.98\\CON8-1 & 755.77 & 0.87 & 
783.02 & 1.17 & \bf{740.85} & 
2.01\\CON8-2 & 729.70 & 0.81 & 
738.98 & 0.88 & \bf{712.89} & 
2.36\\CON8-3 & 821.22 & 1.05 & 
853.32 & 0.89 & \bf{811.07} & 
1.25\\CON8-4 & 838.60 & 0.66 & 
840.42 & 0.88 & \bf{772.25} & 
8.59\\CON8-5 & 765.74 & 1.18 & 
792.66 & 0.93 & \bf{754.88} & 
1.44\\CON8-6 & 693.76 & 0.84 & 
718.68 & 0.94 & \bf{678.92} & 
2.19\\CON8-7 & 825.08 & 1.40 & 
868.50 & 0.97 & \bf{811.96} & 
1.62\\CON8-8 & 799.89 & 0.92 & 
817.52 & 0.76 & \bf{767.53} & 
4.22\\CON8-9 & 814.72 & 0.95 & 
835.13 & 0.81 & \bf{809.00} & 
0.71\\\bf{PROM.} & 
\bf{778.67} & \bf{1.07} & \bf{794.58} & \bf{1.04} & \bf{758.54} & \bf{2.49}\\[1ex]\hline
\end{tabular}
\label{table:nonlin}
\end{table} \clearpage
\begin{table}[ht]
\caption{Resultados de la ejecución de la metaheurística ILS, utilizando instancias de Dethloff con la configuración -n 5.0 -LS 70.0}
\centering
\small
\begin{tabular}{c c c c c c c}
\hline\hline
Instancia & Costo mínimo & Tiempo(seg.) & Costo promedio & Tiempo promedio(seg.) & Costo ILS & \%Gap \\ [0.5ex]
\hline
SCA3-0 & 640.55 & 1.36 & 
642.46 & 1.45 & \bf{635.62} & 
0.78\\SCA3-1 & 712.78 & 1.71 & 
718.16 & 1.40 & \bf{697.84} & 
2.14\\SCA3-2 & 677.10 & 1.18 & 
682.54 & 1.21 & \bf{659.34} & 
2.69\\SCA3-3 & 684.67 & 1.26 & 
696.03 & 1.09 & \bf{680.04} & 
0.68\\SCA3-4 & 693.23 & 1.35 & 
705.92 & 1.20 & \bf{690.50} & 
0.40\\SCA3-5 & 662.75 & 1.59 & 
677.01 & 1.83 & \bf{659.90} & 
0.43\\SCA3-6 & 661.28 & 1.29 & 
665.53 & 1.35 & \bf{651.09} & 
1.57\\SCA3-7 & 671.77 & 1.31 & 
680.79 & 1.15 & \bf{659.17} & 
1.91\\SCA3-8 & 726.86 & 2.04 & 
736.71 & 1.52 & \bf{719.47} & 
1.03\\SCA3-9 & 690.07 & 1.42 & 
692.78 & 1.23 & \bf{681.00} & 
1.33\\SCA8-0 & 1007.80 & 1.00 & 
1019.70 & 1.18 & \bf{961.50} & 
4.82\\SCA8-1 & 1101.06 & 1.14 & 
1127.76 & 0.90 & \bf{1049.65} & 
4.90\\SCA8-2 & 1091.71 & 1.00 & 
1119.01 & 0.84 & \bf{1039.64} & 
5.01\\SCA8-3 & 1036.79 & 0.84 & 
1048.63 & 0.95 & \bf{983.34} & 
5.44\\SCA8-4 & 1120.29 & 0.85 & 
1145.07 & 0.95 & \bf{1065.49} & 
5.14\\SCA8-5 & 1090.77 & 1.38 & 
1104.89 & 1.10 & \bf{1027.08} & 
6.20\\SCA8-6 & 1022.17 & 1.25 & 
1039.74 & 1.14 & \bf{971.82} & 
5.18\\SCA8-7 & 1081.50 & 0.96 & 
1095.93 & 1.33 & \bf{1051.28} & 
2.87\\SCA8-8 & 1099.22 & 1.46 & 
1124.08 & 1.13 & \bf{1071.18} & 
2.62\\SCA8-9 & 1126.82 & 1.39 & 
1145.95 & 1.06 & \bf{1060.50} & 
6.25\\CON3-0 & 633.24 & 1.22 & 
650.21 & 1.34 & \bf{616.52} & 
2.71\\CON3-1 & 568.89 & 1.62 & 
576.41 & 1.34 & \bf{554.47} & 
2.60\\CON3-2 & 521.38 & 1.32 & 
525.77 & 1.45 & \bf{518.00} & 
0.65\\CON3-3 & 594.31 & 1.28 & 
608.52 & 1.45 & \bf{591.19} & 
0.53\\CON3-4 & 605.94 & 1.79 & 
612.84 & 1.55 & \bf{588.79} & 
2.91\\CON3-5 & 568.69 & 1.00 & 
585.65 & 1.09 & \bf{563.70} & 
0.89\\CON3-6 & 516.86 & 1.35 & 
524.24 & 1.34 & \bf{499.05} & 
3.57\\CON3-7 & 599.51 & 1.00 & 
609.58 & 1.18 & \bf{576.48} & 
3.99\\CON3-8 & 526.59 & 1.52 & 
535.61 & 1.49 & \bf{523.05} & 
0.68\\CON3-9 & 591.24 & 1.74 & 
599.07 & 1.54 & \bf{578.24} & 
2.25\\CON8-0 & 907.51 & 1.12 & 
931.72 & 1.25 & \bf{857.17} & 
5.87\\CON8-1 & 786.81 & 0.98 & 
794.82 & 0.91 & \bf{740.85} & 
6.20\\CON8-2 & 727.20 & 1.18 & 
764.16 & 1.00 & \bf{712.89} & 
2.01\\CON8-3 & 850.32 & 1.03 & 
853.11 & 1.19 & \bf{811.07} & 
4.84\\CON8-4 & 827.58 & 0.72 & 
844.98 & 0.99 & \bf{772.25} & 
7.16\\CON8-5 & 785.71 & 0.92 & 
814.90 & 1.08 & \bf{754.88} & 
4.08\\CON8-6 & 702.18 & 1.22 & 
719.81 & 1.19 & \bf{678.92} & 
3.43\\CON8-7 & 842.21 & 1.27 & 
865.39 & 0.93 & \bf{811.96} & 
3.73\\CON8-8 & 790.87 & 1.02 & 
819.07 & 0.92 & \bf{767.53} & 
3.04\\CON8-9 & 815.58 & 1.51 & 
872.30 & 1.02 & \bf{809.00} & 
0.81\\\bf{PROM.} & 
\bf{784.05} & \bf{1.26} & \bf{799.42} & \bf{1.21} & \bf{758.54} & \bf{3.08}\\[1ex]\hline
\end{tabular}
\label{table:nonlin}
\end{table} \clearpage
\begin{table}[ht]
\caption{Resultados de la ejecución de la metaheurística ILS, utilizando instancias de Dethloff con la configuración -n 5.0 -LS 80.0}
\centering
\small
\begin{tabular}{c c c c c c c}
\hline\hline
Instancia & Costo mínimo & Tiempo(seg.) & Costo promedio & Tiempo promedio(seg.) & Costo ILS & \%Gap \\ [0.5ex]
\hline
SCA3-0 & 644.06 & 1.54 & 
645.09 & 1.52 & \bf{635.62} & 
1.33\\SCA3-1 & 707.56 & 1.64 & 
714.15 & 1.53 & \bf{697.84} & 
1.39\\SCA3-2 & 664.18 & 1.47 & 
674.73 & 1.49 & \bf{659.34} & 
0.73\\SCA3-3 & 681.74 & 1.55 & 
690.89 & 1.70 & \bf{680.04} & 
0.25\\SCA3-4 & 693.23 & 1.24 & 
704.35 & 1.40 & \bf{690.50} & 
0.40\\SCA3-5 & \bf{659.90} & 1.24 & 
680.93 & 1.52 & 659.90 & 0.00\\
SCA3-6 & 661.52 & 1.58 & 
665.64 & 1.55 & \bf{651.09} & 
1.60\\SCA3-7 & 669.89 & 1.28 & 
673.65 & 1.35 & \bf{659.17} & 
1.63\\SCA3-8 & 723.99 & 1.69 & 
732.07 & 1.50 & \bf{719.47} & 
0.63\\SCA3-9 & 687.61 & 1.77 & 
699.06 & 1.48 & \bf{681.00} & 
0.97\\SCA8-0 & 1034.57 & 2.26 & 
1052.75 & 1.52 & \bf{961.50} & 
7.60\\SCA8-1 & 1085.51 & 1.08 & 
1102.78 & 1.17 & \bf{1049.65} & 
3.42\\SCA8-2 & 1086.84 & 0.94 & 
1116.02 & 1.12 & \bf{1039.64} & 
4.54\\SCA8-3 & 1013.59 & 1.36 & 
1043.61 & 1.19 & \bf{983.34} & 
3.08\\SCA8-4 & 1071.64 & 1.23 & 
1126.14 & 1.36 & \bf{1065.49} & 
0.58\\SCA8-5 & 1090.21 & 1.68 & 
1092.32 & 1.43 & \bf{1027.08} & 
6.15\\SCA8-6 & 1018.47 & 1.44 & 
1027.58 & 1.35 & \bf{971.82} & 
4.80\\SCA8-7 & 1099.62 & 0.89 & 
1111.05 & 1.32 & \bf{1051.28} & 
4.60\\SCA8-8 & 1097.30 & 1.65 & 
1114.59 & 1.49 & \bf{1071.18} & 
2.44\\SCA8-9 & 1101.47 & 1.20 & 
1104.92 & 1.23 & \bf{1060.50} & 
3.86\\CON3-0 & 640.11 & 1.52 & 
648.70 & 1.58 & \bf{616.52} & 
3.83\\CON3-1 & 560.75 & 2.17 & 
568.92 & 1.50 & \bf{554.47} & 
1.13\\CON3-2 & 521.63 & 1.37 & 
527.35 & 1.36 & \bf{518.00} & 
0.70\\CON3-3 & 599.26 & 1.22 & 
603.44 & 1.77 & \bf{591.19} & 
1.37\\CON3-4 & 605.94 & 1.91 & 
613.23 & 1.50 & \bf{588.79} & 
2.91\\CON3-5 & 564.88 & 1.15 & 
590.86 & 1.49 & \bf{563.70} & 
0.21\\CON3-6 & 517.70 & 1.49 & 
524.11 & 1.48 & \bf{499.05} & 
3.74\\CON3-7 & 586.01 & 1.25 & 
593.00 & 1.31 & \bf{576.48} & 
1.65\\CON3-8 & 535.32 & 1.60 & 
553.09 & 1.72 & \bf{523.05} & 
2.35\\CON3-9 & 594.95 & 1.21 & 
597.79 & 1.58 & \bf{578.24} & 
2.89\\CON8-0 & 879.18 & 0.83 & 
901.95 & 0.99 & \bf{857.17} & 
2.57\\CON8-1 & 772.74 & 2.88 & 
789.19 & 1.67 & \bf{740.85} & 
4.30\\CON8-2 & 722.68 & 1.02 & 
739.51 & 1.19 & \bf{712.89} & 
1.37\\CON8-3 & 831.49 & 1.02 & 
849.96 & 1.11 & \bf{811.07} & 
2.52\\CON8-4 & 819.51 & 1.23 & 
855.20 & 1.07 & \bf{772.25} & 
6.12\\CON8-5 & 783.36 & 1.31 & 
801.02 & 1.26 & \bf{754.88} & 
3.77\\CON8-6 & 710.73 & 1.37 & 
731.11 & 1.26 & \bf{678.92} & 
4.69\\CON8-7 & 828.54 & 1.09 & 
860.23 & 1.11 & \bf{811.96} & 
2.04\\CON8-8 & 779.43 & 2.26 & 
800.21 & 1.36 & \bf{767.53} & 
1.55\\CON8-9 & 821.72 & 1.88 & 
857.25 & 1.24 & \bf{809.00} & 
1.57\\\bf{PROM.} & 
\bf{779.22} & \bf{1.46} & \bf{794.46} & \bf{1.39} & \bf{758.54} & \bf{2.53}\\[1ex]\hline
\end{tabular}
\label{table:nonlin}
\end{table} \clearpage
\begin{table}[ht]
\caption{Resultados de la ejecución de la metaheurística ILS, utilizando instancias de Dethloff con la configuración -n 55.0 -LS 10.0}
\centering
\small
\begin{tabular}{c c c c c c c}
\hline\hline
Instancia & Costo mínimo & Tiempo(seg.) & Costo promedio & Tiempo promedio(seg.) & Costo ILS & \%Gap \\ [0.5ex]
\hline
SCA3-0 & 640.55 & 4.11 & 
644.64 & 3.42 & \bf{635.62} & 
0.78\\SCA3-1 & \bf{697.84} & 3.50 & 
703.73 & 3.81 & 697.84 & 0.00\\
SCA3-2 & 661.13 & 3.57 & 
667.09 & 3.41 & \bf{659.34} & 
0.27\\SCA3-3 & 680.60 & 3.58 & 
681.16 & 3.68 & \bf{680.04} & 
0.08\\SCA3-4 & \bf{690.50} & 4.93 & 
700.65 & 3.58 & 690.50 & 0.00\\
SCA3-5 & 673.39 & 3.49 & 
677.86 & 3.65 & \bf{659.90} & 
2.04\\SCA3-6 & 652.94 & 2.92 & 
656.54 & 3.15 & \bf{651.09} & 
0.28\\SCA3-7 & 671.67 & 3.56 & 
671.86 & 3.32 & \bf{659.17} & 
1.90\\SCA3-8 & 719.77 & 3.36 & 
723.80 & 3.38 & \bf{719.47} & 
0.04\\SCA3-9 & 685.00 & 3.43 & 
688.84 & 3.44 & \bf{681.00} & 
0.59\\SCA8-0 & 977.93 & 3.46 & 
1015.19 & 3.50 & \bf{961.50} & 
1.71\\SCA8-1 & 1081.72 & 3.30 & 
1091.99 & 3.13 & \bf{1049.65} & 
3.06\\SCA8-2 & 1065.43 & 3.08 & 
1071.13 & 2.87 & \bf{1039.64} & 
2.48\\SCA8-3 & 1030.78 & 3.44 & 
1056.09 & 3.18 & \bf{983.34} & 
4.82\\SCA8-4 & 1095.76 & 2.78 & 
1127.76 & 2.86 & \bf{1065.49} & 
2.84\\SCA8-5 & 1078.65 & 3.86 & 
1092.81 & 3.31 & \bf{1027.08} & 
5.02\\SCA8-6 & 1005.01 & 3.79 & 
1022.01 & 3.42 & \bf{971.82} & 
3.42\\SCA8-7 & 1088.59 & 2.77 & 
1100.14 & 2.92 & \bf{1051.28} & 
3.55\\SCA8-8 & 1097.68 & 3.15 & 
1104.15 & 3.08 & \bf{1071.18} & 
2.47\\SCA8-9 & 1104.83 & 2.53 & 
1124.07 & 2.92 & \bf{1060.50} & 
4.18\\CON3-0 & 625.35 & 3.64 & 
633.19 & 3.58 & \bf{616.52} & 
1.43\\CON3-1 & 561.63 & 3.56 & 
563.12 & 3.64 & \bf{554.47} & 
1.29\\CON3-2 & 521.38 & 4.09 & 
528.34 & 3.50 & \bf{518.00} & 
0.65\\CON3-3 & 594.31 & 3.43 & 
606.44 & 3.39 & \bf{591.19} & 
0.53\\CON3-4 & 591.43 & 3.64 & 
608.94 & 3.85 & \bf{588.79} & 
0.45\\CON3-5 & 569.04 & 4.24 & 
572.44 & 3.81 & \bf{563.70} & 
0.95\\CON3-6 & 511.05 & 3.74 & 
514.37 & 3.45 & \bf{499.05} & 
2.40\\CON3-7 & 576.87 & 3.21 & 
584.77 & 3.25 & \bf{576.48} & 
0.07\\CON3-8 & \bf{523.05} & 3.39 & 
531.55 & 3.42 & 523.05 & 0.00\\
CON3-9 & 590.50 & 4.64 & 
592.45 & 3.74 & \bf{578.24} & 
2.12\\CON8-0 & 893.61 & 2.79 & 
903.69 & 2.86 & \bf{857.17} & 
4.25\\CON8-1 & 763.99 & 3.64 & 
768.45 & 3.40 & \bf{740.85} & 
3.12\\CON8-2 & 731.85 & 3.85 & 
739.77 & 3.60 & \bf{712.89} & 
2.66\\CON8-3 & 833.78 & 3.62 & 
840.26 & 3.41 & \bf{811.07} & 
2.80\\CON8-4 & 795.93 & 3.24 & 
808.00 & 3.21 & \bf{772.25} & 
3.07\\CON8-5 & 772.89 & 3.51 & 
776.65 & 3.23 & \bf{754.88} & 
2.39\\CON8-6 & 699.79 & 3.42 & 
707.08 & 3.23 & \bf{678.92} & 
3.07\\CON8-7 & 815.60 & 2.99 & 
839.10 & 3.34 & \bf{811.96} & 
0.45\\CON8-8 & 784.36 & 3.70 & 
801.54 & 3.12 & \bf{767.53} & 
2.19\\CON8-9 & 838.80 & 3.33 & 
842.97 & 3.44 & \bf{809.00} & 
3.68\\\bf{PROM.} & 
\bf{774.87} & \bf{3.51} & \bf{784.62} & \bf{3.36} & \bf{758.54} & \bf{1.93}\\[1ex]\hline
\end{tabular}
\label{table:nonlin}
\end{table} \clearpage
\begin{table}[ht]
\caption{Resultados de la ejecución de la metaheurística ILS, utilizando instancias de Dethloff con la configuración -n 55.0 -LS 20.0}
\centering
\small
\begin{tabular}{c c c c c c c}
\hline\hline
Instancia & Costo mínimo & Tiempo(seg.) & Costo promedio & Tiempo promedio(seg.) & Costo ILS & \%Gap \\ [0.5ex]
\hline
SCA3-0 & 640.55 & 5.47 & 
641.42 & 5.44 & \bf{635.62} & 
0.78\\SCA3-1 & 700.50 & 5.26 & 
704.36 & 5.19 & \bf{697.84} & 
0.38\\SCA3-2 & \bf{659.34} & 3.44 & 
663.75 & 4.50 & 659.34 & 0.00\\
SCA3-3 & 680.60 & 5.82 & 
681.24 & 5.53 & \bf{680.04} & 
0.08\\SCA3-4 & \bf{690.50} & 5.52 & 
697.56 & 4.97 & 690.50 & 0.00\\
SCA3-5 & \bf{659.90} & 6.86 & 
679.01 & 5.42 & 659.90 & 0.00\\
SCA3-6 & \bf{651.09} & 5.50 & 
652.35 & 4.92 & 651.09 & 0.00\\
SCA3-7 & 667.24 & 4.70 & 
670.51 & 5.16 & \bf{659.17} & 
1.22\\SCA3-8 & 723.99 & 9.27 & 
728.17 & 6.05 & \bf{719.47} & 
0.63\\SCA3-9 & \bf{681.00} & 5.68 & 
689.72 & 5.22 & 681.00 & 0.00\\
SCA8-0 & 1006.52 & 5.79 & 
1032.53 & 5.54 & \bf{961.50} & 
4.68\\SCA8-1 & 1080.32 & 4.30 & 
1093.20 & 4.41 & \bf{1049.65} & 
2.92\\SCA8-2 & 1056.87 & 5.04 & 
1068.80 & 4.63 & \bf{1039.64} & 
1.66\\SCA8-3 & 1019.35 & 4.35 & 
1036.38 & 4.56 & \bf{983.34} & 
3.66\\SCA8-4 & 1088.71 & 3.94 & 
1110.86 & 4.30 & \bf{1065.49} & 
2.18\\SCA8-5 & 1064.02 & 4.90 & 
1071.52 & 5.00 & \bf{1027.08} & 
3.60\\SCA8-6 & 1001.19 & 3.08 & 
1011.74 & 4.36 & \bf{971.82} & 
3.02\\SCA8-7 & 1070.53 & 4.31 & 
1090.37 & 4.74 & \bf{1051.28} & 
1.83\\SCA8-8 & 1097.63 & 5.26 & 
1098.88 & 4.67 & \bf{1071.18} & 
2.47\\SCA8-9 & 1083.28 & 4.59 & 
1093.26 & 4.04 & \bf{1060.50} & 
2.15\\CON3-0 & 630.73 & 5.95 & 
635.34 & 5.18 & \bf{616.52} & 
2.30\\CON3-1 & 560.75 & 6.38 & 
560.75 & 6.10 & \bf{554.47} & 
1.13\\CON3-2 & 521.38 & 6.72 & 
526.02 & 5.67 & \bf{518.00} & 
0.65\\CON3-3 & \bf{591.19} & 6.44 & 
594.50 & 5.23 & 591.19 & 0.00\\
CON3-4 & 593.78 & 5.40 & 
601.78 & 4.94 & \bf{588.79} & 
0.85\\CON3-5 & 575.00 & 6.17 & 
577.63 & 4.87 & \bf{563.70} & 
2.00\\CON3-6 & 502.85 & 5.77 & 
507.73 & 5.37 & \bf{499.05} & 
0.76\\CON3-7 & 578.41 & 5.40 & 
587.29 & 5.08 & \bf{576.48} & 
0.33\\CON3-8 & 527.82 & 5.44 & 
531.75 & 5.44 & \bf{523.05} & 
0.91\\CON3-9 & 588.40 & 5.29 & 
590.61 & 5.21 & \bf{578.24} & 
1.76\\CON8-0 & 905.02 & 4.44 & 
921.03 & 4.33 & \bf{857.17} & 
5.58\\CON8-1 & 754.51 & 4.96 & 
786.11 & 4.53 & \bf{740.85} & 
1.84\\CON8-2 & 730.92 & 5.13 & 
735.01 & 5.24 & \bf{712.89} & 
2.53\\CON8-3 & 829.24 & 5.10 & 
846.52 & 4.72 & \bf{811.07} & 
2.24\\CON8-4 & 807.41 & 4.15 & 
819.18 & 4.79 & \bf{772.25} & 
4.55\\CON8-5 & 765.73 & 5.44 & 
776.59 & 5.22 & \bf{754.88} & 
1.44\\CON8-6 & 697.71 & 5.24 & 
704.56 & 5.09 & \bf{678.92} & 
2.77\\CON8-7 & 815.14 & 5.34 & 
835.88 & 4.62 & \bf{811.96} & 
0.39\\CON8-8 & 791.13 & 4.86 & 
802.87 & 5.07 & \bf{767.53} & 
3.07\\CON8-9 & 835.01 & 5.50 & 
847.93 & 4.66 & \bf{809.00} & 
3.22\\\bf{PROM.} & 
\bf{773.13} & \bf{5.31} & \bf{782.62} & \bf{5.00} & \bf{758.54} & \bf{1.74}\\[1ex]\hline
\end{tabular}
\label{table:nonlin}
\end{table} \clearpage
\begin{table}[ht]
\caption{Resultados de la ejecución de la metaheurística ILS, utilizando instancias de Dethloff con la configuración -n 55.0 -LS 30.0}
\centering
\small
\begin{tabular}{c c c c c c c}
\hline\hline
Instancia & Costo mínimo & Tiempo(seg.) & Costo promedio & Tiempo promedio(seg.) & Costo ILS & \%Gap \\ [0.5ex]
\hline
SCA3-0 & 640.55 & 7.97 & 
641.89 & 7.56 & \bf{635.62} & 
0.78\\SCA3-1 & 701.53 & 6.86 & 
705.25 & 7.32 & \bf{697.84} & 
0.53\\SCA3-2 & \bf{659.34} & 7.30 & 
664.85 & 6.95 & 659.34 & 0.00\\
SCA3-3 & 681.16 & 7.56 & 
683.93 & 7.66 & \bf{680.04} & 
0.16\\SCA3-4 & \bf{690.50} & 7.87 & 
692.38 & 6.91 & 690.50 & 0.00\\
SCA3-5 & 673.39 & 5.64 & 
681.28 & 6.39 & \bf{659.90} & 
2.04\\SCA3-6 & 653.93 & 8.48 & 
658.38 & 6.42 & \bf{651.09} & 
0.44\\SCA3-7 & 671.67 & 6.65 & 
671.83 & 6.79 & \bf{659.17} & 
1.90\\SCA3-8 & 721.45 & 6.12 & 
727.19 & 6.71 & \bf{719.47} & 
0.28\\SCA3-9 & \bf{681.00} & 7.01 & 
688.71 & 6.57 & 681.00 & 0.00\\
SCA8-0 & 985.54 & 5.99 & 
1002.95 & 6.47 & \bf{961.50} & 
2.50\\SCA8-1 & 1077.98 & 5.79 & 
1085.13 & 5.71 & \bf{1049.65} & 
2.70\\SCA8-2 & 1064.50 & 4.90 & 
1072.19 & 5.86 & \bf{1039.64} & 
2.39\\SCA8-3 & 1025.28 & 5.53 & 
1041.36 & 5.29 & \bf{983.34} & 
4.27\\SCA8-4 & 1071.86 & 7.95 & 
1092.26 & 6.39 & \bf{1065.49} & 
0.60\\SCA8-5 & 1065.27 & 6.38 & 
1075.92 & 6.32 & \bf{1027.08} & 
3.72\\SCA8-6 & 993.42 & 6.02 & 
997.07 & 6.35 & \bf{971.82} & 
2.22\\SCA8-7 & 1074.37 & 7.42 & 
1087.78 & 6.53 & \bf{1051.28} & 
2.20\\SCA8-8 & 1080.58 & 4.89 & 
1097.07 & 6.32 & \bf{1071.18} & 
0.88\\SCA8-9 & 1079.39 & 7.70 & 
1095.62 & 6.51 & \bf{1060.50} & 
1.78\\CON3-0 & 637.14 & 6.36 & 
644.06 & 6.43 & \bf{616.52} & 
3.34\\CON3-1 & 560.75 & 7.06 & 
567.07 & 6.78 & \bf{554.47} & 
1.13\\CON3-2 & 521.63 & 6.40 & 
527.90 & 6.99 & \bf{518.00} & 
0.70\\CON3-3 & 597.34 & 8.67 & 
604.41 & 7.74 & \bf{591.19} & 
1.04\\CON3-4 & \bf{588.79} & 5.98 & 
600.12 & 6.06 & 588.79 & 0.00\\
CON3-5 & 564.88 & 7.69 & 
572.37 & 7.26 & \bf{563.70} & 
0.21\\CON3-6 & 503.97 & 6.97 & 
510.48 & 6.91 & \bf{499.05} & 
0.99\\CON3-7 & 578.22 & 6.19 & 
582.58 & 6.49 & \bf{576.48} & 
0.30\\CON3-8 & 524.30 & 7.18 & 
527.67 & 6.79 & \bf{523.05} & 
0.24\\CON3-9 & 588.40 & 6.99 & 
590.86 & 6.84 & \bf{578.24} & 
1.76\\CON8-0 & 887.49 & 6.06 & 
905.71 & 6.08 & \bf{857.17} & 
3.54\\CON8-1 & 754.65 & 7.29 & 
768.85 & 6.36 & \bf{740.85} & 
1.86\\CON8-2 & 729.05 & 6.39 & 
736.52 & 6.40 & \bf{712.89} & 
2.27\\CON8-3 & 818.41 & 6.13 & 
830.07 & 5.92 & \bf{811.07} & 
0.90\\CON8-4 & 803.08 & 6.90 & 
814.53 & 6.54 & \bf{772.25} & 
3.99\\CON8-5 & 762.34 & 6.60 & 
777.25 & 6.06 & \bf{754.88} & 
0.99\\CON8-6 & 694.03 & 5.56 & 
708.62 & 5.98 & \bf{678.92} & 
2.23\\CON8-7 & 826.35 & 6.37 & 
839.89 & 5.96 & \bf{811.96} & 
1.77\\CON8-8 & 789.64 & 7.10 & 
806.24 & 5.93 & \bf{767.53} & 
2.88\\CON8-9 & 823.63 & 5.42 & 
830.94 & 6.11 & \bf{809.00} & 
1.81\\\bf{PROM.} & 
\bf{771.17} & \bf{6.68} & \bf{780.23} & \bf{6.52} & \bf{758.54} & \bf{1.53}\\[1ex]\hline
\end{tabular}
\label{table:nonlin}
\end{table} \clearpage
\begin{table}[ht]
\caption{Resultados de la ejecución de la metaheurística ILS, utilizando instancias de Dethloff con la configuración -n 55.0 -LS 40.0}
\centering
\small
\begin{tabular}{c c c c c c c}
\hline\hline
Instancia & Costo mínimo & Tiempo(seg.) & Costo promedio & Tiempo promedio(seg.) & Costo ILS & \%Gap \\ [0.5ex]
\hline
SCA3-0 & 640.55 & 8.78 & 
643.15 & 8.81 & \bf{635.62} & 
0.78\\SCA3-1 & 700.50 & 9.12 & 
704.36 & 8.38 & \bf{697.84} & 
0.38\\SCA3-2 & 664.18 & 10.02 & 
668.66 & 8.59 & \bf{659.34} & 
0.73\\SCA3-3 & \bf{680.04} & 7.87 & 
681.21 & 7.90 & 680.04 & 0.00\\
SCA3-4 & \bf{690.50} & 7.56 & 
697.88 & 7.89 & 690.50 & 0.00\\
SCA3-5 & 666.67 & 8.84 & 
673.11 & 9.06 & \bf{659.90} & 
1.03\\SCA3-6 & \bf{651.09} & 8.83 & 
654.19 & 8.28 & 651.09 & 0.00\\
SCA3-7 & 671.67 & 8.18 & 
671.72 & 8.79 & \bf{659.17} & 
1.90\\SCA3-8 & 719.77 & 7.76 & 
723.08 & 8.57 & \bf{719.47} & 
0.04\\SCA3-9 & \bf{681.00} & 9.16 & 
682.65 & 8.60 & 681.00 & 0.00\\
SCA8-0 & 1013.72 & 6.68 & 
1024.13 & 7.82 & \bf{961.50} & 
5.43\\SCA8-1 & 1062.35 & 6.65 & 
1083.20 & 6.86 & \bf{1049.65} & 
1.21\\SCA8-2 & 1055.65 & 6.85 & 
1066.67 & 6.75 & \bf{1039.64} & 
1.54\\SCA8-3 & 1017.97 & 6.73 & 
1025.57 & 7.32 & \bf{983.34} & 
3.52\\SCA8-4 & 1081.36 & 6.94 & 
1089.46 & 6.54 & \bf{1065.49} & 
1.49\\SCA8-5 & 1082.58 & 7.09 & 
1088.39 & 7.20 & \bf{1027.08} & 
5.40\\SCA8-6 & 981.41 & 6.97 & 
994.52 & 7.22 & \bf{971.82} & 
0.99\\SCA8-7 & 1082.76 & 7.40 & 
1090.41 & 7.30 & \bf{1051.28} & 
2.99\\SCA8-8 & \bf{1071.18} & 8.02 & 
1088.61 & 7.58 & 1071.18 & 0.00\\
SCA8-9 & 1079.82 & 6.39 & 
1101.23 & 6.53 & \bf{1060.50} & 
1.82\\CON3-0 & 633.86 & 8.36 & 
636.21 & 9.04 & \bf{616.52} & 
2.81\\CON3-1 & 558.16 & 8.55 & 
560.54 & 9.83 & \bf{554.47} & 
0.67\\CON3-2 & 521.38 & 8.41 & 
522.10 & 7.96 & \bf{518.00} & 
0.65\\CON3-3 & 591.20 & 9.60 & 
598.81 & 9.09 & \bf{591.19} & 
0.00\\CON3-4 & 595.40 & 7.70 & 
604.52 & 8.34 & \bf{588.79} & 
1.12\\CON3-5 & 567.94 & 8.63 & 
573.34 & 9.02 & \bf{563.70} & 
0.75\\CON3-6 & 504.09 & 9.22 & 
508.40 & 8.63 & \bf{499.05} & 
1.01\\CON3-7 & 592.52 & 8.98 & 
594.85 & 8.53 & \bf{576.48} & 
2.78\\CON3-8 & 523.14 & 9.52 & 
528.90 & 9.29 & \bf{523.05} & 
0.02\\CON3-9 & 588.99 & 7.36 & 
589.75 & 8.24 & \bf{578.24} & 
1.86\\CON8-0 & 876.12 & 7.94 & 
902.92 & 7.37 & \bf{857.17} & 
2.21\\CON8-1 & 754.51 & 7.00 & 
769.29 & 7.60 & \bf{740.85} & 
1.84\\CON8-2 & 727.18 & 8.20 & 
735.67 & 7.56 & \bf{712.89} & 
2.00\\CON8-3 & 824.03 & 8.44 & 
836.62 & 7.91 & \bf{811.07} & 
1.60\\CON8-4 & 781.78 & 7.02 & 
802.40 & 7.04 & \bf{772.25} & 
1.23\\CON8-5 & 758.84 & 7.16 & 
771.14 & 7.08 & \bf{754.88} & 
0.52\\CON8-6 & 701.20 & 9.42 & 
707.88 & 7.79 & \bf{678.92} & 
3.28\\CON8-7 & 822.22 & 7.32 & 
841.55 & 7.63 & \bf{811.96} & 
1.26\\CON8-8 & 782.40 & 8.67 & 
795.62 & 7.43 & \bf{767.53} & 
1.94\\CON8-9 & 826.47 & 6.77 & 
831.01 & 7.45 & \bf{809.00} & 
2.16\\\bf{PROM.} & 
\bf{770.66} & \bf{8.00} & \bf{779.09} & \bf{7.97} & \bf{758.54} & \bf{1.47}\\[1ex]\hline
\end{tabular}
\label{table:nonlin}
\end{table} \clearpage
\begin{table}[ht]
\caption{Resultados de la ejecución de la metaheurística ILS, utilizando instancias de Dethloff con la configuración -n 55.0 -LS 50.0}
\centering
\small
\begin{tabular}{c c c c c c c}
\hline\hline
Instancia & Costo mínimo & Tiempo(seg.) & Costo promedio & Tiempo promedio(seg.) & Costo ILS & \%Gap \\ [0.5ex]
\hline
SCA3-0 & 636.06 & 10.10 & 
640.22 & 10.74 & \bf{635.62} & 
0.07\\SCA3-1 & \bf{697.84} & 9.73 & 
701.19 & 10.60 & 697.84 & 0.00\\
SCA3-2 & 664.18 & 10.53 & 
666.25 & 10.26 & \bf{659.34} & 
0.73\\SCA3-3 & \bf{680.04} & 11.49 & 
680.65 & 10.58 & 680.04 & 0.00\\
SCA3-4 & \bf{690.50} & 10.00 & 
691.18 & 10.50 & 690.50 & 0.00\\
SCA3-5 & 673.39 & 10.95 & 
677.82 & 10.55 & \bf{659.90} & 
2.04\\SCA3-6 & \bf{651.09} & 11.07 & 
653.19 & 11.13 & 651.09 & 0.00\\
SCA3-7 & 671.67 & 10.86 & 
671.72 & 10.06 & \bf{659.17} & 
1.90\\SCA3-8 & \bf{719.47} & 10.30 & 
721.21 & 11.45 & 719.47 & 0.00\\
SCA3-9 & \bf{681.00} & 10.93 & 
682.00 & 9.40 & 681.00 & 0.00\\
SCA8-0 & 975.32 & 11.08 & 
991.12 & 9.68 & \bf{961.50} & 
1.44\\SCA8-1 & 1054.81 & 9.39 & 
1073.76 & 7.95 & \bf{1049.65} & 
0.49\\SCA8-2 & 1063.96 & 9.21 & 
1067.40 & 8.95 & \bf{1039.64} & 
2.34\\SCA8-3 & 1026.39 & 11.20 & 
1034.57 & 8.74 & \bf{983.34} & 
4.38\\SCA8-4 & \bf{1065.49} & 8.71 & 
1090.01 & 7.61 & 1065.49 & 0.00\\
SCA8-5 & 1071.34 & 9.38 & 
1077.42 & 8.61 & \bf{1027.08} & 
4.31\\SCA8-6 & 986.90 & 11.76 & 
998.96 & 9.02 & \bf{971.82} & 
1.55\\SCA8-7 & 1075.36 & 8.38 & 
1090.14 & 8.86 & \bf{1051.28} & 
2.29\\SCA8-8 & 1082.11 & 9.59 & 
1097.35 & 8.37 & \bf{1071.18} & 
1.02\\SCA8-9 & 1087.05 & 7.41 & 
1093.12 & 7.78 & \bf{1060.50} & 
2.50\\CON3-0 & 629.51 & 10.11 & 
634.29 & 9.87 & \bf{616.52} & 
2.11\\CON3-1 & 560.75 & 10.54 & 
564.29 & 10.02 & \bf{554.47} & 
1.13\\CON3-2 & 521.38 & 9.42 & 
522.86 & 10.15 & \bf{518.00} & 
0.65\\CON3-3 & 591.20 & 10.25 & 
602.09 & 10.74 & \bf{591.19} & 
0.00\\CON3-4 & 591.43 & 11.83 & 
598.80 & 10.91 & \bf{588.79} & 
0.45\\CON3-5 & \bf{563.70} & 10.27 & 
569.73 & 10.79 & 563.70 & 0.00\\
CON3-6 & 502.16 & 10.32 & 
507.19 & 9.93 & \bf{499.05} & 
0.62\\CON3-7 & 577.68 & 10.35 & 
585.18 & 11.73 & \bf{576.48} & 
0.21\\CON3-8 & 524.59 & 11.52 & 
527.77 & 11.14 & \bf{523.05} & 
0.29\\CON3-9 & 588.40 & 10.00 & 
589.30 & 10.97 & \bf{578.24} & 
1.76\\CON8-0 & 866.11 & 8.04 & 
902.32 & 7.87 & \bf{857.17} & 
1.04\\CON8-1 & 766.65 & 11.36 & 
776.47 & 9.36 & \bf{740.85} & 
3.48\\CON8-2 & 722.80 & 8.11 & 
728.42 & 8.71 & \bf{712.89} & 
1.39\\CON8-3 & 826.26 & 10.98 & 
834.29 & 9.15 & \bf{811.07} & 
1.87\\CON8-4 & 794.75 & 8.81 & 
798.76 & 8.60 & \bf{772.25} & 
2.91\\CON8-5 & 763.13 & 9.94 & 
766.25 & 8.94 & \bf{754.88} & 
1.09\\CON8-6 & 696.41 & 9.64 & 
701.60 & 8.96 & \bf{678.92} & 
2.58\\CON8-7 & 814.79 & 9.10 & 
830.68 & 8.74 & \bf{811.96} & 
0.35\\CON8-8 & 783.01 & 8.63 & 
792.99 & 8.62 & \bf{767.53} & 
2.02\\CON8-9 & 827.05 & 7.53 & 
833.80 & 7.96 & \bf{809.00} & 
2.23\\\bf{PROM.} & 
\bf{769.14} & \bf{9.97} & \bf{776.66} & \bf{9.60} & \bf{758.54} & \bf{1.28}\\[1ex]\hline
\end{tabular}
\label{table:nonlin}
\end{table} \clearpage
\begin{table}[ht]
\caption{Resultados de la ejecución de la metaheurística ILS, utilizando instancias de Dethloff con la configuración -n 55.0 -LS 60.0}
\centering
\small
\begin{tabular}{c c c c c c c}
\hline\hline
Instancia & Costo mínimo & Tiempo(seg.) & Costo promedio & Tiempo promedio(seg.) & Costo ILS & \%Gap \\ [0.5ex]
\hline
SCA3-0 & 636.06 & 10.86 & 
639.43 & 11.96 & \bf{635.62} & 
0.07\\SCA3-1 & \bf{697.84} & 13.18 & 
700.15 & 13.08 & 697.84 & 0.00\\
SCA3-2 & 664.21 & 12.73 & 
666.46 & 12.29 & \bf{659.34} & 
0.74\\SCA3-3 & \bf{680.04} & 12.31 & 
680.64 & 13.44 & 680.04 & 0.00\\
SCA3-4 & \bf{690.50} & 11.88 & 
691.18 & 11.36 & 690.50 & 0.00\\
SCA3-5 & \bf{659.90} & 12.92 & 
664.28 & 12.44 & 659.90 & 0.00\\
SCA3-6 & 652.94 & 12.50 & 
655.13 & 12.20 & \bf{651.09} & 
0.28\\SCA3-7 & \bf{659.17} & 13.66 & 
668.10 & 12.36 & 659.17 & 0.00\\
SCA3-8 & 722.05 & 11.83 & 
725.49 & 13.44 & \bf{719.47} & 
0.36\\SCA3-9 & \bf{681.00} & 12.13 & 
683.81 & 11.88 & 681.00 & 0.00\\
SCA8-0 & 990.52 & 9.94 & 
1013.45 & 10.56 & \bf{961.50} & 
3.02\\SCA8-1 & 1074.03 & 9.22 & 
1079.92 & 10.09 & \bf{1049.65} & 
2.32\\SCA8-2 & 1066.60 & 10.31 & 
1073.02 & 9.99 & \bf{1039.64} & 
2.59\\SCA8-3 & 1023.68 & 9.58 & 
1029.70 & 9.56 & \bf{983.34} & 
4.10\\SCA8-4 & 1107.90 & 9.34 & 
1113.82 & 9.54 & \bf{1065.49} & 
3.98\\SCA8-5 & 1060.43 & 9.80 & 
1064.70 & 10.30 & \bf{1027.08} & 
3.25\\SCA8-6 & 983.38 & 10.10 & 
998.47 & 10.23 & \bf{971.82} & 
1.19\\SCA8-7 & 1083.34 & 10.30 & 
1087.56 & 10.65 & \bf{1051.28} & 
3.05\\SCA8-8 & \bf{1071.18} & 8.90 & 
1082.11 & 10.26 & 1071.18 & 0.00\\
SCA8-9 & 1077.21 & 9.24 & 
1087.34 & 8.76 & \bf{1060.50} & 
1.58\\CON3-0 & 625.35 & 12.45 & 
630.31 & 12.37 & \bf{616.52} & 
1.43\\CON3-1 & 556.92 & 12.47 & 
561.46 & 12.18 & \bf{554.47} & 
0.44\\CON3-2 & 521.38 & 13.27 & 
524.48 & 11.73 & \bf{518.00} & 
0.65\\CON3-3 & 591.20 & 13.86 & 
591.92 & 13.18 & \bf{591.19} & 
0.00\\CON3-4 & \bf{588.79} & 13.48 & 
592.96 & 12.93 & 588.79 & 0.00\\
CON3-5 & 570.22 & 10.16 & 
572.96 & 11.78 & \bf{563.70} & 
1.16\\CON3-6 & 502.16 & 10.46 & 
505.13 & 12.55 & \bf{499.05} & 
0.62\\CON3-7 & 577.54 & 12.74 & 
580.67 & 12.45 & \bf{576.48} & 
0.18\\CON3-8 & 523.14 & 13.85 & 
524.73 & 12.46 & \bf{523.05} & 
0.02\\CON3-9 & 578.25 & 11.91 & 
586.42 & 12.33 & \bf{578.24} & 
0.00\\CON8-0 & 883.39 & 11.31 & 
897.05 & 10.39 & \bf{857.17} & 
3.06\\CON8-1 & 754.51 & 10.52 & 
764.58 & 10.13 & \bf{740.85} & 
1.84\\CON8-2 & 723.29 & 9.81 & 
730.03 & 10.30 & \bf{712.89} & 
1.46\\CON8-3 & 820.31 & 10.47 & 
832.65 & 10.36 & \bf{811.07} & 
1.14\\CON8-4 & 799.14 & 12.26 & 
808.67 & 10.70 & \bf{772.25} & 
3.48\\CON8-5 & 776.52 & 10.78 & 
778.13 & 10.36 & \bf{754.88} & 
2.87\\CON8-6 & 692.51 & 11.17 & 
697.33 & 10.54 & \bf{678.92} & 
2.00\\CON8-7 & 816.67 & 9.20 & 
821.47 & 10.27 & \bf{811.96} & 
0.58\\CON8-8 & 785.82 & 10.70 & 
791.37 & 10.49 & \bf{767.53} & 
2.38\\CON8-9 & 813.68 & 9.66 & 
828.39 & 9.85 & \bf{809.00} & 
0.58\\\bf{PROM.} & 
\bf{769.57} & \bf{11.28} & \bf{775.64} & \bf{11.29} & \bf{758.54} & \bf{1.26}\\[1ex]\hline
\end{tabular}
\label{table:nonlin}
\end{table} \clearpage
\begin{table}[ht]
\caption{Resultados de la ejecución de la metaheurística ILS, utilizando instancias de Dethloff con la configuración -n 55.0 -LS 70.0}
\centering
\small
\begin{tabular}{c c c c c c c}
\hline\hline
Instancia & Costo mínimo & Tiempo(seg.) & Costo promedio & Tiempo promedio(seg.) & Costo ILS & \%Gap \\ [0.5ex]
\hline
SCA3-0 & 636.06 & 13.19 & 
639.43 & 13.64 & \bf{635.62} & 
0.07\\SCA3-1 & \bf{697.84} & 14.05 & 
703.68 & 13.92 & 697.84 & 0.00\\
SCA3-2 & 661.13 & 12.55 & 
664.80 & 13.95 & \bf{659.34} & 
0.27\\SCA3-3 & 680.60 & 14.15 & 
680.88 & 13.36 & \bf{680.04} & 
0.08\\SCA3-4 & \bf{690.50} & 16.02 & 
691.18 & 13.95 & 690.50 & 0.00\\
SCA3-5 & 666.67 & 14.45 & 
673.10 & 12.99 & \bf{659.90} & 
1.03\\SCA3-6 & 652.94 & 14.01 & 
655.38 & 13.03 & \bf{651.09} & 
0.28\\SCA3-7 & 669.89 & 13.67 & 
671.93 & 14.20 & \bf{659.17} & 
1.63\\SCA3-8 & \bf{719.47} & 15.05 & 
723.79 & 13.82 & 719.47 & 0.00\\
SCA3-9 & 684.44 & 11.65 & 
685.36 & 14.11 & \bf{681.00} & 
0.51\\SCA8-0 & 990.21 & 11.79 & 
1000.62 & 12.32 & \bf{961.50} & 
2.99\\SCA8-1 & 1070.29 & 10.80 & 
1078.63 & 10.87 & \bf{1049.65} & 
1.97\\SCA8-2 & 1054.47 & 11.95 & 
1061.10 & 11.15 & \bf{1039.64} & 
1.43\\SCA8-3 & 1015.23 & 11.72 & 
1022.01 & 10.76 & \bf{983.34} & 
3.24\\SCA8-4 & 1069.87 & 10.34 & 
1092.29 & 11.25 & \bf{1065.49} & 
0.41\\SCA8-5 & 1052.02 & 14.64 & 
1062.53 & 12.94 & \bf{1027.08} & 
2.43\\SCA8-6 & 990.61 & 13.74 & 
997.28 & 12.02 & \bf{971.82} & 
1.93\\SCA8-7 & 1067.11 & 12.20 & 
1084.92 & 12.30 & \bf{1051.28} & 
1.51\\SCA8-8 & \bf{1071.18} & 12.54 & 
1092.09 & 11.62 & 1071.18 & 0.00\\
SCA8-9 & 1069.83 & 13.35 & 
1078.73 & 11.76 & \bf{1060.50} & 
0.88\\CON3-0 & 620.76 & 13.82 & 
630.55 & 14.34 & \bf{616.52} & 
0.69\\CON3-1 & 556.04 & 13.81 & 
558.62 & 15.12 & \bf{554.47} & 
0.28\\CON3-2 & 521.38 & 15.15 & 
521.57 & 14.56 & \bf{518.00} & 
0.65\\CON3-3 & \bf{591.19} & 15.58 & 
595.30 & 14.44 & 591.19 & 0.00\\
CON3-4 & 592.58 & 13.94 & 
594.49 & 14.19 & \bf{588.79} & 
0.64\\CON3-5 & 569.04 & 13.60 & 
569.35 & 13.67 & \bf{563.70} & 
0.95\\CON3-6 & 502.16 & 13.08 & 
504.39 & 20.38 & \bf{499.05} & 
0.62\\CON3-7 & 578.41 & 13.98 & 
587.20 & 14.02 & \bf{576.48} & 
0.33\\CON3-8 & 523.68 & 16.10 & 
524.51 & 14.98 & \bf{523.05} & 
0.12\\CON3-9 & 588.40 & 14.70 & 
589.51 & 14.59 & \bf{578.24} & 
1.76\\CON8-0 & 875.60 & 11.29 & 
889.83 & 11.27 & \bf{857.17} & 
2.15\\CON8-1 & 746.81 & 10.68 & 
763.05 & 11.83 & \bf{740.85} & 
0.80\\CON8-2 & 723.85 & 14.17 & 
730.02 & 13.56 & \bf{712.89} & 
1.54\\CON8-3 & 813.40 & 13.54 & 
828.90 & 12.70 & \bf{811.07} & 
0.29\\CON8-4 & 803.90 & 11.58 & 
813.55 & 11.63 & \bf{772.25} & 
4.10\\CON8-5 & 762.61 & 12.82 & 
769.04 & 12.23 & \bf{754.88} & 
1.02\\CON8-6 & 683.16 & 11.64 & 
692.35 & 11.73 & \bf{678.92} & 
0.62\\CON8-7 & 815.91 & 10.76 & 
824.77 & 11.31 & \bf{811.96} & 
0.49\\CON8-8 & 779.43 & 11.27 & 
784.75 & 11.41 & \bf{767.53} & 
1.55\\CON8-9 & 812.35 & 13.80 & 
821.01 & 12.31 & \bf{809.00} & 
0.41\\\bf{PROM.} & 
\bf{766.78} & \bf{13.18} & \bf{773.81} & \bf{13.11} & \bf{758.54} & \bf{0.99}\\[1ex]\hline
\end{tabular}
\label{table:nonlin}
\end{table} \clearpage
\begin{table}[ht]
\caption{Resultados de la ejecución de la metaheurística ILS, utilizando instancias de Dethloff con la configuración -n 55.0 -LS 80.0}
\centering
\small
\begin{tabular}{c c c c c c c}
\hline\hline
Instancia & Costo mínimo & Tiempo(seg.) & Costo promedio & Tiempo promedio(seg.) & Costo ILS & \%Gap \\ [0.5ex]
\hline
SCA3-0 & 640.55 & 17.98 & 
642.18 & 16.09 & \bf{635.62} & 
0.78\\SCA3-1 & 701.53 & 14.73 & 
701.53 & 16.10 & \bf{697.84} & 
0.53\\SCA3-2 & \bf{659.34} & 15.01 & 
661.53 & 15.87 & 659.34 & 0.00\\
SCA3-3 & \bf{680.04} & 17.35 & 
683.79 & 16.95 & 680.04 & 0.00\\
SCA3-4 & \bf{690.50} & 16.72 & 
690.50 & 16.24 & 690.50 & 0.00\\
SCA3-5 & \bf{659.90} & 16.29 & 
663.32 & 16.14 & 659.90 & 0.00\\
SCA3-6 & \bf{651.09} & 15.14 & 
655.00 & 15.22 & 651.09 & 0.00\\
SCA3-7 & 667.75 & 16.85 & 
669.75 & 15.72 & \bf{659.17} & 
1.30\\SCA3-8 & \bf{719.47} & 14.32 & 
722.79 & 15.18 & 719.47 & 0.00\\
SCA3-9 & \bf{681.00} & 15.03 & 
683.67 & 14.84 & 681.00 & 0.00\\
SCA8-0 & 989.54 & 11.30 & 
1001.47 & 13.29 & \bf{961.50} & 
2.92\\SCA8-1 & 1066.80 & 11.10 & 
1080.70 & 11.43 & \bf{1049.65} & 
1.63\\SCA8-2 & 1060.09 & 10.30 & 
1069.61 & 12.43 & \bf{1039.64} & 
1.97\\SCA8-3 & 1014.45 & 11.97 & 
1026.39 & 11.82 & \bf{983.34} & 
3.16\\SCA8-4 & 1071.64 & 12.17 & 
1079.84 & 12.81 & \bf{1065.49} & 
0.58\\SCA8-5 & 1038.59 & 12.03 & 
1062.07 & 12.52 & \bf{1027.08} & 
1.12\\SCA8-6 & 987.08 & 13.47 & 
1001.40 & 12.61 & \bf{971.82} & 
1.57\\SCA8-7 & 1073.57 & 16.25 & 
1085.10 & 13.81 & \bf{1051.28} & 
2.12\\SCA8-8 & 1089.91 & 11.45 & 
1096.03 & 11.86 & \bf{1071.18} & 
1.75\\SCA8-9 & 1087.65 & 11.46 & 
1093.37 & 12.03 & \bf{1060.50} & 
2.56\\CON3-0 & 629.09 & 16.05 & 
632.49 & 15.80 & \bf{616.52} & 
2.04\\CON3-1 & 560.75 & 16.00 & 
560.75 & 16.27 & \bf{554.47} & 
1.13\\CON3-2 & 521.38 & 17.00 & 
524.70 & 16.46 & \bf{518.00} & 
0.65\\CON3-3 & \bf{591.19} & 15.08 & 
595.86 & 16.10 & 591.19 & 0.00\\
CON3-4 & 591.43 & 14.68 & 
592.91 & 15.17 & \bf{588.79} & 
0.45\\CON3-5 & 569.57 & 16.06 & 
573.99 & 15.46 & \bf{563.70} & 
1.04\\CON3-6 & 500.80 & 16.77 & 
501.82 & 15.76 & \bf{499.05} & 
0.35\\CON3-7 & 583.65 & 15.84 & 
588.52 & 16.26 & \bf{576.48} & 
1.24\\CON3-8 & 523.68 & 16.84 & 
524.36 & 16.09 & \bf{523.05} & 
0.12\\CON3-9 & 588.40 & 18.68 & 
589.36 & 16.82 & \bf{578.24} & 
1.76\\CON8-0 & 861.87 & 13.24 & 
882.43 & 12.71 & \bf{857.17} & 
0.55\\CON8-1 & 755.23 & 14.04 & 
762.93 & 13.44 & \bf{740.85} & 
1.94\\CON8-2 & 719.17 & 12.69 & 
726.14 & 12.71 & \bf{712.89} & 
0.88\\CON8-3 & 824.04 & 15.20 & 
835.21 & 14.60 & \bf{811.07} & 
1.60\\CON8-4 & 789.20 & 11.21 & 
795.85 & 12.62 & \bf{772.25} & 
2.19\\CON8-5 & 762.61 & 12.37 & 
769.27 & 12.59 & \bf{754.88} & 
1.02\\CON8-6 & 697.44 & 12.73 & 
702.50 & 13.86 & \bf{678.92} & 
2.73\\CON8-7 & 815.72 & 13.00 & 
820.12 & 13.11 & \bf{811.96} & 
0.46\\CON8-8 & 783.39 & 12.36 & 
788.39 & 12.07 & \bf{767.53} & 
2.07\\CON8-9 & 822.18 & 14.30 & 
836.10 & 12.86 & \bf{809.00} & 
1.63\\\bf{PROM.} & 
\bf{768.03} & \bf{14.38} & \bf{774.34} & \bf{14.34} & \bf{758.54} & \bf{1.15}\\[1ex]\hline
\end{tabular}
\label{table:nonlin}
\end{table} \clearpage
\begin{table}[ht]
\caption{Resultados de la ejecución de la metaheurística ILS, utilizando instancias de Dethloff con la configuración -n 65.0 -LS 10.0}
\centering
\small
\begin{tabular}{c c c c c c c}
\hline\hline
Instancia & Costo mínimo & Tiempo(seg.) & Costo promedio & Tiempo promedio(seg.) & Costo ILS & \%Gap \\ [0.5ex]
\hline
SCA3-0 & 640.55 & 4.63 & 
641.89 & 4.11 & \bf{635.62} & 
0.78\\SCA3-1 & 706.90 & 4.20 & 
712.04 & 4.12 & \bf{697.84} & 
1.30\\SCA3-2 & \bf{659.34} & 3.57 & 
667.53 & 3.78 & 659.34 & 0.00\\
SCA3-3 & 680.60 & 3.73 & 
682.36 & 4.03 & \bf{680.04} & 
0.08\\SCA3-4 & \bf{690.50} & 4.40 & 
706.38 & 3.85 & 690.50 & 0.00\\
SCA3-5 & 673.39 & 4.42 & 
677.70 & 3.94 & \bf{659.90} & 
2.04\\SCA3-6 & 652.47 & 4.32 & 
656.18 & 3.85 & \bf{651.09} & 
0.21\\SCA3-7 & 666.15 & 4.15 & 
677.20 & 3.87 & \bf{659.17} & 
1.06\\SCA3-8 & \bf{719.47} & 3.35 & 
724.31 & 4.01 & 719.47 & 0.00\\
SCA3-9 & \bf{681.00} & 3.44 & 
687.06 & 3.90 & 681.00 & 0.00\\
SCA8-0 & 1004.86 & 4.48 & 
1019.69 & 4.37 & \bf{961.50} & 
4.51\\SCA8-1 & 1078.10 & 3.25 & 
1084.90 & 3.76 & \bf{1049.65} & 
2.71\\SCA8-2 & 1064.72 & 3.70 & 
1086.05 & 3.48 & \bf{1039.64} & 
2.41\\SCA8-3 & 1012.77 & 4.21 & 
1032.53 & 3.61 & \bf{983.34} & 
2.99\\SCA8-4 & 1074.63 & 3.95 & 
1098.36 & 3.62 & \bf{1065.49} & 
0.86\\SCA8-5 & 1066.35 & 4.31 & 
1087.46 & 3.75 & \bf{1027.08} & 
3.82\\SCA8-6 & 995.52 & 3.44 & 
1002.61 & 3.58 & \bf{971.82} & 
2.44\\SCA8-7 & 1070.92 & 4.65 & 
1082.83 & 4.14 & \bf{1051.28} & 
1.87\\SCA8-8 & 1096.45 & 3.45 & 
1109.39 & 3.50 & \bf{1071.18} & 
2.36\\SCA8-9 & 1072.76 & 3.93 & 
1087.85 & 3.45 & \bf{1060.50} & 
1.16\\CON3-0 & 632.57 & 3.74 & 
639.32 & 3.71 & \bf{616.52} & 
2.60\\CON3-1 & 560.41 & 4.08 & 
566.12 & 4.09 & \bf{554.47} & 
1.07\\CON3-2 & 521.63 & 4.99 & 
525.05 & 4.08 & \bf{518.00} & 
0.70\\CON3-3 & 610.17 & 3.96 & 
620.03 & 3.90 & \bf{591.19} & 
3.21\\CON3-4 & 593.78 & 4.06 & 
599.50 & 4.12 & \bf{588.79} & 
0.85\\CON3-5 & 569.74 & 4.35 & 
573.81 & 4.38 & \bf{563.70} & 
1.07\\CON3-6 & 502.16 & 4.18 & 
505.48 & 4.30 & \bf{499.05} & 
0.62\\CON3-7 & \bf{576.48} & 3.98 & 
591.05 & 4.01 & 576.48 & 0.00\\
CON3-8 & \bf{523.05} & 4.41 & 
530.57 & 4.32 & 523.05 & 0.00\\
CON3-9 & 590.64 & 3.97 & 
591.90 & 4.07 & \bf{578.24} & 
2.14\\CON8-0 & 904.21 & 4.39 & 
920.71 & 3.69 & \bf{857.17} & 
5.49\\CON8-1 & 769.67 & 3.70 & 
783.31 & 3.79 & \bf{740.85} & 
3.89\\CON8-2 & 720.47 & 4.14 & 
727.52 & 3.93 & \bf{712.89} & 
1.06\\CON8-3 & 830.96 & 4.37 & 
840.90 & 3.70 & \bf{811.07} & 
2.45\\CON8-4 & 802.57 & 3.52 & 
816.54 & 3.62 & \bf{772.25} & 
3.93\\CON8-5 & 780.16 & 3.81 & 
789.60 & 3.69 & \bf{754.88} & 
3.35\\CON8-6 & 689.21 & 4.11 & 
704.38 & 3.92 & \bf{678.92} & 
1.52\\CON8-7 & 815.82 & 3.22 & 
833.12 & 3.54 & \bf{811.96} & 
0.48\\CON8-8 & 769.65 & 3.76 & 
794.77 & 4.00 & \bf{767.53} & 
0.28\\CON8-9 & 817.56 & 3.79 & 
827.68 & 4.15 & \bf{809.00} & 
1.06\\\bf{PROM.} & 
\bf{772.21} & \bf{4.00} & \bf{782.64} & \bf{3.89} & \bf{758.54} & \bf{1.66}\\[1ex]\hline
\end{tabular}
\label{table:nonlin}
\end{table} \clearpage
\begin{table}[ht]
\caption{Resultados de la ejecución de la metaheurística ILS, utilizando instancias de Dethloff con la configuración -n 65.0 -LS 20.0}
\centering
\small
\begin{tabular}{c c c c c c c}
\hline\hline
Instancia & Costo mínimo & Tiempo(seg.) & Costo promedio & Tiempo promedio(seg.) & Costo ILS & \%Gap \\ [0.5ex]
\hline
SCA3-0 & 636.06 & 6.36 & 
641.10 & 6.43 & \bf{635.62} & 
0.07\\SCA3-1 & 701.53 & 7.00 & 
704.21 & 6.88 & \bf{697.84} & 
0.53\\SCA3-2 & \bf{659.34} & 5.45 & 
662.98 & 5.70 & 659.34 & 0.00\\
SCA3-3 & 680.60 & 6.06 & 
685.48 & 6.35 & \bf{680.04} & 
0.08\\SCA3-4 & \bf{690.50} & 6.11 & 
697.60 & 6.24 & 690.50 & 0.00\\
SCA3-5 & 666.67 & 6.43 & 
678.73 & 6.12 & \bf{659.90} & 
1.03\\SCA3-6 & 652.47 & 5.20 & 
654.05 & 5.32 & \bf{651.09} & 
0.21\\SCA3-7 & 671.67 & 4.90 & 
672.07 & 5.95 & \bf{659.17} & 
1.90\\SCA3-8 & 724.29 & 5.08 & 
731.99 & 6.20 & \bf{719.47} & 
0.67\\SCA3-9 & \bf{681.00} & 4.98 & 
684.46 & 5.20 & 681.00 & 0.00\\
SCA8-0 & 1001.48 & 5.66 & 
1009.08 & 6.04 & \bf{961.50} & 
4.16\\SCA8-1 & 1079.07 & 5.02 & 
1083.23 & 5.17 & \bf{1049.65} & 
2.80\\SCA8-2 & 1064.13 & 4.42 & 
1070.38 & 5.08 & \bf{1039.64} & 
2.36\\SCA8-3 & 1022.58 & 5.76 & 
1033.38 & 5.42 & \bf{983.34} & 
3.99\\SCA8-4 & 1074.10 & 5.17 & 
1085.85 & 5.10 & \bf{1065.49} & 
0.81\\SCA8-5 & 1063.39 & 5.78 & 
1073.72 & 5.09 & \bf{1027.08} & 
3.54\\SCA8-6 & 999.81 & 6.43 & 
1006.26 & 5.57 & \bf{971.82} & 
2.88\\SCA8-7 & 1067.49 & 5.37 & 
1082.08 & 5.08 & \bf{1051.28} & 
1.54\\SCA8-8 & \bf{1071.18} & 5.57 & 
1088.37 & 5.14 & 1071.18 & 0.00\\
SCA8-9 & 1104.99 & 5.35 & 
1115.62 & 4.87 & \bf{1060.50} & 
4.20\\CON3-0 & 623.84 & 6.35 & 
631.22 & 6.06 & \bf{616.52} & 
1.19\\CON3-1 & 564.55 & 7.29 & 
569.67 & 6.24 & \bf{554.47} & 
1.82\\CON3-2 & 521.38 & 6.83 & 
525.95 & 5.85 & \bf{518.00} & 
0.65\\CON3-3 & \bf{591.19} & 6.58 & 
595.86 & 6.45 & 591.19 & 0.00\\
CON3-4 & 598.36 & 5.85 & 
601.48 & 5.86 & \bf{588.79} & 
1.63\\CON3-5 & 573.41 & 7.16 & 
580.55 & 6.33 & \bf{563.70} & 
1.72\\CON3-6 & 502.16 & 6.82 & 
507.49 & 6.50 & \bf{499.05} & 
0.62\\CON3-7 & 578.41 & 5.03 & 
590.57 & 6.26 & \bf{576.48} & 
0.33\\CON3-8 & 523.14 & 7.31 & 
524.81 & 6.92 & \bf{523.05} & 
0.02\\CON3-9 & 588.99 & 5.76 & 
589.14 & 6.01 & \bf{578.24} & 
1.86\\CON8-0 & 884.44 & 5.97 & 
889.60 & 5.75 & \bf{857.17} & 
3.18\\CON8-1 & 763.07 & 6.04 & 
767.07 & 5.77 & \bf{740.85} & 
3.00\\CON8-2 & 729.85 & 6.38 & 
739.75 & 5.44 & \bf{712.89} & 
2.38\\CON8-3 & 830.41 & 5.00 & 
841.91 & 5.78 & \bf{811.07} & 
2.38\\CON8-4 & 791.67 & 6.21 & 
806.60 & 5.78 & \bf{772.25} & 
2.51\\CON8-5 & 763.52 & 5.24 & 
780.23 & 5.90 & \bf{754.88} & 
1.14\\CON8-6 & 693.99 & 5.97 & 
705.47 & 5.76 & \bf{678.92} & 
2.22\\CON8-7 & 825.69 & 5.94 & 
832.83 & 5.21 & \bf{811.96} & 
1.69\\CON8-8 & 793.10 & 5.32 & 
797.38 & 5.66 & \bf{767.53} & 
3.33\\CON8-9 & 819.81 & 5.77 & 
833.73 & 5.83 & \bf{809.00} & 
1.34\\\bf{PROM.} & 
\bf{771.83} & \bf{5.87} & \bf{779.30} & \bf{5.81} & \bf{758.54} & \bf{1.59}\\[1ex]\hline
\end{tabular}
\label{table:nonlin}
\end{table} \clearpage
\begin{table}[ht]
\caption{Resultados de la ejecución de la metaheurística ILS, utilizando instancias de Dethloff con la configuración -n 65.0 -LS 30.0}
\centering
\small
\begin{tabular}{c c c c c c c}
\hline\hline
Instancia & Costo mínimo & Tiempo(seg.) & Costo promedio & Tiempo promedio(seg.) & Costo ILS & \%Gap \\ [0.5ex]
\hline
SCA3-0 & 640.55 & 8.51 & 
641.81 & 8.51 & \bf{635.62} & 
0.78\\SCA3-1 & \bf{697.84} & 8.68 & 
705.28 & 8.69 & 697.84 & 0.00\\
SCA3-2 & 664.18 & 7.96 & 
665.78 & 8.03 & \bf{659.34} & 
0.73\\SCA3-3 & 680.60 & 8.15 & 
683.67 & 8.17 & \bf{680.04} & 
0.08\\SCA3-4 & \bf{690.50} & 9.32 & 
691.18 & 8.29 & 690.50 & 0.00\\
SCA3-5 & 670.10 & 9.13 & 
678.20 & 8.46 & \bf{659.90} & 
1.55\\SCA3-6 & 652.94 & 8.74 & 
654.39 & 8.03 & \bf{651.09} & 
0.28\\SCA3-7 & 671.77 & 7.52 & 
672.95 & 7.85 & \bf{659.17} & 
1.91\\SCA3-8 & \bf{719.47} & 7.51 & 
727.72 & 7.67 & 719.47 & 0.00\\
SCA3-9 & \bf{681.00} & 8.45 & 
686.52 & 7.71 & 681.00 & 0.00\\
SCA8-0 & 993.77 & 7.10 & 
1008.65 & 7.03 & \bf{961.50} & 
3.36\\SCA8-1 & 1071.11 & 6.81 & 
1083.19 & 7.09 & \bf{1049.65} & 
2.04\\SCA8-2 & 1055.32 & 6.83 & 
1061.02 & 6.75 & \bf{1039.64} & 
1.51\\SCA8-3 & 1033.18 & 8.51 & 
1039.60 & 7.14 & \bf{983.34} & 
5.07\\SCA8-4 & 1090.66 & 6.82 & 
1105.82 & 6.83 & \bf{1065.49} & 
2.36\\SCA8-5 & 1063.25 & 6.63 & 
1071.76 & 7.35 & \bf{1027.08} & 
3.52\\SCA8-6 & 987.58 & 6.46 & 
1011.76 & 6.22 & \bf{971.82} & 
1.62\\SCA8-7 & 1067.49 & 6.41 & 
1080.68 & 6.99 & \bf{1051.28} & 
1.54\\SCA8-8 & 1092.06 & 7.08 & 
1097.72 & 6.98 & \bf{1071.18} & 
1.95\\SCA8-9 & 1082.53 & 6.22 & 
1095.69 & 6.47 & \bf{1060.50} & 
2.08\\CON3-0 & 632.82 & 8.69 & 
635.74 & 8.10 & \bf{616.52} & 
2.64\\CON3-1 & 560.61 & 8.42 & 
561.16 & 8.57 & \bf{554.47} & 
1.11\\CON3-2 & 521.38 & 7.92 & 
524.87 & 8.39 & \bf{518.00} & 
0.65\\CON3-3 & \bf{591.19} & 8.23 & 
598.98 & 8.22 & 591.19 & 0.00\\
CON3-4 & 591.43 & 7.51 & 
600.67 & 7.60 & \bf{588.79} & 
0.45\\CON3-5 & 570.22 & 7.86 & 
574.32 & 8.48 & \bf{563.70} & 
1.16\\CON3-6 & 502.16 & 8.41 & 
504.78 & 8.99 & \bf{499.05} & 
0.62\\CON3-7 & 578.22 & 8.94 & 
590.56 & 8.61 & \bf{576.48} & 
0.30\\CON3-8 & 523.68 & 7.65 & 
528.33 & 8.30 & \bf{523.05} & 
0.12\\CON3-9 & 588.99 & 8.08 & 
590.54 & 8.34 & \bf{578.24} & 
1.86\\CON8-0 & 865.86 & 7.50 & 
884.53 & 7.24 & \bf{857.17} & 
1.01\\CON8-1 & 764.41 & 7.21 & 
782.42 & 7.07 & \bf{740.85} & 
3.18\\CON8-2 & 716.53 & 7.49 & 
729.55 & 7.09 & \bf{712.89} & 
0.51\\CON8-3 & 832.18 & 7.88 & 
840.20 & 7.45 & \bf{811.07} & 
2.60\\CON8-4 & 797.50 & 7.44 & 
821.52 & 6.13 & \bf{772.25} & 
3.27\\CON8-5 & 779.68 & 6.23 & 
781.12 & 7.12 & \bf{754.88} & 
3.29\\CON8-6 & 695.12 & 7.79 & 
704.05 & 7.59 & \bf{678.92} & 
2.39\\CON8-7 & 817.98 & 7.00 & 
841.08 & 6.67 & \bf{811.96} & 
0.74\\CON8-8 & 785.81 & 8.39 & 
793.53 & 7.45 & \bf{767.53} & 
2.38\\CON8-9 & 825.19 & 8.04 & 
831.84 & 7.70 & \bf{809.00} & 
2.00\\\bf{PROM.} & 
\bf{771.17} & \bf{7.74} & \bf{779.58} & \bf{7.63} & \bf{758.54} & \bf{1.52}\\[1ex]\hline
\end{tabular}
\label{table:nonlin}
\end{table} \clearpage
\begin{table}[ht]
\caption{Resultados de la ejecución de la metaheurística ILS, utilizando instancias de Dethloff con la configuración -n 65.0 -LS 40.0}
\centering
\small
\begin{tabular}{c c c c c c c}
\hline\hline
Instancia & Costo mínimo & Tiempo(seg.) & Costo promedio & Tiempo promedio(seg.) & Costo ILS & \%Gap \\ [0.5ex]
\hline
SCA3-0 & 640.55 & 9.70 & 
641.12 & 10.33 & \bf{635.62} & 
0.78\\SCA3-1 & 701.53 & 9.48 & 
703.87 & 9.99 & \bf{697.84} & 
0.53\\SCA3-2 & 661.13 & 9.39 & 
663.99 & 10.07 & \bf{659.34} & 
0.27\\SCA3-3 & 680.60 & 10.85 & 
685.19 & 10.37 & \bf{680.04} & 
0.08\\SCA3-4 & \bf{690.50} & 10.27 & 
691.87 & 10.35 & 690.50 & 0.00\\
SCA3-5 & 673.39 & 9.28 & 
680.84 & 10.16 & \bf{659.90} & 
2.04\\SCA3-6 & 652.94 & 11.96 & 
653.97 & 10.97 & \bf{651.09} & 
0.28\\SCA3-7 & 666.60 & 9.53 & 
671.50 & 9.69 & \bf{659.17} & 
1.13\\SCA3-8 & 719.77 & 11.51 & 
725.67 & 10.23 & \bf{719.47} & 
0.04\\SCA3-9 & 684.25 & 9.05 & 
687.93 & 9.31 & \bf{681.00} & 
0.48\\SCA8-0 & 1000.13 & 8.09 & 
1007.94 & 9.66 & \bf{961.50} & 
4.02\\SCA8-1 & 1086.74 & 8.44 & 
1087.48 & 8.00 & \bf{1049.65} & 
3.53\\SCA8-2 & 1050.37 & 8.07 & 
1058.13 & 8.74 & \bf{1039.64} & 
1.03\\SCA8-3 & 1030.71 & 7.87 & 
1040.55 & 8.13 & \bf{983.34} & 
4.82\\SCA8-4 & 1082.87 & 7.93 & 
1090.85 & 8.64 & \bf{1065.49} & 
1.63\\SCA8-5 & 1074.49 & 8.11 & 
1083.70 & 8.51 & \bf{1027.08} & 
4.62\\SCA8-6 & 987.54 & 9.27 & 
1001.21 & 9.00 & \bf{971.82} & 
1.62\\SCA8-7 & 1070.53 & 8.53 & 
1088.26 & 9.54 & \bf{1051.28} & 
1.83\\SCA8-8 & 1092.82 & 8.73 & 
1096.81 & 8.95 & \bf{1071.18} & 
2.02\\SCA8-9 & 1082.56 & 7.70 & 
1092.38 & 9.02 & \bf{1060.50} & 
2.08\\CON3-0 & 632.00 & 9.88 & 
637.53 & 9.52 & \bf{616.52} & 
2.51\\CON3-1 & 559.52 & 10.41 & 
563.05 & 10.38 & \bf{554.47} & 
0.91\\CON3-2 & 521.38 & 9.56 & 
522.93 & 9.65 & \bf{518.00} & 
0.65\\CON3-3 & \bf{591.19} & 11.17 & 
594.17 & 11.05 & 591.19 & 0.00\\
CON3-4 & 593.78 & 11.04 & 
596.31 & 10.47 & \bf{588.79} & 
0.85\\CON3-5 & 569.04 & 11.20 & 
572.12 & 10.85 & \bf{563.70} & 
0.95\\CON3-6 & 502.16 & 8.59 & 
505.75 & 10.27 & \bf{499.05} & 
0.62\\CON3-7 & 578.41 & 10.49 & 
588.24 & 9.80 & \bf{576.48} & 
0.33\\CON3-8 & 524.59 & 8.76 & 
525.90 & 9.75 & \bf{523.05} & 
0.29\\CON3-9 & 578.25 & 10.38 & 
587.98 & 10.07 & \bf{578.24} & 
0.00\\CON8-0 & 890.01 & 8.35 & 
894.68 & 8.92 & \bf{857.17} & 
3.83\\CON8-1 & 752.72 & 10.16 & 
763.66 & 9.11 & \bf{740.85} & 
1.60\\CON8-2 & 716.69 & 8.83 & 
725.38 & 9.15 & \bf{712.89} & 
0.53\\CON8-3 & 822.09 & 10.38 & 
839.79 & 9.29 & \bf{811.07} & 
1.36\\CON8-4 & 802.11 & 8.31 & 
809.93 & 8.03 & \bf{772.25} & 
3.87\\CON8-5 & 766.55 & 7.46 & 
783.10 & 7.55 & \bf{754.88} & 
1.55\\CON8-6 & 699.10 & 9.91 & 
708.58 & 9.48 & \bf{678.92} & 
2.97\\CON8-7 & 814.86 & 9.07 & 
833.77 & 8.89 & \bf{811.96} & 
0.36\\CON8-8 & 789.40 & 9.68 & 
792.23 & 9.12 & \bf{767.53} & 
2.85\\CON8-9 & 816.47 & 8.39 & 
837.99 & 8.64 & \bf{809.00} & 
0.92\\\bf{PROM.} & 
\bf{771.26} & \bf{9.39} & \bf{778.41} & \bf{9.49} & \bf{758.54} & \bf{1.49}\\[1ex]\hline
\end{tabular}
\label{table:nonlin}
\end{table} \clearpage
\begin{table}[ht]
\caption{Resultados de la ejecución de la metaheurística ILS, utilizando instancias de Dethloff con la configuración -n 65.0 -LS 50.0}
\centering
\small
\begin{tabular}{c c c c c c c}
\hline\hline
Instancia & Costo mínimo & Tiempo(seg.) & Costo promedio & Tiempo promedio(seg.) & Costo ILS & \%Gap \\ [0.5ex]
\hline
SCA3-0 & 636.06 & 11.33 & 
640.18 & 12.12 & \bf{635.62} & 
0.07\\SCA3-1 & \bf{697.84} & 13.89 & 
704.01 & 13.49 & 697.84 & 0.00\\
SCA3-2 & \bf{659.34} & 12.61 & 
662.67 & 12.75 & 659.34 & 0.00\\
SCA3-3 & \bf{680.04} & 12.32 & 
682.18 & 11.97 & 680.04 & 0.00\\
SCA3-4 & \bf{690.50} & 13.46 & 
690.50 & 12.46 & 690.50 & 0.00\\
SCA3-5 & 665.64 & 12.29 & 
673.45 & 12.54 & \bf{659.90} & 
0.87\\SCA3-6 & \bf{651.09} & 10.69 & 
655.30 & 11.60 & 651.09 & 0.00\\
SCA3-7 & 671.67 & 12.61 & 
671.72 & 12.96 & \bf{659.17} & 
1.90\\SCA3-8 & 724.66 & 11.53 & 
725.92 & 12.40 & \bf{719.47} & 
0.72\\SCA3-9 & \bf{681.00} & 12.71 & 
683.27 & 12.47 & 681.00 & 0.00\\
SCA8-0 & 987.08 & 11.30 & 
1006.67 & 10.68 & \bf{961.50} & 
2.66\\SCA8-1 & 1082.51 & 9.62 & 
1087.66 & 9.14 & \bf{1049.65} & 
3.13\\SCA8-2 & 1064.69 & 10.15 & 
1067.63 & 10.01 & \bf{1039.64} & 
2.41\\SCA8-3 & 1022.76 & 9.22 & 
1027.16 & 9.94 & \bf{983.34} & 
4.01\\SCA8-4 & 1069.71 & 10.44 & 
1084.03 & 11.24 & \bf{1065.49} & 
0.40\\SCA8-5 & 1059.39 & 11.02 & 
1063.96 & 9.91 & \bf{1027.08} & 
3.15\\SCA8-6 & 981.41 & 12.04 & 
994.55 & 10.86 & \bf{971.82} & 
0.99\\SCA8-7 & 1071.53 & 9.64 & 
1078.66 & 10.18 & \bf{1051.28} & 
1.93\\SCA8-8 & \bf{1071.18} & 10.31 & 
1090.75 & 10.45 & 1071.18 & 0.00\\
SCA8-9 & 1078.30 & 8.29 & 
1083.36 & 8.77 & \bf{1060.50} & 
1.68\\CON3-0 & 629.03 & 11.62 & 
632.92 & 12.30 & \bf{616.52} & 
2.03\\CON3-1 & 560.61 & 13.78 & 
561.44 & 13.13 & \bf{554.47} & 
1.11\\CON3-2 & 521.38 & 12.76 & 
521.38 & 12.49 & \bf{518.00} & 
0.65\\CON3-3 & 594.10 & 12.25 & 
596.39 & 12.17 & \bf{591.19} & 
0.49\\CON3-4 & 592.58 & 11.64 & 
593.68 & 12.81 & \bf{588.79} & 
0.64\\CON3-5 & 567.94 & 12.23 & 
570.30 & 12.23 & \bf{563.70} & 
0.75\\CON3-6 & 502.16 & 12.90 & 
507.96 & 12.48 & \bf{499.05} & 
0.62\\CON3-7 & 578.41 & 12.56 & 
586.52 & 12.88 & \bf{576.48} & 
0.33\\CON3-8 & 523.68 & 12.85 & 
525.36 & 12.83 & \bf{523.05} & 
0.12\\CON3-9 & 578.25 & 12.29 & 
587.24 & 12.28 & \bf{578.24} & 
0.00\\CON8-0 & 865.86 & 10.71 & 
894.88 & 10.36 & \bf{857.17} & 
1.01\\CON8-1 & 754.22 & 11.20 & 
764.98 & 10.59 & \bf{740.85} & 
1.80\\CON8-2 & 722.29 & 9.68 & 
725.76 & 10.37 & \bf{712.89} & 
1.32\\CON8-3 & 830.79 & 10.62 & 
839.78 & 10.17 & \bf{811.07} & 
2.43\\CON8-4 & 775.49 & 10.03 & 
793.73 & 10.00 & \bf{772.25} & 
0.42\\CON8-5 & 763.13 & 10.49 & 
773.30 & 10.78 & \bf{754.88} & 
1.09\\CON8-6 & 696.58 & 10.44 & 
707.41 & 10.11 & \bf{678.92} & 
2.60\\CON8-7 & 815.72 & 10.69 & 
825.53 & 11.10 & \bf{811.96} & 
0.46\\CON8-8 & 786.86 & 11.10 & 
788.54 & 11.00 & \bf{767.53} & 
2.52\\CON8-9 & 822.87 & 10.82 & 
825.74 & 11.00 & \bf{809.00} & 
1.71\\\bf{PROM.} & 
\bf{768.21} & \bf{11.40} & \bf{774.91} & \bf{11.43} & \bf{758.54} & \bf{1.15}\\[1ex]\hline
\end{tabular}
\label{table:nonlin}
\end{table} \clearpage
\begin{table}[ht]
\caption{Resultados de la ejecución de la metaheurística ILS, utilizando instancias de Dethloff con la configuración -n 65.0 -LS 60.0}
\centering
\small
\begin{tabular}{c c c c c c c}
\hline\hline
Instancia & Costo mínimo & Tiempo(seg.) & Costo promedio & Tiempo promedio(seg.) & Costo ILS & \%Gap \\ [0.5ex]
\hline
SCA3-0 & 636.06 & 14.80 & 
639.43 & 14.60 & \bf{635.62} & 
0.07\\SCA3-1 & \bf{697.84} & 15.74 & 
700.35 & 14.72 & 697.84 & 0.00\\
SCA3-2 & \bf{659.34} & 13.62 & 
661.95 & 14.40 & 659.34 & 0.00\\
SCA3-3 & 680.60 & 15.71 & 
681.71 & 15.10 & \bf{680.04} & 
0.08\\SCA3-4 & \bf{690.50} & 14.47 & 
692.17 & 16.54 & 690.50 & 0.00\\
SCA3-5 & 665.64 & 15.64 & 
674.53 & 15.04 & \bf{659.90} & 
0.87\\SCA3-6 & 652.47 & 13.39 & 
654.00 & 14.73 & \bf{651.09} & 
0.21\\SCA3-7 & 671.67 & 14.50 & 
671.75 & 14.68 & \bf{659.17} & 
1.90\\SCA3-8 & \bf{719.47} & 14.01 & 
721.88 & 15.48 & 719.47 & 0.00\\
SCA3-9 & \bf{681.00} & 12.94 & 
683.75 & 14.16 & 681.00 & 0.00\\
SCA8-0 & 977.93 & 14.64 & 
1003.11 & 13.68 & \bf{961.50} & 
1.71\\SCA8-1 & 1072.05 & 10.01 & 
1083.67 & 11.58 & \bf{1049.65} & 
2.13\\SCA8-2 & 1066.63 & 10.58 & 
1068.79 & 12.05 & \bf{1039.64} & 
2.60\\SCA8-3 & 1005.33 & 12.14 & 
1020.12 & 11.20 & \bf{983.34} & 
2.24\\SCA8-4 & 1074.63 & 12.27 & 
1090.36 & 12.11 & \bf{1065.49} & 
0.86\\SCA8-5 & 1048.25 & 14.14 & 
1067.30 & 13.39 & \bf{1027.08} & 
2.06\\SCA8-6 & 985.96 & 13.37 & 
992.51 & 12.88 & \bf{971.82} & 
1.46\\SCA8-7 & 1067.49 & 11.78 & 
1079.65 & 12.11 & \bf{1051.28} & 
1.54\\SCA8-8 & 1082.91 & 11.90 & 
1092.47 & 11.34 & \bf{1071.18} & 
1.10\\SCA8-9 & 1081.53 & 11.34 & 
1087.37 & 11.65 & \bf{1060.50} & 
1.98\\CON3-0 & 619.09 & 14.98 & 
626.08 & 14.76 & \bf{616.52} & 
0.42\\CON3-1 & 560.75 & 13.38 & 
563.42 & 15.38 & \bf{554.47} & 
1.13\\CON3-2 & 521.38 & 13.61 & 
522.89 & 15.12 & \bf{518.00} & 
0.65\\CON3-3 & 591.48 & 15.22 & 
593.32 & 14.77 & \bf{591.19} & 
0.05\\CON3-4 & 591.43 & 14.28 & 
593.38 & 13.79 & \bf{588.79} & 
0.45\\CON3-5 & 564.88 & 12.72 & 
572.18 & 14.69 & \bf{563.70} & 
0.21\\CON3-6 & 502.64 & 15.05 & 
508.10 & 14.80 & \bf{499.05} & 
0.72\\CON3-7 & 578.41 & 14.44 & 
583.40 & 14.78 & \bf{576.48} & 
0.33\\CON3-8 & 524.59 & 16.43 & 
526.21 & 15.61 & \bf{523.05} & 
0.29\\CON3-9 & 587.97 & 14.18 & 
588.51 & 15.08 & \bf{578.24} & 
1.68\\CON8-0 & 889.90 & 10.68 & 
899.58 & 11.32 & \bf{857.17} & 
3.82\\CON8-1 & 755.71 & 14.99 & 
764.11 & 12.47 & \bf{740.85} & 
2.01\\CON8-2 & 722.51 & 12.07 & 
728.61 & 12.59 & \bf{712.89} & 
1.35\\CON8-3 & 822.00 & 13.54 & 
830.68 & 13.45 & \bf{811.07} & 
1.35\\CON8-4 & 805.03 & 11.00 & 
808.08 & 10.70 & \bf{772.25} & 
4.24\\CON8-5 & 764.17 & 12.72 & 
773.07 & 13.31 & \bf{754.88} & 
1.23\\CON8-6 & 698.58 & 12.53 & 
701.60 & 13.13 & \bf{678.92} & 
2.90\\CON8-7 & 814.50 & 11.73 & 
821.64 & 11.75 & \bf{811.96} & 
0.31\\CON8-8 & 782.61 & 11.55 & 
787.34 & 11.87 & \bf{767.53} & 
1.96\\CON8-9 & 812.03 & 12.44 & 
825.00 & 12.05 & \bf{809.00} & 
0.37\\\bf{PROM.} & 
\bf{768.17} & \bf{13.36} & \bf{774.60} & \bf{13.57} & \bf{758.54} & \bf{1.16}\\[1ex]\hline
\end{tabular}
\label{table:nonlin}
\end{table} \clearpage
\begin{table}[ht]
\caption{Resultados de la ejecución de la metaheurística ILS, utilizando instancias de Dethloff con la configuración -n 65.0 -LS 70.0}
\centering
\small
\begin{tabular}{c c c c c c c}
\hline\hline
Instancia & Costo mínimo & Tiempo(seg.) & Costo promedio & Tiempo promedio(seg.) & Costo ILS & \%Gap \\ [0.5ex]
\hline
SCA3-0 & 636.06 & 16.90 & 
639.43 & 17.51 & \bf{635.62} & 
0.07\\SCA3-1 & \bf{697.84} & 18.88 & 
701.99 & 16.93 & 697.84 & 0.00\\
SCA3-2 & \bf{659.34} & 16.74 & 
661.01 & 16.98 & 659.34 & 0.00\\
SCA3-3 & \bf{680.04} & 18.57 & 
681.39 & 16.30 & 680.04 & 0.00\\
SCA3-4 & \bf{690.50} & 15.18 & 
690.50 & 16.09 & 690.50 & 0.00\\
SCA3-5 & 670.02 & 16.93 & 
674.43 & 17.68 & \bf{659.90} & 
1.53\\SCA3-6 & \bf{651.09} & 15.76 & 
653.27 & 16.70 & 651.09 & 0.00\\
SCA3-7 & 669.89 & 16.03 & 
670.80 & 16.39 & \bf{659.17} & 
1.63\\SCA3-8 & \bf{719.47} & 16.92 & 
723.93 & 16.88 & 719.47 & 0.00\\
SCA3-9 & \bf{681.00} & 16.01 & 
684.40 & 16.03 & 681.00 & 0.00\\
SCA8-0 & 981.19 & 14.91 & 
993.13 & 15.56 & \bf{961.50} & 
2.05\\SCA8-1 & 1074.98 & 11.87 & 
1084.69 & 12.97 & \bf{1049.65} & 
2.41\\SCA8-2 & 1054.30 & 14.36 & 
1060.10 & 14.90 & \bf{1039.64} & 
1.41\\SCA8-3 & 1002.63 & 12.55 & 
1019.11 & 13.00 & \bf{983.34} & 
1.96\\SCA8-4 & 1069.71 & 15.56 & 
1081.55 & 13.63 & \bf{1065.49} & 
0.40\\SCA8-5 & 1043.65 & 14.58 & 
1061.36 & 14.31 & \bf{1027.08} & 
1.61\\SCA8-6 & 991.11 & 13.23 & 
993.87 & 13.82 & \bf{971.82} & 
1.98\\SCA8-7 & 1071.53 & 14.63 & 
1086.40 & 13.70 & \bf{1051.28} & 
1.93\\SCA8-8 & 1087.01 & 13.30 & 
1090.57 & 13.73 & \bf{1071.18} & 
1.48\\SCA8-9 & 1067.42 & 13.73 & 
1083.48 & 12.77 & \bf{1060.50} & 
0.65\\CON3-0 & 632.57 & 16.22 & 
635.98 & 16.27 & \bf{616.52} & 
2.60\\CON3-1 & 556.92 & 18.99 & 
559.79 & 17.96 & \bf{554.47} & 
0.44\\CON3-2 & 521.38 & 17.40 & 
522.32 & 16.40 & \bf{518.00} & 
0.65\\CON3-3 & 591.20 & 17.81 & 
592.36 & 17.05 & \bf{591.19} & 
0.00\\CON3-4 & 591.43 & 15.55 & 
594.45 & 16.61 & \bf{588.79} & 
0.45\\CON3-5 & 569.57 & 15.81 & 
570.16 & 17.48 & \bf{563.70} & 
1.04\\CON3-6 & 502.16 & 17.10 & 
505.64 & 16.05 & \bf{499.05} & 
0.62\\CON3-7 & 578.41 & 15.65 & 
583.66 & 16.04 & \bf{576.48} & 
0.33\\CON3-8 & \bf{523.05} & 19.62 & 
524.78 & 17.91 & 523.05 & 0.00\\
CON3-9 & 588.40 & 17.25 & 
588.55 & 17.77 & \bf{578.24} & 
1.76\\CON8-0 & 889.13 & 11.98 & 
897.17 & 13.59 & \bf{857.17} & 
3.73\\CON8-1 & 754.51 & 13.38 & 
761.72 & 13.11 & \bf{740.85} & 
1.84\\CON8-2 & 726.54 & 16.29 & 
731.47 & 15.34 & \bf{712.89} & 
1.91\\CON8-3 & 826.27 & 13.41 & 
832.63 & 14.35 & \bf{811.07} & 
1.87\\CON8-4 & 780.54 & 14.93 & 
802.58 & 14.23 & \bf{772.25} & 
1.07\\CON8-5 & 771.87 & 15.16 & 
773.28 & 16.16 & \bf{754.88} & 
2.25\\CON8-6 & 698.31 & 14.02 & 
703.16 & 14.08 & \bf{678.92} & 
2.86\\CON8-7 & 815.32 & 14.39 & 
818.69 & 14.88 & \bf{811.96} & 
0.41\\CON8-8 & 791.37 & 13.71 & 
793.73 & 13.35 & \bf{767.53} & 
3.11\\CON8-9 & 820.94 & 17.81 & 
828.65 & 15.40 & \bf{809.00} & 
1.48\\\bf{PROM.} & 
\bf{768.22} & \bf{15.58} & \bf{773.90} & \bf{15.50} & \bf{758.54} & \bf{1.19}\\[1ex]\hline
\end{tabular}
\label{table:nonlin}
\end{table} \clearpage
\begin{table}[ht]
\caption{Resultados de la ejecución de la metaheurística ILS, utilizando instancias de Dethloff con la configuración -n 65.0 -LS 80.0}
\centering
\small
\begin{tabular}{c c c c c c c}
\hline\hline
Instancia & Costo mínimo & Tiempo(seg.) & Costo promedio & Tiempo promedio(seg.) & Costo ILS & \%Gap \\ [0.5ex]
\hline
SCA3-0 & 640.55 & 17.55 & 
640.84 & 18.79 & \bf{635.62} & 
0.78\\SCA3-1 & \bf{697.84} & 19.22 & 
703.44 & 18.54 & 697.84 & 0.00\\
SCA3-2 & \bf{659.34} & 19.33 & 
662.61 & 19.23 & 659.34 & 0.00\\
SCA3-3 & \bf{680.04} & 20.44 & 
680.18 & 20.28 & 680.04 & 0.00\\
SCA3-4 & \bf{690.50} & 16.04 & 
690.50 & 17.27 & 690.50 & 0.00\\
SCA3-5 & 662.75 & 18.50 & 
670.50 & 18.00 & \bf{659.90} & 
0.43\\SCA3-6 & \bf{651.09} & 18.14 & 
652.48 & 18.28 & 651.09 & 0.00\\
SCA3-7 & 671.67 & 19.08 & 
671.72 & 19.02 & \bf{659.17} & 
1.90\\SCA3-8 & \bf{719.47} & 19.59 & 
723.93 & 18.97 & 719.47 & 0.00\\
SCA3-9 & \bf{681.00} & 19.35 & 
686.56 & 18.14 & 681.00 & 0.00\\
SCA8-0 & \bf{961.50} & 19.61 & 
977.64 & 16.09 & 961.50 & 0.00\\
SCA8-1 & 1058.33 & 15.96 & 
1069.45 & 14.38 & \bf{1049.65} & 
0.83\\SCA8-2 & 1054.95 & 16.85 & 
1059.74 & 14.91 & \bf{1039.64} & 
1.47\\SCA8-3 & 1016.62 & 16.68 & 
1025.42 & 14.95 & \bf{983.34} & 
3.38\\SCA8-4 & 1069.71 & 15.93 & 
1072.51 & 15.61 & \bf{1065.49} & 
0.40\\SCA8-5 & 1059.41 & 15.68 & 
1073.54 & 14.94 & \bf{1027.08} & 
3.15\\SCA8-6 & 985.48 & 15.86 & 
995.88 & 15.03 & \bf{971.82} & 
1.41\\SCA8-7 & 1067.11 & 14.92 & 
1070.51 & 15.57 & \bf{1051.28} & 
1.51\\SCA8-8 & 1089.55 & 16.32 & 
1092.01 & 15.74 & \bf{1071.18} & 
1.71\\SCA8-9 & 1067.42 & 11.80 & 
1086.37 & 14.70 & \bf{1060.50} & 
0.65\\CON3-0 & 619.09 & 19.15 & 
628.52 & 19.48 & \bf{616.52} & 
0.42\\CON3-1 & 560.32 & 19.38 & 
560.64 & 20.04 & \bf{554.47} & 
1.06\\CON3-2 & 521.38 & 18.15 & 
521.38 & 18.00 & \bf{518.00} & 
0.65\\CON3-3 & \bf{591.19} & 21.46 & 
594.81 & 20.59 & 591.19 & 0.00\\
CON3-4 & 591.43 & 17.24 & 
592.61 & 18.71 & \bf{588.79} & 
0.45\\CON3-5 & 564.88 & 17.32 & 
570.58 & 18.57 & \bf{563.70} & 
0.21\\CON3-6 & 501.33 & 21.25 & 
505.56 & 19.08 & \bf{499.05} & 
0.46\\CON3-7 & 578.41 & 19.21 & 
586.88 & 18.95 & \bf{576.48} & 
0.33\\CON3-8 & 524.59 & 19.50 & 
527.16 & 19.43 & \bf{523.05} & 
0.29\\CON3-9 & 588.40 & 20.92 & 
589.79 & 19.20 & \bf{578.24} & 
1.76\\CON8-0 & 883.19 & 14.66 & 
892.24 & 15.38 & \bf{857.17} & 
3.04\\CON8-1 & 754.95 & 13.73 & 
762.99 & 15.51 & \bf{740.85} & 
1.90\\CON8-2 & 720.12 & 15.59 & 
727.61 & 15.02 & \bf{712.89} & 
1.01\\CON8-3 & 815.38 & 16.06 & 
832.18 & 17.51 & \bf{811.07} & 
0.53\\CON8-4 & 780.48 & 12.66 & 
799.21 & 14.27 & \bf{772.25} & 
1.07\\CON8-5 & 760.72 & 16.05 & 
769.60 & 15.36 & \bf{754.88} & 
0.77\\CON8-6 & 700.32 & 15.40 & 
703.84 & 16.60 & \bf{678.92} & 
3.15\\CON8-7 & 825.09 & 13.42 & 
827.32 & 15.16 & \bf{811.96} & 
1.62\\CON8-8 & 779.43 & 16.42 & 
790.90 & 15.21 & \bf{767.53} & 
1.55\\CON8-9 & 818.23 & 14.39 & 
823.59 & 14.62 & \bf{809.00} & 
1.14\\\bf{PROM.} & 
\bf{766.58} & \bf{17.22} & \bf{772.83} & \bf{17.13} & \bf{758.54} & \bf{0.98}\\[1ex]\hline
\end{tabular}
\label{table:nonlin}
\end{table} \clearpage
\begin{table}[ht]
\caption{Resultados de la ejecución de la metaheurística ILS, utilizando instancias de SalhiNagy con la configuración -n 15.0 -LS 10.0}
\centering
\small
\begin{tabular}{c c c c c c c}
\hline\hline
Instancia & Costo mínimo & Tiempo(seg.) & Costo promedio & Tiempo promedio(seg.) & Costo ILS & \%Gap \\ [0.5ex]
\hline
CMT1X & 490.81 & 0.97 & 
501.76 & 0.83 & \bf{466.77} & 
5.15\\CMT1Y & 485.71 & 1.08 & 
498.10 & 0.71 & \bf{466.77} & 
4.06\\CMT2X & 722.89 & 2.41 & 
728.89 & 2.55 & \bf{684.21} & 
5.65\\CMT2Y & 722.28 & 2.32 & 
724.83 & 2.25 & \bf{684.21} & 
5.56\\CMT3X & 735.88 & 6.29 & 
746.94 & 6.08 & \bf{721.40} & 
2.01\\CMT3Y & 736.75 & 6.25 & 
742.77 & 6.00 & \bf{721.40} & 
2.13\\CMT4X & 891.65 & 19.14 & 
906.33 & 20.52 & \bf{852.83} & 
4.55\\CMT4Y & 910.90 & 20.03 & 
916.58 & 21.67 & \bf{852.46} & 
6.86\\CMT5X & 1109.35 & 68.60 & 
1111.90 & 51.84 & \bf{1030.55} & 
7.65\\CMT5Y & 1108.63 & 46.61 & 
1116.23 & 51.76 & \bf{1031.17} & 
7.51\\CMT11X & 908.96 & 14.70 & 
921.65 & 16.09 & \bf{839.39} & 
8.29\\CMT11Y & 885.84 & 14.03 & 
896.32 & 15.99 & \bf{841.88} & 
5.22\\CMT12X & 679.87 & 5.26 & 
690.46 & 5.60 & \bf{662.22} & 
2.67\\CMT12Y & 685.38 & 5.80 & 
686.17 & 6.38 & \bf{662.22} & 
3.50\\\bf{PROM.} & 
\bf{791.06} & \bf{15.25} & \bf{799.21} & \bf{14.88} & \bf{751.25} & \bf{5.06}\\[1ex]\hline
\end{tabular}
\label{table:nonlin}
\end{table} \clearpage
\begin{table}[ht]
\caption{Resultados de la ejecución de la metaheurística ILS, utilizando instancias de SalhiNagy con la configuración -n 15.0 -LS 20.0}
\centering
\small
\begin{tabular}{c c c c c c c}
\hline\hline
Instancia & Costo mínimo & Tiempo(seg.) & Costo promedio & Tiempo promedio(seg.) & Costo ILS & \%Gap \\ [0.5ex]
\hline
CMT1X & 486.26 & 1.05 & 
491.09 & 1.18 & \bf{466.77} & 
4.18\\CMT1Y & 485.04 & 1.66 & 
497.56 & 1.64 & \bf{466.77} & 
3.91\\CMT2X & 712.75 & 3.75 & 
719.04 & 3.54 & \bf{684.21} & 
4.17\\CMT2Y & 708.06 & 3.76 & 
713.36 & 3.29 & \bf{684.21} & 
3.49\\CMT3X & 737.20 & 8.68 & 
743.57 & 7.32 & \bf{721.40} & 
2.19\\CMT3Y & 736.49 & 8.11 & 
744.03 & 8.20 & \bf{721.40} & 
2.09\\CMT4X & 884.09 & 32.59 & 
907.08 & 24.07 & \bf{852.83} & 
3.67\\CMT4Y & 910.57 & 20.26 & 
919.45 & 21.61 & \bf{852.46} & 
6.82\\CMT5X & 1104.28 & 54.94 & 
1127.42 & 53.84 & \bf{1030.55} & 
7.15\\CMT5Y & 1104.90 & 58.52 & 
1119.76 & 60.12 & \bf{1031.17} & 
7.15\\CMT11X & 888.18 & 23.47 & 
902.48 & 18.95 & \bf{839.39} & 
5.81\\CMT11Y & 894.92 & 16.93 & 
910.85 & 16.28 & \bf{841.88} & 
6.30\\CMT12X & 676.42 & 6.88 & 
687.13 & 7.35 & \bf{662.22} & 
2.14\\CMT12Y & 682.07 & 6.44 & 
688.30 & 7.54 & \bf{662.22} & 
3.00\\\bf{PROM.} & 
\bf{786.52} & \bf{17.65} & \bf{797.94} & \bf{16.78} & \bf{751.25} & \bf{4.43}\\[1ex]\hline
\end{tabular}
\label{table:nonlin}
\end{table} \clearpage
\begin{table}[ht]
\caption{Resultados de la ejecución de la metaheurística ILS, utilizando instancias de SalhiNagy con la configuración -n 15.0 -LS 30.0}
\centering
\small
\begin{tabular}{c c c c c c c}
\hline\hline
Instancia & Costo mínimo & Tiempo(seg.) & Costo promedio & Tiempo promedio(seg.) & Costo ILS & \%Gap \\ [0.5ex]
\hline
CMT1X & 485.80 & 1.56 & 
495.89 & 1.67 & \bf{466.77} & 
4.08\\CMT1Y & 476.43 & 1.69 & 
485.94 & 2.10 & \bf{466.77} & 
2.07\\CMT2X & 712.38 & 3.82 & 
719.23 & 3.69 & \bf{684.21} & 
4.12\\CMT2Y & 716.86 & 3.05 & 
721.10 & 3.47 & \bf{684.21} & 
4.77\\CMT3X & 733.82 & 9.36 & 
742.40 & 8.53 & \bf{721.40} & 
1.72\\CMT3Y & 736.53 & 8.22 & 
744.82 & 10.14 & \bf{721.40} & 
2.10\\CMT4X & 910.27 & 24.74 & 
914.12 & 26.60 & \bf{852.83} & 
6.74\\CMT4Y & 888.83 & 25.13 & 
905.68 & 25.77 & \bf{852.46} & 
4.27\\CMT5X & 1108.02 & 66.79 & 
1114.48 & 68.50 & \bf{1030.55} & 
7.52\\CMT5Y & 1102.41 & 59.63 & 
1115.61 & 71.81 & \bf{1031.17} & 
6.91\\CMT11X & 878.61 & 20.86 & 
897.95 & 24.50 & \bf{839.39} & 
4.67\\CMT11Y & 893.62 & 23.08 & 
904.36 & 19.47 & \bf{841.88} & 
6.15\\CMT12X & 692.89 & 7.64 & 
698.56 & 9.01 & \bf{662.22} & 
4.63\\CMT12Y & 685.12 & 8.11 & 
691.99 & 8.68 & \bf{662.22} & 
3.46\\\bf{PROM.} & 
\bf{787.26} & \bf{18.83} & \bf{796.58} & \bf{20.28} & \bf{751.25} & \bf{4.51}\\[1ex]\hline
\end{tabular}
\label{table:nonlin}
\end{table} \clearpage
\begin{table}[ht]
\caption{Resultados de la ejecución de la metaheurística ILS, utilizando instancias de SalhiNagy con la configuración -n 15.0 -LS 40.0}
\centering
\small
\begin{tabular}{c c c c c c c}
\hline\hline
Instancia & Costo mínimo & Tiempo(seg.) & Costo promedio & Tiempo promedio(seg.) & Costo ILS & \%Gap \\ [0.5ex]
\hline
CMT1X & 483.81 & 1.58 & 
487.36 & 1.95 & \bf{466.77} & 
3.65\\CMT1Y & 485.61 & 2.02 & 
492.14 & 1.81 & \bf{466.77} & 
4.04\\CMT2X & 719.57 & 4.65 & 
722.80 & 4.65 & \bf{684.21} & 
5.17\\CMT2Y & 707.29 & 4.18 & 
716.39 & 4.33 & \bf{684.21} & 
3.37\\CMT3X & 738.24 & 11.05 & 
745.11 & 11.86 & \bf{721.40} & 
2.33\\CMT3Y & 733.03 & 10.48 & 
737.23 & 10.92 & \bf{721.40} & 
1.61\\CMT4X & 886.86 & 30.14 & 
911.71 & 32.65 & \bf{852.83} & 
3.99\\CMT4Y & 906.85 & 31.16 & 
913.06 & 32.60 & \bf{852.46} & 
6.38\\CMT5X & 1096.95 & 64.51 & 
1117.64 & 66.93 & \bf{1030.55} & 
6.44\\CMT5Y & 1097.04 & 65.53 & 
1112.25 & 71.24 & \bf{1031.17} & 
6.39\\CMT11X & 885.76 & 23.56 & 
910.45 & 22.46 & \bf{839.39} & 
5.52\\CMT11Y & 890.29 & 30.85 & 
903.69 & 24.84 & \bf{841.88} & 
5.75\\CMT12X & 688.70 & 9.95 & 
691.88 & 9.90 & \bf{662.22} & 
4.00\\CMT12Y & 674.52 & 12.13 & 
680.92 & 9.64 & \bf{662.22} & 
1.86\\\bf{PROM.} & 
\bf{785.32} & \bf{21.56} & \bf{795.90} & \bf{21.84} & \bf{751.25} & \bf{4.32}\\[1ex]\hline
\end{tabular}
\label{table:nonlin}
\end{table} \clearpage
\begin{table}[ht]
\caption{Resultados de la ejecución de la metaheurística ILS, utilizando instancias de SalhiNagy con la configuración -n 15.0 -LS 50.0}
\centering
\small
\begin{tabular}{c c c c c c c}
\hline\hline
Instancia & Costo mínimo & Tiempo(seg.) & Costo promedio & Tiempo promedio(seg.) & Costo ILS & \%Gap \\ [0.5ex]
\hline
CMT1X & 482.58 & 2.42 & 
488.69 & 2.31 & \bf{466.77} & 
3.39\\CMT1Y & 484.35 & 3.17 & 
487.91 & 2.83 & \bf{466.77} & 
3.77\\CMT2X & 703.29 & 6.15 & 
715.84 & 5.59 & \bf{684.21} & 
2.79\\CMT2Y & 697.22 & 5.51 & 
710.04 & 5.49 & \bf{684.21} & 
1.90\\CMT3X & 738.81 & 14.16 & 
745.38 & 12.72 & \bf{721.40} & 
2.41\\CMT3Y & 736.83 & 14.25 & 
746.04 & 12.09 & \bf{721.40} & 
2.14\\CMT4X & 885.47 & 42.58 & 
908.59 & 35.96 & \bf{852.83} & 
3.83\\CMT4Y & 885.93 & 32.07 & 
901.18 & 33.30 & \bf{852.46} & 
3.93\\CMT5X & 1106.75 & 75.47 & 
1120.88 & 89.86 & \bf{1030.55} & 
7.39\\CMT5Y & 1106.52 & 78.43 & 
1113.08 & 73.41 & \bf{1031.17} & 
7.31\\CMT11X & 881.37 & 23.47 & 
905.50 & 26.91 & \bf{839.39} & 
5.00\\CMT11Y & 917.08 & 26.52 & 
927.80 & 27.64 & \bf{841.88} & 
8.93\\CMT12X & 675.84 & 11.72 & 
682.70 & 11.37 & \bf{662.22} & 
2.06\\CMT12Y & 681.45 & 11.44 & 
698.13 & 11.65 & \bf{662.22} & 
2.90\\\bf{PROM.} & 
\bf{784.54} & \bf{24.81} & \bf{796.55} & \bf{25.08} & \bf{751.25} & \bf{4.12}\\[1ex]\hline
\end{tabular}
\label{table:nonlin}
\end{table} \clearpage
\begin{table}[ht]
\caption{Resultados de la ejecución de la metaheurística ILS, utilizando instancias de SalhiNagy con la configuración -n 15.0 -LS 60.0}
\centering
\small
\begin{tabular}{c c c c c c c}
\hline\hline
Instancia & Costo mínimo & Tiempo(seg.) & Costo promedio & Tiempo promedio(seg.) & Costo ILS & \%Gap \\ [0.5ex]
\hline
CMT1X & 472.58 & 3.09 & 
479.25 & 3.19 & \bf{466.77} & 
1.24\\CMT1Y & 479.58 & 2.76 & 
491.56 & 2.46 & \bf{466.77} & 
2.74\\CMT2X & 709.75 & 6.40 & 
716.60 & 5.98 & \bf{684.21} & 
3.73\\CMT2Y & 706.79 & 7.23 & 
716.12 & 6.52 & \bf{684.21} & 
3.30\\CMT3X & 737.52 & 13.66 & 
741.51 & 14.49 & \bf{721.40} & 
2.23\\CMT3Y & 732.16 & 15.61 & 
740.49 & 13.80 & \bf{721.40} & 
1.49\\CMT4X & 892.10 & 35.67 & 
908.05 & 40.24 & \bf{852.83} & 
4.60\\CMT4Y & 907.31 & 31.36 & 
917.20 & 35.60 & \bf{852.46} & 
6.43\\CMT5X & 1094.53 & 87.30 & 
1115.61 & 94.43 & \bf{1030.55} & 
6.21\\CMT5Y & 1097.57 & 90.85 & 
1112.60 & 92.58 & \bf{1031.17} & 
6.44\\CMT11X & 906.83 & 28.35 & 
917.29 & 25.84 & \bf{839.39} & 
8.03\\CMT11Y & 903.57 & 28.24 & 
918.31 & 28.97 & \bf{841.88} & 
7.33\\CMT12X & 680.97 & 13.52 & 
684.69 & 14.19 & \bf{662.22} & 
2.83\\CMT12Y & 682.02 & 13.99 & 
689.30 & 12.93 & \bf{662.22} & 
2.99\\\bf{PROM.} & 
\bf{785.95} & \bf{27.00} & \bf{796.33} & \bf{27.94} & \bf{751.25} & \bf{4.26}\\[1ex]\hline
\end{tabular}
\label{table:nonlin}
\end{table} \clearpage
\begin{table}[ht]
\caption{Resultados de la ejecución de la metaheurística ILS, utilizando instancias de SalhiNagy con la configuración -n 15.0 -LS 70.0}
\centering
\small
\begin{tabular}{c c c c c c c}
\hline\hline
Instancia & Costo mínimo & Tiempo(seg.) & Costo promedio & Tiempo promedio(seg.) & Costo ILS & \%Gap \\ [0.5ex]
\hline
CMT1X & 475.26 & 4.34 & 
485.46 & 3.38 & \bf{466.77} & 
1.82\\CMT1Y & 481.35 & 3.20 & 
484.10 & 3.21 & \bf{466.77} & 
3.12\\CMT2X & 710.07 & 7.60 & 
716.92 & 7.26 & \bf{684.21} & 
3.78\\CMT2Y & 707.44 & 9.03 & 
714.15 & 7.36 & \bf{684.21} & 
3.40\\CMT3X & 731.70 & 16.21 & 
742.18 & 16.43 & \bf{721.40} & 
1.43\\CMT3Y & 740.17 & 15.13 & 
747.96 & 15.23 & \bf{721.40} & 
2.60\\CMT4X & 902.54 & 49.64 & 
916.48 & 44.45 & \bf{852.83} & 
5.83\\CMT4Y & 884.88 & 40.24 & 
904.55 & 40.63 & \bf{852.46} & 
3.80\\CMT5X & 1082.12 & 91.86 & 
1106.05 & 95.26 & \bf{1030.55} & 
5.00\\CMT5Y & 1092.17 & 86.99 & 
1095.57 & 102.14 & \bf{1031.17} & 
5.92\\CMT11X & 892.34 & 31.33 & 
903.21 & 32.70 & \bf{839.39} & 
6.31\\CMT11Y & 881.48 & 26.31 & 
895.35 & 32.22 & \bf{841.88} & 
4.70\\CMT12X & 683.86 & 13.20 & 
686.20 & 13.97 & \bf{662.22} & 
3.27\\CMT12Y & 677.36 & 13.16 & 
687.69 & 13.82 & \bf{662.22} & 
2.29\\\bf{PROM.} & 
\bf{781.62} & \bf{29.16} & \bf{791.85} & \bf{30.58} & \bf{751.25} & \bf{3.80}\\[1ex]\hline
\end{tabular}
\label{table:nonlin}
\end{table} \clearpage
\begin{table}[ht]
\caption{Resultados de la ejecución de la metaheurística ILS, utilizando instancias de SalhiNagy con la configuración -n 15.0 -LS 80.0}
\centering
\small
\begin{tabular}{c c c c c c c}
\hline\hline
Instancia & Costo mínimo & Tiempo(seg.) & Costo promedio & Tiempo promedio(seg.) & Costo ILS & \%Gap \\ [0.5ex]
\hline
CMT1X & 478.36 & 3.05 & 
481.78 & 3.39 & \bf{466.77} & 
2.48\\CMT1Y & 482.00 & 2.94 & 
484.41 & 4.03 & \bf{466.77} & 
3.26\\CMT2X & 701.91 & 6.38 & 
712.60 & 7.57 & \bf{684.21} & 
2.59\\CMT2Y & 704.88 & 10.15 & 
712.70 & 7.63 & \bf{684.21} & 
3.02\\CMT3X & 743.73 & 14.66 & 
747.17 & 17.75 & \bf{721.40} & 
3.10\\CMT3Y & 731.57 & 20.33 & 
739.73 & 19.57 & \bf{721.40} & 
1.41\\CMT4X & 892.92 & 49.28 & 
902.73 & 48.09 & \bf{852.83} & 
4.70\\CMT4Y & 901.33 & 47.54 & 
911.64 & 46.09 & \bf{852.46} & 
5.73\\CMT5X & 1110.97 & 116.63 & 
1118.00 & 102.77 & \bf{1030.55} & 
7.80\\CMT5Y & 1101.65 & 96.35 & 
1114.53 & 112.58 & \bf{1031.17} & 
6.83\\CMT11X & 883.45 & 34.88 & 
898.17 & 36.09 & \bf{839.39} & 
5.25\\CMT11Y & 878.31 & 38.18 & 
886.20 & 34.47 & \bf{841.88} & 
4.33\\CMT12X & 691.35 & 15.08 & 
693.81 & 15.86 & \bf{662.22} & 
4.40\\CMT12Y & 673.67 & 19.01 & 
679.04 & 17.70 & \bf{662.22} & 
1.73\\\bf{PROM.} & 
\bf{784.01} & \bf{33.89} & \bf{791.61} & \bf{33.83} & \bf{751.25} & \bf{4.05}\\[1ex]\hline
\end{tabular}
\label{table:nonlin}
\end{table} \clearpage
\begin{table}[ht]
\caption{Resultados de la ejecución de la metaheurística ILS, utilizando instancias de SalhiNagy con la configuración -n 25.0 -LS 10.0}
\centering
\small
\begin{tabular}{c c c c c c c}
\hline\hline
Instancia & Costo mínimo & Tiempo(seg.) & Costo promedio & Tiempo promedio(seg.) & Costo ILS & \%Gap \\ [0.5ex]
\hline
CMT1X & 490.15 & 1.66 & 
494.18 & 1.33 & \bf{466.77} & 
5.01\\CMT1Y & 489.42 & 1.41 & 
492.26 & 1.55 & \bf{466.77} & 
4.85\\CMT2X & 704.22 & 3.82 & 
724.59 & 3.81 & \bf{684.21} & 
2.92\\CMT2Y & 703.82 & 3.96 & 
713.62 & 3.49 & \bf{684.21} & 
2.87\\CMT3X & 741.47 & 10.50 & 
745.74 & 10.06 & \bf{721.40} & 
2.78\\CMT3Y & 735.73 & 8.48 & 
739.81 & 10.60 & \bf{721.40} & 
1.99\\CMT4X & 892.45 & 43.86 & 
911.79 & 42.52 & \bf{852.83} & 
4.65\\CMT4Y & 902.31 & 31.57 & 
907.36 & 32.34 & \bf{852.46} & 
5.85\\CMT5X & 1102.21 & 73.59 & 
1109.08 & 85.14 & \bf{1030.55} & 
6.95\\CMT5Y & 1109.76 & 77.71 & 
1112.57 & 76.46 & \bf{1031.17} & 
7.62\\CMT11X & 890.23 & 24.01 & 
895.05 & 28.53 & \bf{839.39} & 
6.06\\CMT11Y & 888.47 & 24.83 & 
900.26 & 24.82 & \bf{841.88} & 
5.53\\CMT12X & 681.38 & 13.38 & 
688.92 & 11.01 & \bf{662.22} & 
2.89\\CMT12Y & 681.86 & 8.73 & 
691.88 & 9.89 & \bf{662.22} & 
2.97\\\bf{PROM.} & 
\bf{786.68} & \bf{23.39} & \bf{794.79} & \bf{24.40} & \bf{751.25} & \bf{4.50}\\[1ex]\hline
\end{tabular}
\label{table:nonlin}
\end{table} \clearpage
\begin{table}[ht]
\caption{Resultados de la ejecución de la metaheurística ILS, utilizando instancias de SalhiNagy con la configuración -n 25.0 -LS 20.0}
\centering
\small
\begin{tabular}{c c c c c c c}
\hline\hline
Instancia & Costo mínimo & Tiempo(seg.) & Costo promedio & Tiempo promedio(seg.) & Costo ILS & \%Gap \\ [0.5ex]
\hline
CMT1X & 480.08 & 1.68 & 
484.31 & 1.92 & \bf{466.77} & 
2.85\\CMT1Y & 488.44 & 2.14 & 
489.74 & 2.29 & \bf{466.77} & 
4.64\\CMT2X & 716.26 & 5.58 & 
723.76 & 5.10 & \bf{684.21} & 
4.68\\CMT2Y & 713.09 & 5.80 & 
717.26 & 5.53 & \bf{684.21} & 
4.22\\CMT3X & 731.72 & 12.10 & 
742.85 & 13.11 & \bf{721.40} & 
1.43\\CMT3Y & 738.54 & 12.26 & 
740.61 & 12.84 & \bf{721.40} & 
2.38\\CMT4X & 889.67 & 38.84 & 
901.94 & 40.45 & \bf{852.83} & 
4.32\\CMT4Y & 899.13 & 55.79 & 
903.86 & 45.58 & \bf{852.46} & 
5.47\\CMT5X & 1105.10 & 134.82 & 
1120.84 & 97.94 & \bf{1030.55} & 
7.23\\CMT5Y & 1107.39 & 81.12 & 
1117.60 & 94.31 & \bf{1031.17} & 
7.39\\CMT11X & 885.37 & 30.56 & 
907.00 & 32.92 & \bf{839.39} & 
5.48\\CMT11Y & 888.07 & 29.10 & 
906.77 & 31.89 & \bf{841.88} & 
5.49\\CMT12X & 681.36 & 10.46 & 
686.10 & 12.89 & \bf{662.22} & 
2.89\\CMT12Y & 677.05 & 11.45 & 
686.04 & 12.24 & \bf{662.22} & 
2.24\\\bf{PROM.} & 
\bf{785.81} & \bf{30.84} & \bf{794.91} & \bf{29.21} & \bf{751.25} & \bf{4.34}\\[1ex]\hline
\end{tabular}
\label{table:nonlin}
\end{table} \clearpage
\begin{table}[ht]
\caption{Resultados de la ejecución de la metaheurística ILS, utilizando instancias de SalhiNagy con la configuración -n 25.0 -LS 30.0}
\centering
\small
\begin{tabular}{c c c c c c c}
\hline\hline
Instancia & Costo mínimo & Tiempo(seg.) & Costo promedio & Tiempo promedio(seg.) & Costo ILS & \%Gap \\ [0.5ex]
\hline
CMT1X & 470.67 & 2.84 & 
475.77 & 2.94 & \bf{466.77} & 
0.84\\CMT1Y & 475.37 & 3.02 & 
488.05 & 3.22 & \bf{466.77} & 
1.84\\CMT2X & 710.84 & 6.51 & 
720.15 & 6.58 & \bf{684.21} & 
3.89\\CMT2Y & 705.23 & 7.23 & 
715.59 & 6.71 & \bf{684.21} & 
3.07\\CMT3X & 726.96 & 19.64 & 
731.96 & 17.45 & \bf{721.40} & 
0.77\\CMT3Y & 732.48 & 19.61 & 
742.38 & 18.11 & \bf{721.40} & 
1.54\\CMT4X & 900.50 & 41.08 & 
914.13 & 43.97 & \bf{852.83} & 
5.59\\CMT4Y & 899.98 & 44.90 & 
908.57 & 48.31 & \bf{852.46} & 
5.57\\CMT5X & 1092.74 & 101.37 & 
1103.72 & 117.66 & \bf{1030.55} & 
6.03\\CMT5Y & 1099.14 & 97.37 & 
1112.46 & 101.02 & \bf{1031.17} & 
6.59\\CMT11X & 882.82 & 30.56 & 
892.04 & 36.93 & \bf{839.39} & 
5.17\\CMT11Y & 880.05 & 44.79 & 
888.14 & 42.83 & \bf{841.88} & 
4.53\\CMT12X & 683.24 & 13.06 & 
687.65 & 13.37 & \bf{662.22} & 
3.17\\CMT12Y & 674.11 & 17.56 & 
681.28 & 15.33 & \bf{662.22} & 
1.80\\\bf{PROM.} & 
\bf{781.01} & \bf{32.11} & \bf{790.14} & \bf{33.89} & \bf{751.25} & \bf{3.60}\\[1ex]\hline
\end{tabular}
\label{table:nonlin}
\end{table} \clearpage
\begin{table}[ht]
\caption{Resultados de la ejecución de la metaheurística ILS, utilizando instancias de SalhiNagy con la configuración -n 25.0 -LS 40.0}
\centering
\small
\begin{tabular}{c c c c c c c}
\hline\hline
Instancia & Costo mínimo & Tiempo(seg.) & Costo promedio & Tiempo promedio(seg.) & Costo ILS & \%Gap \\ [0.5ex]
\hline
CMT1X & 474.87 & 3.20 & 
484.67 & 3.12 & \bf{466.77} & 
1.74\\CMT1Y & 483.62 & 3.31 & 
486.94 & 3.52 & \bf{466.77} & 
3.61\\CMT2X & 700.95 & 8.79 & 
712.74 & 9.11 & \bf{684.21} & 
2.45\\CMT2Y & 702.20 & 7.48 & 
712.05 & 7.17 & \bf{684.21} & 
2.63\\CMT3X & 731.81 & 18.62 & 
738.23 & 18.23 & \bf{721.40} & 
1.44\\CMT3Y & 728.82 & 16.50 & 
736.32 & 17.63 & \bf{721.40} & 
1.03\\CMT4X & 897.65 & 60.45 & 
907.02 & 58.60 & \bf{852.83} & 
5.26\\CMT4Y & 889.25 & 48.09 & 
898.76 & 47.65 & \bf{852.46} & 
4.32\\CMT5X & 1114.50 & 105.38 & 
1119.56 & 112.93 & \bf{1030.55} & 
8.15\\CMT5Y & 1113.74 & 115.56 & 
1118.84 & 126.28 & \bf{1031.17} & 
8.01\\CMT11X & 887.06 & 36.75 & 
894.50 & 39.01 & \bf{839.39} & 
5.68\\CMT11Y & 856.44 & 38.01 & 
889.32 & 37.06 & \bf{841.88} & 
1.73\\CMT12X & 676.14 & 16.58 & 
680.85 & 17.56 & \bf{662.22} & 
2.10\\CMT12Y & 680.75 & 15.81 & 
685.56 & 16.95 & \bf{662.22} & 
2.80\\\bf{PROM.} & 
\bf{781.27} & \bf{35.32} & \bf{790.38} & \bf{36.77} & \bf{751.25} & \bf{3.64}\\[1ex]\hline
\end{tabular}
\label{table:nonlin}
\end{table} \clearpage
\begin{table}[ht]
\caption{Resultados de la ejecución de la metaheurística ILS, utilizando instancias de SalhiNagy con la configuración -n 25.0 -LS 50.0}
\centering
\small
\begin{tabular}{c c c c c c c}
\hline\hline
Instancia & Costo mínimo & Tiempo(seg.) & Costo promedio & Tiempo promedio(seg.) & Costo ILS & \%Gap \\ [0.5ex]
\hline
CMT1X & 475.58 & 3.55 & 
480.69 & 3.47 & \bf{466.77} & 
1.89\\CMT1Y & 478.54 & 4.95 & 
489.57 & 4.04 & \bf{466.77} & 
2.52\\CMT2X & 709.51 & 8.45 & 
716.02 & 9.20 & \bf{684.21} & 
3.70\\CMT2Y & 709.97 & 10.26 & 
714.93 & 8.85 & \bf{684.21} & 
3.76\\CMT3X & 728.82 & 19.54 & 
742.10 & 19.73 & \bf{721.40} & 
1.03\\CMT3Y & 729.45 & 24.63 & 
736.70 & 21.90 & \bf{721.40} & 
1.12\\CMT4X & 905.49 & 52.92 & 
912.29 & 62.55 & \bf{852.83} & 
6.17\\CMT4Y & 900.75 & 54.95 & 
909.07 & 60.89 & \bf{852.46} & 
5.66\\CMT5X & 1107.29 & 126.22 & 
1120.72 & 140.26 & \bf{1030.55} & 
7.45\\CMT5Y & 1114.55 & 128.69 & 
1117.46 & 133.17 & \bf{1031.17} & 
8.09\\CMT11X & 883.35 & 56.40 & 
901.88 & 45.45 & \bf{839.39} & 
5.24\\CMT11Y & 882.11 & 51.88 & 
890.33 & 46.38 & \bf{841.88} & 
4.78\\CMT12X & 676.38 & 18.48 & 
684.46 & 20.20 & \bf{662.22} & 
2.14\\CMT12Y & 679.40 & 19.25 & 
682.63 & 18.06 & \bf{662.22} & 
2.59\\\bf{PROM.} & 
\bf{784.37} & \bf{41.44} & \bf{792.77} & \bf{42.44} & \bf{751.25} & \bf{4.01}\\[1ex]\hline
\end{tabular}
\label{table:nonlin}
\end{table} \clearpage
\begin{table}[ht]
\caption{Resultados de la ejecución de la metaheurística ILS, utilizando instancias de SalhiNagy con la configuración -n 25.0 -LS 60.0}
\centering
\small
\begin{tabular}{c c c c c c c}
\hline\hline
Instancia & Costo mínimo & Tiempo(seg.) & Costo promedio & Tiempo promedio(seg.) & Costo ILS & \%Gap \\ [0.5ex]
\hline
CMT1X & 470.67 & 4.70 & 
475.33 & 4.46 & \bf{466.77} & 
0.84\\CMT1Y & 476.32 & 3.53 & 
485.00 & 3.94 & \bf{466.77} & 
2.05\\CMT2X & 719.00 & 9.02 & 
720.37 & 9.96 & \bf{684.21} & 
5.08\\CMT2Y & 704.49 & 10.22 & 
709.54 & 10.14 & \bf{684.21} & 
2.96\\CMT3X & 738.80 & 22.58 & 
744.17 & 23.48 & \bf{721.40} & 
2.41\\CMT3Y & 729.20 & 27.21 & 
737.64 & 25.27 & \bf{721.40} & 
1.08\\CMT4X & 900.67 & 61.23 & 
906.35 & 60.92 & \bf{852.83} & 
5.61\\CMT4Y & 900.03 & 85.55 & 
907.50 & 71.91 & \bf{852.46} & 
5.58\\CMT5X & 1104.11 & 150.61 & 
1110.77 & 150.23 & \bf{1030.55} & 
7.14\\CMT5Y & 1098.76 & 147.48 & 
1103.68 & 168.17 & \bf{1031.17} & 
6.55\\CMT11X & 880.32 & 55.62 & 
889.77 & 51.03 & \bf{839.39} & 
4.88\\CMT11Y & 890.05 & 60.99 & 
903.24 & 50.20 & \bf{841.88} & 
5.72\\CMT12X & 683.03 & 22.69 & 
685.32 & 22.77 & \bf{662.22} & 
3.14\\CMT12Y & 681.96 & 18.42 & 
683.26 & 21.20 & \bf{662.22} & 
2.98\\\bf{PROM.} & 
\bf{784.10} & \bf{48.56} & \bf{790.14} & \bf{48.12} & \bf{751.25} & \bf{4.00}\\[1ex]\hline
\end{tabular}
\label{table:nonlin}
\end{table} \clearpage
\begin{table}[ht]
\caption{Resultados de la ejecución de la metaheurística ILS, utilizando instancias de SalhiNagy con la configuración -n 25.0 -LS 70.0}
\centering
\small
\begin{tabular}{c c c c c c c}
\hline\hline
Instancia & Costo mínimo & Tiempo(seg.) & Costo promedio & Tiempo promedio(seg.) & Costo ILS & \%Gap \\ [0.5ex]
\hline
CMT1X & 473.58 & 5.30 & 
483.88 & 5.04 & \bf{466.77} & 
1.46\\CMT1Y & 470.67 & 6.72 & 
478.57 & 6.29 & \bf{466.77} & 
0.84\\CMT2X & 704.81 & 11.58 & 
710.85 & 12.01 & \bf{684.21} & 
3.01\\CMT2Y & 694.72 & 11.70 & 
707.78 & 12.37 & \bf{684.21} & 
1.54\\CMT3X & 733.81 & 26.76 & 
741.53 & 28.99 & \bf{721.40} & 
1.72\\CMT3Y & 727.29 & 26.07 & 
733.97 & 27.26 & \bf{721.40} & 
0.82\\CMT4X & 896.90 & 74.00 & 
908.06 & 75.66 & \bf{852.83} & 
5.17\\CMT4Y & 900.29 & 81.60 & 
902.37 & 77.46 & \bf{852.46} & 
5.61\\CMT5X & 1109.05 & 147.64 & 
1117.55 & 149.89 & \bf{1030.55} & 
7.62\\CMT5Y & 1094.29 & 162.55 & 
1105.45 & 175.39 & \bf{1031.17} & 
6.12\\CMT11X & 882.54 & 52.31 & 
896.37 & 54.47 & \bf{839.39} & 
5.14\\CMT11Y & 885.35 & 49.60 & 
895.38 & 53.16 & \bf{841.88} & 
5.16\\CMT12X & 686.96 & 25.22 & 
690.09 & 24.14 & \bf{662.22} & 
3.74\\CMT12Y & 678.17 & 29.93 & 
681.74 & 25.16 & \bf{662.22} & 
2.41\\\bf{PROM.} & 
\bf{781.32} & \bf{50.78} & \bf{789.54} & \bf{51.95} & \bf{751.25} & \bf{3.60}\\[1ex]\hline
\end{tabular}
\label{table:nonlin}
\end{table} \clearpage
\begin{table}[ht]
\caption{Resultados de la ejecución de la metaheurística ILS, utilizando instancias de SalhiNagy con la configuración -n 25.0 -LS 80.0}
\centering
\small
\begin{tabular}{c c c c c c c}
\hline\hline
Instancia & Costo mínimo & Tiempo(seg.) & Costo promedio & Tiempo promedio(seg.) & Costo ILS & \%Gap \\ [0.5ex]
\hline
CMT1X & 479.19 & 5.09 & 
483.94 & 5.32 & \bf{466.77} & 
2.66\\CMT1Y & 477.72 & 5.56 & 
483.01 & 6.59 & \bf{466.77} & 
2.35\\CMT2X & 703.84 & 12.28 & 
714.18 & 13.04 & \bf{684.21} & 
2.87\\CMT2Y & 713.21 & 13.11 & 
715.29 & 12.09 & \bf{684.21} & 
4.24\\CMT3X & 737.09 & 30.42 & 
742.26 & 28.95 & \bf{721.40} & 
2.17\\CMT3Y & 735.96 & 30.71 & 
739.75 & 30.63 & \bf{721.40} & 
2.02\\CMT4X & 907.23 & 75.74 & 
911.97 & 75.89 & \bf{852.83} & 
6.38\\CMT4Y & 897.81 & 89.16 & 
906.46 & 78.70 & \bf{852.46} & 
5.32\\CMT5X & 1079.55 & 167.50 & 
1101.63 & 179.78 & \bf{1030.55} & 
4.75\\CMT5Y & 1095.85 & 165.92 & 
1105.98 & 182.75 & \bf{1031.17} & 
6.27\\CMT11X & 885.66 & 52.77 & 
893.21 & 57.49 & \bf{839.39} & 
5.51\\CMT11Y & 869.25 & 56.86 & 
907.09 & 59.48 & \bf{841.88} & 
3.25\\CMT12X & 673.17 & 29.22 & 
685.43 & 29.00 & \bf{662.22} & 
1.65\\CMT12Y & 675.11 & 25.01 & 
681.58 & 26.20 & \bf{662.22} & 
1.95\\\bf{PROM.} & 
\bf{780.76} & \bf{54.24} & \bf{790.84} & \bf{56.14} & \bf{751.25} & \bf{3.67}\\[1ex]\hline
\end{tabular}
\label{table:nonlin}
\end{table} \clearpage
\begin{table}[ht]
\caption{Resultados de la ejecución de la metaheurística ILS, utilizando instancias de SalhiNagy con la configuración -n 35.0 -LS 10.0}
\centering
\small
\begin{tabular}{c c c c c c c}
\hline\hline
Instancia & Costo mínimo & Tiempo(seg.) & Costo promedio & Tiempo promedio(seg.) & Costo ILS & \%Gap \\ [0.5ex]
\hline
CMT1X & 480.65 & 1.66 & 
487.08 & 1.76 & \bf{466.77} & 
2.97\\CMT1Y & 482.31 & 3.00 & 
485.87 & 2.19 & \bf{466.77} & 
3.33\\CMT2X & 702.99 & 7.47 & 
713.05 & 5.81 & \bf{684.21} & 
2.74\\CMT2Y & 711.53 & 5.03 & 
714.12 & 4.86 & \bf{684.21} & 
3.99\\CMT3X & 736.81 & 19.40 & 
740.75 & 15.01 & \bf{721.40} & 
2.14\\CMT3Y & 730.68 & 13.74 & 
735.83 & 16.21 & \bf{721.40} & 
1.29\\CMT4X & 893.70 & 42.45 & 
907.88 & 42.98 & \bf{852.83} & 
4.79\\CMT4Y & 905.89 & 39.80 & 
908.34 & 48.09 & \bf{852.46} & 
6.27\\CMT5X & 1108.43 & 106.01 & 
1116.91 & 105.17 & \bf{1030.55} & 
7.56\\CMT5Y & 1104.68 & 109.46 & 
1117.45 & 124.38 & \bf{1031.17} & 
7.13\\CMT11X & 896.15 & 54.29 & 
901.72 & 39.33 & \bf{839.39} & 
6.76\\CMT11Y & 888.02 & 31.38 & 
896.26 & 33.45 & \bf{841.88} & 
5.48\\CMT12X & 685.46 & 12.57 & 
688.43 & 14.85 & \bf{662.22} & 
3.51\\CMT12Y & 673.58 & 10.43 & 
680.89 & 11.84 & \bf{662.22} & 
1.72\\\bf{PROM.} & 
\bf{785.78} & \bf{32.62} & \bf{792.47} & \bf{33.28} & \bf{751.25} & \bf{4.26}\\[1ex]\hline
\end{tabular}
\label{table:nonlin}
\end{table} \clearpage
\begin{table}[ht]
\caption{Resultados de la ejecución de la metaheurística ILS, utilizando instancias de SalhiNagy con la configuración -n 35.0 -LS 20.0}
\centering
\small
\begin{tabular}{c c c c c c c}
\hline\hline
Instancia & Costo mínimo & Tiempo(seg.) & Costo promedio & Tiempo promedio(seg.) & Costo ILS & \%Gap \\ [0.5ex]
\hline
CMT1X & 475.20 & 2.30 & 
480.70 & 2.58 & \bf{466.77} & 
1.81\\CMT1Y & 475.91 & 3.30 & 
488.33 & 3.10 & \bf{466.77} & 
1.96\\CMT2X & 709.22 & 6.94 & 
714.07 & 7.55 & \bf{684.21} & 
3.66\\CMT2Y & 710.10 & 7.58 & 
714.74 & 7.23 & \bf{684.21} & 
3.78\\CMT3X & 736.97 & 17.89 & 
743.93 & 19.11 & \bf{721.40} & 
2.16\\CMT3Y & 726.94 & 17.33 & 
737.33 & 17.72 & \bf{721.40} & 
0.77\\CMT4X & 896.89 & 50.19 & 
907.10 & 56.79 & \bf{852.83} & 
5.17\\CMT4Y & 894.91 & 52.91 & 
904.43 & 51.61 & \bf{852.46} & 
4.98\\CMT5X & 1095.20 & 116.77 & 
1105.70 & 123.27 & \bf{1030.55} & 
6.27\\CMT5Y & 1100.66 & 121.22 & 
1108.82 & 132.37 & \bf{1031.17} & 
6.74\\CMT11X & 887.34 & 59.45 & 
895.92 & 46.10 & \bf{839.39} & 
5.71\\CMT11Y & 884.08 & 57.23 & 
896.14 & 49.60 & \bf{841.88} & 
5.01\\CMT12X & 680.02 & 21.09 & 
688.53 & 17.42 & \bf{662.22} & 
2.69\\CMT12Y & 674.10 & 15.09 & 
683.21 & 16.52 & \bf{662.22} & 
1.79\\\bf{PROM.} & 
\bf{781.97} & \bf{39.23} & \bf{790.64} & \bf{39.36} & \bf{751.25} & \bf{3.75}\\[1ex]\hline
\end{tabular}
\label{table:nonlin}
\end{table} \clearpage
\begin{table}[ht]
\caption{Resultados de la ejecución de la metaheurística ILS, utilizando instancias de SalhiNagy con la configuración -n 35.0 -LS 30.0}
\centering
\small
\begin{tabular}{c c c c c c c}
\hline\hline
Instancia & Costo mínimo & Tiempo(seg.) & Costo promedio & Tiempo promedio(seg.) & Costo ILS & \%Gap \\ [0.5ex]
\hline
CMT1X & 471.25 & 3.64 & 
476.93 & 3.95 & \bf{466.77} & 
0.96\\CMT1Y & 480.50 & 3.70 & 
486.17 & 3.72 & \bf{466.77} & 
2.94\\CMT2X & 707.55 & 8.03 & 
713.78 & 9.37 & \bf{684.21} & 
3.41\\CMT2Y & 712.85 & 8.66 & 
718.33 & 8.49 & \bf{684.21} & 
4.19\\CMT3X & 734.04 & 20.23 & 
736.78 & 23.63 & \bf{721.40} & 
1.75\\CMT3Y & 732.60 & 21.84 & 
739.58 & 23.14 & \bf{721.40} & 
1.55\\CMT4X & 891.04 & 56.28 & 
906.68 & 68.34 & \bf{852.83} & 
4.48\\CMT4Y & 890.20 & 60.40 & 
900.87 & 69.36 & \bf{852.46} & 
4.43\\CMT5X & 1085.84 & 146.25 & 
1107.93 & 138.92 & \bf{1030.55} & 
5.37\\CMT5Y & 1105.80 & 143.08 & 
1112.36 & 151.21 & \bf{1031.17} & 
7.24\\CMT11X & 894.17 & 44.84 & 
896.62 & 44.33 & \bf{839.39} & 
6.53\\CMT11Y & 885.63 & 46.07 & 
903.72 & 50.86 & \bf{841.88} & 
5.20\\CMT12X & 682.37 & 18.88 & 
684.55 & 22.17 & \bf{662.22} & 
3.04\\CMT12Y & 673.57 & 24.15 & 
682.07 & 23.50 & \bf{662.22} & 
1.71\\\bf{PROM.} & 
\bf{781.96} & \bf{43.29} & \bf{790.45} & \bf{45.79} & \bf{751.25} & \bf{3.77}\\[1ex]\hline
\end{tabular}
\label{table:nonlin}
\end{table} \clearpage
\begin{table}[ht]
\caption{Resultados de la ejecución de la metaheurística ILS, utilizando instancias de SalhiNagy con la configuración -n 35.0 -LS 40.0}
\centering
\small
\begin{tabular}{c c c c c c c}
\hline\hline
Instancia & Costo mínimo & Tiempo(seg.) & Costo promedio & Tiempo promedio(seg.) & Costo ILS & \%Gap \\ [0.5ex]
\hline
CMT1X & 479.19 & 4.59 & 
482.98 & 4.58 & \bf{466.77} & 
2.66\\CMT1Y & 478.67 & 3.63 & 
484.44 & 4.65 & \bf{466.77} & 
2.55\\CMT2X & 696.69 & 10.98 & 
712.02 & 11.52 & \bf{684.21} & 
1.82\\CMT2Y & 707.94 & 8.82 & 
711.63 & 10.74 & \bf{684.21} & 
3.47\\CMT3X & 730.15 & 27.22 & 
736.46 & 26.86 & \bf{721.40} & 
1.21\\CMT3Y & 737.76 & 24.24 & 
740.97 & 25.89 & \bf{721.40} & 
2.27\\CMT4X & 891.81 & 67.86 & 
901.27 & 72.48 & \bf{852.83} & 
4.57\\CMT4Y & 879.41 & 69.09 & 
897.97 & 68.81 & \bf{852.46} & 
3.16\\CMT5X & 1105.26 & 210.24 & 
1115.35 & 184.05 & \bf{1030.55} & 
7.25\\CMT5Y & 1101.83 & 160.60 & 
1110.40 & 156.52 & \bf{1031.17} & 
6.85\\CMT11X & 872.84 & 51.69 & 
889.76 & 60.92 & \bf{839.39} & 
3.99\\CMT11Y & 877.90 & 56.41 & 
891.41 & 57.51 & \bf{841.88} & 
4.28\\CMT12X & 678.19 & 22.57 & 
686.68 & 22.11 & \bf{662.22} & 
2.41\\CMT12Y & 679.25 & 22.55 & 
683.32 & 23.64 & \bf{662.22} & 
2.57\\\bf{PROM.} & 
\bf{779.78} & \bf{52.89} & \bf{788.90} & \bf{52.16} & \bf{751.25} & \bf{3.50}\\[1ex]\hline
\end{tabular}
\label{table:nonlin}
\end{table} \clearpage
\begin{table}[ht]
\caption{Resultados de la ejecución de la metaheurística ILS, utilizando instancias de SalhiNagy con la configuración -n 35.0 -LS 50.0}
\centering
\small
\begin{tabular}{c c c c c c c}
\hline\hline
Instancia & Costo mínimo & Tiempo(seg.) & Costo promedio & Tiempo promedio(seg.) & Costo ILS & \%Gap \\ [0.5ex]
\hline
CMT1X & 472.58 & 7.24 & 
477.00 & 5.71 & \bf{466.77} & 
1.24\\CMT1Y & 486.09 & 5.95 & 
489.19 & 6.09 & \bf{466.77} & 
4.14\\CMT2X & 694.46 & 12.92 & 
714.31 & 12.18 & \bf{684.21} & 
1.50\\CMT2Y & 703.05 & 13.36 & 
708.58 & 12.71 & \bf{684.21} & 
2.75\\CMT3X & 723.97 & 33.97 & 
734.62 & 31.53 & \bf{721.40} & 
0.36\\CMT3Y & 729.13 & 33.52 & 
735.13 & 30.86 & \bf{721.40} & 
1.07\\CMT4X & 897.02 & 75.86 & 
904.77 & 89.86 & \bf{852.83} & 
5.18\\CMT4Y & 899.71 & 76.75 & 
909.27 & 80.83 & \bf{852.46} & 
5.54\\CMT5X & 1100.51 & 175.85 & 
1105.17 & 184.79 & \bf{1030.55} & 
6.79\\CMT5Y & 1096.60 & 215.68 & 
1105.88 & 178.39 & \bf{1031.17} & 
6.35\\CMT11X & 880.26 & 55.39 & 
892.43 & 55.45 & \bf{839.39} & 
4.87\\CMT11Y & 889.14 & 54.53 & 
895.15 & 54.77 & \bf{841.88} & 
5.61\\CMT12X & 677.78 & 26.35 & 
683.68 & 27.67 & \bf{662.22} & 
2.35\\CMT12Y & 675.21 & 25.42 & 
681.25 & 26.82 & \bf{662.22} & 
1.96\\\bf{PROM.} & 
\bf{780.39} & \bf{58.06} & \bf{788.32} & \bf{56.97} & \bf{751.25} & \bf{3.55}\\[1ex]\hline
\end{tabular}
\label{table:nonlin}
\end{table} \clearpage
\begin{table}[ht]
\caption{Resultados de la ejecución de la metaheurística ILS, utilizando instancias de SalhiNagy con la configuración -n 35.0 -LS 60.0}
\centering
\small
\begin{tabular}{c c c c c c c}
\hline\hline
Instancia & Costo mínimo & Tiempo(seg.) & Costo promedio & Tiempo promedio(seg.) & Costo ILS & \%Gap \\ [0.5ex]
\hline
CMT1X & 471.25 & 7.21 & 
478.02 & 7.39 & \bf{466.77} & 
0.96\\CMT1Y & 475.22 & 7.12 & 
479.76 & 6.74 & \bf{466.77} & 
1.81\\CMT2X & 704.62 & 15.17 & 
715.20 & 14.76 & \bf{684.21} & 
2.98\\CMT2Y & 707.57 & 15.16 & 
710.74 & 14.38 & \bf{684.21} & 
3.41\\CMT3X & 731.07 & 34.90 & 
735.83 & 34.72 & \bf{721.40} & 
1.34\\CMT3Y & 726.61 & 32.47 & 
738.35 & 36.31 & \bf{721.40} & 
0.72\\CMT4X & 898.24 & 77.33 & 
904.94 & 88.78 & \bf{852.83} & 
5.32\\CMT4Y & 872.20 & 82.90 & 
893.86 & 94.88 & \bf{852.46} & 
2.32\\CMT5X & 1109.17 & 176.76 & 
1116.39 & 184.71 & \bf{1030.55} & 
7.63\\CMT5Y & 1089.72 & 208.38 & 
1098.62 & 206.40 & \bf{1031.17} & 
5.68\\CMT11X & 852.51 & 65.71 & 
886.16 & 67.14 & \bf{839.39} & 
1.56\\CMT11Y & 883.44 & 55.95 & 
894.98 & 64.75 & \bf{841.88} & 
4.94\\CMT12X & 674.28 & 31.42 & 
678.00 & 30.08 & \bf{662.22} & 
1.82\\CMT12Y & 676.66 & 27.75 & 
681.34 & 28.00 & \bf{662.22} & 
2.18\\\bf{PROM.} & 
\bf{776.61} & \bf{59.87} & \bf{786.58} & \bf{62.79} & \bf{751.25} & \bf{3.05}\\[1ex]\hline
\end{tabular}
\label{table:nonlin}
\end{table} \clearpage
\begin{table}[ht]
\caption{Resultados de la ejecución de la metaheurística ILS, utilizando instancias de SalhiNagy con la configuración -n 35.0 -LS 70.0}
\centering
\small
\begin{tabular}{c c c c c c c}
\hline\hline
Instancia & Costo mínimo & Tiempo(seg.) & Costo promedio & Tiempo promedio(seg.) & Costo ILS & \%Gap \\ [0.5ex]
\hline
CMT1X & 472.58 & 8.10 & 
478.69 & 7.31 & \bf{466.77} & 
1.24\\CMT1Y & 474.91 & 7.36 & 
479.20 & 7.27 & \bf{466.77} & 
1.74\\CMT2X & 702.26 & 17.80 & 
714.93 & 15.41 & \bf{684.21} & 
2.64\\CMT2Y & 696.36 & 17.91 & 
707.10 & 16.48 & \bf{684.21} & 
1.78\\CMT3X & 730.78 & 47.65 & 
735.71 & 38.80 & \bf{721.40} & 
1.30\\CMT3Y & 732.10 & 37.11 & 
736.29 & 36.57 & \bf{721.40} & 
1.48\\CMT4X & 890.31 & 111.61 & 
903.22 & 106.92 & \bf{852.83} & 
4.39\\CMT4Y & 901.78 & 103.32 & 
908.63 & 101.89 & \bf{852.46} & 
5.79\\CMT5X & 1068.98 & 216.80 & 
1096.78 & 241.09 & \bf{1030.55} & 
3.73\\CMT5Y & 1098.63 & 223.68 & 
1106.12 & 228.23 & \bf{1031.17} & 
6.54\\CMT11X & 880.79 & 71.90 & 
892.37 & 71.11 & \bf{839.39} & 
4.93\\CMT11Y & 860.42 & 71.38 & 
887.13 & 71.57 & \bf{841.88} & 
2.20\\CMT12X & 678.40 & 32.83 & 
681.12 & 34.23 & \bf{662.22} & 
2.44\\CMT12Y & 683.51 & 35.25 & 
686.01 & 32.00 & \bf{662.22} & 
3.21\\\bf{PROM.} & 
\bf{776.56} & \bf{71.62} & \bf{786.66} & \bf{72.06} & \bf{751.25} & \bf{3.10}\\[1ex]\hline
\end{tabular}
\label{table:nonlin}
\end{table} \clearpage
\begin{table}[ht]
\caption{Resultados de la ejecución de la metaheurística ILS, utilizando instancias de SalhiNagy con la configuración -n 35.0 -LS 80.0}
\centering
\small
\begin{tabular}{c c c c c c c}
\hline\hline
Instancia & Costo mínimo & Tiempo(seg.) & Costo promedio & Tiempo promedio(seg.) & Costo ILS & \%Gap \\ [0.5ex]
\hline
CMT1X & 476.66 & 7.34 & 
478.83 & 7.03 & \bf{466.77} & 
2.12\\CMT1Y & 475.72 & 8.42 & 
481.51 & 8.33 & \bf{466.77} & 
1.92\\CMT2X & 712.07 & 16.75 & 
718.12 & 18.12 & \bf{684.21} & 
4.07\\CMT2Y & 690.24 & 20.17 & 
705.73 & 19.14 & \bf{684.21} & 
0.88\\CMT3X & 730.80 & 39.83 & 
738.58 & 39.59 & \bf{721.40} & 
1.30\\CMT3Y & 731.74 & 41.82 & 
733.20 & 44.49 & \bf{721.40} & 
1.43\\CMT4X & 895.51 & 97.50 & 
900.15 & 108.87 & \bf{852.83} & 
5.00\\CMT4Y & 893.75 & 99.69 & 
904.84 & 97.93 & \bf{852.46} & 
4.84\\CMT5X & 1103.34 & 274.41 & 
1106.00 & 237.80 & \bf{1030.55} & 
7.06\\CMT5Y & 1095.28 & 231.64 & 
1099.38 & 246.88 & \bf{1031.17} & 
6.22\\CMT11X & 874.94 & 106.31 & 
884.56 & 88.19 & \bf{839.39} & 
4.24\\CMT11Y & 884.56 & 79.52 & 
889.35 & 78.45 & \bf{841.88} & 
5.07\\CMT12X & 675.24 & 32.27 & 
680.92 & 35.58 & \bf{662.22} & 
1.97\\CMT12Y & 674.59 & 39.05 & 
683.78 & 36.30 & \bf{662.22} & 
1.87\\\bf{PROM.} & 
\bf{779.60} & \bf{78.19} & \bf{786.07} & \bf{76.19} & \bf{751.25} & \bf{3.43}\\[1ex]\hline
\end{tabular}
\label{table:nonlin}
\end{table} \clearpage
\begin{table}[ht]
\caption{Resultados de la ejecución de la metaheurística ILS, utilizando instancias de SalhiNagy con la configuración -n 45.0 -LS 10.0}
\centering
\small
\begin{tabular}{c c c c c c c}
\hline\hline
Instancia & Costo mínimo & Tiempo(seg.) & Costo promedio & Tiempo promedio(seg.) & Costo ILS & \%Gap \\ [0.5ex]
\hline
CMT1X & 472.58 & 2.90 & 
483.95 & 2.30 & \bf{466.77} & 
1.24\\CMT1Y & 486.98 & 2.28 & 
488.73 & 2.59 & \bf{466.77} & 
4.33\\CMT2X & 712.31 & 7.48 & 
719.32 & 6.84 & \bf{684.21} & 
4.11\\CMT2Y & 712.10 & 6.62 & 
715.02 & 6.90 & \bf{684.21} & 
4.08\\CMT3X & 738.36 & 18.26 & 
740.20 & 17.47 & \bf{721.40} & 
2.35\\CMT3Y & 727.83 & 18.83 & 
737.43 & 19.70 & \bf{721.40} & 
0.89\\CMT4X & 898.88 & 57.44 & 
904.49 & 64.51 & \bf{852.83} & 
5.40\\CMT4Y & 890.39 & 54.91 & 
902.96 & 55.83 & \bf{852.46} & 
4.45\\CMT5X & 1107.23 & 132.19 & 
1109.94 & 153.86 & \bf{1030.55} & 
7.44\\CMT5Y & 1084.64 & 208.57 & 
1102.72 & 156.24 & \bf{1031.17} & 
5.19\\CMT11X & 888.61 & 42.21 & 
895.32 & 43.45 & \bf{839.39} & 
5.86\\CMT11Y & 877.15 & 42.23 & 
902.29 & 43.63 & \bf{841.88} & 
4.19\\CMT12X & 681.01 & 17.82 & 
685.77 & 16.09 & \bf{662.22} & 
2.84\\CMT12Y & 676.34 & 15.62 & 
681.71 & 18.20 & \bf{662.22} & 
2.13\\\bf{PROM.} & 
\bf{782.46} & \bf{44.81} & \bf{790.70} & \bf{43.40} & \bf{751.25} & \bf{3.89}\\[1ex]\hline
\end{tabular}
\label{table:nonlin}
\end{table} \clearpage
\begin{table}[ht]
\caption{Resultados de la ejecución de la metaheurística ILS, utilizando instancias de SalhiNagy con la configuración -n 45.0 -LS 20.0}
\centering
\small
\begin{tabular}{c c c c c c c}
\hline\hline
Instancia & Costo mínimo & Tiempo(seg.) & Costo promedio & Tiempo promedio(seg.) & Costo ILS & \%Gap \\ [0.5ex]
\hline
CMT1X & 475.95 & 3.99 & 
480.81 & 3.59 & \bf{466.77} & 
1.97\\CMT1Y & 473.62 & 2.84 & 
487.29 & 3.56 & \bf{466.77} & 
1.47\\CMT2X & 709.63 & 8.68 & 
713.97 & 8.79 & \bf{684.21} & 
3.72\\CMT2Y & 702.28 & 9.67 & 
707.51 & 9.51 & \bf{684.21} & 
2.64\\CMT3X & 727.54 & 21.90 & 
738.73 & 22.73 & \bf{721.40} & 
0.85\\CMT3Y & 732.84 & 22.30 & 
739.32 & 24.58 & \bf{721.40} & 
1.59\\CMT4X & 892.30 & 63.59 & 
899.92 & 71.36 & \bf{852.83} & 
4.63\\CMT4Y & 889.41 & 62.53 & 
904.05 & 65.02 & \bf{852.46} & 
4.33\\CMT5X & 1104.84 & 216.95 & 
1109.93 & 176.45 & \bf{1030.55} & 
7.21\\CMT5Y & 1088.07 & 220.81 & 
1106.46 & 175.90 & \bf{1031.17} & 
5.52\\CMT11X & 888.82 & 52.90 & 
893.28 & 57.02 & \bf{839.39} & 
5.89\\CMT11Y & 881.49 & 51.88 & 
897.75 & 56.74 & \bf{841.88} & 
4.70\\CMT12X & 671.94 & 19.77 & 
678.29 & 19.82 & \bf{662.22} & 
1.47\\CMT12Y & 675.22 & 20.73 & 
684.42 & 23.53 & \bf{662.22} & 
1.96\\\bf{PROM.} & 
\bf{779.57} & \bf{55.61} & \bf{788.70} & \bf{51.33} & \bf{751.25} & \bf{3.42}\\[1ex]\hline
\end{tabular}
\label{table:nonlin}
\end{table} \clearpage
\begin{table}[ht]
\caption{Resultados de la ejecución de la metaheurística ILS, utilizando instancias de SalhiNagy con la configuración -n 45.0 -LS 30.0}
\centering
\small
\begin{tabular}{c c c c c c c}
\hline\hline
Instancia & Costo mínimo & Tiempo(seg.) & Costo promedio & Tiempo promedio(seg.) & Costo ILS & \%Gap \\ [0.5ex]
\hline
CMT1X & 478.54 & 5.41 & 
484.64 & 4.99 & \bf{466.77} & 
2.52\\CMT1Y & 481.74 & 4.28 & 
488.26 & 4.88 & \bf{466.77} & 
3.21\\CMT2X & 704.47 & 11.77 & 
709.52 & 11.16 & \bf{684.21} & 
2.96\\CMT2Y & 706.88 & 12.47 & 
709.90 & 11.32 & \bf{684.21} & 
3.31\\CMT3X & 729.25 & 27.58 & 
733.91 & 27.29 & \bf{721.40} & 
1.09\\CMT3Y & 725.87 & 35.78 & 
734.03 & 28.96 & \bf{721.40} & 
0.62\\CMT4X & 884.84 & 76.26 & 
893.29 & 75.05 & \bf{852.83} & 
3.75\\CMT4Y & 885.76 & 79.02 & 
898.56 & 92.55 & \bf{852.46} & 
3.91\\CMT5X & 1088.14 & 175.41 & 
1103.87 & 196.63 & \bf{1030.55} & 
5.59\\CMT5Y & 1095.85 & 184.22 & 
1100.96 & 206.82 & \bf{1031.17} & 
6.27\\CMT11X & 885.18 & 59.06 & 
891.29 & 59.30 & \bf{839.39} & 
5.46\\CMT11Y & 867.28 & 54.30 & 
883.41 & 58.33 & \bf{841.88} & 
3.02\\CMT12X & 679.44 & 24.39 & 
680.31 & 25.76 & \bf{662.22} & 
2.60\\CMT12Y & 678.28 & 30.81 & 
686.12 & 26.38 & \bf{662.22} & 
2.43\\\bf{PROM.} & 
\bf{777.97} & \bf{55.77} & \bf{785.58} & \bf{59.24} & \bf{751.25} & \bf{3.34}\\[1ex]\hline
\end{tabular}
\label{table:nonlin}
\end{table} \clearpage
\begin{table}[ht]
\caption{Resultados de la ejecución de la metaheurística ILS, utilizando instancias de SalhiNagy con la configuración -n 45.0 -LS 40.0}
\centering
\small
\begin{tabular}{c c c c c c c}
\hline\hline
Instancia & Costo mínimo & Tiempo(seg.) & Costo promedio & Tiempo promedio(seg.) & Costo ILS & \%Gap \\ [0.5ex]
\hline
CMT1X & 471.25 & 4.64 & 
476.48 & 5.66 & \bf{466.77} & 
0.96\\CMT1Y & 478.39 & 6.99 & 
485.67 & 6.93 & \bf{466.77} & 
2.49\\CMT2X & 707.59 & 11.61 & 
715.32 & 13.54 & \bf{684.21} & 
3.42\\CMT2Y & 707.83 & 14.63 & 
711.33 & 13.83 & \bf{684.21} & 
3.45\\CMT3X & 730.46 & 39.50 & 
734.32 & 34.94 & \bf{721.40} & 
1.26\\CMT3Y & 731.91 & 33.50 & 
733.70 & 34.05 & \bf{721.40} & 
1.46\\CMT4X & 871.26 & 84.44 & 
898.88 & 92.08 & \bf{852.83} & 
2.16\\CMT4Y & 887.72 & 85.79 & 
901.17 & 92.97 & \bf{852.46} & 
4.14\\CMT5X & 1093.28 & 201.98 & 
1106.19 & 202.75 & \bf{1030.55} & 
6.09\\CMT5Y & 1107.75 & 193.46 & 
1115.09 & 194.77 & \bf{1031.17} & 
7.43\\CMT11X & 885.66 & 64.61 & 
889.90 & 78.42 & \bf{839.39} & 
5.51\\CMT11Y & 880.85 & 65.32 & 
892.69 & 71.95 & \bf{841.88} & 
4.63\\CMT12X & 677.06 & 27.70 & 
680.90 & 29.34 & \bf{662.22} & 
2.24\\CMT12Y & 674.15 & 30.19 & 
680.73 & 31.36 & \bf{662.22} & 
1.80\\\bf{PROM.} & 
\bf{778.94} & \bf{61.74} & \bf{787.31} & \bf{64.47} & \bf{751.25} & \bf{3.36}\\[1ex]\hline
\end{tabular}
\label{table:nonlin}
\end{table} \clearpage
\begin{table}[ht]
\caption{Resultados de la ejecución de la metaheurística ILS, utilizando instancias de SalhiNagy con la configuración -n 45.0 -LS 50.0}
\centering
\small
\begin{tabular}{c c c c c c c}
\hline\hline
Instancia & Costo mínimo & Tiempo(seg.) & Costo promedio & Tiempo promedio(seg.) & Costo ILS & \%Gap \\ [0.5ex]
\hline
CMT1X & 478.41 & 6.76 & 
482.16 & 7.97 & \bf{466.77} & 
2.49\\CMT1Y & 470.67 & 7.86 & 
478.98 & 6.75 & \bf{466.77} & 
0.84\\CMT2X & 708.97 & 15.84 & 
714.20 & 16.29 & \bf{684.21} & 
3.62\\CMT2Y & 705.28 & 15.65 & 
709.29 & 16.45 & \bf{684.21} & 
3.08\\CMT3X & 727.57 & 36.17 & 
735.28 & 38.44 & \bf{721.40} & 
0.86\\CMT3Y & 734.44 & 37.77 & 
735.90 & 38.87 & \bf{721.40} & 
1.81\\CMT4X & 897.37 & 95.09 & 
905.72 & 95.37 & \bf{852.83} & 
5.22\\CMT4Y & 892.42 & 97.52 & 
901.66 & 99.05 & \bf{852.46} & 
4.69\\CMT5X & 1097.00 & 224.34 & 
1104.52 & 237.39 & \bf{1030.55} & 
6.45\\CMT5Y & 1090.74 & 218.41 & 
1105.05 & 241.63 & \bf{1031.17} & 
5.78\\CMT11X & 884.30 & 96.73 & 
896.86 & 82.06 & \bf{839.39} & 
5.35\\CMT11Y & 871.64 & 74.98 & 
880.69 & 81.72 & \bf{841.88} & 
3.53\\CMT12X & 674.04 & 33.91 & 
678.68 & 34.95 & \bf{662.22} & 
1.78\\CMT12Y & 677.38 & 38.88 & 
680.94 & 35.94 & \bf{662.22} & 
2.29\\\bf{PROM.} & 
\bf{779.30} & \bf{71.42} & \bf{786.42} & \bf{73.78} & \bf{751.25} & \bf{3.41}\\[1ex]\hline
\end{tabular}
\label{table:nonlin}
\end{table} \clearpage
\begin{table}[ht]
\caption{Resultados de la ejecución de la metaheurística ILS, utilizando instancias de SalhiNagy con la configuración -n 45.0 -LS 60.0}
\centering
\small
\begin{tabular}{c c c c c c c}
\hline\hline
Instancia & Costo mínimo & Tiempo(seg.) & Costo promedio & Tiempo promedio(seg.) & Costo ILS & \%Gap \\ [0.5ex]
\hline
CMT1X & 475.37 & 8.24 & 
481.20 & 7.81 & \bf{466.77} & 
1.84\\CMT1Y & 473.62 & 9.70 & 
482.98 & 9.09 & \bf{466.77} & 
1.47\\CMT2X & 703.62 & 20.52 & 
707.26 & 20.84 & \bf{684.21} & 
2.84\\CMT2Y & 698.05 & 16.05 & 
709.31 & 18.54 & \bf{684.21} & 
2.02\\CMT3X & 729.48 & 43.02 & 
732.36 & 43.17 & \bf{721.40} & 
1.12\\CMT3Y & 729.30 & 51.33 & 
735.71 & 50.73 & \bf{721.40} & 
1.10\\CMT4X & 889.21 & 103.43 & 
898.44 & 124.19 & \bf{852.83} & 
4.27\\CMT4Y & 879.41 & 110.01 & 
889.53 & 107.47 & \bf{852.46} & 
3.16\\CMT5X & 1101.64 & 282.21 & 
1108.62 & 255.22 & \bf{1030.55} & 
6.90\\CMT5Y & 1086.52 & 258.09 & 
1099.92 & 271.40 & \bf{1031.17} & 
5.37\\CMT11X & 882.77 & 87.68 & 
893.27 & 85.80 & \bf{839.39} & 
5.17\\CMT11Y & 875.29 & 102.34 & 
884.62 & 84.93 & \bf{841.88} & 
3.97\\CMT12X & 680.73 & 36.53 & 
682.34 & 39.59 & \bf{662.22} & 
2.80\\CMT12Y & 680.21 & 64.25 & 
681.03 & 49.13 & \bf{662.22} & 
2.72\\\bf{PROM.} & 
\bf{777.52} & \bf{85.24} & \bf{784.76} & \bf{83.42} & \bf{751.25} & \bf{3.19}\\[1ex]\hline
\end{tabular}
\label{table:nonlin}
\end{table} \clearpage
\begin{table}[ht]
\caption{Resultados de la ejecución de la metaheurística ILS, utilizando instancias de SalhiNagy con la configuración -n 45.0 -LS 70.0}
\centering
\small
\begin{tabular}{c c c c c c c}
\hline\hline
Instancia & Costo mínimo & Tiempo(seg.) & Costo promedio & Tiempo promedio(seg.) & Costo ILS & \%Gap \\ [0.5ex]
\hline
CMT1X & 476.41 & 11.52 & 
479.61 & 10.35 & \bf{466.77} & 
2.07\\CMT1Y & 479.25 & 10.62 & 
482.76 & 9.47 & \bf{466.77} & 
2.67\\CMT2X & 697.29 & 22.22 & 
704.10 & 21.70 & \bf{684.21} & 
1.91\\CMT2Y & 699.75 & 21.46 & 
706.00 & 20.80 & \bf{684.21} & 
2.27\\CMT3X & 727.38 & 45.54 & 
730.67 & 47.32 & \bf{721.40} & 
0.83\\CMT3Y & 733.41 & 47.49 & 
736.27 & 48.55 & \bf{721.40} & 
1.66\\CMT4X & 886.76 & 121.26 & 
891.80 & 133.63 & \bf{852.83} & 
3.98\\CMT4Y & 896.68 & 124.91 & 
900.97 & 140.72 & \bf{852.46} & 
5.19\\CMT5X & 1102.12 & 337.96 & 
1109.63 & 291.88 & \bf{1030.55} & 
6.94\\CMT5Y & 1091.12 & 340.11 & 
1102.90 & 312.98 & \bf{1031.17} & 
5.81\\CMT11X & 882.96 & 109.13 & 
889.97 & 102.02 & \bf{839.39} & 
5.19\\CMT11Y & 868.75 & 121.28 & 
885.03 & 97.53 & \bf{841.88} & 
3.19\\CMT12X & 681.60 & 52.57 & 
684.32 & 49.90 & \bf{662.22} & 
2.93\\CMT12Y & 675.06 & 51.52 & 
680.42 & 44.93 & \bf{662.22} & 
1.94\\\bf{PROM.} & 
\bf{778.47} & \bf{101.26} & \bf{784.60} & \bf{95.13} & \bf{751.25} & \bf{3.33}\\[1ex]\hline
\end{tabular}
\label{table:nonlin}
\end{table} \clearpage
\begin{table}[ht]
\caption{Resultados de la ejecución de la metaheurística ILS, utilizando instancias de SalhiNagy con la configuración -n 45.0 -LS 80.0}
\centering
\small
\begin{tabular}{c c c c c c c}
\hline\hline
Instancia & Costo mínimo & Tiempo(seg.) & Costo promedio & Tiempo promedio(seg.) & Costo ILS & \%Gap \\ [0.5ex]
\hline
CMT1X & 470.67 & 9.30 & 
477.31 & 11.55 & \bf{466.77} & 
0.84\\CMT1Y & 480.08 & 12.01 & 
483.02 & 11.14 & \bf{466.77} & 
2.85\\CMT2X & 698.55 & 22.04 & 
707.85 & 23.34 & \bf{684.21} & 
2.10\\CMT2Y & 699.32 & 22.54 & 
707.05 & 22.33 & \bf{684.21} & 
2.21\\CMT3X & 737.18 & 51.76 & 
740.95 & 56.45 & \bf{721.40} & 
2.19\\CMT3Y & 734.76 & 61.43 & 
738.94 & 54.94 & \bf{721.40} & 
1.85\\CMT4X & 894.77 & 131.80 & 
897.18 & 142.61 & \bf{852.83} & 
4.92\\CMT4Y & 889.98 & 159.95 & 
894.48 & 142.74 & \bf{852.46} & 
4.40\\CMT5X & 1079.84 & 301.11 & 
1100.29 & 307.53 & \bf{1030.55} & 
4.78\\CMT5Y & 1090.28 & 354.27 & 
1100.26 & 321.39 & \bf{1031.17} & 
5.73\\CMT11X & 861.14 & 123.81 & 
881.90 & 111.82 & \bf{839.39} & 
2.59\\CMT11Y & 897.87 & 102.12 & 
905.93 & 107.41 & \bf{841.88} & 
6.65\\CMT12X & 682.01 & 52.28 & 
685.15 & 52.12 & \bf{662.22} & 
2.99\\CMT12Y & 673.59 & 57.44 & 
680.91 & 51.83 & \bf{662.22} & 
1.72\\\bf{PROM.} & 
\bf{777.86} & \bf{104.42} & \bf{785.80} & \bf{101.23} & \bf{751.25} & \bf{3.27}\\[1ex]\hline
\end{tabular}
\label{table:nonlin}
\end{table} \clearpage
\begin{table}[ht]
\caption{Resultados de la ejecución de la metaheurística ILS, utilizando instancias de SalhiNagy con la configuración -n 5.0 -LS 10.0}
\centering
\small
\begin{tabular}{c c c c c c c}
\hline\hline
Instancia & Costo mínimo & Tiempo(seg.) & Costo promedio & Tiempo promedio(seg.) & Costo ILS & \%Gap \\ [0.5ex]
\hline
CMT1X & 484.42 & 0.32 & 
490.96 & 0.30 & \bf{466.77} & 
3.78\\CMT1Y & 510.07 & 0.18 & 
516.21 & 0.22 & \bf{466.77} & 
9.28\\CMT2X & 718.33 & 1.06 & 
730.50 & 0.95 & \bf{684.21} & 
4.99\\CMT2Y & 718.80 & 0.91 & 
724.77 & 0.80 & \bf{684.21} & 
5.06\\CMT3X & 740.20 & 2.08 & 
747.31 & 2.30 & \bf{721.40} & 
2.61\\CMT3Y & 746.93 & 1.90 & 
753.51 & 2.12 & \bf{721.40} & 
3.54\\CMT4X & 925.51 & 5.44 & 
930.82 & 7.30 & \bf{852.83} & 
8.52\\CMT4Y & 914.77 & 6.05 & 
927.38 & 7.01 & \bf{852.46} & 
7.31\\CMT5X & 1105.30 & 23.41 & 
1140.11 & 16.27 & \bf{1030.55} & 
7.25\\CMT5Y & 1112.66 & 15.98 & 
1129.79 & 19.07 & \bf{1031.17} & 
7.90\\CMT11X & 894.90 & 4.28 & 
924.56 & 4.77 & \bf{839.39} & 
6.61\\CMT11Y & 907.63 & 5.65 & 
944.20 & 5.66 & \bf{841.88} & 
7.81\\CMT12X & 689.21 & 1.89 & 
708.75 & 1.97 & \bf{662.22} & 
4.08\\CMT12Y & 690.47 & 1.72 & 
697.54 & 1.91 & \bf{662.22} & 
4.27\\\bf{PROM.} & 
\bf{797.09} & \bf{5.06} & \bf{811.88} & \bf{5.05} & \bf{751.25} & \bf{5.93}\\[1ex]\hline
\end{tabular}
\label{table:nonlin}
\end{table} \clearpage
\begin{table}[ht]
\caption{Resultados de la ejecución de la metaheurística ILS, utilizando instancias de SalhiNagy con la configuración -n 5.0 -LS 20.0}
\centering
\small
\begin{tabular}{c c c c c c c}
\hline\hline
Instancia & Costo mínimo & Tiempo(seg.) & Costo promedio & Tiempo promedio(seg.) & Costo ILS & \%Gap \\ [0.5ex]
\hline
CMT1X & 485.80 & 0.43 & 
491.84 & 0.38 & \bf{466.77} & 
4.08\\CMT1Y & 503.24 & 0.58 & 
510.12 & 0.47 & \bf{466.77} & 
7.81\\CMT2X & 723.80 & 1.26 & 
743.86 & 1.13 & \bf{684.21} & 
5.79\\CMT2Y & 723.43 & 0.97 & 
742.33 & 0.88 & \bf{684.21} & 
5.73\\CMT3X & 747.26 & 2.68 & 
758.12 & 2.44 & \bf{721.40} & 
3.58\\CMT3Y & 737.36 & 3.39 & 
750.76 & 2.87 & \bf{721.40} & 
2.21\\CMT4X & 916.49 & 8.04 & 
931.40 & 8.40 & \bf{852.83} & 
7.46\\CMT4Y & 896.33 & 10.86 & 
913.93 & 8.70 & \bf{852.46} & 
5.15\\CMT5X & 1116.06 & 26.42 & 
1133.48 & 19.41 & \bf{1030.55} & 
8.30\\CMT5Y & 1128.26 & 23.79 & 
1141.40 & 18.84 & \bf{1031.17} & 
9.42\\CMT11X & 873.94 & 6.71 & 
921.84 & 6.45 & \bf{839.39} & 
4.12\\CMT11Y & 896.05 & 5.84 & 
931.10 & 5.58 & \bf{841.88} & 
6.43\\CMT12X & 695.13 & 2.80 & 
710.98 & 2.36 & \bf{662.22} & 
4.97\\CMT12Y & 697.79 & 2.64 & 
703.93 & 2.28 & \bf{662.22} & 
5.37\\\bf{PROM.} & 
\bf{795.78} & \bf{6.89} & \bf{813.22} & \bf{5.73} & \bf{751.25} & \bf{5.74}\\[1ex]\hline
\end{tabular}
\label{table:nonlin}
\end{table} \clearpage
\begin{table}[ht]
\caption{Resultados de la ejecución de la metaheurística ILS, utilizando instancias de SalhiNagy con la configuración -n 5.0 -LS 30.0}
\centering
\small
\begin{tabular}{c c c c c c c}
\hline\hline
Instancia & Costo mínimo & Tiempo(seg.) & Costo promedio & Tiempo promedio(seg.) & Costo ILS & \%Gap \\ [0.5ex]
\hline
CMT1X & 497.31 & 0.38 & 
502.64 & 0.48 & \bf{466.77} & 
6.54\\CMT1Y & 481.74 & 0.42 & 
482.20 & 0.43 & \bf{466.77} & 
3.21\\CMT2X & 720.25 & 1.13 & 
728.34 & 1.22 & \bf{684.21} & 
5.27\\CMT2Y & 714.85 & 1.78 & 
719.85 & 1.47 & \bf{684.21} & 
4.48\\CMT3X & 755.44 & 2.71 & 
759.68 & 3.17 & \bf{721.40} & 
4.72\\CMT3Y & 738.34 & 3.85 & 
749.65 & 3.68 & \bf{721.40} & 
2.35\\CMT4X & 901.20 & 8.76 & 
929.04 & 9.96 & \bf{852.83} & 
5.67\\CMT4Y & 917.78 & 9.12 & 
938.28 & 8.74 & \bf{852.46} & 
7.66\\CMT5X & 1121.84 & 17.65 & 
1137.85 & 20.78 & \bf{1030.55} & 
8.86\\CMT5Y & 1111.86 & 30.30 & 
1129.17 & 23.96 & \bf{1031.17} & 
7.83\\CMT11X & 902.95 & 8.60 & 
932.91 & 7.99 & \bf{839.39} & 
7.57\\CMT11Y & 927.82 & 7.57 & 
949.87 & 8.16 & \bf{841.88} & 
10.21\\CMT12X & 691.07 & 2.85 & 
698.11 & 2.72 & \bf{662.22} & 
4.36\\CMT12Y & 678.58 & 3.19 & 
691.46 & 3.31 & \bf{662.22} & 
2.47\\\bf{PROM.} & 
\bf{797.22} & \bf{7.02} & \bf{810.65} & \bf{6.86} & \bf{751.25} & \bf{5.80}\\[1ex]\hline
\end{tabular}
\label{table:nonlin}
\end{table} \clearpage
\begin{table}[ht]
\caption{Resultados de la ejecución de la metaheurística ILS, utilizando instancias de SalhiNagy con la configuración -n 5.0 -LS 40.0}
\centering
\small
\begin{tabular}{c c c c c c c}
\hline\hline
Instancia & Costo mínimo & Tiempo(seg.) & Costo promedio & Tiempo promedio(seg.) & Costo ILS & \%Gap \\ [0.5ex]
\hline
CMT1X & 479.75 & 0.42 & 
492.08 & 0.79 & \bf{466.77} & 
2.78\\CMT1Y & 478.23 & 0.78 & 
485.48 & 0.85 & \bf{466.77} & 
2.46\\CMT2X & 729.34 & 1.54 & 
747.66 & 1.50 & \bf{684.21} & 
6.60\\CMT2Y & 710.00 & 1.24 & 
719.57 & 1.47 & \bf{684.21} & 
3.77\\CMT3X & 743.31 & 4.07 & 
754.11 & 3.97 & \bf{721.40} & 
3.04\\CMT3Y & 746.77 & 3.76 & 
758.77 & 3.65 & \bf{721.40} & 
3.52\\CMT4X & 906.11 & 11.64 & 
922.71 & 12.06 & \bf{852.83} & 
6.25\\CMT4Y & 916.65 & 11.62 & 
928.42 & 10.91 & \bf{852.46} & 
7.53\\CMT5X & 1117.42 & 18.10 & 
1138.58 & 21.73 & \bf{1030.55} & 
8.43\\CMT5Y & 1106.24 & 20.25 & 
1113.81 & 22.92 & \bf{1031.17} & 
7.28\\CMT11X & 921.82 & 11.32 & 
934.40 & 9.31 & \bf{839.39} & 
9.82\\CMT11Y & 896.89 & 7.55 & 
951.83 & 7.92 & \bf{841.88} & 
6.53\\CMT12X & 678.73 & 3.55 & 
705.51 & 3.23 & \bf{662.22} & 
2.49\\CMT12Y & 696.28 & 3.99 & 
705.68 & 3.50 & \bf{662.22} & 
5.14\\\bf{PROM.} & 
\bf{794.82} & \bf{7.13} & \bf{811.33} & \bf{7.42} & \bf{751.25} & \bf{5.40}\\[1ex]\hline
\end{tabular}
\label{table:nonlin}
\end{table} \clearpage
\begin{table}[ht]
\caption{Resultados de la ejecución de la metaheurística ILS, utilizando instancias de SalhiNagy con la configuración -n 5.0 -LS 50.0}
\centering
\small
\begin{tabular}{c c c c c c c}
\hline\hline
Instancia & Costo mínimo & Tiempo(seg.) & Costo promedio & Tiempo promedio(seg.) & Costo ILS & \%Gap \\ [0.5ex]
\hline
CMT1X & 492.44 & 1.05 & 
503.56 & 0.80 & \bf{466.77} & 
5.50\\CMT1Y & 484.06 & 0.93 & 
488.46 & 0.83 & \bf{466.77} & 
3.70\\CMT2X & 720.08 & 2.09 & 
726.70 & 1.96 & \bf{684.21} & 
5.24\\CMT2Y & 713.53 & 1.84 & 
725.79 & 1.85 & \bf{684.21} & 
4.29\\CMT3X & 734.07 & 4.62 & 
739.99 & 4.56 & \bf{721.40} & 
1.76\\CMT3Y & 739.72 & 5.01 & 
752.96 & 4.51 & \bf{721.40} & 
2.54\\CMT4X & 902.15 & 10.86 & 
933.16 & 10.86 & \bf{852.83} & 
5.78\\CMT4Y & 920.34 & 13.61 & 
934.35 & 12.62 & \bf{852.46} & 
7.96\\CMT5X & 1080.66 & 24.03 & 
1127.05 & 23.43 & \bf{1030.55} & 
4.86\\CMT5Y & 1112.18 & 28.95 & 
1135.44 & 25.26 & \bf{1031.17} & 
7.86\\CMT11X & 858.65 & 7.82 & 
914.60 & 10.10 & \bf{839.39} & 
2.29\\CMT11Y & 899.20 & 7.83 & 
924.64 & 10.22 & \bf{841.88} & 
6.81\\CMT12X & 677.87 & 3.74 & 
690.03 & 3.82 & \bf{662.22} & 
2.36\\CMT12Y & 691.95 & 5.57 & 
695.88 & 4.92 & \bf{662.22} & 
4.49\\\bf{PROM.} & 
\bf{787.64} & \bf{8.42} & \bf{806.62} & \bf{8.27} & \bf{751.25} & \bf{4.67}\\[1ex]\hline
\end{tabular}
\label{table:nonlin}
\end{table} \clearpage
\begin{table}[ht]
\caption{Resultados de la ejecución de la metaheurística ILS, utilizando instancias de SalhiNagy con la configuración -n 5.0 -LS 60.0}
\centering
\small
\begin{tabular}{c c c c c c c}
\hline\hline
Instancia & Costo mínimo & Tiempo(seg.) & Costo promedio & Tiempo promedio(seg.) & Costo ILS & \%Gap \\ [0.5ex]
\hline
CMT1X & 486.69 & 0.71 & 
491.60 & 0.57 & \bf{466.77} & 
4.27\\CMT1Y & 500.22 & 0.79 & 
507.10 & 0.62 & \bf{466.77} & 
7.17\\CMT2X & 718.82 & 2.29 & 
723.13 & 2.14 & \bf{684.21} & 
5.06\\CMT2Y & 709.85 & 2.29 & 
725.71 & 2.31 & \bf{684.21} & 
3.75\\CMT3X & 743.59 & 4.75 & 
746.62 & 5.26 & \bf{721.40} & 
3.08\\CMT3Y & 737.60 & 3.99 & 
744.11 & 4.48 & \bf{721.40} & 
2.25\\CMT4X & 920.39 & 19.39 & 
930.29 & 16.55 & \bf{852.83} & 
7.92\\CMT4Y & 918.61 & 14.48 & 
929.34 & 13.33 & \bf{852.46} & 
7.76\\CMT5X & 1123.29 & 31.12 & 
1136.18 & 31.14 & \bf{1030.55} & 
9.00\\CMT5Y & 1118.87 & 34.52 & 
1137.55 & 31.46 & \bf{1031.17} & 
8.50\\CMT11X & 893.56 & 12.28 & 
909.56 & 9.79 & \bf{839.39} & 
6.45\\CMT11Y & 887.03 & 7.84 & 
924.81 & 9.56 & \bf{841.88} & 
5.36\\CMT12X & 689.94 & 5.40 & 
703.84 & 4.29 & \bf{662.22} & 
4.19\\CMT12Y & 681.38 & 4.99 & 
696.75 & 4.34 & \bf{662.22} & 
2.89\\\bf{PROM.} & 
\bf{794.99} & \bf{10.35} & \bf{807.61} & \bf{9.70} & \bf{751.25} & \bf{5.55}\\[1ex]\hline
\end{tabular}
\label{table:nonlin}
\end{table} \clearpage
\begin{table}[ht]
\caption{Resultados de la ejecución de la metaheurística ILS, utilizando instancias de SalhiNagy con la configuración -n 5.0 -LS 70.0}
\centering
\small
\begin{tabular}{c c c c c c c}
\hline\hline
Instancia & Costo mínimo & Tiempo(seg.) & Costo promedio & Tiempo promedio(seg.) & Costo ILS & \%Gap \\ [0.5ex]
\hline
CMT1X & 475.71 & 1.28 & 
493.12 & 1.12 & \bf{466.77} & 
1.92\\CMT1Y & 484.83 & 0.84 & 
499.49 & 1.07 & \bf{466.77} & 
3.87\\CMT2X & 715.03 & 2.06 & 
723.79 & 2.53 & \bf{684.21} & 
4.50\\CMT2Y & 706.80 & 3.22 & 
719.04 & 2.85 & \bf{684.21} & 
3.30\\CMT3X & 749.50 & 6.07 & 
753.81 & 6.01 & \bf{721.40} & 
3.90\\CMT3Y & 738.80 & 7.47 & 
747.14 & 5.88 & \bf{721.40} & 
2.41\\CMT4X & 912.11 & 14.57 & 
922.41 & 13.18 & \bf{852.83} & 
6.95\\CMT4Y & 918.45 & 14.59 & 
926.63 & 14.45 & \bf{852.46} & 
7.74\\CMT5X & 1103.35 & 27.40 & 
1114.52 & 28.09 & \bf{1030.55} & 
7.06\\CMT5Y & 1112.39 & 32.62 & 
1121.66 & 34.88 & \bf{1031.17} & 
7.88\\CMT11X & 888.89 & 9.67 & 
949.62 & 9.13 & \bf{839.39} & 
5.90\\CMT11Y & 896.91 & 12.76 & 
933.68 & 12.34 & \bf{841.88} & 
6.54\\CMT12X & 685.84 & 4.89 & 
699.79 & 4.46 & \bf{662.22} & 
3.57\\CMT12Y & 683.03 & 4.83 & 
699.02 & 5.20 & \bf{662.22} & 
3.14\\\bf{PROM.} & 
\bf{790.83} & \bf{10.16} & \bf{807.41} & \bf{10.09} & \bf{751.25} & \bf{4.91}\\[1ex]\hline
\end{tabular}
\label{table:nonlin}
\end{table} \clearpage
\begin{table}[ht]
\caption{Resultados de la ejecución de la metaheurística ILS, utilizando instancias de SalhiNagy con la configuración -n 5.0 -LS 80.0}
\centering
\small
\begin{tabular}{c c c c c c c}
\hline\hline
Instancia & Costo mínimo & Tiempo(seg.) & Costo promedio & Tiempo promedio(seg.) & Costo ILS & \%Gap \\ [0.5ex]
\hline
CMT1X & 481.61 & 1.36 & 
507.47 & 1.30 & \bf{466.77} & 
3.18\\CMT1Y & 490.30 & 2.12 & 
505.35 & 1.36 & \bf{466.77} & 
5.04\\CMT2X & 708.75 & 3.04 & 
720.83 & 2.89 & \bf{684.21} & 
3.59\\CMT2Y & 711.40 & 2.60 & 
714.64 & 2.49 & \bf{684.21} & 
3.97\\CMT3X & 734.99 & 6.30 & 
746.03 & 6.55 & \bf{721.40} & 
1.88\\CMT3Y & 732.79 & 8.46 & 
741.17 & 6.91 & \bf{721.40} & 
1.58\\CMT4X & 924.48 & 12.86 & 
942.26 & 15.70 & \bf{852.83} & 
8.40\\CMT4Y & 905.90 & 19.23 & 
919.52 & 16.14 & \bf{852.46} & 
6.27\\CMT5X & 1104.65 & 29.04 & 
1125.07 & 33.68 & \bf{1030.55} & 
7.19\\CMT5Y & 1080.50 & 36.47 & 
1112.76 & 37.15 & \bf{1031.17} & 
4.78\\CMT11X & 881.30 & 11.69 & 
919.11 & 11.04 & \bf{839.39} & 
4.99\\CMT11Y & 901.16 & 11.90 & 
947.28 & 12.01 & \bf{841.88} & 
7.04\\CMT12X & 674.46 & 6.45 & 
693.88 & 5.50 & \bf{662.22} & 
1.85\\CMT12Y & 684.83 & 7.58 & 
692.94 & 6.01 & \bf{662.22} & 
3.41\\\bf{PROM.} & 
\bf{786.94} & \bf{11.36} & \bf{806.31} & \bf{11.34} & \bf{751.25} & \bf{4.51}\\[1ex]\hline
\end{tabular}
\label{table:nonlin}
\end{table} \clearpage
\begin{table}[ht]
\caption{Resultados de la ejecución de la metaheurística ILS, utilizando instancias de SalhiNagy con la configuración -n 55.0 -LS 10.0}
\centering
\small
\begin{tabular}{c c c c c c c}
\hline\hline
Instancia & Costo mínimo & Tiempo(seg.) & Costo promedio & Tiempo promedio(seg.) & Costo ILS & \%Gap \\ [0.5ex]
\hline
CMT1X & 477.47 & 2.30 & 
486.59 & 2.96 & \bf{466.77} & 
2.29\\CMT1Y & 486.63 & 2.46 & 
489.68 & 3.05 & \bf{466.77} & 
4.25\\CMT2X & 705.69 & 8.20 & 
710.01 & 8.43 & \bf{684.21} & 
3.14\\CMT2Y & 709.71 & 9.21 & 
716.71 & 8.08 & \bf{684.21} & 
3.73\\CMT3X & 727.76 & 21.76 & 
733.67 & 21.55 & \bf{721.40} & 
0.88\\CMT3Y & 726.14 & 21.24 & 
734.23 & 21.77 & \bf{721.40} & 
0.66\\CMT4X & 883.82 & 100.83 & 
897.66 & 83.64 & \bf{852.83} & 
3.63\\CMT4Y & 885.78 & 67.68 & 
896.36 & 83.11 & \bf{852.46} & 
3.91\\CMT5X & 1089.68 & 161.79 & 
1107.14 & 169.18 & \bf{1030.55} & 
5.74\\CMT5Y & 1083.29 & 175.29 & 
1097.57 & 194.01 & \bf{1031.17} & 
5.05\\CMT11X & 881.85 & 88.17 & 
891.13 & 68.36 & \bf{839.39} & 
5.06\\CMT11Y & 855.88 & 53.37 & 
889.37 & 61.41 & \bf{841.88} & 
1.66\\CMT12X & 679.60 & 26.58 & 
688.33 & 20.86 & \bf{662.22} & 
2.62\\CMT12Y & 676.06 & 28.26 & 
682.80 & 23.64 & \bf{662.22} & 
2.09\\\bf{PROM.} & 
\bf{776.38} & \bf{54.80} & \bf{787.23} & \bf{55.00} & \bf{751.25} & \bf{3.19}\\[1ex]\hline
\end{tabular}
\label{table:nonlin}
\end{table} \clearpage
\begin{table}[ht]
\caption{Resultados de la ejecución de la metaheurística ILS, utilizando instancias de SalhiNagy con la configuración -n 55.0 -LS 20.0}
\centering
\small
\begin{tabular}{c c c c c c c}
\hline\hline
Instancia & Costo mínimo & Tiempo(seg.) & Costo promedio & Tiempo promedio(seg.) & Costo ILS & \%Gap \\ [0.5ex]
\hline
CMT1X & 476.71 & 2.34 & 
486.17 & 4.37 & \bf{466.77} & 
2.13\\CMT1Y & 479.20 & 3.24 & 
483.69 & 4.39 & \bf{466.77} & 
2.66\\CMT2X & 695.35 & 14.24 & 
707.83 & 12.64 & \bf{684.21} & 
1.63\\CMT2Y & 709.60 & 12.39 & 
713.93 & 11.27 & \bf{684.21} & 
3.71\\CMT3X & 729.00 & 35.65 & 
737.48 & 29.73 & \bf{721.40} & 
1.05\\CMT3Y & 733.56 & 36.58 & 
736.72 & 31.85 & \bf{721.40} & 
1.69\\CMT4X & 896.39 & 109.45 & 
899.25 & 95.81 & \bf{852.83} & 
5.11\\CMT4Y & 897.05 & 117.95 & 
905.39 & 100.22 & \bf{852.46} & 
5.23\\CMT5X & 1098.51 & 277.19 & 
1111.41 & 238.47 & \bf{1030.55} & 
6.59\\CMT5Y & 1103.06 & 285.77 & 
1108.43 & 215.52 & \bf{1031.17} & 
6.97\\CMT11X & 880.50 & 62.37 & 
897.81 & 61.97 & \bf{839.39} & 
4.90\\CMT11Y & 882.97 & 64.30 & 
890.93 & 62.15 & \bf{841.88} & 
4.88\\CMT12X & 675.42 & 27.60 & 
685.18 & 26.97 & \bf{662.22} & 
1.99\\CMT12Y & 680.50 & 25.25 & 
684.41 & 26.45 & \bf{662.22} & 
2.76\\\bf{PROM.} & 
\bf{781.27} & \bf{76.74} & \bf{789.19} & \bf{65.84} & \bf{751.25} & \bf{3.66}\\[1ex]\hline
\end{tabular}
\label{table:nonlin}
\end{table} \clearpage
\begin{table}[ht]
\caption{Resultados de la ejecución de la metaheurística ILS, utilizando instancias de SalhiNagy con la configuración -n 55.0 -LS 30.0}
\centering
\small
\begin{tabular}{c c c c c c c}
\hline\hline
Instancia & Costo mínimo & Tiempo(seg.) & Costo promedio & Tiempo promedio(seg.) & Costo ILS & \%Gap \\ [0.5ex]
\hline
CMT1X & 479.89 & 6.14 & 
487.87 & 7.58 & \bf{466.77} & 
2.81\\CMT1Y & 471.25 & 7.08 & 
476.19 & 6.40 & \bf{466.77} & 
0.96\\CMT2X & 706.04 & 16.57 & 
714.08 & 15.54 & \bf{684.21} & 
3.19\\CMT2Y & 699.95 & 14.60 & 
707.48 & 14.03 & \bf{684.21} & 
2.30\\CMT3X & 734.44 & 32.81 & 
739.19 & 32.78 & \bf{721.40} & 
1.81\\CMT3Y & 730.16 & 33.31 & 
734.64 & 34.28 & \bf{721.40} & 
1.21\\CMT4X & 904.04 & 95.32 & 
906.48 & 92.12 & \bf{852.83} & 
6.00\\CMT4Y & 884.89 & 97.95 & 
895.42 & 103.42 & \bf{852.46} & 
3.80\\CMT5X & 1092.00 & 213.90 & 
1098.79 & 218.65 & \bf{1030.55} & 
5.96\\CMT5Y & 1082.54 & 207.84 & 
1099.09 & 219.28 & \bf{1031.17} & 
4.98\\CMT11X & 860.58 & 77.82 & 
882.38 & 81.09 & \bf{839.39} & 
2.52\\CMT11Y & 882.11 & 78.78 & 
891.53 & 82.31 & \bf{841.88} & 
4.78\\CMT12X & 677.55 & 31.75 & 
684.18 & 32.12 & \bf{662.22} & 
2.31\\CMT12Y & 675.33 & 30.36 & 
680.02 & 32.14 & \bf{662.22} & 
1.98\\\bf{PROM.} & 
\bf{777.20} & \bf{67.44} & \bf{785.52} & \bf{69.41} & \bf{751.25} & \bf{3.19}\\[1ex]\hline
\end{tabular}
\label{table:nonlin}
\end{table} \clearpage
\begin{table}[ht]
\caption{Resultados de la ejecución de la metaheurística ILS, utilizando instancias de SalhiNagy con la configuración -n 55.0 -LS 40.0}
\centering
\small
\begin{tabular}{c c c c c c c}
\hline\hline
Instancia & Costo mínimo & Tiempo(seg.) & Costo promedio & Tiempo promedio(seg.) & Costo ILS & \%Gap \\ [0.5ex]
\hline
CMT1X & 470.67 & 6.01 & 
479.78 & 6.93 & \bf{466.77} & 
0.84\\CMT1Y & 475.71 & 7.17 & 
481.28 & 6.81 & \bf{466.77} & 
1.92\\CMT2X & 693.78 & 19.71 & 
706.89 & 21.07 & \bf{684.21} & 
1.40\\CMT2Y & 687.95 & 17.48 & 
704.28 & 17.59 & \bf{684.21} & 
0.55\\CMT3X & 732.97 & 38.95 & 
737.03 & 41.02 & \bf{721.40} & 
1.60\\CMT3Y & 725.60 & 50.40 & 
731.08 & 42.28 & \bf{721.40} & 
0.58\\CMT4X & 894.80 & 109.61 & 
898.52 & 105.34 & \bf{852.83} & 
4.92\\CMT4Y & 896.83 & 106.96 & 
905.00 & 108.51 & \bf{852.46} & 
5.20\\CMT5X & 1094.12 & 231.54 & 
1108.03 & 235.13 & \bf{1030.55} & 
6.17\\CMT5Y & 1089.38 & 232.81 & 
1101.41 & 266.93 & \bf{1031.17} & 
5.65\\CMT11X & 870.61 & 82.22 & 
879.31 & 82.70 & \bf{839.39} & 
3.72\\CMT11Y & 895.30 & 79.86 & 
896.97 & 93.92 & \bf{841.88} & 
6.35\\CMT12X & 685.31 & 46.27 & 
689.60 & 41.01 & \bf{662.22} & 
3.49\\CMT12Y & 674.92 & 35.18 & 
681.37 & 34.77 & \bf{662.22} & 
1.92\\\bf{PROM.} & 
\bf{777.71} & \bf{76.01} & \bf{785.75} & \bf{78.86} & \bf{751.25} & \bf{3.16}\\[1ex]\hline
\end{tabular}
\label{table:nonlin}
\end{table} \clearpage
\begin{table}[ht]
\caption{Resultados de la ejecución de la metaheurística ILS, utilizando instancias de SalhiNagy con la configuración -n 55.0 -LS 50.0}
\centering
\small
\begin{tabular}{c c c c c c c}
\hline\hline
Instancia & Costo mínimo & Tiempo(seg.) & Costo promedio & Tiempo promedio(seg.) & Costo ILS & \%Gap \\ [0.5ex]
\hline
CMT1X & 470.67 & 5.59 & 
476.00 & 8.55 & \bf{466.77} & 
0.84\\CMT1Y & 472.85 & 6.00 & 
485.75 & 7.31 & \bf{466.77} & 
1.30\\CMT2X & 707.76 & 19.60 & 
711.20 & 20.16 & \bf{684.21} & 
3.44\\CMT2Y & 702.53 & 25.52 & 
708.70 & 20.82 & \bf{684.21} & 
2.68\\CMT3X & 730.13 & 56.76 & 
733.41 & 47.42 & \bf{721.40} & 
1.21\\CMT3Y & 729.62 & 56.16 & 
734.07 & 50.69 & \bf{721.40} & 
1.14\\CMT4X & 901.98 & 128.84 & 
907.36 & 120.86 & \bf{852.83} & 
5.76\\CMT4Y & 902.15 & 152.15 & 
906.81 & 128.03 & \bf{852.46} & 
5.83\\CMT5X & 1079.14 & 344.34 & 
1100.13 & 304.98 & \bf{1030.55} & 
4.71\\CMT5Y & 1092.86 & 361.21 & 
1099.81 & 298.55 & \bf{1031.17} & 
5.98\\CMT11X & 881.54 & 88.44 & 
886.79 & 93.00 & \bf{839.39} & 
5.02\\CMT11Y & 879.31 & 94.88 & 
887.50 & 97.19 & \bf{841.88} & 
4.45\\CMT12X & 676.87 & 43.21 & 
679.14 & 45.80 & \bf{662.22} & 
2.21\\CMT12Y & 677.39 & 53.95 & 
683.28 & 47.40 & \bf{662.22} & 
2.29\\\bf{PROM.} & 
\bf{778.91} & \bf{102.62} & \bf{785.71} & \bf{92.20} & \bf{751.25} & \bf{3.35}\\[1ex]\hline
\end{tabular}
\label{table:nonlin}
\end{table} \clearpage
\begin{table}[ht]
\caption{Resultados de la ejecución de la metaheurística ILS, utilizando instancias de SalhiNagy con la configuración -n 55.0 -LS 60.0}
\centering
\small
\begin{tabular}{c c c c c c c}
\hline\hline
Instancia & Costo mínimo & Tiempo(seg.) & Costo promedio & Tiempo promedio(seg.) & Costo ILS & \%Gap \\ [0.5ex]
\hline
CMT1X & 478.39 & 9.21 & 
485.06 & 9.69 & \bf{466.77} & 
2.49\\CMT1Y & 473.62 & 10.59 & 
476.21 & 10.13 & \bf{466.77} & 
1.47\\CMT2X & 703.21 & 23.15 & 
710.00 & 23.20 & \bf{684.21} & 
2.78\\CMT2Y & 698.65 & 25.38 & 
706.81 & 23.43 & \bf{684.21} & 
2.11\\CMT3X & 729.64 & 50.76 & 
737.43 & 54.39 & \bf{721.40} & 
1.14\\CMT3Y & 727.65 & 49.41 & 
734.35 & 57.92 & \bf{721.40} & 
0.87\\CMT4X & 883.95 & 133.22 & 
894.33 & 146.65 & \bf{852.83} & 
3.65\\CMT4Y & 897.89 & 125.88 & 
902.98 & 131.37 & \bf{852.46} & 
5.33\\CMT5X & 1092.25 & 306.73 & 
1103.41 & 297.02 & \bf{1030.55} & 
5.99\\CMT5Y & 1082.45 & 312.29 & 
1092.26 & 320.70 & \bf{1031.17} & 
4.97\\CMT11X & 867.53 & 129.11 & 
881.14 & 116.82 & \bf{839.39} & 
3.35\\CMT11Y & 871.04 & 93.81 & 
886.33 & 111.26 & \bf{841.88} & 
3.46\\CMT12X & 673.99 & 55.75 & 
675.62 & 49.33 & \bf{662.22} & 
1.78\\CMT12Y & 674.09 & 44.94 & 
679.17 & 45.63 & \bf{662.22} & 
1.79\\\bf{PROM.} & 
\bf{775.31} & \bf{97.87} & \bf{783.22} & \bf{99.82} & \bf{751.25} & \bf{2.94}\\[1ex]\hline
\end{tabular}
\label{table:nonlin}
\end{table} \clearpage
\begin{table}[ht]
\caption{Resultados de la ejecución de la metaheurística ILS, utilizando instancias de SalhiNagy con la configuración -n 55.0 -LS 70.0}
\centering
\small
\begin{tabular}{c c c c c c c}
\hline\hline
Instancia & Costo mínimo & Tiempo(seg.) & Costo promedio & Tiempo promedio(seg.) & Costo ILS & \%Gap \\ [0.5ex]
\hline
CMT1X & 477.20 & 11.25 & 
480.61 & 11.51 & \bf{466.77} & 
2.23\\CMT1Y & 477.73 & 10.94 & 
480.75 & 11.92 & \bf{466.77} & 
2.35\\CMT2X & 705.58 & 23.12 & 
709.04 & 25.59 & \bf{684.21} & 
3.12\\CMT2Y & 698.03 & 26.12 & 
701.49 & 28.02 & \bf{684.21} & 
2.02\\CMT3X & 726.68 & 53.21 & 
732.48 & 55.62 & \bf{721.40} & 
0.73\\CMT3Y & 729.23 & 65.42 & 
733.51 & 59.69 & \bf{721.40} & 
1.09\\CMT4X & 897.23 & 147.05 & 
903.36 & 151.91 & \bf{852.83} & 
5.21\\CMT4Y & 891.29 & 145.44 & 
898.71 & 159.12 & \bf{852.46} & 
4.56\\CMT5X & 1087.16 & 388.24 & 
1097.24 & 346.76 & \bf{1030.55} & 
5.49\\CMT5Y & 1093.74 & 340.83 & 
1103.84 & 335.81 & \bf{1031.17} & 
6.07\\CMT11X & 870.20 & 137.53 & 
879.29 & 125.19 & \bf{839.39} & 
3.67\\CMT11Y & 890.71 & 109.09 & 
895.59 & 115.56 & \bf{841.88} & 
5.80\\CMT12X & 676.36 & 48.51 & 
678.41 & 52.97 & \bf{662.22} & 
2.14\\CMT12Y & 674.38 & 50.70 & 
678.05 & 53.15 & \bf{662.22} & 
1.84\\\bf{PROM.} & 
\bf{778.25} & \bf{111.25} & \bf{783.74} & \bf{109.49} & \bf{751.25} & \bf{3.31}\\[1ex]\hline
\end{tabular}
\label{table:nonlin}
\end{table} \clearpage
\begin{table}[ht]
\caption{Resultados de la ejecución de la metaheurística ILS, utilizando instancias de SalhiNagy con la configuración -n 55.0 -LS 80.0}
\centering
\small
\begin{tabular}{c c c c c c c}
\hline\hline
Instancia & Costo mínimo & Tiempo(seg.) & Costo promedio & Tiempo promedio(seg.) & Costo ILS & \%Gap \\ [0.5ex]
\hline
CMT1X & 474.85 & 13.10 & 
478.73 & 12.40 & \bf{466.77} & 
1.73\\CMT1Y & 475.22 & 13.60 & 
477.55 & 13.68 & \bf{466.77} & 
1.81\\CMT2X & 704.58 & 26.99 & 
709.19 & 27.97 & \bf{684.21} & 
2.98\\CMT2Y & 692.37 & 30.62 & 
700.51 & 29.95 & \bf{684.21} & 
1.19\\CMT3X & 727.10 & 63.73 & 
734.75 & 65.50 & \bf{721.40} & 
0.79\\CMT3Y & 728.12 & 63.90 & 
731.20 & 66.71 & \bf{721.40} & 
0.93\\CMT4X & 884.17 & 161.42 & 
894.02 & 160.37 & \bf{852.83} & 
3.67\\CMT4Y & 881.71 & 199.47 & 
897.42 & 173.26 & \bf{852.46} & 
3.43\\CMT5X & 1094.05 & 335.46 & 
1097.41 & 378.99 & \bf{1030.55} & 
6.16\\CMT5Y & 1078.81 & 373.06 & 
1089.79 & 365.23 & \bf{1031.17} & 
4.62\\CMT11X & 878.41 & 153.82 & 
889.95 & 137.27 & \bf{839.39} & 
4.65\\CMT11Y & 880.39 & 121.95 & 
888.88 & 118.90 & \bf{841.88} & 
4.57\\CMT12X & 677.54 & 57.06 & 
680.55 & 57.05 & \bf{662.22} & 
2.31\\CMT12Y & 675.95 & 62.77 & 
679.94 & 64.66 & \bf{662.22} & 
2.07\\\bf{PROM.} & 
\bf{775.23} & \bf{119.78} & \bf{782.14} & \bf{119.42} & \bf{751.25} & \bf{2.92}\\[1ex]\hline
\end{tabular}
\label{table:nonlin}
\end{table} \clearpage
\begin{table}[ht]
\caption{Resultados de la ejecución de la metaheurística ILS, utilizando instancias de SalhiNagy con la configuración -n 65.0 -LS 10.0}
\centering
\small
\begin{tabular}{c c c c c c c}
\hline\hline
Instancia & Costo mínimo & Tiempo(seg.) & Costo promedio & Tiempo promedio(seg.) & Costo ILS & \%Gap \\ [0.5ex]
\hline
CMT1X & 479.05 & 3.54 & 
481.76 & 3.62 & \bf{466.77} & 
2.63\\CMT1Y & 481.39 & 3.01 & 
488.85 & 3.37 & \bf{466.77} & 
3.13\\CMT2X & 706.75 & 10.18 & 
712.11 & 10.01 & \bf{684.21} & 
3.29\\CMT2Y & 709.03 & 10.24 & 
712.02 & 9.66 & \bf{684.21} & 
3.63\\CMT3X & 730.96 & 36.30 & 
734.78 & 29.99 & \bf{721.40} & 
1.33\\CMT3Y & 728.95 & 25.27 & 
732.84 & 28.29 & \bf{721.40} & 
1.05\\CMT4X & 890.45 & 81.63 & 
905.43 & 91.34 & \bf{852.83} & 
4.41\\CMT4Y & 899.25 & 81.83 & 
903.17 & 83.64 & \bf{852.46} & 
5.49\\CMT5X & 1095.74 & 211.25 & 
1107.51 & 229.18 & \bf{1030.55} & 
6.33\\CMT5Y & 1096.09 & 209.48 & 
1104.99 & 251.43 & \bf{1031.17} & 
6.30\\CMT11X & 890.88 & 99.36 & 
895.66 & 81.39 & \bf{839.39} & 
6.13\\CMT11Y & 874.78 & 65.28 & 
879.99 & 64.55 & \bf{841.88} & 
3.91\\CMT12X & 674.73 & 31.60 & 
684.62 & 28.29 & \bf{662.22} & 
1.89\\CMT12Y & 674.34 & 24.04 & 
677.30 & 27.95 & \bf{662.22} & 
1.83\\\bf{PROM.} & 
\bf{780.88} & \bf{63.79} & \bf{787.22} & \bf{67.34} & \bf{751.25} & \bf{3.67}\\[1ex]\hline
\end{tabular}
\label{table:nonlin}
\end{table} \clearpage
\begin{table}[ht]
\caption{Resultados de la ejecución de la metaheurística ILS, utilizando instancias de SalhiNagy con la configuración -n 65.0 -LS 20.0}
\centering
\small
\begin{tabular}{c c c c c c c}
\hline\hline
Instancia & Costo mínimo & Tiempo(seg.) & Costo promedio & Tiempo promedio(seg.) & Costo ILS & \%Gap \\ [0.5ex]
\hline
CMT1X & 471.25 & 6.62 & 
476.19 & 5.47 & \bf{466.77} & 
0.96\\CMT1Y & 480.02 & 5.60 & 
486.50 & 5.25 & \bf{466.77} & 
2.84\\CMT2X & 706.32 & 13.36 & 
708.96 & 13.35 & \bf{684.21} & 
3.23\\CMT2Y & 693.13 & 13.50 & 
708.24 & 13.96 & \bf{684.21} & 
1.30\\CMT3X & 727.53 & 32.18 & 
737.80 & 32.35 & \bf{721.40} & 
0.85\\CMT3Y & 728.14 & 35.31 & 
732.01 & 36.15 & \bf{721.40} & 
0.93\\CMT4X & 895.02 & 94.16 & 
901.76 & 108.97 & \bf{852.83} & 
4.95\\CMT4Y & 883.33 & 94.71 & 
888.16 & 103.77 & \bf{852.46} & 
3.62\\CMT5X & 1098.91 & 212.67 & 
1104.01 & 270.07 & \bf{1030.55} & 
6.63\\CMT5Y & 1100.28 & 227.69 & 
1107.65 & 254.31 & \bf{1031.17} & 
6.70\\CMT11X & 881.34 & 110.89 & 
889.10 & 83.97 & \bf{839.39} & 
5.00\\CMT11Y & 885.02 & 109.11 & 
889.31 & 102.83 & \bf{841.88} & 
5.12\\CMT12X & 674.91 & 38.01 & 
679.95 & 34.05 & \bf{662.22} & 
1.92\\CMT12Y & 674.87 & 27.80 & 
682.06 & 28.51 & \bf{662.22} & 
1.91\\\bf{PROM.} & 
\bf{778.58} & \bf{72.97} & \bf{785.12} & \bf{78.07} & \bf{751.25} & \bf{3.28}\\[1ex]\hline
\end{tabular}
\label{table:nonlin}
\end{table} \clearpage
\begin{table}[ht]
\caption{Resultados de la ejecución de la metaheurística ILS, utilizando instancias de SalhiNagy con la configuración -n 65.0 -LS 30.0}
\centering
\small
\begin{tabular}{c c c c c c c}
\hline\hline
Instancia & Costo mínimo & Tiempo(seg.) & Costo promedio & Tiempo promedio(seg.) & Costo ILS & \%Gap \\ [0.5ex]
\hline
CMT1X & 470.67 & 7.89 & 
477.46 & 7.09 & \bf{466.77} & 
0.84\\CMT1Y & 475.35 & 8.14 & 
480.63 & 6.94 & \bf{466.77} & 
1.84\\CMT2X & 708.80 & 14.92 & 
714.60 & 16.67 & \bf{684.21} & 
3.59\\CMT2Y & 706.08 & 20.24 & 
714.54 & 17.36 & \bf{684.21} & 
3.20\\CMT3X & 739.24 & 51.03 & 
740.60 & 45.03 & \bf{721.40} & 
2.47\\CMT3Y & 731.43 & 42.70 & 
734.20 & 42.44 & \bf{721.40} & 
1.39\\CMT4X & 896.26 & 107.20 & 
904.59 & 107.51 & \bf{852.83} & 
5.09\\CMT4Y & 896.58 & 111.40 & 
902.41 & 117.45 & \bf{852.46} & 
5.18\\CMT5X & 1090.69 & 257.56 & 
1102.18 & 287.56 & \bf{1030.55} & 
5.84\\CMT5Y & 1088.29 & 259.91 & 
1097.49 & 279.70 & \bf{1031.17} & 
5.54\\CMT11X & 851.65 & 83.62 & 
879.16 & 94.43 & \bf{839.39} & 
1.46\\CMT11Y & 874.86 & 92.12 & 
892.72 & 84.15 & \bf{841.88} & 
3.92\\CMT12X & 680.03 & 35.06 & 
683.26 & 35.70 & \bf{662.22} & 
2.69\\CMT12Y & 674.65 & 33.70 & 
678.00 & 35.28 & \bf{662.22} & 
1.88\\\bf{PROM.} & 
\bf{777.47} & \bf{80.39} & \bf{785.85} & \bf{84.09} & \bf{751.25} & \bf{3.21}\\[1ex]\hline
\end{tabular}
\label{table:nonlin}
\end{table} \clearpage
\begin{table}[ht]
\caption{Resultados de la ejecución de la metaheurística ILS, utilizando instancias de SalhiNagy con la configuración -n 65.0 -LS 40.0}
\centering
\small
\begin{tabular}{c c c c c c c}
\hline\hline
Instancia & Costo mínimo & Tiempo(seg.) & Costo promedio & Tiempo promedio(seg.) & Costo ILS & \%Gap \\ [0.5ex]
\hline
CMT1X & 475.37 & 8.09 & 
481.59 & 8.25 & \bf{466.77} & 
1.84\\CMT1Y & 476.66 & 7.33 & 
479.63 & 8.38 & \bf{466.77} & 
2.12\\CMT2X & 705.23 & 19.06 & 
710.59 & 20.09 & \bf{684.21} & 
3.07\\CMT2Y & 697.78 & 20.57 & 
704.37 & 21.76 & \bf{684.21} & 
1.98\\CMT3X & 724.87 & 56.08 & 
732.34 & 50.64 & \bf{721.40} & 
0.48\\CMT3Y & 722.63 & 46.40 & 
729.83 & 48.79 & \bf{721.40} & 
0.17\\CMT4X & 896.24 & 132.03 & 
905.08 & 145.01 & \bf{852.83} & 
5.09\\CMT4Y & 878.80 & 123.82 & 
900.19 & 142.41 & \bf{852.46} & 
3.09\\CMT5X & 1093.44 & 385.22 & 
1103.22 & 361.29 & \bf{1030.55} & 
6.10\\CMT5Y & 1079.55 & 297.11 & 
1099.29 & 297.75 & \bf{1031.17} & 
4.69\\CMT11X & 874.27 & 92.55 & 
886.63 & 102.22 & \bf{839.39} & 
4.16\\CMT11Y & 879.58 & 129.26 & 
887.58 & 126.23 & \bf{841.88} & 
4.48\\CMT12X & 673.92 & 51.33 & 
679.75 & 44.42 & \bf{662.22} & 
1.77\\CMT12Y & 680.35 & 46.29 & 
683.94 & 42.12 & \bf{662.22} & 
2.74\\\bf{PROM.} & 
\bf{775.62} & \bf{101.08} & \bf{784.57} & \bf{101.38} & \bf{751.25} & \bf{2.98}\\[1ex]\hline
\end{tabular}
\label{table:nonlin}
\end{table} \clearpage
\begin{table}[ht]
\caption{Resultados de la ejecución de la metaheurística ILS, utilizando instancias de SalhiNagy con la configuración -n 65.0 -LS 50.0}
\centering
\small
\begin{tabular}{c c c c c c c}
\hline\hline
Instancia & Costo mínimo & Tiempo(seg.) & Costo promedio & Tiempo promedio(seg.) & Costo ILS & \%Gap \\ [0.5ex]
\hline
CMT1X & 473.96 & 9.28 & 
478.63 & 10.21 & \bf{466.77} & 
1.54\\CMT1Y & 470.48 & 9.61 & 
476.88 & 9.40 & \bf{466.77} & 
0.79\\CMT2X & 695.73 & 24.81 & 
709.80 & 23.98 & \bf{684.21} & 
1.68\\CMT2Y & 708.90 & 22.98 & 
711.76 & 23.97 & \bf{684.21} & 
3.61\\CMT3X & 731.39 & 64.80 & 
739.55 & 57.27 & \bf{721.40} & 
1.38\\CMT3Y & 729.39 & 54.10 & 
736.56 & 57.27 & \bf{721.40} & 
1.11\\CMT4X & 890.85 & 171.63 & 
899.41 & 146.97 & \bf{852.83} & 
4.46\\CMT4Y & 893.46 & 143.00 & 
902.28 & 158.32 & \bf{852.46} & 
4.81\\CMT5X & 1079.78 & 325.49 & 
1093.13 & 330.57 & \bf{1030.55} & 
4.78\\CMT5Y & 1080.33 & 319.15 & 
1098.41 & 351.53 & \bf{1031.17} & 
4.77\\CMT11X & 880.18 & 106.44 & 
890.03 & 116.41 & \bf{839.39} & 
4.86\\CMT11Y & 880.79 & 104.33 & 
889.73 & 106.72 & \bf{841.88} & 
4.62\\CMT12X & 674.63 & 45.71 & 
678.88 & 52.44 & \bf{662.22} & 
1.87\\CMT12Y & 674.52 & 59.62 & 
678.53 & 53.71 & \bf{662.22} & 
1.86\\\bf{PROM.} & 
\bf{776.03} & \bf{104.35} & \bf{784.54} & \bf{107.05} & \bf{751.25} & \bf{3.01}\\[1ex]\hline
\end{tabular}
\label{table:nonlin}
\end{table} \clearpage
\begin{table}[ht]
\caption{Resultados de la ejecución de la metaheurística ILS, utilizando instancias de SalhiNagy con la configuración -n 65.0 -LS 60.0}
\centering
\small
\begin{tabular}{c c c c c c c}
\hline\hline
Instancia & Costo mínimo & Tiempo(seg.) & Costo promedio & Tiempo promedio(seg.) & Costo ILS & \%Gap \\ [0.5ex]
\hline
CMT1X & 472.58 & 13.85 & 
476.54 & 12.17 & \bf{466.77} & 
1.24\\CMT1Y & 477.73 & 11.42 & 
480.37 & 10.22 & \bf{466.77} & 
2.35\\CMT2X & 693.67 & 27.40 & 
705.13 & 27.63 & \bf{684.21} & 
1.38\\CMT2Y & 703.54 & 27.23 & 
707.02 & 26.39 & \bf{684.21} & 
2.83\\CMT3X & 729.80 & 60.68 & 
732.32 & 59.83 & \bf{721.40} & 
1.16\\CMT3Y & 728.60 & 73.70 & 
735.05 & 66.97 & \bf{721.40} & 
1.00\\CMT4X & 896.40 & 151.07 & 
903.64 & 153.20 & \bf{852.83} & 
5.11\\CMT4Y & 894.03 & 149.53 & 
900.76 & 155.46 & \bf{852.46} & 
4.88\\CMT5X & 1089.90 & 362.26 & 
1104.00 & 352.00 & \bf{1030.55} & 
5.76\\CMT5Y & 1094.83 & 368.60 & 
1098.26 & 426.10 & \bf{1031.17} & 
6.17\\CMT11X & 878.86 & 112.92 & 
888.00 & 118.16 & \bf{839.39} & 
4.70\\CMT11Y & 878.11 & 154.68 & 
890.23 & 135.71 & \bf{841.88} & 
4.30\\CMT12X & 676.16 & 53.73 & 
680.49 & 56.48 & \bf{662.22} & 
2.11\\CMT12Y & 681.02 & 54.42 & 
682.50 & 55.65 & \bf{662.22} & 
2.84\\\bf{PROM.} & 
\bf{778.23} & \bf{115.82} & \bf{784.59} & \bf{118.28} & \bf{751.25} & \bf{3.27}\\[1ex]\hline
\end{tabular}
\label{table:nonlin}
\end{table} \clearpage
\begin{table}[ht]
\caption{Resultados de la ejecución de la metaheurística ILS, utilizando instancias de SalhiNagy con la configuración -n 65.0 -LS 70.0}
\centering
\small
\begin{tabular}{c c c c c c c}
\hline\hline
Instancia & Costo mínimo & Tiempo(seg.) & Costo promedio & Tiempo promedio(seg.) & Costo ILS & \%Gap \\ [0.5ex]
\hline
CMT1X & 476.93 & 14.61 & 
479.96 & 12.77 & \bf{466.77} & 
2.18\\CMT1Y & 478.84 & 13.00 & 
480.76 & 14.21 & \bf{466.77} & 
2.59\\CMT2X & 707.42 & 31.58 & 
711.80 & 30.91 & \bf{684.21} & 
3.39\\CMT2Y & 700.38 & 29.33 & 
707.35 & 31.00 & \bf{684.21} & 
2.36\\CMT3X & 736.54 & 80.94 & 
737.75 & 71.56 & \bf{721.40} & 
2.10\\CMT3Y & 726.79 & 68.60 & 
732.32 & 72.69 & \bf{721.40} & 
0.75\\CMT4X & 882.70 & 208.51 & 
893.84 & 189.25 & \bf{852.83} & 
3.50\\CMT4Y & 883.43 & 175.48 & 
894.68 & 178.16 & \bf{852.46} & 
3.63\\CMT5X & 1087.99 & 611.04 & 
1099.57 & 462.85 & \bf{1030.55} & 
5.57\\CMT5Y & 1092.31 & 506.29 & 
1097.11 & 449.00 & \bf{1031.17} & 
5.93\\CMT11X & 878.78 & 129.08 & 
890.98 & 146.91 & \bf{839.39} & 
4.69\\CMT11Y & 888.71 & 126.54 & 
893.79 & 129.77 & \bf{841.88} & 
5.56\\CMT12X & 675.41 & 60.79 & 
680.32 & 60.84 & \bf{662.22} & 
1.99\\CMT12Y & 674.21 & 56.85 & 
679.05 & 66.18 & \bf{662.22} & 
1.81\\\bf{PROM.} & 
\bf{777.89} & \bf{150.90} & \bf{784.23} & \bf{136.86} & \bf{751.25} & \bf{3.29}\\[1ex]\hline
\end{tabular}
\label{table:nonlin}
\end{table} \clearpage
\begin{table}[ht]
\caption{Resultados de la ejecución de la metaheurística ILS, utilizando instancias de SalhiNagy con la configuración -n 65.0 -LS 80.0}
\centering
\small
\begin{tabular}{c c c c c c c}
\hline\hline
Instancia & Costo mínimo & Tiempo(seg.) & Costo promedio & Tiempo promedio(seg.) & Costo ILS & \%Gap \\ [0.5ex]
\hline
CMT1X & 475.22 & 14.47 & 
480.72 & 13.95 & \bf{466.77} & 
1.81\\CMT1Y & 475.26 & 16.03 & 
479.17 & 16.13 & \bf{466.77} & 
1.82\\CMT2X & 702.48 & 31.34 & 
707.45 & 33.75 & \bf{684.21} & 
2.67\\CMT2Y & 698.74 & 36.73 & 
708.63 & 32.64 & \bf{684.21} & 
2.12\\CMT3X & 727.97 & 74.77 & 
731.61 & 75.04 & \bf{721.40} & 
0.91\\CMT3Y & 730.44 & 89.16 & 
734.41 & 80.67 & \bf{721.40} & 
1.25\\CMT4X & 896.89 & 188.41 & 
900.54 & 210.01 & \bf{852.83} & 
5.17\\CMT4Y & 888.04 & 194.10 & 
900.04 & 197.07 & \bf{852.46} & 
4.17\\CMT5X & 1075.89 & 445.93 & 
1096.47 & 467.52 & \bf{1030.55} & 
4.40\\CMT5Y & 1073.56 & 537.95 & 
1090.10 & 476.30 & \bf{1031.17} & 
4.11\\CMT11X & 870.22 & 136.41 & 
880.75 & 142.70 & \bf{839.39} & 
3.67\\CMT11Y & 871.52 & 181.71 & 
882.20 & 151.97 & \bf{841.88} & 
3.52\\CMT12X & 676.05 & 70.74 & 
679.97 & 71.23 & \bf{662.22} & 
2.09\\CMT12Y & 674.87 & 65.52 & 
677.47 & 69.69 & \bf{662.22} & 
1.91\\\bf{PROM.} & 
\bf{774.08} & \bf{148.81} & \bf{782.11} & \bf{145.62} & \bf{751.25} & \bf{2.83}\\[1ex]\hline
\end{tabular}
\label{table:nonlin}
\end{table} \clearpage

\begin{table}[ht]
\caption{Resultados de la ejecución de la metaheurística SCA, utilizando instancias de Dethloff con la configuración -n 100.0 -b 10 -y 0.1}
\centering
\small
\begin{tabular}{c c c c c c c}
\hline\hline
Instancia & Costo mínimo & Tiempo(seg.) & Costo promedio & Tiempo promedio(seg.) & Costo SCA & \%Gap \\ [0.5ex]
\hline
SCA3-0 & 640.55 & 13.06 & 
640.55 & 16.09 & \bf{636.06} & 
0.71\\SCA3-1 & 700.50 & 5.45 & 
700.50 & 6.45 & \bf{697.84} & 
0.38\\SCA3-2 & 664.18 & 10.43 & 
665.33 & 14.60 & \bf{659.34} & 
0.73\\SCA3-3 & 680.60 & 12.77 & 
680.60 & 10.29 & \bf{680.04} & 
0.08\\SCA3-4 & \bf{690.50} & 14.16 & 
690.50 & 15.05 & 690.50 & 0.00\\
SCA3-5 & 668.48 & 9.26 & 
668.59 & 13.33 & \bf{659.90} & 
1.30\\SCA3-6 & \bf{651.09} & 11.48 & 
651.09 & 10.54 & 651.09 & 0.00\\
SCA3-7 & 666.15 & 7.74 & 
666.15 & 9.77 & \bf{659.17} & 
1.06\\SCA3-8 & \bf{719.47} & 11.94 & 
719.47 & 10.95 & 719.47 & 0.00\\
SCA3-9 & \bf{681.00} & 15.59 & 
683.69 & 13.52 & 681.00 & 0.00\\
SCA8-0 & 982.58 & 39.81 & 
986.35 & 29.21 & \bf{961.50} & 
2.19\\SCA8-1 & \bf{1050.20} & 30.27 & 
1065.41 & 32.12 & 1050.20 & 0.00\\
SCA8-2 & 1050.37 & 56.93 & 
1052.74 & 45.61 & \bf{1039.64} & 
1.03\\SCA8-3 & 1010.82 & 27.93 & 
1011.69 & 31.54 & \bf{983.34} & 
2.79\\SCA8-4 & 1068.48 & 30.33 & 
1071.97 & 32.77 & \bf{1065.49} & 
0.28\\SCA8-5 & 1043.05 & 67.36 & 
1048.34 & 45.79 & \bf{1027.08} & 
1.55\\SCA8-6 & 972.48 & 29.57 & 
977.22 & 29.23 & \bf{971.82} & 
0.07\\SCA8-7 & 1070.53 & 25.17 & 
1073.87 & 27.12 & \bf{1052.17} & 
1.74\\SCA8-8 & \bf{1071.18} & 54.43 & 
1081.34 & 48.02 & 1071.18 & 0.00\\
SCA8-9 & 1067.42 & 22.67 & 
1073.62 & 31.48 & \bf{1060.50} & 
0.65\\CON3-0 & 617.59 & 9.40 & 
621.93 & 12.48 & \bf{616.52} & 
0.17\\CON3-1 & 560.75 & 14.13 & 
560.75 & 14.48 & \bf{554.47} & 
1.13\\CON3-2 & 521.38 & 24.20 & 
521.38 & 15.36 & \bf{519.26} & 
0.41\\CON3-3 & 591.20 & 11.29 & 
591.20 & 12.21 & \bf{591.19} & 
0.00\\CON3-4 & 591.43 & 9.10 & 
591.43 & 12.57 & \bf{589.32} & 
0.36\\CON3-5 & 564.89 & 6.21 & 
564.89 & 6.53 & \bf{563.70} & 
0.21\\CON3-6 & 502.16 & 11.02 & 
502.16 & 11.16 & \bf{500.80} & 
0.27\\CON3-7 & 578.79 & 12.01 & 
582.25 & 11.71 & \bf{576.48} & 
0.40\\CON3-8 & \bf{523.05} & 8.70 & 
524.32 & 11.88 & 523.05 & 0.00\\
CON3-9 & 588.40 & 7.77 & 
588.40 & 11.25 & \bf{580.05} & 
1.44\\CON8-0 & 863.53 & 29.00 & 
874.45 & 32.90 & \bf{857.17} & 
0.74\\CON8-1 & 742.47 & 43.57 & 
748.43 & 34.79 & \bf{740.85} & 
0.22\\CON8-2 & \bf{\underline{713.05}} & 45.22 & 
715.81 & 48.02 & 713.44 & 
\bf{-0.05}\\CON8-3 & 826.06 & 44.00 & 
828.76 & 37.33 & \bf{811.07} & 
1.85\\CON8-4 & 776.72 & 26.60 & 
783.43 & 29.30 & \bf{772.25} & 
0.58\\CON8-5 & \bf{\underline{754.95}} & 36.88 & 
760.23 & 35.12 & 756.91 & 
\bf{-0.26}\\CON8-6 & 688.00 & 24.66 & 
689.99 & 27.53 & \bf{678.92} & 
1.34\\CON8-7 & 814.77 & 30.82 & 
815.18 & 43.87 & \bf{811.96} & 
0.35\\CON8-8 & 784.82 & 23.84 & 
789.49 & 31.65 & \bf{767.53} & 
2.25\\CON8-9 & 816.07 & 33.90 & 
821.27 & 29.36 & \bf{809.00} & 
0.87\\\bf{PROM.} & 
\bf{764.24} & \bf{23.72} & \bf{767.12} & \bf{23.57} & \bf{758.78} & \bf{0.67}\\[1ex]\hline
\end{tabular}
\label{table:nonlin}
\end{table} \clearpage
\begin{table}[ht]
\caption{Resultados de la ejecución de la metaheurística SCA, utilizando instancias de Dethloff con la configuración -n 100.0 -b 10 -y .2}
\centering
\small
\begin{tabular}{c c c c c c c}
\hline\hline
Instancia & Costo mínimo & Tiempo(seg.) & Costo promedio & Tiempo promedio(seg.) & Costo SCA & \%Gap \\ [0.5ex]
\hline
SCA3-0 & 640.55 & 8.18 & 
640.84 & 11.89 & \bf{636.06} & 
0.71\\SCA3-1 & \bf{697.84} & 14.91 & 
699.43 & 15.33 & 697.84 & 0.00\\
SCA3-2 & 666.01 & 14.93 & 
668.20 & 14.91 & \bf{659.34} & 
1.01\\SCA3-3 & 680.60 & 16.77 & 
681.13 & 12.38 & \bf{680.04} & 
0.08\\SCA3-4 & 692.57 & 14.46 & 
692.57 & 14.64 & \bf{690.50} & 
0.30\\SCA3-5 & 665.04 & 12.50 & 
669.64 & 12.27 & \bf{659.90} & 
0.78\\SCA3-6 & 653.81 & 9.20 & 
654.43 & 13.88 & \bf{651.09} & 
0.42\\SCA3-7 & 666.60 & 6.54 & 
668.25 & 7.64 & \bf{659.17} & 
1.13\\SCA3-8 & \bf{719.47} & 12.14 & 
719.54 & 15.48 & 719.47 & 0.00\\
SCA3-9 & \bf{681.00} & 9.98 & 
681.00 & 10.80 & 681.00 & 0.00\\
SCA8-0 & 987.33 & 33.62 & 
998.16 & 26.16 & \bf{961.50} & 
2.69\\SCA8-1 & 1063.58 & 33.33 & 
1073.78 & 40.24 & \bf{1050.20} & 
1.27\\SCA8-2 & 1050.37 & 32.75 & 
1051.45 & 34.45 & \bf{1039.64} & 
1.03\\SCA8-3 & 995.60 & 22.20 & 
1014.75 & 28.63 & \bf{983.34} & 
1.25\\SCA8-4 & 1074.91 & 22.44 & 
1078.71 & 26.56 & \bf{1065.49} & 
0.88\\SCA8-5 & 1049.44 & 40.25 & 
1052.34 & 41.43 & \bf{1027.08} & 
2.18\\SCA8-6 & 972.48 & 53.54 & 
974.88 & 42.30 & \bf{971.82} & 
0.07\\SCA8-7 & 1063.22 & 37.91 & 
1068.95 & 27.86 & \bf{1052.17} & 
1.05\\SCA8-8 & 1082.12 & 28.13 & 
1089.04 & 29.50 & \bf{1071.18} & 
1.02\\SCA8-9 & 1072.91 & 62.32 & 
1075.61 & 43.10 & \bf{1060.50} & 
1.17\\CON3-0 & 617.98 & 7.33 & 
623.23 & 8.74 & \bf{616.52} & 
0.24\\CON3-1 & 560.75 & 8.78 & 
560.75 & 10.81 & \bf{554.47} & 
1.13\\CON3-2 & 521.38 & 14.84 & 
522.68 & 12.32 & \bf{519.26} & 
0.41\\CON3-3 & \bf{591.19} & 11.90 & 
591.19 & 16.61 & 591.19 & 0.00\\
CON3-4 & 591.43 & 7.65 & 
591.43 & 10.46 & \bf{589.32} & 
0.36\\CON3-5 & \bf{563.70} & 8.47 & 
563.70 & 9.38 & 563.70 & 0.00\\
CON3-6 & 501.34 & 13.67 & 
503.44 & 9.80 & \bf{500.80} & 
0.11\\CON3-7 & 578.22 & 18.94 & 
581.14 & 14.30 & \bf{576.48} & 
0.30\\CON3-8 & 523.68 & 9.03 & 
523.68 & 9.49 & \bf{523.05} & 
0.12\\CON3-9 & 588.40 & 8.76 & 
588.96 & 9.03 & \bf{580.05} & 
1.44\\CON8-0 & 871.84 & 23.38 & 
879.49 & 33.48 & \bf{857.17} & 
1.71\\CON8-1 & 742.47 & 46.22 & 
743.95 & 41.46 & \bf{740.85} & 
0.22\\CON8-2 & 714.06 & 25.69 & 
715.66 & 29.50 & \bf{713.44} & 
0.09\\CON8-3 & 823.24 & 55.48 & 
829.10 & 40.83 & \bf{811.07} & 
1.50\\CON8-4 & 777.81 & 24.05 & 
778.93 & 31.83 & \bf{772.25} & 
0.72\\CON8-5 & \bf{\underline{755.86}} & 35.93 & 
762.37 & 41.38 & 756.91 & 
\bf{-0.14}\\CON8-6 & 681.11 & 24.54 & 
688.87 & 26.36 & \bf{678.92} & 
0.32\\CON8-7 & 814.86 & 37.35 & 
820.43 & 42.32 & \bf{811.96} & 
0.36\\CON8-8 & 784.31 & 42.73 & 
789.22 & 40.16 & \bf{767.53} & 
2.19\\CON8-9 & 810.18 & 62.18 & 
818.71 & 43.20 & \bf{809.00} & 
0.15\\\bf{PROM.} & 
\bf{764.73} & \bf{24.33} & \bf{768.24} & \bf{23.77} & \bf{758.78} & \bf{0.71}\\[1ex]\hline
\end{tabular}
\label{table:nonlin}
\end{table} \clearpage
\begin{table}[ht]
\caption{Resultados de la ejecución de la metaheurística SCA, utilizando instancias de Dethloff con la configuración -n 100.0 -b 10 -y .3}
\centering
\small
\begin{tabular}{c c c c c c c}
\hline\hline
Instancia & Costo mínimo & Tiempo(seg.) & Costo promedio & Tiempo promedio(seg.) & Costo SCA & \%Gap \\ [0.5ex]
\hline
SCA3-0 & 640.55 & 13.34 & 
640.55 & 11.45 & \bf{636.06} & 
0.71\\SCA3-1 & \bf{697.84} & 9.84 & 
700.09 & 12.13 & 697.84 & 0.00\\
SCA3-2 & \bf{659.34} & 16.50 & 
662.99 & 15.65 & 659.34 & 0.00\\
SCA3-3 & \bf{680.04} & 21.05 & 
680.60 & 13.45 & 680.04 & 0.00\\
SCA3-4 & \bf{690.50} & 21.74 & 
691.02 & 17.75 & 690.50 & 0.00\\
SCA3-5 & 672.49 & 5.90 & 
672.49 & 8.46 & \bf{659.90} & 
1.91\\SCA3-6 & 652.94 & 11.64 & 
653.76 & 11.64 & \bf{651.09} & 
0.28\\SCA3-7 & 666.15 & 11.27 & 
666.81 & 8.81 & \bf{659.17} & 
1.06\\SCA3-8 & 719.77 & 14.14 & 
719.77 & 13.95 & \bf{719.47} & 
0.04\\SCA3-9 & \bf{681.00} & 11.34 & 
682.03 & 11.81 & 681.00 & 0.00\\
SCA8-0 & 984.75 & 27.91 & 
987.61 & 26.94 & \bf{961.50} & 
2.42\\SCA8-1 & 1050.93 & 44.93 & 
1072.99 & 46.67 & \bf{1050.20} & 
0.07\\SCA8-2 & 1051.42 & 45.65 & 
1051.59 & 39.98 & \bf{1039.64} & 
1.13\\SCA8-3 & \bf{983.34} & 24.59 & 
1008.88 & 32.21 & 983.34 & 0.00\\
SCA8-4 & 1075.55 & 26.42 & 
1079.24 & 29.59 & \bf{1065.49} & 
0.94\\SCA8-5 & 1049.44 & 47.48 & 
1053.05 & 34.48 & \bf{1027.08} & 
2.18\\SCA8-6 & 972.48 & 51.58 & 
976.85 & 54.85 & \bf{971.82} & 
0.07\\SCA8-7 & 1067.11 & 49.53 & 
1073.33 & 39.05 & \bf{1052.17} & 
1.42\\SCA8-8 & 1080.58 & 27.95 & 
1086.11 & 42.12 & \bf{1071.18} & 
0.88\\SCA8-9 & 1080.96 & 62.27 & 
1090.32 & 43.46 & \bf{1060.50} & 
1.93\\CON3-0 & 620.76 & 8.97 & 
625.78 & 9.10 & \bf{616.52} & 
0.69\\CON3-1 & \bf{554.47} & 10.48 & 
557.50 & 11.06 & 554.47 & 0.00\\
CON3-2 & 521.38 & 13.73 & 
521.38 & 10.84 & \bf{519.26} & 
0.41\\CON3-3 & 591.20 & 22.66 & 
591.20 & 17.64 & \bf{591.19} & 
0.00\\CON3-4 & 591.43 & 20.94 & 
591.43 & 13.91 & \bf{589.32} & 
0.36\\CON3-5 & 564.89 & 19.25 & 
567.02 & 14.35 & \bf{563.70} & 
0.21\\CON3-6 & 502.16 & 12.53 & 
502.16 & 11.94 & \bf{500.80} & 
0.27\\CON3-7 & 578.41 & 12.68 & 
582.61 & 13.98 & \bf{576.48} & 
0.33\\CON3-8 & 523.19 & 9.95 & 
523.19 & 12.63 & \bf{523.05} & 
0.03\\CON3-9 & 588.28 & 18.17 & 
588.37 & 12.15 & \bf{580.05} & 
1.42\\CON8-0 & 858.63 & 25.28 & 
871.73 & 31.45 & \bf{857.17} & 
0.17\\CON8-1 & 742.28 & 60.53 & 
749.00 & 41.74 & \bf{740.85} & 
0.19\\CON8-2 & \bf{\underline{713.05}} & 59.75 & 
714.94 & 40.22 & 713.44 & 
\bf{-0.05}\\CON8-3 & 817.22 & 36.25 & 
826.28 & 48.29 & \bf{811.07} & 
0.76\\CON8-4 & 781.56 & 23.14 & 
784.98 & 32.85 & \bf{772.25} & 
1.21\\CON8-5 & \bf{\underline{754.95}} & 46.54 & 
757.36 & 40.77 & 756.91 & 
\bf{-0.26}\\CON8-6 & 683.06 & 25.23 & 
684.25 & 32.04 & \bf{678.92} & 
0.61\\CON8-7 & 814.86 & 44.93 & 
816.37 & 39.62 & \bf{811.96} & 
0.36\\CON8-8 & 784.80 & 40.13 & 
788.26 & 39.24 & \bf{767.53} & 
2.25\\CON8-9 & 814.65 & 28.73 & 
817.07 & 34.49 & \bf{809.00} & 
0.70\\\bf{PROM.} & 
\bf{763.96} & \bf{27.12} & \bf{767.77} & \bf{25.57} & \bf{758.78} & \bf{0.62}\\[1ex]\hline
\end{tabular}
\label{table:nonlin}
\end{table} \clearpage
\begin{table}[ht]
\caption{Resultados de la ejecución de la metaheurística SCA, utilizando instancias de Dethloff con la configuración -n 100.0 -b 10 -y .4}
\centering
\small
\begin{tabular}{c c c c c c c}
\hline\hline
Instancia & Costo mínimo & Tiempo(seg.) & Costo promedio & Tiempo promedio(seg.) & Costo SCA & \%Gap \\ [0.5ex]
\hline
SCA3-0 & 640.55 & 27.58 & 
640.55 & 16.49 & \bf{636.06} & 
0.71\\SCA3-1 & \bf{697.84} & 10.33 & 
700.61 & 11.75 & 697.84 & 0.00\\
SCA3-2 & 661.13 & 9.42 & 
661.13 & 10.24 & \bf{659.34} & 
0.27\\SCA3-3 & 680.60 & 14.04 & 
680.60 & 17.00 & \bf{680.04} & 
0.08\\SCA3-4 & \bf{690.50} & 10.80 & 
692.55 & 12.56 & 690.50 & 0.00\\
SCA3-5 & 665.04 & 9.56 & 
668.70 & 15.19 & \bf{659.90} & 
0.78\\SCA3-6 & 652.94 & 19.57 & 
653.31 & 16.47 & \bf{651.09} & 
0.28\\SCA3-7 & 666.15 & 17.06 & 
666.81 & 12.59 & \bf{659.17} & 
1.06\\SCA3-8 & \bf{719.47} & 20.26 & 
719.47 & 17.04 & 719.47 & 0.00\\
SCA3-9 & \bf{681.00} & 16.97 & 
682.22 & 14.64 & 681.00 & 0.00\\
SCA8-0 & 970.64 & 41.61 & 
977.92 & 36.33 & \bf{961.50} & 
0.95\\SCA8-1 & 1060.19 & 34.70 & 
1074.67 & 29.33 & \bf{1050.20} & 
0.95\\SCA8-2 & 1050.37 & 48.84 & 
1051.92 & 41.01 & \bf{1039.64} & 
1.03\\SCA8-3 & 1014.19 & 26.84 & 
1022.00 & 31.87 & \bf{983.34} & 
3.14\\SCA8-4 & \bf{1065.49} & 44.25 & 
1084.76 & 37.67 & 1065.49 & 0.00\\
SCA8-5 & 1043.52 & 39.75 & 
1048.47 & 49.41 & \bf{1027.08} & 
1.60\\SCA8-6 & 977.87 & 33.38 & 
980.46 & 35.23 & \bf{971.82} & 
0.62\\SCA8-7 & 1067.67 & 30.14 & 
1071.49 & 33.22 & \bf{1052.17} & 
1.47\\SCA8-8 & 1088.65 & 34.27 & 
1089.65 & 46.78 & \bf{1071.18} & 
1.63\\SCA8-9 & 1069.70 & 28.10 & 
1075.93 & 33.94 & \bf{1060.50} & 
0.87\\CON3-0 & 620.76 & 8.22 & 
624.77 & 8.82 & \bf{616.52} & 
0.69\\CON3-1 & 560.75 & 11.29 & 
560.75 & 11.08 & \bf{554.47} & 
1.13\\CON3-2 & 521.38 & 10.48 & 
521.38 & 10.49 & \bf{519.26} & 
0.41\\CON3-3 & 591.20 & 8.13 & 
591.20 & 9.79 & \bf{591.19} & 
0.00\\CON3-4 & 591.43 & 16.15 & 
591.43 & 11.94 & \bf{589.32} & 
0.36\\CON3-5 & \bf{563.70} & 13.81 & 
563.70 & 9.99 & 563.70 & 0.00\\
CON3-6 & 502.16 & 16.77 & 
502.16 & 13.72 & \bf{500.80} & 
0.27\\CON3-7 & 578.41 & 17.58 & 
581.47 & 15.78 & \bf{576.48} & 
0.33\\CON3-8 & \bf{523.05} & 9.80 & 
523.28 & 10.48 & 523.05 & 0.00\\
CON3-9 & 588.40 & 7.80 & 
588.84 & 7.78 & \bf{580.05} & 
1.44\\CON8-0 & 869.68 & 49.25 & 
871.53 & 32.75 & \bf{857.17} & 
1.46\\CON8-1 & 742.47 & 74.39 & 
751.32 & 50.14 & \bf{740.85} & 
0.22\\CON8-2 & 716.56 & 28.55 & 
718.03 & 33.55 & \bf{713.44} & 
0.44\\CON8-3 & 828.85 & 30.56 & 
830.75 & 41.42 & \bf{811.07} & 
2.19\\CON8-4 & 781.83 & 25.33 & 
784.38 & 36.58 & \bf{772.25} & 
1.24\\CON8-5 & 762.61 & 39.74 & 
763.85 & 38.22 & \bf{756.91} & 
0.75\\CON8-6 & 695.85 & 36.59 & 
700.17 & 41.61 & \bf{678.92} & 
2.49\\CON8-7 & 815.72 & 27.32 & 
817.11 & 36.16 & \bf{811.96} & 
0.46\\CON8-8 & 776.55 & 40.61 & 
779.55 & 33.15 & \bf{767.53} & 
1.18\\CON8-9 & 812.25 & 45.99 & 
821.18 & 57.33 & \bf{809.00} & 
0.40\\\bf{PROM.} & 
\bf{765.18} & \bf{25.90} & \bf{768.25} & \bf{25.74} & \bf{758.78} & \bf{0.77}\\[1ex]\hline
\end{tabular}
\label{table:nonlin}
\end{table} \clearpage
\begin{table}[ht]
\caption{Resultados de la ejecución de la metaheurística SCA, utilizando instancias de Dethloff con la configuración -n 100.0 -b 10 -y .5}
\centering
\small
\begin{tabular}{c c c c c c c}
\hline\hline
Instancia & Costo mínimo & Tiempo(seg.) & Costo promedio & Tiempo promedio(seg.) & Costo SCA & \%Gap \\ [0.5ex]
\hline
SCA3-0 & 640.55 & 14.77 & 
640.55 & 15.20 & \bf{636.06} & 
0.71\\SCA3-1 & \bf{697.84} & 12.24 & 
698.76 & 14.47 & 697.84 & 0.00\\
SCA3-2 & 664.21 & 14.55 & 
665.56 & 18.03 & \bf{659.34} & 
0.74\\SCA3-3 & 680.60 & 12.98 & 
680.60 & 10.95 & \bf{680.04} & 
0.08\\SCA3-4 & \bf{690.50} & 23.41 & 
691.02 & 15.04 & 690.50 & 0.00\\
SCA3-5 & 673.56 & 10.50 & 
678.76 & 9.86 & \bf{659.90} & 
2.07\\SCA3-6 & 652.94 & 9.88 & 
653.93 & 13.16 & \bf{651.09} & 
0.28\\SCA3-7 & 666.60 & 9.51 & 
666.60 & 11.51 & \bf{659.17} & 
1.13\\SCA3-8 & \bf{719.47} & 12.81 & 
719.70 & 12.72 & 719.47 & 0.00\\
SCA3-9 & \bf{681.00} & 11.64 & 
682.03 & 16.19 & 681.00 & 0.00\\
SCA8-0 & 974.40 & 33.32 & 
986.23 & 37.33 & \bf{961.50} & 
1.34\\SCA8-1 & 1061.47 & 40.32 & 
1068.70 & 32.09 & \bf{1050.20} & 
1.07\\SCA8-2 & 1051.21 & 37.37 & 
1053.26 & 43.08 & \bf{1039.64} & 
1.11\\SCA8-3 & 1022.69 & 45.34 & 
1023.98 & 34.27 & \bf{983.34} & 
4.00\\SCA8-4 & 1074.87 & 29.39 & 
1076.86 & 33.31 & \bf{1065.49} & 
0.88\\SCA8-5 & 1049.44 & 60.21 & 
1049.44 & 48.94 & \bf{1027.08} & 
2.18\\SCA8-6 & 972.48 & 71.24 & 
979.43 & 65.90 & \bf{971.82} & 
0.07\\SCA8-7 & 1067.49 & 32.21 & 
1074.17 & 35.75 & \bf{1052.17} & 
1.46\\SCA8-8 & 1082.91 & 35.69 & 
1085.58 & 40.56 & \bf{1071.18} & 
1.10\\SCA8-9 & 1072.10 & 32.11 & 
1075.93 & 38.67 & \bf{1060.50} & 
1.09\\CON3-0 & 617.59 & 11.80 & 
623.78 & 9.70 & \bf{616.52} & 
0.17\\CON3-1 & 560.75 & 21.67 & 
560.75 & 15.77 & \bf{554.47} & 
1.13\\CON3-2 & 521.38 & 19.85 & 
521.70 & 15.02 & \bf{519.26} & 
0.41\\CON3-3 & \bf{591.19} & 12.84 & 
591.20 & 14.06 & 591.19 & 0.00\\
CON3-4 & \bf{\underline{588.79}} & 20.24 & 
590.38 & 14.21 & 589.32 & 
\bf{-0.09}\\CON3-5 & \bf{563.70} & 8.46 & 
564.29 & 13.21 & 563.70 & 0.00\\
CON3-6 & 502.16 & 12.92 & 
502.16 & 12.81 & \bf{500.80} & 
0.27\\CON3-7 & 585.42 & 15.29 & 
585.86 & 11.80 & \bf{576.48} & 
1.55\\CON3-8 & \bf{523.05} & 11.16 & 
523.15 & 9.09 & 523.05 & 0.00\\
CON3-9 & 588.40 & 15.62 & 
588.40 & 10.80 & \bf{580.05} & 
1.44\\CON8-0 & 867.76 & 30.24 & 
876.05 & 26.85 & \bf{857.17} & 
1.24\\CON8-1 & 740.93 & 31.10 & 
745.58 & 40.13 & \bf{740.85} & 
0.01\\CON8-2 & \bf{\underline{712.89}} & 62.79 & 
719.77 & 42.40 & 713.44 & 
\bf{-0.08}\\CON8-3 & 816.27 & 23.03 & 
822.21 & 31.84 & \bf{811.07} & 
0.64\\CON8-4 & 784.83 & 34.32 & 
786.07 & 34.55 & \bf{772.25} & 
1.63\\CON8-5 & 760.41 & 35.31 & 
768.68 & 43.35 & \bf{756.91} & 
0.46\\CON8-6 & 697.24 & 32.83 & 
699.91 & 40.99 & \bf{678.92} & 
2.70\\CON8-7 & 815.72 & 30.27 & 
819.00 & 30.58 & \bf{811.96} & 
0.46\\CON8-8 & 782.68 & 25.51 & 
786.85 & 35.79 & \bf{767.53} & 
1.97\\CON8-9 & 819.56 & 56.31 & 
824.24 & 36.94 & \bf{809.00} & 
1.31\\\bf{PROM.} & 
\bf{765.93} & \bf{26.53} & \bf{768.78} & \bf{25.92} & \bf{758.78} & \bf{0.86}\\[1ex]\hline
\end{tabular}
\label{table:nonlin}
\end{table} \clearpage
\begin{table}[ht]
\caption{Resultados de la ejecución de la metaheurística SCA, utilizando instancias de Dethloff con la configuración -n 125.0 -b 10 -y 0.1}
\centering
\small
\begin{tabular}{c c c c c c c}
\hline\hline
Instancia & Costo mínimo & Tiempo(seg.) & Costo promedio & Tiempo promedio(seg.) & Costo SCA & \%Gap \\ [0.5ex]
\hline
SCA3-0 & 640.55 & 14.30 & 
640.55 & 13.81 & \bf{636.06} & 
0.71\\SCA3-1 & \bf{697.84} & 13.98 & 
698.50 & 13.89 & 697.84 & 0.00\\
SCA3-2 & 661.13 & 27.50 & 
662.08 & 15.54 & \bf{659.34} & 
0.27\\SCA3-3 & 680.60 & 12.98 & 
680.60 & 10.59 & \bf{680.04} & 
0.08\\SCA3-4 & \bf{690.50} & 9.44 & 
692.05 & 9.99 & 690.50 & 0.00\\
SCA3-5 & 665.04 & 11.55 & 
665.04 & 11.61 & \bf{659.90} & 
0.78\\SCA3-6 & 653.81 & 11.69 & 
654.46 & 13.21 & \bf{651.09} & 
0.42\\SCA3-7 & 666.15 & 16.36 & 
667.75 & 17.64 & \bf{659.17} & 
1.06\\SCA3-8 & \bf{719.47} & 9.65 & 
719.54 & 11.61 & 719.47 & 0.00\\
SCA3-9 & 681.68 & 15.82 & 
683.42 & 14.42 & \bf{681.00} & 
0.10\\SCA8-0 & \bf{961.50} & 36.81 & 
977.13 & 29.41 & 961.50 & 0.00\\
SCA8-1 & 1062.88 & 30.80 & 
1068.81 & 42.00 & \bf{1050.20} & 
1.21\\SCA8-2 & 1050.37 & 31.33 & 
1051.85 & 39.39 & \bf{1039.64} & 
1.03\\SCA8-3 & 1017.03 & 40.11 & 
1020.26 & 31.48 & \bf{983.34} & 
3.43\\SCA8-4 & 1071.35 & 40.02 & 
1079.47 & 39.85 & \bf{1065.49} & 
0.55\\SCA8-5 & 1047.41 & 37.79 & 
1049.36 & 34.77 & \bf{1027.08} & 
1.98\\SCA8-6 & 972.48 & 63.51 & 
975.17 & 43.56 & \bf{971.82} & 
0.07\\SCA8-7 & 1067.03 & 28.63 & 
1070.31 & 41.23 & \bf{1052.17} & 
1.41\\SCA8-8 & \bf{1071.18} & 35.06 & 
1088.73 & 39.08 & 1071.18 & 0.00\\
SCA8-9 & 1070.71 & 31.81 & 
1082.07 & 40.69 & \bf{1060.50} & 
0.96\\CON3-0 & 624.84 & 8.66 & 
627.68 & 11.35 & \bf{616.52} & 
1.35\\CON3-1 & 560.75 & 15.95 & 
560.75 & 13.00 & \bf{554.47} & 
1.13\\CON3-2 & 521.38 & 10.18 & 
521.38 & 13.10 & \bf{519.26} & 
0.41\\CON3-3 & 591.20 & 13.43 & 
591.20 & 10.73 & \bf{591.19} & 
0.00\\CON3-4 & 591.43 & 14.43 & 
591.43 & 10.31 & \bf{589.32} & 
0.36\\CON3-5 & \bf{563.70} & 12.68 & 
564.29 & 11.30 & 563.70 & 0.00\\
CON3-6 & 502.16 & 12.16 & 
502.16 & 10.54 & \bf{500.80} & 
0.27\\CON3-7 & 585.42 & 13.43 & 
585.42 & 14.26 & \bf{576.48} & 
1.55\\CON3-8 & 523.14 & 8.48 & 
523.29 & 8.55 & \bf{523.05} & 
0.02\\CON3-9 & 588.40 & 12.20 & 
588.40 & 12.03 & \bf{580.05} & 
1.44\\CON8-0 & 874.79 & 25.49 & 
874.79 & 23.41 & \bf{857.17} & 
2.06\\CON8-1 & 748.85 & 40.54 & 
753.87 & 38.94 & \bf{740.85} & 
1.08\\CON8-2 & 713.60 & 69.54 & 
717.12 & 44.87 & \bf{713.44} & 
0.02\\CON8-3 & 822.65 & 40.17 & 
825.73 & 34.16 & \bf{811.07} & 
1.43\\CON8-4 & 785.86 & 38.50 & 
788.92 & 40.62 & \bf{772.25} & 
1.76\\CON8-5 & 759.82 & 39.80 & 
762.95 & 48.09 & \bf{756.91} & 
0.38\\CON8-6 & 692.81 & 23.10 & 
694.36 & 34.27 & \bf{678.92} & 
2.05\\CON8-7 & 814.50 & 50.23 & 
816.00 & 40.48 & \bf{811.96} & 
0.31\\CON8-8 & 783.13 & 40.45 & 
790.93 & 32.82 & \bf{767.53} & 
2.03\\CON8-9 & 821.42 & 32.55 & 
823.05 & 36.68 & \bf{809.00} & 
1.54\\\bf{PROM.} & 
\bf{765.46} & \bf{26.03} & \bf{768.27} & \bf{25.08} & \bf{758.78} & \bf{0.83}\\[1ex]\hline
\end{tabular}
\label{table:nonlin}
\end{table} \clearpage
\begin{table}[ht]
\caption{Resultados de la ejecución de la metaheurística SCA, utilizando instancias de Dethloff con la configuración -n 125.0 -b 10 -y .2}
\centering
\small
\begin{tabular}{c c c c c c c}
\hline\hline
Instancia & Costo mínimo & Tiempo(seg.) & Costo promedio & Tiempo promedio(seg.) & Costo SCA & \%Gap \\ [0.5ex]
\hline
SCA3-0 & 640.55 & 27.14 & 
640.55 & 17.44 & \bf{636.06} & 
0.71\\SCA3-1 & 701.78 & 13.93 & 
701.84 & 10.93 & \bf{697.84} & 
0.56\\SCA3-2 & 661.13 & 17.48 & 
664.44 & 20.01 & \bf{659.34} & 
0.27\\SCA3-3 & 681.74 & 17.03 & 
681.91 & 11.58 & \bf{680.04} & 
0.25\\SCA3-4 & \bf{690.50} & 19.20 & 
691.70 & 16.11 & 690.50 & 0.00\\
SCA3-5 & 665.04 & 12.79 & 
668.84 & 11.18 & \bf{659.90} & 
0.78\\SCA3-6 & 652.94 & 11.70 & 
653.16 & 12.10 & \bf{651.09} & 
0.28\\SCA3-7 & 667.24 & 10.34 & 
668.35 & 10.28 & \bf{659.17} & 
1.22\\SCA3-8 & \bf{719.47} & 16.42 & 
720.44 & 16.21 & 719.47 & 0.00\\
SCA3-9 & \bf{681.00} & 15.35 & 
681.06 & 11.95 & 681.00 & 0.00\\
SCA8-0 & 995.46 & 32.28 & 
999.43 & 36.02 & \bf{961.50} & 
3.53\\SCA8-1 & 1058.59 & 40.34 & 
1070.04 & 40.22 & \bf{1050.20} & 
0.80\\SCA8-2 & 1053.78 & 30.36 & 
1053.88 & 36.05 & \bf{1039.64} & 
1.36\\SCA8-3 & 1002.63 & 43.35 & 
1008.15 & 35.12 & \bf{983.34} & 
1.96\\SCA8-4 & 1067.55 & 32.74 & 
1071.33 & 28.52 & \bf{1065.49} & 
0.19\\SCA8-5 & 1042.51 & 45.30 & 
1046.45 & 51.40 & \bf{1027.08} & 
1.50\\SCA8-6 & \bf{971.82} & 33.56 & 
978.65 & 32.88 & 971.82 & 0.00\\
SCA8-7 & 1070.89 & 51.21 & 
1074.02 & 34.99 & \bf{1052.17} & 
1.78\\SCA8-8 & \bf{1071.18} & 24.26 & 
1081.14 & 28.45 & 1071.18 & 0.00\\
SCA8-9 & 1072.10 & 34.11 & 
1078.19 & 39.95 & \bf{1060.50} & 
1.09\\CON3-0 & 617.59 & 6.23 & 
618.34 & 7.07 & \bf{616.52} & 
0.17\\CON3-1 & 560.75 & 11.69 & 
560.75 & 13.75 & \bf{554.47} & 
1.13\\CON3-2 & 521.38 & 12.40 & 
521.38 & 8.91 & \bf{519.26} & 
0.41\\CON3-3 & \bf{591.19} & 13.98 & 
591.19 & 12.64 & 591.19 & 0.00\\
CON3-4 & 591.43 & 8.88 & 
591.43 & 9.15 & \bf{589.32} & 
0.36\\CON3-5 & 564.88 & 11.38 & 
564.89 & 9.21 & \bf{563.70} & 
0.21\\CON3-6 & 502.16 & 11.38 & 
502.16 & 10.71 & \bf{500.80} & 
0.27\\CON3-7 & 578.41 & 16.86 & 
578.41 & 13.42 & \bf{576.48} & 
0.33\\CON3-8 & 523.19 & 13.98 & 
523.19 & 13.56 & \bf{523.05} & 
0.03\\CON3-9 & 588.40 & 14.83 & 
589.00 & 11.32 & \bf{580.05} & 
1.44\\CON8-0 & 873.77 & 37.66 & 
876.41 & 33.73 & \bf{857.17} & 
1.94\\CON8-1 & 748.85 & 35.51 & 
751.35 & 33.22 & \bf{740.85} & 
1.08\\CON8-2 & 713.60 & 31.68 & 
717.57 & 42.83 & \bf{713.44} & 
0.02\\CON8-3 & 816.87 & 26.90 & 
817.97 & 28.07 & \bf{811.07} & 
0.72\\CON8-4 & 776.98 & 33.77 & 
776.98 & 25.56 & \bf{772.25} & 
0.61\\CON8-5 & \bf{756.91} & 38.51 & 
768.97 & 35.74 & 756.91 & 0.00\\
CON8-6 & 698.41 & 34.13 & 
703.06 & 26.19 & \bf{678.92} & 
2.87\\CON8-7 & 815.79 & 20.99 & 
819.13 & 26.19 & \bf{811.96} & 
0.47\\CON8-8 & 781.42 & 45.17 & 
784.49 & 40.33 & \bf{767.53} & 
1.81\\CON8-9 & 821.09 & 25.52 & 
822.78 & 34.47 & \bf{809.00} & 
1.49\\\bf{PROM.} & 
\bf{765.27} & \bf{24.51} & \bf{767.83} & \bf{23.44} & \bf{758.78} & \bf{0.79}\\[1ex]\hline
\end{tabular}
\label{table:nonlin}
\end{table} \clearpage
\begin{table}[ht]
\caption{Resultados de la ejecución de la metaheurística SCA, utilizando instancias de Dethloff con la configuración -n 125.0 -b 10 -y .3}
\centering
\small
\begin{tabular}{c c c c c c c}
\hline\hline
Instancia & Costo mínimo & Tiempo(seg.) & Costo promedio & Tiempo promedio(seg.) & Costo SCA & \%Gap \\ [0.5ex]
\hline
SCA3-0 & 640.55 & 7.97 & 
640.55 & 13.95 & \bf{636.06} & 
0.71\\SCA3-1 & \bf{697.84} & 13.83 & 
698.76 & 11.76 & 697.84 & 0.00\\
SCA3-2 & 664.92 & 15.75 & 
665.12 & 17.70 & \bf{659.34} & 
0.85\\SCA3-3 & 680.60 & 14.57 & 
680.60 & 10.91 & \bf{680.04} & 
0.08\\SCA3-4 & \bf{690.50} & 11.08 & 
690.50 & 13.05 & 690.50 & 0.00\\
SCA3-5 & 665.64 & 9.61 & 
667.77 & 14.46 & \bf{659.90} & 
0.87\\SCA3-6 & \bf{651.09} & 25.45 & 
652.88 & 21.14 & 651.09 & 0.00\\
SCA3-7 & 671.67 & 9.20 & 
671.67 & 8.70 & \bf{659.17} & 
1.90\\SCA3-8 & \bf{719.47} & 26.74 & 
719.54 & 15.40 & 719.47 & 0.00\\
SCA3-9 & \bf{681.00} & 13.72 & 
682.03 & 13.08 & 681.00 & 0.00\\
SCA8-0 & 979.79 & 27.58 & 
980.41 & 32.00 & \bf{961.50} & 
1.90\\SCA8-1 & 1050.93 & 39.55 & 
1059.98 & 40.08 & \bf{1050.20} & 
0.07\\SCA8-2 & 1051.42 & 61.29 & 
1052.56 & 42.79 & \bf{1039.64} & 
1.13\\SCA8-3 & 1021.22 & 22.78 & 
1021.22 & 24.25 & \bf{983.34} & 
3.85\\SCA8-4 & 1071.61 & 41.97 & 
1078.99 & 33.33 & \bf{1065.49} & 
0.57\\SCA8-5 & 1038.59 & 34.81 & 
1048.83 & 37.85 & \bf{1027.08} & 
1.12\\SCA8-6 & 977.87 & 34.59 & 
984.33 & 34.21 & \bf{971.82} & 
0.62\\SCA8-7 & 1063.60 & 69.02 & 
1074.69 & 52.28 & \bf{1052.17} & 
1.09\\SCA8-8 & 1090.71 & 26.15 & 
1091.05 & 29.29 & \bf{1071.18} & 
1.82\\SCA8-9 & 1068.65 & 26.58 & 
1080.24 & 35.06 & \bf{1060.50} & 
0.77\\CON3-0 & 617.59 & 11.57 & 
622.61 & 9.88 & \bf{616.52} & 
0.17\\CON3-1 & 556.04 & 11.63 & 
558.39 & 11.23 & \bf{554.47} & 
0.28\\CON3-2 & 521.38 & 8.48 & 
521.38 & 9.11 & \bf{519.26} & 
0.41\\CON3-3 & \bf{591.19} & 10.45 & 
591.19 & 11.52 & 591.19 & 0.00\\
CON3-4 & 591.43 & 10.60 & 
591.43 & 11.34 & \bf{589.32} & 
0.36\\CON3-5 & \bf{563.70} & 12.60 & 
564.70 & 12.94 & 563.70 & 0.00\\
CON3-6 & 502.16 & 7.41 & 
502.16 & 11.09 & \bf{500.80} & 
0.27\\CON3-7 & 578.41 & 11.57 & 
579.79 & 12.26 & \bf{576.48} & 
0.33\\CON3-8 & \bf{523.05} & 14.81 & 
523.67 & 17.22 & 523.05 & 0.00\\
CON3-9 & 582.79 & 7.53 & 
582.79 & 10.18 & \bf{580.05} & 
0.47\\CON8-0 & 870.61 & 19.70 & 
874.31 & 28.30 & \bf{857.17} & 
1.57\\CON8-1 & 742.47 & 29.58 & 
745.72 & 42.20 & \bf{740.85} & 
0.22\\CON8-2 & 713.60 & 33.18 & 
714.65 & 39.52 & \bf{713.44} & 
0.02\\CON8-3 & 823.06 & 27.92 & 
825.69 & 27.97 & \bf{811.07} & 
1.48\\CON8-4 & 777.24 & 27.42 & 
783.94 & 36.00 & \bf{772.25} & 
0.65\\CON8-5 & 763.13 & 34.76 & 
766.91 & 32.40 & \bf{756.91} & 
0.82\\CON8-6 & 693.83 & 29.78 & 
696.01 & 45.40 & \bf{678.92} & 
2.20\\CON8-7 & 814.79 & 73.33 & 
815.53 & 49.73 & \bf{811.96} & 
0.35\\CON8-8 & 777.28 & 27.25 & 
777.28 & 28.05 & \bf{767.53} & 
1.27\\CON8-9 & 825.61 & 30.44 & 
829.43 & 38.61 & \bf{809.00} & 
2.05\\\bf{PROM.} & 
\bf{765.18} & \bf{24.31} & \bf{767.73} & \bf{24.66} & \bf{758.78} & \bf{0.76}\\[1ex]\hline
\end{tabular}
\label{table:nonlin}
\end{table} \clearpage
\begin{table}[ht]
\caption{Resultados de la ejecución de la metaheurística SCA, utilizando instancias de Dethloff con la configuración -n 125.0 -b 10 -y .4}
\centering
\small
\begin{tabular}{c c c c c c c}
\hline\hline
Instancia & Costo mínimo & Tiempo(seg.) & Costo promedio & Tiempo promedio(seg.) & Costo SCA & \%Gap \\ [0.5ex]
\hline
SCA3-0 & 640.55 & 12.56 & 
640.55 & 13.18 & \bf{636.06} & 
0.71\\SCA3-1 & \bf{697.84} & 10.68 & 
699.68 & 13.60 & 697.84 & 0.00\\
SCA3-2 & \bf{659.34} & 15.29 & 
663.16 & 15.89 & 659.34 & 0.00\\
SCA3-3 & 681.74 & 12.43 & 
681.74 & 11.32 & \bf{680.04} & 
0.25\\SCA3-4 & \bf{690.50} & 14.30 & 
690.50 & 12.67 & 690.50 & 0.00\\
SCA3-5 & 670.02 & 9.41 & 
675.91 & 15.20 & \bf{659.90} & 
1.53\\SCA3-6 & 652.94 & 13.09 & 
653.35 & 15.43 & \bf{651.09} & 
0.28\\SCA3-7 & 666.60 & 9.92 & 
670.40 & 15.48 & \bf{659.17} & 
1.13\\SCA3-8 & \bf{719.47} & 28.27 & 
719.47 & 16.43 & 719.47 & 0.00\\
SCA3-9 & \bf{681.00} & 13.99 & 
682.03 & 14.73 & 681.00 & 0.00\\
SCA8-0 & 989.06 & 38.17 & 
998.03 & 32.22 & \bf{961.50} & 
2.87\\SCA8-1 & 1054.07 & 34.24 & 
1064.55 & 44.39 & \bf{1050.20} & 
0.37\\SCA8-2 & 1051.95 & 33.64 & 
1053.96 & 42.83 & \bf{1039.64} & 
1.18\\SCA8-3 & 1012.89 & 35.51 & 
1021.92 & 25.79 & \bf{983.34} & 
3.01\\SCA8-4 & 1074.19 & 36.37 & 
1076.33 & 30.39 & \bf{1065.49} & 
0.82\\SCA8-5 & 1032.79 & 54.27 & 
1046.16 & 39.14 & \bf{1027.08} & 
0.56\\SCA8-6 & 972.48 & 39.51 & 
974.66 & 38.09 & \bf{971.82} & 
0.07\\SCA8-7 & 1066.65 & 29.25 & 
1068.20 & 45.20 & \bf{1052.17} & 
1.38\\SCA8-8 & 1085.34 & 64.51 & 
1093.42 & 38.54 & \bf{1071.18} & 
1.32\\SCA8-9 & 1067.27 & 42.32 & 
1067.62 & 35.39 & \bf{1060.50} & 
0.64\\CON3-0 & 619.09 & 8.96 & 
619.09 & 9.20 & \bf{616.52} & 
0.42\\CON3-1 & 556.92 & 9.40 & 
556.92 & 9.18 & \bf{554.47} & 
0.44\\CON3-2 & 521.38 & 13.57 & 
521.38 & 15.74 & \bf{519.26} & 
0.41\\CON3-3 & 591.20 & 10.45 & 
591.20 & 11.86 & \bf{591.19} & 
0.00\\CON3-4 & 591.43 & 8.49 & 
591.43 & 8.89 & \bf{589.32} & 
0.36\\CON3-5 & \bf{563.70} & 10.04 & 
563.70 & 9.26 & 563.70 & 0.00\\
CON3-6 & 502.16 & 13.44 & 
502.16 & 15.40 & \bf{500.80} & 
0.27\\CON3-7 & 581.39 & 14.72 & 
581.75 & 16.21 & \bf{576.48} & 
0.85\\CON3-8 & 523.19 & 7.42 & 
523.19 & 7.91 & \bf{523.05} & 
0.03\\CON3-9 & 582.79 & 12.72 & 
585.60 & 9.78 & \bf{580.05} & 
0.47\\CON8-0 & 873.79 & 41.99 & 
876.65 & 35.25 & \bf{857.17} & 
1.94\\CON8-1 & 742.47 & 30.15 & 
746.87 & 45.78 & \bf{740.85} & 
0.22\\CON8-2 & 716.07 & 31.12 & 
717.49 & 32.16 & \bf{713.44} & 
0.37\\CON8-3 & 816.87 & 43.03 & 
826.73 & 51.82 & \bf{811.07} & 
0.72\\CON8-4 & 784.99 & 30.82 & 
789.71 & 33.76 & \bf{772.25} & 
1.65\\CON8-5 & 765.06 & 37.45 & 
771.59 & 31.11 & \bf{756.91} & 
1.08\\CON8-6 & 693.82 & 60.59 & 
697.85 & 46.62 & \bf{678.92} & 
2.19\\CON8-7 & 814.79 & 67.48 & 
816.78 & 53.78 & \bf{811.96} & 
0.35\\CON8-8 & 781.96 & 27.25 & 
787.05 & 36.10 & \bf{767.53} & 
1.88\\CON8-9 & 817.29 & 43.57 & 
817.60 & 36.53 & \bf{809.00} & 
1.02\\\bf{PROM.} & 
\bf{765.18} & \bf{26.76} & \bf{768.16} & \bf{25.81} & \bf{758.78} & \bf{0.77}\\[1ex]\hline
\end{tabular}
\label{table:nonlin}
\end{table} \clearpage
\begin{table}[ht]
\caption{Resultados de la ejecución de la metaheurística SCA, utilizando instancias de Dethloff con la configuración -n 125.0 -b 10 -y .5}
\centering
\small
\begin{tabular}{c c c c c c c}
\hline\hline
Instancia & Costo mínimo & Tiempo(seg.) & Costo promedio & Tiempo promedio(seg.) & Costo SCA & \%Gap \\ [0.5ex]
\hline
SCA3-0 & 640.55 & 17.72 & 
640.55 & 14.93 & \bf{636.06} & 
0.71\\SCA3-1 & \bf{697.84} & 16.38 & 
697.84 & 12.84 & 697.84 & 0.00\\
SCA3-2 & \bf{659.34} & 22.54 & 
662.99 & 22.30 & 659.34 & 0.00\\
SCA3-3 & 680.60 & 9.62 & 
680.60 & 12.11 & \bf{680.04} & 
0.08\\SCA3-4 & \bf{690.50} & 22.30 & 
690.50 & 18.62 & 690.50 & 0.00\\
SCA3-5 & 668.48 & 15.88 & 
669.25 & 13.64 & \bf{659.90} & 
1.30\\SCA3-6 & \bf{651.09} & 11.76 & 
651.09 & 11.79 & 651.09 & 0.00\\
SCA3-7 & 666.15 & 9.85 & 
669.28 & 12.01 & \bf{659.17} & 
1.06\\SCA3-8 & \bf{719.47} & 10.62 & 
719.47 & 13.48 & 719.47 & 0.00\\
SCA3-9 & \bf{681.00} & 12.71 & 
683.07 & 12.16 & 681.00 & 0.00\\
SCA8-0 & 970.64 & 38.21 & 
973.81 & 32.73 & \bf{961.50} & 
0.95\\SCA8-1 & 1063.83 & 30.92 & 
1064.85 & 32.00 & \bf{1050.20} & 
1.30\\SCA8-2 & 1052.94 & 45.49 & 
1052.94 & 34.68 & \bf{1039.64} & 
1.28\\SCA8-3 & 1013.56 & 45.11 & 
1026.33 & 38.95 & \bf{983.34} & 
3.07\\SCA8-4 & 1068.97 & 37.37 & 
1072.12 & 38.37 & \bf{1065.49} & 
0.33\\SCA8-5 & 1044.70 & 27.70 & 
1052.76 & 32.67 & \bf{1027.08} & 
1.72\\SCA8-6 & 972.48 & 49.77 & 
974.66 & 57.11 & \bf{971.82} & 
0.07\\SCA8-7 & 1063.22 & 35.83 & 
1067.60 & 32.55 & \bf{1052.17} & 
1.05\\SCA8-8 & 1095.67 & 37.78 & 
1096.11 & 33.38 & \bf{1071.18} & 
2.29\\SCA8-9 & 1070.71 & 51.93 & 
1078.14 & 43.81 & \bf{1060.50} & 
0.96\\CON3-0 & 619.09 & 8.78 & 
624.49 & 11.49 & \bf{616.52} & 
0.42\\CON3-1 & 560.75 & 15.51 & 
560.75 & 17.04 & \bf{554.47} & 
1.13\\CON3-2 & 521.38 & 8.93 & 
521.38 & 14.44 & \bf{519.26} & 
0.41\\CON3-3 & \bf{591.19} & 12.48 & 
591.19 & 12.41 & 591.19 & 0.00\\
CON3-4 & 591.43 & 11.12 & 
591.43 & 11.67 & \bf{589.32} & 
0.36\\CON3-5 & 564.89 & 15.53 & 
564.89 & 11.09 & \bf{563.70} & 
0.21\\CON3-6 & 502.16 & 9.90 & 
502.16 & 10.80 & \bf{500.80} & 
0.27\\CON3-7 & 578.79 & 16.79 & 
581.76 & 14.40 & \bf{576.48} & 
0.40\\CON3-8 & 523.14 & 14.04 & 
523.93 & 13.41 & \bf{523.05} & 
0.02\\CON3-9 & 582.79 & 8.06 & 
586.99 & 11.55 & \bf{580.05} & 
0.47\\CON8-0 & 864.56 & 25.87 & 
866.91 & 36.63 & \bf{857.17} & 
0.86\\CON8-1 & 742.47 & 45.29 & 
749.22 & 43.07 & \bf{740.85} & 
0.22\\CON8-2 & 714.06 & 38.85 & 
721.48 & 36.14 & \bf{713.44} & 
0.09\\CON8-3 & 821.24 & 55.57 & 
823.86 & 43.55 & \bf{811.07} & 
1.25\\CON8-4 & \bf{772.25} & 28.63 & 
780.77 & 31.62 & 772.25 & 0.00\\
CON8-5 & 761.27 & 44.83 & 
764.37 & 46.01 & \bf{756.91} & 
0.58\\CON8-6 & 698.87 & 34.82 & 
702.97 & 33.63 & \bf{678.92} & 
2.94\\CON8-7 & 815.43 & 38.67 & 
817.13 & 48.87 & \bf{811.96} & 
0.43\\CON8-8 & 784.98 & 20.23 & 
785.84 & 27.33 & \bf{767.53} & 
2.27\\CON8-9 & 812.35 & 29.61 & 
817.00 & 27.62 & \bf{809.00} & 
0.41\\\bf{PROM.} & 
\bf{764.87} & \bf{25.82} & \bf{767.56} & \bf{25.57} & \bf{758.78} & \bf{0.72}\\[1ex]\hline
\end{tabular}
\label{table:nonlin}
\end{table} \clearpage
\begin{table}[ht]
\caption{Resultados de la ejecución de la metaheurística SCA, utilizando instancias de Dethloff con la configuración -n 50.0 -b 10 -y 0.1}
\centering
\small
\begin{tabular}{c c c c c c c}
\hline\hline
Instancia & Costo mínimo & Tiempo(seg.) & Costo promedio & Tiempo promedio(seg.) & Costo SCA & \%Gap \\ [0.5ex]
\hline
SCA3-0 & 640.55 & 14.32 & 
640.55 & 14.48 & \bf{636.06} & 
0.71\\SCA3-1 & \bf{697.84} & 10.07 & 
698.76 & 11.62 & 697.84 & 0.00\\
SCA3-2 & 661.13 & 13.29 & 
664.82 & 14.20 & \bf{659.34} & 
0.27\\SCA3-3 & 680.60 & 9.86 & 
680.96 & 13.81 & \bf{680.04} & 
0.08\\SCA3-4 & \bf{690.50} & 19.16 & 
691.53 & 10.88 & 690.50 & 0.00\\
SCA3-5 & 670.10 & 12.90 & 
673.49 & 7.99 & \bf{659.90} & 
1.55\\SCA3-6 & 652.94 & 10.56 & 
652.94 & 13.09 & \bf{651.09} & 
0.28\\SCA3-7 & 671.21 & 17.33 & 
671.55 & 12.24 & \bf{659.17} & 
1.83\\SCA3-8 & \bf{719.47} & 10.33 & 
720.12 & 11.04 & 719.47 & 0.00\\
SCA3-9 & \bf{681.00} & 12.77 & 
682.03 & 12.71 & 681.00 & 0.00\\
SCA8-0 & 973.22 & 51.53 & 
989.72 & 39.96 & \bf{961.50} & 
1.22\\SCA8-1 & 1062.88 & 49.42 & 
1069.50 & 34.15 & \bf{1050.20} & 
1.21\\SCA8-2 & 1051.42 & 40.26 & 
1052.40 & 44.93 & \bf{1039.64} & 
1.13\\SCA8-3 & 1021.31 & 22.58 & 
1023.88 & 30.54 & \bf{983.34} & 
3.86\\SCA8-4 & \bf{1065.49} & 38.28 & 
1077.07 & 41.09 & 1065.49 & 0.00\\
SCA8-5 & 1040.18 & 42.54 & 
1050.17 & 36.56 & \bf{1027.08} & 
1.28\\SCA8-6 & 972.48 & 49.21 & 
978.82 & 58.23 & \bf{971.82} & 
0.07\\SCA8-7 & 1064.06 & 30.38 & 
1065.59 & 31.41 & \bf{1052.17} & 
1.13\\SCA8-8 & \bf{1071.18} & 42.21 & 
1087.02 & 40.00 & 1071.18 & 0.00\\
SCA8-9 & 1072.10 & 54.28 & 
1078.92 & 51.05 & \bf{1060.50} & 
1.09\\CON3-0 & 617.59 & 4.56 & 
620.33 & 7.83 & \bf{616.52} & 
0.17\\CON3-1 & 560.75 & 11.40 & 
560.75 & 10.12 & \bf{554.47} & 
1.13\\CON3-2 & 521.38 & 27.03 & 
521.38 & 15.75 & \bf{519.26} & 
0.41\\CON3-3 & \bf{591.19} & 10.80 & 
591.20 & 12.13 & 591.19 & 0.00\\
CON3-4 & \bf{\underline{588.79}} & 10.38 & 
589.45 & 12.44 & 589.32 & 
\bf{-0.09}\\CON3-5 & 564.88 & 5.69 & 
564.88 & 12.30 & \bf{563.70} & 
0.21\\CON3-6 & \bf{\underline{499.05}} & 7.90 & 
501.38 & 9.72 & 500.80 & 
\bf{-0.35}\\CON3-7 & 578.41 & 9.32 & 
582.49 & 9.90 & \bf{576.48} & 
0.33\\CON3-8 & \bf{523.05} & 6.99 & 
523.82 & 10.22 & 523.05 & 0.00\\
CON3-9 & 588.40 & 8.90 & 
588.40 & 7.83 & \bf{580.05} & 
1.44\\CON8-0 & 879.90 & 18.18 & 
885.24 & 26.66 & \bf{857.17} & 
2.65\\CON8-1 & \bf{740.85} & 29.25 & 
750.50 & 40.45 & 740.85 & 0.00\\
CON8-2 & 714.44 & 35.92 & 
716.98 & 42.59 & \bf{713.44} & 
0.14\\CON8-3 & 823.06 & 40.54 & 
829.67 & 41.35 & \bf{811.07} & 
1.48\\CON8-4 & 780.03 & 37.72 & 
786.98 & 29.98 & \bf{772.25} & 
1.01\\CON8-5 & 760.41 & 30.13 & 
762.49 & 31.32 & \bf{756.91} & 
0.46\\CON8-6 & 689.23 & 39.01 & 
692.66 & 34.26 & \bf{678.92} & 
1.52\\CON8-7 & 814.79 & 79.99 & 
817.82 & 49.43 & \bf{811.96} & 
0.35\\CON8-8 & 779.96 & 43.22 & 
786.02 & 45.56 & \bf{767.53} & 
1.62\\CON8-9 & 815.09 & 38.91 & 
823.54 & 45.53 & \bf{809.00} & 
0.75\\\bf{PROM.} & 
\bf{764.77} & \bf{26.18} & \bf{768.65} & \bf{25.63} & \bf{758.78} & \bf{0.72}\\[1ex]\hline
\end{tabular}
\label{table:nonlin}
\end{table} \clearpage
\begin{table}[ht]
\caption{Resultados de la ejecución de la metaheurística SCA, utilizando instancias de Dethloff con la configuración -n 50.0 -b 10 -y .2}
\centering
\small
\begin{tabular}{c c c c c c c}
\hline\hline
Instancia & Costo mínimo & Tiempo(seg.) & Costo promedio & Tiempo promedio(seg.) & Costo SCA & \%Gap \\ [0.5ex]
\hline
SCA3-0 & 640.55 & 15.38 & 
640.55 & 13.54 & \bf{636.06} & 
0.71\\SCA3-1 & 700.50 & 12.39 & 
700.76 & 11.93 & \bf{697.84} & 
0.38\\SCA3-2 & 661.13 & 18.58 & 
666.28 & 15.76 & \bf{659.34} & 
0.27\\SCA3-3 & 680.60 & 11.91 & 
680.88 & 11.16 & \bf{680.04} & 
0.08\\SCA3-4 & \bf{690.50} & 15.01 & 
691.02 & 14.34 & 690.50 & 0.00\\
SCA3-5 & 670.10 & 11.84 & 
673.53 & 11.26 & \bf{659.90} & 
1.55\\SCA3-6 & \bf{651.09} & 11.75 & 
652.26 & 12.92 & 651.09 & 0.00\\
SCA3-7 & 666.60 & 17.59 & 
669.96 & 12.87 & \bf{659.17} & 
1.13\\SCA3-8 & \bf{719.47} & 9.50 & 
719.62 & 11.38 & 719.47 & 0.00\\
SCA3-9 & \bf{681.00} & 12.26 & 
682.21 & 14.49 & 681.00 & 0.00\\
SCA8-0 & 984.75 & 40.68 & 
996.60 & 32.06 & \bf{961.50} & 
2.42\\SCA8-1 & 1058.43 & 32.95 & 
1071.32 & 42.10 & \bf{1050.20} & 
0.78\\SCA8-2 & 1051.55 & 35.46 & 
1053.22 & 36.61 & \bf{1039.64} & 
1.15\\SCA8-3 & 1015.27 & 51.16 & 
1021.73 & 35.41 & \bf{983.34} & 
3.25\\SCA8-4 & 1071.64 & 33.47 & 
1078.29 & 36.90 & \bf{1065.49} & 
0.58\\SCA8-5 & 1043.05 & 33.45 & 
1050.81 & 46.76 & \bf{1027.08} & 
1.55\\SCA8-6 & 972.48 & 36.74 & 
977.35 & 48.68 & \bf{971.82} & 
0.07\\SCA8-7 & 1064.06 & 27.22 & 
1066.54 & 39.40 & \bf{1052.17} & 
1.13\\SCA8-8 & 1089.28 & 27.68 & 
1094.48 & 42.44 & \bf{1071.18} & 
1.69\\SCA8-9 & 1067.42 & 44.02 & 
1074.45 & 41.91 & \bf{1060.50} & 
0.65\\CON3-0 & 619.09 & 13.28 & 
621.88 & 8.38 & \bf{616.52} & 
0.42\\CON3-1 & 560.75 & 9.20 & 
560.75 & 12.86 & \bf{554.47} & 
1.13\\CON3-2 & 521.38 & 15.30 & 
523.24 & 11.83 & \bf{519.26} & 
0.41\\CON3-3 & 591.20 & 11.02 & 
591.20 & 13.51 & \bf{591.19} & 
0.00\\CON3-4 & 591.43 & 10.85 & 
591.72 & 12.98 & \bf{589.32} & 
0.36\\CON3-5 & 564.88 & 11.74 & 
567.15 & 11.04 & \bf{563.70} & 
0.21\\CON3-6 & 502.16 & 10.94 & 
503.02 & 13.45 & \bf{500.80} & 
0.27\\CON3-7 & 578.79 & 14.42 & 
584.21 & 11.84 & \bf{576.48} & 
0.40\\CON3-8 & \bf{523.05} & 8.84 & 
523.49 & 12.88 & 523.05 & 0.00\\
CON3-9 & 588.40 & 10.37 & 
588.40 & 8.77 & \bf{580.05} & 
1.44\\CON8-0 & 860.72 & 25.30 & 
871.95 & 28.05 & \bf{857.17} & 
0.41\\CON8-1 & 741.70 & 27.16 & 
747.79 & 35.99 & \bf{740.85} & 
0.11\\CON8-2 & 713.60 & 35.42 & 
716.01 & 36.75 & \bf{713.44} & 
0.02\\CON8-3 & 824.69 & 37.27 & 
831.27 & 36.19 & \bf{811.07} & 
1.68\\CON8-4 & 781.90 & 58.77 & 
785.96 & 40.17 & \bf{772.25} & 
1.25\\CON8-5 & \bf{\underline{754.95}} & 31.45 & 
761.30 & 43.12 & 756.91 & 
\bf{-0.26}\\CON8-6 & 691.74 & 40.72 & 
696.12 & 42.32 & \bf{678.92} & 
1.89\\CON8-7 & 815.43 & 54.91 & 
815.59 & 45.87 & \bf{811.96} & 
0.43\\CON8-8 & 784.16 & 37.49 & 
786.43 & 33.55 & \bf{767.53} & 
2.17\\CON8-9 & 818.66 & 22.65 & 
824.19 & 34.31 & \bf{809.00} & 
1.19\\\bf{PROM.} & 
\bf{765.20} & \bf{24.65} & \bf{768.84} & \bf{25.64} & \bf{758.78} & \bf{0.77}\\[1ex]\hline
\end{tabular}
\label{table:nonlin}
\end{table} \clearpage
\begin{table}[ht]
\caption{Resultados de la ejecución de la metaheurística SCA, utilizando instancias de Dethloff con la configuración -n 50.0 -b 10 -y .3}
\centering
\small
\begin{tabular}{c c c c c c c}
\hline\hline
Instancia & Costo mínimo & Tiempo(seg.) & Costo promedio & Tiempo promedio(seg.) & Costo SCA & \%Gap \\ [0.5ex]
\hline
SCA3-0 & \bf{636.06} & 10.13 & 
639.43 & 17.09 & 636.06 & 0.00\\
SCA3-1 & \bf{697.84} & 8.75 & 
698.76 & 9.08 & 697.84 & 0.00\\
SCA3-2 & \bf{659.34} & 11.91 & 
665.53 & 19.70 & 659.34 & 0.00\\
SCA3-3 & \bf{680.04} & 17.78 & 
680.78 & 14.15 & 680.04 & 0.00\\
SCA3-4 & \bf{690.50} & 24.38 & 
692.05 & 16.15 & 690.50 & 0.00\\
SCA3-5 & 670.10 & 13.45 & 
673.03 & 12.38 & \bf{659.90} & 
1.55\\SCA3-6 & 652.47 & 24.99 & 
653.01 & 17.97 & \bf{651.09} & 
0.21\\SCA3-7 & 667.24 & 13.49 & 
668.35 & 13.68 & \bf{659.17} & 
1.22\\SCA3-8 & \bf{719.47} & 10.94 & 
719.47 & 13.14 & 719.47 & 0.00\\
SCA3-9 & \bf{681.00} & 16.52 & 
681.00 & 17.43 & 681.00 & 0.00\\
SCA8-0 & 983.13 & 30.99 & 
994.50 & 36.81 & \bf{961.50} & 
2.25\\SCA8-1 & 1062.08 & 27.71 & 
1064.86 & 40.84 & \bf{1050.20} & 
1.13\\SCA8-2 & 1047.63 & 39.31 & 
1050.45 & 38.09 & \bf{1039.64} & 
0.77\\SCA8-3 & 1013.77 & 34.51 & 
1020.12 & 31.38 & \bf{983.34} & 
3.09\\SCA8-4 & 1071.86 & 33.04 & 
1082.69 & 31.52 & \bf{1065.49} & 
0.60\\SCA8-5 & 1045.89 & 105.03 & 
1049.26 & 58.78 & \bf{1027.08} & 
1.83\\SCA8-6 & \bf{971.82} & 35.83 & 
973.91 & 38.55 & 971.82 & 0.00\\
SCA8-7 & 1063.22 & 32.80 & 
1068.31 & 40.88 & \bf{1052.17} & 
1.05\\SCA8-8 & 1084.41 & 49.40 & 
1088.62 & 40.93 & \bf{1071.18} & 
1.24\\SCA8-9 & 1073.62 & 35.13 & 
1077.70 & 39.83 & \bf{1060.50} & 
1.24\\CON3-0 & 620.76 & 10.31 & 
627.11 & 11.78 & \bf{616.52} & 
0.69\\CON3-1 & 560.75 & 19.47 & 
560.75 & 15.87 & \bf{554.47} & 
1.13\\CON3-2 & 521.38 & 15.50 & 
521.38 & 10.34 & \bf{519.26} & 
0.41\\CON3-3 & \bf{591.19} & 15.24 & 
591.20 & 16.20 & 591.19 & 0.00\\
CON3-4 & 591.43 & 10.47 & 
591.43 & 13.98 & \bf{589.32} & 
0.36\\CON3-5 & \bf{563.70} & 8.51 & 
564.00 & 14.87 & 563.70 & 0.00\\
CON3-6 & 502.16 & 7.96 & 
502.16 & 9.07 & \bf{500.80} & 
0.27\\CON3-7 & 580.40 & 8.00 & 
582.77 & 9.84 & \bf{576.48} & 
0.68\\CON3-8 & \bf{523.05} & 9.67 & 
523.07 & 9.57 & 523.05 & 0.00\\
CON3-9 & 583.86 & 7.92 & 
587.26 & 10.00 & \bf{580.05} & 
0.66\\CON8-0 & 876.12 & 31.48 & 
881.12 & 41.49 & \bf{857.17} & 
2.21\\CON8-1 & 741.70 & 41.21 & 
747.20 & 43.35 & \bf{740.85} & 
0.11\\CON8-2 & \bf{\underline{713.10}} & 35.53 & 
716.73 & 42.20 & 713.44 & 
\bf{-0.05}\\CON8-3 & 812.70 & 60.29 & 
826.86 & 59.08 & \bf{811.07} & 
0.20\\CON8-4 & 786.91 & 43.47 & 
788.21 & 40.24 & \bf{772.25} & 
1.90\\CON8-5 & 758.12 & 47.89 & 
760.97 & 44.09 & \bf{756.91} & 
0.16\\CON8-6 & 693.87 & 41.11 & 
698.46 & 37.55 & \bf{678.92} & 
2.20\\CON8-7 & 816.07 & 33.01 & 
818.25 & 44.49 & \bf{811.96} & 
0.51\\CON8-8 & 782.09 & 20.92 & 
786.07 & 33.62 & \bf{767.53} & 
1.90\\CON8-9 & 819.64 & 26.96 & 
823.87 & 39.46 & \bf{809.00} & 
1.32\\\bf{PROM.} & 
\bf{765.26} & \bf{26.78} & \bf{768.52} & \bf{27.39} & \bf{758.78} & \bf{0.77}\\[1ex]\hline
\end{tabular}
\label{table:nonlin}
\end{table} \clearpage
\begin{table}[ht]
\caption{Resultados de la ejecución de la metaheurística SCA, utilizando instancias de Dethloff con la configuración -n 50.0 -b 10 -y .4}
\centering
\small
\begin{tabular}{c c c c c c c}
\hline\hline
Instancia & Costo mínimo & Tiempo(seg.) & Costo promedio & Tiempo promedio(seg.) & Costo SCA & \%Gap \\ [0.5ex]
\hline
SCA3-0 & 640.55 & 22.46 & 
640.55 & 16.31 & \bf{636.06} & 
0.71\\SCA3-1 & \bf{697.84} & 28.08 & 
698.76 & 16.91 & 697.84 & 0.00\\
SCA3-2 & \bf{659.34} & 18.07 & 
660.24 & 20.24 & 659.34 & 0.00\\
SCA3-3 & \bf{680.04} & 16.06 & 
680.64 & 10.91 & 680.04 & 0.00\\
SCA3-4 & \bf{690.50} & 16.22 & 
691.02 & 14.87 & 690.50 & 0.00\\
SCA3-5 & 661.07 & 9.26 & 
668.05 & 9.16 & \bf{659.90} & 
0.18\\SCA3-6 & 652.94 & 17.61 & 
653.60 & 18.14 & \bf{651.09} & 
0.28\\SCA3-7 & 666.60 & 11.16 & 
669.96 & 12.58 & \bf{659.17} & 
1.13\\SCA3-8 & \bf{719.47} & 8.84 & 
719.47 & 16.20 & 719.47 & 0.00\\
SCA3-9 & \bf{681.00} & 16.98 & 
684.33 & 16.83 & 681.00 & 0.00\\
SCA8-0 & 976.79 & 32.01 & 
992.08 & 38.64 & \bf{961.50} & 
1.59\\SCA8-1 & 1060.62 & 49.42 & 
1076.76 & 32.15 & \bf{1050.20} & 
0.99\\SCA8-2 & 1050.37 & 49.72 & 
1051.22 & 45.08 & \bf{1039.64} & 
1.03\\SCA8-3 & 1024.37 & 62.38 & 
1028.48 & 44.51 & \bf{983.34} & 
4.17\\SCA8-4 & 1074.63 & 34.14 & 
1082.83 & 33.63 & \bf{1065.49} & 
0.86\\SCA8-5 & 1030.08 & 40.04 & 
1041.43 & 44.72 & \bf{1027.08} & 
0.29\\SCA8-6 & 972.48 & 44.25 & 
978.68 & 50.09 & \bf{971.82} & 
0.07\\SCA8-7 & 1066.65 & 49.32 & 
1069.76 & 39.48 & \bf{1052.17} & 
1.38\\SCA8-8 & 1085.98 & 49.50 & 
1093.79 & 45.08 & \bf{1071.18} & 
1.38\\SCA8-9 & 1069.70 & 48.87 & 
1073.41 & 36.54 & \bf{1060.50} & 
0.87\\CON3-0 & 622.11 & 11.05 & 
630.45 & 9.01 & \bf{616.52} & 
0.91\\CON3-1 & 560.75 & 12.72 & 
561.38 & 9.66 & \bf{554.47} & 
1.13\\CON3-2 & 521.38 & 11.70 & 
523.06 & 11.71 & \bf{519.26} & 
0.41\\CON3-3 & 591.20 & 7.44 & 
591.20 & 9.08 & \bf{591.19} & 
0.00\\CON3-4 & 591.43 & 13.39 & 
591.43 & 11.19 & \bf{589.32} & 
0.36\\CON3-5 & \bf{563.70} & 8.33 & 
564.00 & 8.16 & 563.70 & 0.00\\
CON3-6 & 502.16 & 8.35 & 
502.65 & 10.32 & \bf{500.80} & 
0.27\\CON3-7 & 578.22 & 15.48 & 
583.86 & 12.43 & \bf{576.48} & 
0.30\\CON3-8 & \bf{523.05} & 6.95 & 
523.24 & 12.60 & 523.05 & 0.00\\
CON3-9 & \bf{\underline{578.25}} & 16.94 & 
583.33 & 13.68 & 580.05 & 
\bf{-0.31}\\CON8-0 & 874.49 & 25.98 & 
877.32 & 29.24 & \bf{857.17} & 
2.02\\CON8-1 & 742.47 & 43.24 & 
749.62 & 44.61 & \bf{740.85} & 
0.22\\CON8-2 & \bf{713.44} & 62.54 & 
715.01 & 53.02 & 713.44 & 0.00\\
CON8-3 & 818.51 & 50.38 & 
830.23 & 37.95 & \bf{811.07} & 
0.92\\CON8-4 & 772.32 & 26.54 & 
784.52 & 44.62 & \bf{772.25} & 
0.01\\CON8-5 & \bf{\underline{755.14}} & 37.04 & 
759.82 & 37.90 & 756.91 & 
\bf{-0.23}\\CON8-6 & 684.05 & 25.99 & 
693.77 & 46.38 & \bf{678.92} & 
0.76\\CON8-7 & 814.79 & 50.27 & 
815.55 & 43.72 & \bf{811.96} & 
0.35\\CON8-8 & 782.68 & 29.63 & 
788.28 & 33.05 & \bf{767.53} & 
1.97\\CON8-9 & 813.68 & 25.08 & 
819.28 & 31.09 & \bf{809.00} & 
0.58\\\bf{PROM.} & 
\bf{764.12} & \bf{27.84} & \bf{768.58} & \bf{26.79} & \bf{758.78} & \bf{0.61}\\[1ex]\hline
\end{tabular}
\label{table:nonlin}
\end{table} \clearpage
\begin{table}[ht]
\caption{Resultados de la ejecución de la metaheurística SCA, utilizando instancias de Dethloff con la configuración -n 50.0 -b 10 -y .5}
\centering
\small
\begin{tabular}{c c c c c c c}
\hline\hline
Instancia & Costo mínimo & Tiempo(seg.) & Costo promedio & Tiempo promedio(seg.) & Costo SCA & \%Gap \\ [0.5ex]
\hline
SCA3-0 & 640.55 & 16.21 & 
640.55 & 11.96 & \bf{636.06} & 
0.71\\SCA3-1 & \bf{697.84} & 13.43 & 
697.84 & 12.44 & 697.84 & 0.00\\
SCA3-2 & 661.13 & 17.11 & 
663.89 & 17.92 & \bf{659.34} & 
0.27\\SCA3-3 & \bf{680.04} & 12.03 & 
680.50 & 11.79 & 680.04 & 0.00\\
SCA3-4 & \bf{690.50} & 11.38 & 
691.02 & 12.47 & 690.50 & 0.00\\
SCA3-5 & \bf{659.90} & 10.23 & 
665.69 & 9.60 & 659.90 & 0.00\\
SCA3-6 & 652.94 & 23.12 & 
655.07 & 17.83 & \bf{651.09} & 
0.28\\SCA3-7 & 666.60 & 14.70 & 
669.29 & 13.96 & \bf{659.17} & 
1.13\\SCA3-8 & \bf{719.47} & 20.07 & 
719.47 & 16.21 & 719.47 & 0.00\\
SCA3-9 & \bf{681.00} & 10.34 & 
682.03 & 14.86 & 681.00 & 0.00\\
SCA8-0 & 973.22 & 56.93 & 
991.37 & 38.71 & \bf{961.50} & 
1.22\\SCA8-1 & 1050.38 & 32.53 & 
1058.97 & 47.84 & \bf{1050.20} & 
0.02\\SCA8-2 & 1051.21 & 35.72 & 
1052.27 & 42.75 & \bf{1039.64} & 
1.11\\SCA8-3 & 1008.89 & 34.95 & 
1021.97 & 34.89 & \bf{983.34} & 
2.60\\SCA8-4 & 1073.78 & 35.44 & 
1082.86 & 32.34 & \bf{1065.49} & 
0.78\\SCA8-5 & 1050.06 & 36.43 & 
1051.84 & 43.11 & \bf{1027.08} & 
2.24\\SCA8-6 & 972.48 & 36.51 & 
978.96 & 32.15 & \bf{971.82} & 
0.07\\SCA8-7 & 1063.60 & 45.51 & 
1067.59 & 41.28 & \bf{1052.17} & 
1.09\\SCA8-8 & 1082.12 & 28.12 & 
1085.84 & 33.27 & \bf{1071.18} & 
1.02\\SCA8-9 & 1072.10 & 54.41 & 
1085.23 & 42.74 & \bf{1060.50} & 
1.09\\CON3-0 & 621.22 & 9.54 & 
629.59 & 9.03 & \bf{616.52} & 
0.76\\CON3-1 & 560.75 & 8.46 & 
560.75 & 11.43 & \bf{554.47} & 
1.13\\CON3-2 & 521.38 & 6.90 & 
521.38 & 8.26 & \bf{519.26} & 
0.41\\CON3-3 & 591.20 & 12.90 & 
591.20 & 12.26 & \bf{591.19} & 
0.00\\CON3-4 & \bf{\underline{588.79}} & 11.06 & 
590.77 & 15.59 & 589.32 & 
\bf{-0.09}\\CON3-5 & \bf{563.70} & 9.70 & 
564.59 & 9.06 & 563.70 & 0.00\\
CON3-6 & 502.16 & 8.69 & 
502.65 & 12.96 & \bf{500.80} & 
0.27\\CON3-7 & 578.41 & 14.26 & 
584.11 & 14.33 & \bf{576.48} & 
0.33\\CON3-8 & \bf{523.05} & 10.98 & 
525.05 & 9.56 & 523.05 & 0.00\\
CON3-9 & 587.23 & 8.39 & 
587.96 & 10.08 & \bf{580.05} & 
1.24\\CON8-0 & 858.63 & 35.51 & 
872.16 & 32.77 & \bf{857.17} & 
0.17\\CON8-1 & 741.70 & 52.39 & 
746.01 & 42.99 & \bf{740.85} & 
0.11\\CON8-2 & \bf{\underline{713.05}} & 38.96 & 
715.14 & 45.13 & 713.44 & 
\bf{-0.05}\\CON8-3 & 817.44 & 31.09 & 
829.03 & 43.44 & \bf{811.07} & 
0.79\\CON8-4 & 780.10 & 55.99 & 
788.96 & 44.69 & \bf{772.25} & 
1.02\\CON8-5 & \bf{\underline{755.86}} & 67.58 & 
759.84 & 52.42 & 756.91 & 
\bf{-0.14}\\CON8-6 & 687.70 & 63.07 & 
693.41 & 38.40 & \bf{678.92} & 
1.29\\CON8-7 & 815.43 & 37.80 & 
818.25 & 44.81 & \bf{811.96} & 
0.43\\CON8-8 & 782.06 & 50.52 & 
784.77 & 33.15 & \bf{767.53} & 
1.89\\CON8-9 & 817.71 & 57.98 & 
821.70 & 41.04 & \bf{809.00} & 
1.08\\\bf{PROM.} & 
\bf{763.88} & \bf{28.42} & \bf{768.24} & \bf{26.49} & \bf{758.78} & \bf{0.61}\\[1ex]\hline
\end{tabular}
\label{table:nonlin}
\end{table} \clearpage
\begin{table}[ht]
\caption{Resultados de la ejecución de la metaheurística SCA, utilizando instancias de Dethloff con la configuración -n 75.0 -b 10 -y 0.1}
\centering
\small
\begin{tabular}{c c c c c c c}
\hline\hline
Instancia & Costo mínimo & Tiempo(seg.) & Costo promedio & Tiempo promedio(seg.) & Costo SCA & \%Gap \\ [0.5ex]
\hline
SCA3-0 & 640.55 & 11.25 & 
640.55 & 14.37 & \bf{636.06} & 
0.71\\SCA3-1 & 700.50 & 10.39 & 
701.27 & 9.14 & \bf{697.84} & 
0.38\\SCA3-2 & 664.21 & 8.43 & 
664.21 & 8.71 & \bf{659.34} & 
0.74\\SCA3-3 & \bf{680.04} & 11.74 & 
680.61 & 17.29 & 680.04 & 0.00\\
SCA3-4 & \bf{690.50} & 15.71 & 
691.53 & 13.61 & 690.50 & 0.00\\
SCA3-5 & 672.49 & 13.31 & 
675.13 & 10.05 & \bf{659.90} & 
1.91\\SCA3-6 & 652.94 & 21.50 & 
654.38 & 16.67 & \bf{651.09} & 
0.28\\SCA3-7 & 671.67 & 10.55 & 
671.67 & 10.03 & \bf{659.17} & 
1.90\\SCA3-8 & \bf{719.47} & 8.96 & 
719.70 & 10.21 & 719.47 & 0.00\\
SCA3-9 & \bf{681.00} & 9.68 & 
681.17 & 9.36 & 681.00 & 0.00\\
SCA8-0 & 983.04 & 31.91 & 
992.63 & 34.07 & \bf{961.50} & 
2.24\\SCA8-1 & 1072.18 & 29.76 & 
1078.59 & 29.75 & \bf{1050.20} & 
2.09\\SCA8-2 & 1050.37 & 42.05 & 
1052.32 & 35.80 & \bf{1039.64} & 
1.03\\SCA8-3 & 1010.01 & 32.39 & 
1013.11 & 28.80 & \bf{983.34} & 
2.71\\SCA8-4 & 1067.29 & 50.67 & 
1073.31 & 37.88 & \bf{1065.49} & 
0.17\\SCA8-5 & 1050.64 & 26.01 & 
1058.75 & 26.52 & \bf{1027.08} & 
2.29\\SCA8-6 & 972.48 & 57.22 & 
978.54 & 45.55 & \bf{971.82} & 
0.07\\SCA8-7 & 1070.92 & 39.91 & 
1073.18 & 42.98 & \bf{1052.17} & 
1.78\\SCA8-8 & 1082.12 & 31.66 & 
1083.16 & 34.62 & \bf{1071.18} & 
1.02\\SCA8-9 & 1078.80 & 58.62 & 
1087.30 & 47.25 & \bf{1060.50} & 
1.73\\CON3-0 & 620.49 & 11.28 & 
624.10 & 7.47 & \bf{616.52} & 
0.64\\CON3-1 & 560.13 & 6.72 & 
560.28 & 11.85 & \bf{554.47} & 
1.02\\CON3-2 & 521.38 & 12.74 & 
521.38 & 13.07 & \bf{519.26} & 
0.41\\CON3-3 & \bf{591.19} & 8.66 & 
591.20 & 10.30 & 591.19 & 0.00\\
CON3-4 & \bf{\underline{588.79}} & 17.63 & 
590.77 & 13.71 & 589.32 & 
\bf{-0.09}\\CON3-5 & 564.88 & 13.03 & 
564.99 & 10.49 & \bf{563.70} & 
0.21\\CON3-6 & 502.16 & 15.85 & 
502.16 & 13.32 & \bf{500.80} & 
0.27\\CON3-7 & 578.79 & 16.59 & 
583.91 & 13.41 & \bf{576.48} & 
0.40\\CON3-8 & 523.19 & 17.56 & 
524.27 & 15.06 & \bf{523.05} & 
0.03\\CON3-9 & 588.28 & 12.61 & 
589.35 & 11.63 & \bf{580.05} & 
1.42\\CON8-0 & 871.92 & 23.62 & 
880.47 & 28.18 & \bf{857.17} & 
1.72\\CON8-1 & 742.28 & 82.98 & 
747.81 & 50.10 & \bf{740.85} & 
0.19\\CON8-2 & \bf{\underline{713.05}} & 34.82 & 
715.27 & 35.33 & 713.44 & 
\bf{-0.05}\\CON8-3 & 824.71 & 33.47 & 
833.11 & 44.65 & \bf{811.07} & 
1.68\\CON8-4 & 773.27 & 28.24 & 
778.08 & 41.43 & \bf{772.25} & 
0.13\\CON8-5 & 759.87 & 71.44 & 
762.09 & 53.11 & \bf{756.91} & 
0.39\\CON8-6 & 688.77 & 34.08 & 
695.24 & 38.44 & \bf{678.92} & 
1.45\\CON8-7 & 814.79 & 38.38 & 
819.58 & 37.88 & \bf{811.96} & 
0.35\\CON8-8 & 779.43 & 28.57 & 
782.63 & 30.08 & \bf{767.53} & 
1.55\\CON8-9 & 817.79 & 24.48 & 
819.89 & 38.25 & \bf{809.00} & 
1.09\\\bf{PROM.} & 
\bf{765.91} & \bf{26.36} & \bf{768.94} & \bf{25.01} & \bf{758.78} & \bf{0.85}\\[1ex]\hline
\end{tabular}
\label{table:nonlin}
\end{table} \clearpage
\begin{table}[ht]
\caption{Resultados de la ejecución de la metaheurística SCA, utilizando instancias de Dethloff con la configuración -n 75.0 -b 10 -y .2}
\centering
\small
\begin{tabular}{c c c c c c c}
\hline\hline
Instancia & Costo mínimo & Tiempo(seg.) & Costo promedio & Tiempo promedio(seg.) & Costo SCA & \%Gap \\ [0.5ex]
\hline
SCA3-0 & 640.55 & 11.71 & 
640.55 & 13.82 & \bf{636.06} & 
0.71\\SCA3-1 & \bf{697.84} & 14.59 & 
697.84 & 11.69 & 697.84 & 0.00\\
SCA3-2 & 661.13 & 12.46 & 
663.11 & 16.04 & \bf{659.34} & 
0.27\\SCA3-3 & 680.60 & 12.08 & 
680.88 & 12.22 & \bf{680.04} & 
0.08\\SCA3-4 & 692.57 & 9.73 & 
693.07 & 9.98 & \bf{690.50} & 
0.30\\SCA3-5 & 665.64 & 13.28 & 
674.38 & 10.22 & \bf{659.90} & 
0.87\\SCA3-6 & 652.94 & 7.98 & 
652.94 & 9.71 & \bf{651.09} & 
0.28\\SCA3-7 & 667.34 & 8.34 & 
670.59 & 12.38 & \bf{659.17} & 
1.24\\SCA3-8 & \bf{719.47} & 10.94 & 
719.47 & 10.04 & 719.47 & 0.00\\
SCA3-9 & \bf{681.00} & 18.57 & 
683.08 & 15.18 & 681.00 & 0.00\\
SCA8-0 & 980.51 & 53.60 & 
990.81 & 41.47 & \bf{961.50} & 
1.98\\SCA8-1 & 1064.55 & 41.60 & 
1074.64 & 33.59 & \bf{1050.20} & 
1.37\\SCA8-2 & 1051.21 & 49.63 & 
1053.02 & 42.48 & \bf{1039.64} & 
1.11\\SCA8-3 & 1010.01 & 22.90 & 
1019.29 & 25.56 & \bf{983.34} & 
2.71\\SCA8-4 & 1072.06 & 33.66 & 
1078.79 & 37.24 & \bf{1065.49} & 
0.62\\SCA8-5 & 1049.44 & 36.51 & 
1056.46 & 69.40 & \bf{1027.08} & 
2.18\\SCA8-6 & 972.48 & 34.45 & 
972.48 & 51.11 & \bf{971.82} & 
0.07\\SCA8-7 & 1070.67 & 27.77 & 
1072.52 & 29.13 & \bf{1052.17} & 
1.76\\SCA8-8 & 1075.00 & 56.25 & 
1087.65 & 41.40 & \bf{1071.18} & 
0.36\\SCA8-9 & 1085.18 & 77.98 & 
1087.69 & 47.08 & \bf{1060.50} & 
2.33\\CON3-0 & 620.29 & 7.97 & 
628.48 & 10.17 & \bf{616.52} & 
0.61\\CON3-1 & 560.75 & 17.72 & 
560.75 & 18.33 & \bf{554.47} & 
1.13\\CON3-2 & 521.36 & 11.14 & 
521.38 & 14.15 & \bf{519.26} & 
0.40\\CON3-3 & \bf{591.19} & 7.74 & 
591.19 & 7.47 & 591.19 & 0.00\\
CON3-4 & 591.43 & 9.20 & 
592.02 & 14.65 & \bf{589.32} & 
0.36\\CON3-5 & \bf{563.70} & 10.81 & 
563.70 & 11.06 & 563.70 & 0.00\\
CON3-6 & 502.16 & 10.43 & 
502.31 & 9.46 & \bf{500.80} & 
0.27\\CON3-7 & 578.22 & 11.76 & 
584.06 & 9.79 & \bf{576.48} & 
0.30\\CON3-8 & \bf{523.05} & 8.44 & 
523.61 & 10.36 & 523.05 & 0.00\\
CON3-9 & 588.40 & 10.12 & 
588.40 & 10.28 & \bf{580.05} & 
1.44\\CON8-0 & 871.19 & 23.04 & 
874.23 & 26.51 & \bf{857.17} & 
1.64\\CON8-1 & \bf{740.85} & 40.91 & 
748.19 & 40.20 & 740.85 & 0.00\\
CON8-2 & \bf{713.44} & 60.11 & 
717.76 & 44.61 & 713.44 & 0.00\\
CON8-3 & 814.52 & 40.39 & 
820.00 & 37.47 & \bf{811.07} & 
0.43\\CON8-4 & 783.47 & 45.34 & 
785.87 & 49.26 & \bf{772.25} & 
1.45\\CON8-5 & \bf{\underline{754.95}} & 32.54 & 
761.14 & 36.01 & 756.91 & 
\bf{-0.26}\\CON8-6 & 687.70 & 28.64 & 
696.55 & 33.25 & \bf{678.92} & 
1.29\\CON8-7 & 815.43 & 41.46 & 
818.21 & 45.11 & \bf{811.96} & 
0.43\\CON8-8 & 783.65 & 36.53 & 
789.99 & 29.30 & \bf{767.53} & 
2.10\\CON8-9 & 814.77 & 36.81 & 
821.26 & 28.16 & \bf{809.00} & 
0.71\\\bf{PROM.} & 
\bf{765.27} & \bf{26.13} & \bf{768.96} & \bf{25.63} & \bf{758.78} & \bf{0.76}\\[1ex]\hline
\end{tabular}
\label{table:nonlin}
\end{table} \clearpage
\begin{table}[ht]
\caption{Resultados de la ejecución de la metaheurística SCA, utilizando instancias de Dethloff con la configuración -n 75.0 -b 10 -y .3}
\centering
\small
\begin{tabular}{c c c c c c c}
\hline\hline
Instancia & Costo mínimo & Tiempo(seg.) & Costo promedio & Tiempo promedio(seg.) & Costo SCA & \%Gap \\ [0.5ex]
\hline
SCA3-0 & 640.55 & 12.42 & 
640.55 & 13.77 & \bf{636.06} & 
0.71\\SCA3-1 & \bf{697.84} & 8.77 & 
698.85 & 10.55 & 697.84 & 0.00\\
SCA3-2 & 664.21 & 14.66 & 
664.84 & 18.33 & \bf{659.34} & 
0.74\\SCA3-3 & \bf{680.04} & 10.94 & 
680.47 & 12.13 & 680.04 & 0.00\\
SCA3-4 & \bf{690.50} & 14.60 & 
692.05 & 12.34 & 690.50 & 0.00\\
SCA3-5 & 662.75 & 12.14 & 
662.75 & 14.40 & \bf{659.90} & 
0.43\\SCA3-6 & 652.94 & 10.84 & 
653.16 & 14.76 & \bf{651.09} & 
0.28\\SCA3-7 & 666.15 & 11.18 & 
667.64 & 15.70 & \bf{659.17} & 
1.06\\SCA3-8 & \bf{719.47} & 9.87 & 
720.19 & 10.64 & 719.47 & 0.00\\
SCA3-9 & \bf{681.00} & 15.34 & 
681.06 & 12.97 & 681.00 & 0.00\\
SCA8-0 & 982.18 & 30.32 & 
991.33 & 36.02 & \bf{961.50} & 
2.15\\SCA8-1 & 1056.87 & 24.77 & 
1063.27 & 34.19 & \bf{1050.20} & 
0.64\\SCA8-2 & 1051.95 & 51.33 & 
1053.40 & 45.52 & \bf{1039.64} & 
1.18\\SCA8-3 & 1022.69 & 41.59 & 
1028.02 & 38.19 & \bf{983.34} & 
4.00\\SCA8-4 & 1068.27 & 28.76 & 
1071.05 & 29.79 & \bf{1065.49} & 
0.26\\SCA8-5 & 1050.44 & 36.53 & 
1054.51 & 40.23 & \bf{1027.08} & 
2.27\\SCA8-6 & 972.48 & 45.42 & 
978.57 & 60.34 & \bf{971.82} & 
0.07\\SCA8-7 & 1063.22 & 40.55 & 
1069.19 & 35.28 & \bf{1052.17} & 
1.05\\SCA8-8 & 1088.20 & 45.55 & 
1089.62 & 43.08 & \bf{1071.18} & 
1.59\\SCA8-9 & 1070.71 & 52.17 & 
1083.38 & 34.66 & \bf{1060.50} & 
0.96\\CON3-0 & 619.09 & 4.46 & 
619.09 & 5.87 & \bf{616.52} & 
0.42\\CON3-1 & 560.75 & 11.78 & 
561.19 & 12.30 & \bf{554.47} & 
1.13\\CON3-2 & 521.38 & 22.92 & 
521.38 & 12.36 & \bf{519.26} & 
0.41\\CON3-3 & \bf{591.19} & 12.76 & 
591.20 & 16.69 & 591.19 & 0.00\\
CON3-4 & 591.43 & 15.60 & 
591.43 & 14.57 & \bf{589.32} & 
0.36\\CON3-5 & 564.88 & 11.72 & 
564.88 & 11.25 & \bf{563.70} & 
0.21\\CON3-6 & 502.16 & 8.78 & 
502.16 & 14.63 & \bf{500.80} & 
0.27\\CON3-7 & 578.41 & 15.65 & 
582.06 & 15.22 & \bf{576.48} & 
0.33\\CON3-8 & \bf{523.05} & 13.63 & 
523.67 & 9.22 & 523.05 & 0.00\\
CON3-9 & 581.06 & 15.12 & 
586.83 & 18.84 & \bf{580.05} & 
0.17\\CON8-0 & 873.17 & 20.55 & 
880.85 & 30.33 & \bf{857.17} & 
1.87\\CON8-1 & 741.70 & 23.63 & 
746.65 & 38.21 & \bf{740.85} & 
0.11\\CON8-2 & \bf{713.44} & 43.87 & 
715.53 & 42.67 & 713.44 & 0.00\\
CON8-3 & 824.71 & 30.19 & 
830.92 & 35.50 & \bf{811.07} & 
1.68\\CON8-4 & 778.31 & 52.74 & 
787.34 & 39.36 & \bf{772.25} & 
0.78\\CON8-5 & 758.12 & 32.79 & 
762.43 & 40.66 & \bf{756.91} & 
0.16\\CON8-6 & 693.83 & 42.12 & 
697.82 & 44.56 & \bf{678.92} & 
2.20\\CON8-7 & 815.43 & 31.30 & 
816.37 & 42.66 & \bf{811.96} & 
0.43\\CON8-8 & 775.56 & 56.99 & 
785.34 & 35.65 & \bf{767.53} & 
1.05\\CON8-9 & 819.78 & 27.88 & 
822.67 & 31.37 & \bf{809.00} & 
1.33\\\bf{PROM.} & 
\bf{765.25} & \bf{25.31} & \bf{768.34} & \bf{26.12} & \bf{758.78} & \bf{0.76}\\[1ex]\hline
\end{tabular}
\label{table:nonlin}
\end{table} \clearpage
\begin{table}[ht]
\caption{Resultados de la ejecución de la metaheurística SCA, utilizando instancias de Dethloff con la configuración -n 75.0 -b 10 -y .4}
\centering
\small
\begin{tabular}{c c c c c c c}
\hline\hline
Instancia & Costo mínimo & Tiempo(seg.) & Costo promedio & Tiempo promedio(seg.) & Costo SCA & \%Gap \\ [0.5ex]
\hline
SCA3-0 & 640.55 & 18.56 & 
640.55 & 14.73 & \bf{636.06} & 
0.71\\SCA3-1 & \bf{697.84} & 11.92 & 
699.68 & 12.86 & 697.84 & 0.00\\
SCA3-2 & 661.13 & 13.41 & 
662.08 & 16.29 & \bf{659.34} & 
0.27\\SCA3-3 & \bf{680.04} & 17.22 & 
680.61 & 16.59 & 680.04 & 0.00\\
SCA3-4 & \bf{690.50} & 10.24 & 
690.50 & 11.41 & 690.50 & 0.00\\
SCA3-5 & 665.04 & 9.79 & 
670.50 & 9.04 & \bf{659.90} & 
0.78\\SCA3-6 & \bf{651.09} & 9.19 & 
651.55 & 13.76 & 651.09 & 0.00\\
SCA3-7 & 666.60 & 13.28 & 
668.69 & 12.93 & \bf{659.17} & 
1.13\\SCA3-8 & \bf{719.47} & 13.56 & 
719.47 & 13.81 & 719.47 & 0.00\\
SCA3-9 & \bf{681.00} & 12.36 & 
681.00 & 10.86 & 681.00 & 0.00\\
SCA8-0 & 982.58 & 42.73 & 
989.09 & 42.96 & \bf{961.50} & 
2.19\\SCA8-1 & 1057.18 & 31.28 & 
1060.64 & 36.30 & \bf{1050.20} & 
0.66\\SCA8-2 & 1050.17 & 74.92 & 
1052.88 & 52.51 & \bf{1039.64} & 
1.01\\SCA8-3 & 1014.18 & 32.03 & 
1019.44 & 37.43 & \bf{983.34} & 
3.14\\SCA8-4 & 1067.82 & 35.25 & 
1075.37 & 37.37 & \bf{1065.49} & 
0.22\\SCA8-5 & 1036.97 & 35.87 & 
1048.13 & 37.91 & \bf{1027.08} & 
0.96\\SCA8-6 & 977.87 & 62.27 & 
981.97 & 42.42 & \bf{971.82} & 
0.62\\SCA8-7 & 1063.22 & 30.42 & 
1073.81 & 34.35 & \bf{1052.17} & 
1.05\\SCA8-8 & 1085.22 & 32.87 & 
1091.83 & 32.28 & \bf{1071.18} & 
1.31\\SCA8-9 & 1067.42 & 27.43 & 
1080.05 & 41.54 & \bf{1060.50} & 
0.65\\CON3-0 & 619.09 & 8.30 & 
625.82 & 7.59 & \bf{616.52} & 
0.42\\CON3-1 & 556.79 & 8.99 & 
559.76 & 12.53 & \bf{554.47} & 
0.42\\CON3-2 & 521.38 & 9.57 & 
521.38 & 10.32 & \bf{519.26} & 
0.41\\CON3-3 & \bf{591.19} & 10.05 & 
591.20 & 10.59 & 591.19 & 0.00\\
CON3-4 & 589.88 & 14.60 & 
590.65 & 13.91 & \bf{589.32} & 
0.10\\CON3-5 & \bf{563.70} & 11.61 & 
564.59 & 9.06 & 563.70 & 0.00\\
CON3-6 & 502.16 & 10.04 & 
502.16 & 11.55 & \bf{500.80} & 
0.27\\CON3-7 & 578.22 & 18.66 & 
582.12 & 16.52 & \bf{576.48} & 
0.30\\CON3-8 & \bf{523.05} & 9.86 & 
523.12 & 11.08 & 523.05 & 0.00\\
CON3-9 & 582.79 & 14.06 & 
586.70 & 12.49 & \bf{580.05} & 
0.47\\CON8-0 & 872.20 & 21.89 & 
877.76 & 25.58 & \bf{857.17} & 
1.75\\CON8-1 & 742.47 & 50.56 & 
745.74 & 44.99 & \bf{740.85} & 
0.22\\CON8-2 & 713.60 & 29.84 & 
714.89 & 34.27 & \bf{713.44} & 
0.02\\CON8-3 & 828.01 & 47.69 & 
833.01 & 48.49 & \bf{811.07} & 
2.09\\CON8-4 & 777.64 & 49.64 & 
785.77 & 40.74 & \bf{772.25} & 
0.70\\CON8-5 & \bf{\underline{754.95}} & 41.73 & 
762.13 & 42.55 & 756.91 & 
\bf{-0.26}\\CON8-6 & 684.29 & 29.80 & 
694.24 & 30.34 & \bf{678.92} & 
0.79\\CON8-7 & 814.77 & 45.60 & 
815.20 & 41.87 & \bf{811.96} & 
0.35\\CON8-8 & 779.69 & 26.21 & 
783.59 & 31.50 & \bf{767.53} & 
1.58\\CON8-9 & 813.57 & 28.50 & 
822.41 & 41.56 & \bf{809.00} & 
0.56\\\bf{PROM.} & 
\bf{764.13} & \bf{25.54} & \bf{768.00} & \bf{25.62} & \bf{758.78} & \bf{0.62}\\[1ex]\hline
\end{tabular}
\label{table:nonlin}
\end{table} \clearpage
\begin{table}[ht]
\caption{Resultados de la ejecución de la metaheurística SCA, utilizando instancias de Dethloff con la configuración -n 75.0 -b 10 -y .5}
\centering
\small
\begin{tabular}{c c c c c c c}
\hline\hline
Instancia & Costo mínimo & Tiempo(seg.) & Costo promedio & Tiempo promedio(seg.) & Costo SCA & \%Gap \\ [0.5ex]
\hline
SCA3-0 & 640.55 & 14.05 & 
640.55 & 14.27 & \bf{636.06} & 
0.71\\SCA3-1 & \bf{697.84} & 13.37 & 
698.76 & 13.53 & 697.84 & 0.00\\
SCA3-2 & 661.13 & 13.00 & 
661.90 & 15.36 & \bf{659.34} & 
0.27\\SCA3-3 & 680.60 & 11.42 & 
680.98 & 9.19 & \bf{680.04} & 
0.08\\SCA3-4 & \bf{690.50} & 8.11 & 
691.18 & 15.18 & 690.50 & 0.00\\
SCA3-5 & 665.64 & 9.05 & 
668.92 & 13.54 & \bf{659.90} & 
0.87\\SCA3-6 & 652.94 & 11.66 & 
655.04 & 16.75 & \bf{651.09} & 
0.28\\SCA3-7 & 666.60 & 11.76 & 
670.40 & 10.56 & \bf{659.17} & 
1.13\\SCA3-8 & \bf{719.47} & 17.22 & 
719.47 & 14.33 & 719.47 & 0.00\\
SCA3-9 & 685.00 & 17.96 & 
685.29 & 15.69 & \bf{681.00} & 
0.59\\SCA8-0 & 980.51 & 35.83 & 
981.68 & 35.35 & \bf{961.50} & 
1.98\\SCA8-1 & 1060.62 & 48.94 & 
1068.86 & 45.97 & \bf{1050.20} & 
0.99\\SCA8-2 & 1053.90 & 29.24 & 
1053.93 & 35.64 & \bf{1039.64} & 
1.37\\SCA8-3 & 1011.28 & 35.01 & 
1018.67 & 32.16 & \bf{983.34} & 
2.84\\SCA8-4 & 1074.19 & 34.41 & 
1084.19 & 31.81 & \bf{1065.49} & 
0.82\\SCA8-5 & 1049.44 & 34.27 & 
1051.96 & 39.12 & \bf{1027.08} & 
2.18\\SCA8-6 & 979.28 & 70.33 & 
984.75 & 47.79 & \bf{971.82} & 
0.77\\SCA8-7 & 1070.53 & 25.05 & 
1073.34 & 40.33 & \bf{1052.17} & 
1.74\\SCA8-8 & 1085.34 & 50.66 & 
1089.00 & 41.47 & \bf{1071.18} & 
1.32\\SCA8-9 & 1073.62 & 45.85 & 
1078.95 & 38.58 & \bf{1060.50} & 
1.24\\CON3-0 & 620.76 & 8.12 & 
622.30 & 8.03 & \bf{616.52} & 
0.69\\CON3-1 & 558.09 & 11.04 & 
560.09 & 13.56 & \bf{554.47} & 
0.65\\CON3-2 & 521.38 & 10.18 & 
521.38 & 11.86 & \bf{519.26} & 
0.41\\CON3-3 & 591.20 & 17.41 & 
591.20 & 15.95 & \bf{591.19} & 
0.00\\CON3-4 & 591.43 & 16.96 & 
591.43 & 15.52 & \bf{589.32} & 
0.36\\CON3-5 & \bf{563.70} & 12.25 & 
564.76 & 11.32 & 563.70 & 0.00\\
CON3-6 & 502.16 & 8.54 & 
502.16 & 9.64 & \bf{500.80} & 
0.27\\CON3-7 & 577.54 & 13.68 & 
581.48 & 16.44 & \bf{576.48} & 
0.18\\CON3-8 & 523.19 & 13.02 & 
524.98 & 15.26 & \bf{523.05} & 
0.03\\CON3-9 & 588.38 & 10.66 & 
588.39 & 10.47 & \bf{580.05} & 
1.44\\CON8-0 & 871.77 & 47.47 & 
880.84 & 37.56 & \bf{857.17} & 
1.70\\CON8-1 & 742.47 & 41.69 & 
744.07 & 53.23 & \bf{740.85} & 
0.22\\CON8-2 & \bf{\underline{712.89}} & 54.77 & 
713.38 & 42.53 & 713.44 & 
\bf{-0.08}\\CON8-3 & 826.08 & 37.31 & 
833.29 & 36.97 & \bf{811.07} & 
1.85\\CON8-4 & 780.46 & 26.08 & 
785.52 & 42.84 & \bf{772.25} & 
1.06\\CON8-5 & \bf{\underline{754.95}} & 34.51 & 
757.44 & 40.90 & 756.91 & 
\bf{-0.26}\\CON8-6 & 694.76 & 44.92 & 
698.83 & 42.85 & \bf{678.92} & 
2.33\\CON8-7 & 814.79 & 47.96 & 
815.45 & 42.77 & \bf{811.96} & 
0.35\\CON8-8 & 785.30 & 21.08 & 
788.46 & 27.19 & \bf{767.53} & 
2.32\\CON8-9 & 822.42 & 35.63 & 
826.25 & 37.50 & \bf{809.00} & 
1.66\\\bf{PROM.} & 
\bf{766.07} & \bf{26.26} & \bf{768.74} & \bf{26.47} & \bf{758.78} & \bf{0.86}\\[1ex]\hline
\end{tabular}
\label{table:nonlin}
\end{table} \clearpage

% Crea el glosario 
\printglossaries

% Establece las citas y bibliografia ieee.bst
\bibliographystyle{ieee.bst}
\bibliography{myrefs}

% Crea el apendice
\appendix
\clearpage % 
\addappheadtotoc
\appendixpage
\chapter{Pseudocódigos de algoritmos utilizados}\label{chap:apendiceA}

A continuación se presentan pseudocódigos de algoritmos descritos en distintas secciones del libro.

\begin{code}[includerangemarker=false,frame=single,label=alg:ConstRutas,caption=Pseudocódigo de construcción de las rutas de los vehículos a partir de la estructura de datos de representación de la solución,firstnumber=100, mathescape]
$\textbf{Entrada:}$ Numero de rutas $k$, Arreglo de enteros $inicio$, Arreglo de enteros $proximo$
 
i := 0
j := 0
Cadena de enteros s := < 0 >
while (i $\leqslant$ k) do
	j := $inicio$[i]
	while($proximo$[j] $\neq$ 0)			
		s := s||< $proximo$[j] >
		j := $proximo$[j]
	end while
	s :=  s||< 0 >
	i := i+1
end while 

$\textbf{return}$ s 
\end{code}

%\begin{code}[includerangemarker=false,frame=single,label=alg:AlgIns,caption=Pseudocódigo de Algoritmo de Inserción ,firstnumber=100, mathescape]
%$\textbf{Entrada:}$ Numero de rutas $r$, Factor $\gamma$
%    
%Sea S un arreglo compuesto por $r$ rutas vacias
%InicializarListaDeCandidatos(LC)
%entero i := 1
%while (i $\leqslant$ r) do
%	S[i] := t $\in$ LC seleccionado aleatoriamente
%	Actualizar(LC)
%	i := i+1
%end while
%
%while (LC $\neq \emptyset$) do
%	Evaluar f(t,r) $\forall t \in LC$
%	g^{min} := Min{f(t,r)|t$\in$LC}
%	m := cliente $t$ asociado con g^{min}
%	Insertar(S,m)
%	Actualizar(LC)
%end while
%
%$\textbf{return}$ S 
%\end{code}

\begin{code}[includerangemarker=false,frame=single,label=alg:VND,caption=Pseudocódigo de VND,firstnumber=100, mathescape]
$\textbf{Entrada}$: $S_0$
k := 1,...,$K_{max}$ 
Seleccionar vecindades $N_{k}$ para 
//$N_{k}$ es el conjunto de vecindades inter-ruta utilizados en el siguiente orden:
//k=1 shift(1,0), k=2 crossover(), k=3 Swap(1,1), k=4 Shift(2,0), k=5 Swap(2,1) y 
//k=6 Swap(2,2).

$S_{mejor}$ := $S_0$
k := 1
while (k < $k_{max}$+1)
    $S^{'}$ := GenerarVecindadVND($N_{k}$,($S_{mejor}$))
	if(Costo($S_{mejor}$) $>$ Costo($S^{'}$))		
		$S^{''}$ := $Or-opt$($S^{'}$)
		$S^{'''}$ := $2-opt$($S^{''}$)
		$S^{''''}$ := $Swap$($S^{'''}$)		
		$S^{'''''}$ := $Reverse$($S^{''''}$)
		if(Costo($S^{'}$) $>$ Costo($S^{'''''}$))
			$S_{mejor}$ := $S^{'''''}$
		else
			$S_{mejor}$ := $S^{'}$
		end if	
		k := 1
	else
	 k := k+1
	end if 			
end while

$\textbf{return}$ $S_{mejor}$
\end{code}

\begin{code}[includerangemarker=false,frame=single,label=alg:ILS-VND-M,caption=Pseudocódigo de ILS-VND-M,firstnumber=100, mathescape]
$\textbf{Entrada}$: $y$, n, LS, k

i:=0
Costo($S_{mejor}$) := $\infty$
while (i $<$ n) do
	$S$ := ConstruirSolucionInicial($y$,k)
	$S$ := VND($S$)		
	iterILS := 0
	while (iterILS $<$ LS) do
		iterILS := iterILS+1
		$S^{'}$ := Perturbar($S$)
		$S_{actual}$ := VND($S^{'}$)	
		if(Costo($S$) $>$ Costo($S_{actual}$))		
			$S$ := $S_{actual}$
			iterILS := 0		
		end if									
	end while
	if(Costo($S_{mejor}$) $>$ Costo($S$))		
		$S_{mejor}$ := $S$		
	end if			
	i := i+1
end while

$\textbf{return}$ $S_{mejor}$	
\end{code}

\begin{code}[includerangemarker=false,frame=single,label=alg:GTS,caption=Pseudocódigo de GTS-M,firstnumber=100, mathescape]
$\textbf{Entrada}$: $mni$

Lista $Lista\_Tabu$ := $\emptyset$
$S_0$ := ConstruirSolucionInicial()
$S_{mejor}$ := $S_0$
$S_{actual}$ := $S_0$
$it$ = 0
while ($it$ < $mni$) do
	vecindad := ConstruirVecindad($S_{actual}$)
	$Costo(S_{optima})$ := $\infty$ 			
	$\textbf{foreach}$ $x \in vecindad$ do
		if (Costo($x$) > Costo($S_{mejor}$))
			$S_{optima}$ := $x$
		else
			if ($x \notin Lista\_Tabu$)
				if (Costo($x$) < Costo($S_{optima}$))
					$S_{optima}$ := $x$
				end if
			end if
		end if
	end $\textbf{foreach}$
	$S_{actual}$ := $S_{optima}$
	AgregarMovimientoReverso($Lista\_Tabu$)
	if (Costo($S_{actual}$) < Costo($S_{mejor}$))
		$S_{mejor}$ := $S_{actual}$
		$it$ := 0
	end if	
	$peor\_arco$ := ArcoMaximiceUtilidad($S_{actual}$)
	PenalizarArco($peor\_arco$)
	SustraerMovimiento($Lista\_Tabu$)
	$it$ := $it$ + 1
end while

$\textbf{return}$ $S_{mejor}$	
\end{code}

\begin{code}[includerangemarker=false,frame=single,label=alg:MACO,caption=Pseudocódigo de AS-M,firstnumber=100, mathescape]
$\textbf{Entrada}$: $n$

ComputarVisibilidad()
$S_0$ := ConstruirSolucionInicial()
InicializarFeromonas($S_0$)
$S_{mejor}$ := $S_0$
$hormigas$ := CrearHormigas()
$Costo(S_{mejor})$ := $\infty$ 
$it$ := 0
while ($it$ < $n$) do
	$\textbf{foreach}$ $h \in hormigas$ do
		while ($\neg$todos_clientes_visitados) do
			Seleccionar proximo cliente a visitar y agregarlo a la ruta
			if ($\neg$clientes_factibles)
				Retornar al deposito y crear una nueva ruta
			end if
		end while
		$S_{actual}$ := ObtenerSolucion($h$)
		VND($S_{actual}$)
		if (Costo($S_{actual}$) < Costo($S_{mejor}$))
			$S_{mejor}$ := $S_{actual}$
		end if
	end $\textbf{foreach}$
	EvaporarReforzarFeromonas()
	$it$ := $it$ + 1
end while

$\textbf{return}$ $S_{mejor}$	
\end{code}

\begin{code}[includerangemarker=false,frame=single,label=alg:SCA-M,caption=Pseudocódigo de SS-M,firstnumber=100, mathescape]
$Metodo$ $de$ $Generacion$ $y$ $Diversificacion$ - Crea un conjunto de soluciones diversas $P$ de tamano $Psize$

Construir el $RefSet$ con $b$ mejores y diversas soluciones de $P$, RefSet := $\lbrace S^1,...,S^b \rbrace$. Ordenarlas de manera creciente en base a su costo.  

$S_{mejor}$ := $S^1$
$NewSolutions$ := True
while ($NewSolutions$) do
	$NewSubset$ := GenerarSubconjuntos($RefSet$)
	$NewSolutions$ := False
    while ($NewSubset\neq\emptyset $)do
    	$Ss$ := Seleccionar($NewSubset$)
    	$S$ := Combinar(Ss) 
    	$x$ := Mejorar(S)
		if(($x$ $\notin$ $RefSet$) $\wedge$ ($Costo(x)<Costo(S^b$)))
			$S^b$ := $x$
			Ordenar($RefSet$)
			$NewSolutions$ := True											
		end if
		Eliminar($Ss$, $NewSubset$)    	    				
	end while
end while

i := 0
while(i$<$b)do
	$S^{i+1}$ := VND($S^{i+1}$)
end while
Ordenar($Refset$)
$S_{mejor}$ := $S^1$

$\textbf{return}$ $S_{mejor}$
\end{code}


\begin{code}[includerangemarker=false,frame=single,label=algimp:GA,caption=Pseudocódigo de GA-M,firstnumber=100, mathescape]
$\textbf{Entrada}$: p, maxgen, cprob, mprob

Pop := GenerarPoblacionInicial(p)
fitnessP := CalcularFitness(Pop,p)

i:=0
while (i $<$ n) do		
	j:=0
	while (j $<$ (popsize/2)) do
		padres := Seleccion(Pop)
		k := ((randomInt())$Mod$100})+1
		if(k $\leqslant$ cprob)			
			hijos := CrossOver(padres)
			k := ((randomInt())$Mod$100)+1
			if(k $\leqslant$ mprob)
				hijos := Mutacion(hijos)
			end if	
			fitnessHijos := CalcularFitness(hijos)
			Pop := ActualizarPoblacion(hijos, fitnessHijos, Pop,fitnessP)
		end if
		j:=j+1	
	end while
	i := i+1
end while

i:=0
while(i$<$p)do
	$P_i$ := VND($P_i$) 
	i:=i+1
end while
S := SeleccionarMejor(P)
$\textbf{return}$ $S$	

\end{code}

\begin{code}[includerangemarker=false,frame=single,label=alg:MPSO,caption=Pseudocódigo de PSO-M,firstnumber=100, mathescape]
$\textbf{Entrada}$: $n$
$\textbf{Entrada}$: $L$

$S_{mejor}$ := $\infty$
$enjambre$ := CrearParticulas(L)
$\textbf{foreach}$ $particula$ $\in$ $enjambre$ do
	$S_{actual}$ := ConstruirSolucionInicial()	
	$particula$ := Codificar($S_{actual}$)
	Velocidad($particula$) := 0
end $\textbf{foreach}$

$it$ := 0
while ($it$ < $n$) do
	$\textbf{foreach}$ $particula$ $\in$ $enjambre$ do
		$S_{actual}$ := Decodificar($particula$)
		VND($S_{actual}$)
		$particula$ := Codificar($S_{actual}$)
		ActualizarPBest($particula$)
		ActualizarLBest($particula$)
		ActualizarNBest($particula$)
		if (Costo($S_{actual}$) < Costo($S_{mejor}$))
			$S_{mejor}$ := $S_{actual}$
			ActualizarGBest()
		end if
		ActualizarInercia()
		ActualizarVelocidad($particula$)				
		ActualizarPosicion($particula$)
	end $\textbf{foreach}$
	$it := $it + 1
end while

$\textbf{return}$ $S_{mejor}$	
\end{code}
\chapter{Resultados de las pruebas unitarias realizadas sobre el Buscador Focalizado de Información}\label{chap:apendiceb}

%DESIGNACIONES
%------------------------
\begin{landscape}
\begin{table}
\centering
\caption{ Resultados de la evaluación del Extractor Focalizado - Dominio: Designaciones. UnitHit Measure mínimo:1.0}
\centering
\scriptsize
\begin{tabular*}{1\textwidth}{@{\extracolsep{\fill}} !{\vrule width 1pt} c !{\vrule width 1pt} c !{\vrule width 1pt} c | c | c !{\vrule width 1pt} c | c | c !{\vrule width 1pt}}
\hline
Campo & Prob. Campo Faltante & \multicolumn{3}{c!{\vrule width 1pt}}{\bf{P. Aprobadas}} & \multicolumn{3}{c!{\vrule width 1pt}}{\bf{P. No aprobadas}}\\
\hline
\multicolumn{2}{!{\vrule width 1pt}c|}{ } & Correctos & Con texto de más & Aprobados & Incompletos & Incorrectos & No aprobados\\
\hline
\multirow{3}{*}{EsAsignado.calificacion} 

	& 0.0
	& 74,86\% & 9,5\% & \bf{84,36\%} & 12,85\% & 2,79\% & \bf{15,64\%} \\
	\cline{3-8}

	& 0.33
	& 65,92\% & 7,26\% & \bf{73,18\%} & 24,58\% & 2,23\% & \bf{26,82\%} \\
	\cline{3-8}

	& 0.66
	& 39,11\% & 5,59\% & \bf{44,69\%} & 54,19\% & 1,12\% & \bf{55,31\%} \\
	\cline{3-8}

\hline
\multirow{3}{*}{EsAsignado.fechaAsignacion} 

	& 0.0
	& 84,92\% & 1,12\% & \bf{86,03\%} & 13,41\% & 0,56\% & \bf{13,97\%} \\
	\cline{3-8}

	& 0.33
	& 73,18\% & 1,68\% & \bf{74,86\%} & 24,58\% & 0,56\% & \bf{25,14\%} \\
	\cline{3-8}

	& 0.66
	& 44,13\% & 1,12\% & \bf{45,25\%} & 54,75\% & 0\% & \bf{54,75\%} \\
	\cline{3-8}

\hline
\multirow{3}{*}{EsAsignado.fechaFinal} 

	& 0.0
	& 96,09\% & 1,68\% & \bf{97,77\%} & 2,23\% & 0\% & \bf{2,23\%} \\
	\cline{3-8}

	& 0.33
	& 94,41\% & 3,91\% & \bf{98,32\%} & 1,68\% & 0\% & \bf{1,68\%} \\
	\cline{3-8}

	& 0.66
	& 87,15\% & 9,5\% & \bf{96,65\%} & 3,35\% & 0\% & \bf{3,35\%} \\
	\cline{3-8}

\hline
\multirow{3}{*}{EsAsignado.motivo} 

	& 0.0
	& 88,83\% & 7,82\% & \bf{96,65\%} & 2,79\% & 0,56\% & \bf{3,35\%} \\
	\cline{3-8}

	& 0.33
	& 78,77\% & 13,97\% & \bf{92,74\%} & 6,7\% & 0,56\% & \bf{7,26\%} \\
	\cline{3-8}

	& 0.66
	& 63,69\% & 17,88\% & \bf{81,56\%} & 18,44\% & 0\% & \bf{18,44\%} \\
	\cline{3-8}

\hline
\multirow{3}{*}{Profesor.Nombre} 

	& 0.0
	& 86,03\% & 2,79\% & \bf{88,83\%} & 10,61\% & 0,56\% & \bf{11,17\%} \\
	\cline{3-8}

	& 0.33
	& 70,95\% & 2,23\% & \bf{73,18\%} & 26,82\% & 0\% & \bf{26,82\%} \\
	\cline{3-8}

	& 0.66
	& 47,49\% & 2,79\% & \bf{50,28\%} & 49,16\% & 0,56\% & \bf{49,72\%} \\
	\cline{3-8}

\hline
\end{tabular*}
\label{tabla-resultados-EFDesignaciones1.0}
\\
Prob. Campo Faltante es la probabilidad de que no se tenga el valor uno de los campos que se utilizan para hacer extracción focalizada.
\end{table}
\end{landscape}
\begin{landscape}
\begin{table}
\centering
\caption{ Resultados de la evaluación del Extractor Focalizado - Dominio: Designaciones. UnitHit Measure mínimo:0.66}
\centering
\scriptsize
\begin{tabular*}{1\textwidth}{@{\extracolsep{\fill}} !{\vrule width 1pt} c !{\vrule width 1pt} c !{\vrule width 1pt} c | c | c !{\vrule width 1pt} c | c | c !{\vrule width 1pt}}
\hline
Campo & Prob. Campo Faltante & \multicolumn{3}{c!{\vrule width 1pt}}{\bf{P. Aprobadas}} & \multicolumn{3}{c!{\vrule width 1pt}}{\bf{P. No aprobadas}}\\
\hline
\multicolumn{2}{!{\vrule width 1pt}c|}{ } & Correctos & Con texto de más & Aprobados & Incompletos & Incorrectos & No aprobados\\
\hline
\multirow{3}{*}{EsAsignado.calificacion} 

	& 0.0
	& 79,33\% & 9,5\% & \bf{88,83\%} & 8,38\% & 2,79\% & \bf{11,17\%} \\
	\cline{3-8}

	& 0.33
	& 64,8\% & 8,38\% & \bf{73,18\%} & 24,02\% & 2,79\% & \bf{26,82\%} \\
	\cline{3-8}

	& 0.66
	& 39,66\% & 4,47\% & \bf{44,13\%} & 54,75\% & 1,12\% & \bf{55,87\%} \\
	\cline{3-8}

\hline
\multirow{3}{*}{EsAsignado.fechaAsignacion} 

	& 0.0
	& 83,8\% & 2,79\% & \bf{86,59\%} & 12,85\% & 0,56\% & \bf{13,41\%} \\
	\cline{3-8}

	& 0.33
	& 74,86\% & 2,79\% & \bf{77,65\%} & 22,35\% & 0\% & \bf{22,35\%} \\
	\cline{3-8}

	& 0.66
	& 50,28\% & 2,23\% & \bf{52,51\%} & 46,93\% & 0,56\% & \bf{47,49\%} \\
	\cline{3-8}

\hline
\multirow{3}{*}{EsAsignado.fechaFinal} 

	& 0.0
	& 96,09\% & 2,79\% & \bf{98,88\%} & 1,12\% & 0\% & \bf{1,12\%} \\
	\cline{3-8}

	& 0.33
	& 92,18\% & 6,15\% & \bf{98,32\%} & 1,68\% & 0\% & \bf{1,68\%} \\
	\cline{3-8}

	& 0.66
	& 85,47\% & 10,61\% & \bf{96,09\%} & 3,91\% & 0\% & \bf{3,91\%} \\
	\cline{3-8}

\hline
\multirow{3}{*}{EsAsignado.motivo} 

	& 0.0
	& 84,92\% & 12,85\% & \bf{97,77\%} & 1,68\% & 0,56\% & \bf{2,23\%} \\
	\cline{3-8}

	& 0.33
	& 72,63\% & 20,11\% & \bf{92,74\%} & 7,26\% & 0\% & \bf{7,26\%} \\
	\cline{3-8}

	& 0.66
	& 64,25\% & 20,11\% & \bf{84,36\%} & 15,64\% & 0\% & \bf{15,64\%} \\
	\cline{3-8}

\hline
\multirow{3}{*}{Profesor.Nombre} 

	& 0.0
	& 89,39\% & 2,79\% & \bf{92,18\%} & 7,26\% & 0,56\% & \bf{7,82\%} \\
	\cline{3-8}

	& 0.33
	& 68,16\% & 2,79\% & \bf{70,95\%} & 28,49\% & 0,56\% & \bf{29,05\%} \\
	\cline{3-8}

	& 0.66
	& 40,78\% & 2,23\% & \bf{43,02\%} & 56,98\% & 0\% & \bf{56,98\%} \\
	\cline{3-8}

\hline
\end{tabular*}
\label{tabla-resultados-EFDesignaciones0.66}
\\
Prob. Campo Faltante es la probabilidad de que no se tenga el valor uno de los campos que se utilizan para hacer extracción focalizada.
\end{table}
\end{landscape}
\begin{landscape}
\begin{table}
\centering
\caption{ Resultados de la evaluación del Extractor Focalizado - Dominio: Designaciones. UnitHit Measure mínimo:0.33}
\centering
\scriptsize
\begin{tabular*}{1\textwidth}{@{\extracolsep{\fill}} !{\vrule width 1pt} c !{\vrule width 1pt} c !{\vrule width 1pt} c | c | c !{\vrule width 1pt} c | c | c !{\vrule width 1pt}}
\hline
Campo & Prob. Campo Faltante & \multicolumn{3}{c!{\vrule width 1pt}}{\bf{P. Aprobadas}} & \multicolumn{3}{c!{\vrule width 1pt}}{\bf{P. No aprobadas}}\\
\hline
\multicolumn{2}{!{\vrule width 1pt}c|}{ } & Correctos & Con texto de más & Aprobados & Incompletos & Incorrectos & No aprobados\\
\hline
\multirow{3}{*}{EsAsignado.calificacion} 

	& 0.0
	& 80,45\% & 9,5\% & \bf{89,94\%} & 7,26\% & 2,79\% & \bf{10,06\%} \\
	\cline{3-8}

	& 0.33
	& 63,69\% & 8,94\% & \bf{72,63\%} & 25,14\% & 2,23\% & \bf{27,37\%} \\
	\cline{3-8}

	& 0.66
	& 40,78\% & 5,59\% & \bf{46,37\%} & 53,07\% & 0,56\% & \bf{53,63\%} \\
	\cline{3-8}

\hline
\multirow{3}{*}{EsAsignado.fechaAsignacion} 

	& 0.0
	& 83,24\% & 3,35\% & \bf{86,59\%} & 12,85\% & 0,56\% & \bf{13,41\%} \\
	\cline{3-8}

	& 0.33
	& 74,3\% & 2,79\% & \bf{77,09\%} & 22,35\% & 0,56\% & \bf{22,91\%} \\
	\cline{3-8}

	& 0.66
	& 49,72\% & 3,35\% & \bf{53,07\%} & 46,93\% & 0\% & \bf{46,93\%} \\
	\cline{3-8}

\hline
\multirow{3}{*}{EsAsignado.fechaFinal} 

	& 0.0
	& 82,68\% & 16,2\% & \bf{98,88\%} & 1,12\% & 0\% & \bf{1,12\%} \\
	\cline{3-8}

	& 0.33
	& 58,1\% & 40,78\% & \bf{98,88\%} & 1,12\% & 0\% & \bf{1,12\%} \\
	\cline{3-8}

	& 0.66
	& 65,36\% & 30,73\% & \bf{96,09\%} & 3,91\% & 0\% & \bf{3,91\%} \\
	\cline{3-8}

\hline
\multirow{3}{*}{EsAsignado.motivo} 

	& 0.0
	& 59,78\% & 37,99\% & \bf{97,77\%} & 1,68\% & 0,56\% & \bf{2,23\%} \\
	\cline{3-8}

	& 0.33
	& 49,16\% & 44,69\% & \bf{93,85\%} & 5,59\% & 0,56\% & \bf{6,15\%} \\
	\cline{3-8}

	& 0.66
	& 53,07\% & 31,28\% & \bf{84,36\%} & 15,08\% & 0,56\% & \bf{15,64\%} \\
	\cline{3-8}

\hline
\multirow{3}{*}{Profesor.Nombre} 

	& 0.0
	& 90,5\% & 2,79\% & \bf{93,3\%} & 6,15\% & 0,56\% & \bf{6,7\%} \\
	\cline{3-8}

	& 0.33
	& 72,07\% & 1,68\% & \bf{73,74\%} & 25,7\% & 0,56\% & \bf{26,26\%} \\
	\cline{3-8}

	& 0.66
	& 39,11\% & 1,68\% & \bf{40,78\%} & 58,66\% & 0,56\% & \bf{59,22\%} \\
	\cline{3-8}

\hline
\end{tabular*}
\label{tabla-resultados-EFDesignaciones0.33}
\\
Prob. Campo Faltante es la probabilidad de que no se tenga el valor uno de los campos que se utilizan para hacer extracción focalizada.
\end{table}
\end{landscape}

%ESCALAFON
%------------------------
\begin{landscape}
\begin{table}
\centering
\caption{ Resultados de la evaluación del Extractor Focalizado - Dominio: Escalafon. UnitHit Measure mínimo:1.0}
\centering
\scriptsize
\begin{tabular*}{1\textwidth}{@{\extracolsep{\fill}} !{\vrule width 1pt} c !{\vrule width 1pt} c !{\vrule width 1pt} c | c | c !{\vrule width 1pt} c | c | c !{\vrule width 1pt}}
\hline
Campo & Prob. Campo Faltante & \multicolumn{3}{c!{\vrule width 1pt}}{\bf{P. Aprobadas}} & \multicolumn{3}{c!{\vrule width 1pt}}{\bf{P. No aprobadas}}\\
\hline
\multicolumn{2}{!{\vrule width 1pt}c|}{ } & Correctos & Con texto de más & Aprobados & Incompletos & Incorrectos & No aprobados\\
\hline
\multirow{3}{*}{EsAscendido.Nombre} 

	& 0.0
	& 92,31\% & 1,54\% & \bf{93,85\%} & 5,38\% & 0,77\% & \bf{6,15\%} \\
	\cline{3-8}

	& 0.33
	& 77,69\% & 0\% & \bf{77,69\%} & 21,54\% & 0,77\% & \bf{22,31\%} \\
	\cline{3-8}

	& 0.66
	& 43,85\% & 1,54\% & \bf{45,38\%} & 53,85\% & 0,77\% & \bf{54,62\%} \\
	\cline{3-8}

\hline
\multirow{3}{*}{EsAscendido.NombreTrabajo} 

	& 0.0
	& 94,62\% & 1,54\% & \bf{96,15\%} & 3,08\% & 0,77\% & \bf{3,85\%} \\
	\cline{3-8}

	& 0.33
	& 73,85\% & 1,54\% & \bf{75,38\%} & 23,85\% & 0,77\% & \bf{24,62\%} \\
	\cline{3-8}

	& 0.66
	& 42,31\% & 0,77\% & \bf{43,08\%} & 56,15\% & 0,77\% & \bf{56,92\%} \\
	\cline{3-8}

\hline
\multirow{3}{*}{EsAscendido.escalafon} 

	& 0.0
	& 97,69\% & 0\% & \bf{97,69\%} & 2,31\% & 0\% & \bf{2,31\%} \\
	\cline{3-8}

	& 0.33
	& 90\% & 0\% & \bf{90\%} & 10\% & 0\% & \bf{10\%} \\
	\cline{3-8}

	& 0.66
	& 65,38\% & 0\% & \bf{65,38\%} & 34,62\% & 0\% & \bf{34,62\%} \\
	\cline{3-8}

\hline
\multirow{3}{*}{EsAscendido.fecha} 

	& 0.0
	& 96,15\% & 1,54\% & \bf{97,69\%} & 2,31\% & 0\% & \bf{2,31\%} \\
	\cline{3-8}

	& 0.33
	& 84,62\% & 4,62\% & \bf{89,23\%} & 10,77\% & 0\% & \bf{10,77\%} \\
	\cline{3-8}

	& 0.66
	& 54,62\% & 6,15\% & \bf{60,77\%} & 39,23\% & 0\% & \bf{39,23\%} \\
	\cline{3-8}

\hline
\multirow{3}{*}{EsAscendido.postergado} 

	& 0.0
	& 99,23\% & 0\% & \bf{99,23\%} & 0,77\% & 0\% & \bf{0,77\%} \\
	\cline{3-8}

	& 0.33
	& 94,62\% & 3,85\% & \bf{98,46\%} & 1,54\% & 0\% & \bf{1,54\%} \\
	\cline{3-8}

	& 0.66
	& 89,23\% & 7,69\% & \bf{96,92\%} & 3,08\% & 0\% & \bf{3,08\%} \\
	\cline{3-8}

\hline
\end{tabular*}
\label{tabla-resultados-EFEscalafon1.0}
\\
Prob. Campo Faltante es la probabilidad de que no se tenga el valor uno de los campos que se utilizan para hacer extracción focalizada.
\end{table}
\end{landscape}
\begin{landscape}
\begin{table}
\centering
\caption{ Resultados de la evaluación del Extractor Focalizado - Dominio: Escalafon. UnitHit Measure mínimo:0.66}
\centering
\scriptsize
\begin{tabular*}{1\textwidth}{@{\extracolsep{\fill}} !{\vrule width 1pt} c !{\vrule width 1pt} c !{\vrule width 1pt} c | c | c !{\vrule width 1pt} c | c | c !{\vrule width 1pt}}
\hline
Campo & Prob. Campo Faltante & \multicolumn{3}{c!{\vrule width 1pt}}{\bf{P. Aprobadas}} & \multicolumn{3}{c!{\vrule width 1pt}}{\bf{P. No aprobadas}}\\
\hline
\multicolumn{2}{!{\vrule width 1pt}c|}{ } & Correctos & Con texto de más & Aprobados & Incompletos & Incorrectos & No aprobados\\
\hline
\multirow{3}{*}{EsAscendido.Nombre} 

	& 0.0
	& 92,31\% & 1,54\% & \bf{93,85\%} & 5,38\% & 0,77\% & \bf{6,15\%} \\
	\cline{3-8}

	& 0.33
	& 76,92\% & 1,54\% & \bf{78,46\%} & 20,77\% & 0,77\% & \bf{21,54\%} \\
	\cline{3-8}

	& 0.66
	& 43,08\% & 0\% & \bf{43,08\%} & 56,15\% & 0,77\% & \bf{56,92\%} \\
	\cline{3-8}

\hline
\multirow{3}{*}{EsAscendido.NombreTrabajo} 

	& 0.0
	& 94,62\% & 1,54\% & \bf{96,15\%} & 2,31\% & 1,54\% & \bf{3,85\%} \\
	\cline{3-8}

	& 0.33
	& 70,77\% & 1,54\% & \bf{72,31\%} & 26,92\% & 0,77\% & \bf{27,69\%} \\
	\cline{3-8}

	& 0.66
	& 43,08\% & 1,54\% & \bf{44,62\%} & 53,85\% & 1,54\% & \bf{55,38\%} \\
	\cline{3-8}

\hline
\multirow{3}{*}{EsAscendido.escalafon} 

	& 0.0
	& 97,69\% & 0\% & \bf{97,69\%} & 2,31\% & 0\% & \bf{2,31\%} \\
	\cline{3-8}

	& 0.33
	& 90,77\% & 0\% & \bf{90,77\%} & 9,23\% & 0\% & \bf{9,23\%} \\
	\cline{3-8}

	& 0.66
	& 71,54\% & 0\% & \bf{71,54\%} & 28,46\% & 0\% & \bf{28,46\%} \\
	\cline{3-8}

\hline
\multirow{3}{*}{EsAscendido.fecha} 

	& 0.0
	& 95,38\% & 2,31\% & \bf{97,69\%} & 2,31\% & 0\% & \bf{2,31\%} \\
	\cline{3-8}

	& 0.33
	& 76,15\% & 9,23\% & \bf{85,38\%} & 14,62\% & 0\% & \bf{14,62\%} \\
	\cline{3-8}

	& 0.66
	& 60,77\% & 8,46\% & \bf{69,23\%} & 30,77\% & 0\% & \bf{30,77\%} \\
	\cline{3-8}

\hline
\multirow{3}{*}{EsAscendido.postergado} 

	& 0.0
	& 99,23\% & 0\% & \bf{99,23\%} & 0,77\% & 0\% & \bf{0,77\%} \\
	\cline{3-8}

	& 0.33
	& 96,15\% & 2,31\% & \bf{98,46\%} & 1,54\% & 0\% & \bf{1,54\%} \\
	\cline{3-8}

	& 0.66
	& 84,62\% & 12,31\% & \bf{96,92\%} & 3,08\% & 0\% & \bf{3,08\%} \\
	\cline{3-8}

\hline
\end{tabular*}
\label{tabla-resultados-EFEscalafon0.66}
\\
Prob. Campo Faltante es la probabilidad de que no se tenga el valor uno de los campos que se utilizan para hacer extracción focalizada.
\end{table}
\end{landscape}
\begin{landscape}
\begin{table}
\centering
\caption{ Resultados de la evaluación del Extractor Focalizado - Dominio: Escalafon. UnitHit Measure mínimo:0.33}
\centering
\scriptsize
\begin{tabular*}{1\textwidth}{@{\extracolsep{\fill}} !{\vrule width 1pt} c !{\vrule width 1pt} c !{\vrule width 1pt} c | c | c !{\vrule width 1pt} c | c | c !{\vrule width 1pt}}
\hline
Campo & Prob. Campo Faltante & \multicolumn{3}{c!{\vrule width 1pt}}{\bf{P. Aprobadas}} & \multicolumn{3}{c!{\vrule width 1pt}}{\bf{P. No aprobadas}}\\
\hline
\multicolumn{2}{!{\vrule width 1pt}c|}{ } & Correctos & Con texto de más & Aprobados & Incompletos & Incorrectos & No aprobados\\
\hline
\multirow{3}{*}{EsAscendido.Nombre} 

	& 0.0
	& 92,31\% & 1,54\% & \bf{93,85\%} & 5,38\% & 0,77\% & \bf{6,15\%} \\
	\cline{3-8}

	& 0.33
	& 70,77\% & 1,54\% & \bf{72,31\%} & 26,92\% & 0,77\% & \bf{27,69\%} \\
	\cline{3-8}

	& 0.66
	& 41,54\% & 0,77\% & \bf{42,31\%} & 56,92\% & 0,77\% & \bf{57,69\%} \\
	\cline{3-8}

\hline
\multirow{3}{*}{EsAscendido.NombreTrabajo} 

	& 0.0
	& 94,62\% & 1,54\% & \bf{96,15\%} & 2,31\% & 1,54\% & \bf{3,85\%} \\
	\cline{3-8}

	& 0.33
	& 72,31\% & 1,54\% & \bf{73,85\%} & 24,62\% & 1,54\% & \bf{26,15\%} \\
	\cline{3-8}

	& 0.66
	& 41,54\% & 0,77\% & \bf{42,31\%} & 56,15\% & 1,54\% & \bf{57,69\%} \\
	\cline{3-8}

\hline
\multirow{3}{*}{EsAscendido.escalafon} 

	& 0.0
	& 99,23\% & 0\% & \bf{99,23\%} & 0,77\% & 0\% & \bf{0,77\%} \\
	\cline{3-8}

	& 0.33
	& 92,31\% & 0\% & \bf{92,31\%} & 7,69\% & 0\% & \bf{7,69\%} \\
	\cline{3-8}

	& 0.66
	& 70,77\% & 0\% & \bf{70,77\%} & 29,23\% & 0\% & \bf{29,23\%} \\
	\cline{3-8}

\hline
\multirow{3}{*}{EsAscendido.fecha} 

	& 0.0
	& 80\% & 17,69\% & \bf{97,69\%} & 2,31\% & 0\% & \bf{2,31\%} \\
	\cline{3-8}

	& 0.33
	& 70,77\% & 16,92\% & \bf{87,69\%} & 12,31\% & 0\% & \bf{12,31\%} \\
	\cline{3-8}

	& 0.66
	& 56,92\% & 14,62\% & \bf{71,54\%} & 28,46\% & 0\% & \bf{28,46\%} \\
	\cline{3-8}

\hline
\multirow{3}{*}{EsAscendido.postergado} 

	& 0.0
	& 83,08\% & 16,15\% & \bf{99,23\%} & 0,77\% & 0\% & \bf{0,77\%} \\
	\cline{3-8}

	& 0.33
	& 52,31\% & 46,92\% & \bf{99,23\%} & 0,77\% & 0\% & \bf{0,77\%} \\
	\cline{3-8}

	& 0.66
	& 66,15\% & 31,54\% & \bf{97,69\%} & 2,31\% & 0\% & \bf{2,31\%} \\
	\cline{3-8}

\hline
\end{tabular*}
\label{tabla-resultados-EFEscalafon0.33}
\\
Prob. Campo Faltante es la probabilidad de que no se tenga el valor uno de los campos que se utilizan para hacer extracción focalizada.
\end{table}
\end{landscape}

%JURADOS DE ASCENSO 
%------------------------
\begin{landscape}
\begin{table}
\centering
\caption{ Resultados de la evaluación del Extractor Focalizado - Dominio: JuradosAscenso. UnitHit Measure mínimo:1.0}
\centering
\scriptsize
\begin{tabular*}{1\textwidth}{@{\extracolsep{\fill}} !{\vrule width 1pt} c !{\vrule width 1pt} c !{\vrule width 1pt} c | c | c !{\vrule width 1pt} c | c | c !{\vrule width 1pt}}
\hline
Campo & Prob. Campo Faltante & \multicolumn{3}{c!{\vrule width 1pt}}{\bf{P. Aprobadas}} & \multicolumn{3}{c!{\vrule width 1pt}}{\bf{P. No aprobadas}}\\
\hline
\multicolumn{2}{!{\vrule width 1pt}c|}{ } & Correctos & Con texto de más & Aprobados & Incompletos & Incorrectos & No aprobados\\
\hline
\multirow{3}{*}{Jurado.MiembroPrincipalExterno} 

	& 0.0
	& 89,63\% & 0\% & \bf{89,63\%} & 5,93\% & 4,44\% & \bf{10,37\%} \\
	\cline{3-8}

	& 0.33
	& 89,63\% & 0\% & \bf{89,63\%} & 5,93\% & 4,44\% & \bf{10,37\%} \\
	\cline{3-8}

	& 0.66
	& 77,78\% & 2,22\% & \bf{80\%} & 14,81\% & 5,19\% & \bf{20\%} \\
	\cline{3-8}

\hline
\multirow{3}{*}{Jurado.MiembroPrincipalInterno} 

	& 0.0
	& 97,04\% & 0\% & \bf{97,04\%} & 2,96\% & 0\% & \bf{2,96\%} \\
	\cline{3-8}

	& 0.33
	& 97,04\% & 0\% & \bf{97,04\%} & 2,96\% & 0\% & \bf{2,96\%} \\
	\cline{3-8}

	& 0.66
	& 81,48\% & 0,74\% & \bf{82,22\%} & 17,78\% & 0\% & \bf{17,78\%} \\
	\cline{3-8}

\hline
\multirow{3}{*}{Jurado.Presidente} 

	& 0.0
	& 89,63\% & 0\% & \bf{89,63\%} & 10,37\% & 0\% & \bf{10,37\%} \\
	\cline{3-8}

	& 0.33
	& 91,11\% & 0\% & \bf{91,11\%} & 8,89\% & 0\% & \bf{8,89\%} \\
	\cline{3-8}

	& 0.66
	& 81,48\% & 0\% & \bf{81,48\%} & 18,52\% & 0\% & \bf{18,52\%} \\
	\cline{3-8}

\hline
\multirow{3}{*}{Jurado.SuplenteExterno} 

	& 0.0
	& 89,63\% & 0\% & \bf{89,63\%} & 10,37\% & 0\% & \bf{10,37\%} \\
	\cline{3-8}

	& 0.33
	& 92,59\% & 0\% & \bf{92,59\%} & 6,67\% & 0,74\% & \bf{7,41\%} \\
	\cline{3-8}

	& 0.66
	& 79,26\% & 2,96\% & \bf{82,22\%} & 16,3\% & 1,48\% & \bf{17,78\%} \\
	\cline{3-8}

\hline
\multirow{3}{*}{Jurado.SuplenteInterno} 

	& 0.0
	& 93,33\% & 0\% & \bf{93,33\%} & 6,67\% & 0\% & \bf{6,67\%} \\
	\cline{3-8}

	& 0.33
	& 94,07\% & 0\% & \bf{94,07\%} & 5,93\% & 0\% & \bf{5,93\%} \\
	\cline{3-8}

	& 0.66
	& 82,22\% & 0\% & \bf{82,22\%} & 17,78\% & 0\% & \bf{17,78\%} \\
	\cline{3-8}

\hline
\multirow{3}{*}{Profesor.Departamento} 

	& 0.0
	& 88,89\% & 0\% & \bf{88,89\%} & 11,11\% & 0\% & \bf{11,11\%} \\
	\cline{3-8}

	& 0.33
	& 91,85\% & 0\% & \bf{91,85\%} & 8,15\% & 0\% & \bf{8,15\%} \\
	\cline{3-8}

	& 0.66
	& 88,15\% & 0\% & \bf{88,15\%} & 11,85\% & 0\% & \bf{11,85\%} \\
	\cline{3-8}

\hline
\multirow{3}{*}{Profesor.Nombre} 

	& 0.0
	& 88,89\% & 0\% & \bf{88,89\%} & 10,37\% & 0,74\% & \bf{11,11\%} \\
	\cline{3-8}

	& 0.33
	& 88,89\% & 0\% & \bf{88,89\%} & 10,37\% & 0,74\% & \bf{11,11\%} \\
	\cline{3-8}

	& 0.66
	& 80,74\% & 0\% & \bf{80,74\%} & 18,52\% & 0,74\% & \bf{19,26\%} \\
	\cline{3-8}

\hline
\multirow{3}{*}{Trabajo.Escalafon} 

	& 0.0
	& 89,63\% & 0\% & \bf{89,63\%} & 10,37\% & 0\% & \bf{10,37\%} \\
	\cline{3-8}

	& 0.33
	& 91,85\% & 0\% & \bf{91,85\%} & 8,15\% & 0\% & \bf{8,15\%} \\
	\cline{3-8}

	& 0.66
	& 85,19\% & 0\% & \bf{85,19\%} & 14,81\% & 0\% & \bf{14,81\%} \\
	\cline{3-8}

\hline
\multirow{3}{*}{Trabajo.Nombre} 

	& 0.0
	& 89,63\% & 0\% & \bf{89,63\%} & 9,63\% & 0,74\% & \bf{10,37\%} \\
	\cline{3-8}

	& 0.33
	& 91,11\% & 0\% & \bf{91,11\%} & 8,15\% & 0,74\% & \bf{8,89\%} \\
	\cline{3-8}

	& 0.66
	& 80,74\% & 0\% & \bf{80,74\%} & 19,26\% & 0\% & \bf{19,26\%} \\
	\cline{3-8}

\hline
\end{tabular*}
\label{tabla-resultados-EFJuradosAscenso1.0}
\\
Prob. Campo Faltante es la probabilidad de que no se tenga el valor uno de los campos que se utilizan para hacer extracción focalizada.
\end{table}
\end{landscape}
\begin{landscape}
\begin{table}
\centering
\caption{ Resultados de la evaluación del Extractor Focalizado - Dominio: JuradosAscenso. UnitHit Measure mínimo:0.66}
\centering
\scriptsize
\begin{tabular*}{1\textwidth}{@{\extracolsep{\fill}} !{\vrule width 1pt} c !{\vrule width 1pt} c !{\vrule width 1pt} c | c | c !{\vrule width 1pt} c | c | c !{\vrule width 1pt}}
\hline
Campo & Prob. Campo Faltante & \multicolumn{3}{c!{\vrule width 1pt}}{\bf{P. Aprobadas}} & \multicolumn{3}{c!{\vrule width 1pt}}{\bf{P. No aprobadas}}\\
\hline
\multicolumn{2}{!{\vrule width 1pt}c|}{ } & Correctos & Con texto de más & Aprobados & Incompletos & Incorrectos & No aprobados\\
\hline
\multirow{3}{*}{Jurado.MiembroPrincipalExterno} 

	& 0.0
	& 90,37\% & 0\% & \bf{90,37\%} & 2,22\% & 7,41\% & \bf{9,63\%} \\
	\cline{3-8}

	& 0.33
	& 88,89\% & 0,74\% & \bf{89,63\%} & 2,96\% & 7,41\% & \bf{10,37\%} \\
	\cline{3-8}

	& 0.66
	& 75,56\% & 7,41\% & \bf{82,96\%} & 12,59\% & 4,44\% & \bf{17,04\%} \\
	\cline{3-8}

\hline
\multirow{3}{*}{Jurado.MiembroPrincipalInterno} 

	& 0.0
	& 97,78\% & 0\% & \bf{97,78\%} & 2,22\% & 0\% & \bf{2,22\%} \\
	\cline{3-8}

	& 0.33
	& 97,04\% & 0\% & \bf{97,04\%} & 2,96\% & 0\% & \bf{2,96\%} \\
	\cline{3-8}

	& 0.66
	& 81,48\% & 1,48\% & \bf{82,96\%} & 17,04\% & 0\% & \bf{17,04\%} \\
	\cline{3-8}

\hline
\multirow{3}{*}{Jurado.Presidente} 

	& 0.0
	& 97,78\% & 0\% & \bf{97,78\%} & 2,22\% & 0\% & \bf{2,22\%} \\
	\cline{3-8}

	& 0.33
	& 97,78\% & 0\% & \bf{97,78\%} & 2,22\% & 0\% & \bf{2,22\%} \\
	\cline{3-8}

	& 0.66
	& 82,96\% & 0\% & \bf{82,96\%} & 17,04\% & 0\% & \bf{17,04\%} \\
	\cline{3-8}

\hline
\multirow{3}{*}{Jurado.SuplenteExterno} 

	& 0.0
	& 94,81\% & 0\% & \bf{94,81\%} & 2,22\% & 2,96\% & \bf{5,19\%} \\
	\cline{3-8}

	& 0.33
	& 91,85\% & 2,96\% & \bf{94,81\%} & 2,22\% & 2,96\% & \bf{5,19\%} \\
	\cline{3-8}

	& 0.66
	& 77,04\% & 6,67\% & \bf{83,7\%} & 14,07\% & 2,22\% & \bf{16,3\%} \\
	\cline{3-8}

\hline
\multirow{3}{*}{Jurado.SuplenteInterno} 

	& 0.0
	& 95,56\% & 2,22\% & \bf{97,78\%} & 2,22\% & 0\% & \bf{2,22\%} \\
	\cline{3-8}

	& 0.33
	& 95,56\% & 2,22\% & \bf{97,78\%} & 2,22\% & 0\% & \bf{2,22\%} \\
	\cline{3-8}

	& 0.66
	& 74,81\% & 2,96\% & \bf{77,78\%} & 22,22\% & 0\% & \bf{22,22\%} \\
	\cline{3-8}

\hline
\multirow{3}{*}{Profesor.Departamento} 

	& 0.0
	& 97,78\% & 0\% & \bf{97,78\%} & 2,22\% & 0\% & \bf{2,22\%} \\
	\cline{3-8}

	& 0.33
	& 97,04\% & 0\% & \bf{97,04\%} & 2,96\% & 0\% & \bf{2,96\%} \\
	\cline{3-8}

	& 0.66
	& 83,7\% & 0\% & \bf{83,7\%} & 16,3\% & 0\% & \bf{16,3\%} \\
	\cline{3-8}

\hline
\multirow{3}{*}{Profesor.Nombre} 

	& 0.0
	& 97,78\% & 0\% & \bf{97,78\%} & 1,48\% & 0,74\% & \bf{2,22\%} \\
	\cline{3-8}

	& 0.33
	& 96,3\% & 0\% & \bf{96,3\%} & 2,96\% & 0,74\% & \bf{3,7\%} \\
	\cline{3-8}

	& 0.66
	& 77,78\% & 0\% & \bf{77,78\%} & 21,48\% & 0,74\% & \bf{22,22\%} \\
	\cline{3-8}

\hline
\multirow{3}{*}{Trabajo.Escalafon} 

	& 0.0
	& 98,52\% & 0\% & \bf{98,52\%} & 1,48\% & 0\% & \bf{1,48\%} \\
	\cline{3-8}

	& 0.33
	& 97,04\% & 0\% & \bf{97,04\%} & 2,96\% & 0\% & \bf{2,96\%} \\
	\cline{3-8}

	& 0.66
	& 87,41\% & 0\% & \bf{87,41\%} & 12,59\% & 0\% & \bf{12,59\%} \\
	\cline{3-8}

\hline
\multirow{3}{*}{Trabajo.Nombre} 

	& 0.0
	& 97,78\% & 0\% & \bf{97,78\%} & 1,48\% & 0,74\% & \bf{2,22\%} \\
	\cline{3-8}

	& 0.33
	& 97,78\% & 0\% & \bf{97,78\%} & 1,48\% & 0,74\% & \bf{2,22\%} \\
	\cline{3-8}

	& 0.66
	& 77,78\% & 0\% & \bf{77,78\%} & 21,48\% & 0,74\% & \bf{22,22\%} \\
	\cline{3-8}

\hline
\end{tabular*}
\label{tabla-resultados-EFJuradosAscenso0.66}
\\
Prob. Campo Faltante es la probabilidad de que no se tenga el valor uno de los campos que se utilizan para hacer extracción focalizada.
\end{table}
\end{landscape}
\begin{landscape}
\begin{table}
\centering
\caption{ Resultados de la evaluación del Extractor Focalizado - Dominio: JuradosAscenso. UnitHit Measure mínimo:0.33}
\centering
\scriptsize
\begin{tabular*}{1\textwidth}{@{\extracolsep{\fill}} !{\vrule width 1pt} c !{\vrule width 1pt} c !{\vrule width 1pt} c | c | c !{\vrule width 1pt} c | c | c !{\vrule width 1pt}}
\hline
Campo & Prob. Campo Faltante & \multicolumn{3}{c!{\vrule width 1pt}}{\bf{P. Aprobadas}} & \multicolumn{3}{c!{\vrule width 1pt}}{\bf{P. No aprobadas}}\\
\hline
\multicolumn{2}{!{\vrule width 1pt}c|}{ } & Correctos & Con texto de más & Aprobados & Incompletos & Incorrectos & No aprobados\\
\hline
\multirow{3}{*}{Jurado.MiembroPrincipalExterno} 

	& 0.0
	& 67,41\% & 22,96\% & \bf{90,37\%} & 2,22\% & 7,41\% & \bf{9,63\%} \\
	\cline{3-8}

	& 0.33
	& 70,37\% & 20\% & \bf{90,37\%} & 2,22\% & 7,41\% & \bf{9,63\%} \\
	\cline{3-8}

	& 0.66
	& 62,22\% & 17,04\% & \bf{79,26\%} & 13,33\% & 7,41\% & \bf{20,74\%} \\
	\cline{3-8}

\hline
\multirow{3}{*}{Jurado.MiembroPrincipalInterno} 

	& 0.0
	& 91,85\% & 5,93\% & \bf{97,78\%} & 2,22\% & 0\% & \bf{2,22\%} \\
	\cline{3-8}

	& 0.33
	& 91,11\% & 5,19\% & \bf{96,3\%} & 3,7\% & 0\% & \bf{3,7\%} \\
	\cline{3-8}

	& 0.66
	& 77,78\% & 4,44\% & \bf{82,22\%} & 17,78\% & 0\% & \bf{17,78\%} \\
	\cline{3-8}

\hline
\multirow{3}{*}{Jurado.Presidente} 

	& 0.0
	& 97,78\% & 0\% & \bf{97,78\%} & 2,22\% & 0\% & \bf{2,22\%} \\
	\cline{3-8}

	& 0.33
	& 97,04\% & 0\% & \bf{97,04\%} & 2,96\% & 0\% & \bf{2,96\%} \\
	\cline{3-8}

	& 0.66
	& 82,96\% & 0\% & \bf{82,96\%} & 17,04\% & 0\% & \bf{17,04\%} \\
	\cline{3-8}

\hline
\multirow{3}{*}{Jurado.SuplenteExterno} 

	& 0.0
	& 71,85\% & 22,96\% & \bf{94,81\%} & 2,22\% & 2,96\% & \bf{5,19\%} \\
	\cline{3-8}

	& 0.33
	& 74,81\% & 20\% & \bf{94,81\%} & 2,22\% & 2,96\% & \bf{5,19\%} \\
	\cline{3-8}

	& 0.66
	& 68,89\% & 19,26\% & \bf{88,15\%} & 10,37\% & 1,48\% & \bf{11,85\%} \\
	\cline{3-8}

\hline
\multirow{3}{*}{Jurado.SuplenteInterno} 

	& 0.0
	& 93,33\% & 4,44\% & \bf{97,78\%} & 2,22\% & 0\% & \bf{2,22\%} \\
	\cline{3-8}

	& 0.33
	& 93,33\% & 4,44\% & \bf{97,78\%} & 2,22\% & 0\% & \bf{2,22\%} \\
	\cline{3-8}

	& 0.66
	& 82,96\% & 2,96\% & \bf{85,93\%} & 14,07\% & 0\% & \bf{14,07\%} \\
	\cline{3-8}

\hline
\multirow{3}{*}{Profesor.Departamento} 

	& 0.0
	& 98,52\% & 0,74\% & \bf{99,26\%} & 0,74\% & 0\% & \bf{0,74\%} \\
	\cline{3-8}

	& 0.33
	& 98,52\% & 0\% & \bf{98,52\%} & 1,48\% & 0\% & \bf{1,48\%} \\
	\cline{3-8}

	& 0.66
	& 89,63\% & 0\% & \bf{89,63\%} & 10,37\% & 0\% & \bf{10,37\%} \\
	\cline{3-8}

\hline
\multirow{3}{*}{Profesor.Nombre} 

	& 0.0
	& 97,78\% & 0\% & \bf{97,78\%} & 1,48\% & 0,74\% & \bf{2,22\%} \\
	\cline{3-8}

	& 0.33
	& 95,56\% & 0\% & \bf{95,56\%} & 3,7\% & 0,74\% & \bf{4,44\%} \\
	\cline{3-8}

	& 0.66
	& 78,52\% & 0\% & \bf{78,52\%} & 21,48\% & 0\% & \bf{21,48\%} \\
	\cline{3-8}

\hline
\multirow{3}{*}{Trabajo.Escalafon} 

	& 0.0
	& 99,26\% & 0\% & \bf{99,26\%} & 0,74\% & 0\% & \bf{0,74\%} \\
	\cline{3-8}

	& 0.33
	& 97,78\% & 0\% & \bf{97,78\%} & 2,22\% & 0\% & \bf{2,22\%} \\
	\cline{3-8}

	& 0.66
	& 89,63\% & 0\% & \bf{89,63\%} & 10,37\% & 0\% & \bf{10,37\%} \\
	\cline{3-8}

\hline
\multirow{3}{*}{Trabajo.Nombre} 

	& 0.0
	& 97,78\% & 0\% & \bf{97,78\%} & 1,48\% & 0,74\% & \bf{2,22\%} \\
	\cline{3-8}

	& 0.33
	& 94,81\% & 0\% & \bf{94,81\%} & 4,44\% & 0,74\% & \bf{5,19\%} \\
	\cline{3-8}

	& 0.66
	& 74,07\% & 0\% & \bf{74,07\%} & 25,93\% & 0\% & \bf{25,93\%} \\
	\cline{3-8}

\hline
\end{tabular*}
\label{tabla-resultados-EFJuradosAscenso0.33}
\\
Prob. Campo Faltante es la probabilidad de que no se tenga el valor uno de los campos que se utilizan para hacer extracción focalizada.
\end{table}
\end{landscape}
 
%\chapter{Tablas de resultados de entonación de parámetros de metaheurísticas}\label{chap:apendiceC}

\section{ILS-VND-M}

\subsection{Dethloff}

\begin{table}[h]
\caption{Resultados de la ejecución de la metaheurística ILS-VND-M, utilizando instancias de Dethloff con la configuración -n 5.0 -LS 10.0}
\centering
\small
\begin{tabular}{c c c c c c c}
\hline\hline
Instancia & Costo mínimo & Tiempo(seg.) & Costo promedio & Tiempo promedio(seg.) & Costo ILS & \%Gap \\ [0.5ex]
\hline
SCA3-0 & 640.55 & 0.30 & 
641.97 & 0.29 & \bf{635.62} & 
0.78\\SCA3-1 & 708.40 & 0.38 & 
722.56 & 0.33 & \bf{697.84} & 
1.51\\SCA3-2 & 674.29 & 0.29 & 
697.58 & 0.27 & \bf{659.34} & 
2.27\\SCA3-3 & 712.47 & 0.36 & 
717.69 & 0.32 & \bf{680.04} & 
4.77\\SCA3-4 & \bf{690.50} & 0.30 & 
697.63 & 0.29 & 690.50 & 0.00\\
SCA3-5 & 689.59 & 0.37 & 
701.64 & 0.30 & \bf{659.90} & 
4.50\\SCA3-6 & 696.72 & 0.38 & 
709.08 & 0.26 & \bf{651.09} & 
7.01\\SCA3-7 & 669.89 & 0.39 & 
676.31 & 0.31 & \bf{659.17} & 
1.63\\SCA3-8 & 745.33 & 0.28 & 
754.95 & 0.31 & \bf{719.47} & 
3.59\\SCA3-9 & 711.93 & 0.27 & 
719.45 & 0.32 & \bf{681.00} & 
4.54\\SCA8-0 & 1001.33 & 0.28 & 
1015.17 & 0.28 & \bf{961.50} & 
4.14\\SCA8-1 & 1080.67 & 0.34 & 
1112.55 & 0.27 & \bf{1049.65} & 
2.96\\SCA8-2 & 1100.07 & 0.24 & 
1103.89 & 0.26 & \bf{1039.64} & 
5.81\\SCA8-3 & 1044.10 & 0.36 & 
1053.97 & 0.32 & \bf{983.34} & 
6.18\\SCA8-4 & 1138.23 & 0.40 & 
1153.97 & 0.37 & \bf{1065.49} & 
6.83\\SCA8-5 & 1088.81 & 0.31 & 
1106.23 & 0.39 & \bf{1027.08} & 
6.01\\SCA8-6 & 1021.99 & 0.27 & 
1036.04 & 0.30 & \bf{971.82} & 
5.16\\SCA8-7 & 1114.68 & 0.31 & 
1127.69 & 0.27 & \bf{1051.28} & 
6.03\\SCA8-8 & \bf{1071.18} & 0.24 & 
1130.81 & 0.26 & 1071.18 & 0.00\\
SCA8-9 & 1114.34 & 0.26 & 
1133.72 & 0.25 & \bf{1060.50} & 
5.08\\CON3-0 & 650.16 & 0.31 & 
669.65 & 0.33 & \bf{616.52} & 
5.46\\CON3-1 & 568.45 & 0.38 & 
573.28 & 0.28 & \bf{554.47} & 
2.52\\CON3-2 & 521.63 & 0.39 & 
540.90 & 0.38 & \bf{518.00} & 
0.70\\CON3-3 & 611.33 & 0.30 & 
635.93 & 0.27 & \bf{591.19} & 
3.41\\CON3-4 & 611.21 & 0.33 & 
628.71 & 0.33 & \bf{588.79} & 
3.81\\CON3-5 & 569.04 & 0.31 & 
572.26 & 0.27 & \bf{563.70} & 
0.95\\CON3-6 & 511.00 & 0.30 & 
529.03 & 0.39 & \bf{499.05} & 
2.39\\CON3-7 & 618.48 & 0.36 & 
630.72 & 0.30 & \bf{576.48} & 
7.29\\CON3-8 & 546.26 & 0.46 & 
566.07 & 0.39 & \bf{523.05} & 
4.44\\CON3-9 & 599.04 & 0.31 & 
599.84 & 0.30 & \bf{578.24} & 
3.60\\CON8-0 & 919.96 & 0.23 & 
934.16 & 0.23 & \bf{857.17} & 
7.33\\CON8-1 & 777.08 & 0.47 & 
815.24 & 0.37 & \bf{740.85} & 
4.89\\CON8-2 & 776.70 & 0.26 & 
787.67 & 0.26 & \bf{712.89} & 
8.95\\CON8-3 & 857.98 & 0.24 & 
858.49 & 0.27 & \bf{811.07} & 
5.78\\CON8-4 & 825.32 & 0.44 & 
848.55 & 0.36 & \bf{772.25} & 
6.87\\CON8-5 & 825.97 & 0.26 & 
832.06 & 0.33 & \bf{754.88} & 
9.42\\CON8-6 & 724.97 & 0.29 & 
741.21 & 0.27 & \bf{678.92} & 
6.78\\CON8-7 & 839.18 & 0.24 & 
868.98 & 0.34 & \bf{811.96} & 
3.35\\CON8-8 & 794.22 & 0.28 & 
809.92 & 0.29 & \bf{767.53} & 
3.48\\CON8-9 & 825.74 & 0.28 & 
851.28 & 0.40 & \bf{809.00} & 
2.07\\\bf{PROM.} & 
\bf{792.22} & \bf{0.32} & \bf{807.67} & \bf{0.31} & \bf{758.54} & \bf{4.31}\\[1ex]\hline
\end{tabular}
\label{table:ILS-VND-M-5-10}
\end{table}

\begin{table}[h]
\caption{Resultados de la ejecución de la metaheurística ILS-VND-M, utilizando instancias de Dethloff con la configuración -n 5.0 -LS 40.0}
\centering
\small
\begin{tabular}{c c c c c c c}
\hline\hline
Instancia & Costo mínimo & Tiempo(seg.) & Costo promedio & Tiempo promedio(seg.) & Costo ILS & \%Gap \\ [0.5ex]
\hline
SCA3-0 & 640.55 & 1.25 & 
643.75 & 1.08 & \bf{635.62} & 
0.78\\SCA3-1 & 712.59 & 1.06 & 
729.38 & 0.87 & \bf{697.84} & 
2.11\\SCA3-2 & 680.00 & 0.84 & 
700.79 & 0.72 & \bf{659.34} & 
3.13\\SCA3-3 & 685.47 & 1.06 & 
687.22 & 0.94 & \bf{680.04} & 
0.80\\SCA3-4 & \bf{690.50} & 0.93 & 
706.32 & 0.84 & 690.50 & 0.00\\
SCA3-5 & 686.44 & 0.52 & 
686.53 & 0.78 & \bf{659.90} & 
4.02\\SCA3-6 & 654.79 & 0.91 & 
666.88 & 0.73 & \bf{651.09} & 
0.57\\SCA3-7 & 669.89 & 0.68 & 
672.23 & 0.76 & \bf{659.17} & 
1.63\\SCA3-8 & 726.44 & 0.78 & 
750.00 & 0.79 & \bf{719.47} & 
0.97\\SCA3-9 & 683.57 & 1.22 & 
699.63 & 0.96 & \bf{681.00} & 
0.38\\SCA8-0 & 1018.78 & 0.68 & 
1023.67 & 0.63 & \bf{961.50} & 
5.96\\SCA8-1 & 1104.59 & 0.61 & 
1141.33 & 0.62 & \bf{1049.65} & 
5.23\\SCA8-2 & 1066.96 & 0.79 & 
1105.54 & 0.57 & \bf{1039.64} & 
2.63\\SCA8-3 & 1049.88 & 0.80 & 
1087.59 & 0.66 & \bf{983.34} & 
6.77\\SCA8-4 & 1109.30 & 0.73 & 
1119.46 & 0.79 & \bf{1065.49} & 
4.11\\SCA8-5 & 1064.55 & 0.72 & 
1068.26 & 0.89 & \bf{1027.08} & 
3.65\\SCA8-6 & 1026.04 & 0.64 & 
1027.85 & 0.65 & \bf{971.82} & 
5.58\\SCA8-7 & 1110.98 & 1.02 & 
1138.58 & 0.74 & \bf{1051.28} & 
5.68\\SCA8-8 & 1103.10 & 0.53 & 
1139.92 & 0.53 & \bf{1071.18} & 
2.98\\SCA8-9 & 1084.48 & 0.74 & 
1137.56 & 0.60 & \bf{1060.50} & 
2.26\\CON3-0 & 661.33 & 0.72 & 
665.72 & 0.81 & \bf{616.52} & 
7.27\\CON3-1 & 561.63 & 1.04 & 
575.74 & 0.86 & \bf{554.47} & 
1.29\\CON3-2 & 523.99 & 0.87 & 
526.53 & 0.76 & \bf{518.00} & 
1.16\\CON3-3 & 591.20 & 1.18 & 
614.17 & 0.94 & \bf{591.19} & 
0.00\\CON3-4 & 620.15 & 0.79 & 
636.84 & 0.77 & \bf{588.79} & 
5.33\\CON3-5 & 571.63 & 0.83 & 
589.09 & 0.86 & \bf{563.70} & 
1.41\\CON3-6 & 513.13 & 0.85 & 
519.79 & 0.94 & \bf{499.05} & 
2.82\\CON3-7 & 604.95 & 1.23 & 
616.76 & 0.89 & \bf{576.48} & 
4.94\\CON3-8 & 529.65 & 0.88 & 
541.35 & 0.90 & \bf{523.05} & 
1.26\\CON3-9 & 590.64 & 1.01 & 
596.36 & 0.93 & \bf{578.24} & 
2.14\\CON8-0 & 923.06 & 0.68 & 
956.84 & 0.64 & \bf{857.17} & 
7.69\\CON8-1 & 777.60 & 1.00 & 
789.87 & 0.88 & \bf{740.85} & 
4.96\\CON8-2 & 729.26 & 1.09 & 
746.58 & 0.81 & \bf{712.89} & 
2.30\\CON8-3 & 851.97 & 0.99 & 
863.30 & 0.69 & \bf{811.07} & 
5.04\\CON8-4 & 824.36 & 0.71 & 
868.05 & 0.61 & \bf{772.25} & 
6.75\\CON8-5 & 768.63 & 0.55 & 
788.72 & 0.61 & \bf{754.88} & 
1.82\\CON8-6 & 699.30 & 0.63 & 
714.86 & 0.63 & \bf{678.92} & 
3.00\\CON8-7 & 857.87 & 0.61 & 
882.75 & 0.57 & \bf{811.96} & 
5.65\\CON8-8 & 800.37 & 0.64 & 
826.65 & 0.63 & \bf{767.53} & 
4.28\\CON8-9 & 851.07 & 0.72 & 
865.30 & 0.67 & \bf{809.00} & 
5.20\\\bf{PROM.} & 
\bf{785.52} & \bf{0.84} & \bf{802.94} & \bf{0.76} & \bf{758.54} & \bf{3.34}\\[1ex]\hline
\end{tabular}
\label{table:ILS-VND-M-5-40}
\end{table}

\begin{table}[h]
\caption{Resultados de la ejecución de la metaheurística ILS-VND-M, utilizando instancias de Dethloff con la configuración -n 5.0 -LS 70.0}
\centering
\small
\begin{tabular}{c c c c c c c}
\hline\hline
Instancia & Costo mínimo & Tiempo(seg.) & Costo promedio & Tiempo promedio(seg.) & Costo ILS & \%Gap \\ [0.5ex]
\hline
SCA3-0 & 640.55 & 1.36 & 
642.46 & 1.45 & \bf{635.62} & 
0.78\\SCA3-1 & 712.78 & 1.71 & 
718.16 & 1.40 & \bf{697.84} & 
2.14\\SCA3-2 & 677.10 & 1.18 & 
682.54 & 1.21 & \bf{659.34} & 
2.69\\SCA3-3 & 684.67 & 1.26 & 
696.03 & 1.09 & \bf{680.04} & 
0.68\\SCA3-4 & 693.23 & 1.35 & 
705.92 & 1.20 & \bf{690.50} & 
0.40\\SCA3-5 & 662.75 & 1.59 & 
677.01 & 1.83 & \bf{659.90} & 
0.43\\SCA3-6 & 661.28 & 1.29 & 
665.53 & 1.35 & \bf{651.09} & 
1.57\\SCA3-7 & 671.77 & 1.31 & 
680.79 & 1.15 & \bf{659.17} & 
1.91\\SCA3-8 & 726.86 & 2.04 & 
736.71 & 1.52 & \bf{719.47} & 
1.03\\SCA3-9 & 690.07 & 1.42 & 
692.78 & 1.23 & \bf{681.00} & 
1.33\\SCA8-0 & 1007.80 & 1.00 & 
1019.70 & 1.18 & \bf{961.50} & 
4.82\\SCA8-1 & 1101.06 & 1.14 & 
1127.76 & 0.90 & \bf{1049.65} & 
4.90\\SCA8-2 & 1091.71 & 1.00 & 
1119.01 & 0.84 & \bf{1039.64} & 
5.01\\SCA8-3 & 1036.79 & 0.84 & 
1048.63 & 0.95 & \bf{983.34} & 
5.44\\SCA8-4 & 1120.29 & 0.85 & 
1145.07 & 0.95 & \bf{1065.49} & 
5.14\\SCA8-5 & 1090.77 & 1.38 & 
1104.89 & 1.10 & \bf{1027.08} & 
6.20\\SCA8-6 & 1022.17 & 1.25 & 
1039.74 & 1.14 & \bf{971.82} & 
5.18\\SCA8-7 & 1081.50 & 0.96 & 
1095.93 & 1.33 & \bf{1051.28} & 
2.87\\SCA8-8 & 1099.22 & 1.46 & 
1124.08 & 1.13 & \bf{1071.18} & 
2.62\\SCA8-9 & 1126.82 & 1.39 & 
1145.95 & 1.06 & \bf{1060.50} & 
6.25\\CON3-0 & 633.24 & 1.22 & 
650.21 & 1.34 & \bf{616.52} & 
2.71\\CON3-1 & 568.89 & 1.62 & 
576.41 & 1.34 & \bf{554.47} & 
2.60\\CON3-2 & 521.38 & 1.32 & 
525.77 & 1.45 & \bf{518.00} & 
0.65\\CON3-3 & 594.31 & 1.28 & 
608.52 & 1.45 & \bf{591.19} & 
0.53\\CON3-4 & 605.94 & 1.79 & 
612.84 & 1.55 & \bf{588.79} & 
2.91\\CON3-5 & 568.69 & 1.00 & 
585.65 & 1.09 & \bf{563.70} & 
0.89\\CON3-6 & 516.86 & 1.35 & 
524.24 & 1.34 & \bf{499.05} & 
3.57\\CON3-7 & 599.51 & 1.00 & 
609.58 & 1.18 & \bf{576.48} & 
3.99\\CON3-8 & 526.59 & 1.52 & 
535.61 & 1.49 & \bf{523.05} & 
0.68\\CON3-9 & 591.24 & 1.74 & 
599.07 & 1.54 & \bf{578.24} & 
2.25\\CON8-0 & 907.51 & 1.12 & 
931.72 & 1.25 & \bf{857.17} & 
5.87\\CON8-1 & 786.81 & 0.98 & 
794.82 & 0.91 & \bf{740.85} & 
6.20\\CON8-2 & 727.20 & 1.18 & 
764.16 & 1.00 & \bf{712.89} & 
2.01\\CON8-3 & 850.32 & 1.03 & 
853.11 & 1.19 & \bf{811.07} & 
4.84\\CON8-4 & 827.58 & 0.72 & 
844.98 & 0.99 & \bf{772.25} & 
7.16\\CON8-5 & 785.71 & 0.92 & 
814.90 & 1.08 & \bf{754.88} & 
4.08\\CON8-6 & 702.18 & 1.22 & 
719.81 & 1.19 & \bf{678.92} & 
3.43\\CON8-7 & 842.21 & 1.27 & 
865.39 & 0.93 & \bf{811.96} & 
3.73\\CON8-8 & 790.87 & 1.02 & 
819.07 & 0.92 & \bf{767.53} & 
3.04\\CON8-9 & 815.58 & 1.51 & 
872.30 & 1.02 & \bf{809.00} & 
0.81\\\bf{PROM.} & 
\bf{784.05} & \bf{1.26} & \bf{799.42} & \bf{1.21} & \bf{758.54} & \bf{3.08}\\[1ex]\hline
\end{tabular}
\label{table:ILS-VND-M-5-70}
\end{table}

\begin{table}[h]
\caption{Resultados de la ejecución de la metaheurística ILS-VND-M, utilizando instancias de Dethloff con la configuración -n 25.0 -LS 10.0}
\centering
\small
\begin{tabular}{c c c c c c c}
\hline\hline
Instancia & Costo mínimo & Tiempo(seg.) & Costo promedio & Tiempo promedio(seg.) & Costo ILS & \%Gap \\ [0.5ex]
\hline
SCA3-0 & 640.55 & 1.60 & 
643.19 & 1.53 & \bf{635.62} & 
0.78\\SCA3-1 & 700.50 & 2.12 & 
726.25 & 1.63 & \bf{697.84} & 
0.38\\SCA3-2 & 675.12 & 1.42 & 
683.29 & 1.53 & \bf{659.34} & 
2.39\\SCA3-3 & 682.46 & 1.54 & 
688.54 & 1.49 & \bf{680.04} & 
0.36\\SCA3-4 & \bf{690.50} & 1.38 & 
701.73 & 1.44 & 690.50 & 0.00\\
SCA3-5 & \bf{659.90} & 1.78 & 
673.82 & 1.62 & 659.90 & 0.00\\
SCA3-6 & 652.94 & 1.67 & 
655.67 & 1.55 & \bf{651.09} & 
0.28\\SCA3-7 & 672.85 & 1.42 & 
673.85 & 1.54 & \bf{659.17} & 
2.08\\SCA3-8 & \bf{719.47} & 1.52 & 
733.67 & 1.65 & 719.47 & 0.00\\
SCA3-9 & 690.07 & 1.58 & 
700.76 & 1.62 & \bf{681.00} & 
1.33\\SCA8-0 & 1015.61 & 1.98 & 
1027.39 & 1.87 & \bf{961.50} & 
5.63\\SCA8-1 & 1097.18 & 1.29 & 
1117.42 & 1.26 & \bf{1049.65} & 
4.53\\SCA8-2 & 1092.94 & 1.69 & 
1102.00 & 1.41 & \bf{1039.64} & 
5.13\\SCA8-3 & 1025.40 & 1.03 & 
1044.06 & 1.36 & \bf{983.34} & 
4.28\\SCA8-4 & 1121.03 & 1.63 & 
1156.94 & 1.47 & \bf{1065.49} & 
5.21\\SCA8-5 & 1071.46 & 1.36 & 
1091.91 & 1.50 & \bf{1027.08} & 
4.32\\SCA8-6 & 1013.70 & 1.62 & 
1025.38 & 1.25 & \bf{971.82} & 
4.31\\SCA8-7 & 1083.63 & 1.21 & 
1101.53 & 1.32 & \bf{1051.28} & 
3.08\\SCA8-8 & 1114.83 & 1.36 & 
1120.43 & 1.32 & \bf{1071.18} & 
4.07\\SCA8-9 & 1102.71 & 1.18 & 
1118.88 & 1.22 & \bf{1060.50} & 
3.98\\CON3-0 & 636.77 & 1.72 & 
639.11 & 1.68 & \bf{616.52} & 
3.28\\CON3-1 & 564.81 & 1.46 & 
566.50 & 1.46 & \bf{554.47} & 
1.86\\CON3-2 & 523.23 & 1.45 & 
529.21 & 1.70 & \bf{518.00} & 
1.01\\CON3-3 & 594.31 & 1.34 & 
613.26 & 1.34 & \bf{591.19} & 
0.53\\CON3-4 & 603.13 & 1.98 & 
609.41 & 1.61 & \bf{588.79} & 
2.44\\CON3-5 & 569.04 & 1.52 & 
586.28 & 1.57 & \bf{563.70} & 
0.95\\CON3-6 & 502.95 & 2.05 & 
507.77 & 1.80 & \bf{499.05} & 
0.78\\CON3-7 & 600.35 & 1.54 & 
607.71 & 1.48 & \bf{576.48} & 
4.14\\CON3-8 & 528.59 & 1.38 & 
533.48 & 1.61 & \bf{523.05} & 
1.06\\CON3-9 & 588.99 & 1.71 & 
596.03 & 1.38 & \bf{578.24} & 
1.86\\CON8-0 & 898.15 & 1.76 & 
928.87 & 1.43 & \bf{857.17} & 
4.78\\CON8-1 & 763.10 & 2.39 & 
772.95 & 1.82 & \bf{740.85} & 
3.00\\CON8-2 & 731.70 & 1.88 & 
737.10 & 1.66 & \bf{712.89} & 
2.64\\CON8-3 & 836.03 & 1.34 & 
850.09 & 1.31 & \bf{811.07} & 
3.08\\CON8-4 & 802.74 & 1.38 & 
824.89 & 1.43 & \bf{772.25} & 
3.95\\CON8-5 & 776.35 & 1.65 & 
802.90 & 1.55 & \bf{754.88} & 
2.84\\CON8-6 & 710.16 & 1.58 & 
718.62 & 1.39 & \bf{678.92} & 
4.60\\CON8-7 & 841.49 & 1.34 & 
844.21 & 1.36 & \bf{811.96} & 
3.64\\CON8-8 & 778.39 & 1.60 & 
798.16 & 1.52 & \bf{767.53} & 
1.41\\CON8-9 & 840.92 & 1.47 & 
865.62 & 1.57 & \bf{809.00} & 
3.95\\\bf{PROM.} & 
\bf{780.35} & \bf{1.57} & \bf{792.97} & \bf{1.51} & \bf{758.54} & \bf{2.60}\\[1ex]\hline
\end{tabular}
\label{table:ILS-VND-M-25-10}
\end{table}

\begin{table}[h]
\caption{Resultados de la ejecución de la metaheurística ILS-VND-M, utilizando instancias de Dethloff con la configuración -n 25.0 -LS 40.0}
\centering
\small
\begin{tabular}{c c c c c c c}
\hline\hline
Instancia & Costo mínimo & Tiempo(seg.) & Costo promedio & Tiempo promedio(seg.) & Costo ILS & \%Gap \\ [0.5ex]
\hline
SCA3-0 & 641.69 & 4.63 & 
642.63 & 4.25 & \bf{635.62} & 
0.95\\SCA3-1 & 700.50 & 4.10 & 
709.55 & 4.01 & \bf{697.84} & 
0.38\\SCA3-2 & 664.21 & 4.28 & 
672.85 & 3.95 & \bf{659.34} & 
0.74\\SCA3-3 & 681.74 & 3.90 & 
685.25 & 3.65 & \bf{680.04} & 
0.25\\SCA3-4 & \bf{690.50} & 3.94 & 
692.22 & 4.13 & 690.50 & 0.00\\
SCA3-5 & 680.80 & 4.25 & 
686.26 & 4.26 & \bf{659.90} & 
3.17\\SCA3-6 & 652.47 & 3.46 & 
654.19 & 3.58 & \bf{651.09} & 
0.21\\SCA3-7 & 671.77 & 3.50 & 
676.11 & 3.55 & \bf{659.17} & 
1.91\\SCA3-8 & \bf{719.47} & 3.61 & 
730.11 & 3.52 & 719.47 & 0.00\\
SCA3-9 & \bf{681.00} & 4.48 & 
690.88 & 4.03 & 681.00 & 0.00\\
SCA8-0 & 981.73 & 3.74 & 
1005.63 & 3.87 & \bf{961.50} & 
2.10\\SCA8-1 & 1085.54 & 3.06 & 
1086.60 & 3.27 & \bf{1049.65} & 
3.42\\SCA8-2 & 1062.62 & 5.10 & 
1067.91 & 3.54 & \bf{1039.64} & 
2.21\\SCA8-3 & 1032.69 & 3.24 & 
1040.32 & 3.33 & \bf{983.34} & 
5.02\\SCA8-4 & 1074.28 & 3.17 & 
1112.38 & 3.25 & \bf{1065.49} & 
0.82\\SCA8-5 & 1049.98 & 3.55 & 
1072.99 & 3.35 & \bf{1027.08} & 
2.23\\SCA8-6 & 991.49 & 3.53 & 
998.24 & 2.87 & \bf{971.82} & 
2.02\\SCA8-7 & 1066.65 & 4.56 & 
1093.26 & 3.54 & \bf{1051.28} & 
1.46\\SCA8-8 & 1088.20 & 3.50 & 
1104.36 & 3.13 & \bf{1071.18} & 
1.59\\SCA8-9 & 1119.56 & 3.74 & 
1130.42 & 3.09 & \bf{1060.50} & 
5.57\\CON3-0 & 617.59 & 4.37 & 
631.20 & 4.28 & \bf{616.52} & 
0.17\\CON3-1 & 560.61 & 3.27 & 
564.99 & 3.62 & \bf{554.47} & 
1.11\\CON3-2 & 524.13 & 4.51 & 
528.31 & 4.25 & \bf{518.00} & 
1.18\\CON3-3 & 591.48 & 4.85 & 
600.22 & 4.17 & \bf{591.19} & 
0.05\\CON3-4 & 599.13 & 4.19 & 
601.95 & 4.05 & \bf{588.79} & 
1.76\\CON3-5 & 569.04 & 4.48 & 
574.53 & 4.47 & \bf{563.70} & 
0.95\\CON3-6 & 505.14 & 4.36 & 
511.65 & 4.10 & \bf{499.05} & 
1.22\\CON3-7 & 586.53 & 3.84 & 
598.45 & 3.92 & \bf{576.48} & 
1.74\\CON3-8 & 524.59 & 4.96 & 
532.31 & 4.20 & \bf{523.05} & 
0.29\\CON3-9 & 589.57 & 3.64 & 
591.51 & 4.06 & \bf{578.24} & 
1.96\\CON8-0 & 872.79 & 3.52 & 
894.27 & 3.54 & \bf{857.17} & 
1.82\\CON8-1 & 773.48 & 3.21 & 
777.26 & 3.63 & \bf{740.85} & 
4.40\\CON8-2 & 722.56 & 3.74 & 
732.73 & 3.62 & \bf{712.89} & 
1.36\\CON8-3 & 845.27 & 3.21 & 
848.23 & 3.18 & \bf{811.07} & 
4.22\\CON8-4 & 813.27 & 2.85 & 
825.95 & 2.63 & \bf{772.25} & 
5.31\\CON8-5 & 767.45 & 4.07 & 
775.88 & 3.61 & \bf{754.88} & 
1.67\\CON8-6 & 703.00 & 3.05 & 
710.01 & 3.77 & \bf{678.92} & 
3.55\\CON8-7 & 816.07 & 4.33 & 
829.87 & 3.35 & \bf{811.96} & 
0.51\\CON8-8 & 787.24 & 4.46 & 
800.91 & 3.83 & \bf{767.53} & 
2.57\\CON8-9 & 823.94 & 4.56 & 
849.08 & 4.19 & \bf{809.00} & 
1.85\\\bf{PROM.} & 
\bf{773.24} & \bf{3.92} & \bf{783.29} & \bf{3.72} & \bf{758.54} & \bf{1.79}\\[1ex]\hline
\end{tabular}
\label{table:ILS-VND-M-25-40}
\end{table}

\begin{table}[h]
\caption{Resultados de la ejecución de la metaheurística ILS-VND-M, utilizando instancias de Dethloff con la configuración -n 25.0 -LS 70.0}
\centering
\small
\begin{tabular}{c c c c c c c}
\hline\hline
Instancia & Costo mínimo & Tiempo(seg.) & Costo promedio & Tiempo promedio(seg.) & Costo ILS & \%Gap \\ [0.5ex]
\hline
SCA3-0 & 640.55 & 6.50 & 
642.00 & 6.48 & \bf{635.62} & 
0.78\\SCA3-1 & 700.50 & 5.67 & 
701.34 & 6.39 & \bf{697.84} & 
0.38\\SCA3-2 & 668.28 & 6.17 & 
676.70 & 6.05 & \bf{659.34} & 
1.36\\SCA3-3 & 681.74 & 6.30 & 
684.20 & 6.52 & \bf{680.04} & 
0.25\\SCA3-4 & \bf{690.50} & 6.95 & 
691.53 & 6.39 & 690.50 & 0.00\\
SCA3-5 & 661.07 & 6.88 & 
676.80 & 6.29 & \bf{659.90} & 
0.18\\SCA3-6 & \bf{651.09} & 5.87 & 
652.81 & 6.37 & 651.09 & 0.00\\
SCA3-7 & 671.77 & 5.99 & 
672.83 & 6.12 & \bf{659.17} & 
1.91\\SCA3-8 & 727.64 & 6.96 & 
731.02 & 6.23 & \bf{719.47} & 
1.14\\SCA3-9 & \bf{681.00} & 6.69 & 
686.16 & 6.22 & 681.00 & 0.00\\
SCA8-0 & 1004.83 & 6.18 & 
1015.99 & 5.32 & \bf{961.50} & 
4.51\\SCA8-1 & 1066.73 & 5.42 & 
1079.99 & 5.66 & \bf{1049.65} & 
1.63\\SCA8-2 & 1062.47 & 6.14 & 
1072.18 & 5.41 & \bf{1039.64} & 
2.20\\SCA8-3 & 1011.09 & 5.51 & 
1022.09 & 4.94 & \bf{983.34} & 
2.82\\SCA8-4 & 1096.60 & 6.17 & 
1102.88 & 5.68 & \bf{1065.49} & 
2.92\\SCA8-5 & 1065.60 & 5.54 & 
1073.88 & 5.49 & \bf{1027.08} & 
3.75\\SCA8-6 & 993.19 & 5.53 & 
1001.61 & 5.12 & \bf{971.82} & 
2.20\\SCA8-7 & 1089.45 & 4.63 & 
1102.95 & 5.19 & \bf{1051.28} & 
3.63\\SCA8-8 & 1092.82 & 6.61 & 
1107.96 & 5.33 & \bf{1071.18} & 
2.02\\SCA8-9 & 1081.51 & 5.34 & 
1093.93 & 5.40 & \bf{1060.50} & 
1.98\\CON3-0 & 620.76 & 5.70 & 
631.13 & 5.95 & \bf{616.52} & 
0.69\\CON3-1 & 556.04 & 6.20 & 
557.55 & 6.08 & \bf{554.47} & 
0.28\\CON3-2 & 521.63 & 6.66 & 
528.12 & 6.38 & \bf{518.00} & 
0.70\\CON3-3 & 599.26 & 6.28 & 
604.55 & 6.48 & \bf{591.19} & 
1.37\\CON3-4 & 591.43 & 6.60 & 
597.55 & 6.65 & \bf{588.79} & 
0.45\\CON3-5 & 573.23 & 7.94 & 
576.64 & 6.24 & \bf{563.70} & 
1.69\\CON3-6 & 504.20 & 7.02 & 
506.30 & 6.50 & \bf{499.05} & 
1.03\\CON3-7 & 591.91 & 5.98 & 
599.38 & 6.80 & \bf{576.48} & 
2.68\\CON3-8 & 524.59 & 5.91 & 
534.28 & 6.33 & \bf{523.05} & 
0.29\\CON3-9 & 590.50 & 7.40 & 
592.07 & 6.83 & \bf{578.24} & 
2.12\\CON8-0 & 888.34 & 5.03 & 
900.55 & 5.09 & \bf{857.17} & 
3.64\\CON8-1 & 761.69 & 4.48 & 
774.60 & 4.75 & \bf{740.85} & 
2.81\\CON8-2 & 725.72 & 6.17 & 
734.64 & 5.49 & \bf{712.89} & 
1.80\\CON8-3 & 822.12 & 5.98 & 
837.72 & 6.23 & \bf{811.07} & 
1.36\\CON8-4 & 781.78 & 6.84 & 
811.40 & 5.36 & \bf{772.25} & 
1.23\\CON8-5 & 769.55 & 4.99 & 
781.85 & 5.52 & \bf{754.88} & 
1.94\\CON8-6 & 699.04 & 5.97 & 
704.47 & 5.50 & \bf{678.92} & 
2.96\\CON8-7 & 817.70 & 5.43 & 
827.51 & 6.01 & \bf{811.96} & 
0.71\\CON8-8 & 783.97 & 5.09 & 
799.11 & 5.12 & \bf{767.53} & 
2.14\\CON8-9 & 815.52 & 7.19 & 
835.05 & 5.58 & \bf{809.00} & 
0.81\\\bf{PROM.} & 
\bf{771.94} & \bf{6.10} & \bf{780.58} & \bf{5.89} & \bf{758.54} & \bf{1.61}\\[1ex]\hline
\end{tabular}
\label{table:ILS-VND-M-25-70}
\end{table}

\begin{table}[h]
\caption{Resultados de la ejecución de la metaheurística ILS-VND-M, utilizando instancias de Dethloff con la configuración -n 55.0 -LS 10.0}
\centering
\small
\begin{tabular}{c c c c c c c}
\hline\hline
Instancia & Costo mínimo & Tiempo(seg.) & Costo promedio & Tiempo promedio(seg.) & Costo ILS & \%Gap \\ [0.5ex]
\hline
SCA3-0 & 640.55 & 4.11 & 
644.64 & 3.42 & \bf{635.62} & 
0.78\\SCA3-1 & \bf{697.84} & 3.50 & 
703.73 & 3.81 & 697.84 & 0.00\\
SCA3-2 & 661.13 & 3.57 & 
667.09 & 3.41 & \bf{659.34} & 
0.27\\SCA3-3 & 680.60 & 3.58 & 
681.16 & 3.68 & \bf{680.04} & 
0.08\\SCA3-4 & \bf{690.50} & 4.93 & 
700.65 & 3.58 & 690.50 & 0.00\\
SCA3-5 & 673.39 & 3.49 & 
677.86 & 3.65 & \bf{659.90} & 
2.04\\SCA3-6 & 652.94 & 2.92 & 
656.54 & 3.15 & \bf{651.09} & 
0.28\\SCA3-7 & 671.67 & 3.56 & 
671.86 & 3.32 & \bf{659.17} & 
1.90\\SCA3-8 & 719.77 & 3.36 & 
723.80 & 3.38 & \bf{719.47} & 
0.04\\SCA3-9 & 685.00 & 3.43 & 
688.84 & 3.44 & \bf{681.00} & 
0.59\\SCA8-0 & 977.93 & 3.46 & 
1015.19 & 3.50 & \bf{961.50} & 
1.71\\SCA8-1 & 1081.72 & 3.30 & 
1091.99 & 3.13 & \bf{1049.65} & 
3.06\\SCA8-2 & 1065.43 & 3.08 & 
1071.13 & 2.87 & \bf{1039.64} & 
2.48\\SCA8-3 & 1030.78 & 3.44 & 
1056.09 & 3.18 & \bf{983.34} & 
4.82\\SCA8-4 & 1095.76 & 2.78 & 
1127.76 & 2.86 & \bf{1065.49} & 
2.84\\SCA8-5 & 1078.65 & 3.86 & 
1092.81 & 3.31 & \bf{1027.08} & 
5.02\\SCA8-6 & 1005.01 & 3.79 & 
1022.01 & 3.42 & \bf{971.82} & 
3.42\\SCA8-7 & 1088.59 & 2.77 & 
1100.14 & 2.92 & \bf{1051.28} & 
3.55\\SCA8-8 & 1097.68 & 3.15 & 
1104.15 & 3.08 & \bf{1071.18} & 
2.47\\SCA8-9 & 1104.83 & 2.53 & 
1124.07 & 2.92 & \bf{1060.50} & 
4.18\\CON3-0 & 625.35 & 3.64 & 
633.19 & 3.58 & \bf{616.52} & 
1.43\\CON3-1 & 561.63 & 3.56 & 
563.12 & 3.64 & \bf{554.47} & 
1.29\\CON3-2 & 521.38 & 4.09 & 
528.34 & 3.50 & \bf{518.00} & 
0.65\\CON3-3 & 594.31 & 3.43 & 
606.44 & 3.39 & \bf{591.19} & 
0.53\\CON3-4 & 591.43 & 3.64 & 
608.94 & 3.85 & \bf{588.79} & 
0.45\\CON3-5 & 569.04 & 4.24 & 
572.44 & 3.81 & \bf{563.70} & 
0.95\\CON3-6 & 511.05 & 3.74 & 
514.37 & 3.45 & \bf{499.05} & 
2.40\\CON3-7 & 576.87 & 3.21 & 
584.77 & 3.25 & \bf{576.48} & 
0.07\\CON3-8 & \bf{523.05} & 3.39 & 
531.55 & 3.42 & 523.05 & 0.00\\
CON3-9 & 590.50 & 4.64 & 
592.45 & 3.74 & \bf{578.24} & 
2.12\\CON8-0 & 893.61 & 2.79 & 
903.69 & 2.86 & \bf{857.17} & 
4.25\\CON8-1 & 763.99 & 3.64 & 
768.45 & 3.40 & \bf{740.85} & 
3.12\\CON8-2 & 731.85 & 3.85 & 
739.77 & 3.60 & \bf{712.89} & 
2.66\\CON8-3 & 833.78 & 3.62 & 
840.26 & 3.41 & \bf{811.07} & 
2.80\\CON8-4 & 795.93 & 3.24 & 
808.00 & 3.21 & \bf{772.25} & 
3.07\\CON8-5 & 772.89 & 3.51 & 
776.65 & 3.23 & \bf{754.88} & 
2.39\\CON8-6 & 699.79 & 3.42 & 
707.08 & 3.23 & \bf{678.92} & 
3.07\\CON8-7 & 815.60 & 2.99 & 
839.10 & 3.34 & \bf{811.96} & 
0.45\\CON8-8 & 784.36 & 3.70 & 
801.54 & 3.12 & \bf{767.53} & 
2.19\\CON8-9 & 838.80 & 3.33 & 
842.97 & 3.44 & \bf{809.00} & 
3.68\\\bf{PROM.} & 
\bf{774.87} & \bf{3.51} & \bf{784.62} & \bf{3.36} & \bf{758.54} & \bf{1.93}\\[1ex]\hline
\end{tabular}
\label{table:ILS-VND-M-55-10}
\end{table}

\begin{table}[h]
\caption{Resultados de la ejecución de la metaheurística ILS-VND-M, utilizando instancias de Dethloff con la configuración -n 55.0 -LS 40.0}
\centering
\small
\begin{tabular}{c c c c c c c}
\hline\hline
Instancia & Costo mínimo & Tiempo(seg.) & Costo promedio & Tiempo promedio(seg.) & Costo ILS & \%Gap \\ [0.5ex]
\hline
SCA3-0 & 640.55 & 8.78 & 
643.15 & 8.81 & \bf{635.62} & 
0.78\\SCA3-1 & 700.50 & 9.12 & 
704.36 & 8.38 & \bf{697.84} & 
0.38\\SCA3-2 & 664.18 & 10.02 & 
668.66 & 8.59 & \bf{659.34} & 
0.73\\SCA3-3 & \bf{680.04} & 7.87 & 
681.21 & 7.90 & 680.04 & 0.00\\
SCA3-4 & \bf{690.50} & 7.56 & 
697.88 & 7.89 & 690.50 & 0.00\\
SCA3-5 & 666.67 & 8.84 & 
673.11 & 9.06 & \bf{659.90} & 
1.03\\SCA3-6 & \bf{651.09} & 8.83 & 
654.19 & 8.28 & 651.09 & 0.00\\
SCA3-7 & 671.67 & 8.18 & 
671.72 & 8.79 & \bf{659.17} & 
1.90\\SCA3-8 & 719.77 & 7.76 & 
723.08 & 8.57 & \bf{719.47} & 
0.04\\SCA3-9 & \bf{681.00} & 9.16 & 
682.65 & 8.60 & 681.00 & 0.00\\
SCA8-0 & 1013.72 & 6.68 & 
1024.13 & 7.82 & \bf{961.50} & 
5.43\\SCA8-1 & 1062.35 & 6.65 & 
1083.20 & 6.86 & \bf{1049.65} & 
1.21\\SCA8-2 & 1055.65 & 6.85 & 
1066.67 & 6.75 & \bf{1039.64} & 
1.54\\SCA8-3 & 1017.97 & 6.73 & 
1025.57 & 7.32 & \bf{983.34} & 
3.52\\SCA8-4 & 1081.36 & 6.94 & 
1089.46 & 6.54 & \bf{1065.49} & 
1.49\\SCA8-5 & 1082.58 & 7.09 & 
1088.39 & 7.20 & \bf{1027.08} & 
5.40\\SCA8-6 & 981.41 & 6.97 & 
994.52 & 7.22 & \bf{971.82} & 
0.99\\SCA8-7 & 1082.76 & 7.40 & 
1090.41 & 7.30 & \bf{1051.28} & 
2.99\\SCA8-8 & \bf{1071.18} & 8.02 & 
1088.61 & 7.58 & 1071.18 & 0.00\\
SCA8-9 & 1079.82 & 6.39 & 
1101.23 & 6.53 & \bf{1060.50} & 
1.82\\CON3-0 & 633.86 & 8.36 & 
636.21 & 9.04 & \bf{616.52} & 
2.81\\CON3-1 & 558.16 & 8.55 & 
560.54 & 9.83 & \bf{554.47} & 
0.67\\CON3-2 & 521.38 & 8.41 & 
522.10 & 7.96 & \bf{518.00} & 
0.65\\CON3-3 & 591.20 & 9.60 & 
598.81 & 9.09 & \bf{591.19} & 
0.00\\CON3-4 & 595.40 & 7.70 & 
604.52 & 8.34 & \bf{588.79} & 
1.12\\CON3-5 & 567.94 & 8.63 & 
573.34 & 9.02 & \bf{563.70} & 
0.75\\CON3-6 & 504.09 & 9.22 & 
508.40 & 8.63 & \bf{499.05} & 
1.01\\CON3-7 & 592.52 & 8.98 & 
594.85 & 8.53 & \bf{576.48} & 
2.78\\CON3-8 & 523.14 & 9.52 & 
528.90 & 9.29 & \bf{523.05} & 
0.02\\CON3-9 & 588.99 & 7.36 & 
589.75 & 8.24 & \bf{578.24} & 
1.86\\CON8-0 & 876.12 & 7.94 & 
902.92 & 7.37 & \bf{857.17} & 
2.21\\CON8-1 & 754.51 & 7.00 & 
769.29 & 7.60 & \bf{740.85} & 
1.84\\CON8-2 & 727.18 & 8.20 & 
735.67 & 7.56 & \bf{712.89} & 
2.00\\CON8-3 & 824.03 & 8.44 & 
836.62 & 7.91 & \bf{811.07} & 
1.60\\CON8-4 & 781.78 & 7.02 & 
802.40 & 7.04 & \bf{772.25} & 
1.23\\CON8-5 & 758.84 & 7.16 & 
771.14 & 7.08 & \bf{754.88} & 
0.52\\CON8-6 & 701.20 & 9.42 & 
707.88 & 7.79 & \bf{678.92} & 
3.28\\CON8-7 & 822.22 & 7.32 & 
841.55 & 7.63 & \bf{811.96} & 
1.26\\CON8-8 & 782.40 & 8.67 & 
795.62 & 7.43 & \bf{767.53} & 
1.94\\CON8-9 & 826.47 & 6.77 & 
831.01 & 7.45 & \bf{809.00} & 
2.16\\\bf{PROM.} & 
\bf{770.66} & \bf{8.00} & \bf{779.09} & \bf{7.97} & \bf{758.54} & \bf{1.47}\\[1ex]\hline
\end{tabular}
\label{table:ILS-VND-M-55-40}
\end{table}

\begin{table}[h]
\caption{Resultados de la ejecución de la metaheurística ILS-VND-M, utilizando instancias de Dethloff con la configuración -n 55.0 -LS 70.0}
\centering
\small
\begin{tabular}{c c c c c c c}
\hline\hline
Instancia & Costo mínimo & Tiempo(seg.) & Costo promedio & Tiempo promedio(seg.) & Costo ILS-VND-M & \%Gap \\ [0.5ex]
\hline
SCA3-0 & 636.06 & 13.19 & 
639.43 & 13.64 & \bf{635.62} & 
0.07\\SCA3-1 & \bf{697.84} & 14.05 & 
703.68 & 13.92 & 697.84 & 0.00\\
SCA3-2 & 661.13 & 12.55 & 
664.80 & 13.95 & \bf{659.34} & 
0.27\\SCA3-3 & 680.60 & 14.15 & 
680.88 & 13.36 & \bf{680.04} & 
0.08\\SCA3-4 & \bf{690.50} & 16.02 & 
691.18 & 13.95 & 690.50 & 0.00\\
SCA3-5 & 666.67 & 14.45 & 
673.10 & 12.99 & \bf{659.90} & 
1.03\\SCA3-6 & 652.94 & 14.01 & 
655.38 & 13.03 & \bf{651.09} & 
0.28\\SCA3-7 & 669.89 & 13.67 & 
671.93 & 14.20 & \bf{659.17} & 
1.63\\SCA3-8 & \bf{719.47} & 15.05 & 
723.79 & 13.82 & 719.47 & 0.00\\
SCA3-9 & 684.44 & 11.65 & 
685.36 & 14.11 & \bf{681.00} & 
0.51\\SCA8-0 & 990.21 & 11.79 & 
1000.62 & 12.32 & \bf{961.50} & 
2.99\\SCA8-1 & 1070.29 & 10.80 & 
1078.63 & 10.87 & \bf{1049.65} & 
1.97\\SCA8-2 & 1054.47 & 11.95 & 
1061.10 & 11.15 & \bf{1039.64} & 
1.43\\SCA8-3 & 1015.23 & 11.72 & 
1022.01 & 10.76 & \bf{983.34} & 
3.24\\SCA8-4 & 1069.87 & 10.34 & 
1092.29 & 11.25 & \bf{1065.49} & 
0.41\\SCA8-5 & 1052.02 & 14.64 & 
1062.53 & 12.94 & \bf{1027.08} & 
2.43\\SCA8-6 & 990.61 & 13.74 & 
997.28 & 12.02 & \bf{971.82} & 
1.93\\SCA8-7 & 1067.11 & 12.20 & 
1084.92 & 12.30 & \bf{1051.28} & 
1.51\\SCA8-8 & \bf{1071.18} & 12.54 & 
1092.09 & 11.62 & 1071.18 & 0.00\\
SCA8-9 & 1069.83 & 13.35 & 
1078.73 & 11.76 & \bf{1060.50} & 
0.88\\CON3-0 & 620.76 & 13.82 & 
630.55 & 14.34 & \bf{616.52} & 
0.69\\CON3-1 & 556.04 & 13.81 & 
558.62 & 15.12 & \bf{554.47} & 
0.28\\CON3-2 & 521.38 & 15.15 & 
521.57 & 14.56 & \bf{518.00} & 
0.65\\CON3-3 & \bf{591.19} & 15.58 & 
595.30 & 14.44 & 591.19 & 0.00\\
CON3-4 & 592.58 & 13.94 & 
594.49 & 14.19 & \bf{588.79} & 
0.64\\CON3-5 & 569.04 & 13.60 & 
569.35 & 13.67 & \bf{563.70} & 
0.95\\CON3-6 & 502.16 & 13.08 & 
504.39 & 20.38 & \bf{499.05} & 
0.62\\CON3-7 & 578.41 & 13.98 & 
587.20 & 14.02 & \bf{576.48} & 
0.33\\CON3-8 & 523.68 & 16.10 & 
524.51 & 14.98 & \bf{523.05} & 
0.12\\CON3-9 & 588.40 & 14.70 & 
589.51 & 14.59 & \bf{578.24} & 
1.76\\CON8-0 & 875.60 & 11.29 & 
889.83 & 11.27 & \bf{857.17} & 
2.15\\CON8-1 & 746.81 & 10.68 & 
763.05 & 11.83 & \bf{740.85} & 
0.80\\CON8-2 & 723.85 & 14.17 & 
730.02 & 13.56 & \bf{712.89} & 
1.54\\CON8-3 & 813.40 & 13.54 & 
828.90 & 12.70 & \bf{811.07} & 
0.29\\CON8-4 & 803.90 & 11.58 & 
813.55 & 11.63 & \bf{772.25} & 
4.10\\CON8-5 & 762.61 & 12.82 & 
769.04 & 12.23 & \bf{754.88} & 
1.02\\CON8-6 & 683.16 & 11.64 & 
692.35 & 11.73 & \bf{678.92} & 
0.62\\CON8-7 & 815.91 & 10.76 & 
824.77 & 11.31 & \bf{811.96} & 
0.49\\CON8-8 & 779.43 & 11.27 & 
784.75 & 11.41 & \bf{767.53} & 
1.55\\CON8-9 & 812.35 & 13.80 & 
821.01 & 12.31 & \bf{809.00} & 
0.41\\\bf{PROM.} & 
\bf{766.78} & \bf{13.18} & \bf{773.81} & \bf{13.11} & \bf{758.54} & \bf{0.99}\\[1ex]\hline
\end{tabular}
\label{table:ILS-VND-M-55-70}
\end{table}

\clearpage
\subsection{SalhiNagy}

\begin{table}[h]
\caption{Resultados de la ejecución de la metaheurística ILS-VND-M, utilizando instancias de SalhiNagy con la configuración -n 15.0 -LS 10.0}
\centering
\small
\begin{tabular}{c c c c c c c}
\hline\hline
Instancia & Costo mínimo & Tiempo(seg.) & Costo promedio & Tiempo promedio(seg.) & Costo ILS & \%Gap \\ [0.5ex]
\hline
CMT1X & 490.81 & 0.97 & 
501.76 & 0.83 & \bf{466.77} & 
5.15\\CMT1Y & 485.71 & 1.08 & 
498.10 & 0.71 & \bf{466.77} & 
4.06\\CMT2X & 722.89 & 2.41 & 
728.89 & 2.55 & \bf{684.21} & 
5.65\\CMT2Y & 722.28 & 2.32 & 
724.83 & 2.25 & \bf{684.21} & 
5.56\\CMT3X & 735.88 & 6.29 & 
746.94 & 6.08 & \bf{721.40} & 
2.01\\CMT3Y & 736.75 & 6.25 & 
742.77 & 6.00 & \bf{721.40} & 
2.13\\CMT4X & 891.65 & 19.14 & 
906.33 & 20.52 & \bf{852.83} & 
4.55\\CMT4Y & 910.90 & 20.03 & 
916.58 & 21.67 & \bf{852.46} & 
6.86\\CMT5X & 1109.35 & 68.60 & 
1111.90 & 51.84 & \bf{1030.55} & 
7.65\\CMT5Y & 1108.63 & 46.61 & 
1116.23 & 51.76 & \bf{1031.17} & 
7.51\\CMT11X & 908.96 & 14.70 & 
921.65 & 16.09 & \bf{839.39} & 
8.29\\CMT11Y & 885.84 & 14.03 & 
896.32 & 15.99 & \bf{841.88} & 
5.22\\CMT12X & 679.87 & 5.26 & 
690.46 & 5.60 & \bf{662.22} & 
2.67\\CMT12Y & 685.38 & 5.80 & 
686.17 & 6.38 & \bf{662.22} & 
3.50\\\bf{PROM.} & 
\bf{791.06} & \bf{15.25} & \bf{799.21} & \bf{14.88} & \bf{751.25} & \bf{5.06}\\[1ex]\hline
\end{tabular}
\label{table:ILS-VND-M-15-10-S}
\end{table}

\begin{table}[h]
\caption{Resultados de la ejecución de la metaheurística ILS-VND-M, utilizando instancias de SalhiNagy con la configuración -n 25.0 -LS 50.0}
\centering
\small
\begin{tabular}{c c c c c c c}
\hline\hline
Instancia & Costo mínimo & Tiempo(seg.) & Costo promedio & Tiempo promedio(seg.) & Costo ILS & \%Gap \\ [0.5ex]
\hline
CMT1X & 475.58 & 3.55 & 
480.69 & 3.47 & \bf{466.77} & 
1.89\\CMT1Y & 478.54 & 4.95 & 
489.57 & 4.04 & \bf{466.77} & 
2.52\\CMT2X & 709.51 & 8.45 & 
716.02 & 9.20 & \bf{684.21} & 
3.70\\CMT2Y & 709.97 & 10.26 & 
714.93 & 8.85 & \bf{684.21} & 
3.76\\CMT3X & 728.82 & 19.54 & 
742.10 & 19.73 & \bf{721.40} & 
1.03\\CMT3Y & 729.45 & 24.63 & 
736.70 & 21.90 & \bf{721.40} & 
1.12\\CMT4X & 905.49 & 52.92 & 
912.29 & 62.55 & \bf{852.83} & 
6.17\\CMT4Y & 900.75 & 54.95 & 
909.07 & 60.89 & \bf{852.46} & 
5.66\\CMT5X & 1107.29 & 126.22 & 
1120.72 & 140.26 & \bf{1030.55} & 
7.45\\CMT5Y & 1114.55 & 128.69 & 
1117.46 & 133.17 & \bf{1031.17} & 
8.09\\CMT11X & 883.35 & 56.40 & 
901.88 & 45.45 & \bf{839.39} & 
5.24\\CMT11Y & 882.11 & 51.88 & 
890.33 & 46.38 & \bf{841.88} & 
4.78\\CMT12X & 676.38 & 18.48 & 
684.46 & 20.20 & \bf{662.22} & 
2.14\\CMT12Y & 679.40 & 19.25 & 
682.63 & 18.06 & \bf{662.22} & 
2.59\\\bf{PROM.} & 
\bf{784.37} & \bf{41.44} & \bf{792.77} & \bf{42.44} & \bf{751.25} & \bf{4.01}\\[1ex]\hline
\end{tabular}
\label{table:ILS-VND-M-15-50-S}
\end{table}

\begin{table}[h]
\caption{Resultados de la ejecución de la metaheurística ILS-VND-M, utilizando instancias de SalhiNagy con la configuración -n 15.0 -LS 80.0}
\centering
\small
\begin{tabular}{c c c c c c c}
\hline\hline
Instancia & Costo mínimo & Tiempo(seg.) & Costo promedio & Tiempo promedio(seg.) & Costo ILS & \%Gap \\ [0.5ex]
\hline
CMT1X & 478.36 & 3.05 & 
481.78 & 3.39 & \bf{466.77} & 
2.48\\CMT1Y & 482.00 & 2.94 & 
484.41 & 4.03 & \bf{466.77} & 
3.26\\CMT2X & 701.91 & 6.38 & 
712.60 & 7.57 & \bf{684.21} & 
2.59\\CMT2Y & 704.88 & 10.15 & 
712.70 & 7.63 & \bf{684.21} & 
3.02\\CMT3X & 743.73 & 14.66 & 
747.17 & 17.75 & \bf{721.40} & 
3.10\\CMT3Y & 731.57 & 20.33 & 
739.73 & 19.57 & \bf{721.40} & 
1.41\\CMT4X & 892.92 & 49.28 & 
902.73 & 48.09 & \bf{852.83} & 
4.70\\CMT4Y & 901.33 & 47.54 & 
911.64 & 46.09 & \bf{852.46} & 
5.73\\CMT5X & 1110.97 & 116.63 & 
1118.00 & 102.77 & \bf{1030.55} & 
7.80\\CMT5Y & 1101.65 & 96.35 & 
1114.53 & 112.58 & \bf{1031.17} & 
6.83\\CMT11X & 883.45 & 34.88 & 
898.17 & 36.09 & \bf{839.39} & 
5.25\\CMT11Y & 878.31 & 38.18 & 
886.20 & 34.47 & \bf{841.88} & 
4.33\\CMT12X & 691.35 & 15.08 & 
693.81 & 15.86 & \bf{662.22} & 
4.40\\CMT12Y & 673.67 & 19.01 & 
679.04 & 17.70 & \bf{662.22} & 
1.73\\\bf{PROM.} & 
\bf{784.01} & \bf{33.89} & \bf{791.61} & \bf{33.83} & \bf{751.25} & \bf{4.05}\\[1ex]\hline
\end{tabular}
\label{table:ILS-VND-M-15-80-S}
\end{table}

\begin{table}[h]
\caption{Resultados de la ejecución de la metaheurística ILS-VND-M, utilizando instancias de SalhiNagy con la configuración -n 35.0 -LS 10.0}
\centering
\small
\begin{tabular}{c c c c c c c}
\hline\hline
Instancia & Costo mínimo & Tiempo(seg.) & Costo promedio & Tiempo promedio(seg.) & Costo ILS & \%Gap \\ [0.5ex]
\hline
CMT1X & 480.65 & 1.66 & 
487.08 & 1.76 & \bf{466.77} & 
2.97\\CMT1Y & 482.31 & 3.00 & 
485.87 & 2.19 & \bf{466.77} & 
3.33\\CMT2X & 702.99 & 7.47 & 
713.05 & 5.81 & \bf{684.21} & 
2.74\\CMT2Y & 711.53 & 5.03 & 
714.12 & 4.86 & \bf{684.21} & 
3.99\\CMT3X & 736.81 & 19.40 & 
740.75 & 15.01 & \bf{721.40} & 
2.14\\CMT3Y & 730.68 & 13.74 & 
735.83 & 16.21 & \bf{721.40} & 
1.29\\CMT4X & 893.70 & 42.45 & 
907.88 & 42.98 & \bf{852.83} & 
4.79\\CMT4Y & 905.89 & 39.80 & 
908.34 & 48.09 & \bf{852.46} & 
6.27\\CMT5X & 1108.43 & 106.01 & 
1116.91 & 105.17 & \bf{1030.55} & 
7.56\\CMT5Y & 1104.68 & 109.46 & 
1117.45 & 124.38 & \bf{1031.17} & 
7.13\\CMT11X & 896.15 & 54.29 & 
901.72 & 39.33 & \bf{839.39} & 
6.76\\CMT11Y & 888.02 & 31.38 & 
896.26 & 33.45 & \bf{841.88} & 
5.48\\CMT12X & 685.46 & 12.57 & 
688.43 & 14.85 & \bf{662.22} & 
3.51\\CMT12Y & 673.58 & 10.43 & 
680.89 & 11.84 & \bf{662.22} & 
1.72\\\bf{PROM.} & 
\bf{785.78} & \bf{32.62} & \bf{792.47} & \bf{33.28} & \bf{751.25} & \bf{4.26}\\[1ex]\hline
\end{tabular}
\label{table:ILS-VND-M-35-10-S}
\end{table}

\begin{table}[h]
\caption{Resultados de la ejecución de la metaheurística ILS-VND-M, utilizando instancias de SalhiNagy con la configuración -n 35.0 -LS 50.0}
\centering
\small
\begin{tabular}{c c c c c c c}
\hline\hline
Instancia & Costo mínimo & Tiempo(seg.) & Costo promedio & Tiempo promedio(seg.) & Costo ILS & \%Gap \\ [0.5ex]
\hline
CMT1X & 472.58 & 7.24 & 
477.00 & 5.71 & \bf{466.77} & 
1.24\\CMT1Y & 486.09 & 5.95 & 
489.19 & 6.09 & \bf{466.77} & 
4.14\\CMT2X & 694.46 & 12.92 & 
714.31 & 12.18 & \bf{684.21} & 
1.50\\CMT2Y & 703.05 & 13.36 & 
708.58 & 12.71 & \bf{684.21} & 
2.75\\CMT3X & 723.97 & 33.97 & 
734.62 & 31.53 & \bf{721.40} & 
0.36\\CMT3Y & 729.13 & 33.52 & 
735.13 & 30.86 & \bf{721.40} & 
1.07\\CMT4X & 897.02 & 75.86 & 
904.77 & 89.86 & \bf{852.83} & 
5.18\\CMT4Y & 899.71 & 76.75 & 
909.27 & 80.83 & \bf{852.46} & 
5.54\\CMT5X & 1100.51 & 175.85 & 
1105.17 & 184.79 & \bf{1030.55} & 
6.79\\CMT5Y & 1096.60 & 215.68 & 
1105.88 & 178.39 & \bf{1031.17} & 
6.35\\CMT11X & 880.26 & 55.39 & 
892.43 & 55.45 & \bf{839.39} & 
4.87\\CMT11Y & 889.14 & 54.53 & 
895.15 & 54.77 & \bf{841.88} & 
5.61\\CMT12X & 677.78 & 26.35 & 
683.68 & 27.67 & \bf{662.22} & 
2.35\\CMT12Y & 675.21 & 25.42 & 
681.25 & 26.82 & \bf{662.22} & 
1.96\\\bf{PROM.} & 
\bf{780.39} & \bf{58.06} & \bf{788.32} & \bf{56.97} & \bf{751.25} & \bf{3.55}\\[1ex]\hline
\end{tabular}
\label{table:ILS-VND-M-35-50-S}
\end{table}

\begin{table}[h]
\caption{Resultados de la ejecución de la metaheurística ILS-VND-M, utilizando instancias de SalhiNagy con la configuración -n 35.0 -LS 80.0}
\centering
\small
\begin{tabular}{c c c c c c c}
\hline\hline
Instancia & Costo mínimo & Tiempo(seg.) & Costo promedio & Tiempo promedio(seg.) & Costo ILS & \%Gap \\ [0.5ex]
\hline
CMT1X & 476.66 & 7.34 & 
478.83 & 7.03 & \bf{466.77} & 
2.12\\CMT1Y & 475.72 & 8.42 & 
481.51 & 8.33 & \bf{466.77} & 
1.92\\CMT2X & 712.07 & 16.75 & 
718.12 & 18.12 & \bf{684.21} & 
4.07\\CMT2Y & 690.24 & 20.17 & 
705.73 & 19.14 & \bf{684.21} & 
0.88\\CMT3X & 730.80 & 39.83 & 
738.58 & 39.59 & \bf{721.40} & 
1.30\\CMT3Y & 731.74 & 41.82 & 
733.20 & 44.49 & \bf{721.40} & 
1.43\\CMT4X & 895.51 & 97.50 & 
900.15 & 108.87 & \bf{852.83} & 
5.00\\CMT4Y & 893.75 & 99.69 & 
904.84 & 97.93 & \bf{852.46} & 
4.84\\CMT5X & 1103.34 & 274.41 & 
1106.00 & 237.80 & \bf{1030.55} & 
7.06\\CMT5Y & 1095.28 & 231.64 & 
1099.38 & 246.88 & \bf{1031.17} & 
6.22\\CMT11X & 874.94 & 106.31 & 
884.56 & 88.19 & \bf{839.39} & 
4.24\\CMT11Y & 884.56 & 79.52 & 
889.35 & 78.45 & \bf{841.88} & 
5.07\\CMT12X & 675.24 & 32.27 & 
680.92 & 35.58 & \bf{662.22} & 
1.97\\CMT12Y & 674.59 & 39.05 & 
683.78 & 36.30 & \bf{662.22} & 
1.87\\\bf{PROM.} & 
\bf{779.60} & \bf{78.19} & \bf{786.07} & \bf{76.19} & \bf{751.25} & \bf{3.43}\\[1ex]\hline
\end{tabular}
\label{table:ILS-VND-M-35-80-S}
\end{table}

\begin{table}[h]
\caption{Resultados de la ejecución de la metaheurística ILS-VND-M, utilizando instancias de SalhiNagy con la configuración -n 55.0 -LS 10.0}
\centering
\small
\begin{tabular}{c c c c c c c}
\hline\hline
Instancia & Costo mínimo & Tiempo(seg.) & Costo promedio & Tiempo promedio(seg.) & Costo ILS & \%Gap \\ [0.5ex]
\hline
CMT1X & 477.47 & 2.30 & 
486.59 & 2.96 & \bf{466.77} & 
2.29\\CMT1Y & 486.63 & 2.46 & 
489.68 & 3.05 & \bf{466.77} & 
4.25\\CMT2X & 705.69 & 8.20 & 
710.01 & 8.43 & \bf{684.21} & 
3.14\\CMT2Y & 709.71 & 9.21 & 
716.71 & 8.08 & \bf{684.21} & 
3.73\\CMT3X & 727.76 & 21.76 & 
733.67 & 21.55 & \bf{721.40} & 
0.88\\CMT3Y & 726.14 & 21.24 & 
734.23 & 21.77 & \bf{721.40} & 
0.66\\CMT4X & 883.82 & 100.83 & 
897.66 & 83.64 & \bf{852.83} & 
3.63\\CMT4Y & 885.78 & 67.68 & 
896.36 & 83.11 & \bf{852.46} & 
3.91\\CMT5X & 1089.68 & 161.79 & 
1107.14 & 169.18 & \bf{1030.55} & 
5.74\\CMT5Y & 1083.29 & 175.29 & 
1097.57 & 194.01 & \bf{1031.17} & 
5.05\\CMT11X & 881.85 & 88.17 & 
891.13 & 68.36 & \bf{839.39} & 
5.06\\CMT11Y & 855.88 & 53.37 & 
889.37 & 61.41 & \bf{841.88} & 
1.66\\CMT12X & 679.60 & 26.58 & 
688.33 & 20.86 & \bf{662.22} & 
2.62\\CMT12Y & 676.06 & 28.26 & 
682.80 & 23.64 & \bf{662.22} & 
2.09\\\bf{PROM.} & 
\bf{776.38} & \bf{54.80} & \bf{787.23} & \bf{55.00} & \bf{751.25} & \bf{3.19}\\[1ex]\hline
\end{tabular}
\label{table:ILS-VND-M-55-10-S}
\end{table}

\begin{table}[h]
\caption{Resultados de la ejecución de la metaheurística ILS-VND-M, utilizando instancias de SalhiNagy con la configuración -n 55.0 -LS 50.0}
\centering
\small
\begin{tabular}{c c c c c c c}
\hline\hline
Instancia & Costo mínimo & Tiempo(seg.) & Costo promedio & Tiempo promedio(seg.) & Costo ILS & \%Gap \\ [0.5ex]
\hline
CMT1X & 470.67 & 5.59 & 
476.00 & 8.55 & \bf{466.77} & 
0.84\\CMT1Y & 472.85 & 6.00 & 
485.75 & 7.31 & \bf{466.77} & 
1.30\\CMT2X & 707.76 & 19.60 & 
711.20 & 20.16 & \bf{684.21} & 
3.44\\CMT2Y & 702.53 & 25.52 & 
708.70 & 20.82 & \bf{684.21} & 
2.68\\CMT3X & 730.13 & 56.76 & 
733.41 & 47.42 & \bf{721.40} & 
1.21\\CMT3Y & 729.62 & 56.16 & 
734.07 & 50.69 & \bf{721.40} & 
1.14\\CMT4X & 901.98 & 128.84 & 
907.36 & 120.86 & \bf{852.83} & 
5.76\\CMT4Y & 902.15 & 152.15 & 
906.81 & 128.03 & \bf{852.46} & 
5.83\\CMT5X & 1079.14 & 344.34 & 
1100.13 & 304.98 & \bf{1030.55} & 
4.71\\CMT5Y & 1092.86 & 361.21 & 
1099.81 & 298.55 & \bf{1031.17} & 
5.98\\CMT11X & 881.54 & 88.44 & 
886.79 & 93.00 & \bf{839.39} & 
5.02\\CMT11Y & 879.31 & 94.88 & 
887.50 & 97.19 & \bf{841.88} & 
4.45\\CMT12X & 676.87 & 43.21 & 
679.14 & 45.80 & \bf{662.22} & 
2.21\\CMT12Y & 677.39 & 53.95 & 
683.28 & 47.40 & \bf{662.22} & 
2.29\\\bf{PROM.} & 
\bf{778.91} & \bf{102.62} & \bf{785.71} & \bf{92.20} & \bf{751.25} & \bf{3.35}\\[1ex]\hline
\end{tabular}
\label{table:ILS-VND-M-55-50-S}
\end{table}

\begin{table}[h]
\caption{Resultados de la ejecución de la metaheurística ILS-VND-M, utilizando instancias de SalhiNagy con la configuración -n 55.0 -LS 80.0}
\centering
\small
\begin{tabular}{c c c c c c c}
\hline\hline
Instancia & Costo mínimo & Tiempo(seg.) & Costo promedio & Tiempo promedio(seg.) & Costo ILS & \%Gap \\ [0.5ex]
\hline
CMT1X & 474.85 & 13.10 & 
478.73 & 12.40 & \bf{466.77} & 
1.73\\CMT1Y & 475.22 & 13.60 & 
477.55 & 13.68 & \bf{466.77} & 
1.81\\CMT2X & 704.58 & 26.99 & 
709.19 & 27.97 & \bf{684.21} & 
2.98\\CMT2Y & 692.37 & 30.62 & 
700.51 & 29.95 & \bf{684.21} & 
1.19\\CMT3X & 727.10 & 63.73 & 
734.75 & 65.50 & \bf{721.40} & 
0.79\\CMT3Y & 728.12 & 63.90 & 
731.20 & 66.71 & \bf{721.40} & 
0.93\\CMT4X & 884.17 & 161.42 & 
894.02 & 160.37 & \bf{852.83} & 
3.67\\CMT4Y & 881.71 & 199.47 & 
897.42 & 173.26 & \bf{852.46} & 
3.43\\CMT5X & 1094.05 & 335.46 & 
1097.41 & 378.99 & \bf{1030.55} & 
6.16\\CMT5Y & 1078.81 & 373.06 & 
1089.79 & 365.23 & \bf{1031.17} & 
4.62\\CMT11X & 878.41 & 153.82 & 
889.95 & 137.27 & \bf{839.39} & 
4.65\\CMT11Y & 880.39 & 121.95 & 
888.88 & 118.90 & \bf{841.88} & 
4.57\\CMT12X & 677.54 & 57.06 & 
680.55 & 57.05 & \bf{662.22} & 
2.31\\CMT12Y & 675.95 & 62.77 & 
679.94 & 64.66 & \bf{662.22} & 
2.07\\\bf{PROM.} & 
\bf{775.23} & \bf{119.78} & \bf{782.14} & \bf{119.42} & \bf{751.25} & \bf{2.92}\\[1ex]\hline
\end{tabular}
\label{table:ILS-VND-M-55-80}
\end{table}

\clearpage
\section{SS-M}

\subsection{Dethloff}
\begin{table}[h]
\caption{Resultados de la ejecución de la metaheurística SS-M, utilizando instancias de Dethloff con la configuración -n 50.0 -b 10 -y 0.1}
\centering
\small
\begin{tabular}{c c c c c c c c}
\hline\hline
Instancia & Costo mínimo & Tiempo(seg.) & Costo promedio & Tiempo promedio(seg.) & CME & \%G & \%GP \\ [0.5ex]
\hline
SCA3-0 & 636.06 & 2.50 & 
638.30 & 2.68 & \bf{635.62} & 
0.07 & 0.42\\SCA3-1 & \bf{697.84} & 1.94 & 
698.50 & 2.69 & 697.84 & 0.00
 & 0.10\\SCA3-2 & 661.13 & 2.27 & 
663.89 & 3.13 & \bf{659.34} & 
0.27 & 0.69\\SCA3-3 & 681.74 & 1.80 & 
681.74 & 2.74 & \bf{680.04} & 
0.25 & 0.25\\SCA3-4 & 692.57 & 2.71 & 
693.42 & 3.00 & \bf{690.50} & 
0.30 & 0.42\\SCA3-5 & 665.04 & 1.76 & 
674.17 & 2.27 & \bf{659.90} & 
0.78 & 2.16\\SCA3-6 & \bf{651.09} & 1.45 & 
651.09 & 1.38 & 651.09 & 0.00
 & 0.00\\
SCA3-7 & 671.67 & 3.22 & 
671.67 & 2.62 & \bf{659.17} & 
1.90 & 1.90\\SCA3-8 & \bf{719.47} & 3.34 & 
719.47 & 2.97 & 719.47 & 0.00
 & 0.00\\
SCA3-9 & 685.14 & 3.62 & 
685.75 & 2.83 & \bf{681.00} & 
0.61 & 0.70\\SCA8-0 & 974.40 & 10.60 & 
989.38 & 10.00 & \bf{961.50} & 
1.34 & 2.90\\SCA8-1 & 1057.41 & 11.32 & 
1067.80 & 11.60 & \bf{1049.65} & 
0.74 & 1.73\\SCA8-2 & 1053.00 & 10.54 & 
1053.44 & 10.48 & \bf{1039.64} & 
1.29 & 1.33\\SCA8-3 & 1019.57 & 6.69 & 
1022.20 & 7.71 & \bf{983.34} & 
3.68 & 3.95\\SCA8-4 & 1072.75 & 6.35 & 
1075.28 & 7.42 & \bf{1065.49} & 
0.68 & 0.92\\SCA8-5 & 1050.44 & 14.44 & 
1053.92 & 10.57 & \bf{1027.08} & 
2.27 & 2.61\\SCA8-6 & 972.48 & 12.97 & 
980.33 & 10.23 & \bf{971.82} & 
0.07 & 0.88\\SCA8-7 & 1063.60 & 9.24 & 
1068.39 & 10.22 & \bf{1051.28} & 
1.17 & 1.63\\SCA8-8 & 1085.98 & 6.58 & 
1090.76 & 6.97 & \bf{1071.18} & 
1.38 & 1.83\\SCA8-9 & 1070.34 & 9.07 & 
1080.14 & 9.61 & \bf{1060.50} & 
0.93 & 1.85\\CON3-0 & \bf{616.52} & 1.09 & 
616.52 & 1.08 & 616.52 & 0.00
 & 0.00\\
CON3-1 & 556.79 & 2.98 & 
559.21 & 3.24 & \bf{554.47} & 
0.42 & 0.86\\CON3-2 & 521.38 & 2.91 & 
521.57 & 2.69 & \bf{518.00} & 
0.65 & 0.69\\CON3-3 & \bf{591.19} & 3.39 & 
591.20 & 3.83 & 591.19 & 0.00
 & 0.00\\CON3-4 & \bf{588.79} & 2.14 & 
589.45 & 2.31 & 588.79 & 0.00
 & 0.11\\CON3-5 & 568.69 & 1.91 & 
568.69 & 1.97 & \bf{563.70} & 
0.89 & 0.89\\CON3-6 & 502.16 & 3.35 & 
502.16 & 4.05 & \bf{499.05} & 
0.62 & 0.62\\CON3-7 & 578.41 & 4.79 & 
582.64 & 3.87 & \bf{576.48} & 
0.33 & 1.07\\CON3-8 & 523.14 & 3.67 & 
523.66 & 2.95 & \bf{523.05} & 
0.02 & 0.12\\CON3-9 & 588.40 & 2.39 & 
588.55 & 2.38 & \bf{578.24} & 
1.76 & 1.78\\CON8-0 & 881.17 & 8.65 & 
885.22 & 8.59 & \bf{857.17} & 
2.80 & 3.27\\CON8-1 & 742.47 & 8.56 & 
747.61 & 11.40 & \bf{740.85} & 
0.22 & 0.91\\CON8-2 & 713.05 & 7.47 & 
716.73 & 9.12 & \bf{712.89} & 
0.02 & 0.54\\CON8-3 & 832.18 & 12.78 & 
835.44 & 11.52 & \bf{811.07} & 
2.60 & 3.00\\CON8-4 & 778.37 & 10.24 & 
786.81 & 8.44 & \bf{772.25} & 
0.79 & 1.89\\CON8-5 & 754.95 & 8.22 & 
758.88 & 10.62 & \bf{754.88} & 
0.01 & 0.53\\CON8-6 & 686.39 & 5.52 & 
695.23 & 6.82 & \bf{678.92} & 
1.10 & 2.40\\CON8-7 & 815.44 & 8.79 & 
817.42 & 10.06 & \bf{811.96} & 
0.43 & 0.67\\CON8-8 & 775.49 & 8.24 & 
781.99 & 7.36 & \bf{767.53} & 
1.04 & 1.88\\CON8-9 & 820.76 & 6.71 & 
829.91 & 7.77 & \bf{809.00} & 
1.45 & 2.58\\\bf{PROM.} & 
\bf{765.44} & \bf{5.91} & \bf{768.96} & \bf{6.03} & \bf{758.54} & \bf{0.82} & \bf{1.25}\\[1ex]\hline
\end{tabular}
\label{table:SS-M-50-0.1}
\end{table}

\begin{table}[h]
\caption{Resultados de la ejecución de la metaheurística SS-M, utilizando instancias de Dethloff con la configuración -n 50.0 -b 10 -y 0.5}
\centering
\small
\begin{tabular}{c c c c c c c c}
\hline\hline
Instancia & Costo mínimo & Tiempo(seg.) & Costo promedio & Tiempo promedio(seg.) & CME & \%G & \%GP \\ [0.5ex]
\hline
SCA3-0 & 640.55 & 3.40 & 
640.55 & 3.49 & \bf{635.62} & 
0.78 & 0.78\\SCA3-1 & \bf{697.84} & 2.11 & 
697.84 & 2.14 & 697.84 & 0.00
 & 0.00\\
SCA3-2 & 661.13 & 3.04 & 
664.06 & 3.80 & \bf{659.34} & 
0.27 & 0.72\\SCA3-3 & 680.60 & 3.24 & 
680.60 & 2.78 & \bf{680.04} & 
0.08 & 0.08\\SCA3-4 & \bf{690.50} & 3.09 & 
690.50 & 2.30 & 690.50 & 0.00
 & 0.00\\
SCA3-5 & 674.01 & 2.14 & 
679.03 & 2.36 & \bf{659.90} & 
2.14 & 2.90\\SCA3-6 & 652.94 & 3.93 & 
654.55 & 4.12 & \bf{651.09} & 
0.28 & 0.53\\SCA3-7 & 669.89 & 2.69 & 
670.78 & 2.54 & \bf{659.17} & 
1.63 & 1.76\\SCA3-8 & \bf{719.47} & 3.95 & 
719.47 & 3.44 & 719.47 & 0.00
 & 0.00\\
SCA3-9 & \bf{681.00} & 3.73 & 
682.00 & 3.33 & 681.00 & 0.00
 & 0.15\\SCA8-0 & 982.03 & 8.08 & 
983.96 & 7.83 & \bf{961.50} & 
2.14 & 2.34\\SCA8-1 & 1053.90 & 12.40 & 
1077.41 & 9.87 & \bf{1049.65} & 
0.40 & 2.64\\SCA8-2 & 1051.95 & 8.87 & 
1052.45 & 9.72 & \bf{1039.64} & 
1.18 & 1.23\\SCA8-3 & 1024.97 & 7.12 & 
1032.10 & 8.09 & \bf{983.34} & 
4.23 & 4.96\\SCA8-4 & 1069.30 & 36.13 & 
1088.76 & 16.74 & \bf{1065.49} & 
0.36 & 2.18\\SCA8-5 & 1050.64 & 8.80 & 
1057.90 & 8.89 & \bf{1027.08} & 
2.29 & 3.00\\SCA8-6 & 972.48 & 14.98 & 
975.41 & 13.62 & \bf{971.82} & 
0.07 & 0.37\\SCA8-7 & 1066.82 & 12.52 & 
1069.57 & 10.88 & \bf{1051.28} & 
1.48 & 1.74\\SCA8-8 & 1085.98 & 15.17 & 
1091.49 & 10.56 & \bf{1071.18} & 
1.38 & 1.90\\SCA8-9 & 1072.10 & 9.98 & 
1078.36 & 9.43 & \bf{1060.50} & 
1.09 & 1.68\\CON3-0 & 617.59 & 3.15 & 
621.36 & 2.47 & \bf{616.52} & 
0.17 & 0.78\\CON3-1 & 560.75 & 3.23 & 
560.75 & 3.29 & \bf{554.47} & 
1.13 & 1.13\\CON3-2 & 521.38 & 3.42 & 
521.38 & 2.88 & \bf{518.00} & 
0.65 & 0.65\\CON3-3 & 591.20 & 2.75 & 
591.20 & 4.99 & \bf{591.19} & 
0.00 & 0.00\\CON3-4 & 591.43 & 2.63 & 
591.43 & 3.13 & \bf{588.79} & 
0.45 & 0.45\\CON3-5 & 564.88 & 3.11 & 
565.92 & 2.05 & \bf{563.70} & 
0.21 & 0.39\\CON3-6 & 502.16 & 3.01 & 
504.25 & 2.71 & \bf{499.05} & 
0.62 & 1.04\\CON3-7 & 586.01 & 2.84 & 
586.01 & 3.26 & \bf{576.48} & 
1.65 & 1.65\\CON3-8 & 523.68 & 1.90 & 
525.49 & 2.19 & \bf{523.05} & 
0.12 & 0.47\\CON3-9 & 588.38 & 2.63 & 
588.53 & 2.87 & \bf{578.24} & 
1.75 & 1.78\\CON8-0 & 881.90 & 8.96 & 
886.37 & 8.28 & \bf{857.17} & 
2.89 & 3.41\\CON8-1 & 743.42 & 10.49 & 
747.79 & 11.10 & \bf{740.85} & 
0.35 & 0.94\\CON8-2 & 713.05 & 13.06 & 
716.27 & 11.59 & \bf{712.89} & 
0.02 & 0.47\\CON8-3 & 817.57 & 10.56 & 
830.30 & 8.79 & \bf{811.07} & 
0.80 & 2.37\\CON8-4 & 782.59 & 8.43 & 
787.20 & 9.34 & \bf{772.25} & 
1.34 & 1.94\\CON8-5 & 760.41 & 11.82 & 
765.65 & 9.86 & \bf{754.88} & 
0.73 & 1.43\\CON8-6 & 696.64 & 9.43 & 
698.09 & 9.70 & \bf{678.92} & 
2.61 & 2.82\\CON8-7 & 814.50 & 13.93 & 
816.76 & 20.17 & \bf{811.96} & 
0.31 & 0.59\\CON8-8 & 771.32 & 5.62 & 
783.38 & 6.70 & \bf{767.53} & 
0.49 & 2.07\\CON8-9 & 813.57 & 9.50 & 
822.80 & 8.44 & \bf{809.00} & 
0.56 & 1.71\\\bf{PROM.} & 
\bf{766.01} & \bf{7.40} & \bf{769.94} & \bf{6.74} & \bf{758.54} & \bf{0.92} & \bf{1.38}\\[1ex]\hline
\end{tabular}
\label{table:SS-M-50-0.5}
\end{table}

\begin{table}[h]
\caption{Resultados de la ejecución de la metaheurística SS-M, utilizando instancias de Dethloff con la configuración -n 50.0 -b 10 -y 1.0}
\centering
\small
\begin{tabular}{c c c c c c c c}
\hline\hline
Instancia & Costo mínimo & Tiempo(seg.) & Costo promedio & Tiempo promedio(seg.) & CME & \%G & \%GP \\ [0.5ex]
\hline
SCA3-0 & 640.55 & 3.94 & 
640.55 & 3.22 & \bf{635.62} & 
0.78 & 0.78\\SCA3-1 & \bf{697.84} & 5.00 & 
700.43 & 6.41 & 697.84 & 0.00
 & 0.37\\SCA3-2 & 661.13 & 4.31 & 
663.60 & 3.57 & \bf{659.34} & 
0.27 & 0.65\\SCA3-3 & 680.60 & 2.93 & 
680.96 & 3.35 & \bf{680.04} & 
0.08 & 0.13\\SCA3-4 & \bf{690.50} & 3.75 & 
690.50 & 3.58 & 690.50 & 0.00
 & 0.00\\
SCA3-5 & 661.07 & 3.18 & 
676.62 & 3.24 & \bf{659.90} & 
0.18 & 2.53\\SCA3-6 & 653.68 & 5.16 & 
654.54 & 4.57 & \bf{651.09} & 
0.40 & 0.53\\SCA3-7 & 671.67 & 3.76 & 
671.67 & 3.20 & \bf{659.17} & 
1.90 & 1.90\\SCA3-8 & \bf{719.47} & 4.79 & 
719.47 & 3.50 & 719.47 & 0.00
 & 0.00\\
SCA3-9 & \bf{681.00} & 6.29 & 
683.07 & 4.03 & 681.00 & 0.00
 & 0.30\\SCA8-0 & 988.41 & 11.67 & 
993.34 & 9.93 & \bf{961.50} & 
2.80 & 3.31\\SCA8-1 & 1067.36 & 11.45 & 
1072.48 & 9.96 & \bf{1049.65} & 
1.69 & 2.18\\SCA8-2 & 1051.48 & 11.26 & 
1053.04 & 12.36 & \bf{1039.64} & 
1.14 & 1.29\\SCA8-3 & 1019.54 & 9.98 & 
1031.45 & 9.54 & \bf{983.34} & 
3.68 & 4.89\\SCA8-4 & 1067.28 & 11.20 & 
1074.67 & 10.86 & \bf{1065.49} & 
0.17 & 0.86\\SCA8-5 & 1050.64 & 8.24 & 
1064.45 & 9.86 & \bf{1027.08} & 
2.29 & 3.64\\SCA8-6 & 972.48 & 8.98 & 
977.31 & 11.09 & \bf{971.82} & 
0.07 & 0.56\\SCA8-7 & 1063.22 & 10.48 & 
1069.30 & 10.33 & \bf{1051.28} & 
1.14 & 1.71\\SCA8-8 & 1082.91 & 10.96 & 
1090.99 & 9.52 & \bf{1071.18} & 
1.10 & 1.85\\SCA8-9 & 1067.42 & 13.77 & 
1075.34 & 11.33 & \bf{1060.50} & 
0.65 & 1.40\\CON3-0 & 632.98 & 1.86 & 
633.17 & 1.98 & \bf{616.52} & 
2.67 & 2.70\\CON3-1 & 560.75 & 2.51 & 
560.75 & 2.40 & \bf{554.47} & 
1.13 & 1.13\\CON3-2 & 521.38 & 2.03 & 
521.38 & 2.29 & \bf{518.00} & 
0.65 & 0.65\\CON3-3 & 591.20 & 2.11 & 
591.20 & 2.53 & \bf{591.19} & 
0.00 & 0.00\\CON3-4 & \bf{588.79} & 3.76 & 
590.21 & 2.85 & 588.79 & 0.00
 & 0.24\\CON3-5 & \bf{563.70} & 4.46 & 
566.80 & 3.56 & 563.70 & 0.00
 & 0.55\\CON3-6 & 504.15 & 2.26 & 
504.88 & 2.50 & \bf{499.05} & 
1.02 & 1.17\\CON3-7 & 584.01 & 3.22 & 
585.33 & 3.25 & \bf{576.48} & 
1.31 & 1.54\\CON3-8 & \bf{523.05} & 3.36 & 
523.21 & 3.07 & 523.05 & 0.00
 & 0.03\\CON3-9 & 588.40 & 1.85 & 
588.40 & 2.80 & \bf{578.24} & 
1.76 & 1.76\\CON8-0 & 877.17 & 9.05 & 
880.45 & 8.82 & \bf{857.17} & 
2.33 & 2.72\\CON8-1 & 742.47 & 7.02 & 
744.09 & 9.41 & \bf{740.85} & 
0.22 & 0.44\\CON8-2 & 713.60 & 11.58 & 
716.59 & 11.46 & \bf{712.89} & 
0.10 & 0.52\\CON8-3 & 832.75 & 8.54 & 
835.43 & 9.49 & \bf{811.07} & 
2.67 & 3.00\\CON8-4 & 777.99 & 7.85 & 
786.67 & 9.66 & \bf{772.25} & 
0.74 & 1.87\\CON8-5 & 758.84 & 8.80 & 
760.00 & 10.46 & \bf{754.88} & 
0.52 & 0.68\\CON8-6 & 689.23 & 11.18 & 
693.27 & 10.43 & \bf{678.92} & 
1.52 & 2.11\\CON8-7 & 814.79 & 11.57 & 
816.30 & 12.08 & \bf{811.96} & 
0.35 & 0.54\\CON8-8 & 779.43 & 6.28 & 
785.41 & 7.15 & \bf{767.53} & 
1.55 & 2.33\\CON8-9 & 815.02 & 9.66 & 
823.94 & 8.36 & \bf{809.00} & 
0.74 & 1.85\\\bf{PROM.} & 
\bf{766.20} & \bf{6.75} & \bf{769.78} & \bf{6.70} & \bf{758.54} & \bf{0.94} & \bf{1.37}\\[1ex]\hline
\end{tabular}
\label{table:SS-M-50-1.0}
\end{table}

\begin{table}[h]
\caption{Resultados de la ejecución de la metaheurística SS-M, utilizando instancias de Dethloff con la configuración -n 100.0 -b 10 -y 0.1}
\centering
\small
\begin{tabular}{c c c c c c c c}
\hline\hline
Instancia & Costo mínimo & Tiempo(seg.) & Costo promedio & Tiempo promedio(seg.) & CME & \%G & \%GP \\ [0.5ex]
\hline
SCA3-0 & 640.55 & 3.12 & 
640.55 & 3.13 & \bf{635.62} & 
0.78 & 0.78\\SCA3-1 & 701.53 & 2.94 & 
701.61 & 2.92 & \bf{697.84} & 
0.53 & 0.54\\SCA3-2 & \bf{659.34} & 5.23 & 
667.78 & 3.94 & 659.34 & 0.00
 & 1.28\\SCA3-3 & 681.31 & 2.78 & 
681.31 & 2.67 & \bf{680.04} & 
0.19 & 0.19\\SCA3-4 & 692.57 & 2.87 & 
692.57 & 2.79 & \bf{690.50} & 
0.30 & 0.30\\SCA3-5 & 670.10 & 2.22 & 
673.50 & 2.27 & \bf{659.90} & 
1.55 & 2.06\\SCA3-6 & 653.93 & 1.32 & 
653.93 & 1.33 & \bf{651.09} & 
0.44 & 0.44\\SCA3-7 & 666.15 & 2.70 & 
666.15 & 2.69 & \bf{659.17} & 
1.06 & 1.06\\SCA3-8 & \bf{719.47} & 2.85 & 
719.47 & 3.85 & 719.47 & 0.00
 & 0.00\\
SCA3-9 & \bf{681.00} & 3.43 & 
683.07 & 4.03 & 681.00 & 0.00
 & 0.30\\SCA8-0 & 985.12 & 8.62 & 
991.28 & 10.82 & \bf{961.50} & 
2.46 & 3.10\\SCA8-1 & 1053.44 & 13.86 & 
1071.02 & 17.20 & \bf{1049.65} & 
0.36 & 2.04\\SCA8-2 & 1053.78 & 9.63 & 
1054.01 & 11.54 & \bf{1039.64} & 
1.36 & 1.38\\SCA8-3 & 1029.00 & 14.64 & 
1032.12 & 9.43 & \bf{983.34} & 
4.64 & 4.96\\SCA8-4 & 1071.86 & 12.46 & 
1092.77 & 11.52 & \bf{1065.49} & 
0.60 & 2.56\\SCA8-5 & 1029.95 & 9.36 & 
1044.70 & 10.26 & \bf{1027.08} & 
0.28 & 1.72\\SCA8-6 & 972.48 & 11.75 & 
982.84 & 8.95 & \bf{971.82} & 
0.07 & 1.13\\SCA8-7 & 1067.03 & 9.88 & 
1071.47 & 10.66 & \bf{1051.28} & 
1.50 & 1.92\\SCA8-8 & 1092.01 & 4.88 & 
1094.74 & 6.19 & \bf{1071.18} & 
1.94 & 2.20\\SCA8-9 & 1082.94 & 8.41 & 
1089.82 & 7.44 & \bf{1060.50} & 
2.12 & 2.76\\CON3-0 & 620.76 & 2.38 & 
629.38 & 2.40 & \bf{616.52} & 
0.69 & 2.09\\CON3-1 & 556.04 & 2.79 & 
557.22 & 2.92 & \bf{554.47} & 
0.28 & 0.50\\CON3-2 & 521.38 & 2.47 & 
521.38 & 2.60 & \bf{518.00} & 
0.65 & 0.65\\CON3-3 & \bf{591.19} & 3.37 & 
591.20 & 3.75 & 591.19 & 0.00
 & 0.00\\CON3-4 & 589.32 & 2.59 & 
589.32 & 2.58 & \bf{588.79} & 
0.09 & 0.09\\CON3-5 & 569.04 & 3.14 & 
570.69 & 2.80 & \bf{563.70} & 
0.95 & 1.24\\CON3-6 & 502.16 & 2.38 & 
502.16 & 2.22 & \bf{499.05} & 
0.62 & 0.62\\CON3-7 & 582.33 & 3.15 & 
585.09 & 3.43 & \bf{576.48} & 
1.01 & 1.49\\CON3-8 & \bf{523.05} & 3.03 & 
524.25 & 2.77 & 523.05 & 0.00
 & 0.23\\CON3-9 & 581.06 & 1.86 & 
586.93 & 2.17 & \bf{578.24} & 
0.49 & 1.50\\CON8-0 & 872.09 & 8.14 & 
877.49 & 8.04 & \bf{857.17} & 
1.74 & 2.37\\CON8-1 & 741.70 & 10.36 & 
749.20 & 9.27 & \bf{740.85} & 
0.11 & 1.13\\CON8-2 & 714.06 & 49.88 & 
721.03 & 18.67 & \bf{712.89} & 
0.16 & 1.14\\CON8-3 & 832.89 & 9.62 & 
834.91 & 8.89 & \bf{811.07} & 
2.69 & 2.94\\CON8-4 & 786.79 & 9.87 & 
788.03 & 7.66 & \bf{772.25} & 
1.88 & 2.04\\CON8-5 & 757.75 & 11.22 & 
761.28 & 9.40 & \bf{754.88} & 
0.38 & 0.85\\CON8-6 & 687.70 & 9.91 & 
699.46 & 8.05 & \bf{678.92} & 
1.29 & 3.02\\CON8-7 & 814.50 & 10.99 & 
815.41 & 9.77 & \bf{811.96} & 
0.31 & 0.43\\CON8-8 & 789.44 & 10.06 & 
789.44 & 8.23 & \bf{767.53} & 
2.85 & 2.85\\CON8-9 & 814.36 & 10.32 & 
824.84 & 8.38 & \bf{809.00} & 
0.66 & 1.96\\\bf{PROM.} & 
\bf{766.28} & \bf{7.51} & \bf{770.59} & \bf{6.44} & \bf{758.54} & \bf{0.93} & \bf{1.45}\\[1ex]\hline
\end{tabular}
\label{table:SS-M-100-0.1}
\end{table}

\begin{table}[h]
\caption{Resultados de la ejecución de la metaheurística SS-M, utilizando instancias de Dethloff con la configuración -n 100.0 -b 10 -y 0.5}
\centering
\small
\begin{tabular}{c c c c c c c c}
\hline\hline
Instancia & Costo mínimo & Tiempo(seg.) & Costo promedio & Tiempo promedio(seg.) & CME & \%G & \%GP \\ [0.5ex]
\hline
SCA3-0 & 640.55 & 2.71 & 
640.55 & 3.15 & \bf{635.62} & 
0.78 & 0.78\\SCA3-1 & 700.50 & 2.26 & 
700.50 & 2.35 & \bf{697.84} & 
0.38 & 0.38\\SCA3-2 & 666.01 & 3.02 & 
666.01 & 3.55 & \bf{659.34} & 
1.01 & 1.01\\SCA3-3 & \bf{680.04} & 1.59 & 
680.99 & 2.88 & 680.04 & 0.00
 & 0.14\\SCA3-4 & 692.57 & 3.88 & 
693.25 & 4.21 & \bf{690.50} & 
0.30 & 0.40\\SCA3-5 & 669.80 & 3.05 & 
672.51 & 3.10 & \bf{659.90} & 
1.50 & 1.91\\SCA3-6 & 652.94 & 4.04 & 
653.16 & 4.11 & \bf{651.09} & 
0.28 & 0.32\\SCA3-7 & 666.60 & 3.54 & 
669.13 & 3.75 & \bf{659.17} & 
1.13 & 1.51\\SCA3-8 & \bf{719.47} & 1.27 & 
719.47 & 1.28 & 719.47 & 0.00
 & 0.00\\
SCA3-9 & \bf{681.00} & 4.52 & 
684.47 & 3.78 & 681.00 & 0.00
 & 0.51\\SCA8-0 & 978.76 & 10.18 & 
983.44 & 8.87 & \bf{961.50} & 
1.80 & 2.28\\SCA8-1 & 1067.02 & 6.95 & 
1070.99 & 7.85 & \bf{1049.65} & 
1.65 & 2.03\\SCA8-2 & 1051.95 & 8.69 & 
1052.96 & 11.53 & \bf{1039.64} & 
1.18 & 1.28\\SCA8-3 & 1018.32 & 7.82 & 
1021.88 & 8.92 & \bf{983.34} & 
3.56 & 3.92\\SCA8-4 & 1086.81 & 5.86 & 
1089.60 & 7.38 & \bf{1065.49} & 
2.00 & 2.26\\SCA8-5 & 1050.64 & 8.94 & 
1057.71 & 10.95 & \bf{1027.08} & 
2.29 & 2.98\\SCA8-6 & 972.48 & 13.18 & 
975.93 & 12.16 & \bf{971.82} & 
0.07 & 0.42\\SCA8-7 & 1063.22 & 10.48 & 
1069.38 & 9.56 & \bf{1051.28} & 
1.14 & 1.72\\SCA8-8 & 1086.54 & 8.88 & 
1091.65 & 10.56 & \bf{1071.18} & 
1.43 & 1.91\\SCA8-9 & 1072.60 & 8.58 & 
1085.00 & 8.87 & \bf{1060.50} & 
1.14 & 2.31\\CON3-0 & 633.22 & 1.74 & 
633.22 & 1.78 & \bf{616.52} & 
2.71 & 2.71\\CON3-1 & 558.34 & 1.67 & 
559.54 & 2.30 & \bf{554.47} & 
0.70 & 0.92\\CON3-2 & 521.38 & 2.98 & 
523.88 & 3.18 & \bf{518.00} & 
0.65 & 1.14\\CON3-3 & 591.20 & 3.29 & 
591.20 & 2.76 & \bf{591.19} & 
0.00 & 0.00\\CON3-4 & 589.32 & 2.76 & 
590.90 & 3.46 & \bf{588.79} & 
0.09 & 0.36\\CON3-5 & \bf{563.70} & 2.65 & 
563.70 & 1.94 & 563.70 & 0.00
 & 0.00\\
CON3-6 & 502.16 & 2.26 & 
502.31 & 2.49 & \bf{499.05} & 
0.62 & 0.65\\CON3-7 & 577.68 & 2.91 & 
577.68 & 3.02 & \bf{576.48} & 
0.21 & 0.21\\CON3-8 & 526.59 & 3.58 & 
533.46 & 2.14 & \bf{523.05} & 
0.68 & 1.99\\CON3-9 & 588.40 & 2.20 & 
588.40 & 2.20 & \bf{578.24} & 
1.76 & 1.76\\CON8-0 & 873.22 & 9.08 & 
877.00 & 8.44 & \bf{857.17} & 
1.87 & 2.31\\CON8-1 & 742.47 & 12.73 & 
752.79 & 8.82 & \bf{740.85} & 
0.22 & 1.61\\CON8-2 & 712.94 & 9.74 & 
719.17 & 10.54 & \bf{712.89} & 
0.01 & 0.88\\CON8-3 & 817.57 & 6.27 & 
827.22 & 14.55 & \bf{811.07} & 
0.80 & 1.99\\CON8-4 & 780.51 & 12.09 & 
784.79 & 11.92 & \bf{772.25} & 
1.07 & 1.62\\CON8-5 & 755.14 & 12.81 & 
759.15 & 10.81 & \bf{754.88} & 
0.03 & 0.57\\CON8-6 & 698.41 & 9.56 & 
701.97 & 9.58 & \bf{678.92} & 
2.87 & 3.39\\CON8-7 & 814.50 & 16.83 & 
816.09 & 12.52 & \bf{811.96} & 
0.31 & 0.51\\CON8-8 & 781.74 & 9.03 & 
786.14 & 9.32 & \bf{767.53} & 
1.85 & 2.42\\CON8-9 & 817.16 & 6.47 & 
826.89 & 9.24 & \bf{809.00} & 
1.01 & 2.21\\\bf{PROM.} & 
\bf{766.59} & \bf{6.25} & \bf{769.85} & \bf{6.50} & \bf{758.54} & \bf{0.98} & \bf{1.38}\\[1ex]\hline
\end{tabular}
\label{table:SS-M-100-0.5}
\end{table}

\begin{table}[h]
\caption{Resultados de la ejecución de la metaheurística SS-M, utilizando instancias de Dethloff con la configuración -n 100.0 -b 10 -y 1.0}
\centering
\small
\begin{tabular}{c c c c c c c c}
\hline\hline
Instancia & Costo mínimo & Tiempo(seg.) & Costo promedio & Tiempo promedio(seg.) & CME & \%G & \%GP \\ [0.5ex]
\hline
SCA3-0 & 640.55 & 4.38 & 
640.55 & 3.97 & \bf{635.62} & 
0.78 & 0.78\\SCA3-1 & \bf{697.84} & 4.82 & 
699.68 & 4.76 & 697.84 & 0.00
 & 0.26\\SCA3-2 & 661.13 & 3.48 & 
661.13 & 3.37 & \bf{659.34} & 
0.27 & 0.27\\SCA3-3 & 680.60 & 3.00 & 
681.06 & 3.58 & \bf{680.04} & 
0.08 & 0.15\\SCA3-4 & \bf{690.50} & 3.46 & 
691.02 & 3.85 & 690.50 & 0.00
 & 0.08\\SCA3-5 & 681.30 & 2.32 & 
681.43 & 2.60 & \bf{659.90} & 
3.24 & 3.26\\SCA3-6 & 652.94 & 4.68 & 
653.68 & 2.70 & \bf{651.09} & 
0.28 & 0.40\\SCA3-7 & 666.60 & 2.92 & 
667.87 & 2.95 & \bf{659.17} & 
1.13 & 1.32\\SCA3-8 & \bf{719.47} & 5.55 & 
719.47 & 5.59 & 719.47 & 0.00
 & 0.00\\
SCA3-9 & \bf{681.00} & 4.24 & 
683.07 & 4.20 & 681.00 & 0.00
 & 0.30\\SCA8-0 & 984.75 & 12.83 & 
990.95 & 12.51 & \bf{961.50} & 
2.42 & 3.06\\SCA8-1 & 1063.76 & 10.70 & 
1069.78 & 10.10 & \bf{1049.65} & 
1.34 & 1.92\\SCA8-2 & 1052.94 & 10.12 & 
1054.06 & 10.69 & \bf{1039.64} & 
1.28 & 1.39\\SCA8-3 & 1018.37 & 7.21 & 
1021.04 & 8.02 & \bf{983.34} & 
3.56 & 3.83\\SCA8-4 & 1069.53 & 12.64 & 
1084.12 & 10.89 & \bf{1065.49} & 
0.38 & 1.75\\SCA8-5 & 1043.05 & 14.46 & 
1049.65 & 12.38 & \bf{1027.08} & 
1.55 & 2.20\\SCA8-6 & 972.48 & 13.68 & 
976.96 & 11.49 & \bf{971.82} & 
0.07 & 0.53\\SCA8-7 & 1064.06 & 11.43 & 
1073.54 & 11.35 & \bf{1051.28} & 
1.22 & 2.12\\SCA8-8 & 1091.20 & 8.20 & 
1092.13 & 9.11 & \bf{1071.18} & 
1.87 & 1.96\\SCA8-9 & 1070.71 & 13.66 & 
1086.18 & 12.24 & \bf{1060.50} & 
0.96 & 2.42\\CON3-0 & 630.73 & 3.44 & 
630.73 & 3.41 & \bf{616.52} & 
2.30 & 2.30\\CON3-1 & 560.55 & 2.61 & 
560.65 & 2.73 & \bf{554.47} & 
1.10 & 1.11\\CON3-2 & 521.38 & 2.58 & 
521.38 & 2.79 & \bf{518.00} & 
0.65 & 0.65\\CON3-3 & 591.20 & 1.94 & 
591.20 & 2.18 & \bf{591.19} & 
0.00 & 0.00\\CON3-4 & 591.43 & 2.44 & 
591.43 & 2.45 & \bf{588.79} & 
0.45 & 0.45\\CON3-5 & 564.89 & 3.73 & 
565.22 & 3.54 & \bf{563.70} & 
0.21 & 0.27\\CON3-6 & 502.16 & 2.52 & 
502.16 & 3.11 & \bf{499.05} & 
0.62 & 0.62\\CON3-7 & 585.42 & 2.91 & 
585.86 & 3.24 & \bf{576.48} & 
1.55 & 1.63\\CON3-8 & \bf{523.05} & 3.30 & 
523.43 & 2.69 & 523.05 & 0.00
 & 0.07\\CON3-9 & 588.40 & 3.88 & 
588.40 & 3.46 & \bf{578.24} & 
1.76 & 1.76\\CON8-0 & 875.11 & 7.67 & 
880.36 & 7.91 & \bf{857.17} & 
2.09 & 2.71\\CON8-1 & 742.47 & 10.80 & 
744.00 & 11.50 & \bf{740.85} & 
0.22 & 0.42\\CON8-2 & 716.32 & 9.76 & 
720.70 & 12.14 & \bf{712.89} & 
0.48 & 1.10\\CON8-3 & 812.54 & 11.42 & 
828.24 & 9.93 & \bf{811.07} & 
0.18 & 2.12\\CON8-4 & 785.55 & 11.60 & 
789.55 & 12.68 & \bf{772.25} & 
1.72 & 2.24\\CON8-5 & 755.14 & 9.94 & 
762.53 & 9.63 & \bf{754.88} & 
0.03 & 1.01\\CON8-6 & 686.39 & 9.17 & 
689.18 & 9.91 & \bf{678.92} & 
1.10 & 1.51\\CON8-7 & 815.04 & 11.18 & 
820.36 & 8.79 & \bf{811.96} & 
0.38 & 1.03\\CON8-8 & 784.36 & 11.68 & 
786.65 & 9.57 & \bf{767.53} & 
2.19 & 2.49\\CON8-9 & 816.55 & 12.39 & 
826.63 & 9.46 & \bf{809.00} & 
0.93 & 2.18\\\bf{PROM.} & 
\bf{766.29} & \bf{7.22} & \bf{769.65} & \bf{6.94} & \bf{758.54} & \bf{0.96} & \bf{1.34}\\[1ex]\hline
\end{tabular}
\label{table:SS-M-100-1.0}
\end{table}

\begin{table}[h]
\caption{Resultados de la ejecución de la metaheurística SS-M, utilizando instancias de Dethloff con la configuración -n 150.0 -b 10 -y 0.1}
\centering
\small
\begin{tabular}{c c c c c c c c}
\hline\hline
Instancia & Costo mínimo & Tiempo(seg.) & Costo promedio & Tiempo promedio(seg.) & CME & \%G & \%GP \\ [0.5ex]
\hline
SCA3-0 & 640.55 & 4.72 & 
640.55 & 8.17 & \bf{635.62} & 
0.78 & 0.78\\SCA3-1 & \bf{697.84} & 2.47 & 
697.84 & 2.56 & 697.84 & 0.00
 & 0.00\\
SCA3-2 & 661.13 & 3.63 & 
670.16 & 2.05 & \bf{659.34} & 
0.27 & 1.64\\SCA3-3 & 680.60 & 4.48 & 
680.88 & 4.19 & \bf{680.04} & 
0.08 & 0.12\\SCA3-4 & \bf{690.50} & 5.48 & 
690.50 & 4.64 & 690.50 & 0.00
 & 0.00\\
SCA3-5 & 665.04 & 2.76 & 
677.62 & 2.27 & \bf{659.90} & 
0.78 & 2.68\\SCA3-6 & \bf{651.09} & 4.55 & 
652.48 & 2.46 & 651.09 & 0.00
 & 0.21\\SCA3-7 & 667.34 & 3.46 & 
667.34 & 3.44 & \bf{659.17} & 
1.24 & 1.24\\SCA3-8 & \bf{719.47} & 3.66 & 
719.47 & 3.47 & 719.47 & 0.00
 & 0.00\\
SCA3-9 & \bf{681.00} & 4.80 & 
681.00 & 4.80 & 681.00 & 0.00
 & 0.00\\
SCA8-0 & 987.51 & 7.26 & 
989.46 & 9.85 & \bf{961.50} & 
2.71 & 2.91\\SCA8-1 & 1057.41 & 9.26 & 
1068.66 & 15.75 & \bf{1049.65} & 
0.74 & 1.81\\SCA8-2 & 1050.37 & 15.52 & 
1050.79 & 14.44 & \bf{1039.64} & 
1.03 & 1.07\\SCA8-3 & 1004.25 & 7.76 & 
1019.72 & 8.03 & \bf{983.34} & 
2.13 & 3.70\\SCA8-4 & 1068.97 & 9.93 & 
1078.50 & 9.71 & \bf{1065.49} & 
0.33 & 1.22\\SCA8-5 & 1049.44 & 8.91 & 
1053.40 & 9.96 & \bf{1027.08} & 
2.18 & 2.56\\SCA8-6 & 977.83 & 6.30 & 
977.83 & 6.94 & \bf{971.82} & 
0.62 & 0.62\\SCA8-7 & 1067.88 & 6.82 & 
1069.30 & 6.06 & \bf{1051.28} & 
1.58 & 1.71\\SCA8-8 & \bf{1071.18} & 9.17 & 
1085.23 & 8.62 & 1071.18 & 0.00
 & 1.31\\SCA8-9 & 1072.60 & 8.81 & 
1081.75 & 9.97 & \bf{1060.50} & 
1.14 & 2.00\\CON3-0 & 617.59 & 2.43 & 
618.26 & 2.30 & \bf{616.52} & 
0.17 & 0.28\\CON3-1 & 560.75 & 3.39 & 
560.75 & 2.89 & \bf{554.47} & 
1.13 & 1.13\\CON3-2 & 521.38 & 2.66 & 
521.38 & 1.93 & \bf{518.00} & 
0.65 & 0.65\\CON3-3 & 591.48 & 2.43 & 
591.48 & 2.54 & \bf{591.19} & 
0.05 & 0.05\\CON3-4 & 591.43 & 3.01 & 
591.43 & 4.00 & \bf{588.79} & 
0.45 & 0.45\\CON3-5 & 565.30 & 2.06 & 
565.30 & 2.04 & \bf{563.70} & 
0.28 & 0.28\\CON3-6 & 502.16 & 1.96 & 
502.16 & 1.97 & \bf{499.05} & 
0.62 & 0.62\\CON3-7 & 581.37 & 5.06 & 
581.58 & 4.63 & \bf{576.48} & 
0.85 & 0.89\\CON3-8 & \bf{523.05} & 2.74 & 
523.12 & 2.87 & 523.05 & 0.00
 & 0.01\\CON3-9 & 588.40 & 3.08 & 
588.40 & 2.87 & \bf{578.24} & 
1.76 & 1.76\\CON8-0 & 874.01 & 9.13 & 
874.16 & 8.50 & \bf{857.17} & 
1.96 & 1.98\\CON8-1 & 742.47 & 9.91 & 
745.25 & 9.85 & \bf{740.85} & 
0.22 & 0.59\\CON8-2 & 713.60 & 10.72 & 
716.90 & 11.21 & \bf{712.89} & 
0.10 & 0.56\\CON8-3 & 822.73 & 10.53 & 
831.38 & 9.98 & \bf{811.07} & 
1.44 & 2.50\\CON8-4 & 777.24 & 52.96 & 
779.77 & 20.37 & \bf{772.25} & 
0.65 & 0.97\\CON8-5 & 755.86 & 12.68 & 
760.77 & 12.01 & \bf{754.88} & 
0.13 & 0.78\\CON8-6 & 693.34 & 7.96 & 
698.94 & 7.66 & \bf{678.92} & 
2.12 & 2.95\\CON8-7 & 814.79 & 12.42 & 
814.99 & 13.75 & \bf{811.96} & 
0.35 & 0.37\\CON8-8 & 786.80 & 5.52 & 
789.30 & 6.42 & \bf{767.53} & 
2.51 & 2.84\\CON8-9 & 821.70 & 7.89 & 
821.70 & 7.03 & \bf{809.00} & 
1.57 & 1.57\\\bf{PROM.} & 
\bf{765.19} & \bf{7.46} & \bf{768.24} & \bf{6.80} & \bf{758.54} & \bf{0.82} & \bf{1.17}\\[1ex]\hline
\end{tabular}
\label{table:SS-M-150-0.1}
\end{table}

\begin{table}[h]
\caption{Resultados de la ejecución de la metaheurística SS-M, utilizando instancias de Dethloff con la configuración -n 150.0 -b 10 -y 0.5}
\centering
\small
\begin{tabular}{c c c c c c c c}
\hline\hline
Instancia & Costo mínimo & Tiempo(seg.) & Costo promedio & Tiempo promedio(seg.) & CME & \%G & \%GP \\ [0.5ex]
\hline
SCA3-0 & 640.55 & 3.86 & 
640.55 & 3.38 & \bf{635.62} & 
0.78 & 0.78\\SCA3-1 & 701.53 & 3.81 & 
701.53 & 3.39 & \bf{697.84} & 
0.53 & 0.53\\SCA3-2 & 665.71 & 3.46 & 
665.71 & 3.73 & \bf{659.34} & 
0.97 & 0.97\\SCA3-3 & 680.60 & 3.63 & 
681.17 & 3.03 & \bf{680.04} & 
0.08 & 0.17\\SCA3-4 & \bf{690.50} & 3.96 & 
690.50 & 4.01 & 690.50 & 0.00
 & 0.00\\
SCA3-5 & 666.67 & 2.46 & 
666.67 & 2.49 & \bf{659.90} & 
1.03 & 1.03\\SCA3-6 & 654.26 & 3.47 & 
654.26 & 3.30 & \bf{651.09} & 
0.49 & 0.49\\SCA3-7 & 666.60 & 2.77 & 
666.60 & 2.87 & \bf{659.17} & 
1.13 & 1.13\\SCA3-8 & \bf{719.47} & 3.44 & 
719.47 & 3.60 & 719.47 & 0.00
 & 0.00\\
SCA3-9 & 685.14 & 3.23 & 
685.14 & 3.45 & \bf{681.00} & 
0.61 & 0.61\\SCA8-0 & 984.96 & 12.25 & 
996.72 & 9.20 & \bf{961.50} & 
2.44 & 3.66\\SCA8-1 & 1065.72 & 10.93 & 
1068.84 & 11.06 & \bf{1049.65} & 
1.53 & 1.83\\SCA8-2 & 1051.95 & 11.50 & 
1052.69 & 11.47 & \bf{1039.64} & 
1.18 & 1.26\\SCA8-3 & 1017.21 & 8.32 & 
1017.76 & 8.99 & \bf{983.34} & 
3.44 & 3.50\\SCA8-4 & 1078.39 & 8.74 & 
1080.53 & 9.91 & \bf{1065.49} & 
1.21 & 1.41\\SCA8-5 & 1042.51 & 12.29 & 
1049.60 & 11.30 & \bf{1027.08} & 
1.50 & 2.19\\SCA8-6 & 972.48 & 9.30 & 
973.83 & 11.41 & \bf{971.82} & 
0.07 & 0.21\\SCA8-7 & 1067.49 & 10.12 & 
1073.93 & 8.94 & \bf{1051.28} & 
1.54 & 2.15\\SCA8-8 & \bf{1071.18} & 10.41 & 
1084.98 & 9.33 & 1071.18 & 0.00
 & 1.29\\SCA8-9 & 1077.03 & 9.42 & 
1080.55 & 9.20 & \bf{1060.50} & 
1.56 & 1.89\\CON3-0 & 633.24 & 2.59 & 
633.24 & 2.55 & \bf{616.52} & 
2.71 & 2.71\\CON3-1 & 563.72 & 2.45 & 
563.72 & 2.56 & \bf{554.47} & 
1.67 & 1.67\\CON3-2 & 521.38 & 2.43 & 
521.38 & 2.49 & \bf{518.00} & 
0.65 & 0.65\\CON3-3 & 591.20 & 2.67 & 
591.20 & 2.87 & \bf{591.19} & 
0.00 & 0.00\\CON3-4 & 591.43 & 3.16 & 
591.43 & 3.15 & \bf{588.79} & 
0.45 & 0.45\\CON3-5 & \bf{563.70} & 2.02 & 
567.71 & 3.19 & 563.70 & 0.00
 & 0.71\\CON3-6 & 502.16 & 3.63 & 
502.16 & 3.77 & \bf{499.05} & 
0.62 & 0.62\\CON3-7 & 586.01 & 2.79 & 
586.01 & 3.07 & \bf{576.48} & 
1.65 & 1.65\\CON3-8 & \bf{523.05} & 3.03 & 
523.12 & 3.15 & 523.05 & 0.00
 & 0.01\\CON3-9 & 588.99 & 2.15 & 
588.99 & 2.15 & \bf{578.24} & 
1.86 & 1.86\\CON8-0 & 873.65 & 9.32 & 
881.06 & 10.06 & \bf{857.17} & 
1.92 & 2.79\\CON8-1 & 748.30 & 8.64 & 
751.35 & 10.39 & \bf{740.85} & 
1.01 & 1.42\\CON8-2 & 713.05 & 12.39 & 
717.61 & 9.85 & \bf{712.89} & 
0.02 & 0.66\\CON8-3 & 826.06 & 7.75 & 
827.04 & 7.00 & \bf{811.07} & 
1.85 & 1.97\\CON8-4 & 776.98 & 11.42 & 
781.70 & 10.89 & \bf{772.25} & 
0.61 & 1.22\\CON8-5 & 760.82 & 11.83 & 
762.19 & 11.90 & \bf{754.88} & 
0.79 & 0.97\\CON8-6 & 698.41 & 10.35 & 
703.11 & 8.33 & \bf{678.92} & 
2.87 & 3.56\\CON8-7 & 814.79 & 10.55 & 
815.59 & 9.32 & \bf{811.96} & 
0.35 & 0.45\\CON8-8 & 782.82 & 8.76 & 
785.17 & 9.07 & \bf{767.53} & 
1.99 & 2.30\\CON8-9 & 818.17 & 12.75 & 
820.37 & 9.89 & \bf{809.00} & 
1.13 & 1.41\\\bf{PROM.} & 
\bf{766.95} & \bf{6.70} & \bf{769.13} & \bf{6.49} & \bf{758.54} & \bf{1.06} & \bf{1.30}\\[1ex]\hline
\end{tabular}
\label{table:SS-M-150-0.5}
\end{table}

\begin{table}[h]
\caption{Resultados de la ejecución de la metaheurística SS-M, utilizando instancias de Dethloff con la configuración -n 150.0 -b 10 -y 1.0}
\centering
\small
\begin{tabular}{c c c c c c c c}
\hline\hline
Instancia & Costo mínimo & Tiempo(seg.) & Costo promedio & Tiempo promedio(seg.) & CME & \%G & \%GP \\ [0.5ex]
\hline
SCA3-0 & 640.55 & 3.70 & 
640.55 & 3.66 & \bf{635.62} & 
0.78 & 0.78\\SCA3-1 & 700.50 & 3.67 & 
700.76 & 3.89 & \bf{697.84} & 
0.38 & 0.42\\SCA3-2 & \bf{659.34} & 3.85 & 
661.18 & 4.59 & 659.34 & 0.00
 & 0.28\\SCA3-3 & 681.74 & 1.48 & 
681.74 & 3.11 & \bf{680.04} & 
0.25 & 0.25\\SCA3-4 & \bf{690.50} & 4.41 & 
692.55 & 3.50 & 690.50 & 0.00
 & 0.30\\SCA3-5 & 679.54 & 3.07 & 
679.54 & 3.19 & \bf{659.90} & 
2.98 & 2.98\\SCA3-6 & 652.94 & 3.59 & 
653.43 & 3.64 & \bf{651.09} & 
0.28 & 0.36\\SCA3-7 & 666.60 & 3.26 & 
666.60 & 3.52 & \bf{659.17} & 
1.13 & 1.13\\SCA3-8 & \bf{719.47} & 4.73 & 
719.47 & 3.77 & 719.47 & 0.00
 & 0.00\\
SCA3-9 & \bf{681.00} & 3.31 & 
681.00 & 3.73 & 681.00 & 0.00
 & 0.00\\
SCA8-0 & 975.50 & 13.43 & 
986.08 & 10.58 & \bf{961.50} & 
1.46 & 2.56\\SCA8-1 & 1050.38 & 8.27 & 
1055.87 & 10.00 & \bf{1049.65} & 
0.07 & 0.59\\SCA8-2 & 1052.56 & 13.52 & 
1053.19 & 11.59 & \bf{1039.64} & 
1.24 & 1.30\\SCA8-3 & 1013.56 & 10.74 & 
1026.66 & 12.17 & \bf{983.34} & 
3.07 & 4.41\\SCA8-4 & 1078.86 & 16.96 & 
1089.70 & 11.71 & \bf{1065.49} & 
1.25 & 2.27\\SCA8-5 & 1043.17 & 16.91 & 
1050.84 & 13.12 & \bf{1027.08} & 
1.57 & 2.31\\SCA8-6 & 972.48 & 10.70 & 
986.46 & 19.02 & \bf{971.82} & 
0.07 & 1.51\\SCA8-7 & 1070.53 & 9.59 & 
1073.11 & 10.43 & \bf{1051.28} & 
1.83 & 2.08\\SCA8-8 & 1082.11 & 10.09 & 
1089.93 & 9.30 & \bf{1071.18} & 
1.02 & 1.75\\SCA8-9 & 1081.06 & 8.18 & 
1091.49 & 8.29 & \bf{1060.50} & 
1.94 & 2.92\\CON3-0 & 630.73 & 1.56 & 
630.80 & 1.32 & \bf{616.52} & 
2.30 & 2.32\\CON3-1 & 559.25 & 2.53 & 
559.25 & 2.67 & \bf{554.47} & 
0.86 & 0.86\\CON3-2 & 521.38 & 3.72 & 
521.38 & 3.72 & \bf{518.00} & 
0.65 & 0.65\\CON3-3 & 591.20 & 3.50 & 
591.51 & 3.25 & \bf{591.19} & 
0.00 & 0.05\\CON3-4 & 591.43 & 3.47 & 
591.43 & 3.75 & \bf{588.79} & 
0.45 & 0.45\\CON3-5 & \bf{563.70} & 2.43 & 
563.70 & 2.48 & 563.70 & 0.00
 & 0.00\\
CON3-6 & 502.16 & 2.76 & 
502.16 & 2.53 & \bf{499.05} & 
0.62 & 0.62\\CON3-7 & 586.01 & 3.24 & 
586.01 & 2.83 & \bf{576.48} & 
1.65 & 1.65\\CON3-8 & \bf{523.05} & 1.57 & 
523.67 & 1.96 & 523.05 & 0.00
 & 0.12\\CON3-9 & 588.40 & 2.10 & 
588.84 & 2.26 & \bf{578.24} & 
1.76 & 1.83\\CON8-0 & 883.76 & 8.87 & 
893.42 & 8.67 & \bf{857.17} & 
3.10 & 4.23\\CON8-1 & 748.39 & 13.46 & 
752.37 & 9.28 & \bf{740.85} & 
1.02 & 1.56\\CON8-2 & 713.10 & 10.38 & 
713.48 & 7.96 & \bf{712.89} & 
0.03 & 0.08\\CON8-3 & 817.57 & 6.82 & 
824.40 & 7.97 & \bf{811.07} & 
0.80 & 1.64\\CON8-4 & 785.55 & 10.03 & 
789.18 & 9.54 & \bf{772.25} & 
1.72 & 2.19\\CON8-5 & 760.41 & 10.04 & 
762.05 & 11.29 & \bf{754.88} & 
0.73 & 0.95\\CON8-6 & 698.32 & 10.54 & 
700.23 & 8.08 & \bf{678.92} & 
2.86 & 3.14\\CON8-7 & 814.77 & 11.90 & 
815.58 & 11.44 & \bf{811.96} & 
0.35 & 0.45\\CON8-8 & 783.92 & 9.54 & 
788.04 & 9.42 & \bf{767.53} & 
2.14 & 2.67\\CON8-9 & 814.45 & 11.82 & 
825.20 & 11.40 & \bf{809.00} & 
0.67 & 2.00\\\bf{PROM.} & 
\bf{766.75} & \bf{7.09} & \bf{770.07} & \bf{6.87} & \bf{758.54} & \bf{1.03} & \bf{1.39}\\[1ex]\hline
\end{tabular}
\label{table:SS-M-150-1.0}
\end{table}

\clearpage
\subsection{SalhiNagy}

\begin{table}[h]
\caption{Resultados de la ejecución de la metaheurística SS-M, utilizando instancias de SalhiNagy con la configuración -n 50.0 -b 10 -y 0.1}
\centering
\small
\begin{tabular}{c c c c c c c c}
\hline\hline
Instancia & Costo mínimo & Tiempo(seg.) & Costo promedio & Tiempo promedio(seg.) & CME & \%G & \%GP \\ [0.5ex]
\hline
CMT1X & 472.37 & 3.50 & 
473.33 & 2.94 & \bf{470.48} & 
0.40 & 0.61\\CMT1Y & 472.87 & 2.58 & 
474.63 & 1.58 & \bf{470.48} & 
0.51 & 0.88\\CMT2X & 704.15 & 15.88 & 
708.62 & 12.10 & \bf{682.39} & 
3.19 & 3.84\\CMT2Y & 704.57 & 16.85 & 
705.94 & 17.72 & \bf{682.39} & 
3.25 & 3.45\\CMT3X & 726.32 & 29.41 & 
737.06 & 31.96 & \bf{719.06} & 
1.01 & 2.50\\CMT3Y & 737.83 & 42.83 & 
741.20 & 37.65 & \bf{719.06} & 
2.61 & 3.08\\CMT4X & 900.80 & 237.52 & 
906.71 & 234.19 & \bf{854.21} & 
5.45 & 6.15\\CMT4Y & 881.86 & 402.81 & 
898.65 & 231.40 & \bf{852.46} & 
3.45 & 5.42\\CMT5X & 1100.80 & 1063.01 & 
1100.80 & 1063.01 & \bf{1030.56} & 
6.82 & 6.82\\CMT5Y & 1107.29 & 1085.01 & 
1114.28 & 1135.02 & \bf{1031.69} & 
7.33 & 8.01\\CMT11X & 890.64 & 44.83 & 
900.71 & 47.67 & \bf{831.09} & 
7.17 & 8.38\\CMT11Y & 906.29 & 36.56 & 
913.19 & 34.48 & \bf{829.85} & 
9.21 & 10.04\\CMT12X & 685.32 & 38.33 & 
686.50 & 58.66 & \bf{658.83} & 
4.02 & 4.20\\CMT12Y & 682.79 & 82.46 & 
689.40 & 47.61 & \bf{660.47} & 
3.38 & 4.38\\\bf{PROM.} & 
\bf{783.85} & \bf{221.54} & \bf{789.36} & \bf{211.14} & \bf{749.50} & \bf{4.13} & \bf{4.84}\\[1ex]\hline
\end{tabular}
\label{table:SS-M-50-0.1-S}
\end{table}

\begin{table}[h]
\caption{Resultados de la ejecución de la metaheurística SS-M, utilizando instancias de SalhiNagy con la configuración -n 50.0 -b 10 -y 0.2}
\centering
\small
\begin{tabular}{c c c c c c c c}
\hline\hline
Instancia & Costo mínimo & Tiempo(seg.) & Costo promedio & Tiempo promedio(seg.) & CME & \%G & \%GP \\ [0.5ex]
\hline
CMT1X & 479.10 & 2.01 & 
479.10 & 1.94 & \bf{470.48} & 
1.83 & 1.83\\CMT1Y & 472.87 & 2.88 & 
476.40 & 2.24 & \bf{470.48} & 
0.51 & 1.26\\CMT2X & 706.82 & 26.86 & 
710.29 & 16.96 & \bf{682.39} & 
3.58 & 4.09\\CMT2Y & 706.74 & 14.97 & 
709.04 & 19.32 & \bf{682.39} & 
3.57 & 3.91\\CMT3X & 738.72 & 33.61 & 
740.89 & 30.29 & \bf{719.06} & 
2.73 & 3.04\\CMT3Y & 734.49 & 37.84 & 
738.16 & 29.61 & \bf{719.06} & 
2.15 & 2.66\\CMT4X & 901.10 & 216.33 & 
910.23 & 174.28 & \bf{854.21} & 
5.49 & 6.56\\CMT4Y & 887.24 & 170.92 & 
898.74 & 285.46 & \bf{852.46} & 
4.08 & 5.43\\CMT5X & 1103.93 & 1082.73 & 
1111.23 & 1085.23 & \bf{1030.56} & 
7.12 & 7.83\\CMT5Y & 1104.45 & 1025.23 & 
1115.77 & 1092.86 & \bf{1031.69} & 
7.05 & 8.15\\CMT11X & 907.87 & 22.25 & 
910.17 & 43.40 & \bf{831.09} & 
9.24 & 9.52\\CMT11Y & 890.90 & 27.79 & 
903.55 & 41.80 & \bf{829.85} & 
7.36 & 8.88\\CMT12X & 676.66 & 48.77 & 
680.56 & 37.36 & \bf{658.83} & 
2.71 & 3.30\\CMT12Y & 674.95 & 79.03 & 
681.67 & 76.12 & \bf{660.47} & 
2.19 & 3.21\\\bf{PROM.} & 
\bf{784.70} & \bf{199.37} & \bf{790.41} & \bf{209.78} & \bf{749.50} & \bf{4.26} & \bf{4.97}\\[1ex]\hline
\end{tabular}
\label{table:SS-M-50-0.2-S}
\end{table}

\begin{table}[h]
\caption{Resultados de la ejecución de la metaheurística SS-M, utilizando instancias de SalhiNagy con la configuración -n 150.0 -b 10 -y 0.1}
\centering
\small
\begin{tabular}{c c c c c c c c}
\hline\hline
Instancia & Costo mínimo & Tiempo(seg.) & Costo promedio & Tiempo promedio(seg.) & CME & \%G & \%GP \\ [0.5ex]
\hline
CMT1X & 472.37 & 3.21 & 
473.08 & 2.84 & \bf{470.48} & 
0.40 & 0.55\\CMT1Y & 480.15 & 1.15 & 
480.15 & 1.17 & \bf{470.48} & 
2.06 & 2.06\\CMT2X & 711.65 & 16.82 & 
713.04 & 13.55 & \bf{682.39} & 
4.29 & 4.49\\CMT2Y & 703.00 & 22.52 & 
710.14 & 17.81 & \bf{682.39} & 
3.02 & 4.07\\CMT3X & 729.86 & 51.99 & 
732.82 & 45.16 & \bf{719.06} & 
1.50 & 1.91\\CMT3Y & 727.10 & 34.85 & 
737.27 & 30.00 & \bf{719.06} & 
1.12 & 2.53\\CMT4X & 897.25 & 173.62 & 
906.74 & 202.63 & \bf{854.21} & 
5.04 & 6.15\\CMT4Y & 892.03 & 173.72 & 
901.90 & 183.08 & \bf{852.46} & 
4.64 & 5.80\\CMT5X & 1099.51 & 1025.47 & 
1123.62 & 704.51 & \bf{1030.56} & 
6.69 & 9.03\\CMT5Y & 1110.02 & 1055.47 & 
1117.78 & 758.73 & \bf{1031.69} & 
7.59 & 8.34\\CMT11X & 908.56 & 40.80 & 
914.40 & 38.80 & \bf{831.09} & 
9.32 & 10.02\\CMT11Y & 893.14 & 63.46 & 
904.82 & 58.56 & \bf{829.85} & 
7.63 & 9.03\\CMT12X & 682.69 & 30.59 & 
686.61 & 36.78 & \bf{658.83} & 
3.62 & 4.22\\CMT12Y & 678.93 & 37.66 & 
680.04 & 43.23 & \bf{660.47} & 
2.79 & 2.96\\\bf{PROM.} & 
\bf{784.73} & \bf{195.09} & \bf{791.60} & \bf{152.63} & \bf{749.50} & \bf{4.27} & \bf{5.08}\\[1ex]\hline
\end{tabular}
\label{table:SS-M-150-0.1-S}
\end{table}

\begin{table}[h]
\caption{Resultados de la ejecución de la metaheurística SS-M, utilizando instancias de SalhiNagy con la configuración -n 150.0 -b 10 -y 0.2}
\centering
\small
\begin{tabular}{c c c c c c c c}
\hline\hline
Instancia & Costo mínimo & Tiempo(seg.) & Costo promedio & Tiempo promedio(seg.) & CME & \%G & \%GP \\ [0.5ex]
\hline
CMT1X & 475.22 & 1.28 & 
480.87 & 1.85 & \bf{470.48} & 
1.01 & 2.21\\CMT1Y & 473.01 & 1.52 & 
473.87 & 1.44 & \bf{470.48} & 
0.54 & 0.72\\CMT2X & 704.90 & 17.29 & 
707.20 & 17.41 & \bf{682.39} & 
3.30 & 3.64\\CMT2Y & 695.04 & 15.84 & 
708.48 & 15.51 & \bf{682.39} & 
1.85 & 3.82\\CMT3X & 729.83 & 35.87 & 
739.69 & 38.77 & \bf{719.06} & 
1.50 & 2.87\\CMT3Y & 734.07 & 35.46 & 
737.18 & 39.56 & \bf{719.06} & 
2.09 & 2.52\\CMT4X & 897.78 & 201.25 & 
911.53 & 211.32 & \bf{854.21} & 
5.10 & 6.71\\CMT4Y & 905.39 & 197.35 & 
908.59 & 208.71 & \bf{852.46} & 
6.21 & 6.58\\CMT5X & 1102.97 & 1031.15 & 
1111.08 & 873.47 & \bf{1030.56} & 
7.03 & 7.81\\CMT5Y & 1091.65 & 694.26 & 
1107.65 & 823.40 & \bf{1031.69} & 
5.81 & 7.36\\CMT11X & 881.09 & 43.05 & 
898.58 & 36.48 & \bf{831.09} & 
6.02 & 8.12\\CMT11Y & 884.31 & 47.38 & 
909.57 & 42.63 & \bf{829.85} & 
6.56 & 9.61\\CMT12X & 680.80 & 35.51 & 
684.72 & 49.56 & \bf{658.83} & 
3.33 & 3.93\\CMT12Y & 675.76 & 95.29 & 
683.65 & 63.20 & \bf{660.47} & 
2.32 & 3.51\\\bf{PROM.} & 
\bf{780.84} & \bf{175.18} & \bf{790.19} & \bf{173.09} & \bf{749.50} & \bf{3.76} & \bf{4.96}\\[1ex]\hline
\end{tabular}
\label{table:SS-M-150-0.2-S}
\end{table}

\begin{table}[h]
\caption{Resultados de la ejecución de la metaheurística SS-M, utilizando instancias de SalhiNagy con la configuración -n 250.0 -b 10 -y 0.1}
\centering
\small
\begin{tabular}{c c c c c c c c}
\hline\hline
Instancia & Costo mínimo & Tiempo(seg.) & Costo promedio & Tiempo promedio(seg.) & CME & \%G & \%GP \\ [0.5ex]
\hline
CMT1X & 480.26 & 1.74 & 
480.26 & 1.75 & \bf{470.48} & 
2.08 & 2.08\\CMT1Y & 474.72 & 2.23 & 
474.72 & 2.38 & \bf{470.48} & 
0.90 & 0.90\\CMT2X & 705.95 & 21.29 & 
710.14 & 16.34 & \bf{682.39} & 
3.45 & 4.07\\CMT2Y & 699.34 & 11.63 & 
709.63 & 18.58 & \bf{682.39} & 
2.48 & 3.99\\CMT3X & 734.18 & 42.26 & 
741.47 & 36.02 & \bf{719.06} & 
2.10 & 3.12\\CMT3Y & 729.67 & 29.82 & 
733.91 & 31.25 & \bf{719.06} & 
1.48 & 2.07\\CMT4X & 896.72 & 281.80 & 
903.79 & 151.81 & \bf{854.21} & 
4.98 & 5.80\\CMT4Y & 898.85 & 179.21 & 
910.69 & 149.28 & \bf{852.46} & 
5.44 & 6.83\\CMT5X & 1097.08 & 801.31 & 
1113.54 & 1032.43 & \bf{1030.56} & 
6.45 & 8.05\\CMT5Y & 1135.61 & 933.42 & 
1115.89 & 850.22 & \bf{1031.69} & 
10.07 & 8.16\\CMT11X & 909.72 & 25.86 & 
914.96 & 20.85 & \bf{831.09} & 
9.46 & 10.09\\CMT11Y & 898.00 & 24.95 & 
904.78 & 37.72 & \bf{829.85} & 
8.21 & 9.03\\CMT12X & 680.36 & 63.91 & 
684.43 & 64.93 & \bf{658.83} & 
3.27 & 3.89\\CMT12Y & 674.58 & 85.38 & 
681.42 & 66.05 & \bf{660.47} & 
2.14 & 3.17\\\bf{PROM.} & 
\bf{786.79} & \bf{178.91} & \bf{791.40} & \bf{177.11} & \bf{749.50} & \bf{4.47} & \bf{5.09}\\[1ex]\hline
\end{tabular}
\label{table:SS-M-250-0.1-S}
\end{table}

\begin{table}[h]
\caption{Resultados de la ejecución de la metaheurística SS-M, utilizando instancias de SalhiNagy con la configuración -n 250.0 -b 10 -y .2}
\centering
\small
\begin{tabular}{c c c c c c c c}
\hline\hline
Instancia & Costo mínimo & Tiempo(seg.) & Costo promedio & Tiempo promedio(seg.) & CME & \%G & \%GP \\ [0.5ex]
\hline
CMT1X & 474.87 & 2.27 & 
476.14 & 2.45 & \bf{470.48} & 
0.93 & 1.20\\CMT1Y & 472.37 & 2.39 & 
478.13 & 2.29 & \bf{470.48} & 
0.40 & 1.63\\CMT2X & 711.02 & 13.50 & 
714.33 & 14.04 & \bf{682.39} & 
4.20 & 4.68\\CMT2Y & 703.34 & 28.24 & 
703.34 & 29.33 & \bf{682.39} & 
3.07 & 3.07\\CMT3X & 734.12 & 47.39 & 
735.71 & 44.76 & \bf{719.06} & 
2.09 & 2.32\\CMT3Y & 730.23 & 38.07 & 
737.06 & 28.14 & \bf{719.06} & 
1.55 & 2.50\\CMT4X & 901.41 & 180.22 & 
907.23 & 188.48 & \bf{854.21} & 
5.53 & 6.21\\CMT4Y & 903.43 & 260.59 & 
912.51 & 239.54 & \bf{852.46} & 
5.98 & 7.04\\CMT5X & 1095.96 & 735.86 & 
1109.12 & 1023.52 & \bf{1030.56} & 
6.35 & 7.62\\CMT5Y & 1102.60 & 653.38 & 
1113.41 & 760.59 & \bf{1031.69} & 
6.87 & 7.92\\CMT11X & 890.59 & 36.43 & 
903.66 & 40.21 & \bf{831.09} & 
7.16 & 8.73\\CMT11Y & 875.15 & 31.45 & 
893.36 & 29.15 & \bf{829.85} & 
5.46 & 7.65\\CMT12X & 681.87 & 81.61 & 
688.16 & 52.16 & \bf{658.83} & 
3.50 & 4.45\\CMT12Y & 675.89 & 81.87 & 
680.98 & 53.60 & \bf{660.47} & 
2.33 & 3.11\\\bf{PROM.} & 
\bf{782.35} & \bf{156.66} & \bf{789.51} & \bf{179.16} & \bf{749.50} & \bf{3.96} & \bf{4.87}\\[1ex]\hline
\end{tabular}
\label{table:SS-M-250-0.2-S}
\end{table}

\clearpage
\section{GTS-M}\label{tablas-entonacion-GTS-M}

\subsection{Dethloff}\label{tablas-entonacion-GTS-M-dethloff}

\begin{table}[h]
\caption{Resultados de la ejecución de la metaheurística GTS-M, utilizando instancias de Dethloff con la configuración -mni 3000 -lambda1 0.05 -lambda2 0.05 -tabu 13}
\centering
\small
\begin{tabular}{c c c c c c c c}
\hline\hline
Instancia & Costo mínimo & Tiempo(seg.) & Costo promedio & Tiempo promedio(seg.) & CME & \%G & \%GP \\ [0.5ex]
\hline
SCA3-0 & 636.06 & 1.68 & 
638.78 & 1.76 & \bf{635.62} & 
0.07 & 0.50\\SCA3-1 & \bf{697.84} & 1.98 & 
697.84 & 1.75 & 697.84 & 0.00
 & 0.00\\
SCA3-2 & \bf{659.34} & 3.00 & 
659.34 & 2.06 & 659.34 & 0.00
 & 0.00\\
SCA3-3 & 680.60 & 1.30 & 
683.17 & 1.60 & \bf{680.04} & 
0.08 & 0.46\\SCA3-4 & \bf{690.50} & 3.15 & 
690.50 & 2.23 & 690.50 & 0.00
 & 0.00\\
SCA3-5 & \bf{659.90} & 1.18 & 
666.85 & 1.69 & 659.90 & 0.00
 & 1.05\\SCA3-6 & \bf{651.09} & 2.22 & 
654.73 & 1.99 & 651.09 & 0.00
 & 0.56\\SCA3-7 & 664.88 & 1.29 & 
668.15 & 1.75 & \bf{659.17} & 
0.87 & 1.36\\SCA3-8 & \bf{719.47} & 2.70 & 
719.47 & 2.69 & 719.47 & 0.00
 & 0.00\\
SCA3-9 & \bf{681.00} & 2.41 & 
681.00 & 2.34 & 681.00 & 0.00
 & 0.00\\
SCA8-0 & \bf{961.50} & 3.28 & 
971.55 & 2.61 & 961.50 & 0.00
 & 1.05\\SCA8-1 & 1065.36 & 1.09 & 
1073.92 & 2.06 & \bf{1049.65} & 
1.50 & 2.31\\SCA8-2 & \bf{1039.64} & 2.16 & 
1049.75 & 2.19 & 1039.64 & 0.00
 & 0.97\\SCA8-3 & \bf{983.34} & 2.44 & 
1004.53 & 2.30 & 983.34 & 0.00
 & 2.15\\SCA8-4 & 1067.55 & 1.96 & 
1071.03 & 2.97 & \bf{1065.49} & 
0.19 & 0.52\\SCA8-5 & \bf{1027.08} & 3.95 & 
1051.81 & 2.18 & 1027.08 & 0.00
 & 2.41\\SCA8-6 & 972.48 & 2.15 & 
985.33 & 2.13 & \bf{971.82} & 
0.07 & 1.39\\SCA8-7 & 1063.22 & 1.59 & 
1082.05 & 2.04 & \bf{1051.28} & 
1.14 & 2.93\\SCA8-8 & \bf{1071.18} & 1.50 & 
1086.54 & 1.18 & 1071.18 & 0.00
 & 1.43\\SCA8-9 & 1063.68 & 2.76 & 
1076.48 & 1.80 & \bf{1060.50} & 
0.30 & 1.51\\CON3-0 & \bf{616.52} & 1.53 & 
625.33 & 2.27 & 616.52 & 0.00
 & 1.43\\CON3-1 & \bf{554.47} & 1.83 & 
558.11 & 2.15 & 554.47 & 0.00
 & 0.66\\CON3-2 & 519.61 & 2.32 & 
526.50 & 2.04 & \bf{518.00} & 
0.31 & 1.64\\CON3-3 & \bf{591.19} & 2.63 & 
591.19 & 1.70 & 591.19 & 0.00
 & 0.00\\
CON3-4 & 591.43 & 0.89 & 
597.97 & 1.50 & \bf{588.79} & 
0.45 & 1.56\\CON3-5 & \bf{563.70} & 1.00 & 
574.71 & 1.49 & 563.70 & 0.00
 & 1.95\\CON3-6 & \bf{499.05} & 2.83 & 
501.54 & 1.97 & 499.05 & 0.00
 & 0.50\\CON3-7 & \bf{576.48} & 4.16 & 
582.58 & 2.89 & 576.48 & 0.00
 & 1.06\\CON3-8 & \bf{523.05} & 2.26 & 
523.05 & 2.05 & 523.05 & 0.00
 & 0.00\\
CON3-9 & 582.79 & 2.04 & 
590.94 & 1.83 & \bf{578.24} & 
0.79 & 2.20\\CON8-0 & 857.40 & 6.12 & 
867.25 & 3.27 & \bf{857.17} & 
0.03 & 1.18\\CON8-1 & 752.18 & 3.52 & 
771.45 & 2.56 & \bf{740.85} & 
1.53 & 4.13\\CON8-2 & 718.64 & 4.48 & 
728.78 & 3.49 & \bf{712.89} & 
0.81 & 2.23\\CON8-3 & 811.23 & 1.46 & 
830.33 & 2.80 & \bf{811.07} & 
0.02 & 2.37\\CON8-4 & \bf{772.25} & 1.71 & 
775.62 & 1.80 & 772.25 & 0.00
 & 0.44\\CON8-5 & 756.91 & 2.42 & 
762.33 & 2.10 & \bf{754.88} & 
0.27 & 0.99\\CON8-6 & 688.68 & 1.92 & 
698.43 & 1.74 & \bf{678.92} & 
1.44 & 2.87\\CON8-7 & 814.79 & 5.72 & 
817.90 & 3.21 & \bf{811.96} & 
0.35 & 0.73\\CON8-8 & \bf{767.53} & 2.42 & 
774.29 & 2.81 & 767.53 & 0.00
 & 0.88\\CON8-9 & 812.23 & 3.96 & 
831.40 & 2.38 & \bf{809.00} & 
0.40 & 2.77\\\bf{PROM.} & 
\bf{760.65} & \bf{2.48} & \bf{768.56} & \bf{2.18} & \bf{758.54} & \bf{0.26} & \bf{1.25}\\[1ex]\hline
\end{tabular}
\label{table:GTS-M-dethloff-3000-13}
\end{table}
\begin{table}[h]
\caption{Resultados de la ejecución de la metaheurística GTS-M, utilizando instancias de Dethloff con la configuración -mni 3000 -lambda1 0.05 -lambda2 0.05 -tabu 29}
\centering
\small
\begin{tabular}{c c c c c c c c}
\hline\hline
Instancia & Costo mínimo & Tiempo(seg.) & Costo promedio & Tiempo promedio(seg.) & CME & \%G & \%GP \\ [0.5ex]
\hline
SCA3-0 & 640.55 & 1.53 & 
640.55 & 2.19 & \bf{635.62} & 
0.78 & 0.78\\SCA3-1 & \bf{697.84} & 2.10 & 
699.17 & 2.77 & 697.84 & 0.00
 & 0.19\\SCA3-2 & \bf{659.34} & 1.53 & 
659.34 & 2.44 & 659.34 & 0.00
 & 0.00\\
SCA3-3 & \bf{680.04} & 1.32 & 
680.32 & 2.09 & 680.04 & 0.00
 & 0.04\\SCA3-4 & \bf{690.50} & 2.52 & 
690.50 & 2.82 & 690.50 & 0.00
 & 0.00\\
SCA3-5 & \bf{659.90} & 3.39 & 
666.42 & 2.09 & 659.90 & 0.00
 & 0.99\\SCA3-6 & \bf{651.09} & 2.53 & 
651.37 & 3.03 & 651.09 & 0.00
 & 0.04\\SCA3-7 & 666.15 & 1.97 & 
667.09 & 2.08 & \bf{659.17} & 
1.06 & 1.20\\SCA3-8 & \bf{719.47} & 2.39 & 
719.47 & 2.67 & 719.47 & 0.00
 & 0.00\\
SCA3-9 & \bf{681.00} & 2.46 & 
681.00 & 1.76 & 681.00 & 0.00
 & 0.00\\
SCA8-0 & \bf{961.50} & 2.70 & 
981.89 & 2.04 & 961.50 & 0.00
 & 2.12\\SCA8-1 & 1050.38 & 1.64 & 
1060.47 & 2.48 & \bf{1049.65} & 
0.07 & 1.03\\SCA8-2 & 1039.71 & 1.45 & 
1042.99 & 1.62 & \bf{1039.64} & 
0.01 & 0.32\\SCA8-3 & 1002.38 & 1.44 & 
1012.59 & 1.50 & \bf{983.34} & 
1.94 & 2.97\\SCA8-4 & 1067.28 & 1.58 & 
1068.64 & 2.40 & \bf{1065.49} & 
0.17 & 0.30\\SCA8-5 & 1042.43 & 1.48 & 
1058.44 & 1.93 & \bf{1027.08} & 
1.49 & 3.05\\SCA8-6 & 972.48 & 1.66 & 
977.43 & 2.14 & \bf{971.82} & 
0.07 & 0.58\\SCA8-7 & 1053.84 & 1.65 & 
1067.64 & 3.18 & \bf{1051.28} & 
0.24 & 1.56\\SCA8-8 & 1082.12 & 3.28 & 
1091.10 & 1.88 & \bf{1071.18} & 
1.02 & 1.86\\SCA8-9 & \bf{1060.50} & 8.53 & 
1070.11 & 3.91 & 1060.50 & 0.00
 & 0.91\\CON3-0 & \bf{616.52} & 2.40 & 
626.18 & 1.75 & 616.52 & 0.00
 & 1.57\\CON3-1 & 556.28 & 4.71 & 
558.74 & 2.79 & \bf{554.47} & 
0.33 & 0.77\\CON3-2 & 519.61 & 2.08 & 
523.94 & 1.96 & \bf{518.00} & 
0.31 & 1.15\\CON3-3 & \bf{591.19} & 4.78 & 
603.40 & 3.71 & 591.19 & 0.00
 & 2.06\\CON3-4 & \bf{588.79} & 2.42 & 
595.61 & 2.73 & 588.79 & 0.00
 & 1.16\\CON3-5 & \bf{563.70} & 1.09 & 
569.64 & 1.26 & 563.70 & 0.00
 & 1.05\\CON3-6 & \bf{499.05} & 2.41 & 
500.40 & 2.32 & 499.05 & 0.00
 & 0.27\\CON3-7 & \bf{576.48} & 2.18 & 
586.02 & 2.38 & 576.48 & 0.00
 & 1.65\\CON3-8 & \bf{523.05} & 2.11 & 
523.21 & 1.93 & 523.05 & 0.00
 & 0.03\\CON3-9 & 578.25 & 3.23 & 
586.41 & 2.55 & \bf{578.24} & 
0.00 & 1.41\\CON8-0 & 857.40 & 1.97 & 
868.43 & 5.08 & \bf{857.17} & 
0.03 & 1.31\\CON8-1 & \bf{740.85} & 3.10 & 
756.38 & 3.14 & 740.85 & 0.00
 & 2.10\\CON8-2 & \bf{712.89} & 2.00 & 
727.86 & 3.02 & 712.89 & 0.00
 & 2.10\\CON8-3 & 826.12 & 3.29 & 
828.89 & 1.98 & \bf{811.07} & 
1.86 & 2.20\\CON8-4 & \bf{772.25} & 2.58 & 
778.42 & 3.17 & 772.25 & 0.00
 & 0.80\\CON8-5 & 759.93 & 2.26 & 
760.17 & 2.80 & \bf{754.88} & 
0.67 & 0.70\\CON8-6 & \bf{678.92} & 1.33 & 
689.98 & 2.01 & 678.92 & 0.00
 & 1.63\\CON8-7 & 813.91 & 3.88 & 
837.75 & 2.33 & \bf{811.96} & 
0.24 & 3.18\\CON8-8 & \bf{767.53} & 2.30 & 
772.67 & 1.95 & 767.53 & 0.00
 & 0.67\\CON8-9 & \bf{809.00} & 4.25 & 
817.08 & 2.88 & 809.00 & 0.00
 & 1.00\\\bf{PROM.} & 
\bf{760.76} & \bf{2.54} & \bf{767.44} & \bf{2.47} & \bf{758.54} & \bf{0.26} & \bf{1.12}\\[1ex]\hline
\end{tabular}
\label{table:nonlin}
\end{table}
\begin{table}[h]
\caption{Resultados de la ejecución de la metaheurística GTS-M, utilizando instancias de Dethloff con la configuración -mni 3000 -lambda1 0.05 -lambda2 0.05 -tabu 37}
\centering
\small
\begin{tabular}{c c c c c c c c}
\hline\hline
Instancia & Costo mínimo & Tiempo(seg.) & Costo promedio & Tiempo promedio(seg.) & CME & \%G & \%GP \\ [0.5ex]
\hline
SCA3-0 & 636.45 & 1.81 & 
638.59 & 2.32 & \bf{635.62} & 
0.13 & 0.47\\SCA3-1 & \bf{697.84} & 4.87 & 
700.09 & 2.52 & 697.84 & 0.00
 & 0.32\\SCA3-2 & \bf{659.34} & 1.75 & 
666.01 & 1.71 & 659.34 & 0.00
 & 1.01\\SCA3-3 & \bf{680.04} & 3.23 & 
682.94 & 3.17 & 680.04 & 0.00
 & 0.43\\SCA3-4 & \bf{690.50} & 1.58 & 
690.50 & 3.13 & 690.50 & 0.00
 & 0.00\\
SCA3-5 & \bf{659.90} & 2.10 & 
663.16 & 2.10 & 659.90 & 0.00
 & 0.49\\SCA3-6 & \bf{651.09} & 1.11 & 
656.65 & 1.31 & 651.09 & 0.00
 & 0.85\\SCA3-7 & 666.15 & 2.02 & 
667.09 & 2.38 & \bf{659.17} & 
1.06 & 1.20\\SCA3-8 & \bf{719.47} & 3.56 & 
726.01 & 2.58 & 719.47 & 0.00
 & 0.91\\SCA3-9 & \bf{681.00} & 3.92 & 
681.00 & 3.08 & 681.00 & 0.00
 & 0.00\\
SCA8-0 & \bf{961.50} & 4.69 & 
982.80 & 2.83 & 961.50 & 0.00
 & 2.22\\SCA8-1 & \bf{1049.65} & 7.26 & 
1066.14 & 3.63 & 1049.65 & 0.00
 & 1.57\\SCA8-2 & 1050.37 & 2.40 & 
1062.19 & 2.53 & \bf{1039.64} & 
1.03 & 2.17\\SCA8-3 & \bf{983.34} & 3.56 & 
1007.40 & 2.59 & 983.34 & 0.00
 & 2.45\\SCA8-4 & \bf{1065.49} & 4.02 & 
1068.96 & 3.12 & 1065.49 & 0.00
 & 0.33\\SCA8-5 & \bf{1027.08} & 1.75 & 
1041.95 & 1.62 & 1027.08 & 0.00
 & 1.45\\SCA8-6 & 972.48 & 3.38 & 
972.48 & 2.22 & \bf{971.82} & 
0.07 & 0.07\\SCA8-7 & 1052.17 & 4.70 & 
1064.93 & 3.14 & \bf{1051.28} & 
0.08 & 1.30\\SCA8-8 & 1082.12 & 3.95 & 
1083.40 & 1.96 & \bf{1071.18} & 
1.02 & 1.14\\SCA8-9 & \bf{1060.50} & 5.33 & 
1065.17 & 3.69 & 1060.50 & 0.00
 & 0.44\\CON3-0 & 617.59 & 2.02 & 
626.84 & 1.90 & \bf{616.52} & 
0.17 & 1.67\\CON3-1 & \bf{554.47} & 1.44 & 
555.36 & 1.95 & 554.47 & 0.00
 & 0.16\\CON3-2 & 519.61 & 2.59 & 
522.07 & 2.56 & \bf{518.00} & 
0.31 & 0.79\\CON3-3 & \bf{591.19} & 3.58 & 
597.89 & 2.99 & 591.19 & 0.00
 & 1.13\\CON3-4 & \bf{588.79} & 2.63 & 
594.81 & 2.04 & 588.79 & 0.00
 & 1.02\\CON3-5 & \bf{563.70} & 2.42 & 
567.38 & 2.29 & 563.70 & 0.00
 & 0.65\\CON3-6 & 502.16 & 2.77 & 
505.10 & 2.11 & \bf{499.05} & 
0.62 & 1.21\\CON3-7 & \bf{576.48} & 2.47 & 
576.96 & 3.83 & 576.48 & 0.00
 & 0.08\\CON3-8 & \bf{523.05} & 1.14 & 
523.05 & 2.38 & 523.05 & 0.00
 & 0.00\\
CON3-9 & 578.25 & 2.48 & 
579.57 & 3.41 & \bf{578.24} & 
0.00 & 0.23\\CON8-0 & 867.04 & 2.39 & 
878.76 & 2.84 & \bf{857.17} & 
1.15 & 2.52\\CON8-1 & \bf{740.85} & 4.23 & 
750.94 & 2.62 & 740.85 & 0.00
 & 1.36\\CON8-2 & 716.03 & 5.23 & 
725.38 & 3.61 & \bf{712.89} & 
0.44 & 1.75\\CON8-3 & \bf{811.07} & 4.02 & 
825.47 & 3.39 & 811.07 & 0.00
 & 1.77\\CON8-4 & \bf{772.25} & 3.40 & 
780.75 & 2.90 & 772.25 & 0.00
 & 1.10\\CON8-5 & 755.14 & 6.35 & 
761.58 & 3.08 & \bf{754.88} & 
0.03 & 0.89\\CON8-6 & 692.75 & 2.76 & 
700.62 & 2.30 & \bf{678.92} & 
2.04 & 3.20\\CON8-7 & 812.89 & 4.42 & 
817.78 & 3.16 & \bf{811.96} & 
0.11 & 0.72\\CON8-8 & \bf{767.53} & 3.50 & 
778.01 & 2.54 & 767.53 & 0.00
 & 1.37\\CON8-9 & 812.03 & 2.64 & 
815.42 & 2.50 & \bf{809.00} & 
0.37 & 0.79\\\bf{PROM.} & 
\bf{760.23} & \bf{3.24} & \bf{766.78} & \bf{2.65} & \bf{758.54} & \bf{0.22} & \bf{1.03}\\[1ex]\hline
\end{tabular}
\label{table:nonlin}
\end{table}
\begin{table}[h]
\caption{Resultados de la ejecución de la metaheurística GTS-M, utilizando instancias de Dethloff con la configuración -mni 3000 -lambda1 0.05 -lambda2 0.05 -tabu 5}
\centering
\small
\begin{tabular}{c c c c c c c c}
\hline\hline
Instancia & Costo mínimo & Tiempo(seg.) & Costo promedio & Tiempo promedio(seg.) & CME & \%G & \%GP \\ [0.5ex]
\hline
SCA3-0 & 640.55 & 6.08 & 
640.55 & 3.58 & \bf{635.62} & 
0.78 & 0.78\\SCA3-1 & \bf{697.84} & 2.35 & 
699.17 & 1.67 & 697.84 & 0.00
 & 0.19\\SCA3-2 & \bf{659.34} & 1.77 & 
659.34 & 2.47 & 659.34 & 0.00
 & 0.00\\
SCA3-3 & \bf{680.04} & 1.16 & 
680.46 & 2.00 & 680.04 & 0.00
 & 0.06\\SCA3-4 & \bf{690.50} & 2.32 & 
699.27 & 1.84 & 690.50 & 0.00
 & 1.27\\SCA3-5 & \bf{659.90} & 1.78 & 
670.22 & 1.37 & 659.90 & 0.00
 & 1.56\\SCA3-6 & \bf{651.09} & 1.62 & 
654.81 & 2.02 & 651.09 & 0.00
 & 0.57\\SCA3-7 & 666.15 & 1.14 & 
667.83 & 1.45 & \bf{659.17} & 
1.06 & 1.31\\SCA3-8 & \bf{719.47} & 2.53 & 
719.47 & 1.88 & 719.47 & 0.00
 & 0.00\\
SCA3-9 & \bf{681.00} & 1.62 & 
686.34 & 1.18 & 681.00 & 0.00
 & 0.78\\SCA8-0 & 970.64 & 3.35 & 
984.62 & 2.35 & \bf{961.50} & 
0.95 & 2.40\\SCA8-1 & 1067.45 & 2.80 & 
1072.91 & 1.99 & \bf{1049.65} & 
1.70 & 2.22\\SCA8-2 & 1042.17 & 2.52 & 
1053.30 & 1.98 & \bf{1039.64} & 
0.24 & 1.31\\SCA8-3 & \bf{983.34} & 2.00 & 
1004.75 & 1.61 & 983.34 & 0.00
 & 2.18\\SCA8-4 & 1067.29 & 2.74 & 
1072.30 & 1.89 & \bf{1065.49} & 
0.17 & 0.64\\SCA8-5 & 1048.65 & 3.40 & 
1075.27 & 1.88 & \bf{1027.08} & 
2.10 & 4.69\\SCA8-6 & 972.48 & 4.31 & 
987.13 & 2.27 & \bf{971.82} & 
0.07 & 1.58\\SCA8-7 & 1066.65 & 1.31 & 
1075.46 & 2.01 & \bf{1051.28} & 
1.46 & 2.30\\SCA8-8 & 1082.12 & 1.87 & 
1083.19 & 1.45 & \bf{1071.18} & 
1.02 & 1.12\\SCA8-9 & \bf{1060.50} & 2.94 & 
1065.52 & 2.39 & 1060.50 & 0.00
 & 0.47\\CON3-0 & \bf{616.52} & 1.94 & 
624.04 & 1.78 & 616.52 & 0.00
 & 1.22\\CON3-1 & 556.04 & 3.02 & 
558.39 & 1.62 & \bf{554.47} & 
0.28 & 0.71\\CON3-2 & 523.23 & 2.45 & 
526.01 & 1.59 & \bf{518.00} & 
1.01 & 1.55\\CON3-3 & \bf{591.19} & 1.66 & 
612.59 & 1.47 & 591.19 & 0.00
 & 3.62\\CON3-4 & 591.43 & 3.68 & 
598.15 & 1.78 & \bf{588.79} & 
0.45 & 1.59\\CON3-5 & \bf{563.70} & 2.17 & 
569.89 & 2.14 & 563.70 & 0.00
 & 1.10\\CON3-6 & \bf{499.05} & 1.85 & 
501.86 & 1.98 & 499.05 & 0.00
 & 0.56\\CON3-7 & \bf{576.48} & 1.18 & 
586.83 & 1.79 & 576.48 & 0.00
 & 1.79\\CON3-8 & \bf{523.05} & 2.60 & 
523.05 & 1.45 & 523.05 & 0.00
 & 0.00\\
CON3-9 & 587.23 & 1.32 & 
592.85 & 1.70 & \bf{578.24} & 
1.55 & 2.53\\CON8-0 & 866.68 & 1.81 & 
890.44 & 2.18 & \bf{857.17} & 
1.11 & 3.88\\CON8-1 & \bf{740.85} & 1.73 & 
757.38 & 2.02 & 740.85 & 0.00
 & 2.23\\CON8-2 & 718.64 & 4.03 & 
737.71 & 3.58 & \bf{712.89} & 
0.81 & 3.48\\CON8-3 & \bf{811.07} & 2.22 & 
827.27 & 3.39 & 811.07 & 0.00
 & 2.00\\CON8-4 & \bf{772.25} & 2.06 & 
790.01 & 2.17 & 772.25 & 0.00
 & 2.30\\CON8-5 & \bf{754.88} & 5.19 & 
759.27 & 2.79 & 754.88 & 0.00
 & 0.58\\CON8-6 & 690.58 & 2.67 & 
700.02 & 2.13 & \bf{678.92} & 
1.72 & 3.11\\CON8-7 & 814.79 & 1.90 & 
832.91 & 1.52 & \bf{811.96} & 
0.35 & 2.58\\CON8-8 & 776.55 & 1.73 & 
790.69 & 1.92 & \bf{767.53} & 
1.18 & 3.02\\CON8-9 & \bf{809.00} & 1.78 & 
813.65 & 3.04 & 809.00 & 0.00
 & 0.58\\\bf{PROM.} & 
\bf{762.26} & \bf{2.42} & \bf{771.12} & \bf{2.03} & \bf{758.54} & \bf{0.45} & \bf{1.60}\\[1ex]\hline
\end{tabular}
\label{table:nonlin}
\end{table}
\begin{table}[h]
\caption{Resultados de la ejecución de la metaheurística GTS-M, utilizando instancias de Dethloff con la configuración -mni 5000 -lambda1 0.05 -lambda2 0.05 -tabu 13}
\centering
\small
\begin{tabular}{c c c c c c c c}
\hline\hline
Instancia & Costo mínimo & Tiempo(seg.) & Costo promedio & Tiempo promedio(seg.) & CME & \%G & \%GP \\ [0.5ex]
\hline
SCA3-0 & 636.06 & 5.80 & 
637.18 & 3.06 & \bf{635.62} & 
0.07 & 0.25\\SCA3-1 & \bf{697.84} & 6.81 & 
699.17 & 3.93 & 697.84 & 0.00
 & 0.19\\SCA3-2 & \bf{659.34} & 3.19 & 
659.34 & 4.96 & 659.34 & 0.00
 & 0.00\\
SCA3-3 & \bf{680.04} & 2.87 & 
687.37 & 4.00 & 680.04 & 0.00
 & 1.08\\SCA3-4 & \bf{690.50} & 3.27 & 
690.50 & 3.82 & 690.50 & 0.00
 & 0.00\\
SCA3-5 & \bf{659.90} & 4.34 & 
663.16 & 3.47 & 659.90 & 0.00
 & 0.49\\SCA3-6 & \bf{651.09} & 2.50 & 
651.09 & 2.76 & 651.09 & 0.00
 & 0.00\\
SCA3-7 & 666.15 & 4.47 & 
666.15 & 4.84 & \bf{659.17} & 
1.06 & 1.06\\SCA3-8 & \bf{719.47} & 7.40 & 
722.50 & 3.81 & 719.47 & 0.00
 & 0.42\\SCA3-9 & \bf{681.00} & 3.28 & 
681.00 & 3.89 & 681.00 & 0.00
 & 0.00\\
SCA8-0 & 970.64 & 3.92 & 
979.52 & 3.00 & \bf{961.50} & 
0.95 & 1.87\\SCA8-1 & 1050.20 & 4.36 & 
1058.59 & 3.19 & \bf{1049.65} & 
0.05 & 0.85\\SCA8-2 & 1042.10 & 6.18 & 
1064.62 & 4.29 & \bf{1039.64} & 
0.24 & 2.40\\SCA8-3 & 1007.85 & 4.88 & 
1013.70 & 3.55 & \bf{983.34} & 
2.49 & 3.09\\SCA8-4 & 1067.28 & 2.62 & 
1071.35 & 3.35 & \bf{1065.49} & 
0.17 & 0.55\\SCA8-5 & 1042.30 & 2.62 & 
1048.70 & 3.85 & \bf{1027.08} & 
1.48 & 2.10\\SCA8-6 & 972.48 & 1.84 & 
972.48 & 3.15 & \bf{971.82} & 
0.07 & 0.07\\SCA8-7 & 1070.72 & 7.45 & 
1072.75 & 4.70 & \bf{1051.28} & 
1.85 & 2.04\\SCA8-8 & \bf{1071.18} & 5.53 & 
1083.67 & 3.57 & 1071.18 & 0.00
 & 1.17\\SCA8-9 & \bf{1060.50} & 2.86 & 
1063.96 & 4.52 & 1060.50 & 0.00
 & 0.33\\CON3-0 & \bf{616.52} & 1.88 & 
619.75 & 2.71 & 616.52 & 0.00
 & 0.52\\CON3-1 & \bf{554.47} & 2.49 & 
557.61 & 3.95 & 554.47 & 0.00
 & 0.57\\CON3-2 & 523.23 & 3.93 & 
523.35 & 3.54 & \bf{518.00} & 
1.01 & 1.03\\CON3-3 & \bf{591.19} & 4.92 & 
594.58 & 3.25 & 591.19 & 0.00
 & 0.57\\CON3-4 & \bf{588.79} & 2.34 & 
595.12 & 3.71 & 588.79 & 0.00
 & 1.08\\CON3-5 & \bf{563.70} & 1.74 & 
565.92 & 2.94 & 563.70 & 0.00
 & 0.39\\CON3-6 & \bf{499.05} & 3.84 & 
501.60 & 3.54 & 499.05 & 0.00
 & 0.51\\CON3-7 & \bf{576.48} & 9.98 & 
581.75 & 5.44 & 576.48 & 0.00
 & 0.91\\CON3-8 & \bf{523.05} & 2.82 & 
523.05 & 2.22 & 523.05 & 0.00
 & 0.00\\
CON3-9 & 578.25 & 5.48 & 
585.61 & 5.61 & \bf{578.24} & 
0.00 & 1.27\\CON8-0 & \bf{857.17} & 5.78 & 
864.60 & 4.71 & 857.17 & 0.00
 & 0.87\\CON8-1 & \bf{740.85} & 4.56 & 
766.76 & 3.18 & 740.85 & 0.00
 & 3.50\\CON8-2 & 718.64 & 9.61 & 
723.54 & 4.13 & \bf{712.89} & 
0.81 & 1.49\\CON8-3 & \bf{811.07} & 5.86 & 
811.11 & 3.66 & 811.07 & 0.00
 & 0.00\\CON8-4 & \bf{772.25} & 6.79 & 
776.01 & 4.61 & 772.25 & 0.00
 & 0.49\\CON8-5 & 754.95 & 4.72 & 
759.31 & 4.67 & \bf{754.88} & 
0.01 & 0.59\\CON8-6 & 685.45 & 2.10 & 
697.00 & 3.21 & \bf{678.92} & 
0.96 & 2.66\\CON8-7 & 813.31 & 4.18 & 
826.56 & 3.42 & \bf{811.96} & 
0.17 & 1.80\\CON8-8 & \bf{767.53} & 2.50 & 
785.13 & 3.16 & 767.53 & 0.00
 & 2.29\\CON8-9 & 811.14 & 4.01 & 
814.90 & 4.00 & \bf{809.00} & 
0.26 & 0.73\\\bf{PROM.} & 
\bf{761.09} & \bf{4.39} & \bf{766.50} & \bf{3.78} & \bf{758.54} & \bf{0.29} & \bf{0.98}\\[1ex]\hline
\end{tabular}
\label{table:nonlin}
\end{table}
\begin{table}[h]
\caption{Resultados de la ejecución de la metaheurística GTS-M, utilizando instancias de Dethloff con la configuración -mni 5000 -lambda1 0.05 -lambda2 0.05 -tabu 29}
\centering
\small
\begin{tabular}{c c c c c c c c}
\hline\hline
Instancia & Costo mínimo & Tiempo(seg.) & Costo promedio & Tiempo promedio(seg.) & CME & \%G & \%GP \\ [0.5ex]
\hline
SCA3-0 & 636.06 & 3.06 & 
638.30 & 3.88 & \bf{635.62} & 
0.07 & 0.42\\SCA3-1 & \bf{697.84} & 2.10 & 
698.50 & 3.98 & 697.84 & 0.00
 & 0.10\\SCA3-2 & \bf{659.34} & 2.44 & 
659.34 & 3.67 & 659.34 & 0.00
 & 0.00\\
SCA3-3 & \bf{680.04} & 8.85 & 
680.32 & 5.89 & 680.04 & 0.00
 & 0.04\\SCA3-4 & \bf{690.50} & 10.47 & 
690.50 & 7.42 & 690.50 & 0.00
 & 0.00\\
SCA3-5 & \bf{659.90} & 3.75 & 
663.16 & 4.25 & 659.90 & 0.00
 & 0.49\\SCA3-6 & \bf{651.09} & 4.28 & 
651.09 & 4.32 & 651.09 & 0.00
 & 0.00\\
SCA3-7 & 666.15 & 2.59 & 
667.09 & 2.99 & \bf{659.17} & 
1.06 & 1.20\\SCA3-8 & \bf{719.47} & 5.04 & 
721.90 & 5.95 & 719.47 & 0.00
 & 0.34\\SCA3-9 & \bf{681.00} & 5.14 & 
681.00 & 4.54 & 681.00 & 0.00
 & 0.00\\
SCA8-0 & \bf{961.50} & 5.78 & 
975.16 & 3.68 & 961.50 & 0.00
 & 1.42\\SCA8-1 & 1050.20 & 4.29 & 
1062.09 & 3.39 & \bf{1049.65} & 
0.05 & 1.19\\SCA8-2 & \bf{1039.64} & 4.32 & 
1055.00 & 4.32 & 1039.64 & 0.00
 & 1.48\\SCA8-3 & \bf{983.34} & 6.90 & 
990.89 & 6.25 & 983.34 & 0.00
 & 0.77\\SCA8-4 & 1067.55 & 2.50 & 
1068.70 & 3.99 & \bf{1065.49} & 
0.19 & 0.30\\SCA8-5 & \bf{1027.08} & 4.50 & 
1042.58 & 4.23 & 1027.08 & 0.00
 & 1.51\\SCA8-6 & \bf{971.82} & 2.18 & 
972.32 & 2.72 & 971.82 & 0.00
 & 0.05\\SCA8-7 & 1062.66 & 2.87 & 
1066.24 & 4.37 & \bf{1051.28} & 
1.08 & 1.42\\SCA8-8 & \bf{1071.18} & 5.92 & 
1073.91 & 4.16 & 1071.18 & 0.00
 & 0.25\\SCA8-9 & \bf{1060.50} & 3.65 & 
1063.91 & 3.69 & 1060.50 & 0.00
 & 0.32\\CON3-0 & \bf{616.52} & 7.00 & 
620.84 & 5.05 & 616.52 & 0.00
 & 0.70\\CON3-1 & \bf{554.47} & 6.44 & 
556.43 & 5.21 & 554.47 & 0.00
 & 0.35\\CON3-2 & \bf{518.00} & 3.76 & 
521.13 & 5.85 & 518.00 & 0.00
 & 0.61\\CON3-3 & \bf{591.19} & 6.58 & 
591.19 & 7.26 & 591.19 & 0.00
 & 0.00\\
CON3-4 & \bf{588.79} & 3.25 & 
593.38 & 3.59 & 588.79 & 0.00
 & 0.78\\CON3-5 & \bf{563.70} & 4.65 & 
567.42 & 3.48 & 563.70 & 0.00
 & 0.66\\CON3-6 & \bf{499.05} & 4.15 & 
500.72 & 5.99 & 499.05 & 0.00
 & 0.34\\CON3-7 & \bf{576.48} & 5.39 & 
576.48 & 5.95 & 576.48 & 0.00
 & 0.00\\
CON3-8 & \bf{523.05} & 4.17 & 
523.05 & 4.02 & 523.05 & 0.00
 & 0.00\\
CON3-9 & 578.98 & 3.39 & 
584.86 & 4.11 & \bf{578.24} & 
0.13 & 1.15\\CON8-0 & 857.38 & 1.92 & 
863.91 & 3.57 & \bf{857.17} & 
0.02 & 0.79\\CON8-1 & \bf{740.85} & 2.77 & 
753.43 & 4.88 & 740.85 & 0.00
 & 1.70\\CON8-2 & \bf{712.89} & 9.53 & 
718.66 & 5.56 & 712.89 & 0.00
 & 0.81\\CON8-3 & \bf{811.07} & 7.72 & 
845.44 & 5.38 & 811.07 & 0.00
 & 4.24\\CON8-4 & \bf{772.25} & 3.12 & 
783.80 & 3.67 & 772.25 & 0.00
 & 1.50\\CON8-5 & \bf{754.88} & 3.23 & 
756.04 & 4.84 & 754.88 & 0.00
 & 0.15\\CON8-6 & \bf{678.92} & 4.15 & 
691.58 & 4.79 & 678.92 & 0.00
 & 1.86\\CON8-7 & 812.89 & 7.68 & 
813.84 & 4.25 & \bf{811.96} & 
0.11 & 0.23\\CON8-8 & \bf{767.53} & 6.04 & 
794.22 & 4.30 & 767.53 & 0.00
 & 3.48\\CON8-9 & \bf{809.00} & 4.66 & 
816.45 & 4.35 & 809.00 & 0.00
 & 0.92\\\bf{PROM.} & 
\bf{759.12} & \bf{4.76} & \bf{764.87} & \bf{4.59} & \bf{758.54} & \bf{0.07} & \bf{0.79}\\[1ex]\hline
\end{tabular}
\label{table:nonlin}
\end{table}
\begin{table}[h]
\caption{Resultados de la ejecución de la metaheurística GTS-M, utilizando instancias de Dethloff con la configuración -mni 5000 -lambda1 0.05 -lambda2 0.05 -tabu 37}
\centering
\small
\begin{tabular}{c c c c c c c c}
\hline\hline
Instancia & Costo mínimo & Tiempo(seg.) & Costo promedio & Tiempo promedio(seg.) & CME & \%G & \%GP \\ [0.5ex]
\hline
SCA3-0 & 636.06 & 7.16 & 
637.18 & 5.71 & \bf{635.62} & 
0.07 & 0.25\\SCA3-1 & \bf{697.84} & 4.38 & 
697.84 & 4.59 & 697.84 & 0.00
 & 0.00\\
SCA3-2 & \bf{659.34} & 3.73 & 
659.34 & 4.69 & 659.34 & 0.00
 & 0.00\\
SCA3-3 & \bf{680.04} & 6.40 & 
680.46 & 6.03 & 680.04 & 0.00
 & 0.06\\SCA3-4 & \bf{690.50} & 8.30 & 
690.50 & 5.34 & 690.50 & 0.00
 & 0.00\\
SCA3-5 & \bf{659.90} & 3.40 & 
663.16 & 4.40 & 659.90 & 0.00
 & 0.49\\SCA3-6 & \bf{651.09} & 4.08 & 
656.49 & 3.56 & 651.09 & 0.00
 & 0.83\\SCA3-7 & 666.15 & 2.75 & 
666.26 & 5.43 & \bf{659.17} & 
1.06 & 1.08\\SCA3-8 & \bf{719.47} & 8.67 & 
719.47 & 5.46 & 719.47 & 0.00
 & 0.00\\
SCA3-9 & \bf{681.00} & 2.54 & 
681.00 & 3.60 & 681.00 & 0.00
 & 0.00\\
SCA8-0 & \bf{961.50} & 3.82 & 
984.43 & 4.53 & 961.50 & 0.00
 & 2.39\\SCA8-1 & \bf{1049.65} & 6.98 & 
1058.93 & 4.28 & 1049.65 & 0.00
 & 0.88\\SCA8-2 & \bf{1039.64} & 8.42 & 
1051.15 & 5.54 & 1039.64 & 0.00
 & 1.11\\SCA8-3 & \bf{983.34} & 12.80 & 
990.89 & 7.06 & 983.34 & 0.00
 & 0.77\\SCA8-4 & 1068.97 & 6.31 & 
1071.20 & 5.15 & \bf{1065.49} & 
0.33 & 0.54\\SCA8-5 & \bf{1027.08} & 5.18 & 
1043.38 & 4.48 & 1027.08 & 0.00
 & 1.59\\SCA8-6 & 972.48 & 3.74 & 
976.62 & 3.62 & \bf{971.82} & 
0.07 & 0.49\\SCA8-7 & \bf{1051.28} & 3.65 & 
1063.92 & 5.83 & 1051.28 & 0.00
 & 1.20\\SCA8-8 & \bf{1071.18} & 2.40 & 
1073.91 & 5.11 & 1071.18 & 0.00
 & 0.25\\SCA8-9 & \bf{1060.50} & 3.55 & 
1063.70 & 5.04 & 1060.50 & 0.00
 & 0.30\\CON3-0 & \bf{616.52} & 4.25 & 
622.47 & 6.22 & 616.52 & 0.00
 & 0.96\\CON3-1 & \bf{554.47} & 4.86 & 
556.16 & 6.07 & 554.47 & 0.00
 & 0.30\\CON3-2 & 521.33 & 8.78 & 
522.75 & 5.00 & \bf{518.00} & 
0.64 & 0.92\\CON3-3 & \bf{591.19} & 5.06 & 
591.19 & 6.50 & 591.19 & 0.00
 & 0.00\\
CON3-4 & \bf{588.79} & 5.12 & 
597.23 & 4.52 & 588.79 & 0.00
 & 1.43\\CON3-5 & \bf{563.70} & 6.46 & 
563.70 & 5.22 & 563.70 & 0.00
 & 0.00\\
CON3-6 & \bf{499.05} & 3.89 & 
500.64 & 4.64 & 499.05 & 0.00
 & 0.32\\CON3-7 & \bf{576.48} & 7.53 & 
580.84 & 7.29 & 576.48 & 0.00
 & 0.76\\CON3-8 & \bf{523.05} & 7.40 & 
523.05 & 4.88 & 523.05 & 0.00
 & 0.00\\
CON3-9 & 578.25 & 7.86 & 
584.36 & 8.21 & \bf{578.24} & 
0.00 & 1.06\\CON8-0 & 858.16 & 5.70 & 
863.87 & 4.29 & \bf{857.17} & 
0.12 & 0.78\\CON8-1 & \bf{740.85} & 2.74 & 
748.04 & 4.15 & 740.85 & 0.00
 & 0.97\\CON8-2 & 716.03 & 8.42 & 
728.59 & 5.54 & \bf{712.89} & 
0.44 & 2.20\\CON8-3 & 821.26 & 8.54 & 
827.25 & 5.25 & \bf{811.07} & 
1.26 & 1.99\\CON8-4 & \bf{772.25} & 5.22 & 
775.39 & 5.34 & 772.25 & 0.00
 & 0.41\\CON8-5 & 754.95 & 3.66 & 
756.80 & 4.62 & \bf{754.88} & 
0.01 & 0.26\\CON8-6 & 681.33 & 4.39 & 
688.75 & 4.58 & \bf{678.92} & 
0.35 & 1.45\\CON8-7 & 814.50 & 8.41 & 
837.74 & 4.51 & \bf{811.96} & 
0.31 & 3.17\\CON8-8 & \bf{767.53} & 3.67 & 
782.86 & 5.41 & 767.53 & 0.00
 & 2.00\\CON8-9 & 811.18 & 3.52 & 
817.45 & 3.23 & \bf{809.00} & 
0.27 & 1.04\\\bf{PROM.} & 
\bf{759.45} & \bf{5.59} & \bf{764.97} & \bf{5.12} & \bf{758.54} & \bf{0.12} & \bf{0.81}\\[1ex]\hline
\end{tabular}
\label{table:nonlin}
\end{table}
\begin{table}[h]
\caption{Resultados de la ejecución de la metaheurística GTS-M, utilizando instancias de Dethloff con la configuración -mni 5000 -lambda1 0.05 -lambda2 0.05 -tabu 5}
\centering
\small
\begin{tabular}{c c c c c c c c}
\hline\hline
Instancia & Costo mínimo & Tiempo(seg.) & Costo promedio & Tiempo promedio(seg.) & CME & \%G & \%GP \\ [0.5ex]
\hline
SCA3-0 & 640.55 & 2.07 & 
640.55 & 2.85 & \bf{635.62} & 
0.78 & 0.78\\SCA3-1 & \bf{697.84} & 2.82 & 
698.50 & 2.74 & 697.84 & 0.00
 & 0.10\\SCA3-2 & \bf{659.34} & 2.20 & 
659.34 & 3.15 & 659.34 & 0.00
 & 0.00\\
SCA3-3 & 680.60 & 1.63 & 
685.60 & 2.28 & \bf{680.04} & 
0.08 & 0.82\\SCA3-4 & \bf{690.50} & 4.27 & 
690.50 & 2.99 & 690.50 & 0.00
 & 0.00\\
SCA3-5 & \bf{659.90} & 2.06 & 
659.90 & 3.88 & 659.90 & 0.00
 & 0.00\\
SCA3-6 & \bf{651.09} & 4.20 & 
651.74 & 3.01 & 651.09 & 0.00
 & 0.10\\SCA3-7 & 666.15 & 2.30 & 
669.36 & 3.08 & \bf{659.17} & 
1.06 & 1.55\\SCA3-8 & \bf{719.47} & 7.99 & 
719.47 & 4.31 & 719.47 & 0.00
 & 0.00\\
SCA3-9 & \bf{681.00} & 3.56 & 
681.00 & 3.64 & 681.00 & 0.00
 & 0.00\\
SCA8-0 & 970.64 & 2.87 & 
981.08 & 6.10 & \bf{961.50} & 
0.95 & 2.04\\SCA8-1 & 1050.20 & 2.29 & 
1060.43 & 3.10 & \bf{1049.65} & 
0.05 & 1.03\\SCA8-2 & \bf{1039.64} & 4.25 & 
1044.67 & 4.50 & 1039.64 & 0.00
 & 0.48\\SCA8-3 & 1010.50 & 2.98 & 
1012.73 & 3.32 & \bf{983.34} & 
2.76 & 2.99\\SCA8-4 & 1067.55 & 4.62 & 
1073.34 & 2.87 & \bf{1065.49} & 
0.19 & 0.74\\SCA8-5 & \bf{1027.08} & 3.18 & 
1042.09 & 2.83 & 1027.08 & 0.00
 & 1.46\\SCA8-6 & \bf{971.82} & 2.34 & 
976.28 & 2.58 & 971.82 & 0.00
 & 0.46\\SCA8-7 & 1063.22 & 4.13 & 
1070.55 & 3.84 & \bf{1051.28} & 
1.14 & 1.83\\SCA8-8 & 1080.58 & 2.04 & 
1085.86 & 2.99 & \bf{1071.18} & 
0.88 & 1.37\\SCA8-9 & \bf{1060.50} & 3.79 & 
1065.19 & 4.46 & 1060.50 & 0.00
 & 0.44\\CON3-0 & 628.47 & 1.84 & 
635.11 & 2.37 & \bf{616.52} & 
1.94 & 3.01\\CON3-1 & \bf{554.47} & 4.78 & 
558.00 & 3.67 & 554.47 & 0.00
 & 0.64\\CON3-2 & 519.26 & 3.53 & 
522.76 & 3.13 & \bf{518.00} & 
0.24 & 0.92\\CON3-3 & \bf{591.19} & 2.99 & 
591.19 & 2.83 & 591.19 & 0.00
 & 0.00\\
CON3-4 & \bf{588.79} & 4.26 & 
590.11 & 3.20 & 588.79 & 0.00
 & 0.22\\CON3-5 & \bf{563.70} & 3.47 & 
563.70 & 2.87 & 563.70 & 0.00
 & 0.00\\
CON3-6 & 502.16 & 3.91 & 
502.66 & 3.15 & \bf{499.05} & 
0.62 & 0.72\\CON3-7 & 578.22 & 2.85 & 
580.22 & 3.16 & \bf{576.48} & 
0.30 & 0.65\\CON3-8 & \bf{523.05} & 4.25 & 
523.05 & 2.78 & 523.05 & 0.00
 & 0.00\\
CON3-9 & 578.25 & 2.33 & 
584.18 & 3.99 & \bf{578.24} & 
0.00 & 1.03\\CON8-0 & 857.40 & 3.72 & 
866.99 & 3.43 & \bf{857.17} & 
0.03 & 1.15\\CON8-1 & \bf{740.85} & 2.51 & 
766.76 & 2.42 & 740.85 & 0.00
 & 3.50\\CON8-2 & 722.22 & 4.12 & 
734.71 & 2.76 & \bf{712.89} & 
1.31 & 3.06\\CON8-3 & \bf{811.07} & 3.62 & 
832.79 & 4.44 & 811.07 & 0.00
 & 2.68\\CON8-4 & \bf{772.25} & 5.57 & 
787.29 & 3.59 & 772.25 & 0.00
 & 1.95\\CON8-5 & 755.67 & 4.15 & 
762.24 & 3.93 & \bf{754.88} & 
0.10 & 0.97\\CON8-6 & 690.63 & 6.23 & 
697.87 & 3.95 & \bf{678.92} & 
1.72 & 2.79\\CON8-7 & \bf{811.96} & 2.72 & 
814.08 & 2.65 & 811.96 & 0.00
 & 0.26\\CON8-8 & \bf{767.53} & 3.09 & 
770.84 & 3.15 & 767.53 & 0.00
 & 0.43\\CON8-9 & \bf{809.00} & 7.80 & 
812.76 & 3.64 & 809.00 & 0.00
 & 0.46\\\bf{PROM.} & 
\bf{761.36} & \bf{3.58} & \bf{766.64} & \bf{3.34} & \bf{758.54} & \bf{0.35} & \bf{1.02}\\[1ex]\hline
\end{tabular}
\label{table:nonlin}
\end{table}
\begin{table}[h]
\caption{Resultados de la ejecución de la metaheurística GTS-M, utilizando instancias de Dethloff con la configuración -mni 6000 -lambda1 0.05 -lambda2 0.05 -tabu 13}
\centering
\small
\begin{tabular}{c c c c c c c c}
\hline\hline
Instancia & Costo mínimo & Tiempo(seg.) & Costo promedio & Tiempo promedio(seg.) & CME & \%G & \%GP \\ [0.5ex]
\hline
SCA3-0 & 636.06 & 3.31 & 
639.43 & 3.84 & \bf{635.62} & 
0.07 & 0.60\\SCA3-1 & \bf{697.84} & 6.40 & 
697.84 & 3.60 & 697.84 & 0.00
 & 0.00\\
SCA3-2 & \bf{659.34} & 5.47 & 
659.34 & 4.71 & 659.34 & 0.00
 & 0.00\\
SCA3-3 & \bf{680.04} & 4.20 & 
680.46 & 4.42 & 680.04 & 0.00
 & 0.06\\SCA3-4 & \bf{690.50} & 3.81 & 
696.64 & 3.54 & 690.50 & 0.00
 & 0.89\\SCA3-5 & \bf{659.90} & 3.02 & 
663.16 & 5.50 & 659.90 & 0.00
 & 0.49\\SCA3-6 & \bf{651.09} & 2.96 & 
651.09 & 4.95 & 651.09 & 0.00
 & 0.00\\
SCA3-7 & 666.15 & 3.31 & 
667.09 & 3.28 & \bf{659.17} & 
1.06 & 1.20\\SCA3-8 & \bf{719.47} & 4.30 & 
719.47 & 3.62 & 719.47 & 0.00
 & 0.00\\
SCA3-9 & \bf{681.00} & 3.72 & 
681.00 & 3.77 & 681.00 & 0.00
 & 0.00\\
SCA8-0 & 979.79 & 5.54 & 
986.95 & 4.91 & \bf{961.50} & 
1.90 & 2.65\\SCA8-1 & 1050.20 & 5.98 & 
1063.57 & 4.35 & \bf{1049.65} & 
0.05 & 1.33\\SCA8-2 & \bf{1039.64} & 2.59 & 
1049.10 & 3.69 & 1039.64 & 0.00
 & 0.91\\SCA8-3 & \bf{983.34} & 4.59 & 
989.85 & 4.54 & 983.34 & 0.00
 & 0.66\\SCA8-4 & 1067.55 & 3.92 & 
1069.84 & 4.85 & \bf{1065.49} & 
0.19 & 0.41\\SCA8-5 & \bf{1027.08} & 2.85 & 
1043.65 & 4.23 & 1027.08 & 0.00
 & 1.61\\SCA8-6 & \bf{971.82} & 9.55 & 
981.05 & 5.61 & 971.82 & 0.00
 & 0.95\\SCA8-7 & 1064.34 & 2.54 & 
1073.82 & 3.89 & \bf{1051.28} & 
1.24 & 2.14\\SCA8-8 & \bf{1071.18} & 2.98 & 
1079.39 & 3.79 & 1071.18 & 0.00
 & 0.77\\SCA8-9 & 1063.68 & 7.01 & 
1075.99 & 3.65 & \bf{1060.50} & 
0.30 & 1.46\\CON3-0 & \bf{616.52} & 4.92 & 
622.50 & 5.06 & 616.52 & 0.00
 & 0.97\\CON3-1 & \bf{554.47} & 6.50 & 
555.39 & 5.91 & 554.47 & 0.00
 & 0.17\\CON3-2 & 519.61 & 4.68 & 
521.60 & 4.43 & \bf{518.00} & 
0.31 & 0.69\\CON3-3 & \bf{591.19} & 3.38 & 
591.19 & 6.42 & 591.19 & 0.00
 & 0.00\\
CON3-4 & \bf{588.79} & 5.43 & 
604.85 & 4.35 & 588.79 & 0.00
 & 2.73\\CON3-5 & \bf{563.70} & 3.12 & 
568.13 & 5.52 & 563.70 & 0.00
 & 0.79\\CON3-6 & \bf{499.05} & 3.07 & 
500.16 & 5.46 & 499.05 & 0.00
 & 0.22\\CON3-7 & \bf{576.48} & 8.46 & 
576.48 & 6.39 & 576.48 & 0.00
 & 0.00\\
CON3-8 & \bf{523.05} & 4.59 & 
523.05 & 3.85 & 523.05 & 0.00
 & 0.00\\
CON3-9 & 582.79 & 3.63 & 
586.95 & 4.22 & \bf{578.24} & 
0.79 & 1.51\\CON8-0 & 876.15 & 2.76 & 
886.57 & 5.36 & \bf{857.17} & 
2.21 & 3.43\\CON8-1 & 756.57 & 3.28 & 
759.24 & 2.91 & \bf{740.85} & 
2.12 & 2.48\\CON8-2 & 713.05 & 5.03 & 
720.43 & 5.42 & \bf{712.89} & 
0.02 & 1.06\\CON8-3 & 821.26 & 5.78 & 
828.29 & 5.00 & \bf{811.07} & 
1.26 & 2.12\\CON8-4 & \bf{772.25} & 3.89 & 
777.61 & 4.93 & 772.25 & 0.00
 & 0.69\\CON8-5 & 758.99 & 3.71 & 
759.94 & 4.54 & \bf{754.88} & 
0.54 & 0.67\\CON8-6 & 685.45 & 2.90 & 
692.76 & 3.66 & \bf{678.92} & 
0.96 & 2.04\\CON8-7 & 812.89 & 5.66 & 
813.14 & 4.44 & \bf{811.96} & 
0.11 & 0.15\\CON8-8 & 776.55 & 5.87 & 
783.10 & 4.22 & \bf{767.53} & 
1.18 & 2.03\\CON8-9 & 811.14 & 4.54 & 
838.29 & 3.91 & \bf{809.00} & 
0.26 & 3.62\\\bf{PROM.} & 
\bf{761.50} & \bf{4.48} & \bf{766.96} & \bf{4.52} & \bf{758.54} & \bf{0.36} & \bf{1.04}\\[1ex]\hline
\end{tabular}
\label{table:nonlin}
\end{table}
\begin{table}[h]
\caption{Resultados de la ejecución de la metaheurística GTS-M, utilizando instancias de Dethloff con la configuración -mni 6000 -lambda1 0.05 -lambda2 0.05 -tabu 29}
\centering
\small
\begin{tabular}{c c c c c c c c}
\hline\hline
Instancia & Costo mínimo & Tiempo(seg.) & Costo promedio & Tiempo promedio(seg.) & CME & \%G & \%GP \\ [0.5ex]
\hline
SCA3-0 & 636.06 & 5.11 & 
638.30 & 5.81 & \bf{635.62} & 
0.07 & 0.42\\SCA3-1 & \bf{697.84} & 6.58 & 
697.84 & 5.79 & 697.84 & 0.00
 & 0.00\\
SCA3-2 & \bf{659.34} & 5.91 & 
659.34 & 4.66 & 659.34 & 0.00
 & 0.00\\
SCA3-3 & \bf{680.04} & 8.95 & 
682.62 & 5.49 & 680.04 & 0.00
 & 0.38\\SCA3-4 & \bf{690.50} & 5.27 & 
690.50 & 4.82 & 690.50 & 0.00
 & 0.00\\
SCA3-5 & \bf{659.90} & 3.26 & 
663.16 & 3.90 & 659.90 & 0.00
 & 0.49\\SCA3-6 & \bf{651.09} & 4.56 & 
651.55 & 5.56 & 651.09 & 0.00
 & 0.07\\SCA3-7 & 666.15 & 4.68 & 
666.15 & 6.04 & \bf{659.17} & 
1.06 & 1.06\\SCA3-8 & \bf{719.47} & 4.60 & 
719.47 & 5.20 & 719.47 & 0.00
 & 0.00\\
SCA3-9 & \bf{681.00} & 4.23 & 
681.00 & 3.76 & 681.00 & 0.00
 & 0.00\\
SCA8-0 & \bf{961.50} & 3.87 & 
981.35 & 4.01 & 961.50 & 0.00
 & 2.06\\SCA8-1 & 1050.20 & 4.50 & 
1064.59 & 4.19 & \bf{1049.65} & 
0.05 & 1.42\\SCA8-2 & \bf{1039.64} & 4.97 & 
1047.69 & 3.73 & 1039.64 & 0.00
 & 0.77\\SCA8-3 & \bf{983.34} & 10.02 & 
998.42 & 6.74 & 983.34 & 0.00
 & 1.53\\SCA8-4 & \bf{1065.49} & 4.61 & 
1068.23 & 4.79 & 1065.49 & 0.00
 & 0.26\\SCA8-5 & \bf{1027.08} & 3.94 & 
1037.87 & 3.40 & 1027.08 & 0.00
 & 1.05\\SCA8-6 & 972.48 & 4.41 & 
972.48 & 5.03 & \bf{971.82} & 
0.07 & 0.07\\SCA8-7 & \bf{1051.28} & 7.62 & 
1076.45 & 4.68 & 1051.28 & 0.00
 & 2.39\\SCA8-8 & \bf{1071.18} & 3.23 & 
1079.57 & 3.38 & 1071.18 & 0.00
 & 0.78\\SCA8-9 & \bf{1060.50} & 9.08 & 
1060.50 & 7.22 & 1060.50 & 0.00
 & 0.00\\
CON3-0 & \bf{616.52} & 8.40 & 
619.22 & 7.62 & 616.52 & 0.00
 & 0.44\\CON3-1 & \bf{554.47} & 3.76 & 
556.83 & 4.99 & 554.47 & 0.00
 & 0.42\\CON3-2 & \bf{518.00} & 2.91 & 
521.98 & 5.66 & 518.00 & 0.00
 & 0.77\\CON3-3 & \bf{591.19} & 8.10 & 
591.19 & 7.54 & 591.19 & 0.00
 & 0.00\\
CON3-4 & \bf{588.79} & 2.64 & 
593.44 & 4.07 & 588.79 & 0.00
 & 0.79\\CON3-5 & \bf{563.70} & 3.85 & 
569.43 & 4.79 & 563.70 & 0.00
 & 1.02\\CON3-6 & 500.37 & 9.42 & 
502.17 & 5.67 & \bf{499.05} & 
0.26 & 0.62\\CON3-7 & \bf{576.48} & 3.37 & 
581.37 & 4.81 & 576.48 & 0.00
 & 0.85\\CON3-8 & \bf{523.05} & 2.59 & 
523.05 & 3.48 & 523.05 & 0.00
 & 0.00\\
CON3-9 & 578.25 & 8.27 & 
583.61 & 6.62 & \bf{578.24} & 
0.00 & 0.93\\CON8-0 & 857.40 & 9.75 & 
865.50 & 6.38 & \bf{857.17} & 
0.03 & 0.97\\CON8-1 & \bf{740.85} & 15.36 & 
751.68 & 6.51 & 740.85 & 0.00
 & 1.46\\CON8-2 & 713.05 & 12.66 & 
718.15 & 8.79 & \bf{712.89} & 
0.02 & 0.74\\CON8-3 & 811.23 & 2.67 & 
823.73 & 5.96 & \bf{811.07} & 
0.02 & 1.56\\CON8-4 & \bf{772.25} & 3.91 & 
779.71 & 3.94 & 772.25 & 0.00
 & 0.97\\CON8-5 & 755.67 & 5.21 & 
757.96 & 5.15 & \bf{754.88} & 
0.10 & 0.41\\CON8-6 & \bf{678.92} & 3.74 & 
688.26 & 8.03 & 678.92 & 0.00
 & 1.38\\CON8-7 & 812.89 & 4.15 & 
813.29 & 5.94 & \bf{811.96} & 
0.11 & 0.16\\CON8-8 & \bf{767.53} & 2.69 & 
769.05 & 4.94 & 767.53 & 0.00
 & 0.20\\CON8-9 & \bf{809.00} & 5.19 & 
811.70 & 6.15 & 809.00 & 0.00
 & 0.33\\\bf{PROM.} & 
\bf{758.84} & \bf{5.70} & \bf{763.96} & \bf{5.38} & \bf{758.54} & \bf{0.05} & \bf{0.67}\\[1ex]\hline
\end{tabular}
\label{table:nonlin}
\end{table}
\begin{table}[h]
\caption{Resultados de la ejecución de la metaheurística GTS-M, utilizando instancias de Dethloff con la configuración -mni 6000 -lambda1 0.05 -lambda2 0.05 -tabu 37}
\centering
\small
\begin{tabular}{c c c c c c c c}
\hline\hline
Instancia & Costo mínimo & Tiempo(seg.) & Costo promedio & Tiempo promedio(seg.) & CME & \%G & \%GP \\ [0.5ex]
\hline
SCA3-0 & \bf{635.62} & 11.29 & 
638.20 & 7.42 & 635.62 & 0.00
 & 0.41\\SCA3-1 & \bf{697.84} & 5.71 & 
698.50 & 6.87 & 697.84 & 0.00
 & 0.10\\SCA3-2 & \bf{659.34} & 7.00 & 
659.34 & 7.16 & 659.34 & 0.00
 & 0.00\\
SCA3-3 & \bf{680.04} & 7.92 & 
685.65 & 5.82 & 680.04 & 0.00
 & 0.83\\SCA3-4 & \bf{690.50} & 4.05 & 
690.50 & 7.05 & 690.50 & 0.00
 & 0.00\\
SCA3-5 & \bf{659.90} & 3.76 & 
659.90 & 6.32 & 659.90 & 0.00
 & 0.00\\
SCA3-6 & \bf{651.09} & 4.79 & 
651.43 & 4.44 & 651.09 & 0.00
 & 0.05\\SCA3-7 & 666.15 & 4.19 & 
666.15 & 6.23 & \bf{659.17} & 
1.06 & 1.06\\SCA3-8 & \bf{719.47} & 3.43 & 
722.30 & 4.28 & 719.47 & 0.00
 & 0.39\\SCA3-9 & \bf{681.00} & 3.96 & 
681.00 & 4.74 & 681.00 & 0.00
 & 0.00\\
SCA8-0 & \bf{961.50} & 4.89 & 
972.05 & 4.43 & 961.50 & 0.00
 & 1.10\\SCA8-1 & \bf{1049.65} & 5.21 & 
1054.77 & 4.59 & 1049.65 & 0.00
 & 0.49\\SCA8-2 & \bf{1039.64} & 8.34 & 
1044.92 & 6.25 & 1039.64 & 0.00
 & 0.51\\SCA8-3 & \bf{983.34} & 8.17 & 
1000.26 & 6.70 & 983.34 & 0.00
 & 1.72\\SCA8-4 & 1067.28 & 4.15 & 
1071.41 & 3.65 & \bf{1065.49} & 
0.17 & 0.56\\SCA8-5 & \bf{1027.08} & 11.83 & 
1043.49 & 6.05 & 1027.08 & 0.00
 & 1.60\\SCA8-6 & \bf{971.82} & 8.14 & 
981.79 & 5.12 & 971.82 & 0.00
 & 1.03\\SCA8-7 & \bf{1051.28} & 4.69 & 
1061.01 & 6.35 & 1051.28 & 0.00
 & 0.93\\SCA8-8 & \bf{1071.18} & 4.46 & 
1073.92 & 5.84 & 1071.18 & 0.00
 & 0.26\\SCA8-9 & \bf{1060.50} & 4.74 & 
1063.35 & 5.61 & 1060.50 & 0.00
 & 0.27\\CON3-0 & \bf{616.52} & 3.80 & 
625.42 & 3.37 & 616.52 & 0.00
 & 1.44\\CON3-1 & 556.04 & 5.84 & 
556.66 & 4.81 & \bf{554.47} & 
0.28 & 0.40\\CON3-2 & 521.38 & 5.29 & 
522.21 & 5.10 & \bf{518.00} & 
0.65 & 0.81\\CON3-3 & \bf{591.19} & 7.94 & 
591.19 & 6.33 & 591.19 & 0.00
 & 0.00\\
CON3-4 & \bf{588.79} & 6.62 & 
590.66 & 5.47 & 588.79 & 0.00
 & 0.32\\CON3-5 & \bf{563.70} & 4.41 & 
567.42 & 5.93 & 563.70 & 0.00
 & 0.66\\CON3-6 & \bf{499.05} & 6.42 & 
501.50 & 8.28 & 499.05 & 0.00
 & 0.49\\CON3-7 & \bf{576.48} & 13.39 & 
580.34 & 8.21 & 576.48 & 0.00
 & 0.67\\CON3-8 & \bf{523.05} & 5.90 & 
523.05 & 5.67 & 523.05 & 0.00
 & 0.00\\
CON3-9 & 578.25 & 14.09 & 
581.85 & 8.46 & \bf{578.24} & 
0.00 & 0.62\\CON8-0 & 858.03 & 6.62 & 
872.27 & 5.01 & \bf{857.17} & 
0.10 & 1.76\\CON8-1 & \bf{740.85} & 2.84 & 
744.76 & 3.25 & 740.85 & 0.00
 & 0.53\\CON8-2 & 713.84 & 3.88 & 
719.11 & 4.39 & \bf{712.89} & 
0.13 & 0.87\\CON8-3 & \bf{811.07} & 4.84 & 
824.48 & 5.20 & 811.07 & 0.00
 & 1.65\\CON8-4 & \bf{772.25} & 4.06 & 
788.24 & 4.29 & 772.25 & 0.00
 & 2.07\\CON8-5 & 758.12 & 8.34 & 
759.48 & 4.88 & \bf{754.88} & 
0.43 & 0.61\\CON8-6 & 683.68 & 6.74 & 
685.90 & 6.45 & \bf{678.92} & 
0.70 & 1.03\\CON8-7 & 813.00 & 4.82 & 
814.24 & 4.65 & \bf{811.96} & 
0.13 & 0.28\\CON8-8 & \bf{767.53} & 7.32 & 
776.13 & 5.52 & 767.53 & 0.00
 & 1.12\\CON8-9 & 814.50 & 3.07 & 
831.16 & 3.61 & \bf{809.00} & 
0.68 & 2.74\\\bf{PROM.} & 
\bf{759.29} & \bf{6.17} & \bf{764.40} & \bf{5.60} & \bf{758.54} & \bf{0.11} & \bf{0.73}\\[1ex]\hline
\end{tabular}
\label{table:nonlin}
\end{table}
\begin{table}[h]
\caption{Resultados de la ejecución de la metaheurística GTS-M, utilizando instancias de Dethloff con la configuración -mni 6000 -lambda1 0.05 -lambda2 0.05 -tabu 5}
\centering
\small
\begin{tabular}{c c c c c c c c}
\hline\hline
Instancia & Costo mínimo & Tiempo(seg.) & Costo promedio & Tiempo promedio(seg.) & CME & \%G & \%GP \\ [0.5ex]
\hline
SCA3-0 & 636.06 & 6.14 & 
638.30 & 4.72 & \bf{635.62} & 
0.07 & 0.42\\SCA3-1 & \bf{697.84} & 3.31 & 
697.84 & 3.30 & 697.84 & 0.00
 & 0.00\\
SCA3-2 & \bf{659.34} & 2.13 & 
659.34 & 3.90 & 659.34 & 0.00
 & 0.00\\
SCA3-3 & \bf{680.04} & 3.94 & 
680.46 & 3.44 & 680.04 & 0.00
 & 0.06\\SCA3-4 & \bf{690.50} & 5.48 & 
690.50 & 4.64 & 690.50 & 0.00
 & 0.00\\
SCA3-5 & \bf{659.90} & 3.40 & 
663.16 & 3.29 & 659.90 & 0.00
 & 0.49\\SCA3-6 & 652.94 & 2.83 & 
655.70 & 2.91 & \bf{651.09} & 
0.28 & 0.71\\SCA3-7 & 666.15 & 6.94 & 
667.37 & 3.58 & \bf{659.17} & 
1.06 & 1.24\\SCA3-8 & \bf{719.47} & 6.07 & 
719.47 & 4.33 & 719.47 & 0.00
 & 0.00\\
SCA3-9 & \bf{681.00} & 4.81 & 
681.00 & 4.42 & 681.00 & 0.00
 & 0.00\\
SCA8-0 & 979.79 & 2.08 & 
986.82 & 3.16 & \bf{961.50} & 
1.90 & 2.63\\SCA8-1 & 1067.45 & 4.04 & 
1071.08 & 3.89 & \bf{1049.65} & 
1.70 & 2.04\\SCA8-2 & 1050.37 & 3.45 & 
1058.39 & 3.42 & \bf{1039.64} & 
1.03 & 1.80\\SCA8-3 & \bf{983.34} & 3.65 & 
995.85 & 4.50 & 983.34 & 0.00
 & 1.27\\SCA8-4 & 1067.28 & 5.15 & 
1069.53 & 4.86 & \bf{1065.49} & 
0.17 & 0.38\\SCA8-5 & 1047.55 & 6.97 & 
1056.38 & 3.98 & \bf{1027.08} & 
1.99 & 2.85\\SCA8-6 & 972.48 & 5.74 & 
977.09 & 4.93 & \bf{971.82} & 
0.07 & 0.54\\SCA8-7 & \bf{1051.28} & 4.86 & 
1073.03 & 3.82 & 1051.28 & 0.00
 & 2.07\\SCA8-8 & 1080.58 & 1.64 & 
1092.19 & 3.06 & \bf{1071.18} & 
0.88 & 1.96\\SCA8-9 & \bf{1060.50} & 8.60 & 
1068.74 & 4.58 & 1060.50 & 0.00
 & 0.78\\CON3-0 & 628.47 & 8.73 & 
634.36 & 5.28 & \bf{616.52} & 
1.94 & 2.89\\CON3-1 & \bf{554.47} & 5.21 & 
555.59 & 5.11 & 554.47 & 0.00
 & 0.20\\CON3-2 & \bf{518.00} & 3.40 & 
521.27 & 3.56 & 518.00 & 0.00
 & 0.63\\CON3-3 & \bf{591.19} & 3.77 & 
594.58 & 2.74 & 591.19 & 0.00
 & 0.57\\CON3-4 & 591.43 & 3.85 & 
595.07 & 3.19 & \bf{588.79} & 
0.45 & 1.07\\CON3-5 & \bf{563.70} & 3.11 & 
567.21 & 3.91 & 563.70 & 0.00
 & 0.62\\CON3-6 & \bf{499.05} & 4.40 & 
501.04 & 4.01 & 499.05 & 0.00
 & 0.40\\CON3-7 & \bf{576.48} & 3.70 & 
582.18 & 4.63 & 576.48 & 0.00
 & 0.99\\CON3-8 & \bf{523.05} & 2.60 & 
523.37 & 2.74 & 523.05 & 0.00
 & 0.06\\CON3-9 & 578.25 & 2.47 & 
581.65 & 6.38 & \bf{578.24} & 
0.00 & 0.59\\CON8-0 & 857.40 & 3.82 & 
881.57 & 4.95 & \bf{857.17} & 
0.03 & 2.85\\CON8-1 & \bf{740.85} & 3.02 & 
750.74 & 3.56 & 740.85 & 0.00
 & 1.33\\CON8-2 & 717.67 & 5.25 & 
733.24 & 4.58 & \bf{712.89} & 
0.67 & 2.85\\CON8-3 & \bf{811.07} & 2.27 & 
818.50 & 4.07 & 811.07 & 0.00
 & 0.92\\CON8-4 & \bf{772.25} & 5.86 & 
786.11 & 5.33 & 772.25 & 0.00
 & 1.79\\CON8-5 & \bf{754.88} & 4.91 & 
757.67 & 3.28 & 754.88 & 0.00
 & 0.37\\CON8-6 & 683.83 & 4.00 & 
691.56 & 3.41 & \bf{678.92} & 
0.72 & 1.86\\CON8-7 & 812.89 & 6.93 & 
825.80 & 4.42 & \bf{811.96} & 
0.11 & 1.70\\CON8-8 & 773.60 & 2.86 & 
779.75 & 3.91 & \bf{767.53} & 
0.79 & 1.59\\CON8-9 & 813.46 & 5.16 & 
828.73 & 6.18 & \bf{809.00} & 
0.55 & 2.44\\\bf{PROM.} & 
\bf{761.65} & \bf{4.41} & \bf{767.81} & \bf{4.10} & \bf{758.54} & \bf{0.36} & \bf{1.13}\\[1ex]\hline
\end{tabular}
\label{table:nonlin}
\end{table}

\clearpage
\subsection{SalhiNagy}\label{tablas-entonacion-GTS-M-salhinagy}

\begin{table}[h]
\caption{Resultados de la ejecución de la metaheurística GTS-M, utilizando instancias de SalhiNagy con la configuración -mni 3000 -lambda1 0.05 -lambda2 0.05 -tabu 13}
\centering
\small
\begin{tabular}{c c c c c c c c}
\hline\hline
Instancia & Costo mínimo & Tiempo(seg.) & Costo promedio & Tiempo promedio(seg.) & CME & \%G & \%GP \\ [0.5ex]
\hline
CMT1X & \bf{470.48} & 2.49 & 
474.49 & 2.33 & 470.48 & 0.00
 & 0.85\\CMT1Y & \bf{470.48} & 1.14 & 
474.14 & 2.53 & 470.48 & 0.00
 & 0.78\\CMT2X & 682.97 & 9.30 & 
689.71 & 6.97 & \bf{682.39} & 
0.08 & 1.07\\CMT2Y & 685.83 & 3.53 & 
692.28 & 4.19 & \bf{682.39} & 
0.50 & 1.45\\CMT3X & 727.10 & 5.56 & 
731.30 & 4.82 & \bf{719.06} & 
1.12 & 1.70\\CMT3Y & 725.24 & 3.82 & 
734.83 & 6.14 & \bf{719.06} & 
0.86 & 2.19\\CMT4X & 857.10 & 44.51 & 
880.01 & 29.54 & \bf{854.21} & 
0.34 & 3.02\\CMT4Y & 864.65 & 30.39 & 
878.79 & 21.78 & \bf{852.46} & 
1.43 & 3.09\\CMT5X & 1042.41 & 59.10 & 
1076.56 & 51.16 & \bf{1030.56} & 
1.15 & 4.46\\CMT5Y & 1051.63 & 112.91 & 
1064.77 & 62.12 & \bf{1031.69} & 
1.93 & 3.21\\CMT11X & 874.16 & 30.02 & 
927.10 & 17.56 & \bf{831.09} & 
5.18 & 11.55\\CMT11Y & 877.73 & 13.89 & 
890.64 & 20.76 & \bf{829.85} & 
5.77 & 7.33\\CMT12X & 670.63 & 11.53 & 
682.35 & 7.75 & \bf{658.83} & 
1.79 & 3.57\\CMT12Y & 673.80 & 10.35 & 
677.40 & 8.68 & \bf{660.47} & 
2.02 & 2.56\\\bf{PROM.} & 
\bf{762.44} & \bf{24.18} & \bf{776.74} & \bf{17.60} & \bf{749.50} & \bf{1.58} & \bf{3.35}\\[1ex]\hline
\end{tabular}
\label{table:nonlin}
\end{table}
\begin{table}[h]
\caption{Resultados de la ejecución de la metaheurística GTS-M, utilizando instancias de SalhiNagy con la configuración -mni 3000 -lambda1 0.05 -lambda2 0.05 -tabu 29}
\centering
\small
\begin{tabular}{c c c c c c c c}
\hline\hline
Instancia & Costo mínimo & Tiempo(seg.) & Costo promedio & Tiempo promedio(seg.) & CME & \%G & \%GP \\ [0.5ex]
\hline
CMT1X & \bf{470.48} & 1.87 & 
470.92 & 2.22 & 470.48 & 0.00
 & 0.09\\CMT1Y & 472.37 & 2.26 & 
476.12 & 1.91 & \bf{470.48} & 
0.40 & 1.20\\CMT2X & 683.24 & 3.37 & 
688.90 & 3.79 & \bf{682.39} & 
0.12 & 0.95\\CMT2Y & 684.24 & 4.74 & 
686.00 & 6.74 & \bf{682.39} & 
0.27 & 0.53\\CMT3X & 725.25 & 9.50 & 
730.82 & 7.51 & \bf{719.06} & 
0.86 & 1.64\\CMT3Y & 723.67 & 3.33 & 
726.16 & 8.43 & \bf{719.06} & 
0.64 & 0.99\\CMT4X & 861.29 & 19.30 & 
874.26 & 24.07 & \bf{854.21} & 
0.83 & 2.35\\CMT4Y & 857.86 & 44.85 & 
870.08 & 40.63 & \bf{852.46} & 
0.63 & 2.07\\CMT5X & 1055.29 & 71.62 & 
1066.52 & 58.18 & \bf{1030.56} & 
2.40 & 3.49\\CMT5Y & 1054.81 & 81.07 & 
1068.03 & 51.84 & \bf{1031.69} & 
2.24 & 3.52\\CMT11X & 883.60 & 27.35 & 
923.70 & 25.00 & \bf{831.09} & 
6.32 & 11.14\\CMT11Y & 879.29 & 53.60 & 
903.25 & 26.61 & \bf{829.85} & 
5.96 & 8.85\\CMT12X & 673.13 & 16.37 & 
677.44 & 9.21 & \bf{658.83} & 
2.17 & 2.82\\CMT12Y & 670.27 & 18.54 & 
676.88 & 10.79 & \bf{660.47} & 
1.48 & 2.48\\\bf{PROM.} & 
\bf{763.91} & \bf{25.56} & \bf{774.22} & \bf{19.78} & \bf{749.50} & \bf{1.74} & \bf{3.01}\\[1ex]\hline
\end{tabular}
\label{table:nonlin}
\end{table}
\begin{table}[h]
\caption{Resultados de la ejecución de la metaheurística GTS-M, utilizando instancias de SalhiNagy con la configuración -mni 3000 -lambda1 0.05 -lambda2 0.05 -tabu 37}
\centering
\small
\begin{tabular}{c c c c c c c c}
\hline\hline
Instancia & Costo mínimo & Tiempo(seg.) & Costo promedio & Tiempo promedio(seg.) & CME & \%G & \%GP \\ [0.5ex]
\hline
CMT1X & \bf{470.48} & 1.92 & 
471.27 & 3.78 & 470.48 & 0.00
 & 0.17\\CMT1Y & \bf{470.48} & 2.93 & 
470.53 & 3.10 & 470.48 & 0.00
 & 0.01\\CMT2X & 687.47 & 2.84 & 
691.24 & 3.85 & \bf{682.39} & 
0.74 & 1.30\\CMT2Y & 683.64 & 5.18 & 
692.29 & 4.39 & \bf{682.39} & 
0.18 & 1.45\\CMT3X & 726.56 & 6.52 & 
730.58 & 8.29 & \bf{719.06} & 
1.04 & 1.60\\CMT3Y & \bf{\underline{718.40}} & 9.59 & 
726.25 & 9.36 & 719.06 & 
\bf{-0.09} & 1.00\\CMT4X & 860.34 & 62.94 & 
871.92 & 36.80 & \bf{854.21} & 
0.72 & 2.07\\CMT4Y & 871.94 & 15.54 & 
876.66 & 24.44 & \bf{852.46} & 
2.29 & 2.84\\CMT5X & 1049.05 & 73.98 & 
1063.42 & 61.17 & \bf{1030.56} & 
1.79 & 3.19\\CMT5Y & 1063.25 & 58.15 & 
1075.56 & 49.43 & \bf{1031.69} & 
3.06 & 4.25\\CMT11X & 862.52 & 17.70 & 
908.98 & 35.20 & \bf{831.09} & 
3.78 & 9.37\\CMT11Y & 877.99 & 41.80 & 
922.88 & 25.49 & \bf{829.85} & 
5.80 & 11.21\\CMT12X & 673.32 & 7.85 & 
675.05 & 9.27 & \bf{658.83} & 
2.20 & 2.46\\CMT12Y & 674.00 & 9.58 & 
680.58 & 8.52 & \bf{660.47} & 
2.05 & 3.05\\\bf{PROM.} & 
\bf{763.53} & \bf{22.61} & \bf{775.51} & \bf{20.22} & \bf{749.50} & \bf{1.68} & \bf{3.14}\\[1ex]\hline
\end{tabular}
\label{table:nonlin}
\end{table}
\begin{table}[h]
\caption{Resultados de la ejecución de la metaheurística GTS-M, utilizando instancias de SalhiNagy con la configuración -mni 3000 -lambda1 0.05 -lambda2 0.05 -tabu 5}
\centering
\small
\begin{tabular}{c c c c c c c c}
\hline\hline
Instancia & Costo mínimo & Tiempo(seg.) & Costo promedio & Tiempo promedio(seg.) & CME & \%G & \%GP \\ [0.5ex]
\hline
CMT1X & \bf{470.48} & 2.39 & 
476.37 & 2.30 & 470.48 & 0.00
 & 1.25\\CMT1Y & \bf{470.48} & 1.45 & 
471.43 & 1.86 & 470.48 & 0.00
 & 0.20\\CMT2X & 690.50 & 6.81 & 
695.57 & 5.34 & \bf{682.39} & 
1.19 & 1.93\\CMT2Y & 684.33 & 8.14 & 
690.32 & 8.03 & \bf{682.39} & 
0.28 & 1.16\\CMT3X & 731.11 & 7.23 & 
734.15 & 5.92 & \bf{719.06} & 
1.68 & 2.10\\CMT3Y & 728.31 & 13.55 & 
731.19 & 10.36 & \bf{719.06} & 
1.29 & 1.69\\CMT4X & 860.05 & 37.86 & 
873.55 & 31.84 & \bf{854.21} & 
0.68 & 2.26\\CMT4Y & 875.32 & 26.85 & 
881.91 & 19.92 & \bf{852.46} & 
2.68 & 3.46\\CMT5X & 1056.38 & 55.30 & 
1072.32 & 73.74 & \bf{1030.56} & 
2.51 & 4.05\\CMT5Y & 1046.01 & 76.00 & 
1058.52 & 58.35 & \bf{1031.69} & 
1.39 & 2.60\\CMT11X & 886.55 & 15.48 & 
904.07 & 20.26 & \bf{831.09} & 
6.67 & 8.78\\CMT11Y & 876.59 & 60.21 & 
922.32 & 32.83 & \bf{829.85} & 
5.63 & 11.14\\CMT12X & 670.73 & 11.38 & 
688.26 & 8.30 & \bf{658.83} & 
1.81 & 4.47\\CMT12Y & 673.64 & 15.00 & 
675.70 & 10.71 & \bf{660.47} & 
1.99 & 2.31\\\bf{PROM.} & 
\bf{765.75} & \bf{24.12} & \bf{776.83} & \bf{20.70} & \bf{749.50} & \bf{1.99} & \bf{3.39}\\[1ex]\hline
\end{tabular}
\label{table:nonlin}
\end{table}
\begin{table}[h]
\caption{Resultados de la ejecución de la metaheurística GTS-M, utilizando instancias de SalhiNagy con la configuración -mni 5000 -lambda1 0.05 -lambda2 0.05 -tabu 13}
\centering
\small
\begin{tabular}{c c c c c c c c}
\hline\hline
Instancia & Costo mínimo & Tiempo(seg.) & Costo promedio & Tiempo promedio(seg.) & CME & \%G & \%GP \\ [0.5ex]
\hline
CMT1X & 470.67 & 5.28 & 
474.65 & 4.79 & \bf{470.48} & 
0.04 & 0.89\\CMT1Y & \bf{470.48} & 5.17 & 
471.90 & 4.96 & 470.48 & 0.00
 & 0.30\\CMT2X & 687.04 & 10.30 & 
688.10 & 7.61 & \bf{682.39} & 
0.68 & 0.84\\CMT2Y & 683.95 & 8.23 & 
687.21 & 6.09 & \bf{682.39} & 
0.23 & 0.71\\CMT3X & 720.85 & 19.41 & 
727.44 & 13.26 & \bf{719.06} & 
0.25 & 1.17\\CMT3Y & 727.67 & 23.26 & 
730.82 & 16.68 & \bf{719.06} & 
1.20 & 1.64\\CMT4X & 856.96 & 53.35 & 
875.27 & 53.15 & \bf{854.21} & 
0.32 & 2.47\\CMT4Y & 861.87 & 42.58 & 
873.45 & 43.27 & \bf{852.46} & 
1.10 & 2.46\\CMT5X & 1044.32 & 67.27 & 
1061.08 & 86.57 & \bf{1030.56} & 
1.34 & 2.96\\CMT5Y & 1070.09 & 67.93 & 
1075.29 & 88.73 & \bf{1031.69} & 
3.72 & 4.23\\CMT11X & 874.78 & 24.59 & 
929.53 & 35.62 & \bf{831.09} & 
5.26 & 11.85\\CMT11Y & 877.95 & 35.18 & 
929.10 & 26.55 & \bf{829.85} & 
5.80 & 11.96\\CMT12X & 669.67 & 15.38 & 
675.60 & 13.96 & \bf{658.83} & 
1.65 & 2.55\\CMT12Y & 674.38 & 16.92 & 
682.40 & 11.98 & \bf{660.47} & 
2.11 & 3.32\\\bf{PROM.} & 
\bf{763.62} & \bf{28.20} & \bf{777.27} & \bf{29.52} & \bf{749.50} & \bf{1.69} & \bf{3.38}\\[1ex]\hline
\end{tabular}
\label{table:nonlin}
\end{table}
\begin{table}[h]
\caption{Resultados de la ejecución de la metaheurística GTS-M, utilizando instancias de SalhiNagy con la configuración -mni 5000 -lambda1 0.05 -lambda2 0.05 -tabu 29}
\centering
\small
\begin{tabular}{c c c c c c c c}
\hline\hline
Instancia & Costo mínimo & Tiempo(seg.) & Costo promedio & Tiempo promedio(seg.) & CME & \%G & \%GP \\ [0.5ex]
\hline
CMT1X & 470.67 & 4.63 & 
472.06 & 4.64 & \bf{470.48} & 
0.04 & 0.34\\CMT1Y & \bf{470.48} & 4.63 & 
471.62 & 5.22 & 470.48 & 0.00
 & 0.24\\CMT2X & 683.95 & 12.96 & 
688.00 & 8.90 & \bf{682.39} & 
0.23 & 0.82\\CMT2Y & 685.40 & 3.82 & 
689.71 & 6.48 & \bf{682.39} & 
0.44 & 1.07\\CMT3X & 720.85 & 18.09 & 
728.09 & 13.77 & \bf{719.06} & 
0.25 & 1.26\\CMT3Y & \bf{\underline{718.40}} & 17.16 & 
724.92 & 15.59 & 719.06 & 
\bf{-0.09} & 0.81\\CMT4X & 869.88 & 39.87 & 
875.35 & 48.16 & \bf{854.21} & 
1.83 & 2.47\\CMT4Y & 857.21 & 40.89 & 
866.88 & 39.30 & \bf{852.46} & 
0.56 & 1.69\\CMT5X & 1053.03 & 90.68 & 
1066.16 & 87.20 & \bf{1030.56} & 
2.18 & 3.45\\CMT5Y & 1052.09 & 55.50 & 
1069.69 & 73.75 & \bf{1031.69} & 
1.98 & 3.68\\CMT11X & 887.38 & 64.62 & 
927.71 & 33.04 & \bf{831.09} & 
6.77 & 11.63\\CMT11Y & 864.03 & 39.67 & 
881.12 & 39.27 & \bf{829.85} & 
4.12 & 6.18\\CMT12X & 664.46 & 26.26 & 
668.91 & 22.70 & \bf{658.83} & 
0.85 & 1.53\\CMT12Y & 667.08 & 20.05 & 
674.38 & 14.53 & \bf{660.47} & 
1.00 & 2.11\\\bf{PROM.} & 
\bf{761.78} & \bf{31.35} & \bf{771.76} & \bf{29.47} & \bf{749.50} & \bf{1.44} & \bf{2.66}\\[1ex]\hline
\end{tabular}
\label{table:nonlin}
\end{table}
\begin{table}[h]
\caption{Resultados de la ejecución de la metaheurística GTS-M, utilizando instancias de SalhiNagy con la configuración -mni 5000 -lambda1 0.05 -lambda2 0.05 -tabu 37}
\centering
\small
\begin{tabular}{c c c c c c c c}
\hline\hline
Instancia & Costo mínimo & Tiempo(seg.) & Costo promedio & Tiempo promedio(seg.) & CME & \%G & \%GP \\ [0.5ex]
\hline
CMT1X & \bf{470.48} & 3.23 & 
471.05 & 4.15 & 470.48 & 0.00
 & 0.12\\CMT1Y & 472.37 & 7.38 & 
472.75 & 4.57 & \bf{470.48} & 
0.40 & 0.48\\CMT2X & 684.40 & 8.12 & 
686.93 & 7.93 & \bf{682.39} & 
0.29 & 0.67\\CMT2Y & 684.61 & 9.51 & 
685.83 & 6.67 & \bf{682.39} & 
0.33 & 0.50\\CMT3X & 720.08 & 17.55 & 
724.32 & 13.92 & \bf{719.06} & 
0.14 & 0.73\\CMT3Y & 719.24 & 12.88 & 
722.30 & 21.97 & \bf{719.06} & 
0.03 & 0.45\\CMT4X & 859.91 & 42.04 & 
868.65 & 45.02 & \bf{854.21} & 
0.67 & 1.69\\CMT4Y & 855.50 & 30.56 & 
872.61 & 40.44 & \bf{852.46} & 
0.36 & 2.36\\CMT5X & 1034.47 & 149.62 & 
1074.57 & 100.35 & \bf{1030.56} & 
0.38 & 4.27\\CMT5Y & 1058.07 & 188.37 & 
1072.12 & 114.88 & \bf{1031.69} & 
2.56 & 3.92\\CMT11X & 886.09 & 11.84 & 
927.52 & 26.26 & \bf{831.09} & 
6.62 & 11.60\\CMT11Y & 877.73 & 62.72 & 
934.12 & 42.12 & \bf{829.85} & 
5.77 & 12.56\\CMT12X & 670.50 & 14.44 & 
673.27 & 12.21 & \bf{658.83} & 
1.77 & 2.19\\CMT12Y & 673.77 & 13.84 & 
682.41 & 14.23 & \bf{660.47} & 
2.01 & 3.32\\\bf{PROM.} & 
\bf{761.94} & \bf{40.86} & \bf{776.32} & \bf{32.48} & \bf{749.50} & \bf{1.52} & \bf{3.21}\\[1ex]\hline
\end{tabular}
\label{table:nonlin}
\end{table}
\begin{table}[h]
\caption{Resultados de la ejecución de la metaheurística GTS-M, utilizando instancias de SalhiNagy con la configuración -mni 5000 -lambda1 0.05 -lambda2 0.05 -tabu 5}
\centering
\small
\begin{tabular}{c c c c c c c c}
\hline\hline
Instancia & Costo mínimo & Tiempo(seg.) & Costo promedio & Tiempo promedio(seg.) & CME & \%G & \%GP \\ [0.5ex]
\hline
CMT1X & \bf{470.48} & 3.24 & 
474.14 & 3.19 & 470.48 & 0.00
 & 0.78\\CMT1Y & \bf{470.48} & 5.62 & 
472.26 & 5.56 & 470.48 & 0.00
 & 0.38\\CMT2X & 683.52 & 12.75 & 
688.38 & 8.01 & \bf{682.39} & 
0.17 & 0.88\\CMT2Y & \bf{682.39} & 7.22 & 
693.66 & 9.75 & 682.39 & 0.00
 & 1.65\\CMT3X & \bf{\underline{718.40}} & 14.93 & 
724.80 & 13.71 & 719.06 & 
\bf{-0.09} & 0.80\\CMT3Y & 726.28 & 11.27 & 
729.32 & 12.80 & \bf{719.06} & 
1.00 & 1.43\\CMT4X & 857.77 & 48.73 & 
880.70 & 43.04 & \bf{854.21} & 
0.42 & 3.10\\CMT4Y & 871.01 & 28.47 & 
883.92 & 31.14 & \bf{852.46} & 
2.18 & 3.69\\CMT5X & 1053.60 & 78.61 & 
1073.71 & 75.47 & \bf{1030.56} & 
2.24 & 4.19\\CMT5Y & 1065.33 & 89.55 & 
1084.73 & 86.14 & \bf{1031.69} & 
3.26 & 5.14\\CMT11X & 885.58 & 55.43 & 
914.03 & 37.53 & \bf{831.09} & 
6.56 & 9.98\\CMT11Y & 864.51 & 16.59 & 
897.43 & 25.27 & \bf{829.85} & 
4.18 & 8.14\\CMT12X & 672.47 & 22.52 & 
673.72 & 12.43 & \bf{658.83} & 
2.07 & 2.26\\CMT12Y & 673.59 & 14.16 & 
681.59 & 11.20 & \bf{660.47} & 
1.99 & 3.20\\\bf{PROM.} & 
\bf{763.96} & \bf{29.22} & \bf{776.60} & \bf{26.80} & \bf{749.50} & \bf{1.71} & \bf{3.26}\\[1ex]\hline
\end{tabular}
\label{table:nonlin}
\end{table}
\begin{table}[h]
\caption{Resultados de la ejecución de la metaheurística GTS-M, utilizando instancias de SalhiNagy con la configuración -mni 6000 -lambda1 0.05 -lambda2 0.05 -tabu 13}
\centering
\small
\begin{tabular}{c c c c c c c c}
\hline\hline
Instancia & Costo mínimo & Tiempo(seg.) & Costo promedio & Tiempo promedio(seg.) & CME & \%G & \%GP \\ [0.5ex]
\hline
CMT1X & \bf{470.48} & 3.54 & 
470.95 & 5.05 & 470.48 & 0.00
 & 0.10\\CMT1Y & \bf{470.48} & 6.14 & 
470.95 & 3.78 & 470.48 & 0.00
 & 0.10\\CMT2X & 683.95 & 11.29 & 
688.55 & 9.14 & \bf{682.39} & 
0.23 & 0.90\\CMT2Y & 684.29 & 7.24 & 
685.10 & 7.54 & \bf{682.39} & 
0.28 & 0.40\\CMT3X & 723.52 & 10.92 & 
727.06 & 21.06 & \bf{719.06} & 
0.62 & 1.11\\CMT3Y & 720.08 & 40.20 & 
723.09 & 19.58 & \bf{719.06} & 
0.14 & 0.56\\CMT4X & 861.31 & 113.35 & 
865.77 & 61.74 & \bf{854.21} & 
0.83 & 1.35\\CMT4Y & 860.99 & 44.63 & 
872.36 & 46.59 & \bf{852.46} & 
1.00 & 2.33\\CMT5X & 1046.69 & 205.14 & 
1061.31 & 129.06 & \bf{1030.56} & 
1.57 & 2.98\\CMT5Y & 1048.16 & 105.49 & 
1061.84 & 82.57 & \bf{1031.69} & 
1.60 & 2.92\\CMT11X & 869.87 & 32.10 & 
916.02 & 54.55 & \bf{831.09} & 
4.67 & 10.22\\CMT11Y & 873.14 & 33.83 & 
880.33 & 36.66 & \bf{829.85} & 
5.22 & 6.08\\CMT12X & 664.07 & 15.73 & 
680.90 & 20.53 & \bf{658.83} & 
0.80 & 3.35\\CMT12Y & 674.40 & 22.88 & 
682.37 & 12.76 & \bf{660.47} & 
2.11 & 3.32\\\bf{PROM.} & 
\bf{760.82} & \bf{46.61} & \bf{770.47} & \bf{36.47} & \bf{749.50} & \bf{1.36} & \bf{2.55}\\[1ex]\hline
\end{tabular}
\label{table:nonlin}
\end{table} 
\begin{table}[h]
\caption{Resultados de la ejecución de la metaheurística GTS-M, utilizando instancias de SalhiNagy con la configuración -mni 6000 -lambda1 0.05 -lambda2 0.05 -tabu 29}
\centering
\small
\begin{tabular}{c c c c c c c c}
\hline\hline
Instancia & Costo mínimo & Tiempo(seg.) & Costo promedio & Tiempo promedio(seg.) & CME & \%G & \%GP \\ [0.5ex]
\hline
CMT1X & \bf{470.48} & 6.68 & 
471.00 & 6.77 & 470.48 & 0.00
 & 0.11\\CMT1Y & \bf{470.48} & 5.09 & 
470.48 & 6.37 & 470.48 & 0.00
 & 0.00\\
CMT2X & \bf{682.39} & 15.27 & 
687.72 & 10.73 & 682.39 & 0.00
 & 0.78\\CMT2Y & 684.24 & 9.03 & 
687.19 & 9.91 & \bf{682.39} & 
0.27 & 0.70\\CMT3X & \bf{\underline{718.40}} & 33.52 & 
724.05 & 19.34 & 719.06 & 
\bf{-0.09} & 0.69\\CMT3Y & 724.61 & 13.60 & 
726.73 & 14.06 & \bf{719.06} & 
0.77 & 1.07\\CMT4X & 865.48 & 31.88 & 
874.62 & 38.20 & \bf{854.21} & 
1.32 & 2.39\\CMT4Y & 854.60 & 62.67 & 
858.79 & 62.72 & \bf{852.46} & 
0.25 & 0.74\\CMT5X & 1048.92 & 100.90 & 
1065.85 & 107.50 & \bf{1030.56} & 
1.78 & 3.42\\CMT5Y & 1063.53 & 84.99 & 
1066.91 & 71.83 & \bf{1031.69} & 
3.09 & 3.41\\CMT11X & 878.62 & 20.20 & 
919.99 & 24.25 & \bf{831.09} & 
5.72 & 10.70\\CMT11Y & 877.30 & 69.82 & 
909.59 & 39.31 & \bf{829.85} & 
5.72 & 9.61\\CMT12X & 664.03 & 10.90 & 
672.56 & 21.02 & \bf{658.83} & 
0.79 & 2.08\\CMT12Y & 674.35 & 17.00 & 
683.50 & 16.16 & \bf{660.47} & 
2.10 & 3.49\\\bf{PROM.} & 
\bf{762.67} & \bf{34.40} & \bf{772.78} & \bf{32.01} & \bf{749.50} & \bf{1.55} & \bf{2.80}\\[1ex]\hline
\end{tabular}
\label{table:nonlin}
\end{table} 
\begin{table}[h]
\caption{Resultados de la ejecución de la metaheurística GTS-M, utilizando instancias de SalhiNagy con la configuración -mni 6000 -lambda1 0.05 -lambda2 0.05 -tabu 37}
\centering
\small
\begin{tabular}{c c c c c c c c}
\hline\hline
Instancia & Costo mínimo & Tiempo(seg.) & Costo promedio & Tiempo promedio(seg.) & CME & \%G & \%GP \\ [0.5ex]
\hline
CMT1X & \bf{470.48} & 4.91 & 
471.62 & 5.01 & 470.48 & 0.00
 & 0.24\\CMT1Y & \bf{470.48} & 5.88 & 
471.62 & 4.64 & 470.48 & 0.00
 & 0.24\\CMT2X & 684.29 & 9.88 & 
688.68 & 8.28 & \bf{682.39} & 
0.28 & 0.92\\CMT2Y & 686.36 & 16.65 & 
689.14 & 9.42 & \bf{682.39} & 
0.58 & 0.99\\CMT3X & 720.18 & 32.90 & 
722.04 & 25.22 & \bf{719.06} & 
0.16 & 0.41\\CMT3Y & \bf{\underline{718.40}} & 11.80 & 
726.87 & 12.34 & 719.06 & 
\bf{-0.09} & 1.09\\CMT4X & 863.50 & 58.77 & 
865.25 & 51.08 & \bf{854.21} & 
1.09 & 1.29\\CMT4Y & 853.87 & 37.43 & 
873.71 & 51.29 & \bf{852.46} & 
0.17 & 2.49\\CMT5X & 1052.84 & 143.98 & 
1068.72 & 101.87 & \bf{1030.56} & 
2.16 & 3.70\\CMT5Y & \bf{\underline{1031.58}} & 170.46 & 
1061.21 & 120.06 & 1031.69 & 
\bf{-0.01} & 2.86\\CMT11X & 899.00 & 97.67 & 
926.85 & 43.22 & \bf{831.09} & 
8.17 & 11.52\\CMT11Y & 887.56 & 20.82 & 
927.96 & 35.98 & \bf{829.85} & 
6.95 & 11.82\\CMT12X & 667.81 & 15.18 & 
670.24 & 18.77 & \bf{658.83} & 
1.36 & 1.73\\CMT12Y & 681.37 & 34.00 & 
682.54 & 17.79 & \bf{660.47} & 
3.16 & 3.34\\\bf{PROM.} & 
\bf{763.41} & \bf{47.17} & \bf{774.75} & \bf{36.07} & \bf{749.50} & \bf{1.71} & \bf{3.05}\\[1ex]\hline
\end{tabular}
\label{table:nonlin}
\end{table} 
\begin{table}[h]
\caption{Resultados de la ejecución de la metaheurística GTS-M, utilizando instancias de SalhiNagy con la configuración -mni 6000 -lambda1 0.05 -lambda2 0.05 -tabu 5}
\centering
\small
\begin{tabular}{c c c c c c c c}
\hline\hline
Instancia & Costo mínimo & Tiempo(seg.) & Costo promedio & Tiempo promedio(seg.) & CME & \%G & \%GP \\ [0.5ex]
\hline
CMT1X & \bf{470.48} & 3.80 & 
470.95 & 3.93 & 470.48 & 0.00
 & 0.10\\CMT1Y & \bf{470.48} & 7.44 & 
470.95 & 6.68 & 470.48 & 0.00
 & 0.10\\CMT2X & 688.06 & 9.45 & 
689.90 & 8.47 & \bf{682.39} & 
0.83 & 1.10\\CMT2Y & 684.29 & 8.03 & 
690.79 & 8.82 & \bf{682.39} & 
0.28 & 1.23\\CMT3X & 720.17 & 14.80 & 
727.20 & 12.06 & \bf{719.06} & 
0.15 & 1.13\\CMT3Y & 725.02 & 15.33 & 
730.14 & 16.30 & \bf{719.06} & 
0.83 & 1.54\\CMT4X & 863.01 & 68.36 & 
879.09 & 59.50 & \bf{854.21} & 
1.03 & 2.91\\CMT4Y & 861.32 & 67.47 & 
868.18 & 47.98 & \bf{852.46} & 
1.04 & 1.84\\CMT5X & 1055.39 & 118.95 & 
1068.93 & 83.02 & \bf{1030.56} & 
2.41 & 3.72\\CMT5Y & 1049.57 & 79.66 & 
1070.86 & 105.15 & \bf{1031.69} & 
1.73 & 3.80\\CMT11X & 896.87 & 48.70 & 
944.09 & 40.97 & \bf{831.09} & 
7.91 & 13.60\\CMT11Y & 875.47 & 41.09 & 
895.22 & 41.19 & \bf{829.85} & 
5.50 & 7.88\\CMT12X & 667.62 & 11.11 & 
671.05 & 13.10 & \bf{658.83} & 
1.33 & 1.86\\CMT12Y & 674.44 & 19.16 & 
682.28 & 21.88 & \bf{660.47} & 
2.12 & 3.30\\\bf{PROM.} & 
\bf{764.44} & \bf{36.67} & \bf{775.69} & \bf{33.50} & \bf{749.50} & \bf{1.80} & \bf{3.15}\\[1ex]\hline
\end{tabular}
\label{table:GTS-M-salhinagy-6000-5}
\end{table}

\clearpage
\section{AS-M}\label{tablas-entonacion-AS-M}

\subsection{Dethloff}\label{tablas-entonacion-AS-M-dethloff}
\begin{table}[h]
\caption{Resultados de la ejecución de la metaheurística AS-M, utilizando instancias de Dethloff con la configuración -n 3.0 -alpha 1.0 -beta 3.0 -q 0.1 -ro 0.015}
\centering
\small
\begin{tabular}{c c c c c c c c}
\hline\hline
Instancia & Costo mínimo & Tiempo(seg.) & Costo promedio & Tiempo promedio(seg.) & CME & \%G & \%GP \\ [0.5ex]
\hline
SCA3-0 & 636.06 & 2.15 & 
636.13 & 2.13 & \bf{635.62} & 
0.07 & 0.08\\SCA3-1 & \bf{697.84} & 2.35 & 
697.84 & 2.34 & 697.84 & 0.00
 & 0.00\\
SCA3-2 & \bf{659.34} & 2.24 & 
662.99 & 2.23 & 659.34 & 0.00
 & 0.55\\SCA3-3 & \bf{680.04} & 2.02 & 
680.32 & 2.13 & 680.04 & 0.00
 & 0.04\\SCA3-4 & \bf{690.50} & 2.20 & 
690.50 & 2.33 & 690.50 & 0.00
 & 0.00\\
SCA3-5 & \bf{659.90} & 2.28 & 
662.61 & 2.26 & 659.90 & 0.00
 & 0.41\\SCA3-6 & \bf{651.09} & 2.24 & 
652.79 & 2.22 & 651.09 & 0.00
 & 0.26\\SCA3-7 & 666.15 & 1.93 & 
667.80 & 1.93 & \bf{659.17} & 
1.06 & 1.31\\SCA3-8 & \bf{719.47} & 2.22 & 
720.26 & 2.25 & 719.47 & 0.00
 & 0.11\\SCA3-9 & \bf{681.00} & 2.03 & 
681.81 & 1.91 & 681.00 & 0.00
 & 0.12\\SCA8-0 & \bf{961.50} & 2.24 & 
977.14 & 2.31 & 961.50 & 0.00
 & 1.63\\SCA8-1 & 1054.08 & 2.16 & 
1067.20 & 2.10 & \bf{1049.65} & 
0.42 & 1.67\\SCA8-2 & 1050.35 & 1.90 & 
1051.86 & 1.95 & \bf{1039.64} & 
1.03 & 1.18\\SCA8-3 & 1000.72 & 2.11 & 
1007.16 & 2.21 & \bf{983.34} & 
1.77 & 2.42\\SCA8-4 & 1067.55 & 2.34 & 
1068.66 & 2.28 & \bf{1065.49} & 
0.19 & 0.30\\SCA8-5 & 1052.18 & 2.64 & 
1054.87 & 2.64 & \bf{1027.08} & 
2.44 & 2.71\\SCA8-6 & 972.48 & 2.60 & 
978.37 & 2.54 & \bf{971.82} & 
0.07 & 0.67\\SCA8-7 & 1066.65 & 2.56 & 
1070.78 & 2.50 & \bf{1051.28} & 
1.46 & 1.85\\SCA8-8 & \bf{1071.18} & 2.55 & 
1081.24 & 2.50 & 1071.18 & 0.00
 & 0.94\\SCA8-9 & 1067.42 & 2.10 & 
1068.15 & 2.13 & \bf{1060.50} & 
0.65 & 0.72\\CON3-0 & 620.76 & 2.38 & 
621.92 & 2.40 & \bf{616.52} & 
0.69 & 0.88\\CON3-1 & \bf{554.47} & 2.19 & 
558.37 & 2.33 & 554.47 & 0.00
 & 0.70\\CON3-2 & 521.38 & 2.25 & 
521.38 & 2.29 & \bf{518.00} & 
0.65 & 0.65\\CON3-3 & 591.20 & 2.31 & 
591.24 & 2.35 & \bf{591.19} & 
0.00 & 0.01\\CON3-4 & \bf{588.79} & 2.15 & 
590.11 & 2.21 & 588.79 & 0.00
 & 0.22\\CON3-5 & \bf{563.70} & 2.30 & 
567.50 & 2.18 & 563.70 & 0.00
 & 0.67\\CON3-6 & 500.80 & 2.64 & 
502.42 & 2.60 & \bf{499.05} & 
0.35 & 0.68\\CON3-7 & 578.22 & 2.06 & 
581.57 & 1.99 & \bf{576.48} & 
0.30 & 0.88\\CON3-8 & 524.38 & 2.01 & 
526.31 & 2.06 & \bf{523.05} & 
0.25 & 0.62\\CON3-9 & 578.25 & 2.25 & 
584.43 & 2.14 & \bf{578.24} & 
0.00 & 1.07\\CON8-0 & 871.77 & 2.38 & 
875.49 & 2.40 & \bf{857.17} & 
1.70 & 2.14\\CON8-1 & \bf{740.85} & 2.40 & 
748.45 & 2.34 & 740.85 & 0.00
 & 1.03\\CON8-2 & 713.68 & 3.16 & 
713.85 & 2.80 & \bf{712.89} & 
0.11 & 0.13\\CON8-3 & \bf{811.07} & 2.36 & 
814.29 & 2.37 & 811.07 & 0.00
 & 0.40\\CON8-4 & 776.34 & 2.27 & 
783.53 & 2.27 & \bf{772.25} & 
0.53 & 1.46\\CON8-5 & 758.84 & 2.29 & 
764.03 & 2.35 & \bf{754.88} & 
0.52 & 1.21\\CON8-6 & 691.42 & 2.69 & 
694.53 & 2.69 & \bf{678.92} & 
1.84 & 2.30\\CON8-7 & 814.79 & 1.99 & 
814.84 & 2.15 & \bf{811.96} & 
0.35 & 0.36\\CON8-8 & 778.40 & 2.62 & 
781.59 & 2.55 & \bf{767.53} & 
1.42 & 1.83\\CON8-9 & 810.61 & 2.68 & 
814.49 & 2.67 & \bf{809.00} & 
0.20 & 0.68\\\bf{PROM.} & 
\bf{762.38} & \bf{2.31} & \bf{765.72} & \bf{2.30} & \bf{758.54} & \bf{0.45} & \bf{0.87}\\[1ex]\hline
\end{tabular}
\label{table:AS-M-dethloff}
\end{table}
\begin{table}[h]
\caption{Resultados de la ejecución de la metaheurística AS-M, utilizando instancias de Dethloff con la configuración -n 3.0 -alpha 1.0 -beta 3.0 -q .3 -ro 0.015}
\centering
\small
\begin{tabular}{c c c c c c c c}
\hline\hline
Instancia & Costo mínimo & Tiempo(seg.) & Costo promedio & Tiempo promedio(seg.) & CME & \%G & \%GP \\ [0.5ex]
\hline
SCA3-0 & 636.06 & 2.12 & 
636.20 & 2.18 & \bf{635.62} & 
0.07 & 0.09\\SCA3-1 & \bf{697.84} & 2.34 & 
697.84 & 2.42 & 697.84 & 0.00
 & 0.00\\
SCA3-2 & \bf{659.34} & 2.00 & 
659.79 & 2.07 & 659.34 & 0.00
 & 0.07\\SCA3-3 & \bf{680.04} & 2.15 & 
680.46 & 2.10 & 680.04 & 0.00
 & 0.06\\SCA3-4 & \bf{690.50} & 2.32 & 
690.50 & 2.27 & 690.50 & 0.00
 & 0.00\\
SCA3-5 & \bf{659.90} & 2.18 & 
663.17 & 2.20 & 659.90 & 0.00
 & 0.50\\SCA3-6 & \bf{651.09} & 2.19 & 
652.01 & 2.31 & 651.09 & 0.00
 & 0.14\\SCA3-7 & 666.15 & 1.88 & 
667.20 & 1.91 & \bf{659.17} & 
1.06 & 1.22\\SCA3-8 & \bf{719.47} & 2.21 & 
721.88 & 2.33 & 719.47 & 0.00
 & 0.33\\SCA3-9 & \bf{681.00} & 2.07 & 
681.64 & 1.86 & 681.00 & 0.00
 & 0.09\\SCA8-0 & \bf{961.50} & 2.34 & 
974.22 & 2.35 & 961.50 & 0.00
 & 1.32\\SCA8-1 & 1058.98 & 2.00 & 
1065.31 & 2.08 & \bf{1049.65} & 
0.89 & 1.49\\SCA8-2 & 1047.63 & 1.69 & 
1049.89 & 1.86 & \bf{1039.64} & 
0.77 & 0.99\\SCA8-3 & 991.84 & 2.08 & 
1007.50 & 2.18 & \bf{983.34} & 
0.86 & 2.46\\SCA8-4 & 1067.28 & 2.30 & 
1068.84 & 2.32 & \bf{1065.49} & 
0.17 & 0.31\\SCA8-5 & 1051.47 & 2.57 & 
1055.03 & 2.56 & \bf{1027.08} & 
2.37 & 2.72\\SCA8-6 & 977.83 & 2.26 & 
980.03 & 2.34 & \bf{971.82} & 
0.62 & 0.85\\SCA8-7 & 1067.20 & 2.48 & 
1073.25 & 2.48 & \bf{1051.28} & 
1.51 & 2.09\\SCA8-8 & \bf{1071.18} & 2.53 & 
1079.95 & 2.49 & 1071.18 & 0.00
 & 0.82\\SCA8-9 & 1067.42 & 2.47 & 
1067.42 & 2.10 & \bf{1060.50} & 
0.65 & 0.65\\CON3-0 & 619.09 & 2.37 & 
622.42 & 2.37 & \bf{616.52} & 
0.42 & 0.96\\CON3-1 & \bf{554.47} & 2.09 & 
556.53 & 2.67 & 554.47 & 0.00
 & 0.37\\CON3-2 & 521.38 & 2.22 & 
522.03 & 2.18 & \bf{518.00} & 
0.65 & 0.78\\CON3-3 & \bf{591.19} & 2.40 & 
591.20 & 2.46 & 591.19 & 0.00
 & 0.00\\CON3-4 & \bf{588.79} & 2.06 & 
589.87 & 2.01 & 588.79 & 0.00
 & 0.18\\CON3-5 & 564.88 & 2.28 & 
566.78 & 2.38 & \bf{563.70} & 
0.21 & 0.55\\CON3-6 & 502.88 & 2.50 & 
503.83 & 2.45 & \bf{499.05} & 
0.77 & 0.96\\CON3-7 & 578.22 & 1.92 & 
580.22 & 1.87 & \bf{576.48} & 
0.30 & 0.65\\CON3-8 & 523.14 & 2.15 & 
525.98 & 2.10 & \bf{523.05} & 
0.02 & 0.56\\CON3-9 & 578.98 & 2.27 & 
586.16 & 2.17 & \bf{578.24} & 
0.13 & 1.37\\CON8-0 & 867.28 & 2.35 & 
874.08 & 2.31 & \bf{857.17} & 
1.18 & 1.97\\CON8-1 & \bf{740.85} & 2.31 & 
743.89 & 2.33 & 740.85 & 0.00
 & 0.41\\CON8-2 & 714.06 & 2.75 & 
716.35 & 2.72 & \bf{712.89} & 
0.16 & 0.49\\CON8-3 & 816.45 & 2.40 & 
820.67 & 2.34 & \bf{811.07} & 
0.66 & 1.18\\CON8-4 & 780.48 & 2.10 & 
788.00 & 2.15 & \bf{772.25} & 
1.07 & 2.04\\CON8-5 & 760.91 & 2.28 & 
762.53 & 2.20 & \bf{754.88} & 
0.80 & 1.01\\CON8-6 & 689.11 & 3.21 & 
691.61 & 2.72 & \bf{678.92} & 
1.50 & 1.87\\CON8-7 & 814.86 & 2.24 & 
818.69 & 2.04 & \bf{811.96} & 
0.36 & 0.83\\CON8-8 & 791.31 & 2.54 & 
791.87 & 2.58 & \bf{767.53} & 
3.10 & 3.17\\CON8-9 & \bf{809.00} & 2.72 & 
812.99 & 2.55 & 809.00 & 0.00
 & 0.49\\\bf{PROM.} & 
\bf{762.78} & \bf{2.28} & \bf{765.95} & \bf{2.27} & \bf{758.54} & \bf{0.51} & \bf{0.90}\\[1ex]\hline
\end{tabular}
\label{table:nonlin}
\end{table}
\begin{table}[h]
\caption{Resultados de la ejecución de la metaheurística AS-M, utilizando instancias de Dethloff con la configuración -n 3.0 -alpha 1.0 -beta 3.0 -q .4 -ro 0.015}
\centering
\small
\begin{tabular}{c c c c c c c c}
\hline\hline
Instancia & Costo mínimo & Tiempo(seg.) & Costo promedio & Tiempo promedio(seg.) & CME & \%G & \%GP \\ [0.5ex]
\hline
SCA3-0 & 636.06 & 2.05 & 
637.47 & 2.03 & \bf{635.62} & 
0.07 & 0.29\\SCA3-1 & \bf{697.84} & 2.35 & 
698.76 & 2.29 & 697.84 & 0.00
 & 0.13\\SCA3-2 & \bf{659.34} & 2.09 & 
663.23 & 2.09 & 659.34 & 0.00
 & 0.59\\SCA3-3 & \bf{680.04} & 2.04 & 
680.64 & 2.13 & 680.04 & 0.00
 & 0.09\\SCA3-4 & \bf{690.50} & 2.24 & 
690.50 & 2.26 & 690.50 & 0.00
 & 0.00\\
SCA3-5 & 665.04 & 2.20 & 
666.47 & 2.22 & \bf{659.90} & 
0.78 & 1.00\\SCA3-6 & \bf{651.09} & 2.05 & 
653.34 & 2.21 & 651.09 & 0.00
 & 0.35\\SCA3-7 & 666.15 & 1.82 & 
667.31 & 1.84 & \bf{659.17} & 
1.06 & 1.23\\SCA3-8 & \bf{719.47} & 2.11 & 
720.19 & 2.13 & 719.47 & 0.00
 & 0.10\\SCA3-9 & \bf{681.00} & 1.74 & 
681.00 & 1.76 & 681.00 & 0.00
 & 0.00\\
SCA8-0 & \bf{961.50} & 2.51 & 
981.24 & 2.42 & 961.50 & 0.00
 & 2.05\\SCA8-1 & 1056.11 & 1.84 & 
1061.02 & 1.89 & \bf{1049.65} & 
0.62 & 1.08\\SCA8-2 & 1050.17 & 1.84 & 
1050.53 & 1.84 & \bf{1039.64} & 
1.01 & 1.05\\SCA8-3 & 1000.96 & 2.20 & 
1010.46 & 2.15 & \bf{983.34} & 
1.79 & 2.76\\SCA8-4 & \bf{1065.49} & 2.50 & 
1067.04 & 2.31 & 1065.49 & 0.00
 & 0.15\\SCA8-5 & 1038.59 & 2.42 & 
1049.78 & 2.55 & \bf{1027.08} & 
1.12 & 2.21\\SCA8-6 & 972.48 & 2.37 & 
978.93 & 2.40 & \bf{971.82} & 
0.07 & 0.73\\SCA8-7 & 1067.20 & 2.48 & 
1069.38 & 2.59 & \bf{1051.28} & 
1.51 & 1.72\\SCA8-8 & \bf{1071.18} & 2.47 & 
1081.82 & 2.45 & 1071.18 & 0.00
 & 0.99\\SCA8-9 & 1067.42 & 2.12 & 
1068.36 & 2.02 & \bf{1060.50} & 
0.65 & 0.74\\CON3-0 & \bf{616.52} & 2.28 & 
622.44 & 2.31 & 616.52 & 0.00
 & 0.96\\CON3-1 & 557.21 & 2.34 & 
559.02 & 2.32 & \bf{554.47} & 
0.49 & 0.82\\CON3-2 & 521.38 & 2.43 & 
522.10 & 2.27 & \bf{518.00} & 
0.65 & 0.79\\CON3-3 & \bf{591.19} & 2.44 & 
591.20 & 2.52 & 591.19 & 0.00
 & 0.00\\CON3-4 & \bf{588.79} & 1.98 & 
591.64 & 2.02 & 588.79 & 0.00
 & 0.48\\CON3-5 & 564.88 & 2.07 & 
565.92 & 2.12 & \bf{563.70} & 
0.21 & 0.39\\CON3-6 & 500.80 & 2.42 & 
503.09 & 2.52 & \bf{499.05} & 
0.35 & 0.81\\CON3-7 & 578.22 & 2.04 & 
580.37 & 1.93 & \bf{576.48} & 
0.30 & 0.67\\CON3-8 & \bf{523.05} & 2.19 & 
524.71 & 2.19 & 523.05 & 0.00
 & 0.32\\CON3-9 & 585.25 & 1.89 & 
587.19 & 2.01 & \bf{578.24} & 
1.21 & 1.55\\CON8-0 & 869.15 & 2.10 & 
876.19 & 2.28 & \bf{857.17} & 
1.40 & 2.22\\CON8-1 & 742.44 & 2.47 & 
746.95 & 2.31 & \bf{740.85} & 
0.21 & 0.82\\CON8-2 & 713.60 & 2.78 & 
716.54 & 2.74 & \bf{712.89} & 
0.10 & 0.51\\CON8-3 & 812.54 & 2.30 & 
816.31 & 2.36 & \bf{811.07} & 
0.18 & 0.65\\CON8-4 & 784.87 & 2.10 & 
789.59 & 2.16 & \bf{772.25} & 
1.63 & 2.24\\CON8-5 & 759.93 & 2.23 & 
763.37 & 2.21 & \bf{754.88} & 
0.67 & 1.12\\CON8-6 & 690.85 & 2.65 & 
694.59 & 2.54 & \bf{678.92} & 
1.76 & 2.31\\CON8-7 & 814.79 & 1.94 & 
818.30 & 2.00 & \bf{811.96} & 
0.35 & 0.78\\CON8-8 & 785.30 & 2.38 & 
787.58 & 2.45 & \bf{767.53} & 
2.32 & 2.61\\CON8-9 & 814.37 & 2.45 & 
816.09 & 2.46 & \bf{809.00} & 
0.66 & 0.88\\\bf{PROM.} & 
\bf{762.82} & \bf{2.22} & \bf{766.27} & \bf{2.23} & \bf{758.54} & \bf{0.53} & \bf{0.96}\\[1ex]\hline
\end{tabular}
\label{table:nonlin}
\end{table}
\begin{table}[h]
\caption{Resultados de la ejecución de la metaheurística AS-M, utilizando instancias de Dethloff con la configuración -n 4.0 -alpha 1.0 -beta 3.0 -q 0.1 -ro 0.015}
\centering
\small
\begin{tabular}{c c c c c c c c}
\hline\hline
Instancia & Costo mínimo & Tiempo(seg.) & Costo promedio & Tiempo promedio(seg.) & CME & \%G & \%GP \\ [0.5ex]
\hline
SCA3-0 & 636.06 & 2.97 & 
636.06 & 3.11 & \bf{635.62} & 
0.07 & 0.07\\SCA3-1 & \bf{697.84} & 3.06 & 
697.84 & 3.23 & 697.84 & 0.00
 & 0.00\\
SCA3-2 & \bf{659.34} & 2.79 & 
659.79 & 2.85 & 659.34 & 0.00
 & 0.07\\SCA3-3 & \bf{680.04} & 2.69 & 
680.36 & 2.81 & 680.04 & 0.00
 & 0.05\\SCA3-4 & \bf{690.50} & 3.00 & 
690.50 & 3.02 & 690.50 & 0.00
 & 0.00\\
SCA3-5 & 662.75 & 2.94 & 
663.47 & 3.08 & \bf{659.90} & 
0.43 & 0.54\\SCA3-6 & \bf{651.09} & 3.12 & 
652.48 & 3.09 & 651.09 & 0.00
 & 0.21\\SCA3-7 & 664.88 & 2.92 & 
665.95 & 2.66 & \bf{659.17} & 
0.87 & 1.03\\SCA3-8 & \bf{719.47} & 2.92 & 
719.54 & 3.01 & 719.47 & 0.00
 & 0.01\\SCA3-9 & \bf{681.00} & 2.65 & 
681.00 & 2.56 & 681.00 & 0.00
 & 0.00\\
SCA8-0 & \bf{961.50} & 2.93 & 
979.07 & 3.11 & 961.50 & 0.00
 & 1.83\\SCA8-1 & 1059.55 & 2.83 & 
1067.80 & 2.72 & \bf{1049.65} & 
0.94 & 1.73\\SCA8-2 & 1050.17 & 2.52 & 
1050.47 & 2.48 & \bf{1039.64} & 
1.01 & 1.04\\SCA8-3 & 1005.22 & 3.11 & 
1009.93 & 2.97 & \bf{983.34} & 
2.23 & 2.70\\SCA8-4 & \bf{1065.49} & 3.25 & 
1068.75 & 3.06 & 1065.49 & 0.00
 & 0.31\\SCA8-5 & 1034.74 & 4.26 & 
1049.29 & 3.69 & \bf{1027.08} & 
0.75 & 2.16\\SCA8-6 & 979.28 & 3.24 & 
979.71 & 3.31 & \bf{971.82} & 
0.77 & 0.81\\SCA8-7 & 1066.65 & 3.28 & 
1070.93 & 3.10 & \bf{1051.28} & 
1.46 & 1.87\\SCA8-8 & \bf{1071.18} & 3.22 & 
1081.12 & 3.32 & 1071.18 & 0.00
 & 0.93\\SCA8-9 & 1067.42 & 2.76 & 
1067.42 & 3.00 & \bf{1060.50} & 
0.65 & 0.65\\CON3-0 & \bf{616.52} & 3.01 & 
620.82 & 3.17 & 616.52 & 0.00
 & 0.70\\CON3-1 & \bf{554.47} & 3.29 & 
555.84 & 3.17 & 554.47 & 0.00
 & 0.25\\CON3-2 & 519.61 & 2.90 & 
521.00 & 2.92 & \bf{518.00} & 
0.31 & 0.58\\CON3-3 & \bf{591.19} & 3.48 & 
591.20 & 3.25 & 591.19 & 0.00
 & 0.00\\CON3-4 & \bf{588.79} & 2.93 & 
588.79 & 2.87 & 588.79 & 0.00
 & 0.00\\
CON3-5 & 564.88 & 2.89 & 
567.00 & 2.92 & \bf{563.70} & 
0.21 & 0.59\\CON3-6 & 500.80 & 3.23 & 
502.21 & 3.28 & \bf{499.05} & 
0.35 & 0.63\\CON3-7 & 578.22 & 2.58 & 
580.68 & 2.77 & \bf{576.48} & 
0.30 & 0.73\\CON3-8 & \bf{523.05} & 3.25 & 
523.77 & 3.02 & 523.05 & 0.00
 & 0.14\\CON3-9 & 588.40 & 3.06 & 
588.42 & 2.86 & \bf{578.24} & 
1.76 & 1.76\\CON8-0 & 870.22 & 3.16 & 
871.56 & 3.09 & \bf{857.17} & 
1.52 & 1.68\\CON8-1 & \bf{740.85} & 3.05 & 
742.18 & 3.07 & 740.85 & 0.00
 & 0.18\\CON8-2 & 713.44 & 3.72 & 
714.84 & 3.63 & \bf{712.89} & 
0.08 & 0.27\\CON8-3 & 812.11 & 3.08 & 
816.32 & 3.08 & \bf{811.07} & 
0.13 & 0.65\\CON8-4 & 780.09 & 3.06 & 
783.16 & 2.91 & \bf{772.25} & 
1.02 & 1.41\\CON8-5 & 758.84 & 3.26 & 
762.70 & 3.27 & \bf{754.88} & 
0.52 & 1.04\\CON8-6 & 686.97 & 3.40 & 
693.16 & 3.49 & \bf{678.92} & 
1.19 & 2.10\\CON8-7 & 814.79 & 2.74 & 
816.97 & 2.80 & \bf{811.96} & 
0.35 & 0.62\\CON8-8 & 782.09 & 3.40 & 
784.41 & 3.50 & \bf{767.53} & 
1.90 & 2.20\\CON8-9 & 811.59 & 3.50 & 
814.12 & 3.46 & \bf{809.00} & 
0.32 & 0.63\\\bf{PROM.} & 
\bf{762.53} & \bf{3.09} & \bf{765.27} & \bf{3.07} & \bf{758.54} & \bf{0.48} & \bf{0.80}\\[1ex]\hline
\end{tabular}
\label{table:nonlin}
\end{table}
\begin{table}[h]
\caption{Resultados de la ejecución de la metaheurística AS-M, utilizando instancias de Dethloff con la configuración -n 4.0 -alpha 1.0 -beta 3.0 -q .3 -ro 0.015}
\centering
\small
\begin{tabular}{c c c c c c c c}
\hline\hline
Instancia & Costo mínimo & Tiempo(seg.) & Costo promedio & Tiempo promedio(seg.) & CME & \%G & \%GP \\ [0.5ex]
\hline
SCA3-0 & 636.06 & 2.73 & 
636.06 & 2.79 & \bf{635.62} & 
0.07 & 0.07\\SCA3-1 & \bf{697.84} & 3.10 & 
697.84 & 3.12 & 697.84 & 0.00
 & 0.00\\
SCA3-2 & \bf{659.34} & 2.92 & 
661.76 & 2.82 & 659.34 & 0.00
 & 0.37\\SCA3-3 & \bf{680.04} & 2.65 & 
680.04 & 2.79 & 680.04 & 0.00
 & 0.00\\
SCA3-4 & \bf{690.50} & 3.13 & 
690.50 & 3.02 & 690.50 & 0.00
 & 0.00\\
SCA3-5 & 665.04 & 3.04 & 
665.90 & 2.94 & \bf{659.90} & 
0.78 & 0.91\\SCA3-6 & \bf{651.09} & 2.71 & 
652.01 & 2.83 & 651.09 & 0.00
 & 0.14\\SCA3-7 & 666.15 & 2.50 & 
666.15 & 2.42 & \bf{659.17} & 
1.06 & 1.06\\SCA3-8 & \bf{719.47} & 2.72 & 
720.12 & 2.84 & 719.47 & 0.00
 & 0.09\\SCA3-9 & \bf{681.00} & 2.43 & 
681.00 & 2.41 & 681.00 & 0.00
 & 0.00\\
SCA8-0 & 968.79 & 2.90 & 
980.97 & 3.06 & \bf{961.50} & 
0.76 & 2.03\\SCA8-1 & 1060.41 & 2.81 & 
1065.82 & 2.72 & \bf{1049.65} & 
1.03 & 1.54\\SCA8-2 & 1050.37 & 2.30 & 
1050.94 & 2.33 & \bf{1039.64} & 
1.03 & 1.09\\SCA8-3 & 998.59 & 2.86 & 
1007.59 & 2.90 & \bf{983.34} & 
1.55 & 2.47\\SCA8-4 & 1067.28 & 2.98 & 
1069.28 & 3.02 & \bf{1065.49} & 
0.17 & 0.36\\SCA8-5 & 1045.15 & 3.42 & 
1054.64 & 3.51 & \bf{1027.08} & 
1.76 & 2.68\\SCA8-6 & 977.03 & 3.09 & 
981.62 & 3.25 & \bf{971.82} & 
0.54 & 1.01\\SCA8-7 & 1067.20 & 3.14 & 
1071.31 & 3.20 & \bf{1051.28} & 
1.51 & 1.91\\SCA8-8 & \bf{1071.18} & 3.34 & 
1072.13 & 3.34 & 1071.18 & 0.00
 & 0.09\\SCA8-9 & 1067.42 & 2.76 & 
1068.24 & 2.63 & \bf{1060.50} & 
0.65 & 0.73\\CON3-0 & 617.59 & 3.25 & 
620.40 & 3.31 & \bf{616.52} & 
0.17 & 0.63\\CON3-1 & \bf{554.47} & 2.94 & 
556.69 & 3.03 & 554.47 & 0.00
 & 0.40\\CON3-2 & 521.38 & 2.63 & 
522.62 & 2.89 & \bf{518.00} & 
0.65 & 0.89\\CON3-3 & \bf{591.19} & 3.24 & 
591.23 & 3.20 & 591.19 & 0.00
 & 0.01\\CON3-4 & \bf{588.79} & 2.78 & 
590.11 & 2.81 & 588.79 & 0.00
 & 0.22\\CON3-5 & \bf{563.70} & 2.88 & 
565.24 & 2.88 & 563.70 & 0.00
 & 0.27\\CON3-6 & 502.16 & 3.20 & 
502.33 & 3.41 & \bf{499.05} & 
0.62 & 0.66\\CON3-7 & 578.41 & 2.56 & 
580.11 & 2.69 & \bf{576.48} & 
0.33 & 0.63\\CON3-8 & 523.14 & 2.80 & 
523.64 & 2.78 & \bf{523.05} & 
0.02 & 0.11\\CON3-9 & 578.25 & 2.82 & 
585.18 & 2.77 & \bf{578.24} & 
0.00 & 1.20\\CON8-0 & 869.43 & 3.07 & 
876.69 & 2.94 & \bf{857.17} & 
1.43 & 2.28\\CON8-1 & \bf{740.85} & 3.05 & 
743.49 & 3.21 & 740.85 & 0.00
 & 0.36\\CON8-2 & \bf{712.89} & 3.62 & 
715.24 & 3.69 & 712.89 & 0.00
 & 0.33\\CON8-3 & 815.14 & 3.04 & 
816.88 & 3.08 & \bf{811.07} & 
0.50 & 0.72\\CON8-4 & 776.72 & 2.98 & 
782.70 & 3.16 & \bf{772.25} & 
0.58 & 1.35\\CON8-5 & 755.14 & 3.14 & 
759.85 & 2.99 & \bf{754.88} & 
0.03 & 0.66\\CON8-6 & 693.80 & 3.26 & 
694.38 & 3.40 & \bf{678.92} & 
2.19 & 2.28\\CON8-7 & 814.86 & 2.67 & 
820.13 & 2.77 & \bf{811.96} & 
0.36 & 1.01\\CON8-8 & 783.02 & 3.18 & 
788.58 & 3.40 & \bf{767.53} & 
2.02 & 2.74\\CON8-9 & 811.32 & 3.37 & 
813.45 & 3.36 & \bf{809.00} & 
0.29 & 0.55\\\bf{PROM.} & 
\bf{762.81} & \bf{2.95} & \bf{765.57} & \bf{2.99} & \bf{758.54} & \bf{0.50} & \bf{0.85}\\[1ex]\hline
\end{tabular}
\label{table:nonlin}
\end{table}
\begin{table}[h]
\caption{Resultados de la ejecución de la metaheurística AS-M, utilizando instancias de Dethloff con la configuración -n 4.0 -alpha 1.0 -beta 3.0 -q .4 -ro 0.015}
\centering
\small
\begin{tabular}{c c c c c c c c}
\hline\hline
Instancia & Costo mínimo & Tiempo(seg.) & Costo promedio & Tiempo promedio(seg.) & CME & \%G & \%GP \\ [0.5ex]
\hline
SCA3-0 & 636.06 & 3.03 & 
637.47 & 2.84 & \bf{635.62} & 
0.07 & 0.29\\SCA3-1 & \bf{697.84} & 3.16 & 
697.84 & 3.17 & 697.84 & 0.00
 & 0.00\\
SCA3-2 & \bf{659.34} & 2.95 & 
660.24 & 2.85 & 659.34 & 0.00
 & 0.14\\SCA3-3 & \bf{680.04} & 2.89 & 
680.04 & 2.80 & 680.04 & 0.00
 & 0.00\\
SCA3-4 & \bf{690.50} & 2.92 & 
690.50 & 2.90 & 690.50 & 0.00
 & 0.00\\
SCA3-5 & \bf{659.90} & 3.10 & 
662.05 & 2.97 & 659.90 & 0.00
 & 0.33\\SCA3-6 & 652.94 & 2.80 & 
653.13 & 2.83 & \bf{651.09} & 
0.28 & 0.31\\SCA3-7 & 666.15 & 2.44 & 
666.15 & 2.45 & \bf{659.17} & 
1.06 & 1.06\\SCA3-8 & \bf{719.47} & 2.82 & 
720.60 & 2.73 & 719.47 & 0.00
 & 0.16\\SCA3-9 & \bf{681.00} & 2.45 & 
681.00 & 2.41 & 681.00 & 0.00
 & 0.00\\
SCA8-0 & \bf{961.50} & 3.11 & 
970.75 & 3.01 & 961.50 & 0.00
 & 0.96\\SCA8-1 & 1059.55 & 2.65 & 
1062.33 & 2.72 & \bf{1049.65} & 
0.94 & 1.21\\SCA8-2 & 1045.64 & 2.34 & 
1049.73 & 2.35 & \bf{1039.64} & 
0.58 & 0.97\\SCA8-3 & 1011.52 & 2.98 & 
1015.71 & 2.88 & \bf{983.34} & 
2.87 & 3.29\\SCA8-4 & \bf{1065.49} & 3.06 & 
1067.78 & 2.90 & 1065.49 & 0.00
 & 0.21\\SCA8-5 & 1034.74 & 3.41 & 
1042.27 & 3.39 & \bf{1027.08} & 
0.75 & 1.48\\SCA8-6 & 977.03 & 3.44 & 
979.95 & 3.25 & \bf{971.82} & 
0.54 & 0.84\\SCA8-7 & 1067.11 & 3.19 & 
1069.92 & 3.22 & \bf{1051.28} & 
1.51 & 1.77\\SCA8-8 & \bf{1071.18} & 3.16 & 
1076.64 & 3.25 & 1071.18 & 0.00
 & 0.51\\SCA8-9 & 1067.42 & 2.76 & 
1067.42 & 2.59 & \bf{1060.50} & 
0.65 & 0.65\\CON3-0 & \bf{616.52} & 3.37 & 
619.11 & 3.17 & 616.52 & 0.00
 & 0.42\\CON3-1 & \bf{554.47} & 2.83 & 
556.30 & 2.91 & 554.47 & 0.00
 & 0.33\\CON3-2 & 519.11 & 2.86 & 
521.96 & 2.80 & \bf{518.00} & 
0.21 & 0.76\\CON3-3 & \bf{591.19} & 3.19 & 
591.54 & 3.08 & 591.19 & 0.00
 & 0.06\\CON3-4 & \bf{588.79} & 2.65 & 
590.38 & 2.74 & 588.79 & 0.00
 & 0.27\\CON3-5 & \bf{563.70} & 2.80 & 
566.49 & 2.93 & 563.70 & 0.00
 & 0.49\\CON3-6 & 501.05 & 3.37 & 
501.69 & 3.42 & \bf{499.05} & 
0.40 & 0.53\\CON3-7 & \bf{576.48} & 2.68 & 
579.43 & 2.58 & 576.48 & 0.00
 & 0.51\\CON3-8 & 523.14 & 2.90 & 
523.57 & 2.78 & \bf{523.05} & 
0.02 & 0.10\\CON3-9 & 588.48 & 3.09 & 
588.88 & 2.78 & \bf{578.24} & 
1.77 & 1.84\\CON8-0 & 866.22 & 3.07 & 
870.49 & 3.00 & \bf{857.17} & 
1.06 & 1.55\\CON8-1 & 742.29 & 3.14 & 
746.61 & 3.05 & \bf{740.85} & 
0.19 & 0.78\\CON8-2 & 713.60 & 3.40 & 
715.66 & 3.61 & \bf{712.89} & 
0.10 & 0.39\\CON8-3 & 815.80 & 2.92 & 
817.17 & 2.97 & \bf{811.07} & 
0.58 & 0.75\\CON8-4 & 781.56 & 2.90 & 
790.07 & 2.86 & \bf{772.25} & 
1.21 & 2.31\\CON8-5 & 758.84 & 2.94 & 
762.87 & 2.91 & \bf{754.88} & 
0.52 & 1.06\\CON8-6 & 683.83 & 3.29 & 
689.90 & 3.35 & \bf{678.92} & 
0.72 & 1.62\\CON8-7 & 814.86 & 2.71 & 
816.77 & 2.67 & \bf{811.96} & 
0.36 & 0.59\\CON8-8 & 779.43 & 3.17 & 
786.30 & 3.34 & \bf{767.53} & 
1.55 & 2.45\\CON8-9 & 812.60 & 3.71 & 
814.81 & 3.48 & \bf{809.00} & 
0.44 & 0.72\\\bf{PROM.} & 
\bf{762.41} & \bf{2.99} & \bf{765.04} & \bf{2.95} & \bf{758.54} & \bf{0.46} & \bf{0.79}\\[1ex]\hline
\end{tabular}
\label{table:nonlin}
\end{table}
\begin{table}[h]
\caption{Resultados de la ejecución de la metaheurística AS-M, utilizando instancias de Dethloff con la configuración -n 5.0 -alpha 1.0 -beta 3.0 -q 0.1 -ro 0.015}
\centering
\small
\begin{tabular}{c c c c c c c c}
\hline\hline
Instancia & Costo mínimo & Tiempo(seg.) & Costo promedio & Tiempo promedio(seg.) & CME & \%G & \%GP \\ [0.5ex]
\hline
SCA3-0 & 636.06 & 3.38 & 
636.13 & 3.51 & \bf{635.62} & 
0.07 & 0.08\\SCA3-1 & \bf{697.84} & 3.82 & 
697.84 & 3.92 & 697.84 & 0.00
 & 0.00\\
SCA3-2 & \bf{659.34} & 3.50 & 
659.79 & 3.52 & 659.34 & 0.00
 & 0.07\\SCA3-3 & \bf{680.04} & 3.26 & 
680.04 & 3.59 & 680.04 & 0.00
 & 0.00\\
SCA3-4 & \bf{690.50} & 3.64 & 
690.50 & 3.73 & 690.50 & 0.00
 & 0.00\\
SCA3-5 & \bf{659.90} & 3.54 & 
664.73 & 3.66 & 659.90 & 0.00
 & 0.73\\SCA3-6 & \bf{651.09} & 4.07 & 
652.01 & 3.73 & 651.09 & 0.00
 & 0.14\\SCA3-7 & 664.88 & 3.41 & 
666.11 & 3.41 & \bf{659.17} & 
0.87 & 1.05\\SCA3-8 & \bf{719.47} & 3.74 & 
719.62 & 3.80 & 719.47 & 0.00
 & 0.02\\SCA3-9 & \bf{681.00} & 3.31 & 
681.59 & 3.23 & 681.00 & 0.00
 & 0.09\\SCA8-0 & 968.79 & 4.09 & 
974.18 & 3.86 & \bf{961.50} & 
0.76 & 1.32\\SCA8-1 & 1053.44 & 3.17 & 
1058.36 & 3.36 & \bf{1049.65} & 
0.36 & 0.83\\SCA8-2 & 1046.29 & 3.25 & 
1049.30 & 3.41 & \bf{1039.64} & 
0.64 & 0.93\\SCA8-3 & 997.48 & 3.77 & 
1005.78 & 3.62 & \bf{983.34} & 
1.44 & 2.28\\SCA8-4 & 1067.55 & 3.86 & 
1069.46 & 3.78 & \bf{1065.49} & 
0.19 & 0.37\\SCA8-5 & 1034.74 & 4.23 & 
1041.65 & 4.76 & \bf{1027.08} & 
0.75 & 1.42\\SCA8-6 & 980.91 & 4.20 & 
980.91 & 4.98 & \bf{971.82} & 
0.94 & 0.94\\SCA8-7 & 1066.65 & 4.01 & 
1067.99 & 4.11 & \bf{1051.28} & 
1.46 & 1.59\\SCA8-8 & \bf{1071.18} & 4.00 & 
1074.49 & 4.12 & 1071.18 & 0.00
 & 0.31\\SCA8-9 & 1067.42 & 3.38 & 
1067.42 & 3.41 & \bf{1060.50} & 
0.65 & 0.65\\CON3-0 & 617.59 & 4.31 & 
619.17 & 4.03 & \bf{616.52} & 
0.17 & 0.43\\CON3-1 & \bf{554.47} & 3.96 & 
555.15 & 3.85 & 554.47 & 0.00
 & 0.12\\CON3-2 & 519.11 & 3.64 & 
520.25 & 3.69 & \bf{518.00} & 
0.21 & 0.43\\CON3-3 & \bf{591.19} & 4.28 & 
591.24 & 4.01 & 591.19 & 0.00
 & 0.01\\CON3-4 & \bf{588.79} & 3.16 & 
589.58 & 3.36 & 588.79 & 0.00
 & 0.13\\CON3-5 & 564.88 & 3.68 & 
565.64 & 3.75 & \bf{563.70} & 
0.21 & 0.35\\CON3-6 & 500.80 & 4.02 & 
501.49 & 4.32 & \bf{499.05} & 
0.35 & 0.49\\CON3-7 & 577.54 & 4.48 & 
578.14 & 3.64 & \bf{576.48} & 
0.18 & 0.29\\CON3-8 & \bf{523.05} & 3.72 & 
523.10 & 3.43 & 523.05 & 0.00
 & 0.01\\CON3-9 & 584.05 & 3.21 & 
586.96 & 3.48 & \bf{578.24} & 
1.00 & 1.51\\CON8-0 & 861.35 & 3.91 & 
876.04 & 4.00 & \bf{857.17} & 
0.49 & 2.20\\CON8-1 & 740.93 & 3.87 & 
741.95 & 3.92 & \bf{740.85} & 
0.01 & 0.15\\CON8-2 & 713.21 & 4.58 & 
714.34 & 4.54 & \bf{712.89} & 
0.04 & 0.20\\CON8-3 & \bf{811.07} & 3.86 & 
814.00 & 3.96 & 811.07 & 0.00
 & 0.36\\CON8-4 & 776.34 & 3.61 & 
783.93 & 3.70 & \bf{772.25} & 
0.53 & 1.51\\CON8-5 & 758.12 & 3.95 & 
759.02 & 3.90 & \bf{754.88} & 
0.43 & 0.55\\CON8-6 & 693.10 & 4.49 & 
696.34 & 4.38 & \bf{678.92} & 
2.09 & 2.57\\CON8-7 & 814.79 & 3.55 & 
815.13 & 3.34 & \bf{811.96} & 
0.35 & 0.39\\CON8-8 & 778.34 & 4.78 & 
782.71 & 4.63 & \bf{767.53} & 
1.41 & 1.98\\CON8-9 & 812.60 & 4.06 & 
815.19 & 4.17 & \bf{809.00} & 
0.44 & 0.76\\\bf{PROM.} & 
\bf{761.90} & \bf{3.82} & \bf{764.18} & \bf{3.84} & \bf{758.54} & \bf{0.40} & \bf{0.68}\\[1ex]\hline
\end{tabular}
\label{table:nonlin}
\end{table}
\begin{table}[h]
\caption{Resultados de la ejecución de la metaheurística AS-M, utilizando instancias de Dethloff con la configuración -n 5.0 -alpha 1.0 -beta 3.0 -q .3 -ro 0.015}
\centering
\small
\begin{tabular}{c c c c c c c c}
\hline\hline
Instancia & Costo mínimo & Tiempo(seg.) & Costo promedio & Tiempo promedio(seg.) & CME & \%G & \%GP \\ [0.5ex]
\hline
SCA3-0 & 636.06 & 3.52 & 
636.13 & 3.42 & \bf{635.62} & 
0.07 & 0.08\\SCA3-1 & \bf{697.84} & 3.85 & 
697.84 & 3.85 & 697.84 & 0.00
 & 0.00\\
SCA3-2 & \bf{659.34} & 3.57 & 
661.77 & 3.54 & 659.34 & 0.00
 & 0.37\\SCA3-3 & \bf{680.04} & 3.42 & 
680.04 & 3.50 & 680.04 & 0.00
 & 0.00\\
SCA3-4 & \bf{690.50} & 3.58 & 
690.50 & 3.78 & 690.50 & 0.00
 & 0.00\\
SCA3-5 & \bf{659.90} & 3.67 & 
662.76 & 3.84 & 659.90 & 0.00
 & 0.43\\SCA3-6 & \bf{651.09} & 3.53 & 
652.48 & 3.62 & 651.09 & 0.00
 & 0.21\\SCA3-7 & 666.15 & 3.14 & 
667.09 & 3.05 & \bf{659.17} & 
1.06 & 1.20\\SCA3-8 & \bf{719.47} & 3.80 & 
720.61 & 3.69 & 719.47 & 0.00
 & 0.16\\SCA3-9 & \bf{681.00} & 2.93 & 
681.00 & 3.04 & 681.00 & 0.00
 & 0.00\\
SCA8-0 & \bf{961.50} & 3.75 & 
969.59 & 3.94 & 961.50 & 0.00
 & 0.84\\SCA8-1 & 1056.70 & 3.34 & 
1061.04 & 3.27 & \bf{1049.65} & 
0.67 & 1.09\\SCA8-2 & 1046.29 & 3.11 & 
1049.35 & 3.06 & \bf{1039.64} & 
0.64 & 0.93\\SCA8-3 & 1015.17 & 4.01 & 
1016.97 & 3.74 & \bf{983.34} & 
3.24 & 3.42\\SCA8-4 & 1067.28 & 3.78 & 
1068.05 & 3.76 & \bf{1065.49} & 
0.17 & 0.24\\SCA8-5 & 1045.30 & 4.25 & 
1052.74 & 4.37 & \bf{1027.08} & 
1.77 & 2.50\\SCA8-6 & 977.87 & 4.15 & 
979.87 & 4.00 & \bf{971.82} & 
0.62 & 0.83\\SCA8-7 & 1065.61 & 4.10 & 
1071.63 & 4.20 & \bf{1051.28} & 
1.36 & 1.94\\SCA8-8 & \bf{1071.18} & 4.10 & 
1073.91 & 4.03 & 1071.18 & 0.00
 & 0.25\\SCA8-9 & 1067.42 & 3.34 & 
1067.42 & 3.29 & \bf{1060.50} & 
0.65 & 0.65\\CON3-0 & \bf{616.52} & 3.78 & 
619.69 & 3.88 & 616.52 & 0.00
 & 0.51\\CON3-1 & \bf{554.47} & 3.93 & 
556.76 & 3.90 & 554.47 & 0.00
 & 0.41\\CON3-2 & 519.11 & 3.66 & 
520.88 & 3.49 & \bf{518.00} & 
0.21 & 0.56\\CON3-3 & \bf{591.19} & 3.98 & 
591.19 & 3.99 & 591.19 & 0.00
 & 0.00\\
CON3-4 & \bf{588.79} & 3.47 & 
589.05 & 3.46 & 588.79 & 0.00
 & 0.05\\CON3-5 & \bf{563.70} & 3.62 & 
565.54 & 3.56 & 563.70 & 0.00
 & 0.33\\CON3-6 & 500.80 & 4.39 & 
502.39 & 4.33 & \bf{499.05} & 
0.35 & 0.67\\CON3-7 & 578.22 & 3.42 & 
578.32 & 3.28 & \bf{576.48} & 
0.30 & 0.32\\CON3-8 & \bf{523.05} & 3.31 & 
523.44 & 3.36 & 523.05 & 0.00
 & 0.07\\CON3-9 & 578.25 & 3.72 & 
581.98 & 3.58 & \bf{578.24} & 
0.00 & 0.65\\CON8-0 & 866.43 & 3.87 & 
870.64 & 3.77 & \bf{857.17} & 
1.08 & 1.57\\CON8-1 & \bf{740.85} & 3.83 & 
743.63 & 3.86 & 740.85 & 0.00
 & 0.38\\CON8-2 & 713.44 & 4.60 & 
714.92 & 4.50 & \bf{712.89} & 
0.08 & 0.28\\CON8-3 & \bf{811.07} & 3.55 & 
814.69 & 3.75 & 811.07 & 0.00
 & 0.45\\CON8-4 & \bf{772.25} & 3.45 & 
781.84 & 3.59 & 772.25 & 0.00
 & 1.24\\CON8-5 & 758.99 & 3.64 & 
762.01 & 3.65 & \bf{754.88} & 
0.54 & 0.95\\CON8-6 & 689.23 & 4.31 & 
693.69 & 4.37 & \bf{678.92} & 
1.52 & 2.18\\CON8-7 & 814.79 & 3.74 & 
816.43 & 3.39 & \bf{811.96} & 
0.35 & 0.55\\CON8-8 & 778.22 & 4.16 & 
784.75 & 4.18 & \bf{767.53} & 
1.39 & 2.24\\CON8-9 & 812.60 & 4.07 & 
815.73 & 4.12 & \bf{809.00} & 
0.44 & 0.83\\\bf{PROM.} & 
\bf{762.19} & \bf{3.74} & \bf{764.71} & \bf{3.72} & \bf{758.54} & \bf{0.41} & \bf{0.73}\\[1ex]\hline
\end{tabular}
\label{table:nonlin}
\end{table}
\begin{table}[h]
\caption{Resultados de la ejecución de la metaheurística AS-M, utilizando instancias de Dethloff con la configuración -n 5.0 -alpha 1.0 -beta 3.0 -q .4 -ro 0.015}
\centering
\small
\begin{tabular}{c c c c c c c c}
\hline\hline
Instancia & Costo mínimo & Tiempo(seg.) & Costo promedio & Tiempo promedio(seg.) & CME & \%G & \%GP \\ [0.5ex]
\hline
SCA3-0 & 636.06 & 3.47 & 
636.06 & 3.48 & \bf{635.62} & 
0.07 & 0.07\\SCA3-1 & \bf{697.84} & 3.72 & 
697.84 & 3.73 & 697.84 & 0.00
 & 0.00\\
SCA3-2 & \bf{659.34} & 3.58 & 
661.45 & 3.46 & 659.34 & 0.00
 & 0.32\\SCA3-3 & \bf{680.04} & 3.61 & 
680.04 & 3.60 & 680.04 & 0.00
 & 0.00\\
SCA3-4 & \bf{690.50} & 3.94 & 
690.50 & 3.79 & 690.50 & 0.00
 & 0.00\\
SCA3-5 & \bf{659.90} & 3.83 & 
660.61 & 3.67 & 659.90 & 0.00
 & 0.11\\SCA3-6 & \bf{651.09} & 3.23 & 
652.48 & 3.41 & 651.09 & 0.00
 & 0.21\\SCA3-7 & 664.88 & 3.06 & 
665.83 & 3.08 & \bf{659.17} & 
0.87 & 1.01\\SCA3-8 & \bf{719.47} & 3.46 & 
721.21 & 3.52 & 719.47 & 0.00
 & 0.24\\SCA3-9 & \bf{681.00} & 2.94 & 
681.17 & 2.92 & 681.00 & 0.00
 & 0.02\\SCA8-0 & 968.79 & 3.76 & 
984.28 & 3.95 & \bf{961.50} & 
0.76 & 2.37\\SCA8-1 & 1054.87 & 2.98 & 
1062.53 & 3.14 & \bf{1049.65} & 
0.50 & 1.23\\SCA8-2 & 1046.29 & 2.90 & 
1049.35 & 2.88 & \bf{1039.64} & 
0.64 & 0.93\\SCA8-3 & \bf{983.34} & 3.87 & 
1002.51 & 3.74 & 983.34 & 0.00
 & 1.95\\SCA8-4 & 1067.66 & 3.42 & 
1068.73 & 3.62 & \bf{1065.49} & 
0.20 & 0.30\\SCA8-5 & 1034.74 & 4.32 & 
1047.76 & 4.26 & \bf{1027.08} & 
0.75 & 2.01\\SCA8-6 & 977.03 & 4.08 & 
980.36 & 4.08 & \bf{971.82} & 
0.54 & 0.88\\SCA8-7 & 1067.03 & 3.94 & 
1069.62 & 3.94 & \bf{1051.28} & 
1.50 & 1.74\\SCA8-8 & \bf{1071.18} & 3.91 & 
1071.18 & 4.00 & 1071.18 & 0.00
 & 0.00\\
SCA8-9 & 1067.42 & 3.74 & 
1067.42 & 3.16 & \bf{1060.50} & 
0.65 & 0.65\\CON3-0 & 617.59 & 4.10 & 
621.09 & 3.95 & \bf{616.52} & 
0.17 & 0.74\\CON3-1 & 556.04 & 3.73 & 
557.39 & 3.71 & \bf{554.47} & 
0.28 & 0.53\\CON3-2 & 519.11 & 3.57 & 
520.90 & 3.61 & \bf{518.00} & 
0.21 & 0.56\\CON3-3 & \bf{591.19} & 4.10 & 
591.19 & 3.92 & 591.19 & 0.00
 & 0.00\\CON3-4 & \bf{588.79} & 3.37 & 
590.24 & 3.26 & 588.79 & 0.00
 & 0.25\\CON3-5 & 564.88 & 3.67 & 
566.79 & 3.63 & \bf{563.70} & 
0.21 & 0.55\\CON3-6 & 500.80 & 4.23 & 
502.13 & 4.21 & \bf{499.05} & 
0.35 & 0.62\\CON3-7 & 577.68 & 3.24 & 
578.89 & 3.31 & \bf{576.48} & 
0.21 & 0.42\\CON3-8 & \bf{523.05} & 3.28 & 
524.06 & 3.36 & 523.05 & 0.00
 & 0.19\\CON3-9 & 578.98 & 3.67 & 
584.51 & 3.64 & \bf{578.24} & 
0.13 & 1.08\\CON8-0 & 870.90 & 3.94 & 
872.29 & 3.94 & \bf{857.17} & 
1.60 & 1.76\\CON8-1 & \bf{740.85} & 3.78 & 
742.69 & 3.79 & 740.85 & 0.00
 & 0.25\\CON8-2 & 713.44 & 4.48 & 
714.19 & 4.52 & \bf{712.89} & 
0.08 & 0.18\\CON8-3 & \bf{811.07} & 3.71 & 
814.57 & 3.73 & 811.07 & 0.00
 & 0.43\\CON8-4 & 776.37 & 3.64 & 
780.99 & 3.64 & \bf{772.25} & 
0.53 & 1.13\\CON8-5 & 754.95 & 3.42 & 
759.46 & 3.54 & \bf{754.88} & 
0.01 & 0.61\\CON8-6 & 688.00 & 4.06 & 
692.85 & 4.04 & \bf{678.92} & 
1.34 & 2.05\\CON8-7 & 814.77 & 3.31 & 
816.39 & 3.33 & \bf{811.96} & 
0.35 & 0.55\\CON8-8 & 784.28 & 4.29 & 
787.47 & 4.24 & \bf{767.53} & 
2.18 & 2.60\\CON8-9 & 810.18 & 4.10 & 
812.71 & 4.03 & \bf{809.00} & 
0.15 & 0.46\\\bf{PROM.} & 
\bf{761.53} & \bf{3.69} & \bf{764.54} & \bf{3.67} & \bf{758.54} & \bf{0.36} & \bf{0.73}\\[1ex]\hline
\end{tabular}
\label{table:nonlin}
\end{table}
\begin{table}[h]
\caption{Resultados de la ejecución de la metaheurística AS-M, utilizando instancias de Dethloff con la configuración -n 6.0 -alpha 1.0 -beta 3.0 -q 0.1 -ro 0.015}
\centering
\small
\begin{tabular}{c c c c c c c c}
\hline\hline
Instancia & Costo mínimo & Tiempo(seg.) & Costo promedio & Tiempo promedio(seg.) & CME & \%G & \%GP \\ [0.5ex]
\hline
SCA3-0 & 636.06 & 4.31 & 
636.06 & 4.40 & \bf{635.62} & 
0.07 & 0.07\\SCA3-1 & \bf{697.84} & 4.94 & 
697.84 & 4.83 & 697.84 & 0.00
 & 0.00\\
SCA3-2 & \bf{659.34} & 3.93 & 
660.68 & 4.21 & 659.34 & 0.00
 & 0.20\\SCA3-3 & \bf{680.04} & 4.29 & 
680.04 & 4.28 & 680.04 & 0.00
 & 0.00\\
SCA3-4 & \bf{690.50} & 4.53 & 
690.50 & 4.62 & 690.50 & 0.00
 & 0.00\\
SCA3-5 & 662.75 & 4.70 & 
663.32 & 4.51 & \bf{659.90} & 
0.43 & 0.52\\SCA3-6 & \bf{651.09} & 4.46 & 
652.70 & 4.35 & 651.09 & 0.00
 & 0.25\\SCA3-7 & 666.15 & 3.86 & 
666.15 & 3.90 & \bf{659.17} & 
1.06 & 1.06\\SCA3-8 & \bf{719.47} & 4.31 & 
720.67 & 4.24 & 719.47 & 0.00
 & 0.17\\SCA3-9 & \bf{681.00} & 3.94 & 
682.40 & 3.77 & 681.00 & 0.00
 & 0.21\\SCA8-0 & \bf{961.50} & 4.64 & 
971.77 & 4.64 & 961.50 & 0.00
 & 1.07\\SCA8-1 & 1053.09 & 3.91 & 
1060.39 & 4.07 & \bf{1049.65} & 
0.33 & 1.02\\SCA8-2 & 1050.37 & 3.63 & 
1050.68 & 3.58 & \bf{1039.64} & 
1.03 & 1.06\\SCA8-3 & \bf{983.34} & 4.44 & 
995.78 & 4.33 & 983.34 & 0.00
 & 1.26\\SCA8-4 & \bf{1065.49} & 4.44 & 
1066.45 & 4.43 & 1065.49 & 0.00
 & 0.09\\SCA8-5 & 1050.46 & 5.13 & 
1054.02 & 5.21 & \bf{1027.08} & 
2.28 & 2.62\\SCA8-6 & 977.03 & 4.83 & 
979.87 & 4.80 & \bf{971.82} & 
0.54 & 0.83\\SCA8-7 & 1067.03 & 4.85 & 
1069.09 & 4.92 & \bf{1051.28} & 
1.50 & 1.69\\SCA8-8 & \bf{1071.18} & 5.06 & 
1076.50 & 4.98 & 1071.18 & 0.00
 & 0.50\\SCA8-9 & 1067.42 & 4.07 & 
1067.42 & 4.15 & \bf{1060.50} & 
0.65 & 0.65\\CON3-0 & 620.29 & 5.10 & 
623.04 & 4.83 & \bf{616.52} & 
0.61 & 1.06\\CON3-1 & \bf{554.47} & 4.43 & 
555.55 & 4.58 & 554.47 & 0.00
 & 0.19\\CON3-2 & 521.36 & 4.25 & 
521.38 & 4.38 & \bf{518.00} & 
0.65 & 0.65\\CON3-3 & \bf{591.19} & 4.66 & 
591.20 & 4.84 & 591.19 & 0.00
 & 0.00\\CON3-4 & \bf{588.79} & 4.10 & 
588.92 & 4.04 & 588.79 & 0.00
 & 0.02\\CON3-5 & \bf{563.70} & 4.24 & 
565.24 & 4.43 & 563.70 & 0.00
 & 0.27\\CON3-6 & 502.09 & 4.96 & 
502.14 & 5.09 & \bf{499.05} & 
0.61 & 0.62\\CON3-7 & \bf{576.48} & 3.97 & 
577.88 & 4.10 & 576.48 & 0.00
 & 0.24\\CON3-8 & \bf{523.05} & 4.42 & 
523.54 & 4.20 & 523.05 & 0.00
 & 0.09\\CON3-9 & 582.79 & 4.15 & 
587.04 & 4.19 & \bf{578.24} & 
0.79 & 1.52\\CON8-0 & 858.63 & 4.76 & 
869.89 & 4.69 & \bf{857.17} & 
0.17 & 1.48\\CON8-1 & \bf{740.85} & 4.49 & 
743.24 & 4.68 & 740.85 & 0.00
 & 0.32\\CON8-2 & 713.44 & 5.11 & 
714.94 & 5.54 & \bf{712.89} & 
0.08 & 0.29\\CON8-3 & \bf{811.07} & 4.73 & 
812.96 & 4.68 & 811.07 & 0.00
 & 0.23\\CON8-4 & 780.48 & 4.34 & 
783.07 & 4.24 & \bf{772.25} & 
1.07 & 1.40\\CON8-5 & 760.03 & 4.65 & 
761.37 & 4.60 & \bf{754.88} & 
0.68 & 0.86\\CON8-6 & 685.06 & 5.33 & 
690.66 & 5.30 & \bf{678.92} & 
0.90 & 1.73\\CON8-7 & 814.79 & 4.19 & 
814.83 & 4.14 & \bf{811.96} & 
0.35 & 0.35\\CON8-8 & 779.76 & 5.30 & 
786.02 & 5.18 & \bf{767.53} & 
1.59 & 2.41\\CON8-9 & 812.03 & 5.07 & 
812.46 & 5.30 & \bf{809.00} & 
0.37 & 0.43\\\bf{PROM.} & 
\bf{761.79} & \bf{4.51} & \bf{764.19} & \bf{4.53} & \bf{758.54} & \bf{0.39} & \bf{0.69}\\[1ex]\hline
\end{tabular}
\label{table:nonlin}
\end{table}
\begin{table}[h]
\caption{Resultados de la ejecución de la metaheurística AS-M, utilizando instancias de Dethloff con la configuración -n 6.0 -alpha 1.0 -beta 3.0 -q .3 -ro 0.015}
\centering
\small
\begin{tabular}{c c c c c c c c}
\hline\hline
Instancia & Costo mínimo & Tiempo(seg.) & Costo promedio & Tiempo promedio(seg.) & CME & \%G & \%GP \\ [0.5ex]
\hline
SCA3-0 & 636.06 & 4.46 & 
636.06 & 4.29 & \bf{635.62} & 
0.07 & 0.07\\SCA3-1 & \bf{697.84} & 4.64 & 
697.84 & 4.50 & 697.84 & 0.00
 & 0.00\\
SCA3-2 & \bf{659.34} & 4.11 & 
659.79 & 4.14 & 659.34 & 0.00
 & 0.07\\SCA3-3 & \bf{680.04} & 4.15 & 
680.04 & 4.05 & 680.04 & 0.00
 & 0.00\\
SCA3-4 & \bf{690.50} & 4.36 & 
690.50 & 4.51 & 690.50 & 0.00
 & 0.00\\
SCA3-5 & 662.75 & 4.31 & 
664.20 & 4.44 & \bf{659.90} & 
0.43 & 0.65\\SCA3-6 & \bf{651.09} & 4.37 & 
652.48 & 4.28 & 651.09 & 0.00
 & 0.21\\SCA3-7 & 666.15 & 4.03 & 
666.15 & 4.01 & \bf{659.17} & 
1.06 & 1.06\\SCA3-8 & \bf{719.47} & 4.25 & 
721.64 & 4.27 & 719.47 & 0.00
 & 0.30\\SCA3-9 & \bf{681.00} & 3.43 & 
681.00 & 3.50 & 681.00 & 0.00
 & 0.00\\
SCA8-0 & 968.79 & 4.59 & 
976.77 & 4.65 & \bf{961.50} & 
0.76 & 1.59\\SCA8-1 & 1054.87 & 3.68 & 
1057.44 & 3.75 & \bf{1049.65} & 
0.50 & 0.74\\SCA8-2 & 1050.17 & 3.66 & 
1050.74 & 3.46 & \bf{1039.64} & 
1.01 & 1.07\\SCA8-3 & 995.50 & 4.32 & 
1005.90 & 4.30 & \bf{983.34} & 
1.24 & 2.29\\SCA8-4 & \bf{1065.49} & 4.68 & 
1065.94 & 4.54 & 1065.49 & 0.00
 & 0.04\\SCA8-5 & 1034.74 & 4.70 & 
1046.68 & 4.95 & \bf{1027.08} & 
0.75 & 1.91\\SCA8-6 & 972.48 & 5.19 & 
978.93 & 4.96 & \bf{971.82} & 
0.07 & 0.73\\SCA8-7 & 1067.03 & 4.88 & 
1069.12 & 4.97 & \bf{1051.28} & 
1.50 & 1.70\\SCA8-8 & \bf{1071.18} & 4.58 & 
1074.87 & 4.80 & 1071.18 & 0.00
 & 0.34\\SCA8-9 & 1065.60 & 4.06 & 
1066.97 & 3.86 & \bf{1060.50} & 
0.48 & 0.61\\CON3-0 & 617.59 & 4.61 & 
620.93 & 4.73 & \bf{616.52} & 
0.17 & 0.72\\CON3-1 & \bf{554.47} & 4.83 & 
556.34 & 4.74 & 554.47 & 0.00
 & 0.34\\CON3-2 & 519.11 & 4.44 & 
519.80 & 4.49 & \bf{518.00} & 
0.21 & 0.35\\CON3-3 & \bf{591.19} & 4.50 & 
591.50 & 4.64 & 591.19 & 0.00
 & 0.05\\CON3-4 & \bf{588.79} & 4.22 & 
589.72 & 4.08 & 588.79 & 0.00
 & 0.16\\CON3-5 & \bf{563.70} & 4.42 & 
564.59 & 4.46 & 563.70 & 0.00
 & 0.16\\CON3-6 & 502.09 & 4.99 & 
503.14 & 5.03 & \bf{499.05} & 
0.61 & 0.82\\CON3-7 & 578.22 & 3.64 & 
578.36 & 3.81 & \bf{576.48} & 
0.30 & 0.33\\CON3-8 & \bf{523.05} & 4.14 & 
523.36 & 4.21 & 523.05 & 0.00
 & 0.06\\CON3-9 & 583.32 & 4.14 & 
587.32 & 4.21 & \bf{578.24} & 
0.88 & 1.57\\CON8-0 & 866.22 & 4.51 & 
869.91 & 4.53 & \bf{857.17} & 
1.06 & 1.49\\CON8-1 & \bf{740.85} & 4.94 & 
742.95 & 4.95 & 740.85 & 0.00
 & 0.28\\CON8-2 & 713.44 & 5.39 & 
713.86 & 5.38 & \bf{712.89} & 
0.08 & 0.14\\CON8-3 & \bf{811.07} & 4.39 & 
814.40 & 4.45 & 811.07 & 0.00
 & 0.41\\CON8-4 & 784.87 & 4.34 & 
787.99 & 4.45 & \bf{772.25} & 
1.63 & 2.04\\CON8-5 & 759.93 & 5.11 & 
760.97 & 4.84 & \bf{754.88} & 
0.67 & 0.81\\CON8-6 & 692.75 & 5.04 & 
693.80 & 5.09 & \bf{678.92} & 
2.04 & 2.19\\CON8-7 & 814.79 & 4.02 & 
818.25 & 4.03 & \bf{811.96} & 
0.35 & 0.78\\CON8-8 & 777.60 & 5.72 & 
786.37 & 5.29 & \bf{767.53} & 
1.31 & 2.45\\CON8-9 & 810.18 & 5.19 & 
812.96 & 5.25 & \bf{809.00} & 
0.15 & 0.49\\\bf{PROM.} & 
\bf{762.08} & \bf{4.48} & \bf{764.49} & \bf{4.47} & \bf{758.54} & \bf{0.43} & \bf{0.73}\\[1ex]\hline
\end{tabular}
\label{table:nonlin}
\end{table}
\begin{table}[h]
\caption{Resultados de la ejecución de la metaheurística AS-M, utilizando instancias de Dethloff con la configuración -n 6.0 -alpha 1.0 -beta 3.0 -q .4 -ro 0.015}
\centering
\small
\begin{tabular}{c c c c c c c c}
\hline\hline
Instancia & Costo mínimo & Tiempo(seg.) & Costo promedio & Tiempo promedio(seg.) & CME & \%G & \%GP \\ [0.5ex]
\hline
SCA3-0 & 636.06 & 4.20 & 
636.06 & 4.09 & \bf{635.62} & 
0.07 & 0.07\\SCA3-1 & \bf{697.84} & 4.35 & 
697.84 & 4.48 & 697.84 & 0.00
 & 0.00\\
SCA3-2 & \bf{659.34} & 5.08 & 
661.45 & 4.36 & 659.34 & 0.00
 & 0.32\\SCA3-3 & \bf{680.04} & 4.13 & 
680.04 & 4.22 & 680.04 & 0.00
 & 0.00\\
SCA3-4 & \bf{690.50} & 4.36 & 
690.50 & 4.42 & 690.50 & 0.00
 & 0.00\\
SCA3-5 & \bf{659.90} & 5.17 & 
662.34 & 4.50 & 659.90 & 0.00
 & 0.37\\SCA3-6 & 652.94 & 3.91 & 
653.19 & 4.63 & \bf{651.09} & 
0.28 & 0.32\\SCA3-7 & 666.15 & 3.58 & 
666.42 & 3.73 & \bf{659.17} & 
1.06 & 1.10\\SCA3-8 & \bf{719.47} & 4.20 & 
719.47 & 4.14 & 719.47 & 0.00
 & 0.00\\
SCA3-9 & \bf{681.00} & 3.48 & 
681.00 & 3.60 & 681.00 & 0.00
 & 0.00\\
SCA8-0 & 971.49 & 4.68 & 
981.11 & 4.66 & \bf{961.50} & 
1.04 & 2.04\\SCA8-1 & 1054.87 & 3.90 & 
1062.60 & 3.70 & \bf{1049.65} & 
0.50 & 1.23\\SCA8-2 & 1045.64 & 3.29 & 
1049.19 & 3.39 & \bf{1039.64} & 
0.58 & 0.92\\SCA8-3 & 991.84 & 4.74 & 
1000.33 & 4.52 & \bf{983.34} & 
0.86 & 1.73\\SCA8-4 & \bf{1065.49} & 4.57 & 
1068.28 & 4.73 & 1065.49 & 0.00
 & 0.26\\SCA8-5 & 1034.74 & 5.04 & 
1047.00 & 4.95 & \bf{1027.08} & 
0.75 & 1.94\\SCA8-6 & 972.48 & 4.54 & 
976.87 & 5.70 & \bf{971.82} & 
0.07 & 0.52\\SCA8-7 & 1067.03 & 5.14 & 
1070.41 & 4.82 & \bf{1051.28} & 
1.50 & 1.82\\SCA8-8 & \bf{1071.18} & 4.76 & 
1073.91 & 4.82 & 1071.18 & 0.00
 & 0.25\\SCA8-9 & 1065.60 & 3.85 & 
1066.97 & 3.89 & \bf{1060.50} & 
0.48 & 0.61\\CON3-0 & \bf{616.52} & 4.88 & 
621.37 & 4.85 & 616.52 & 0.00
 & 0.79\\CON3-1 & \bf{554.47} & 4.50 & 
555.39 & 4.66 & 554.47 & 0.00
 & 0.17\\CON3-2 & 521.38 & 4.55 & 
521.38 & 4.52 & \bf{518.00} & 
0.65 & 0.65\\CON3-3 & \bf{591.19} & 4.86 & 
591.20 & 4.69 & 591.19 & 0.00
 & 0.00\\CON3-4 & \bf{588.79} & 3.92 & 
588.79 & 3.75 & 588.79 & 0.00
 & 0.00\\
CON3-5 & \bf{563.70} & 4.18 & 
564.59 & 4.40 & 563.70 & 0.00
 & 0.16\\CON3-6 & 502.16 & 5.12 & 
503.04 & 5.05 & \bf{499.05} & 
0.62 & 0.80\\CON3-7 & 578.41 & 4.26 & 
579.15 & 4.00 & \bf{576.48} & 
0.33 & 0.46\\CON3-8 & 523.14 & 3.88 & 
523.41 & 4.09 & \bf{523.05} & 
0.02 & 0.07\\CON3-9 & 578.25 & 4.16 & 
582.77 & 4.07 & \bf{578.24} & 
0.00 & 0.78\\CON8-0 & 865.86 & 4.61 & 
868.67 & 4.46 & \bf{857.17} & 
1.01 & 1.34\\CON8-1 & 740.93 & 4.61 & 
741.99 & 4.55 & \bf{740.85} & 
0.01 & 0.15\\CON8-2 & 713.84 & 5.55 & 
715.41 & 5.72 & \bf{712.89} & 
0.13 & 0.35\\CON8-3 & 815.14 & 4.30 & 
816.79 & 4.48 & \bf{811.07} & 
0.50 & 0.70\\CON8-4 & 772.32 & 4.64 & 
779.12 & 4.33 & \bf{772.25} & 
0.01 & 0.89\\CON8-5 & 758.84 & 4.30 & 
761.73 & 4.50 & \bf{754.88} & 
0.52 & 0.91\\CON8-6 & 695.66 & 5.07 & 
696.91 & 5.12 & \bf{678.92} & 
2.47 & 2.65\\CON8-7 & 814.79 & 4.01 & 
814.89 & 3.90 & \bf{811.96} & 
0.35 & 0.36\\CON8-8 & 782.86 & 4.84 & 
784.71 & 5.03 & \bf{767.53} & 
2.00 & 2.24\\CON8-9 & 810.18 & 5.19 & 
812.23 & 5.18 & \bf{809.00} & 
0.15 & 0.40\\\bf{PROM.} & 
\bf{761.80} & \bf{4.46} & \bf{764.21} & \bf{4.47} & \bf{758.54} & \bf{0.40} & \bf{0.68}\\[1ex]\hline
\end{tabular}
\label{table:nonlin}
\end{table}
\clearpage
\subsection{SalhiNagy}\label{tablas-entonacion-AS-M-salhinagy}
\begin{table}[h]
\caption{Resultados de la ejecución de la metaheurística AS-M, utilizando instancias de SalhiNagy con la configuración -n 3.0 -alpha 1.0 -beta 3.0 -q 0.1 -ro 0.015}
\centering
\small
\begin{tabular}{c c c c c c c c}
\hline\hline
Instancia & Costo mínimo & Tiempo(seg.) & Costo promedio & Tiempo promedio(seg.) & CME & \%G & \%GP \\ [0.5ex]
\hline
CMT1X & 476.70 & 1.84 & 
477.65 & 1.93 & \bf{470.48} & 
1.32 & 1.53\\CMT1Y & 477.00 & 1.80 & 
477.11 & 1.95 & \bf{470.48} & 
1.39 & 1.41\\CMT2X & 707.56 & 9.02 & 
708.16 & 8.98 & \bf{682.39} & 
3.69 & 3.78\\CMT2Y & 704.36 & 8.15 & 
705.82 & 8.52 & \bf{682.39} & 
3.22 & 3.43\\CMT3X & 731.61 & 28.72 & 
733.67 & 28.33 & \bf{719.06} & 
1.75 & 2.03\\CMT3Y & 728.04 & 29.10 & 
729.72 & 28.79 & \bf{719.06} & 
1.25 & 1.48\\CMT4X & 879.38 & 121.93 & 
884.29 & 122.60 & \bf{854.21} & 
2.95 & 3.52\\CMT4Y & 891.66 & 125.77 & 
893.05 & 122.78 & \bf{852.46} & 
4.60 & 4.76\\CMT5X & 1067.33 & 354.90 & 
1072.17 & 357.15 & \bf{1030.56} & 
3.57 & 4.04\\CMT5Y & 1095.90 & 379.71 & 
1095.99 & 370.67 & \bf{1031.69} & 
6.22 & 6.23\\CMT11X & 849.07 & 51.71 & 
863.19 & 51.98 & \bf{831.09} & 
2.16 & 3.86\\CMT11Y & 893.21 & 47.78 & 
894.27 & 46.76 & \bf{829.85} & 
7.64 & 7.76\\CMT12X & 674.82 & 23.84 & 
675.88 & 24.30 & \bf{658.83} & 
2.43 & 2.59\\CMT12Y & 680.30 & 24.77 & 
680.67 & 24.43 & \bf{660.47} & 
3.00 & 3.06\\\bf{PROM.} & 
\bf{775.50} & \bf{86.36} & \bf{777.97} & \bf{85.65} & \bf{749.50} & \bf{3.23} & \bf{3.53}\\[1ex]\hline
\end{tabular}
\label{table:nonlin}
\end{table}
\begin{table}[h]
\caption{Resultados de la ejecución de la metaheurística AS-M, utilizando instancias de SalhiNagy con la configuración -n 3.0 -alpha 1.0 -beta 3.0 -q .3 -ro 0.015}
\centering
\small
\begin{tabular}{c c c c c c c c}
\hline\hline
Instancia & Costo mínimo & Tiempo(seg.) & Costo promedio & Tiempo promedio(seg.) & CME & \%G & \%GP \\ [0.5ex]
\hline
CMT1X & 476.71 & 1.84 & 
476.92 & 1.90 & \bf{470.48} & 
1.32 & 1.37\\CMT1Y & 472.37 & 1.74 & 
473.54 & 1.81 & \bf{470.48} & 
0.40 & 0.65\\CMT2X & 696.28 & 8.97 & 
701.77 & 8.91 & \bf{682.39} & 
2.04 & 2.84\\CMT2Y & 707.08 & 8.98 & 
708.40 & 9.02 & \bf{682.39} & 
3.62 & 3.81\\CMT3X & 733.71 & 26.87 & 
734.40 & 27.36 & \bf{719.06} & 
2.04 & 2.13\\CMT3Y & 731.81 & 29.07 & 
733.43 & 28.59 & \bf{719.06} & 
1.77 & 2.00\\CMT4X & 879.06 & 118.74 & 
882.26 & 121.38 & \bf{854.21} & 
2.91 & 3.28\\CMT4Y & 886.46 & 121.27 & 
888.43 & 121.99 & \bf{852.46} & 
3.99 & 4.22\\CMT5X & 1097.03 & 371.51 & 
1099.79 & 360.80 & \bf{1030.56} & 
6.45 & 6.72\\CMT5Y & 1078.60 & 344.92 & 
1079.73 & 354.44 & \bf{1031.69} & 
4.55 & 4.66\\CMT11X & 859.99 & 54.65 & 
880.09 & 54.96 & \bf{831.09} & 
3.48 & 5.90\\CMT11Y & 896.28 & 46.71 & 
903.67 & 47.96 & \bf{829.85} & 
8.01 & 8.90\\CMT12X & 677.28 & 25.64 & 
677.32 & 23.54 & \bf{658.83} & 
2.80 & 2.81\\CMT12Y & 678.07 & 22.76 & 
678.63 & 22.86 & \bf{660.47} & 
2.66 & 2.75\\\bf{PROM.} & 
\bf{776.48} & \bf{84.55} & \bf{779.88} & \bf{84.68} & \bf{749.50} & \bf{3.29} & \bf{3.72}\\[1ex]\hline
\end{tabular}
\label{table:nonlin}
\end{table}
\begin{table}[h]
\caption{Resultados de la ejecución de la metaheurística AS-M, utilizando instancias de SalhiNagy con la configuración -n 3.0 -alpha 1.0 -beta 3.0 -q .4 -ro 0.015}
\centering
\small
\begin{tabular}{c c c c c c c c}
\hline\hline
Instancia & Costo mínimo & Tiempo(seg.) & Costo promedio & Tiempo promedio(seg.) & CME & \%G & \%GP \\ [0.5ex]
\hline
CMT1X & 477.20 & 1.85 & 
479.00 & 1.82 & \bf{470.48} & 
1.43 & 1.81\\CMT1Y & 471.25 & 2.02 & 
474.71 & 1.97 & \bf{470.48} & 
0.16 & 0.90\\CMT2X & 689.46 & 8.86 & 
695.48 & 8.94 & \bf{682.39} & 
1.04 & 1.92\\CMT2Y & 701.54 & 8.73 & 
704.24 & 8.89 & \bf{682.39} & 
2.81 & 3.20\\CMT3X & 736.81 & 27.36 & 
737.33 & 26.68 & \bf{719.06} & 
2.47 & 2.54\\CMT3Y & 728.81 & 28.37 & 
731.99 & 29.32 & \bf{719.06} & 
1.36 & 1.80\\CMT4X & 884.71 & 115.10 & 
889.33 & 115.58 & \bf{854.21} & 
3.57 & 4.11\\CMT4Y & 888.67 & 119.54 & 
892.54 & 119.23 & \bf{852.46} & 
4.25 & 4.70\\CMT5X & 1089.32 & 345.02 & 
1091.24 & 352.59 & \bf{1030.56} & 
5.70 & 5.89\\CMT5Y & 1082.96 & 355.85 & 
1089.63 & 358.55 & \bf{1031.69} & 
4.97 & 5.62\\CMT11X & 864.72 & 53.23 & 
871.75 & 52.06 & \bf{831.09} & 
4.05 & 4.89\\CMT11Y & 888.34 & 44.67 & 
891.84 & 47.52 & \bf{829.85} & 
7.05 & 7.47\\CMT12X & 677.81 & 22.46 & 
677.94 & 22.03 & \bf{658.83} & 
2.88 & 2.90\\CMT12Y & 674.03 & 20.52 & 
679.54 & 20.84 & \bf{660.47} & 
2.05 & 2.89\\\bf{PROM.} & 
\bf{775.40} & \bf{82.40} & \bf{779.04} & \bf{83.29} & \bf{749.50} & \bf{3.13} & \bf{3.62}\\[1ex]\hline
\end{tabular}
\label{table:nonlin}
\end{table}
\begin{table}[h]
\caption{Resultados de la ejecución de la metaheurística AS-M, utilizando instancias de SalhiNagy con la configuración -n 4.0 -alpha 1.0 -beta 3.0 -q 0.1 -ro 0.015}
\centering
\small
\begin{tabular}{c c c c c c c c}
\hline\hline
Instancia & Costo mínimo & Tiempo(seg.) & Costo promedio & Tiempo promedio(seg.) & CME & \%G & \%GP \\ [0.5ex]
\hline
CMT1X & 474.91 & 2.60 & 
478.94 & 2.52 & \bf{470.48} & 
0.94 & 1.80\\CMT1Y & 474.72 & 2.56 & 
476.78 & 2.60 & \bf{470.48} & 
0.90 & 1.34\\CMT2X & 698.66 & 11.86 & 
702.43 & 12.04 & \bf{682.39} & 
2.38 & 2.94\\CMT2Y & 698.20 & 11.69 & 
702.35 & 11.85 & \bf{682.39} & 
2.32 & 2.93\\CMT3X & 733.61 & 35.87 & 
734.20 & 35.70 & \bf{719.06} & 
2.02 & 2.11\\CMT3Y & 732.39 & 38.05 & 
733.18 & 39.04 & \bf{719.06} & 
1.85 & 1.96\\CMT4X & 876.09 & 158.14 & 
886.90 & 160.46 & \bf{854.21} & 
2.56 & 3.83\\CMT4Y & 893.40 & 162.86 & 
894.73 & 162.59 & \bf{852.46} & 
4.80 & 4.96\\CMT5X & 1076.36 & 463.69 & 
1081.05 & 494.47 & \bf{1030.56} & 
4.44 & 4.90\\CMT5Y & 1081.99 & 499.68 & 
1088.77 & 496.27 & \bf{1031.69} & 
4.88 & 5.53\\CMT11X & 842.10 & 70.23 & 
867.16 & 69.14 & \bf{831.09} & 
1.32 & 4.34\\CMT11Y & 862.98 & 68.71 & 
879.25 & 69.78 & \bf{829.85} & 
3.99 & 5.95\\CMT12X & 674.79 & 30.61 & 
675.29 & 30.77 & \bf{658.83} & 
2.42 & 2.50\\CMT12Y & 678.10 & 30.16 & 
679.05 & 30.00 & \bf{660.47} & 
2.67 & 2.81\\\bf{PROM.} & 
\bf{771.31} & \bf{113.34} & \bf{777.15} & \bf{115.52} & \bf{749.50} & \bf{2.68} & \bf{3.42}\\[1ex]\hline
\end{tabular}
\label{table:nonlin}
\end{table}
\begin{table}[h]
\caption{Resultados de la ejecución de la metaheurística AS-M, utilizando instancias de SalhiNagy con la configuración -n 4.0 -alpha 1.0 -beta 3.0 -q .3 -ro 0.015}
\centering
\small
\begin{tabular}{c c c c c c c c}
\hline\hline
Instancia & Costo mínimo & Tiempo(seg.) & Costo promedio & Tiempo promedio(seg.) & CME & \%G & \%GP \\ [0.5ex]
\hline
CMT1X & 472.37 & 2.52 & 
476.04 & 2.63 & \bf{470.48} & 
0.40 & 1.18\\CMT1Y & 474.41 & 2.51 & 
477.21 & 2.50 & \bf{470.48} & 
0.84 & 1.43\\CMT2X & 691.69 & 11.74 & 
698.67 & 12.03 & \bf{682.39} & 
1.36 & 2.39\\CMT2Y & 701.44 & 10.91 & 
701.90 & 11.38 & \bf{682.39} & 
2.79 & 2.86\\CMT3X & 729.13 & 35.54 & 
733.76 & 36.40 & \bf{719.06} & 
1.40 & 2.04\\CMT3Y & 728.91 & 42.67 & 
731.51 & 40.35 & \bf{719.06} & 
1.37 & 1.73\\CMT4X & 886.04 & 164.93 & 
886.47 & 161.44 & \bf{854.21} & 
3.73 & 3.78\\CMT4Y & 889.66 & 161.03 & 
889.74 & 170.28 & \bf{852.46} & 
4.36 & 4.37\\CMT5X & 1080.32 & 474.04 & 
1085.86 & 474.11 & \bf{1030.56} & 
4.83 & 5.37\\CMT5Y & 1078.95 & 475.12 & 
1084.78 & 479.30 & \bf{1031.69} & 
4.58 & 5.15\\CMT11X & 855.77 & 67.37 & 
870.47 & 70.45 & \bf{831.09} & 
2.97 & 4.74\\CMT11Y & 853.00 & 60.41 & 
854.34 & 59.88 & \bf{829.85} & 
2.79 & 2.95\\CMT12X & 675.00 & 30.70 & 
676.03 & 29.64 & \bf{658.83} & 
2.45 & 2.61\\CMT12Y & 672.60 & 27.45 & 
676.11 & 28.50 & \bf{660.47} & 
1.84 & 2.37\\\bf{PROM.} & 
\bf{770.66} & \bf{111.92} & \bf{774.49} & \bf{112.78} & \bf{749.50} & \bf{2.55} & \bf{3.07}\\[1ex]\hline
\end{tabular}
\label{table:nonlin}
\end{table}
\begin{table}[h]
\caption{Resultados de la ejecución de la metaheurística AS-M, utilizando instancias de SalhiNagy con la configuración -n 4.0 -alpha 1.0 -beta 3.0 -q .4 -ro 0.015}
\centering
\small
\begin{tabular}{c c c c c c c c}
\hline\hline
Instancia & Costo mínimo & Tiempo(seg.) & Costo promedio & Tiempo promedio(seg.) & CME & \%G & \%GP \\ [0.5ex]
\hline
CMT1X & 472.87 & 2.48 & 
474.89 & 2.52 & \bf{470.48} & 
0.51 & 0.94\\CMT1Y & 472.01 & 2.43 & 
474.69 & 2.51 & \bf{470.48} & 
0.33 & 0.90\\CMT2X & 706.66 & 11.95 & 
708.45 & 11.88 & \bf{682.39} & 
3.56 & 3.82\\CMT2Y & 695.12 & 11.26 & 
701.43 & 11.40 & \bf{682.39} & 
1.87 & 2.79\\CMT3X & 732.71 & 35.86 & 
734.28 & 35.48 & \bf{719.06} & 
1.90 & 2.12\\CMT3Y & 734.62 & 36.93 & 
734.69 & 37.60 & \bf{719.06} & 
2.16 & 2.17\\CMT4X & 882.04 & 154.90 & 
887.68 & 155.34 & \bf{854.21} & 
3.26 & 3.92\\CMT4Y & 880.37 & 155.88 & 
887.30 & 156.56 & \bf{852.46} & 
3.27 & 4.09\\CMT5X & 1090.07 & 496.77 & 
1090.93 & 482.64 & \bf{1030.56} & 
5.77 & 5.86\\CMT5Y & 1072.19 & 469.75 & 
1078.80 & 470.04 & \bf{1031.69} & 
3.93 & 4.57\\CMT11X & 859.99 & 76.33 & 
863.21 & 72.75 & \bf{831.09} & 
3.48 & 3.86\\CMT11Y & 853.57 & 67.32 & 
869.67 & 63.69 & \bf{829.85} & 
2.86 & 4.80\\CMT12X & 675.72 & 31.70 & 
676.74 & 31.72 & \bf{658.83} & 
2.56 & 2.72\\CMT12Y & 668.42 & 27.88 & 
673.22 & 28.41 & \bf{660.47} & 
1.20 & 1.93\\\bf{PROM.} & 
\bf{771.17} & \bf{112.96} & \bf{775.43} & \bf{111.61} & \bf{749.50} & \bf{2.62} & \bf{3.18}\\[1ex]\hline
\end{tabular}
\label{table:nonlin}
\end{table}
\begin{table}[h]
\caption{Resultados de la ejecución de la metaheurística AS-M, utilizando instancias de SalhiNagy con la configuración -n 5.0 -alpha 1.0 -beta 3.0 -q 0.1 -ro 0.015}
\centering
\small
\begin{tabular}{c c c c c c c c}
\hline\hline
Instancia & Costo mínimo & Tiempo(seg.) & Costo promedio & Tiempo promedio(seg.) & CME & \%G & \%GP \\ [0.5ex]
\hline
CMT1X & 474.41 & 3.49 & 
476.51 & 3.50 & \bf{470.48} & 
0.84 & 1.28\\CMT1Y & 475.22 & 3.20 & 
475.40 & 3.16 & \bf{470.48} & 
1.01 & 1.05\\CMT2X & 696.29 & 15.01 & 
699.96 & 14.68 & \bf{682.39} & 
2.04 & 2.57\\CMT2Y & 700.56 & 14.88 & 
703.67 & 14.53 & \bf{682.39} & 
2.66 & 3.12\\CMT3X & 724.14 & 44.89 & 
726.48 & 45.26 & \bf{719.06} & 
0.71 & 1.03\\CMT3Y & 721.81 & 47.45 & 
724.25 & 47.20 & \bf{719.06} & 
0.38 & 0.72\\CMT4X & 883.57 & 203.36 & 
886.51 & 204.58 & \bf{854.21} & 
3.44 & 3.78\\CMT4Y & 882.16 & 197.48 & 
886.42 & 199.82 & \bf{852.46} & 
3.48 & 3.98\\CMT5X & 1083.56 & 603.75 & 
1086.90 & 614.36 & \bf{1030.56} & 
5.14 & 5.47\\CMT5Y & 1081.01 & 600.45 & 
1087.33 & 588.44 & \bf{1031.69} & 
4.78 & 5.39\\CMT11X & 882.60 & 89.41 & 
882.66 & 92.17 & \bf{831.09} & 
6.20 & 6.21\\CMT11Y & 838.84 & 75.93 & 
855.70 & 77.39 & \bf{829.85} & 
1.08 & 3.11\\CMT12X & 669.35 & 40.06 & 
671.24 & 39.80 & \bf{658.83} & 
1.60 & 1.88\\CMT12Y & 662.46 & 36.05 & 
668.60 & 37.15 & \bf{660.47} & 
0.30 & 1.23\\\bf{PROM.} & 
\bf{769.71} & \bf{141.10} & \bf{773.69} & \bf{141.57} & \bf{749.50} & \bf{2.40} & \bf{2.92}\\[1ex]\hline
\end{tabular}
\label{table:nonlin}
\end{table}
\begin{table}[h]
\caption{Resultados de la ejecución de la metaheurística AS-M, utilizando instancias de SalhiNagy con la configuración -n 5.0 -alpha 1.0 -beta 3.0 -q .3 -ro 0.015}
\centering
\small
\begin{tabular}{c c c c c c c c}
\hline\hline
Instancia & Costo mínimo & Tiempo(seg.) & Costo promedio & Tiempo promedio(seg.) & CME & \%G & \%GP \\ [0.5ex]
\hline
CMT1X & 470.67 & 3.19 & 
474.61 & 3.15 & \bf{470.48} & 
0.04 & 0.88\\CMT1Y & 475.37 & 3.20 & 
476.64 & 3.04 & \bf{470.48} & 
1.04 & 1.31\\CMT2X & 697.30 & 14.31 & 
701.66 & 14.39 & \bf{682.39} & 
2.18 & 2.82\\CMT2Y & 694.79 & 14.39 & 
697.64 & 14.38 & \bf{682.39} & 
1.82 & 2.24\\CMT3X & 726.85 & 47.55 & 
727.23 & 46.10 & \bf{719.06} & 
1.08 & 1.14\\CMT3Y & 732.70 & 46.26 & 
733.92 & 47.01 & \bf{719.06} & 
1.90 & 2.07\\CMT4X & 884.79 & 195.24 & 
885.15 & 195.69 & \bf{854.21} & 
3.58 & 3.62\\CMT4Y & 878.35 & 205.48 & 
885.23 & 201.16 & \bf{852.46} & 
3.04 & 3.84\\CMT5X & 1076.69 & 574.24 & 
1080.76 & 582.87 & \bf{1030.56} & 
4.48 & 4.87\\CMT5Y & 1081.86 & 618.64 & 
1087.61 & 618.00 & \bf{1031.69} & 
4.86 & 5.42\\CMT11X & 852.51 & 95.24 & 
865.27 & 90.78 & \bf{831.09} & 
2.58 & 4.11\\CMT11Y & 847.88 & 77.54 & 
847.96 & 76.77 & \bf{829.85} & 
2.17 & 2.18\\CMT12X & 674.88 & 38.02 & 
677.86 & 38.45 & \bf{658.83} & 
2.44 & 2.89\\CMT12Y & 674.33 & 35.29 & 
677.83 & 36.09 & \bf{660.47} & 
2.10 & 2.63\\\bf{PROM.} & 
\bf{769.21} & \bf{140.61} & \bf{772.81} & \bf{140.56} & \bf{749.50} & \bf{2.38} & \bf{2.86}\\[1ex]\hline
\end{tabular}
\label{table:nonlin}
\end{table}
\begin{table}[h]
\caption{Resultados de la ejecución de la metaheurística AS-M, utilizando instancias de SalhiNagy con la configuración -n 5.0 -alpha 1.0 -beta 3.0 -q .4 -ro 0.015}
\centering
\small
\begin{tabular}{c c c c c c c c}
\hline\hline
Instancia & Costo mínimo & Tiempo(seg.) & Costo promedio & Tiempo promedio(seg.) & CME & \%G & \%GP \\ [0.5ex]
\hline
CMT1X & 472.87 & 3.46 & 
475.30 & 3.32 & \bf{470.48} & 
0.51 & 1.02\\CMT1Y & 471.09 & 3.15 & 
473.25 & 3.25 & \bf{470.48} & 
0.13 & 0.59\\CMT2X & 700.26 & 15.92 & 
702.00 & 15.52 & \bf{682.39} & 
2.62 & 2.87\\CMT2Y & 696.51 & 14.78 & 
698.41 & 14.43 & \bf{682.39} & 
2.07 & 2.35\\CMT3X & 726.47 & 45.78 & 
729.82 & 45.08 & \bf{719.06} & 
1.03 & 1.50\\CMT3Y & 731.62 & 45.49 & 
733.32 & 47.52 & \bf{719.06} & 
1.75 & 1.98\\CMT4X & 873.05 & 201.00 & 
880.00 & 196.01 & \bf{854.21} & 
2.21 & 3.02\\CMT4Y & 880.95 & 201.32 & 
885.86 & 203.00 & \bf{852.46} & 
3.34 & 3.92\\CMT5X & 1088.15 & 603.70 & 
1090.07 & 620.89 & \bf{1030.56} & 
5.59 & 5.77\\CMT5Y & 1086.61 & 587.27 & 
1089.18 & 583.59 & \bf{1031.69} & 
5.32 & 5.57\\CMT11X & 877.52 & 93.34 & 
877.58 & 92.02 & \bf{831.09} & 
5.59 & 5.59\\CMT11Y & 849.74 & 80.47 & 
869.46 & 81.64 & \bf{829.85} & 
2.40 & 4.77\\CMT12X & 674.29 & 38.61 & 
675.30 & 37.77 & \bf{658.83} & 
2.35 & 2.50\\CMT12Y & 672.55 & 38.69 & 
678.02 & 36.80 & \bf{660.47} & 
1.83 & 2.66\\\bf{PROM.} & 
\bf{771.55} & \bf{140.93} & \bf{775.54} & \bf{141.49} & \bf{749.50} & \bf{2.62} & \bf{3.15}\\[1ex]\hline
\end{tabular}
\label{table:nonlin}
\end{table}
\begin{table}[h]
\caption{Resultados de la ejecución de la metaheurística AS-M, utilizando instancias de SalhiNagy con la configuración -n 6.0 -alpha 1.0 -beta 3.0 -q 0.1 -ro 0.015}
\centering
\small
\begin{tabular}{c c c c c c c c}
\hline\hline
Instancia & Costo mínimo & Tiempo(seg.) & Costo promedio & Tiempo promedio(seg.) & CME & \%G & \%GP \\ [0.5ex]
\hline
CMT1X & 472.37 & 3.91 & 
474.38 & 3.73 & \bf{470.48} & 
0.40 & 0.83\\CMT1Y & 472.87 & 3.79 & 
474.79 & 3.89 & \bf{470.48} & 
0.51 & 0.92\\CMT2X & 707.59 & 17.75 & 
707.94 & 18.12 & \bf{682.39} & 
3.69 & 3.74\\CMT2Y & 692.84 & 17.46 & 
695.51 & 17.46 & \bf{682.39} & 
1.53 & 1.92\\CMT3X & 731.48 & 68.94 & 
731.48 & 65.67 & \bf{719.06} & 
1.73 & 1.73\\CMT3Y & 725.12 & 56.05 & 
728.42 & 55.62 & \bf{719.06} & 
0.84 & 1.30\\CMT4X & 890.27 & 237.22 & 
891.14 & 240.45 & \bf{854.21} & 
4.22 & 4.32\\CMT4Y & 883.64 & 236.18 & 
888.63 & 258.64 & \bf{852.46} & 
3.66 & 4.24\\CMT5X & 1077.77 & 711.55 & 
1085.11 & 714.21 & \bf{1030.56} & 
4.58 & 5.29\\CMT5Y & 1085.11 & 711.55 & 
1087.93 & 357.28 & \bf{1031.69} & 
5.18 & 5.45\\CMT11X & 851.62 & 104.75 & 
855.48 & 107.10 & \bf{831.09} & 
2.47 & 2.93\\CMT11Y & 842.27 & 94.41 & 
844.58 & 97.94 & \bf{829.85} & 
1.50 & 1.78\\CMT12X & 673.32 & 45.98 & 
673.50 & 45.17 & \bf{658.83} & 
2.20 & 2.23\\CMT12Y & 671.73 & 47.80 & 
673.52 & 47.37 & \bf{660.47} & 
1.70 & 1.98\\\bf{PROM.} & 
\bf{769.86} & \bf{168.38} & \bf{772.32} & \bf{145.19} & \bf{749.50} & \bf{2.44} & \bf{2.76}\\[1ex]\hline
\end{tabular}
\label{table:nonlin}
\end{table}
\begin{table}[h]
\caption{Resultados de la ejecución de la metaheurística AS-M, utilizando instancias de SalhiNagy con la configuración -n 6.0 -alpha 1.0 -beta 3.0 -q .3 -ro 0.015}
\centering
\small
\begin{tabular}{c c c c c c c c}
\hline\hline
Instancia & Costo mínimo & Tiempo(seg.) & Costo promedio & Tiempo promedio(seg.) & CME & \%G & \%GP \\ [0.5ex]
\hline
CMT1X & 474.72 & 4.03 & 
477.20 & 3.91 & \bf{470.48} & 
0.90 & 1.43\\CMT1Y & 474.41 & 3.50 & 
474.56 & 3.63 & \bf{470.48} & 
0.84 & 0.87\\CMT2X & 701.00 & 17.42 & 
701.53 & 17.57 & \bf{682.39} & 
2.73 & 2.80\\CMT2Y & 699.88 & 17.50 & 
701.62 & 17.32 & \bf{682.39} & 
2.56 & 2.82\\CMT3X & 732.74 & 55.14 & 
732.92 & 55.13 & \bf{719.06} & 
1.90 & 1.93\\CMT3Y & 727.10 & 60.45 & 
729.36 & 57.66 & \bf{719.06} & 
1.12 & 1.43\\CMT4X & 883.00 & 233.07 & 
889.30 & 233.16 & \bf{854.21} & 
3.37 & 4.11\\CMT4Y & 884.90 & 243.10 & 
888.06 & 242.22 & \bf{852.46} & 
3.81 & 4.18\\CMT5X & 1088.88 & 716.46 & 
1090.60 & 695.30 & \bf{1030.56} & 
5.66 & 5.83\\CMT5Y & 1084.99 & 706.68 & 
1088.22 & 707.80 & \bf{1031.69} & 
5.17 & 5.48\\CMT11X & 863.81 & 104.32 & 
871.64 & 102.93 & \bf{831.09} & 
3.94 & 4.88\\CMT11Y & 861.68 & 100.27 & 
872.79 & 95.83 & \bf{829.85} & 
3.84 & 5.18\\CMT12X & 676.05 & 44.71 & 
676.06 & 45.96 & \bf{658.83} & 
2.61 & 2.62\\CMT12Y & 670.27 & 42.27 & 
673.05 & 44.69 & \bf{660.47} & 
1.48 & 1.91\\\bf{PROM.} & 
\bf{773.10} & \bf{167.78} & \bf{776.21} & \bf{165.94} & \bf{749.50} & \bf{2.85} & \bf{3.25}\\[1ex]\hline
\end{tabular}
\label{table:nonlin}
\end{table}
\begin{table}[h]
\caption{Resultados de la ejecución de la metaheurística AS-M, utilizando instancias de SalhiNagy con la configuración -n 6.0 -alpha 1.0 -beta 3.0 -q .4 -ro 0.015}
\centering
\small
\begin{tabular}{c c c c c c c c}
\hline\hline
Instancia & Costo mínimo & Tiempo(seg.) & Costo promedio & Tiempo promedio(seg.) & CME & \%G & \%GP \\ [0.5ex]
\hline
CMT1X & 476.38 & 3.78 & 
476.75 & 3.79 & \bf{470.48} & 
1.25 & 1.33\\CMT1Y & 472.87 & 3.70 & 
473.80 & 3.71 & \bf{470.48} & 
0.51 & 0.70\\CMT2X & 701.09 & 18.37 & 
701.25 & 18.31 & \bf{682.39} & 
2.74 & 2.76\\CMT2Y & 704.13 & 17.56 & 
704.86 & 17.43 & \bf{682.39} & 
3.19 & 3.29\\CMT3X & 729.80 & 55.76 & 
730.01 & 56.73 & \bf{719.06} & 
1.49 & 1.52\\CMT3Y & 728.03 & 54.68 & 
728.79 & 55.38 & \bf{719.06} & 
1.25 & 1.35\\CMT4X & 888.83 & 241.75 & 
891.51 & 242.74 & \bf{854.21} & 
4.05 & 4.37\\CMT4Y & 875.18 & 243.20 & 
881.96 & 242.04 & \bf{852.46} & 
2.67 & 3.46\\CMT5X & 1076.30 & 729.69 & 
1084.36 & 722.86 & \bf{1030.56} & 
4.44 & 5.22\\CMT5Y & 1081.46 & 716.50 & 
1083.43 & 760.62 & \bf{1031.69} & 
4.82 & 5.02\\CMT11X & 865.40 & 104.85 & 
875.25 & 103.21 & \bf{831.09} & 
4.13 & 5.31\\CMT11Y & 863.12 & 92.05 & 
873.37 & 92.59 & \bf{829.85} & 
4.01 & 5.24\\CMT12X & 677.01 & 48.09 & 
678.55 & 46.95 & \bf{658.83} & 
2.76 & 2.99\\CMT12Y & 669.13 & 44.57 & 
669.33 & 44.81 & \bf{660.47} & 
1.31 & 1.34\\\bf{PROM.} & 
\bf{772.05} & \bf{169.61} & \bf{775.23} & \bf{172.23} & \bf{749.50} & \bf{2.76} & \bf{3.14}\\[1ex]\hline
\end{tabular}
\label{table:AS-M-salhinagy}
\end{table}
\clearpage
\section{PSO-M}\label{tablas-entonacion-PSO-M}
\subsection{SalhiNagy}\label{tablas-entonacion-PSO-M-salhinagy}

\begin{table}[h]
\caption{Resultados de la ejecución de la metaheurística PSO-M, utilizando instancias de SalhiNagy con la configuración -n 10.0 -L 10.0 -cp 1 -cg 0 -cl 1 -cn 2 -w1 0.9 -wt 0.1 -K 5}
\centering
\small
\begin{tabular}{c c c c c c c c}
\hline\hline
Instancia & Costo mínimo & Tiempo(seg.) & Costo promedio & Tiempo promedio(seg.) & CME & \%G & \%GP \\ [0.5ex]
\hline
CMT1X & 475.35 & 0.55 & 
478.83 & 0.55 & \bf{470.48} & 
1.04 & 1.77\\CMT1Y & 474.70 & 0.84 & 
475.73 & 0.83 & \bf{470.48} & 
0.90 & 1.11\\CMT2X & 722.41 & 0.51 & 
728.73 & 0.85 & \bf{682.39} & 
5.86 & 6.79\\CMT2Y & 719.83 & 1.06 & 
720.38 & 1.23 & \bf{682.39} & 
5.49 & 5.57\\CMT3X & 738.73 & 6.16 & 
738.86 & 6.78 & \bf{719.06} & 
2.74 & 2.75\\CMT3Y & 743.96 & 8.98 & 
757.29 & 8.11 & \bf{719.06} & 
3.46 & 5.32\\CMT4X & 908.59 & 10.83 & 
909.48 & 13.15 & \bf{854.21} & 
6.37 & 6.47\\CMT4Y & 900.50 & 10.93 & 
906.30 & 12.97 & \bf{852.46} & 
5.64 & 6.32\\CMT5X & 1092.93 & 32.14 & 
1138.10 & 38.85 & \bf{1030.56} & 
6.05 & 10.44\\CMT5Y & 1158.65 & 13.44 & 
1160.84 & 14.86 & \bf{1031.69} & 
12.31 & 12.52\\CMT11X & 943.91 & 1.82 & 
981.16 & 1.49 & \bf{831.09} & 
13.57 & 18.06\\CMT11Y & 932.84 & 2.06 & 
1006.66 & 1.54 & \bf{829.85} & 
12.41 & 21.31\\CMT12X & 732.20 & 0.92 & 
746.05 & 0.76 & \bf{658.83} & 
11.14 & 13.24\\CMT12Y & 798.11 & 0.57 & 
798.11 & 0.55 & \bf{660.47} & 
20.84 & 20.84\\\bf{PROM.} & 
\bf{810.19} & \bf{6.49} & \bf{824.75} & \bf{7.32} & \bf{749.50} & \bf{7.70} & \bf{9.46}\\[1ex]\hline
\end{tabular}
\label{table:PSO-M-dethloff}
\end{table}
\begin{table}[h]
\caption{Resultados de la ejecución de la metaheurística PSO-M, utilizando instancias de SalhiNagy con la configuración -n 10.0 -L 30.0 -cp 1 -cg 0 -cl 1 -cn 2 -w1 0.9 -wt 0.1 -K 5}
\centering
\small
\begin{tabular}{c c c c c c c c}
\hline\hline
Instancia & Costo mínimo & Tiempo(seg.) & Costo promedio & Tiempo promedio(seg.) & CME & \%G & \%GP \\ [0.5ex]
\hline
CMT1X & 474.41 & 1.80 & 
476.42 & 1.89 & \bf{470.48} & 
0.84 & 1.26\\CMT1Y & 477.21 & 1.98 & 
482.05 & 2.18 & \bf{470.48} & 
1.43 & 2.46\\CMT2X & 706.14 & 3.98 & 
734.48 & 3.13 & \bf{682.39} & 
3.48 & 7.63\\CMT2Y & 694.50 & 2.42 & 
707.60 & 2.58 & \bf{682.39} & 
1.77 & 3.69\\CMT3X & 733.20 & 25.64 & 
737.93 & 25.25 & \bf{719.06} & 
1.97 & 2.62\\CMT3Y & 728.71 & 26.41 & 
736.84 & 26.38 & \bf{719.06} & 
1.34 & 2.47\\CMT4X & 935.39 & 27.35 & 
951.11 & 31.12 & \bf{854.21} & 
9.50 & 11.34\\CMT4Y & 955.59 & 27.20 & 
958.10 & 39.60 & \bf{852.46} & 
12.10 & 12.39\\CMT5X & 1130.51 & 43.06 & 
1140.12 & 36.24 & \bf{1030.56} & 
9.70 & 10.63\\CMT5Y & 1168.08 & 19.64 & 
1180.84 & 32.60 & \bf{1031.69} & 
13.22 & 14.46\\CMT11X & 910.23 & 4.01 & 
940.16 & 5.29 & \bf{831.09} & 
9.52 & 13.12\\CMT11Y & 916.08 & 4.31 & 
920.65 & 5.57 & \bf{829.85} & 
10.39 & 10.94\\CMT12X & 775.11 & 1.99 & 
782.45 & 2.64 & \bf{658.83} & 
17.65 & 18.76\\CMT12Y & 773.76 & 1.92 & 
831.21 & 2.56 & \bf{660.47} & 
17.15 & 25.85\\\bf{PROM.} & 
\bf{812.78} & \bf{13.69} & \bf{827.14} & \bf{15.50} & \bf{749.50} & \bf{7.86} & \bf{9.83}\\[1ex]\hline
\end{tabular}
\label{table:nonlin}
\end{table}
\begin{table}[h]
\caption{Resultados de la ejecución de la metaheurística PSO-M, utilizando instancias de SalhiNagy con la configuración -n 10.0 -L 50.0 -cp 1 -cg 0 -cl 1 -cn 2 -w1 0.9 -wt 0.1 -K 5}
\centering
\small
\begin{tabular}{c c c c c c c c}
\hline\hline
Instancia & Costo mínimo & Tiempo(seg.) & Costo promedio & Tiempo promedio(seg.) & CME & \%G & \%GP \\ [0.5ex]
\hline
CMT1X & 470.67 & 3.31 & 
470.96 & 3.21 & \bf{470.48} & 
0.04 & 0.10\\CMT1Y & 476.51 & 3.80 & 
478.17 & 3.60 & \bf{470.48} & 
1.28 & 1.63\\CMT2X & 709.90 & 4.63 & 
751.30 & 5.25 & \bf{682.39} & 
4.03 & 10.10\\CMT2Y & 707.96 & 5.06 & 
749.18 & 4.53 & \bf{682.39} & 
3.75 & 9.79\\CMT3X & 744.85 & 41.54 & 
748.92 & 40.95 & \bf{719.06} & 
3.59 & 4.15\\CMT3Y & 735.62 & 46.49 & 
736.10 & 42.53 & \bf{719.06} & 
2.30 & 2.37\\CMT4X & 894.65 & 56.73 & 
914.04 & 62.83 & \bf{854.21} & 
4.73 & 7.00\\CMT4Y & 903.49 & 69.63 & 
931.51 & 75.05 & \bf{852.46} & 
5.99 & 9.27\\CMT5X & 1095.96 & 69.39 & 
1118.35 & 69.39 & \bf{1030.56} & 
6.35 & 8.52\\CMT5Y & 1138.83 & 80.81 & 
1142.50 & 78.66 & \bf{1031.69} & 
10.38 & 10.74\\CMT11X & 914.63 & 12.26 & 
931.06 & 13.52 & \bf{831.09} & 
10.05 & 12.03\\CMT11Y & 908.78 & 13.28 & 
914.17 & 11.28 & \bf{829.85} & 
9.51 & 10.16\\CMT12X & 701.26 & 3.60 & 
737.68 & 3.48 & \bf{658.83} & 
6.44 & 11.97\\CMT12Y & 769.18 & 3.36 & 
794.11 & 3.29 & \bf{660.47} & 
16.46 & 20.23\\\bf{PROM.} & 
\bf{798.02} & \bf{29.56} & \bf{815.58} & \bf{29.83} & \bf{749.50} & \bf{6.06} & \bf{8.43}\\[1ex]\hline
\end{tabular}
\label{table:nonlin}
\end{table}
\begin{table}[h]
\caption{Resultados de la ejecución de la metaheurística PSO-M, utilizando instancias de SalhiNagy con la configuración -n 10.0 -L 70.0 -cp 1 -cg 0 -cl 1 -cn 2 -w1 0.9 -wt 0.1 -K 5}
\centering
\small
\begin{tabular}{c c c c c c c c}
\hline\hline
Instancia & Costo mínimo & Tiempo(seg.) & Costo promedio & Tiempo promedio(seg.) & CME & \%G & \%GP \\ [0.5ex]
\hline
CMT1X & 470.67 & 5.39 & 
472.05 & 5.34 & \bf{470.48} & 
0.04 & 0.33\\CMT1Y & 472.37 & 5.50 & 
472.62 & 5.70 & \bf{470.48} & 
0.40 & 0.45\\CMT2X & 709.80 & 6.60 & 
710.29 & 6.55 & \bf{682.39} & 
4.02 & 4.09\\CMT2Y & 721.62 & 5.44 & 
747.51 & 6.20 & \bf{682.39} & 
5.75 & 9.54\\CMT3X & 730.41 & 63.17 & 
732.65 & 60.74 & \bf{719.06} & 
1.58 & 1.89\\CMT3Y & 735.06 & 59.16 & 
739.25 & 59.55 & \bf{719.06} & 
2.23 & 2.81\\CMT4X & 884.08 & 84.86 & 
905.39 & 77.67 & \bf{854.21} & 
3.50 & 5.99\\CMT4Y & 888.72 & 92.65 & 
896.08 & 92.65 & \bf{852.46} & 
4.25 & 5.12\\CMT5X & 1123.18 & 69.03 & 
1124.80 & 53.33 & \bf{1030.56} & 
8.99 & 9.14\\CMT5Y & 1153.22 & 89.56 & 
1155.42 & 81.66 & \bf{1031.69} & 
11.78 & 11.99\\CMT11X & 918.67 & 11.93 & 
923.12 & 17.33 & \bf{831.09} & 
10.54 & 11.07\\CMT11Y & 913.68 & 16.32 & 
967.60 & 13.67 & \bf{829.85} & 
10.10 & 16.60\\CMT12X & 784.39 & 4.40 & 
792.32 & 4.41 & \bf{658.83} & 
19.06 & 20.26\\CMT12Y & 816.05 & 4.71 & 
835.10 & 5.95 & \bf{660.47} & 
23.56 & 26.44\\\bf{PROM.} & 
\bf{808.71} & \bf{37.05} & \bf{819.59} & \bf{35.05} & \bf{749.50} & \bf{7.56} & \bf{8.98}\\[1ex]\hline
\end{tabular}
\label{table:nonlin}
\end{table}
\begin{table}[h]
\caption{Resultados de la ejecución de la metaheurística PSO-M, utilizando instancias de SalhiNagy con la configuración -n 10.0 -L 90.0 -cp 1 -cg 0 -cl 1 -cn 2 -w1 0.9 -wt 0.1 -K 5}
\centering
\small
\begin{tabular}{c c c c c c c c}
\hline\hline
Instancia & Costo mínimo & Tiempo(seg.) & Costo promedio & Tiempo promedio(seg.) & CME & \%G & \%GP \\ [0.5ex]
\hline
CMT1X & \bf{470.48} & 6.22 & 
479.21 & 6.51 & 470.48 & 0.00
 & 1.86\\CMT1Y & 472.37 & 6.30 & 
484.45 & 6.55 & \bf{470.48} & 
0.40 & 2.97\\CMT2X & 715.54 & 8.80 & 
729.17 & 8.81 & \bf{682.39} & 
4.86 & 6.86\\CMT2Y & 726.07 & 10.33 & 
734.93 & 8.86 & \bf{682.39} & 
6.40 & 7.70\\CMT3X & 723.86 & 76.96 & 
726.55 & 76.88 & \bf{719.06} & 
0.67 & 1.04\\CMT3Y & 733.71 & 74.21 & 
742.61 & 77.59 & \bf{719.06} & 
2.04 & 3.28\\CMT4X & 904.10 & 103.19 & 
917.59 & 94.97 & \bf{854.21} & 
5.84 & 7.42\\CMT4Y & 898.59 & 93.94 & 
937.05 & 97.83 & \bf{852.46} & 
5.41 & 9.92\\CMT5X & 1158.73 & 69.98 & 
1164.42 & 109.78 & \bf{1030.56} & 
12.44 & 12.99\\CMT5Y & 1147.76 & 111.20 & 
1153.11 & 96.53 & \bf{1031.69} & 
11.25 & 11.77\\CMT11X & 907.53 & 14.79 & 
911.03 & 14.99 & \bf{831.09} & 
9.20 & 9.62\\CMT11Y & 903.36 & 15.34 & 
920.98 & 19.52 & \bf{829.85} & 
8.86 & 10.98\\CMT12X & 711.75 & 6.24 & 
730.71 & 6.32 & \bf{658.83} & 
8.03 & 10.91\\CMT12Y & 854.42 & 8.96 & 
888.57 & 7.82 & \bf{660.47} & 
29.37 & 34.54\\\bf{PROM.} & 
\bf{809.16} & \bf{43.32} & \bf{822.88} & \bf{45.21} & \bf{749.50} & \bf{7.48} & \bf{9.42}\\[1ex]\hline
\end{tabular}
\label{table:nonlin}
\end{table}
\begin{table}[h]
\caption{Resultados de la ejecución de la metaheurística PSO-M, utilizando instancias de SalhiNagy con la configuración -n 30.0 -L 10.0 -cp 1 -cg 0 -cl 1 -cn 2 -w1 0.9 -wt 0.1 -K 5}
\centering
\small
\begin{tabular}{c c c c c c c c}
\hline\hline
Instancia & Costo mínimo & Tiempo(seg.) & Costo promedio & Tiempo promedio(seg.) & CME & \%G & \%GP \\ [0.5ex]
\hline
CMT1X & 472.87 & 1.82 & 
475.80 & 1.86 & \bf{470.48} & 
0.51 & 1.13\\CMT1Y & 495.40 & 1.31 & 
513.36 & 1.93 & \bf{470.48} & 
5.30 & 9.11\\CMT2X & 719.51 & 1.84 & 
728.50 & 3.12 & \bf{682.39} & 
5.44 & 6.76\\CMT2Y & 773.70 & 3.80 & 
787.73 & 3.23 & \bf{682.39} & 
13.38 & 15.44\\CMT3X & 737.86 & 23.18 & 
743.43 & 26.66 & \bf{719.06} & 
2.61 & 3.39\\CMT3Y & 731.93 & 30.18 & 
744.86 & 26.04 & \bf{719.06} & 
1.79 & 3.59\\CMT4X & 917.13 & 28.78 & 
934.46 & 29.30 & \bf{854.21} & 
7.37 & 9.39\\CMT4Y & 908.64 & 20.27 & 
932.59 & 16.29 & \bf{852.46} & 
6.59 & 9.40\\CMT5X & 1165.70 & 160.05 & 
1168.04 & 87.94 & \bf{1030.56} & 
13.11 & 13.34\\CMT5Y & 1131.63 & 15.73 & 
1159.22 & 38.98 & \bf{1031.69} & 
9.69 & 12.36\\CMT11X & 931.33 & 5.15 & 
941.22 & 4.49 & \bf{831.09} & 
12.06 & 13.25\\CMT11Y & 898.85 & 7.12 & 
899.40 & 8.98 & \bf{829.85} & 
8.31 & 8.38\\CMT12X & 722.55 & 2.16 & 
751.77 & 2.17 & \bf{658.83} & 
9.67 & 14.11\\CMT12Y & 798.30 & 1.41 & 
799.28 & 1.40 & \bf{660.47} & 
20.87 & 21.02\\\bf{PROM.} & 
\bf{814.67} & \bf{21.63} & \bf{827.12} & \bf{18.03} & \bf{749.50} & \bf{8.34} & \bf{10.05}\\[1ex]\hline
\end{tabular}
\label{table:nonlin}
\end{table}
\begin{table}[h]
\caption{Resultados de la ejecución de la metaheurística PSO-M, utilizando instancias de SalhiNagy con la configuración -n 30.0 -L 30.0 -cp 1 -cg 0 -cl 1 -cn 2 -w1 0.9 -wt 0.1 -K 5}
\centering
\small
\begin{tabular}{c c c c c c c c}
\hline\hline
Instancia & Costo mínimo & Tiempo(seg.) & Costo promedio & Tiempo promedio(seg.) & CME & \%G & \%GP \\ [0.5ex]
\hline
CMT1X & 472.37 & 4.96 & 
472.37 & 5.83 & \bf{470.48} & 
0.40 & 0.40\\CMT1Y & 471.25 & 5.93 & 
475.52 & 5.58 & \bf{470.48} & 
0.16 & 1.07\\CMT2X & 701.79 & 10.36 & 
737.60 & 10.65 & \bf{682.39} & 
2.84 & 8.09\\CMT2Y & 741.54 & 12.35 & 
757.29 & 11.29 & \bf{682.39} & 
8.67 & 10.98\\CMT3X & 725.16 & 65.53 & 
731.46 & 66.11 & \bf{719.06} & 
0.85 & 1.72\\CMT3Y & 722.97 & 62.61 & 
746.34 & 69.03 & \bf{719.06} & 
0.54 & 3.79\\CMT4X & 914.28 & 85.78 & 
937.26 & 80.34 & \bf{854.21} & 
7.03 & 9.72\\CMT4Y & 871.22 & 66.15 & 
885.53 & 107.36 & \bf{852.46} & 
2.20 & 3.88\\CMT5X & 1111.68 & 46.21 & 
1122.70 & 131.13 & \bf{1030.56} & 
7.87 & 8.94\\CMT5Y & 1159.74 & 248.55 & 
1175.49 & 247.18 & \bf{1031.69} & 
12.41 & 13.94\\CMT11X & 901.31 & 19.30 & 
925.38 & 14.54 & \bf{831.09} & 
8.45 & 11.35\\CMT11Y & 901.51 & 15.44 & 
903.36 & 23.37 & \bf{829.85} & 
8.64 & 8.86\\CMT12X & 705.34 & 4.92 & 
755.54 & 5.99 & \bf{658.83} & 
7.06 & 14.68\\CMT12Y & 683.58 & 4.31 & 
741.89 & 4.36 & \bf{660.47} & 
3.50 & 12.33\\\bf{PROM.} & 
\bf{791.70} & \bf{46.60} & \bf{811.98} & \bf{55.91} & \bf{749.50} & \bf{5.04} & \bf{7.84}\\[1ex]\hline
\end{tabular}
\label{table:nonlin}
\end{table}
\begin{table}[h]
\caption{Resultados de la ejecución de la metaheurística PSO-M, utilizando instancias de SalhiNagy con la configuración -n 30.0 -L 50.0 -cp 1 -cg 0 -cl 1 -cn 2 -w1 0.9 -wt 0.1 -K 5}
\centering
\small
\begin{tabular}{c c c c c c c c}
\hline\hline
Instancia & Costo mínimo & Tiempo(seg.) & Costo promedio & Tiempo promedio(seg.) & CME & \%G & \%GP \\ [0.5ex]
\hline
CMT1X & 472.87 & 8.87 & 
473.89 & 9.49 & \bf{470.48} & 
0.51 & 0.72\\CMT1Y & \bf{470.48} & 9.64 & 
470.87 & 10.31 & 470.48 & 0.00
 & 0.08\\CMT2X & 711.98 & 18.66 & 
723.34 & 16.23 & \bf{682.39} & 
4.34 & 6.00\\CMT2Y & 699.23 & 18.49 & 
702.34 & 17.45 & \bf{682.39} & 
2.47 & 2.92\\CMT3X & 732.75 & 117.75 & 
741.67 & 114.59 & \bf{719.06} & 
1.90 & 3.14\\CMT3Y & 731.01 & 113.47 & 
733.44 & 115.78 & \bf{719.06} & 
1.66 & 2.00\\CMT4X & 884.04 & 228.28 & 
888.49 & 195.35 & \bf{854.21} & 
3.49 & 4.01\\CMT4Y & 892.82 & 174.11 & 
898.60 & 167.23 & \bf{852.46} & 
4.73 & 5.41\\CMT5X & 1190.01 & 223.55 & 
1202.96 & 209.81 & \bf{1030.56} & 
15.47 & 16.73\\CMT5Y & 1113.32 & 469.13 & 
1130.37 & 284.38 & \bf{1031.69} & 
7.91 & 9.56\\CMT11X & 915.19 & 49.25 & 
918.00 & 47.55 & \bf{831.09} & 
10.12 & 10.46\\CMT11Y & 897.07 & 33.22 & 
909.38 & 36.27 & \bf{829.85} & 
8.10 & 9.58\\CMT12X & 738.91 & 7.41 & 
742.55 & 7.32 & \bf{658.83} & 
12.15 & 12.71\\CMT12Y & 733.79 & 7.64 & 
772.86 & 10.35 & \bf{660.47} & 
11.10 & 17.02\\\bf{PROM.} & 
\bf{798.82} & \bf{105.68} & \bf{807.77} & \bf{88.72} & \bf{749.50} & \bf{6.00} & \bf{7.17}\\[1ex]\hline
\end{tabular}
\label{table:nonlin}
\end{table}
\begin{table}[h]
\caption{Resultados de la ejecución de la metaheurística PSO-M, utilizando instancias de SalhiNagy con la configuración -n 30.0 -L 70.0 -cp 1 -cg 0 -cl 1 -cn 2 -w1 0.9 -wt 0.1 -K 5}
\centering
\small
\begin{tabular}{c c c c c c c c}
\hline\hline
Instancia & Costo mínimo & Tiempo(seg.) & Costo promedio & Tiempo promedio(seg.) & CME & \%G & \%GP \\ [0.5ex]
\hline
CMT1X & 470.67 & 14.91 & 
471.52 & 15.95 & \bf{470.48} & 
0.04 & 0.22\\CMT1Y & 470.67 & 11.86 & 
482.54 & 12.18 & \bf{470.48} & 
0.04 & 2.56\\CMT2X & 699.28 & 22.03 & 
715.43 & 19.05 & \bf{682.39} & 
2.48 & 4.84\\CMT2Y & 699.09 & 16.63 & 
725.11 & 21.24 & \bf{682.39} & 
2.45 & 6.26\\CMT3X & 730.30 & 160.07 & 
731.71 & 161.41 & \bf{719.06} & 
1.56 & 1.76\\CMT3Y & 725.30 & 159.14 & 
729.00 & 168.12 & \bf{719.06} & 
0.87 & 1.38\\CMT4X & 892.29 & 216.43 & 
903.11 & 205.19 & \bf{854.21} & 
4.46 & 5.72\\CMT4Y & 887.34 & 209.72 & 
908.81 & 232.06 & \bf{852.46} & 
4.09 & 6.61\\CMT5X & 1135.42 & 342.10 & 
1166.31 & 338.54 & \bf{1030.56} & 
10.18 & 13.17\\CMT5Y & 1103.07 & 162.21 & 
1130.86 & 195.93 & \bf{1031.69} & 
6.92 & 9.61\\CMT11X & 883.11 & 112.29 & 
884.25 & 78.28 & \bf{831.09} & 
6.26 & 6.40\\CMT11Y & 892.43 & 44.67 & 
896.29 & 59.37 & \bf{829.85} & 
7.54 & 8.01\\CMT12X & 701.22 & 11.26 & 
753.39 & 10.94 & \bf{658.83} & 
6.43 & 14.35\\CMT12Y & 694.27 & 10.96 & 
759.38 & 11.74 & \bf{660.47} & 
5.12 & 14.98\\\bf{PROM.} & 
\bf{784.60} & \bf{106.73} & \bf{804.12} & \bf{109.28} & \bf{749.50} & \bf{4.17} & \bf{6.85}\\[1ex]\hline
\end{tabular}
\label{table:nonlin}
\end{table}
\begin{table}[h]
\caption{Resultados de la ejecución de la metaheurística PSO-M, utilizando instancias de SalhiNagy con la configuración -n 30.0 -L 90.0 -cp 1 -cg 0 -cl 1 -cn 2 -w1 0.9 -wt 0.1 -K 5}
\centering
\small
\begin{tabular}{c c c c c c c c}
\hline\hline
Instancia & Costo mínimo & Tiempo(seg.) & Costo promedio & Tiempo promedio(seg.) & CME & \%G & \%GP \\ [0.5ex]
\hline
CMT1X & 472.37 & 19.09 & 
489.41 & 17.64 & \bf{470.48} & 
0.40 & 4.02\\CMT1Y & 471.25 & 17.19 & 
471.81 & 17.86 & \bf{470.48} & 
0.16 & 0.28\\CMT2X & 703.80 & 24.19 & 
707.75 & 22.39 & \bf{682.39} & 
3.14 & 3.72\\CMT2Y & 706.85 & 20.71 & 
745.30 & 23.94 & \bf{682.39} & 
3.58 & 9.22\\CMT3X & 725.93 & 205.58 & 
726.74 & 209.85 & \bf{719.06} & 
0.96 & 1.07\\CMT3Y & 731.17 & 194.34 & 
734.29 & 200.57 & \bf{719.06} & 
1.68 & 2.12\\CMT4X & 930.31 & 277.54 & 
941.03 & 338.06 & \bf{854.21} & 
8.91 & 10.16\\CMT4Y & 884.77 & 263.89 & 
922.38 & 301.71 & \bf{852.46} & 
3.79 & 8.20\\CMT5X & 1134.02 & 413.46 & 
1148.45 & 340.15 & \bf{1030.56} & 
10.04 & 11.44\\CMT5Y & 1093.02 & 476.60 & 
1101.95 & 435.88 & \bf{1031.69} & 
5.94 & 6.81\\CMT11X & 892.48 & 83.22 & 
910.72 & 91.33 & \bf{831.09} & 
7.39 & 9.58\\CMT11Y & 910.24 & 92.67 & 
912.60 & 74.20 & \bf{829.85} & 
9.69 & 9.97\\CMT12X & 676.18 & 15.97 & 
769.18 & 15.03 & \bf{658.83} & 
2.63 & 16.75\\CMT12Y & 725.35 & 15.14 & 
794.06 & 15.09 & \bf{660.47} & 
9.82 & 20.23\\\bf{PROM.} & 
\bf{789.84} & \bf{151.40} & \bf{812.55} & \bf{150.26} & \bf{749.50} & \bf{4.87} & \bf{8.11}\\[1ex]\hline
\end{tabular}
\label{table:nonlin}
\end{table}
\begin{table}[h]
\caption{Resultados de la ejecución de la metaheurística PSO-M, utilizando instancias de SalhiNagy con la configuración -n 50.0 -L 10.0 -cp 1 -cg 0 -cl 1 -cn 2 -w1 0.9 -wt 0.1 -K 5}
\centering
\small
\begin{tabular}{c c c c c c c c}
\hline\hline
Instancia & Costo mínimo & Tiempo(seg.) & Costo promedio & Tiempo promedio(seg.) & CME & \%G & \%GP \\ [0.5ex]
\hline
CMT1X & 471.25 & 3.84 & 
471.81 & 4.24 & \bf{470.48} & 
0.16 & 0.28\\CMT1Y & 476.53 & 3.81 & 
482.31 & 3.48 & \bf{470.48} & 
1.29 & 2.51\\CMT2X & 704.31 & 6.51 & 
719.52 & 4.74 & \bf{682.39} & 
3.21 & 5.44\\CMT2Y & 717.09 & 6.59 & 
717.41 & 4.12 & \bf{682.39} & 
5.09 & 5.13\\CMT3X & 735.43 & 35.01 & 
740.28 & 34.91 & \bf{719.06} & 
2.28 & 2.95\\CMT3Y & 733.04 & 42.21 & 
744.48 & 41.38 & \bf{719.06} & 
1.94 & 3.54\\CMT4X & 955.07 & 84.12 & 
1035.63 & 95.45 & \bf{854.21} & 
11.81 & 21.24\\CMT4Y & 887.68 & 16.17 & 
907.20 & 55.52 & \bf{852.46} & 
4.13 & 6.42\\CMT5X & 1146.53 & 22.53 & 
1148.85 & 50.94 & \bf{1030.56} & 
11.25 & 11.48\\CMT5Y & 1208.80 & 12.85 & 
1217.37 & 127.27 & \bf{1031.69} & 
17.17 & 18.00\\CMT11X & 907.94 & 5.53 & 
916.14 & 16.17 & \bf{831.09} & 
9.25 & 10.23\\CMT11Y & 894.19 & 10.12 & 
909.83 & 10.61 & \bf{829.85} & 
7.75 & 9.64\\CMT12X & 737.78 & 2.49 & 
759.64 & 2.90 & \bf{658.83} & 
11.98 & 15.30\\CMT12Y & 675.61 & 3.92 & 
701.05 & 4.59 & \bf{660.47} & 
2.29 & 6.14\\\bf{PROM.} & 
\bf{803.66} & \bf{18.26} & \bf{819.39} & \bf{32.59} & \bf{749.50} & \bf{6.40} & \bf{8.45}\\[1ex]\hline
\end{tabular}
\label{table:nonlin}
\end{table}
\begin{table}[h]
\caption{Resultados de la ejecución de la metaheurística PSO-M, utilizando instancias de SalhiNagy con la configuración -n 50.0 -L 30.0 -cp 1 -cg 0 -cl 1 -cn 2 -w1 0.9 -wt 0.1 -K 5}
\centering
\small
\begin{tabular}{c c c c c c c c}
\hline\hline
Instancia & Costo mínimo & Tiempo(seg.) & Costo promedio & Tiempo promedio(seg.) & CME & \%G & \%GP \\ [0.5ex]
\hline
CMT1X & 472.37 & 9.00 & 
472.37 & 9.34 & \bf{470.48} & 
0.40 & 0.40\\CMT1Y & 472.37 & 7.91 & 
474.39 & 8.60 & \bf{470.48} & 
0.40 & 0.83\\CMT2X & 705.76 & 8.17 & 
706.54 & 10.19 & \bf{682.39} & 
3.42 & 3.54\\CMT2Y & 697.44 & 17.94 & 
763.76 & 19.52 & \bf{682.39} & 
2.21 & 11.92\\CMT3X & 752.79 & 101.67 & 
762.11 & 94.98 & \bf{719.06} & 
4.69 & 5.99\\CMT3Y & 723.67 & 108.12 & 
730.59 & 115.36 & \bf{719.06} & 
0.64 & 1.60\\CMT4X & 882.12 & 175.82 & 
908.82 & 158.59 & \bf{854.21} & 
3.27 & 6.39\\CMT4Y & 903.63 & 188.86 & 
939.82 & 215.71 & \bf{852.46} & 
6.00 & 10.25\\CMT5X & 1147.12 & 524.72 & 
1154.86 & 437.89 & \bf{1030.56} & 
11.31 & 12.06\\CMT5Y & 1132.69 & 69.30 & 
1177.20 & 125.66 & \bf{1031.69} & 
9.79 & 14.10\\CMT11X & 885.38 & 74.00 & 
942.62 & 56.35 & \bf{831.09} & 
6.53 & 13.42\\CMT11Y & 903.44 & 36.59 & 
905.38 & 40.98 & \bf{829.85} & 
8.87 & 9.10\\CMT12X & 750.03 & 6.97 & 
810.80 & 7.13 & \bf{658.83} & 
13.84 & 23.07\\CMT12Y & 683.58 & 7.69 & 
728.83 & 7.12 & \bf{660.47} & 
3.50 & 10.35\\\bf{PROM.} & 
\bf{793.74} & \bf{95.48} & \bf{819.86} & \bf{93.39} & \bf{749.50} & \bf{5.35} & \bf{8.79}\\[1ex]\hline
\end{tabular}
\label{table:nonlin}
\end{table}
\begin{table}[h]
\caption{Resultados de la ejecución de la metaheurística PSO-M, utilizando instancias de SalhiNagy con la configuración -n 50.0 -L 50.0 -cp 1 -cg 0 -cl 1 -cn 2 -w1 0.9 -wt 0.1 -K 5}
\centering
\small
\begin{tabular}{c c c c c c c c}
\hline\hline
Instancia & Costo mínimo & Tiempo(seg.) & Costo promedio & Tiempo promedio(seg.) & CME & \%G & \%GP \\ [0.5ex]
\hline
CMT1X & 470.67 & 13.02 & 
472.79 & 14.22 & \bf{470.48} & 
0.04 & 0.49\\CMT1Y & 470.67 & 15.64 & 
474.30 & 14.98 & \bf{470.48} & 
0.04 & 0.81\\CMT2X & 701.09 & 26.70 & 
715.48 & 22.65 & \bf{682.39} & 
2.74 & 4.85\\CMT2Y & 700.57 & 35.45 & 
743.18 & 34.73 & \bf{682.39} & 
2.66 & 8.91\\CMT3X & 723.67 & 169.42 & 
749.00 & 173.84 & \bf{719.06} & 
0.64 & 4.16\\CMT3Y & 725.72 & 184.05 & 
729.92 & 215.19 & \bf{719.06} & 
0.93 & 1.51\\CMT4X & 889.19 & 330.33 & 
894.43 & 327.23 & \bf{854.21} & 
4.10 & 4.71\\CMT4Y & 884.99 & 329.40 & 
918.78 & 335.67 & \bf{852.46} & 
3.82 & 7.78\\CMT5X & 1209.12 & 212.25 & 
1211.50 & 653.50 & \bf{1030.56} & 
17.33 & 17.56\\CMT5Y & 1105.97 & 399.16 & 
1138.25 & 420.59 & \bf{1031.69} & 
7.20 & 10.33\\CMT11X & 888.31 & 52.25 & 
937.61 & 61.49 & \bf{831.09} & 
6.88 & 12.82\\CMT11Y & 912.89 & 99.40 & 
957.08 & 98.62 & \bf{829.85} & 
10.01 & 15.33\\CMT12X & 789.52 & 11.95 & 
837.99 & 12.06 & \bf{658.83} & 
19.84 & 27.19\\CMT12Y & 703.08 & 14.82 & 
721.81 & 17.68 & \bf{660.47} & 
6.45 & 9.29\\\bf{PROM.} & 
\bf{798.25} & \bf{135.27} & \bf{821.58} & \bf{171.60} & \bf{749.50} & \bf{5.90} & \bf{8.98}\\[1ex]\hline
\end{tabular}
\label{table:nonlin}
\end{table}
\begin{table}[h]
\caption{Resultados de la ejecución de la metaheurística PSO-M, utilizando instancias de SalhiNagy con la configuración -n 50.0 -L 70.0 -cp 1 -cg 0 -cl 1 -cn 2 -w1 0.9 -wt 0.1 -K 5}
\centering
\small
\begin{tabular}{c c c c c c c c}
\hline\hline
Instancia & Costo mínimo & Tiempo(seg.) & Costo promedio & Tiempo promedio(seg.) & CME & \%G & \%GP \\ [0.5ex]
\hline
CMT1X & 471.25 & 22.10 & 
471.25 & 21.90 & \bf{470.48} & 
0.16 & 0.16\\CMT1Y & \bf{470.48} & 21.01 & 
470.87 & 22.02 & 470.48 & 0.00
 & 0.08\\CMT2X & 705.28 & 38.62 & 
721.83 & 36.45 & \bf{682.39} & 
3.35 & 5.78\\CMT2Y & 702.94 & 22.65 & 
707.25 & 27.29 & \bf{682.39} & 
3.01 & 3.64\\CMT3X & 747.16 & 274.61 & 
752.57 & 269.39 & \bf{719.06} & 
3.91 & 4.66\\CMT3Y & 725.57 & 223.59 & 
726.18 & 230.81 & \bf{719.06} & 
0.91 & 0.99\\CMT4X & 914.03 & 388.71 & 
945.83 & 425.42 & \bf{854.21} & 
7.00 & 10.73\\CMT4Y & 925.07 & 328.78 & 
929.32 & 387.35 & \bf{852.46} & 
8.52 & 9.02\\CMT5X & 1127.06 & 469.50 & 
1148.85 & 445.80 & \bf{1030.56} & 
9.36 & 11.48\\CMT5Y & 1155.25 & 500.64 & 
1187.38 & 592.57 & \bf{1031.69} & 
11.98 & 15.09\\CMT11X & 893.06 & 101.30 & 
940.77 & 50.65 & \bf{831.09} & 
7.46 & 13.20\\CMT11Y & 889.74 & 87.47 & 
896.95 & 93.68 & \bf{829.85} & 
7.22 & 8.09\\CMT12X & 705.55 & 18.66 & 
727.97 & 19.20 & \bf{658.83} & 
7.09 & 10.49\\CMT12Y & 708.37 & 18.82 & 
748.34 & 18.71 & \bf{660.47} & 
7.25 & 13.30\\\bf{PROM.} & 
\bf{795.77} & \bf{179.75} & \bf{812.52} & \bf{188.66} & \bf{749.50} & \bf{5.52} & \bf{7.62}\\[1ex]\hline
\end{tabular}
\label{table:nonlin}
\end{table}
\begin{table}[h]
\caption{Resultados de la ejecución de la metaheurística PSO-M, utilizando instancias de SalhiNagy con la configuración -n 50.0 -L 90.0 -cp 1 -cg 0 -cl 1 -cn 2 -w1 0.9 -wt 0.1 -K 5}
\centering
\small
\begin{tabular}{c c c c c c c c}
\hline\hline
Instancia & Costo mínimo & Tiempo(seg.) & Costo promedio & Tiempo promedio(seg.) & CME & \%G & \%GP \\ [0.5ex]
\hline
CMT1X & 470.67 & 26.67 & 
470.88 & 30.02 & \bf{470.48} & 
0.04 & 0.09\\CMT1Y & \bf{470.48} & 29.14 & 
470.57 & 29.32 & 470.48 & 0.00
 & 0.02\\CMT2X & 703.231224 & 34.11 & 
721.18 & 34.11 & \bf{682.39} & 
3.05 & 5.68\\CMT2Y & 692.15 & 35.81 & 
727.80 & 44.34 & \bf{682.39} & 
1.43 & 6.65\\CMT3X & 723.52 & 392.04 & 
763.20 & 358.35 & \bf{719.06} & 
0.62 & 6.14\\CMT3Y & 725.90 & 329.92 & 
727.47 & 327.12 & \bf{719.06} & 
0.95 & 1.17\\CMT4X & 878.88 & 561.14 & 
901.75 & 503.18 & \bf{854.21} & 
2.89 & 5.56\\CMT4Y & 896.19 & 552.75 & 
898.96 & 579.33 & \bf{852.46} & 
5.13 & 5.45\\CMT5X & 1135.60 & 886.99 & 
1136.95 & 828.08 & \bf{1030.56} & 
10.19 & 10.32\\CMT5Y & 1122.67 & 591.57 & 
1145.68 & 799.82 & \bf{1031.69} & 
8.82 & 11.05\\CMT11X & 903.66 & 177.27 & 
959.13 & 169.28 & \bf{831.09} & 
8.73 & 15.41\\CMT11Y & 882.72 & 156.44 & 
888.63 & 176.46 & \bf{829.85} & 
6.37 & 7.08\\CMT12X & 802.01 & 37.15 & 
856.42 & 30.32 & \bf{658.83} & 
21.73 & 29.99\\CMT12Y & 751.72 & 23.86 & 
755.09 & 24.62 & \bf{660.47} & 
13.82 & 14.33\\\bf{PROM.} & 
\bf{797.10} & \bf{273.92} & \bf{815.98} & \bf{281.02} & \bf{749.50} & \bf{5.98} & \bf{8.50}\\[1ex]\hline
\end{tabular}
\label{table:PSO-M-salhinagy}
\end{table}
\clearpage

\section{GA-M}\label{tablas-entonacion-GA-M}

\subsection{Dethloff}\label{tablas-entonacion-GA-M-dethloff}


\begin{table}[ht]
\caption{Resultados de la ejecución de la metaheurística GA-M, utilizando instancias de Dethloff con la configuración -n 100 -p 300 -cprob 40 -mprob 70}
\centering
\small
\begin{tabular}{c c c c c c c c c}
\hline\hline
Instancia & Costo mínimo & Tiempo(seg.) & Costo promedio & T. prom(seg.) & Por & CME & \%G & \%GP \\ [0.5ex]
\hline
SCA3-0 & 636.06 & 5.52 & 
640.40 & 5.70 & 13.33\% & \bf{635.62} & 
0.07 & 0.75\\SCA3-1 & \bf{697.84} & 5.64 & 
697.84 & 5.44 & 30\% & 697.84 & 0.00
 & 0.00\\
SCA3-2 & \bf{659.34} & 5.71 & 
660.48 & 5.11 & 66.66\% & 659.34 & 0.00
 & 0.17\\SCA3-3 & \bf{680.04} & 4.79 & 
680.59 & 5.31 & 86.66\% & 680.04 & 0.00
 & 0.08\\SCA3-4 & \bf{690.50} & 5.47 & 
690.50 & 5.34 & 10\% & 690.50 & 0.00
 & 0.00\\
SCA3-5 & 661.07 & 5.12 & 
662.96 & 5.36 & 90\% & \bf{659.90} & 
0.18 & 0.46\\SCA3-6 & 652.94 & 4.70 & 
652.94 & 5.24 & 16.66\% & \bf{651.09} & 
0.28 & 0.28\\SCA3-7 & 666.15 & 5.02 & 
666.15 & 5.38 & 0\% & \bf{659.17} & 
1.06 & 1.06\\SCA3-8 & \bf{719.47} & 5.07 & 
720.24 & 5.43 & 86.66\% & 719.47 & 0.00
 & 0.11\\SCA3-9 & \bf{681.00} & 5.69 & 
681.00 & 5.48 & 3.33\% & 681.00 & 0.00
 & 0.00\\
SCA8-0 & 970.64 & 5.71 & 
975.20 & 5.22 & 100\% & \bf{961.50} & 
0.95 & 1.43\\SCA8-1 & 1061.29 & 6.57 & 
1061.77 & 5.50 & 100\% & \bf{1049.65} & 
1.11 & 1.15\\SCA8-2 & 1050.37 & 4.41 & 
1050.37 & 5.48 & 40\% & \bf{1039.64} & 
1.03 & 1.03\\SCA8-3 & 995.60 & 6.15 & 
999.51 & 5.50 & 100\% & \bf{983.34} & 
1.25 & 1.64\\SCA8-4 & 1068.97 & 4.84 & 
1071.81 & 5.13 & 93.33\% & \bf{1065.49} & 
0.33 & 0.59\\SCA8-5 & 1036.88 & 4.29 & 
1042.12 & 5.32 & 100\% & \bf{1027.08} & 
0.95 & 1.46\\SCA8-6 & 976.37 & 6.60 & 
977.26 & 5.70 & 96.66\% & \bf{971.82} & 
0.47 & 0.56\\SCA8-7 & 1067.49 & 5.30 & 
1069.31 & 5.05 & 100\% & \bf{1051.28} & 
1.54 & 1.72\\SCA8-8 & \bf{1071.18} & 5.24 & 
1079.20 & 5.28 & 100\% & 1071.18 & 0.00
 & 0.75\\SCA8-9 & 1067.42 & 5.06 & 
1070.22 & 5.08 & 96.66\% & \bf{1060.50} & 
0.65 & 0.92\\CON3-0 & 617.59 & 6.58 & 
620.44 & 5.74 & 93.33\% & \bf{616.52} & 
0.17 & 0.64\\CON3-1 & 556.04 & 5.47 & 
557.31 & 5.72 & 86.66\% & \bf{554.47} & 
0.28 & 0.51\\CON3-2 & 519.26 & 5.53 & 
520.96 & 6.15 & 23.33\% & \bf{518.00} & 
0.24 & 0.57\\CON3-3 & \bf{591.19} & 6.25 & 
591.89 & 5.60 & 93.33\% & 591.19 & 0.00
 & 0.12\\CON3-4 & 589.32 & 5.29 & 
591.49 & 5.52 & 80\% & \bf{588.79} & 
0.09 & 0.46\\CON3-5 & \bf{563.70} & 4.91 & 
564.46 & 5.83 & 90\% & 563.70 & 0.00
 & 0.14\\CON3-6 & 502.16 & 5.61 & 
503.32 & 5.94 & 86.66\% & \bf{499.05} & 
0.62 & 0.86\\CON3-7 & \bf{576.48} & 5.17 & 
578.43 & 5.45 & 96.66\% & 576.48 & 0.00
 & 0.34\\CON3-8 & \bf{523.05} & 6.59 & 
525.13 & 5.79 & 83.33\% & 523.05 & 0.00
 & 0.40\\CON3-9 & 581.06 & 6.20 & 
583.41 & 5.92 & 83.33\% & \bf{578.24} & 
0.49 & 0.89\\CON8-0 & 860.48 & 5.06 & 
865.41 & 5.67 & 93.33\% & \bf{857.17} & 
0.39 & 0.96\\CON8-1 & 741.70 & 6.46 & 
750.95 & 5.61 & 90\% & \bf{740.85} & 
0.11 & 1.36\\CON8-2 & 716.22 & 4.97 & 
716.70 & 5.86 & 100\% & \bf{712.89} & 
0.47 & 0.53\\CON8-3 & 814.50 & 5.35 & 
816.32 & 5.79 & 100\% & \bf{811.07} & 
0.42 & 0.65\\CON8-4 & 772.76 & 5.75 & 
778.10 & 5.22 & 100\% & \bf{772.25} & 
0.07 & 0.76\\CON8-5 & 761.01 & 4.85 & 
762.18 & 5.71 & 63.33\% & \bf{754.88} & 
0.81 & 0.97\\CON8-6 & 689.56 & 5.60 & 
691.51 & 5.62 & 100\% & \bf{678.92} & 
1.57 & 1.85\\CON8-7 & 814.79 & 5.00 & 
815.74 & 5.43 & 100\% & \bf{811.96} & 
0.35 & 0.47\\CON8-8 & 775.36 & 5.50 & 
780.40 & 5.59 & 96.66\% & \bf{767.53} & 
1.02 & 1.68\\CON8-9 & 815.52 & 5.33 & 
820.70 & 5.76 & 100\% & \bf{809.00} & 
0.81 & 1.45\\\bf{PROM.} & 
\bf{762.31} & \bf{5.46} & \bf{764.62} & \bf{5.52} & - & \bf{758.54} & \bf{0.44} & \bf{0.74}\\[1ex]\hline
\end{tabular}
\label{table:GA-M-porcentajeD}
\end{table} \clearpage

\begin{table}[ht]
\caption{Resultados de la ejecución de la metaheurística GA-M, utilizando instancias de Dethloff con la configuración -n 200 -p 40 -cprob 10.0 -mprob 10.0}
\centering
\small
\begin{tabular}{c c c c c c c c}
\hline\hline
Instancia & Costo mínimo & Tiempo(seg.) & Costo promedio & Tiempo promedio(seg.) & CME & \%G & \%GP \\ [0.5ex]
\hline
SCA3-0 & 640.55 & 0.45 & 
641.40 & 0.43 & \bf{635.62} & 
0.78 & 0.91\\SCA3-1 & 701.78 & 0.55 & 
703.34 & 0.55 & \bf{697.84} & 
0.56 & 0.79\\SCA3-2 & \bf{659.34} & 0.40 & 
660.24 & 0.41 & 659.34 & 0.00
 & 0.14\\SCA3-3 & 685.47 & 0.41 & 
685.47 & 0.42 & \bf{680.04} & 
0.80 & 0.80\\SCA3-4 & \bf{690.50} & 0.40 & 
690.50 & 0.43 & 690.50 & 0.00
 & 0.00\\
SCA3-5 & 679.27 & 0.38 & 
681.16 & 0.39 & \bf{659.90} & 
2.94 & 3.22\\SCA3-6 & 652.94 & 0.44 & 
652.94 & 0.47 & \bf{651.09} & 
0.28 & 0.28\\SCA3-7 & 667.24 & 0.71 & 
667.24 & 0.58 & \bf{659.17} & 
1.22 & 1.22\\SCA3-8 & 726.57 & 0.56 & 
726.57 & 0.60 & \bf{719.47} & 
0.99 & 0.99\\SCA3-9 & \bf{681.00} & 0.44 & 
681.00 & 0.42 & 681.00 & 0.00
 & 0.00\\
SCA8-0 & 1007.19 & 0.49 & 
1007.19 & 0.45 & \bf{961.50} & 
4.75 & 4.75\\SCA8-1 & 1074.57 & 0.40 & 
1083.59 & 0.45 & \bf{1049.65} & 
2.37 & 3.23\\SCA8-2 & 1054.47 & 0.57 & 
1054.53 & 0.54 & \bf{1039.64} & 
1.43 & 1.43\\SCA8-3 & 1025.19 & 0.41 & 
1028.83 & 0.41 & \bf{983.34} & 
4.26 & 4.63\\SCA8-4 & 1080.08 & 0.41 & 
1094.22 & 0.39 & \bf{1065.49} & 
1.37 & 2.70\\SCA8-5 & 1057.68 & 0.39 & 
1058.36 & 0.47 & \bf{1027.08} & 
2.98 & 3.05\\SCA8-6 & 976.74 & 0.46 & 
976.74 & 0.45 & \bf{971.82} & 
0.51 & 0.51\\SCA8-7 & 1078.81 & 0.37 & 
1078.81 & 0.41 & \bf{1051.28} & 
2.62 & 2.62\\SCA8-8 & 1087.21 & 0.49 & 
1088.38 & 0.45 & \bf{1071.18} & 
1.50 & 1.61\\SCA8-9 & 1078.49 & 0.65 & 
1078.49 & 0.47 & \bf{1060.50} & 
1.70 & 1.70\\CON3-0 & 620.76 & 0.73 & 
620.76 & 0.51 & \bf{616.52} & 
0.69 & 0.69\\CON3-1 & 556.04 & 0.44 & 
562.04 & 0.46 & \bf{554.47} & 
0.28 & 1.37\\CON3-2 & 521.38 & 0.47 & 
521.38 & 0.48 & \bf{518.00} & 
0.65 & 0.65\\CON3-3 & 592.57 & 0.41 & 
592.76 & 0.41 & \bf{591.19} & 
0.23 & 0.27\\CON3-4 & 591.43 & 0.42 & 
594.82 & 0.41 & \bf{588.79} & 
0.45 & 1.02\\CON3-5 & 568.76 & 0.58 & 
569.60 & 0.48 & \bf{563.70} & 
0.90 & 1.05\\CON3-6 & 504.20 & 0.57 & 
507.52 & 0.57 & \bf{499.05} & 
1.03 & 1.70\\CON3-7 & \bf{576.48} & 0.55 & 
578.92 & 0.50 & 576.48 & 0.00
 & 0.42\\CON3-8 & \bf{523.05} & 0.50 & 
525.82 & 0.47 & 523.05 & 0.00
 & 0.53\\CON3-9 & 590.67 & 0.46 & 
591.01 & 0.46 & \bf{578.24} & 
2.15 & 2.21\\CON8-0 & 857.40 & 0.45 & 
878.45 & 0.46 & \bf{857.17} & 
0.03 & 2.48\\CON8-1 & 760.08 & 0.49 & 
760.08 & 0.46 & \bf{740.85} & 
2.60 & 2.60\\CON8-2 & 725.93 & 0.50 & 
731.18 & 0.50 & \bf{712.89} & 
1.83 & 2.57\\CON8-3 & 833.50 & 0.59 & 
836.37 & 0.47 & \bf{811.07} & 
2.77 & 3.12\\CON8-4 & 780.34 & 0.49 & 
780.34 & 0.48 & \bf{772.25} & 
1.05 & 1.05\\CON8-5 & 763.90 & 0.44 & 
763.90 & 0.42 & \bf{754.88} & 
1.19 & 1.19\\CON8-6 & 684.69 & 0.44 & 
705.55 & 0.46 & \bf{678.92} & 
0.85 & 3.92\\CON8-7 & 821.28 & 0.39 & 
821.86 & 0.48 & \bf{811.96} & 
1.15 & 1.22\\CON8-8 & 793.77 & 0.42 & 
797.03 & 0.41 & \bf{767.53} & 
3.42 & 3.84\\CON8-9 & 819.88 & 0.40 & 
819.88 & 0.46 & \bf{809.00} & 
1.34 & 1.34\\\bf{PROM.} & 
\bf{769.78} & \bf{0.48} & \bf{772.46} & \bf{0.46} & \bf{758.54} & \bf{1.34} & \bf{1.69}\\[1ex]\hline
\end{tabular}
\label{table:GA-M-10-10}
\end{table} 

\begin{table}[ht]
\caption{Resultados de la ejecución de la metaheurística GA-M, utilizando instancias de Dethloff con la configuración -n 200 -p 40 -cprob 10.0 -mprob 70.0}
\centering
\small
\begin{tabular}{c c c c c c c c}
\hline\hline
Instancia & Costo mínimo & Tiempo(seg.) & Costo promedio & Tiempo promedio(seg.) & CME & \%G & \%GP \\ [0.5ex]
\hline
SCA3-0 & 640.55 & 0.66 & 
640.84 & 0.51 & \bf{635.62} & 
0.78 & 0.82\\SCA3-1 & 701.53 & 0.52 & 
701.53 & 0.48 & \bf{697.84} & 
0.53 & 0.53\\SCA3-2 & 666.19 & 0.41 & 
669.90 & 0.46 & \bf{659.34} & 
1.04 & 1.60\\SCA3-3 & 681.16 & 0.42 & 
681.21 & 0.55 & \bf{680.04} & 
0.16 & 0.17\\SCA3-4 & \bf{690.50} & 0.39 & 
690.50 & 0.42 & 690.50 & 0.00
 & 0.00\\
SCA3-5 & 677.56 & 0.45 & 
679.27 & 0.45 & \bf{659.90} & 
2.68 & 2.94\\SCA3-6 & 652.94 & 0.44 & 
654.00 & 0.47 & \bf{651.09} & 
0.28 & 0.45\\SCA3-7 & 666.15 & 0.50 & 
666.38 & 0.45 & \bf{659.17} & 
1.06 & 1.09\\SCA3-8 & 724.29 & 0.92 & 
727.71 & 0.53 & \bf{719.47} & 
0.67 & 1.14\\SCA3-9 & \bf{681.00} & 0.41 & 
682.03 & 0.41 & 681.00 & 0.00
 & 0.15\\SCA8-0 & 987.26 & 0.45 & 
987.26 & 0.57 & \bf{961.50} & 
2.68 & 2.68\\SCA8-1 & 1095.15 & 0.41 & 
1095.15 & 0.40 & \bf{1049.65} & 
4.33 & 4.33\\SCA8-2 & 1053.59 & 0.40 & 
1053.59 & 0.40 & \bf{1039.64} & 
1.34 & 1.34\\SCA8-3 & 1014.10 & 0.49 & 
1014.10 & 0.47 & \bf{983.34} & 
3.13 & 3.13\\SCA8-4 & 1074.18 & 0.39 & 
1074.18 & 0.42 & \bf{1065.49} & 
0.82 & 0.82\\SCA8-5 & 1038.26 & 0.66 & 
1053.77 & 0.55 & \bf{1027.08} & 
1.09 & 2.60\\SCA8-6 & 982.65 & 0.38 & 
985.85 & 0.52 & \bf{971.82} & 
1.11 & 1.44\\SCA8-7 & 1084.88 & 0.37 & 
1084.88 & 0.51 & \bf{1051.28} & 
3.20 & 3.20\\SCA8-8 & 1089.91 & 0.44 & 
1089.91 & 0.53 & \bf{1071.18} & 
1.75 & 1.75\\SCA8-9 & 1067.27 & 0.38 & 
1067.27 & 0.44 & \bf{1060.50} & 
0.64 & 0.64\\CON3-0 & 628.47 & 0.46 & 
631.54 & 0.50 & \bf{616.52} & 
1.94 & 2.44\\CON3-1 & 558.67 & 0.46 & 
560.23 & 0.45 & \bf{554.47} & 
0.76 & 1.04\\CON3-2 & 521.38 & 0.63 & 
522.30 & 0.53 & \bf{518.00} & 
0.65 & 0.83\\CON3-3 & 591.20 & 0.41 & 
597.07 & 0.43 & \bf{591.19} & 
0.00 & 0.99\\CON3-4 & 592.58 & 0.40 & 
597.47 & 0.46 & \bf{588.79} & 
0.64 & 1.47\\CON3-5 & \bf{563.70} & 0.44 & 
567.85 & 0.45 & 563.70 & 0.00
 & 0.74\\CON3-6 & 502.16 & 0.48 & 
504.76 & 0.47 & \bf{499.05} & 
0.62 & 1.14\\CON3-7 & 578.41 & 0.41 & 
588.70 & 0.40 & \bf{576.48} & 
0.33 & 2.12\\CON3-8 & 525.30 & 0.57 & 
532.36 & 0.48 & \bf{523.05} & 
0.43 & 1.78\\CON3-9 & 588.40 & 0.49 & 
589.07 & 0.46 & \bf{578.24} & 
1.76 & 1.87\\CON8-0 & 891.49 & 0.58 & 
891.49 & 0.49 & \bf{857.17} & 
4.00 & 4.00\\CON8-1 & 760.30 & 0.46 & 
767.24 & 0.55 & \bf{740.85} & 
2.63 & 3.56\\CON8-2 & 718.80 & 0.56 & 
722.15 & 0.56 & \bf{712.89} & 
0.83 & 1.30\\CON8-3 & 832.93 & 0.46 & 
833.39 & 0.53 & \bf{811.07} & 
2.70 & 2.75\\CON8-4 & 787.50 & 0.38 & 
792.90 & 0.39 & \bf{772.25} & 
1.97 & 2.67\\CON8-5 & 763.13 & 0.47 & 
763.13 & 0.47 & \bf{754.88} & 
1.09 & 1.09\\CON8-6 & 700.80 & 0.62 & 
700.84 & 0.52 & \bf{678.92} & 
3.22 & 3.23\\CON8-7 & 816.18 & 0.77 & 
826.36 & 0.49 & \bf{811.96} & 
0.52 & 1.77\\CON8-8 & 787.58 & 0.49 & 
787.58 & 0.48 & \bf{767.53} & 
2.61 & 2.61\\CON8-9 & 833.49 & 0.42 & 
838.12 & 0.47 & \bf{809.00} & 
3.03 & 3.60\\\bf{PROM.} & 
\bf{770.29} & \bf{0.49} & \bf{772.85} & \bf{0.48} & \bf{758.54} & \bf{1.43} & \bf{1.80}\\[1ex]\hline
\end{tabular}
\label{table:nonlin}
\end{table}

\begin{table}[ht]
\caption{Resultados de la ejecución de la metaheurística GA-M, utilizando instancias de Dethloff con la configuración -n 200 -p 40 -cprob 10.0 -mprob 100.0}
\centering
\small
\begin{tabular}{c c c c c c c c}
\hline\hline
Instancia & Costo mínimo & Tiempo(seg.) & Costo promedio & Tiempo promedio(seg.) & CME & \%G & \%GP \\ [0.5ex]
\hline
SCA3-0 & 640.55 & 0.54 & 
640.55 & 0.48 & \bf{635.62} & 
0.78 & 0.78\\SCA3-1 & 701.53 & 0.93 & 
701.53 & 0.61 & \bf{697.84} & 
0.53 & 0.53\\SCA3-2 & 668.65 & 0.42 & 
668.86 & 0.43 & \bf{659.34} & 
1.41 & 1.44\\SCA3-3 & 681.35 & 0.44 & 
685.11 & 0.43 & \bf{680.04} & 
0.19 & 0.75\\SCA3-4 & \bf{690.50} & 0.46 & 
690.50 & 0.43 & 690.50 & 0.00
 & 0.00\\
SCA3-5 & 668.48 & 0.57 & 
669.70 & 0.53 & \bf{659.90} & 
1.30 & 1.48\\SCA3-6 & 660.26 & 0.90 & 
660.26 & 0.67 & \bf{651.09} & 
1.41 & 1.41\\SCA3-7 & 671.77 & 0.45 & 
672.00 & 0.46 & \bf{659.17} & 
1.91 & 1.95\\SCA3-8 & 719.77 & 0.48 & 
725.70 & 0.50 & \bf{719.47} & 
0.04 & 0.87\\SCA3-9 & \bf{681.00} & 0.46 & 
681.00 & 0.43 & 681.00 & 0.00
 & 0.00\\
SCA8-0 & 976.98 & 0.41 & 
986.80 & 0.52 & \bf{961.50} & 
1.61 & 2.63\\SCA8-1 & 1078.78 & 0.47 & 
1081.80 & 0.44 & \bf{1049.65} & 
2.78 & 3.06\\SCA8-2 & 1054.85 & 0.41 & 
1054.85 & 0.60 & \bf{1039.64} & 
1.46 & 1.46\\SCA8-3 & 1024.73 & 0.44 & 
1024.87 & 0.43 & \bf{983.34} & 
4.21 & 4.22\\SCA8-4 & 1092.53 & 0.38 & 
1092.53 & 0.47 & \bf{1065.49} & 
2.54 & 2.54\\SCA8-5 & 1060.65 & 0.41 & 
1060.65 & 0.45 & \bf{1027.08} & 
3.27 & 3.27\\SCA8-6 & 981.24 & 0.40 & 
983.65 & 0.47 & \bf{971.82} & 
0.97 & 1.22\\SCA8-7 & 1075.21 & 0.47 & 
1075.21 & 0.47 & \bf{1051.28} & 
2.28 & 2.28\\SCA8-8 & 1090.39 & 0.44 & 
1090.39 & 0.54 & \bf{1071.18} & 
1.79 & 1.79\\SCA8-9 & 1081.16 & 0.38 & 
1081.16 & 0.41 & \bf{1060.50} & 
1.95 & 1.95\\CON3-0 & 624.96 & 0.45 & 
631.68 & 0.44 & \bf{616.52} & 
1.37 & 2.46\\CON3-1 & 560.75 & 0.48 & 
562.73 & 0.52 & \bf{554.47} & 
1.13 & 1.49\\CON3-2 & 521.38 & 0.50 & 
521.38 & 0.51 & \bf{518.00} & 
0.65 & 0.65\\CON3-3 & 599.26 & 0.66 & 
599.46 & 0.51 & \bf{591.19} & 
1.37 & 1.40\\CON3-4 & 591.43 & 0.50 & 
593.56 & 0.56 & \bf{588.79} & 
0.45 & 0.81\\CON3-5 & 567.94 & 0.54 & 
567.94 & 0.48 & \bf{563.70} & 
0.75 & 0.75\\CON3-6 & 502.26 & 0.46 & 
504.83 & 0.51 & \bf{499.05} & 
0.64 & 1.16\\CON3-7 & 582.14 & 0.40 & 
586.23 & 0.41 & \bf{576.48} & 
0.98 & 1.69\\CON3-8 & 526.59 & 0.52 & 
532.61 & 0.51 & \bf{523.05} & 
0.68 & 1.83\\CON3-9 & 588.18 & 0.48 & 
588.84 & 0.59 & \bf{578.24} & 
1.72 & 1.83\\CON8-0 & 873.62 & 0.43 & 
884.50 & 0.45 & \bf{857.17} & 
1.92 & 3.19\\CON8-1 & 752.70 & 0.73 & 
768.00 & 0.61 & \bf{740.85} & 
1.60 & 3.66\\CON8-2 & 713.44 & 0.49 & 
715.34 & 0.62 & \bf{712.89} & 
0.08 & 0.34\\CON8-3 & 830.96 & 0.44 & 
830.96 & 0.47 & \bf{811.07} & 
2.45 & 2.45\\CON8-4 & 797.16 & 0.44 & 
797.16 & 0.41 & \bf{772.25} & 
3.23 & 3.23\\CON8-5 & 766.01 & 1.00 & 
768.51 & 0.60 & \bf{754.88} & 
1.47 & 1.81\\CON8-6 & 705.16 & 0.47 & 
705.16 & 0.62 & \bf{678.92} & 
3.86 & 3.86\\CON8-7 & 821.91 & 0.47 & 
822.22 & 0.48 & \bf{811.96} & 
1.23 & 1.26\\CON8-8 & 802.70 & 0.46 & 
803.24 & 0.52 & \bf{767.53} & 
4.58 & 4.65\\CON8-9 & 822.33 & 0.42 & 
823.31 & 0.49 & \bf{809.00} & 
1.65 & 1.77\\\bf{PROM.} & 
\bf{771.28} & \bf{0.51} & \bf{773.37} & \bf{0.50} & \bf{758.54} & \bf{1.56} & \bf{1.85}\\[1ex]\hline
\end{tabular}
\label{table:nonlin}
\end{table} 
\clearpage
\begin{table}[ht]
\caption{Resultados de la ejecución de la metaheurística GA-M, utilizando instancias de Dethloff con la configuración -n 200 -p 40 -cprob 40.0 -mprob 10.0}
\centering
\small
\begin{tabular}{c c c c c c c c}
\hline\hline
Instancia & Costo mínimo & Tiempo(seg.) & Costo promedio & Tiempo promedio(seg.) & CME & \%G & \%GP \\ [0.5ex]
\hline
SCA3-0 & 636.06 & 0.70 & 
640.28 & 0.57 & \bf{635.62} & 
0.07 & 0.73\\SCA3-1 & 700.50 & 0.54 & 
701.01 & 0.66 & \bf{697.84} & 
0.38 & 0.45\\SCA3-2 & 669.06 & 0.52 & 
673.08 & 0.53 & \bf{659.34} & 
1.47 & 2.08\\SCA3-3 & \bf{680.04} & 0.69 & 
680.04 & 0.67 & 680.04 & 0.00
 & 0.00\\
SCA3-4 & \bf{690.50} & 0.67 & 
691.53 & 0.52 & 690.50 & 0.00
 & 0.15\\SCA3-5 & 680.80 & 0.68 & 
680.80 & 0.65 & \bf{659.90} & 
3.17 & 3.17\\SCA3-6 & 652.94 & 0.70 & 
652.94 & 0.65 & \bf{651.09} & 
0.28 & 0.28\\SCA3-7 & 671.77 & 0.47 & 
672.50 & 0.60 & \bf{659.17} & 
1.91 & 2.02\\SCA3-8 & 723.99 & 0.69 & 
724.07 & 0.55 & \bf{719.47} & 
0.63 & 0.64\\SCA3-9 & \bf{681.00} & 0.46 & 
681.00 & 0.53 & 681.00 & 0.00
 & 0.00\\
SCA8-0 & 998.79 & 0.50 & 
1008.02 & 0.80 & \bf{961.50} & 
3.88 & 4.84\\SCA8-1 & 1079.39 & 0.58 & 
1079.39 & 0.59 & \bf{1049.65} & 
2.83 & 2.83\\SCA8-2 & 1054.69 & 0.49 & 
1054.69 & 0.59 & \bf{1039.64} & 
1.45 & 1.45\\SCA8-3 & 1023.67 & 0.39 & 
1023.67 & 0.56 & \bf{983.34} & 
4.10 & 4.10\\SCA8-4 & 1091.52 & 0.57 & 
1091.52 & 0.48 & \bf{1065.49} & 
2.44 & 2.44\\SCA8-5 & 1040.87 & 0.75 & 
1040.87 & 0.67 & \bf{1027.08} & 
1.34 & 1.34\\SCA8-6 & 987.58 & 0.56 & 
989.51 & 0.57 & \bf{971.82} & 
1.62 & 1.82\\SCA8-7 & 1081.07 & 0.40 & 
1081.07 & 0.41 & \bf{1051.28} & 
2.83 & 2.83\\SCA8-8 & 1087.30 & 0.45 & 
1089.65 & 0.53 & \bf{1071.18} & 
1.50 & 1.72\\SCA8-9 & 1075.03 & 0.43 & 
1075.03 & 0.58 & \bf{1060.50} & 
1.37 & 1.37\\CON3-0 & 628.47 & 0.57 & 
628.47 & 0.52 & \bf{616.52} & 
1.94 & 1.94\\CON3-1 & 556.92 & 0.52 & 
560.75 & 0.60 & \bf{554.47} & 
0.44 & 1.13\\CON3-2 & 521.38 & 0.52 & 
524.01 & 0.64 & \bf{518.00} & 
0.65 & 1.16\\CON3-3 & 601.33 & 0.52 & 
601.95 & 0.51 & \bf{591.19} & 
1.72 & 1.82\\CON3-4 & 592.58 & 0.50 & 
597.26 & 0.65 & \bf{588.79} & 
0.64 & 1.44\\CON3-5 & 564.89 & 0.59 & 
566.81 & 0.55 & \bf{563.70} & 
0.21 & 0.55\\CON3-6 & 505.97 & 0.48 & 
508.88 & 0.50 & \bf{499.05} & 
1.39 & 1.97\\CON3-7 & 581.46 & 0.42 & 
583.13 & 0.42 & \bf{576.48} & 
0.86 & 1.15\\CON3-8 & 524.59 & 0.52 & 
524.59 & 0.53 & \bf{523.05} & 
0.29 & 0.29\\CON3-9 & 588.11 & 0.53 & 
588.75 & 0.62 & \bf{578.24} & 
1.71 & 1.82\\CON8-0 & 878.85 & 0.51 & 
878.85 & 0.53 & \bf{857.17} & 
2.53 & 2.53\\CON8-1 & 764.90 & 0.53 & 
764.90 & 0.55 & \bf{740.85} & 
3.25 & 3.25\\CON8-2 & 722.22 & 0.84 & 
722.22 & 0.78 & \bf{712.89} & 
1.31 & 1.31\\CON8-3 & 825.14 & 0.54 & 
833.86 & 0.65 & \bf{811.07} & 
1.73 & 2.81\\CON8-4 & 782.31 & 0.44 & 
794.30 & 0.57 & \bf{772.25} & 
1.30 & 2.86\\CON8-5 & 760.62 & 0.47 & 
760.62 & 0.58 & \bf{754.88} & 
0.76 & 0.76\\CON8-6 & 701.80 & 0.74 & 
702.46 & 0.62 & \bf{678.92} & 
3.37 & 3.47\\CON8-7 & 816.18 & 0.45 & 
820.93 & 0.53 & \bf{811.96} & 
0.52 & 1.11\\CON8-8 & 795.86 & 0.56 & 
796.92 & 0.57 & \bf{767.53} & 
3.69 & 3.83\\CON8-9 & 835.63 & 0.64 & 
839.62 & 0.82 & \bf{809.00} & 
3.29 & 3.78\\\bf{PROM.} & 
\bf{771.39} & \bf{0.55} & \bf{773.25} & \bf{0.59} & \bf{758.54} & \bf{1.57} & \bf{1.83}\\[1ex]\hline
\end{tabular}
\label{table:nonlin}
\end{table} 
\clearpage
\begin{table}[ht]
\caption{Resultados de la ejecución de la metaheurística GA-M, utilizando instancias de Dethloff con la configuración -n 200 -p 40 -cprob 40.0 -mprob 70.0}
\centering
\small
\begin{tabular}{c c c c c c c c}
\hline\hline
Instancia & Costo mínimo & Tiempo(seg.) & Costo promedio & Tiempo promedio(seg.) & CME & \%G & \%GP \\ [0.5ex]
\hline
SCA3-0 & 641.69 & 0.56 & 
641.88 & 0.52 & \bf{635.62} & 
0.95 & 0.98\\SCA3-1 & \bf{697.84} & 0.56 & 
697.84 & 0.82 & 697.84 & 0.00
 & 0.00\\
SCA3-2 & 666.33 & 0.45 & 
669.67 & 0.57 & \bf{659.34} & 
1.06 & 1.57\\SCA3-3 & 681.16 & 0.58 & 
681.40 & 0.58 & \bf{680.04} & 
0.16 & 0.20\\SCA3-4 & \bf{690.50} & 0.42 & 
690.50 & 0.59 & 690.50 & 0.00
 & 0.00\\
SCA3-5 & 673.56 & 0.72 & 
673.56 & 0.63 & \bf{659.90} & 
2.07 & 2.07\\SCA3-6 & 656.23 & 0.65 & 
659.43 & 0.57 & \bf{651.09} & 
0.79 & 1.28\\SCA3-7 & 664.88 & 0.52 & 
665.83 & 0.66 & \bf{659.17} & 
0.87 & 1.01\\SCA3-8 & 728.24 & 0.76 & 
729.24 & 0.62 & \bf{719.47} & 
1.22 & 1.36\\SCA3-9 & \bf{681.00} & 0.92 & 
681.17 & 0.80 & 681.00 & 0.00
 & 0.02\\SCA8-0 & 973.22 & 0.63 & 
976.04 & 0.56 & \bf{961.50} & 
1.22 & 1.51\\SCA8-1 & 1065.38 & 1.02 & 
1079.83 & 0.67 & \bf{1049.65} & 
1.50 & 2.88\\SCA8-2 & 1051.60 & 0.56 & 
1053.14 & 0.63 & \bf{1039.64} & 
1.15 & 1.30\\SCA8-3 & 1002.86 & 0.66 & 
1002.86 & 0.79 & \bf{983.34} & 
1.99 & 1.99\\SCA8-4 & 1104.93 & 0.95 & 
1109.41 & 0.67 & \bf{1065.49} & 
3.70 & 4.12\\SCA8-5 & 1037.54 & 0.98 & 
1043.76 & 0.75 & \bf{1027.08} & 
1.02 & 1.62\\SCA8-6 & 979.60 & 0.69 & 
983.10 & 0.66 & \bf{971.82} & 
0.80 & 1.16\\SCA8-7 & 1075.42 & 0.44 & 
1083.66 & 0.58 & \bf{1051.28} & 
2.30 & 3.08\\SCA8-8 & 1075.00 & 0.45 & 
1075.00 & 0.59 & \bf{1071.18} & 
0.36 & 0.36\\SCA8-9 & 1076.67 & 0.40 & 
1076.67 & 0.47 & \bf{1060.50} & 
1.52 & 1.52\\CON3-0 & 625.14 & 0.60 & 
627.60 & 0.72 & \bf{616.52} & 
1.40 & 1.80\\CON3-1 & 557.38 & 0.72 & 
559.07 & 0.65 & \bf{554.47} & 
0.52 & 0.83\\CON3-2 & 521.38 & 0.80 & 
521.38 & 0.71 & \bf{518.00} & 
0.65 & 0.65\\CON3-3 & 591.20 & 0.68 & 
597.53 & 0.68 & \bf{591.19} & 
0.00 & 1.07\\CON3-4 & 592.58 & 0.62 & 
596.88 & 0.69 & \bf{588.79} & 
0.64 & 1.37\\CON3-5 & 566.96 & 0.49 & 
568.16 & 0.68 & \bf{563.70} & 
0.58 & 0.79\\CON3-6 & 504.15 & 0.71 & 
504.80 & 0.68 & \bf{499.05} & 
1.02 & 1.15\\CON3-7 & 582.14 & 0.47 & 
582.14 & 0.55 & \bf{576.48} & 
0.98 & 0.98\\CON3-8 & 523.14 & 0.54 & 
531.60 & 0.69 & \bf{523.05} & 
0.02 & 1.63\\CON3-9 & 590.39 & 0.80 & 
591.39 & 0.81 & \bf{578.24} & 
2.10 & 2.27\\CON8-0 & 874.78 & 0.52 & 
875.55 & 0.66 & \bf{857.17} & 
2.05 & 2.14\\CON8-1 & 768.44 & 0.60 & 
772.53 & 0.79 & \bf{740.85} & 
3.72 & 4.28\\CON8-2 & 717.31 & 0.51 & 
717.34 & 0.60 & \bf{712.89} & 
0.62 & 0.62\\CON8-3 & 822.54 & 0.70 & 
822.54 & 0.77 & \bf{811.07} & 
1.41 & 1.41\\CON8-4 & 784.78 & 0.96 & 
784.78 & 0.79 & \bf{772.25} & 
1.62 & 1.62\\CON8-5 & 762.01 & 0.43 & 
762.01 & 0.61 & \bf{754.88} & 
0.94 & 0.94\\CON8-6 & 688.93 & 0.99 & 
697.49 & 0.78 & \bf{678.92} & 
1.47 & 2.73\\CON8-7 & 822.83 & 0.46 & 
825.37 & 0.56 & \bf{811.96} & 
1.34 & 1.65\\CON8-8 & 785.64 & 0.52 & 
791.52 & 0.76 & \bf{767.53} & 
2.36 & 3.13\\CON8-9 & 818.99 & 0.42 & 
822.87 & 0.56 & \bf{809.00} & 
1.23 & 1.71\\\bf{PROM.} & 
\bf{768.11} & \bf{0.64} & \bf{770.66} & \bf{0.66} & \bf{758.54} & \bf{1.18} & \bf{1.52}\\[1ex]\hline
\end{tabular}
\label{table:nonlin}
\end{table} 
\clearpage
\begin{table}[ht]
\caption{Resultados de la ejecución de la metaheurística GA-M, utilizando instancias de Dethloff con la configuración -n 200 -p 40 -cprob 40.0 -mprob 100.0}
\centering
\small
\begin{tabular}{c c c c c c c c}
\hline\hline
Instancia & Costo mínimo & Tiempo(seg.) & Costo promedio & Tiempo promedio(seg.) & CME & \%G & \%GP \\ [0.5ex]
\hline
SCA3-0 & 640.55 & 0.82 & 
640.55 & 0.83 & \bf{635.62} & 
0.78 & 0.78\\SCA3-1 & 707.07 & 0.66 & 
707.32 & 0.64 & \bf{697.84} & 
1.32 & 1.36\\SCA3-2 & 664.18 & 0.54 & 
664.64 & 0.60 & \bf{659.34} & 
0.73 & 0.80\\SCA3-3 & \bf{680.04} & 0.72 & 
681.29 & 0.70 & 680.04 & 0.00
 & 0.18\\SCA3-4 & \bf{690.50} & 0.53 & 
692.55 & 0.64 & 690.50 & 0.00
 & 0.30\\SCA3-5 & \bf{659.90} & 0.59 & 
662.47 & 0.79 & 659.90 & 0.00
 & 0.39\\SCA3-6 & \bf{651.09} & 0.63 & 
655.30 & 0.76 & 651.09 & 0.00
 & 0.65\\SCA3-7 & 666.15 & 0.65 & 
666.15 & 0.64 & \bf{659.17} & 
1.06 & 1.06\\SCA3-8 & 719.77 & 0.57 & 
721.47 & 0.63 & \bf{719.47} & 
0.04 & 0.28\\SCA3-9 & \bf{681.00} & 0.51 & 
681.00 & 0.66 & 681.00 & 0.00
 & 0.00\\
SCA8-0 & 972.59 & 0.56 & 
982.18 & 0.61 & \bf{961.50} & 
1.15 & 2.15\\SCA8-1 & 1074.65 & 0.69 & 
1074.90 & 0.61 & \bf{1049.65} & 
2.38 & 2.41\\SCA8-2 & 1050.37 & 0.50 & 
1050.37 & 0.66 & \bf{1039.64} & 
1.03 & 1.03\\SCA8-3 & 1022.58 & 0.56 & 
1023.37 & 0.54 & \bf{983.34} & 
3.99 & 4.07\\SCA8-4 & 1074.87 & 0.49 & 
1075.76 & 0.79 & \bf{1065.49} & 
0.88 & 0.96\\SCA8-5 & 1045.69 & 0.53 & 
1062.05 & 0.55 & \bf{1027.08} & 
1.81 & 3.40\\SCA8-6 & 976.74 & 0.68 & 
982.43 & 0.69 & \bf{971.82} & 
0.51 & 1.09\\SCA8-7 & 1075.60 & 0.51 & 
1075.60 & 0.51 & \bf{1051.28} & 
2.31 & 2.31\\SCA8-8 & 1084.41 & 0.68 & 
1084.41 & 0.77 & \bf{1071.18} & 
1.24 & 1.24\\SCA8-9 & 1077.86 & 0.95 & 
1091.96 & 0.89 & \bf{1060.50} & 
1.64 & 2.97\\CON3-0 & 620.76 & 0.94 & 
630.75 & 0.64 & \bf{616.52} & 
0.69 & 2.31\\CON3-1 & 557.21 & 0.95 & 
562.96 & 0.67 & \bf{554.47} & 
0.49 & 1.53\\CON3-2 & 521.38 & 1.02 & 
522.26 & 0.90 & \bf{518.00} & 
0.65 & 0.82\\CON3-3 & 598.45 & 0.52 & 
600.79 & 0.72 & \bf{591.19} & 
1.23 & 1.62\\CON3-4 & 595.25 & 0.57 & 
600.20 & 0.70 & \bf{588.79} & 
1.10 & 1.94\\CON3-5 & 564.88 & 0.62 & 
565.64 & 0.65 & \bf{563.70} & 
0.21 & 0.35\\CON3-6 & 503.97 & 1.03 & 
504.55 & 0.73 & \bf{499.05} & 
0.99 & 1.10\\CON3-7 & 578.22 & 0.65 & 
578.27 & 0.58 & \bf{576.48} & 
0.30 & 0.31\\CON3-8 & 524.38 & 0.96 & 
529.30 & 0.77 & \bf{523.05} & 
0.25 & 1.19\\CON3-9 & 590.16 & 0.61 & 
590.25 & 0.69 & \bf{578.24} & 
2.06 & 2.08\\CON8-0 & 879.38 & 1.01 & 
889.84 & 0.94 & \bf{857.17} & 
2.59 & 3.81\\CON8-1 & 771.70 & 1.00 & 
771.70 & 0.90 & \bf{740.85} & 
4.16 & 4.16\\CON8-2 & 716.19 & 0.73 & 
717.46 & 0.87 & \bf{712.89} & 
0.46 & 0.64\\CON8-3 & 831.73 & 0.55 & 
835.04 & 0.69 & \bf{811.07} & 
2.55 & 2.96\\CON8-4 & 790.59 & 1.02 & 
790.73 & 0.78 & \bf{772.25} & 
2.37 & 2.39\\CON8-5 & 762.61 & 1.02 & 
762.61 & 0.77 & \bf{754.88} & 
1.02 & 1.02\\CON8-6 & 698.19 & 0.85 & 
704.55 & 0.85 & \bf{678.92} & 
2.84 & 3.78\\CON8-7 & 817.98 & 0.68 & 
824.23 & 0.59 & \bf{811.96} & 
0.74 & 1.51\\CON8-8 & 784.32 & 0.71 & 
786.37 & 0.62 & \bf{767.53} & 
2.19 & 2.45\\CON8-9 & 830.79 & 0.94 & 
834.54 & 0.72 & \bf{809.00} & 
2.69 & 3.16\\\bf{PROM.} & 
\bf{768.84} & \bf{0.72} & \bf{771.94} & \bf{0.71} & \bf{758.54} & \bf{1.26} & \bf{1.66}\\[1ex]\hline
\end{tabular}
\label{table:nonlin}
\end{table} 
\clearpage
\begin{table}[ht]
\caption{Resultados de la ejecución de la metaheurística GA-M, utilizando instancias de Dethloff con la configuración -n 200 -p 40 -cprob 90.0 -mprob 10.0}
\centering
\small
\begin{tabular}{c c c c c c c c}
\hline\hline
Instancia & Costo mínimo & Tiempo(seg.) & Costo promedio & Tiempo promedio(seg.) & CME & \%G & \%GP \\ [0.5ex]
\hline
SCA3-0 & 640.55 & 0.71 & 
641.31 & 0.71 & \bf{635.62} & 
0.78 & 0.89\\SCA3-1 & 700.50 & 0.85 & 
700.50 & 0.74 & \bf{697.84} & 
0.38 & 0.38\\SCA3-2 & 674.45 & 0.68 & 
674.95 & 0.69 & \bf{659.34} & 
2.29 & 2.37\\SCA3-3 & \bf{680.04} & 0.72 & 
680.65 & 0.72 & 680.04 & 0.00
 & 0.09\\SCA3-4 & \bf{690.50} & 0.69 & 
690.50 & 0.69 & 690.50 & 0.00
 & 0.00\\
SCA3-5 & 662.75 & 0.74 & 
676.29 & 0.73 & \bf{659.90} & 
0.43 & 2.48\\SCA3-6 & 652.94 & 0.69 & 
652.94 & 0.71 & \bf{651.09} & 
0.28 & 0.28\\SCA3-7 & 667.24 & 0.68 & 
670.64 & 0.70 & \bf{659.17} & 
1.22 & 1.74\\SCA3-8 & 731.10 & 0.70 & 
732.23 & 0.71 & \bf{719.47} & 
1.62 & 1.77\\SCA3-9 & \bf{681.00} & 0.70 & 
683.03 & 0.70 & 681.00 & 0.00
 & 0.30\\SCA8-0 & 999.14 & 0.68 & 
1011.17 & 0.73 & \bf{961.50} & 
3.91 & 5.17\\SCA8-1 & 1054.87 & 0.76 & 
1059.72 & 0.75 & \bf{1049.65} & 
0.50 & 0.96\\SCA8-2 & 1053.59 & 0.74 & 
1053.59 & 0.84 & \bf{1039.64} & 
1.34 & 1.34\\SCA8-3 & 1014.58 & 1.03 & 
1014.58 & 0.94 & \bf{983.34} & 
3.18 & 3.18\\SCA8-4 & 1077.80 & 0.74 & 
1077.98 & 0.76 & \bf{1065.49} & 
1.16 & 1.17\\SCA8-5 & 1058.57 & 0.58 & 
1059.61 & 0.73 & \bf{1027.08} & 
3.07 & 3.17\\SCA8-6 & 978.03 & 0.77 & 
978.03 & 0.84 & \bf{971.82} & 
0.64 & 0.64\\SCA8-7 & 1089.40 & 0.74 & 
1101.11 & 0.72 & \bf{1051.28} & 
3.63 & 4.74\\SCA8-8 & 1092.02 & 0.69 & 
1092.02 & 0.66 & \bf{1071.18} & 
1.95 & 1.95\\SCA8-9 & 1074.19 & 0.54 & 
1074.19 & 0.73 & \bf{1060.50} & 
1.29 & 1.29\\CON3-0 & 617.59 & 0.94 & 
622.33 & 0.78 & \bf{616.52} & 
0.17 & 0.94\\CON3-1 & 561.87 & 0.77 & 
562.20 & 0.69 & \bf{554.47} & 
1.33 & 1.39\\CON3-2 & 521.38 & 0.78 & 
522.32 & 0.78 & \bf{518.00} & 
0.65 & 0.83\\CON3-3 & 591.20 & 0.83 & 
596.88 & 0.76 & \bf{591.19} & 
0.00 & 0.96\\CON3-4 & 592.58 & 0.75 & 
593.38 & 0.75 & \bf{588.79} & 
0.64 & 0.78\\CON3-5 & 568.76 & 0.74 & 
570.30 & 0.72 & \bf{563.70} & 
0.90 & 1.17\\CON3-6 & 504.44 & 0.76 & 
504.79 & 0.76 & \bf{499.05} & 
1.08 & 1.15\\CON3-7 & 578.22 & 0.70 & 
581.35 & 0.71 & \bf{576.48} & 
0.30 & 0.85\\CON3-8 & \bf{523.05} & 0.80 & 
527.00 & 0.77 & 523.05 & 0.00
 & 0.75\\CON3-9 & 588.11 & 0.76 & 
588.64 & 0.71 & \bf{578.24} & 
1.71 & 1.80\\CON8-0 & 870.49 & 0.60 & 
870.49 & 0.70 & \bf{857.17} & 
1.55 & 1.55\\CON8-1 & 755.30 & 0.60 & 
757.18 & 0.72 & \bf{740.85} & 
1.95 & 2.20\\CON8-2 & 727.69 & 0.82 & 
728.10 & 0.87 & \bf{712.89} & 
2.08 & 2.13\\CON8-3 & 831.59 & 0.73 & 
833.33 & 0.76 & \bf{811.07} & 
2.53 & 2.74\\CON8-4 & 793.13 & 0.73 & 
801.38 & 0.75 & \bf{772.25} & 
2.70 & 3.77\\CON8-5 & 761.01 & 0.81 & 
761.01 & 0.78 & \bf{754.88} & 
0.81 & 0.81\\CON8-6 & 701.09 & 0.83 & 
701.09 & 0.80 & \bf{678.92} & 
3.27 & 3.27\\CON8-7 & 830.80 & 0.80 & 
832.28 & 0.74 & \bf{811.96} & 
2.32 & 2.50\\CON8-8 & 795.08 & 0.78 & 
795.08 & 0.84 & \bf{767.53} & 
3.59 & 3.59\\CON8-9 & 824.39 & 0.74 & 
834.81 & 0.76 & \bf{809.00} & 
1.90 & 3.19\\\bf{PROM.} & 
\bf{770.28} & \bf{0.74} & \bf{772.72} & \bf{0.75} & \bf{758.54} & \bf{1.43} & \bf{1.76}\\[1ex]\hline
\end{tabular}
\label{table:nonlin}
\end{table} 
\clearpage
\begin{table}[ht]
\caption{Resultados de la ejecución de la metaheurística GA-M, utilizando instancias de Dethloff con la configuración -n 200 -p 40 -cprob 90.0 -mprob 70.0}
\centering
\small
\begin{tabular}{c c c c c c c c}
\hline\hline
Instancia & Costo mínimo & Tiempo(seg.) & Costo promedio & Tiempo promedio(seg.) & CME & \%G & \%GP \\ [0.5ex]
\hline
SCA3-0 & 641.64 & 0.96 & 
641.64 & 0.94 & \bf{635.62} & 
0.95 & 0.95\\SCA3-1 & 700.50 & 0.77 & 
700.76 & 0.76 & \bf{697.84} & 
0.38 & 0.42\\SCA3-2 & 664.21 & 0.68 & 
664.21 & 0.79 & \bf{659.34} & 
0.74 & 0.74\\SCA3-3 & \bf{680.04} & 0.70 & 
683.51 & 0.71 & 680.04 & 0.00
 & 0.51\\SCA3-4 & \bf{690.50} & 0.89 & 
691.70 & 0.84 & 690.50 & 0.00
 & 0.17\\SCA3-5 & 673.56 & 0.91 & 
673.56 & 0.81 & \bf{659.90} & 
2.07 & 2.07\\SCA3-6 & 652.94 & 0.70 & 
653.81 & 0.85 & \bf{651.09} & 
0.28 & 0.42\\SCA3-7 & 666.15 & 0.74 & 
668.73 & 0.73 & \bf{659.17} & 
1.06 & 1.45\\SCA3-8 & 724.29 & 0.95 & 
729.85 & 0.90 & \bf{719.47} & 
0.67 & 1.44\\SCA3-9 & \bf{681.00} & 0.84 & 
681.00 & 0.77 & 681.00 & 0.00
 & 0.00\\
SCA8-0 & 982.36 & 0.96 & 
982.36 & 0.92 & \bf{961.50} & 
2.17 & 2.17\\SCA8-1 & 1062.35 & 1.04 & 
1062.35 & 0.96 & \bf{1049.65} & 
1.21 & 1.21\\SCA8-2 & 1069.86 & 0.98 & 
1070.50 & 0.84 & \bf{1039.64} & 
2.91 & 2.97\\SCA8-3 & 1000.75 & 0.60 & 
1005.91 & 0.75 & \bf{983.34} & 
1.77 & 2.30\\SCA8-4 & 1067.66 & 0.47 & 
1067.66 & 0.62 & \bf{1065.49} & 
0.20 & 0.20\\SCA8-5 & 1060.37 & 0.98 & 
1060.37 & 0.93 & \bf{1027.08} & 
3.24 & 3.24\\SCA8-6 & \bf{971.82} & 0.97 & 
976.14 & 0.92 & 971.82 & 0.00
 & 0.44\\SCA8-7 & 1079.05 & 0.67 & 
1079.05 & 0.84 & \bf{1051.28} & 
2.64 & 2.64\\SCA8-8 & 1075.00 & 0.97 & 
1075.00 & 0.91 & \bf{1071.18} & 
0.36 & 0.36\\SCA8-9 & 1074.21 & 0.99 & 
1077.26 & 0.90 & \bf{1060.50} & 
1.29 & 1.58\\CON3-0 & 624.84 & 0.79 & 
629.04 & 0.86 & \bf{616.52} & 
1.35 & 2.03\\CON3-1 & 556.92 & 0.94 & 
559.79 & 0.94 & \bf{554.47} & 
0.44 & 0.96\\CON3-2 & 521.38 & 0.96 & 
522.73 & 0.90 & \bf{518.00} & 
0.65 & 0.91\\CON3-3 & \bf{591.19} & 0.75 & 
599.75 & 0.79 & 591.19 & 0.00
 & 1.45\\CON3-4 & 604.71 & 0.92 & 
604.71 & 0.92 & \bf{588.79} & 
2.70 & 2.70\\CON3-5 & 568.66 & 0.72 & 
571.32 & 0.81 & \bf{563.70} & 
0.88 & 1.35\\CON3-6 & 504.15 & 0.79 & 
505.05 & 0.80 & \bf{499.05} & 
1.02 & 1.20\\CON3-7 & 581.27 & 0.62 & 
588.70 & 0.68 & \bf{576.48} & 
0.83 & 2.12\\CON3-8 & 524.30 & 0.82 & 
524.30 & 0.88 & \bf{523.05} & 
0.24 & 0.24\\CON3-9 & 588.11 & 0.81 & 
588.40 & 0.83 & \bf{578.24} & 
1.71 & 1.76\\CON8-0 & 877.38 & 0.96 & 
877.38 & 0.95 & \bf{857.17} & 
2.36 & 2.36\\CON8-1 & 758.03 & 1.05 & 
761.15 & 0.91 & \bf{740.85} & 
2.32 & 2.74\\CON8-2 & 716.26 & 0.88 & 
716.61 & 0.92 & \bf{712.89} & 
0.47 & 0.52\\CON8-3 & 836.01 & 0.98 & 
836.81 & 1.01 & \bf{811.07} & 
3.07 & 3.17\\CON8-4 & 772.76 & 0.76 & 
772.81 & 0.72 & \bf{772.25} & 
0.07 & 0.07\\CON8-5 & 771.15 & 1.02 & 
771.15 & 0.90 & \bf{754.88} & 
2.16 & 2.16\\CON8-6 & 691.83 & 0.86 & 
691.83 & 0.88 & \bf{678.92} & 
1.90 & 1.90\\CON8-7 & 826.02 & 0.83 & 
826.11 & 0.88 & \bf{811.96} & 
1.73 & 1.74\\CON8-8 & 791.19 & 0.96 & 
802.00 & 0.90 & \bf{767.53} & 
3.08 & 4.49\\CON8-9 & 813.67 & 0.94 & 
825.78 & 0.88 & \bf{809.00} & 
0.58 & 2.07\\\bf{PROM.} & 
\bf{768.45} & \bf{0.85} & \bf{770.52} & \bf{0.85} & \bf{758.54} & \bf{1.24} & \bf{1.53}\\[1ex]\hline
\end{tabular}
\label{table:nonlin}
\end{table} 

\begin{table}[ht]
\caption{Resultados de la ejecución de la metaheurística GA-M, utilizando instancias de Dethloff con la configuración -n 200 -p 40 -cprob 90.0 -mprob 100.0}
\centering
\small
\begin{tabular}{c c c c c c c c}
\hline\hline
Instancia & Costo mínimo & Tiempo(seg.) & Costo promedio & Tiempo promedio(seg.) & CME & \%G & \%GP \\ [0.5ex]
\hline
SCA3-0 & 640.55 & 0.92 & 
640.84 & 0.93 & \bf{635.62} & 
0.78 & 0.82\\SCA3-1 & 706.23 & 0.90 & 
706.23 & 0.90 & \bf{697.84} & 
1.20 & 1.20\\SCA3-2 & 661.13 & 0.90 & 
661.89 & 0.77 & \bf{659.34} & 
0.27 & 0.39\\SCA3-3 & 681.35 & 0.92 & 
681.35 & 0.94 & \bf{680.04} & 
0.19 & 0.19\\SCA3-4 & \bf{690.50} & 0.91 & 
690.50 & 0.77 & 690.50 & 0.00
 & 0.00\\
SCA3-5 & 677.56 & 0.94 & 
677.56 & 0.86 & \bf{659.90} & 
2.68 & 2.68\\SCA3-6 & 653.81 & 0.93 & 
654.41 & 0.88 & \bf{651.09} & 
0.42 & 0.51\\SCA3-7 & 672.74 & 0.91 & 
672.74 & 0.86 & \bf{659.17} & 
2.06 & 2.06\\SCA3-8 & 724.29 & 0.98 & 
726.81 & 0.88 & \bf{719.47} & 
0.67 & 1.02\\SCA3-9 & \bf{681.00} & 0.75 & 
683.07 & 0.85 & 681.00 & 0.00
 & 0.30\\SCA8-0 & 1008.50 & 0.70 & 
1009.00 & 0.94 & \bf{961.50} & 
4.89 & 4.94\\SCA8-1 & 1086.98 & 1.01 & 
1086.98 & 0.97 & \bf{1049.65} & 
3.56 & 3.56\\SCA8-2 & 1051.95 & 0.80 & 
1051.95 & 0.96 & \bf{1039.64} & 
1.18 & 1.18\\SCA8-3 & 1019.22 & 0.99 & 
1021.33 & 0.91 & \bf{983.34} & 
3.65 & 3.86\\SCA8-4 & 1098.79 & 0.97 & 
1101.78 & 0.94 & \bf{1065.49} & 
3.13 & 3.41\\SCA8-5 & 1061.90 & 0.91 & 
1061.90 & 0.91 & \bf{1027.08} & 
3.39 & 3.39\\SCA8-6 & 972.48 & 0.98 & 
974.68 & 0.96 & \bf{971.82} & 
0.07 & 0.29\\SCA8-7 & 1073.05 & 0.94 & 
1073.25 & 1.03 & \bf{1051.28} & 
2.07 & 2.09\\SCA8-8 & \bf{1071.18} & 1.00 & 
1076.18 & 0.97 & 1071.18 & 0.00
 & 0.47\\SCA8-9 & 1073.96 & 1.07 & 
1073.96 & 0.81 & \bf{1060.50} & 
1.27 & 1.27\\CON3-0 & 624.96 & 0.94 & 
627.53 & 0.89 & \bf{616.52} & 
1.37 & 1.79\\CON3-1 & 556.92 & 0.93 & 
559.72 & 0.89 & \bf{554.47} & 
0.44 & 0.95\\CON3-2 & 521.38 & 0.79 & 
522.26 & 0.94 & \bf{518.00} & 
0.65 & 0.82\\CON3-3 & 601.26 & 0.75 & 
604.53 & 0.77 & \bf{591.19} & 
1.70 & 2.26\\CON3-4 & 595.25 & 0.96 & 
603.07 & 0.87 & \bf{588.79} & 
1.10 & 2.42\\CON3-5 & \bf{563.70} & 0.95 & 
566.99 & 0.95 & 563.70 & 0.00
 & 0.58\\CON3-6 & 504.44 & 0.95 & 
507.06 & 0.93 & \bf{499.05} & 
1.08 & 1.60\\CON3-7 & 582.12 & 0.94 & 
584.06 & 0.91 & \bf{576.48} & 
0.98 & 1.32\\CON3-8 & 532.86 & 0.98 & 
534.74 & 0.97 & \bf{523.05} & 
1.88 & 2.23\\CON3-9 & 589.73 & 0.94 & 
589.85 & 0.95 & \bf{578.24} & 
1.99 & 2.01\\CON8-0 & 886.27 & 0.96 & 
888.29 & 0.93 & \bf{857.17} & 
3.39 & 3.63\\CON8-1 & 769.90 & 1.02 & 
771.00 & 0.94 & \bf{740.85} & 
3.92 & 4.07\\CON8-2 & 726.07 & 1.02 & 
727.90 & 1.02 & \bf{712.89} & 
1.85 & 2.11\\CON8-3 & 840.01 & 1.03 & 
840.01 & 1.00 & \bf{811.07} & 
3.57 & 3.57\\CON8-4 & 800.50 & 0.95 & 
800.50 & 0.95 & \bf{772.25} & 
3.66 & 3.66\\CON8-5 & 758.12 & 1.01 & 
761.71 & 0.99 & \bf{754.88} & 
0.43 & 0.91\\CON8-6 & 691.83 & 0.97 & 
702.44 & 0.98 & \bf{678.92} & 
1.90 & 3.46\\CON8-7 & 815.72 & 0.85 & 
815.72 & 0.92 & \bf{811.96} & 
0.46 & 0.46\\CON8-8 & 810.83 & 0.87 & 
810.83 & 0.87 & \bf{767.53} & 
5.64 & 5.64\\CON8-9 & 814.45 & 1.00 & 
815.79 & 0.98 & \bf{809.00} & 
0.67 & 0.84\\\bf{PROM.} & 
\bf{772.34} & \bf{0.93} & \bf{774.01} & \bf{0.92} & \bf{758.54} & \bf{1.70} & \bf{1.95}\\[1ex]\hline
\end{tabular}
\label{table:nonlin}
\end{table} 

\begin{table}[ht]
\caption{Resultados de la ejecución de la metaheurística GA-M, utilizando instancias de Dethloff con la configuración -n 100.0 -p 150.0 -cprob 40 -mprob 70}
\centering
\small
\begin{tabular}{c c c c c c c c}
\hline\hline
Instancia & Costo mínimo & Tiempo(seg.) & Costo promedio & Tiempo promedio(seg.) & CME & \%G & \%GP \\ [0.5ex]
\hline
SCA3-0 & 640.55 & 2.34 & 
640.55 & 2.08 & \bf{635.62} & 
0.78 & 0.78\\SCA3-1 & \bf{697.84} & 3.55 & 
699.17 & 2.47 & 697.84 & 0.00
 & 0.19\\SCA3-2 & 664.18 & 1.88 & 
664.29 & 1.70 & \bf{659.34} & 
0.73 & 0.75\\SCA3-3 & 680.60 & 1.69 & 
681.96 & 2.06 & \bf{680.04} & 
0.08 & 0.28\\SCA3-4 & \bf{690.50} & 1.81 & 
690.50 & 2.00 & 690.50 & 0.00
 & 0.00\\
SCA3-5 & 665.64 & 2.40 & 
665.64 & 1.97 & \bf{659.90} & 
0.87 & 0.87\\SCA3-6 & 652.94 & 1.54 & 
652.94 & 1.99 & \bf{651.09} & 
0.28 & 0.28\\SCA3-7 & 664.88 & 1.97 & 
665.20 & 2.09 & \bf{659.17} & 
0.87 & 0.91\\SCA3-8 & \bf{719.47} & 1.78 & 
720.75 & 1.82 & 719.47 & 0.00
 & 0.18\\SCA3-9 & \bf{681.00} & 1.68 & 
681.00 & 1.81 & 681.00 & 0.00
 & 0.00\\
SCA8-0 & 982.18 & 1.89 & 
982.18 & 2.02 & \bf{961.50} & 
2.15 & 2.15\\SCA8-1 & 1053.09 & 1.54 & 
1053.09 & 1.82 & \bf{1049.65} & 
0.33 & 0.33\\SCA8-2 & 1053.78 & 1.96 & 
1053.78 & 1.93 & \bf{1039.64} & 
1.36 & 1.36\\SCA8-3 & 1010.01 & 2.19 & 
1012.10 & 1.87 & \bf{983.34} & 
2.71 & 2.92\\SCA8-4 & 1067.66 & 2.24 & 
1067.66 & 2.08 & \bf{1065.49} & 
0.20 & 0.20\\SCA8-5 & 1053.01 & 1.74 & 
1054.06 & 1.94 & \bf{1027.08} & 
2.52 & 2.63\\SCA8-6 & 976.69 & 1.94 & 
976.69 & 2.08 & \bf{971.82} & 
0.50 & 0.50\\SCA8-7 & 1070.92 & 1.70 & 
1070.92 & 1.56 & \bf{1051.28} & 
1.87 & 1.87\\SCA8-8 & \bf{1071.18} & 1.54 & 
1082.28 & 1.66 & 1071.18 & 0.00
 & 1.04\\SCA8-9 & 1068.65 & 1.86 & 
1068.65 & 1.89 & \bf{1060.50} & 
0.77 & 0.77\\CON3-0 & 619.09 & 2.04 & 
620.28 & 1.86 & \bf{616.52} & 
0.42 & 0.61\\CON3-1 & 557.21 & 1.63 & 
559.87 & 1.93 & \bf{554.47} & 
0.49 & 0.97\\CON3-2 & 521.38 & 2.31 & 
521.38 & 2.07 & \bf{518.00} & 
0.65 & 0.65\\CON3-3 & 592.43 & 1.81 & 
594.89 & 2.02 & \bf{591.19} & 
0.21 & 0.63\\CON3-4 & 591.43 & 2.51 & 
593.19 & 2.07 & \bf{588.79} & 
0.45 & 0.75\\CON3-5 & 564.88 & 2.17 & 
566.41 & 2.10 & \bf{563.70} & 
0.21 & 0.48\\CON3-6 & 502.26 & 1.86 & 
503.86 & 2.17 & \bf{499.05} & 
0.64 & 0.96\\CON3-7 & 578.41 & 1.98 & 
580.37 & 1.99 & \bf{576.48} & 
0.33 & 0.67\\CON3-8 & 524.30 & 2.02 & 
524.45 & 2.04 & \bf{523.05} & 
0.24 & 0.27\\CON3-9 & 588.11 & 1.92 & 
588.11 & 1.98 & \bf{578.24} & 
1.71 & 1.71\\CON8-0 & 860.48 & 1.86 & 
865.74 & 1.98 & \bf{857.17} & 
0.39 & 1.00\\CON8-1 & 753.57 & 2.29 & 
754.04 & 2.12 & \bf{740.85} & 
1.72 & 1.78\\CON8-2 & 717.84 & 1.92 & 
717.84 & 2.19 & \bf{712.89} & 
0.69 & 0.69\\CON8-3 & 822.51 & 1.97 & 
823.78 & 1.85 & \bf{811.07} & 
1.41 & 1.57\\CON8-4 & 780.51 & 2.43 & 
780.51 & 2.05 & \bf{772.25} & 
1.07 & 1.07\\CON8-5 & 765.69 & 1.96 & 
766.05 & 2.17 & \bf{754.88} & 
1.43 & 1.48\\CON8-6 & 688.24 & 2.07 & 
688.24 & 1.98 & \bf{678.92} & 
1.37 & 1.37\\CON8-7 & 815.43 & 1.66 & 
818.14 & 1.83 & \bf{811.96} & 
0.43 & 0.76\\CON8-8 & 782.68 & 2.13 & 
782.68 & 1.99 & \bf{767.53} & 
1.97 & 1.97\\CON8-9 & 818.50 & 1.92 & 
818.54 & 2.01 & \bf{809.00} & 
1.17 & 1.18\\\bf{PROM.} & 
\bf{765.24} & \bf{1.99} & \bf{766.29} & \bf{1.98} & \bf{758.54} & \bf{0.83} & \bf{0.96}\\[1ex]\hline
\end{tabular}
\label{table:nonlin}
\end{table} \clearpage

\begin{table}[ht]
\caption{Resultados de la ejecución de la metaheurística GA-M, utilizando instancias de Dethloff con la configuración -n 100.0 -p 250.0 -cprob 40 -mprob 70}
\centering
\small
\begin{tabular}{c c c c c c c c}
\hline\hline
Instancia & Costo mínimo & Tiempo(seg.) & Costo promedio & Tiempo promedio(seg.) & CME & \%G & \%GP \\ [0.5ex]
\hline
SCA3-0 & 640.55 & 3.07 & 
640.55 & 3.17 & \bf{635.62} & 
0.78 & 0.78\\SCA3-1 & \bf{697.84} & 3.49 & 
697.84 & 3.12 & 697.84 & 0.00
 & 0.00\\
SCA3-2 & 661.13 & 3.23 & 
662.67 & 2.98 & \bf{659.34} & 
0.27 & 0.51\\SCA3-3 & 681.16 & 2.98 & 
681.40 & 3.29 & \bf{680.04} & 
0.16 & 0.20\\SCA3-4 & \bf{690.50} & 3.59 & 
690.50 & 3.42 & 690.50 & 0.00
 & 0.00\\
SCA3-5 & 665.64 & 2.76 & 
665.64 & 3.05 & \bf{659.90} & 
0.87 & 0.87\\SCA3-6 & 652.94 & 3.51 & 
652.94 & 3.45 & \bf{651.09} & 
0.28 & 0.28\\SCA3-7 & 666.15 & 2.78 & 
666.15 & 3.25 & \bf{659.17} & 
1.06 & 1.06\\SCA3-8 & \bf{719.47} & 2.88 & 
719.54 & 3.15 & 719.47 & 0.00
 & 0.01\\SCA3-9 & \bf{681.00} & 3.58 & 
681.00 & 3.46 & 681.00 & 0.00
 & 0.00\\
SCA8-0 & 970.64 & 3.41 & 
976.86 & 3.43 & \bf{961.50} & 
0.95 & 1.60\\SCA8-1 & 1058.43 & 4.08 & 
1060.93 & 3.69 & \bf{1049.65} & 
0.84 & 1.07\\SCA8-2 & 1050.37 & 3.29 & 
1051.93 & 3.21 & \bf{1039.64} & 
1.03 & 1.18\\SCA8-3 & 1011.49 & 3.60 & 
1011.49 & 3.35 & \bf{983.34} & 
2.86 & 2.86\\SCA8-4 & 1067.55 & 3.06 & 
1067.55 & 3.28 & \bf{1065.49} & 
0.19 & 0.19\\SCA8-5 & 1048.59 & 2.86 & 
1048.59 & 3.29 & \bf{1027.08} & 
2.09 & 2.09\\SCA8-6 & 976.69 & 3.49 & 
976.77 & 3.33 & \bf{971.82} & 
0.50 & 0.51\\SCA8-7 & 1070.53 & 3.56 & 
1070.53 & 3.14 & \bf{1051.28} & 
1.83 & 1.83\\SCA8-8 & \bf{1071.18} & 2.92 & 
1076.56 & 3.10 & 1071.18 & 0.00
 & 0.50\\SCA8-9 & 1072.10 & 3.33 & 
1073.64 & 2.88 & \bf{1060.50} & 
1.09 & 1.24\\CON3-0 & 617.59 & 3.39 & 
619.73 & 3.35 & \bf{616.52} & 
0.17 & 0.52\\CON3-1 & 557.21 & 3.41 & 
557.21 & 3.35 & \bf{554.47} & 
0.49 & 0.49\\CON3-2 & 521.38 & 3.50 & 
521.38 & 3.36 & \bf{518.00} & 
0.65 & 0.65\\CON3-3 & 591.20 & 3.20 & 
591.50 & 3.35 & \bf{591.19} & 
0.00 & 0.05\\CON3-4 & 593.78 & 3.48 & 
594.51 & 3.27 & \bf{588.79} & 
0.85 & 0.97\\CON3-5 & \bf{563.70} & 2.77 & 
563.70 & 3.24 & 563.70 & 0.00
 & 0.00\\
CON3-6 & \bf{499.05} & 3.27 & 
503.15 & 3.98 & 499.05 & 0.00
 & 0.82\\CON3-7 & 577.91 & 2.82 & 
579.01 & 3.15 & \bf{576.48} & 
0.25 & 0.44\\CON3-8 & \bf{523.05} & 3.99 & 
524.08 & 3.63 & 523.05 & 0.00
 & 0.20\\CON3-9 & 587.79 & 4.14 & 
587.95 & 3.66 & \bf{578.24} & 
1.65 & 1.68\\CON8-0 & 858.88 & 3.56 & 
870.36 & 3.12 & \bf{857.17} & 
0.20 & 1.54\\CON8-1 & \bf{740.85} & 3.82 & 
743.94 & 3.44 & 740.85 & 0.00
 & 0.42\\CON8-2 & 716.03 & 3.44 & 
716.03 & 3.74 & \bf{712.89} & 
0.44 & 0.44\\CON8-3 & 821.26 & 2.97 & 
822.27 & 3.21 & \bf{811.07} & 
1.26 & 1.38\\CON8-4 & 777.62 & 3.01 & 
780.64 & 2.91 & \bf{772.25} & 
0.70 & 1.09\\CON8-5 & 758.12 & 3.75 & 
758.12 & 3.37 & \bf{754.88} & 
0.43 & 0.43\\CON8-6 & 687.81 & 3.21 & 
687.81 & 3.31 & \bf{678.92} & 
1.31 & 1.31\\CON8-7 & 814.77 & 3.13 & 
814.91 & 3.20 & \bf{811.96} & 
0.35 & 0.36\\CON8-8 & 783.47 & 3.17 & 
785.46 & 3.29 & \bf{767.53} & 
2.08 & 2.34\\CON8-9 & 813.04 & 3.88 & 
816.45 & 3.59 & \bf{809.00} & 
0.50 & 0.92\\\bf{PROM.} & 
\bf{763.96} & \bf{3.33} & \bf{765.28} & \bf{3.31} & \bf{758.54} & \bf{0.65} & \bf{0.82}\\[1ex]\hline
\end{tabular}
\label{table:nonlin}
\end{table} \clearpage

\begin{table}[ht]
\caption{Resultados de la ejecución de la metaheurística GA-M, utilizando instancias de Dethloff con la configuración -n 100.0 -p 300.0 -cprob 40 -mprob 70}
\centering
\small
\begin{tabular}{c c c c c c c c}
\hline\hline
Instancia & Costo mínimo & Tiempo(seg.) & Costo promedio & Tiempo promedio(seg.) & CME & \%G & \%GP \\ [0.5ex]
\hline
SCA3-0 & 636.06 & 3.51 & 
638.38 & 3.81 & \bf{635.62} & 
0.07 & 0.43\\SCA3-1 & \bf{697.84} & 3.48 & 
698.50 & 3.67 & 697.84 & 0.00
 & 0.10\\SCA3-2 & 661.13 & 3.56 & 
661.13 & 3.81 & \bf{659.34} & 
0.27 & 0.27\\SCA3-3 & \bf{680.04} & 4.15 & 
680.04 & 4.04 & 680.04 & 0.00
 & 0.00\\
SCA3-4 & \bf{690.50} & 3.50 & 
690.50 & 3.75 & 690.50 & 0.00
 & 0.00\\
SCA3-5 & \bf{659.90} & 3.80 & 
663.75 & 4.48 & 659.90 & 0.00
 & 0.58\\SCA3-6 & 652.94 & 4.28 & 
652.94 & 4.20 & \bf{651.09} & 
0.28 & 0.28\\SCA3-7 & \bf{659.17} & 4.56 & 
664.40 & 3.92 & 659.17 & 0.00
 & 0.79\\SCA3-8 & 719.77 & 3.78 & 
719.77 & 4.42 & \bf{719.47} & 
0.04 & 0.04\\SCA3-9 & \bf{681.00} & 3.83 & 
681.00 & 3.90 & 681.00 & 0.00
 & 0.00\\
SCA8-0 & 975.50 & 3.18 & 
979.36 & 3.58 & \bf{961.50} & 
1.46 & 1.86\\SCA8-1 & 1054.11 & 4.02 & 
1063.61 & 4.09 & \bf{1049.65} & 
0.42 & 1.33\\SCA8-2 & 1050.17 & 3.82 & 
1050.22 & 3.72 & \bf{1039.64} & 
1.01 & 1.02\\SCA8-3 & \bf{983.34} & 4.20 & 
983.34 & 4.32 & 983.34 & 0.00
 & 0.00\\
SCA8-4 & 1068.97 & 3.70 & 
1068.97 & 3.87 & \bf{1065.49} & 
0.33 & 0.33\\SCA8-5 & 1037.06 & 4.43 & 
1037.06 & 3.86 & \bf{1027.08} & 
0.97 & 0.97\\SCA8-6 & 972.48 & 3.66 & 
972.48 & 3.83 & \bf{971.82} & 
0.07 & 0.07\\SCA8-7 & 1070.92 & 4.55 & 
1073.30 & 3.88 & \bf{1051.28} & 
1.87 & 2.09\\SCA8-8 & \bf{1071.18} & 3.40 & 
1071.18 & 3.63 & 1071.18 & 0.00
 & 0.00\\
SCA8-9 & 1073.62 & 3.04 & 
1073.91 & 3.38 & \bf{1060.50} & 
1.24 & 1.26\\CON3-0 & 619.09 & 4.52 & 
619.09 & 4.20 & \bf{616.52} & 
0.42 & 0.42\\CON3-1 & 556.04 & 4.63 & 
557.49 & 4.09 & \bf{554.47} & 
0.28 & 0.54\\CON3-2 & 521.38 & 5.07 & 
521.38 & 4.66 & \bf{518.00} & 
0.65 & 0.65\\CON3-3 & \bf{591.19} & 3.90 & 
591.80 & 4.30 & 591.19 & 0.00
 & 0.10\\CON3-4 & 592.58 & 3.73 & 
592.88 & 3.95 & \bf{588.79} & 
0.64 & 0.69\\CON3-5 & \bf{563.70} & 4.14 & 
564.29 & 4.27 & 563.70 & 0.00
 & 0.10\\CON3-6 & 502.16 & 4.89 & 
502.34 & 4.44 & \bf{499.05} & 
0.62 & 0.66\\CON3-7 & 577.54 & 4.50 & 
579.03 & 3.85 & \bf{576.48} & 
0.18 & 0.44\\CON3-8 & 523.14 & 5.22 & 
524.96 & 4.78 & \bf{523.05} & 
0.02 & 0.37\\CON3-9 & 582.79 & 4.44 & 
582.79 & 4.58 & \bf{578.24} & 
0.79 & 0.79\\CON8-0 & 869.08 & 3.66 & 
869.08 & 4.02 & \bf{857.17} & 
1.39 & 1.39\\CON8-1 & 750.36 & 4.24 & 
750.36 & 4.16 & \bf{740.85} & 
1.28 & 1.28\\CON8-2 & 717.20 & 4.71 & 
717.92 & 4.89 & \bf{712.89} & 
0.60 & 0.71\\CON8-3 & 821.85 & 4.02 & 
821.85 & 4.28 & \bf{811.07} & 
1.33 & 1.33\\CON8-4 & 781.39 & 3.94 & 
784.48 & 4.22 & \bf{772.25} & 
1.18 & 1.58\\CON8-5 & 758.84 & 3.59 & 
758.84 & 3.91 & \bf{754.88} & 
0.52 & 0.52\\CON8-6 & 687.17 & 4.35 & 
687.17 & 4.51 & \bf{678.92} & 
1.22 & 1.22\\CON8-7 & 814.50 & 3.21 & 
814.50 & 3.72 & \bf{811.96} & 
0.31 & 0.31\\CON8-8 & 780.80 & 3.97 & 
782.06 & 3.93 & \bf{767.53} & 
1.73 & 1.89\\CON8-9 & 816.54 & 4.14 & 
818.60 & 4.41 & \bf{809.00} & 
0.93 & 1.19\\\bf{PROM.} & 
\bf{763.08} & \bf{4.03} & \bf{764.12} & \bf{4.08} & \bf{758.54} & \bf{0.55} & \bf{0.69}\\[1ex]\hline
\end{tabular}
\label{table:nonlin}
\end{table} \clearpage

\subsection{SalhiNagy}\label{tablas-entonacion-GA-M-salhinagy}

\begin{table}[ht]
\caption{Resultados de la ejecución de la metaheurística GA-M, utilizando instancias de SalhiNagy con la configuración -n 100 -p 350 -cprob 90 -mprob 70}
\centering
\small
\begin{tabular}{c c c c c c c c c}
\hline\hline
Instancia & Costo mínimo & Tiempo(seg.) & Costo promedio & T. prom(seg.) & Por & CME & \%G & \%GP \\ [0.5ex]
\hline
CMT1X & 474.91 & 7.26 & 
476.30 & 6.70 & 86.66\% & \bf{470.48} & 
0.94 & 1.24\\CMT1Y & 472.37 & 7.08 & 
476.58 & 6.39 & 96.66\% & \bf{470.48} & 
0.40 & 1.30\\CMT2X & 694.31 & 13.61 & 
702.97 & 14.80 & 100\% & \bf{682.39} & 
1.75 & 3.02\\CMT2Y & 693.46 & 14.56 & 
702.54 & 14.72 & 100\% & \bf{682.39} & 
1.62 & 2.95\\CMT3X & 723.46 & 33.76 & 
734.23 & 32.97 & 100\% & \bf{719.06} & 
0.61 & 2.11\\CMT3Y & 725.24 & 32.33 & 
734.62 & 32.25 & 100\% & \bf{719.06} & 
0.86 & 2.16\\CMT4X & 865.53 & 88.79 & 
895.27 & 88.27 & 100\% & \bf{854.21} & 
1.33 & 4.81\\CMT4Y & 886.82 & 89.15 & 
901.01 & 88.80 & 96.66\% & \bf{852.46} & 
4.03 & 5.70\\CMT5X & 1087.15 & 179.20 & 
1100.22 & 180.73 & 93.33\% & \bf{1030.56} & 
5.49 & 6.76\\CMT5Y & 1070.68 & 184.15 & 
1100.07 & 183.53 & 86.66\% & \bf{1031.69} & 
3.78 & 6.63\\CMT11X & 875.48 & 55.28 & 
893.37 & 55.07 & 100\% & \bf{831.09} & 
5.34 & 7.49\\CMT11Y & 846.38 & 61.17 & 
871.31 & 61.62 & 100\% & \bf{829.85} & 
1.99 & 5.00\\CMT12X & 670.29 & 33.64 & 
675.03 & 33.70 & 100\% & \bf{658.83} & 
1.74 & 2.46\\CMT12Y & 669.49 & 34.14 & 
674.08 & 33.06 & 96.66\% & \bf{660.47} & 
1.37 & 2.06\\\bf{PROM.} & 
\bf{768.25} & \bf{59.58} & \bf{781.26} & \bf{59.47} & - & \bf{749.50} & \bf{2.23} & \bf{3.83}\\[1ex]\hline
\end{tabular}
\label{table:GA-M-porcentajeS}
\end{table} \clearpage

\begin{table}[h]
\caption{Resultados de la ejecución de la metaheurística GA-M, utilizando instancias de SalhiNagy con la configuración -n 200 -p 40 -cprob 10.0 -mprob 10.0}
\centering
\small
\begin{tabular}{c c c c c c c c}
\hline\hline
Instancia & Costo mínimo & Tiempo(seg.) & Costo promedio & Tiempo promedio(seg.) & CME & \%G & \%GP \\ [0.5ex]
\hline
CMT1X & 480.02 & 0.40 & 
486.13 & 0.40 & \bf{470.48} & 
2.03 & 3.33\\CMT1Y & 477.06 & 0.42 & 
483.85 & 0.37 & \bf{470.48} & 
1.40 & 2.84\\CMT2X & 703.41 & 1.08 & 
708.40 & 1.04 & \bf{682.39} & 
3.08 & 3.81\\CMT2Y & 711.30 & 1.01 & 
712.00 & 1.01 & \bf{682.39} & 
4.24 & 4.34\\CMT3X & 742.27 & 2.66 & 
744.50 & 2.60 & \bf{719.06} & 
3.23 & 3.54\\CMT3Y & 740.27 & 2.36 & 
744.13 & 2.44 & \bf{719.06} & 
2.95 & 3.49\\CMT4X & 898.12 & 7.24 & 
914.44 & 7.42 & \bf{854.21} & 
5.14 & 7.05\\CMT4Y & 892.37 & 7.20 & 
913.15 & 7.29 & \bf{852.46} & 
4.68 & 7.12\\CMT5X & 1093.27 & 15.22 & 
1121.11 & 15.44 & \bf{1030.56} & 
6.09 & 8.79\\CMT5Y & 1116.16 & 16.28 & 
1132.67 & 15.86 & \bf{1031.69} & 
8.19 & 9.79\\CMT11X & 898.36 & 4.38 & 
911.32 & 4.49 & \bf{831.09} & 
8.09 & 9.65\\CMT11Y & 883.51 & 5.05 & 
907.64 & 5.14 & \bf{829.85} & 
6.47 & 9.37\\CMT12X & 674.53 & 2.78 & 
678.89 & 2.76 & \bf{658.83} & 
2.38 & 3.05\\CMT12Y & 680.71 & 2.36 & 
681.07 & 2.50 & \bf{660.47} & 
3.06 & 3.12\\\bf{PROM.} & 
\bf{785.10} & \bf{4.89} & \bf{795.66} & \bf{4.91} & \bf{749.50} & \bf{4.36} & \bf{5.66}\\[1ex]\hline
\end{tabular}
\label{table:nonlin}
\end{table} 

\begin{table}[ht]
\caption{Resultados de la ejecución de la metaheurística GA-M, utilizando instancias de SalhiNagy con la configuración -n 200 -p 40 -cprob 10.0 -mprob 70.0}
\centering
\small
\begin{tabular}{c c c c c c c c}
\hline\hline
Instancia & Costo mínimo & Tiempo(seg.) & Costo promedio & Tiempo promedio(seg.) & CME & \%G & \%GP \\ [0.5ex]
\hline
CMT1X & 482.53 & 0.39 & 
483.85 & 0.40 & \bf{470.48} & 
2.56 & 2.84\\CMT1Y & 479.65 & 0.34 & 
481.20 & 0.39 & \bf{470.48} & 
1.95 & 2.28\\CMT2X & 712.85 & 0.95 & 
717.94 & 1.03 & \bf{682.39} & 
4.46 & 5.21\\CMT2Y & 720.06 & 1.00 & 
723.66 & 1.17 & \bf{682.39} & 
5.52 & 6.05\\CMT3X & 743.40 & 2.56 & 
744.52 & 2.54 & \bf{719.06} & 
3.38 & 3.54\\CMT3Y & 746.95 & 2.39 & 
751.00 & 2.45 & \bf{719.06} & 
3.88 & 4.44\\CMT4X & 910.89 & 7.59 & 
914.77 & 7.48 & \bf{854.21} & 
6.64 & 7.09\\CMT4Y & 896.62 & 7.79 & 
907.25 & 7.68 & \bf{852.46} & 
5.18 & 6.43\\CMT5X & 1110.94 & 16.15 & 
1117.30 & 15.46 & \bf{1030.56} & 
7.80 & 8.42\\CMT5Y & 1105.72 & 15.40 & 
1121.09 & 15.76 & \bf{1031.69} & 
7.18 & 8.67\\CMT11X & 905.00 & 4.48 & 
912.13 & 4.67 & \bf{831.09} & 
8.89 & 9.75\\CMT11Y & 869.06 & 4.86 & 
887.79 & 4.91 & \bf{829.85} & 
4.72 & 6.98\\CMT12X & 674.30 & 2.63 & 
680.43 & 2.69 & \bf{658.83} & 
2.35 & 3.28\\CMT12Y & 674.94 & 2.44 & 
675.72 & 2.63 & \bf{660.47} & 
2.19 & 2.31\\\bf{PROM.} & 
\bf{788.06} & \bf{4.93} & \bf{794.19} & \bf{4.95} & \bf{749.50} & \bf{4.76} & \bf{5.52}\\[1ex]\hline
\end{tabular}
\label{table:nonlin}
\end{table}

\begin{table}[h]
\caption{Resultados de la ejecución de la metaheurística GA-M, utilizando instancias de SalhiNagy con la configuración -n 200 -p 40 -cprob 10.0 -mprob 100.0}
\centering
\small
\begin{tabular}{c c c c c c c c}
\hline\hline
Instancia & Costo mínimo & Tiempo(seg.) & Costo promedio & Tiempo promedio(seg.) & CME & \%G & \%GP \\ [0.5ex]
\hline
CMT1X & 483.83 & 0.38 & 
486.00 & 0.48 & \bf{470.48} & 
2.84 & 3.30\\CMT1Y & 488.48 & 0.30 & 
490.30 & 0.33 & \bf{470.48} & 
3.83 & 4.21\\CMT2X & 707.74 & 0.90 & 
711.26 & 1.03 & \bf{682.39} & 
3.71 & 4.23\\CMT2Y & 706.50 & 0.92 & 
708.19 & 1.04 & \bf{682.39} & 
3.53 & 3.78\\CMT3X & 737.63 & 2.36 & 
745.83 & 2.46 & \bf{719.06} & 
2.58 & 3.72\\CMT3Y & 733.14 & 2.28 & 
741.17 & 2.40 & \bf{719.06} & 
1.96 & 3.07\\CMT4X & 902.13 & 7.55 & 
910.31 & 7.14 & \bf{854.21} & 
5.61 & 6.57\\CMT4Y & 912.23 & 7.19 & 
918.72 & 7.25 & \bf{852.46} & 
7.01 & 7.77\\CMT5X & 1101.55 & 15.33 & 
1117.89 & 15.26 & \bf{1030.56} & 
6.89 & 8.47\\CMT5Y & 1116.18 & 15.18 & 
1123.25 & 15.72 & \bf{1031.69} & 
8.19 & 8.87\\CMT11X & 885.85 & 4.66 & 
905.14 & 4.60 & \bf{831.09} & 
6.59 & 8.91\\CMT11Y & 849.52 & 5.10 & 
897.51 & 4.98 & \bf{829.85} & 
2.37 & 8.15\\CMT12X & 673.99 & 2.72 & 
679.58 & 2.63 & \bf{658.83} & 
2.30 & 3.15\\CMT12Y & 677.04 & 2.55 & 
679.02 & 2.70 & \bf{660.47} & 
2.51 & 2.81\\\bf{PROM.} & 
\bf{783.99} & \bf{4.82} & \bf{793.87} & \bf{4.86} & \bf{749.50} & \bf{4.28} & \bf{5.50}\\[1ex]\hline
\end{tabular}
\label{table:nonlin}
\end{table} 

\begin{table}[h]
\caption{Resultados de la ejecución de la metaheurística GA-M, utilizando instancias de SalhiNagy con la configuración -n 200 -p 40 -cprob 40.0 -mprob 10.0}
\centering
\small
\begin{tabular}{c c c c c c c c}
\hline\hline
Instancia & Costo mínimo & Tiempo(seg.) & Costo promedio & Tiempo promedio(seg.) & CME & \%G & \%GP \\ [0.5ex]
\hline
CMT1X & 478.84 & 0.46 & 
486.30 & 0.45 & \bf{470.48} & 
1.78 & 3.36\\CMT1Y & 481.64 & 0.59 & 
484.14 & 0.48 & \bf{470.48} & 
2.37 & 2.90\\CMT2X & 712.09 & 1.12 & 
719.28 & 1.18 & \bf{682.39} & 
4.35 & 5.41\\CMT2Y & 705.06 & 1.36 & 
710.07 & 1.22 & \bf{682.39} & 
3.32 & 4.06\\CMT3X & 729.33 & 2.98 & 
742.21 & 3.09 & \bf{719.06} & 
1.43 & 3.22\\CMT3Y & 738.62 & 2.45 & 
739.85 & 2.57 & \bf{719.06} & 
2.72 & 2.89\\CMT4X & 906.49 & 7.72 & 
913.06 & 8.01 & \bf{854.21} & 
6.12 & 6.89\\CMT4Y & 904.96 & 7.56 & 
915.36 & 7.46 & \bf{852.46} & 
6.16 & 7.38\\CMT5X & 1079.48 & 15.86 & 
1115.01 & 16.16 & \bf{1030.56} & 
4.75 & 8.19\\CMT5Y & 1108.23 & 17.50 & 
1124.05 & 16.66 & \bf{1031.69} & 
7.42 & 8.95\\CMT11X & 881.52 & 4.47 & 
908.50 & 5.00 & \bf{831.09} & 
6.07 & 9.31\\CMT11Y & 855.05 & 5.19 & 
878.80 & 5.28 & \bf{829.85} & 
3.04 & 5.90\\CMT12X & 675.69 & 2.75 & 
680.51 & 2.67 & \bf{658.83} & 
2.56 & 3.29\\CMT12Y & 675.07 & 3.28 & 
678.80 & 2.88 & \bf{660.47} & 
2.21 & 2.78\\\bf{PROM.} & 
\bf{780.86} & \bf{5.23} & \bf{792.57} & \bf{5.22} & \bf{749.50} & \bf{3.88} & \bf{5.32}\\[1ex]\hline
\end{tabular}
\label{table:nonlin}
\end{table} 

\begin{table}[h]
\caption{Resultados de la ejecución de la metaheurística GA-M, utilizando instancias de SalhiNagy con la configuración -n 200 -p 40 -cprob 40.0 -mprob 70.0}
\centering
\small
\begin{tabular}{c c c c c c c c}
\hline\hline
Instancia & Costo mínimo & Tiempo(seg.) & Costo promedio & Tiempo promedio(seg.) & CME & \%G & \%GP \\ [0.5ex]
\hline
CMT1X & 478.82 & 0.50 & 
480.37 & 0.73 & \bf{470.48} & 
1.77 & 2.10\\CMT1Y & 489.68 & 0.82 & 
489.68 & 0.82 & \bf{470.48} & 
4.08 & 4.08\\CMT2X & 699.53 & 1.11 & 
701.81 & 1.12 & \bf{682.39} & 
2.51 & 2.85\\CMT2Y & 697.50 & 1.18 & 
704.64 & 1.38 & \bf{682.39} & 
2.21 & 3.26\\CMT3X & 737.01 & 3.39 & 
747.27 & 3.01 & \bf{719.06} & 
2.50 & 3.92\\CMT3Y & 730.43 & 2.75 & 
739.94 & 2.90 & \bf{719.06} & 
1.58 & 2.90\\CMT4X & 874.20 & 7.71 & 
909.50 & 8.03 & \bf{854.21} & 
2.34 & 6.47\\CMT4Y & 900.12 & 8.57 & 
916.16 & 8.29 & \bf{852.46} & 
5.59 & 7.47\\CMT5X & 1115.08 & 16.67 & 
1124.91 & 16.09 & \bf{1030.56} & 
8.20 & 9.16\\CMT5Y & 1130.57 & 17.02 & 
1137.87 & 16.56 & \bf{1031.69} & 
9.58 & 10.29\\CMT11X & 902.47 & 5.05 & 
917.15 & 5.07 & \bf{831.09} & 
8.59 & 10.36\\CMT11Y & 898.41 & 5.62 & 
912.28 & 5.81 & \bf{829.85} & 
8.26 & 9.93\\CMT12X & 674.69 & 3.02 & 
676.08 & 3.06 & \bf{658.83} & 
2.41 & 2.62\\CMT12Y & 675.40 & 3.30 & 
678.62 & 2.92 & \bf{660.47} & 
2.26 & 2.75\\\bf{PROM.} & 
\bf{785.99} & \bf{5.48} & \bf{795.45} & \bf{5.41} & \bf{749.50} & \bf{4.42} & \bf{5.58}\\[1ex]\hline
\end{tabular}
\label{table:nonlin}
\end{table} 

\begin{table}[h]
\caption{Resultados de la ejecución de la metaheurística GA-M, utilizando instancias de SalhiNagy con la configuración -n 200 -p 40 -cprob 40.0 -mprob 100.0}
\centering
\small
\begin{tabular}{c c c c c c c c}
\hline\hline
Instancia & Costo mínimo & Tiempo(seg.) & Costo promedio & Tiempo promedio(seg.) & CME & \%G & \%GP \\ [0.5ex]
\hline
CMT1X & 476.53 & 0.52 & 
486.06 & 0.76 & \bf{470.48} & 
1.29 & 3.31\\CMT1Y & 483.02 & 0.54 & 
485.81 & 0.54 & \bf{470.48} & 
2.67 & 3.26\\CMT2X & 703.21 & 1.51 & 
712.71 & 1.47 & \bf{682.39} & 
3.05 & 4.44\\CMT2Y & 716.02 & 1.54 & 
719.77 & 1.32 & \bf{682.39} & 
4.93 & 5.48\\CMT3X & 734.38 & 2.65 & 
739.02 & 3.09 & \bf{719.06} & 
2.13 & 2.78\\CMT3Y & 735.08 & 2.55 & 
748.39 & 3.00 & \bf{719.06} & 
2.23 & 4.08\\CMT4X & 900.05 & 8.04 & 
905.88 & 7.66 & \bf{854.21} & 
5.37 & 6.05\\CMT4Y & 908.78 & 8.23 & 
916.92 & 7.93 & \bf{852.46} & 
6.61 & 7.56\\CMT5X & 1108.60 & 16.24 & 
1115.09 & 16.10 & \bf{1030.56} & 
7.57 & 8.20\\CMT5Y & 1106.28 & 17.35 & 
1117.31 & 17.02 & \bf{1031.69} & 
7.23 & 8.30\\CMT11X & 907.01 & 4.92 & 
916.09 & 5.10 & \bf{831.09} & 
9.13 & 10.23\\CMT11Y & 885.25 & 6.17 & 
895.28 & 5.55 & \bf{829.85} & 
6.68 & 7.89\\CMT12X & 675.73 & 3.01 & 
683.49 & 2.94 & \bf{658.83} & 
2.57 & 3.74\\CMT12Y & 681.33 & 3.36 & 
684.09 & 2.98 & \bf{660.47} & 
3.16 & 3.58\\\bf{PROM.} & 
\bf{787.23} & \bf{5.47} & \bf{794.71} & \bf{5.39} & \bf{749.50} & \bf{4.61} & \bf{5.63}\\[1ex]\hline
\end{tabular}
\label{table:nonlin}
\end{table} 

\begin{table}[h]
\caption{Resultados de la ejecución de la metaheurística GA-M, utilizando instancias de SalhiNagy con la configuración -n 200 -p 40 -cprob 90.0 -mprob 70.0}
\centering
\small
\begin{tabular}{c c c c c c c c}
\hline\hline
Instancia & Costo mínimo & Tiempo(seg.) & Costo promedio & Tiempo promedio(seg.) & CME & \%G & \%GP \\ [0.5ex]
\hline
CMT1X & 480.19 & 0.82 & 
483.09 & 0.82 & \bf{470.48} & 
2.06 & 2.68\\CMT1Y & 477.00 & 1.52 & 
483.12 & 0.90 & \bf{470.48} & 
1.39 & 2.69\\CMT2X & 703.44 & 1.68 & 
713.65 & 1.62 & \bf{682.39} & 
3.08 & 4.58\\CMT2Y & 704.88 & 1.73 & 
713.29 & 1.62 & \bf{682.39} & 
3.30 & 4.53\\CMT3X & 733.34 & 3.40 & 
736.12 & 3.12 & \bf{719.06} & 
1.99 & 2.37\\CMT3Y & 729.62 & 3.47 & 
735.80 & 3.31 & \bf{719.06} & 
1.47 & 2.33\\CMT4X & 903.54 & 8.17 & 
912.16 & 8.11 & \bf{854.21} & 
5.77 & 6.78\\CMT4Y & 889.48 & 8.68 & 
905.24 & 8.68 & \bf{852.46} & 
4.34 & 6.19\\CMT5X & 1087.66 & 16.62 & 
1116.59 & 16.49 & \bf{1030.56} & 
5.54 & 8.35\\CMT5Y & 1115.55 & 17.37 & 
1128.86 & 17.46 & \bf{1031.69} & 
8.13 & 9.42\\CMT11X & 896.75 & 5.17 & 
921.97 & 5.14 & \bf{831.09} & 
7.90 & 10.94\\CMT11Y & 852.23 & 6.38 & 
892.54 & 5.95 & \bf{829.85} & 
2.70 & 7.55\\CMT12X & 676.64 & 3.22 & 
678.61 & 3.45 & \bf{658.83} & 
2.70 & 3.00\\CMT12Y & 674.87 & 3.26 & 
676.52 & 3.19 & \bf{660.47} & 
2.18 & 2.43\\\bf{PROM.} & 
\bf{780.37} & \bf{5.82} & \bf{792.68} & \bf{5.71} & \bf{749.50} & \bf{3.75} & \bf{5.27}\\[1ex]\hline
\end{tabular}
\label{table:nonlin}
\end{table}

\begin{table}[h]
\caption{Resultados de la ejecución de la metaheurística GA-M, utilizando instancias de SalhiNagy con la configuración -n 200 -p 40 -cprob 90.0 -mprob 100.0}
\centering
\small
\begin{tabular}{c c c c c c c c}
\hline\hline
Instancia & Costo mínimo & Tiempo(seg.) & Costo promedio & Tiempo promedio(seg.) & CME & \%G & \%GP \\ [0.5ex]
\hline
CMT1X & 484.51 & 0.83 & 
486.91 & 0.81 & \bf{470.48} & 
2.98 & 3.49\\CMT1Y & 481.85 & 0.82 & 
481.85 & 0.77 & \bf{470.48} & 
2.42 & 2.42\\CMT2X & 710.11 & 1.70 & 
718.44 & 1.68 & \bf{682.39} & 
4.06 & 5.28\\CMT2Y & 709.29 & 1.74 & 
711.18 & 1.65 & \bf{682.39} & 
3.94 & 4.22\\CMT3X & 738.81 & 3.51 & 
745.97 & 3.48 & \bf{719.06} & 
2.75 & 3.74\\CMT3Y & 742.21 & 3.36 & 
747.68 & 3.31 & \bf{719.06} & 
3.22 & 3.98\\CMT4X & 899.99 & 8.68 & 
909.11 & 8.52 & \bf{854.21} & 
5.36 & 6.43\\CMT4Y & 907.36 & 8.61 & 
916.09 & 8.69 & \bf{852.46} & 
6.44 & 7.46\\CMT5X & 1103.51 & 17.46 & 
1118.76 & 17.39 & \bf{1030.56} & 
7.08 & 8.56\\CMT5Y & 1102.73 & 18.39 & 
1125.69 & 17.80 & \bf{1031.69} & 
6.89 & 9.11\\CMT11X & 897.15 & 5.54 & 
907.26 & 5.65 & \bf{831.09} & 
7.95 & 9.17\\CMT11Y & 900.07 & 5.91 & 
915.51 & 6.06 & \bf{829.85} & 
8.46 & 10.32\\CMT12X & 675.28 & 5.21 & 
680.98 & 3.77 & \bf{658.83} & 
2.50 & 3.36\\CMT12Y & 673.49 & 3.40 & 
677.92 & 3.38 & \bf{660.47} & 
1.97 & 2.64\\\bf{PROM.} & 
\bf{787.60} & \bf{6.08} & \bf{795.95} & \bf{5.93} & \bf{749.50} & \bf{4.72} & \bf{5.73}\\[1ex]\hline
\end{tabular}
\label{table:nonlin}
\end{table}

\begin{table}[h]
\caption{Resultados de la ejecución de la metaheurística GA-M, utilizando instancias de SalhiNagy con la configuración -n 100.0 -p 150.0 -cprob 90 -mprob 70}
\centering
\small
\begin{tabular}{c c c c c c c c}
\hline\hline
Instancia & Costo mínimo & Tiempo(seg.) & Costo promedio & Tiempo promedio(seg.) & CME & \%G & \%GP \\ [0.5ex]
\hline
CMT1X & 476.66 & 2.10 & 
477.61 & 2.06 & \bf{470.48} & 
1.31 & 1.52\\CMT1Y & 474.91 & 1.72 & 
477.88 & 1.89 & \bf{470.48} & 
0.94 & 1.57\\CMT2X & 704.33 & 4.71 & 
706.86 & 4.59 & \bf{682.39} & 
3.22 & 3.59\\CMT2Y & 697.61 & 4.58 & 
708.14 & 4.29 & \bf{682.39} & 
2.23 & 3.77\\CMT3X & 739.58 & 10.05 & 
742.50 & 10.17 & \bf{719.06} & 
2.85 & 3.26\\CMT3Y & 737.23 & 10.64 & 
740.26 & 10.04 & \bf{719.06} & 
2.53 & 2.95\\CMT4X & 897.35 & 27.72 & 
901.54 & 28.54 & \bf{854.21} & 
5.05 & 5.54\\CMT4Y & 887.82 & 28.65 & 
901.00 & 29.27 & \bf{852.46} & 
4.15 & 5.69\\CMT5X & 1093.53 & 57.13 & 
1104.10 & 57.73 & \bf{1030.56} & 
6.11 & 7.14\\CMT5Y & 1097.56 & 59.94 & 
1104.60 & 59.17 & \bf{1031.69} & 
6.38 & 7.07\\CMT11X & 895.94 & 18.10 & 
903.25 & 17.58 & \bf{831.09} & 
7.80 & 8.68\\CMT11Y & 863.65 & 19.81 & 
881.75 & 21.39 & \bf{829.85} & 
4.07 & 6.25\\CMT12X & 674.72 & 10.69 & 
676.28 & 10.66 & \bf{658.83} & 
2.41 & 2.65\\CMT12Y & 674.87 & 10.60 & 
675.93 & 10.65 & \bf{660.47} & 
2.18 & 2.34\\\bf{PROM.} & 
\bf{779.70} & \bf{19.03} & \bf{785.84} & \bf{19.15} & \bf{749.50} & \bf{3.66} & \bf{4.43}\\[1ex]\hline
\end{tabular}
\label{table:nonlin}
\end{table} 

\begin{table}[h]
\caption{Resultados de la ejecución de la metaheurística GA-M, utilizando instancias de SalhiNagy con la configuración -n 100.0 -p 250.0 -cprob 90 -mprob 70}
\centering
\small
\begin{tabular}{c c c c c c c c}
\hline\hline
Instancia & Costo mínimo & Tiempo(seg.) & Costo promedio & Tiempo promedio(seg.) & CME & \%G & \%GP \\ [0.5ex]
\hline
CMT1X & 478.84 & 3.21 & 
479.55 & 3.40 & \bf{470.48} & 
1.78 & 1.93\\CMT1Y & 475.53 & 2.84 & 
478.90 & 3.37 & \bf{470.48} & 
1.07 & 1.79\\CMT2X & 705.12 & 8.20 & 
706.38 & 8.03 & \bf{682.39} & 
3.33 & 3.51\\CMT2Y & 702.79 & 7.52 & 
707.14 & 7.46 & \bf{682.39} & 
2.99 & 3.63\\CMT3X & 735.58 & 17.30 & 
738.93 & 17.23 & \bf{719.06} & 
2.30 & 2.76\\CMT3Y & 725.20 & 16.16 & 
730.15 & 16.89 & \bf{719.06} & 
0.85 & 1.54\\CMT4X & 884.54 & 47.17 & 
892.93 & 46.43 & \bf{854.21} & 
3.55 & 4.53\\CMT4Y & 897.49 & 47.75 & 
906.99 & 47.59 & \bf{852.46} & 
5.28 & 6.40\\CMT5X & 1091.98 & 96.68 & 
1098.78 & 95.91 & \bf{1030.56} & 
5.96 & 6.62\\CMT5Y & 1091.82 & 98.94 & 
1105.65 & 98.36 & \bf{1031.69} & 
5.83 & 7.17\\CMT11X & 861.70 & 29.52 & 
889.88 & 28.83 & \bf{831.09} & 
3.68 & 7.07\\CMT11Y & 877.16 & 31.56 & 
888.71 & 32.42 & \bf{829.85} & 
5.70 & 7.09\\CMT12X & 671.23 & 18.42 & 
672.35 & 17.67 & \bf{658.83} & 
1.88 & 2.05\\CMT12Y & 668.95 & 17.41 & 
673.56 & 17.10 & \bf{660.47} & 
1.28 & 1.98\\\bf{PROM.} & 
\bf{776.28} & \bf{31.62} & \bf{783.56} & \bf{31.48} & \bf{749.50} & \bf{3.25} & \bf{4.15}\\[1ex]\hline
\end{tabular}
\label{table:nonlin}
\end{table} 

\begin{table}[h]
\caption{Resultados de la ejecución de la metaheurística GA-M, utilizando instancias de SalhiNagy con la configuración -n 100.0 -p 350.0 -cprob 90 -mprob 70}
\centering
\small
\begin{tabular}{c c c c c c c c}
\hline\hline
Instancia & Costo mínimo & Tiempo(seg.) & Costo promedio & Tiempo promedio(seg.) & CME & \%G & \%GP \\ [0.5ex]
\hline
CMT1X & 475.37 & 4.94 & 
476.89 & 4.91 & \bf{470.48} & 
1.04 & 1.36\\CMT1Y & 477.21 & 3.96 & 
478.35 & 4.15 & \bf{470.48} & 
1.43 & 1.67\\CMT2X & 695.58 & 10.70 & 
701.59 & 11.20 & \bf{682.39} & 
1.93 & 2.81\\CMT2Y & 692.39 & 11.43 & 
703.65 & 11.01 & \bf{682.39} & 
1.47 & 3.12\\CMT3X & 728.88 & 24.93 & 
732.16 & 24.99 & \bf{719.06} & 
1.37 & 1.82\\CMT3Y & 733.44 & 23.79 & 
736.68 & 23.88 & \bf{719.06} & 
2.00 & 2.45\\CMT4X & 896.29 & 64.68 & 
899.29 & 64.94 & \bf{854.21} & 
4.93 & 5.28\\CMT4Y & 872.44 & 66.17 & 
895.17 & 66.64 & \bf{852.46} & 
2.34 & 5.01\\CMT5X & 1091.97 & 135.16 & 
1097.95 & 134.75 & \bf{1030.56} & 
5.96 & 6.54\\CMT5Y & 1074.86 & 135.76 & 
1099.76 & 135.59 & \bf{1031.69} & 
4.18 & 6.60\\CMT11X & 881.94 & 41.26 & 
889.67 & 41.73 & \bf{831.09} & 
6.12 & 7.05\\CMT11Y & 863.57 & 45.09 & 
874.57 & 46.31 & \bf{829.85} & 
4.06 & 5.39\\CMT12X & 673.00 & 23.98 & 
674.30 & 24.90 & \bf{658.83} & 
2.15 & 2.35\\CMT12Y & 670.65 & 24.33 & 
673.51 & 23.98 & \bf{660.47} & 
1.54 & 1.97\\\bf{PROM.} & 
\bf{773.40} & \bf{44.01} & \bf{780.97} & \bf{44.21} & \bf{749.50} & \bf{2.89} & \bf{3.82}\\[1ex]\hline
\end{tabular}
\label{table:GA-M-90-70}
\end{table} 

\end{document}
