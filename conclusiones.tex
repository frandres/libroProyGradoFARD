% Conclusiones
\chapter*{Conclusiones y recomendaciones} \label{chap:conclusiones}
\addcontentsline{toc}{chapter}{Conclusiones y recomendaciones}

El presente proyecto tuvo como objetivo diseñar e implementar un mecanismo de extracción focalizada según lo especificado en la Arquitectura para la Interrogación de Documentos (DIA) propuesta por Abad Mota \cite{documentInterrogationArchitecture}.
\textless Resumir aquí brevemente los resultados \emph{finales} obtenidos \textgreater. \\

A manera de conclusión, es pertinente resumir algunas reflexiones derivadas de este trabajo que pueden ser de interés para profundizar la investigación en esta área. \\

En primera instancia, la premisa fundamental sobre la cual opera DIA al definir la extracción focalizada es que existe información presente en la ontología con la cual se puede realizar una segunda extracción sobre un conjunto de documentos para descubrir información faltante. Si bien este proyecto ha sido un primer paso importante en el razonamiento sobre la extracción focalizada, existen varias interrogantes que deben ser respondidas en proyectos que busquen probar cosas más específicas con tecnologías más poderosas. \\

En primer lugar, algo muy importante sobre lo cual se debe razonar para ampliar el concepto de extracción focalizada es qué metadatos se pueden derivar o tener presentes en la ontología para guiar la extracción focalizada. Esto es quizás la principal fortaleza que tiene este mecanismo y merece muchísima atención. En este trabajo se utilizaron valores de atributos conocidos para realizar la búsqueda focalizada y se obtuvieron resultados satisfactorios, dada la naturaleza de los dominios de información con los que se trabajó y otras consideraciones que se han mencionado anteriormente. Sin embargo, este modelo es probablemente simple y se puede asociar a casos en los que los documentos contengan unidades de información, según lo definido anteriormente, que puedan ser explotados por extractores. Una posible profundización en el desarrollo de DIA es el estudio de nuevos metadatos e información que se pueda incluir en lo que se definió como un contexto de extracción en este trabajo. \\

Esto lleva a otra segunda conclusión de altísima relevancia. La tarea de la extracción (y el contexto de extracción) depende de las tecnologías existentes para ello. En este trabajo se utilizaron expresiones regulares. Se mencionó, sin embargo, que la investigación en esta área ha estado guiada en los últimos años por la utilización de métodos estocásticos y de inteligencia artificial que han demostrado mejores resultados. La utilización de dichos mecanismos escapa por completo del alcance definido para este proyecto, concebido como un primer paso en el área de extracción focalizada. Sin embargo, un reto que queda pendiente es razonar en la construcción de extractores de información con las últimas y las mejores tecnologías existentes a partir de los metadatos que se puedan extraer de la ontología. \\

Otro resultado derivado del trabajo es lo importante que supone la tarea de hacer \emph{ingeniería} de documentos para la extracción focalizada. Ademas de construir el extractor de información y el contexto de extracción, es necesario realizar tareas adicionales cuando la información está presente en más de un tipo de documento, cuando los documentos no son los documentos originales, cuando se quiere trabajar con el World Wide Web, cuando los estilos y tipos de documentos son heterógeneos en su estilo y forma de redacción o simplemente poco estructurados, entre otros.\\

Además de lo ya expuesto anteriormente, es importante tomar en cuenta es que se decidió trabajar con sólo un tipo de origen de incompletitud: la incompletitud por el desconocimiento de algunos valores de atributos en la ontología. Dicha decisión fue tomada para acotar y delimitar los alcances de este proyecto. Sin embargo queda pendiente razonar sobre incompletitudes por inexistencia de una instancia en la ontología o por el desconocimiento en la interrelación entre dos conceptos. \\

De forma que en conclusión, este proyecto de grado puede interpretarse como un primer paso en un tema como la extracción focalizada que resulta bastante prometedor pero que aún tiene muchísimas interrogantes y áreas de crecimiento. En este último capítulo se ha buscado resumir las principales áreas. Es de alto interés evaluar los resultados obtenidos a la luz de tecnologías y avances científicos que permitan respondar las interrogantes planteadas. \\