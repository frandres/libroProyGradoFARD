% file thesis.tex
% Archivo thesis.tex
% Documento maestro que incluye todos los paquetes necesarios para el documento
% principal.

% Documento obtenido por un sinfin de iteraciones de administradores del LDC
% Estructura actual hecha por:
% Jairo Lopez <jairo@ldc.usb.ve>
% Actualizado ligeramente por:
% Alexander Tough 

\documentclass[oneside,11pt,letterpaper]{report}
\tolerance=1000  
\hbadness=10000  
\raggedbottom

% Paquetes para manejar graficos
\usepackage{epsf}
\usepackage[pdftex]{graphicx}
\usepackage{epsfig}
% Simbolos matematicos
\usepackage{latexsym,amssymb}
% Paquetes para presentar una tesis decente.
\usepackage{setspace,cite} % Doble espacio para texto, espacio singular para
                           % los caption y pie de pagina

% Paquetes no utilizados para citas
%\usepackage{mcite} 
%\usepackage{draft} 

\usepackage{wrapfig}
\usepackage{alltt}

% Acentos 
\usepackage[spanish,activeacute]{babel}
\usepackage[spanish]{translator}
\usepackage[utf8]{inputenc}
\usepackage{color, xcolor, colortbl}
\usepackage{multirow}
\usepackage{subfig}
\usepackage[OT1]{fontenc}
\usepackage{tocbibind}
\usepackage{anysize}
\usepackage{listings} 

% Para poder tener texto asiatico
%\usepackage{CJK}

% Opciones para los glosarios
\usepackage[style=altlist,toc,numberline,acronym]{glossaries}
\usepackage{url}
\usepackage{amsthm}
\usepackage{amsmath}
\usepackage{fancyhdr} % Necesario para los encabezados
\usepackage{fancyvrb}
\usepackage{makeidx} % En caso de necesitar indices.
\usepackage{tikz}
\usepackage{float}
\usepackage{caption}
\DeclareCaptionFont{white}{\color{white}}
\DeclareCaptionFormat{listing}{\colorbox{white}{\parbox{\textwidth}{#1#2#3}}}
\lstset{
 		 language=Pascal,
         numbers=left,               % Ort der Zeilennummern
         numberstyle=\tiny,          % Stil der Zeilennummern
         numbersep=5pt,              % Abstand der Nummern zum Text
         tabsize=5,                  % Groesse von Tabs
         extendedchars=true,         %
         breaklines=true,            % Zeilen werden Umgebrochen
         keywordstyle=\bfseries,
         frame=b,
         showspaces=false,           % Leerzeichen anzeigen ?
         showtabs=false,             % Tabs anzeigen ?
         xleftmargin=17pt,
         framexleftmargin=17pt,
         framexrightmargin=5pt,
         framexbottommargin=4pt,
         backgroundcolor=\color{white},
         showstringspaces=false      % Leerzeichen in Strings anzeigen ?        
}
\floatstyle{boxed} 
\restylefloat{figure}

\makeindex  % Necesitado para los indices

% Definiciones para definicions, teoremas y lemas
\theoremstyle{definition} \newtheorem{definicion}{Definici\'{o}n}
\theoremstyle{plain} \newtheorem{teorema}{Teorema}
\theoremstyle{plain} \newtheorem{lema}{Lema}

% Para la creacion de los pdfs
\usepackage{hyperref}

% Para resolver el lio del Unicode para la informacion de los PDFs
% En pdftitle coloca el nombre de su proyecto de grado/pasantia.
% En pdfauthor coloca su nombre.
\hypersetup{
    pdftitle = {ANÁLISIS DE METAHEURÍSTICAS PARA RESOLVER EL PROBLEMA DE ENRUTAMIENTO DE VEHÍCULOS CON ENVÍO Y RECEPCIÓN SIMULTÁNEA (VRPSPD)},
    pdfauthor={ALESSANDRO CHACÓN, RICARDO SANTANA},
    colorlinks,
    citecolor=black,
    filecolor=black,
    linkcolor=black,
    urlcolor=black,
    backref,
    pdftex
}

% Crea el glosario
\makeglossaries

% Incluye el glosario
\newacronym{as-h}{as-H}{proceso autosimilar con par\'{a}metro autosimilar $H$}

\newacronym{asie-h}{asie-H}{proceso autosimilar con par\'{a}metro autosimilar
$H$ e incrementos estacionarios}

\newacronym{aas-h}{aas-H}{proceso asint\'{o}ticamente autosimilar con
par\'{a}metro autosimilar $H$}


% Para crear la hoja escaneada de las firmas
\usepackage[absolute]{textpos}

% Pone los nombres y las opciones para mostrar los codigos fuentes
\newcommand{\vrp}{Problema de Enrutamiento de Vehículos }
\newcommand{\vrpingles}{Vehicle Routing Problem }
\newcommand{\vrpspd}{Problema de Enrutamiento de Vehículos con Entrega y Recepción Simultánea }
\renewcommand\lstlistingname{Algoritmo}
\renewcommand\lstlistlistingname{Lista de Algoritmos}

% Dimensiones de la pagina
\setlength{\headheight}{15pt}
\marginsize{2cm}{2cm}{2cm}{2cm}

% Se pueden omitir para que no compile ciertos capitulos.
\includeonly{header, resultados, conclusiones}

%%%%%%%%%%%%%%%%%%%%%%%%%%%%%%%%%%%%%%%%%%%%%%%%%%%%%%%%%%%%%%%%%%%%%%%%%%%
%%%%%%%%%%%%%%%%      end of preamble and start of document     %%%%%%%%%%%
%%%%%%%%%%%%%%%%%%%%%%%%%%%%%%%%%%%%%%%%%%%%%%%%%%%%%%%%%%%%%%%%%%%%%%%%%%%
\begin{document}

% Pagina de titulo
% Pagina de titulo
\begin{titlepage}
\begin{center}

% Upper part (aqui ya esta incluido el logo de la USB).
\includegraphics[scale=0.5,type=png,ext=.png,read=.png]{figures/cebolla} \\

% Encabezado
\textsc {\large UNIVERSIDAD SIMÓN BOLÍVAR} \\
\textsc{\bfseries DECANATO DE ESTUDIOS PROFESIONALES\\
COORDINACIÓN DE INGENIERÍA DE LA COMPUTACIÓN}

\bigskip
\bigskip
\bigskip
\bigskip
\bigskip
\bigskip
\bigskip
\bigskip
\bigskip

% Title/Titulo
% Aqui ponga el nombre de su proyecto de grado/pasantia larga
\textsc{\bfseries EXTRACCIÓN DE INFORMACIÓN FOCALIZADA BASADA EN RESPUESTAS INCOMPLETAS (FIE)}
\bigskip
\bigskip
\bigskip
\bigskip
\bigskip

% Author and supervisor/Autor y tutor
\begin{minipage}{\textwidth}
\centering
Por: \\
FRANCISCO RODRÍGUEZ DRUMOND \\

\bigskip
\bigskip
\bigskip

Realizado con la asesoría de: \\
PROF. SORAYA ABAD
\end{minipage}

\bigskip
\bigskip
\bigskip
\bigskip
\bigskip
\bigskip
\bigskip
\bigskip
\bigskip

% Bottom half
{PROYECTO DE GRADO \\ Presentado ante la Ilustre Universidad Simón Bolívar \\
como requisito parcial para optar al título de \\ Ingeniero de Computación} \\

\bigskip
\bigskip
\vfill

% Date/Fecha 
{\large \bfseries Sartenejas, Septiembre de 2012}

\end{center}
\end{titlepage}


% Pagina de acta final (vacio)
%% Pagina del acta final
\begin{titlepage}
\begin{center}

% Upper part
\includegraphics[scale=0.5,type=png,ext=.png,read=.png]{figures/cebolla} \\

\textsc {\large UNIVERSIDAD SIM'ON BOL'IVAR} \\
\textsc{DECANATO DE ESTUDIOS PROFESIONALES\\
COORDINACI'ON DE INGENIER'IA DE LA COMPUTACI'ON}

\bigskip
\bigskip
\bigskip
\bigskip
\bigskip
\bigskip

% Title
\textsc{ACTA FINAL PROYECTO DE GRADO}

\bigskip
\bigskip

% Aqui coloca el nombre de su proyecto de grado/pasantia larga.
\textsc{\bfseries  EXTRACCIÓN DE INFORMACIÓN FOCALIZADA BASADA EN RESPUESTAS INCOMPLETAS (FIE)}

\bigskip
\bigskip
\bigskip
\bigskip

\begin{minipage}{\textwidth}
\centering
Presentado por: \\
% Aqui coloca su nombre.
\textsc{\bfseries FRANCISCO RODRÍGUEZ DRUMOND} \\

\bigskip
\bigskip
\bigskip
\bigskip

Este Proyecto de Grado ha sido aprobado por el siguiente jurado examinador: \\

\bigskip
\bigskip

% Despues de cada line coloca el (los) nombre(s) de
% cada uno de los integrantes del jurado.
\line(1,0){200} \\
PROF. 1\\

\bigskip
\bigskip

\line(1,0){200} \\
PROF. 2\\

\bigskip
\bigskip

\line(1,0){200} \\
PROF. 3\\
\end{minipage}

\bigskip
\bigskip
\vfill

% Date/Fecha
{\large \bfseries Sartenejas, X/X/12}

\end{center}
\end{titlepage}

\setcounter{secnumdepth}{3}
\setcounter{tocdepth}{4}

% Define encabezado numeros romanos y como se separan los captiulos y las
% secciones
\addtolength{\headheight}{3pt}
\pagenumbering{roman}
\pagestyle{fancyplain}

\renewcommand{\chaptermark}[1]{\markboth{\chaptername\ \thechapter:\,\ #1}{}}
\renewcommand{\sectionmark}[1]{\markright{\thesection\,\ #1}}

\onehalfspacing

\lhead{}
\chead{}
\rhead{}
\renewcommand{\headrulewidth}{0.0pt}
\lfoot{}
\cfoot{\fancyplain{}{\thepage}}
\rfoot{}


% Pagina de resumen
\setcounter{page}{4}
\chapter*{Resumen}
%
%El Problema de Enrutamiento de Vehículos (VRP por las siglas de Vehicle Routing Problem) es un 
%problema de optimización combinatoria del área de logística y distribución de bienes. Este problema busca 
%determinar las rutas que deben transitar un grupo de vehículos con el objetivo de transportar la cantidad de
%productos demandada por un conjunto de consumidores de forma tal que se minimice la distancia total recorrida. 
%El Problema de Enrutamiento de Vehículos con Entrega y Recepción Simúltanea (VRPSPD por las siglas de Vehicle Routing 
%Problem with Simultaneus Pick-up and Delivery) es una variante de VRP que permite a los clientes tanto recibir como entregar bienes de manera simultánea. Este problema está clasificado como un problema complejo, pues resolverlo de manera exacta por un algoritmo implica tiempos que se incrementan de manera exponencial a medida que el número de clientes aumenta de tamaño. Las metaheurísticas constituyen un conjunto de técnicas que  permiten conseguir buena calidad de soluciones y tiempos computacionales aceptables para este timpo de problemas. En este trabajo se presenta la implementación de seis metaheurísticas para la resolución de VRPSPD. Experimentos son realizados, así como entonación de los parámetros más influyentes para cada metaheurística. Finalmente, se comparan los resultados obtenidos entre las mejores metaheurísticas y se recomienda la utilización de la mejor de ellas, la cual tiene resultados competitivos en comparación con los mejores algoritmos en la literatura.

El problema de enrutamiento de vehículos (VRP por las siglas de Vehicle Routing Problem) es un problema de optimización combinatoria del área de logística y distribución de bienes. Una de sus variantes es VRPSPD, el cual tiene la particularidad que los clientes pueden recibir y entregar bienes de manera simultánea.\\

En este trabajo se busca resolver VRPSPD por medio de diferentes metaheurísticas híbridas, con el propósito de comparar sus resultados y poder recomendar la que genere la solución de mejor calidad y tiempo.\\

Para ello se realizó una investigación sobre las metaheurísticas más utilizadas para resolver VRPSPD y se seleccionaron seis. Esas seis se implementaron y para cada una de ellas se estudiaron los parámetros que más influyen en su comportamiento, para luego entonar cada uno de ellos. Una vez entonada cada metaheurística, se comparó su desempeño con los resultados del trabajo utilizado como base para la implementación de la metaheurística. Las metaheurísticas son ejecutadas con un conjunto de instancias del problema recomendadas en la literatura, mayormente utilizadas en los trabajos referenciados. Finalmente, se seleccionaron las metaheurísticas que proporcionaron la solución de mejor calidad y menor tiempo según cada clase de instancia, y se recomendaron las mejores.

\newpage



% Pagina de dedicatoria (opcional)
\setcounter{page}{5}

\vspace*{8cm} 
\pdfbookmark[0]{Dedication}{dedication} % Sets a PDF bookmark for the dedication
\begin{center} 
\large DEDICATORIA
\end{center}
\newpage


% Pagina de agradecimientos (opcional)
\setcounter{page}{6}

\chapter*{Agradecimientos
\markboth{Agradecimientos}{Agradecimientos}}
AQU\'I VAN LOS AGRADECIMIENTOS.





% Crea la tabla de contenidos
\tableofcontents

% Crea la lista de cuadros
\listoftables

% Crea la lista de figuras
\listoffigures

% Crea la lista de codigos fuentes
%\lstlistoflistings

\clearpage

% Define encabezado en numeros arabicos  
\pagenumbering{arabic}

\fancyhf{} % Redefine el encabezado 
\lhead{}
\chead{}
\rhead{\fancyplain{}{\thepage}}
\renewcommand{\headrulewidth}{0.0pt}
\lfoot{}
\cfoot{}
\rfoot{}

%\doublespacing


% Incluye los archivos deseados - El contenido de
% su proyecto de grado/pasantia larga.
% Marco Teorico.
\chapter{El Problema de Enrutamiento de Vehículos} \label{chap:vrp}


La distribución de bienes es una tarea fundamental hoy en día para muchas empresas y organizaciones. Pro\-ble\-mas como conseguir la distancia más corta entre dos establecimientos, satisfacer demandas de clientes y mejorar el tiempo de la entrega son factores que influyen en el costo final de un servicio. Teniendo en cuenta lo importante que es la distribución para muchos aspectos en la sociedad los investigadores han estudiado formas de reducir los costos, haciendo la distribución más eficiente. El Problema de Enrutamiento de Vehículos (\textbf{VRP} por sus siglas en inglés de Vehicle Routing Problem) encapsula las restricciones y definiciones de la distribución de bienes en su forma más general. El problema fue propuesto originalmente por Dantzig y Ramser en 1959 \cite{primervrp} mientras intentaban encontrar maneras más eficientes de entregar gasolina a estaciones de servicio utilizando camiones como medio de transporte. \\

VRP está formado de los siguientes elementos:
\begin{itemize}
	
	\item Un depósito que almacena los bienes.
	\item Un conjunto de clientes que demandan una cantidad exacta de bienes.
	\item Un conjunto de vehículos cuyo objetivo es transportar los bienes entre el depósito y los clientes.

\end{itemize}

La resolución exacta del problema es encontrar las rutas que deben recorrer los vehículos para satisfacer las demandas de los clientes de tal manera de minimizar el costo (o distancia) de las rutas.

\section{Descripción} \label{sect:descripcion}

Se tiene un depósito que contiene una cantidad finita de bienes, los cuales se deben distribuir utilizando una cantidad de vehículos. Éstos deben recorrer los clientes satisfaciendo sus demandas de productos cumpliendo las siguientes restricciones:

\begin{itemize}

	\item Todos los vehículos deben comenzar su recorrido en el depósito y teminar en el depósito.
	\item Cada cliente debe ser visitado exactamente por un sólo vehículo una única vez.
	\item Se deben satisfacer las demandas de todos los clientes en su totalidad.
	
\end{itemize}

La resolución del problema consiste en minimizar el costo total de todos los recorridos de los vehículos cum\-plien\-do las restricciones mencionadas anteriormente. \\


VRP pertenece al conjunto de problemas \textit{NP-Hard
}\footnote{NP-hard: una clase de problemas de decisión que pueden ser resueltos por una máquina de turing no determinista en tiempo polinomial.}. Debido a esto, la utilización de métodos exactos resulta inconveniente para resolver el problema, ya que su tiempo de resolución incrementa de forma exponencial a medida de que las instancias aumentan de tamaño (número de clientes). Generalmente para resolver este tipo de problemas se utilizan algoritmos que logran resultados aproximados de buena calidad con tiempos reducidos. Estos algoritmos son llamados metaheurísticas y han demostrado ser de gran ayuda, siendo así un área de investigación de gran interés dentro de la computación.

Más adelante se describirán las metaheurísticas utilizadas.

\begin{comment}
\section{Formulación} \label{sect:formulacion}

VRP puede ser definido formalmente utilizando la formulación propuesta por Level y Palhazi\cite{tesisdanielernesto}: \\

Usualmente el problema es modelado utilizando un grafo, donde los vértices representan los clientes (incluyendo el depósito) y los arcos llevan asociados un costo (que usualmente es visto como la distancia entre dos puntos).

Sean $n$ el número de clientes, $k$ el número de vehículos disponibles y $G = (V, E)$ un grafo completo donde $V = \{0,1,...,n\}$ es el conjunto de vértices y $E = \{(i,j)|(i,j) \in V \times V \ \wedge \ i \not= j\}$ es el conjunto de arcos. El vértice $0$ representa al depósito y los vértices $q = 1,2,...,n$ corresponden a los clientes que deben ser suplidos. Los arcos representan los costos de los caminos que puede utilizar la red de transporte.

Una solución esta formada por una conjunto de rutas $R = \{r_1,r_2,...,r_k\}$, las cuales deben cumplir las siguientes condiciones:

\begin{itemize}

	\item Cada ruta $r_i \in R$ debe iniciar y terminar en el depósito.
	\item Cada vértice $v_i \in V - \{0\}$ debe ser visitado una y sólo una vez.

\end{itemize}
\end{comment}


\section{Variantes de VRP} \label{sect:variantes}

En la vida real muchos problemas de distribución han tratado de modelarse utilizando VRP, sin embargo resulta complejo, por lo que ha surgido la necesidad de crear nuevas formas de modelado más específicas de acuerdo a cada tipo de problema, en algunos casos agregando restricciones y en otros relajando algunas de ellas. Estas nuevas formas de modelado son llamadas variantes. A continuación se mencionan algunas:

El Problema de Enrutamiento de Vehículos con Capacidad (CVRP por sus siglas en inglés de Capacited Vehicle Routing Problem) se define por contar con una flota de vehículos homogéneos que poseen una capacidad máxima fija. Este elemento adicional crea una nueva restricción al problema: no permitir que los bienes dentro del vehículo excedan su capacidad máxima en ningún momento.

En el \vrp con Ventanas de Tiempo (VRPTW por sus siglas en inglés de \vrpingles with Time Windows) se define para cada cliente un rango de tiempo o \emph{ventana de tiempo}, durante el cual debe ser atendido por algún vehículo. Por lo tanto un vehículo puede atender a un cliente si y sólo si el vehículo llega al cliente dentro su ventana de tiempo.

El \vrp con Recolección (VRPB por sus siglas en inglés de Vehicle Routing Problem with Backhauls) requiere que los clientes se clasifiquen en dos conjuntos disjuntos: clientes que reciben bienes del depósito (llamados consumidores o \textit{linehauls}) y clientes que entregan bienes al depósito (llamados pro\-ve\-e\-do\-res o \textit{backhauls}). En esta variante se requiere que los consumidores sean atendidos antes que los proveedores, permitiendo así asegurar que los vehículos nunca superen su capacidad máxima, entregando todo lo que necesitan entregar antes de recoger nuevos bienes.

El \vrp Estocástico (SVRP por sus siglas en inglés de Stochastic Vehicle Routing Problem) se caracteriza por poseer uno o varios componentes aleatorios. Podrían haber nuevos clientes durante la ejecución, las demandas podrían ser aleatorias y conocidas sólo al momento de que un vehículo llegue al cliente.

%El \vrp Dinámico (DVRP por las siglas en inglés de Dynamic Vehicle Routing Problem) es una variante compleja ya que  mientras se construye la ruta aún no se sabe información completa de las demandas de ciertos clientes. Además, luego de construidas las rutas la información de las demandas podría cambiar dinámicamente, por lo tanto mientras los vehículos se desplazan podría haber cambios que influyan positivamente o negativamente en el costo total de las rutas. 

El Problema de Enrutamiento de Vehículos con Múltiples Depósitos (VRPMD por sus siglas en inglés de \vrpingles with Multiple Depots) agrega depósitos adicionales al problema general. Por lo tanto una flota de vehículos distinta se encarga de atender a los clientes.

El Problema de Enrutamiento de Vehículos con Múltiples Vehículos (VRPMV por las siglas en inglés de Vehicle Routing Problem with Multiple Vehicles) tiene como elemento clave que la distribución de los bienes no esté limitada a un sólo vehículo por cliente, ya que en algunas situaciones la demanda de bienes de parte de un cliente es mayor a la capacidad máxima del vehículo. Elimina la restricción de que un cliente sea atendido por sólo un vehículo.

A continuación se pasa a describir una variante de particular interés de este proyecto, como lo es VRPSPD.

%\section{Aplicaciones de VRP} \label{sect:aplicaciones}
%
%En los sectores productivos es muy común observar la necesidad de distribuir bienes, así como recolectar materiales. Cada día aumenta más la cantidad de clientes en una cadena de distribución, por lo que ha sido necesario desarrollar formas más eficientes de realizar entregas y recolecciones óptimas con el fin de disminuir los costos. El \vrp se ha investigado con mucho hincapié para resolver este problema. La utilización de algoritmos computacionales ha resultado, a medida que se profundizan las investigaciones, beneficiosa para reducir los costes de distribución en las organizaciones.
%
%VRP ha sido modelado matemáticamente con el objetivo de que pueda ser representado computacionalmente. Esta representación ha llevado a muchas empresas e industrias a llevar a cabo implementaciones en máquina que resuelvan VRP y así contribuir a minimizar los costos de distribución. Algunas aplicaciones fueron implementadas por algunos investigadores:
%
%\begin{description}
%\item[Golden y Wasil\cite{aplicacionvrp1}: ] Utilizaron un software para optimizar la distribución de la industria de bebidas. Tanto de ventas preordenadas por los clientes, como entrega de bebidas compradas por los clientes al conductor del vehículo. Estudios concluyeron en que en Estados Unidos el 49.8\% de las ventas fueron realizadas directamente por el chofer, lo que implica que los costos asociados a la distribución de las bebidas eran enormes. Se realizaron implementaciones de tres casos de estudio, buscando reducir los costos en un 2-5\%.
%\item[Evans y Norback\cite{aplicacionvrp2}: ] Implementaron un programa para reducir los costos de distribución de alimentos en la empresa Kraft Inc. Utilizaron un sistema basado en heurísticas utilizando gráficos de computadora para ilustrar las largas rutas de su cadena de distribución. En pruebas, el personal de Kraft logró reducir los costos operativos en 10.7\% en problemas de hasta 223 paradas en clientes.
%
%\end{desocription}
\section{VRP con Recepción y Entrega Simultánea (VRPSPD)} \label{sect:vrpspd}

El \vrpspd (VRPSPD por sus siglas en inglés de Vehicle Routing Problem with Simultaneous Pickup and Delivery) fue abordado por primera vez por Min \cite{primervrpspd} en el año 1989. En esta variante los clientes pueden, de manera simultánea, recibir y entregar bienes. Por lo tanto, cada cliente tiene asociado una cantidad de bienes por recibir y por entregar. 

\subsection{Modelo Matemático} \label{subsect:vrpspdFor}

\subsubsection{Características del Problema}

\begin{itemize}
\item Las demandas de entrega y recepción de bienes son determinísticas.
\item Las demandas de recepción son atendidas exclusivamente desde el depósito.
\item No hay intercambio de bienes entre clientes.
\item Las demandas de entrega y recepción de bienes no pueden ser divididas.
\item Todo cliente es visitado una sola vez.
\item No hay restricción de distancia máxima recorrida para los vehículos.

\end{itemize}

\subsubsection{Notación}

\begin{tabular}{ll}
{$n:\,$} 		& {Número total de clientes.}  \\ 
{$Q:\,$} 		& {Capacidad de los vehículos.} \\ 
{$k:\,$}  		& {Número de vehículos.}\\
{$i,j:\,$}		& {Índices para el deposito y los clientes. El depósito se denota con el número 0.}\\
{$l:\,$}    	& {Índice para los vehículos.}\\
{$c_{ij}:\,$} 	& {Distancia (costo) entre el cliente $i$ y el cliente $j$.}\\
{$p_j:\,$}		& {Bienes a ser entregados por un cliente $j$, donde $j=1,2,…,n$.}	\\	
{$d_j:\,$}		& {Bienes a ser recogidos por un cliente $j$, donde $j=1,2,…,n$.} \\
{$x_{ijl}$} 	& {Variable de decisión que toma el valor de 1 si el vehículo $l$ visita al cliente $i$ antes } \\
{$ $}              & {que al cliente $j$, de lo contrario es 0.} \\
{$y_{ij}$} 		& {Bienes recogidos hasta el cliente $i$ y transportados hasta el cliente $j$.} \\
{$z_{ij}$}		& {Bienes por entregar desde el cliente $i$ y transportados hasta el cliente $j$.}\\

\end{tabular} 

\subsubsection{Formulación Matemática}

La función objetivo ~\eqref{eq:FOBJ} correspondiente a esta formulación matemática es la siguiente: 


\begin{equation}\label{eq:FOBJ}
 Min \sum_{l=1}^k\sum_{i=0}^n\sum_{j=0}^n  c_{ij} x_{ijl} 
\end{equation}

Sujeto a:

\begin{equation}\label{eq:FM1}
\sum_{i=0}^n \sum_{l=1}^k x_{ijl}=1 , \: j=1,2,…,n 
\end{equation}


	La restricción ~\eqref{eq:FM1} asegura que cada cliente es visitado exactamente una vez por un sólo vehículo.


\begin{equation}\label{eq:Ft2}
\sum_{i=0}^n x_{ijl} - \sum_{i=0}^n x_{jil} = 0, \: j=0,1,…,n, l=1,2,…,k  
\end{equation}	
	
	La restricción ~\eqref{eq:Ft2} asegura que el mismo vehículo entra y sale de cada cliente que visita.


\begin{equation}\label{eq:FM3}
\sum_{j=1}^n x_{0jl} \leq 1 , \: l=1,2,…,k   
\end{equation}

	La restricción ~\eqref{eq:FM3} representa que la cantidad de vehículos utilizados es menor o igual que  $k$. 



\begin{equation}\label{eq:FM4}
\sum_{i=0}^n y_{ji} - \sum_{i=0}^n y_{ij} = p_j, \: \forall j \neq 0  
\end{equation}

\begin{equation}\label{eq:FM5}
\sum_{i=0}^n z_{ji} - \sum_{i=0}^n z_{ij} = d_j, \: \forall j \neq 0  
\end{equation}


Las restricciones ~\eqref{eq:FM4} y ~\eqref{eq:FM5} son ecuaciones de flujo para la demanda de recepción y de entrega de mercancía.


\begin{equation}\label{eq:FM6}
(y_{ij}+ z_{ij})  \leq Q \sum_{l=1}^k x_{ijl}   , \: i,j=0,1,…,n  
\end{equation}

La restricción ~\eqref{eq:FM6} establece que los bienes  de recepción y de entrega solo pueden ser transportados por arcos pertenecientes a la solución y para ninguno de ellos se excede de la máxima capacidad permitida.

\begin{equation}\label{eq:FM7}
x_{ijl}\in \lbrace0,1\rbrace,  y_{ij} \geq 0 ,z_{ij} \geq 0,   i,j=0,1,…,n,   l=0,1,…,k     
\end{equation}		

		Finalmente  la restricción ~\eqref{eq:FM7} muestra la naturaleza de las variables de decisión.

%	Existen  $k$ vehículos en el depósito, V representa al conjunto de clientes que serán visitados, donde $n = |V|$ es el número total de clientes. La ubicación del depósito y los clientes es conocida.  Todo cliente posee una cantidad conocida de mercancía demandada y de mercancía a devolver, representadas por $d_j$   y $p_j$ respectivamente con valores desde $j=1,2,…,n$.  Toda ruta debe comenzar en el depósito y debe terminar en el mismo. Todos los clientes son visitados exactamente una vez y sus necesidades son satisfechas por un solo vehículo. $V_0 = V \cup {0}$  , es el conjunto de clientes más el depósito; $c_{ij}$ es la distancia entre el cliente $i$ y el cliente $j$, y la capacidad del vehículo es representada por $Q$. $x_{ijl}$ representa la variable de decisión que toma el valor de 1 si el vehículo $l$ visita visita al cliente $i$ antes que al cliente $j$, de lo contrario es 0. $y_{ij}$  representa los bienes recogidos hasta el cliente $i$ y transportados hasta el cliente $j$. $z_{ij}$ Representa los bienes por entregar desde el cliente $i$ y transportados hasta el cliente $j$.


%Puedes quitar esto(es opcional)
\vspace{5 mm}



\chapter{Metaheurísticas} \label{chap:metaheuristicas}

\section{Definición} \label{sect:definicion}

Ya que el problema de enrutamiento de vehículos es un problema complejo, resolverlo exactamente implica un incremento exponencial del tiempo de cómputo a medida que las instancias aumentan de tamaño. Para problemas complejos usualmente se trata de conseguir una buena solución que tome un tiempo aceptable. Una forma de lograr esto es utilizando las denominadas \emph{metaheurísticas}. El objetivo de esta clase de algoritmos  es lograr un balance entre obtener buenos tiempos de ejecución y buenas soluciones a través de un proceso iterativo. La palabra \emph{heurística} proviene del griego $\varepsilon\upsilon\rho\iota\sigma\kappa\varepsilon\iota\nu$, que significa \emph{``hallar, encontrar''}. El prefijo \emph{meta} quiere decir \emph{``nivel superior''}. Juntas podrían interpretarse como `mejorar lo encontrado'.

Las metaheurísticas son una familia de algoritmos aproximados de propósitos generales y no determinísticos que consiste de procesos iterativos que guían una heurística subordinada, realizando sobre el espacio de búsqueda exploración y explotación.
\begin{itemize}
\item \textbf{Exploración:} Busca en el espacio de soluciones lugares nunca antes visitados. 
\item \textbf{Explotación:} Regresa a lugares del espacio de soluciones ya visitados para obtener soluciones anteriormente calificadas como buenas.
\end{itemize}

\section{Clasificación} \label{sect:clasificacion}
Una metaheurística puede ser:

\begin{itemize}
\item \textbf{Constructiva:} Construye paso a paso una solución del problema, basándose en la mejor elección de cada iteración.
\item \textbf{De trayectoria:} Parten de una solución inicial y la mejoran progresivamente hasta lograr el óptimo.
\item \textbf{Poblacional:} Basado en conjunto de soluciones que evolucionan sobre el espacio de búsqueda.
\end{itemize}

\section{Metaheurísticas de interés} \label{sect:interes}

A continuación una breve descripción de las metaheurísticas más relevantes para este trabajo.

\subsection{Búsqueda Local} \label{subsect:ls}

La \emph{búsqueda local} (LS por sus siglas en inglés de \emph{Local Search}) es uno de los métodos más empleados para resolver problemas complejos. Su orígen proviene del año 1950 donde fue utilizada por primera vez para resolver el problema TSP.\footnote{Siglas en inglés de Traveling Salesman Problem, problema en donde un vendedor debe visitar todos los clientes una única vez, reduciendo su desplazamiento al mínimo} 
\\
Como descrito en \cite{LocalSearchVNDDef} el algoritmo parte de una solución inicial, que se representa en la solución actual, de la cual se obtiene una vecindad parcial o completa de nuevas soluciones. A partir de esta vecindad se elige una nueva solución actual, siempre y cuando  la solución sea mejor que la solución actual de acuerdo a un criterio previamente establecido. Este proceso se repite hasta alcanzar un criterio de parada previamente definido. 

El pseudocódigo general de LS se puede ver en el \textbf{Algoritmo~\ref{alg:LSS}}.

\begin{code}[includerangemarker=false,frame=single,label=alg:LSS,caption=Pseudocódigo de Búsqueda Local,firstnumber=100, mathescape]
$S_0$ := ConstruirSolucionInicial()
$S_{actual}$ := $S_0$

while ($\neg$criterio_parada) do
	vecindad := ConstruirVecindad($S_{actual}$)
	Seleccionar $S$ $\in$ vecindad | $S$ := Min{Costo($S$)}
	if (Costo($S$) $<$ Costo($S_{actual}$))	
		$S_{actual}$ := S
	end if		
end while

$\textbf{return}$ $S_{actual}$
\end{code}

\subsection{Búsqueda Local Iterada} \label{subsect:ils}

La \emph{búsqueda local iterada} (ILS por sus siglas en inglés de \textit{Iterated Local Search}) es una metaheurística de trayectoria que consiste en iterativamente perturbar la solución actual a la cual se le aplica una heurística de mejoramiento. Una perturbación consiste en aplicar un operador de manera aleatoria, obteniendo de esta forma una nueva solución. La fuerza de una perturbación se refiere al número de componentes de la solución que son modificadas, es importante destacar que si la perturbación es muy fuerte el ILS puede comportarse como si su solución actual fuera aleatoria, mientras que si la perturbación es muy leve la solución actual normalmente caerá de nuevo en el mismo óptimo local, dejando un espacio de diversificación muy limitado \cite{ILSDef}. 
\\
El pseudocódigo general de ILS se puede ver en el \textbf{Algoritmo~\ref{alg:ILS}}.

\begin{code}[includerangemarker=false,frame=single,label=alg:ILS,caption=Pseudocódigo de Búsqueda Local Iterada,firstnumber=100, mathescape]
$S_0$ := ConstruirSolucionInicial()
$S_{mejor}$ := LocalSearch($S_0$)
while ($\neg$criterio_parada) do

	$S_{perturb}$ := Perturbar($S_{mejor}$)
	$S_{actual}$ := LocalSearch($S_{perturb}$)

	if(Costo($S_{mejor}$) $>$ Costo($S_{actual}$))		
		$S_{mejor}$ := $S_{actual}$		
	end if
			
end while

$\textbf{return}$ $S_{mejor}$
\end{code}

\subsection{Búsqueda en Vecindades Variables} \label{subsect:vnd}

La \emph{búsqueda en vecindades variables} (VNS por sus siglas \emph{Variable Neightborhood Search}) es una metaheurística de trayectoria que consiste en cambiar sistemáticamente de vecindad dentro de una búsqueda local, además de un método de perturbación para diversificar la solución. Una de sus variantes, VND (\emph{Variable Neightborhood Descent} o \emph{Búsqueda Descendente en Vecindades Variables}) se obtiene cuando los cambios de vecindarios son realizados de manera determinística. 
\\
El pseudocódigo general de VND se puede ver en el \textbf{Algoritmo~\ref{alg:VNS}} \cite{VNDDef}.

\begin{code}[includerangemarker=false,frame=single,label=alg:VNS,caption=Pseudocódigo de Búsqueda Descendente en Vecindades Variables,firstnumber=100, mathescape]
Seleccionar vecindades $N_{k}$ para 
k := 1,...,$K_{max}$ 
$S_0$ := ConstruirSolucionInicial()
$S_{mejor}$ := $S_0$
while ($\neg$criterio_parada) do
    k := 1
    while (k < $k_{max}$+1) do 
    	$vecindad$ := ConstruirVecindadVND($N_{k}$,($S_{mejor}$))
		Seleccionar $S_{actual}$ $\in$ $vecindad$ | $S_{actual}$ := Min{Costo($S_{actual}$)}	
		if(Costo($S_{mejor}$) > Costo($S_{actual}$))		
			$S_{mejor}$ := $S_{actual}$
			k := 1
		else
		 k := k+1
		end if 			
	end while
end while

$\textbf{return}$ $S_{mejor}$
\end{code}

\subsection{Búsqueda Tabú} \label{subsect:ts}

La \emph{búsqueda tabú} (TS por sus siglas en inglés de \emph{Tabu Search}) es una metaheurística de trayectoria propuesta por Fred Glover en 1986 \cite{tabusearch} con el objetivo de ayudar a la \emph{búsqueda local} a evitar el estancamiento en óptimos locales, permitiendo movimientos dentro del espacio de soluciones que no necesariamente impliquen un mejor valor de función objetivo que el valor actual.
Para evitar el fenómeno cíclico de la búsqueda local y así impedir volver a visitar soluciones anteriormente visitadas, éstas se prohiben mediante la utilización de memoria a corto plazo, empleando la llamada \emph{lista tabú}, cuyo propósito es almacenar la historia reciente de la búsqueda. 
Estas soluciones ya visitadas son declaradas \emph{tabú}, es decir, su posterior exploración está prohibida por un determinado número de iteraciones definidas por el valor de la \emph{tenencia tabú}. 

Usualmente es impráctico guardar soluciones en la lista tabú por razones de uso excesivo de recursos. Por este motivo normalmente se guardan movimientos específicos entre soluciones permitiendo así un uso eficiente de recursos y buenos resultados.

Suele ocurrir que la búsqueda tabú evite movimientos que dirijan a muy buenas soluciones como resultado de la utilización de la lista tabú, por lo que generalmente se define un \emph{criterio de aspiración}, que permite a la búsqueda ignorar el hecho de que una solución se encuentre en la lista tabú sólo si esa solución conlleva un valor de función objetivo mejor que el de la mejor solución encontrada hasta el momento.

El pseudocódigo general de TS se puede ver en el \textbf{Algoritmo~\ref{alg:TS}}:

\begin{code}[includerangemarker=false,frame=single,label=alg:TS,caption=Pseudocódigo de Búsqueda Tabú,firstnumber=100, mathescape]
$S_0$ := ConstruirSolucionInicial()
$S_{actual}$ := $S_0$
$S_{mejor}$ := $S_0$
$Lista\_Tabu$ := $\emptyset$

while ($\neg$criterio_parada) do
	$vecindad$ := ConstruirVecindad($S_{actual}$)
	Seleccionar $S$ $\in$ $vecindad$ | $S$ = Min{Costo($S$)}
	if (Costo($S$) < Costo($S_{mejor}$))	        // criterio de aspiracion
		$S_{mejor}$ := $S$
	else
		if ($S$ $\in$ $Lista\_Tabu$)
			Seleccionar $S$ $\in$ $vecindad$ | $S$ := Min{Costo($S$)} $\wedge$ $S$ $\notin$ $Lista\_Tabu$
		end if
	end if
	$S_{actual}$ := $S$
	Guardar $S_{actual}$ en $Lista\_Tabu$ y actualizar tenencia si necesario
end while

$\textbf{return}$ $S_{mejor}$
\end{code}

\subsection{Búsqueda Local Guiada} \label{subsect:gls}

La \emph{búsqueda local guiada} (GLS por sus siglas en inglés de \emph{Guided Local Search}) es una metaheurística de trayectoria diseñada para guiar la \emph{búsqueda local} penalizando características no deseables de la solución actual, para así evitar el estacamiento en óptimos locales. Primeramente se definen las características de una solución candidata, cuando la búsqueda se atasca en un óptimo local las soluciones que posean  ciertas características consideradas no deseables son penalizadas. Usualmente, se utiliza una función de utilidad tal que la característica que maximice la función será penalizada. La característica penalizada y el número de veces que ha sido penalizada será almacenada de manera que en búsquedas posteriores el valor de la función objetivo se vea alterado. De esta manera, la búsqueda se concentra en explorar mayormente soluciones calificadas como deseables y permite prevenir que dirija todo su esfuerzo en explorar sólo una región del espacio de soluciones.

El pseudocódigo general de GLS se puede ver en el \textbf{Algoritmo~\ref{alg:GLS}}:

\begin{code}[includerangemarker=false,frame=single,label=alg:GLS,caption=Pseudocódigo de Búsqueda Local Guiada,firstnumber=100, mathescape]
Funcion $U$
Conjunto $C$

$S_0$ := ConstruirSolucionInicial()
$S_{actual}$ := $S_0$
$S_{mejor}$ := $S_0$

while ($\neg$criterio_parada) do
	$S_{mejor}$ := LocalSearch($S_{actual}$)
	Encontrar la caracteristica $C_i$ | $C_i$ = Max {$U(i)$}
	Penalizar($C_i$)
	$S_{actual}$ := $S_{mejor}$
end while

$\textbf{return}$ $S_{mejor}$
\end{code}

\subsection{Optimización Basada en Colonia de Hormigas} \label{subsect:aco}

La \emph{optimización basada en colonia de hormigas} (ACO por sus siglas en inglés de \emph{Ant Colony Optimization}) es una metaheurística poblacional inspirada en los rastros de feromonas dejados por las hormigas al caminar y el comportamiento de ellas ante las feromonas. Estas feromonas son utilizadas por las hormigas como medio de comunicación con el fin de conseguir una forma óptima para llegar a la comida. Análogamente ACO está basado en la comunicación indirecta de las hormigas artificiales mediante los rastros de feromonas artificiales. 

Se tiene un número fijo de hormigas, cada hormiga construye una solución propia basándose en información subyacente a caminos de feromonas dejados por otras hormigas. Si no existen feromonas en los caminos a tomar, se escoje un camino aleatorio. Si existen feromonas en los caminos a tomar, se elige probabilísticamente dependiendo de la cantidad de feromonas depositada en los caminos. Mientras más feromonas tenga un camino mayor probabilidad tiene de ser elegido. Cada hormiga deposita feromonas en su recorrido. 

Para evitar que el flujo de hormigas se dirija siempre por el camino de más feromonas, acarreando un es\-tan\-ca\-mien\-to en un óptimo local, se define un factor de evaporación, que reduce la cantidad de  feromonas en los caminos cada cierta cantidad de iteraciones o cada cierto tiempo. De esta manera se explorará eficazmente el espacio de soluciones, logrando soluciones de calidad.

El pseudocódigo general de ACO se puede ver en el \textbf{Algoritmo~\ref{alg:ACO}}:

\begin{code}[includerangemarker=false,frame=single,label=alg:ACO,caption=Pseudocódigo de Optimización basada en Colonia de Hormigas,firstnumber=100, mathescape]
hormiguero := CrearHormigas()
InicializarFeromonas()
$S_{mejor}$ := ConstruirSolucionInicial()

while ($\neg$criterio_parada) do
	$\textbf{foreach}$ $hormiga \in$ hormiguero do
		$S_{actual}$ := ConstruirSolucion($hormiga$)
		ActualizarFeromonas($S_{actual}$)
		if (Costo($S_{actual}$) < Costo($S_{mejor}$))
			$S_{mejor}$ := $S_{actual}$
		end if
	end $\textbf{foreach}$	
	EvaporarFeromonas()
end while

$\textbf{return}$ $S_{mejor}$
\end{code}

\subsection{Optimización por Enjambre de Partículas} \label{subsect:pso}

La \emph{optimización por enjambre de partículas} (PSO por sus siglas en inglés de \emph{Particle Swarm Optimization}) es una metaheurística poblacional inspirada en el comportamiento en grupo de enjambres. PSO imita los movimientos físicos individuales en los enjambres como método de búsqueda. 

Un enjambre está compuesto por un número definido de partículas, las partículas tienen dos características esenciales: posición y  velocidad. La posición de una partícula representa una solución al problema, mientras que la velocidad representa la habilidad de la partícula  para cambiar de posición. Utilizando estos dos factores, se intenta encontrar la mejor posición posible dentro de un espacio de búsqueda, la  cual se espera represente una buena solución al problema.

La velocidad de una partícula es constantemente actualizada basándose en tres términos: (1) Inercia, obliga a la partícula a moverse en la misma dirección. (2) Aprendizaje cognitivo, obliga a la partícula a moverse a la mejor posición encontrada hasta el momento por ella. (3) Aprendizaje social, obliga a la partícula a moverse a la mejor posición encontrada hasta el momento por todas las partículas del enjambre.
La velocidad de cada partícula es influenciada por los factores cognitivos y sociales con el fin de dirigir el enjambre a encontrar mejores soluciones al problema. Además, se incluye un factor estocástico para evitar que todas las partículas con misma posición se muevan en las mismas direcciones y así explorar extensivamente el espacio de soluciones.

El pseudocódigo general de PSO se puede ver en el \textbf{Algoritmo~\ref{alg:PSO}}:

\begin{code}[includerangemarker=false,frame=single,label=alg:PSO,caption=Pseudocódigo de Optimización por Enjambre de Partículas,firstnumber=100, mathescape]
$\emph{enjambre}$ := CrearEnjambre()
InicializarPosiciones($\emph{enjambre}$)
InicializarVelocidades($\emph{enjambre}$)
while ($\neg$criterio_parada) do
	$\textbf{foreach}$ $\emph{p}$ $\in$ $\emph{enjambre}$ do
		ActualizarVelocidad($\emph{p}$)
		ActualizarPosicion($\emph{p}$)
		if (Costo($\emph{p}$) < MejorCosto($\emph{p}$))
			ActualizarMejorCosto($\emph{p}$)
		end if
		if (Costo($\emph{p}$) < MejorCosto($\emph{enjambre}$))
			mejor_particula := $\emph{p}$
		end if
	end $\textbf{foreach}$
end while
$S_{mejor}$ := ObtenerPosicion(mejor_particula)

$\textbf{return}$ $S_{mejor}$
\end{code}

\subsection{Búsqueda en Dispersión} \label{subsect:sca}

La \emph{búsqueda en dispersión} (SS por sus siglas en inglés de \emph{Scatter Search}) es una metaheurística poblacional la cual está diseñada para operar a través de un conjunto de soluciones, llamado conjunto de referencia, el cual está constituido por buenas soluciones obtenidas hasta el momento. Es importante destacar que una buena so\-lu\-ción puede ser considerada de esta forma tanto por su diversidad, como por el valor de su función objetivo. Este enfoque genera sistemáticamente nuevas soluciones a partir del conjunto de referencia a través de combinaciones de soluciones pertenecientes al mismo.    
\\
El algoritmo básico SS consta de cinco métodos:
\\
\begin{enumerate}
 
\item Un \textit{Método de generación diversificada} que crea un conjunto de soluciones iniciales.

\item Un \textit{Método de mejora} que mejora la calidad de las soluciones.

\item Un \textit{Método de actualización del conjunto de referencia} que construye el conjunto de referencia con soluciones buenas y diversas.

\item Un \textit{Método de generación de subconjuntos} que determina cuales soluciones del conjunto de referencia sirven como base para la creación de nuevas soluciones.

\item Un \textit{Método de combinación de soluciones} que genera nuevas soluciones a partir de los subconjuntos creados por el método de generación de subconjuntos.

\end{enumerate}

El pseudocódigo general de SS se puede ver en el \textbf{Algoritmo~\ref{alg:SCA1}}

\begin{code}[includerangemarker=false,frame=single,label=alg:SCA1,caption=Pseudocódigo de Búsqueda en Dispersión,firstnumber=100, mathescape]
$Metodo$ $de$ $Generacion$ $y$ $Diversificacion$ - Crea un conjunto de soluciones diversas $P$ de tamano $Psize$

Construir el $RefSet$ con $b$ mejores y diversas soluciones de $P$, RefSet := $\lbrace S^1,...,S^b \rbrace$. Ordenarlas de manera creciente en base a su costo.  

$S_{mejor}$ := $S^1$
$NewSolutions$ := True
while ($NewSolutions$) do
	$NewSubset$ := GenerarSubconjuntos($RefSet$)
	$NewSolutions$ := False
    while ($NewSubset\neq\emptyset $) do
    	$Ss$ := Seleccionar($NewSubset$)
    	$S$ := Combinar(Ss) 
    	$x$ := Mejorar(S)
		if(($x$ $\notin$ $RefSet$) $\wedge$ ($Costo(x)<Costo(S^b$)))
			$S^b$ := $x$
			Ordenar($RefSet$)
			$NewSolutions$ := True											
		end if
		Eliminar($Ss$, $NewSubset$)    	    				
	end while
end while
$S_{mejor}$ := $S^1$

$\textbf{return}$ $S_{mejor}$
\end{code}


\subsection{Algoritmo Genético} \label{subsect:GA}

El \emph{algoritmo genético} (GA por sus siglas en inglés de \emph{Genetic Algorithm}) es una metaheurística poblacional utilizada por primera vez por John Holland en el año 1975. El enfoque de GA representa una analogía biológica en donde se simula el proceso evolutivo de una \textit{población} que refiere a un conjunto donde cada individuo representa una solución del problema. 

Cada solución perteneciente a la \textit{población} posee un valor asociado que recibe el nombre de \textit{fitness}. Éste determina si una solución es considerada como buena o no y aquellas que tengan mayor \textit{fitness} tendrán mayor oportunidad para aparearse y sobrevivir al paso del tiempo, lo cual permite que la población evolucione y obtenga cada vez mejores soluciones. Lo población inicial es generada de manera aleatoria, luego el algoritmo posee tres procedimientos fundamentales:

\begin{enumerate}
 
\item\textit{Selección:} Se encarga de elegir un subconjunto de la población, tal que las soluciones de mayor \textit{fitness} o calidad tengan mayor oportunidad de reproducirse. 

\item\textit{Cruce:} Se encarga de simular la reproducción de dos soluciones dentro de la población, las cuales para el efecto del GA serán llamadas ``padres''. El cruce da como resultado una o más soluciones que poseen características en común de ambos padres. Estas soluciones reciben el nombre de ``hijos'' para efectos del GA.

\item\textit{Mutación:} realiza una leve modificación de las características de una solución. Este proceso se utiliza para diversificar una nueva población sin necesidad de depender exclusivamente del proceso de cruce. 

\end{enumerate}

El pseudocódigo general de GA se puede ver en el \textbf{Algoritmo~\ref{alg:GA}}.
 
\begin{code}[includerangemarker=false,frame=single,label=alg:GA,caption=Pseudocódigo de Algoritmo Genético,firstnumber=100, mathescape]
t := 0
$P_0$ := GenerarPoblacionInicial()
Evaluar($P_t$)
while ($\neg$criterio_parada) do
	t := t+1
	$P_t$ := Seleccionar($P_{t-1}$)
	Cruzar($P_{t}$)
	Mutar($P_t$)
	Evaluar($P_t$)	
end while
$S$ := SeleccionarMejor($P_t$)

$\textbf{return}$ $S$
\end{code}
% Resultados
\chapter{Resultados} \label{chap:resultados}

A continuación se muestran los resultados de las pruebas realizadas.

\section{Resultados de las pruebas unitarias sobre el Módulo de Determinación de Origen de la Incompletitud.} 

\section{Resultados de las pruebas unitarias sobre el Módulo de Búsqueda Focalizada de Información}


\begin{table}[h]
\caption{Resultados detallados la evaluación del Preprocesador de Textos}
\centering
\scriptsize
\begin{tabular*}{1\textwidth}{@{\extracolsep{\fill}} !{\vrule width 1pt} c !{\vrule width 1pt} c | c | c!{\vrule width 1pt} c | c | c!{\vrule width 1pt}}
\hline
Dominio & \multicolumn{3}{c!{\vrule width 1pt}}{\bf{Aprobados}} & \multicolumn{3}{c |}{\bf{No aprobados}}\\
\hline
 & Correctos & Con texto de más & \bf{Total aprobados} & Incompletos & Incorrectos & \bf{Total no aprobados}\\
\hline
Designaciones & 87.69\% & 7.69\% & \bf{95.39\%} & 3.07\% & 1.54\% & \bf{4.61\%}\\
\hline
Escalafón & 80.51\% & 12.98\% & \bf{93.6\%}  & 6.4\% & 0\% & \bf{6.4\%} \\
\hline
Jurados de Ascenso & 77.55\% & 16.32\% & \bf{93.88\%} & 0\% & 6.12\% & \bf{6.12\%} \\
\hline
\end{tabular*}
\label{tabla-resultados-preprocesamientoDatosDesignacion}

\end{table}

%------------------------

\begin{comment}
\begin{table}[h]
\caption{Resultados detallados la evaluación del Preprocesador de Textos: Escalafón}
\centering
\scriptsize
\begin{tabular*}{.68\textwidth}{@{\extracolsep{\fill}} !{\vrule width 1pt} c !{\vrule width 1pt} c | c !{\vrule width 1pt} c | c !{\vrule width 1pt}}
\hline
Dominio & \multicolumn{2}{c!{\vrule width 1pt}}{\bf{Aprobados}} & \multicolumn{2}{c |}{\bf{No aprobados}}\\
\hline
 & Correctos & Con texto de más & Incompletos & Incorrectos \\
\hline
\multirow{2}{*}{Designaciones} & 80.51\% & 12.98\% & 6.4\% & 0\% \\\cline{2-5}
& \multicolumn{2}{c!{\vrule width 1pt}}{Aprobados: \bf{93.6}} & \multicolumn{2}{c |}{No aprobados: \bf{6.4\%}}\\
\hline
\end{tabular*}
\label{tabla-resultados-preprocesamientoDatosEscalafon}
\\El conjunto de prueba es de tama~no 77, un tercio de la población.
\end{table}
\end{comment}

%------------------------

\begin{table}[h]
\caption{ Resultados de la evaluación del Extractor Focalizado - Dominio: Designaciones. UnitHit Measure mínimo:.75}
\centering
\scriptsize
\begin{tabular*}{1\textwidth}{@{\extracolsep{\fill}} | c | c | c | c | c | c |}
\hline
Campo & Prob. Campo Faltante & \multicolumn{2}{c|}{\bf{P. Aprobados}} & \multicolumn{2}{c |}{\bf{P. No aprobados}}\\
\hline

\multicolumn{2}{|c|}{ } & Correctos & Con texto de más & Incompletos & Incorrectos \\
\hline
\multirow{8}{*}{EsAsignado.calificacion} 

	& \multirow{2}{*}{0} 
	& 87.69\% & 7.69\% & 3.07\% & 1.54\% \\
	\cline{3-6}
	& & \multicolumn{2}{c|}{Aprobados: \bf{95.38\%}} & \multicolumn{2}{c |}{No aprobados: \bf{4.61\%}}\\
	\cline{2-6}
	
	& \multirow{2}{*}{0.10} 
	& 87.69\% & 7.69\% & 3.07\% & 1.54\% \\
	\cline{3-6}
	& & \multicolumn{2}{c|}{Aprobados: \bf{95.38\%}} & \multicolumn{2}{c |}{No aprobados: \bf{4.61\%}}\\
	\cline{2-6}

	& \multirow{2}{*}{0.25} 
	& 87.69\% & 7.69\% & 3.07\% & 1.54\% \\
	\cline{3-6}
	& & \multicolumn{2}{c|}{Aprobados: \bf{95.38\%}} & \multicolumn{2}{c |}{No aprobados: \bf{4.61\%}}\\
	\cline{2-6}
	
	& \multirow{2}{*}{0.50} 
	& 87.69\% & 7.69\% & 3.07\% & 1.54\% \\
	\cline{3-6}
	& & \multicolumn{2}{c|}{Aprobados: \bf{95.38\%}} & \multicolumn{2}{c |}{No aprobados: \bf{4.61\%}}\\

\hline
	
\multirow{8}{*}{EsAsignado.fechaAsignacion} 

	& \multirow{2}{*}{0} 
	& 87.69\% & 7.69\% & 3.07\% & 1.54\% \\
	\cline{3-6}
	& & \multicolumn{2}{c|}{Aprobados: \bf{95.38\%}} & \multicolumn{2}{c |}{No aprobados: \bf{4.61\%}}\\
	\cline{2-6}
	
	& \multirow{2}{*}{0.10} 
	& 87.69\% & 7.69\% & 3.07\% & 1.54\% \\
	\cline{3-6}
	& & \multicolumn{2}{c|}{Aprobados: \bf{95.38\%}} & \multicolumn{2}{c |}{No aprobados: \bf{4.61\%}}\\
	\cline{2-6}

	& \multirow{2}{*}{0.25} 
	& 87.69\% & 7.69\% & 3.07\% & 1.54\% \\
	\cline{3-6}
	& & \multicolumn{2}{c|}{Aprobados: \bf{95.38\%}} & \multicolumn{2}{c |}{No aprobados: \bf{4.61\%}}\\
	\cline{2-6}
	
	& \multirow{2}{*}{0.50} 
	& 87.69\% & 7.69\% & 3.07\% & 1.54\% \\
	\cline{3-6}
	& & \multicolumn{2}{c|}{Aprobados: \bf{95.38\%}} & \multicolumn{2}{c |}{No aprobados: \bf{4.61\%}}\\
	\cline{2-6}
\hline

\multirow{8}{*}{EsAsignado.fechaFinal} 

	& \multirow{2}{*}{0} 
	& 87.69\% & 7.69\% & 3.07\% & 1.54\% \\
	\cline{3-6}
	& & \multicolumn{2}{c|}{Aprobados: \bf{95.38\%}} & \multicolumn{2}{c |}{No aprobados: \bf{4.61\%}}\\
	\cline{2-6}
	
	& \multirow{2}{*}{0.10} 
	& 87.69\% & 7.69\% & 3.07\% & 1.54\% \\
	\cline{3-6}
	& & \multicolumn{2}{c|}{Aprobados: \bf{95.38\%}} & \multicolumn{2}{c |}{No aprobados: \bf{4.61\%}}\\
	\cline{2-6}

	& \multirow{2}{*}{0.25} 
	& 87.69\% & 7.69\% & 3.07\% & 1.54\% \\
	\cline{3-6}
	& & \multicolumn{2}{c|}{Aprobados: \bf{95.38\%}} & \multicolumn{2}{c |}{No aprobados: \bf{4.61\%}}\\
	\cline{2-6}
	
	& \multirow{2}{*}{0.50} 
	& 87.69\% & 7.69\% & 3.07\% & 1.54\% \\
	\cline{3-6}
	& & \multicolumn{2}{c|}{Aprobados: \bf{95.38\%}} & \multicolumn{2}{c |}{No aprobados: \bf{4.61\%}}\\
	\cline{2-6}
\hline

\multirow{8}{*}{EsAsignado.motivo} 

	& \multirow{2}{*}{0} 
	& 87.69\% & 7.69\% & 3.07\% & 1.54\% \\
	\cline{3-6}
	& & \multicolumn{2}{c|}{Aprobados: \bf{95.38\%}} & \multicolumn{2}{c |}{No aprobados: \bf{4.61\%}}\\
	\cline{2-6}
	
	& \multirow{2}{*}{0.10} 
	& 87.69\% & 7.69\% & 3.07\% & 1.54\% \\
	\cline{3-6}
	& & \multicolumn{2}{c|}{Aprobados: \bf{95.38\%}} & \multicolumn{2}{c |}{No aprobados: \bf{4.61\%}}\\
	\cline{2-6}

	& \multirow{2}{*}{0.25} 
	& 87.69\% & 7.69\% & 3.07\% & 1.54\% \\
	\cline{3-6}
	& & \multicolumn{2}{c|}{Aprobados: \bf{95.38\%}} & \multicolumn{2}{c |}{No aprobados: \bf{4.61\%}}\\
	\cline{2-6}
	
	& \multirow{2}{*}{0.50} 
	& 87.69\% & 7.69\% & 3.07\% & 1.54\% \\
	\cline{3-6}
	& & \multicolumn{2}{c|}{Aprobados: \bf{95.38\%}} & \multicolumn{2}{c |}{No aprobados: \bf{4.61\%}}\\
	\cline{2-6}
\hline

\multirow{8}{*}{Profesor.Nombre} 

	& \multirow{2}{*}{0} 
	& 87.69\% & 7.69\% & 3.07\% & 1.54\% \\
	\cline{3-6}
	& & \multicolumn{2}{c|}{Aprobados: \bf{95.38\%}} & \multicolumn{2}{c |}{No aprobados: \bf{4.61\%}}\\
	\cline{2-6}
	
	& \multirow{2}{*}{0.10} 
	& 87.69\% & 7.69\% & 3.07\% & 1.54\% \\
	\cline{3-6}
	& & \multicolumn{2}{c|}{Aprobados: \bf{95.38\%}} & \multicolumn{2}{c |}{No aprobados: \bf{4.61\%}}\\
	\cline{2-6}

	& \multirow{2}{*}{0.25} 
	& 87.69\% & 7.69\% & 3.07\% & 1.54\% \\
	\cline{3-6}
	& & \multicolumn{2}{c|}{Aprobados: \bf{95.38\%}} & \multicolumn{2}{c |}{No aprobados: \bf{4.61\%}}\\
	\cline{2-6}
	
	& \multirow{2}{*}{0.50} 
	& 87.69\% & 7.69\% & 3.07\% & 1.54\% \\
	\cline{3-6}
	& & \multicolumn{2}{c|}{Aprobados: \bf{95.38\%}} & \multicolumn{2}{c |}{No aprobados: \bf{4.61\%}}\\
	\cline{2-6}
\hline
\end{tabular*}
\label{tabla-resultados-EFDesignaciones.75}
\\
Prob. Campo Faltante es la probabilidad de que no se tenga el valor uno de los campos que se utilizan para hacer extracción focalizada.
\end{table}

Por que el extractor falla con designaciones? (Observaciones hechas al hacer las pruebas que pueden ser útiles en el análisis de datos)

1. Las unidades de informacion no son precisas. A veces hay multiples designaciones en una misma linea. \\
2. Hay multiples designaciones que coinciden en un mismo día para una misma persona. \\
3. Hay ratificaciones, correcciones, posteriores a la designacoines que modifican el resultado que uno pensaria correcto.\\
4. Hay errores de tipeo por parte de la secretaria.\\
5. Hay casos "patologicos" que es imposible generalizar con expresiones regulares que no sean hechas a la medida. \\
6. Hay casos en los que los campos tienden a tener muchos valores null y al no encontrar la respuesta en el mejor hit, se procede al segundo. Arreglar esto en el extractor? Hay muchos casos en los que un segundo match ayuda. \\

% Conclusiones
\chapter*{Conclusiones y recomendaciones} \label{chap:conclusiones}
\addcontentsline{toc}{chapter}{Conclusiones y recomendaciones}

En este capítulo se presentan los hallazgos y contribuciones de este trabajo y se dan algunas recomendaciones.\\


%\begin{table}[ht]
\caption{Resultados de la ejecución de la metaheurística GTS, utilizando instancias de Dethloff con la configuración XXX}
\centering
\begin{tabular}{c c c c c}
\hline\hline
Instancia & Costo mínimo & Tiempo(seg.) & Costo promedio & Tiempo promedio(seg.) \\ [0.5ex]
\hline
c0530.txt & 640.549071 & 1.19 & 640.549 & 1.196 \\
c0531.txt & 697.835518 & 1.06 & 698.368 & 1.56 \\
c0532.txt & 659.335981 & 1.74 & 659.336 & 1.572 \\
c0533.txt & 680.604414 & 1.09 & 682.989 & 1.374 \\
c0534.txt & 690.499806 & 1.75 & 693.912 & 2.472 \\
[1ex]\hline
\end{tabular}
\label{table:nonlin}
\end{table}
\begin{table}[ht]
\caption{Resultados de la ejecución de la metaheurística GTS, utilizando instancias de Dethloff con la configuración XXX}
\centering
\begin{tabular}{c c c c c}
\hline\hline
Instancia & Costo mínimo & Tiempo(seg.) & Costo promedio & Tiempo promedio(seg.) \\ [0.5ex]
\hline
c0530.txt & 636.336875 & 3.68 & 639.707 & 2.166 \\
c0531.txt & 697.835518 & 1.35 & 699.381 & 1.756 \\
c0532.txt & 659.335981 & 1.24 & 659.336 & 1.402 \\
c0533.txt & 680.041387 & 4.57 & 683.589 & 2.258 \\
c0534.txt & 690.499806 & 2.02 & 697.517 & 2.062 \\
[1ex]\hline
\end{tabular}
\label{table:nonlin}
\end{table}

\begin{table}[ht]
\caption{Resultados de la ejecución de la metaheurística GTS, utilizando instancias de SalhiNagy con la configuración XXX}
\centering
\begin{tabular}{c c c c c}
\hline\hline
Instancia & Costo mínimo & Tiempo(seg.) & Costo promedio & Tiempo promedio(seg.) \\ [0.5ex]
\hline
vrpnc1x.txt & 470.479372 & 3.06 & 475.716 & 2.376 \\
vrpnc1y.txt & 470.479372 & 2.34 & 473.529 & 2.536 \\
vrpnc2x.txt & 686.772205 & 5.43 & 691.376 & 5.348 \\
vrpnc2y.txt & 682.387217 & 3.03 & 690.264 & 4.622 \\
vrpnc3x.txt & 721.270834 & 14.85 & 725.903 & 11.458 \\
[1ex]\hline
\end{tabular}
\label{table:nonlin}
\end{table}

% Crea el glosario 
\printglossaries

% Establece las citas y bibliografia
\bibliographystyle{alpha.bst}
\bibliography{myrefs}

% Crea el apendice
\appendix

\end{document}
