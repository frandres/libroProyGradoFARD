\chapter{Marco Teórico} \label{chap:marcoTeorico}

A continuación se resume brevemente algunos trabajos de interés. Posteriormente se introducen algunos conceptos relacionados con DIA y el lenguaje utilizado para interrogar documentos, ODIL, que son necesarios para el entendimiento de este proyecto.

%En particular, se revisaron "“A survey of Uncertain Data Algorithms and Applications”, escrito por Aggarwal y Fellow; “Evaluating Probabilistic Queries over Imprecise Data” de Cheng, Kalashnikov y Prabhakar; “Cleaning Uncertain Data with Quality Guarentees” de  Chen, Chen y Xie; “On databases with Incomplete Information” de Lipski; “Query type classification for web document retrieval” de Kang y Kim; “Representing and querying XML with incomplete information” de Rocquenco, Segoufin y Viano. \\

\section{Revisión de trabajos sobre consultas con datos incompletos.}

En “A survey of Uncertain Data Algorithms and Applications” (2005) en \cite{surveyUDAA}, Aggarwal  y Fellow presentan un  estudio de las tecnologías que se han desarrollado para trabajar con datos inciertos. Los autores definen los datos inciertos como datos que pueden  estar incompletos o que pueden contener errores. Por ejemplo, se puede tener una base de datos con datos de mediciones meteorológicas cuya precisión depende de los instrumentos utilizados y que por ende sus datos pueden contener errores. Puede darse también el caso en el que no se tenga la información completa, y que haya la presencia de varios atributos nulls en la base de datos. En el caso particular de este proyecto de grado se trabaja con datos incompletos. \\

En base a la definición de datos inciertos, los autores examinan las técnicas más recientes en tres campos fundamentales: modelado de datos inciertos, manejo de datos inciertos y minería de datos inciertos. Este proyecto de grado está enmarcado en las dos primeras categorías: modelado y manejo de datos inciertos. Esto se debe a que se quiere estudiar la forma como modelar el problema así como un conjunto de mecanismos que permitan completar la información que falta mediante extracción focalizada. \\

Los autores definen  una base de datos probabilística como un espacio de probabilidad finito en el cual los resultados son las posibles bases de datos consistentes con un esquema dado. En pocas palabras, se tiene una representación de los “posibles mundos”. Existen varias soluciones para ello, como modelar la probabilidad de que una tupla esté en la base de datos o, si se quiere más detalle, la probabilidad de que un atributo de una tupla de base de datos esté presente. Sobre esta base se pueden construir técnicas para el manejo de datos y para realizar minería sobre ellos. \\

Por otro lado, Aggarwal y Fellow (2006) en \cite{surveyUDAA} proponen  utilizar una función de probabilidad sobre la correctitud de los datos presentes en la base de datos. Si bien el objetivo de este proyecto es el desarrollo de un mecanismo para contestar una consulta cuya respuesta no está en su totalidad en la ontología, la aproximación que hacen estos autores puede ser útil para definir una medida de la calidad de los datos presentes en la base de datos. \\

Es importante destacar sin embargo, una referencia que se hace al trabajo “Evaluating Probabilistic Queries over Imprecise Data” (2003), de Cheng, Kalashnikov y Prabhakar en \cite{evaluatingProbabilisticQueriesOverImpreciseData}. En el mismo se hace un análisis sobre las consultas con datos imprecisos o propensos a errores y proponen una clasificación de consultas probabilísticas que puede ser tomada en cuenta para clasificar las consultas con respuestas incompletas. Básicamente los autores toman en consideración dos criterios: el tipo de elemento devuelto por la consulta y el uso de funciones agregadas. En pocas palabras, cuando se toma en consideración el tipo de elemento devuelto por la consulta se tiene que puede devolver un valor puntual o un conjunto de tuplas. En el contexto de las consultas probabilísticas esto puede tener implicaciones: para las consultas que buscan un valor los autores definen cotas superiores e inferiores que definen intervalos en los que los valores de la función deben estar, con una probabilidad acumulada de 1. La segunda clasificación toma en cuenta si existen agregados o no. La presencia de agregados puede afectar la evaluación de consultas incompletas. \\

Además de estos dos trabajos se han examinado otros tres. Sin embargo, su aporte y utilidad  para el presente de trabajo de investigación es menor. Chen, Chen y Xie proponen en “Cleaning Uncertain Data with Quality Guarentees” (2008)\cite{cleaningUncertainDataWithQualityGuarantees}, una métrica para cuantificar la ambigüedad de una respuesta de consulta bajo semánticas de mundos posibles. Sobre esta base, se podría construir un mecanismo para “limpiar” la base de datos. \\

El trabajo “On databases with Incomplete Information” de Lipski (1981)\cite{onDatabasesWithIncompleteInformation} es muy citado por otros investigadores en el área de consultas incerteras y sin duda alguna constituye un hito muy importante en ésta area. Sin embargo, en dicho trabajo se busca tratar el tema con un formalismo matemático que va más allá de los alcances de este proyecto. \\

Por otro lado, Kang y Kim proponen en  “Query type classification for web document retrieval” (2003)\cite{queryTypeClasWebDocumentRetrieval} un mecanismo de clasificación de consultas para extracción de información del WEB. Dicha clasificación corresponde con el tipo de operación que se desea: úbicar algo por tópico, ubicar un homepage o ubicar un servicio. \\

Por último, Rocquenco, Segoufin y Viano presentan en “Representing and querying XML with incomplete information” (2001)\cite{repAndQueryXMLIncompInfo} un modelo para hacer consultas incompletas sobre XML.  Dicho trabajo no fue de utilidad dado que la naturaleza de las mismas es diferente de las que se hacen en el contexto de DIA, basadas en una ontología y en un modelo de datos más formal que el de un XML. \\

%Existe cierto paralelismo con lo que se quiere hacer en este trabajo de investigación: realizar una segunda extracción de información cuando no se pueda contestar una consulta con lo que está en la ontología.

\section{DIA y el lenguaje ODIL}

 DIA es una arquitectura que permite realizar consultas sobre un conjunto de documentos  utilizando una representación basada en ontologías por medio de un lenguaje llamado \emph{ODIL} (Ontology-based Document Interrogation Language). Una ontología, en un contexto computacional, es una especificación basada en primitivas con los cuales se quiere modelar un dominio de conocimiento o del discurso (Tom Gruber 2009 \cite{defOntology}). Dichas primitivas a menudo son clases o entidades, atributos y relaciones. Una ontología puede ser vista consecuentemente como una abstracción en un modelos de datos cuyo propósito es modelar conocimiento (Ibidem).Para hacer consultas DIA funciona en 3 fases: preparación de datos, extracción de datos e interrogación de documento. \\

La primera fase de DIA consiste en la elección de un conjunto de documentos sobre el cual se realizará la interrogación y en la construcción de un extractor de información. El extractor de información puede realizarse utilizando mecanismos de procesamiento de lenguaje natural, de aprendizaje de máquina, estadísticos, entre otros posibles.\\

La segunda fase consiste en realizar una primera Extracción Amplia de Información \emph{(BIE)} del conjunto de documentos utilizando el extractor ya construido par poblar una Base de Datos relacional. \\

En la tercera fase los usuarios, toda vez que se haya poblado la base de datos con datos, pueden realizar preguntas en la base de datos de información extraida de los documentos. En este punto se debe determinar la respuesta basado en la información presente en la Base de Datos. Para ello es necesario revisar si hay algún tipo de incompletitud y en el caso de haberla, buscar de nuevo en el conjunto de documentos. El proceso de realizar una segunda extracción sobre documentos es el denominado Extracción Focalizada de Información. En esencia busca encontrar la información faltante para poblar la base de datos para luego responder la pregunta realizada. La figura \ref{figura-DIA} sintetiza las fases de DIA explicadas\\

\begin{figure}[h]
  \centering
  \includegraphics[width=1\textwidth]%
    {arquitecturaDIA}
  \caption{Arquitectura de Interrogación de documentos}
\end{figure}\label{figura-DIA}

El objetivo de la Extracción Focalizada de Información es encontrar información desconocida en la ontología realizando una extracción sobre un conjunto de documentos utilizando información presente en la ontología. El conjunto de documentos sobre el cual se hace la segunda extracción puede ser el mismo sobre el cual se hizo Extracción Amplia de Información o no. Una de las ventajas de esto es que se puede explorar un conjunto de documentos de mayor tamaño al inicial, incluyendo por ejemplo la \emph{WWW} (World Wide Web), sobre el cual es más complejo realizar una extracción inicial debido a su magnitud. La utilización de información conocida puede ser explotada por el extractor para simplificar considerablemente la búsqueda de la información faltante.\\

Para poder realizar la interrogación de documentos en DIA, se diseñó e implementó el lenguaje ODIL. ODIL, que significa Ontology-based Document Interrogation Language (Lenguaje basado en ontología para la interrogación de documentos), fue definido por Abad y Helman en \cite{ODILdefinition}  para realizar preguntas sobre los documentos en la arquitectura DIA. Para ello, ODIL cuenta con dos módulos diferentes: el Módulo de Definición de Conceptos y el Módulo de Cláusulas de Interrogación. El primer modulo busca definir la ontología que se utilizará para hacer preguntas sobre los documentos. El segundo tiene como objetivo permitir al usuario hacer las preguntas.\\

El eje central de ODIL es el \emph{concepto}. Un concepto se refiere a un elemento del dominio de información sobre el cual se define la ontología. En lo sucesivo se utilizará  la expresión \emph{dominio de información} para referirse al ámbito o temática de los documentos sobre los cuales se hace la extracción. El concepto debe tener un nombre, una definición basada en cinco tipo de abstracciones: concepto primitivo, agregación, enumeración y generalización, un origen y un conjunto de sinónimos.\\

La abstracción se refiere a la tipología del concepto. Un concepto primitivo puede verse como un átomo de información (un pedazo de información que no se puede descomponer), como por ejemplo un atributo del modelo ER. Un concepto puede verse también como un agregado de otros conceptos, lo cual permite darle estructura al modelo de datos. Por ejemplo, manteniendo el paralelismo con el modelo ER, un concepto puede verse como una entidad que a su vez está definida por un conjunto de atributos. Además de la agregación, ODIL permite establecer jerarquías de especialización-generalización, similar al modelo OMT. Por último, la abstracción enumeración se refiere a la especificación de instancias de un concepto.\\

Otro elemento importante para entender ODIL es que el concepto puede tener dos orígenes diferentes: puede estar basado en información contenida en la Base de Datos o puede estar definido exclusivamente a nivel de la ontología. Si se encuentra en la base de datos, un concepto puede verse como algún elemento del modelo ER, necesario para la persistencia de los datos. Se puede suponer que la base de datos tiene un modelo relacional ya que ODIL y DIA operan bajo esta suposición. Un concepto con origen en la base de datos puede ser, por ejemplo, un atributo o una entidad. \\

Por otro lado, un concepto definido a nivel de la ontología tiene como objetivo, en lugar de permitir la persistencia de datos, permitir a un usuario realizar preguntas. Por ejemplo, supongamos que se tiene una ontología sobre estudiantes de la Universidad Simón Bolívar. Un concepto con origen en la base de datos puede ser un carnet (concepto primitivo), puede ser un estudiante (agregación de varios conceptos), o puede ser un estudiante de postgrado (especialización-generalización). Un concepto definido en la ontología puede ser un equipo de trabajo para algún proyecto, que a su vez es una agregación sobre conceptos (los estudiantes) con origen en la Base de Datos. Al hacer consultas, el usuario puede definir la estructura de qué es un equipo de trabajo y puede definir instancias de equipos de trabajos basadas en instancias de estudiantes contenidas en la Base de Datos.\\

Entender que pueden haber conceptos definidos a nivel de la ontología es importante para razonar sobre los orígenes de incompletitud. Puede haber información en los documentos sobre los cuales se hizo la primera extracción de DIA que no fue extraída porque en el momento en el cual se hizo dicha primera extracción el modelo no contemplaba dichos conceptos. \\

Por último es de interés para el lector entender algunos aspectos sobre la implementación del lenguaje ODIL. Dicha implementación fue hecha por Apaza y Mirasola en \cite{ODILImplementation} y consiste en un sistema de software que analiza texto escrito con la sintaxis de ODIL para definir estructura y hacer consultas y basado en el mismo, contesta preguntas con el contenido de la Base de Datos. Lo relevante para la comprensión de este proyecto de grado es saber que dichas consultas son respondidas traduciendo la consulta hecha en ODIL a expresiones del algebra relacional. Es útil saber esto ya que como se verá en los próximos capítulos, para descubrir los orígenes de incompletitud de las consultas es necesario razonar sobre la consulta propiamente dicha, y es conveniente suponer que esa consulta se realiza utilizando álgebra relacional.\\