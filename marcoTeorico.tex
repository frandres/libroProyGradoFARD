\chapter{Marco Teórico} \label{chap:marcoTeorico}

A continuación se resume brevemente algunos trabajos relacionados con el presente proyecto cuyos resultados son de interés. \\

En “A survey of Uncertain Data Algorithms and Applications” (2005) \cite{surveyUDAA}, Aggarwal  y Fellow presentan un  estudio de las tecnologías que se han desarrollado para trabajar con datos inciertos. Los autores definen los datos inciertos como datos que pueden tener estar incompletos o que pueden contener errores. Por ejemplo, se puede tener una base de datos con datos de mediciones meteorológicas cuya precisión depende de los instrumentos utilizados y que por ende sus datos pueden contener errores. Puede darse también el caso en el que no se tenga la información completa, como en la base de datos un censo en la cual no se tiene la información de todos los ciudadanos de un país. En el caso particular de este proyecto de grado se trabaja con datos incompletos. \\

En base a la definición de datos inciertos, los autores examinan las técnicas más recientes en tres campos fundamentales: modelado de datos inciertos, manejo de datos inciertos y minería de datos inciertos. Este proyecto de grado está enmarcado en las dos primeras categorías: modelado y manejo de datos inciertos. Esto se debe a que se quiere estudiar la forma como modelar el problema así como un conjunto de mecanismos que permitan completar la información que falta mediante extracción focalizada. \\

Los autores definen  una base de datos probabilística como un espacio de probabilidad finito en el cual los resultados son las posibles bases de datos consistentes con un esquema dado. En pocas palabras, se tiene una representación de los “posibles mundos”. Existen varias soluciones para ello, como modelar la probabilidad de que una tupla esté en la base de datos o, si se quiere más detalle, la probabilidad de que un atributo de una tupla de base de datos esté presente. Sobre esta base se pueden construir técnicas para el manejo de datos y para realizar minería sobre ellos. \\

Por otro lado, Aggarwal y Fellow (2006) \cite{surveyUDAA} proponen  utilizar una función de probabilidad sobre la correctitud de los datos presentes en la base de datos. Si bien el objetivo de este proyecto es el desarrollo de un mecanismo para contestar una consulta cuya respuesta no está en su totalidad en la ontología, la aproximación que hacen estos autores puede ser útil para definir una medida de la calidad de los datos presentes en la Base de Datos. \\

Es importante destacar sin embargo, una referencia que se hace al trabajo “Evaluating Probabilistic Queries over Imprecise Data” (2003), de Cheng, Kalashnikov y Prabhakar \cite{evaluatingProbabilisticQueriesOverImpreciseData}. En el mismo se hace un análisis sobre las consultas con datos imprecisos o propensos a errores. Una vez más, esto es diferente de lo que se quiere en este trabajo de investigación: trabajar consultas incompletas. Sin embargo, Cheng et al proponen una clasificación de consultas probabilísticas que puede ser tomada en cuenta para clasificar las consultas incompletas. \\

Básicamente los autores toman en consideración dos criterios: el tipo de elemento devuelto por la consulta y el uso de funciones agregados. En pocas palabras, cuando se toma en consideración el tipo de elemento devuelto por la consulta se tiene que puede devolver un valor puntual o un conjunto de tuplas. En el contexto de las consultas probabilísticas esto puede tener implicaciones: para las consultas que buscan un valor los autores definen cotas superiores e inferiores que definen intervalos en los que los valores de la función deben estar, con una probabilidad acumulada de 1.  \\

La segunda clasificación toma en cuenta si existen agregados o no. La presencia de agregados puede afectar como se verá la evaluación de consultas incompletas. \\

Además de estos dos trabajos se han examinado otros 3. Sin embargo, su aporte y utilidad  para el presente de trabajo de investigación es menor. Chen, Chen y Xie proponen en “Cleaning Uncertain Data with Quality Guarentees” (2008)\cite{cleaningUncertainDataWithQualityGuarantees}, una métrica para cuantificar la ambiguedad de una respuesta de consulta bajo semánticas de mundo posibles. Sobre esta base, se podría construir un mecanismo para “limpiar” la base de datos. \\

El trabajo “On databases with Incomplete Information” de Lipski (1981)\cite{onDatabasesWithIncompleteInformation} es muy citado por otros investigadores en el área de consultas incerteras y sin duda alguna constituye un hito muy importante en ésta area. Sin embargo, en dicho trabajo se busca tratar el tema con un formalismo matemático que va más allá de los alcances de este proyecto. \\

Por otro lado, Kang y Kim proponen en  “Query type classification for web document retrieval” (2003)\cite{queryTypeClasWebDocumentRetrieval} un mecanismo de clasificación de consultas para extracción de información del WEB. Sin embargo, dicha clasificación corresponde con el tipo de operación que se desea: úbicar algo por tópico, ubicar un homepage o ubicar un servicio. Dicha clasificación no está enmarcada dentro del presente trabajo de investigación y por ende tiene poca utilidad. \\

Por último, Rocquenco, Segoufin y Viano presentan en “Representing and querying XML with incomplete information” (2001)\cite{repAndQueryXMLIncompInfo} un modelo para hacer consultas incompletas sobre XML. Existe cierto paralelismo con lo que se quiere hacer en este trabajo de investigación: realizar una segunda extracción de información cuando no se pueda contestar una consulta con lo que está en la ontología. Sin embargo, dicho trabajo no fue de utilidad para la clasificación de las consultas. \\