\chapter*{Introducción} \label{chap:intro}
\addcontentsline{toc}{chapter}{Introducción}

Uno de los retos aún presentes en el mundo de la computación hoy por hoy es el de construir herramientas que permitan automatizar operaciones sobre un lenguaje natural. Se entiende por lenguaje natural un idioma como el inglés, español, italiano, que no es reconocible por una computadora sin herramientas pertientes. Cada día surgen más y más tecnologías en este ámbito: existen procesadores de comandos de voz que son capaces de entender instrucciones dadas por un usuario, programas que permiten tomar dictados de un humano, traductores de textos y programas que colocan signos de puntuación automáticamente, entre otros. A la par de dichos avances, sin embargo, hay muchísimos campos los cuales se hacen más difíciles para la computación; en la medida en la que los objetivos trazados requieren una mayor ``comprensión'' semántica del lenguaje natural se hace más complicado construir herramientas que resuelvan los problemas planteados. \\ 

Uno de las áreas del Procesamiento de Lenguaje Natural (NLP) que se está desarrollando actualmente y cuya utilidad es altísima es la extracción de información (IE, que significa Information Extraction). La IE tiene como objetivo obtener datos a partir de textos (o en general información en cualquier medio que se encuentre en algún lenguaje humano). Por ejemplo, puede ser deseable construir un extractor que analice noticias de un periódico para poblar una base de datos con los principales eventos que han ocurrido. Dicha base de datos puede luego ser consultada de forma que un usuario pueda buscar algún dato que sea necesario sin tener que leer todas las noticias. Generalizando, los avances en esta  área pueden ser aplicados en cualquier dominio en el cual se tenga una amplia cantidad de información en lenguaje natural no entendible por una máquina: ámbitos corporativos, educativos, diplomáticos, etcétera. Las posibles aplicaciones de ésto son numerosas.\\

Abad Mota (2009) \cite{documentInterrogationArchitecture} propuso una arquitectura para la Interrogación de Documentos (DIA, por sus siglas Document Interrogation Architecture) que busca resolver el problema de extracción de información en documentos escritos. La interrogación de documentos se entiende como un proceso en el cual tienen lugar varias actividades: una primera extracción de información de un conjunto de documentos, la realización de consultas sobre una ontología poblada con información previa, y la resolución de consultas con información que no se encuentra presente en la ontología. \\

Ruiz \cite{SemistructuredTextExtraction} \cite{ruiz-HMM} propuso mecanismos para realizar la primera extracción definida en DIA para poblar inicialmente la ontología. Paza y Mirisola \cite{ODILImplementation} implementaron el lenguaje ODIL (Lenguage de interrogación de documentos con ontologías) diseñado a la par de DIA para la interrogación de Documentos. Todos estos avances están relacionados con las primeras fases propuestas en DIA.  \\

\shadowbox{Duda: ¿Hay alguién más que merece ser citado en trabajos sobre DIA?}\\


El propósito del presente proyecto de grado es diseñar e implementar un mecanismo de \emph{extracción focalizada}, que permita resolver consultas cuando la ontología no se encuentra completa. Esto es, el objetivo del mismo es construir un mecanismo que permita mediante búsquedas enfocadas encontrar datos faltantes en la ontología a partir de un conjunto de documentos e información presente en la ontología. Esta fase es la última propuesta en DIA y opera bajo la premisa de que es posible obtener mejores resultados en extracción de información cuando se tiene información conocida. \\

Para poder realizar la extracción focalizada es necesario una serie de pasos que serán descritos en profundidad en el presente informe. Por ahora conviene mencionar como objetivos específicos o pasos intermedios: el razonamiento sobre el problema de las respuestas incompletas o imprecisas a consultas a la base de datos a partir del lenguaje ODIL y el razonamiento sobre cómo aprovechar la información presente en la ontología para construir un extractor de información y el diseño e implementación de un extractor focalizado. \\


El presente informe busca sintetizar los principales resultados obtenidos en dicoh proyecto. Para ello está estructurado en 7 capítulos. Primeramente se hace una planteamiento del problema introduciendo más en detalle la arquitectura DIA y presentando en profundidad el concepto de extracción focalizada. Seguidamente se presente una breve revisión de algunos trabajos existentes en el ámbito de extracción de información. Posteriormente se presente en los capítulos 3 y 4 el diseño propuesto para la extracción focalizada de la información y el mecanismo concebido para determinar los origenes de incompletitud propios de una ontología incompleta. Luego se realizan algunos comentarios relevantes sobre la implementación del sistema, se presentan las pruebas realizadas para probar el sistema hecho en conjunto con sus resultados y el análisis pertinente. \\


