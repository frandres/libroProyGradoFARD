% Planteamiento del Problema
\chapter{Planteamiento del problema} \label{chap:planning}


AQU\'I VA EL DESARROLLO DEL PLANTEAMIENTO DEL PROBLEMA.
%Puedes quitar esto(es opcional)
\vspace{5 mm}

Para verificar si los m'etodos para detecci'on de ataques de denegaci'on de
servicio mediante el uso del par'ametro de Hurst, descritos en la secci'on
son utilizables en tiempo real necesitamos 
implementarlos e incluirlos en una herramienta que adem'as maneje las trazas
y sea capaz de graficar los resultados. 

Los requerimientos funcionales y no funcionales para la creaci'on del programa,
son descritos en las secci'on \ref{sect:requirements}.  

\section{Requerimientos} \label{sect:requirements}

Los requerimientos de la herramienta o software a implementar se mencionan a
continuaci'on: 

\begin{itemize}
\item Por posibles cuestiones legales, la implementaci'on de la soluci'on 
debe ser de c'odigo libre para poder, luego de su validaci'on, ser
modificados y los m'odulos e incluidos en el AIR-NMS del laboratorio Kinoshita
de la Universidad de Tohoku.
\item La herramienta de l'inea de comando debe desarrollarse en el lenguaje
{\tt C}, ya que se quiere utilizar la librer'ia {\tt libpcap} para manipular
las trazas de red como fuese necesario.
\item Con el uso de la liber'ia {\tt libpcap} se debe extraer todo la
informaci'on posible sobre los tiempos de llegada, cantidad y tama~no de los
paquetes de distintos protocolos que componen TCP/IP en la traza que alimenta
el programa.
\item Debido a que algunos m'etodos de estimaci'on del par'ametro de Hurst
utilizan gr'aficas, la herramienta debe tener capacidades gr'aficas. 
El software debe tambi'en poder graficar el cambio del par'ametro en el tiempo
cuando se usa el mecanismo de ventanas deslizantes, y la serie de tiempo
creada a partir de la velocidad de captura ($c$).
\item El programa debe ser flexible. Esto incluye el hacer todos los aspectos
importantes de la estimaci'on parametrizables, tales como la velocidad de 
captura ($c$), el tama~no de la ventana ($w$) y el tama~no de la ventana
deslizante ($s$).
\item El programa debe incluir al menos 3 m'etodos para la estimaci'on del 
par'ametro de Hurst. Dichos m'etodos fueron escogidos desde un principio por
el laboratorio Kinoshita y los criterios de selecci'on se exponen a
continuaci'on. Ellos ya ten'ian una implementaci'on del estad'istico R/S y
quer'ian seguir teniendo este m'etodo como alternativa. Debido a los art'iculos 
\cite{intelligentfuzzy} y \cite{xiang:292} d'onde se hace un an'alisis
exhaustivo de la estimaci'on del par'ametro de Hurst mediante las gr'aficas
varianza-tiempo y su comportamiento en ataques de denegaci'on de servicio era
tambi'en razonable su implementaci'on. Por 'ultimo, el m'etodo de la varianza
modificada de Allan fue seleccionado ya que hoy en d'ia es uno de los m'etodos 
m'as novedosos utilizados para la estimaci'on del par'ametro de Hurst y se ha
demostrado que es bastante preciso, especialmente con muestras peque~nas
\cite{MAVARStefano}. 
\item  Por tratarse de un prototipo, todas sus funciones ser'ian para el 
an'alisis de forma {\it offline}. Los resultados dar'ian una idea del
comportamiento en un ambiente real.
\end{itemize}


