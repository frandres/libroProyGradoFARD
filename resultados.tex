% Resultados
\chapter{Resultados} \label{chap:resultados}

A continuación se muestran los resultados de las pruebas realizadas.

\section{Resultados de las pruebas unitarias sobre el Módulo de Determinación de Origen de la Incompletitud.} 

\section{Resultados de las pruebas unitarias sobre el Módulo de Búsqueda Focalizada de Información}


\begin{table}[h]
\caption{Resultados detallados la evaluación del Preprocesador de Textos}
\centering
\scriptsize
\begin{tabular*}{1\textwidth}{@{\extracolsep{\fill}} !{\vrule width 1pt} c !{\vrule width 1pt} c | c | c!{\vrule width 1pt} c | c | c!{\vrule width 1pt}}
\hline
Dominio & \multicolumn{3}{c!{\vrule width 1pt}}{\bf{Aprobados}} & \multicolumn{3}{c |}{\bf{No aprobados}}\\
\hline
 & Correctos & Con texto de más & \bf{Total aprobados} & Incompletos & Incorrectos & \bf{Total no aprobados}\\
\hline
Designaciones & 87.69\% & 7.69\% & \bf{95.39\%} & 3.07\% & 1.54\% & \bf{4.61\%}\\
\hline
Escalafón & 80.51\% & 12.98\% & \bf{93.6\%}  & 6.4\% & 0\% & \bf{6.4\%} \\
\hline
Jurados de Ascenso & 77.55\% & 16.32\% & \bf{93.88\%} & 0\% & 6.12\% & \bf{6.12\%} \\
\hline
\end{tabular*}
\label{tabla-resultados-preprocesamientoDatosDesignacion}

\end{table}

\begin{comment}
\begin{table}[h]
\caption{Resultados detallados la evaluación del Preprocesador de Textos: Escalafón}
\centering
\scriptsize
\begin{tabular*}{.68\textwidth}{@{\extracolsep{\fill}} !{\vrule width 1pt} c !{\vrule width 1pt} c | c !{\vrule width 1pt} c | c !{\vrule width 1pt}}
\hline
Dominio & \multicolumn{2}{c!{\vrule width 1pt}}{\bf{Aprobados}} & \multicolumn{2}{c |}{\bf{No aprobados}}\\
\hline
 & Correctos & Con texto de más & Incompletos & Incorrectos \\
\hline
\multirow{2}{*}{Designaciones} & 80.51\% & 12.98\% & 6.4\% & 0\% \\\cline{2-5}
& \multicolumn{2}{c!{\vrule width 1pt}}{Aprobados: \bf{93.6}} & \multicolumn{2}{c |}{No aprobados: \bf{6.4\%}}\\
\hline 
\end{tabular*}
\label{tabla-resultados-preprocesamientoDatosEscalafon}
\\El conjunto de prueba es de tama~no 77, un tercio de la población.
\end{table}
\end{comment}

%------------------------


Por que el extractor falla con designaciones? (Observaciones hechas al hacer las pruebas que pueden ser útiles en el análisis de datos)

1. Las unidades de informacion no son precisas. A veces hay multiples designaciones en una misma linea. \\
2. Hay multiples designaciones que coinciden en un mismo día para una misma persona. \\
3. Hay ratificaciones, correcciones, posteriores a la designacoines que modifican el resultado que uno pensaria correcto.\\
4. Hay errores de tipeo por parte de la secretaria.\\
5. Hay casos "patologicos" que es imposible generalizar con expresiones regulares que no sean hechas a la medida. \\
6. Hay casos en los que los campos tienden a tener muchos valores null y al no encontrar la respuesta en el mejor hit, se procede al segundo. Arreglar esto en el extractor? Hay muchos casos en los que un segundo match ayuda. \\
