% Resultados
\chapter{Resultados computacionales} \label{chap:resultados}

\section{Pruebas experimentales}\label{sect:pruebas}

Las seis metaheurísticas implementadas fueron ejecutadas utilizando dos conjuntos de instancias VRPSPD conocidas en el área y utilizadas por muchos autores. Cada conjunto de instancias incluye datos relevantes a la resolución del problema tales como el número de clientes, posición del depósito y los clientes en un plano cartesiano, capacidad máxima de los vehículos, y demandas de recolección y entrega exigidas por los clientes. Los dos conjuntos de instancias se describen en la próxima sección.

\subsubsection{Ambiente de ejecución}\label{subsect:ambiente}

Las pruebas se realizaron en un servidor dedicado, corriendo Ubuntu 11.10 como sistema operativo, un procesador Pentium Dual-Core E5700 a 3.00GHz y 2 GB RAM DDR3.

\subsection{Instancias utilizadas}\label{subsect:instancias}

El primer conjunto denominado \emph{Dethloff} fue introducido por Dethloff \cite{Dethloff} en 2001, donde 50 clientes son generados considerando dos diferentes escenarios geográficos: en el escenario SCA, las coordenadas de los clientes están uniformemente distribuidas en el intervalo [0,100]. En el escenario CON, la mitad de las coordenadas están distribuidas de la misma manera que SCA, mientras que la otra mitad están uniformemente distribuidas en el intervalo [$\frac{100}{3}$,$\frac{200}{3}$]. La demanda de entrega de cada cliente $j$, $D_j$, está uniformemente distribuida en el intervalo [0,100], mientras que la demanda de recepción del cliente $j$, $P_j$, es calculada utilizando un número aleatorio $r_j$ uniformemente distribuido en [0,1] tal que $P_j$ = (0.5 + $r_j$)$D_j$. Las instancias son generadas con diferentes capacidades de vehículos escogiendo el mínimo de vehículos $\mu$. Entonces la capacidad correspondiente es establecida como $C = \sum_{s \in J}D_s/\mu$, donde $\mu$ se escoge como 3 u 8. Se crean 40 instancias en total.

El segundo conjunto denominado \emph{SalhiNagy} fue introducido por Salhi y Nagy \cite{SalhiNagy} en 1999, basado en 14 problemas propuestos en \cite{Christofides}. El número de clientes en estos datos varía desde 50 a 199 clientes. Dos clases de instancias son generadas aquí, la clase X y la clase Y. La diferencia entre estas dos es que sus demandas de entrega y recolección están intercambiadas. Sólo un subconjunto de estas instancias son utilizadas por ser las más representativas y diversas, de la misma manera que lo hacen muchos autores. Un total de 14 instancias SalhiNagy son utilizadas.

\subsection{Entonación de parámetros}\label{subsect:entonacion}

La entonación de los parámetros de cada metaheurística es un proceso de importancia fundamental, ya que de estos depende la calidad de una solución. No existen fundamentos teóricos que permitan predecir la influencia de cada parámetro en la calidad de una solución. En ese sentido, se realizaron pruebas experimentales con la finalidad de seleccionar las combinaciones de parámetros que generen mejores resultados.

Se realizaron muestreos de cada metaheurística aislando cada parámetro por separado y manteniendo los restantes fijos en sus valores recomendados según los trabajos referidos. Cada parámetro fue probado a lo largo de su intervalo o creando un intervalo a partir del valor recomendado por el trabajo, para comprobar su influencia en el resultado final. Finalmente, los dos parámetros considerados como los más influyentes son escogidos para su posterior entonación. Éstos parámetros son acotados superior e inferiormente según su influencia en el resultado final. Los parámetros restantes son mantenidos fijos en sus valores recomendados.

Para la entonación se realizaron 4 corridas de cada instancia (para un total de 54 instancias) variando los dos parámetros más influyentes de cada metaheurística en su rango acotado. Estas pruebas se encuentran en el \textbf{Apéndice~\ref{chap:apendiceC}}. Para cada metaheurística, los resultados de la entonación son comparados contra \emph{la mejor solución reportada hasta el momento}\footnote{Según su costo promedio, para instancias de Dethloff: \cite{ils-vnd} y para instancias de SalhiNagy: \cite{gts}} para cada instancia y los parámetros finales son escogidos en base a su porcentaje de mejoría promedio (\%GP) y a su tiempo de ejecución promedio. Se escogieron parámetros distintos para cada tipo de instancia, ya que la topología de cada clase de problema  influye en la solución del problema.

\subsubsection{Pruebas de entonación} \label{subsect:pruebasentonacion}

A continuación se presenta una descripción de las pruebas de entonación realizadas para cada metaheurística implementada.

%\begin{minipage}{1\linewidth}
%\smallskip
%\smallskip
%\smallskip
%\smallskip
%\footnotesize
%$^1$Para una descripción de los parámetros ver \textbf{Sección~\ref{sect:implementacion-heuristicas-hibridas}, Tabla~\ref{table:param-gts}}\\
%$^2$Para una descripción de los parámetros ver \textbf{Sección~\ref{sect:implementacion-heuristicas-hibridas}, Tabla~\ref{table:param-ils}}\\
%$^3$Para una descripción de los parámetros ver \textbf{Sección~\ref{sect:implementacion-heuristicas-hibridas}, Tabla~\ref{table:param-aco}}\\
%$^4$Para una descripción de los parámetros ver \textbf{Sección~\ref{sect:implementacion-heuristicas-hibridas}, Tabla~\ref{}}\\
%$^5$Para una descripción de los parámetros ver \textbf{Sección~\ref{sect:implementacion-heuristicas-hibridas}, Tabla~\ref{table:param-pso}}\\
%$^6$Para una descripción de los parámetros ver \textbf{Sección~\ref{sect:implementacion-heuristicas-hibridas}, Tabla~\ref{}}\\
%\end{minipage}

\subsubsection*{GTS-M}

Se determinó a través de los resultados del muestreo previamente mencionado que los parámetros más influyentes para GTS-M son \emph{mni}\footnote{Para una descripción de los parámetros de GTS-M ver \textbf{Sección~\ref{sect:implementacion-heuristicas-hibridas}, Tabla~\ref{table:param-gts}}} y \emph{tabu}. Los parámetros restantes de la metaheurística, \emph{lambda1} y \emph{lambda2}, se establecieron en los valores recomendados del  trabajo referenciado.

Las pruebas fueron realizadas en los siguientes rangos:

\begin{itemize}
\item \emph{mni:} [3000,6000], con un salto de 500.
\item \emph{tabu:} [5,37], con un salto de 4.
\end{itemize}

Un total de 63 pruebas fueron realizadas utilizando todas las combinaciones de parámetros según los rangos anteriores. Un conjunto de las mismas se presentan en el \textbf{Apéndice~\ref{chap:apendiceC}}, 
\textbf{Tabla~\ref{table:GTS-M-dethloff-3000-13}} a \textbf{Tabla~\ref{table:GTS-M-salhinagy-6000-5}}.

Se determinó que el número de iteraciones (\emph{mni}) tiene en promedio una alta incidencia en el tiempo del algoritmo y un aumento en calidad de la solución considerable. Por lo tanto, el valor de este parámetro se estableció de tal manera que lograra conseguir soluciones de calidad en tiempos aceptables.

Se determinó que el tamaño de la lista tabú (\emph{tabu}) a medida que se incrementa consigue en promedio mejorar la calidad de las soluciones hasta cierto punto, en el cual la lista tabú resulta ser muy grande y no permite al algoritmo explorar de forma correcta el espacio de soluciones, lo que implica soluciones de peor calidad. Por lo tanto, el valor de este parámetro se estableció en un punto intermedio.

Los valores de los parámetros se establecieron de igual manera para la clase de instancias Dethloff como para SalhiNagy por poseer igual comportamiento en ambos.\\

\subsubsection*{ILS-VND-M}

Se determinó a través de los resultados del muestreo previamente mencionado que los parámetros más influyentes para ILS-VND-M son \emph{n}\footnote{Para una descripción de los parámetros de ILS-VND-M ver \textbf{Sección~\ref{sect:implementacion-heuristicas-hibridas}, Tabla~\ref{table:param-ils}}} y \emph{LS}. El parámetro \emph{y} se deja fijo para las pruebas en los mejores propuestos por el trabajo referenciado.

Las pruebas fueron realizadas en los siguientes rangos:

\begin{itemize}
\item \emph{n:} [5,65], con un salto de 10.
\item \emph{LS:} [10,80], con un salto de 10.
\end{itemize}

Un total de 56 pruebas fueron realizadas utilizando todas las combinaciones de parámetros según los rangos anteriores. Un conjunto de las mismas se presentan en el \textbf{Apéndice~\ref{chap:apendiceC}}, \textbf{Tabla~\ref{table:ILS-VND-M-5-10}}  
a \textbf{Tabla~\ref{table:ILS-VND-M-55-80}}.

En las pruebas finales, para cada una de las instancias se ejecutó 30 veces ILS-VND-M para cada uno de los valores de \textit{y}. Para las instancias de Dethloff el rango se encuentra entre [0,1] con saltos de 0.1, mientras que para las instancias de SalhiNagy el rango se encuentra entre [0,0.5] con saltos de 0.05.\\ 

Se determinó en la entonación de \emph{n} y \emph{LS} que a medida que se incrementan, el costo promedio de las soluciones mejora hasta cierto punto, en el cual este valor se mantiene en el mismo rango, pero sus tiempos incrementan significativamente. Por los tanto, estos parámetros se establecieron en un punto intermedio.

Existe una diferencia en \emph{LS} para cada clase de instancia, probablemente debido a que la clase SalhiNagy posee instancias más grandes que las de Dethloff, necesitando así una exploración más exhaustiva dentro de la sección del método de mejoramiento y perturbación de la solución.\\

\subsubsection*{AS-M}

Se determinó a través de los resultados del muestreo previamente mencionado que los parámetros más influyentes para AS-M son \emph{n}\footnote{Para una descripción de los parámetros de AS-M ver \textbf{Sección~\ref{sect:implementacion-heuristicas-hibridas}, Tabla~\ref{table:param-aco}}} y \emph{q}. Los parámetros restantes de la metaheurística, \emph{alpha}, \emph{beta} y \emph{ro}, se establecieron en los valores recomendados del  trabajo referenciado.

Las pruebas fueron realizadas en los siguientes rangos:

\begin{itemize}
\item \emph{n:} [3,6], con un salto de 1.
\item \emph{q:} [0.1,0.4], con un salto de 0.1.
\end{itemize}

Un total de 16 pruebas fueron realizadas utilizando todas las combinaciones de parámetros según los rangos anteriores. Un conjunto de las mismas se presentan en el \textbf{Apéndice~\ref{chap:apendiceC}}, 
\textbf{Tabla~\ref{table:AS-M-dethloff}} a \textbf{Tabla~\ref{table:AS-M-salhinagy}}.

Se determinó que el número de iteraciones (\emph{n}) tiene en promedio una alta incidencia en la calidad de las so\-lu\-cio\-nes. Sin embargo, luego de ciertas iteraciones las hormigas construyen soluciones sin ninguna mejoría relevante. Por lo tanto, el valor de este parámetro se estableció de tal manera que se lograra conseguir las mejores soluciones sin malgastar cómputo.

Se determinó que la exploración en la metaheurística juega un valor fundamental, y a medida que se realiza mayor explotación la calidad de la solución empeora. El mejoramiento de la solución beneficiando la exploración por encima de la explotación mejora considerablemente la calidad de la solución a costa de un despreciable tiempo de cómputo. Por lo tanto, el valor de \emph{q} se estableció de tal manera de beneficiar en gran medida la exploración sobre la explotación.

Existe una diferencia en \emph{n} para cada clase de instancia, probablemente debido a que la clase SalhiNagy posee instancias más grandes que las de Dethloff, necesitando así una exploración más exhaustiva de las hormigas para mejorar la solución.\\

\subsubsection*{SS-M}

Se determinó a través de los resultados del muestreo previamente mencionado que los parámetros más influyentes para SS-M son \emph{n}\footnote{Para una descripción de los parámetros de SS-M ver \textbf{Sección~\ref{sect:implementacion-heuristicas-hibridas}, Tabla~\ref{table:param-ss}}}, \emph{y} y \emph{b}. Sin embargo, el parámetro \emph{b} se establece en el valor recomendado del  trabajo referenciado, ya que al incrementarlo afecta significativamente los tiempos de ejecución.

Las pruebas fueron realizadas en los siguientes rangos para las instancias de Dethloff:

\begin{itemize}
\item \emph{y:} [0.1,1], con un salto de 0.1.
\item \emph{n:} [50,200], con un salto de 50.
\end{itemize}

Al establecer la combinación parámetros que tiene mejor comportamiento para este tipo de instancias se es\-ta\-ble\-cie\-ron los siguientes rangos para las pruebas de SalhiNagy:

\begin{itemize}
\item \emph{y:} [0.1,0.3], con un salto de 0.1.
\item \emph{n:} [50,250], con un salto de 50.
\end{itemize}
 

Un total de 55 pruebas fueron realizadas utilizando todas las combinaciones de parámetros según los rangos anteriores. Un conjunto de las mismas se presentan en el \textbf{Apéndice~\ref{chap:apendiceC}}, 

\textbf{Tabla~\ref{table:SS-M-50-0.1}} a \textbf{Tabla~\ref{table:SS-M-250-0.2-S}}.

Se determinó que tamaño del \textit{Pool} \emph{n} no tiene incidencia en el tiempo del algoritmo y provee hasta cierto punto un aumento en calidad de la solución. Por lo tanto, el valor de este parámetro se estableció de tal manera que el \textit{Pool} sea diverso y halla obtenido en las pruebas el mejor comportamiento en su resultado promedio.

Se determinó que la variable que define la función de costo del método de diversificación \emph{y} no tiene incidencia en el tiempo del algoritmo y provee hasta cierto punto un aumento en calidad de la solución, mientras que sacrifica la diversidad de las soluciones. 

Existe una diferencia en \emph{y} y \emph{n} para cada clase de instancia, probablemente debido a que la clase SalhiNagy posee instancias más grandes que las de Dethloff, necesitando así un Pool inicial más diverso para obtener mejores soluciones.\\

\subsubsection*{PSO-M}

Se determinó a través de los resultados del muestreo previamente mencionado que los parámetros más influyentes para PSO-M son \emph{n} y \emph{L}. Los parámetros restantes de la metaheurística, \emph{cp}, \emph{cg}, \emph{cl}, \emph{cn}, \emph{w1}, \emph{wt} y \emph{K} se establecieron en los valores recomendados del trabajo referenciado.

Las pruebas fueron realizadas en los siguientes rangos:

\begin{itemize}
\item \emph{n:} [10,50], con un salto de 20.
\item \emph{L:} [10,90], con un salto de 20.
\end{itemize}

Un total de 15 pruebas fueron realizadas utilizando todas las combinaciones de parámetros según los rangos anteriores. Un conjunto de las mismas se presentan en el \textbf{Apéndice~\ref{chap:apendiceC}}, \textbf{Tabla~\ref{table:PSO-M-dethloff}} a \textbf{Tabla~\ref{table:PSO-M-salhinagy}}.

Se determinó que el número de iteraciones (\emph{n}\footnote{Para una descripción de los parámetros de PSO-M ver \textbf{Sección~\ref{sect:implementacion-heuristicas-hibridas}, Tabla~\ref{table:param-pso}}}) tiene en promedio una alta incidencia en el tiempo del algoritmo y un aumento en calidad de la solución considerable. Por lo tanto, el valor de este parámetro se estableció de tal manera que lograra conseguir soluciones de calidad en tiempos aceptables.

Se determinó que el número de partículas (\emph{L}) tiene en promedio también una alta incidencia en el tiempo del algoritmo y un aumento en calidad de la solución considerable. Por lo tanto, el valor de este parámetro se estableció de tal manera que lograra conseguir soluciones de calidad en tiempos aceptables.

Los valores de los parámetros sólo se pudieron probar para la clase de instancia SalhiNagy.\\

\subsubsection*{GA-M} 

Se determinó a través de los resultados del muestreo previamente mencionado que los parámetros más influyentes para GA-M son \emph{cprob}\footnote{Para una descripción de los parámetros de GA-M ver \textbf{Sección~\ref{sect:implementacion-heuristicas-hibridas}, Tabla~\ref{table:param-ga}}}, \emph{mprob} y \emph{p}. El parámetro restante \emph{n} se estableció en un valor fijo para las pruebas finales dado a su baja influencia en la calidad de la solución.

Las pruebas fueron realizadas de la siguiente manera. Se establecieron \emph{n} y \emph{p} en los valores recomendados por el trabajo referenciado y se realizaron pruebas para \emph{cprob} y \emph{mprob} en los siguientes rangos:

\begin{itemize}
\item \emph{cprob} [10,100], con un salto de 10.
\item \emph{mprob} [10,100], con un salto de 10.
\end{itemize}

Una vez establecida la mejor combinación de los parámetros \emph{cprob} y \emph{mprob} para ambos tipo de instancia, se procedió a realizar pruebas para \emph{p} dejando fijos \emph{cprob} y \emph{mprob}. Los rangos son los siguientes:

\begin{itemize}
\item \emph{p} [100,400], con un salto de 50.
\end{itemize}

Un total de 107 pruebas fueron realizadas utilizando todas las combinaciones de parámetros según los rangos anteriores. Un conjunto de las mismas se presentan en el \textbf{Apéndice~\ref{chap:apendiceC}}, 
\textbf{Tabla~\ref{table:GA-M-10-10}} a \textbf{Tabla~\ref{table:GA-M-90-70}}.

Se determinó que la probabilidad porcentual de cruce (\emph{cprob}) y la probabilidad porcentual de mutación tiene en promedio una alta incidencia en el aumento en calidad de la solución considerable. Por lo tanto, el valor de este parámetro se estableció de tal manera de que la combinación de ambos lograra conseguir soluciones de calidad.

Se determinó que el tamaño de la población (\emph{p}) tiene en promedio una alta incidencia en el tiempo del algoritmo y un aumento en calidad de la solución considerable. Por lo tanto, el valor de este parámetro se estableció de tal manera que lograra conseguir soluciones de calidad en tiempos aceptables.

Existe una diferencia en \emph{cprob} y \emph{p} para cada clase de instancia, probablemente debido a que la clase SalhiNagy posee instancias más grandes que las de Dethloff, necesitando de una población de mayor diversidad para obtener mejores resultados.\\

Con los parámetros seleccionados para ambos tipo de instancia se realizó una prueba la cual consistía en determinar el porcentaje total de veces que el algoritmo da como resultado la solución de mejor fitness antes de aplicar Vecindades Variables a las soluciones pertenecientes a la población. El resultado se encuentra en el \textbf{Apéndice~\ref{chap:apendiceC}}, 
\textbf{Tabla~\ref{table:GA-M-porcentajeD}} y \textbf{Tabla~\ref{table:GA-M-porcentajeS}}. La columna de nombre \textit{Por} en ambas tablas representa de forma porcentual la cantidad de veces que la solución de mejor fitness continúa siendo la de mejor costo luego de aplicar VND a toda la población. \\


En la \textbf{Tabla~\ref{table:parametros-entonados}} se presenta la combinación de parámetros recomendada para la ejecución de cada metaheurística.\\


\begin{table}[h]
\centering
\caption{Parámetros finales}
\label{table:parametros-entonados}
%\begin{threeparttable}
\begin{minipage}[ht]{0.386\linewidth}
\begin{table}[H]
\centering
\footnotesize
\begin{tabular}{c || ll}
\hline\hline
\textbf{Metaheurística} & \multicolumn{2}{c|}{\textbf{Parámetros finales}}\\
						& \multicolumn{2}{c|}{\textbf{Dethloff}}\\
\hline\hline\\
\emph{$GTS-M$} & mni: 5000     & lambda1: 0.05\\  
			& lambda2: 0.05 & tabu: 29      \\ [0.7ex]\cline{2-3}\\
\emph{$ILS-VND-M$} & n: 55 & LS: 70  \\ [0.7ex]\cline{2-3}\\
\emph{$AS-M$} & n: 5 		& alpha: 1.0 	\\  
			& beta: 3.0 & q: 0.1		\\
			& ro: 0.015 &	 			\\ [0.7ex]\cline{2-3}\\
\emph{$SS-M$} & n: 150 & b: 10 \\  
			& y: 0.1 & 		 \\ [0.7ex]\cline{2-3}\\
\emph{$PSO-M$} & \multicolumn{2}{c}{N/A} \\
			& \multicolumn{2}{c}{N/A} \\
			& \multicolumn{2}{c}{N/A} \\ [0.7ex]\cline{2-3}\\
\emph{$GA-M$} & n: 100    & p: 300  	 \\  
		   & cprob: 40 & mprob: 70   \\
\\ \hline\hline
\end{tabular}
\end{table}
\end{minipage}
\begin{minipage}[ht]{0.386\linewidth}
\begin{table}[H]
\centering
\footnotesize
\begin{tabular}{|ll}
\hline\hline
\multicolumn{2}{c}{\textbf{Parámetros finales}}\\
\multicolumn{2}{c}{\textbf{SalhiNagy}}\\
\hline\hline\\
			mni: 5000     & lambda1: 0.05\\  
			lambda2: 0.05 & tabu: 29 \\ [0.7ex]\cline{1-2}\\
			n: 55 & LS: 80 \\ [0.7ex]\cline{1-2}\\
			n: 6 		& alpha: 1.0 \\  
			beta: 3.0 & q: 0.1\\
			ro: 0.015 & \\ [0.7ex]\cline{1-2}\\
			n: 250 & b: 10\\  
			y: 0.2 & \\ [0.7ex]\cline{1-2}\\
			n: 30 & L: 70\\
			cp: 1 & cg: 0\\
			cl: 1 & \\ [0.7ex]\cline{1-2}\\
			n: 100    & p: 350 \\  
		   cprob: 90 & mprob: 70 \\
\\ \hline\hline
\end{tabular}
\end{table}
\end{minipage}
%\begin{tablenotes}
%       \item[1] Ver \textbf{Tabla~\ref{table:param-gts}}
%       \item[2] Ver \textbf{Tabla~\ref{table:param-gts}}
%       \item[3] Ver \textbf{Tabla~\ref{table:param-aco}}
%       \item[4] Ver \textbf{Tabla~\ref{table:param-gts}}
%       \item[5] Ver \textbf{Tabla~\ref{table:param-pso}}
%       \item[6] Ver \textbf{Tabla~\ref{table:param-gts}}
%\end{tablenotes}
%\end{threeparttable}
\end{table}

\subsection{Evaluación comparativa de la calidad de las metaheurísticas} \label{subsect:evaluacioncomparativa}

Se realizaron pruebas con la finalidad de comparar la calidad de los resultados de las metaheurísticas im\-ple\-men\-ta\-das con respecto a los resultados de los trabajos reproducidos. Para determinar el número de repeticiones se hace uso del \textit{teorema del límite central} (TLC)\\

De acuerdo con \cite{TCL} el TLC establece lo siguiente:\\

Si $x_1,x_2,...,x_n$ son variables aleatorias independientes, con idéntico modelo de probabilidad, de valor medio $\mu$ y varianza $\sigma^2$, entonces la distribución de las variables 

\begin{equation}
z = \dfrac{\sum_{i=1}^n x_i - n \mu}{\sigma\sqrt{n}}
\end{equation}

Se aproxima a la de una variable normal tipificada N(0,1). Este resultado prueba que la media muestral 

\begin{equation}
\bar{x} = \sum_{i=1}^n \dfrac{x_i}{n}
\end{equation}

se distribuye aproximadamente como una variable N($\mu$ , $\dfrac{\sigma}{\sqrt{n}}$).

Al menos en los modelos de probabilidad clásicos, se admite una aproximación aceptable al modelo normal, siempre que $n \geqslant 30$, a pesar de que esta cifra es insuficiente en determinados casos y excesiva en otros.\\

Por lo especificado anteriormente cada metaheurística fue corrida 30 veces con sus parámetros finales utilizando las instancias de Dethloff y SalhiNagy. De esta manera, se tiene una aproximación del comportamiento promedio de cada metaheurística para distintas clases de instancias en cuanto a tiempo de cómputo y costo de la solución. 

A continuación se presentan los resultados de las pruebas finales para cada metaheurística implementada. Pri\-me\-ra\-men\-te, se comparan los resultados obtenidos con cada uno de los trabajos referenciados para luego comparar las mejores metaheurísticas. Estas comparaciones se realizan tomando en cuenta el mejor costo obtenido y el tiempo de cómputo utilizado. Los resultados en negrilla son iguales a los de su metaheurística análoga, mientras que los resultados subrayados son mejores. \emph{\%Gap} representa el porcentaje de mejoría de la metaheurística implementada con respecto a su análoga, porcentajes negativos representan un mejoramiento o reducción del costo mínimo, mientras que porcentajes positivos implican empeoramiento del resultado.

Para el caso particular de las metaheurísticas GA-M y PSO-M, ciertas consideraciones fueron establecidas. En el caso de GA-M, el trabajo referenciado  no incluye resultados de las instancias utilizadas, por lo tanto la tabla de resultados sólo incluye los resultados de la metaheurística implementada (GA-M). En el caso de PSO-M, las instancias de Dethloff utilizadas no pueden ser usadas como entradas al algoritmo, ya que éstas contienen una matriz de costo entre cada uno de los clientes en lugar de la posición de cada cliente en el plano cartesiano. PSO-M necesita las coordenadas de cada cliente para la construcción de las soluciones. Ya que las instancias utilizadas no contienen esta información, esta clase de instancias no pudo ser utilizada, por lo tanto sólo se muestra la tabla correspondiente a las instancias de SalhiNagy.

Los tiempos de los trabajos referenciados fueron convertidos de manera aproximada al procesador utilizado para las pruebas finales, para ello se utilizaron los datos de \cite{website:maquinas}, que miden los megaflops\footnote{Millones de operaciones de punto flotante por segundo} (MFLOPS) empleados utilizando la librería \emph{Linpack}\footnote{Paquete de rutinas de álgebra lineal donde se utilizan constantemente operaciones de punto flotante} en las máquinas de cada uno de los trabajos re\-fe\-ren\-cia\-dos (ver \textbf{Tabla~\ref{table:comparacion-procesadores}}). Esto con la finalidad de permitir la comparación entre los tiempos de los trabajos reproducidos y las metaheurísticas implementadas. En la \textbf{Tabla~\ref{table:comparacion-procesadores}} se muestran las relaciones de tiempo entre el procesador \emph{Pentium Dual-Core 3.0GHz} y los otros procesadores. Este valor se utiliza como factor de conversión de los tiempos.\\

\begin{table}[h]
\centering
\footnotesize
\caption{Especificación de los procesadores utilizados en las pruebas y factor de conversión correspondiente}
\begin{tabular}{ c | c || c }
\hline\hline
\textbf{Procesador} & \textbf{MFLOPS} & \textbf{Relación de tiempo}\\ \hline\hline
Pentium IV 2.4GHz & 687.10 & 0.75:1\\
Core 2 Duo 2.5GHz & 1642.00 & 1.79:1\\
Pentium D 2.8GHz & 853.06 & 0.93:1\\
Athlon 2.0GHz & 734.64 & 0.80:1\\
Pentium IV 3.4GHz & 941.67 & 1.03:1\\
Pentium IV 3.0GHz & 830.88 & 0.91:1\\ \hline
Pentium Dual-Core 3.0GHz & 912.78 & 1:1\\
\hline\hline
\end{tabular}
\label{table:comparacion-procesadores}
\end{table}

Para cada metaheurística se presentan dos tablas correspondientes a los resultados comparativos utilizando las clases de instancias Dethloff y SalhiNagy. La tabla correspondiente a instancias Dethloff se muestra en su forma re\-du\-ci\-da, combinando instancias con las mismas características. Las tablas completas se pueden ver en el \textbf{Apéndice~\ref{chap:apendiceB}}.\\

\subsection{Resultados para Metaheurística GTS-M}

A continuación se presentan los resultados de la metaheurística denominada GTS-M, comparada con la metaheurística propuesta por \cite{gts} que se denominará GTS.\\

En la tabla \textbf{Tabla~\ref{tabla-final-gtsD}} se aprecia que el mejor costo promedio de GTS-M en Dethloff se reduce ligeramente en comparación con el de GTS (-0.04\%). Específicamente, según la \textbf{Tabla~\ref{apendice-tabla-final-gtsD}} el mejor costo se reduce en 9 de 40 instancias, mientras que en 30 se logra el mismo costo. GTS-M logra reducir los tiempos de cómputo promedio en aproximadamente un 30\% para esta clase de instancia.

En la tabla \textbf{Tabla~\ref{tabla-final-gtsS}} correspondiente a instancias de SalhiNagy se observa un empeoramiento del mejor costo promedio con respecto a GTS (0.77\%), reduciendo los mejores costos reportados en GTS en sólo 2 instancias sobre un total de 14 (\textbf{Tabla~\ref{apendice-tabla-final-gtsS}}). Sin embargo, se aprecia una reducción aproximada de 8\% en los tiempos de cómputo promedio.

El mejoramiento en el mejor costo promedio en instancias Dethloff puede ser atribuido a la implementación de una búsqueda en vecindades variables en lugar de los movimientos de búsqueda local utilizados por GTS. Sin embargo, se puede observar que en instancias de SalhiNagy no ocurre tal mejoramiento. Esto puede deberse a un criterio de selección distinto al momento de seleccionar un operador en el algoritmo de búsqueda local, lo cual puede incidir en la calidad de la solución si se utiliza información de la topología del problema en esta selección. El tamaño de la lista tabú fue descartado como causante de esta diferencia, ya que forma parte de la lista de parámetros que fueron entonados por clase de instancia (Dethloff y SalhiNagy). La reducción de los tiempos de cómputo en ambas clases de instancia se puede atribuir a una implementación eficiente de la representación de la solución y la lista tabú, así como un criterio de selección de operador distinto establecido para favorecer los tiempos de cómputo (ver \textbf{ Sección~\ref{sect:implementacion-gts}}).\\

GTS-M obtiene un mejor costo promedio que prácticamente no difiere de GTS para instancias Dethloff, mejorando los tiempos de cómputo. Por lo tanto se recomienda su utilización para este tipo de instancias. Sin embargo, para instancias SalhiNagy no se recomienda pues la diferencia entre mejor costo promedio es significativa.

\begin{table}[h]
\caption{ Resultados de GTS-M, utilizando instancias de Dethloff}
\centering
\scriptsize
\begin{tabular*}{1.00\textwidth}{@{\extracolsep{\fill}} |c||c c||c c c c c c|}
\hline
 & \multicolumn{2}{c||}{\bf{GTS}} & \multicolumn{6}{c|}{\bf{GTS-M}}\\\hline
Instancia & Mejor costo & T.(seg) & & Mejor costo & T.(seg) & Costo prom. & T. prom.(seg) & \%Gap\\ [0.5ex]
\hline\hline
SCA3 & 
673.44 & 2.29 & & \bf{\underline{673.402}} & 
1.56 & 675.22 & 1.59 & -0.01\\SCA8 & 
1028.29 & 2.73 & & \bf{\underline{1028.15}} & 
1.72 & 1039.80 & 1.50 & -0.01\\CON3 & 
561.48 & 2.48 & & \bf{\underline{560.95}} & 
2.30 & 564.70 & 1.55 & -0.10\\CON8 & 
771.91 & 2.78 & & \bf{\underline{771.745}} & 
1.90 & 781.84 & 1.50 & -0.02\\\hline\hline\bf{PROM.} & 
\bf{758.78} & \bf{2.57} & & \bf{758.56} & \bf{1.87} & \bf{765.39} & \bf{1.53} & \bf{-0.04}\\[1ex]\hline
\end{tabular*}
\label{tabla-final-gtsD}
\end{table}

%\begin{table}[H]
%\caption{\footnotesize Resultados del experimento intensivo de la metaheurística GTS, utilizando instancias de Dethloff}
%\centering
%\scriptsize
%\begin{tabular}{|c||c c||c c c c c c|}
%\hline
%& \multicolumn{2}{c||}{\bf{GTS}} & \multicolumn{6}{c|}{\bf{GTS-M}}\\\hline
%Instancia & Costo & T. & & Costo & T. & Costo prom. & T. prom. & \%Gap\\ [0.5ex]
%\hline\hline
%SCA3-0 & 636.06 & 2.12 & & \bf{\underline{635.67}} & 
%1.52 & 638.59 & 1.50 & -0.06\\SCA3-1 & 697.84 & 1.59 & & \bf{697.84} & 
%1.62 & 698.24 & 1.57 & 0.00\\SCA3-2 & 659.34 & 1.93 & & \bf{659.34} & 
%0.92 & 659.34 & 1.41 & 0.00\\SCA3-3 & 680.04 & 2.34 & & \bf{680.04} & 
%3.84 & 681.62 & 1.72 & 0.00\\SCA3-4 & 690.50 & 2.01 & & \bf{690.50} & 
%2.05 & 692.99 & 1.92 & 0.00\\SCA3-5 & 659.90 & 1.92 & & \bf{659.90} & 
%1.11 & 662.09 & 1.48 & 0.00\\SCA3-6 & 651.09 & 3.30 & & \bf{651.09} & 
%1.55 & 651.52 & 1.43 & 0.00\\SCA3-7 & 659.17 & 2.23 & & \bf{659.17} & 
%0.68 & 666.43 & 1.47 & 0.00\\SCA3-8 & 719.47 & 2.53 & & \bf{719.47} & 
%1.11 & 719.80 & 1.74 & 0.00\\SCA3-9 & 681.00 & 2.89 & & \bf{681.00} & 
%1.21 & 681.55 & 1.65 & 0.00\\SCA8-0 & 961.50 & 2.40 & & \bf{961.50} & 
%1.00 & 977.46 & 1.95 & 0.00\\SCA8-1 & 1050.20 & 2.66 & & \bf{\underline{1049.65}} & 
%2.34 & 1064.18 & 1.37 & -0.05\\SCA8-2 & 1039.64 & 3.50 & & \bf{1039.64} & 
%1.43 & 1049.81 & 1.45 & 0.00\\SCA8-3 & 983.34 & 2.46 & & \bf{983.34} & 
%2.40 & 1006.85 & 1.66 & 0.00\\SCA8-4 & 1065.49 & 2.01 & & \bf{1065.49} & 
%1.92 & 1069.93 & 1.61 & 0.00\\SCA8-5 & 1027.08 & 3.37 & & \bf{1027.08} & 
%2.01 & 1043.66 & 1.42 & 0.00\\SCA8-6 & 971.82 & 2.00 & & \bf{971.82} & 
%0.95 & 978.22 & 1.34 & 0.00\\SCA8-7 & 1052.17 & 3.24 & & \bf{\underline{1051.28}} & 
%1.67 & 1064.38 & 1.55 & -0.08\\SCA8-8 & 1071.18 & 2.57 & & \bf{1071.18} & 
%1.30 & 1076.29 & 1.15 & 0.00\\SCA8-9 & 1060.50 & 3.09 & & \bf{1060.50} & 
%2.20 & 1067.23 & 1.45 & 0.00\\CON3-0 & 616.52 & 2.91 & & \bf{616.52} & 
%2.26 & 623.07 & 1.82 & 0.00\\CON3-1 & 554.47 & 2.22 & & \bf{554.47} & 
%1.40 & 556.06 & 1.46 & 0.00\\CON3-2 & 519.26 & 2.49 & & \bf{\underline{518.00}} & 
%3.58 & 522.35 & 1.62 & -0.24\\CON3-3 & 591.19 & 2.08 & & \bf{591.19} & 
%2.49 & 593.81 & 1.56 & 0.00\\CON3-4 & 589.32 & 2.34 & & \bf{\underline{588.79}} & 
%1.59 & 592.91 & 1.49 & -0.09\\CON3-5 & 563.70 & 2.58 & & \bf{563.70} & 
%1.77 & 567.71 & 1.39 & 0.00\\CON3-6 & 500.80 & 2.23 & & \bf{\underline{499.05}} & 
%1.06 & 501.21 & 1.62 & -0.35\\CON3-7 & 576.48 & 1.80 & & \bf{576.48} & 
%2.51 & 582.81 & 1.70 & 0.00\\CON3-8 & 523.05 & 3.76 & & \bf{523.05} & 
%3.91 & 523.05 & 1.25 & 0.00\\CON3-9 & 580.05 & 2.35 & & \bf{\underline{578.25}} & 
%2.44 & 583.99 & 1.62 & -0.31\\CON8-0 & 857.17 & 2.55 & & \bf{857.17} & 
%2.22 & 877.23 & 1.41 & 0.00\\CON8-1 & 740.85 & 2.79 & & \bf{740.85} & 
%1.31 & 755.43 & 1.44 & 0.00\\CON8-2 & 713.44 & 2.15 & & \bf{\underline{712.89}} & 
%1.50 & 721.38 & 1.56 & -0.08\\CON8-3 & 811.07 & 2.86 & & \bf{811.07} & 
%2.67 & 822.00 & 1.63 & 0.00\\CON8-4 & 772.25 & 2.23 & & \bf{772.25} & 
%0.74 & 781.12 & 1.55 & 0.00\\CON8-5 & 756.91 & 4.32 & & \bf{\underline{754.88}} & 
%1.25 & 758.81 & 1.40 & -0.27\\CON8-6 & 678.92 & 3.00 & & \bf{678.92} & 
%3.97 & 692.10 & 1.61 & 0.00\\CON8-7 & \bf{811.96} & 1.84 & & 
%812.89 & 2.35 & 819.24 & 1.54 & 0.11\\CON8-8 & 767.53 & 3.15 & & \bf{767.53} & 
%2.19 & 775.47 & 1.36 & 0.00\\CON8-9 & 809.00 & 2.90 & & \bf{809.00} & 
%0.76 & 815.65 & 1.50 & 0.00\\\hline\hline\bf{PROM.} & 
%\bf{758.78} & \bf{2.57} & & \bf{758.56} & \bf{1.87} & \bf{765.39} & \bf{1.53} & \bf{-0.04}\\[1ex]\hline
%\end{tabular}
%\label{tabla-final-gtsD}
%\end{table}

\begin{table}[h]
\caption{ Resultados de GTS-M, utilizando instancias de SalhiNagy}
\centering
\scriptsize
\begin{tabular*}{1.00\textwidth}{@{\extracolsep{\fill}} |c||c c||c c c c c c|}
\hline
& \multicolumn{2}{c||}{\bf{GTS}} & \multicolumn{6}{c|}{\bf{GTS-M}}\\\hline
Instancia & Mejor costo & T.(seg) & & Mejor costo & T.(seg) & Costo prom. & T. prom.(seg) & \%Gap\\ [0.5ex]
\hline\hline
CMT1X & 470.48 & 3.04 & & \bf{470.48} & 
1.74 & 472.08 & 1.69 & 0.00\\CMT1Y & 470.48 & 2.40 & & \bf{470.48} & 
0.80 & 472.85 & 1.50 & 0.00\\CMT2X & 682.39 & 4.89 & & \bf{682.39} & 
2.94 & 686.99 & 2.56 & 0.00\\CMT2Y & 682.39 & 5.95 & & \bf{682.39} & 
4.84 & 687.45 & 2.77 & 0.00\\CMT3X & 719.06 & 7.89 & & \bf{\underline{718.40}} & 
7.32 & 727.00 & 5.60 & -0.09\\CMT3Y & \bf{719.06} & 9.93 & & 
723.40 & 3.02 & 728.30 & 4.77 & 0.60\\CMT4X & \bf{854.21} & 17.23 & & 
855.09 & 29.81 & 870.76 & 17.27 & 0.10\\CMT4Y & \bf{852.46} & 21.48 & & 
854.68 & 12.06 & 874.23 & 14.87 & 0.26\\CMT5X & \bf{1030.56} & 43.21 & & 
1036.17 & 34.35 & 1066.62 & 28.63 & 0.54\\CMT5Y & 1031.69 & 40.35 & & \bf{\underline{1027.60}} & 
46.03 & 1066.36 & 27.14 & -0.40\\CMT11X & \bf{831.09} & 12.61 & & 
871.93 & 5.20 & 911.93 & 15.14 & 4.91\\CMT11Y & \bf{829.85} & 11.44 & & 
851.74 & 18.23 & 905.47 & 12.58 & 2.64\\CMT12X & \bf{658.83} & 9.03 & & 
663.64 & 6.53 & 677.07 & 4.58 & 0.73\\CMT12Y & \bf{660.47} & 7.82 & & 
670.10 & 8.19 & 680.86 & 5.68 & 1.46\\\hline\hline\bf{PROM.} & 
\bf{749.50} & \bf{14.09} & & \bf{755.61} & \bf{12.93} & \bf{773.43} & \bf{10.34} & \bf{0.77}\\[1ex]\hline
\end{tabular*}
\label{tabla-final-gtsS}
\end{table}

\subsection{Resultados para Metaheurística ILS-VND-M} 

A continuación se presentan los resultados de la metaheurística denominada ILS-VND-M, comparada con la metaheurística propuesta por \cite{ils-vnd} que se denominará ILS-VND.\\

En la tabla \textbf{Tabla~\ref{table:finalD-ILS}} se aprecia que el mejor costo promedio de ILS-VND-M en Dethloff aumenta ligeramente en comparación con el de ILS-VND (0.35\%). Específicamente, según la \textbf{Tabla~\ref{apendice-table:finalD-ILS}} en 20 instancias se logra el mismo costo, mientras que en las otras 20 empeora ligeramente con un máximo de 1.96\% . ILS-VND-M aumenta los tiempos de cómputo promedio en aproximadamete un 8\% para esta clase de instancia.\\

En la tabla \textbf{Tabla~\ref{table:finalS-ILS}} correspondiente a instancias de SalhiNagy se observa un empeoramiento del mejor costo promedio con respecto a ILS-VND, con un máximo de 3.16\% y el promedio de 1.31. Sin embargo, se aprecia una reducción del 14\% en los tiempos de cómputo promedio.


ILS-VND-M obtiene un costo promedio que prácticamente no difiere de ILS-VND para instancias Dethloff, empeorando los tiempos de cómputo. Por lo tanto no se recomienda su utilización para este tipo de instancias. Para instancias SalhiNagy no se recomienda pues la diferencia entre mejor costo promedio es significativa.


En la \textbf{Tabla~\ref{table:finalD-ILS}} y \textbf{Tabla~\ref{table:finalS-ILS}}.

%Pentium 4E         3000    912.78          
%Core 2 Duo  	   2500    1642

%Relacion ILS-M  : ILS
%		 1		: 1.79
		 
\begin{table}[h]
\caption{ Resultados de ILS-VND-M, utilizando instancias de Dethloff}
\centering
\scriptsize
\begin{tabular*}{1.00\textwidth}{@{\extracolsep{\fill}} |c||c c||c c c c c c|}
\hline
 & \multicolumn{2}{c||}{\bf{ILS-VND}} & \multicolumn{6}{c|}{\bf{ILS-VND-M}}\\\hline
Instancia & Mejor costo & T.(seg) & & Mejor costo & T.(seg) & Costo prom. & T. prom.(seg) & \%Gap\\ [0.5ex]
\hline\hline
SCA3 & 
\bf{673.40} & 2.12 & & 
674.01 & 5.10 & 682.27 & 4.92 & 0.09\\SCA8 & 
\bf{1028.15} & 5.63 & & 
1035.59 & 4.14 & 1072.68 & 3.96 & 0.73\\CON3 & 
\bf{560.95} & 3.10 & & 
561.29 & 4.66 & 573.01 & 4.91 & 0.07\\CON8 & 
\bf{771.65} & 5.60 & & 
775.56 & 3.98 & 806.86 & 4.03 & 0.53\\\hline\hline\bf{PROM.} & 
\bf{758.54} & \bf{4.11} & & \bf{761.61} & \bf{4.47} & \bf{783.71} & \bf{4.45} & \bf{0.35}\\[1ex]\hline
\end{tabular*}
\label{table:finalD-ILS}
\end{table}		 
		 
%\begin{table}[H]
%\caption{\footnotesize Resultados del experimento intensivo de la metaheurística ILS, utilizando instancias de Dethloff}
%\centering
%\scriptsize
%\begin{tabular}{|c||c c c||c c c c c c c|}
%\hline
% & \multicolumn{3}{c||}{\bf{ILS}} & \multicolumn{7}{c|}{\bf{ILS-M}}\\\hline
%Instancia & Costo & T. & $\gamma$ & & Costo & T. & Costo prom. & T. prom. & $\gamma$ & \%Gap\\ [0.5ex]
%\hline\hline
%SCA3-0 & \bf{635.62} & 1.60 & 0.40 & & 
%636.06 & 5.05 & 641.14 & 4.74 & 0.50 & 0.07\\SCA3-1 & 697.84 & 2.00 & 0.00 & & \bf{697.84} & 
%4.66 & 704.58 & 4.91 & 0.30 & 0.00\\SCA3-2 & 659.34 & 2.13 & 0.00 & & \bf{659.34} & 
%5.61 & 670.74 & 5.12 & 0.70 & 0.00\\SCA3-3 & 680.04 & 2.02 & 0.10 & & \bf{680.04} & 
%4.90 & 684.47 & 5.01 & 0.40 & 0.00\\SCA3-4 & 690.50 & 2.36 & 0.00 & & \bf{690.50} & 
%4.78 & 697.71 & 4.89 & 0.50 & 0.00\\SCA3-5 & 659.90 & 2.09 & 0.10 & & \bf{659.90} & 
%5.09 & 683.41 & 4.75 & 0.30 & 0.00\\SCA3-6 & 651.09 & 2.20 & 0.00 & & \bf{651.09} & 
%5.46 & 654.98 & 5.08 & 0.00 & 0.00\\SCA3-7 & \bf{659.17} & 3.02 & 1.00 & & 
%664.88 & 5.12 & 671.43 & 4.97 & 0.50 & 0.87\\SCA3-8 & 719.47 & 1.93 & 0.00 & & \bf{719.47} & 
%5.75 & 724.92 & 5.08 & 1.00 & 0.00\\SCA3-9 & 681.00 & 1.84 & 0.00 & & \bf{681.00} & 
%4.56 & 689.38 & 4.64 & 0.50 & 0.00\\SCA8-0 & \bf{961.50} & 4.51 & 0.10 & & 
%970.64 & 5.41 & 1026.34 & 3.88 & 0.30 & 0.95\\SCA8-1 & \bf{1049.65} & 5.33 & 0.60 & & 
%1050.93 & 4.73 & 1092.84 & 3.97 & 0.60 & 0.12\\SCA8-2 & \bf{1039.64} & 6.12 & 0.40 & & 
%1051.21 & 3.61 & 1078.00 & 3.74 & 0.70 & 1.11\\SCA8-3 & \bf{983.34} & 6.15 & 0.80 & & 
%1002.63 & 3.14 & 1034.67 & 3.76 & 1.00 & 1.96\\SCA8-4 & \bf{1065.49} & 4.90 & 0.90 & & 
%1067.55 & 3.96 & 1109.56 & 4.01 & 0.50 & 0.19\\SCA8-5 & \bf{1027.08} & 6.15 & 0.70 & & 
%1042.30 & 4.10 & 1081.86 & 3.93 & 0.80 & 1.48\\SCA8-6 & 971.82 & 4.43 & 0.00 & & \bf{971.82} & 
%3.39 & 1009.95 & 4.29 & 0.90 & 0.00\\SCA8-7 & \bf{1051.28} & 9.55 & 0.40 & & 
%1067.11 & 4.63 & 1096.02 & 4.19 & 0.10 & 1.51\\SCA8-8 & 1071.18 & 3.66 & 0.00 & & \bf{1071.18} & 
%3.92 & 1097.02 & 3.88 & 0.70 & 0.00\\SCA8-9 & 1060.50 & 5.54 & 0.60 & & \bf{1060.50} & 
%4.55 & 1100.56 & 3.97 & 0.10 & 0.00\\CON3-0 & 616.52 & 3.61 & 1.00 & & \bf{616.52} & 
%3.88 & 637.12 & 4.90 & 0.40 & 0.00\\CON3-1 & 554.47 & 3.27 & 1.00 & & \bf{554.47} & 
%4.97 & 562.30 & 4.87 & 0.20 & 0.00\\CON3-2 & \bf{518.00} & 3.75 & 0.70 & & 
%521.38 & 5.12 & 526.63 & 4.80 & 0.00 & 0.65\\CON3-3 & 591.19 & 2.39 & 0.00 & & \bf{591.19} & 
%4.25 & 598.74 & 4.95 & 0.40 & 0.00\\CON3-4 & 588.79 & 3.20 & 0.90 & & \bf{588.79} & 
%4.88 & 599.75 & 4.70 & 0.60 & 0.00\\CON3-5 & 563.70 & 3.06 & 0.50 & & \bf{563.70} & 
%4.50 & 581.84 & 4.79 & 0.40 & 0.00\\CON3-6 & 499.05 & 3.45 & 1.00 & & \bf{499.05} & 
%4.75 & 509.25 & 4.92 & 0.60 & 0.00\\CON3-7 & 576.48 & 2.72 & 0.60 & & \bf{576.48} & 
%4.18 & 593.89 & 4.93 & 0.30 & 0.00\\CON3-8 & 523.05 & 2.70 & 0.10 & & \bf{523.05} & 
%5.19 & 530.15 & 5.01 & 0.90 & 0.00\\CON3-9 & \bf{578.24} & 2.82 & 0.70 & & 
%578.25 & 4.90 & 590.40 & 5.21 & 0.30 & 0.00\\CON8-0 & 857.17 & 6.69 & 0.90 & & \bf{857.17} & 
%4.26 & 908.34 & 3.84 & 0.20 & 0.00\\CON8-1 & \bf{740.85} & 5.04 & 0.90 & & 
%742.29 & 3.10 & 773.51 & 4.05 & 0.80 & 0.19\\CON8-2 & \bf{712.89} & 4.40 & 1.00 & & 
%713.60 & 4.36 & 736.12 & 4.18 & 0.30 & 0.10\\CON8-3 & \bf{811.07} & 5.04 & 0.80 & & 
%814.50 & 5.04 & 843.77 & 4.31 & 0.80 & 0.42\\CON8-4 & \bf{772.25} & 6.03 & 0.80 & & 
%776.60 & 4.74 & 825.05 & 3.71 & 0.90 & 0.56\\CON8-5 & \bf{754.88} & 5.90 & 0.40 & & 
%759.87 & 2.84 & 786.40 & 3.83 & 0.30 & 0.66\\CON8-6 & \bf{678.92} & 5.44 & 0.80 & & 
%691.30 & 2.94 & 712.07 & 4.33 & 1.00 & 1.82\\CON8-7 & \bf{811.96} & 4.88 & 0.30 & & 
%814.50 & 4.18 & 836.31 & 3.92 & 0.60 & 0.31\\CON8-8 & \bf{767.53} & 6.20 & 0.80 & & 
%775.62 & 4.45 & 799.86 & 4.00 & 0.40 & 1.05\\CON8-9 & \bf{809.00} & 6.44 & 0.70 & & 
%810.18 & 3.92 & 847.20 & 4.09 & 0.20 & 0.15\\\hline\hline\bf{PROM.} & 
%\bf{758.54} & \bf{4.11} & \bf{} & & \bf{761.61} & \bf{4.47} & \bf{783.71} & \bf{4.45} & & \bf{0.35}\\[1ex]\hline
%\end{tabular}
%\label{table:finalD-ILS}
%\end{table}

\begin{table}[h]
\caption{ Resultados de ILS-VND-M, utilizando instancias de SalhiNagy}
\centering
\scriptsize
\begin{tabular*}{1.00\textwidth}{@{\extracolsep{\fill}} |c||c c c||c c c c c c c|}
\hline
 & \multicolumn{3}{c||}{\bf{ILS-VND}} & \multicolumn{7}{c|}{\bf{ILS-VND-M}}\\\hline
Instancia & Mejor costo & T.(seg) & $\gamma$ & & Mejor costo & T.(seg) & Costo prom. & T. prom.(seg) & $\gamma$ & \%Gap\\ [0.5ex]
\hline\hline
CMT1X & \bf{466.77} &  1.96 & 0.30 & & 
470.48 & 4.43 & 482.96 & 4.61 & 0.05 & 0.79\\CMT1Y & \bf{466.77} & 1.93 & 0.35 & & 
470.48 & 4.68 & 481.25 & 4.36 & 0.15 & 0.79\\CMT2X & \bf{684.21} & 12.51 & 0.35 & & 
685.96 & 10.26 & 711.33 & 9.79 & 0.00 & 0.26\\CMT2Y & \bf{684.21} & 10.45 & 0.10 & & 
694.80 & 11.78 & 710.06 & 10.16 & 0.20 & 1.55\\CMT3X & \bf{721.40} & 13.90 & 0.50 & & 
722.84 & 19.64 & 737.43 & 18.71 & 0.00 & 0.20\\CMT3Y & \bf{721.40} &  11.45 & 0.50 & & 
723.67 & 19.83 & 741.25 & 19.67 & 0.30 & 0.31\\CMT4X & \bf{852.83} & 90.78 & 0.40 & & 
867.22 & 47.10 & 894.29 & 45.44 & 0.00 & 1.69\\CMT4Y & \bf{852.46} & 82.44 & 0.45 & & 
867.74 & 46.34 & 895.34 & 47.15 & 0.15 & 1.79\\CMT5X & \bf{1030.55} & 95.78 & 0.15 & & 
1058.84 & 78.09 & 1096.99 & 91.58 & 0.40 & 2.75\\CMT5Y & \bf{1031.17} & 105.14 & 0.30 & & 
1063.75 & 84.06 & 1097.46 & 90.52 & 0.05 & 3.16\\CMT11X & \bf{839.39} &  22.51 & 0.45  & & 
850.66 & 38.34 & 892.22 & 32.60 & 0.30 & 1.34\\CMT11Y & \bf{841.88} & 26.49 & 0.45 & & 
847.67 & 30.42 & 876.59 & 32.21 & 0.35 & 0.69\\CMT12X & \bf{662.22} & 14.35 & 0.05  & & 
671.76 & 19.01 & 683.18 & 17.09 & 0.35 & 1.44\\CMT12Y & \bf{662.22} & 10.82 & 0.10 & & 
673.01 & 17.64 & 683.36 & 17.46 & 0.00 & 1.63\\\hline\hline\bf{PROM.} & 
\bf{751.25} & \bf{35.75} & \bf{} & & \bf{762.06} & \bf{30.83} & \bf{784.55} & \bf{31.53} & & \bf{1.31}\\[1ex]\hline
\end{tabular*}
\label{table:finalS-ILS}
\end{table}

\subsection{Resultados para Metaheurística AS-M}

A continuación se presentan los resultados de la metaheurística denominada AS-M, comparada con la metaheurística propuesta por \cite{maco} que se denominará AS.\\

Para la clase de instancias Dethloff (\textbf{Tabla~\ref{table:aco-finalD}}) se aprecia un ligero mejoramiento del mejor costo  promedio con respecto a AS (-0.29\%), consiguiendo mejorar los costos de 27 instancias de un total de 40 (\textbf{Tabla~\ref{apendice-table:aco-finalD}}). El tiempo de cómputo promedio se redujo de una manera llamativa en aproximadamente   99\%.

Para la clase SalhiNagy se aprecia un mejoramiento del mejor costo promedio con respecto a AS (-2.18\%), consiguiendo reducir todos los mejores costos de AS, es decir, 14 instancias de 14. El tiempo de cómputo promedio fue reducido en aproximadamente 94\%.

El mejoramiento del mejor costo promedio se atribuye a la variedad de vecindades visitadas por VND en comparación con las utilizadas en AS. La reducción contundente del tiempo de cómputo promedio ocurre gracias a la reducción de  iteraciones (n=5 para Dethloff, n=6 para SalhiNagy) en comparación con las recomendadas en AS (n=1000), ya que durante la etapa de entonación de este parámetro se determinó que luego de determinado número de iteraciones la solución dejaba de mejorar (ver \textbf{Sección~\ref{subsect:pruebasentonacion}}). El número de iteraciones pudo ser reducido gracias al aporte que lleva VND al  mejoramiento de las soluciones.

AS-M mejora el mejor costo promedio de las dos clases de instancias utilizadas, así como también mejora contundentemente los tiempos de cómputo, por lo tanto AS-M se recomienda para instancias Dethloff y SalhiNagy.


%Pentium 4E         3000    912.78          
%Pentium 4E         2800    853.06        
%
%Relacion ACO-M  : ACO
%		 1		: 0.93

\begin{table}[h]
\caption{ Resultados de AS-M, utilizando instancias de Dethloff }
\centering
\scriptsize
\begin{tabular*}{1.00\textwidth}{@{\extracolsep{\fill}} |c||c c||c c c c c c|}
\hline
 & \multicolumn{2}{c||}{\bf{AS}} & \multicolumn{6}{c|}{\bf{AS-M}}\\\hline
Instancia & Mejor costo & T.(seg) & & Mejor costo & T.(seg) & Costo prom. & T. prom.(seg) & \%Gap\\ [0.5ex]
\hline\hline
SCA3 & 
675.38 & N/A & & \bf{\underline{673.441}} & 
1.28 & 675.05 & 1.27 & -0.29\\SCA8 & 
1037.26 & N/A & & \bf{\underline{1032.01}} & 
1.32 & 1040.47 & 1.31 & -0.49\\CON3 & 
561.57 & N/A & & \bf{\underline{561.063}} & 
1.35 & 563.83 & 1.34 & -0.09\\CON8 & 
776.08 & N/A & & \bf{\underline{773.938}} & 
1.39 & 780.09 & 1.40 & -0.30\\\hline\hline\bf{PROM.} & 
\bf{762.57} & \bf{419.43} & & \bf{760.11} & \bf{1.33} & \bf{764.86} & \bf{1.33} & \bf{-0.29}\\[1ex]\hline
\end{tabular*}
\label{table:aco-finalD}
\end{table}

%\begin{table}[H]
%\caption{\footnotesize Resultados del experimento intensivo de la metaheurística ACO, utilizando instancias de Dethloff}
%\centering
%\scriptsize
%\begin{tabular}{|c||c c||c c c c c c|}
%\hline
% & \multicolumn{2}{c||}{\bf{ACO}} & \multicolumn{6}{c|}{\bf{ACO-M}}\\\hline
%Instancia & Costo & T. & & Costo & T. & Costo prom. & T. prom. & \%Gap\\ [0.5ex]
%\hline\hline
%SCA3-0 & 636.10 & N/A & & \bf{\underline{636.06}} & 
%1.21 & 636.23 & 1.26 & -0.01\\SCA3-1 & 700.10 & N/A & & \bf{\underline{697.84}} & 
%1.35 & 697.84 & 1.40 & -0.32\\SCA3-2 & \bf{659.30} & N/A & & 
%659.34 & 1.33 & 661.75 & 1.25 & 0.01\\SCA3-3 & \bf{680.00} & N/A & & 
%680.04 & 1.26 & 680.08 & 1.24 & 0.01\\SCA3-4 & 690.50 & N/A & & \bf{690.50} & 
%1.34 & 690.50 & 1.34 & 0.00\\SCA3-5 & 670.10 & N/A & & \bf{\underline{659.90}} & 
%1.30 & 663.62 & 1.34 & -1.52\\SCA3-6 & 651.10 & N/A & & \bf{\underline{651.09}} & 
%1.32 & 653.03 & 1.30 & -0.00\\SCA3-7 & 666.10 & N/A & & \bf{\underline{659.17}} & 
%1.26 & 666.01 & 1.18 & -1.04\\SCA3-8 & 719.50 & N/A & & \bf{\underline{719.47}} & 
%1.29 & 720.34 & 1.30 & -0.00\\SCA3-9 & 681.00 & N/A & & \bf{681.00} & 
%1.15 & 681.07 & 1.11 & 0.00\\SCA8-0 & 961.50 & N/A & & \bf{961.50} & 
%1.32 & 979.44 & 1.36 & 0.00\\SCA8-1 & 1063.00 & N/A & & \bf{\underline{1052.71}} & 
%1.20 & 1060.59 & 1.19 & -0.97\\SCA8-2 & \bf{1040.6} & N/A & & 
%1044.24 & 1.13 & 1049.75 & 1.07 & 0.35\\SCA8-3 & 985.90 & N/A & & \bf{\underline{985.47}} & 
%1.28 & 1004.04 & 1.26 & -0.04\\SCA8-4 & 1071.00 & N/A & & \bf{\underline{1065.49}} & 
%1.36 & 1069.27 & 1.32 & -0.51\\SCA8-5 & 1054.30 & N/A & & \bf{\underline{1034.74}} & 
%1.44 & 1049.81 & 1.46 & -1.86\\SCA8-6 & 972.50 & N/A & & \bf{\underline{972.48}} & 
%1.44 & 978.84 & 1.43 & -0.00\\SCA8-7 & \bf{1059.70} & N/A & & 
%1066.65 & 1.36 & 1069.62 & 1.38 & 0.66\\SCA8-8 & 1082.70 & N/A & & \bf{\underline{1071.18}} & 
%1.48 & 1075.89 & 1.45 & -1.06\\SCA8-9 & 1081.40 & N/A & & \bf{\underline{1065.60}} & 
%1.20 & 1067.40 & 1.16 & -1.46\\CON3-0 & \bf{616.50} & N/A & & 
%616.52 & 1.46 & 622.14 & 1.42 & 0.00\\CON3-1 & 555.60 & N/A & & \bf{\underline{554.47}} & 
%1.38 & 557.09 & 1.38 & -0.20\\CON3-2 & 521.40 & N/A & & \bf{\underline{519.11}} & 
%1.31 & 521.03 & 1.29 & -0.44\\CON3-3 & 591.20 & N/A & & \bf{\underline{591.19}} & 
%1.40 & 591.24 & 1.44 & -0.00\\CON3-4 & 589.30 & N/A & & \bf{\underline{588.79}} & 
%1.27 & 590.28 & 1.22 & -0.09\\CON3-5 & 563.70 & N/A & & \bf{563.70} & 
%1.27 & 566.64 & 1.32 & 0.00\\CON3-6 & 499.20 & N/A & & \bf{\underline{499.07}} & 
%1.56 & 502.31 & 1.51 & -0.03\\CON3-7 & 577.50 & N/A & & \bf{\underline{576.48}} & 
%1.24 & 578.59 & 1.21 & -0.18\\CON3-8 & 523.10 & N/A & & \bf{\underline{523.05}} & 
%1.28 & 523.61 & 1.30 & -0.01\\CON3-9 & \bf{578.20} & N/A & & 
%578.25 & 1.32 & 585.38 & 1.29 & 0.01\\CON8-0 & \bf{858.20} & N/A & & 
%859.74 & 1.28 & 872.98 & 1.32 & 0.18\\CON8-1 & 740.90 & N/A & & \bf{\underline{740.85}} & 
%1.41 & 744.43 & 1.38 & -0.01\\CON8-2 & 714.30 & N/A & & \bf{\underline{712.89}} & 
%1.63 & 715.34 & 1.57 & -0.20\\CON8-3 & 812.30 & N/A & & \bf{\underline{811.07}} & 
%1.38 & 816.57 & 1.37 & -0.15\\CON8-4 & \bf{770.10} & N/A & & 
%772.76 & 1.30 & 780.95 & 1.25 & 0.35\\CON8-5 & 766.60 & N/A & & \bf{\underline{754.95}} & 
%1.31 & 760.71 & 1.33 & -1.52\\CON8-6 & 697.20 & N/A & & \bf{\underline{684.05}} & 
%1.46 & 694.02 & 1.54 & -1.89\\CON8-7 & 814.80 & N/A & & \bf{\underline{814.50}} & 
%1.22 & 816.40 & 1.23 & -0.04\\CON8-8 & \bf{771.30} & N/A & & 
%778.39 & 1.46 & 785.59 & 1.53 & 0.92\\CON8-9 & 815.10 & N/A & & \bf{\underline{810.18}} & 
%1.41 & 813.93 & 1.45 & -0.60\\\hline\hline\bf{PROM.} & 
%\bf{762.57} & \bf{419.43} & & \bf{760.11} & \bf{1.33} & \bf{764.86} & \bf{1.33} & \bf{-0.29}\\[1ex]\hline
%\end{tabular}
%\label{table:aco-finalD}
%\end{table}

\begin{table}[h]
\caption{ Resultados de AS-M, utilizando instancias de SalhiNagy}
\centering
\scriptsize
\begin{tabular*}{1.00\textwidth}{@{\extracolsep{\fill}} |c||c c||c c c c c c|}
\hline
 & \multicolumn{2}{c||}{\bf{AS}} & \multicolumn{6}{c|}{\bf{AS-M}}\\\hline
Instancia & Mejor costo & T.(seg) & & Mejor costo & T.(seg) & Costo prom. & T. prom.(seg) & \%Gap\\ [0.5ex]
\hline\hline
CMT1X & 470.67 & N/A & & \bf{\underline{470.48}} & 
1.74 & 474.70 & 1.42 & -0.04\\CMT1Y & 472.37 & N/A & & \bf{\underline{471.25}} & 
1.35 & 474.94 & 1.41 & -0.24\\CMT2X & 705.21 & N/A & & \bf{\underline{691.13}} & 
6.64 & 700.84 & 6.86 & -2.00\\CMT2Y & 704.16 & N/A & & \bf{\underline{686.76}} & 
6.76 & 700.44 & 6.59 & -2.47\\CMT3X & 726.55 & N/A & & \bf{\underline{724.21}} & 
21.06 & 729.24 & 21.57 & -0.32\\CMT3Y & 729.02 & N/A & & \bf{\underline{723.52}} & 
22.69 & 731.28 & 22.67 & -0.75\\CMT4X & 893.90 & N/A & & \bf{\underline{865.47}} & 
94.77 & 883.96 & 96.93 & -3.18\\CMT4Y & 895.25 & N/A & & \bf{\underline{870.98}} & 
97.12 & 886.17 & 97.64 & -2.71\\CMT5X & 1115.75 & N/A & & \bf{\underline{1077.34}} & 
273.74 & 1085.90 & 274.51 & -3.44\\CMT5Y & 1112.61 & N/A & & \bf{\underline{1070.53}} & 
274.44 & 1085.51 & 275.57 & -3.78\\CMT11X & 887.36 & N/A & & \bf{\underline{847.74}} & 
43.28 & 865.98 & 42.96 & -4.46\\CMT11Y & 874.13 & N/A & & \bf{\underline{845.47}} & 
37.33 & 869.98 & 37.90 & -3.28\\CMT12X & 681.02 & N/A & & \bf{\underline{663.01}} & 
18.74 & 675.11 & 18.70 & -2.64\\CMT12Y & 671.32 & N/A & & \bf{\underline{663.19}} & 
18.18 & 674.76 & 17.89 & -1.21\\\hline\hline\bf{PROM.} & 
\bf{781.38} & \bf{1131.81} & & \bf{762.22} & \bf{65.56} & \bf{774.20} & \bf{65.90} & \bf{-2.18}\\[1ex]\hline
\end{tabular*}
\label{table:aco-finalS}
\end{table}

\subsection{Resultados para Metaheurística SS-M}


A continuación se presentan los resultados de la metaheurística denominada GTS-M, comparada con la metaheurística propuesta por \cite{SCAimp} que se denominará SS.\\

En la tabla  \textbf{Tabla~\ref{table:finalD-SCA}} se aprecia que el mejor costo promedio de SS-M en Dethloff se reduce ligeramente en comparación con el de SS (-0.13\%). Específicamente, según la \textbf{Tabla~\ref{apendice-table:finalD-SCA}} el mejor costo se reduce en 18 de 40 instancias, mientras que en 11 se logra el mismo costo. SS-M aumenta significativamente los tiempos de cómputo promedio en aproximadamete un 63.5\% para esta clase de instancia.

En la tabla  \textbf{Tabla~\ref{table:finalS-SCA}}. correspondiente a instancias de SalhiNagy se observa un empeoramiento del mejor costo promedio con respecto a SS, reduciendo los mejores costos reportados en SS en sólo 6 instancias sobre un total de 14 (\textbf{Tabla~\ref{apendice-tabla-final-gtsS}}). SS-M aumenta significativamente los tiempos de cómputo promedio en aproximadamete un 34.2\% para esta clase de instancia.

El aumento significativo de los tiempos de cómputo promedio en las instancias de Dethloff y SalhiNagy, puede ser atribuido a la implementación de un mecanismo de inserción para el método de combinación. En SS se realiza la mejor inserción posible de un solo cliente escogido arbitrariamente, mientras que el mecanismo de inserción utilizado en SS-M se escoge entre todos los clientes no asignados aquel cliente que dé el menor costo añadido a la solución parcial.


SS-M obtiene un mejor costo promedio que prácticamente no difiere de SS para instancias Dethloff. Por lo tanto se recomienda su utilización para este tipo de instancias. Para instancias SalhiNagy no se recomienda, pues el mejor costo promedio, a pesar de no ser significativo, es mayor para SS-M, adem\'{a}s de tener un aumento notable en los tiempos de cómputo.


%Pentium 4E         3000   912.78
%Athlon XP          2000   734.64
%
%Relacion SCA-M  : SCA
%		 1		: 0.80
		 
		 
\begin{table}[h]
\caption{ Resultados de SS-M, utilizando instancias de Dethloff}
\centering
\scriptsize
\begin{tabular*}{1.00\textwidth}{@{\extracolsep{\fill}} |c||c c||c c c c c c|}
\hline
 & \multicolumn{2}{c||}{\bf{SS}} & \multicolumn{6}{c|}{\bf{SS-M}}\\\hline
Instancia & Mejor costo & T.(seg) & & Mejor costo & T.(seg) & Costo prom. & T. prom.(seg) & \%Gap\\ [0.5ex]
\hline\hline
SCA3 & 
\bf{674.16} & N/A & & 
675.29 & 3.19 & 677.67 & 3.34 & 0.17\\SCA8 & 
1044.35 & N/A & & \bf{\underline{1037.37}} & 
11.18 & 1050.07 & 9.53 & -0.65\\CON3 & 
564.17 & N/A & & \bf{\underline{561.897}} & 
2.88 & 564.71 & 2.77 & -0.38\\CON8 & \bf{774.305} & N/A & & 
777.01 & 10.04 & 785.72 & 9.20 & 0.34\\\hline\hline\bf{PROM.} & 
\bf{764.24} & \bf{2.49} & & \bf{762.89} & \bf{6.82} & \bf{769.54} & \bf{6.21} & \bf{-0.13}\\[1ex]\hline
\end{tabular*}
\label{table:finalD-SCA}
\end{table}		 
		 
%\begin{table}[H]
%\caption{\footnotesize Resultados del experimento intensivo de la metaheurística SS, utilizando instancias de Dethloff}
%\centering
%\scriptsize
%\begin{tabular}{|c||c c||c c c c c c|}
%\hline
% & \multicolumn{2}{c||}{\bf{SS}} & \multicolumn{6}{c|}{\bf{SS-M}}\\\hline
%Instancia & Costo & T. & & Costo & T. & Costo prom. & T. prom. & \%Gap\\ [0.5ex]
%\hline\hline
%SCA3-0 & 640.55 & N/A & & \bf{640.55} & 
%2.45 & 640.55 & 2.95 & 0.00\\SCA3-1 & 697.84 & N/A & & \bf{697.84} & 
%1.91 & 700.30 & 3.70 & 0.00\\SCA3-2 & 659.34 & N/A & & \bf{659.34} & 
%3.01 & 664.13 & 3.74 & 0.00\\SCA3-3 & 680.04 & N/A & & \bf{680.04} & 
%3.26 & 681.08 & 3.54 & 0.00\\SCA3-4 & 690.50 & N/A & & \bf{690.50} & 
%4.81 & 690.98 & 3.14 & 0.00\\SCA3-5 & \bf{659.90} & N/A & & 
%665.04 & 2.29 & 673.31 & 2.57 & 0.78\\SCA3-6 & 653.81 & N/A & & \bf{\underline{652.94}} & 
%2.97 & 653.52 & 3.03 & -0.13\\SCA3-7 & \bf{659.17} & N/A & & 
%666.15 & 3.51 & 670.58 & 3.39 & 1.06\\SCA3-8 & 719.47 & N/A & & \bf{719.47} & 
%3.88 & 719.70 & 3.31 & 0.00\\SCA3-9 & 681.00 & N/A & & \bf{681.00} & 
%3.80 & 682.57 & 4.01 & 0.00\\SCA8-0 & 981.47 & N/A & & \bf{\underline{965.26}} & 
%9.36 & 985.82 & 8.54 & -1.65\\SCA8-1 & 1077.44 & N/A & & \bf{\underline{1053.57}} & 
%17.13 & 1069.26 & 9.85 & -2.22\\SCA8-2 & 1050.98 & N/A & & \bf{\underline{1050.37}} & 
%13.90 & 1052.30 & 9.72 & -0.06\\SCA8-3 & \bf{983.34} & N/A & & 
%1014.10 & 8.64 & 1027.24 & 7.63 & 3.13\\SCA8-4 & 1073.46 & N/A & & \bf{\underline{1065.49}} & 
%9.08 & 1080.71 & 8.69 & -0.74\\SCA8-5 & 1047.24 & N/A & & \bf{\underline{1040.18}} & 
%13.45 & 1057.39 & 10.34 & -0.67\\SCA8-6 & 995.59 & N/A & & \bf{\underline{972.48}} & 
%10.50 & 980.72 & 9.08 & -2.32\\SCA8-7 & 1068.56 & N/A & & \bf{\underline{1067.49}} & 
%9.42 & 1073.92 & 9.24 & -0.10\\SCA8-8 & 1080.58 & N/A & & \bf{\underline{1071.18}} & 
%8.71 & 1090.33 & 8.87 & -0.87\\SCA8-9 & 1084.80 & N/A & & \bf{\underline{1073.62}} & 
%11.62 & 1082.97 & 13.35 & -1.03\\CON3-0 & 631.39 & N/A & & \bf{\underline{617.59}} & 
%1.85 & 621.83 & 2.32 & -2.19\\CON3-1 & 554.47 & N/A & & \bf{554.47} & 
%3.18 & 559.74 & 2.97 & 0.00\\CON3-2 & 522.86 & N/A & & \bf{\underline{521.38}} & 
%3.99 & 521.60 & 2.64 & -0.28\\CON3-3 & 591.19 & N/A & & \bf{591.19} & 
%1.59 & 591.40 & 2.89 & 0.00\\CON3-4 & 591.12 & N/A & & \bf{\underline{588.79}} & 
%3.17 & 591.10 & 2.88 & -0.39\\CON3-5 & \bf{563.70} & N/A & & 
%564.88 & 3.59 & 565.86 & 2.46 & 0.21\\CON3-6 & 506.19 & N/A & & \bf{\underline{502.16}} & 
%2.07 & 502.95 & 1.97 & -0.80\\CON3-7 & 577.68 & N/A & & \bf{\underline{576.48}} & 
%3.92 & 581.42 & 3.52 & -0.21\\CON3-8 & \bf{523.00} & N/A & & 
%523.05 & 3.21 & 523.70 & 3.46 & 0.01\\CON3-9 & 580.05 & N/A & & \bf{\underline{578.98}} & 
%2.21 & 587.54 & 2.65 & -0.18\\CON8-0 & \bf{860.48} & N/A & & 
%869.08 & 5.89 & 884.50 & 8.02 & 1.00\\CON8-1 & 740.85 & N/A & & \bf{740.85} & 
%9.85 & 748.62 & 9.70 & 0.00\\CON8-2 & 723.32 & N/A & & \bf{\underline{713.05}} & 
%12.08 & 717.21 & 8.79 & -1.42\\CON8-3 & \bf{811.23} & N/A & & 
%815.71 & 15.30 & 830.53 & 10.34 & 0.55\\CON8-4 & \bf{772.25} & N/A & & 
%777.24 & 10.68 & 787.88 & 10.06 & 0.65\\CON8-5 & 756.91 & N/A & & \bf{\underline{754.95}} & 
%9.39 & 763.88 & 10.53 & -0.26\\CON8-6 & \bf{678.92} & N/A & & 
%686.34 & 8.60 & 694.54 & 8.24 & 1.09\\CON8-7 & 814.50 & N/A & & \bf{814.50} & 
%11.56 & 816.88 & 10.20 & 0.00\\CON8-8 & \bf{775.59} & N/A & & 
%785.30 & 8.20 & 790.54 & 7.04 & 1.25\\CON8-9 & \bf{809.00} & N/A & & 
%813.10 & 8.80 & 822.63 & 9.10 & 0.51\\\hline\hline\bf{PROM.} & 
%\bf{764.24} & \bf{2.49} & & \bf{762.89} & \bf{6.82} & \bf{769.54} & \bf{6.21} & \bf{-0.13}\\[1ex]\hline
%\end{tabular}
%\label{table:finalD-SCA}
%\end{table}

\begin{table}[h]
\caption{ Resultados de SS-M, utilizando instancias de SalhiNagy}
\centering
\scriptsize
\begin{tabular*}{1.00\textwidth}{@{\extracolsep{\fill}} |c||c c||c c c c c c|}
\hline
 & \multicolumn{2}{c||}{\bf{SS}} & \multicolumn{6}{c|}{\bf{SS-M}}\\\hline
Instancia & Mejor costo & T.(seg) & & Mejor costo & T.(seg) & Costo prom. & T. prom.(seg) & \%Gap\\ [0.5ex]
\hline\hline
CMT1X & 473.23 & 1.15 & & \bf{\underline{470.67}} & 
1.80 & 473.84 & 1.73 & -0.54\\CMT1Y & 475.14 & 0.92 & & \bf{\underline{472.37}} & 
2.60 & 473.23 & 2.43 & -0.58\\CMT2X & \bf{691.11} & 12.61 & & 
699.80 & 14.75 & 708.96 & 17.39 & 1.26\\CMT2Y & \bf{690.50} & 11.42 & & 
700.28 & 18.89 & 711.13 & 14.05 & 1.42\\CMT3X & 730.11 & 33.82 & & \bf{\underline{726.98}} & 
30.34 & 738.05 & 35.62 & -0.43\\CMT3Y & \bf{725.26} & 25.79 & & 
727.77 & 27.80 & 737.32 & 36.29 & 0.35\\CMT4X & \bf{860.97} & 214.03 & & 
866.65 & 421.30 & 905.41 & 231.34 & 0.66\\CMT4Y & \bf{868.50} & 230.74 & & 
872.44 & 321.56 & 905.47 & 232.94 & 0.45\\CMT5X & \bf{1041.19} & 955.04 & & 
1072.03 & 773.75 & 1110.77 & 1071.87 & 2.96\\CMT5Y & \bf{1069.32} & 448.69 & & 
1077.14 & 1142.85 & 1110.33 & 949.78 & 0.73\\CMT11X & 883.63 & 28.63 & & \bf{\underline{874.80}} & 
68.83 & 903.88 & 40.83 & -1.00\\CMT11Y & \bf{873.19} & 83.94 & & 
880.42 & 92.37 & 906.81 & 44.55 & 0.83\\CMT12X & 686.38 & 25.99 & & \bf{\underline{673.49}} & 
232.82 & 682.30 & 59.59 & -1.88\\CMT12Y & 679.17 & 11.68 & & \bf{\underline{675.05}} & 
74.83 & 684.71 & 63.02 & -0.61\\\hline\hline\bf{PROM.} & 
\bf{767.69} & \bf{151.48} & & \bf{770.71} & \bf{230.32} & \bf{789.44} & \bf{200.10} & \bf{0.26}\\[1ex]\hline
\end{tabular*}
\label{table:finalS-SCA}
\end{table}

\subsection{Resultados para Metaheurística PSO-M}

A continuación se presentan los resultados de la metaheurística denominada PSO-M, comparada con la metaheurística propuesta por \cite{mpso} que se denominará PSO.\\

Como lo mencionado a principios de la \textbf{Sección~\ref{subsect:evaluacioncomparativa}} las pruebas de PSO-M fueron realizadas sólo en instancias de la clase SalhiNagy, por lo tanto sólo se muestra la \textbf{Tabla~\ref{table:pso-finalS}} correspondiente a instancias SalhiNagy. 

Se puede observar un mejoramiento considerable del mejor costo promedio para SalhiNagy (\textbf{Tabla~\ref{table:pso-finalS}}) en comparación con PSO (-2.27\%). Específicamente, el mejor costo fue mejorado en 10 de 14 instancias. Además, se evidencia una reducción importante del tiempo de cómputo promedio de un 72\% aproximadamente.

El mejor costo promedio mejora gracias a la aplicación de VND al final de cada iteración de PSO-M, en lugar de aplicar sólo el operador de búsqueda local \emph{2-opt} utilizado en PSO. La mejora del tiempo se atribuye a la reducción significativa del número de iteraciones (n=30) en comparación con el número de iteraciones en PSO (n=1000), reducido gracias al aporte que lleva VND al  mejoramiento de las soluciones.

PSO-M mejora tanto el mejor costo promedio como el tiempo de cómputo para el tipo de instancia utilizada (SalhiNagy), por lo tanto, se recomienda su uso.

%Pentium 4E         3000   912.78          
%Pentium 4          3400   941.67
%
%Relacion PSO-M  : PSO
%		 1		: 1.03

\begin{table}[h]
\caption{ Resultados de PSO-M, utilizando instancias de SalhiNagy}
\centering
\scriptsize
\begin{tabular*}{1.00\textwidth}{@{\extracolsep{\fill}} |c||c c||c c c c c c|}
\hline
 & \multicolumn{2}{c||}{\bf{PSO}} & \multicolumn{6}{c|}{\bf{PSO-M}}\\\hline
Instancia & Mejor costo & T.(seg) & & Mejor costo & T.(seg) & Costo prom. & T. prom.(seg) & \%Gap\\ [0.5ex]
\hline\hline
CMT1X & \bf{467.00} & 41.20 & & 
470.48 & 5.06 & 477.81 & 4.98 & 0.75\\CMT1Y & \bf{467.00} & 41.20 & & 
470.48 & 4.89 & 476.11 & 5.13 & 0.75\\CMT2X & 710.00 & 55.62 & & \bf{\underline{693.96}} & 
8.25 & 735.46 & 6.99 & -2.26\\CMT2Y & 710.00 & 55.62 & & \bf{\underline{693.74}} & 
5.95 & 721.04 & 6.77 & -2.29\\CMT3X & 738.00 & 117.42 & & \bf{\underline{723.67}} & 
60.99 & 735.31 & 62.71 & -1.94\\CMT3Y & 740.00 & 116.39 & & \bf{\underline{723.23}} & 
63.47 & 737.00 & 63.39 & -2.27\\CMT4X & 912.00 & 213.21 & & \bf{\underline{878.06}} & 
103.76 & 916.43 & 101.35 & -3.72\\CMT4Y & 913.00 & 210.12 & & \bf{\underline{872.63}} & 
79.17 & 906.71 & 101.04 & -4.42\\CMT5X & 1167.00 & 293.55 & & \bf{\underline{1090.08}} & 
115.89 & 1151.52 & 103.82 & -6.59\\CMT5Y & 1142.00 & 294.58 & & \bf{\underline{1089.93}} & 
102.17 & 1141.93 & 117.49 & -4.56\\CMT11X & 895.00 & 232.78 & & \bf{\underline{883.28}} & 
14.86 & 913.41 & 20.12 & -1.31\\CMT11Y & 900.00 & 234.84 & & \bf{\underline{887.01}} & 
16.95 & 910.41 & 21.66 & -1.44\\CMT12X & \bf{691.00} & 118.45 & & 
691.19 & 3.83 & 749.96 & 3.93 & 0.03\\CMT12Y & 697.00 & 117.42 & & \bf{\underline{679.79}} & 
3.81 & 755.03 & 3.95 & -2.47\\\hline\hline\bf{PROM.} & 
\bf{796.36} & \bf{153.02} & & \bf{774.82} & \bf{42.08} & \bf{809.15} & \bf{44.52} & \bf{-2.27}\\[1ex]\hline
\end{tabular*}
\label{table:pso-finalS}
\end{table}

\subsection{Resultados para Metaheurística GA-M}


A continuación se presentan los resultados de la metaheurística denominada GA-M en la \textbf{Tabla~\ref{table:finalD-IGA}} y \textbf{Tabla~\ref{table:finalS-IGA}}. Debido a que \cite{IGAimp} no presenta resultados para ninguno de los tipos de instancia utilizados en este trabajo, resulta imposible hacer una comparación



\begin{minipage}[h]{0.45\linewidth}
\begin{table}[H]
\caption{ Resultados de GA-M, utilizando instancias de Dethloff}
\centering
\scriptsize 
\begin{tabular}{|c||c c c c|}
\hline
 \multicolumn{5}{|c|}{\bf{GA-M}}\\\hline
Instancia & Mejor & T. & Costo prom. & T. prom.\\
& costo & (seg) & & (seg)\\ [0.5ex]
\hline\hline
SCA3 & 673.76 & 4.18 & 675.28 & 4.06\\
SCA8 & 1036.44 & 3.88 & 1040.17 & 3.99\\
CON3 & 562.35 & 4.61 & 563.59 & 4.28\\
CON8 & 776.01 & 4.49 & 778.37 & 4.17\\\hline\hline\bf{PROM.} & 
\bf{762.14} & \bf{4.29} & \bf{764.35} & \bf{4.12}\\[1ex]\hline
\end{tabular}
\label{table:finalD-IGA}
\end{table}
%\begin{table}[H]
%\scriptsize 
%\caption{\footnotesize Resultados del experimento intensivo de la metaheurística GA, utilizando instancias de Dethloff }
%\centering 
%\begin{tabular}{|c||c c c c|}
%\hline
% \multicolumn{5}{|c|}{\bf{GA-M}}\\\hline
%Instancia & Costo & T. & Costo prom. & T. prom.\\ [0.5ex]
%\hline\hline
%SCA3-0 & 636.06 & 4.43 & 639.50 & 4.24\\
%SCA3-1 & 697.84 & 4.67 & 698.20 & 4.14\\
%SCA3-2 & 661.13 & 3.77 & 662.05 & 3.93\\
%SCA3-3 & 680.04 & 3.81 & 680.37 & 4.08\\
%SCA3-4 & 690.50 & 4.08 & 690.50 & 3.91\\
%SCA3-5 & 659.90 & 3.92 & 663.87 & 4.25\\
%SCA3-6 & 652.47 & 4.24 & 652.92 & 4.07\\
%SCA3-7 & 659.17 & 4.27 & 664.29 & 3.93\\
%SCA3-8 & 719.47 & 4.15 & 720.10 & 4.11\\
%SCA3-9 & 681.00 & 4.51 & 681.00 & 3.93\\
%SCA8-0 & 973.03 & 4.16 & 974.19 & 4.15\\
%SCA8-1 & 1059.16 & 3.77 & 1061.80 & 3.89\\
%SCA8-2 & 1050.37 & 4.16 & 1051.48 & 4.12\\
%SCA8-3 & 985.60 & 3.77 & 1008.15 & 4.25\\
%SCA8-4 & 1069.71 & 4.27 & 1072.14 & 3.83\\
%SCA8-5 & 1043.52 & 3.46 & 1045.61 & 3.98\\
%SCA8-6 & 976.69 & 3.98 & 976.69 & 3.99\\
%SCA8-7 & 1067.03 & 3.36 & 1068.46 & 3.92\\
%SCA8-8 & 1071.18 & 4.20 & 1071.31 & 3.94\\
%SCA8-9 & 1068.10 & 3.66 & 1071.91 & 3.82\\
%CON3-0 & 617.59 & 4.40 & 619.16 & 4.26\\
%CON3-1 & 556.04 & 7.15 & 557.63 & 4.34\\
%CON3-2 & 521.38 & 4.56 & 521.38 & 4.57\\
%CON3-3 & 591.20 & 5.13 & 591.31 & 4.10\\
%CON3-4 & 589.32 & 5.40 & 592.13 & 4.23\\
%CON3-5 & 563.70 & 3.74 & 565.23 & 4.29\\
%CON3-6 & 500.88 & 4.04 & 502.67 & 4.29\\
%CON3-7 & 577.54 & 3.67 & 578.91 & 3.93\\
%CON3-8 & 523.05 & 4.02 & 524.11 & 4.36\\
%CON3-9 & 582.79 & 4.02 & 583.33 & 4.43\\
%CON8-0 & 869.15 & 4.99 & 870.06 & 4.17\\
%CON8-1 & 741.70 & 4.60 & 751.80 & 4.31\\
%CON8-2 & 713.44 & 4.64 & 713.49 & 4.40\\
%CON8-3 & 811.07 & 4.72 & 811.07 & 4.23\\
%CON8-4 & 772.25 & 4.74 & 777.48 & 3.97\\
%CON8-5 & 759.44 & 4.56 & 759.60 & 4.04\\
%CON8-6 & 685.80 & 4.83 & 689.52 & 4.20\\
%CON8-7 & 814.79 & 3.67 & 814.79 & 3.84\\
%CON8-8 & 780.80 & 4.10 & 784.11 & 4.13\\
%CON8-9 & 811.66 & 4.06 & 811.81 & 4.36\\\hline\hline\bf{PROM.} & 
% \bf{762.14} & \bf{4.29} & \bf{764.35} & \bf{4.12}\\[1ex]\hline
%\end{tabular}
%\label{table:finalD-IGA}
%\end{table}
\end{minipage}
\hspace{0.5cm}
\begin{minipage}[h]{0.45\linewidth}
\begin{table}[H]
\scriptsize 
\caption{ Resultados de GA-M, utilizando instancias de SalhiNagy }
\centering
\begin{tabular}{|c||c c c c|}
\hline
\multicolumn{5}{|c|}{\bf{GA-M}}\\\hline
Instancia & Mejor & T. & Costo prom. & T. prom.\\
& costo & (seg) & & (seg)\\ [0.5ex]
\hline\hline
CMT1X & 476.38 & 5.06 & 476.71 & 5.04\\
CMT1Y & 472.87 & 4.56 & 477.66 & 4.64\\
CMT2X & 697.28 & 10.72 & 705.32 & 11.00\\
CMT2Y & 697.76 & 10.15 & 706.51 & 10.63\\
CMT3X & 728.50 & 24.66 & 736.12 & 24.48\\
CMT3Y & 725.30 & 24.70 & 734.65 & 24.09\\
CMT4X & 880.66 & 63.16 & 896.91 & 65.36\\
CMT4Y & 892.05 & 65.54 & 902.53 & 66.57\\
CMT5X & 1068.63 & 134.22 & 1098.41 & 134.47\\
CMT5Y & 1087.40 & 137.00 & 1104.05 & 137.40\\
CMT11X & 858.94 & 40.64 & 891.06 & 41.90\\
CMT11Y & 848.68 & 45.25 & 877.11 & 45.61\\
CMT12X & 668.48 & 24.83 & 673.47 & 24.73\\
CMT12Y & 672.60 & 24.47 & 674.22 & 24.43\\\hline\hline\bf{PROM.} & 
\bf{769.68} & \bf{43.93} & \bf{782.48} & \bf{44.31}\\[1ex]\hline
\end{tabular}
\label{table:finalS-IGA}
\end{table}
\end{minipage}

\section{Comparación de las mejores metaheurísticas}

A continuación se presenta una comparativa de las mejores metaheurísticas y una breve descripción de los resultados.

En la tabla \textbf{Tabla~\ref{tabla-final-comparativaD}} se encuentran los resultados de las mejores metaheurísticas para instancias Dethloff.

En la tabla \textbf{Tabla~\ref{tabla-final-comparativaS}} se encuentran los resultados de las mejores metaheurísticas para instancias SalhiNagy.\\

Se puede observar que para instancias de Dethloff (\textbf{Tabla~\ref{tabla-final-comparativaD}}) la metaheurística ILS-VND logró obtener los mejores costos para todas las instancias con respecto a las otras. Sin embargo, la metaheurística implementada GTS-M logró obtener el mejor costo en 37 instancias de 40, reduciendo el tiempo promedio en más de un 40\% con respecto a ILS-VND. Además, la metaheurística implementada AS-M logró el mejor costo en 22 instancias de 40, obteniendo un mejor costo promedio que se diferencia de ILS-VND en sólo un 0.20\%, reduciendo el tiempo promedio en más de un 60\%.

Para instancias de SalhiNagy (\textbf{Tabla~\ref{tabla-final-comparativaS}}) ninguna de las metaheurísticas implementadas escogidas logró conseguir los mejores resultados. La metaheurística GTS logra el mejor costo promedio consiguiendo el mejor costo en 9 de 14 instancias, seguido de ILS-VND que consigue 6 de 14 instancias. La metaheurística implementada que muestra mejores resultados para SalhiNagy es AS-M, que consiguió un mejor costo promedio que se diferencia de GTS en sólo un 1.66\%, con tiempos bastante competitivos.

\begin{table}[h]
\caption{ Resultados comparativos finales utilizando instancias de Dethloff}
\centering
\scriptsize
\begin{tabular*}{1.00\textwidth}{@{\extracolsep{\fill}} |c||c c|c c|c c|c c|c c|}
\hline
\textbf{Instancia} & \multicolumn{2}{c|}{\textbf{AS-M}} & \multicolumn{2}{c|}{\textbf{GTS-M}} & 
\multicolumn{2}{c|}{\textbf{GA-M}} & \multicolumn{2}{c|}{\textbf{ILS}} & \multicolumn{2}{c|}{\textbf{SS-M}} \\\hline
 & Mejor & T. & Mejor & T. & Mejor & T. & Mejor & T. & Mejor & T.\\
 & costo & (seg) & costo & (seg) & costo & (seg) & costo & (seg) & costo & (seg)\\\hline\hline
SCA3-0 & 636.06 & 1.21 & 635.67 & 1.52 & 636.06 & 4.43 & \textbf{635.62} & \underline{1.60} & 640.55 & 2.45\\
SCA3-1 & \textbf{697.84} & \underline{1.35} & \textbf{697.84} & 1.62 & \textbf{697.84} & 4.67 & \textbf{697.84} & 2.00 & \textbf{697.84} & 1.91\\
SCA3-2 & \textbf{659.34} & 1.33 & \textbf{659.34} & \underline{0.92} & 661.13 & 3.77 & \textbf{659.34} & 2.13 & \textbf{659.34} & 3.01\\
SCA3-3 & \textbf{680.04} & \underline{1.26} & \textbf{680.04} & 3.84 & \textbf{680.04} & 3.81 & \textbf{680.04} & 2.02 & \textbf{680.04} & 3.26\\
SCA3-4 & \textbf{690.50} & \underline{1.34} & \textbf{690.50} & 2.05 & \textbf{690.50} & 4.08 & \textbf{690.50} & 2.36 & \textbf{690.50} & 4.81\\
SCA3-5 & \textbf{659.90} & 1.30 & \textbf{659.90} & \underline{1.11} & \textbf{659.90} & 3.92 & \textbf{659.90} & 2.09 & 665.04 & 2.29\\
SCA3-6 & \textbf{651.09} & \underline{1.32} & \textbf{651.09} & 1.55 & 652.47 & 4.24 & \textbf{651.09} & 2.20 & 652.94 & 2.97\\
SCA3-7 & \textbf{659.17} & 1.26 & \textbf{659.17} & \underline{0.68} & \textbf{659.17} & 4.27 & \textbf{659.17} & 3.02 & 666.15 & 3.51\\
SCA3-8 & \textbf{719.47} & 1.29 & \textbf{719.47} & \underline{1.11} & \textbf{719.47} & 4.15 & \textbf{719.47} & 1.93 & \textbf{719.47} & 3.88\\
SCA3-9 & \textbf{681.00} & \underline{1.15} & \textbf{681.00} & 1.21 & \textbf{681.00} & 4.51 & \textbf{681.00} & 1.84 & \textbf{681.00} & 3.80\\
SCA8-0 & \textbf{961.50} & 1.32 & \textbf{961.50} & \underline{1.00} & 973.03 & 4.16 & \textbf{961.50} & 4.51 & 965.26 & 9.36\\
SCA8-1 & 1052.71 & 1.20 & \textbf{1049.65} & \underline{2.34} & 1059.16 & 3.77 & \textbf{1049.65} & 5.33 & 1053.57 & 17.13\\
SCA8-2 & 1044.24 & 1.13 & \textbf{1039.64} & \underline{1.43} & 1050.37 & 4.16 & \textbf{1039.64} & 6.12 & 1050.37 & 13.90\\
SCA8-3 & 985.47 & 1.28 & \textbf{983.34} & \underline{2.40} & 985.60 & 3.77 & \textbf{983.34} & 6.15 & 1014.10 & 8.64\\
SCA8-4 & \textbf{1065.49} & \underline{1.36} & \textbf{1065.49} & 1.92 & 1069.71 & 4.27 & \textbf{1065.49} & 4.90 & \textbf{1065.49} & 9.08\\
SCA8-5 & 1034.74 & 1.44 & \textbf{1027.08} & \underline{2.01} & 1043.52 & 3.46 & \textbf{1027.08} & 6.15 & 1040.18 & 13.45\\
SCA8-6 & 972.48 & 1.44 & \textbf{971.82} & \underline{0.95} & 976.69 & 3.98 & \textbf{971.82} & 4.43 & 972.48 & 10.50\\
SCA8-7 & 1066.65 & 1.36 & \textbf{1051.28} & \underline{1.67} & 1067.03 & 3.36 & \textbf{1051.28} & 9.55 & 1067.49 & 9.42\\
SCA8-8 & \textbf{1071.18} & 1.48 & \textbf{1071.18} & \underline{1.30} & \textbf{1071.18} & 4.20 & \textbf{1071.18} & 3.66 & \textbf{1071.18} & 8.71\\
SCA8-9 & 1065.60 & 1.20 & \textbf{1060.50} & \underline{2.20} & 1068.10 & 3.66 & \textbf{1060.50} & 5.54 & 1073.62 & 11.62\\
CON3-0 & \textbf{616.52} & \underline{1.46} & \textbf{616.52} & 2.26 & 617.59 & 4.40 & \textbf{616.52} & 3.61 & 617.59 & 1.85\\
CON3-1 & \textbf{554.47} & \underline{1.38} & \textbf{554.47} & 1.40 & 556.04 & 7.15 & \textbf{554.47} & 3.27 & \textbf{554.47} & 3.18\\
CON3-2 & 519.11 & 1.31 & \textbf{518.00} & \underline{3.58} & 521.38 & 4.56 & \textbf{518.00} & 3.75 & 521.38 & 3.99\\
CON3-3 & \textbf{591.19} & \underline{1.40} & \textbf{591.19} & 2.49 & 591.20 & 5.13 & \textbf{591.19} & 2.39 & \textbf{591.19} & 1.59\\
CON3-4 & \textbf{588.79} & \underline{1.27} & \textbf{588.79} & 1.59 & 589.32 & 5.40 & \textbf{588.79} & 3.20 & \textbf{588.79} & 3.17\\
CON3-5 & \textbf{563.70} & \underline{1.27} & \textbf{563.70} & 1.77 & \textbf{563.70} & 3.74 & \textbf{563.70} & 3.06 & 564.88 & 3.59\\
CON3-6 & 499.07 & 1.56 & \textbf{499.05} & \underline{1.06} & 500.88 & 4.04 & \textbf{499.05} & 3.45 & 502.16 & 2.07\\
CON3-7 & \textbf{576.48} & 1.24 & \textbf{576.48} & 2.51 & 577.54 & 3.67 & \textbf{576.48} & 2.72 & \textbf{576.48} & 3.92\\
CON3-8 & \textbf{523.05} & \underline{1.28} & \textbf{523.05} & 3.91 & \textbf{523.05} & 4.02 & \textbf{523.05} & 2.70 & \textbf{523.05} & 3.21\\
CON3-9 & 578.25 & 1.32 & 578.25 & 2.44 & 582.79 & 4.02 & \textbf{578.24} & \underline{2.82} & 578.98 & 2.21\\
CON8-0 & 859.74 & 1.28 & \textbf{857.17} & \underline{2.22} & 869.15 & 4.99 & \textbf{857.17} & 6.69 & 869.08 & 5.89\\
CON8-1 & \textbf{740.85} & 1.41 & \textbf{740.85} & \underline{1.31} & 741.70 & 4.60 & \textbf{740.85} & 5.04 & \textbf{740.85} & 9.85\\
CON8-2 & \textbf{712.89} & 1.63 & \textbf{712.89} & \underline{1.50} & 713.44 & 4.64 & \textbf{712.89} & 4.40 & 713.05 & 12.08\\
CON8-3 & \textbf{811.07} & \underline{1.38} & \textbf{811.07} & 2.67 & \textbf{811.07} & 4.72 & \textbf{811.07} & 5.04 & 815.71 & 15.30\\
CON8-4 & 772.76 & 1.30 & \textbf{772.25} & \underline{0.74} & \textbf{772.25} & 4.74 & \textbf{772.25} & 6.03 & 777.24 & 10.68\\
CON8-5 & 754.95 & 1.31 & \textbf{754.88} & \underline{1.25} & 759.44 & 4.56 & \textbf{754.88} & 5.90 & 754.95 & 9.39\\
CON8-6 & 684.05 & 1.46 & \textbf{678.92} & \underline{3.97} & 685.80 & 4.83 & \textbf{678.92} & 5.44 & 686.34 & 8.60\\
CON8-7 & 814.50 & 1.22 & 812.89 & 2.35 & 814.79 & 3.67 & \textbf{811.96} & \underline{4.88} & 814.50 & 11.56\\
CON8-8 & 778.39 & 1.46 & \textbf{767.53} & \underline{2.19} & 780.80 & 4.10 & \textbf{767.53} & 6.20 & 785.30 & 8.20\\
CON8-9 & 810.18 & 1.41 & \textbf{809.00} & \underline{0.76} & 811.66 & 4.06 & \textbf{809.00} & 6.44 & 813.10 & 8.80\\
\hline\hline
\textbf{PROM.} & \textbf{760.11} & \textbf{1.33} & \textbf{758.56} & \textbf{1.87} & \textbf{764.35} & \textbf{4.12} & \textbf{758.54} & \textbf{4.11} & \textbf{762.89} & \textbf{6.82} \\[1ex]\hline
\end{tabular*}
\label{tabla-final-comparativaD}
\end{table}

\begin{table}[h]
\caption{ Resultados comparativos finales utilizando instancias de SalhiNagy}
\centering
\scriptsize
\begin{tabular}{|c||c c|c c|c c|c c|c c|c c|}
\hline
\textbf{Instancia} & \multicolumn{2}{c|}{\textbf{AS-M}} & \multicolumn{2}{c|}{\textbf{GTS}} & 
\multicolumn{2}{c|}{\textbf{GA-M}} & \multicolumn{2}{c|}{\textbf{ILS}} & \multicolumn{2}{c|}{\textbf{PSO-M}} & \multicolumn{2}{c|}{\textbf{SS}} \\\hline
 & Mejor & T. & Mejor & T. & Mejor & T. & Mejor & T. & Mejor & T. & Mejor & T.\\
 & costo & (seg) & costo & (seg) & costo & (seg) & costo & (seg) & costo & (seg) & costo & (seg)\\\hline\hline
CMT1X & 470.48 & 1.74 & 470.48 & 3.04 & 476.38 & 5.06 & \textbf{466.77} & \underline{1.96} & 470.48 & 5.06 & 473.23 & 1.15\\
CMT1Y & 471.25 & 1.35 & 470.48 & 2.40 & 472.87 & 4.56 & \textbf{466.77} & \underline{1.93} & 470.48 & 4.89 & 475.14 & 0.92\\
CMT2X & 691.13 & 6.64 & \textbf{682.39} & \underline{4.89} & 697.28 & 10.72 & 684.21 & 12.51 & 693.96 & 8.25 & 691.11 & 12.61\\
CMT2Y & 686.76 & 6.76 & \textbf{682.39} & \underline{5.95} & 697.76 & 10.15 & 684.21 & 10.45 & 693.74 & 5.95 & 690.50 & 11.42\\
CMT3X & 724.21 & 21.06 & \textbf{719.06} & \underline{7.89} & 728.50 & 24.66 & 721.40 & 13.90 & 723.67 & 60.99 & 730.11 & 33.82\\
CMT3Y & 723.52 & 22.69 & \textbf{719.06} & \underline{9.93} & 725.30 & 24.70 & 721.40 & 11.45 & 723.23 & 63.47 & 725.26 & 25.79\\
CMT4X & 865.47 & 94.77 & 854.21 & 17.23 & 880.66 & 63.16 & \textbf{852.83} & \underline{90.78} & 878.06 & 103.76 & 860.97 & 214.03\\
CMT4Y & 870.98 & 97.12 & \textbf{852.46} & \underline{21.48} & 892.05 & 65.54 & \textbf{852.46} & 82.44 & 872.63 & 79.17 & 868.50 & 230.74\\
CMT5X & 1077.34 & 273.74 & 1030.56 & 43.21 & 1068.63 & 134.22 & \textbf{1030.55} & \underline{95.78} & 1090.08 & 115.89 & 1041.19 & 955.04\\
CMT5Y & 1070.53 & 274.44 & 1031.69 & 40.35 & 1087.40 & 137.00 & \textbf{1031.17} & \underline{105.14} & 1089.93 & 102.17 & 1069.32 & 448.69\\
CMT11X & 847.74 & 43.28 & \textbf{831.09} & \underline{12.61} & 858.94 & 40.64 & 839.39 & 22.51 & 883.28 & 14.86 & 883.63 & 28.63\\
CMT11Y & 845.47 & 37.33 & \textbf{829.85} & \underline{11.44} & 848.68 & 45.25 & 841.88 & 26.49 & 887.01 & 16.95 & 873.19 & 83.94\\
CMT12X & 663.01 & 18.74 & \textbf{658.83} & \underline{9.03} & 668.48 & 24.83 & 662.22 & 14.35 & 691.19 & 3.83 & 686.38 & 25.99\\
CMT12Y & 663.19 & 18.18 & \textbf{660.47} & \underline{7.82} & 672.60 & 24.47 & 662.22 & 10.82 & 679.79 & 3.81 & 679.17 & 11.68\\
\hline\hline
\textbf{PROM.} & \textbf{762.22} & \textbf{65.56} & \textbf{749.50} & \textbf{14.09} & \textbf{769.68} & \textbf{43.93} & \textbf{751.25} & \textbf{35.75} & \textbf{774.82} & \textbf{42.08} & \textbf{767.69} & \textbf{151.48} \\[1ex]\hline
\end{tabular}
\label{tabla-final-comparativaS}
\end{table}
