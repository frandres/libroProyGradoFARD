\setcounter{page}{4}
\chapter*{Resumen}
%
%El Problema de Enrutamiento de Vehículos (VRP por las siglas de Vehicle Routing Problem) es un 
%problema de optimización combinatoria del área de logística y distribución de bienes. Este problema busca 
%determinar las rutas que deben transitar un grupo de vehículos con el objetivo de transportar la cantidad de
%productos demandada por un conjunto de consumidores de forma tal que se minimice la distancia total recorrida. 
%El Problema de Enrutamiento de Vehículos con Entrega y Recepción Simúltanea (VRPSPD por las siglas de Vehicle Routing 
%Problem with Simultaneus Pick-up and Delivery) es una variante de VRP que permite a los clientes tanto recibir como entregar bienes de manera simultánea. Este problema está clasificado como un problema complejo, pues resolverlo de manera exacta por un algoritmo implica tiempos que se incrementan de manera exponencial a medida que el número de clientes aumenta de tamaño. Las metaheurísticas constituyen un conjunto de técnicas que  permiten conseguir buena calidad de soluciones y tiempos computacionales aceptables para este timpo de problemas. En este trabajo se presenta la implementación de seis metaheurísticas para la resolución de VRPSPD. Experimentos son realizados, así como entonación de los parámetros más influyentes para cada metaheurística. Finalmente, se comparan los resultados obtenidos entre las mejores metaheurísticas y se recomienda la utilización de la mejor de ellas, la cual tiene resultados competitivos en comparación con los mejores algoritmos en la literatura.

El problema de enrutamiento de vehículos (VRP por las siglas de Vehicle Routing Problem) es un problema de optimización combinatoria del área de logística y distribución de bienes. Una de sus variantes es VRPSPD, el cual tiene la particularidad que los clientes pueden recibir y entregar bienes de manera simultánea.\\

En este trabajo se busca resolver VRPSPD por medio de diferentes metaheurísticas híbridas, con el propósito de comparar sus resultados y poder recomendar la que genere la solución de mejor calidad y tiempo.\\

Para ello se realizó una investigación sobre las metaheurísticas más utilizadas para resolver VRPSPD y se seleccionaron seis. Esas seis se implementaron y para cada una de ellas se estudiaron los parámetros que más influyen en su comportamiento, para luego entonar cada uno de ellos. Una vez entonada cada metaheurística, se comparó su desempeño con los resultados del trabajo utilizado como base para la implementación de la metaheurística. Las metaheurísticas son ejecutadas con un conjunto de instancias del problema recomendadas en la literatura, mayormente utilizadas en los trabajos referenciados. Finalmente, se seleccionaron las metaheurísticas que proporcionaron la solución de mejor calidad y menor tiempo según cada clase de instancia, y se recomendaron las mejores.

\newpage

