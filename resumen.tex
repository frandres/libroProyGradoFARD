\setcounter{page}{4}
\chapter*{Resumen}

DIA es una arquitectura para realizar consultas sobre información presente en un conjunto de documentos. Para ello, contempla poblar una base de datos con información extraída en una búsqueda amplia sobre un conjunto de documentos de interés; posteriormente se pueden realizar preguntas sobre esa información utilizando una ontología. La extracción focalizada es una fase que contempla DIA para buscar información en documentos una vez que se haya determinado que falta información para poder responder alguna consulta. El objetivo de este trabajo fue diseñar e implementar una propuesta para la tarea de Extracción Focalizada. Para ello se estudió el problema de la determinación de los orígenes de incompletitud en consultas hechas en DIA, se razonó sobre la información presente en la ontología y en la base de datos que se pueda utilizar para encontrar la información faltante y se construyó un extractor de información que trabaja con expresiones regulares. La solución propuesta fue probada utilizando actas de Consejos Directivos y Académicos de la USB, para realizar extracción sobre designaciones de jurados de ascenso, ascenso en el escalafón y designaciones administrativas. Los resultados obtenidos fueron: \\

Máximo 250 palabras.

\newpage

