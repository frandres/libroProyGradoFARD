\chapter{Pruebas} \label{chap:pruebas}

Descripción del conjunto de pruebas < Hablar de los dominios: se trabajó sobre designaciones, etc>

Duda: esto es una especie de marco metodológico?

\section{Selección de Documentos de Prueba}

La selección de los documentos se realizó sobre el conjunto de las actas de los Consejos Directivos y Académicos de la Universidad Simón Bolívar. Con el objetivo de probar apropiadamente el Extractor, se decidió trabajar con las actas desde el año 1998 hasta el año 2012. Esto permite que las pruebas se hagan sobre un conjunto de documentos semiestructurado, pero que puede mostrar variabilidad en su redacción y estructura en los años.

Del conjunto de actas en el intervalo de tiempo especificado, se hizo una preselección de los documentos para utilizar las actas de los Consejos Extraordinarios. Adicionalmente, según cada dominio de aplicación se extrajeron las actas que sí contienen información sobre los dominios. Es decir, se utilizaron los documentos que contienen información sobre los dominios elegidos.

Para probar el preprocesador, se tomó un conjunto de archivos seleccionados aleatoriamente con uniformidad sobre los años. La prueba consistió en ver si el fragmento de texto extraído por el preprocesador coincidía exactamente, estaba contenido, contenia o era completamente disjunto con el segmento de texto que debería ser extraído. Esta prueba se hizo manualmente. 

\section{Pruebas de Extractor Focalizado}

Para probar el extractor focalizado, se tomó un conjunto de prueba seleccionado aleatoriamente con uniformidad sobre los años. De cada archivo seleccionado, se tomaban hasta 3 designaciones (falta explicalro para los otros dominios), que se utilizaban para realizar pruebas. Una prueba consite en hacer una búsqueda focalizada sobre cada uno de los campos de una designación: se asume que se tienen algunos valores de esa designación (un contexto de extracción) y se buscan los campos faltantes. 

Estas pruebas se hicieron variando 2 parámetros. En primera instancia, se varió sobre el minimum hit measure (varió entre 0.25;0.5;0.75;1) que mide la proporción de los campos con valores conocidos que aparecen en una unidad de extracción (designación). Luego para cada hit measure se varió sobre la probabilidad de que cada uno de los campos fuese desconocido. Esto es, para cada hit measure al hacer una prueba se determinaba aleatoriamente cuantos campos tenían valores conocidos. Esto para poder simular los casos de la vida real en los que no se tienen todos los campos de una designación menos el conocido. Las probabilidades de no tener un valor puntual variaron entre 0.1;0.25;0.75. 

