\chapter{Pruebas} \label{chap:pruebas}

\section{Consideraciones Generales} 

Con el objetivo de realizar las pruebas sobre el sistema se trabajó como ya se ha dicho sobre tres dominios relacionados con el funcionamiento interno de la Universidad Simón Bolívar: las designaciones de cargos, los ingresos y ascensos dentro del escalafón de Profesores y la designación de Jurados para Trabajos de Ascenso. Para trabajar con esos 3 dominios se trabajó con las Actas de Consejos Directivos y Actas de Consejos Académicos. La elección de los 3 dominios se tomó buscando tener una variedad de estilos de texto semiestructurado para obtener resultados diversos sobre el funcionamiento. \\

La selección de los documentos se realizó sobre el conjunto de las actas de los Consejos Directivos y Académicos de la Universidad Simón Bolívar desde el año 1998 hasta el año 2012. Esto permite que las pruebas se hagan sobre un conjunto de documentos que pueden tener variabilidad en su estilo, redacción y estructura en los años. Del conjunto de actas en el intervalo de tiempo especificado, se hizo una preselección de los documentos para utilizar las actas de los Consejos Ordinarios. Adicionalmente, según cada dominio de aplicación se extrajeron las actas que sí contienen información sobre los dominios. Es decir, se utilizaron los documentos que contienen información sobre los dominios elegidos, descartando aquellas actas que no contenían texto de interés.\\

Una vez definidos los dominios se definieron varios conjuntos de prueba. En primera instancia se probaron el Determinador de Origen de Incompletitud y el uscador Focalizado de Información por separado. Luego se hicieron pruebas sobre ambos compontentes integrados. A continuación se describen con mayor profundidad las pruebas realizadas sobre el sistema. \\

\section{Pruebas unitarias sobre el Determinador de Origen de Incompletitud.} 

\section{Pruebas unitarias sobre el Buscador Focalizado de Información}

Para probar el Buscador Focalizado de Información se hicieron diseñaron 2 pruebas por dominio. Por un lado de hicieron pruebas para el Preprocesador de Texto y en segunda instancia Pruebas para el Extractor Focalizado. 

\subsection{Pruebas del Preprocesador de Texto}

Para probar el preprocesador, se tomó un conjunto de archivos seleccionados aleatoriamente con uniformidad sobre los años. La muestra elegida fue de un tercio de la población total de actas existentes, para cada dominio. Para las designaciones se trabajó con 73 documentos, para las incorporaciones y ascensos dentro del escalafón se trabajó con 77 documentos y para la designación de Trabajos de Jurado de Ascenso se trabajó con 49 documentos. \\

Las pruebas en esencia consistieron en ver si el fragmento de texto extraído por el preprocesador coincidía exactamente, estaba contenido, contenía o era completamente disjunto con el segmento de texto que debería ser extraído. Esto es: pueden verse ambos fragmentos de textos como secuencias de caracteres y se puede examinar si ambas secuencias coinciden por completo, tienen relaciones de contención o son completamente diferentes. \\

Esta prueba se hizo manualmente: se examinó el resultado de la extracción realizada por el preprocesador y se buscó dentro de cada documento el fragmento que debería extraerse. \\

Se definió como \emph{aprobado} que o bien el fragmento extraído sea completamente igual a la respuesta esperada (caso correcto) o si el fragmento extraído contiene a la respuseta esperada y algunos caracteres de más. Dicho caso se considera aprobado porque en esencia el que el documento preprocesado no contenga algunas líneas más de código no afecta considerablemente la extracción focalizada. Como \emph{no aprobado} se entienden los casos en los que la respuesta está incompleta (la respuesta esperada contiene al fragmento de texto obtenido) o que los fragmentos son completamente disjuntos. \\

\subsection{Pruebas de Extractor Focalizado}

Para probar el extractor focalizado, se tomó al igual que en las pruebas del Preprocesador de texto un conjunto de prueba seleccionado aleatoriamente con uniformidad sobre los años. La muestra es de un tercio de la población total. Para hacer esto de cada archivo seleccionado, se tomaban hasta 3 unidades de documento según lo explicado en \ref{sect:implementacion-extractorFocalizado}. Dichas unidades fueron vaciadas en un archivo de pruebas que posteriormente era leído por un módulo de pruebas que se encarga de probar el extractor focalizado automáticamente. \\

Una prueba individual consiste en hacer una búsqueda focalizada sobre un campo presente en una unidad de documento. Una unidad de documento da por lo tanto para hacer tantas pruebas como campos con valor tenga esa unidad. Si el dominio tiene por ejemplo 3 campos y se tiene una unidad de documento con 2 campos, se realizan 2 pruebas para esa unidad: una para cada campo. Las pruebas se hacen iterando entonces sobre cada uno de los campos con valores en la unidad y buscando extraer el valor del mismo utilizando un contexto de extracción dado por los valores de los campos que sí se tienen. \\ 

Para simular condiciones en las que no se tienen todos los valores de los campos relacionados con una unidad - esto es, que a pesar de que la unidad contenga valores para varios campos a la hora de hacer la búsqueda focalizada el \emph{contexto de extracción dado por la consulta} no contenga todos esos valores -, los contextos de extracción se construyeron con un subconjunto de los valores de los campos presentes en la unidad de documento. Este subconjunto se determina aleatoriamente: se genera un número al azar y si ese número es mayor que una \emph{probabilidad de tener cada campo} se incluye en el contexto de extracción. De esta manera se evitar suponer para estas pruebas que los contextos de extracción son completos.\\

Las pruebas se realizaron iterando el valor de la \emph{probabilidad de tener cada campo}. Se tomaron como \emph{probabilidades de tener cada campo} 1; 0.66 y 0.33. Adicionalmente, se iteró también sobre el \emph{Minimum Hit Measure} (según lo definido en \ref{sect:implementacion-extractorFocalizado}) con el objetivo de probar el efecto que tiene la elección de este parámetro en el funcionamiento del extractor focalizado. Los valores sobre los que se realizaron las pruebas fueron: 1; 0.66 y .33.\\

De esta manera, y resumiendo, las pruebas se realizaron tomando tres unidades de documento de cada documento de cada dominio. Para cada unidad se itera sobre la \emph{probabilidad de tener cada campo} y el \emph{Minimum Hit Measure}. Los resultados fueron posteriormente totalizados por dominio y tabulados. \\

