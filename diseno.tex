\chapter{Diseño} \label{chap:diseno}

La tarea de Extracción Focalizada de Información está compuesta por dos grandes labores: la Determinación del Origen de Incompletitud (DOI) y la Búsqueda Focalizada de Información (BFI). La Determinación del Origen de Incompletitud busca, como se ha dicho anteriormente, encontrar qué información está incompleta dentro de la ontología. Los objetivos de la misma deben ser seleccionar un dato que se quiera obtener en la búsqueda focalizada y un contexto que facilite la búsqueda.  La Búsqueda Focalizada de Información toma los resultados de la Determinación del Origen de Incompletitud y busca la información faltante en un conjunto de documentos relacionados al dominio de información de los conceptos involucrados. \\

A continuación se describirá en detalle esta solución propuesta para realizar ambos pasos. Ambas tareas son hechas por dos módulos diferentes: el Determinador del Origen de Incompletitud y el Buscador Focalizado de Información. Se hablará de cada uno de ellos y se harán algunos comentarios sobre el contexto de extracción que es el producto intermedio entre ambos módulos. 

\section{El Determinador del Origen de Incompletitud (DOI)}\label{sect:diseno-DOI}

Extractor plano: no se toman en cuenta ors. Quizas se debería.

\section{El Buscador Focalizado de Información (BFI)}\label{sect:diseno-BFI}

El Buscador Focalizado de Información es el encargado de encontrar los valores de los campos cuyos valores son desconocidos o incompletos para una consulta dada. Dichos valores son buscados en un conjunto de documentos. En el caso de este proyecto de grado, se requiere que los documentos estén presentes en el sistema de archivos donde se ejecute el sistema utilizando información de contexto proporcionada por el Determinador del Origen de Incompletitud (DOI). Sin embargo, como se ha dicho, la arquitectura DIA no presenta restricciones sobre la naturaleza de los documentos y podrían ser accedidos remotamente e inclusive utilizarse la World Wide Web.\\

En el caso del presente proyecto de grado, para llevar a cabo esta tarea, el BFI tiene dos componentes: un preprocesador de texto y un extractor focalizado de información. \\

\begin{figure}[h]
  \centering
  \includegraphics[width=1\textwidth]%
    {BFIdiagram}
  \caption[Diagrama del Buscador Focalizado de Información]
   {Diagrama del Buscador Focalizado de Información}
\end{figure}

\begin{comment}
\begin{figure}[hb]
  \centering
  \includegraphics[width=1\textwidth]%
    {DOIdiagram}
  \caption[Diagrama del Determinador del Origen de Incompletitud]
   {Diagrama del Determinador del Origen de Incompletitud}
\end{figure}
\end{comment}

El Preprocesador de Texto tiene como objetivo preparar el conjunto de documentos sobre el cual se hará la Búsqueda Focalizada de Información y dejarlo listo para la realización de la extracción. Las tareas para ello incluyen leer el documento en su formato original y extraer el fragmento o el conjunto de fragmentos del documento que están relacionados con el dominio sobre el cual se hará la Búsqueda Focalizada de Información. \\

En este proyecto de grado, los dominios elegidos se caracterizan por tener un conjunto de documentos bastante homogéneos y el preprocesamiento consiste en esencia en elegir los fragmentos de documentos que están relacionados con el dominio informativo sobre el cual se hará la extracción focalizada. Sin embargo, en general el preprocesamiento de documentos podría incluir tareas más amplias: puede darse que el conjunto de documentos sea heterógeneo y que se requiera una preparación adicional; puede darse analógamente que un mismo documento tenga información de diferentes dominios que se quieran trabajar. Puede ocurrir también que los documentos esten en más de un idioma. En todo caso lo relevante en todo esto es que el preprocesamiento es una labor que tiene como objetivo preparar a los documentos para realizar la extracción propiamente dicha y que puede incluir muchas subtareas.   \\

De igual forma, es importante destacar que el preprocesamiento de texto es concebido como una tarea que se realiza \emph{offline}. Es decir, cada conjunto de documentos puede ser preprocesado previamente de forma para que las consultas sean respondidas lo más rápidamente posible. \\

Por su parte, el Extractor Focalizado de Información (EFI) tiene como objetivo extraer el valor del campo faltante utilizando como materia prima el conjunto de documentos preprocesados y un contexto de extracción. El contexto de extracción, como se ha dicho anteriormente, consiste en toda la información que el DOI puede recopilar sobre la consulta que se está realizando que puede ser de útil para hacer la extracción focalizada de información. \\

La tarea de extracción es probablemente la que tiene una mayor complejidad asociada y depende muchísimo del tipo de documento que se trate y de que tan estructurado es. Cómo se ha dicho, en la medida en la que los textos sean mas estructurados - esto es, en la medida en la que el estilo sea regular y homogéneo - se hace más fácil la construcción del extractor. Sin embargo, esto supone un reto abierto en el mundo de la computación y los avances más recientes en el área de NLP incluyen extractores basados en modelos estocásticos y Machine Learning. \\

Anteriormente se han diseñado extractores para DIA con resultados bastante buenos. Ruiz (\emph{año}) \cite{ruiz-HMM} construyó un extractor basado en Hidden Markov Models (Modelos Ocultos de Markov) con resultados muy buenos. En el caso particular de este proyecto, no obstante, el objetivo principal no es la construcción de un extractor focalizado sofisticado. El extractor focalizado concebido trabaja con expresiones regulares, que buscan explotar al máximo la estructura de los documentos. En esencia, lo que se tiene para ello son una serie de reglas de extracción que especifican delimitadores anteriores y posteriores a la unidad de texto que se quiere extraer. Dicha tecnología se utiliza en este proyecto para el preprocesador de texto y para el extractor focalizado.\\

El motivo por el cual se eligió el uso de expresiones regulares es que los dominios de información tienen documentos semiestructurados,  que en general tienen un estilo de diagramación y redacción que permanece bastante similar. Esto es, es posible identificar frases y palabras que pueden servir de delimitadores para encontrar los datos requeridos. Para ello, es necesario que un usuario experto pueda identificarlas y compilar una lista de expresiones regulares. En todo caso, conviene destacar que cualquier extractor focalizado requiere intervención humana para poder configurar la manera como se extrae la información, bien sea porque involucra marcar documentos (Document Tagging) o porque requiera aprendizaje de máquina en el cual se requiere una configuración mínima por parte del usuario.\\

Es importante aclarar que la solución propuesta no depende del tipo de extractor que se construya. En general DIA y la FIE definen pasos necesarios para interrogar documentos; la construcción del extractor es una tarea que depende fundamentalmente del dominio sobre el cual se trabaje. \\

A manera de conclusión, es relevante decir que el DOI trabaja directamente con el Extractor Focalizado de Información para encontrar los valores faltantes. Esto es, una vez que se determinan las fuentes de incompletitud, el EFI recibe solicitudes de búsqueda para cada uno de los campos con valores faltantes. Esas solicitudes permiten descubrir y completar la ontología y dar respuesta a las consultas que tiene el usuario. El preprocesador, como se ha dicho, se ejecuta una única vez por conjunto de documentos. \\

