\chapter*{Acrónimos}

\newcommand{\TERM}[3]{
	\textbf{#1} & #3 & #2 \\[5pt]
}

\begin{longtable}{p{1in}p{2.2in}p{3in}}

		\TERM{FIE}{Extracción Focalizada de Información}{Focalized Information Extraction}
		\TERM{DIA}{Arquitectura de Interrogación de Documentos}{Document Interrogation Architecture}
		\TERM{BFR}{Buscador Focalizado de Información}{Buscador Focalizado de Información}
		\TERM{DOI}{Determinador del Origen de Incompletitud}{Determinador del Origen de Incompletitud}		
		\TERM{EFI}{Extractor Focalizado de Información}{Extractor Focalizado de Información}
		\TERM{NLP}{Natural Language Processing}{Procesamiento del Lenguaje Natural}				
		\TERM{BIE}{Broad Information Extraction}{Extracción Amplia de Información}				
		\TERM{NCF}{Normal Conjunctive Form}{Forma Normal Conjuntiva}		
%		\TERM{}{}{}		

\end{longtable}

		\textbf{DUDA: Que hago con los acrónimos que son sólo en español? Deberia poner los acronimos siempre en inglés?}
		
\newpage
